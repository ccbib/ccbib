\usepackage[ngerman]{babel}
\usepackage[T1]{fontenc}
\hyphenation{wa-rum}


%\setlength{\emergencystretch}{1ex}

%\renewcommand*{\tb}[1]{\begin{center}#1\end{center}}

\newcommand\bigpar\medskip

%\hyphenation{Bar-ring-ton}

\begin{document}
\raggedbottom
\begin{center}
\textbf{\huge\textsf{Die Jesaja-Mission}}

\medskip
Alexandra Keller

\end{center}

\bigskip

\begin{flushleft}
Dieser Text wurde erstmals veröffentlicht in:
\begin{center}
Die Steampunk-Chroniken\\
Band I -- Æthergarn
\end{center}

\bigskip

Der ganze Band steht unter einer 
\href{http://creativecommons.org/licenses/by-nc-nd/2.0/de/}{Creative-Commons-Lizenz.} \\ 
(CC BY-NC-ND)

\bigskip

Spenden werden auf der 
\href{http://steampunk-chroniken.de/download}{Downloadseite}
des Projekts gerne entgegen genommen. 

\vfill

Alexandra Keller studierte Geschichtswissenschaften in Tübingen,
Edinburgh und Köln.

\bigpar

Ihre Leidenschaft für England und viktorianische Kultur lebt sie
derzeit in der Entwicklung eines Spieles im Steampunk-Genre für
iPad und PC/Mac aus. Wenn sie nicht gerade Korsett tragende
Heroinen programmiert, die ihre Luftschiffe in gefährlicher Mission
durch die Atmosphäre des Planeten Sky steuern, schreibt sie für
ihren Firmen-Blog
\texttt{http://eye3ware.com/blog} auf
Englisch über Viktorianisches, sowie auf ihrem deutschen Blog
\texttt{http://stoerstoff.wordpress.com}
über alles andere.

\bigpar

Derzeit entsteht die Story zum Spiel »Sky Alchemist« in Roman-form.
Die Autorin lebt in Köln.
\end{flushleft}

\section{Die Jesaja-Mission}

\textit{»Niemals zuvor hatte die Menschheit eine so erschütternde Erfahrung
gemacht und niemals mehr wird sie eine ähnliche durchleben, es sei
denn, dass eines Tages ein anderer Globus auftauchte, Millionen von
Kilometern von unserem entfernt und ebenfalls von denkenden Wesen
bewohnt. Immerhin wissen wir heute, dass solche Entfernungen
theoretisch zu überwinden sind, während die ersten Seefahrer
fürchteten, dem Nichts zu begegnen.«
}

\begin{flushright}
Claude Lévi-Strauss, Traurige Tropen
\end{flushright}

\bigpar

Am 18. November des Jahres 1898 fing Douglas McLoughlin,
diensthabender Funker der Æther-Morsestation \textit{Ben Nevis} in den
schottischen Highlands, ein rätselhaftes Signal auf. Es war fünf
Uhr morgens, Douglas hatte am Vorabend ein paar Gläschen Scotch
über den Durst getrunken und sein Schädel fühlte sich an wie ein
zum Platzen reifer Kürbis. Douglas griff mit unsicheren Fingern
nach einem Bleistift und begann den Funkspruch niederzuschreiben.

Das Signal war schwach und brach hin und wieder ab, trotzdem gelang
es ihm, eine wenn auch lückenhafte Folge von Zeichen aufzunehmen.
Der Code war ihm fremd. Keines der Luftschiffe der Æthereal Fleet
verwendete eine derartige Verschlüsselung. Douglas trug die
Zeichenfolge in sein Logbuch ein und leitete den Vorgang weiter an
die zentrale Auswertungsstelle in London.

\bigpar

Funkstationen in Deutschland, Frankreich und China fingen von nun
an ebenfalls regelmäßig Funksprüche auf, die aus derselben Quelle
zu stammen schienen, jedoch andere Zeichensequenzen sendeten. Unter
den Geheimdiensten der Großmächte setzte ein Wettlauf um das
Entschlüsseln der Funksprüche ein.

\bigpar

In London vermutete man, es handele sich um die chiffrierte
Botschaft eines Militär-Luftschiffes, und sandte die aufgefangenen
Sequenzen zum Entschlüsseln ins militärische Kalkulatoren-Zentrum
in Compton Hall. Hier standen die großen dampfbetriebenen
Analysemaschinen, die mit Hilfe von Lochkarten komplexe
Rechenoperationen durchführen konnten. Das Militär nutzte sie
hauptsächlich zum Dechiffrieren feindlicher Funksprüche.

\bigpar

Fünf Analysemaschinen, jede von der Größe eines kleinstädtischen
Rathauses, arbeiteten eine Woche unter Volllast an der
Entschlüsselung. Sämtliche bekannten Dekodierungs-Algorithmen
wurden angewandt. Die Rechenarbeit musste allerdings einmal
unterbrochen werden, weil eine Ratte in den Mechanismus von
Maschine fünf geraten und zerquetscht worden war.

\bigpar

Schließlich gelang es den Analysemaschinen, eine der Sequenzen zu
dekodieren:

»Es wird \ldots{} Krachen \ldots{} brechen, … \ldots{} und fallen«.

\bigpar

Colonel George Fitzwilliam, Vorsteher des Kalkulatoren-Zentrums saß
in seinem Arbeitszimmer und sinnierte über die Bedeutung des
rätselhaften Funkspruchs, als Mrs Hall, seine Haushälterin, den Tee
brachte.

Mrs. Halls praktische Erfahrung mit Dechiffrierung beschränkte sich
auf das Entziffern der stets unleserlich geschriebenen
Metzgerrechnung und der vom Zimmermädchen Lizzy schlampig geführten
Wäschelisten. »Sie sollten die Heilige Schrift lieber im Original
studieren, Sir«, bemerkte sie in strengem Ton nach einem
beiläufigen Blick auf das Blatt, über dem der Colonel brütete. Der
Colonel schaute fragend auf. »Jesaia 24«, erklärte Mrs Hall. Und
laut wie die Posaune des jüngsten Gerichtes verkündete sie: »Es
wird die Erde mit Krachen zerbrechen, zerbersten und zerfallen«.

\bigpar

Eine weitere Woche verging, dann kam eine hochrangige
Gelehrtenrunde aus Æther-Spezialisten, Physikern, Kalkulatoren und
Theologen zusammen, um den Funkspruch zu analysieren. Die Experten
waren uneins: Die Theologen vertraten die Meinung, der Spruch sei
eine direkte Offenbarung Gottes und kündige das Ende der Welt an,
während die Æther-Spezialisten die These vertraten, das Signal
beweise die Existenz intelligenten Lebens im All. Die Physiker
wandten ein, in 30 Jahren bemannter Schifffahrt im Æther habe man
noch keine Spur intelligenter Lebensformen gefunden und überdies:
warum sollten Außerirdische, falls sie denn existierten, Verse aus
der Bibel senden?

Der Vertreter der Kalkulatoren, ein schüchternes Männchen in
abgewetztem Gehrock und mit einer übergroßen runden Brille schlug
vor, den Funkspruch erneut dechiffrieren zu lassen. Es könne ja
schließlich ein Fehler beim Entschlüsseln passiert sein. Immerhin
habe man den Rechenvorgang einer Analysemaschine unterbrechen
müssen, da ein organisches Objekt den Mechanismus blockiert habe.

Die Mehrheit der versammelten Experten war jedoch der Auffassung,
der Spruch sei eindeutig, ein Fehler bei der Dekodierung könne
daher nicht vorliegen.

Schließlich bekam die Presse Wind von der Sache, und damit stand
die Regierung unter Zugzwang.

\bigpar

Egal ob Hilferuf eines unbekannten Æther-Schiffes, Signale einer
außerirdischen Lebensform, oder eine direkte Nachricht DES HERRN:
es war zwingend notwendig eine Mission ins All zu senden, um dem
Ursprung des Signals auf die Spur zu kommen. Da sich die
Regierungen Englands, Frankreichs und Chinas nicht über die
Aufteilung der Kosten einigen konnten, unterbreitete der bekannte
Abenteurer und Analysemaschinen-Millionär Frederick Barrington-Ward
der britischen Regierung ein Angebot, das diese nicht ablehnen
konnte: die Regierung solle das Schiff und die militärische
Besatzung stellen, er werde ein Team hochrangiger Wissenschaftler
beisteuern. Überdies versprach er, die Ausrüstung des Schiffes,
sowie Löhne und großzügige Boni für die Expeditionsteilnehmer zu
finanzieren. Um die Unterstützung der konservativen Kreise Englands
zu erhalten, verpflichtete er sich zudem, zwei Plätze auf dem
Ætherschiff für Missionare zu reservieren.

\bigpar

Und so startete am 3. Februar 1899 der schwere Kreuzer \textit{Phönix} unter
seinem erfahrenen und im Dienst der Æthereal Fleet in Ehren
ergrauten Kapitän Jennings zu seiner großen Fahrt. Im
Aufenthaltsraum der \textit{Phönix} drängten sich die zivilen Teilnehmer der
Expedition um die großen Panoramafenster und genossen mit lautem
»Ahh« und »Ohh« das Spektakel des Ablegens.

Die Docks der Æthereal Fleet waren über und über mit lustig
flatternden Fahnen und Wimpeln geschmückt. Es war ein Fest. Die
Kleider der Damen leuchteten in zarten Pastellfarben, die Herren
trugen bunte Einstecktüchlein, Kinder hielten bunte Fähnchen in den
von Zuckerwatte verklebten Händchen. Militärkapellen spielten
flotte Märsche, ein Chor der Heilsarmee trug freudige Hymnen vor.
Fundamentalistische Prediger forderten zu Umkehr und Buße auf
angesichts der drohenden Apokalypse. Die Huren aus London und
Umgebung machen Extraschichten.

Es wurde bereits dunkel, als die \textit{Phönix} schließlich aufstieg.

Das nächtliche London lag unter ihnen, rote Lampions leuchteten wie
Mohnblüten um die Hafenbecken und spiegelten sich vieltausendfach
in der Schwärze des Hafenwassers. Salutschüsse dröhnten, die
zuckenden Lichtblitze des Abschieds-Feuerwerks erhellten das
zerklüftete Gebirgsmassiv der neugotischen Parlamentsgebäude.
Bleich und löchrig wie ein vergammelter Cheddar-Käse hing am
südlichen Horizont der volle Mond.

\bigpar

Manchmal kam ein besonders großer feuerspeiender Drache oder eines
der großen Sternenräder dem Luftschiff so nahe, dass Penelope den
Geruch verbrannten Pulvers zu riechen glaubte.

Penelope dachte an die Wasserstofftanks des Luftschiffes, und
daran, dass nur eine dünne Hülle aus gummierter Leinwand zwischen
ihr und dem freien Fall aus mehreren hundert Fuß Höhe stand.
Frederick Barrington-Ward, Mäzen und wissenschaftlicher Leiter der
Expedition, schlenderte zu ihr herüber. Lässig lehnte er sich gegen
die Fensterbrüstung und hob sein Champagnerglas: »Auf eine
erfolgreiche Mission -- wen immer wir antreffen«. Penelope lächelte
höflich. Ihr Blick fiel auf einen drahtigen Mann mit militärischem
Bürstenhaarschnitt und buschigem grauem Schnauzbart. Seine runden
Brillengläser blitzten im Licht der Gaslampen wie übergroße Augen
eines nachtaktiven Tiers. Dr. von Todt, Astronom an einer
mitteleuropäischen Universität, hatte anhand der aufgefangenen
Radiosignale die Zielkoordinaten der Expedition berechnet. Es war
ein Punkt in der Umlaufbahn des Mars.

Eine voluminöse Dame in züchtigem Schwarz und vernünftigen Schuhen
gesellte sich zu ihnen, ihren verschüchterten Ehemann hinter sich
her zerrend. »Ist das Feuerwerk nicht furchtbar gefährlich?«,
wandte sie sich direkt an Mr. Barrington-Ward und vermied es,
Penelope anzuschauen.

»Aber nein, meine Gnädigste«, antwortete der Millionär, »unsere
chinesischen Feuerwerker können die Höhe und Flugbahnen ihrer
Feuerwerkskörper genau vorausberechnen. Mit charmantem Lächeln
fügte er hinzu: »Es sei denn, sie wollten uns hochgehen lassen.«

Mrs Collins, als Gattin eines Missionars auch ausländischen Heiden
gegenüber zu christlicher Milde verpflichtet, schien dieser Gedanke
zu schaffen zu machen. Mr Collins wagte einen Blick auf Penelope,
ihr glänzendes schwarzes Haar, ihren makellosen elfenbeinfarbenen
Teint und die großen dunklen Augen und sagte: »Ethel, meine Liebe,
der HERR wird unsere Reise schützen und leiten und er wird uns
nähren mit Milch und Honig.« Sein Blick saugte sich an Penelopes
Brüsten fest, die durch die hochgeschlossene Uniform noch betont
wurden.

Penelope Fairfax, Tochter eines streng atheistischen Mathematikers,
der mit der Massen-Produktion billiger Analysemaschinen reich
geworden war, hob amüsiert eine Augenbraue.

\bigpar

Ethel und Nathaniel Collins hatten 15 Jahre als Missionare
segensreich unter den Heiden Indiens gewirkt und dort feste
Grundsätze hinsichtlich des richtigen Umgangs mit fremden Spezies
erworben. Aufgrund ihrer überdurchschnittlichen Taufquote waren sie
von der Missionsgesellschaft »First Missionary Society for Godly
Predestination« ausgewählt worden, den Außerirdischen, die sich ja
bereits durch ihren Funkspruch als empfänglich für das Wort Gottes
erwiesen hatten, das Licht des Evangeliums zu bringen.

»Was macht eine so hübsche junge Lady wie Sie denn an Bord eines
Luftschiffes?«, wandte sich nun Dr. von Todt an Penelope. Sein
Englisch hatte einen schwer zu lokalisierenden Akzent. Das »R«
sprach er rollend, sein »S« zischelte feucht. Seine vollen Lippen
machten kleine nervöse Bewegungen und die Enden seines Schnauzbarts
zuckten. Penelope nahm wahr, dass aus den Ohren des Doktors dicke
graue Haarbüschel wucherten. In die allgemeine Stille hinein sagte
Penelope: »Ich bin die Navigatorin.«

\tb

Attila hatte sich hinter einem Fass mit Pökelfleisch verkrochen.
Die Kälte im Frachtraum machte ihm nichts aus, sein dickes Fell
schützte ihn. Unter seinen Pfoten fühlte er das Vibrieren der
Maschinen. Das Leben als Æther-Hund war leichter, als in den
Hundezwingern von Soho, wo er als Welpe die ersten Monate seines
Lebens verbracht hatte. Essensreste gab es genug in der Kombüse,
und ein geschickter Rattenjäger konnte sich im Frachtraum noch mit
zusätzlichem Fleisch versorgen. Doch er vermisste den
Geruchswirrwarr der Erde, den Duft nach Gras, Abfall und den
Markierungen anderer Rüden. Hier oben gab es weit weniger Gerüche,
dafür nahm seine Nase sie intensiver wahr. Daher war der Mensch
ohne Geruch für Attila eine beständige Quelle der Irritation. Der
geruchlose Zweibeiner hatte soeben den Frachtraum betreten und
bewegte sich zielstrebig zwischen Kisten und Fässern auf eine
unscheinbare braune Holzkiste zu. Attila hörte ein leises
Quietschen, als er die Kiste öffnete und einige schwere Gegenstände
heraus nahm. »Schinken«, roch Attila. Kurz darauf stach der Geruch
leicht erhitzten Metalls in seine Nase und seine empfindlichen
Hundeohren nahmen ein dünnes hohes Sirren wahr, weit außerhalb
dessen, was ein Mensch hören konnte. Ein prägnanter Rhythmus,
mehrfach wiederholt. Attila heulte laut auf. Seine Ohren
schmerzten. Er schoss hinter seinem Pökelfass hervor und zwängte
sich durch eine enge Luke in den angrenzenden Laderaum. Der
geruchlose Mensch blieb zurück.

\tb

Die \textit{Phönix} war bereits vier Monate unterwegs, als Penelope Fairfax,
Erster Navigations-Offizier, eine Unstimmigkeit beim Abgleich der
Schiffkoordinaten mit den Sternenkarten feststellte. Die \textit{Phönix}
steuerte zehn Grad außerhalb des vorberechneten Kurses. Penelope
korrigierte die Navigation, machte einen Logbucheintrag und
informierte Kapitän Jennings. Einige Tage später stellte Penelope
fest, dass das Schiff erneut vom Kurs abgekommen war. Niemand von
der Besatzung – alle hatten bereits mehrere Reisen im Æther hinter
sich – hatte jemals etwas Ähnliches erlebt. Die \textit{Phönix} hätte sich
inzwischen im Sternbild des Wassermanns befinden müssen, nach
Penelopes Messungen bewegte sie sich jedoch auf Pegasus zu. Das
Radiosignal wurde stärker.

\bigpar

Die zivilen Teilnehmer der Expedition erfuhren von den
Navigationsproblemen beim Dinner – Lammkoteletts, Sauerkraut und
Yorkshire-Pudding. Bislang hatte der Koch keine Veranlassung
gesehen, am Proviant zu sparen.

»Wir mussten feststellen, dass bei der Berechnung unserer
Zielkoordinaten ein Fehler aufgetreten ist«, leitete der Kapitän
das Thema ein. »Das Signal scheint aus dem Inneren von Pegasus zu
kommen, nicht von der Umlaufbahn des Mars.« Mr Barrington-Ward
lächelte leichthin. »Dann sollten wir unseren Kurs ändern.« »Aber
wie konnte das denn passieren, Kapitän Jennings?«, fragte Mrs
Collins. »Ethel, meine Liebe, die VORSEHUNG wacht über unser
Schicksal, wo immer wir sind«, sagte Mr. Collins salbungsvoll. Mrs
Collins verstummte, ihre Miene machte jedoch deutlich, dass sie das
Walten der VORSEHUNG in diesem Fall äußerst dilettantisch fand. Dr.
von Todt hielt geziert sein Lammkotelett in seinen überaus
behaarten Händen und nagte mit großen weißen Schneidezähnen das
Fleisch vom Knochen.

\bigpar

Kapitän Jennings versammelte seine Offiziere und die
Wissenschaftler zu einer Beratung über das Schicksal der
Expedition.

Lieutenant Leigh-Hunt, der erste Offizier, schlug die sofortige
Rückkehr zur Erde vor. Eine Verlängerung der Reise stelle ein zu
hohes Risiko hinsichtlich Treibstoff und Lebensmittelreserven da.

Lieutenant Penelope Fairfax, die Navigatorin, riet ebenfalls ab,
dem Signal weiter zu folgen. Dr. von Todt hingegen – seine
Schnurrbartspitzen vibrierten erregt, und Penelope glaubte einen
Moment, auch seine Ohren zuckten – befürwortete eine Fortsetzung
der Reise. Es sei zu erwarten, dass die Außerirdischen, die das
Signal sendeten, über ausreichend Ressourcen verfügten, um das
Schiff für die Rückkehr auszurüsten.

Kapitän Jennings entschied sich, dem Radiosignal zu folgen.

\tb

Die \textit{Phönix} war nun beinahe 10 Monate unterwegs. Das Signal schien
ferner und ferner zu rücken. Das Schiff war weit ins Sternbild des
Pegasus eingedrungen und bewegte sich in Richtung des Großen Bären.
Attila lebte inzwischen ausschließlich von Ratten. Zum Dinner gab
es für die Expeditionsteilnehmer Grütze mit Speckgrieben. Alle paar
Tage kam der geruchlose Zweibeiner in den Laderaum und quälte
Attila mit flirrenden Pfeiftönen.

\tb

Penelope stand auf der leeren Kommandobrücke. Die Offiziere wohnten
der Abendandacht bei. So viele Male in den letzten Monaten hatte
sie die Koordinaten neu berechnet. Das Schiff schien jedoch
beharrlich seinem eigenen Kurs zu folgen, dem undurchschaubaren
Plan eines fremden Willens unterworfen, nicht der peilenden,
messenden Kunst der Navigatorin. Penelope schaute hinaus in die
geheimnisvolle, lockende Schwärze, sah ferne Gestirne, das Licht
toter Sonnen, die ihre rätselhaften Lichtbotschaften ins All
sandten. Die Sterne des Pegasus schienen zum Greifen nahe. Südlich
strahlte rot der Mars, das ursprüngliche Ziel der Mission. In der
Weite des Universums, den unendlichen Räumen des Alls, schrumpften
das Schiff, seine Besatzung und ihrer aller Leben auf die Größe von
Staubkörnchen zusammen. In der Stille der von allen verlassenen
Kommandobrücke sah Penelope nun die Wahrheit, die seit Wochen
darauf wartete ausgesprochen zu werden: Die \textit{Phönix} war dem Signal
zu lange gefolgt. Die Wasser- und Treibstoffvorräte würden noch
etwas mehr als zwei Monate reichen. Wenn sie nicht bald einen
bewohnbaren Planeten erreichten wo sie Treibstoff, Wasser und
Lebensmittel aufnehmen konnten, war das Schiff verloren.

\bigpar

Schritte hinter ihrem Rücken rissen sie aus ihren Gedanken.
Frederick Barrington-Ward hatte die Kommandobrücke betreten. »So
sieht also die Unendlichkeit aus, Lieutenant Fairfax«, – der
Millionär blickte hinaus auf das schillernde Band der Milchstraße.
»Warum haben Sie sich für diese Mission freiwillig gemeldet?«
Penelope zögerte. Niemand auf dem Schiff hatte sie je gefragt,
warum sie, eine junge, reiche Frau mit besten beruflichen
Aussichten in der Æthereal Fleet, sich zu einer so gefährlichen
Mission gemeldet hatte. Für andere wäre ein Platz im
Expeditionsteam ein Sprungbrett für einen schnellen Aufstieg
gewesen. Wieder andere lockten die phantastisch hohen Prämien, die
Barrington-Ward für alle Teilnehmer der Expedition ausgelobt hatte.
Die Collins konnten zusätzlich zu den irdischen Reichtümern auch
noch mit himmlischem Lohn rechnen, wenn sie die Außerirdischen zum
Evangelium bekehrten. Penelope jedoch würde mehr Geld erben, als
sie jemals ausgeben konnte und mit ihrem Vermögen hätte sie sich
jederzeit eine prestigeträchtige Offiziersstelle kaufen können. Ihr
früheres Leben erschien ihr plötzlich klein und unwichtig. Sie
dachte an Charles Osgood, ihren Verlobten und daran, dass sie durch
die Teilnahme an dieser Mission die unvermeidliche Heirat mit einem
ungeliebten Mann hatte aufschieben wollen.

Sie hatte Barrington-Ward in den letzten Monaten als anregenden
Gesprächspartner schätzen gelernt. Zum ersten Mal nahm sie ihn nun
als Mann wahr. Penelope griff nach seiner Hand und zog ihn an sich.
Und während sich ihre Körper aneinander pressten, ihre Lippen
zueinander fanden und sie sich im Angesicht der Sterne liebten, sah
Penelope mit aller Klarheit, dass sie nicht mehr zurückkehren
würden.

\tb

Die Reise dauerte nun schon über ein Jahr. Attila lag im Frachtraum
und träumte vom Wohlgeruch der Abfallhalden in Soho, einer
wundervoll duftenden schwarzen Hündin und davon, wie er einmal eine
Hammelkeule gestohlen hatte. Seine Pfoten zuckten, er bellte leise
im Schlaf. Zuweilen gab ihm der Koch ein paar Knochenreste oder
einen Zwieback, der zu wurmzerfressen war, um ihn der Mannschaft
vorzusetzen. Die Ratten, nun Attilas Hauptmahlzeit, wurden dünn und
ausgemergelt. Der ohne Geruch kam immer noch regelmäßig in den
Frachtraum.

\tb

Der Funker stellte fest, dass sie sich der Quelle des Funksignals
näherten. Am 23. März 1900 fand die \textit{Phönix} in der Umlaufbahn des
Sterns Enif die Quelle des Funkspruchs. Sämtliche Mitglieder des
Expeditionsteams waren auf die Kommandobrücke geladen, während sich
das Schiff dem rätselhaften Objekt näherte. Eine achteckige Kapsel,
groß wie eine englische Kathedrale, schwebte im strahlenden roten
Licht Enifs. Die Oberfläche bestand aus bläulich schimmernden
Metall, das übersät war mit rostigen Pockennarben. »Ein
Meteoritensturm«, vermutete Kapitän Jennings. Das ausgesandte
Vorauskommando kam mit der Nachricht zurück, es gäbe eine Luke an
der oberen Schmalseite der Kapsel.

Die Mitglieder des Expeditionsteams, bestehend aus Frederick
Barrington-Ward, Dr. von Todt, Mr. Collins, sowie fünf
Marinesoldaten, geführt von Penelope Fairfax, stiegen in ihre
Raumanzüge. Mrs Collins, in Erwartung außerirdische Wesen
vorzufinden, die nach der spirituellen Nahrung des Evangeliums
hungerten, drückte ihrem Gatten eine Bibel in die Hand.

Die Gruppe bestieg den kleinen, Wasserstoff-betriebenen Raumgleiter
und legte ab. Als sie sich der Kapsel näherten, fiel Penelopes
Blick auf Dr. von Todt. Der Gelehrte schien merkwürdig erregt,
seine Hände zuckten, seine Brillengläser blitzten im grellen Licht
Enifs. Die Marinesoldaten benutzten schweres Gerät, um die Luke zur
Kapsel aufzustemmen. Schließlich zeigte sich eine schmale Treppe,
die nach unten führte. Die metallenen Treppenstufen waren überzogen
von einem dunklen flechtenartigen Belag. Penelope führte die Gruppe
nach unten. Sie nahm die Maske vom Gesicht. Die Luft roch
metallisch und dumpf. Die Kapsel musste über eine autarke
Sauerstoffversorgung verfügen. Immer tiefer führten die gewundenen
Stufen. Schließlich erreichte Penelope eine sechseckige Tür. Nur
mühsam gelang es vier Marinesoldaten, die Tür nach innen
aufzustemmen.

\bigpar

Das erste, was Penelope wahrnahm war der Geruch. Ein süßlich
fauliger Dunst quoll ihnen entgegen. Penelope trat in den Raum, der
einmal eine Kommandobrücke gewesen war. Im Dämmerlicht sah sie
schmutzverkrustete Sternenkarten, rostige Navigationsinstrumente,
Geräte, deren Funktion Penelope nicht erkennen konnte. Alles war
überzogen von einer dicken Schicht Staub und Unrat. Als nächstes
nahm sie die Ratten wahr. Hunderte, Tausende von ihnen wimmelten
auf den Kartentischen und Instrumentenborden, schmiegten sich in
Bücherregalen aneinander und nisteten in den Rohrsystemen, die sich
die Wände entlangzogen.

\bigpar

Penelope versuchte ein Würgen zu unterdrücken und machte einen
weiteren Schritt in den Raum hinein. Der Raum vibrierte vom
Rascheln, Scharren und Fiepen der Nager. Doch über dem Grundton aus
tierischen Lauten nahmen Penelopes Ohren ein weiteres Geräusch auf:
Technisch, und in gleichmäßigem Rhythmus. Ein Morseapparat. Der
Apparat schien Signale zu empfangen und auszusenden. Auf der
Morsetaste saß, wie auf einer Kinderwippe, eine dicke weiße Ratte
und putzte ihr Fell. Die Morsetaste federte mit klickendem Geräusch
auf und nieder und sendete den Spruch der Ratte hinaus in die Weite
des Alls.

\section{Epilog}

Der schwere Kreuzer \textit{Phönix} blieb verschollen. Der einzige
Überlebende der Mission wurde Monate später in einer Rettungskapsel
von der Fregatte \textit{Princess Albertina} in der Nähe des Mars
aufgegriffen. Es handelte sich um Dr. von Todt. Bei ihm fanden sich
Knochen- und Fellreste, die vermuten ließen, von Todt habe sich von
der Leiche eines Hundes ernährt. Außerdem sprangen zwei weiße
Ratten aus der Raumkapsel und verschwanden blitzschnell in den
Lüftungsschächten der \textit{Princess Albertina}. Dr. von Todt konnte
nichts über das Ende der \textit{Phönix} berichten, da er der menschlichen
Sprache nicht mehr mächtig schien. Gelegentlich, bei Vollmond,
befielen ihn unerklärliche Erregungszustände, unter deren Einfluss
er ein hohes Pfeifen ausstieß.

Er wurde ins \textit{Mountain View Asylum for the Mentally Sick}
eingewiesen, von wo er im Jahr 1905 für immer verschwand.

\bigpar

Colonel Fitzwilliam, der Vorsteher des Kalkulatorenzentrums Compton
Hall ordnete im Jahr 1903 eine Generalüberholung der
Analysemaschine fünf an. Als die Arbeiter die Maschinenteile
auseinandernahmen, fanden sie unterhalb des Metallmechanismus,
unter einem hölzernen Zwischenboden, ein ungeheures Rattennest.
Tausende von Lochkarten waren von den emsigen Nagern zum Nestbau
verwendet worden – »ein verdammtes stinkendes Datenschlamassel«,
wie Bertie Blunt, der Mechaniker, sich ausdrückte. Anhand der
Markierungen auf den Lochkarten stellte man fest, dass die Karten
zum Entschlüsselungsauftrag des Ben-Nevis Funkspruchs gehört
hatten.

\bigpar

Colonel Fitzwilliam hat diese Entdeckung nie verkraftet. Er reichte
seine Demission ein und zog sich aufs Land zurück. Es heißt, er
habe seine Haushälterin geheiratet und sei religiös geworden.

\bigpar

Ausgehend von den Docks der Æthereal Fleet hat sich in London im
Laufe der letzten Jahre eine neue Rattenart ausgebreitet.
Wissenschaftler haben herausgefunden, dass sie weit intelligenter
sind als unsere einheimischen braunen Ratten. In einigen
Lagerhäusern haben sie bereits die Herrschaft übernommen.

\end{document}

