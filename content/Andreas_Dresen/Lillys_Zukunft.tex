\usepackage[ngerman]{babel}
\usepackage[T1]{fontenc}
\hyphenation{wa-rum Fracht-raum}
\hyphenation{schien}
\hyphenation{Tief-ebe-ne Tief-ebe-ne gro-ßen}


%\setlength{\emergencystretch}{1ex}

\renewcommand*{\tb}{\begin{center}* \quad * \quad *\end{center}}

\newcommand\bigpar\medskip
\newcommand\gedanke\textit

\begin{document}
\raggedbottom
\begin{center}
\textbf{\huge\textsf{Lillys Zukunft}}

\medskip
Andreas Dresen
\end{center}

\bigskip

\begin{flushleft}
Dieser Text wurde erstmals veröffentlicht in:
\begin{center}
Die Steampunk-Chroniken\\
Band I -- Æthergarn
\end{center}

\bigskip

Der ganze Band steht unter einer
\href{http://creativecommons.org/licenses/by-nc-nd/2.0/de/}{Creative-Commons-Lizenz.} \\
(CC BY-NC-ND)

\bigskip

Spenden werden auf der
\href{http://steampunk-chroniken.de/download}{Downloadseite}
des Projekts gerne entgegen genommen.

\vfill

Andreas Dresen, Jahrgang 1975, lebt und arbeitet in seiner
Heimatstadt Aachen. Schon immer war er von fremden Welten
fasziniert – von der wilden Atlantik-Küste Südirlands genauso wie
von den Sagen und Legenden seiner Heimat. Und so findet sich in
seinen Kurzgeschichten genauso wie in seinem Debütroman »Ava und
die STADT des schwarzen Engels« eine fesselnde, gleichsam skurrile
und charmante Mischung aus Fantasy-Elementen, klassischer
Mythologie und einem scharfen Blick für die Kuriositäten der
Gesellschaft und des Alltags. Im epospresse-Verlag sind inzwischen
auch eine kleine Anzahl ebooks erschienen.

\bigpar

\texttt{http://www.andreas-dresen.de/}

\texttt{http://www.epospresse-verlag.de/}
\end{flushleft}

\section{Lillys Zukunft}

Du gehörst mir!« Eugene strich Lilly durch das kurze blonde Haar,
bevor er sie fest im Nacken packte und zu sich zog. Sein Gesicht
kam so nah, dass Lilly die kleinen Schweißtropfen auf seiner Stirn
sehen konnte. »Vergiss das nicht. Du bist mein Eigentum. Also hast
du deinen Beitrag zu leisten, klar?«

»Ja, Eugene.« Lilly hielt seinem Blick kurz stand, dann wollte sie
den Kopf abwenden, doch Eugene hielt sie fest. Als sie einander
ansahen, hörten sie das Stampfen der Maschinen, das sonst nur als
Hintergrundgeräusch wahrnehmbar war, laut und deutlich in der
Stille. Auf der untersten Ebene des Raumschiffs \textit{Kleine Hoffnung} war
es feucht und stickig. Die Luft wurde nur unregelmäßig
ausgetauscht, da die wenige Energie die übrig geblieben war, für
das Kolonistendeck genutzt wurde.

Eugene hatte seine Schiebermütze weit in den Nacken geschoben, so
dass sein verschwitztes Haar zum Vorschein kam. Sein
fadenscheiniger Anzug und das weiße, kragenlose Hemd waren
eigentlich viel zu warm für diese Umgebung, aber Eugene achtete
stets auf sein Aussehen. Der erste Eindruck zählt, sagte er immer
wieder zu Lilly, und wenn ich jemandem dafür eine blutige Nase
schlagen muss.

»Und jetzt mach dich ein bisschen hübsch«, sagte er. »Die Reise
geht bald zu Ende und ich habe nicht vor, mit leeren Händen in der
Kolonie anzukommen. Das ist unsere große Chance, die werde ich mir
nicht vermasseln lassen. Ich würde meinen rechten Arm dafür geben.
Dann haben wir ausgesorgt!« Er ließ sie los. Lilly stolperte zurück
und strich sich das einfache Kleid glatt. Der Ausschnitt zeigte
etwas mehr nackte Haut, als es der aktuellen Mode entsprach. Das
sonst übliche Korsett hätte ihrer fülligeren Figur sicher eine
schmalere Form verliehen, doch Eugene war es lieber so. Es sprach
die Kunden mehr an. Auf der unteren Ebene war man eine direkte
Sprache gewohnt.

Trotzig streckte Lilly ihr Kinn nach vorn, so dass ihr zum Bubikopf
geschnittenes Haar nach hinten schnellte.

»Es ist noch gar nicht sicher, ob wir überhaupt aufgenommen
werden.«

»Eben. Und ich will die Wahrscheinlichkeit, dass wir abgewiesen
werden, minimieren.« Er griff in seine Brusttasche und zog ein
Bündel grüner Geldscheine heraus. Geschickt fächerte er sie auf und
wedelte damit vor Lillys Nase herum. »Also, an die Arbeit.«

\tb

Johann saß ungeduldig auf dem Barhocker und starrte die Tür an,
hinter der Eugene vor Minuten verschwunden war. Heiße Wut brodelte
in ihm und es fiel ihm schwer sitzen zu bleiben. Am liebsten hätte
er noch einen Whiskey getrunken, aber er wusste, dass er ihn nicht
vertragen würde. Außerdem wollte er nüchtern sein, jede Minute mit
Lilly genießen können, denn die Zeit mit ihr war teuer erkauft und
viel zu wenig. Doch das würde sich ändern. Er verachtete Eugene,
mit seiner schlechten Haut, seinem rüpelhaften Benehmen und seinem
überheblichen Chauvinismus. Doch er war der einzige Weg zu Lilly.
Und wenn Johann diesen Weg gehen musste, wollte er auch ihren
Beschützer ertragen.

In einer Ecke saß ein betrunkener alter Mann, der auf seinem Banjo
spielte und zahnlos versuchte, ein Volkslied zu singen. Die
Menschen um ihn herum ignorierten ihn genauso wie sie Johnny
ignorierten. Es war nicht ungewöhnlich, dass sich Herren der oberen
Decks unten heimlich amüsierten. Sie waren der Geldhahn, an dem
hier viele hingen. Eine andere Tür öffnete sich, und eine junge
Frau mit asiatischen Gesichtszügen verließ mit einem Tablett voller
Pfeifen den zweiten Hinterraum. Johnny hasste die Personen, die
sich dort hinein begaben. Er hatte gesehen, wie die Menschen sich
unter Opium veränderten. Seine Schwester war dem Rauschgift
verfallen gewesen - ein unschätzbarer Verlust für die Gemeinschaft.
Ihre Bildung war ausgezeichnet, man hatte sie gar als Multiplikator
auserwählt. Es war vorgesehen worden, dass sie als Mutter ihre
Quote übererfüllen sollte – und so die Stellung der Familie in der
Kolonie zu stärken. Doch sie hatte sich lieber umgebracht.

Der Kapitän hatte danach eine Razzia durchführen und die
Schlafdroge verbieten lassen. Fast eine halbe Tonne Rohmaterial
hatte man gefunden und der Maschine übergeben.

\bigpar

Johann fuhr sich mit der Hand über das Gesicht. Ihm würde das nicht
passieren.

\bigpar

Er litt unter der entsetzlichen Wärme. Sein Apartment auf dem
Kolonistendeck war viel kühler, ein Großteil der Energie, die die
Maschine ihnen spendete, kam dem Komfort und dem Leben auf dem
Oberdeck zu Gute. Johann fuhr sich mit dem Finger unter den hohen
Stehkragen, der durch die schwarze Fliege an seinen Hals gedrückt
wurde. Wenn er erst einmal bei Lilly wäre, würde es angenehmer
werden.

Heute würde er wagen, dachte er. Jetzt öffnete sich endlich die
Türe zu Lillys Raum und Eugene trat heraus. Johann stand auf. Sein
Magen fühlte sich flau an. Heute würde er es wagen. »Schnell, noch
einen Whiskey«, herrschte er den Barmann an und legte eine Münze
auf den Tresen. Bis Eugene bei ihm war, hatte er den billigen Fusel
hinunter gestürzt. Dann nahm er seinen Zylinder, setzte ihn auf und
griff nach seinem Spazierstock aus dunklem Holz, in dessen Griff
ein Elefantenkopf geschnitzt war. Als Johann seine Finger über den
Kopf gleiten lies, fühlte er sich wieder sicherer. Er war wie ein
Elefant, dachte er oft. Was er sich einmal in den Kopf gesetzt
hatte, würde er gegen alle Widerstände durchführen. Und er hatte
einen Plan.

Eugene streckte ihm die Hand entgegen, die Johann ergriff und ihm
dabei die Geldscheine übergab, die vereinbart waren.

\tb

»Liebst du mich, Johnny?« Lilly lag neben dem jungen Mann und
streichelte langsam seine nackte Brust. Dieser nahm eine von Lillys
billigen Zigaretten und steckte sie sich an.

»Und wie.« Er richtete sich auf und zog Lilly mit sich. Dabei
verrutschte das Laken, das sie bedeckt hatte, doch Johnny bemerkte
es nicht. »Komm mit mir. Wir werden einen Weg finden. Ich habe
Zukunft. Ich brauche eine Frau! Man erwartet mich bereits auf der
Kolonie. Als ich nach meiner Geburt ausgewählt wurde, habe ich
diesen Posten auf Lebenszeit erhalten. Es wird dir an nichts
mangeln.«

»Eugene wird mich nicht gehen lassen.« Lilly schaute wieder zu
Boden. In ihren Augen sammelten sich Tränen.

»Ich werde das regeln.« Aufgeregt sprang er auf. Nur mit seinen
langen Unterhosen bekleidet, schritt er energisch durch das Zimmer.
Er griff seinen Spazierstock und fuchtelte damit unbeholfen in der
Luft herum.

»Wie?«

»Meine Familie hat Geld. Und Einfluss.« Johann dachte daran, wie
sein Vater auf die Pläne reagiert hatte. Doch er verschob den
Gedanken, er würde es auch alleine schaffen.

»Das wird alles nichts nützen.« Lilly schüttelte den Kopf. »Er wird
mich nie ganz aufgeben. Er wird immer wieder kommen.«

»Wir könnten auf ein anderes Schiff umsiedeln. Die \textit{Ein langer Weg}
ist noch in Reichweite.«

»Er wird uns finden.«

»Verdammt«, fluchte er. »Wenn ich doch nur zur Polizei ausgesucht
worden wäre. Dann könnte ich etwas gegen ihn unternehmen. Aber so
\ldots{} wenn wir erst auf den Kolonien wären, würde ich über mehr Macht
verfügen. Vielleicht könnte Vater … er hat Beziehungen zum
Gouverneur …«

»Willst du deine Eltern wirklich darum bitten?«

Johann sank entmutigt auf einen Sessel. »Nein, das würde er nicht
verstehen. Sie machen mir jetzt schon genug Vorwürfe.«

»Hast du es ihnen erzählt.«

»Noch nicht ganz. Aber das werde ich tun müssen. Seit dem Tod
meiner Schwester setzen sie ihre ganzen Hoffnungen auf mich.«

»Ich muss frei sein, Johnny, bevor ich mit dir kommen kann.«

»Dann gibt es nur noch einen Weg …«

Sie nickte ihm zu. »Ja … genau.«

»Dann …« Er weigerte sich, es auszusprechen. Ihre Hand fuhr an
seinem Hals lang. Sie drückte ihre Brüste gegen seinen Arm.

»Dann …?«

»Dann werde ich Eugene töten müssen \ldots{}«

\subsection{Zwei}

\bigpar

Johann schritt gedankenverloren durch die Gänge des Kolonieschiffs
Kleine Hoffnung. Der Frachter war groß genug, um nicht zu vielen
Leuten über den Weg zu laufen, die einen kannten – und das, obwohl
er schon seit fast zehn Jahren auf dem Raumschiff lebte. Die Kleine
Hoffnung war ein hoffnungslos überfrachtetes Kolonieschiff. Die
sechs riesigen Dampfmaschinen, die tief im Inneren verbaut waren,
hatten damals genug Druck aufgebaut, um das Schiff, das im Orbit
der Erde zusammengesetzt worden war, in die richtige Richtung zu
bringen. Dabei hatten sie fast zwei Drittel der Kohle- und
Energievorräte verbraucht, mit der die Kleine Hoffnung gestartet
war.

Johann lehnte sich mit der Stirn an eins der dicken Fenster und sah
hinaus in die Dunkelheit. Die ferne Sonne schien immer noch
gleichmäßig auf das Schiff, das sich langsam um die eigene Achse
drehte. Durch die Rotation des Rumpfkörpers erzeugte die Maschine,
tief im Inneren, die Schwærkræft - die Macht, die alles
zusammenhielt.

Er sah den Ring aus Glaskugeln, die um das Schiff herum, einer
Perlenkette gleich, angebracht waren. Sie beinhalteten die großen
externen Gärten, die mit einer dünnen Versorgungsschleuse mit dem
Hauptschiff verbunden waren. Er wusste, weiter hinten, außer
Sichtweite, hing der letzte Kohlentender. Die ersten zwei waren
bereits kurz nach dem Start der Reise abgeworfen worden. Die
anderen waren dem Unglück zum Opfer gefallen, als sie in einen
Asteroidenschauer geraten waren.

Die Energiemenge war vor dem Start genau ausgerechnet worden. Den
benötigten Schub erhielt die Kleine Hoffnung durch die riesigen
Sonnensegel, die den Sonnenwind fingen und so das Schiff
vorantrieben. Die gedrosselten Maschinen hingegen sollten die
Menschen am Leben erhalten und jeglichen Komfort an Bord
sicherstellen. Das stampfende Geräusch der Kolben war der
Herzschlag des Schiffes. Sollte es eines Tages verstummen, würden
sie alle erfrieren.

»Die Raumfahrt steckt noch in den Kinderschuhen«, sagte sein Vater
immer wieder, wenn er die primitiven Maschinen sah, die das Leben
an Bord ermöglichten. Und doch war er stolz, bei diesem Projekt als
leitender Ingenieur an Bord sein zu können. Er setzte große
Hoffnungen in Johann, war ihm doch ein hervorragender Posten in der
Kolonialverwaltung sicher. Der Vater hoffte, dass sein Sohn eine
schnelle und zügige Karriere hinlegen würde. Als Beamter per Geburt
würde er auf Lebenszeit versorgt sein, das gleiche galt für seine
Frau und die Kinder bis zum Erwachsenenalter.

Aber was habe ich davon, dachte Johann, wenn ich nicht mit meiner
Liebe zusammen sein kann. Er dachte an Lilly und sein Herz wurde
ihm schwer. Wie sollte er erfüllen, was er ihr versprochen hatte?

Ein Duell, dachte er. Er drückte den Rücken durch und stellte sich
gerade hin. Das war eines Ehrenmannes wie ihm würdig. Jeder eine
Pistole, morgens um sechs, hier auf der Promenade, oder meinetwegen
auch unten auf dem Unterdeck. Oder in den Gärten, mit Blick auf die
gewaltigen Sonnensegel und die vorbeiziehenden Sterne. Jeder einen
Schuss. Johann glühte vor Aufregung, als er sich vorstellte, wie
Lilly ihm dankbar um den Hals fallen würde, voller Bewunderung für
seinen Heldenmut und sein großartiges Augenmaß. »Deine Hand hat
noch nicht einmal gezittert«, würde sie hauchen und ihm einen Kuss
geben. In der Kolonie würde sie seine Frau sein und ihm Kinder
schenken, die Macht der Familie stärken.

Wenn er nicht vorher verhaftet würde. Duelle waren, auch für
Kolonisten, seit den Unruhen im Jahre sechs der Reise verboten
worden und wurden mit dem Tode bestraft. Das gleiche galt für das
Tragen von Waffen.

\gedanke{Wie soll ich ihn denn töten, wenn ich keine Waffe nutzen darf?},
heulte es in Johann auf und er ging entmutigt die Promenade weiter
entlang. Als er von Ferne einen Freund erkannte, bog er schnell in
einen Seitengang ab, der ihn wieder hinunter brachte in die
Maschinenebene. Er hatte lange genug auf dem Schiff gelebt, um die
meisten Bereiche zu kennen.

Dann stand er vor der Tür, die zum Maschinendeck führte. Die
schwere Stahltür war gerade so groß, dass man gebückt
hindurchsteigen konnte. Große Nieten verstärkten die Seiten und in
der Mitte der Tür, direkt unter dem Schild »Zugang verboten«, ragte
das wuchtige Ventilrad aus dem Stahl heraus. Vorsichtig drehte
Johann daran. Sofort hörte er das zischende Geräusch als der Druck
entwich. Dann schienen mächtige Riegel durch neuen Dampfdruck zur
Seite geschoben zu werden, klackend griffen Zahnräder ineinander
und mit einem leisen »Pfff« bewegte sich die Tür.

Der Lärm überflutete ihn, als er die kleine Schleuse zum
Maschinendeck aufzog und hindurch schlüpfte. Kurz sah er sich um,
konnte aber keinen der Machinaisten sehen.

Das Maschinendeck war die Heimat der Machinaisten, selten sah man
sie auf den anderen Ebenen, außer zu offiziellen Anlässen, wie dem
Gottesdienst, den sie regelmäßig gaben.

Langsam und vorsichtig ging er über den schmalen Steg, der ihn
durch die lange Röhre führte. Die Luft war heiß und stickig. Das
wenige Licht schien von kleinen Lampen zu kommen, die in die Wände
eingelassen waren, doch womit sie gespeist wurden, konnte er nicht
erkennen. Überall schien sich etwas zu bewegen. Riesige Schrauben
wälzten sich durch eine träge Flüssigkeit, schoben sie durch das
komplexe Rohrsystem, das hier seinen Anfang nahm und sich wie
Blutgefäße durch den gesamten Schiffskörper zogen. Stampfend fuhren
Kolben auf und nieder, angetrieben durch den Dampf, der in den
Kesseln erzeugt wurde. Damit die ewigen Feuer der Kleine Hoffnung
nicht erloschen und das ganze Schiff in einen tödlichen Eispanzer
gehüllt wurde, benötigte es permanenten Nachschub mit
Brennmaterial. Johann sah von seinem Steg hinab in die Tiefe des
Maschinendecks. Unter ihm rauschte unablässig das Förderband, das
die Kohle aus dem Raumtender nach vorn zu den Maschinen brachte. Im
Moment war nur ein Förderband aktiv, doch sobald die Kleine
Hoffnung in die Nähe der Kolonie käme, würden alle Maschinen wieder
hochgefahren und die Förderbänder würden mit maximaler Leistung die
letzten Energiereserven transportieren. Doch soweit war es noch
nicht.

Johann spürte das Hämmern der Kolben tief in seinen Eingeweiden.
Hier war er gerne, wenn er nachdenken musste. Die rauch- und
ölgeschwängerte Luft benebelte sein Hirn und besänftigte seine
Gedanken.

Als er weiter ging, um zum nahegelegenen Aufstieg in die
Kolonistenebene zu gelangen, blieb er schließlich noch einmal
stehen. Er betrachtete die großen stählernen Walzen, die die
Kohlebrocken zertrümmerten und in kleine, gleichgroße Stücke
hackten. Riesige Hebel bewegten die Zahnräder, die die Rollen in
einer gleichmäßigen, unabdingbar brutalen Bewegung hielten. Nichts
und niemand konnte dieses Uhrwerk stoppen.

Alles was hier in die Räder gerät, wird zermalmt, dachte er. Wenn
hier ein Mensch hineinfallen würde …

\bigpar

»Im Namen der Maschine und der Schwærkræft. Was machen Sie denn
da?« Ein Machinaist hatte sich Johann genähert. Es war zu spät, um
einfach wegzulaufen, und auch die überhebliche Arroganz der oberen
Klasse, mit der Johann normalerweise auf Überraschungen zu
reagieren pflegte, war hier nicht angebracht. Die Machinaisten
waren die höchste Macht auf dem Schiff, selbst der Kapitän hatte
sich ihrem Urteil zu beugen. Sie waren mehr als reine Mechaniker.
Sie glaubten an die Maschine. Sie lebten für die Maschine. Das
Schiff, die Maschine, das war ihre Religion. Sie hielten
Gottesdienste ab, in denen der Maschine gehuldigt wurde. Und es
wurde allen Passagieren geraten, bei den Messen anwesend zu sein.
Sie waren die letzte Instanz, die Herren über Leben und Tod auf
diesem Schiff. Sie waren eine Sekte, eine fundamentalistische Abart
des überall aufkochenden Fortschrittsglaubens. Aber sie waren die
einzigen gewesen, die den Auftrag angenommen hatten, diese Ladung
voller Hoffnungsträger und Verzweifelter zu den Kolonien zu
befördern.

Johann nahm seinen schwarzen Hut vom Kopf, als ihn der Machinaist
strafend anblickte. Der Maschinenpriester trug einen einfachen,
ölverschmierten Overall, der jedoch eine Kapuze besaß, die die
Machinaisten zu Gottesdiensten und zur Rechtsprechung über das
Gesicht zogen.

»Ich erbitte den Segen und die Gnade der Maschine und der
Schwærkræft«, sagte Johann.

»Möge es dir gewährt werden, von der Maschine und der
Schwærkræft.«, beantwortete der Machinaist die übliche Grußformel.
»Also«, sagte er streng. »Was haben Sie hier zu suchen? Auf diesem
Deck besteht Lebensgefahr.« Sein roter Bart zitterte vor echter
Entrüstung.

Johann sah wieder auf, blickte hinunter auf die Walzen und schaute
dann dem Rotbärtigen wieder in die harten Augen. Ihm war eine Idee
gekommen.

»Das war mir bewusst, doch ich kam mit einem Anliegen, dass keinen
Aufschub duldet. Ich erbitte den Beistand und die Segen der
Maschine, denn ich möchte heiraten. Und zwar so schnell wie
möglich.«

\tb

»Das kann ich nicht! Das ist unmöglich, Johnny! Eugene hat doch
meine Papiere!« Lilly war ganz aufgeregt, als Johann ihr von seinem
Plan berichtete.

»Lilly, versteh doch, es gibt keinen anderen Weg. Ich dachte, du
liebst mich?«

Die junge Frau sah ihn entgeistert an, dann warf sie sich in seine
Arme. »Das tue ich doch auch. Aber ich habe solche Angst. Was
geschieht, wenn Eugene das herausfindet?«

»Er darf es nicht bemerken. Stiehl' die Papiere und wir können
heiraten.«

»Aber warum so schnell?«

»Ich will, dass du in Sicherheit bist. Für den Fall dass ich sterbe
oder erwischt werde – du als meine Frau wirst ein sicheres Leben
haben.«

»Wieso bringst du ihn nicht einfach um, Johnny? Erwürge ihn, wie
ein Mann. Oder schicke jemanden, der es für dich tut, du hast doch
Geld!«

»Ich werde niemanden töten! Ich kann es einfach nicht. Aber es wird
einen anderen Weg geben. Vielleicht einen Unfall. Doch zuerst
müssen wir heiraten. Vertraue mir.«

Lilly sah im tief in die Augen. Johann konnte nichts außer
bedingungsloser Liebe darin erkennen.

»Ich werde da sein, Johnny.«

\tb

»Du wurdest für diese Aufgabe ausgewählt, Johann.« Vor den dicken
Bleiglasscheiben schoben sich langsam und träge die Sterne dahin,
während der Vater seinen Monolog beendete. Es war angenehm kühl im
Salon, die Temperatur und die Luftfeuchtigkeit wurden durch ein
ausgeklügeltes System von Röhren und Lüftungsschächten in der Decke
und den Wänden auf einem konstanten Wert gehalten, damit die
Bücher, die sich an den Wänden in den hohen Regalen
aneinanderreihten, keinen Schaden nahmen. Hier hatte er einen
Großteil seines Lebens verbracht, hatte gelesen, gelernt und aus
dem Fenster gesehen. Alles, was er wusste, hatte er von seinem
Privatlehrer gelernt und aus den Büchern seines Vaters. Aus ihnen
wusste er, wie er sein Leben gestalten wollte und wie man sich als
Held zu benehmen hatte. So wie Phileas Fogg seine Aouda in Jules
Vernes »In achtzig Tagen um die Welt« vom Scheiterhaufen der
Brahmanen rettete und aus Indien mit ins zivilisierte London nahm,
so wollte er seine Lilly den Klauen Eugenes entreißen und entführen
in ein besseres Leben an seiner Seite in den Kolonien. Kein
Widerstand, keine Gefahr waren zu groß, um dieses edle Ziel zu
erreichen. Dass sich sein Vater um die Details kümmern müsste,
störte Johann nicht mehr. Denn auch Phileas Fogg wäre ohne seinen
Diener Passepartout wahrscheinlich noch nicht einmal über den
Ärmelkanal gekommen.

»Wir sind nur auf diesem Schiff, um dir diesen Weg zu ebnen«, fuhr
sein Vater fort. Sein weißer Backenbart betonte seine vom Whiskey
und dem Alter rot gefärbte Haut. »Du wirst eine wichtige
Persönlichkeit in der Kolonie werden.«

»Ich weiß. Und ich bin euch dankbar. Wirklich.«

»In der Kolonie wird dich ein gutes Leben erwarten.«

»Ich möchte es mit Lilly teilen.«

»Lilly. Was ist das für ein Name?!« Sein Vater schnaufte abfällig.
Doch Johann wusste, dass er am Ende nachgeben würde.

»Kann ich nicht selbst über mein Leben bestimmen?«, erwiderte er
trotzig, denn er hielt es für seine wichtigste Pflicht, seinen Plan
vor allem gegen seine Eltern durchzusetzen. Sie standen für all die
Zweifler, die es zu überwinden galt.

»Doch, das kannst du. Aber ich hätte mehr von dir erwartet.«

Johann blickte zu Boden. Der weiche Teppich im Salon des Vaters
dämpfte ihre Stimmen. Schwer hin der Zigarrenrauch in der Luft,
wirkte herbe neben dem süßlichen Aroma des Cognacs, dem der Vater
zusprach.

Johann hatte sich immer gewünscht, den folgenden Satz nie sagen zu
müssen. Aber Lilly war wichtiger. Er wollte sie haben, sie retten.
Das allein zählte.

»Wirst du mir helfen?«

»Was ist sie für eine Person?«

»Sie lebt auf dem Unterdeck. Doch sie hat einen tadellosen
Charakter.«

»Was hat sie dann mit Eugene zu schaffen?«

»Du kennst ihn?« Johann war wirklich überrascht. Obwohl es nicht
unüblich war, dass die Herren seiner Klasse sich auf dem Unterdeck
vergnügten, hätte er es von seinem Vater nicht erwartet.

»Vom Namen her. Ich bin leitender Ingenieur des Schiffes, ich muss
Bescheid wissen, was hier vor sich geht.«

»Also, hilfst du mir?« Es lastete schwer auf Johann, dass er einen
Menschen töten sollte. Doch wenn es einen Unfall geben würde \ldots{}
Eugene bräuchte niemals gefunden werden, die Walzen würden das
erledigen. Aber er würde das nicht selbst erledigen. Nach der
Hochzeit würde Eugene einfach verschwinden. Er hatte überlegt, mit
der Hochzeit zu warten, aber das hätte er nicht ausgehalten. Er
wollte Lilly und zwar sofort.

»Und du bist sicher, dass sie dich will und nicht dein Geld?«

»Sie ist schwanger.«

»Von dir?«

»Ja.« Johann hoffte, dass das die Wahrheit war. Aber er war bereit
es als solche zu akzeptieren, da es seiner Vision dienlich war.

»Ich werde sehen, was ich tun kann.«

Johann stand auf und nahm des Vaters Hände. »Danke!«, strahlte er
erleichtert.

\tb

Ein halbes Dutzend Machianisten hatten sich versammelt, um der
Zeremonie beizuwohnen. Der rothaarige Machinaist, den Johann um die
stille Hochzeit gebeten hatte, sang leise und im Rhythmus des
vorbeiziehenden Förderbandes. Er hatte seine Kapuze über den Kopf
gezogen und hielt den Kopf gesenkt. In seinen Händen ruhte eine
kleine Gaslaterne, die oft für die Reparaturen in den dunklen und
unbeleuchteten Gegenden des Schiffs benötigt wurde. Schließlich
verstummte er und sah Johann und Lilly an, die vor ihm knieten.

»Bei der Maschine und der Schwærkræft. So wie diese Flamme Licht in
die Dunkelheit der Maschine bringt, soll sie euch erleuchten und
eure Motive erhellen.« Er stellte die Lampe zwischen die beiden.
Dann zog er ein kleines Kästchen aus den Taschen seines Overalls.
Er öffnete es, und drehte an einem kleinen Schlüssel. Es knarrte
und knarzte, doch als das Uhrwerk aufgezogen war, begann das
Kästchen leise zu ticken.

»Als Zeichen, dass die Maschine immer bei Euch ist, dass sie Euch
am Leben erhält und dass sie für Euch sorgen wird, empfangt nun
dieses Uhrwerk. Geht in Euch. Ein heiliger Ablauf hat begonnen, den
niemand aufzuhalten vermag. Dreimal wird das Uhrwerk läuten, im
Abstand von einer Minute. Wenn die Maschine so ihr Einverständnis
gegeben hat und ihr dann immer noch beide nebeneinander in
Eintracht kniet, dann seid ihr Mann und Frau, für die Welt, das
Schiff und alle Kolonien, die unter den Sternen warten.«

Johann hörte auf das leise Ticken des Uhrwerks. Es war beruhigend
und aufregend zugleich. Drei Minuten, hundertachtzig Sekunden
trennten ihn noch von ihr. Sein Atem ging schneller, seine Brust
bebte. Mit jedem Herzschlag kam sie ihm näher, dachte er. Er
blickte auf und sah sie an. Sie lächelte. Sie trug wieder das
einfache braune Kleid, welches das einzige in ihrem Besitz zu sein
schien. Er konnte sehen, wie auch ihre Brust vor Aufregung bebte.

Über ihre Haare hatte sie einen einfachen, weißen Schleier gelegt,
der ihre Augen verdeckte. Johann konnte es nicht abwarten. So oft
hatte er schon bei ihr gelegen, sie berührt und geküsst, doch heute
sollte alles anders werden. Endlich würde sie ihm gehören. Um
Eugene würde er sich danach kümmern.

Ein leiser Gong ertönte. Johann blickte auf und sah das Kästchen
an. Die erste Minute war bereits verstrichen. Noch hundertzwanzig
Sekunden.

»Habt ihr geglaubt, ich lasse mich so einfach reinlegen? Verraten?
Da habt ihr Euch geschnitten.« Eugenes Stimme schnitt durch die
Stille des Maschinendecks.

Johann fuhr auf, Lilly kreischte. »Das wagst du nicht«, knurrte
Johann den Zuhälter an.

Doch der ignorierte ihn und zeigte mit dem Finger auf die Frau.
»Lilly, komm her. Sofort.«

Johann legte seine Hand auf ihre Schulter und drückte sie zurück
auf den Boden. »Nicht«, sagte er. »Noch zwei Minuten, dann kann er
uns nichts mehr.«

Doch Eugene lachte. »Du dummer Junge. Was denkst du dir denn? Sie
gehört mir, wird immer mir gehören. Sie mag zwar deine Frau sein,
aber ich werde dafür sorgen, dass du für sie blechen musst, wenn du
sie besteigen willst. Und nicht nur du. Sie ist mein bestes Pferd
im Stall, das lass ich mir von dir nicht wegnehmen.«

In Johann kochte der Zorn hoch. Der Gedanke, dass Lilly mit anderen
Männern \ldots{} dann wäre alles umsonst gewesen. Die heimliche
Hochzeit, die Erniedrigung durch den Vater. Ein zweiter Gong
ertönte und Johann zwang sich zur Ruhe. Noch eine Minute. Den Rest
würde sein Vater regeln. Doch Eugene schien seine Gedanken lesen zu
können.

»Du Träumer! Unselbstständiges Kind, das du bist. Dir wurde alles
in die Wiege gelegt. Andere müssen um ihr Leben kämpfen.« Er kam
näher, sein Grinsen wurde breiter. »Glaubst du wirklich, Dein Vater
könnte das regeln? Mich mundtot machen? Ich lasse mich nicht
aushebeln.« Das Uhrwerk tickte weiter, fünfundvierzig Sekunden
schätzte Johann. Ruhig bleiben, schalt er sich. Doch Eugene war
noch nicht zu Ende.

»Als deine Schwester im Opiumrausch in meinem Hinterzimmer starb,
habe ich dabei zugesehen. Danach bin ich zu deinem Vater gegangen
und habe die Schulden eingetrieben, die sie bei mir hatte. Ich
hatte ja ihren Leichnam als Pfand.« Eugene grinste. »Die Schande
wäre ihm doch zu groß gewesen. Was glaubst du nun, wie er sich
verhalten wird, wenn ich durchblicken lasse, welche Vergangenheit
die süße Lilly mit in die gute Familie gebracht hat? Trägt sie ein
Kind unter dem Herzen? Aber ist es auch deins, Johnny?«

Etwas setzte in Johann aus. sprang auf und stürmte auf Eugene zu.
Dieser erstarrte, sein Lächeln gefror zu Eis. Johann rammte ihm die
Schulter in den Magen, brüllte auf und schob ihn zum Rand der
Brücke. Unter ihnen toste das Förderband vorbei, schob unablässig
große und kleine Brocken Kohle in die malmenden Walzen. Eugene
ruderte mit den Armen, um sein Gleichgewicht zu halten. Dann, mit
einem ungläubigen Gesichtsausdruck, rutschte er langsam und fiel,
wie in Zeitlupe, nach hinten. Er fiel, schien einen Moment in der
Luft zu verharren, dann stützte hinunter auf den endlosen Strom aus
Kohle.

Johann sah nicht mehr, wie der Körper aufschlug, weil ihn ein
Machinaist mit einem Fluch auf den Boden warf. Lilly schrie auf.
Unter ihnen, außer Sicht, knirschte es hässlich weich.

Das Uhrwerk tickte weiter, unbeeindruckt von den Ereignissen.

»Wartet!«, rief Johann, als die hinzugerufenen Machinaisten ihn
wegzerren wollten. Mit aller ihm verbliebenen Kraft kämpfte er sich
frei, warf sich auf Lilly und klammerte sich an sie. Die Wachen
warfen sich wiederum auf Johnny und rissen ihn auf die Beine, zogen
seinen Kopf nach hinten und fesselten seine Hände auf den Rücken,
bevor sie ihn wieder zu Boden drückten. Johnny kniete neben Lilly.
In dem Moment erklang der Gong.

Der leitende Machinaist trat nach vorne und legte Johnny und Lilly
die Hände auf den Kopf, dann zeichnete er jedem mit Maschinenöl
einen kleinen Kreis auf die Stirn. »Im Namen der Maschine erkläre
ich Euch zu Mann und Frau. Ihr seid verheiratet.«

\bigpar

Johnny erschlaffte und gab jeden Widerstand auf.

\bigpar

\subsection{Drei}

\bigpar

»Man hat nur noch seinen rechten Arm gefunden.« Der Vater stand mit
gramerfülltem Gesicht in der Zelle, als die Machinaisten kamen, um
Johann zu holen. »Wieso hast du nicht gewartet?«

Johann schwieg, sah seinen Vater nicht an. Das letzte, was er in
seinen letzten Minuten vor der Vollstreckung des Urteils hören
wollte, waren Vorwürfe.

»Kümmert Euch um Lilly. Ich bitte Euch.«

Der Vater schwieg, und die Machinaisten zogen Johann an der
schweren Eisenkette, die seine Hände fesselten hinaus. Mord war das
schwerste Verbrechen auf der Kleine Hoffnung. Jeder Mensch wurde
gebraucht in den Kolonien. Und ein Menschenleben auszulöschen
bedeutete, die Gemeinschaft zu schwächen. Paradoxerweise wurde man
dafür mit dem Tode bestraft.

Es waren viele zur öffentlichen Hinrichtung gekommen. Der große
Saal unter der Sternenkuppel war gut gefüllt, als man Johann
hineinführte. Schlagartig wurde es still. Die Machinaisten stellten
Johann auf eine Luke im Boden. Er schluckte und schloss die Augen.
Er spürte die Maschine unter seinen Füßen rumoren. Jetzt stand er
direkt über der Brennkammer. Darin war das Feuer, das den Kessel
heizte. Das Feuer, das sie alle am Leben hielt in dieser brutalen
Dunkelheit des Universums. Johann blickte noch einmal nach oben,
sah durch die Glaskuppel die großen Segel, in denen sich der
Sonnenwind fing, um sie zu ihrem Ziel zu tragen. Die Kolonie, die
er nun nie erreichen würde. Aber Lilly würde. Schnell suchte er mit
seinem Blick ihr Gesicht in den Massen. Er fand sie in der ersten
Reihe, kummervoll stützte sie sich auf einen Mann.

»\ldots{} wirst du dem ewigen Kreislauf der Maschine überantwortet. Wir
übergeben deinen Körper der Maschine und der Schwærkræft.«

»Lilly!«, rief Johann verzweifelt. Seine Frau hob den Kopf und
lächelte ihn an. Das teure Spitzenkleid ließ ihre Figur plötzlich
wieder sehr jugendhaft aussehen. Sie wollte zu ihm kommen, doch die
Machinaisten hielten sie zurück. Johann wusste jetzt, dass es ihr
gut gehen würde. Die Witwenrente böte ihr die Voraussetzung für ein
hervorragendes Leben. Lilly lächelte ihm zu und Johann zwang sich
ebenfalls zu einem Lächeln. Er straffte sich, als der Machinaist
zum Hebel schritt, der die Falltüre öffnen würde.

Der Mann neben Lilly schaute nun auf. Er hob den Arm, oder was
davon übrig war, und winkte ihm mit dem Stumpf zu. Johann sah, wie
Eugenes Lippen grinsend ein Wort formten: »Danke!«

\bigpar

Bevor jemand etwas bemerkt hatte, war Eugene mit Lilly wieder in
der Menge verschwunden. Dann öffnete sich die Luke.

\end{document}

