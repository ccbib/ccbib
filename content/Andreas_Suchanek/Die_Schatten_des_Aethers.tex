\usepackage[ngerman]{babel}
\usepackage[T1]{fontenc}
\hyphenation{wa-rum}


%\setlength{\emergencystretch}{1ex}

\renewcommand*{\tb}{\begin{center}* \quad * \quad *\end{center}}

\newcommand\bigpar\medskip
\newcommand\gedanke\textit

\begin{document}
\raggedbottom
\begin{center}
\textbf{\huge\textsf{Die Schatten des Æthers}}

\bigskip
Andreas Suchanek
\end{center}

\bigskip

\begin{flushleft}
Dieser Text wurde erstmals veröffentlicht in:
\begin{center}
Die Steampunk-Chroniken\\
Band I -- Æthergarn
\end{center}

\bigskip

Der ganze Band steht unter einer
\href{http://creativecommons.org/licenses/by-nc-nd/2.0/de/}{Creative-Commons-Lizenz.} \\
(CC BY-NC-ND)

\bigskip

Spenden werden auf der
\href{http://steampunk-chroniken.de/download}{Downloadseite}
des Projekts gerne entgegen genommen.

\vfill

Es war der 21.\,März 1982, als die Stadt Landau (i.\,d. Pfalz) um
einen Mitbewohner reicher wurde. Überspringen wir die frühen und
relativ uninteressanten Jahre, und kommen direkt zum interessanten
Teil. Es war das Buch »Der Kristall der Macht«, aus der Professor
Zamorra-Reihe, das mich zum eifrigen Konsumenten von
Romanheftserien machte. Und nach kurzer Zeit stand für mich fest:
Ich will selbst etwas Gleichartiges schreiben. So erschuft ich die
Internetromanserie »Das Erbe der Macht« im jungen Alter von 16
Jahren (angereichert mit einer Vielzahl an Rechtschreib- und
Grammatikfehlern). Es folgten weiter Web-Projekte, Schreibübungen
und der Konsum von Fachliteratur. 2008 folgte ein Book On Demand,
bevor ich mich im Oktober 2010 erfolgreich für die
Bastei-Heftromanserie »Sternenfaust« bewerben konnte. Bisher habe
ich fünf Folgen beigesteuert, die sechste Story entsteht gerade.
Neben einem fertigen Fantasy-Manuskript, das ich aktuell
überarbeite, schreibe ich eifrig Geschichten für Story-Wettbewerbe
(wie hier zu lesen ist) und habe weitere Bewerbungen im Blick.

\bigpar

Vom Schreiben allein lässt es sich aber bisher aber noch schwer
leben. Hauptberuflich ging es von der Ausbildung zum »Technischen
Assistent für Informatik« direkt weiter zum Informatikstudium. Auf
das Diplom folgte der Master und aktuell arbeite ich am
Fraunhofer-Institut in Karlsruhe.
\end{flushleft}

\section{Die Schatten des Æthers}
Vorsichtig lugte Elisabeth um die Gangbiegung. Es war niemand zu
sehen, was jedoch nicht viel bedeuten mochte. Die Wahnsinnigen
waren überall, verbargen sich in den Schatten und schlugen
erbarmungslos zu. Erst vor wenigen Minuten hatte Oberleutnant
Ashbrooke seine Zähne in den Hals eines bedauernswerten Stewarts
geschlagen. Das Blut war über den glänzenden Boden der Kombüse
gespritzt, während Keneth Ashbrooke verzückt gelächelt hatte. Das
darauf einsetzende Schmatzen und Schlürfen würde sie wohl niemals
wieder vergessen können.

\bigpar

Mit einem Kopfschütteln vertrieb Elisabeth die grauenvolle
Erinnerung. Den Hand-Tesla erhoben schritt sie voran. Eine
Glühbirne flackerte und Schatten tanzten über die Wände. Das
Kommandodeck war in gespenstische Stille getaucht. Einzig das
Dröhnen der Dampfturbinen, die das Luxalin erhitzten, war
allgegenwärtig. Das Schott zur Brücke schälte sich aus dem
flackernden Licht. Die Konsolen in den Wänden waren von einem der
Wahnsinnigen zerschmettert worden. Schalter, Knöpfe und Hebel
kullerten über den stählernen Untergrund, getrieben durch das
Vibrieren des Decks. Die Gemälde der königlichen Familie lagen
aufgeschlitzt am Boden, von den Rahmen waren nur Bruchstücke
geblieben.

Ihre Hände zitterten, als Elisabeth ihren Zugangscode mittels der
Schieberegler eingab, dann zog sie einen Hebel nach unten. Mit
einem Rumpeln teilte sich das Schott in der Mitte und beide Hälften
fuhren quietschend in die Wand. Sofort richtete sie den Hand-Tesla
auf die einzige Person, die sich, wie erwartet, in dem Raum befand:
der Kapitän.

\bigpar

Alfred Windsor blickte aus geröteten Augen zu ihr auf. Sein linker
Arm hing schlaff am Körper herab, eine Blutlache hatte sich unter
seinem Sessel ausgebreitet. Sein väterliches Lächeln entgleiste zu
einer Grimasse. Mit schwerer Stimme sprach er: »Ah, Elisabeth. Es
wundert mich nicht, dass Sie als letzte stehen.«

»Sir«, hauchte sie. »Ich muss Ihnen leider mitteilen, dass auch
Oberleutnant Ashbrooke betroffen ist.« Es mutete wie ein Wunder an,
dass der Kapitän nicht ebenfalls zu einem Wahnsinnigen mutiert war.
Bedächtig schob Elisabeth den Hand-Tesla in das Etui an ihrem
Gürtel, bevor sie mit zwei schnellen Handgriffen das Brückenschott
schloss und verriegelte. "Was können wir tun?«

»Tun?«, echote Alfred Windsor. »Nichts. Die Puderdosen haben uns
hereingelegt.«

Elisabeth nickte schwer. Das war ihr längst klar. Trotzdem musste
es doch etwas geben, das sie von hier aus tun konnten. Die
Navigationskonsole war zertrümmert, die Beine von Leutnant
Schwarzmann schauten darunter hervor. Eine Flucht war unmöglich.
Die Waffen konnten nicht verwendet werden, da niemand mehr da war,
um die Torpedo-Schächte und die Flechette-Werfer zu befüllen.

\gedanke{Es gibt immer einen Ausweg, er muss nur gefunden werden}, hallte das
Credo ihres Vaters in ihrer Erinnerung wieder. Vermutlich war er
aber nie Lichtjahre von der Erde entfernt mit Wahnsinnigen auf
einem Sternenschiff eingesperrt gewesen.

\bigpar

»Ich grüße Sie«, hallte eine Stimme über die Brücke.

Der gewölbte Schwarz-Weiß-Bildschirm, der gegenüber dem
Kommandosessel an der Wand hing, erwachte zum Leben.
Interferenzstreifen waberten über das Bild. Die
Graham-Bell-Verbindung zu dem anderen Raumschiff war instabil.

Elisabeth kannte die alte Frau nicht, die ihnen entgegenblicke. Sie
mochte achtzig sein, vielleicht auch neunzig. Zwar lächelte sie,
doch ihre Augen blieben dabei unnatürlich kalt. Die Zähne waren
gelb und schief, das graue Haar fettig und zu einem altmodischen
Dutt hochgesteckt. »Sie wollen sich also widersetzen«, keifte die
Alte. Mit dem Zeigefinger ihrer rechten Hand, dessen Fingernagel
schwarz lackiert war, stach sie in Richtung des Aufnahmeobjektivs.
»Aber was können Sie schon ausrichten? Nichts.«

Tränen rannen aus den Augen von Alfred Windsor. Elisabeth hatte den
alten Bären noch nie weinen sehen. Es war ein ernüchternder Schock.
Er nuschelte etwas in seinen braunen Vollbart, das sie jedoch nicht
verstand.

Ein Lachen hallte über die Brücke. »Kommen Sie schon Kapitän, ihrem
wertvollen Schiff wird schon nichts geschehen. Ich erwarte Sie.«
Eine ölige Flüssigkeit floss aus dem Mund der Alten. Sprenkel davon
flogen auf das Aufnahmeobjektiv.

»Nein, ich will nicht«, wimmerte Alfred Windsor.

Die Alte antwortete nur mit einem weiteren hässlichen Lachen und
das schwarze Öl spritzte erneut. Ein Schauer rann über Elisabeths
Körper. Die Augen des Kapitäns suchten die ihren. Sie nickte.
Bedächtig zog sie den Hand-Tesla hervor. Sie nickte dem alten Bären
ein letztes Mal zu, dann hob sie wie versprochen die Handfeuerwaffe
und betätigte den Auslöser. Aus dem Dorn der Waffe schoss ein
blauer, sich verästelnder Strahl, der den Kapitän mitten in die
Brust traf. Aufseufzend sackte er in sich zusammen, ein Lächeln auf
dem Gesicht.

Ich habe gerade meinen vorgesetzten Offizier erschossen, wurde
Elisabeth klar. Sollte sie dieses Gemetzel irgendwie überleben,
würde sie sich dafür vor der Admiralität ihrer Majestät
verantworten müssen.

\bigpar

»Dummes Mädchen«, hallte es über die Brücke. »Er war mein.«

Elisabeth wandte sich dem Bildschirm zu. Konsolen waren in U-Form
darum herum angeordnet. Bedächtig trat sie neben den Stuhl des
toten Kapitäns, so dass sie der Aufnahmelinse frontal gegenüber
stand. »Du hast dir schon genug genommen«, spie sie der Vettel
entgegen.

Während das Lachen der Alten über die Brücke hallte, setzte sich
Elisabeth an die Navigationskonsole. Die \textit{Europa} konnte die Richtung
nicht mehr ändern, doch das Ziel lag direkt vor dem Bug des
Zeppelin-förmigen Schiffes. Grimmig lächelnd schob sie den Regler
für den Luxalin-Antrieb nach oben.

»Nimm deine Finger weg, dummes Gör«, keifte die Alte. »Das ist mein
Schiff.«

»Es war das Schiff eines stolzen Mannes«, gab Elisabeth kalt
zurück. »Es wird niemals deines sein.«

\bigpar

Die \textit{Europa} setzte sich nur langsam in Bewegung, dann beschleunigte
sie jedoch kontinuierlich. Gleich ist es vorbei. Elisabeth
schluckte. In einem letzten Aufbäumen sollte das Sternenschiff im
Feuer des Luxalin vergehen – und den Feind mit sich reißen. Auf dem
Bildschirm wurde das Ziel übergroß, war nicht mehr vollständig zu
erfassen. Elisabeth lächelte.

\minisec{Ein Jahr später}

»Mister Wittkamp! Schön, dass Sie uns die Ehre Ihrer Anwesenheit
zuteil werden lassen«, erklang die Stimme von Oberleutnant Larkin.
Mit einem Blick auf seine aufgeklappte Taschenuhr fügte er hinzu:
»Und lediglich fünf Minuten zu spät. Melden Sie sich nach Ihrer
Schicht beim Obermaat!«

»Aye, Sir«, bestätigte Alek und salutierte. Rory, dessen Platz an
der Ortungskonsole er nun übernahm, verließ mit beschwingtem Gang
die Brücke. Im Hinausgehen zwinkerte er Alek noch gehässig zu, dann
war der andere Junge verschwunden. Arroganter Marsianer!

Ein Blick auf die Ortung zeigte Alek, dass die \textit{Berlin} ihr Ziel fast
erreicht hatte. Endlich, nach fast einem Jahr, hatte ihre Majestät
der Entsendung eines weiteren Schiffes zugestimmt, dass das
seltsame Verschwinden der \textit{Europa} aufklären sollte.

»Bericht, Mister Wittkamp«, forderte der Oberleutnant.

»Bisher keine neuen Erkenntnisse, Sir«, meldete Alek. Er warf einen
scheuen Blick zur Mitte der Kommandobrücke. Kapitän von
Winterfelden thronte wie immer schweigsam in seinem Sessel, blickte
ab und an auf seinen Kommandobildschirm und zwirbelte sich
ansonsten seinen spitz zulaufenden Lippenbart. Oberleutnant Larkin
hatte neben ihm Aufstellung bezogen, den Rücken gerade
durchgestreckt, die schwarzen Haare akkurat gescheitelt, das Kinn
wie anklagend auf Alek gerichtet.

\gedanke{Ich bin für eine Laufbahn bei der Raummarine einfach nicht
geschaffen}, ging es ihm erneut durch den Kopf. Noch vor einem Jahr
hätte er es sich niemals träumen lassen, auch nur einen Fuß auf ein
Luxalin-Schiff zu setzen. Die großen Zeppelin-förmigen
Sternenschiffe, die von der Kraft jenes geheimnisvollen Minerals
angetrieben wurden, das erst vor wenigen Jahren entdeckt worden
war, hatten ihm stets eine Menge Respekt eingeflößt. In dem
riesigen, metallenen Ovoid über ihnen wurde das Luxalin über
haushohe Dampfturbinen erhitzt, bis es seinen Aggregatzustand
änderte und mit der ihm innewohnenden Kraft den
Tscherenkow-Generator speiste. Noch heute rann ihm ein Schauer über
den Rücken, wenn er sich an jenen Moment zurückerinnerte, als er
die Gondel zum ersten Mal betreten hatte.

\gedanke{Schlag dir das aus dem Kopf! Ich habe schon Elisabeth an diese
verdammten Schiffe verloren, da lass ich mir nicht auch noch mein
zweites Kind nehmen}, hallte die Stimme seiner Mutter aus der
Vergangenheit empor. Kind hatte sie ihn genannt. Wo er doch längst
siebzehn gewesen war. Ein Wunder, dass die Raummarine ihn überhaupt
angenommen hatte, traten die meisten Kadetten doch bereits mit
fünfzehn bei. Ein Jahr hatte die Grundausbildung gedauert, doch
jetzt versah er seinen Dienst auf der \textit{Berlin}, die sich auf zum Rand
des Sonnensystems und darüber hinaus gemacht hatte. Dort war sie
verschwunden, seine Schwester. Oberleutnant Larkin mochte ihn noch
so oft Decks, Bildschirme oder Dampfturbinen schrubben lassen – er
würde Elisabeth mit nach Hause bringen.

Ein Blinken auf seinem Monitor brachte ihn zurück in die Realität.
»Sir, die Ortung hat ein Objekt entdeckt«, meldete er. Auf dem nach
außen gewölbten Schwarz-Weiß-Bildschirm wurden schematische Umrisse
sichtbar. Ein metallener Federkiel, von einem eisernen Arm
gehalten, flog automatisiert über ein Stück Pergament und übertrug
die aufgefangenen Messwerte. Oberleutnant Larkin stapfte an den
Sensortisch, zog das Blatt unter der Feder hervor und überflog die
Ausgabe. »Stahl, Kupfer, ein wenig Chrom. Dazu ein geringer Anteil
an Luxalin. Die übrigen Werte sind vernachlässigbar.«

»Die \textit{Europa}«, hauchte Alek.

»Mister Wittkamp«, fauchte der Oberleutnant. »Unterlassen sie
derlei Mutmaßungen. Wir werden bald nahe genug sein, um uns ein
exaktes Bild zu machen. Vorher will ich so etwas nicht noch einmal
hören.«

»Aye, Sir«, stieß Alek zwischen zusammengebissenen Zähnen hervor,
was ihm einen weiteren grimmigen Blick des Ersten Offiziers
einbrachte.

Der Kapitän griff nach der Papierausgabe und überflog die
Niederschrift der Sensorfeder ebenfalls. »Wie lange noch, Miss
Syriosa?«

Die hochgewachsene Marsianerin mit den rotblonden Haaren drehte an
einem der bronzenen Schieberegler, dann erwiderte sie:
»Sechsundfünfzig Minuten bis zur Ankunft, Sir.« Alek mochte
Marsianer nicht. Rory war Marsianer.

Zwirbelbart – wie Alek den Kapitän stets nannte – nahm wieder
Platz, ohne ein weiteres Wort zu verlieren. Die anderen Fähnriche
behaupteten, dass der Kapitän meistens schwieg, weil er ein weiser
Mann war. Alek hielt ihn einfach nur für faul.

Die darauffolgenden Minuten schienen sich ins Unendliche zu dehnen.
Immer wieder starrte Alek auf die Anzeigen der Sensoren. Im
Normalfall wäre er längst von Leutnant Parcel abgelöst worden, doch
der Ortungsoffizier lag auf der Krankenstation des Schiffes. Eine
der Luxalin-Dichtungen hatte durch einen Haarriss einen Teil der
Strahlung in den Maschinenraum entlassen. Neben dem leitenden
Ingenieur und einem Großteil der Techniker hatte es auch Leutnant
Parcel erwischt. Es würden wohl noch einige Tage vergehen, bis der
Doktor die Strahlenvergiftung vollständig behandelt hatte.

Als sie bis auf zehntausend Kilometer an das Objekt heran waren,
begannen die Bugkameras zu arbeiten. Zwirbelbart und Oberleutnant
Larkin waren hinter ihn getreten, die übrigen Offiziere warfen
neugierige Blicke zum Ortungsmonitor.

»Es handelt sich keinesfalls um die \textit{Europa}«, stellte der Kapitän in
seiner unnachahmlichen Weisheit fest, als er ein weiteres Blatt von
der Feder des Sensorschreibers entgegen nahm. Alek fühlte eine
bleierne Schwere, die sich auf sein Gemüt herabsenkte. Dann,
endlich, hatte sich die Kamera justiert und auf dem Bildschirm
erschien das Objekt.

»Beim heiligen Tesla, was ist das?«, stieß der Oberleutnant
hervor.

Alek konnte nur mit aufgerissenen Augen auf das Ding starren, das
sich auf dem Monitor abzeichnete. Ähnlich wie die \textit{Berlin} sah es
aus, mit dem riesigen, metallenen Zeppelin und der darunter
angebrachten Gondel, die der Mannschaft Platz bot. Doch damit
endeten die Gemeinsamkeiten auch schon. Eine ölige Flüssigkeit
bewegte sich über die Außenhaut des Schiffes, bildete seltsam
anmutende Formen und schien nie stillzustehen. Teile der Gondel
waren zertrümmert, chitinartige Klauen ragten aus den Rissen in der
Außenhaut hervor. Zudem war das Schiff mindestens drei Mal so groß
wie die \textit{Berlin}.

»Mister Takana, stellen sie ein Außenteam zusammen, wir setzen
über«, befahl der Kapitän.

Der asiatische Waffenoffizier erhob sich und nickte Alek zu – als
Fähnrich war er natürlich dabei, sollte Erfahrung im Außeneinsatz
sammeln –, ebenso wie Leutnant Syriosa.

Der Kapitän übergab das Kommando an Oberleutnant Larkin, dann
folgte auch er. Alek war sich sicher den Antworten auf die Frage
nach dem Schicksal seiner Schwester noch nie so nahe gewesen zu
sein.

\tb

Die beiden Konstabler bildeten die Spitze der Gruppe, als das
Schott am Ende der Andockröhre sich langsam öffnete. Alek hatte die
Kurbel gedreht und wischte sich nun keuchend den Schweiß von der
Stirn. Die Hand-Teslas im Anschlag taten sie einen Schritt nach
vorne. Der Linke der beiden – Mister Schell – sog scharf die Luft
ein, während der Rechte – Mister Leclerc – sich abrupt übergab.
Bevor Alek sich fragen konnte, was mit den beiden sonst so
disziplinierten Konstablern los war, drang der Gestank auch an
seine Nase. Eine schwere Süße, gepaart mit einem Hauch von Eisen
und \ldots{} er konnte es nicht benennen. Auch ihm wurde übel.

Auf ein Nicken des Kapitäns setzte sich Leutnant Syriosa ebenso wie
Leutnant Takana in Bewegung.

Der asiatische Chef der Sicherheit hatte den Chrom-Arm
übergestreift, der zwei Hand-Teslas sowie eine Leuchtkugel
beherbergte und obendrein durch ein Servo-Gelenk die Kraft des
normalen Arms erhöhte.

Die übrigen Offiziere beachteten Alek nicht, als sie nun gemeinsam
das fremde Schiff betraten.

Die Wände bestanden aus dunklem Metall, das an mehreren Stellen von
einer ölig stinkenden Flüssigkeit bedeckt war. Die Gemälde waren
nur noch nasse Fetzen, die Holzvertäfelung war feucht und
schimmelig, die Glühbirnen an der Decke zersplittert.

Als das Schott sich hinter ihnen schloss, zuckte Alek zusammen.

»Alles in Ordnung, Mister Wittkamp?« Die Stimme von Leutnant Takana
klang sanft, fast verständnisvoll.

Alek nickte und brachte sogar ein verkrampftes Lächeln zustande.

Aufgrund der hohen Luftfeuchtigkeit begann er stark zu schwitzen,
genau wie die übrigen Offiziere - ausgenommen der Marsianerin.

Leutnant Takana hatte die in seinem Chrom-Arm eingelassene
Leuchtkugel aktiviert. Ein sanfter elektrischer Impuls hatte die
Glühwürmchen in ihrem Inneren dazu angeregt, ihr grün
fluoreszierendes Sekret abzusondern. Die übrigen Offiziere hatten
die Lichtgloben an ihren Uniformgürteln aktiviert, Alek tat es
ihnen mit etwas Verspätung gleich.

»Mister Takana, Sie begeben sich in den Maschinenraum«, befahl der
Kapitän. »Überprüfen Sie den Luxalin-Antrieb und den
Tscheren\-kow-Generator. Mit etwas Glück können wir dieses seltsame
Schiff bergen und mit zur Erde nehmen. Mister Sandsworth, Sie sind
für die Sicherheit der beiden verantwortlich.«

Takana nickte. Er warf Alek einen letzten Blick zu, dann verschwand
er mit seinen Begleitern im Dunkeln. Nun waren sie also nur noch zu
viert. Der Kapitän, die verdammte Marsianerin, der zweite
Konstabler und er.

»Wir begeben uns zur Brücke«, erklärte der Kapitän. Mit forschen
Schritten setzte er sich an die Spitze, was natürlich jedem
Sicherheitsprotokoll widersprach. Die Marsianerin schob Alek in die
Mitte zwischen sich und Mister Leclerc. Im fahlen Licht der
Leuchtkugeln stapften sie durch das tote Schiff. Als hätten
Vandalen in den Gängen getobt fanden sie kaum noch funktionsfähige
Schaltpulte oder Einrichtungsgegenstände. Kissen waren
aufgeschlitzt, Wände mit eingetrocknetem Blut besprenkelt, die
Teppiche zerfetzt.

»Es hat deutliche Merkmale eines Schiffes des Empires«, stellte der
Kapitän fest.

Aber es ist nicht die \textit{Europa}, kam es Alek in den Sinn. Dafür wirkt
es viel zu alt.

Gemeinsam erreichten sie nach kurzer Zeit das Schott zur Brücke.
Alek half dem Kapitän die Kurbel zu drehen, worauf die Hälften mit
einem Quietschen auseinander fuhren. Der Raum war leer, die
Konsolen tot. Ein weiteres Schott führte zur Kajüte des Kapitäns.

Im Gegensatz zum übrigen Schiff war es hier klinisch sauber. Der
schwere Holzschreibtisch war blank poliert, die Glühbirne in der
Decke erhellte den Raum bei ihrem Eintritt und sogar die Schlafkoje
schien erst vor kurzem frisch bezogen worden zu sein. Auf dem Tisch
stand eine dampfende Tasse heißen Tees. Daneben lag ein
aufgeschlagener Foliant.

»Was ist das nur für ein verdammtes Geisterschiff«, fluchte der
Kapitän.

Alek sah sich aufmerksam um, doch es war niemand zu sehen.

In dem langen Raum gab es etliche Ecken und Winkel, wuchtige
Möbelstücke und dicke Vorhänge. Doch nirgends konnte sich jemand
ernstlich verstecken.

Der Kapitän trat an den Tisch heran. Aufgeschlagen lag dort das
Logbuch. Mit krakeliger Handschrift stand auf dem vergilbten
Pergament etwas geschrieben, mit Permanent-Tinte für die Ewigkeit
konserviert. Kapitän von Winterfelden ließ seinen Blick über die
Zeilen wandern, dann stöhnte er auf.

»Es ist die \textit{Einheit}«, hauchte er.

\bigpar

Eiseskälte kroch über Aleks Rücken. Das erste Schiff, das die Erde
vor drei Jahren verlassen hatte, um ein benachbartes Sonnensystem
anzusteuern. Ein experimenteller Luxalin-Antrieb sollte die
Reisedauer auf zwei Monate verkürzen. Man hatte nie wieder etwas
von der \textit{Einheit} gehört.

»Nun schauen Sie doch nicht so«, erklang eine Stimme aus Richtung
des offenen Schotts. »Mein Schiff hat sich doch gar nicht so
schlecht gehalten.«

\tb

Auf einen Wink der alten Frau ließ Leclerc seinen Hand-Tesla sinken
und stakste – einer Marionette gleich – aus dem Raum.

»Sie haben verdammt lange gebraucht«, stellte die Alte in
gespielter Liebenswürdigkeit fest. »Da hat die gute Mary wohl sehr
mit sich gerungen, oh ja.«

Alek wagte es nicht, auch nur einen Ton von sich zu geben. Das
graue Haar der Frau war fettig, die Zähne eitergelb. Von ihren
Mundwinkeln tropfte das schwarze Öl.

"Mister Leclerc!", brüllte der Kapitän.

"Geben Sie sich keine Mühe, er kann sie nicht hören", erklärte die
Alte. Leicht nach vorne gebeugt stakste sie auf den Sessel zu.
Kapitän von Winterfelden sprang förmlich zur Seite. Aufseufzend
ließ sich die Alte hinein sinken, nahm die Tasse auf und schlürfte
genüsslich den Tee.

»Wer sind Sie?«, fragte Leutnant Syriosa mit emotionsloser Stimme.

»Marsianer waren mir schon immer am liebsten«, entgegnete die Alte.
»Sie kommen immer direkt zum Punkt, ohne Schnickschnack.
Andererseits sind sie auch nicht beeinflussbar.« Auf einen Wink
ihrer rechten Hand hin erwachten die Schatten der Kajüte zu
unheimlichen Leben. Metallene Spinnen, mit kleinen Luxalin-Kugeln
auf ihrem Rücken, sprangen daraus hervor. Die mechanischen
Kreaturen waren ebenfalls von der öligen Flüssigkeit umfangen, was
ihre Wendigkeit jedoch nicht beeinträchtigte. Es waren insgesamt
fünf die über die Marsianerin herfielen. Aus Düsen an ihren Beinen
versprühten die metallenen Bestien das schwarze Öl auf Leutnant
Syriosa. Ihre Haut begann Blasen zu werfen, das hübsche Gesicht
verformte sich, als wäre es nicht mehr als einfache Knetmasse.
Seltsamerweise spritzte kein Blut hervor, ihr Körper fiel einfach
in sich zusammen. Alek würde ihre Schreie niemals vergessen. Alles,
was blieb, war ein schwarzer Fleck auf dem Teppich der Kajüte.

»Das war köstlich«, sinnierte die Alte mit genussvoll geschlossenen
Augen. »Und gleichzeitig ein Problem weniger. Kommen wir also zu
Ihnen beiden.« Sie nahm einen weiteren Schluck aus der Tasse. »Was
mache ich nur mit euch?«

Der Kapitän bekreuzigte sich. Alek musst an sich halten, um nicht
zu würgen. Leutnant Syriosa war einfach tot.

»Sie verdammtes Monstrum!«, brüllte der Kapitän. Keuchend blickte
er die Alte an, dann fügte er entsetzt hinzu: »Beim großen Tesla,
Sie sind Kapitän Hollow. Sie befehligten die \textit{Einheit}. Aber warum
sind Sie so \ldots{}«

»\ldots{} alt? Oh mein lieber Kapitän, was Sie hier vor sich sehen, ist
zweifellos der Körper von Kapitän Hollow. Ihr Geist jedoch ist von
uns gegangen.«

\bigpar

Alek erinnerte sich an das hübsche Gesicht von Kapitän Hollow, als
diese vor drei Jahren mit der \textit{Einheit} aufgebrochen war. Sie war
zweiunddreißig gewesen, das Gesicht faltenfrei und gepflegt, die
schwarzen Haare zu einem Pferdeschwanz gebunden. Die Queen
persönlich hatte der Offizierin und ihrer Crew den Segen gewährt.

»Was sind Sie?«, hauchte Kapitän von Winterfelden.

»Ich komme von jenseits der \textit{Schwarzen Tore}.« Bei diesen Worten
lächelte die Alte verzückt. »Dort ist es immer dunkel und feucht,
unsere Essenz kann gedeihen. Bevor ich von eurem Schiff aus meinem
Frieden gerissen wurde, wusste ich nicht einmal, was Hunger
bedeutet.«

»Hunger?« Die Frage des Kapitäns war kaum mehr als ein Krächzen.

»Nach der menschlichen Essenz, mein lieber Kapitän.« Die Alte
lachte boshaft, wodurch ein Schwall der öligen Flüssigkeit aus
ihrem Mund hervor spritzte. »Eure Welten sind so nah. Aber noch bin
ich an den Pakt gebunden – solange sie mir Essenzen schickt.«

\bigpar

Eine dunkle Vorahnung wallte in Alek empor und er konnte den Blick
von dem Gesicht der Alten nicht mehr abwenden.

»Was für ein Pakt? Welch törichtes Wesen würde mit einer solchen
Kreatur wie Sie es sind einen Pakt eingehen?« Der Kapitän trat
einen Schritt zurück.

»Queen Mary natürlich«, gab die Alte, von einem krächzenden Lachen
untermalt, zurück. »Sie schickt mir Schiffe mit Essenzen. So
wunderbaren, kraftvollen Essenzen, wie ihr es seid. Im Gegenzug
versichere ich, ihr Hoheitsgebiet nicht zu betreten und sende ihr
was sie benötigt, um die Essenzen zu mir zu bringen – Luxalin.«

Alek keuchte auf, fassungslos über das Gehörte. Er konnte und
wollte diesem abscheulichen Geschöpf nicht glauben, dass die
Königin des Empires zu einem so grausamen Handel bereit gewesen
war.

\bigpar

Die Alte schnippte mit dem Finger der rechten Hand, worauf der
Schwarz-Weiß-Bildschirm auf dem Schreibtisch zum Leben erwachte.
Der Maschinenraum war darauf zu sehen. Ölige Flecken hatten sich
auf dem Boden ausgebreitet. Alek erkannte eine Chrom-Prothese in
einem von ihnen, einen Hand-Tesla in einer anderen.

\bigpar

»Monstrum!«, schrie der Kapitän erneut und warf sich auf die Alte.

\bigpar

Die Spinnenautomaten sprangen auf ihn zu, ölige Flüssigkeit wurde
versprüht, das Lachen der Alten erklang. Als der Kapitän zu
schmelzen begann, warf Alek sich herum und rannte davon.

\tb

Als Alek aus seiner Bewusstlosigkeit erwachte, fuhr er keuchend in
die Höhe. Die Roboter hatten ihn erreicht, er war gestrauchelt,
dann hatte ihn eines der Ungetüme mit seinem Spinnenbein am Kopf
getroffen.

Seine Hände befühlten weichen Stoff. Er lag in einer Koje und
warmes Licht tauchte den Raum in eine heimelige Atmosphäre. Als er
den Kopf wandte, blickte er in das Gesicht einer grinsenden
Elisabeth.

\bigpar

»Elisabeth«, konnte er nur krächzen. Sein Mund war so trocken wie
Marsstaub.

»Hallo du Langschläfer«, erwiderte seine Schwester. »Hast uns einen
ganz schönen Schrecken eingejagt.«

»Wo bin ich? Wir müssen den Oberleutnant warnen!«

»Beruhig dich erst mal.« Sanft drückte sie ihn zurück in das
Kissen. »Die \textit{Berlin} ist längst wieder auf dem Weg zur Erde.
Oberleutnant Larkin hat einen Alarmstart befohlen, nachdem die
Konstabler dich von der \textit{Einheit} geholt hatten. Das war vor zwei
Tagen.«

Die Bilder des Kapitäns kamen ihm wieder in den Sinn. Die Augen
weit aufgerissen, das Gesicht nicht mehr als eine breiige Masse.

»Die anderen?«

»Niemand hat überlebt. Die Luxalin-Torpedos haben ein Trümmerfeld
aus der \textit{Einheit} gemacht«, hauchte Elisabeth. Nach einigen
Augenblicken fügte sie hinzu: »Ich weiß, wie du dich fühlst. Mir
ist es nicht anders ergangen. Ich bin die einzige Überlebende der
\textit{Europa}. Diese widerlichen Spinnenautomaten haben fast die gesamte
Besatzung weggeschmolzen. Und die übrigen wurden wahnsinnig. Ich
habe die \textit{Einheit} am Ende mit der \textit{Europa} gerammt, konnte mich kurz
zuvor noch mit einem Beiboot absetzen und habe mich im Hangar
dieses Monsterschiffes versteckt.«

»Und dann hast du mich gerettet.« Alek hustete. »Eigentlich hätte
es umgekehrt sein sollen.«

»Soweit kommt es noch«, widersprach Elisabeth. »Vom kleinen Bruder
gerettet. Ich hätte mich nie wieder zuhause blicken lassen können.
Das hat schon alles seine Richtigkeit so.« Sie zwinkerte.

Alek schluckte, als ihm die Worte der Alten wieder in den Sinn
kamen. »Die Queen hat \ldots{}«

»Ja, ich weiß«, unterbrach ihn Elisabeth. »Die Queen und ihr
Hofstaat – diese verdammten Puderdosen – haben uns verkauft. Schlaf
jetzt. Oberleutnant Larkin wird dich bald befragen. Du wirst all
deine Kraft benötigen.«

Alek merkte wie erschöpfte er war. Elisabeth lächelte ihm zu. Alles
war in Ordnung. Für einige Augenblicke spielte ihm das Licht im
Raum einen Streich und er glaubte, in den hellblauen Augen seiner
Schwester ölige Tupfen zu erkennen. Und war ihr Lächeln nicht eine
Spur zu … künstlich? Er rieb sich über die Augen. Als er sie wieder
öffnete, war alles wie zuvor. Es gab keine öligen Tupfer. Alek
entspannte sich. Sein letzter wacher Gedanke war: Bald sind wir
wieder zuhause. Das Lächeln von Elisabeth verfolgte ihn in seine
Träume …
\end{document}

