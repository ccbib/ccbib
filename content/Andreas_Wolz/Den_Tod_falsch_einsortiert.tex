\usepackage[ngerman]{babel}
\usepackage[T1]{fontenc}
\hyphenation{wa-rum Fracht-raum}
\hyphenation{schien}
\hyphenation{Tief-ebe-ne Tief-ebe-ne gro-ßen}


%\setlength{\emergencystretch}{1ex}

\renewcommand*{\tb}{\begin{center}* \quad * \quad *\end{center}}

\newcommand\bigpar\medskip

\begin{document}
\raggedbottom
\begin{center}
\textbf{\huge\textsf{Den Tod falsch einsortiert}}

\bigskip

Andreas Wolz
\end{center}

\bigskip

\begin{flushleft}
Dieser Text wurde erstmals veröffentlicht in:
\begin{center}
Die Steampunk-Chroniken\\
Band I -- Æthergarn
\end{center}

\bigskip

Der ganze Band steht unter einer
\href{http://creativecommons.org/licenses/by-nc-nd/2.0/de/}{Creative-Commons-Lizenz.} \\
(CC BY-NC-ND)

\bigskip

Spenden werden auf der
\href{http://steampunk-chroniken.de/download}{Downloadseite}
des Projekts gerne entgegen genommen.

\vfill

Andreas Wolz wurde in Wilhelmshaven an der Nordseeküste geboren und
wuchs in Bayern auf, wo er noch heute mit seiner Familie lebt. Zum
Schreiben fand er sowohl privat als auch beruflich auf Umwegen.
Nach einer Bankkaufmannslehre landete er in der Marketing-Abteilung
einer großen deutschen Direktbank. Dort ist er Redaktionsmitglied
der Mitarbeiter-Zeitschrift. Und seit er als Jugendlicher bei einem
»Star Trek«-Story-Wettbewerb den ersten Preis gewann, ließ ihn das
Schreiben auch privat nicht mehr los.

In seine Geschichten lässt er gerne Science-Fiction-Elemente
einfließen und auch eine gehörige Portion Humor darf meist nicht
fehlen. Für die Teilnahme an den Steampunk-Chroniken hat er die
Arbeit an seinem aktuellen Projekt unterbrochen, einem Jugend-roman
mit SF-Touch.
\end{flushleft}

\section{Den Tod falsch einsortiert}

In weniger als 26 Stunden sind Sie und alle hier an Bord tot‹,
sagte der Kerl zu meiner Frau, als ihm der weiße Schaum aus dem
Mund quoll und er in ihren Armen starb.«

\bigpar

Salmon Wincover lehnte sich in dem bequemen Ledersessel zurück und
spürte das sanfte Vibrieren der Motoren des Ætherschiffes \textit{Halina}.
Manche Menschen störte es, doch auf ihn wirkte es beruhigend zu
spüren, dass die Zeppelinmotoren gleichmäßig arbeiteten. Er nippte
an dem Portwein und ließ die herb-fruchtige Flüssigkeit einen
Moment auf seiner Zunge wirken. Der Rote Salon der \textit{Halina II}, in
dem er gerade saß, befand sich direkt neben dem Präsentationssaal.
Dort hatte bis vor einer Viertelstunde Edgar Statson einen Vortrag
gehalten. Der Edgar Statson, der nun ihm gegenüber saß, und mit
jenem Satz über seine Frau damit begonnen hatte, eine sehr
spannende Geschichte zu erzählen. Und Salmon Wincover liebte den
Nervenkitzel.

Edgar Statson gehörte offensichtlich zu den Menschen, die einfach
nie aufhören konnten zu reden. Die immer Zuhörer brauchten. Er
hatte eben erst seinen Vortrag über »Integrale Ordnung in
strukturierten Zettelsammlungen durch die Unterstützung von
Suchanfragen mittels feinmotorischer Räderwerke« gehalten. Statson
hatte es verstanden, dieses trocken anmutende Thema zu einer
detektivischen Geschichte zu verwandeln. Als sich eine kleine
Gruppe seines Publikums anschließend in den Salon begab, war er ihr
gefolgt und hatte Ausschau gehalten. Ausschau nach weiteren
Zuhörern. Wincover kannte diese Sorte Menschen. Ohne Publikum
fielen sie in sich zusammen, waren sie nichts. Sie waren süchtig
danach, Menschen mit ihren Erzählungen zu manipulieren. Statson
dachte wohl, Wincover sei ein dankbares Opfer. Wenn er sich da nur
nicht täuschte.

»Wie reagierte Ihre Frau auf diese Drohung?«, fragte Wincover und
sah Statson neugierig an. Das Jackett sorgfältig zugeknöpft, Stoff
und Schnitt nicht sonderlich modisch. Aber Statsons sicherer
Geschmack bei der Wahl des Portweins war Wincover wichtiger.

Statson lächelte. Wincover meinte eine gewisse Bitterkeit in dem
Lächeln zu erkennen, aber er konnte sich auch täuschen. »Der Kerl
hatte vorgesorgt und Ann in eine Falle gelockt. Eine Woche zuvor
hatte er sie schon einmal auf dieselbe Weise bedroht. Mitten auf
halber Strecke des viermonatigen Linienflugs zum Saturnring-Hotel.
Meine Frau informierte sofort den Kapitän. Da sie die
Bordingenieurin war, glaubte man ihr sofort und leitete eine
aufwendige Suche ein. Am Ende fand man \ldots{} einen Koffer voller
kleiner Buddha-Figuren. Der Kapitän machte eine Bemerkung über die
fehlende Menschenkenntnis meiner Frau und ging zur Tagesordnung
über. Sie wusste, beim zweiten Mal hätte man ihr nicht mehr
geglaubt.«

\bigpar

Wincovers Blick wurde von einer Frau an der Theke angezogen. Mit
ihrer Geschäftskleidung, ihrem sicheren Auftreten und den streng
nach hinten gekämmten Haaren entsprach sie dem Bild, das er sich
von Statsons Frau Ann gemacht hatte. Er hätte sich nicht gewundert,
wenn sie auf sie zugekommen wäre, Statson von hinten umarmt und mit
einem gehauchten »Schatz« auf den Nacken geküsst hätte. Die
Weiblichkeit, die viele kühle Geschäftsfrauen ausstrahlten, hatte
schon immer eine Faszination auf Wincover ausgeübt.

»Was machte Ihre Frau denn so sicher, dass der Kerl dieses Mal eine
echte Bombe versteckt hatte?«

Statson nippte an seinem Port. Eine schwer verständliche Durchsage
ertönte über den Lautsprecher, während der Barkeeper der
Geschäftsfrau einen Cocktail hinstellte. Sie nahm ihn, ignorierte
den anzüglichen Blick des Barmannes und setzte sich in den hinteren
Bereich des Salons.

»Der Kerl hatte ihr gesagt, dass es von Anfang an sein Plan gewesen
war, ihre Glaubwürdigkeit zu zerstören. Damit sie sich allein auf
die Suche nach der Bombe machen musste. Er nannte ihr auch den
Grund: Er hatte eine Rechnung mit mir offen. Er wollte, dass der
Erfolg ihrer Suche auch von mir abhing – und der Misserfolg mich
mitschuldig an ihrem Tod machen würde. Alles hing von unserer
Verbindung über Ætherfunk ab.«

»Eine offene Rechnung?«

»Da muss ich kurz ausholen. Sie wissen, dass meine Firma Statson \&
Sun heißt?«

»Ja, das steht auf dem Plakat zu Ihrem Vortrag, das dort drüben an
der Wand hängt.«

Statson nickte. »Der Kerl, der meine Frau bedrohte, war Karl Suhn.
Ein österreichischer Erfinder, mit dem ich zusammen studiert und
erste Pläne zu einer feinmotorischen Suchmaschine entworfen hatte.
Wir gründeten die Firma und ließen für den Firmennamen in Karls
Nachnamen einfach das H weg, damit es besser klang. Anfangs lief es
gut, wir schlugen uns ganz ordentlich durch und kamen mit kleineren
Verkäufen finanziell über die Runden. Doch unsere Pläne zu dieser
einen großen, bahnbrechenden Erfindung führten in eine Sackgasse.
Wir kamen nicht voran und zerstritten uns darüber, wie wir aus
dieser verfahrenen Situation herauskommen konnten. Schließlich
schmiss er hin und verkaufte mir seine Anteile. Nun konnte ich
ungehindert meine Vorstellungen umsetzen und hatte schon bald
Erfolg. Dank der Entwicklung eines funktionierenden Prototypen
gelang mir ein Kooperationsvertrag mit einem Großhändler, der mir
regelmäßige Einnahmen sicherte. Karl dagegen ging es nicht so gut.
Das erfuhr ich, als er eines Tages wieder vor meiner Tür stand und
eine Gewinnbeteiligung forderte. Schließlich hätte ich auf unsere
gemeinsamen Plänen aufgebaut. Ich lehnte ab, da mir der Durchbruch
nur gelungen war, weil ich meine eigene Richtung eingeschlagen
hatte. Er ging sogar vor Gericht – und verlor. Von da an ging es
ihm immer schlechter. Die Schuld daran gab er mir. Er suchte nach
einem Weg, es mir heimzuzahlen.«

Statson machte eine Pause und holte Luft. Er hatte immer schneller
und leidenschaftlicher gesprochen. Wincover spürte Statsons Ärger
über seinen ehemaligen Geschäftspartner. »Nicht jeder Mensch, der
sich betrogen fühlt, wird zu einem Bombenleger«, warf Wincover ein.
»Gab es denn keine Anzeichen für seine radikalen Pläne? Mord ist in
der Regel erst der letzte Schritt. Davor gibt es andere, die
dorthin führen.«

Statson lächelte wieder dieses Lächeln, das Wincover nicht ganz
deuten konnte. »Ich gehe davon aus, dass Sie meinem Vortrag genauso
aufmerksam gefolgt sind, wie Sie jetzt zuhören. Also wissen Sie,
was für eine Maschine ich erfunden habe: Das Sammeln von
Informationen auf genormten Zetteln, das Einsortieren in
intelligente Strukturen, das Extrahieren neuer Informationen aus
der Kombination der Einzelinformationen. Ich habe das getan, weil
ich immer schon der Meinung war, dass der Mensch in seiner
Urteilskraft nicht zuverlässig ist. Er verarbeitet die ihm
gegebenen Informationen nicht objektiv. Er geht mit einem
vorgefassten Urteil an die Dinge heran und sucht nach Bestätigung
seiner Meinung. Wenn man etwas sucht, findet man es auch. Ich habe
für mich schon früh entschieden, dass es beispielsweise die
sogenannte Menschenkenntnis nicht gibt. Wer behauptet, einen
Menschen nach kurzer Zeit beurteilen zu können, handelt nicht
zielgerichteter als ein Horoskop. Er gibt ein paar Vermutungen ab,
die manchmal stimmen, manchmal nicht. Am Ende misst er sich nur an
seinen Erfolgen und denkt wirklich, er könne Menschen einschätzen.
Absurd!« Statson kippte mit einem angewiderten Blick seinen Port
hinunter und deutete mit dem leeren Glas auf Wincover. »Alles
Ammenmärchen, sage ich Ihnen. Der Mensch ist zu einer rationalen
Beurteilung nicht fähig. Dafür braucht man Maschinen, die völlig
vorurteilsfrei an eine Sache herangehen, nur dann bekommt man ein
stimmiges Ergebnis. So eine Maschine zu bauen, war schon immer mein
Traum. Wenn Sie also fragen, ob ich Karls Absichten früher hätte
erkennen können, sage ich nein. Für mich war Karl zwar ein hitziger
Ingenieur, aber eben doch ein Ingenieur, kein Verbrecher. Ich habe
seine Absichten nicht kommen sehen, da ich nicht im mindesten mit
ihnen gerechnet hätte.«

Er schüttelte den Kopf. »Im Nachhinein ist man immer klüger und
erkennt auf einmal die Zeichen. Aber man sieht sie nur mit dem
Wissen, was aus Karl geworden war.«

Wincover kippte nun auch seinen restlichen Port ganz gegen seine
Gewohnheit in einem Zug hinunter. Generell entsprachen die
Ansichten dieses Statsons so überhaupt nicht Wincovers
Überzeugungen. Man könne Menschen nicht einschätzen, dürfe seinem
Urteil nicht trauen? Sein halbes Leben hatte Wincover doch daran
verdient, Menschen einzuschätzen. Dieser Statson schien sich da in
etwas verrannt zu haben. Trotzdem war Wincover neugierig, wie die
Geschichte weiterging.

»Konnte Ihre Maschine Ihnen denn nicht bei der Einschätzung dieses
Kerls helfen?«

Statson lächelte ihn an. »Sie bekommen nur Antworten auf Fragen,
die Sie stellen. Karl war bis zu jenem Zeitpunkt nicht mehr
Bestandteil meines Lebens, daher hatte ich die Maschine nicht mit
Daten über ihn bestückt. Er war buchstäblich ein unbeschriebenes
Blatt für meine Maschine.«

»Also ist Ihre Maschine doch nicht allwissend.«

»Nein, aber das war von vorneherein klar. Das ist sozusagen der
Unterschied zwischen Intelligenz und Bildung. Auch ein
hochintelligenter Mensch kommt ums Lernen nicht herum. Die Frage
ist nur, was macht er aus den Dingen, die er gelernt hat? Und nutzt
er das Potenzial, das ihm sein Schöpfer mitgegeben hat?«

»Nutzt Ihre Maschine das Potenzial, dass Sie Ihr mitgegeben
haben?«

Statson fühlte sich bei diesem Vergleich sichtlich geschmeichelt.
»Ich würde mit Ja antworten, denn meine Maschine war mir in dieser
Sache immer noch von Hilfe. Zumindest bis zu einem gewissen
Punkt.«

»Inwiefern?«, fragte Wincover.

»Karls Plan war perfide, weil er meine Maschine bewusst einplante.
Wie sehr, konnte ich damals nicht ahnen. Er schickte Ann auf eine
einsame Schnitzeljagd nach der Bombe. An verschiedenen Orten an
Bord hatte er Zettel mit Informationshäppchen hinterlassen. Sie
alle verrieten Details zum Aufenthaltsort der Bombe und ihrer
Bauweise. Er wusste, dass meine Frau sich via Ætherfunk mit mir in
Verbindung setzen würde. Dass ich verzweifelt versuchen würde, ihr
Leben zu retten, und dass ich dazu eben jene Maschine einsetzen
würde, wegen der Karl und ich uns damals zerstritten hatten.«

Statson gab dem Kellner einen Wink mit dem Glas. Dieser kam sofort
mit der richtigen Flasche Portwein an und schenkte nach. Statson
drückte ihm ein Euro-Pfundstück in die Hand. Der Kellner verbeugte
sich elegant und nickte auch Wincover kurz zu, bevor er zur Bar
zurückging. Wincover schätzte diese Fluglinie und ihr Personal.
Ausgebildete Leute, die ihr Handwerk noch verstanden. Sie waren
immer zur Stelle, wenn man sie brauchte.

Statson bemerkte seinen Blick. »Gutes Personal, was? Meine Frau
flog damals übrigens mit derselben Fluglinie, auf der
\textit{Migdal}, einem Schwesterschiff der \textit{Halina II}. Sie kannte
viele der Angestellten. Etliche halfen ihr hinter dem Rücken des
Kapitäns und vertrauten ihrem Urteil trotz des ersten Fehlgriffs.
Gemeinsam arbeiteten wir ein Profil heraus, um uns zu den
vorhandenen Informationen von Karl weiter heranzutasten. Wie groß
eine Bombe sein musste, die man an Bord schmuggeln konnte, und ob
sie dort vielleicht erst zusammengebaut wurde. Wir kamen zu dem
Schluss, dass sie so oder so nicht sehr groß sein konnte. Wie viele
Orte also gab es, an denen eine relativ kleine Bombe platziert
werden musste, um in der Lage zu sein, das gesamte Schiff zu
zerstören? Denn davon hatte Karl ja gesprochen, von der Vernichtung
aller. Es gab nur wenige neuralgische Punkte, die diese Bedingung
erfüllten. Etliche halfen ihr bei der Suche, gingen mit uns die
Pläne noch einmal durch, als die erste Suche erfolglos blieb.«

Wincover stutzte. »Kann so eine Suche denn unbemerkt bleiben? Muss
der Kapitän da nicht irgendwann Lunte riechen?«\\ »Weiß Ihr
Vorgesetzter denn alles, was in der Firma vorgeht?«

»Ich arbeite eher im Außendienst, daher ist der Kontakt zu ihm eher
lose.«

Statson hob die Augenbrauen. »Nicht das Schlechteste, keineswegs.
Ich hatte in meiner Lehrzeit den Büroleiter in meinem Rücken
sitzen, da gab es Szenen, die habe ich bis heute nicht vergessen.
Aber lassen wir das. Betrachten Sie allein die Größe des Schiffes,
das uns beide in diesem Moment zur Station auf dem Jupitermond
Ganymed bringt. Spüren Sie die gewaltige Kraft der Turbinen, die
diesen Koloss antreiben. Dieses Schiff ist nichts anderes als eine
mobile Stadt. Bei diesen Dimensionen können Sie nicht alles
kontrollieren.«

Wincover wollte etwas entgegnen, verkniff es sich aber, als Statson
abwehrend die Hand hob. »Dieses Problem war sowieso zweitrangig, da
ein erheblicher Teil der Suche an der Außenwand des Schiffs
stattfand, an einem Ort also, an dem man selten von ungebetenen
Besuchern überrascht wird. In der Tat stellte sich schließlich
heraus, dass sich die Bombe auf einem der Außenträger mit den
Antriebspropellern befand. Sensible Dinger, die von den
Maschinisten jeden Tag geprüft werden. Jeden Tag muss einer auf
einem schmalen Laufsteg rausgehen und nachschauen, ob sie noch
einwandfrei laufen. Unterhalb des Laufstegs eines dieser Propeller
war die Bombe angebracht. So raffiniert, dass ihre Explosion an
dieser Stelle ein Loch in das Gerüst des zeppelinförmigen
Ætherschiffs reißen würde. Die \textit{Migdal} wäre nicht mehr gewesen als
ein unkontrolliert umherschwirrender Luftballon, der in sich
zusammenfällt. Hätte Karl doch seine Energie damals in die
richtigen Bahnen gelenkt, wie viel mehr hätten wir gemeinsam
erreichen können.«

\bigpar

In Statsons Rücken konnte Wincover erkennen, dass die blonde
Geschäftsfrau einen Mann getroffen und herzlich begrüßt hatte. Die
beiden verstanden sich offensichtlich gut und gingen mit einem Glas
in der Hand aus dem Salon. Fast bedauerte Wincover, dass er sich
auf das Gespräch mit Statson eingelassen hatte. Sonst wäre er jetzt
vielleicht der Begleiter der Frau gewesen. Aber wo dachte er hin,
das ließ sein Vorhaben auf Ganymed doch gar nicht zu. Und jetzt
wollte er schon gerne das Ende von Statsons Geschichte erfahren!

»Dann haben Sie die Bombe also rechtzeitig gefunden?«

»Wie man es nimmt«, antwortete Statson. »Wir hatten die Bombe
gefunden. Aber rechtzeitig? Wie lange brauchen Sie, um eine Bombe
zu entschärfen? Mussten Sie das schon einmal tun?« Wincover zuckte
mit den Achseln.

»Eben«, sagte Statson. »Da geht man nicht einfach ran und sucht
nach dem Knopf, den man drücken, nach dem Kabel, das man
durchschneiden muss. Entgegen der landläufigen Meinung funktioniert
das so nicht. Wenn man direkt vor so einer Bombe kniet, noch dazu
in einem Ætheranzug mit dicken Handschuhen, während alle möglichen
interstellaren Partikel an einem vorbeisausen, dann rüttelt man
nicht einfach mal so an einer Bombe. Was ich damit sagen will: Zu
jenem Zeitpunkt begann erst unsere eigentliche Aufgabe, jetzt
musste sich meine Maschine wirklich beweisen!«

Statson hob das Glas auf Augenhöhe, aber sein Blick ging durch die
rötliche Flüssigkeit hindurch. In diesem Moment wirkte er auf
Wincover wie ein Wahrsager, der in seine Glaskugel schaute und dort
nach Erkenntnissen suchte. War Statson nicht sogar eine Art
Wahrsager? Hatte er nicht etwas von den fahrenden Gauklern aus dem
Mittelalter, die ihrem Publikum trickreich vortäuschten, sie
könnten Gedanken lesen?

Statson blickte von seinem Glas zu Wincover, nun wieder ganz der
Wissenschaftler. »Sie müssen sich vergegenwärtigen, was eine Bombe
ist. Viele denken nur an den Moment der Explosion, dabei ist dieser
ja nur der Zeitpunkt, in dem sich die Mission der Bombe erfüllt.
Wer sie entschärfen will, darf daran nicht denken. Wer sie
entschärfen will, muss in ihr ein filigranes, maschinelles
Lebewesen sehen, das treu und zuverlässig die Zeit zählt. Ein Wesen
mit einer Bestimmung, das sich dem Schicksal ergeben wird, das sein
Schöpfer ihm zugedacht hat. Da surren die kleinen Zahnräder,
klicken die Hebel, vibriert die Unruh. Nur wer dieses Zusammenspiel
aller einzelnen Teile verstanden hat, kann eingreifen, kann das
Schicksal der Maschine, das auch sein Schicksal werden würde,
abwenden.«

Statson blickte Wincover entschuldigend an. »Verzeihen Sie, wenn
Ihnen das zu philosophisch ist. Wenn es Ihnen lieber ist, können
Sie die Bombe natürlich einfach als eine Maschine betrachten. Eine
Maschine aus bekannten Bauteilen in unbekannter Kombination. So gut
wie jede Bombe funktioniert nach bekannten Funktionen, das
gefährliche ist die unbekannte Kombination. Und genau da konnte die
Analyse meiner Maschine helfen. Meine Frau gab mir so viele
Beobachtungen zum Äußeren der Bombe durch wie möglich. Diese
Informationen verknüpfte ich mit allem Wissen, das ich zum Denken
und zur Arbeitsweise von Karl hatte. Auf diese Weise wollten wir
die Achillesferse finden, die Stelle, an der die Bombe verletzlich
war.«

Statson machte eine Pause und sah sich im Salon um. Etliche Sessel
hatten sich mittlerweile geleert. Im Speisesalon nebenan hatte das
Buffet eröffnet und lockte mit Delikatessen von drei verschiedenen
Planeten. An den kleinen, runden Fenstern des Salons klappte gerade
der Kellner die rot gepolsterte Läden halb vor, da laut Bordzeit
der Abend begann. Auf der \textit{Halina II} tat man alles, seinen
Passagieren einen Tagesrhythmus zu simulieren. Im freien Spalt
hinter den Läden konnte man Sterne vorbeihuschen sehen. Statson
machte eine Geste in ihre Richtung. »Wir Menschen sind Meister
darin, uns eine eigentlich tödliche Umgebung so anheimelnd
einzurichten, dass wir ganz vergessen, wie wenige Zentimeter uns
nur vom Tod trennen. Wir vergessen es so sehr, dass wir oft
leichtsinnig werden und ein noch größeres Risiko eingehen.«

Er schüttelte den Kopf. »Nein, meine Frau war nicht leichtsinnig.
Sie war hoch konzentriert. Sie wusste, was auf dem Spiel stand. Sie
wusste, wie viele hundert Leben auf der \textit{Migdal} von ihrem Handeln
abhängig waren. Punkt für Punkt ging sie die Skizze durch, die ich
von dem vermutlichen Innenleben der Bombe angefertigt hatte.

Wincover stutzte. »Wie konnte sie Ihre Skizze sehen?«, fragte er.

»Ach, das ist so eine andere kleine Erfindung von mir. Unter einem
Brett mit Glasdeckel befindet sich ein Metallgitter, das
magnetische Felder erzeugt. Auf dem Brett befinden sich feine
Metallspäne. Dieses Gerät steht in Ætherfunk-Verbindung mit einem
identischen Gerät. Wenn ich auf dem Zweitgerät mit Metallspänen
eine Zeichnung anfertige, tastet das Gitter unter dem Brett diese
ab, überträgt sie via Funk und bildet sie auf dem Empfängergerät
durch die magnetischen Felder nach. Glücklicherweise hatte meine
Frau den Prototypen in ihrer Kabine dabei, da wir die Reichweite
und die Zuverlässigkeit der Nachbildung testen wollten. Wir konnten
ja nicht ahnen, dass aus dem Test Ernst werden würde. Sie musste
die Skizze nur noch abzeichnen, dann konnte sie sie mit nach
draußen zur Bombe nehmen. Schneller geht es nicht.«

»Ein interessantes Gerät«, sagte Wincover anerkennend. »Warum haben
Sie das noch nicht veröffentlicht?«

Statson winkte ab. »Es ist sehr kompliziert zu bedienen, muss immer
völlig waagerecht aufgestellt werden und versagt leider oft den
Dienst. Wer will schon ein Gerät kaufen, das oft nicht
funktioniert? Bis zur Serienreife gibt es noch viel zu tun.«

»Aber dank dieses Geräts hatte Ihre Frau also den vermutlichen
Bauplan der Bombe?«

»Das kann man so sagen, ja.« Ein Mann kam mit unsicheren Schritten
in den Salon und ließ sich in einem Sessel unweit der beiden
nieder. Wincover war sich nicht ganz sicher, aber er glaubte in ihm
den Mann zu erkennen, der vorhin mit der kühlen Blonden weggegangen
war. Der hatte sich ja schnell betrunken, wenn er es war. Hatte sie
ihm eine Abfuhr gegeben? Stümper! Aber klar, sie war sicher eine
harte Nuss. Ob er sie geknackt hätte? Ach was. Er ärgerte sich,
dass er sich heute so leicht ablenken ließ. Es war nicht gut, so
leichtfertig zu werden. Nicht bei seinen Jobs.

»Was kam dann?«, fragte er Statson.

»Ann fing mit dem Entschärfen an«, antwortete dieser. »Meine
Erfindung hatte schließlich eine klare Empfehlung abgegeben. Meine
Frau arbeitete sich vor, löste Schraube um Schraube. Wir berieten
uns über jeden Schritt. Es fällt verdammt schwer, sich zu
konzentrieren, wenn das Leben der eigenen Frau auf dem Spiel steht
und man nicht bei ihr ist.«

»Karl hatte also erreicht, was er wollte«, sagte Wincover.

»Fast. Es fehlte nur noch das i-Tüpfelchen seines Plans. Er wollte
mir beweisen, dass ich einen Denkfehler gemacht hatte. Und er hatte
recht!«

Wincover wurde aus Statson nicht schlau. Er hatte versucht, wie man
das als Zuhörer automatisch tat, das Ende der Geschichte zu
erahnen. Erzählte Statson ihm hier eine Tragödie oder kam er im
allerletzten Moment mit einem Happy End um die Ecke? Statsons
Tonfall hatte den gleichen eifrigen und engagierten Klang, der
seinen Vortrag zu einem Genuss gemacht hatte. Aber was beim Vortrag
über eine Maschine angebracht war, wirkte bei der Geschichte über
die Todesgefahr, in der die eigene Frau geschwebt hatte, gefühllos.
Statson ließ sich einfach nicht in die Karten schauen.

»Nun spannen Sie mich nicht länger auf die Folter«, drängte
Wincover. »Ich gebe auch noch eine Runde Portwein aus.« Er winkte
dem Kellner zu, der sich an der Theke mit der Reinigung der
Mocca-Maschine die Zeit vertrieb.

Statson sah Wincover wie ein Professor an, der sich über einen
frechen Einwand eines seiner Studenten ärgerte. Dann gab er sich
einen Ruck. »Sie haben recht, ich sollte aus dieser Situation keine
heischende Geschichte machen.« Er räusperte sich. »Kurz und gut, am
Ende landeten wir doch bei der Frage ›Knopf oder Kabel?‹.
Beziehungsweise in diesem Fall: Soll Ann das kleine Rad blockieren
oder das große? Unglücklicherweise gab meine Maschine hierfür einen
expliziten Rat.«

»Unglücklicherweise?«\\ »Ja, denn wir folgten dem Rat, meine Frau
blockierte gerade das kleine Rad mit einem Draht und kommentierte
dies, als ein hohes Pfeifen scharf in meine Ohren drang, gefolgt
von einem geradezu wütenden Rauschen. Sie \ldots{} – sie \ldots{}« Statson
stockte heiser. Er konnte nur mit leiser Stimme fortfahren. »Ich
habe nie mehr von ihr gehört \ldots{} – die Bombe war explodiert und
hatte alle an Bord in den Tod gerissen.« Statson griff mit
zitternder Hand in seine Jackettasche. »Bitte entschuldigen Sie«,
sagte er und fuhr sich hektisch mit einem Taschentuch über die
Augen.

Wincover saß für einen Moment fassungslos da. Obwohl er mit diesem
Ende hätte rechnen müssen und Statsons Frau nicht kannte, ließ es
ihn nicht kalt.

»Warum hat man von dem Unglück nichts gehört?«

Statson hustete und hatte seine Stimme wieder etwas unter
Kontrolle. »Man wollte negative Auswirkungen auf den Tourismus
vermeiden und hat es buchstäblich totgeschwiegen.«

»Warum ist die Bombe explodiert? Sie sagten doch, Ihre Erfindung
hat ausdrücklich einen Rat zur Entschärfung gegeben.«

»Das hat sie. Basierend auf ihren Parametern war das auch die
richtige Entscheidung. Doch leider war mir bei der Konstruktion ein
elementarer Denkfehler unterlaufen.«

Wincover sah Statson fragend an.

»Karl wollte mich mit meinen eigenen Mitteln schlagen. Daher hatte
er die Bombe bewusst anders konzipiert, als es für ihn typisch
gewesen wäre – er hatte sich am Ende umentschieden. Ich hätte es
besser wissen müssen. Aber ich habe die menschliche Komponente
vernachlässigt und meiner Maschine blind vertraut. Einer Maschine,
die das Menschliche wegen mir auch nicht kannte. Eine Variable
fehlte, und so sortierte die Maschine die letzte Information falsch
ein. Man kann sagen, sie sortierte den Tod falsch ein.«

»Und trotz dieser Katastrophe halten Sie an Ihrer Erfindung
fest?«\\ »Warum nicht?«, sah ihn Statson beinahe erstaunt an. »Es
ist doch nur eine Maschine. Ich hatte auf drastische Weise einen
Konstruktionsfehler erkannt. Und war daher in der Lage, den Fehler
zu beheben. Nun ist die Maschine besser denn je.«

War Statson so herzlos? Oder war es seine Art, mit dem Verlust
fertig zu werden?

»Im Grunde habe ich nur eine Art Zufallsgenerator hinzugefügt«,
erklärte Statson. »Aber das angeblich so rationale Element des
menschlichen Denkens lässt sich mit einer irrationalen Komponente
sehr gut simulieren. Und das hat einen bittersüßen Geschmack von
Ironie.«

»Wünschen die Herren noch etwas?«, überraschte der auf leisen
Sohlen angeschlichene Kellner die beiden.

»Nein, danke«, sagte Statson. »Wir sind so gut wie fertig.«

Der Kellner verabschiedete sich und wünschte ihnen einen schönen
Abend. Wincover wollte sich dem anschließen. »Es hat mich gefreut,
Ihre Bekanntschaft gemacht zu haben«, sagte er. »Vielen Dank für
Ihr Vertrauen, mir diese persönliche Geschichte erzählt zu haben.
Selbstverständlich werde ich das Wissen über dieses Unglück für
mich behalten!«

»Oh, das werden Sie! Denn meine Geschichte ist noch nicht ganz zu
Ende!«

»Nicht?«, sagte Wincover überrascht.

Statson beugte sich vor. »Was glauben Sie, warum ich Ihnen die
Geschichte erzählt habe? Weil ich unbedingt einen Zuhörer finden
wollte, um den Abend rumzubringen?«

»Nun \ldots{}«, setzte Wincover an und schluckte den Satz herunter, dass
er genau dies gedacht hatte. Unruhig fragte er sich, worauf Statson
anspielte.

»Dachten Sie, Sie wären zufällig mein Gesprächspartner an diesem
Abend geworden?«, sagte dieser und schüttelte den Kopf. »Ich habe
nach Ihnen gesucht. Denn vor genau drei Tagen hat meine Erfindung
Ihren Namen ausgespuckt.«

»Meinen Namen?« Wincover schaute sich um und suchte die Ausgänge
ab. Welcher Weg wäre der kürzeste, hier herauszukommen?

»Es war ursprünglich reiner Zufall. Ein Student wollte über diese
Reise berichten und hatte dazu Recherchen und statistische
Hochrechnungen angestellt. Er konnte ja nicht ahnen, dass er damit
einem Mord auf die Spur kam.«

»Ich denke, es wird am besten sein, wir beenden das Gespräch hier«,
sagte Wincover hart.

»Wenn Sie vorhaben, als freier Mann auf Ganymed dieses Schiff zu
verlassen, wäre es für Sie das Beste sitzenzubleiben«, sagte
Statson mit einem nicht minder harten Tonfall. »Sie haben vor zwei
Wochen einen Mann auf der Erde umgebracht, und Sie haben von einem
Ehepaar dafür eine Stange Geld erhalten.«

»Das ist doch \ldots{}«

»Bitte machen Sie sich nicht durch sinnloses Leugnen lächerlich.
Der Mann, den Sie umgebracht haben, hatte das Kind Ihrer
Auftraggeber entführt und getötet. Ihr Mord war Vergeltung.«

»Sie wollen doch nicht allen Ernstes behaupten, dass Ihre Maschine
all diese Dinge für Sie herausgefunden hat?« Wincover grinste
spöttisch.

Statson schüttelte erneut den Kopf. »Natürlich nicht. Aber sie hat
Sie mit dem Mordfall in enge Verbindung gebracht. Zusätzliche
Recherchen und weitere Eingaben in der Maschine haben dann meinen
aufgeregten Studenten und mich zu Ihrem Geheimnis geführt.«

»Wollen Sie mich jetzt etwa erpressen?«, fragte Wincover.

»Nein, ich will Ihnen eine Gelegenheit geben.«

»Was soll das heißen?«

»Sie haben nicht das erste Mal getötet. Sie haben sich auf
Vergeltungsmorde spezialisiert. Der erste war für Sie persönlich,
nachdem man Ihre hochschwangere Frau ermordet hatte. Als die
Polizei wegen Ermittlungsfehlern den Täter laufen lassen musste,
haben Sie das Heft in die Hand genommen. Ich möchte, dass Sie
weitermachen.«

Wincover sah Statson ungläubig an. »Das ist nicht ihr Ernst. Dafür
sind Sie nicht der Typ!«

»Sagt wer? Ihre Menschenkenntnis? Ich habe Karl einen Massenmord
auch nicht zugetraut. Aber man wird nicht gerade ein besserer
Mensch, wenn man hilflos miterlebt, wie die eigene Frau getötet
wird. Und man droht daran zugrunde zu gehen, wenn man den Täter
nicht zur Rechenschaft ziehen kann, weil er sich der Vergeltung
entzogen hat. Ich weiß nicht, warum Karl sich mit einer Giftkapsel
getötet hat, anstatt von Bord zu fliehen. Vielleicht, weil er
gewusst hat, dass ich ihn gejagt hätte. Weil er mir die Genugtuung
nicht gönnte, ihn eines Tages doch zu fangen. Nun gut, dann müssen
eben andere daran glauben. Nennen Sie es ruhig eine perfide Art,
dafür zu sorgen, dass sie nicht umsonst gestorben ist. Aber ich bin
nun einmal nicht der Mensch, der mildtätige Stiftungen ins Leben
ruft, um ihrem Tod einen Sinn zu geben.«

Wincover suchte in Statsons Blick die Ironie oder den Sarkasmus,
fand aber keinen. Dieser Mann meinte wirklich, was er sagte.

»Sie lassen mich also gehen?«

»Ja, aber ich möchte Sie warnen. Man ist Ihnen auf der Spur.
Bekanntlich schätzt das Gesetz es nicht, wenn man ihm eigenmächtig
die Arbeit abnimmt. An Bord sind zwei Polizisten, die sie bei der
Landung festnehmen wollen.«

»Wer? Woran erkenne ich sie?«

»Sie waren vorhin hier. Vielleicht sind sie Ihnen aufgefallen. Eine
blonde Frau und ein Mann. Sie hat die kühle Geschäftsfrau gespielt,
er ihren Verehrer. Er kam vorhin noch einmal zurück, um zu schauen,
ob Sie noch hier sind.«

Wincover fiel es wie Schuppen von den Augen. Das verliebte Paar!
Wie dumm war er nur gewesen, nicht Lunte zu riechen. Er drehte sich
ruckartig zu dem Sessel um, in den sich der Mann bei seiner
Rückkehr vermeintlich angetrunken hatte fallen lassen. Der Sessel
war leer. Wieso hatte er die beiden nicht erkannt? Er wurde
nachlässig.

»Ich sehe, sie sind Ihnen aufgefallen«, sagte Statson. »Es war
nicht ganz einfach, aber meine Maschine hat mir auch dabei
geholfen, die beiden zu entdecken. Das ist alles, was ich für Sie
tun kann. Jetzt sind Ihre Fähigkeiten gefragt, wie Sie sich aus
dieser Situation herauswinden. Ich bitte Sie nur um eines!«

»Das wäre?«, fragte Wincover. Seine Gedanken begannen schon zu
rotieren, welche Möglichkeiten ihm blieben.

»Bringen Sie die beiden Polizisten nicht um. Ich möchte nicht für
den Tod dieser Menschen verantwortlich sein.«

»Keine Sorge«, beruhigte ihn Statson. »Polizistenmord ist schon
seit jeher kein kluger Schachzug für einen Menschen, der nicht auf
allen Planeten dieses Universums von Gesetzeshütern gejagt werden
möchte.«

\bigpar

Statson richtete sich im Sessel auf und zog sich das Jacket glatt.
»Nun, dann wäre alles gesagt und wir können unserer Wege gehen.« Er
streckte Wincover die Hand entgegen. »Es hat mich gefreut, Ihre
Bekanntschaft gemacht zu haben. Ich hätte mich gerne mit Ihnen über
Ihren ungewöhnlichen Beruf unterhalten, aber ich denke, es ist am
besten, wenn ich darüber so wenig wie möglich weiß.«

Wincover ergriff die Hand. Das Händeschütteln war für ihn eine
Vertragsbesiegelung unter zwei Ehrenmännern. Dabei sah er sich und
seinen Beruf schon lange nicht mehr als Ehrensache. Eher als die
Müllabfuhr der Gerechtigkeit. »Auf eine merkwürdige Weise hat es
mich auch gefreut«, sagte er. Hoffentlich überlegen Sie es sich
nicht noch anders.«

»Sie wissen, ich bin ein Mensch der Zahlen. Wenn ich zu einem
Ergebnis gekommen bin, das meiner Meinung nach stimmt, dann stehe
ich dazu!« Der Kellner nickte ihnen von der Theke aus zu und dimmte
das Licht. Sie gaben sich einen Ruck. »Wir sollten den Salon durch
verschiedene Ausgänge verlassen und nicht mehr miteinander reden«,
sagte Statson. Wincover stimmte nickend zu.

»Leben Sie wohl«, sagte Statson.

»Leben Sie wohl«, antwortete Wincover.

\bigpar

Die beiden Männer drehten sich um und verließen den Salon. Im
letzten Moment blickte Wincover noch einmal über seine Schulter zu
seinem Gesprächspartner. Ob es sicherer wäre, wenn Statson Ganymed
nicht lebend verließ? Aber das wäre schon sehr undankbar von ihm.
Oder nicht?

\end{document}

