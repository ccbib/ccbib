\usepackage[ngerman]{babel}
\usepackage[T1]{fontenc}
\usepackage{textcomp}
\hyphenation{wa-rum}


%\setlength{\emergencystretch}{1ex}

\renewcommand*{\tb}{\begin{center}* \quad * \quad *\end{center}}

\newcommand\bigpar\medskip

\begin{document}
\raggedbottom
\begin{center}
\textbf{\huge\textsf{Es ist nicht leicht, kein Held zu sein}}

\bigskip

Bernd Meyer
\end{center}

\bigskip

\begin{flushleft}
Dieser Text wurde erstmals veröffentlicht in:
\begin{center}
Die Steampunk-Chroniken\\
Band I -- Æthergarn
\end{center}

\bigskip

Der ganze Band steht unter einer
\href{http://creativecommons.org/licenses/by-nc-nd/2.0/de/}{Creative-Commons-Lizenz.} \\
(CC BY-NC-ND)

\bigskip

Spenden werden auf der
\href{http://steampunk-chroniken.de/download}{Downloadseite}
des Projekts gerne entgegen genommen.

\vfill

Geboren 1968 in Hamburg, wohnt Bernd Meyer mittlerweile mit seiner Familie
und den Katzen etwas weiter nördlich. Schon früh verfiel er dem
Rollenspiel, wo er recht viel ausprobiert hat, was den Mangel an
Lieblingsgenres erklären könnte. Er machte mehrere Ausbildungen,
unternahm zwischendurch das Risiko der Selbständigkeit und blieb
nebenbei immer wieder beim Schreiben hängen. Im Rahmen von FOLLOW,
eines Fantasy-Literaturvereins, wurden die ersten literarischen
Gehversuche unternommen. Sein erstes Romanmanuskript wartet noch
auf den Feinschliff, während er dann und wann die eine oder andere
Kurzgeschichte schreibt.

\bigpar

Bisherige Veröffentlichungen über die eigene Internetpräsenz
\texttt{http://www.bedlamboys.de/} sowie in
der Anthologie "Berggeister" des Mondwolf-Verlages.
\end{flushleft}

\section{Es ist nicht leicht, kein Held zu sein}
Das fahle Licht des Mondes fiel auf das Aussichtsdeck der World of
Æther und warf die Schatten der beiden an der Reling stehenden
Personen auf die Wand hinter ihnen.

Der Luxusliner glitt gemächlich durch den Æther, fast schien es den
Passagieren so, als befänden sie sich auf einer Kreuzfahrt über
einen der irdischen Ozeane. Doch das gemächliche Schwanken des
Fahrzeugs durch die Wellen fehlte ebenso, wie die frische Brise,
die man auf der Erde gespürt hätte. Stattdessen war um sie herum
der schier endlose Sternenozeans, diese samtene Schwärze, in der
die fernen Lichtpunkte der Sterne schwammen. Eine Glaswand trennte
die Passagiere vom Æther, sorgte aber dafür, dass der großartige
Ausblick nicht getrübt wurde. Fast unmerklich drehte das Schiff ein
wenig nach Steuerbord ab, so dass der Mond ganz langsam achteraus
wanderte. Die kleinere der beiden Gestalten ließ den Kopf sinken
und seufzte leise. Sofort wandte sich die andere Person ihr zu.

\bigpar

»Lady Walsington, ist Euch nicht gut? Soll ich den Steward rufen?«

\bigpar

Wäre ein Beobachter zugegen gewesen, hätte er die schlankere der
beiden Gestalten als weiblich erkannt. Modisch gekleidet, der
kostbare schwarze Stoff die Figur der Trägerin umschmeichelnd, das
lange rote Haar von einem kleinen Hut gekrönt. Die grünen Augen
schimmerten hinter dem hauchdünnen, schwarzen Schleier feucht über
der zierlichen Stupsnase und den vollen Lippen. Sie umfasste die
Reling fester, als würde sich eine Ertrinkende an den rettenden
Halt klammern. Ohne ihren Blick zu heben schüttelte sie den Kopf.

»Es ist nichts, Sir Geoffrey, aber haben Sie Dank für Ihre
Besorgnis. Der Anblick lässt mich nur ein wenig sentimental werden.
Wie Sie vielleicht wissen, war mein Verlobter bei der königlichen
Marine. Er \ldots{} kam von einer Patrouille nicht zurück. In Erfüllung
seiner Pflicht gefallen, sagte man.«

Sie wandte sich ab, beide Hände auf der Reling, und starrte hinaus
in die Schwärze. Sir Geoffrey, noch vom Dinner formal in seinen
Frack gekleidet, für den Besuch auf Deck den Zylinder auf dem
Kopfe, lehnte sich auf seinen Gehstock mit dem Silberknauf und
wusste nicht so recht, was er tun sollte. Sein wohlfrisiertes,
blondes Haar ragte unter der Kopfbedeckung hervor, seine grauen
Augen ruhten auf der Gestalt der Lady vor ihm. Es schmerzte ihn,
sie derart traurig zu sehen, aber der Anstand verbot ihm, sie
tröstend in den Arm zu nehmen, wie sein Herz es ihm sagte.

Er hatte sie erst vor ein paar Tagen kennengelernt, genau an diesem
Ort, auf dem Aussichtsdeck. Sie war jeden Abend hier, betrachtete
den Æther und dachte an die Vergangenheit. Doch selbst wenn sie
keine Trauer getragen hätte, würde es sich nicht schicken, ihr
derart zu nahe zu treten. So musste er sich damit begnügen, in
ihrer Nähe zu bleiben und abzuwarten. Ein Angebot – das wohl nie in
Anspruch genommen werden würde.

Ach, wäre seine Familie doch nur noch so bedeutend wie zu den
Zeiten seines berühmten Vorfahren. Aber nur dessen Ruhm war
geblieben, nicht die Stellung der Familie oder das Vermögen, dafür
hatte sein Onkel mit seiner Vorliebe für kostspielige Bälle und
Pferdewetten gesorgt. Wäre er nicht bereits so früh einem
Jagdunfall zum Opfer gefallen, würde die Familie vielleicht sogar
einer Arbeit nachgehen müssen. Schockierender Gedanke …

Somit galt seine Familie als nicht einflussreich genug, um in der
Gesellschaft ernstzunehmend wahrgenommen zu werden. Schon gar nicht
von einer Frau wie Lady Walsington, die gute Verbindungen zum Hof
hatte, wie es hieß. Aber zumindest wehrte sie sich nicht gegen
seine Anwesenheit, im Gegenteil schien sie ihre gemeinsamen
Spaziergänge über das Aussichtsdeck zu genießen. Nur bei jenen
letzten Wanderungen nach dem Dinner griff verstärkt die Schwermut
nach ihr.

Nach kurzer Zeit reichte sie ihm ihren Arm, den er ergriff und sie
zu ihrer Kabine geleitete. Als er sich kurze Zeit später zur Nacht
umkleiden ließ, grübelte er immer noch, wie er ihr Erleichterung
verschaffen konnte. Doch wie stets kam er auf keine Lösung.

\tb

»Sir, bitte wachen Sie auf. Bitte, Sir, es ist dringend!«

Sein Butler, Lionel, stand vor seinem Bett, eine Tasse frisch
gebrühten Tees in der Hand. Unwillig richtete Sir Geoffrey sich auf
und sah seinen Butler an.

»Ich habe mich doch soeben erst zum Schlafen niedergelassen, warum
wecken Sie mich denn schon? Ich würde gern eine Bemerkung machen,
dass es draußen noch dunkel ist, aber das wäre fruchtlos, im Æther
ist es bekanntlich immer dunkel. Also, Lionel, was ist los? Gibt es
einen Bridge-Notfall?«

Mit einem versöhnlichen Lächeln das seinen Worten die Spitze nahm
und einem dankbaren Nicken griff Sir Geoffrey nach der Teetasse.
Trotzdem er es nicht zeigte und sogar versuchte, es mit Humor zu
überspielen, war er besorgt. Lionel würde ihn nie grundlos wecken,
dafür kannten sie sich zu lange und hatten zu viel miteinander
erlebt. Also musste etwas passiert sein, etwas sehr Ernstes sogar.

»Wir werden angegriffen, Sir. Piraten, vermutet der Kapitän. Er
bittet um Ihre Anwesenheit auf der Brücke. Darf ich Ihnen beim
Anziehen helfen, Sir?«

Sir Geoffrey verschluckte sich an einem Schluck Tee. Während er
hustete, rasten seine Gedanken. Piraten! Und die World of Æther war
ein Vergnügungsschiff, kein Kriegsschiff. Unbewaffnet, auch
Soldaten waren nicht an Bord, von ein paar Passagieren abgesehen.
Doch das waren meistens pensionierte Offiziere, die schon lange
keinen Waffengang mehr erlebt hatten. Wie sollten sie sich also zur
Wehr setzen? Wie die Damen beschützen gegen die unangebrachten
Aufmerksamkeiten, welche diese ungehobelten Rohlinge sicherlich im
Sinn hatten. Sie diesen Banditen kampflos zu überlassen, verbot
sich von selbst. Kurz stieg die Vorstellung, Lady Walsington müsse
ein derartiges Schicksal erleiden, vor seinem geistigen Auge auf,
doch er verwarf diesen Gedanken sofort.

Eilig kleidete er sich mit Hilfe von Lionel an, griff nach seinem
Gehstock, dann begab er sich schnellen Schrittes zur Brücke.

\bigpar

Auf dem Kommandostand herrschte eine Art geordnete Unruhe,
zumindest konnte Sir Geoffrey keine bessere Beschreibung dafür
finden. Man merkte deutlich, dass etwas nicht in Ordnung war, aber
jeder Mann versah seine Aufgabe mit der üblichen Professionalität.
Ihre Augen jedoch huschten unruhig hin und her, suchten die Quelle
der Unruhe. Nur der Kapitän, ein würdevoller, älterer Gentleman in
weißer Uniform und mit einem prächtigen Backenbart, stand wie ein
Fels der Ruhe mitten auf der Brücke. Mit sicherer Hand lenkte er
sein Schiff durch den Æther, näher zum Mond, wo die Zwölfte
Königlich Britische Wachflotille stationiert war. Der
Schiffsheliograph versuchte unterdessen, die Station der »Zwölften«
zu erreichen, um sie über die Notlage zu unterrichten.

»Ah, Sir Geoffrey, gut dass Sie da sind. Winslow mein Name, ich bin
der Kapitän der World of Æther. Vergeben Sie mir die unverzeihliche
Störung, aber wir haben ein kleines Problem. Wenn Sie ihren Blick
gen Steuerbord richten, werden Sie dort einen Verfolger bemerken.
Dort drüben, sehen Sie? Ausgezeichnet! Es handelt sich
bedauernswerterweise nicht um Gentlemen, sondern um Piraten, welche
die unvergleichliche Frechheit besaßen, uns dazu aufzufordern, die
Flagge zu streichen. Unverschämtes Pack! Unglaublich! Leider sind
sie schneller als wir, bei der Konstruktion der World of Æther
wurde auf Geschwindigkeit weniger Wert gelegt als auf
Bequemlichkeit. Wir werden ihnen also nicht davonsegeln können,
ebenso wenig werden wir dazu in der Lage sein, sie niederzukämpfen,
da wir keine nennenswerten Bordgeschütze mitführen. Wenn sie uns
entern, sollten wir uns ihrer eine gewisse Zeit erwehren können, da
wir aller Wahrscheinlichkeit nach über mehr Personen verfügen.
Aber, bei Harry, wir können doch nicht von den Passagieren
verlangen, zu kämpfen. Das wäre schockierend, meinen Sie nicht? Wie
Sie nun sicherlich sehen, haben wir hier ein kleines Ungemach. Aber
bestimmt können Sie uns behilflich sein, immerhin sind Sie ja ein
Nelson, nicht wahr? Steckt im Blut, der Heldenmut, was?«

\bigpar

Sir Geoffrey wand sich innerlich. Natürlich entstammte er einer
berühmten Familie von Helden, aber warum sollten sie deswegen alle
derart veranlagt sein? Er hatte schon genug Probleme, die bohrenden
Fragen von Tante Beth zu antworten, was aber in dem doch recht
geschlossenen Umfeld der Familie um einiges einfacher war. Immerhin
konnte man unter Verwandten den Ruf der Familie nicht gefährden,
nur seinen eigenen. Der aber war Sir Geoffrey herzlich egal,
solange er in Ruhe gelassen wurde.

Es war ja nicht so, dass er feige gewesen wäre. Keineswegs, er
hatte sich schon des Öfteren in halsbrecherische Unternehmungen
gestürzt. Es war nur so, dass sein Interesse an Abenteuern sich auf
Bridge und die Times beschränkte. Er bevorzugte ein gemütliches und
beschauliches Leben. Nun aber hatte ihn wieder einmal der Ruf
seiner Familie eingeholt.

»Natürlich, Kapitän Winslow. Gar keine Frage. Aber wie Sie selbst
bereits sagten, wir dürfen die Damen nicht in Gefahr bringen. Das
wäre unverzeihlich. Haben Sie die Flotte benachrichtigt? Wann
können wir mit Hilfe rechnen, Sir?«

Der Kapitän lächelte, dann nickte er zum Heliographen hinüber.

»Wir haben noch keine Bestätigung erhalten, also müssen wir
befürchten, dass unsere Nachricht die Station nicht erreicht hat.
Unsere Position ist nicht ideal, wir haben die Sonne schräg hinter
uns, deswegen kann unser Signal im Schein des Sterns untergehen.
Wir versuchen uns dem Mond zu nähern, so gut es geht. Erreichen
werden wir ihn nicht, aber möglicherweise entdeckt uns eine
Patrouille und kommt uns zu Hilfe. Oberlippe steif halten und so
weiter, richtig? Auf jeden Fall stehen wir hinter Ihnen, Sir
Geoffrey, wir sind uns ganz sicher, dass Sie uns hier herausholen
werden.«

Großartig, es geht nichts über subtilen Druck, um einen Tag zu
einem großartigen Erlebnis zu machen! Geoffrey zwang sich dazu,
weiterhin zu lächeln. Währenddessen rasten seine Gedanken und
versuchten einen Ausweg zu finden. Der ganz große Nachteil einer
Ætherkreuzfahrt, verglichen mit einer Kreuzfahrt auf den Meeren,
klang zwar vernachlässigbar, war aber gerade jetzt ein ziemlich
großes Problem. Rettungsboote gab es nicht, genauso wenig wie man
einfach von Bord springen und sein Leben dem Schicksal und dem Meer
anvertrauen konnte. Der Æther war nicht so gnädig wie die irdischen
Wassermassen. Leider. Und der Verfolger holte immer weiter auf,
sich seiner Sache völlig sicher, schoss er sie nicht erst
zusammen.

Wäre das Schiff besser bewaffnet gewesen hätten die Piraten das
wohl getan, aber entweder wussten sie nichts von den wenigen
Geschützen oder sie erachteten sie nicht als eine Gefahr. Wobei sie
das wohl auch nicht waren, wenn er den Schiffstyp der Verfolger
richtig erkannte. Eine Korvette. Woher hatte dieses Gesindel nur
eine Korvette? Sollten sie das Schiff der britischen Marine
abgerungen haben? Das würde auf einen respektablen Gegner schließen
lassen.

»Wenn mir die Frage gestattet ist, Sir, welcherart Geschütze führt
die World of Æther mit sich? Ich sollte schon wissen, mit welchen
Kontingenten ich planen kann.«

Er zwang seine Züge dazu, verwegen auszusehen, während er den
Kapitän ansah. Doch so zuversichtlich und draufgängerisch, wie er
sich gab, war er mitnichten.

\bigpar

Wenn nur Bredon hier wäre, der wüsste, was zu tun ist. Immerhin tut
er so etwas öfter, der Verrückte. Bredon war Sir Geoffreys ältester
Bruder. Ein waschechter Held, selbst für die Verhältnisse seiner
Familie, wie die Times einmal schrieb. Er hatte viele Länder
bereist und fühlte sich in unwirtlichen Gebieten wie zu Hause. Er
focht wie der Teufel, verstand gar mit einer Armbrust umzugehen.
Schoss genauer als dieser Schweizer – Hell, oder wie auch immer der
geheißen hatte. Außerdem war Bredon ein ausgezeichneter
Cricket-Spieler und ein angenehmer Bridgepartner. Er liebte die
Gefahr und die Queen hatte ihn bereits ausgezeichnet.

Sir Geoffrey fand ihn suspekt, der Mann hatte keinen Funken
modisches Verständnis. Er kleidete sich praktisch, bei Gott, nicht
nach den Gesichtspunkten der Mode. Wäre da nicht die Tatsache, dass
die Queen ihn geehrt hatte, und die Verwandtschaft, Sir Geoffrey
hätte ihm ganz klar abgesprochen, ein Gentleman zu sein. So aber
musste er ihn wohl oder übel ertragen, auch wenn es schwerfiel.
Aber da Bredon nicht hier war, oblag es offenbar ihm, die
Familienehre hochzuhalten. Er hoffte, er würde sich dieser Aufgabe
gewachsen zeigen.

»Geschütze \ldots{} nun ja,« entgegnete der Kapitän. »Die Konstrukteure
gingen davon aus, dass die königliche Marine im Bedarfsfall den
Schutz übernehmen würde. Deswegen haben wir – wenn man diesen
hochtrabenden Begriff verwenden möchte – zwei Breitseiten zu je
zwei dreieinhalb Zoll-Kanonen und einem Tesla-Geschütz. Eigentlich
eher, um zu zeigen, dass wir bewaffnet sind, nicht wirklich
nützlich, befürchte ich. Dazu noch eine Handvoll Gewehre und
Faustfeuerwaffen, auch diese eher zu Repräsentationszwecken. Sie
sehen also, wir sind nicht viel mehr als ein alter, zahnloser
Hund.«

Der Kapitän lachte dröhnend über seinen Scherz und Sir Geoffrey
fiel höflich mit ein, während sich in seinem Kopf ein Plan zu
formen begann. Ein sehr riskanter Plan, aber er baute darauf, dass
die Piraten die Besatzung der World of Æther nicht für so dumm
hielten, derartige Dinge zu versuchen. Nun musste er die Besatzung
nur noch davon überzeugen, genau das zu tun. Und nebenbei beten,
dass sein wahnwitziger Plan gelang. Warum eigentlich war Bredon nie
da, wenn man ihn einmal brauchte?

\tb

»Sir Geoffrey, sind Sie sich wirklich sicher, das gut überlegt zu
haben? Haben Sie nicht doch möglicherweise etwas übersehen? Keine
andere Option denkbar? Ich möchte Ihnen nicht zu nahe treten, aber
bitte bedenken Sie, dass wir anderen keine Helden sind, nur ganz
normale Untertanen Ihrer Majestät. Was, wenn wir Ihren
Vorstellungen nicht gerecht werden können?«

\bigpar

Sir Geoffrey lächelte sanft.

Nette Worte, aber ich weiß doch ganz genau, dass ihr mich für
verrückt haltet und am liebsten einsperren würdet. Himmel, ich
selbst halte mich ja auch für verrückt, auch nur ansatzweise
anzunehmen, das hier könnte klappen. Aber es ist vielleicht unsere
einzige Chance, den Schurken Herr zu werden. Und wäre ich kein
Nelson, dann würdet ihr mir das auch ganz klar sagen und es nicht
nur denken. Anstatt aber diese Gedanken auszusprechen, legte er
lieber dem Kapitän beruhigend die Hand auf die Schulter.

»Machen Sie sich keine Sorgen. Mein Butler war früher Kanonier,
einer der Besten. Hat unter mir gedient, ich habe ihm schon des
Öfteren mein Leben anvertraut. Und es nie bereut, ansonsten würde
ich hier ja nicht stehen.«

Sein Lachen verbarg seine Nervosität. Zu viel konnte schief gehen,
jetzt. Er konnte nur hoffen und beten, dass sein Plan wie ersonnen
funktionierte. Mit fragend hochgezogener Augenbraue wandte er sich
an den ersten Offizier.

»Sir Geoffrey, wir haben wie befohlen den Piraten signalisiert,
dass wir uns ergeben und beigedreht. Damit wenden wir ihnen nun
unsere Breitseite zu. Keine Antwort bisher von der Königlichen
Flotte. Die Passagiere haben keinen Verdacht geschöpft, die
Stewards haben den Halt als eine bessere Möglichkeit, die Schönheit
des Mondes zu genießen ausgegeben. Die Mannschaft verhält sich
ruhig, um kein Aufsehen zu erregen. Die Verfolger scheinen keinen
Verdacht geschöpft zu haben und nähern sich weiter. Alles läuft
nach Plan, Sir. Und wenn ich das sagen darf, Sir, die Besatzung ist
stolz, unter Ihrem Kommando kämpfen zu dürfen. Wo Sie doch einer
der heldenhaften Nelsons sind, Sir.«

\bigpar

Sir Geoffrey nickte dankend, dann wandte er sich der Brückenseite
zu, von der die Angreifer kamen. Jetzt waren schon Einzelheiten des
Piratenschiffes zu erkennen. Guter Zustand, wie er neidlos
anerkennen musste. Von der Bewaffnung her um ein Vielfaches
überlegen, aber das hatte er vorher gewusst. Gegen die World of
Æther waren selbst die Kutter der Flotte kleine Schlachtschiffe.
Aber es gab eine Chance, wenngleich auch nur eine recht winzige
Chance. Klappte der Versuch nicht, so würden die Piraten keine
Gnade mit ihnen kennen, wie er annahm. Immerhin hätten sie
versucht, diese Schurken zu täuschen. Es musste einfach gelingen.
Nicht auszudenken, wenn \ldots{} Er zwang seine Gedanken fort von Lady
Walsington und wieder zurück zur aktuellen Situation. Das Schiff
der Verfolger drehte jetzt ein, brachte sich in Position, um
längsseits zu gehen. Nun kam es auf den richtigen Zeitpunkt an. Zu
früh oder zu spät und alles wäre vergeblich gewesen.

\bigpar

»Sir Geoffrey, wir müssen jetzt reagieren. Schnell, bevor es zu
spät ist. Bitte, wir \ldots{}«

Der erste Offizier konnte seine Angst nun nicht mehr verbergen. Auf
ein Nicken des Kapitäns hin wurde er von zwei kräftigen Matrosen
von der Brücke geführt. Auch der Kapitän hatte Angst, sogar große
Angst, wie ein Blick in seine Augen zweifelsfrei bestätigte, doch
er beherrschte sich mustergültig. Er wusste, dass seine Mannschaft
nur funktionierte, weil er sich nicht gehen ließ. Sir Geoffrey war
ihm dankbar, ließ aber weiterhin das Schiff der Piraten nicht aus
den Augen.

\bigpar

Gleich \ldots{} nur noch ein klein wenig näher \ldots{} warte \ldots{} warte \ldots{}
nun komm schon \ldots{} jetzt!

Zeitgleich mit dem letzten Gedanken blitzte es an der
Steuerbordseite auf, dann rollte der Donner der Abschüsse durch das
Schiff. Auch wenn er sie nicht sehen konnte, so wusste er, dass die
Kanonenkugeln durch den Æther auf das Schiff der Piraten zu rasten.
Ihr Steuermann versuchte noch, das Ruder herumzureißen, aber es war
zu spät.

Noch während auf der Brücke der Jubel der Besatzung aufbrandete,
sah Sir Geoffrey, wie die Geschosse am Ziel ankamen. Aber sie
prallten nicht an der Panzerung ab, sondern zerschlugen die
Brückenfenster und verursachten im Kommandoraum des Gegners Tod und
Zerstörung!

Guter alter Lionel! Die Wahrscheinlichkeit war minimal gewesen,
aber es hatte ihre einzige Chance dargestellt. Die gepanzerte
Außenhülle hätten sie mit ihren kleinkalibrigen Kanonen nicht
überwinden können, sie mussten einen der wenigen verwundbaren
Punkte treffen. Das aber gelang nur mit einem meisterhaften
Schützen. Dass sein Butler ein derartiger Mensch war, hatte weder
die Besatzung der World of Æther noch die Angreifer wissen können.
Deswegen war der Anflug derart sorglos gewesen.

Und sie hatten es wirklich vollbracht, wenngleich es ihnen vorerst
nur mehr Zeit verschafft hatte. Das Piratenschiff trieb steuerlos
im Æther, aber die Geschütze waren noch einsatzbereit. Sobald sich
das Passagierschiff in den Schussbereich der Kanonen der Schurken
bewegte, würden diese feuern. Und die World of Æther war um ein
Vielfaches verletzlicher als ihr Verfolger.

\bigpar

Dankbar registrierte Sir Geoffrey, dass der Kapitän den Jubel
eindämmte und anordnete, wieder Kurs auf den Mond zu nehmen. Ein
Brückenoffizier wurde abgestellt, die Piraten im Auge zu behalten,
und der Heliograph versuchte erneut, die Königliche Wachflotille zu
erreichen.

\bigpar

Sir Geoffrey jedoch begab sich in die Messe, ohne einen Blick
zurückzuwerfen. Vermissen würde man ihn erst später und der
glückliche Ausgang würde wieder etwas zur Legende um seine Familie
beitragen. Dass ein Scheitern sehr viel wahrscheinlicher gewesen
war, daran würde sich schon nach kurzer Zeit niemand mehr erinnern.
Und warum auch, immerhin war die tollkühne Tat ja gelungen. Jeder
Zweifel hätte nur der Heldenverehrung im Weg gestanden und so etwas
konnte man ja nicht zulassen …

Auch die Rolle seines Butlers Lionel würde in Vergessenheit
geraten, was ihn nicht wenig ärgerte. So würde er selbst dafür
Sorge tragen müssen, dass der wahre Held dieser Begebenheit eine
angemessene Entlohnung erhielt. Wie üblich.

Und vielleicht sickerte ja genug seitens der Besatzung an die
Passagiere durch, dass Lady Walsington ihn ein wenig bewundern
würde. Um sie nicht zu kränken würde er diesen eigentlich
unverdienten Ruhm ertragen.

\bigpar

Aber war das nicht ebenfalls irgendwie heldenhaft?
\end{document}

