\usepackage[ngerman]{babel}
\usepackage[T1]{fontenc}
\hyphenation{wa-rum}


%\setlength{\emergencystretch}{1ex}

%\renewcommand*{\tb}{\begin{center}* * *\end{center}}

\newcommand\bigpar\medskip

\begin{document}
\raggedbottom
\begin{center}
\textbf{\huge\textsf{Triumpf der Wissenschaft}}

\medskip
Bernd Meyer

\end{center}

\bigskip
\begin{flushleft}
Dieser Text wurde erstmals veröffentlicht in:
\begin{center}
Die Steampunk-Chroniken\\
Geschichten aus dem Æther
\end{center}

\bigskip

Der ganze Band steht unter einer 
\href{http://creativecommons.org/licenses/by-nc-nd/2.0/de/}{Creative-Commons-Lizenz.} \\ 
(CC BY-NC-ND)

\bigskip

Spenden werden auf der 
\href{http://steampunk-chroniken.de/download}{Downloadseite}
des Projekts gerne entgegen genommen. 
\end{flushleft}

\newpage

Gemächlich zog die \emph{HMAS Ikarus} durch den Æther, ihre
Außenhülle blitzte in der Sonne. Vor einem Monat in England vom
Stapel gelaufen, befand sie sich nun in den Weiten des Æthers auf
einer Forschungsmission. \emph{Die Augen der Krone}, so hatte es
beim Auslaufen geheißen, \emph{ruhen auf dieser Mission}. Was Lord
Stokesbridge, dem Kapitän, herzlich egal hätte sein können. Er war
zufrieden, sein Schiff durch den Æther zu führen, ungestört sich
diesem Abenteuer zu stellen und weitab von allem Trubel zu sein.
Ein hervorragender Kapitän, aber mit einigen Schwächen, was den
Umgang mit anderen Menschen anging. Seine Offiziere und auch die
Crew wussten das, nahmen ihm aber die daraus resultierende
Schroffheit nicht übel. Warum auch, immerhin war ein
Forschungsschiff Ihrer Majestät kein Vergnügungsdampfer, ein etwas
rauerer Ton war also nichts Ungewöhnliches. Solange ihr Kapitän nur
das Beste für Schiff und Mannschaft im Sinn hatte, stünden sie treu
hinter ihm.

Auf dieser Reise aber war ihr Skipper immer missmutiger geworden,
je weiter sie in die Weiten des Æthers vorgestoßen waren. Und
seitdem sie den Mond passiert hatten, war es ganz schlimm geworden.
Oftmals sah man ihn in sich selbst versunken dastehen und
Unverständliches in seinen prächtigen weißen Bart murmeln. So
kannten sie ihn gar nicht, dabei war dies nicht das erste Schiff,
auf dem sie unter ihm fuhren. Vielleicht sollte man ihn behutsam
darauf ansprechen, schlug der zweite Offizier Willowbrough-Smythe
vor. Doch irgendwie fand sich kein Freiwilliger, der diese Aufgabe
übernehmen wollte, darum geriet der Vorschlag über der Bordroutine
schnell in Vergessenheit.

\bigpar

Lord Stokesbridge blickte in Gedanken versunken in die samtige
Schwärze des Æthers hinaus. Eigentlich liebte er diesen Anblick,
doch heute wirkte er nicht so beruhigend auf ihn wie sonst. Der
Grund lag aber nicht bei dem Schiff oder der Mannschaft, auch wenn
die \emph{Ikarus} nicht ganz auf Kurs zu liegen schien. Fast schon
fahrig winkte Lord Stokesbridge zu seinem Steuermann hinüber.

»Überprüfen Sie den Kurs, Huxley, wir liegen ein Grad zu weit
steuerbord, wenn ich mich nicht irre. Das Mädchen ist nicht ganz
einwandfrei getrimmt, das sollten wir überprüfen.«

Und wieder versank er in seinen Gedanken, ohne noch weiter auf den
Steuermann zu achten, der erstaunt den Kurs anglich und sich unter
einem strafenden Blick des ersten Offiziers zusammenkrümmte. Nein,
der Grund war nicht das Schiff, es war vielmehr ...

»Guten Morgen, Kapitän. Haben Sie schon Anweisung gegeben, den Kurs
anzupassen? Sie wissen ja, unser Auftrag ist von höchster
Wichtigkeit.«

Sir James Saint-John, Professor für Ætherphysik an Ihrer Majestät
Universität war hinter ihn getreten und hatte ihm die Hand auf die
Schulter gelegt. Das alleine war schon Grund genug für Lord
Stokesbridge, verstimmt zu sein. Jedoch war der Professor nur das
kleinere Übel. Das größere Übel würde aber in Kürze auch
auftauchen, man konnte sich stets drauf verlassen, dass die Präsenz
Sir Saint-Johns auch ihre Anwesenheit nach sich zog. Sie, das war
Lady Isabelle Farthington, Hofdame Ihrer Majestät und deren
besondere Vertraute, was sie zur inoffiziellen Leiterin dieser
Mission machte, zumindest in ihren Augen. Und da sie ziemlich
deutlich machte, dass Ihre Majestät von ihr Berichte über den
Verlauf dieser Unternehmung erwartete, war es von großer
Wichtigkeit, die Dame nicht zu verärgern. Was aber zugleich
bedeutete, kein Ventil für seinen Ärger zu haben, denn von ihrer
Wichtigkeit überzeugt, behandelte Lady Farthington jeden an Bord,
als wäre er ihr persönlicher Lakai. Selbst ihn, den Kapitän, hatte
sie schon für Handreichungen verpflichtet. Unverschämte Person.
Aber natürlich konnte er ihr das nicht direkt ins Gesicht sagen, so
sehr ihn auch danach verlangte, denn das hätte das Ende seiner
Karriere und seiner gesellschaftlichen Stellung bedeutet. Und auch
wenn ihn der Verlust seiner Karriere kaum mehr belastet hätte, so
musste er doch auf seine Familie Rücksicht nehmen und sich deswegen
in Zurückhaltung üben. Es fraß ihn innerlich auf, war er doch
bisher immer stolz darauf gewesen, seine Meinung auch zu
vertreten.

\bigpar

»Oh, Kapitän Stackenbridge, da sind Sie ja. Ich habe sie schon
überall gesucht. Sie wollten mir heute das hydroponische Deck
zeigen, erinnern Sie sich noch? Nun, worauf warten wir denn noch?«

Lord Stokesbridge zuckte zusammen und starrte auf das Deck. Ein
geknurrtes »Ich sollte dich als Dünger vergraben, dann wärest du
wenigstens zu etwas nütze« wurde nahezu unhörbar hervorgestoßen,
als er sich umdrehte und ein erfreutes Lächeln auf seine Züge
zwang.

»Wie meinen Sie? Ich konnte Ihre Erwiderung gerade nicht verstehen,
Kapitän.«

Lady Farthington kam näher und hielt ihm die Hand hin, der Ton
ihrer Frage arglos. Der Kapitän beugte sich über ihre Hand, küsste
sie und gab dann seinem ersten Offizier einen Wink. Doch ganz ohne
Gegenwehr wollte er sich nicht fügen.

»Stokesbridge, Mylady. Mein Name ist Stokesbridge, nicht
Stackenbridge. Das ist Ihr Steward, wenn ich mich nicht irre.
Außerdem habe ich hier wichtige Dinge zu erledigen, die meiner
vollen Aufmerksamkeit bedürfen; ich fürchte, im Moment kann ich mir
das Vergnügen der Besichtigungstour schlicht nicht erlauben. Es tut
mir außerordentlich leid.«

Lady Farthington zog einen Schmollmund und wollte sich gerade
verstimmt umdrehen, als der erste Offizier sich räusperte und das
Wort ergriff. Sein Kapitän schien unter einem extremen Stress zu
stehen, dem musste er entgegenwirken.

»Sir, ich würde mich gern dazu bereit erklären, Ihre Wache zu
übernehmen. Dann könnten Sie der Lady wie versprochen zur Verfügung
stehen. Außerdem würde es Ihnen ermöglichen, gleich die eigentlich
heute Abend anstehende Inspektion vorzunehmen, das verbände das
Angenehme mit dem Nützlichen.«

\bigpar

Der Offizier salutierte korrekt und wandte sich dann wieder dem
Steuermann zu. Dadurch entging ihm das Aufblitzen in den Augen
seines Kapitäns, das ihm eine höchst unangenehme Zukunft verhieß.
Angesichts des erfreuten Quietschens, mit dem die Lady diese
Wendung gut zu heißen schien, streckte Lord Stokesbridge die Waffen
und bot ihr seinen Arm. Gemeinsam verließen sie dann die Brücke,
ohne noch auf den sichtlich mit sich selbst zufriedenen Offizier zu
achten.

\tb

»Und dies hier, Mylady, ist unser hydroponisches Deck. Hinter
diesem Begriff verbirgt sich eine ausgeklügelte Anlage, die
einerseits unser Schiff mit atembarer Luft versorgt, andererseits
Mannschaft und Passagiere mit pflanzlicher Nahrung und zum Teil
auch mit Blumenschmuck beliefert. Hier zum Beispiel sehen Sie
unsere Rosenstöcke, sind sie nicht prachtvoll?«

Der Kapitän führte seine Begleiterin durch die Parklandschaft,
welche das oberste Deck der HMAS Ikarus darstellte. Eine Zofe
begleitete sie und bemühte sich, so viel wie möglich von der
Umgebung zu bestaunen, ohne geistesabwesend ihre Herrin
anzurempeln, wenn diese wieder einmal stehen geblieben war.

»Wie sie sehen, Mylady, haben wir hier verschiedene Büsche und
Bäume gepflanzt, wobei wir versucht haben, die Anlage auch für das
Auge ansprechend zu gestalten. Natürlich können wir nur kleinere
Bäume nutzen, andernfalls wäre natürlich eine Eiche als Sinnbild
für die Tugenden der Marine erfreulich gewesen. Aber auch so ist
die Auswahl recht ansprechend gelungen, wenn ich das so sagen darf.
Geschickt die Beengtheit überspielt, meinen Sie nicht?«

Lady Farthington nickte erfreut, während ihre Augen mit leuchtendem
Blick über die Blumen wanderten, die zwischen den größeren Pflanzen
eingesetzt waren. Nie hätte sie gedacht, an einem solchen Ort eine
derartige Vielfalt zu sehen. Es fehlte nur noch das Zwitschern von
Vögeln, aber das wäre nun wirklich zu viel erwartet.

»Aber das sind nur Zierpflanzen, oder täusche ich mich da?
Sicherlich werden Sie den begrenzten Platz hier nicht nur dafür
verwendet haben, oder? Sie erwähnten pflanzliche Nahrung und
frisches Gemüse mitzunehmen, ist ein derartiges Vorhaben nicht viel
zu aufwendig?«

Er lächelte, ohne sich dessen bewusst zu sein.

»Das ist richtig, den Platz können wir nicht erübrigen. Außerdem
würde es nicht lange genug frisch bleiben. Nein, dieses Problem
haben wir anders gelöst, schauen Sie.«

Er führte sie an den Rand eines Weges und bog die Zweige eines
Busches beiseite. Dahinter erstreckte sich ein Feld voller grüner
Pflanzen, in sauberen Reihen gepflanzt.

»Oh wie bezaubernd, Kapitän. Eine kluge Lösung, das muss ich
zugeben. Sie können sich also selbst versorgen auf der
\emph{Ikarus}? Beeindruckend.«

Unbewusst richtete er sich bei dem Lob etwas weiter auf, doch
schüttelte er gleich darauf den Kopf.

»Nein, Mylady, auch wenn es eine gute Sache wäre. Aber die dafür
nötigen Felder könnten wir auf dem Schiff gar nicht unterbringen.
Was hier wächst, ist ein Zusatz, aber damit ist die Verpflegung um
Längen besser, als sie auf den Seeschiffen Ihrer Majestät ist.
Nichts gegen die Verpflegung dort, aber im Vergleich schwelgt meine
Mannschaft geradezu im Luxus. Auch wenn es ein notwendiger Luxus
ist, denn durch frisches Gemüse verhindern wir den Skorbut. Also
ein sehr nutzbringender Aufwand.«

Seine Begleiterin widmete sich wieder den Blumen und roch an ihren
Blüten. Mit einem klaren, glockenhellen Lachen warf sie dann ihren
Kopf zurück und ließ die samtene Schwärze des Æthers auf sich
wirken.

»Ich muss sagen, diese Anlage beeindruckt mich, Kapitän. Aber
sicherlich dient sie nicht nur praktischen Gesichtspunkten, oder?
Dann wären bestimmt die Blumen nicht hier. Oh, was ist das denn da?
Wissen Sie das? Sophie, bring mir doch bitte ein Blatt davon. Ich
möchte es mir ansehen.«

Ihre Zofe tat, wie ihr geheissen und der Kapitän folgte mit dem
Blick ihrer Geste.

»Nun, das ist ... es sieht aus wie ... bestimmt ... hmmm ...
Mylady, ich muss gestehen, ich komme nicht auf den Namen. Aber
irgendetwas war damit, das weiß ich noch ganz genau. Wenn ich mich
nur erinnern könnte. Smithers!«

Ein Steward eilte auf diesen Ruf hin herbei. Der Kapitän deutete
auf die Pflanze, von der die Zofe gerade ein Blatt pflückte, daran
roch und sich dann auf den Rückweg zu ihrer Herrin machte.

»Smithers, wie heißt diese Pflanze? Helfen Sie meinem Gedächtnis
auf die Sprünge.«

Smithers kniff die Augen zu, dann entfuhr ihm ein leiser Aufschrei
des Erstaunens.

»Sir, das ist Giftsumach. Ich befürchte, die Dame hätte ihn nicht
berühren sollen, aber zum Glück trägt sie ja Handschuhe. Das sollte
die Wirkung abhalten. Ich empfehle jedoch, das Blatt beiseite zu
werfen.«

Mit einem erschreckten Quietschen ließ die Zofe das Blatt fallen.
Dann sah sie ihre Herrin an, auf deren Gesicht es merkwürdig
zuckte. Der fragende Blick der Zofe sorgte dann dafür, dass Lady
Farthington sich nicht länger beherrschen konnte und in ein helles
Lachen ausbrach. Sie versuchte, etwas zu sagen, aber durch das
Lachen war sie kaum zu verstehen. Nach mehreren Versuchen konnte
sie mit Anstrengung ein »Die Nase!« hervorbringen, dann übermannte
sie das Lachen wieder. Erschrocken führte die Zofe die Hände zu
ihrem Gesicht, schneller als Smithers, der es auch gesehen hatte,
sie davon abhalten konnte. Als er ihre Handgelenke umfassen konnte,
war es schon zu spät und auf ihren Wangen bildeten sich rote
Flecken, genau wie schon vorher auf ihrer Nase. Tränen erschienen
in ihren Augen, als Smithers dem Kapitän einen Blick zuwarf, den
dieser mit einem Nicken beantwortete. Behutsam auf das weinende
Mädchen einredend führte Smithers sie dann weg, um sie in die
Sanitätsabteilung zu bringen.

\bigpar

»Giftsumach. Genau das war es. Unschönes Zeug, die Berührung ruft
rote Flecken hervor. Aber ansonsten größtenteils harmlos. Gut, dass
ich mich wieder erinnert habe. Weiß gar nicht, warum wir das
gepflanzt haben. Aber ich denke, das hat medizinische Gründe. Ist
Ihnen nicht gut, Mylady?«

Lady Farthington konnte sich kaum noch beherrschen, das Lachen
perlte ohne Halt aus ihr heraus. Und immer, wenn sie gerade
versuchte, sich zu beruhigen, fiel ihr Blick auf den Kapitän und
sie wurde wieder vom Lachen übermannt. Besorgt führte er sie zu
einer Bank und wartete still ab, bis sie wieder Luft schnappen
konnte. Und auch er merkte, dass sein vorhin noch ausgeprägter
Ärger sich im Laufe der Besichtigung verflüchtigt hatte, ja er
begann sogar, sich an dem Augenblick zu erfreuen. Vielleicht war
die Lady ja doch gar nicht so schlimm, wie er gedacht hatte. Nun
gut, sie hatte ihn aus seinen Pflichten herausgerissen, sie hätte
durchaus auch warten können, aber andererseits tat ihm die Ruhe
dieses Ortes sehr gut. Vielleicht war er ja viel zu verkrampft an
dieses Thema herangegangen, er sollte besser …

\bigpar

Es sollte Kapitän Stokesbridge nicht vergönnt sein, diesen Gedanken
zu beenden. Der Klang der Alarmglocke, der durch das Schiff hallte,
ließ ihn zusammenfahren. Sekundenbruchteile später musste er
allerdings zugeben, dass der schrille Schrei, der von Lady
Farthington ausging, die alarmierende Wirkung der Glocke um ein
Vielfaches übertraf. Nur einen Augenblick erlaubte er seinen
Gedanken, an dem Bild festzuhalten, in Zukunft die Dame als
Notsignal zu verwenden, dann wurde er wieder ernst. Die kurz über
seine Gesichtszüge huschende Andeutung eines Lächelns hätte seine
Begleitung bestimmt irritiert, aber glücklicherweise war sie zu
beschäftigt damit, ihre Panik zu pflegen.

Ein herbeieilender Steward wurde von ihm dazu verpflichtet, die
Lady in Sicherheit zu bringen und mithilfe der Dienste einer ihrer
Zofen zu beruhigen. Kurz sah er ihr nach, dann schüttelte er
unwillig den Kopf und eilte zum Niedergang. Er musste zur Brücke
gelangen und den Grund für den Alarm herausfinden. Sollte es sich
um einen Angriff handeln, hätten sie einen schweren Stand, denn die
Ikarus war ein nur minimal bewaffnetes Forschungsschiff. Sie würden
die Flotte zu Hilfe rufen müssen, wobei es fraglich war, ob diese
ihnen rechtzeitig würde beistehen können. Sich im Geist alle
möglichen Szenarien ausmalend, lief er durch das Schiff, vorbei an
der Besatzung, die zwar beunruhigt schien, jedoch ohne zu fragen
ihren Aufgaben nachging. Dann und wann passierte er Matrosen, die
auf ihre Stationen rannten, zwar eilig, jedoch ohne Anzeichen von
Panik. Stolz erfüllte ihn, doch konnte er sich nicht davon
aufhalten lassen. Es gab eine Gefahr, da musste jeder auf seinem
Posten sein. Außer Atem rannte er um die letzte Ecke, betrat die
Brücke und ... erstarrte.

Ihm bot sich das ganz normale Bild effektiver Geschäftigkeit. Gäbe
es nicht den immer noch ertönenden Alarm, niemand könnte vermuten,
dass etwas ungewöhnlich wäre. Mit einer schnellen Geste bedeutete
er einem Fähnrich, den Alarm zu beenden. Als die Glocke verstummte,
wandten sich die Offiziere zu ihm um. Der Steuermannsmaat
salutierte und wollte Meldung machen, aber Lord Stokesbridge kam
ihm zuvor: »Wo ist mein erster Offizier? Er sollte bei einem Alarm
auf der Brücke sein. Warum ist er das nicht? Und vor allem ...
warum gab es den Alarm überhaupt, bei Harry?«

Der Steuermannsmaat lächelte etwas schief, wurde aber sofort wieder
ernst. Erneut salutierte er, dann zeigte sich seine vorherige
Marine-Dienstzeit deutlich in seiner Meldung.

»Sir, der erste Offizier entbietet seinen Gruß, Sir. Wenn der
Kapitän es wünscht, möge er bitte auf das Unterdeck kommen, dort
gibt es eine ... Situation, Sir. Der erste Offizier würde ihn dort
erwarten, Sir.«

Wieder salutierte der Mann, dann erstarrte er und schien auf etwas
zu warten. Der Kapitän aber grübelte schon über die Worte der
Meldung und schien nicht auf ihn zu achten. Das Gesicht des Mannes
begann, Unruhe zu zeigen, als der militärische Drill mit der
Tatsache umzugehen versuchte, dass Lord Stokesbridge sich dem
Schott zuwandte, anstatt ihn zu entlassen. Den Marine-Regularien
entsprechend durfte er sich aber nicht eher bewegen, als dass der
Vorgesetzte es ihm erlaubt hatte. Doch besagter Vorgesetzter schien
ihn vergessen zu haben. Ihn mit einem Räuspern an die eigene
Gegenwart zu erinnern oder ihn gar anzusprechen wäre undenkbar
gewesen. Der Steuermannsmaat versuchte derart verzweifelt, einen
militärisch korrekten Ausweg aus diesem Dilemma zu finden, dass ihm
beinahe entgangen wäre, wie der Kapitän ihm, bevor er durch das
Schott verschwand, geistesabwesend zurief »Ach ja, rühren, Mann!«

Erleichtert machte sich der Steuermannsmaat wieder an seine Arbeit,
die stichelnden Blicke der restlichen Brückenbesatzung bewusst
ignorierend.

Als er erneut durch die Korridore der \emph{Ikarus} eilte, fühlte
Stokesbridge wieder den Ärger in sich aufsteigen. Warum wurde er
wie ein Untergebener im Schiff herumgeschickt, anstatt dass ihm
irgendjemand einfach sagte, was passiert war? So musste er von
Pontius zu Pilatus laufen, um den Grund für den Alarm
herauszufinden. Das musste sich ändern, immerhin war er der Kapitän
dieses Schiffes.

\bigpar

Mit außerordentlich schlechter Laune setzte der Kommandant der
\emph{Ikarus} seinen Fuß auf das Unterdeck. Dort, da war eine
größere Ansammlung von Personen, vielleicht konnten diese ihm
sagen, was auf seinem Schiff vor sich ging. Rücksichtslos drängelte
er sich durch die Menschen, wobei jeder Protest sofort verstummte,
als die Murrenden erkannten, wer er war. Schließlich stand er in
dem freien Zentrum, das die Umstehenden gelassen hatten. Vor ihm
stand sein erster Offizier, der die Arme von Sir Saint-John
festhielt. In den Händen hielt der Professor eine schwere Axt und
blickte zornig den Offizier an.

»Kapitän, Sir. Gott sei Dank, dass Sie da sind. Wir haben hier ein
kleines ... Problem, denke ich. Sehen Sie, Mister Saint-John hier
...«

»Sir Saint-John, das werden auch Sie noch hinbekommen, denke ich.
Unverschämtheit. Kapitän Stokesbridge, ihr Mann hier behindert die
Wissenschaft, ich verlange ...«

Der Kapitän hatte genug. War er vorher nur verärgert und schlecht
gelaunt, so platzte ihm nun der Kragen. Er entriss dem Professor
die Axt und strich mit den Fingern fast liebevoll über das Blatt,
was bei den anderen beiden Männern für Unbehagen sorgte. Beide
öffneten den Mund, um etwas zu sagen, aber er ließ ihnen keine
Chance.

»Ruhe, verdammt. Sir Saint-John, Sie dürfen mich mit Lord
Stokesbridge anreden, wenn Ihnen so viel an Titeln liegt. Und Sie,
Mister Longbottom, werden mir nun erklären, warum auf meinem Schiff
Alarm gegeben wurde. Und ich empfehle Ihnen, die Erklärung so knapp
und umfassend wie möglich zu machen, wenn Sie sich nicht in Eisen
wiederfinden möchten. Und Sie ...«, er deutete wahllos auf einen
der Wissenschaftler in der sie umgebenden Menge, »... Sie bringen
mir ein Glas Sherry! Und das möglichst schnell, wenn ich bitten
darf!«

Der völlig überraschte Wissenschaftler wollte zuerst protestieren,
immerhin war er ein Mann des Geistes und kein Bediensteter, aber
ein Blick in das Gesicht des Kapitäns, der immer noch die Axt
geistesabwesend streichelte, überzeugte ihn davon, seinen
Widerspruch nicht zu äußern. Mit einem knappen Nicken eilte er
davon, um das Gewünschte zu beschaffen. Währenddessen räusperte
sich der erste Offizier und blickte zu Sir Saint-John hinüber, der
immer noch mit knallrotem Kopf um seine Fassung rang.

»Nun ja, Sir, der Alarm wurde gegeben, weil Sir Saint-John
versuchte, mit dieser Axt ... nun, er versuchte, ein Bullauge zu
öffnen, Sir.«

Der Kapitän zuckte zusammen. Sein Blick bohrte sich in den
Wissenschaftler, seine Hände krampften sich um den Axtgriff.

»Sie wollten was tun? Bitte, Sir Saint-John, tun Sie mir den
Gefallen und sagen Sie mir, dass sich Mister Longbottom getäuscht
hat, damit wir alle über dieses Missverständnis lachen und wieder
unseren Aufgaben nachgehen können.«

Der Wissenschaftler senkte unwillkürlich den Kopf, doch dann
straffte er sich, begegnete dem Blick des Kapitäns und deutete
anklagend auf den ersten Offizier.

»Dieser Mann ... Lord Stokesbridge, ich verlange, dass er bestraft
wird. Sperren Sie ihn ein, nein, besser noch, lassen Sie ihn
erschießen. Er hat den Fortschritt der Wissenschaft verhindern
wollen! Dieser Barbar, kein Verständnis für die moderne
Forschung.«

Ein Knurren entrang sich der Kehle des Kapitäns, was den
Wissenschaftler dazu veranlasste, ein paar Schritte
zurückzuweichen. Die Axt unbewusst als Verlängerung seines Armes
benutzend, setzte der Kapitän nach.

»Sie bringen mein Schiff und die gesamte Besatzung durch Ihre
bodenlose Dummheit in äußerste Gefahr und dann haben Sie die
Frechheit, sich auf die Wissenschaft zu berufen? Ist es
wissenschaftlich, nicht nachzudenken? Ist dann der stupideste
marsianische Hinterwäldler eine Leuchte der Wissenschaft? Immerhin
denkt der auch nicht nach und tut einfach das, was ihm in den Sinn
kommt. Wenn mir meine gute Erziehung es nicht verbieten würde, dann
wäre ich nun versucht, Ihnen meine Meinung zu sagen, Sir. Ich
sollte Sie einfach über Bord werfen, damit wäre sowohl der
Wissenschaft als auch der Menschheit gedient. Aber ich ...«

Ein leises Räuspern erklang, dann unterbrach ihn die Stimme Lady
Farthingtons.

»Sie werden natürlich nichts von all dem tun. Ich denke nicht, dass
Ihre Majestät von einem solchen Verhalten sehr angetan wäre.
Stattdessen würde ich Sie bitten, mir zu erklären, warum diese
Lappalie eine derartige Reaktion Ihrerseits hervorruft. Nehmen Sie
einfach an, ich würde von der ganzen Sache nichts verstehen.«

Lord Stokesbridge wirbelte herum, in seinem Gesicht eine Mischung
aus Zorn und Unverständnis. Kurz setzte er an, mit der Axt zu
gestikulieren, aber Lady Farthington nahm sie ihm ohne viel
Federlesens ab und reichte sie einem Mitglied der Mannschaft. Das
verwirrte den Kapitän nur noch mehr, aber bevor er ins Stottern
geraten konnte, erinnerte er sich an seine Erziehung sowie seine
Stellung und seine Gestalt straffte sich wieder. Der anwesende
Offizier trat etwas zurück, als er die hervortretende Ader an der
Schläfe des Kapitäns bemerkte.

»Nun, Mylady, diese ... Lappalie, wie Sie es zu nennen belieben,
war mitnichten so harmlos, wie Sie es darzustellen versuchen. Wie
man eigentlich wissen sollte, befindet sich die Ikarus auf ihrem
Weg durch den Æther. Das ist jenes dunkle Gebiet da draußen, vor
dem Bullauge.«

Ein Anflug eines Lächelns überzog sein Gesicht, verschwand jedoch
sehr schnell wieder, als er den Zorn auf ihren Zügen sah. Und
überraschenderweise dämpfte ihr Zorn den seinen, sodass er etwas
ruhiger weiter sprach.

»Der Æther, wie schon gesagt, umgibt uns. Und er ist keine
Umgebung, in der wir existieren können, fürchte ich. Hätte nun
Professor Saint-John seinen Plan ausführen können, so wäre unsere
durchaus kostbare Luft hinaus in den Æther entschwunden, der sich
daraufhin an Bord der Ikarus breit gemacht hätte. Das Atmen und
damit die fortgesetzte Existenz von Besatzung und Passagieren wäre
hierdurch recht herausfordernd geworden, wenn ich das so sagen
darf. Wir wären wegen dieses ... dieses Versehens alle gestorben.
Was mich zu der Frage bringt, wie ein kluger Kopf wie der des
Professors etwas so unglaublich Dummes hervorbringen kann. Ich
würde nur ungern annehmen müssen, dass der Professor der
\emph{Ikarus} schaden wollte, das würde mich dazu zwingen, ihn
einzusperren und ihm den Prozess zu machen. Damit aber vermisse ich
immer noch seine Motivation für diese Tat, die er mir jedoch
sicherlich gleich logisch begründen wird. Nun, ich warte ...«

\bigpar

Alle Augen richteten sich auf den Angesprochenen, der während der
Aussage des Kapitäns mehrmals sichtlich mit sich gerungen hatte, es
aber schließlich unterlassen hatte, auf der korrekten Adressierung
zu bestehen. Nun aber, im Fokus der Aufmerksamkeit, wurde er immer
nervöser. Fast flehentlich rang er seine Hände und schließlich
senkte er den Kopf und murmelte verschämt etwas Unverständliches.

»Was wollten Sie sagen, Sir Saint-John? Ich fürchte, es konnte
niemand Ihren Worten folgen.«

Die Stimme Lady Farthingtons durchschnitt die peinliche Stimme. Ihr
Gesicht hatte einen mitfühlenden Ausdruck angenommen, der jedoch in
ihrer Stimme fehlte.

»Ich habe nicht daran gedacht. Ich war mit den Gedanken bei meinem
Experiment, da muss mir das ... entfallen sein.«

Die Umstehenden sogen scharf die Luft ein, doch der Kapitän knurrte
nur vor sich hin. Wie leicht könnte er nun diesen Vorfall nutzen,
um das Unternehmen abzubrechen, niemand würde ihm einen Vorwurf
machen können. Doch er hatte den Blick gesehen, den Lady
Farthington ihm zugeworfen hatte, ohne dass es jemand anders
gemerkt hatte. Woher wusste sie, was in ihm vorging? Verdammtes
Weibsbild!

Doch er musste eine Entscheidung treffen, also winkte er einen Maat
heran.

»Simmerson, Sie werden ab nun dafür Sorge tragen, dass Sir
Saint-John in seinem Arbeitseifer nicht wieder die Sicherheit der
\emph{Ikarus} vergisst. Das bedeutet auch, werter Professor, dass
Sie Mister Simmerson fortan vorab über alles zu unterrichten haben,
was notwendig ist, um seine Aufgabe zu erfüllen. Und wenn er etwas
untersagt, dann richten Sie sich danach, als ob die Anweisung von
mir kommt. Haben wir uns verstanden, Sir?«

Der Maat salutierte zackig, während der Professor nur stumm nicken
konnte. Dann wurde er von seinen Mitarbeitern zurück in seine
Kabine begleitet, während der erste Offizier die
Besatzungsmitglieder wieder an ihre Arbeit schickte. Lady
Farthington nickte dem Kapitän lächelnd zu.

»Ich danke Ihnen für Ihre Umsichtigkeit, Mylord. Mit Ihrer
Erlaubnis möchte ich mich nun zurückziehen. Aber vielleicht können
wir die Besichtigung der hydroponischen Anlage ein anderes Mal
fortsetzen. Ich denke, Sie haben im Moment dringlichere Aufgaben,
als Ihre kostbare Zeit mir zu widmen.«

Ihr Lächeln wurde noch eine Spur süßlicher, ohne dass er aber an
ihren Augen erkennen konnte, wie ihre Bemerkungen gemeint waren.
Als sie von ihrer Zofe fortgeleitet wurde, stand der Kapitän immer
noch am selben Platz und versuchte, aus ihrem Verhalten schlau zu
werden.

\bigpar

Auf dem Weg zurück zur Brücke sah Lord Stokesbridge zu seinem
ersten Offizier hinüber.

»Nun, Mister Longbottom, würden Sie mir einmal erklären, was diesen
Vorfall ausgelöst hat? Damit wir dann gemeinsam überlegen können,
wie man so etwas in Zukunft vermeiden kann, immerhin wird dies ja
hoffentlich nicht unsere letzte Reise als Forschungsschiff Ihrer
Majestät sein.«

Der Offizier ließ kurzzeitig ein Lächeln erkennen, das aber schnell
verschwand, als er bemerkte, wie ernst es seinem Kapitän war. Mit
einem Räuspern überspielte er die Entgleisung.

»Nun, Lord Stokesbridge, es ging um ein Experiment. Der Eierkopf
... Verzeihung, Professor Saint-John natürlich, war mit einem
Versuch beschäftigt, für den er eine Probe dem Æther aussetzen
musste. Das gestaltete sich jedoch als schwierig, immerhin verfügt
die Ikarus nicht über derartige Möglichkeiten, wie Sie wissen. Doch
davon wollte der Professor nichts hören, er begann damit, die
Männer und mich zu beschimpfen, dass wir seine Arbeit ruinieren
wollten. Schließlich griff er sich eine der Not-Äxte und kündigte
an, sich notfalls mit Gewalt einen Zugang zum Æther verschaffen zu
wollen. Da habe ich Alarm geben lassen.«

Der Kapitän blieb stehen und sah den ersten Offizier nun direkt
an.

»Und was wollte er dem Æther aussetzen? Vielleicht hätte es eine
einfachere Möglichkeit gegeben, als das gesamte Schiff zu
gefährden. Wäre mir jedenfalls lieber, würde ich sagen.«

Longbottom zuckte unmilitärisch mit den Schultern.

»Verzeihen Sie mir, Kapitän, aber ich habe nicht die geringste
Ahnung. Leider war der Professor nicht wirklich kommunikativ, was
derlei Details betraf.«

Lord Stokesbridge sah ihn verwundert an.

»Bitte tun Sie mir den Gefallen und finden es heraus. Und sprechen
Sie mit dem Schiffszimmermann, damit wir versuchen, eine Lösung für
das Problem zu finden. So störend der Vorfall auch war, hat er doch
aufgezeigt, dass die Ikarus für ihre Aufgabe nicht optimal
ausgerüstet ist. Und das sollten wir ändern, wenn es uns möglich
ist, meinen Sie nicht?«

Der erste Offizier nickte, salutierte knapp und begab sich an die
Ausführung seiner Order.

\tb

Einige Zeit später, Lord Stokesbridge saß gerade im Rauchsalon und
genoss ein wenig Ruhe von seinen Pflichten, versuchte der erste
Offizier mit einem vorsichtigen Hüsteln seine Aufmerksamkeit zu
erregen. Der Kapitän fuhr zusammen und sah sich erschreckt um,
funkelte dann den Offizier an.

»Müssen Sie sich so anschleichen, Mann? Haben mich ganz schön
erschreckt, wie soll ein Mensch denn in Ruhe denken können bei den
ständigen Störungen hier?«

Longbottom verkniff sich die Bemerkung, dass die regelmäßigen
Atemzüge des Kapitäns nicht gerade auf eine rege Denktätigkeit
hingewiesen hatten. Stattdessen versuchte er immerhin,
schuldbewusst auszusehen.

»Bitte um Vergebung, Sir, aber ich hatte Sie so verstanden, dass
Sie von Ergebnissen sofort unterrichtet werden wollten. Wenn ich
mich allerdings später wieder melden soll ...«

Der Kapitän winkte ab, wenn auch seine Bewegung etwas irritiert
wirkte.

»Keineswegs, nun reden Sie schon, Mann. Was für Ergebnisse meinen
Sie?«

Mit der anderen Hand bedeutete Lord Stokesbridge dem Offizier, sich
in den gegenüberliegenden Sessel zu setzen, was dieser dankend
annahm. Eine schlanke Tonpfeife stopfend begann er mit seinem
Bericht.

»Wie mir von Ihnen aufgetragen wurde, habe ich mit dem
Schiffszimmermann geredet. Er war sich noch nicht ganz sicher, aber
er hatte schon ein oder zwei Ideen, wie man das Problem angehen
könnte, während einer Reise einen Zugang zum Æther zu ermöglichen.
Mit Ihrer Erlaubnis würde er gerne beginnen, an dem Problem zu
arbeiten. Er versprach mir, dass der Schiffsrumpf dabei nicht in
Mitleidenschaft gezogen werden würde und ich bin geneigt, ihm in
dieser Hinsicht zu glauben, auch wenn er so ein seltsames Funkeln
in den Augen hatte.«

Lord Stokesbridge nickte zerstreut, während er nach seiner eigenen
Pfeife griff, dabei aber feststellen musste, dass sie bereits
gestopft war. Mit einem Kienspan entzündete er sie, gab seinem
Offizier auch Feuer und sog dann genüsslich den würzigen Rauch
ein.

»Und mit diesem unsäglichen Wissenschaftler ... wissen Sie da
inzwischen Genaueres?«

Longbottom schüttelte bedauernd den Kopf.

»Leider nein. Professor Saint-John hat sich in andere Experimente
vergraben und ist nicht ansprechbar. Immerhin scheint er Ihre
Befehle zu befolgen, zumindest sagte Simmerson mir das. Wenngleich
er auch nicht unbedingt weiß, was die ganzen Versuche für einen
Sinn haben. Aber wenigstens bringen sie die Ikarus nicht in
Gefahr.«

»Verdammte Eierköpfe, warum müssen wir ehrlichen Seeleute uns mit
diesem Haufen abgeben? Weltfremde Spinner, die ohne Hilfe nicht
einmal fähig sind, auf einem Ætherschiff zurechtzukommen. Aber sie
haben das Ohr Ihrer Majestät, warum auch immer. Nun ja, unser Platz
ist immer dort, wo der Wille Ihrer Majestät uns hinstellt, nicht
wahr? Vertrauen wir darauf, dass sie weiß, was am Besten für das
Empire ist. Und nun ruft mich die Pflicht wieder. Die unterbrochene
Führung durch das hydroponische Deck, wissen Sie? Auch wenn ich
Besseres mit meiner Zeit anzufangen wüsste, aber die Pflicht geht
vor, nicht?«

Longbottom beugte sich vor, um seine Pfeife auszuklopfen, deswegen
sah der Kapitän das Lächeln nicht, das über die Züge seines
Offiziers huschte.

»Natürlich, Sir, die Pflicht geht immer vor. Überaus vorbildlich
von Ihnen, Sir.«

Salutierend verabschiedete er sich und machte sich auf den Weg zum
Schiffszimmermann. Der Kapitän aber begab sich gedankenverloren zum
Spiegel und richtete den Sitz seiner Uniform, bevor er sich auf den
Weg zum hydroponischen Deck machte.

\bigpar

»Lord Stokesbridge, wie schön, dass Sie die Zeit fanden, unsere
unterbrochene Besichtigung fortzusetzen. Aber sind Sie wirklich
sicher, dass Ihre anderen Aufgaben nicht dringlicher sind?«

Lady Farthington reichte ihm ihren Arm, den er lächelnd ergriff und
zusammen schritten sie über die Wege durch die Pflanzungen.

»Aber nein, meine Offiziere sind durchaus imstande, sich um diese
Aufgaben zu kümmern. Damit habe ich die Muße, Ihnen ein wenig mehr
von unserem Schiff zu zeigen. Immerhin sollen Sie ja später Ihrer
Majestät ausführlich Bericht erstatten können, wie gut unsere
Forschungsschiffe ihren Aufgaben nachkommen können.«

Da er seine Aufmerksamkeit den Pflanzen zuwandte, um ihr die
beeindruckendsten Exemplare zeigen zu können, entging ihm das
Lächeln, das sich auf ihrem Gesicht ausbreitete. Schnell verbarg
sie ihre Belustigung aber wieder.

»Das bedeutet, Sie sind nur deswegen dazu bereit, sich mit mir zu
belasten, um bei Ihrer Majestät einen guten Eindruck zu machen?
Nun, warum haben Sie das nicht gleich gesagt? Möchten Sie dann
diese Posse beenden und sich wieder Ihren Aufgaben zuwenden? Ich
kann Ihnen versichern, dass mein Bericht deswegen um keinen Deut
schlechter ausfallen wird. Soll meine Zofe mich dann wieder zu
meinen Gemächern führen?«

Der Kapitän wurde bleich und sah sie bestürzt an.

»Aber, Lady Farthington, wie kommen Sie ... ich habe doch nicht ...
würde doch niemals ... bitte, Sie verstehen nicht, es ist ...«

Mit unbewegter Miene winkte sie ihrer Zofe, die daraufhin näher
trat.

»Bitte, Sie müssen verstehen, ich ...«

Mit hochgezogener Augenbraue sah sie den Kapitän an.

»Muss ich? Meinen Sie wirklich?«

Den immer noch um Fassung ringenden Kapitän neben sich kurzzeitig
nicht beachtend, ließ sie sich von ihrer Zofe den Fächer reichen
und bedeutete ihr dann, sich wieder zu entfernen. Ihr Gesicht mit
dem nun geöffneten Fächer beschirmend wandte sie sich wieder dem
Kapitän zu.

»Mylady, ich schwöre Ihnen, es ist nicht so, wie es scheint, bitte
glauben Sie mir. Sie verstehen mich völlig falsch, bitte lassen Sie
mich erklären.«

Froh, dass man hinter dem Fächer das feine Lächeln auf ihren Zügen
nicht sah, warf sie ihm einen Blick zu.

»Ich denke, ich verstehe recht gut, Lord Stokesbridge. Wollen wir
nun die Besichtigung fortsetzen oder möchten Sie mir etwas sagen?
In diesem Fall wäre ich natürlich ganz Ohr.«

Kokett klappte sie den Fächer zu und lächelte ihn an. Wortlos
setzten sie dann ihren Weg fort und erst, als er die nächsten
Besonderheiten der angepflanzten Flora entdeckte, fand er seine
Sprache wieder, immer noch unsicher, was er von dieser Sache halten
sollte. Die Lady aber schien sich auf dieser Führung bestens zu
amüsieren.

\bigpar

Schließlich verließen sie das hydroponische Deck und wollten sich
zum Tee in die Messe begeben, als ein Maat sie erreichte. Der noch
recht junge Mann salutierte, dann wanderte sein Blick unsicher
zwischen der Lady und seinem Kapitän hin und her. Schließlich
platzte diesem der Kragen.

»Nun reden Sie schon, Kerl, was ist los? Oder wollen Sie weiterhin
Löcher in die Luft starren? Gibt es ein Problem?«

Verunsichert salutierte der Maat noch einmal, dann senkte er den
Blick.

»Empfehlung von Mister Longbottom, Sir. Er bittet den Kapitän,
hinunter in den Frachtraum zu kommen, wenn es ihm beliebt. Der Ei
... der Professor ist ebenfalls da und man erwartet Sie. Sir. Im
Frachtraum, Sir. Wenn Sie es einrichten können, Sir. Natürlich
...«

Mit einer Handbewegung schnitt der Kapitän den Redefluss des Mannes
ab und schickte ihn wieder an seine Aufgaben. Was war denn nun
schon wieder, konnte man denn hier niemals seine Ruhe haben? Mit
einer bedauernden Geste wandte er sich zu seiner Begleiterin um.

»Ich bitte um Vergebung, Mylady, aber wie es scheint, werde ich von
meinem ersten Offizier erwartet. Soll ich Eure Zofe bitten, Euch in
die Kabine zu geleiten? Ich fürchte, ich kann Ihnen nun nicht wie
geplant beim Tee Gesellschaft leisten.«

Ihr unerwartetes silberhelles Lachen verblüffte ihn.

»Aber keineswegs, Lord Stokesbridge, ich würde Sie liebend gern
begleiten. Dann kann ich gleich mit eigenen Augen sehen, was der
Eierkopf wieder will und muss nicht zusehen, möglichst unauffällig
hinzuzukommen, meinen Sie nicht? Oder möchten Sie meine Begleitung
nicht? In diesem Fall würde ich mich natürlich in meine Kabine
zurückziehen, schließlich möchte ich Ihnen bei der Erfüllung ihrer
Aufgaben keinesfalls im Weg sein.«

Ein honigsüßes Lächeln lag auf ihrem Gesicht, aber in ihren Augen
blitzte der Schalk. Außerstande, ihr zu antworten, bot er ihr
seinen Arm und zusammen begaben sie sich in den Laderaum.

\bigpar

»Mister Longbottom, würden Sie die Güte haben, mir zu erklären,
warum Sie mich hier heruntergebeten haben?«

Die Worte des Kapitäns schnitten durch die hitzige Diskussion und
sorgten augenblicklich für Ruhe. Die Gruppe, bestehend aus dem
ersten Offizier, dem Maat zusammen mit dem Professor, dem er
zugeteilt war, zwei weiteren Wissenschaftlern, ein paar Matrosen
sowie dem Schiffszimmermann, stand um eine Kiste herum, die vor
einer Tür stand. Der Kapitän hätte schwören können, dass an diesem
Ort noch nie eine Tür gewesen war, aber er wartete die Antwort
seines ersten Offiziers ab. Dieser wandte sich ihm zu, dabei aber
immer wieder einen Blick zu dem Professor werfend, der vor
Aufregung feuerrot im Gesicht war.

»Nun, Lord Stokesbridge, sehen Sie, der Schiffszimmermann hat,
genau wie Sie es befohlen haben ...«

»Dieser Stümper, er hat alles ruiniert, wie sie sehen. Mein ganzer
Versuch, schauen Sie nur!«

Sir Saint-John drängelte sich vor, wild in Richtung der Tür
gestikulierend und sichtlich darum bemüht, seine Fassung
wiederzufinden. Der Kapitän wollte ihn gerade zurechtweisen, als
eine unerwartet scharfe Stimme erklang.

»Beruhigen Sie sich, Sir Saint-John, ansonsten lasse ich Sie von
Bord werfen. Sie haben genug Unsinn angerichtet für ein ganzes Heer
von Eierköpfen, versuchen Sie wenigstens ein Mal, sich wie ein
Gentleman zu benehmen. Immerhin sind Damen anwesend!«

Der Einwurf kam von Lady Farthington, die mit blitzenden Augen und
zornig in die Hüfte gestemmten Armen neben den Kapitän getreten
war. Die Mannschaft und die Wissenschaftler waren sprachlos, der
Professor schrak wie ein Kaninchen vor dem Donner zurück, nur ihre
Zofe grinste völlig undamenhaft. Lord Stokesbridge wollte etwas
sagen, aber sie ließ ihn nicht zu Wort kommen.

»Was ist nun wieder los, wollten Sie erneut ein Fenster
einschlagen? Nun reden Sie schon, Mann, sonst sind Sie doch auch
nicht um Worte verlegen.«

Der Professor war sichtlich geschockt.

»Aber, Mylady, wie reden Sie denn mit mir? Wissen Sie nicht, wer
ich bin? Ich muss doch sehr bitten ...«

»Ich weiß sehr gut, wer Sie sind, James. Ihr Vater hat sich um das
Empire sehr verdient gemacht, deswegen hat man Sie viel zu lange
gewähren lassen, mit Ihrem Unsinn. Hat Ihnen Ehren und Posten
verliehen, für welche Sie ganz offensichtlich nicht geeignet sind,
nur hat sich nie jemand getraut, das auszusprechen, weil sie durch
Ihren Vater in der Gunst Ihrer Majestät standen. Aber wenn wir
wieder zurück sind, wird es damit vorbei sein, mein Guter. Ich habe
Sie lange verteidigt, aber damit ist nun Schluss. Walters, Sie
übernehmen bitte ab sofort die wissenschaftliche Leitung. Nun
schauen Sie nicht so überrascht, meinen Sie wirklich, mir sei nicht
aufgefallen, dass alle brauchbaren Ergebnisse ohnehin von Ihnen
stammen? Ich erwarte dann allerdings von Ihnen, dass Sie das von
diesem Stümper angerichtete Chaos wieder zurechtbiegen können. Na
also, das wäre damit geregelt. Wo waren wir gerade? Ach ja, Sie
hatten eine Frage, Lord Stokesbridge. Bitte, lassen Sie sich nicht
mehr stören und verzeihen Sie mir diese kleine Unterbrechung.«

Alle Augen starrten sie an, ungläubig, bis es dem Kapitän als
Erstem gelang, seine Fassung zurückzugewinnen.

»Oh, ja, danke, Mylady. Nun, Mister Longbottom, Ihr Bericht.«

Der Angesprochene zuckte zusammen, dann richtete er sich auf und
deutete auf die rätselhafte Tür.

»Sir, wie Sie befohlen haben, hat der Schiffszimmermann eine Lösung
gefunden, Sie stehen davor. Durch ein System von zwei Türen, die
luftdicht schließen, ist es möglich, in einem abgetrennten Raum
Dinge dem Æther auszusetzen, ohne Schiff und Mannschaft zu
gefährden. Zumindest versicherte er mir, dass es genau so
funktionieren wird. Ein Test steht noch aus, wir dachten uns, Sie
wären gerne dabei, Sir. Doch der Professor bestand darauf, nicht zu
warten, was zu dem von Ihnen beobachteten Disput führte.«

Der Kapitän starrte ungläubig auf die Tür. Vorsichtig streckte er
die Hand aus, strich behutsam über das Holz und sah dann den
Schiffszimmermann an.

»Und das ... das ist sicher? Keine Gefährdung für das Schiff? Sagte
ich nicht, eigentlich, dass ich vorher über Umbauten informiert
werden möchte, wenn sie den Rumpf betreffen?«

Der muskulöse Zimmermann zuckte mit den Achseln.

»Es ist durchaus möglich, dass ich diesen Teil Ihres Befehls im
Eifer der Planung vergessen habe, Sir. Aber um den Rest Ihrer
Anordnung umzusetzen, musste ich natürlich eine Öffnung nach
draußen schaffen. Und da bisher nichts passiert ist, gehe ich davon
aus, dass meine Konstruktion sicher ist, denken Sie nicht auch,
Sir?«

Diese Antwort verschlug Lord Stokesbridge geradezu den Atem.
\emph{Er ging davon aus}? Aber gerade, als er zu einer
geharnischten Antwort ansetzen wollte, fiel ihm Lady Farthington
erneut mit ihrem Lachen ins Wort. Entrüstet wandte er sich ihr zu,
aber wusste dann nichts zu sagen. Erschüttert musste er mit
ansehen, wie die Lady verzweifelt darum rang, des Gelächters Herr
zu werden, aber kläglich scheiterte. Wobei er zugeben musste, dass
er ihr Lachen gern hörte, es hatte eine erfrischende Note. Und ihre
Grübchen, die sich dabei bildeten, ganz zauberhaft. Fast so
bezaubernd wie der Schwung ihrer Nase, die ... räuspernd rief er
sich ins Gedächtnis zurück, wo er sich befand.

»Nun, wenn sie davon ... ausgehen, dass nichts passiert, sollten
wir vielleicht einen Test durchführen, meinen Sie nicht?«

Sein erster Offizier reagierte überrascht, er hatte mit einem
Wutanfall seines Kapitäns gerechnet. Das waren ganz neue,
unerwartete Seiten. Schnell hatte er jedoch beschlossen, diese
Wendung der Dinge auszunutzen. Er wusste um die Fähigkeiten des
Schiffszimmermannes, der zwar ein wenig verschroben war, aber ein
enormes Wissen und ein fast schon unheimliches Gespür für das
Schiff hatte. Es hätte ihm Leid getan, den Mann bestrafen zu
müssen, auch wenn er selbstredend ganz und gar außerhalb seiner
Kompetenzen gehandelt hatte.

»Natürlich, Lord Stokesbridge, zu Befehl. Schiffszimmermann, würden
Sie dann bitte Ihre Konstruktion testen? Und ... sollten wir ein
wenig Abstand halten, nur zur Sicherheit?«

Der Angesprochene öffnete die Tür, schaute kurz in den
dahinterliegenden Raum in dem eine weitere Tür zu erkennen war,
nickte dann, und wandte sich dem ersten Offizier zu.

»Das können Sie natürlich gerne tun, aber erstens sähen Sie dann
nicht mehr allzu viel und zweitens ... nun, wenn der Test Fehler in
der Konstruktion zeigt, nützt ein Abstand auch nichts, dann werden
wir alle dem Æther ausgesetzt sein. Lincoln, her zu mir, aber
schnell!«

Während der Kapitän erneut mit einer Entgegnung rang, ertönte ein
stampfendes Klirren und ein Matrose in einem seltsamen Aufzug trat
näher. Den Kopf bedeckte ein kugelförmiger Helm, ansonsten steckte
der Mann in einem seltsam bauschigen Anzug mit einem merkwürdigen
Tornister auf dem Rücken, aus dem Schläuche zum Helm führten.
Wirklich wohl fühlte er sich nicht, das zeigte bereits ein Blick in
sein Gesicht. Doch der Schiffszimmermann ignorierte das stille
Flehen und schob ihn in die kleine Kammer hinter der Tür.

»Wie Sie sehen, habe ich hier einen kleinen, luftdichten Raum
konstruiert, der über zwei Türen verfügt. Die eine halte ich
derzeit offen, sie führt hier in den Laderaum. Die andere Tür führt
direkt in den Æther hinaus. Da es furchtbar unklug wäre, wenn unser
Freiwilliger hier zum jetzigen Zeitpunkt die andere Tür öffnen
würde, wird er es bitte unterlassen. Ich denke, hier hat niemand
ein Interesse daran, unserem Schöpfer bereits jetzt
gegenüberzutreten. Folglich werden wir erst die Tür schließen, die
in den Laderaum führt, danach wird auf ein Klopfen von mir die
andere Tür geöffnet. Wenn meine Konstruktion funktioniert, wird der
Æther in den kleinen Raum strömen, dabei die dort in dem Raum
befindliche Luft hinaus aus dem Schiff saugen. Die Außentür wird
dann geschlossen, sodass der Raum mit Æther gefüllt ist. Öffnen wir
nun die Tür in den Laderaum, strömt die Atemluft wieder in den Raum
und alles ist wieder wie vorher. Faszinierend, meinen Sie nicht?
Nun denn, lassen Sie uns beginnen. Ich werde nun die Innentür
schließen.«

Bevor der Kapitän das Gesagte richtig verinnerlichen und etwas
entgegnen konnte, war die Tür geschlossen und ein Klopfen leitete
den nächsten Teil des Experimentes ein. Vorsichtig betätigte der
Matrose den Öffnungsmechanismus und drückte die Tür auf. Nichts
schien zu geschehen, aber der Schiffszimmermann nickte zufrieden.
Dann rüttelte er an der Tür zum Laderaum, doch zur Beruhigung der
Zuschauer ließ die Tür sich nicht öffnen.

»Aber was machen Sie denn da, sind Sie denn wahnsinnig geworden?
Sie können doch nicht einfach ...»

Der Schiffszimmermann zuckte mit den Achseln.

»Nur ein Experiment, Sir. Ich war mir ziemlich sicher, dass es mir
nicht gelingen würde, die Tür zu öffnen, wenn der Raum mit Æther
gefüllt wäre. Kein Grund also zur Besorgnis.«

»Ziemlich sicher? Und wenn es doch möglich gewesen wäre?« Der erste
Offizier war nicht geneigt, diese Aussage ohne Nachfrage zu
akzeptieren.

»Dann wäre meine Theorie falsch gewesen und wir alle wären dem
Æther ausgesetzt worden. Dumme Sache, das. Tödlich vor allem. Gut,
dass meine Theorie nicht falsch war, oder wie sehen Sie das, Sir?«

Ungerührt klopfte der Schiffszimmermann erneut und der Matrose
schloss die Außentür. Nach einer kurzen Wartezeit wurde die
Innentür erneut geöffnet und der recht bleich gewordene Mann trat
mit unziemlich scheinender Hast aus dem Raum hinaus. Der
Schiffszimmermann aber wandte sich zufrieden aussehend dem Kapitän
zu und salutierte.

»Sir, melde den Test als gelungen, meine Konstruktion funktioniert
besser als erwartet. Wir können also mit den Experimenten beginnen,
wenn Sie das wünschen. Wenn Sie gestatten, würde ich aber nun gerne
etwas essen, derartige Vorhaben regen meinen Appetit an.

Wortlos gab Lord Stokesbridge seine Zustimmung, dann drehte er sich
um und blickte seinen ersten Offizier an.

»Sie haben das Kommando, Mister Longbottom. Ich würde mich nun gern
etwas zurückziehen, die Vorführung war doch etwas ... anstrengend.
Begleiten Sie mich zum Tee, Lady Farthington? Das wäre wunderbar,
vielen Dank.«

Arm in Arm gingen die beiden davon und der erste Offizier sah ihnen
hinterher. Wie es schien, würden in Kürze neue Zeiten auf der
\emph{Ikarus} anbrechen. Und er war sich sicher, dass es zwar
aufregendere, aber auch bessere Zeiten sein würden. Vor allem für
ihren Kapitän.

\end{document}
