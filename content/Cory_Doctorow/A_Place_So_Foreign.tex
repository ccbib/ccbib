\hyphenation{mo-no-poly car-ne-gie pro-ject pro-gress mo-dem rou-lette
  browse-wrap Use-net mon-as-tery mo-dems}
\hyphenation{co-me-dic polt-roon stove-pipe Ma-dame scru-ta-ble star-tling}
\hyphenation{heal-thily lim-ou-sines wrest-lers tan-trum push-over un-asked
  bras-siere bro-th-er}
\hyphenation{Can-a-da Fred-rick teen-agers wrest-ler Cha-vez Tho-mas 
  a-nom-a-lies sur-veil-lance ar-mies ref-u-gee ref-u-gees bris-tling
  eve-ning man-chu-ria man-chu-ri-an mid-terms me-di-um jap-a-nese}
\hyphenation{spend-ers googl-ing tour-ist tour-ists leg-end-ary}
\hyphenation{Dan-iel Van-essa Doc-to-row Ste-phen-son}
\hyphenation{de-cade sur-veilled rout-ers Wol-fen-stein teen-ager to-night}
\hyphenation{his-to-gram an-o-nym-ize Ga-la-xy sym-pa-the-tic}
\hyphenation{ar-phid ar-phids Found-ers}
\hyphenation{stran-ger stran-gers shoul-der-blades dump-ling dump-lings}
\hyphenation{ice-pack guard-rail Sep-tem-ber boot-able e-co-nom-ist}
\hyphenation{grown-ups roos-ter shoe-laces li-quid-i-ty}
\hyphenation{side-arm}
\hyphenation{wo-man wo-men tan-trum tan-trums Le-nin-grad zom-bie bunk-house}
\hyphenation{up-tick bio-mass}
\hyphenation{of-fi-cial of-fi-cial-ly gov-ern-ment}
\hyphenation{heal-thy Or-ville spark-ling}
\hyphenation{ves-ti-bule Law-rence au-to-no-mous}
\hyphenation{sau-sage door-step staf-fer}
\hyphenation{tree-trunk}
\hyphenation{to-ron-to}
\hyphenation{qua-dril-lion-aire qua-dril-lion-aires}
\hyphenation{sports-jack-et sports-jack-ets}
\hyphenation{work-space skunk-works}
\hyphenation{kings-ton}


%\newenvironment{sign}{\begin{center}\scshape}{\end{center}}
\newenvironment{authorof}{\begin{flushright}\sffamily}{\end{flushright}}

\begin{document}
\begin{center}
\textbf{\huge\textsf{{A Place So Foreign}}}
\end{center}

%\setlength{\emergencystretch}{1ex}

From ``A Place So Foreign and Eight More,'' a short story
collection published in September, 2003 by Four Walls Eight Windows
Press (ISBN 1568582862). See http://craphound.com/place for more.

Originally Published in Science Fiction Age, January 2000


\section{Blurbs and quotes:}

\begin{itemize}
\item
  Cory Doctorow straps on his miner's helmet and takes you deep into
  the caverns and underground rivers of Pop Culture, here filtered
  through SF-coloured glasses. Enjoy.

  \begin{authorof}
    Neil Gaiman Author of American Gods and Sandman
  \end{authorof}
\item
  Few writers boggle my sense of reality as much as Cory Doctorow.
  His vision is so far out there, you'll need your GPS to find your
  way back.

  \begin{authorof}
    David Marusek Winner of the Theodore Sturgeon Award, Nebula Award
    nominee
  \end{authorof}
\item
  Cory Doctorow is one of our best new writers: smart, daring, savvy,
  entertaining, ambitious, plugged-in, and as good a guide to the
  wired world of the twenty-first century that stretches out before
  us as you're going to find.

  \begin{authorof}
    Gardner Dozois Editor, Asimov's SF
  \end{authorof}
\item
  He sparkles! He fizzes! He does backflips and breaks the furniture!
  Science fiction needs Cory Doctorow!

  \begin{authorof}
    Bruce Sterling Author of The Hacker Crackdown and Distraction
  \end{authorof}
\item
  Cory Doctorow strafes the senses with a geekspeedfreak explosion of
  gomi kings with heart, weirdass shapeshifters from Pleasure Island
  and jumping automotive jazz joints. If this is Canadian science
  fiction, give me more.

  \begin{authorof}
    Nalo Hopkinson Author of Midnight Robber and Brown Girl in the Ring
  \end{authorof}
\item
  Cory Doctorow is the future of science fiction. An nth-generation
  hybrid of the best of Greg Bear, Rudy Rucker, Bruce Sterling and
  Groucho Marx, Doctorow composes stories that are as BPM-stuffed as
  techno music, as idea-rich as the latest issue of NEW SCIENTIST,
  and as funny as humanity's efforts to improve itself. Utopian,
  insightful, somehow simultaneously ironic and heartfelt, these nine
  tales will upgrade your basal metabolism, overwrite your cortex
  with new and efficient subroutines and generally improve your life
  to the point where you'll wonder how you ever got along with them.
  Really, you should need a prescription to ingest this book. Out of
  all the glittering crap life and our society hands us, craphound
  supreme Doctorow has managed to fashion some industrial-grade
  art."

  \begin{authorof}
    Paul Di Filippo Author of The Steampunk Trilogy
  \end{authorof}
\item
  As scary as the future, and twice as funny. In this eclectic and
  electric collection Doctorow strikes sparks off today to illuminate
  tomorrow, which is what SF is supposed to do. And nobody does it
  better.

  \begin{authorof}
    Terry Bisson Author of Bears Discover Fire
  \end{authorof}
\end{itemize}

\section{A note about this story}

This story is from my collection,
``A Place So Foreign and Eight More,'' published by Four Walls
Eight Windows Press in September, 2003, ISBN 1568582862. I've
released this story, along with five others, under the terms of a
Creative Commons license that gives you, the reader, a bunch of
rights that copyright normally reserves for me, the creator.

I recently did the same thing with the entire text of my novel,
\href{http://craphound.com/down}{``Down and Out in the Magic Kingdom''},
and it was an unmitigated success. Hundreds of thousands of people
downloaded the book --- good news --- and thousands of people
bought the book --- also good news. It turns out that, as near as
anyone can tell, distributing free electronic versions of books is
a great way to sell more of the paper editions, while
simultaneously getting the book into the hands of readers who would
otherwise not be exposed to my work.

I still don't know how it is artists will earn a living in the age
of the Internet, but I remain convinced that the way to find out is
to do basic science: that is, to do stuff and observe the outcome.
That's what I'm doing here. The thing to remember is that the very
\emph{worst} thing you can do to me as an artist is to not read my
work --- to let it languish in obscurity and disappear from
posterity. Most of the fiction I grew up on is out-of-print, and
this is doubly true for the short stories. Losing a couple bucks to
people who would have bought the book save for the availability of
the free electronic text is no big deal, at least when compared to
the horror that is being irrelevant and unread. And luckily for me,
it appears that giving away the text for free gets me more paying
customers than it loses me.

You can find the canonical version of this file at\\
\texttt{http://craphound.com/place/download.php}

If you'd like to convert this file to some other format and
distribute it, you have my permission, provided that:

\begin{itemize}
\item
  You don't charge money for the distribution

\item
  You keep the entire text intact, including this notice, the license
  below, and the metadata at the end of the file

\item
  You don't use a file-format that has ``DRM'' or ``copy-protection''
  or any other form of use-restriction turned on

\end{itemize}
If you'd like, you can advertise the existence of your edition by
posting a link to it at http://craphound.com/place/000013.php

\subsection{Here's a summary of the license:}

\begin{verbatim}
http://creativecommons.org/licenses/by-nd-nc/1.0

Attribution. The licensor permits others to copy, distribute,
display, and perform the work. In return, licensees must give the
original author credit.

No Derivative Works. The licensor permits others to copy,
distribute, display and perform only unaltered copies of the work
-- not derivative works based on it.

Noncommercial. The licensor permits others to copy, distribute,
display, and perform the work. In return, licensees may not use
the work for commercial purposes -- unless they get the
licensor's permission.
\end{verbatim}

\subsection{And here's the license itself:}

\begin{verbatim}
http://creativecommons.org/licenses/by-nd-nc/1.0-legalcode

THE WORK (AS DEFINED BELOW) IS PROVIDED UNDER THE TERMS OF THIS
CREATIVE COMMONS PUBLIC LICENSE ("CCPL" OR "LICENSE"). THE WORK
IS PROTECTED BY COPYRIGHT AND/OR OTHER APPLICABLE LAW. ANY USE OF
THE WORK OTHER THAN AS AUTHORIZED UNDER THIS LICENSE IS
PROHIBITED.

BY EXERCISING ANY RIGHTS TO THE WORK PROVIDED HERE, YOU ACCEPT
AND AGREE TO BE BOUND BY THE TERMS OF THIS LICENSE. THE LICENSOR
GRANTS YOU THE RIGHTS CONTAINED HERE IN CONSIDERATION OF YOUR
ACCEPTANCE OF SUCH TERMS AND CONDITIONS.

1. Definitions

    a. "Collective Work" means a work, such as a periodical issue,
    anthology or encyclopedia, in which the Work in its entirety in
    unmodified form, along with a number of other contributions,
    constituting separate and independent works in themselves, are
    assembled into a collective whole. A work that constitutes a
    Collective Work will not be considered a Derivative Work (as
    defined below) for the purposes of this License.

    b. "Derivative Work" means a work based upon the Work or upon the
    Work and other pre-existing works, such as a translation, musical
    arrangement, dramatization, fictionalization, motion picture
    version, sound recording, art reproduction, abridgment,
    condensation, or any other form in which the Work may be recast,
    transformed, or adapted, except that a work that constitutes a
    Collective Work will not be considered a Derivative Work for the
    purpose of this License.

    c. "Licensor" means the individual or entity that offers the Work
    under the terms of this License.

    d. "Original Author" means the individual or entity who created
    the Work.

    e. "Work" means the copyrightable work of authorship offered
    under the terms of this License.

    f. "You" means an individual or entity exercising rights under
    this License who has not previously violated the terms of this
    License with respect to the Work, or who has received express
    permission from the Licensor to exercise rights under this
    License despite a previous violation.

2. Fair Use Rights. Nothing in this license is intended to
reduce, limit, or restrict any rights arising from fair use,
first sale or other limitations on the exclusive rights of the
copyright owner under copyright law or other applicable laws.

3. License Grant. Subject to the terms and conditions of this
License, Licensor hereby grants You a worldwide, royalty-free,
non-exclusive, perpetual (for the duration of the applicable
copyright) license to exercise the rights in the Work as stated
below:

    a. to reproduce the Work, to incorporate the Work into one or
    more Collective Works, and to reproduce the Work as incorporated
    in the Collective Works;

    b. to distribute copies or phonorecords of, display publicly,
    perform publicly, and perform publicly by means of a digital
    audio transmission the Work including as incorporated in
    Collective Works;

The above rights may be exercised in all media and formats
whether now known or hereafter devised. The above rights include
the right to make such modifications as are technically necessary
to exercise the rights in other media and formats. All rights not
expressly granted by Licensor are hereby reserved.

4. Restrictions. The license granted in Section 3 above is
expressly made subject to and limited by the following
restrictions:

    a. You may distribute, publicly display, publicly perform, or
    publicly digitally perform the Work only under the terms of this
    License, and You must include a copy of, or the Uniform Resource
    Identifier for, this License with every copy or phonorecord of
    the Work You distribute, publicly display, publicly perform, or
    publicly digitally perform. You may not offer or impose any terms
    on the Work that alter or restrict the terms of this License or
    the recipients' exercise of the rights granted hereunder. You may
    not sublicense the Work. You must keep intact all notices that
    refer to this License and to the disclaimer of warranties. You
    may not distribute, publicly display, publicly perform, or
    publicly digitally perform the Work with any technological
    measures that control access or use of the Work in a manner
    inconsistent with the terms of this License Agreement. The above
    applies to the Work as incorporated in a Collective Work, but
    this does not require the Collective Work apart from the Work
    itself to be made subject to the terms of this License. If You
    create a Collective Work, upon notice from any Licensor You must,
    to the extent practicable, remove from the Collective Work any
    reference to such Licensor or the Original Author, as requested.

    b. You may not exercise any of the rights granted to You in
    Section 3 above in any manner that is primarily intended for or
    directed toward commercial advantage or private monetary
    compensation. The exchange of the Work for other copyrighted
    works by means of digital file-sharing or otherwise shall not be
    considered to be intended for or directed toward commercial
    advantage or private monetary compensation, provided there is no
    payment of any monetary compensation in connection with the
    exchange of copyrighted works.

    c. If you distribute, publicly display, publicly perform, or
    publicly digitally perform the Work or any Collective Works, You
    must keep intact all copyright notices for the Work and give the
    Original Author credit reasonable to the medium or means You are
    utilizing by conveying the name (or pseudonym if applicable) of
    the Original Author if supplied; the title of the Work if
    supplied. Such credit may be implemented in any reasonable
    manner; provided, however, that in the case of a Collective Work,
    at a minimum such credit will appear where any other comparable
    authorship credit appears and in a manner at least as prominent
    as such other comparable authorship credit.

5. Representations, Warranties and Disclaimer

    a. By offering the Work for public release under this License,
    Licensor represents and warrants that, to the best of Licensor's
    knowledge after reasonable inquiry:

        i. Licensor has secured all rights in the Work necessary to grant
        the license rights hereunder and to permit the lawful exercise of
        the rights granted hereunder without You having any obligation to
        pay any royalties, compulsory license fees, residuals or any
        other payments;

        ii. The Work does not infringe the copyright, trademark,
        publicity rights, common law rights or any other right of any
        third party or constitute defamation, invasion of privacy or
        other tortious injury to any third party.

    b. EXCEPT AS EXPRESSLY STATED IN THIS LICENSE OR OTHERWISE AGREED
    IN WRITING OR REQUIRED BY APPLICABLE LAW, THE WORK IS LICENSED ON
    AN "AS IS" BASIS, WITHOUT WARRANTIES OF ANY KIND, EITHER EXPRESS
    OR IMPLIED INCLUDING, WITHOUT LIMITATION, ANY WARRANTIES
    REGARDING THE CONTENTS OR ACCURACY OF THE WORK.

6. Limitation on Liability. EXCEPT TO THE EXTENT REQUIRED BY
APPLICABLE LAW, AND EXCEPT FOR DAMAGES ARISING FROM LIABILITY TO
A THIRD PARTY RESULTING FROM BREACH OF THE WARRANTIES IN SECTION
5, IN NO EVENT WILL LICENSOR BE LIABLE TO YOU ON ANY LEGAL THEORY
FOR ANY SPECIAL, INCIDENTAL, CONSEQUENTIAL, PUNITIVE OR EXEMPLARY
DAMAGES ARISING OUT OF THIS LICENSE OR THE USE OF THE WORK, EVEN
IF LICENSOR HAS BEEN ADVISED OF THE POSSIBILITY OF SUCH DAMAGES.

7. Termination

    a. This License and the rights granted hereunder will terminate
    automatically upon any breach by You of the terms of this
    License. Individuals or entities who have received Collective
    Works from You under this License, however, will not have their
    licenses terminated provided such individuals or entities remain
    in full compliance with those licenses. Sections 1, 2, 5, 6, 7,
    and 8 will survive any termination of this License.

    b. Subject to the above terms and conditions, the license granted
    here is perpetual (for the duration of the applicable copyright
    in the Work). Notwithstanding the above, Licensor reserves the
    right to release the Work under different license terms or to
    stop distributing the Work at any time; provided, however that
    any such election will not serve to withdraw this License (or any
    other license that has been, or is required to be, granted under
    the terms of this License), and this License will continue in
    full force and effect unless terminated as stated above.

8. Miscellaneous

    a. Each time You distribute or publicly digitally perform the
    Work or a Collective Work, the Licensor offers to the recipient a
    license to the Work on the same terms and conditions as the
    license granted to You under this License.

    b. If any provision of this License is invalid or unenforceable
    under applicable law, it shall not affect the validity or
    enforceability of the remainder of the terms of this License, and
    without further action by the parties to this agreement, such
    provision shall be reformed to the minimum extent necessary to
    make such provision valid and enforceable.

    c. No term or provision of this License shall be deemed waived
    and no breach consented to unless such waiver or consent shall be
    in writing and signed by the party to be charged with such waiver
    or consent.

    d. This License constitutes the entire agreement between the
    parties with respect to the Work licensed here. There are no
    understandings, agreements or representations with respect to the
    Work not specified here. Licensor shall not be bound by any
    additional provisions that may appear in any communication from
    You. This License may not be modified without the mutual written
    agreement of the Licensor and You.
\end{verbatim}

\section{A Place So Foreign}

My Pa disappeared somewhere in the wilds of 1975, when I was just
fourteen years old. He was the Ambassador to 1975, but back home in
1898, in New Jerusalem, Utah, they all thought he was Ambassador to
France. When he disappeared, Mama and I came back through the
triple-bolted door that led from our apt in 1975 to our horsebarn
in 1898. We returned to the dusty streets of New Jerusalem, and I
had to keep on reminding myself that I was supposed to have been in
France, and ``polly-voo'' for my chums, and tell whoppers about the
Eiffel Tower and the fancy bread and the snails and frogs we'd
eaten.

I was born in New Jerusalem, and raised there till I was ten. Then,
one summer's day, my Pa sat me on his knee and told me we'd be
going away for a while, that he had a new job.

``But what about the store?'' I said, scandalised. My Pa's
wonderful store, the only General Store in town not run by the
Saints, was my second home. I'd spent my whole life crawling and
then walking on the dusty wooden floors, checking stock and
unpacking crates with waybills from exotic places like Salt Lake
City and even San Francisco.

Pa looked uncomfortable. ``Mr Johnstone is buying it.''

My mouth dropped. James H Johnstone was as dandified a city-slicker
as you'd ever hope to meet. He'd blown into town on the weekly
Zephyr Speedball, and skinny Tommy Benson had hauled his three huge
steamer trunks to the cowboy hotel. He'd tipped Tommy two dollars,
in Wells-Fargo notes, and later, in the empty lot behind the
smithy, all the kids in New Jerusalem had gathered 'round Tommy to
goggle at the small fortune in queer, never-seen bills.

``Pa, no!'' I said, without thinking. I knew that if my chums
ordered their fathers around like that, they'd get a whipping, but
my Pa almost never whipped me.

He smiled, and stretched his thick moustache across his face.
``James, I know you love the store, but it's already been decided. Once you've 
been to France, you'll see that it has wonders that beat anything that store 
can deliver.''

``Nothing's better than the store,'' I said.

He laughed and rumpled my hair.
``Don't be so sure, son. There are more things in heaven and earth then are 
dreamed of in your philosophy.''
It was one of his sayings, from Shakespeare, who he'd studied back
east, before I was born. It meant that the discussion was closed.

I decided to withhold judgement until I saw France, but still
couldn't shake the feeling that my Pa was going soft in the head.
Mr Johnstone wasn't fit to run an apple-cart. He was short and
skinny and soft, not like my Pa, who, as far as I was concerned,
was the biggest, strongest man in the whole world. I loved my Pa.

\tb

Well, when we packed our bags and Pa went into the horsebarn to
hitch up our team, I figured we'd be taking a short trip out to the
train station. All my chums were waiting there to see us off, and
I'd promised my best pal Oly Sweynsdatter that I'd give him my
coonskin cap to wear until we came back. But instead, Pa rode us to
the edge of town, where the road went to rutted trail and salt
flats, and there was Mr James H Johnstone, in his own fancy-pants
trap. Pa and me moved our luggage into Johnstone's trap and got
inside with Mama and hunkered down so, you couldn't see us from
outside. Mama said,
``You just hush up now, James. There's parts of this trip that we couldn't tell 
you about before we left, but you're going to have to stay quiet and hold onto 
your questions until we get to where we're going.''

I nearly said, ``To where we're going?'' but I didn't, because Mama
had never looked so serious in all my born days. So I spent an hour
hunkered down in there, listening to the clatter of the wheels and
trying to guess where we were going. When I heard the trap stop and
a set of wooden doors close, all my guesses dried up and blew away,
because I couldn't think of anywhere we would've heard those sounds
out in the desert.

So imagine my surprise when I stood up and found us right in our
very own horsebarn, having made a circle around town and back to
where we'd started from! Mama held a finger up to her lips and then
took Mr Johnstone's soft, girlish hand as he helped her down from
the trap.

My Pa and Mr Johnstone started shifting one of the piles of
hay-bales that stacked to the rafters, until they had revealed a
triple-bolted door that looked new and sturdy, fresh-sawn edges
still bright and yellow, and not the weathered brown of the rest of
the barn.

Pa took a key ring out of his vest pocket and unlocked the door,
then swung it open. Each of us shouldered our bags and walked
through, in eerie silence, into a pitch black room.

Pa reached out and pulled the door shut, then there was a sharp
click and we were in 1975.

\tb

1975 was a queer sight. Our apartment was a lozenge of silver,
spoked into the hub of a floating null-gee doughnut. Pa did
something fancy with his hands and the walls went transparent, and
I swear, I dropped to the floor and hugged the nubby rubber tiles
for all I was worth. My eyes were telling me that we were hundreds
of yards off the ground, and while I'd jumped from the rafters of
the horsebarn into the hay countless times, I suddenly discovered
that I was afraid of heights.

After that first dizzying glimpse of 1975, I kept my eyes squeezed
shut and held on for all I was worth. After a minute or two of
this, my stomach told me that I wasn't falling, and I couldn't hear
any rushing wind, any birdcalls, anything except Mama and Pa
laughing, fit to bust. I opened one eye and snuck a peek. My folks
were laughing so hard they had to hold onto each other to stay up,
and they were leaning against thin air, Pa's back pressed up
against nothing at all.

Cautiously, I got to my feet and walked over to the edge. I
extended one finger and it bumped up against an invisible wall,
cool and smooth as glass in winter.

``James,'' said my Pa, smiling so wide that his thick moustache
stretched all the way across his face, ``welcome to 1975.''

\tb

Pa's ambassadorial mission meant that he often spent long weeks
away from home, teleporting in only for Sunday dinner, the stink of
aliens and distant worlds clinging to him even after he washed up.
The last Sunday dinner I had with him, Mama had made mashed
potatoes and corn bread and sausage gravy and turkey, spending the
whole day with the wood-fired cooker back in 1898 (actually, it was
1901 by then, but I always thought of it as 1898). She'd moved the
cooker into the horsebarn after a week of wrestling with the
gadgets we had in our 1975 kitchen, and when Pa had warned her that
the smoke was going to raise questions in New Jerusalem, she
explained that she was going to run some flexible exhaust hose
through the door into 75 and into our apt's air-scrubber. Pa had
shook his head and smiled at her, and every Sunday, she dragged the
exhaust pipe through the door.

That night, Pa sat down and said grace, and he was in his
shirtsleeves with his suspenders down, and it almost felt like home
--- almost felt like a million Sunday dinners eaten by gaslight,
with a sweaty pitcher of lemonade in the middle of the table, and
seasonal wildflowers, and a stinky cheroot for Pa afterwards as he
tipped his chair back and rested one hand on his belly, as if he
couldn't believe how much Mama had managed to stuff him this time.

``How are your studies coming, James?'' he asked me, when the
robutler had finished clearing the plates and clattered away into
its nook.

``Very well, sir. We're starting calculus now.'' Truth be told, I
hated calculus, hated Isaac Newton and asymptotes and the whole
smelly business. Even with the viral learning shots, it was like
swimming in molasses for me.

``Calculus! Well, well, well ---'' this was one of Pa's catch-all
phrases, like ``How \emph{about} that?'' or ``What do you know?''
``Well, well, well. I can't believe how much they stuff into kids' heads here.''

``Yes, sir. There's an awful lot left to learn, yet.'' We did a
subject every two weeks. So far, I'd done French, Molecular and
Cellular Biology, Physics and Astrophysics, Esperanto, Cantonese
and Mandarin, and an alien language whose name translated as
``Standard.'' I'd been exempted from History, of course, along with
the other kids there from the past --- the Chinese girl from the
Ming Dynasty, the Roman boy, and the Injun kid from South America.

Pa laughed around his cigar and crossed his legs. His shoes were so
big, they looked like canoes.
``There surely is, son. There surely is. And how are you doing with your 
classmates? Any tussles your teacher will want to talk to me about?''

``No, sir! We're friendly as all get-out, even the girls.'' The
kids in 75 didn't even notice what they were doing in school. They
just sat down at their workstations and waited to have their brains
filled with whatever was going on, and left at three, and never
complained about something being too hard or too dull.

``That's good to hear, son. You've always been a good boy. Tell you what: you 
bring home a good report this Christmas, and I'll take you to see Saturn's 
rings on vacation.''

Mama shot him a look then, but he pretended he didn't see it. He
stubbed out his cigar, hitched up his suspenders, and put on his
tailcoat and tophat and ambassadorial sash and picked up his
leather case.

``Good night, son. Good night, Ulla. I'll see you on Wednesday,''
he said, and stepped into the teleporter.

That was the last time I ever saw him.

\tb

``He died from bad snails?'' Oly Sweynsdatter said to me, yet
again.

I balled up a fist and stuck it under his nose.
``For the last time, yes. Ask me again, and I'll feed you this.''

I'd been back for a month, and in all that time, Oly had skittered
around me like a shy pony, always nearby but afraid to talk to me.
Finally, I'd grabbed him and shook him and told him not to be such
a ninny, tell me what was on his mind. He wanted to know how my Pa
had died, over in France. I told him the reason that Mama and Mr
Johnstone and the man from the embassy had worked out together.
Now, I regretted it. I couldn't get him to shut up.

``Sorry, all right, sorry!'' he said, taking a step backwards. We
were in the orchard behind the schoolyard, chucking rotten apples
at the tree-trunks to watch them splatter.
``Want to hear something?''

``Sure,'' I said.

``Tommy Benson's sweet on Marta Helprin. It's disgusting. They hold hands --- 
\emph{in church}! None of the fellows will talk to him.''

I didn't see what the big deal was. Back in 75, we had had a
two-week session on sexual reproduction, like all the other
subjects. Most of the kids there were already in couples, sneaking
off to low-gee bounceataria and renting private cubes with
untraceable cash-tokens. I'd even tussled with one girl, Katebe
M'Buto, another exchange student, from United Africa Trading
Sphere. I'd picked her up at her apt, and her father had even
shaken my hand --- they grow up fast in UATS. Of course, I'd never
let on to my folks. Pa would've broken an axle.
``That's pretty disgusting, all right,'' I said, unconvincingly.

``You want to go down to the river? I told Amos and Luke that I'd meet them 
after lunch.''

I didn't much feel like it, but I didn't know what else to do. We
walked down to the swimming hole, where some boys were already
naked, swimming and horsing around. I found myself looking away,
conscious of their nudity in a way that I'd never been before ---
all the boys in town swam there, all summer long.

I turned my back to the group and stripped down, then ran into the
water as quick as I could.

I paddled around a little, half-heartedly, and then I found myself
being pulled under! My sinuses filled with water and I yelled a
stream of bubbles, and closed my mouth on a swallow of water.
Strong hands pulled at my ankles. I kicked out as hard as I could,
and connected with someone's head. The hands loosened and I shot up
like a cork, sputtering and coughing. I ran for the shore, and saw
one of the Allen brothers surfacing, rubbing at his head and
laughing. The four Allen boys lived on a ranch with their parents
out by the salt flats, and we only saw them when they came into
town with their folks for supplies. I'd never liked them, but now,
I saw red.

``You pig!'' I shouted at him.
``You stupid, rotten, pig! What the heck do you think you were doing?''

The Allens kept on laughing --- I used to know some of their names,
but in the time I'd been in 75, they'd grown as indistinguishable
as twins: big, hard boys with their heads shaved for lice. They
pointed at me and laughed. I scooped up a flat stone from the shore
and threw it at the head of the one who'd pulled me under, as hard
as I could.

Lucky for him --- and me! --- I was too angry to aim properly, and
the stone hit him in the shoulder, knocking him backwards. He
shouted at me --- it was like a roar of a wild animal --- and the
four brothers charged.

Oly appeared at my side. ``Run!'' he shouted.

I was too angry. I balled my fists and stood my ground. The first
one shot out of the water towards me, and punched me so hard in the
guts, I saw stars. I fell to the ground, gasping. I looked up at a
forest of strong, bare legs, and knew they'd surrounded me.

``It's the Sheriff!'' Oly shouted. The legs disappeared. I
struggled to my knees.

Oly collapsed to the ground beside me, laughing.
``Did you see the way they ran? The Sheriff never comes down to the river!''

``Thanks,'' I said, around gasps, and started to get dressed.

``Any time,'' he said. ``Now, let's do some swimming.''

``No, I gotta go home and help Mama,'' I lied. I didn't feel like
going skinny dipping anymore --- maybe never again.

Oly gave me a queer look. ``OK. See you.''

\tb

I went straight home, pelting down the road as fast as I could, not
even looking where I was going. I let the door slam behind me and
took the stairs two at a time up to the attic ladder, then bolted
the trap-door shut behind me and sat in the dark, with my knees in
my chest.

Down below, Mama let out a half-hearted, ``James? Is that you?''
like she always did since I came back home. I ignored her, like
always, and she stopped worrying about it, like always.

Pa's last trip had been to the Dalai Lama's court in 1975. The man
from the embassy said that he was going to talk with the monks
about a
``white-paper that the two embassies were jointly presenting on the effect of 
mimetic ambassadorships on the reincarnated soul.''
It was all nonsense to me. He'd never arrived. The teleporter said
that it had put him down gentle as you like on the floor of the
Lama's floating castle over the Caspian Sea, but the monks never
saw him.

And that was that.

It had been a month since our return. I'd ventured out into town
and looked up my chums, and found them so full of gossip that
didn't mean anything to me; so absorbed with games that seemed
childish to me; so strange, that I'd retreated home. I'd prowled
around our house like a burglar at first, and when I came back to
the attic, all the numbness that had enveloped me since the man
from the State Department had teleported into our apt melted away
and I started bawling.

The attic had always been Pa's domain. He'd come up here with
whatever crackpot invention he'd ordered this month out of a
catalog or one of the expensive, foreign journals he subscribed to,
and tinker and swear and hit his thumbnail and tear his pants on a
stray dingus and smoke his cheroots and have a heck of a time.

The muffled tread of his feet and the distant cursing while I sat
in the parlour downstairs had been the homiest sound I knew. Mama
and I would lock eyes every time a particularly forceful round of
hollers shook down, and Mama would get a little smile and her eyes
would crinkle, and I felt like we were sharing a secret.

Now, the attic was my private domain: there was the elixir shelf,
full of patent medicines, hair-tonics, and soothing syrups. There
was the bookcase full of wild theories and fantastic adventure
stories. There were the crates full of dangerous, coal-fired
machines --- an automatic clothes-washing-machine, a cherry-pitter,
and other devices whose nature I couldn't even guess at. None of
them had ever worked, but I liked to run my hands over them, feel
the smooth steel of their parts, disassemble and reassemble them.
Back in 75, I'd once tried to take the robutler apart, just to get
a look at how it was all put together, but it was a lost cause ---
I couldn't even figure out how to get the cover off.

I walked through the cool dark, the only light coming from the
grimy attic window, and fondled each piece. I picked up an oilcan
and started oiling the joints and bearings and axles of each
machine in turn. Pa would have wanted to know that everything was
in good working order.

\tb

``I think you should be going to school, James,'' Mama said, at
breakfast. I'd already done my morning chores, bringing in the
coal, chopping kindling, taking care of the milch-cows and making
my bed.

I took another forkful of sausage, and a spoonful of mush, chewed,
and looked at my plate.

``It's time, it's time. You can't spend the rest of your life sulking around 
here. Your father would have wanted us to get on with our lives.''

Even though I wasn't looking at her when she said this, I knew that
her eyes were bright with tears, the way they always got when she
mentioned Pa. His chair sat, empty, at the head of the table. I had
another bite of sausage.

``James Arthur Nicholson! Look at me when I speak to you!''

I looked up, reflexively, as I always did when she used my full
name. My eyes slid over her face, then focused on a point over her
left shoulder.

``Yes'm.''

``You're going to school. Today. And I expect to get a good report from Mr 
Adelson.''

``Yes'm.''

\tb

We have two schools in New Jerusalem: the elementary school that
was built twenty years before, when they put in the wooden
sidewalks and the town hall; and the non-denominational Academy
that was built just before I left for 1975.

Miss Tannenbaum, a spinster lady with a moustache and a bristling
German accent terrorised the little kids in the elementary school
--- I'd been stuck in her class for five long years. Mr Adelson,
who was raised in San Francisco and who had worked as a roustabout,
a telegraph operator and a merchant seaman taught the Academy, and
his wild stories were all Oly could talk about.

He raised one eyebrow quizzically when I came through the door at
8:00 that morning. He was tall, like my Pa, but Pa had been as big
as an ox, and Mr Adelson was thin and wiry. He wore rumpled pants
and a shirt with a wilted celluloid collar. He had a skinny little
beard that made him look like a gentleman pirate, and used some
shiny pomade to grease his hair straight back from his high
forehead. I caught him reading, thumbing the hand-written pages of
a leatherbound volume.

``Mr Adelson?''

``Why, James Nicholson! What can I do for you, sonny?'' New
Jerusalem only had but 2,000 citizens, and only a hundred or so in
town proper, so of course he knew who I was, but it surprised me to
hear him pronounce my name in his creaky, weatherbeaten voice.

``My mother says I have to go to the Academy.''

``She does, hey? How do you feel about that?''

I snuck a look at his face to see if he was putting me on, but I
couldn't tell --- he'd raised up his other eyebrow now, and was
looking hard at me. There might have been the beginning of a smile
on his face, but it was hard to tell with the beard.
``I guess it don't matter how I feel.''

``Oh, I don't know about that. This is a school, not a prison, after all. How 
old are you?''

``Fourteen. Sir.''

``That would put you in with the seniors. Do you think you can handle their 
course of study? It's half-way through the semester now, and I don't know how 
much they taught you when you were over in,''
he swallowed, ``France.''

I didn't know what to say to that, so I just stared at my hard,
uncomfortable shoes.

``How are your maths? Have you studied geometry? Basic algebra?''

``Yes, sir. They taught us all that.'' And lots more besides. I had
the feeling of icebergs of knowledge floating in my brain, ready to
crest the waves and crash against the walls of my skull.

``Very good. We will be studying maths today in the seniors' class. We'll see 
how you do. Is that all right?''

Again, I didn't know if he was really asking, so I just said,
``Yes, sir.''

``Marvelous. We'll see you at the 8:30 bell, then. And James ---''
he paused, waited until I met his gaze. His eyebrows were at rest.
``I'm sorry about your father. I'd met him several times. He was a good man.''

``Thank you, sir,'' I said, unable to look away from his stare.

\tb

The first half of the day passed with incredible sloth, as I copied
down problems to my slate and pretended to puzzle over them before
writing down the answer I'd known the minute I saw the question.

At lunch I found a seat at the base of the big willow out front of
the school and unwrapped the waxed paper from the thick ham
sandwich Mama had fixed me. I munched it and conjugated Latin verbs
in my head, trying to make the day pass. Oly and the fellows were
roughhousing in the yard, playing follow-the-leader with Amos
Gundersen out front, showing off by walking on his hands and then
springing upright. Amos' mother came from circus people in Russia,
and all the kids in his family wanted to be acrobats when they grew
up.

I tried not to watch them.

I was engrossed in a caterpillar that was crawling up my pants-leg
when Mr Adelson cleared his throat behind me. I started, and the
caterpillar tumbled to the ground, and then Mr Adelson was
squatting on his long haunches at my side.

``How are you liking your first day, James?'' he asked, in his
raspy voice.

``It's fine, sir.''

``And the work? You're able to keep up with the class?''

``It's not a problem for me. We studied this when I was away.''

``Are you bored? Do you need more of a challenge?''

``It's fine, sir.''
\emph{Unless you want to assign me some large-prime factoring problems}.

``Right, then. Don't hesitate to call on me if things are moving too slowly or 
too quickly. I mean that.''

I snuck another look at him. He seemed sincere.

``Why aren't you playing with your chums?''

``I don't feel like it.''

``You just wanted to think?''

``I guess so.'' Why wouldn't he just leave me alone?

``It's hard to come home, isn't it?''

I stared at my shoes. What did he know about it?

``I've been around the world, you know that? I sailed with a tramp steamer, the 
\emph{Slippery Trick}. I saw the naked savages of Polynesia, and the voodoo 
witches that the freed slaves of the Caribbean worship, and the coolies pulling 
rickshaws in Peking. It was so \emph{hard} to come home to Frisco, after five 
years at sea.''

To my surprise, he sat down next to me, in the dirt and roots at
the base of the tree.
``You know, aboard the \emph{Trick}, they called me Runnyguts --- I threw up 
every hour for my first month. I was more reliable than the Watch! But they 
didn't mean anything by it. When you live with a crew for years, you become a 
different person. We'd be out at sea, nothing but water as far as the eye could 
see, and we'd be playing cards on-deck. We'd told each other every joke we knew 
already, and every story about home, and we knew that deck of cards so well, 
which one had salt-water stains on the back and which one turned up at corner 
and which one had been torn, and we'd just scream at the sun, so bored! But 
then we'd put in to port at some foreign city, and we'd come down the plank in 
our best clothes, twenty men who knew each other better than brothers, hard and 
brown from months at sea, and it felt like whatever happened in that strange 
port-of-call, we'd come out on top.''

"And then I came back to the Frisco, and the Captain shook my hand
and gave me a sack of gold and saw me off, and I'd never felt so
alone, and I'd never seen a place so foreign.

``I went back to my old haunts, the saloons where I'd gone for a beer after a 
day's work at the docks, and the dance-halls, and the theatres, and I saw my 
old chums. That was hard, James.''

He stopped then. I found myself saying,
``How was it hard, Mr Adelson?''

He looked surprised, like he'd forgotten that he was talking to me.
``Well, James, it's like this: when you're away that long, you get to invent 
yourself all over again. Of course, everyone invents themselves as they grow 
up. Your chums there ---''
he gestured at the boys, who were now trying, with varying success,
to turn somersaults, dirtying their school clothes "--- they're
inventing themselves right now, whether they know it or not. The
smart one, the strong one, the brave one, the sad one. It's going
on while we watch!

``But when you go away, nobody knows you, and you can be whoever you want. You 
can shed your old skin and grow a new one. When we put out to sea, I was just a 
youngster, eighteen years old and fresh from my Pa's house. He was a cablecar 
engineer, and wanted me to follow in his shoes, get an apprenticeship and join 
him there under the hills, oiling the giant pulleys. But no, not me! I wanted 
to put out to sea and see the world. I`d never been out of the city, can you 
believe that? The first port where I took shore leave was in Haiti, and when I 
stepped onto the dock, it was like my life was starting all over again. I got a 
tattoo, and I drank hard liquor, and gambled in the saloons, and did all the 
things that a man did, as far as I was concerned." He had a faraway look now, 
staring at the boys' game without seeing it.''And
when I got back on-board, sick and tired and broke, there was a new
kid there, a negro from Port-Au-Prince who'd signed on to be a
cabin boy. His name was Jean-Paul, and he didn't speak a word of
English and I didn't speak a word of French. But I took him under
my wing, James, and acted like I'd been at sea all my life, and
showed him the ropes, and taught him to play cards, and bossed him
around, and taught him English, one word at a time.

"And that became the new me. Every time a new hand signed on, I
would be his teacher, his mentor, his guide.

"And then I came home.

"As far as the folks back home were concerned, I was the kid they'd
said good-bye to five years before. My father thought I was still a
kid, even though I'd fought pirates and weathered storms. My chums
wanted me to be the kid I'd been, and do all the boring, kid things
we'd done before I left --- riding the trolleys, watching the
vaudeville shows, fishing off the docks.

``Even though that stuff was still fun, it wasn't \emph{me}, not anymore. I 
missed the old me, and felt him slipping away. So, you know what I did?''

``You moved to New Jerusalem?''

``I moved to New Jerusalem. Well, to Salt Lake City, first. I studied with the 
Jesuits, to be a teacher, then I saw an ad for a teacher in the paper, and I 
packed my bag and caught the next train. And here I am, not the me that came 
home from sea, and not the me who I was before I went to sea, but someone in 
between, a new me --- teaching, but on dry land, and not chasing dangerous 
adventures, but still reading my old log-book and smiling.''

We sat for a moment, in companionable silence. Then, abruptly, he
checked his pocket watch and yelped.
``Damn! Lunch was over twenty minutes ago!'' He leapt to his feet,
as smoothly as a boy, and ran into the schoolhouse to ring the
bell.

I folded up the waxed-paper, and thought about this adult who
talked to me like an adult, who didn't worry about swearing, or
telling me about his adventures, and I made my way back to class.

It went better, the rest of that day.

\tb

In 75, Pa had almost never been home, but his presence was always
around us.

I'd call the robutler out of its closet and have it affix its
electrode fingertips to my temples and juice my endorphins after a
hard day at school, and when I was done, the faint smell of Pa's
hair-oil, picked up from the 'trodes and impossible to be rid of,
would cling to me. Or I'd sit down on the oubliette and find one of
Pa's journals from back home, well-thumbed and open to an article
on mental telepathy. We did ESP in school, and it was all about a
race of alien traders who communicated in geometric thought
pictures that took forever to translate. We'd never learned about
Magnetism and Astral Projection and all the other things Pa's
journals were full of.

And while I never doubted the things in Pa's journals, I never
brought them up in class, neither. There were lots of different
kinds of truth.

``James?''

``Yes, Mama?'' I said, on my way out to chop kindling.

``Did you finish your homework?''

``Yes, Mama.''

``Good boy.''

Homework had been some math, and some biology, and some geology.
I'd done it before I left school.

\tb

The report cards came out in the middle of December. Mr Adelson
sealed them with wax in thick brown envelopes and handed them out
at the end of the day. Sealing them was a dirty trick --- it mean a
boy would have to go home not knowing whether to expect a whipping
or an extra slice of pie, and the fellows were as nervous as
long-tailed cats in a rocking-chair factory when class let out. For
once, there was no horseplay afterwards.

I came home and tossed the envelope on the kitchen table without a
moment's worry. I'd aced every test, I'd done every take-home
assignment, I'd led the class, in a bored, sleepy way,
regurgitating the things they'd stuck in my brain in 1975.

I went up to the attic and started reading one of Pa's adventure
stories, \emph{Tarzan of the Apes}, by the Frenchman, Jules Verne.
Pa had all of Verne's books, each of them crisply autographed on
the inside cover. He'd met Verne on one of his diplomatic missions,
and the two had been like two peas in a pod, to hear him tell of it
--- they both subscribed to all the same crazy journals.

I was reading my favorite part, where Tarzan meets the man in the
balloon, when Mama's voice called from downstairs.
``James Arthur Nicholson! Get your behind down here \emph{now}!''

I jumped like I was stung and rattled down the attic stairs so fast
I nearly broke my neck and then down into the parlour, where Mama
was holding my report card and looking fit to bust.

``Yes, Mama?'' I said. ``What is it?''

She handed me the report card and folded her arms over her chest.
``Explain that, mister. Make it good.''

I read the card and my eyes nearly jumped out of my head. The
rotten so-and-so had given me F's all the way down, in every
subject. Below, in his seaman's hand, he'd written,
``James' performance this semester has disappointed me gravely. I would like it 
very much if I could meet with you and he, Mrs Nicholson, at your earliest 
convenience, to discuss his future at the Academy. Signed, Rbt. Adelson.''

Mama grabbed my ear and twisted. I howled and dropped the card.
Before I knew what was happening, she had me over her knee and was
paddling my bottom with her open hand, hard.

``I don't'' --- whack --- ``know \emph{what}'' --- whack ---
``you think'' --- whack --- ``you're doing, James.'' --- whack ---
``If your \emph{father}'' --- whack, \emph{whack} ---
``were here,'' --- whack --- ``he'd switch you'' --- whack ---
``within an inch of your life.'' And she gave me a load more
whacks.

I was too stunned even to cry or howl. Pa had only beat me twice in
all the time I'd known him. Mama had \emph{never} beat me. My
bottom ached distantly, and I felt tears come to my eyes.

``Well, what do you have to say for yourself?''

``Mama, it's a mistake ---'' I began.

``You're durn right!'' she said.

``No, really! I did all my homework! I passed all the exams! I showed 'em to 
you! You saw 'em!''
The unfairness of it made my heart hammer in time to the throbbing
of my backside.

Mama's breath fumed angrily out of her nose.
``You go straight to your room and \emph{stay there}. We're going to see Mr 
Adelson first thing tomorrow morning.''

``What about my chores?'' I said.

``Oh, don't worry about that. You'll have \emph{plenty} of chores to do when I 
let you out.''

I went to my room and stripped down, and lay on my tummy and
cracked my window so the icy winter air blew over my backside. I
cried a vale of tears, and rained down miserable, mean curses on
everyone: Mama, Pa, and especially the lying, snaky, backstabbing
Runnyguts Adelson.

\tb

Mama didn't get any less mad through the night, but when she came
to my door at cock-crow, she seemed to be holding it in better. My
throat and eyes were sore as sandpaper from crying, and Mama gave
me exactly five minutes to wash up and dress before dragging me out
to the horsebarn. She'd already hitched up our team and refused my
hand when I tried to help her up.

I'd been angry and righteous when I woke, but seeing Mama's
towering, barely controlled fury changed my mood to dire terror. I
stared out at the trees and farms as we rode into town, feeling
like a condemned man being taken to the gallows.

Mama pulled up out front of the Academy and marched me around back
to the teacher's cottage. She rapped on the door and waited,
blowing clouds of steam out of her nose into the frosty morning
air.

Mr Adelson answered the door in shirtsleeves and suspenders,
unshaved and bleary. His hair, normally neatly oiled and slicked,
stuck out like frayed broom-straw. The muscles on his thin arms
stood out like snakes. He blinked at us, standing on his doorstep.
``Mrs Nicholson!'' he said.

``Mr Adelson,'' my mother said.
``We've come to discuss James' report card.''

Mr Adelson smoothed his hair back and stepped aside.
``Please, come in. Can I offer you some coffee?''

``No, thank you,'' Mama said, primly, standing in his foyer. He
held out his hand for her coat and kerchief and she handed them to
him. I took off my coat and struggled out of my boots. He took them
both and put them away in a closet.

``I'm going to have some coffee. Are you sure I can't offer you a cup?''

``No. Thank you, all the same.''

``As you wish.'' He disappeared down the dark hallway, and Mama and
I found our way into his tiny parlour. Books were stacked every
which where, dusty and precarious. Mama and I sat down in a pair of
cushioned chairs, and Mr Adelson came in, holding two mugs of
coffee. He set one down next to Mama on the floor, then smacked
himself in the forehead.
``You said no, didn't you? Sorry, I'm not quite awake yet. Well, leave it there 
--- there's cream in it, maybe the cat will have some.''

He settled himself onto another chair and sipped at his coffee.
``Let's start over, shall we? Hello, Mrs Nicholson. Hello, James. I understand 
you're here to discuss James' report card.''

Mama sat back a little in her chair and let hint of a sardonic
smile show on her face.
``Yes, we are. Forgive my coming by unannounced.''

``Oh, it's nothing.''

Mr Adelson drank more coffee. Mama smoothed her skirts. I kicked my
feet against the rungs of my chair. Finally, it was too much for
me. ``What's the big idea, anyway?'' I said, glaring daggers at
him. ``I don't deserve no F!''

``Any F,'' Mr Adelson corrected. ``Why don't you think so?''

``Well, because I did all my homework. I gave the right answers in class. I 
passed all the tests. It ain't fair!''

``Not fair,'' my Mama corrected, gently. She was staring
distractedly at Mr Adelson.

``What you say is true enough, James. What grade do you suppose you should've 
gotten?''

``Why, an A! An A-plus! Perfect!'' I said, glaring again at him,
daring him to say otherwise.

``Is that what an A-plus is for, James? Perfection?''

``Sure,'' I said, opening my mouth without thinking.

Mama shifted her stare to me. She was looking even more
thoughtful.

``Why do you suppose you go to school?''

``\,'Cause Mama says I have to,'' I said, sullenly.

``James!'' Mama said.

``Oh, I suppose it's to learn things,'' I said.

Mr Adelson smiled and nodded, the way he did when one of the
students got the right answer in class. ``Well?''

``Well, what?'' I said.

``What did you learn this semester?''

``Why, everything you taught! Geometry! Algebra! Latin! Geography! Biology! 
Physics! Grammar!''

``I see,'' he said.
``James, what's the formula for determining the constant in the second 
derivative of an equation?''

I knew that one: it was one of Newton's dirty calculus proofs.
``It's a trick question. There's no way to get the constant of second 
derivative.''

``Exactly right,'' he said.

``Yes,'' I said, and folded my arms across my chest.

``Where did you learn that?''

``In ---'' I started to say 1975, but caught myself.
``In France.''

``Yes.''

``Yes,'' I said. The fingers of dawn crept across my comprehension.
``Oh.''

Mama smiled at me.

``But it's not fair! So what if I already knew everything before I started? I 
still did all the work.''

``Why are you in school, James?'' Mr Nicholson asked me again.

``To learn.''

``Well, then I think you'd better start learning something, don't you? You're 
the brightest student in the class. You're certainly smarter than I am --- I'm 
just an old sailor struggling along with the rest of the class. But you, you've 
\emph{got it}. You've been marking time in class all semester, and I daresay 
you haven't learned a single thing since you started. That's why you got F's.''

``Mr Adelson,'' Mama said.
``Am I to understand that James performed all his assignments satisfactorily?''

It was Mr Adelson's turn to squirm.
``Yes, but madam, you have to understand ---''

Mama waved aside his objections.
``If James satisfactorily completed all the work assigned to him, then I think 
he should have a grade that reflects that, don't you?''
She took a sip of her coffee.

``Yes, well ---''

``However, you do have a point. I didn't send my son to your school so that he 
could mark time, as you put it. I sent him there to learn. To be \emph{taught}. 
Have you taught him anything, Mr Adelson?''

Mr Adelson looked so all-fired sad, I forgave him the report card
and spoke up. ``Yes, Mama.''

Mama swiveled her head to me. ``Really?''

``Yes. He taught me what I was at school for. Just now.''

``I see,'' Mama said. ``This is very good coffee, Mr Adelson.''

``Thank you,'' he said, and sipped at his.

``James,'' Mr Adelson said.
``You've learned your first lesson. What do you propose your second should be?''

``I dunno,'' I said, and went back to kicking the rungs of the
chair.

``What is it that you have been doing since you came back to town, son?''
he asked.

``Hanging around in the attic, mostly. Reading. Tinkering. Like my Pa.''

``My husband's machines and journals are up there,'' Mama
explained.

``And his books,'' I said.

``Books?'' Mr Adelson looked suddenly interested.
``What kind of books?''

``Adventure stories. Stevenson. Wells. Some of it's in French. We have all of 
Verne.''

``Well, perhaps that can be your next assignment. I would like to see an 
original composition of no less than twenty pages, discussing each work of 
Verne's, charting his literary progress. Due January fifth, please.''

``Twenty pages!'' I said. ``But it's the holidays!''

``Very well. Whatever length the piece turns out is fine. But be sure you do 
justice to each work.''

\tb

By the time I got through with the assignment, it was thirty-eight
pages long. I never thought I could write that much but it kept on
coming, new thoughts about each book, each scene, the different
worlds Verne had built: the fantastic slopes of Barsoom, the
sinister Island of Dr Moreau\ldots{} Each one spawned a new
insight. I felt like the Verne's detective, Sherlock Holmes,
assembling all of the seemingly insignificant details into some
kind of coherent picture, finding the improbable links between the
wildly different stories the Frenchman told.

Mama was thrilled to see me working, papers spread out all around
me on the kitchen table --- I could've used Pa's study, but it felt
like an invasion, somehow --- writing until my wrists cramped. She
let me get away without doing my chores, rising early to milk the
cow, bringing in the eggs from the henhouse, even chopping the
kindling. Just so long as I was writing, she was happy to let me go
on shirking my responsibilities.

Even on Christmas Eve, I was too distracted to really enjoy the
smells of goose and ham and the stuffing Mama spent days preparing.
I was still writing when she told me to go change and set the table
for three.

``We're having Mr Johnston to dinner,'' she said.

I made a face. Mr Johnston was the only one in town that I could
have talked to about my time in 1975, but I never did. He had a way
of bossing a fellow around while seeming to be nice to him. He
still ran Pa's store, using ladders to reach the high shelves that
Pa had just plucked things off of. I had to see him when Mama sent
me on errands there, but I made sure that I left as quickly as I
could. Mama kept saying that I should ask him for a job, but I was
pretty good at changing the subject whenever it came up.

I put away my papers and changed into my Sunday clothes. I'd been
hinting to Mama lately that a boy just wasn't complete without a
puppy, so I put an extra shine on my shoes and said a quick prayer
that I wouldn't find socks and picture-books under the tree.

Mr Johnstone arrived with a double-armload of gifts. Well, he
\emph{did} run my Pa's store, after all, so he could get things
wholesale. I took his parcels from him and set them under the tree.
Then that dandified sissy actually \emph{kissed} my Mama on the
cheek, lifting a sprig of mistletoe up with one hand. When Pa and
Mama stood together, she'd barely come up to his shoulder, while Mr
Johnstone had to stand on tiptoe to get the mistletoe over their
heads. ``Merry Christmas, Ulla,'' he said.

She took his hands and said, ``Merry Christmas, James.''

I wanted to be sick.

\tb

Mr Johnstone had a whiskey in our parlour before we ate, sitting in
my Pa's chair, smoking a cigar from my Pa's humidor. Mama ordered
me to keep him company while she set out the meal.

``Do they call you Jimmy?'' he asked me, staring down his long,
pointy nose.

``No, sir. James.''

``It's a fine name, isn't it? Served me well, man and boy.'' He
made a face that was supposed to be funny, like he'd bit into a
lemon.

``I like it fine, sir.''

``Are you having any problems adjusting, now that you're home? Finding it hard 
to relate to the other fellows?''

``No, sir.''

``You don't find it strange, after seeing 1975?''

``No, sir. It's home.''

``Ha!'' he said, as though I'd said something profound.
``I guess it is, at that. Say, why don't you come by the store some time? I 
just got some samples from a new candy company in Oregon, and I need to get an 
unbiased opinion before I order.''
He gave me a pinched smile, like he thought he was Santa Claus.

``Mama doesn't like me eating sweets,'' I said, and stared at my
reflection in my shoes.

Mama rescued me by coming into the parlour then, looking young and
pretty in her best dress. ``Dinner is served, gentlemen.''

We followed her into the dining room, and Mr Johnstone took my Pa's
seat at the head of the table and carved the goose. Even though the
bird was brown and juicy, I found I didn't have any appetite.

``I have word from Pondicherry,'' Mr Johnstone said, as he poured
gravy over his second helping of mashed potatoes.

``Yes?'' Mama said.

``Who's he?'' I asked.

``Your father's successor,'' Mr Johnstone said.
``A British officer from New Delhi. A fat little man, and awfully full of 
himself.''

I repressed a snort. For my money Mr Johnstone was as full of
himself as one man could be. I couldn't imagine a blacker kettle.

``He says that Nussbaum, from 1952 New York, has rolled back relations with 
extraterrestrials by fifty years. He sold a Centurian half a million defective 
umbrellas from his brother-in-law's factory. The New Yorkers are all defending 
him. \emph{Caveat emptor}.''

``I never could keep track of who was friendly and who wasn't,''
Mama said. ``It was all Greek to me. Politics.''

Mr Johnstone opened his mouth to explain, but Mama held up one
hand.
``No, no, I don't \emph{want} to understand. Les used to lecture me about this 
from dawn to dusk.''
She smiled a little sad smile and stared off at the cabbage-roses
on our dining-room walls. Mr Johnstone put one hand over hers.

``He was a good man, Ulla.''

Mama stood and smoothed her skirts. ``I'll get dessert.''

\tb

I didn't get a puppy. Mr Johnstone gave me an air-rifle that I was
sure Mama would have fits over, but she just smiled. She gave me a
beautiful fountain-pen and a green blotter and a ream of creamy,
thick paper.

The pen made the most beautiful, jet-black marks, and the paper
drank it up like a thirsty man in the desert. I recopied my essay
the next day, sitting with Mama in the parlour while she darned
socks. Mr Johnstone had given her a tin of cosmetics from Paris,
that he'd ordered in special. I'd heard Mama say that only
dancehall girls wore makeup, but she blushed when he gave it to
her. I gave her a carving I'd done, of the robutler we'd had in 75.
I'd whittled it out of a block of pine, and sanded it and oiled it
until it was as smooth as silk.

Oly Sweynsdatter came by after supper and asked if I wanted to go
out and play with the fellows. To my surprise, I found I did. We
had a grand afternoon pelting each other with snow-balls, a game
that turned into a full-scale war, as all the older boys back from
high-school came out and joined in, and then, later, all the men,
even the Sheriff and Mr Adelson. I never laughed so much in all my
life, even when I got one right in the ear.

Mr Adelson led a charge of adults against the fort that most of the
Academy boys were hiding behind, but I saw him planning it and
started laying in ammunition long before they made their go, and we
sent them back with their tails between their legs. I hit him smack
in the behind with one ball as he dove for cover.

Oly's mother gave us both good, Svenska hot cocoa afterwards, with
fresh whipped cream, and Oly and I exchanged gifts. He gave me a
tin soldier, a Confederate who was caught in the act of falling
over backwards, clutching his chest. I gave him my best marble. We
followed his mother around their house, recounting the adventures
in the snow until she told me it was time for me to go home.

\tb

School started again, and I went in early the first day to turn in
my paper. Mr Adelson took it without comment and scanned the first
few paragraphs.
``Thank you, James, I think this will do nicely. I'll have it graded for you in 
the afternoon.''

I met Oly out in the orchard, where he was chopping kindling for
the school's stove, a job we all took turns at.
``I hear you might be getting a new Pa for Christmas,'' he said. He
gave me a smile that meant something, but I couldn't guess what.

``What is that supposed to mean?'' I asked.

``My Mama says your Mama had old man Johnstone over for Christmas dinner. And 
the widow Ott told my Mama that she'd connected one or two calls between your 
house and the store every day in the last month. My Mama says that Johnstone is 
courting your Mama.''

``Mrs Ott isn't supposed to talk about the calls she connects,'' I
said, as my mind reeled.
``It's like a telegraph operator: it's a confidential trust.'' Mr
Adelson had told me that, once when he was telling me stories about
his life before he went to sea.

``So, is it true?''

``No!'' I said, surprising myself with my vehemence.
``My Mama just didn't want him to be alone at Christmas.''

Oly swung the axe a few more times.
``Well, sure. But what about all the telephone calls?''

``That's business. The store is still partly ours. Mama's just looking after 
our interests.''

``If you say so,'' he said.

I shoved him hard. I drew a line in the snow with my toe.
``I \emph{do} say so. Step across the line if you say otherwise!''

Oly got to his feet and looked at me.
``I don't want to fight with you, James. I was just tellin' you what my Mama 
said.''

``Well, your Mama ought to mind her own business,'' I said, baiting
him.

That did it. He stepped over and popped me one, right in the nose.
Oly and I had been chums since we could walk, and we'd had a few
fights in our days but this time it was different. I was so
\emph{angry} at him, at my Mama, at my Pa, at New Jerusalem, and we
just kept on swinging at each other until Mr Adelson came out to
ring the bell and separated us. My nose was sore and I was limping,
and I'd torn Oly's jacket and bent his fingers back, so he cradled
his hand in the crook of his arm.

``Boys!'' Mr Adelson said.
``What the hell do you think you're doing? You're supposed to be friends.''

His language shocked me, but I was still plenty angry.
``He's no friend of mine!'' I said.

``That's fine with me,'' Oly said and glared at me.

The other kids were milling around, and Mr Adelson gave us both a
look that could melt steel, then rang the bell.

\tb

I could hardly concentrate in class that day. My Mama getting
married? A new Pa? It couldn't be true. But in my mind, I kept
seeing my Mama and that Johnstone kissing under the mistletoe, and
him sitting in my Pa's chair, drinking his whiskey.

Oly's desk was next to mine, and he kept shooting me dirty looks.
Finally, I leaned over and whispered, ``Cut it out, you idiot.''

Oly said,
``You're the idiot. I think you got your brains scrambled in France, James.''

``I'll scramble your brains!''

``Gentlemen,'' said Mr Adelson.
``Do you have something you'd like to share with the class?''

``No sir,'' we said together, and exchanged glares.

``James, perhaps you'd like to come up to the front and finish the lesson?''

``Sir?'' I said, looking at the blackboard. He'd been going through
quadratics, an elaborate first-principles proof.

``I believe you know this already, don't you? Come up to the front and finish 
the lesson.''

Slowly, I got up from my desk, leaving my slate on my desk, and
made my way up to the front. Some of the kids giggled. I picked up
a piece of chalk from the chalk-well, and started to write on the
board.

Mr Adelson walked back to my seat and sat down. I stopped and
looked over my shoulder, and he gave me a little scooting gesture
that meant go on. I did, and by the end of the hour, I found that I
was enjoying myself. I stopped frequently for questions, and erased
the board over and over again, filling it with steady columns of
numbers and equations. I stopped noticing Mr Adelson in my seat,
and when he stood and thanked me and told us we could eat our
lunches, it seemed like no time at all had passed.

Mr Adelson looked up from my essay.
``James, I'd like to have a chat with you. Stay behind, please.''

``Sit,'' he said, offering me the chair at his desk. He sat on one
of the front-row desks, and stared at me for a long moment.

``What was that mess this morning all about, James?'' he asked.

``Oly and I had an argument,'' I said, sullenly.

``I could see that. What was it about, if you don't mind my asking?''

``He said something about my Mama,'' I said.

``I see,'' he said.
``Well, having met your mother, I feel confident in saying that she's more than 
capable of defending herself. Am I right?''

``Yes, sir,'' I said.

``Then we won't see a repeat?''

``No, sir,'' I said. I didn't plan on talking to Oly ever again.

``Then we'll say no more about it. Now, about this morning's lesson: you did 
very well.''

``It was a dirty trick,'' I said.

He grinned like a pirate.
``I suppose it was. I wouldn't have played it on you if I didn't have every 
confidence in your abilities, though.''
He leaned across and picked up my essay from his desk.
``It was this that convinced me, really. This is as good as anything I've seen 
in scholarly journals. I've half a mind to send it to the \emph{Idler}.''

``I'm just a kid!''

``You're an extraordinary boy. I'm tempted to let you teach all the classes, 
and take up whittling.''

He said it so deadpan, I couldn't tell if he was kidding me.
``Oh, you can't do that! I'm not nearly ready to take over.''

He laughed.
``You're readier than you think, but I expect the town council would stop my 
salary unless I did \emph{some} of the work around here. Still, I think that's 
the most active I've seen you since you came to my class, and I'm running out 
of ideas to keep you busy. Maybe I'll keep you teaching maths. I'll give you my 
lesson plan to take home before school's out.''

``Yes, sir.''

\tb

Mr Adelson gave me a stack of papers tied up with twine after he
dismissed the class for the day. I went home and did my chores,
then unwrapped the parcel in the parlour. The lesson plans were
there, laid out, day by day, and in the centre of them was a
smaller parcel, wrapped in coloured paper. ``Merry Christmas,'' was
written across it, in his hand.

I opened it, and found a slim book. ``War of the Worlds,'' by
Verne. For some reason, it rang a bell. I thought that maybe it had
been on our bookcase in 75, but somehow, it hadn't made it back
home with us. I opened it, and read the inscription he'd written:
``From one traveller to another, Merry Christmas.''

I forced myself to read the lesson plans for the next month before
I allowed myself to start the Verne, and once I started, I found I
couldn't stop. Mama had to drag me away for dinner.

\tb

My trip back to 1975 wasn't planned, but it wasn't an accident,
either. We'd gotten a new load of hay in for our team, and Mama
added stacking it in the horsebarn to my chores. I'd been
consciously avoiding the horsebarn since Pa had disappeared. Every
time I looked at it, I felt a little hexed, a little frightened.

But Mama had a philosophy: a boy should face up to his fears. She'd
been terrified of spiders when she was a girl, and she told me that
she had made a point of picking up every spider she saw and letting
it crawl around on her face. After a year of that, she said, she
never met a spider that frightened her.

Mama had been sending me to the store more and more, too, and
having Mr Johnstone over for dinner every Friday night. She knew I
didn't like him one little bit, and she said that I would just have
to learn to live with what I didn't like, and if that was the only
thing I learned from her, it would be enough.

I preferred the horsebarn.

I worked close to the door the first day, which is no way to do it,
of course: if you blocked the door, it just made it harder to get
at the back when the time came. The way to do it is to first clear
out whatever hay is left over, move it out to the pasture, and then
fill in from the back forward.

Mama told me so, that first night, when she came out to inspect my
work. ``You sure must love working out here,'' she said.
``If you do it that way, you'll be out here stacking for twice as long. Well, 
you have your fun, but I still expect you to be getting your homework and 
regular chores done. Come in and clean up for supper now.''

I jammed the pitchfork into a bale, and washed for supper.

The next afternoon, I resolved to do it right. I moved the bales
I'd stacked up by the door to a corner, and then started cleaning
out the back. Before long, I'd uncovered the door into 1975.
``James,'' Mama called, from the house. ``Dinner!''

I took a long look at the door. The wood on the edges had aged to
the silvery-brown of the rest of the barn-boards, and it looked
like it had been there forever. I could hardly remember a time when
it wasn't there.

I went in for supper.

The next morning, I picked up my lunch and my schoolbooks, kissed
Mama good-bye, and walked out. I stood on our porch for a long
time, staring at the horsebarn. I remembered the brave explorers in
Verne's books. I looked over my shoulder, at the closed door of our
house, then walked slowly to the horsebarn. I swung the door open,
then walked to the back. The triple-bolts had rusted somewhat and
took real shoving to slide back. One of them was stubborn, so I
picked up the rake and pried it back with the handle, thinking of
how ingenious that was.

I gave the door my shoulder and shoved, and it swung back,
complaining on its hinges. On the other side was the still-familiar
dark of our 1975 apartment. I stepped into it, and closed the door
behind me.

``Lights,'' I said, and they came on.

The old place was just like the day we left it. It wasn't even
dusty, and as I heard the familiar trundle of the robutler, I knew
why. My Pa's easy chair sat in the parlour, with a print-out of the
day's \emph{Salt Lake City Bugler} folded on the side-table. I
walked to one wall and laid my palm against it, the familiar cool
glassy stuff it was made of. ``Window,'' I said, and wiped a line
across the wall. Wherever my hand wiped went transparent. It was a
sunny day in 1975 --- 1980, by then, but it would be 75 in my mind
forever. Under the dome, Greater Salt Lake was warm and tranquil. I
saw boys my age scooting around in jet-packs, dodging
hover-traffic.

Pa liked to open a big, square window when he came home, and sit in
his easy chair and smoke a stinky cigar and read the paper and
cluck over it --- ``Well, well, well,'' he'd say, and
``How \emph{about} that?'' Sometimes, he'd have a tumbler of
whiskey. He'd given me some, once, and the stuff had burned like
turpentine and I swore I wouldn't try it again for a long, long
time.

I sat in Pa's easy chair and snapped up the newspaper, the way he
used to. ``Panorama,'' I said, and Pa's square window opened before
me. ``Whiskey,'' I said, and ``Cigar,'' because I was never one for
half-measures. The robutler trundled over to me with a tumbler and
a White Owl in its hover-field. I plucked them out. Cautiously, I
put the cigar between my lips. The robutler extruded a long, snaky
arm with a flame, and lit it. I took a deep puff, and coughed
convulsively. Unthinking, I took a gulp of whiskey. I felt like my
lungs had turned inside-out.

I finished both the whiskey and the cigar before I got up, taking
cautious puffs and tiny sips, forcing myself.

My head swam, and nausea nearly drowned me. I staggered into the
WC, and hung my head in the oubliette for an eternity, but nothing
was coming up. I moved into my old bedroom and splayed out on my
bed, watching the ceiling spin. ``Lights,'' I managed to croak, and
the room went dark.

\tb

When I woke in the morning, the walls were at half-opacity, the
normal 0700 schedule, and I dragged myself out of bed.

The robutler had extruded the table and set out my breakfast, ham
and eggs and a big bulb of milk. One look at it sent me over the
edge, and I left a trail of sick all the way to the WC.

When I was done, I was as wrung-out as a washcloth. My head
pounded. The robutler was quietly cleaning up my mess. I started to
order it to clear away breakfast, but discovered that I was
miraculously hungry. I ate everything on the table and seconds,
besides, and had the robutler juice my temples and clear away my
headache. I dialed the walls to full transparency, and watched the
traffic go by.

The robutler maneuvered itself into my field of vision and flashed
a clock on its chest-plate: 0800 0800 0800. It was my old
school-alarm. It snapped me back to reality. My Mama was going to
whip me raw! She must've been worried sick.

I stood up and ran for the door. It was closed. I punched my code
into its panel, and waited. Nothing happened. I calmed myself and
punched it again. Still nothing. After trying it a hundred times, I
convinced myself that it had been changed.

I summoned the robutler and asked it for the code. Its chest-panel
lit up: BAD PROGRAM.

That's when I started to really worry. I was near to tears when I
remembered the emergency override. I punched it in.

Nothing happened.

I think I started crying around then. I was stuck in 1975!

\tb

I'm not a stupid little kid. I didn't spend much time pewling.
Instead, I went to the phone and dialed the police. The screen
stayed blank. Feeling like I was in a dream, I went to the
teleporter and dialed for my old school and stepped in. I failed to
teleport.

Reality sank in.

All outside services to the apartment had been shut off when we
moved out. The only things that still worked were the ones that ran
off our reactor, a squat armoured box on the apartment's
underbelly. The door in New Jerusalem worked, but on the 1975 side,
it needed to communicate with the central office to approve any
passage.

I thought about sitting tight and waiting. Mama would be sick with
worry, and would check the barn eventually and see the shot bolts.
She'd speak to Mr Johnstone, who would send a telegram to Paris,
and they would relay the message to 1975, and \emph{voila}, I'd be
rescued. I'd get the whipping of my life, and do extra chores until
I was seventy, but it was better than starving to death after the
apartment's pantry ran out. I felt hungry just thinking about it.

Still, there was a better way. The null-gee doughnut that our
apartment was spoked into had a supply of escape-jumpers,
single-use jet-packs with a simple transponder that screamed for
help on all the emergency channels. I could ride one of these down
into Greater Salt Lake, wait for the police. The more I thought
about this plan, the better it sounded. Better, anyway, than
sitting around like a fairytale princess, waiting for rescue. In my
mind, I was the rescuing type, not the kind that needed rescuing.

Besides, there wasn't much better than riding around in one of
those jet-packs.

I cycled the emergency lock into the doughnut, unracked a pack and
a jumpsuit that looked like it would fit me, and suited up. The
packed whined as it powered up and ran through its diagnostics. I
checked the idiot-lights to make sure they were all green, feeling
like a real man of action, then I stepped into the exterior lock
and jumped, arms and legs streamlined, toes pointed.

The jet-pack coughed to life and kicked me gently, then started
lowering me to the ground. The emergency beacon's idiot-light came
on, and I heaved a sigh of relief and got comfortable.

The flight was peaceful and dreamlike, a slow descent over the
gleaming metal city. I was so engrossed with the view that I didn't
see the packjackers until they were already on me. They hit me high
and low, two kids about my age with tricked-out custom jet-packs
with their traffic beacons broken off. One snagged my knees and
hugged them to his chest, while the other took me at the armpits. A
voice shouted in my ear:
``I'm cutting your pack loose. This is a very, very sharp knife, and when I'm 
done, I'll be the only thing holding you up. \emph{Don't squirm}.''

I didn't even have the chance to squirm. By the time the speech was
finished, I was separated from my pack, and I spun over
upside-down, and watched it continue its descent, straps dangling
in the wind. My hair hung down, and blood filled my head,
reawakening my headache. Reflexively, I twisted to get a look at my
kidnappers, but stopped immediately as I felt their grips loosen. I
squeezed my eyes shut and prayed.

The three of us dove fast and hard, and I tasted that second
helping of breakfast again before we leveled off. I risked a peek,
then squeezed my eyes shut again. We were speeding through the
lower levels of Greater Salt Lake, the unmanned freight corridors,
impossibly claustrophobic, and at our speed, dangerous.

We cornered tightly so many times that I lost count, and then we
slowed to a stop. They dumped me to the ground, steel
traction-plate. The wind was knocked out of me, and I was barely
conscious of the hands that untabbed my jumpsuit, then began
methodically turning out the pockets of my clothes.

``What the hell are you wearing, kid?'' one of them asked. It was
the same one who'd warned me about squirming. Hearing his voice a
second time, I realised that he was younger than I was, maybe ten
or eleven. Even then, it didn't occur to me to fight back --- he
had a knife sharp enough to cut through the safety strapping on my
pack.

``Clothes. I'm from 1898 --- my Pa's an ambassador. I don't have any money.''
I struggled into a sitting position, and was knocked onto my back
again.

``Stay down and you won't get hurt,'' the same voice said. It was
young enough that I couldn't tell if it was a boy or a girl. Small
hands pressed into my eyes. ``No peeking, now.''

Another set of hands systematically rifled my coat and pants, then
cut them loose and gave the same treatment to my underpants and
shirt. I blushed as they were cut loose, too.

``You really don't have any money!'' the voice said.

``I said so, didn't I?''

The voice said a dirty word that would've gotten it beaten
black-and-blue back home, and then the hands were gone. I looked up
just in time to see two small figures jetting away upwards.

I was naked, sitting on a catwalk above a freight corridor,
three-quarters of a century and God-knew-how-many miles from home.
I didn't cry. I was too worried to cry. I kicked my ruined clothes
down into the freight corridor and pulled on the jumpsuit.

Some hero I was!

\tb

It was hard work, climbing staircase after staircase, up to the
shopping levels. By the time I reached a level where I could see
the sky, I was dripping with sweat and my headache had returned.

Foot traffic was light, but what there was pretty frightening. I'd
gone walking in 75 before of course, but Greater Salt Lake was a
big place, and there were parts of it that an Ambassador's son
would never get to see.

This was one of them. The shopfronts were all iris-open airlocks,
and had been painted around to look like surprised mouths, or eyes,
or, in one fascinating case, a woman's private parts. Mostly, they
were betting shops, or bars, or low-rent bounceaterias. Even in
1975, the Saints had some influence in Salt Lake, and the bars and
brothels were pretty shameful places, where no respectable person
would be caught.

The other pedestrians on the street were mostly off-worlders,
either spacers in uniform or extees. In some cases, it was hard to
tell which was which.

I kind of slunk along, sticking to the walls, hands in my pockets.
I kept my eyes down, except when I was looking around for a public
phone. After several blocks, I realised that no one was paying any
attention to me, and I took my hands out of my pockets. The sun
filtered down over me, warm through the big dome, and I realised
that even though I'd gotten myself stuck in 75, been 'jacked, and
left in the worst neighbourhood in the whole State, I'd landed on
my feet. The thought made me smile. Another kid, say Oly, wouldn't
have coped nearly as well.

I still hadn't spied a public phone. I figured that the taprooms
would have a phone, otherwise, how could a drunk call his wife and
tell her he was going to be late coming home? I picked a bar, whose
airlock was painted to look like a brick tunnel, and walked in.

The airlock irised shut behind me and I blinked in the gloom. My
nose was assaulted with sickly sweet incense, and stale liquor, and
cigar smoke.

The place was tiny, and crowded with dented metal tables and chairs
that were bolted to the floorplates. A woman stood behind the bar,
looking hard and brassy and cheap, watching a soap opera on her
vid. A spacer sat in one corner, staring at his bulb of beer.

The bartender looked up. ``Get lost, kid,'' she said.
``No minors allowed.''

``Sorry, ma'am,'' I said.
``I just wanted to use your telephone. I was packjacked, and I need to call the 
police.''

The bartender turned back to her soap opera.
``Go peddle it somewhere else, sonny. The phone's for customers only.''

``Please,'' I said.
``My father's an ambassador, from 1898? I don't have any money, and I'm stuck 
here. I won't be a minute.''

The spacer looked up from his drink. ``Get lost, the lady said,''
he slurred at me.

``I'll buy something,'' I said.

``You just said you don't have any money,'' the bartender said.

``I'll pay for it when the police get here. The Embassy will cover it.''

``No credit,'' she said.

``You're not going to let me use your phone?'' I said.

``That's right,'' she said, still staring at her vid.

``I'm a stranger, an ambassador's son, who's been robbed. A kid. Stuck here, 
broke and alone, and you won't let me use your phone to call the police?''

``That's about the size of things,'' she said.

``Well, I guess my Pa was right. The whole world went to hell after 1914. No 
manners, no human decency.''

``You're breaking my heart,'' she said.

``Fine. Be that way. Send me back out on the street, deny me a favour that 
won't cost you one red cent, just because I'm a stranger.''

``Shut \emph{up}, kid, for chrissakes,'' the spacer said.
``I'll stand him to a Coke, if that's what it takes. Just let him use the phone 
and get out of here. He's giving me a headache.''

``Thank you, sir,'' I said, politely.

The bartender switched her vid over to phone mode, poured me a
Coke, and handed me the vid.

\tb

The policeman who showed up a few minutes later stuck me in the
back of his cruiser, listened to my story, scanned my retinas,
confirmed my identity, and retracted the armour between the back
and front seats.

``I'll take you to the station house,'' he said.
``We'll contact your Embassy, let them handle it from there.''

``What about the kids who 'jacked me?'' I asked.

The cop turned the jetcar's conn over to wire-fly mode and turned
around. ``You got any description?''

``Well, they had really nice packs on, with the traffic beacons snapped off. 
One was red, and I think the other was green. And they were young. Ten or 
eleven.''

The cop punched at his screen. ``Kid,'' he said,
``I got over three million minors eight to eleven, flying packs less than a 
year old. The most popular colour is red. Second choice, green. Where would you 
like me to start? Alphabetically?''

``Sorry, sir, I didn't realise.''

``Sure,'' he said. ``Whatever.''

``I guess I'm not thinking very clearly. It's been a long day.''

The cop looked over to me and smiled.
``I guess it has, at that. Don't worry, kid, we'll get you home all right.''

\tb

They gave me a fresh jumpsuit, sat me on a bench, called the
embassy, and forgot about me. A long, boring time later, a fat man
with walrus moustaches and ruddy skin showed up.

``On your feet, lad,'' he said.
``I'm Pondicherry, your father's successor. You've made quite a mess of things, 
haven't you?''
He had a clipped, British accent, with a hint of something else. I
remembered Mr Johnstone saying he'd been in India. He wore a
standard unisex jumpsuit, with his ambassadorial sash overtop of
it. He looked ridiculous.

``Sorry to have disturbed you, sir,'' I said.

``I'm sure you are,'' he said.
``Come along, we'll see about fixing this mess.''

He used the station's teleporter to bring me to his apartment. It
was as ridiculous as his uniform, and in the same way.
He`d taken the basic elegant simplicity of a standard 1975 unit and draped all 
kinds of silly trophies and models overtop of it: lions'
heads and sabers and model ships and framed medals and savage masks
and dolls.

``You may look, but not touch, do you understand me?'' he said, as
we stepped out of the teleporter.

``Yes, sir,'' I said. If anyone else had said it, I would have been
offended, but coming from this puffed-up pigeon, it didn't sting
much.

He went to a vid and punched impatiently at the screen while I
prowled the apartment. The bookcase was full of old friends, books
by the Frenchman, of course, and more, with strange names like
Wells and Burroughs and Shelley. I looked over a long, stone-headed
spear, and the curve of an elephant's tusk, and a collection of
campaign ribbons and medals under glass. I returned to the
bookcase: something had been bothering me. There, there it was:
``War of the Worlds,'' the book that Mr Adelson had given me for
Christmas. But there was something wrong with the spine of this
one: instead of \emph{Jules Verne}, the author name was
\emph{H.G. Wells}. I snuck a look over my shoulder; Pondicherry was
still stabbing at the screen. I snuck the book off the shelf and
turned to the title page:
``War of the Worlds, by Herbert George Wells.'' I turned to the
first chapter:

The Eve of the War

No one would have believed in the last years of the nineteenth
century that this world was being watched keenly and closely by
intelligences greater than man's and yet as mortal as his own; that
as men busied themselves about their various concerns they were
scrutinised and studied, perhaps almost as narrowly as a man with a
microscope might scrutinise the transient creatures that swarm and
multiply in a drop of water.

It was just as I remembered it, every word, just as it was in the
Verne. I couldn't begin to explain it.

A robutler swung out of its niche with a sheaf of papers. I
startled at the noise, then reflexively stuck the book in my
jumpsuit. The roboutler delivered them to Pondicherry, who stuffed
them in a briefcase.

``The embassy will be able to return you home by courier route in three hours. 
Unfortunately, I don't have the luxury of waiting around here until then. I 
have an important meeting to attend --- you'll have to come along.''

``Yes, sir,'' I said, trying to sound eager and helpful.

``Don't say anything, don't touch anything. This is very sensitive.''

``No, sir, I won't. Thank you, sir.''

\tb

The meeting was in a private room in a fancy restaurant, one that
I'd been to before for an embassy Christmas party. Mama had drunk
two glasses of sherry, and had flushed right to the neck of her
dress. We'd had roast beef, and a goose wrapped inside a huge
squash, the size of a barrel, like they grew on the Moon.

Pondicherry whisked through the lobby, and the main dining room,
and then up a narrow set of stairs, without checking to see if I
was following. I dawdled a little, remembering Pa laughing and
raising his glass in toast after toast.

I caught up with Pondicherry just as he was ordering, speaking
brusquely into the table. Another man sat opposite him. Pondicherry
looked up at me and said, ``Have you dined, boy?''

``No, sir.''

He ordered me a plate of calf livers in cream sauce, which is about
the worst thing you can feed a boy, if you ask me, which he didn't.
``Sit down,'' he said.
``Mr Nussbaum, Master James Nicholson. I am temporarily in \emph{loco 
parentis}, until he can be sent home.''

Nussbaum smiled and extended his hand. He was wearing a grey suit,
with a strange cut, and a black tie. His fingers dripped with heavy
gold rings, and his hair, while short, still managed to look fancy
and a little sissy-fied.
``Good to meetcha, son. You Lester's boy?''

``Yes, sir, he was my Pa.''

``Good man. A damned shame. What are you doing here? Playing hooky?''

``I guess I just got lost. I'm going home, soon as they can get me there.''

``Is that so? Well, I'll be sad to see you go. You look like a smart kid. You 
like chocolate cake, I bet.''

``Sometimes,'' I said.

``Like when?''

``When my mama makes it, with a glass of milk, after school,'' I
said.

He laughed, a strangled har-har-har.
``You guys kill me. Your mama, huh? Well, they make some fine chocolate cake 
here, though it may not be as good as the stuff from home.''
He thumbed the table.
``Sweetie, send up the biggest piece of chocolate cake you got down there, and 
a glass a milk, willya?''

The table acknowledged his request with a soft green light.

``Thank you, sir,'' I said.

``That's quite enough, I think,'' Pondicherry said.
``I didn't come here to watch you rot young James's teeth. Can we get to 
business?''

Pondicherry started talking, in rapid, clipped sentences,
punctuated by vicious bites of his food. I tried to follow what it
was about --- trading buffalo steaks for rare metals, I got that
much, but not much more. The calves' livers were worse than I
imagined, and I hid as much of them as I could under the potatoes,
then pushed the plate away and dug into the cake.

I sneaked a look up and saw that Nussbaum was grinning slyly at me.
He hadn't said much, just ate calmly and waited for Pondicherry to
run out of steam. He caught my eye and slipped a wink at me. I
looked over at Pondicherry, who was noisily cudding a piece of
steak, oblivious, and winked back at Nussbaum.

Pondicherry daubbed at his mouth with his napkin. ``Excuse me,'' he
said, ``I'll be right back.'' He stood and walked towards the WC.

Nussbaum suddenly jingled. Distractledly, he patted his pockets
until he located a tiny phone. He flipped it open and grunted
``Nussbaum,'' into it.

``Jules!'' he said a moment later. ``How're things?''

He scowled as he listened to the answer.
``Now, you and I know that there's a difference between \emph{smart} and 
\emph{greedy}. I think it's a bad idea.''

He listened some more and drummed his fingers on the table.

``Because it's not \emph{credible}, dammit! Even the title is anachronistic: no 
one in 1902 is going to understand what \emph{Neuromancer} means. Think about 
it, wouldya? Why don't you do some of Twain's stuff? Those books've got 
\emph{legs}.''

My jaw dropped. Nussbaum was talking to the Frenchman --- and he
was helping him to \emph{cheat}! To steal from Mark Twain! I was
suddenly conscious of ``War of the Worlds,'' down the front of my
jumpsuit. I thought back to Mr Adelson's assignment, and it all
made sudden sense. Verne was a \emph{plagiarist}.

Nussbaum hung up just as Pondicherry re-seated himself. He took a
sip of his drink, then held up a hand. Pondicherry eyed him
coldly.

``Look,'' Nussbaum said.
``We've gone over this a few times, OK? I know where you stand. You know where 
I stand. We're not standing in the same place. Much as I enjoy your company, I 
don't really wanna spend the whole day listening to you repeating yourself. All 
right?''

``Really, I don't think ---'' Pondicherry started, but Nussbaum
held up his hand again.

``That's all right, I'm a rude son-of-a-bitch, and I know it. Let's just take 
it as read that you and me spent the whole afternoon letting the other fella 
know how sincere our positions are. Then we can move onto cocktails, and 
compromise, and maybe have some of the day left over.''
Pondicherry started to talk again, but Nussbaum plowed over him.
``I'll go to six troy ounces per steer. You won't get a better offer. 98\% pure 
ores. Better than anything you'd ever refine back home. It's as far as I go.''

``Sir, is that an ultimatum?'' Pondicherry asked, his eyes
narrowing.

``Call it whatever you please, buster. It's my final, iron-clad offer. You 
don't like it, I can talk to the Chinaman. He seemed pretty eager to get some 
good metal home to the Emperor.''

``You wouldn't --- he's too far back, it would violate the protocols.''

``That's what you say. It may be what the trade court decides. I'll take my 
chances.''

``Six and a half ounces,'' Pondicherry said, in a spoiled-brat
voice.

``You don't hear so good, do you? Six ounces is the offer on the table; take it 
or leave it.''
Nussbaum pushed some papers across the table.

Pondicherry stared at them for a long moment.
``I will sign them, sir, but it is with the expectation of continued trade 
opportunities. This is a good-will gesture, do you understand?''

Nussbaum snorted and reached for his papers.
``This is about steaks and metals. This isn't about the future, it's about 
today, now. That's what's on the table. You can sign it, or you can walk away.''

Pondicherry blew air out his nose like a crazy horse, and signed.
``If you'll excuse me, I need to use the WC again.'' He rose and
left the room, purple from the collar up.

``What a maroon,'' Nussbaum said to the closed door.
``This's gotta be a real blast for you, huh?'' he said.

I grinned. ``It's not so bad. I liked watchin' you hogtie him.''

He laughed.
``I never would've tried that on your father, kid. He was too sharp. But fatso 
there, he's terrified the Chinaman will give the Middle Kingdom an edge when it 
faces down his Royal Navy. All it takes is the slightest hint, and he folds 
like a cheap suit.''

That made me chuckle --- a cheap suit!

I gave him my best innocent look.
``Who else knows about the Frenchman?'' I asked him.

Nussbaum grinned like he'd been caught with his hand in the cookie
jar.
``I realised about halfway through that conversation that being Lester's boy, 
you've probably read just about every word old Jules `wrote.'\,''

``I have,'' I said. I took out ``War of the Worlds.''
``How does Mr Wells feel about this?'' I asked.

``I imagine he's pretty mystified,'' Nussbaum said.
``Would you believe, you're the first one who's caught on?''

I believed it. I knew enough to know that the agencies that policed
the protocols had their hands full keeping track of art and gold
smugglers. I'd never even thought of smuggling \emph{words}. If the
trade courts found out\ldots{} Well, hardly a week went by that
someone didn't propose shutting down the ambassadorships; they'd
talk about how the future kept on leaking pastwards, and if we
thought 1975 looked bad, imagine life in 1492 once the future
reached it! The ambassadors had made a lot of friends in high
places, though: they used their influence to keep things on an even
keel.

Nussbaum raised an eyebrow and studied me.
``I think your father may've figured it out, but he kept it to himself. He and 
Jules got along like a house on fire.''

I kept the innocent look on my face. ``Well, then,'' I said.
``If Pa didn't say anything, you'd think that I wouldn't either, right?''

Nussbaum sighed and gave me a sheepish look.
``I'd \emph{like} to think so,'' he said.

I turned the book over in my hands, keeping my gaze locked with
his. I was about to tell him that I'd keep it to myself, but at the
last minute, some instinct told me to keep my mouth shut.

Nussbaum shrugged as though to say, \emph{I give up}.
``Hey, you're headed home today, right?'' he said, carefully.

``Yes, sir.''

``I've got a message that you could maybe relay for me, you think?''

``I guess so\ldots{}'' I said, doubtfully.

``I'll make it worth your while. It's got to go to a friend of mine in Frisco. 
There's no hurry --- just make sure he gets it in the next ten years or so. 
Once you deliver it, he'll take care of you --- you'll be set for life.''

``Gosh,'' I said, deadpan.

``Are you game?''

``I guess so. Sure.'' My heart skipped. Set for life!

``The man you want to speak to is Reddekop, he's the organist at the Castro 
theatre. Tell him: `Nussbaum says get out by October 29th, 1929.' He'll know 
what it means. You got that?''

``Reddekop, Castro Theatre. October 29th, 1929.''

``Exac-atac-ally.'' He slid ``War of the Worlds'' into his
briefcase. ``You're doin' me a hell of a favour, son.''

He shook my hand. Pondicherry came back in then, and glared at me.
``The embassy contacted me. They can set you at home six months after you left 
--- there's a courier gateway this afternoon.''

``Six months!'' I said.
``My Mama will go crazy! Can't you get me home any sooner?''

Pondicherry smirked.
``Don't complain to me, boy. You dug this hole yourself. The next scheduled 
courier going anywhere near your departure-point is in five years. We'll send 
notice to your mother then, to expect you home mid-July.''

``Tough break, kiddo,'' Nussbaum said, and he shook my hand and
slipped me another wink.

\tb

The courier gateway let me out in an alleyway in Salt Lake City.
The embassy had given me ten Wells Fargo dollars, and fitted me out
with a pair of jeans and a workshirt that were both far too big for
me, so that they slopped around me as I made my way to the train
station and bought my ticket to New Jerusalem.

It was Wednesday, the normal schedule for the Zephyr Speedball, so
I didn't have too long to wait at the station. I bought copies of
the Salt Lake City \emph{Shout}, and the San Francisco
\emph{Chronicle} from a passing newsie. The \emph{Chronicle} was a
week old, but it was filled with all sorts of fascinating big-city
gossip. I read it cover-to-cover on the long ride to New
Jerusalem.

Mama met me at the train station. I'd been expecting a switching,
right then and there, but instead she hugged me fiercely with tears
in her eyes. I remembered that it had been over six months for her
since I'd gone.

``James, you will be the death of me, I swear,'' she said, after
she'd squeezed every last bit of stuffing out of me.

``I'm sorry, Mama,'' I said.

``We had to tell everyone you'd gone away to school in France,'' a
familiar male voice said. I looked up and saw Mr Johnstone standing
a few yards away, with our team and trap. He was glaring at me.
``I've had the barn gateway sealed permanently on both sides.''

``I'm sorry, sir,'' I said. But inside, I wasn't. Even though I'd
only been away for a few days, I'd had the adventure of a lifetime:
smoked and drank and been 'jacked and escaped and received a secret
message. My Mama seemed shorter to me, and frailer, and James H
Johnstone was a puffed-up nothing of a poltroon.

``We'll put it behind us, son,'' he said.
``But from now on, there will be order in our household, do we understand 
each-other?''

\emph{Our} house? I looked up sharply at my Mama. She smiled at me,
nervously. ``We married, James. A month ago. Congratulate me!''

I thought about it. My Mama needed someone around to take care of
her, and vice-versa. After all, it wasn't right for her to be all
alone. With a start, I realised that in my mind, I'd left my Mama's
house. I felt the Wells-Fargo notes in my pocket.

``Congratulations, Mama. Congratulations, Mr Johnstone.''

Mama hugged me again and the Mr Johnstone drove us home in the
trap.

\tb

All through the rest of the day, Mama kept looking worriedly at me,
whenever she thought I wasn't watching. I pretended not to notice,
and did my chores, then took my \emph{Chronicle} out to the apple
orchard behind the Academy. I sat beneath a big, shady tree and
re-read the paper, all the curious bits and pieces of a city frozen
in time.

I was hardly surprised to see Mr Adelson, nor did he seem surprised
to see me.

``Back from France, James?''

``Yes, sir.''

``Looks like it did you some good, though I must say, we missed you around the 
Academy. It just wasn't the same. Have you been keeping up your writing?''

``Sorry, sir, I haven't. There hasn't been time. I'm thinking about writing an 
adventure story, though --- about pirates and space-travellers and airships,''
I said.

``That sound exciting.'' He sat down beside me, and we sat there in
silence for a time, watching the flies buzz around. The air was
sweet with apple blossoms, and the only sound was the wind in the
trees.

``I'm going to miss this place,'' I said, unthinking.

``Me, too,'' Mr Adelson said.

Our eyes locked, and a slow smile spread over his face.
``Well, I know where \emph{I'm} going, but where are you off to, son?''

``You're going away?'' I said.

``Yes, sir. Is that a copy of the \emph{Chronicle}? Give it here, I'll show you 
something.''

He flipped through the pages, and pointed to an advertisement.
``The \emph{Slippery Trick} is in port, and they're signing on crew for a run 
through the south seas, in September. I intend to go as Quartermaster.''

``You're leaving?'' I said, shocked to my boots.

To my surprise, he pulled out a pouch of tobacco and some rolling
papers and rolled himself a cigarette. I'd never seen a
schoolteacher smoking before. He took a thoughtful puff and blew
the smoke out into the sky.
``To tell you the truth, James, I just don't think I'm cut out for this line of 
work. Not enough excitement in a town like this. I've never been happier than I 
was when I was at sea, and that's as good a reason to go back as any. I'll miss 
you, though, son. You were a delight to teach.''

``But what will I do?'' I said.

``Why, I expect your mother will send you back East to go to school. I 
graduated you from the Academy \emph{in absentia} during the last week of 
classes. Your report card and diploma are waiting on my desk.''

``Graduated?'' I said, shocked. I had another year to go at the
Academy.

``Don't look so surprised! There was no earthly reason for you to stay at the 
Academy. I'd say you were ready for college, myself. Maybe Harvard!''
He tousled my hair.

I allowed myself a smile --- I didn't think I was any smarter than
the other kids, but I sure knew a whole lot more about the world
--- the worlds! And maybe, in my heart of hearts, I knew that I was
a \emph{little} smarter. ``I'll miss you, sir,'' I said.

``Call me Robert. School's out. Where are you off to, James?''

I gestured with my copy of the \emph{Chronicle}.

``My home town! Whatever for?''

I looked at my shoes.

``Oh, a secret. I see. Well, I won't pry. Does your mother know about this?''

I felt like kicking myself. If I said no, he'd have to tell her. If
I said yes, I'd only have myself to blame if he spilled the news to
her. I looked at him, and he blew a streamer of smoke into the sky.
``No, sir,'' I said. ``No, Robert.''

He looked at me. He winked. ``Better keep it to ourselves, then,''
he said.

\tb

The ticket-girl at the Castro Theatre wasn't any older than I was,
but she wore her hair shorter than some of the boys I'd known back
home, and more makeup than even the painted ladies at the saloon.
She looked at me like I was some kind of small-town fool. It was a
look I was getting used to seeing.

``Reddekop only plays for the \emph{evening} shows, kid. No organ for the 
\emph{matinee}.''

``Who you calling a kid?'' I said. I'd kept a civil tongue ever
since debarking the train, treating adult and kid with equal
respect, but I was getting sick of being treated like a yokel. I'd
been farther than any of these dusty slickers would ever go, and I
was grown enough that I'd told my Mama and Mr Johnstone that I was
going off on my own, instead of just leaving a note like I'd
originally planned.

``You. Kid. You want to talk to Reddekop, you come back after six. In the 
meantime, you can either buy a ticket to the matinee or get lost.''

On reflection, telling my Mama was probably a mistake. It meant
that I was locked in my room for two consecutive Wednesdays so that
I couldn't catch the train. On the third Wednesday, I climbed out
onto the roof and then went down the rope-ladder I'd hidden behind
a chimney. The Wells Fargo notes I'd started with were almost gone,
mostly spent on the expensive food on the train --- I hadn't dared
try to sneak any food away from home, my Mama was no fool.

I thought about buying a ticket to the matinee. I still had almost
five dollars, but a quick look at the menus in the restaurants had
taught me that if I thought the food on the train was expensive, I
had another think coming. I shouldered my rucksack and wandered
away, taking care to avoid the filth from dogs and people that
littered the sidewalks. I told myself that I wasn't homesick ---
just tired.

\tb

``October 29, 1929, huh?'' Reddekop was a small German with a
greying spade beard and a heavily oiled part in his long hair. His
fingers were long and nimble, but nearly everything else about him
was short and crude. He made me nervous.

``Yes, sir. Mr Nussbaum thought you'd know what it meant.''

Reddekop struck a match off the side of the organist's pit, lighted
a pipe, then tossed the match carelessly into the theatre seats. I
winced and he chuckled. "Not to worry, kid. The place won't burn
down for a few years yet. I have it on the very best authority.

``Now, Nussbaum says October 29, 1929. What else does he say?''

``He said that you'd take care of me.''

He gripped the pipe in his yellow teeth and hissed a laugh around
the stem.
``He did, did he? Well, I suppose I should. Of course, I won't know for sure 
for more than 25 years --- I don't suppose you want to wait that long?''

``No, sir!'' I said. I didn't like this little man --- he reminded
me of some kind of musical rat.

``I thought not. Do you know what a trust is, James?''

We'd covered that in common law --- I could rattle off about thirty
different kinds without blinking. ``I have a general idea,'' I
said.

``Good, good. What I'm thinking is, the best thing is for me to set up a trust 
through a lawyer I know on Market Street. He'll make sure that you're always 
flush, but never so filthy that someone will take a notice in you. How does 
that strike you?''

I thought it over.
``How do I know that the trust fund won't disappear in a few years?''

``You're nobody's fool, huh? Well, how about this --- you find your own 
advocate: a lawyer, a bondsman, someone you trust, and he can look over all the 
books and papers, make sure it's all square-john. How does that strike you?''

Reddekop knew I was a stranger in town, and maybe he was counting
on my not being able to find anyone qualified to audit the trust,
but I had an ace up my sleeve. I wasn't anybody's fool.

``That sounds fair,'' I said.

\tb

Back at my Mama's I'd had long hard days, doing chores: chopping
wood, stacking hay, weeding the garden, carrying water. I'd go to
bed bone-tired, limp as a rag and as exhausted as I thought I could
be.

Boy, was I wrong! By the time I found Mr Adelson's rooming house, I
could barely stand, my mouth was dry as a salt-flat, and it was
hard to keep my eyes open. They've got hills in San Francisco that
must've been some kind of joke God played. His landlady, a worn-out
grey woman whose sour expression seemed directed at everything and
anything, let me in and pointed me up three rickety flights of
stairs to Mr Adelson's room.

I dragged my luggage up with me, bumping it on the stairs, and
rapped on the door. Mr Adelson answered in the same shirtsleeves
and suspenders I'd seen him in that Christmas, an age ago, when my
Mama dragged me to his cottage. ``James!'' he said.

``Mr Adelson,'' I said. ``Sorry to drop in like this.''

He took my bag from me and ushered me into his room, pulling up a
chair. ``What on earth are you doing here?'' he said.
``Do your parents know where you are? Are you all right? Have you eaten? Are 
you hungry?''

``I'm pretty hungry --- I haven't eaten since supper last night on the train,''
I tried to make it sound jaunty, but I'm afraid it came out pretty
tired-sounding."

``I'll fix us sandwiches,'' he said, and started fishing around his
sea-chest. I watched his shoulders move for a moment, and then my
eyes closed.

\tb

``Well, good morning,'' Mr Adelson said, as I sat bolt upright,
disoriented in a strange bed with a strange blanket. ``Coffee?''

He was leaning over a little Sterno stove, heating up a small tin
pot. Morning sun streamed in through the grimy window.

``I wrapped your sandwich up from last night. It's there, on the dresser.''

I stood up and saw that except for my shoes, I was still dressed.
The sandwich was salt beef and cheese, and the sourdough was stale,
and it was the best thing I'd ever eaten. Mr Adelson handed me a
tin cup full of strong coffee, and though I don't much like coffee,
I found myself drinking it as fast as I could.

``Thank you, Mr Adelson,'' I said.

``Robert,'' he said, and sat down on the room's only chair. I
perched on the bed's end.
``Well, you seem to have had quite a day! Let's hear about it.''

I told him as much as I could, fudging around some of the details
--- my Mama surely did know where I was, even if she wasn't very
happy about it; and of course, I couldn't tell him that I'd met
Nussbaum in 1975, so I just moved the locale to France, and caged
around what message he'd asked me to deliver to Reddekop. It still
made for a pretty exciting telling.

``So you want me to go to this lawyer's office with you? To look over the 
papers? James, I'm just a sailor, I'm not qualified.''

I'd prepared for this argument, on the long slog to the rooming
house.
``But \emph{I} know something about this; they won't believe it, though, and 
will slip all kinds of dirty tricks in if they think that the only fellow 
who'll be looking at it is just a kid.''

``Explain to me again why you don't want to wire Mr Johnstone to come and look 
it over? It sounds like an awful lot of money for him not to be involved.''

``He's not my Pa, Robert. I don't even \emph{like} him, and chances are, he'll 
hide away all that money until I'm eighteen or \emph{twenty-one}, and try to 
send me off to school.''

``And what's wrong with that? You have other plans?''

``Sure,'' I said, too loudly --- I hadn't really worked that part
out. I just knew that the next time I set foot in New Jerusalem,
I'd be my own man, a man of the world, and not dependent on anyone.
I'd take Mama and Mr Johnstone out for a big supper, and stay in
the fanciest room at the Stableman's hotel, and hire Tommy Benson
to carry my bags to my room.
``Besides, I'm not asking you to do this for \emph{free}. I'll pay you a --- an 
administrative fee. Five percent, \emph{for life}!''

He looked serious.
``James, if I do this --- mind I said \emph{if} --- I won't take a red cent. 
There are things here that you're not telling me. Now, that's your business, 
but I want to make sure that if anyone ever scrutinises the affair, that it's 
clear that I didn't receive any benefit from it.''

I smiled. I knew I had him --- if he'd thought it that far through,
he wasn't going to say no. Besides, I hadn't even played my trump
card yet: that if he didn't help me, I'd be out on the streets on
my own, and I could tell that he didn't like that idea.

\tb

Mr Adelson wore his teacher clothes for the affair and I wore the
good breeches and shirt I'd packed. We stopped at a barber's
before, Mr Adelson treated me to a haircut from the number-two man
while he took a shave and a trim. We boarded the cablecar to Market
like a couple of proper gentlemen, and if I thought flying in a
jetpack was exciting, it was nothing compared to the terror of
hanging on the running-board of a cablecar as it laboured up and
then --- quickly! --- down a monster hill.

The lawyer was a foreigner, a Frenchie or a Belgian, and his
offices were grubby and filled with stinking cigar smoke and the
din of the trolleys. He asked no embarrassing questions of me. He
just sized up Mr Adelson, then put away the papers on his desk and
presented a set from his briefcase, laying out the terms of the
trust, and retreated from the office. I read over Mr Adelson's
shoulder, the terms scribbled in a hasty hand, but every word of it
legal and binding, near as I could tell.

The amounts in question were staggering. Two hundred dollars, every
month! Indexed for inflation, for seventy years or the duration of
my natural life, whichever was lesser. The records of the trust to
be deposited with the Wells Fargo, subject to scrutiny on demand.
Mr Adelson looked long and hard at me.
``James, I can't begin to imagine what sort of information you've traded for 
this, but son, you're rich as Croesus!''

``Yes, sir,'' I said.

``Do these papers look legal to you?''

``Yes, sir.''

``They seem legal to me, too.''

A bubble of excitement filled my chest and I had to restrain myself
from bouncing on my heels. ``I'm going to sign it,'' I said.
``Will you witness it?''

``I've got a better idea. Let's get that lawyer and take this down to the Wells 
Fargo and have the President of the Bank witness it himself.''

And that's just what we did.

\tb

Mr Adelson had spent the previous night on the floor, while I slept
in his bed. My first month's payment was tucked carefully in my
pocket, and over his protests, I pried loose a few bills and took
my own room in the rooming house, and then the two of us ate out at
a restaurant whose prices had seemed impossibly out-of-reach the
day before. We had oysters and steaks and I had a slab of apple pie
for desert with fresh ice cream and peach syrup, and when I was
done, I felt like new man. Mr Adelson had a bottle of beer with
dinner, and a whiskey afterwards, and I insisted on paying.

``Well, then,'' he said, sipping his whiskey.
``You're a very well-set-up young man. What will you do now?''

All throughout my scheming since my second return from 75, the
prospect of what to do with all the money had niggled away at the
back of my mind. All I knew for sure was that I didn't want to grow
up in New Jerusalem. I wanted adventure, exotic places and people,
danger and excitement. Over dinner, though, a plan had been forming
in my head.

``Does the \emph{Slippery Trick} need a cabin-boy?''

He shook his head and smiled at me.
``I was afraid it was something like that. Son, you could pay for a stateroom 
on a proper liner with all the money you have. Why would you want to be in 
charge of chamber-pots on a leaky old tub?''

``Why do you want to sail off on a leaky old tub instead of teaching in Utah, 
or working on the trolleys here?''

It took me most of the night to convince him, but there was no
doubt in my mind that I would, and when the ship sailed, that I'd
be on it, with a big, leather-bound log, writing stories.

\end{document}
