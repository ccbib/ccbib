\newcommand{\heading}{\chapter*}

\usepackage{url}
\DeclareUrlCommand\url{\def\UrlLeft{\hspace{0pt plus 2ex}}%
        \def\UrlRight{\hspace{0pt plus 2ex}}%
        \urlstyle{tt}}

\hyphenation{mo-no-poly car-ne-gie pro-ject pro-gress mo-dem rou-lette
  browse-wrap Use-net mon-as-tery mo-dems}
\hyphenation{co-me-dic polt-roon stove-pipe Ma-dame scru-ta-ble star-tling}
\hyphenation{heal-thily lim-ou-sines wrest-lers tan-trum push-over un-asked
  bras-siere bro-th-er}
\hyphenation{Can-a-da Fred-rick teen-agers wrest-ler Cha-vez Tho-mas 
  a-nom-a-lies sur-veil-lance ar-mies ref-u-gee ref-u-gees bris-tling
  eve-ning man-chu-ria man-chu-ri-an mid-terms me-di-um jap-a-nese}
\hyphenation{spend-ers googl-ing tour-ist tour-ists leg-end-ary}
\hyphenation{Dan-iel Van-essa Doc-to-row Ste-phen-son}
\hyphenation{de-cade sur-veilled rout-ers Wol-fen-stein teen-ager to-night}
\hyphenation{his-to-gram an-o-nym-ize Ga-la-xy sym-pa-the-tic}
\hyphenation{ar-phid ar-phids Found-ers}
\hyphenation{stran-ger stran-gers shoul-der-blades dump-ling dump-lings}
\hyphenation{ice-pack guard-rail Sep-tem-ber boot-able e-co-nom-ist}
\hyphenation{grown-ups roos-ter shoe-laces li-quid-i-ty}
\hyphenation{side-arm}
\hyphenation{wo-man wo-men tan-trum tan-trums Le-nin-grad zom-bie bunk-house}
\hyphenation{up-tick bio-mass}
\hyphenation{of-fi-cial of-fi-cial-ly gov-ern-ment}
\hyphenation{heal-thy Or-ville spark-ling}
\hyphenation{ves-ti-bule Law-rence au-to-no-mous}
\hyphenation{sau-sage door-step staf-fer}
\hyphenation{tree-trunk}
\hyphenation{to-ron-to}
\hyphenation{qua-dril-lion-aire qua-dril-lion-aires}
\hyphenation{sports-jack-et sports-jack-ets}
\hyphenation{work-space skunk-works}
\hyphenation{kings-ton}

\hyphenation{Va-len-tine}
\begin{document}

\title{After the Siege}
\author{Cory Doctorow}
\date{}
\maketitle

\heading{Forematter}
\begin{flushleft}
\setlength{\parskip}{0.7\baselineskip plus 0.3\baselineskip}
This story is part of Cory Doctorow’s 2007 short story collection
“Overclocked: Stories of the Future Present,” published by
Thunder’s Mouth, a division of Avalon Books. It is licensed under a
Creative Commons Attribution-NonCommercial-ShareAlike 2.5 license,
about which you’ll find more at the end of this file.

This story and the other stories in the volume are available at:

\url{http://craphound.com/overclocked}

You can buy Overclocked at finer bookstores everywhere, including
Amazon:

\url{http://www.amazon.com/exec/obidos/ASIN/1560259817/downandoutint-20}

In the words of Woody Guthrie:

“This song is Copyrighted in U.S., under Seal of Copyright
\#154085, for a period of 28 years, and anybody caught singin it
without our permission, will be mighty good friends of ourn, cause
we don’t give a dern. Publish it. Write it. Sing it. Swing to it.
Yodel it. We wrote it, that’s all we wanted to do.”

Overclocked is dedicated to Pat York, who made my stories better.
\end{flushleft}

\heading{Introduction to After the Siege}

My grandmother, Valentina Rachman (now Valerie Goldman), was a
little girl when Hitler laid siege to Leningrad, 12 years old. All
my life, she told me that she’d experienced horrors during the war
that I’d never comprehend, but I’m afraid that in my callow youth,
I discounted this. My grandmother wasn’t in a concentration camp,
and as far as I knew, all that had happened is that she’d met my
grandfather\dash{}a Red Army conscript\dash{}in Siberia, they’d deserted and
gone to Azerbaijan, and my father had been born in a refugee camp
near Baku. That’s dramatic, but hardly a major trauma.

Then I went to St Petersburg with my family in the summer of 2005,
and my grandmother walked us through the streets of her girlhood,
and for the first time, she opened up about the war to me. She
pointed out the corners where she’d seen frozen, starved corpses,
their asses sliced away by black-market butchers; the windows from
which she’d heaved the bodies of her starved neighbors when she
grew too weak to carry them.

The stories came one after another, washing over the sun-bleached
summertime streets of Petersburg, conjuring up a darker place,
frozen over, years into a siege that killed millions. Harrison E.
Salisbury’s “900 Days” is probably the best account of those years,
and the more I read of it, the more this story fleshed itself out
in my head. I wrote almost all of it on airplanes between London,
Singapore and San Francisco, in great, 5000- and 6000-word gouts.

My grandmother’s stories found an easy marriage with the
contemporary narrative of developing nations being strong-armed
into taking on rich-country copyright and patent laws, even where
this means letting their citizens die by the millions for lack of
AIDS drugs (Mandela’s son died of AIDS\dash{}imagine if one of the Bush
twins died of a disease that would be treatable except for the
greed of a South African company), destroying their education
system, or punishing local artists to preserve imported, expensive
culture.

The USA was a pirate nation for the first 100 years of its
existence, ripping off the patents and trademarks of the imperial
European powers it had liberated itself from with blood. By keeping
their GDP at home, the US revolutionaries were able to bootstrap
their nation into an industrial powerhouse. Now, it seems, their
descendants are bent on ensuring that no other country can pull the
same trick off.

\heading{After the Siege}
\textsf{(Published in The Infinite Matrix, January 2007)}

The day the siege began, Valentine was at the cinema across the
street from her building. The cinema had only grown the night
before and when she got out of bed and saw it there, all gossamer
silver supports and brave sweeping candy-apple red curves, she’d
begged Mata and Popa to let her go. She knew that all the children
in the building would spend the day there\dash{}didn’t the pack of them
explore each fresh marvel as a group? The week before it had been
the clever little flying cars that swooped past each other with
millimeters to spare, like pigeons ripping over your head. Before
that it had been the candy forest where the trees sprouted bon-bons
and sticks of rock, and every boy and girl in the city had been
there, laughing and eating until their bellies and sides ached.
Before that, the swarms of robot insects that had gathered up every
fleck of litter and dust and spirited it all away to the edge of
town where they’d somehow chewed it up and made factories out of
it, brightly colored and airy as an aviary. Before that: fish in
the river. Before that: the new apartment buildings. Before that:
the new hospitals. Before that: the new government offices.

Before that: the revolution, which Valentine barely
re\-mem\-bered\dash{}she’d been a little kid of ten then, not a big girl of
thirteen like now. All she remembered was a long time when she’d
been always a little hungry, and when everything was grey and dirty
and Mata and Popa whispered angrily at each other when they thought
she slept and her little brother Trover had cried thin sickly cries
all night, which made her angry too.

The cine was amazing, the greatest marvel yet as far as she was
concerned. She and the other little girls crowded into one of the
many balconies and tinkered with the controls for it until it
lifted free\dash{}how they’d whooped!\dash{}and sailed off to its own little
spot under the high swoop of the dome. From there the screen was a
little distorted, but they could count the bald spots on the old
war heroes’ heads as they nodded together in solemn congress,
waiting for the films to start. From there they could spy on the
boys who were making spitball mischief that was sure to attract a
reprimand, though for now the airborne robots were doing a flawless
job of silently intercepting the boys’ missiles before they
disturbed any of the other watchers.

The films weren’t very good in Valentine’s opinion. The first one
was all about the revolution as if she hadn’t heard enough about
the revolution! It was all they talked about in school for one
thing. And her parents! The \emph{quantities}, the positive
\emph{quantities} of times they’d sat her down to Explain the
Revolution, which was apparently one of their duties as bona fide
heroes of the revolution.

This was better than most though, because they’d made it with a
game and it was a game that Valentine played quite a lot and
thought was quite good. She recognized the virtual city modeled on
her own city, the avatars’ dance-moves taken from the game too,
along with the combat sequences and the scary zombies that had
finally given rise to the revolution.

That much she knew and that much they all knew: without the
zombies, the revolution would never have come. Zombiism and the
need to cure it had outweighed every other priority. Three
governments had promised that they’d negotiate better prices for
zombiism drugs, and three governments had failed and in the end,
the Cabinet had been overrun by zombies who’d torn three MPs to
bits and infected seven more and the crowd had carried the PM out
of her office and put her in a barrel and driven nails through it
and rolled it down the river-bank into the river, something so
horrible and delicious that Valentine often thought about it like
you poke a sore tooth with your tongue.

After that, the revolution, and a new PM who wouldn’t negotiate the
price of zombiism drugs. After that, a PM who built zombiism drug
factories right there in the city, giving away the drug in spray
and pill and needles. From there, it was only a matter of time
until everything was being made right there, copies of movies and
copies of songs and copies of drugs and copies of buildings and
cars and you name it, and that was the revolution, and Valentine
thought it was probably a good thing for everyone except the old PM
whom they’d put in the barrel.

The next movie was much better and Valentine and Leeza, who was her
best friend that week put their arms around each others’ shoulders
and watched it avidly. It was about a woman who was in love with
two men and the men hated each other and there was fighting and
glorious kissing and sophisticated, cutting insults, and oh they
dressed so \emph{well}! The audio was dubbed over from English but
that was OK, the voice-actors they used were very good.

After the second showing, she and her friends allowed their seats
to lower and set off for the concessions stand where they found the
beaming proprietors of the cinema celebrating their opening day
with chocolates and thick sandwiches and fish pies and bottles of
brown beer for the adults and bottles of fizzy elderflower for the
kids. Valentine saw the cute boy who Leeza liked and tripped him so
he practically fell in Leeza’s lap and that set the two of them to
laughing so hard they nearly didn’t make it back to their seats.

The next picture barely had time to start when it was shut off and
the lights came up and one of the proprietors stepped in front of
the screen, talking into his phone, which must have been dialed
into the cine’s sound system.

“Comrades, your attention please. We have had word that the city is
under attack by our old enemies. They have bombed the east quarter
and many are dead. More bombs are expected soon.” They all spoke at
once, horrified non-words that were like a panic, a sound that made
Valentine want to cover her ears.

“Please, comrades,” the speaker said. He was about sixty and was
getting a new head of hair, but he had the look of the old ones
who’d lived through the zombiism, a finger or two bent at a funny
angle by a secret policeman, a wattle under the chin of skin
loosened by some dark year of starvation. “Please! We must be calm!
If there are shelters in your apartment buildings and you can walk
there in less than ten minutes, you should walk there. If your
building lacks shelters, or if it would take more than ten minutes
to go to your building’s shelter, you may use some of the limited
shelter space here. The seats will lower in order, two at a time,
to prevent a rush, and when yours reaches ground, please leave
calmly and quickly and get to your shelter.”

Leeza clutched at her arm. “Vale! My building is more than ten
minutes’ walk! I’ll have to stay here! Oh, my poor parents! They’ll
think---”

“They’ll think you’re safe with me, Leeze,” Valentine said, hugging
her. “I’ll stay with you and both our parents’ can worry about
us.”

They headed for the shelter together, white-faced and silent in the
slow-moving crowd that shuffled down the steps into the first
basement, the second basement, then the shelter below that. A war
hero was handing out masks to everyone who entered and he had to go
and find more child-sized ones for them so they waited patiently in
the doorway.

“Valentine! You don’t belong here! Go home and leave room for we
who need it!” It was her worst enemy, Reeta, who had been her best
friend the week before. She was red in the face and pointing and
shouting. “She lives across the street! You see how selfish she is!
Across the street is her own shelter and she would take a spot away
from her comrades, send them walking through the street---”

The hero silenced her with a sharp gesture and looked hard at
Valentine. “Is it true?”

“My friend is scared,” she said, squeezing Leeza’s shaking
shoulder. “I will stay with her.”

“You go home now,” the hero said, putting one of the child-sized
masks back in the box. “Your friend will be fine and you’ll see her
in a few minutes when they sound the all clear. Hurry now.” His
voice and his look brooked no argument.

So Valentine fought her way up the stairs\dash{}so many headed for the
shelter!\dash{}and out the doors and when she stepped out, it was like a
different city. The streets, always so busy and cheerful, were
silent. No air-cars flew overhead. It was silent, silent, like the
ringing in your ears after you turn your headphones up too loud. It
was so weird that a laugh escaped her lips, though not one of
mirth, more like a scared laugh.

She stood a moment longer and then there was a sound like far-away
thunder. A second later, a little wind. On its heels, a bigger
wind, icy cold and then hot as the oven when you open the door,
nearly blowing her off her feet. It \emph{smelled} like something
dead or something deadly. She \emph{ran} as fast as she could
across the street, pounding hell for leather to her front door.
Just as she reached for it, there was a much louder thunderclap,
one that lifted her off her feet and tossed her into the air,
spinning her around. As she spun around and around, she saw the
brave red dome of the cine disintegrate, crumble to a million
shards that began to rain down on the street. Then the boom dropped
her hard on the pavement and she saw no more.

\tb

The day after the siege began, the doctor fitted Valentine for her
hearing aid and told her to come back in ten years for a battery
change. She hardly felt it slide under her skin but once it was
there, the funny underwatery sound of everything and everyone
turned back into bright sound as sharp as the cine’s had been.

Now that she could hear, she could speak, and she grabbed Popa’s
hands. “The cine!” she said. “Oh, Popa, the cine, those poor
people! What happened to them?”

“The work crews opened the shelter ten hours later,” Popa said. He
never sugar-coated anything for her, even though Mata disapproved
of talking to her like an adult. “Half of them died from lack of
air\dash{}the air re-circulators were damaged by the bomb, and the
shelter was air-tight. The rest are in hospital.”

She cried. “Leeza---”

Mata took her hands. “Leeza is fine,” she said. “She made sure we
told you that.”

She cried harder, but smiling this time. Trover was on her mother’s
hip, and looking like he didn’t know whether to stay quiet or pitch
one of his famous tantrums. Automatically, Valentine gave him a
tickle that brought a smile that kept him from bursting out in
tears.

They left the hospital together and walked home, though it was far.
The Metro wasn’t running and the air-cars were still grounded. Some
of the buildings they passed were nothing but rubble, and there
robots and people labored to make sense of them and get them
reassembled and back on their feet.

It wasn’t until the next day that she found out that Reeta had been
killed under the cine. She threw up the porridge she’d had for
breakfast and shut herself in her room and cried into her pillow
until she fell asleep.

\tb

Three days after the siege began, Mata went away.

“You can’t go!” Popa shouted at her. “Are you crazy? You can’t go
to the front! You have two small children, woman!” He was red-faced
and his hands were clenching and unclenching. Trover was having a
tantrum that was so loud and horrible that Valentine wanted to rip
her hearing aids out.

Mata’s eyes were red. “Harald, you know I have to do this. It’s not
the ‘front’\dash{}it’s our own city. My country needs me\dash{}if I don’t help
to fight for it, then what will become of our children?”

“You never got over the glory of fighting, did you?” Her father’s
voice was bitter in a way that she’d never heard before. “You’re an
addict!”

She held up her left hand and shook it in his face. “An addict! Is
\emph{that} what you think?” Her middle finger and little finger on
that hand had never bent properly in all of Valentine’s memory, and
when Valentine had asked her about it, she’d said the terrible word
\emph{knucklebreakers} which was the old name for the police. “You
think I’m addicted to \emph{this}? Harald, honor and courage and
patriotism are \emph{virtues} no matter that you would make them
into vices and shame our children with your cowardice. I go to
fight now, Harald, and it’s for \emph{all} of us.”

Popa couldn’t find another word to speak in the two seconds it took
for Mata to give her two children hard kisses on the foreheads and
slam out the door, and then it was Valentine and Popa and Trover,
still screaming. Her father fisted the tears out of his eyes, not
bothering to try to hide them, and said, “Well then, who wants
pancakes?”

But the power was out and he had to make them cereal instead.

\tb

Two weeks after the siege began, her mother didn’t come home and
the city came for her father.

“Every adult, comrade, every adult fights for the city.”

“My children---” he sputtered. Mata hadn’t been home all night, and
it wasn’t the first time. She and Popa barely spoke anymore.

“Your girl there is big enough to look after herself, aren’t you
honey?” The woman from the city was short and plump and wore heavy
armor and was red in the face from walking up ten flights to get to
their flat. The power to the elevators was almost always out.

Valentine hugged her father’s leg. “My Popa will fight for the
city,” she said. “He’s a hero.”

He was. He’d fought in the revolution and he’d been given a medal
for it. Sometimes when no one was looking, Valentine took out her
parents’ medals and looked at their tiny writing, their shining,
unscratchable surfaces, their intricate ribbons.

The woman from the city gave her father a look that said,
\emph{You see, a child understands, what’s your excuse?} Valentine
couldn’t quite feel guilty for taking the woman’s side. Leeza’s
parents fought every day.

“I must leave a note for my wife,” he said. Valentine realized that
for the first time in her life her parents were going to leave her
\emph{all on her own} and felt a thrill.

\tb

Two weeks and one day after the siege began, her Mata came home and
the city came for Valentine.

Mata was grimy and exhausted, and she favored one leg as she went
about the flat making them cold cereal with water\dash{}all the milk had
spoiled\dash{}and dried fruits. Trover looked curiously at her as though
he didn’t recognize her, but eventually he got in her way and she
snapped at him to move already and he pitched a relieved fit,
pounding his fists and howling. How that little boy could howl!

She sat down at the table with Valentine and the two of them ate
their cereal together.

“Your father?”

“He said he was digging trenches\dash{}that’s what he did all day
yesterday.”

Her mother’s eyes glinted. “Good. We need more trenches. We’ll
fortify the whole city with them, spread them out all the way to
their lines, trenches we can move through without being seen or
shot. We’ll take the war to those bastards and slip away before
they know we’ve killed them.” Mata had apparently forgotten all
about not talking to Valentine like a grownup.

The knock at the door came then, and Mata answered it and it was
the woman from the city again. “Your little girl,” she said.

“No,” Mata said. Her voice was flat and would not brook any
contradiction. She’d bossed her nine brothers\dash{}Valentine’s uncles,
now scattered to the winds\dash{}and then commanded a squadron in the
revolution, and no one could win an argument with her. As far as
Valentine knew, no one could win an argument with her.

“No?” The woman from the city said. “No is not an option,
comrade.”

Mata drew herself up. “My husband digs. I fight. My daughter cares
for our son. That’s enough from this family.”

“There are old people in this building who need water brought for
them. There’s a creche for the boy underground, he’ll be happy
enough there. Your little girl is strong and the old people are
weak.”

“No,” her mother said. “I’m very sorry, but no.” She didn’t sound
the least bit sorry.

The woman from the city went away. Mata sat down and went back to
eating her cereal with water without a word, but there was another
knock at the door fifteen minutes later. The woman from the city
had brought along an old hero with one arm and one eye. He greeted
Mata by name and Mata gave him a smart salute. He spoke quietly in
her ear for a moment. She saluted him again and he left.

“You’ll carry water,” Mata said.

Valentine didn’t mind, it was a chance to get out of the flat. One
day of baby-sitting the human tantrum had convinced her that any
chore was preferable to being cooped up with him.

She carried water that day. She’d expected to be balancing buckets
over her shoulders like in the schoolbooks, but they fitted her
with a bubble-suit that distributed the weight over her whole body
and then filled it up with a hose until she weighed nearly twice
what she normally did. Other kids were in the stairwells wearing
identical bubble-suits, sloshing up the steps to old peoples’ flats
that smelled funny. The old women and men that Valentine saw that
day pinched her cheeks and then emptied out her bubble-suit into
their cisterns.

It was exhausting work and by the end of the day she had stopped
making even perfunctory conversation with the other water-carriers.
The old people she met at the day’s end were bitter about being
left alone and thirsty all day and they snapped at her and didn’t
thank her at all.

She picked Trover up from the creche and he demanded that he be
carried and she had half a mind to toss him down the stairs. But
she noticed that he had a bruise over his eye and his hands and
face were sticky and dirty and she decided that he’d had a hard day
too. Mata and Popa weren’t home when they got there so Valentine
made dinner\dash{}more cold cereal and some cabbage with leftover
dumplings kept cool in a bag hung out the window\dash{}and then when they
still hadn’t returned by bedtime, Valentine tucked Trover in and
then fell asleep herself.

\tb

One month after the siege began, Valentine’s mother came home in
tears.

“What is it, Mata?” Valentine said, as soon as her mother came
through the door. “Are you hurt?” Her mother had come home hurt
more than once in the month, bandaged or splinted or covered in
burn ointment or hacking at some deep chemical irritation in her
throat and nose and lungs.

Her mother’s eyes were swollen like they had been the day she’d
been caught by the gas and they’d had to do emergency robot
field-surgery on them. But there were no sutures. Tears had swollen
her eyes.

“New trenchbuster missiles on the eastern front,” she said. “The
anti-missiles are too slow for them.” She sobbed, a terrible
terrifying sound that Valentine had never heard from her mother.
“The bastards are trading with the EU and the Americans for better
weapons, they say they’re on the same side, they say we are lawless
thieves who deprive them all of their royalties---”

Valentine had heard that the Americans and the EU had declared for
the other side, while the Russians and the Koreans and the
Brazilians had declared for the city. The war gossip was
everywhere. The old people didn’t pinch her cheeks when she brought
water, not anymore\dash{}they told her about the war and the enemies
who’d come to drive them back into the dark ages.

“Mata, are you \emph{hurt}?” Her mother was covering her face with
her hands and sobbing so loudly it drowned out the tantrum Trover
threw every night the second she came through the door.

Her shoulders shook. She gulped her sobs. Then she lowered her wet,
snotty, sticky hands and wiped them on the thighs of her jumpsuit.
She hugged Valentine so hard Valentine felt her skinny ribs creak.

“They killed your father, Vale,” she said. “Your father is dead.”

Valentine stood numb for a moment, then pulled free of her mother’s
hug.

“No,” she said, calmly. “Popa is digging away from the front, where
it’s safe.” She’d expected that her \emph{mother} would die, not
her \emph{father}. She’d known that all along, since her mother
stepped out the door of the flat talking of heroism. Known it
fatalistically and never dwelt on it, never even admitted it. In
her mind, though, she’d always seen a future where her father and
Trover and she lived together as heroes of this war, which would
surely be over soon, and visited her mother’s memorial four times a
year, the way they did the memorials for the comrade heroes who’d
been martyred in the revolution.

The death toll was gigantic. Three apartment buildings had
disappeared on her street with no air raid warning, no warning of
any kind. All dead. Why should her brave mother live on?

“No,” she said again. “You’re mistaken.”

“\emph{I saw the body!}” her mother said, shrieking like Trover. “I
held his head! He is \emph{dead}, Vale!”

Valentine didn’t understand what her mother was saying, but she
certainly didn’t want to hang around the flat and listen to this
raving.

She turned on her heel and walked out of the flat. It was full dark
out and there was snow on the ground and wet snow whipping along in
the wind and she didn’t have her too-small winter coat on, but she
wasn’t going to stay and listen to her mother’s nonsense.

On the corner a man from the city told her she was breaking curfew
and told her to go home or she’d end up getting herself shot. She
shivered and glared at him and ignored him and set off in a random
direction. She certainly wasn’t going to stand on that corner and
listen to his lunacy.

There were soldiers drinking in a cellar on another street and they
called out to her and what they said wasn’t the kind of thing you
said to a little girl, though she knew well enough what it meant.
Now she was cold and soaked through and shivering uncontrollably
and she didn’t know where she was and her father was\dash{}

She began to run.

Someone from the city shouted at her to stop and so she pelted
through the ruins of a bombed building and then down one of the old
streets from before the revolution, one of the streets they hadn’t
yet straightened out and rebuilt. The enemy hadn’t bombed it yet,
and she wondered if that was because this was the kind of dark and
broken and smelly street they wanted the city to be returned to, so
they’d left it untouched as an example of what the defenders should
be working towards if they wanted to escape with their lives.

Down the street she ran, and then down an alley and another street.
She stopped running when she came to a dead-end and her chest
heaved. Running had warmed her up a little, but she hadn’t had much
to eat except cabbage and cold cereal with water for weeks and she
couldn’t run like she used to.

The cold stole back over her. It was full dark and the blackout
curtains on the windows meant that not a sliver of light escaped.
The moonless cloudy night made everything as dark as a cave.

Finally, she cried. She hadn’t cried since she found out that Reeta
had died\dash{}she hadn’t even \emph{liked} Reeta, but to have someone
die that soon after your seeing them was scary like you had almost
died, almost.

The wizard came on her there, weeping. He appeared out of the mist
carrying a little light the size of a pea that he cupped in his
hand to muffle most of the light. He was about her father’s age,
but with her mother’s look of having survived something terrible
without having survived altogether. He dressed like it was the old
days, in fancy, bright-colored clothes, and he was well-fed in a
way that no one else in the city was.

“Hello there,” he said. He got down on his hunkers so he could look
her in the eye. “Why are you crying?”

Valentine hated grownups who patronized her and the wizard sounded
like he believed that no little girl could possibly have anything
\emph{real} to cry about.

“My dad died in the war today,” she said. “In a trench.”

“Oh, the American trench-busters,” he said, knowingly. “Lots of
children lost their daddies today, I bet.”

That made her stop crying. Lots of children. Lots of
daddies\dash{}fathers, she hated the baby-word “daddy.” Mothers, too.

“Let’s get you cleaned up, put a coat on you, feed you and send you
home, all right?”

She looked warily at him. She knew all about strange men who
offered to take you home. But she had no idea where she was and she
was dark and shivering and couldn’t stop.

“My mother is a hero, and a soldier, and she’s killed a lot of
men,” Valentine said.

He nodded. “I shall keep that in mind,” he said.

The wizard lived in the old town, in an old building, but inside it
was as new as anything she had ever seen. The walls swooped and
curved, the furniture was gaily colored and new, like it had just
been printed that day. There was so much \emph{light}\dash{}they’d been
saving it at her building. There was so much \emph{food}! He gave
her hamburgers and fizzy elderflower, then steak-frites, then rich
dumplings as big as her fist stuffed with goose livers. He had
working robots, lots of them, and they scurried after him doing the
dishes and tidying and wiping up the slushy footprints.

And when they arrived and he took her coat, old familiar
laser-lights played over her, the kind of every\-where-at-once
measuring lasers that they used to have at the clothing stores. By
the time dinner was done, there were two pairs of fresh trousers,
two wooly jumpers, a heavy winter coat, three pairs of white cotton
pants (all her pants had gone grey once she’d started having to
launder them, rather than get them printed fresh on Sundays) and
a\dash{}

“A bra?” She gave him a hard look. She had the knife she’d used on
the hamburger in her hand. “My mother taught me to kill,” she
said.

The wizard had a face that looked like he spent a lot of time
laughing with it, and so even when he looked scared, he also looked
like he was laughing. He held up his hands. “It wasn’t my idea.
That’s just the programming. If the printer thinks you need a bra,
it makes a bra.”

Leeza had a bra, though Valentine wasn’t convinced she needed it.
But she had noticed a certain uncomfortable jiggling weight
climbing the stairs, hadn’t she? Running? She hadn’t looked in the
mirror in\dash{}Well, since the siege, practically.

“There’s a bathroom there to change into,” he said.

His bathroom was clean and neat and there were six toothbrushes
beside the sink in a holder.

“Who else lives here?” she said, coming out in her new clothes (the
bra felt \emph{really} weird).

“I have a lot of friends who come and see me now and again. I hope
you’ll come back.”

“How come your place is like the war never happened?”

“I’m the wizard, that’s why,” he said. “I can make magic.”

His robots tied up her extra clothes in waterproof grip sheets for
her, then helped her into a warm slicker with a hood. “Tell your
mother that you met someone from the city who fed you and gave you
a change of clothes,” he said, holding open the door. He’d
explained to her where to go from there to get out to the old
shopping street and from there she could manage on her own,
especially since he’d given her one of his little pea-lights to
carry with her.

“You’re not from the city,” she said.

“You got me,” he said. “So tell her you met a wizard.”

She thought about what her mother would say to that, especially
when that was the answer to the question
\emph{Where have you been?} “I’ll tell her I met someone from the
city,” she said.

“You’re a clever girl,” he said.

\tb

One week after her father died, Valentine stopped carrying water.

“There’s not enough food,” her mother said, over a breakfast of
nothing but dried fruit\dash{}the cereal was gone. “If you---” she
swallowed and looked out the window. “If you dig in the trenches,
we’ll get 150 grams of bread a day.”

Valentine looked at Trover. He hadn’t had a tantrum in days. He
didn’t cry or even speak much anymore.

“I’ll dig.”

She dug.

\tb

Six months after her father died, Valentine stood in the queue for
her bread. It was now the full heat of summer and the clothes the
wizard had given her had fallen to bits the way all printer clothes
did. She was wearing her father’s old trousers, cut off just below
the knee, and one of his shirts, with the sleeves and collar cut
off. All to let a little of the lazy air in and to let a little of
the sluggish sweat out. She was dirty and tired the way she always
was at day’s end.

She was also so hungry.

She and her mother didn’t talk much anymore, but they didn’t have
to. Her mother was sometimes away on long missions, and
increasingly longer. She was harrying the enemy with the guerilla
fighters, and living on pine-cone soup and squirrels from the
woods.

Trover stayed over at the creche some nights. A lot of the little
ones did. Who had the strength to carry a little boy up the stairs
at the end of a day’s digging, at the end of three days’ hard
fighting in the woods?

The bread-rations were handed out in the spot where the cine once
stood. She couldn’t really remember what it had been like, though
she remembered Reeta, the things Reeta had said that had made her
leave the shelter, which had probably saved her life. Poor Reeta.
Little bitch.

She was so hungry, and the line moved slowly. She had her chit from
the boy from the city who oversaw the ditch digging in her part of
the ditches. He was only a little older than her but he couldn’t
dig because his hands had been mutilated when a bomb went off near
him. He kept them shoved in his pockets, but she’d seen them and
they looked like the knucklebreakers had given them a good
seeing-to. Every finger pointed a different direction, except for
the ones that were missing altogether. There was also something
wrong with him that made him sometimes stop talking in the middle
of a sentence and sit down for a moment with his head tilted back.

The chit, though\dash{}the boy always gave her her chit, and the chit
could be redeemed for bread. If she left Trover in the creche they
would feed him. If Mata didn’t come home from the fighting again
tonight, the bread would be hers, and the cabbage, too.

\tb

Eight months after her father died, her mother stayed away in the
fighting for three weeks and Valentine decided that she was dead
and started sleeping in her mother’s bed. Valentine cried a little
at first, but she got used to it. She started to negotiate with one
of the women who lived on the floor below to sell her narrow little
bed for 800 grams of bread, 40 grams of butter and\dash{}though she
didn’t really believe in it\dash{}100 grams of ground beef.

She never found out if the woman downstairs had any ground
beef\dash{}where would you get ground beef, anyway? Even the cats and
dogs and rats were all gone! For Valentine’s mother came home after
three weeks and it turned out that she’d been in hospital all that
time having her broken bones mended, something they could still do
for some soldiers.

Mata came through the door like an old woman and Valentine looked
up from the table where she’d been patiently feeding silent Trover
before collapsing to sleep again. Valentine stood and looked at her
and her Mata looked at Valentine and then her mother hobbled across
the room like an old woman and gave Valentine a fierce, hard, long
hug.

Valentine found she was crying and also found that silent little
Trover had gotten up from the table and was hugging them both. He
was tall, she realized dimly, tall enough to reach up and hug her
at the waist instead of the knees, and when had \emph{that}
happened?

Her mother ate some of the dinner they’d had, and took a
painkiller, the old kind that came in pill form that were now
everywhere. Take a few of them and you would forget your problems,
or so hissed the boys she passed in the street, though she passed
them without a glance or a sniff.

Soon Mata was asleep, back in her bed, and Valentine was back in
her bed, too, but she couldn’t sleep.

Under her bed she had the remains of her grip sheet parcel, one of
the precise robot-knots remaining. In that parcel was her winter
galosh, just one, the other had been stolen the winter before while
she’d had them both off to rub some warmth back into her toes
before going back to the digging.

In the toe of the galosh, there was a pea-sized glowing light.
She’d never considered selling it for bread, though it was very
fine. Its light seemed too bright in the dark flat, so she took it
outside into the hot night, and used it to light her way on a
secret walk through the old streets of her dirty city.

\tb

Nine months after her father died, winter had sent autumn as a
threatening envoy. The bread ration was cut to 120 grams, and there
were sometimes pebbles in the bread that everyone knew were there
to increase the weight.

She was proud that when the bread was bad, she and the other
diggers cursed the enemy and not the city. Everyone knew that no
one had it any better. They fought and suffered together.

But she was so hungry all the time, and you couldn’t eat pride. One
day she was in the queue for bread and reached out with her
trembling hands to take her ration and then she turned with it and
in a flash, a man old enough to be her father had snatched it out
of her hands and run away with it!

She chased after him and the shrill cries of the women followed
them, but he knew the rubble-piles well and he dodged and weaved
and she was so tired. Eventually she sat down and wept.

That was when she saw her first zombie. Zombiism had been
eliminated when she was practically a baby, just after the
revolution, years and years ago.

But now it was back. The zombie had been a soldier, so maybe
zombiism was coming back in the gas attacks that wafted over the
trenches. His uniform hung in rags from his loose limbs as he
walked in that funky, disco-dancer shuffle that meant zombie as
clearly as the open drooling mouth and the staring, not-seeing
eyes. They were fast, zombies, though you could hardly believe it
when they were doing that funky walk. Once they saw prey, they
turned into race horses that tore over anything and everything in
their quest to rip and bite and rend and tear, screaming
incoherencies with just enough words in them to make it clear that
they were angry\dash{}so so angry.

She scrambled up from the curb she’d been weeping on and began to
back away slowly, keeping perfectly silent. You needed to get away
from zombies and then tell someone from the city so they could
administer the cure. That’s how you did it, back in the old days.

The zombie was shambling away from her anyway. It would pass by
harmlessly, but she had to \emph{get away} in any event, because it
was a \emph{zombie} and it was \emph{wrong} in just the same way
that a giant hairy spider was wrong (though if she found something
giant and hairy today, she’d take it home for the soup-pot).

She didn’t kick a tin or knock over a pile of rubble. She was
perfectly stealthy. She hardly breathed.

And that zombie saw her anyway. It roared and charged. Its mouth
was almost toothless, but what teeth remained there gleamed. It had
been a soldier and it had good boots, and they crunched the broken
glass and the rubble as it pelted for her. She shrieked and ran,
but she knew even as she did that she would never outrun it. She
was starved and had already used all her energy chasing the
\emph{bastard}, the \emph{fucking bastard} who’d taken her bread.

She ran anyway, but the sound of the zombie’s good boots drew
closer and closer, coming up on her, closing on her. A hand thumped
her shoulder and scrabbled at it and she spied a piece of steel
bar\dash{}maybe it had been a locking post for a hover-car in the golden
days\dash{}and she snatched it up and whirled around.

The zombie grabbed for her and she smashed its wrist like an
old-timey schoolteacher with a ruler. She heard something crack and
the zombie roared again. “Bread fight asshole kill hungry!” is what
it sounded like.

But one of its hands was now useless, flopping at its side. It
charged her, grappling with her, and she couldn’t get her bar back
for a swing. Its good hand was in her hair and it didn’t stink,
that was the worst part. It smelled like fresh-baked bread. It
smelled like flowers. Zombies smelled \emph{delicious}.

The part of her brain that was detached and thinking these thoughts
was not the part in the front. That part was incoherent with equal
parts rage and terror. The zombie would bite her soon and that
would be it. In a day, she’d be a zombie too, in need of medicine,
and how many more would she bite before she got cured.

In that moment, she stopped being angry at the zombie and became
angry at the besiegers. They had been abstract enemies until then,
an unknowable force from outside her world, but in that moment she
realized that they were \emph{people} like her, who could suffer
like her and she wished that they would. She wished that their
children would starve. She wished the parents would die. The old
people shrivel unto death in their dry, unwatered flats. The
toddlers wander the streets until sunburn or cold took them.

She screamed an animal scream and pushed the zombie off her with
her arms and legs, even her head, snapping it into the zombie’s
cheekbone as hard as she could and something broke there too.

The zombie staggered back. They couldn’t feel pain, but their
balance was a little weak. It tottered and she went after it with
the bar. One whack in the knee took it down on its side. It reached
with its good arm and so she smashed that too. Then the heaving
ribs. Then the face, the hateful, leering, mouth-open-stupid face,
three smashes turned it into ruin. The jaw hung down to its chest,
broken off its face.

A hand seized her and she whirled with her bar held high and nearly
brained the soldier who’d grabbed her. He wasn’t a zombie, and he
had his pistol out. It was pointed at her. She dropped her bar like
it was red hot and threw her arms in the air.

He shoved her rudely aside and knelt beside the zombie\dash{}the
\emph{soldier zombie} she realized with a sick lurch\dash{}that she’d
just smashed to pieces.

The soldier’s back was to her, but his chest was heaving like a
bellows and his neck was tight.

“Please,” she said. “After they give him the cure, they can fix his
bones. I had to hit him or he would have killed me. He would have
infected me. You see that, right? I know it was wrong, but---”

The soldier shot the zombie through the head, twice.

He turned around. His face was streaming with tears. “There is no
cure, not for this strain of zombiism. Once you get it, you die. It
takes a week. Slower than the old kind. It gives you more time to
infect new people. Our enemies are crafty crafty, girl.”

The soldier kicked the zombie. “I knew his brother. I commanded him
until he was killed by a trenchbuster. The mother and father were
killed by a shell. Now he’s dead and that’s a whole family gone.”

The soldier cocked his head at her and examined her more closely.
“Have you been bitten?”

“No,” she said, quickly. The gun was still in his hand. There was
no cure.

“You’re sure?” he said. His voice was like her father’s had been
when she skinned her knee, stern but sympathetic. “If you have,
you’d better tell me. Better to go quick and painless than like
this thing.” He kicked the zombie again.

“I’m sure,” she said. “Have you got any bread? A man stole my
ration.”

The soldier lost interest in her when she asked him for bread.
“Goodbye, little girl,” he said.

That night, she had a fever. She was so hot. She got them all the
time, everyone did. Not enough food. No heat. No vegetables and
vitamins. You always got fevers.

But she was so hot. She took off her clothes and let the cool air
blow over her skin on her narrow bed. Trover was sleeping on the
floor nearby\dash{}he had outgrown his crib long since\dash{}and he stirred
irritably as she felt that air cool her sizzling skin.

She ran her fingertips lightly over her body. She was never naked
anymore. If you were lucky, you washed your face and hands every
day, but baths\dash{}they were cold and miserable and who wanted to haul
water for them anyway?

Her breasts were undeniable now. Her blood had started a few months
back, then stopped. Starvation, she knew, that’s what did it. But
there was new hair in her armpits and at her groin.

She crossed her arms over her chest and hugged herself. That’s when
she found the bite on her shoulder, just where it met her neck. It
was swollen like a quail egg\dash{}the chocolate quail eggs from before,
that had grown on the trees, she could taste them even now\dash{}and so
hot it felt like a coal. In the middle of the egg, at its peak, the
seeping wound left behind by one of the zombie’s few teeth.

Now she was cold as ice, shivering nude on her thin ruin of a bed
with her thin ruin of a body. She would be dead in a week. It was a
death sentence, that bite.

And she wouldn’t go clean. She’d shamble and scream and bite. Maybe
Trover. Maybe Mata. Maybe she’d find Leeza and give her a hard bite
before she went.

Her breath was coming in little pants now. She bit her lip to keep
from screaming.

She pulled her clothes back on as quietly as she could and slipped
out into the night to find the wizard, clutching her pea light.
Many times she’d walked toward his house in the night, but she’d
always turned back. Now she had to see him.

She passed three zombies in the dark, two dead on the ground and
riddled with bullet holes, one leaning out a fifth-story window and
screaming its incoherent rage out at the city.

As she drew nearer to the wizard’s door, an unshakable fatal
conclusion gripped her: he was long gone, shot or gassed, or simply
moved to somewhere else. It had been months and months since he’d
given her the printer-clothes and the dumplings and surely he was
dead now. Who wasn’t?

Her steps slowed as she came to his block. Each step was the work
of half a minute or more. She didn’t want to see his old door
hanging off its hinges, didn’t want to see the ruins of the brave
curves and swoops of his flat and his furniture.

But her steps took her to the door, and it was shut and silent as
any of the doors in the street. Nothing marked it beyond the grime
of the city and the scratches and scrapes that no one painted over
any longer.

She tried the knob. It was locked. She knocked. Silence. She
knocked again, harder. Still silence. Crying now, she thundered on
the door with her fists and kicked it with her feet. He was gone,
gone, gone, and she would be dead in a week.

Then the door opened. It wasn’t the wizard, but a well-fed blonde
woman in a housecoat with slippers. She was beautiful, a
movie-star, though maybe that was just because she wasn’t starved
nearly to death.

“Girl, you’d better have a good reason for waking up the whole
fucking street at three in the morning.” Her voice wasn’t unkind,
though she was clearly annoyed.

“I need to see---” She dropped her voice at the last moment. “I need
to see the wizard.”

“Oh,” the woman said, comprehension dawning on her face. “Oh, well
then, come on in. Any friend of the wizard.”

The flat was just as she remembered it from that long ago night.
The woman gestured at the kitchen and coffee-smells began to
emanate from it. Valentine’d forgotten the smell of coffee, but now
she remembered it.

“I’ll go wake up his majesty, then,” the woman said. “Just sit
yourself down.”

Valentine sat perched on the edge of the grand divan that twisted
and curved along one wall of the sitting room. She knew that the
seat of her trousers\dash{}filthy even before her tussle with the
zombie\dash{}would leave black marks on its brave red upholstery.

The conversation from down the corridor was muffled but the tone
was angry. Valentine felt her cheeks go hot, even through the
fever. This place was still civilized and she’d brought the war to
it.

Then the wizard came into the sitting room and waved the lights up
to full bright, wincing away from the sudden illumination. He
squinted at her.

“Do I know you?” he said.

Her tongue caught in her mouth. In his pajamas with his hair
mussed, he still looked every inch the wizard.

“I;” She couldn’t finish. “I---” She tried again. “You gave me
clothes. My mother is a soldier.”

He snapped his fingers and grinned. “Oh, the
\emph{soldier’s daughter}. I remember you now. You counted the
toothbrushes. You’re a bright girl.”

“She’s a walking skeleton.” The beautiful blonde woman was in the
kitchen, tinkering with the cooker. It was the pre-war kind,
capable of printing out food with hardly any intervention.
Valentine was hypnotized by her fingers.

“You want sandwiches and fish-fingers?”

“Start her with some drinking chocolate, Ana,” the wizard said.
“Hot and then a milkshake. Little girls love chocolate.”

She hadn’t tasted chocolate in\dash{}She didn’t know. Her mouth was
flooded with saliva. The woman, Ana, pressed some more buttons and
then took down a bottle of rum from a cupboard.

“Will you have rum in yours, little girl?”

“I---”

“She’s a little young for rum, Ana,” the wizard said. He sat down
on one of his curvy sofas and it embraced him and unfurled a
foot-rest.

“I’ll have rum,” she said. She was dying, and she wouldn’t die
without at least having one drink, once.

“Good girl,” Ana said. “There’s a war on, after all.” She poured a
liquid with the consistency of mud into a tall mug and then added a
glug or two of rum and pushed it across the counter then fixed one
for herself. “Come and get it, no waitress service here.”

Valentine took off her too-small shoes and walked over the carpet
in her dirty bare feet. It felt like something she barely
remembered. Grass?

The chocolate smelled wonderful. Wonderful was the word for it. It
made her full of wonder. Rich. Something from another planet\dash{}from
heaven, maybe.

She lifted the mug and felt its warmth seep into her hands. She
took a tentative sip and held it in her mouth.

It was spicy! Was chocolate spicy? She didn’t think so! The rum
made her tongue tingle and the heat made its fumes rise in her
head, carrying up the chocolate taste and the peppers. Her eyes
streamed. Her ears felt like they were full of chocolate.

She swallowed and gasped and the wizard laughed. She looked at
him.

“Ana’s recipe. She adds the chilies. I think it’s lovely, don’t
you? Aztec chocolate, we call it.”

She took another mouthful, held it, swallowed. The chocolate was in
her tummy too, and there was a feeling there, a greedy feeling, a
\emph{more} feeling. She drained the glass. Ana and the wizard both
laughed.

Ana handed her a tall frosted metal cup with a mountain of whipped
cream on top and a straw sticking out. “Chocolate malted,” she
said. “The perfect chaser.”

Transfixed in her bare feet on the carpet, she drank this. A cold
headache hit her between the eyes and that didn’t stop her from
going on drinking. Wow! Wow! Were there tastes like this? Did
things really taste this good?

The straw made slurping noises as she chased down the last of the
rich liquid.

“Sit now,” the wizard said. “Let that work its magic and then we’ll
put some food down your gullet.”

She walked to the sofa. It was like walking on the deck of a
rocking ship, or on the surface of the moon. Everything slid
beneath her. \emph{I’m drunk}, she thought.
\emph{I’m 14 years old and I am drunk as a skunk.}

She lowered herself carefully and sat up as straight as she could.

“Now, young lady, what brings you to my home in the middle of the
night?”

She remembered the bite on her neck and thought for a panicky
second that she would throw up.

“I needed to talk to you,” she said. “I needed some help.”

“What kind of help?”

She couldn’t say it. She had the new kind of zombiism and the
soldier had explained it clearly\dash{}the cure for zombiism now was a
bullet to the head.

Then she knew what she must say. The chocolate helped. Her family
would love chocolate. “I’m going away soon and my mother and
brother won’t be able to take care of themselves. I need help to
keep them safe once I go.”

“Where are you going?”

The drink made it hard to think, but that was balanced out by that
precious and magical feeling of fullness in her belly. Her mind
flew over all the possible answers.

“I have found someone who’ll take me out of the city and to a safe
place.”

“Are there safe places?” Ana said.

“Oh, Ana, you cynic,” the wizard said. “There are many, many safe
places. The world is full of them. They are the exception, not the
rule. Isn’t that why you’ve come here?”

“We’re not talking about why \emph{I} came here,” Ana said. She
nodded back at Valentine and made a little scooting hand-gesture at
her. Valentine couldn’t decide if she liked Ana, though Ana was
very pretty.

“I need help for my family,” she said.

“And why would I give help?” the wizard said. He was still smiling,
but that face of his, the face that looked like he’d been wounded
and never quite healed, it was set in an expression that scared her
a little. His eyes glittered in the low light of the swooping
sitting room. She found that she had slumped against the sofa and
now it had her in its soft embrace.

“Because you helped me before,” she said.

“I see,” the wizard said. “So you assumed that because I’d been
generous\dash{}very, very generous\dash{}to you once before that I’d be
generous again? You repay my favor with a request for another one?

Valentine shook her head.

“No?”

“I will find a way to repay it,” she said. “I can work for you.”

“I don’t need any ditches dug around here, thank you.”

Somewhere in the flat, a door opened and shut. She heard muffled
voices. Lots of them. The flat was full of people, somewhere.

“I can do lots of kinds of work,” she said. She attempted a smile.
She didn’t know what she was offering him, but she knew that she
was too young to be offering it. And besides, with zombiism, you
shouldn’t do that sort of thing. She would be safe, though, and
careful, so that he would live to help her family.

Ana crossed past her in a flash and then she smacked the wizard, a
crack across the face hard enough to rock his head back. His cheek
glowed with the print of her open hand.

“Don’t you toy with this little girl,” she said. “You see how
desperate she is? Don’t you toy with her.”

She whirled on Valentine, who stood her ground even though she
wanted to shrink away. If she was old enough to offer herself to
the wizard, she was old enough to stand her ground before this
beautiful, well-fed blonde woman.

“And you,” Ana said. “You aren’t a fool, I can tell. So don’t act a
fool. There are a thousand ways to survive that don’t involve lying
on your back, and you must know them or you wouldn’t have survived
this long. Be smart or be gone. I won’t watch you make a tragedy of
yourself.”

“Ana, what do you know about survival?” the wizard said. He had one
hand to his cheek, and he was giving her the same glittering look
he’d given to Valentine a moment before.

“Just don’t play with her,” Ana said. “Help her or get rid of her,
but don’t play with her.”

“Go and see to the others, Anushla,” the wizard said. “I will
negotiate in the best of faith with our friend here and call you in
to review the terms of our deal when it’s all done, all right?”

Ana looked toward the corridor where the voices were coming from
and back to the wizard, then to Valentine. “Be smart, girl,” she
said.

The wizard brought her a plate of goose-liver dumplings smothered
in white gravy and then took a bite out of a big toasted corned
beef sandwich that oozed brown mustard.

“Right,” he said. “No playing. If you want to work for me, there
are jobs that need doing. Have you ever seen stage magic performed,
the kind with tuxedos and white doves?”

She nodded slowly. “Before the war,” she said.

“You know how the magician always has a supply of lovely assistants
on hand?”

She nodded again. They’d worn flattering, tight-fitting calf-high
trousers, cutaway coats, tummy-revealing crop-tops, and feathered
confections for hats.

“Everyone who does magic has an assistant or two. I’m the wizard
and I do the best magic of all, and so I have need of more
assistants than most. I have an army of assistants, and they help
me out and I help them out.”

“I’m leaving in five days,” she said.

“The kind of favor I had in mind from you was the kind of favor
that you could perform the day after tomorrow.”

“And you’d take care of my family?”

“I would do that,” he said. “I always take care of my assistants’
families. Do we have an agreement?”

She stuck her hand out and they shook.

“Eat your dumplings,” he said. “And then we’ll get you some things
to take home to your family.”

\tb

Two days after the wizard agreed to take care of Valentine’s
family, the fever had become her constant companion, so omnipresent
that it she hardly noticed it, though it made her walk like an old
woman and she sometimes had trouble focusing her eyes.

She arose that morning and feasted on brown rolls with hard crusts,
small citrus cakes, green beef tea, porridge with currants and
blueberry concentrate and sweet condensed milk, and a chocolate bun
to top it off.

Trover ate even more than she did, licking up the crumbs. She saw
him hide two of the jackfruits under his shirt and nodded
satisfaction\dash{}he had learned something about surviving, then.

Her mother had not questioned the food nor the clothes nor her
daughter’s absence that night. But oh, she had given Valentine a
look when she came through the door carrying all her parcels, a
look that said, \emph{not my daughter any more}. Not a look that
refused what she bore, but a look that refused \emph{her}.
Valentine didn’t bother trying to explain. She knew what her mother
suspected and it was better in some ways than the truth.

Her mother drank the real coffee reverently, with three sugars and
thick no-refrigeration cream. She ate sardines on toast, green beef
tea, and a heap of fluffy scrambled eggs with minced herring, then
she put on her uniform and took up her gun and went out the door,
without a look back at Valentine.

\emph{By the end of the week, she won’t have to worry about me},
Valentine thought. The fever made her fingers shake, but she still
drank her hot chocolate.

Trover knew his own way to the creche, and so Valentine went forth
to earn her family’s fortune.

The wizard had given her a small sack of little electronic marbles,
and had told her to get them planted in no fewer than three hundred
locations at the front and in the places where the fighting was
likely to move. They were spy-eyes, the kind of thing that she and
her friends had exchanged to keep in one-anothers’ rooms before the
war, so they could sneak midnight conversations in perfect
encrypted secrecy.

“If I’m caught,” she said.

“You’ll be shot,” he said. “You \emph{must be}. The alternative is
that you’ll lead them back to me. And if you do that, the whole
game is up\dash{}your family’s lives, my life, your life, the lives of
all my assistants and friends will be forfeit. It will be terrible.
They will destroy this place. They will destroy your home, too.”

She didn’t report for her digging. That was OK. Lots of people
didn’t show up to dig on the days when they were feeling too weak
to hold a shovel. She wouldn’t be missed.

She had the fastest shoes that the wizard could print for her on
her feet, though she’d carefully covered them in grime and dirt so
they wouldn’t stand out. And she’d taken an inhaler along that
would make her faster still. He’d warned her to keep eating after
she took the inhaler, or she’d starve to death before the day was
out. The pockets on both thighs of her jumpsuit were stuffed with
butterballs wrapped around sugared kidneys and livers, stuff that
would sustain her no matter how many puffs she took.

No one challenged her on the way to the front. There were some her
age who fought and many more who served those who fought, bringing
forward ammunition, digging new trenches right at the front. The
pay for this was better than the pay she’d gotten digging in the
“safe” trenches. She brought a shovel for camouflage.

The first round of trenches were familiar, the same kinds she’d
been digging in for months now. She even saw some of the diggers
she’d dug alongside of, nodding to them though her heart was
thumping. \emph{You’ll be shot}, she thought, and she palmed an
electronic eye and stuck it to the wall of a trench.

She moved forward and forward, closer to the fighting. It had
always been a dull, distant rattle, the fighting, never quite gone,
but not always there, either. Instinctively, she’d kept her
distance from it, always moving away from it. Today she moved
toward it and her blood sang.

One trench over there came the dread \emph{zizz} sound of a
trenchbuster and she threw herself down. There were anti-busters in
the trenches, too, but they didn’t always work. The trench-busters
were mostly up around the front, but they sometimes came back to
the diggers, and they had killed one crew she knew of.

There were screams from the next trench, then a sound like a bag of
gravel being poured out\dash{}that was the anti-buster, she knew\dash{}and the
trenchbuster soared out overhead of her and detonated in the sky,
mortally confused by the counter-logic in the anti-buster.

She realized that she had peed herself. Just a little, just a few
drops that must have escaped when she gave her involuntary shriek.
She planted her hands in the frozen dirt of the trench-floor and
got to her knees. That was when she saw the fingertip, shriveled
and frozen, lying just a few inches from her. It had been cleanly
severed.

She had seen so much death, but the fingertip, cut off and left
here to dry out and be trampled down into the dirt; It made her
stomach do slow somersaults. She threw up a little, and peed
herself a little more, and her eyes watered.

That’s when she knew she couldn’t complete the wizard’s mission.
There was death ahead for her that day, much death to see at the
front, and she couldn’t face that. Not when her pockets were full
of spy-eyes and that meant espionage, meant that the wizard was on
the other side, the side of the bastards whose old people she would
starve in their high flats and whose children she would tear from
their beloved parents.

The fever made her shake hard now. Her head swam and the world
pitched and yawed like a ship in heavy seas.

She stood up and took a step. It was a funky disco-dancer step. Her
next step was, too. Then she was walking normally again.

She reached down into her shirt, between her breasts\dash{}she had a bra
on again, fresh from the wizard’s printer\dash{}and withdrew the inhaler
he’d given her. She’d be dead in a week.

She put the inhaler to her lips and drew in a deep breath while
squeezing it, and then the fever was gone. The horror was gone. The
fear and cold were gone. What was left behind was a hard, frenetic
grin, something that sharpened her every sense and set her feet
alight like the most infectious of dance music.

She ran now, flying through the trenches. The closer she got to the
front the worse it smelled, but that was OK, bad smells were fine
by her. Body parts\dash{}the fingertip had just been a preview, here you
could find jawbones and tongues, hands and feet, curled-in cocks
and viscera that glistened through its dust-crust\dash{}not a problem.

She planted five eyes, then crouched to let a trenchbuster sail
over her head. She resisted a mad urge to reach up and stick an eye
to it, then planted another eye, palming it and sticking it right
under the nose of a gunnery sergeant who was hollering at two old
women who were struggling to maneuver a gigantic, multi-part weapon
into position. To Valentine, the women looked old enough to be from
the same tribe she’d hauled water for, and they were so thin they
looked like they were made of twisted-together wires. Their eyes
were huge and round and showed the whites.

The sergeant paid her no mind as she slipped forward, her shovel
still in one hand. The trench dead-ended ahead of her and she
jigged to a side trench, but soon that, too, dead-ended. Dead end
after dead end\dash{}each got its own eye\dash{}and before long she was at the
end of the road, no more side tunnels. She would have to turn back
and try another path. There were no maps of the trenches, of
course.

She had another puff off her inhaler. Her stomach lurched and then
her knees gave way. She was back in the dirt now and she remembered
what the wizard had told her about eating when she was on the
inhaler or starving to death. Then she went into seizure. Her limbs
thrashed, her head shook back and forth, she banged her forehead
into the dirt. A gargling escaped her throat, nothing at all like
words or any other human sound.

When the seizure passed\dash{}and it did pass, though it felt like it
never would\dash{}she shakily withdrew a fistful of butter-ball and
sugared organ meat and shoved it in her mouth. Most of it escaped,
but some of it got down her throat and her hands were steadier in a
moment, enabling her to eat more. She got to her knees, she got to
her feet, she ate some more and had another puff off the inhaler.

God oh god! She felt \emph{marvelous} now. Food and the inhaler
were magic together. Dead ends, pah! Who had time to go back
through the trenches? She’d be dead in five days. She jammed her
fingers in the frozen dirt on the trench-side and hauled herself up
to the surface.

In her months and months of digging in the trenches, she had never
once peeked over the edge. There were things that watched for
snoopy looks over the trenches, laser scanners and sentry guns. You
could lose the top of your head zip-zap.

Now she was on the surface. It was like the surface of the moon.
Craters, hills, trenches, and great clouds of roiling smoke and
dust. Nothing alive. Broken guns and things that might have been
body parts. She grinned that hard grin, because there was no one
else here and so she was the queen of the surface, the bloody angel
of the battlefield. She fisted more sugared liver into her gob and
\emph{ran}.

Zizz, zizz, zizz. There were bullets and other materiel around her,
as soon as she moved, but the world was so clear now, the grey
light so pure, the domain so utterly hers, there was no chance
she’d be hit by a bullet.

She leapt a trench and skirted a trench, leapt and skirted, heading
further and further toward the lines. She nearly tripped over a
sentry gun, then leapt \emph{on top of it} as it tried to swivel
around to get her in its sights, and she pasted an eye on it and
laughed and leapt away.

She was thinking that she should get back into a trench and was
trying to pick one when it was decided for her\dash{}she was in mid-leap
over a trench when a bullet clipped the heel of her shoe and she
tumbled down into the trench. She did a tremendous, jarring
face-plant into the planks below and lay stunned for a moment with
her mouth filling up with blood. Her tongue throbbed\dash{}it had been
bitten\dash{}and as she carefully rolled it around her mouth, she
discovered that she’d knocked out one of her front teeth. Not such
a pretty girl anymore, but she’d be dead in less than five days.

She got to her knees again and planted an eye as she looked
around.

A soldier was staring at her from the end of her current trench. He
was saying something, but here the trenches boomed with artillery
and zizzed with gunfire and hearing was impossible. She drew closer
to him to hear what he had to say and she was practically upon him
when she realized that he was wearing an enemy uniform.

She was quick quick, but he was quicker and he had her arm in an
iron grip before she could pull away.

He said something in a language that they often spoke in the
movies, back where there was a cine across from her block of flats.
She knew a few words of that language.

“Friend!” she said.

He said something in a different language, but she didn’t recognize
that one. Then he switched to Hindi, but all she knew to say in
Hindi was Love Love Love I’m in Love, which was the chorus to all
the songs in the Hindi movies.

He shook her arm hard. He was angry with her, and his gun was in
his other hand now, a soft, floppy handgun like a length of rope
and he was gesturing at her and shouting. He was as well-fed as the
wizard, and he was not much older than her. She thought that he
didn’t want to kill her and was angry because he was going to have
to.

She tried smiling at him. He scowled hard. She held her hand out to
him and touched his arm softly, placatingly. Then she pointed at
her pocket, where the butterballs were. Very slowly, she reached
into it. He watched her with suspicious eyes, the handgun trained
on her now. She thought that if she was a suicide bomber, he’d be
dead now, and that made her feel a little better about the war: if
this was what a soldier from the other side was like, they all had
a chance after all.

She drew out a butter-ball and took a bite of it, then offered it
to the soldier. He looked like he wanted to cry. She held it to his
mouth so he wouldn’t have to let go of her or the gun in order to
eat it. He took a small, polite bite, chewed and swallowed. She had
a bite, then gave him one. They ate like that until the butter-ball
was gone, and then she drew out another, and another.

She pointed to herself. “Valentine,” she said.

He shook his head. He was the picture of moroseness. “Withnail,” he
said.

“Pleased to make your acquaintance, Withnail,” she said in his
language, another useful phrase culled from the cine, though she
suspected she was pronouncing it all wrong. She held out her hand
to shake his. He holstered his handgun and shook her hand.

“I have to go, Withnail.” She couldn’t say this in his language,
but she spoke slowly and as clearly as she could.

He shook his head again. She covered his hand on her arm with her
own and gave it a squeeze.

“To save my family,” she said. “I’m on a mission for your side
anyway. Let me go, Withnail.” She gave his hand another squeeze.
Slowly, he released her arm.

He was very handsome, she saw now, with a good chin and sensuous
lips. She’d never kissed a boy and she’d be dead in four days and a
little more. Or maybe she’d be dead that afternoon, if she couldn’t
get back into her own trenches.

She put her hand on the back of his neck and pulled his face to
hers and gave him a dry, hard kiss on those pouting lips. It made
her blood sing, and she gave him a hug, too, pressing her body to
him. He kissed her back after a moment, surprised. His tongue
probed at her closed lips and she pulled away, then for a crazy
moment she thought of biting him and giving him a dose of zombiism
to spread to his comrades in the trenches with him. But that
wouldn’t be right. They were friends now.

She stuck her fingers in the trench wall. They hurt\dash{}she must have
broken a finger before. She hauled herself up and began to run,
pawing her pockets for her inhaler. “So long, Withnail!” It was
another phrase she knew from the cine.

\tb

Three days after being bitten by the zombie, Valentine woke up with
her hand curled protectively over the huge hot egg on her
collarbone. She couldn’t move that arm this morning, not without
pain like nothing she’d ever felt. Her face ached. Her limbs ached.
Her new breasts ached like she’d been punched in them, repeatedly.
She got out of bed like an old woman and crept to the table.

She sat gingerly and spooned up some cereal. Her mother sat
opposite her, staring over her shoulder. Valentine ate a spoonful
of cereal, then spat it out as it came into contact with the raw,
toothless spot on her gum.

Her mother looked at her.

“Open your mouth, Vale,” she said.

Valentine did as she was bade, showing the gap in her teeth.

“You were hit?” her mother said. Valentine didn’t answer. She
didn’t trust herself to speak with her mother looking at her like
that. “They won’t want you now you’ve lost that tooth,” she said.
“You can go back to digging now.”

She stood up from the table without a word and went out of the
flat. She was so feverish that she couldn’t tell if the stairs went
down or up, whether she was descending or ascending.

She tottered out on the street. The way she felt, she couldn’t walk
properly. Her hips wanted to give way with every step and so she
walked like a funky disco dancer through the early, cold streets,
toward the wizard’s house.

She didn’t make it. Less than half way there, she sat down on a
pile of rubble and retched. She reached down into her pocket and
pulled out the wizard’s inhaler, but she fumbled it. She couldn’t
bend over to get it, so she let herself slowly fall to the street,
then she crawled one-armed to it. She fitted it to her mouth and
then squeezed it with clumsy fingers.

She dragged herself to her feet, not bothering to take the inhaler
with her. Her limbs burned now and wanted to move, no matter how
much it hurt, and she lurched to the wizard’s door, moaning in the
back of her throat.

Ana let her in, eyes wide. “You did it.” It might have been a
question. Valentine let herself slide to the soft, sweet-smelling
carpet and closed her eyes.

\tb

An unknowable number of hours or days after Valentine got to the
wizard’s flat, she woke up in a soft, fluffy bed that was quietly
massaging her limbs. She was dressed in loose cotton pajamas, and
there was a trolley by the bed piled high with the kind of fruit
that wasn’t a berry and wasn’t an orange, but a little of both and
each one had a different smiley face growing in the peel.

The wizard came into the room.

“You’ll live,” he said. “Probably. It would have been a certainty
if you’d fucking told me you had zombiism, you little idiot.”

Ana came in behind him. “Do you think she would have done your
mission for you if she didn’t have zombiism, wizard?”

He waved her off. “You’ve got your cure,” he said.

“It won’t cure me,” Valentine said. Her voice was like a
gravel-mixer. “Not the kind I have. There’s no cure.”

“Oh ho,” the wizard said. “Would you care to make a wager on that?
How about this: if you die, I take care of your family. If you
live, you work for me\dash{}and I’ll take care of your family.”

“You already must take care of my family,” she said.

The wizard’s eyes glittered. “I think that curing your zombiism is
repayment enough, so I’ve unilaterally renegotiated the terms of
our deal. If you don’t like that, I can arrange to have you
re-infected and we can go back to the original contract.”

“You’ve cured me?”

Ana said, “There are lots of things we have access to here that you
can’t get in the city. What you had would have killed you if he
hadn’t helped.”

“Will you take my bet?”

She thought about the mission, about the soldier, about being queen
of the battlefield. She thought about the way they’d bombed her
city and how she’d just helped them kill the city’s soldiers and
diggers\dash{}like her father.

“I won’t betray my city to its enemies ever again.” She sat up very
straight. “I was a traitor once, but I had a fever and I was dying.
You are a traitor every day and what is \emph{your} excuse?”

“A traitor? What the hell are you talking about?”

“The spy-eyes I planted so our enemies can spy on us, the wealth
you have around here. How many of our people died because you sold
them out?”

“Valentine, you are a smart girl and your mother is a soldier, but
you aren’t so very smart as all that. You are a stupid girl
sometimes. Our little palace here isn’t full of spies. We’re
\emph{documentarians}. We shoot the war and we send it to the
outside world so they can see the tragedy they are wreaking here.
We have a huge activist movement that we fuel through our pictures.
The spy-eyes you planted yesterday are now streaming 24/7 to
activist sites in fifty countries. It is being played in the halls
of the United Nations.”

Ana made a spitting sound. “It’s being played as filler on the
snowy slopes of upper cable. It’s being played as ironic snuff-porn
in dorm rooms. It’s being used as stock footage for avant-garde
performance art. Please, wizard, please. She deserves to know the
real situation, not the things you tell yourself when you can’t
sleep.”

“It’s\dash{}\emph{entertainment}?”

“It’s riveting,” Ana said, like \emph{riveting} meant
\emph{terrible}. “Very highly rated.”

“And it raises consciousness,” the wizard said. “You cynic, Ana,
you can’t see anything except the worst. It is the reason that
anyone except for a few policy wonks have heard of what’s going on
here.”

“Entertainment?”

“Entertainment,” the wizard said. “And more than that.”

“They’re killing us, they’re gassing us, they’re bombing us, and
you’re selling back to them as \emph{entertainment}?”

She climbed out of the bed. She hurt, but not so much as she had
before. The fever had broken, at least.

“Am I cured?” she asked. “Do I need anything else, or am I cured?”

The wizard scowled. “Now wait a moment---”

“You’re cured,” Ana said. “You should rest for a few days and eat
well, but you will get better no matter what.” The wizard turned
and shoved her toward the door, so hard she stumbled and hit the
jam. She spat on the floor and walked out of the room.

Valentine pulled herself out of the bed. The wizard took her wrist
and without hesitating, she jammed her thumb into his eye-socket,
grunting with the effort. He shouted and reeled away and she made
her way out of the bedroom and down the corridor to the brave red
sitting room. Ana had a couple of grip-sheeted, robot-tied parcels
for her. “Clothes,” she said. “Food. Don’t come back. I’m not from
here, but even I know how wrong this is. He\dash{}there’s no excuse for
him. Go.” She handed Valentine some shoes\dash{}good sturdy work-boots,
still warm from the printer.

\tb

Six months after she took home the clothes that Ana had given her,
Valentine was taken off of ditch-digging and put on corpse-duty.
They were dying like flies, and the zombies fed on them, and unless
the meat was disposed of, the zombies would multiply like rats.

There was only bread on alternate days now. The hunger was like a
playmate or a childhood enemy that taunted her. It woke her in the
night like a punch in the gut.

The first body she found was missing its ass-cheeks. You could find
the bodies by the smell, and she was on corpse-detail with a boy
about her age whose face she never saw, because it was covered by a
mask. He had a floppy machine-pistol that she hoped he knew how to
use, because the zombies were everywhere. He’d been hauling meat
for weeks, and grunted out little bits and pieces to help her get
acquainted. Neither of them exchanged names.

“What happened to her---”

“Ever see black-market meat? The ass is the last part to go when
they starve. The mafiyehs take the cheeks and grind them up with
some filler and add flavoring agent and sell it. They used to kill
people and take the meat that way, but they don’t need to do that
anymore. There’s enough meat from natural causes.”

The smell was terrible. It was a woman and she’d been dead for some
time. It was hot out, too. Valentine’s mask didn’t really seem to
help, but when she stuck a finger under it to scratch her sweaty
upper lip, an unfiltered gulp of air went up her nose and she
gagged.

They started in the early hours of the morning before the heat got
too bad. They slept for a few hours at noon, then started again
mid-afternoon. She was so hungry that she was dizzy. The next
corpse was on the fifteenth storey of a block of revolutionary-era
flats. No lift in the city had worked in more than a year. They
climbed and rested, climbed and rested. There was no question of
going straight up. She was too weak to consider it for a second.

It was a man. He was big and tall, and even starved out as he was,
they could barely lift him. He must have been a giant in life.

“We’ll never carry him down all those stairs,” the boy said. “Go
and open the big window.”

Valentine obeyed woodenly. She knew that if you couldn’t carry the
body, you’d have to get it out of the building some other way. She
knew that. She didn’t want to think about what it meant, but she
knew it. There’d been a corpse one floor down from her flat and it
had taken weeks for the city to dispose of it and life had been
almost unbearable for everyone in the building. And that had been
winter, when the cold kept the smell down some.

So you had to get rid of the body. The window was a revolutionary
window. It was marvelous and self-cleaning and it swung easily
open. Forty-five meters below, she could see the building’s
deserted courtyard and the corpse-wagon that the boy drove
haltingly through the city streets. Under other circumstances she
might have felt show-offy and ostentatious riding in a car while
everyone else walked, but she knew that no one envied her her ride
in the corpse-wagon.

“Take his ankles.” With the mask on, the boy looked like a horse,
and she knew she did, too. On the one chair that hadn’t been burned
for fuel the previous winter, the boy had stacked up the few
possessions the corpse had: a ring, a lighter, a clasp-knife, a
little set of headphones with their charge-lights showing red.

She picked up the body by the ankles. The boy had him by the
shoulders. When they alley-\emph{oop}ed it up off the floor, the
body let loose a tremendous, evil fart. It wasn’t the first time a
body had done that on that day, but it was the loudest and evilest
of all the farts. Its ankles were dirty and the smell of its feet
and its fart combined into a grey, fuggy miasma that she could
smell through the mask.

“You should smell his feet,” she said.

“You should smell his breath,” the boy said.

They dragged the body to the window and one, two, three, swung it
out into the wide world. She watched it spin away, fascinated and
wordless. Then it hit the ground and the sound. And the way it
looked. And the splash. And the blood.

There were tears streaming down her face, fouling her mask. She
stepped out into the corridor and ripped the mask off and faced the
wall, groaning.

“It gets easier later,” the boy said, tugging her arm.

He was right.

But they needed shovels to get the body into the corpse wagon. Some
of the bits had gone a long way off and she had to carry them
before her on the spade-end of the shovel. His viscera glistened
like an accusation at her. She lived on the fourteenth floor. When
her time came, she’d go out the window too.

\tb

Two years after the siege began, she awoke deaf. Mata was shaking
her vigorously and her lips were moving, but there was no sound.
Valentine listened hard and made out a distant, underwater sound
that she couldn’t place, though it was familiar.

Mata was thin and hard now, and slept with a gun and only came home
for a few hours at a time. She was taking lots of different pills,
and they made her a little jumpy. Valentine wondered if the pills
had rendered her mother mute, before she realized that she couldn’t
hear \emph{anything}.

She tapped her ear.

“I can’t hear,” she said.

Her mother didn’t appear to understand. She still shook Valentine
hard.

“I’m deaf, Mata,” she said. She shook her head and tugged her
earlobes. She was scared now, and she sat up. She wiggled a finger
in her ear, which was very greasy. Not even the sound of her finger
in her ear carried back to her mind. Stone deaf.

She was breathing heavily, but that happened a lot. The hunger made
her weepy and she sometimes cried for no reason. Sometimes in the
middle of a sentence she had to sit down and stare at the sky while
her tears rolled down her throat, until she felt able to go on
again.

She slowed her breathing. “Mata,” she said.

Her mother made a “stay there” gesture, then repeated it and
mouthed the words at her slowly and obviously. She nodded to show
she understood.

She was supposed to be carrying bodies that day. You could get
bread every day if you carried bodies. One piece on alternate days
from the city, one piece from the black-market in exchange for the
loot you could find in the flats of the starved.

There was a new girl that Valentine was training, too. The boy was
long gone. He’d tried to touch her breast, not just once, either,
and she’d reported him. When the supervisor confronted him, he went
crazy and tried to attack the supervisor and the supervisor sent
him to the front to carry ammunition, where, Valentine supposed, he
was still working. Unless he was dead. She didn’t much care which.

But she wanted bread. The creche had shut down a few months before,
but Trover had some little boys he played with and they sometimes
came home with a little food that he was always careful to share
with her, though she was sure that he didn’t share everything. She
didn’t either. No one did. Mata had a little stash of dried fish
under her pillow. Valentine almost never raided it, though she
could have.

Trover was looking at her. She tugged her earlobes. “I’m deaf,” she
said. She thought she might be speaking very loudly, but she
couldn’t tell.

Trover went out of the flat without looking back at her.

She waited for Mata, but the day crept by and Mata didn’t return.
The more she didn’t return, the more Valentine worried. She cried
some, and tried to sleep. She sucked pebbles for the hunger, and
drank the cistern dry. She carried the chamber-pot downstairs, but
the world in silence was so scary that she practically ran back to
the flat once she’d tipped it out into the reeking
collection-point.

She had finally gotten to sleep when Mata returned. Mata mouthed
something at her in slow, deliberate words, but she couldn’t make
it out. Mata repeated it, and then again. She didn’t get any of it,
but Mata’s expression was clear. No doctors would help her. She
hadn’t expected them to.

No doctors could help her, as far as she was concerned. She knew
exactly what had gone wrong: her hearing aids had failed.
Everything from the golden years after the revolution failed. Old
people died when their artificial hearts or kidneys seized up and
withered. Lifts didn’t work. Printers didn’t work\dash{}they’d nearly all
died the day the siege began. The hospitals couldn’t print drugs.
The sky-cars fell out of the sky.

Nothing worked. Nothing would ever work again. Everything fell
apart. Her hearing aids were of that same magical \emph{stuff} as
everything else from the revolution, so it followed that they would
die too.

Mata must have known this. That’s probably what she was saying. If
Valentine concentrated, she could recall her mother’s voice and
have it say the words.

“It’s OK, Mata,” she said. She knew she was shouting. “It’s OK.”

Mata cried and she cried, but she put herself to sleep as soon as
she could, and once she thought Trover and Mata were sleeping, she
took out her small wizard-light and made her way down the silent
stairs, into the silent streets.

She walked cautiously toward the wizard’s flat. She was deaf, but
it felt like she was a little blind too. Without her hearing, she
couldn’t see right, or balance right. She thought about a life
without ears. She’d probably have to go back to digging, since you
couldn’t haul bodies without a partner and you needed to be able to
talk, even if it was only to say alley-\emph{oop}.

She walked like a drunkard, keeping to the darkest streets where
even the night wardens stayed away. She let only the tiniest glow
escape from her little light.

She was about to turn into the main shopping street when a strong
hand seized her arm and jerked her back into the alley. Her first
thought was \emph{zombie} and she screamed involuntarily and a fist
connected with her mouth, loosening one of the teeth next to her
gap. Her head rang like a bell, the first sound she’d heard since
that morning.

The little bead fell out of her hand and rolled into a crack in the
pavement, crazily illuminating the scene and her attacker. The
alley was filthy and covered in drifts of rubble, and the man who’d
hit her was a young civil defense warden with acne that looked
chemically induced. He didn’t smell good. He smelled very bad.
Sick, maybe. Unclean like everyone, and worse. He was no zombie. He
didn’t smell good enough.

She saw his mouth work and knew he was saying something to her.
“I’m deaf” she said and she knew she said it too loud because he
recoiled and then he punched her harder in the mouth than before.

She fell down this time and he dragged her roughly by one arm away
from the light.

She was cried out, and weak from hunger, and she understood what
was coming next when he threw her down and grabbed the collar of
her shirt and ripped it away from her, then gave her bra the same
treatment. She was dazed from the knocks on the head, but she knew
what was coming.

Valentine’s mother was a soldier. She’d been taught to kill. She’d
taught Valentine to kill. Valentine never left the house without a
clasp-knife, the knife she’d taken from the corpse she’d thrown out
a fifteenth storey window some unknowable time before.

The knife was in her back pocket. She watched the boy’s silhouette
work at the fastener of his trousers, while she stole a hand behind
her and slowly, slowly took out the knife. She let herself make
silent choking dazed sounds.

She knew what was coming next, but the boy didn’t.

But as he knelt down and reached out for the snap on her trousers,
she showed him what was next. She took two of his fingers and just
missed opening her own belly. He tried to jerk his arm away, but
she had him by the wrist before he could, and she pulled him down
on top of her, making sure that her knife was free of the clinch,
free to slip around behind him and take him once-twice-three times
in between his ribs, then again into his kidneys. Seeing the
splatted corpses she tossed out of windows had given her a very
keen idea of how anatomy worked.

She had never felt so clearheaded as she did at this work, and the
boy on her thrashed and got her a couple good knocks on the head,
and his blood soaked her bare chest and her face and her short
hair. But she worked the knife some more, going for the throat and
then the face. She let him go and he rolled away and she pounced on
him. She worked with the knife. Soon he stopped moving.

Her shirt was in rags, but the bra-clasp still worked, once she
bent it back. The pea-light was easy to find\dash{}it glowed like a
beacon. She picked it up and made her way to the wizard’s.

“I’m deaf,” she said to Ana. Ana looked the same, at first. And
then Valentine saw that she was holding a cane and leaning on it
heavily.

She knew that she was half-naked and covered with gore, but she
also knew that Ana would not be fazed by this. She squeezed past
her and into the brave, swooping, just-printed sitting room. She
fixed herself some coffee and poured a glug of rum into it while
Ana stared at her in some wordless emotion.

“I’m deaf,” she repeated, setting down some coffee and rum for Ana.
“I could use a shirt, too. And the wizard, of course.”

She remembered how to use the cooker from the revolutionary days,
but it was like remembering something from a dream. She poked at
it, ignoring Ana, and got it to produce a plate of goose-liver
dumplings in white gravy. She rinsed the blood off her fingers and
then ate the dumplings with them.

Ana stared at her for a long moment, then limped out of the room
and fetched the wizard.

He said something that she couldn’t hear. Everyone in the city was
old, even the young people\dash{}wrinkled with dust in the wrinkles and
missing teeth and torn clothes. The wizard was forever young. He
was clean and unscarred and well-fed as ever.

“Print me some clothes, wizard,” she said. “These ones are covered
in blood. And I’m deaf, so don’t bother talking to me.”

The wizard stared at her. She ate a dumpling and licked the gravy
off her fingers. Her stomach had been in flutters since waking up
deaf, a not entirely unpleasant counterpoint to her constant,
painful hunger. The gravy soothed her stomach, the dumplings
settled it, the pain retreated.

She was deaf. She was a murderess. But there was food and it was
good. Better than no food, anyway.

The wizard brought her a pile of warm, printer-fresh clothes. “Your
printers never stopped working, did they?” she said. She was sure
she was talking very loudly and she didn’t give a festering shit.

“Our printers stopped working the morning of the siege. Everything
did. Everything stops working. That’s the infowar. The infowar
probably is what did for my hearing aids. They were supposed to
last ten years but it’s hardly been two.

“I’m taking a shower now,” she said. “You can write me an answer if
you’d like. I promise to read it afterward.”

She took herself to the bathroom and let the shower wash her. There
were some tears in her head somewhere but they couldn’t find their
way to her eyes. That was all right. It was a war, after all.

She dressed in fine printer-fresh clothes and burped a
printer-fresh belch. The gravy taste wafted gassily into her
mouth.

The wizard had rolled up one of the sofas and unrolled a big screen
in its place, the kind of thing she used to love to play games on,
in the dreamlike fantasy of yore.

YOU’RE DEAF?

She nodded. “I have hearing aids, from a bomb. They weren’t working
when I got out of bed this morning. No warning. They went like
that.” She snapped her fingers.

Some movement caught the corner of her eye and she spun around.
There were four more people in the living room, people she hadn’t
met before though she assumed that they belonged to the distant
voices she’d heard on her earlier visits. They had the well-fed
look of Ana and the wizard, and a couple of them were obviously
foreign. The documentarians. One of them was pointing a camera at
her. She bared her gap-tooth grin at the camera and faked a step
toward it. The camera-woman cringed back and she laughed nastily.

“Your cameras work. Your printers work. You’re not losing the
infowar the way we do. That’s because there’s a way to build things
to resist the infowar agents, right? That’s why the enemy
trench-busters don’t fail the way our weapons do.”

The wizard and Ana conversed briefly, their heads pointed away from
her. She grabbed the camera away from the startled camera-woman and
pointed it at them.

“I want to get a recording of what you’re saying now so once my
hearing comes back I’ll be able to listen. You don’t mind, do you?”
She laughed again and poked her tongue out through the gap in her
teeth. All her teeth were loose now, and running her tongue along
the back of them made then wiggle in a way that was part tickle,
part hurt.

The wizard got the idea. He made a keyboard appear on the screen
again and prodded at it.

IT’S NOT QUITE WHAT YOU THINK VALENTINE

“Sure, what do I know? But you’ve got something, don’t you?”

Ana nodded.

“You can fix my hearing?”

Ana nodded again.

“You could try to kill me while you performed surgery, couldn’t
you?”

Neither of them said anything.

“I’m boobytrapped.” She wasn’t, but it had been known to happen.
“When I die, boom!” She realized that this lie might be too
extravagant. Who’d booby-trap a starved gap-toothed girl? “My
mother arranged it.”

She thought back to the cine. The food she’d eaten was helping her
think, the way it always did, making her realize what a cloud of
fuzz-headed hunger she usually floated through.

“I’ve left a full description of your operation in a sealed
envelope to be opened in the event of my death.”

That was better. She should have gone with that in the first place.
She couldn’t tell if they believed her. Ana was shaking her head.

“You’ve got a doctor here, or someone like a doctor. Whatever’s
been done to your leg, Ana, a doctor did that.”

Ana pointed at the woman from whom Valentine had snatched the
camera. Valentine passed it back to her. “Sorry about that.”

\tb

The day after Valentine killed her first man, her hearing came
back. The surgery took about ten minutes and was largely performed
remotely, reprogramming the hardware in her head with something
that the doctor kept calling “hardened logic.” She liked the sound
of that.

Her hearing came back slowly, in blips and bloops over the course
of a few hours. Then it was back, better than new. She found that
she could hear sounds from much farther away. The camera-woman also
showed her how she could use a terminal to access the memory in her
new ears, which would buffer six months’ worth of audio. Valentine
didn’t think she’d be in a position to make much use of this
feature, as interesting as it was. There weren’t any working
machines in the city.

“I’m going home now,” she said.

Ana was waiting by the printer, making it output clothes and food
as fast as it could, giving it to robots to tie up in grip-sheets.

“Would you have turned us in if we didn’t help you?”

Valentine shook her head and tried not to smile. “No one would have
believed me anyway. I’m not boobytrapped, either.”

“I didn’t think you were,” Ana said. She gave Valentine a long hug
and kissed her cheek. “Be careful, OK?”

“Why don’t you people help us? Why can’t you give our army hardened
logic for their weapons?”

Ana shook her head. She was crying. “You think I haven’t asked
this? To do that would be suicide. Your enemies would never forgive
us. It’s one thing to chide them for their slaughter, another thing
to end it.”

Valentine had Ana print her some convincing rags with bit-mapped
filth right in the weave and wrapped her parcels in them so they
wouldn’t be suspicious. She stepped out into the bright light of a
spring day, every sound sharp as a pin-drop, from distant gunfire
to the nearby hungry whimpering of a baby.

She walked slowly through the streets. She passed a spot that she
thought was the place where the boy grabbed her, where she’d done
her work with the knife. If that was the spot, though, there was no
sign of it. The corpse-carriers were efficient.

She walked the stairs to her flat quickly, her full belly supplying
her with boundless energy. As she reached for the door, though, she
heard something from behind it, some crying. Trover. Once he’d
cried nonstop. But he hadn’t cried in so long she barely remembered
the sound.

She swung open the door and saw what Trover was crying about. Mata
was stretched out on the floor beside the one chair they hadn’t
burned for fuel. She wasn’t moving and one of her eyes was wide
open, the other squeezed shut. Trover was shaking her shoulder and
crying.

“What?” Valentine said to her brother, grabbing and shaking him.
“What happened?”

He opened his mouth and let out a howl. He hadn’t spoken in a long
time.

She knelt at her mother’s side. Her mother’s cheek was cold. Her
arms and hands were stiff. Valentine knew that stiffness. Anyone
who worked on the corpse patrol knew that stiffness. The front of
her mother’s torn trousers were damp with cold piss, Valentine
could smell it. In Mata’s breast pocket were a couple of inhalers,
military grade, the kind of thing you took if you couldn’t afford
to sleep and if you needed to make your body go.

To Valentine, her mother looked like a skeleton, something
long-buried and not freshly dead. Compared to Ana, this woman was
very ugly and skinny and hard. Too hard to be a mother. She must
have taken the drugs to keep herself going when Valentine didn’t
come home. Maybe she’d gone looking for Valentine. Maybe she’d gone
looking for a doctor. Maybe she’d gone to the front to kill some
soldiers. Whatever the cause, Valentine had been the reason. It was
for her that Mata had killed herself.

Valentine pulled Trover to her and hugged him. The little boy
smelled of his own shit. In her parcels, she had the food he needed
so she cut them open and gave him some.

She let him eat and covered Mata with some of the new clothes that
she’d brought home. She knew how to go through a corpse’s pockets
efficiently. She also knew all of Mata’s hiding places in the tiny,
grimy flat. Soon she had Mata’s identification, her sidearm, her
inhaler, her rucksack. There were soldiers Valentine’s age at the
front. She could pass.

“Come on, Trover,” she said, getting him into a change of clothes,
putting good shoes on him. Good shoes would be important. She
didn’t know how much walking they’d do, but it would be a lot.

She took him down the stairs, snuffling and weeping a little still,
but logy from all the rich food. She led him to the civic patrol
office.

“I can win the war,” she said.

The woman from the city wasn’t so fat anymore, but she still had
her armor on. She was the one who’d told father he had to go to the
war. She didn’t seem to recognize Valentine, though.

She stared at Valentine. “I’m busy,” she said.

“I know a---” Valentine searched for the word. “A profiteer who has
access to hardened logic that the infowar doesn’t work against.”

The woman from the city looked at her a little longer this time.
“I’m very busy, little girl.”

“I can bring you to him. He has working printers.”

The woman pretended not to hear her. She stared down at a pile of
papers in front of her, and it was clear to Valentine that she was
only pretending to read them.

Valentine led Trover to the woman’s desk and knocked all the papers
off of it.

“It’s illegal to be a profiteer. Don’t you want to at least arrest
him?”

“I’ll arrest \emph{you},” the woman from the city said, grabbing
her wrist. Valentine was ready for this. Her mother had taught her
what to do about this. She bent the woman’s thumb back and squeezed
it until she tumbled out of the chair and dropped to her knees.

“That’s enough,” said an old, old hero. He sounded like he was
right behind her, but that was just her new ears. When she turned
around she saw that he was in the doorway. He was so old now that
he looked like a zombie, and his one arm was pointing at her with
shaky authority. “Let her go.”

Valentine released the woman from the city.

“Do you want to see the profiteer?” Valentine said, approaching the
hero. Her mother had respected this man, and Valentine decided she
would respect him too.

“I will come with you,” he said.

“Will you bring guards? He is armed.” She thought for a minute. “I
believe he’s armed.”

“It will be fine,” he said. He showed her the heavy pistol he wore
on his belt.

“My brother has to come, too,” Valentine said.

“That will be fine.”

The old hero walked slowly and carefully. The soldiers he passed
nodded to him and saluted him. The old people smiled and waved.
Valentine came to feel proud to be at his side. Normally she was
invisible in the city, just another grey, thin face, but with the
old hero, she was a hero too. And she \emph{was} a hero: she was
about to end the war.

The old man spoke creakingly to her as they walked. He remembered
her mother, and he remembered her father. He told her stories of
her mother’s bravery in the revolution, when he’d been her
commander, and she felt her heart race. Valentine was a hero, like
her mother. The wizard would win the war for them.

Then they came to his door. The old man didn’t need her to point it
out. He went and thumped it three times with the butt of his gun.

Ana answered a moment later. She was dressed in old rags, and had
left behind the cast from her leg, limping to the door on a
makeshift cane.

“Hello, comrade,” she said. She didn’t have her usual accent.

The hero nodded to her. “Comrade Ana.” He knew her name, without
being introduced.

The wizard came to the door. “Comrade hero.”

“Comrade Georg.” The old hero shook the wizard’s hand. The wizard
was wearing rags like Ana’s. He had a cunning glitter in his eye
and he took in the street, took in Valentine. “Hello, Valentine,”
he said.

“This girl tells me you have contraband,” the old hero said. “It’s
my duty to come in and search your premises for it.”

“Valentine,” the wizard said, with unconvincing disappointment.
“The food you took from here wasn’t contraband. It was my savings.”
To the hero, he added, “She took the food and I didn’t blame her.
Surely she was hungry. If I had been a little child in her
circumstances, I might have done the same.”

Valentine squeezed Trover’s hand until he whimpered. She didn’t
trust her tongue enough to say anything.

They went into the vestibule and then turned left into a flat. Now,
until this time, she’d always turned right when visiting the
wizard, but now on the right there was nothing but a smooth,
unbroken wall. And to the left, there was an entirely different
flat, barren of furniture as her own flat, small and dirty and
smelling of death.

“Search away,” the wizard said. He tried to put a hand on
Valentine’s shoulder and she shied away and dropped her hand to the
waistband of the trousers he’d printed for her, where she’d hidden
her mother’s tiny sidearm. “You’ll find nothing, I assure you.”

Valentine could see that they’d find nothing. All the furniture in
the room couldn’t have concealed a single tin of food. This wasn’t
even the right flat. With her amazing ears, she heard the movement
of the wizard’s associates, the documentarians, in the next flat
over.

“I hear them,” she said. “Next door. This isn’t the right flat.”

“This is the flat you led me to,” the old hero said.

“It’s through there!” she said, pointing at the blank wall. “It’s a
false wall!” She thumped it but it was solid and stony. Tears
pricked her eyes. “These clothes!” she said, desperately, plucking
at her shirt and trousers. “He printed them for me! He has hardened
logic printers on the other side of that wall. He could win the
war!”

The wizard shook his head and smiled at her again. His eyes
glittered. “Oh, if only that were true. To win this war---”

She looked imploringly at Ana. Ana looked away.

The old hero shook the wizard’s hand with his one remaining hand.
“I’m sorry to have disturbed you, comrade.”

“Nonsense,” the wizard said. “Anything for the city.”

“Come along,” the old hero said. “Let’s leave these people in
peace.”

Trover let himself be led silently into the street and stayed at
her side even when she let go of his hand to silently palm her
mother’s sidearm.

“Your mother would be ashamed of you,” the old hero said. “She
wouldn’t have wasted the city’s time on her fantasies and
vendettas.”

She kept silent. She knew a nearby alley where no one ventured
except for people who disappeared without a trace. Though she
wanted to shout at him that her mother died for the city that the
old hero had just betrayed, she kept silent.

When they passed the alley-mouth, she hastily shoved Trover into
it. He gave a cry and fell over. She ducked in after him.

“He’s tripped! Help me!” she called.

The old hero slowly negotiated his way into the alley and to her
side. She was holding Trover down as he struggled to rise, but she
hoped it looked like she was helping him up. Maybe it did, for the
old hero bent at her side and she stuck the sidearm under his chin
and pushed it hard into the wattle of skin there.

“My mother died for this city, you traitorous worm,” she said, her
jaws clenching with the effort of not shouting the words. “I would
kill you right now if I didn’t think you could be of use to me.”

The old hero’s eyes were calm. “Lots of people have tried to kill
me, little girl.”

“Lots of the enemy have tried. How many from the city?”

“Lots,” the old hero said. “Lots of them, and yet here I stand,
alive and well.”

“I want to go to see the people who fight the infowar. I’ll kill
you if you don’t take me to them.”

“You want to do what? You stupid little girl.” His tone still
wasn’t angry. “The wizard there is the city’s best friend abroad.
He’s the only reason our enemies haven’t crushed us. You want to
betray him?”

“I will win this war,” she said. But she faltered. She had thought
that he’d just been bought off by the wizard, but maybe it was the
case that he supported the wizard’s work. Was it possible?

“We will win this war, by cooperating with our friends abroad. We
can’t afford to expose them to risk. I don’t expect you to
understand, little girl. This is a very deep game.”

The phrase “deep game” enraged her so much that she almost shot him
there. It was so\dash{}\emph{patronizing}.

She let him lead the way toward the front. Trover was whimpering
now\dash{}he’d twisted his ankle when she’d shoved him\dash{}and she whispered
to him to be still.

Her plan was stupid. The old hero was going to lead her into a
trap, not to the high command, and she knew it.

“I suppose I should just shoot you,” she said.

“Why do you say that?” He was so calm. What kind of man was this?

“You’ll lead me to a trap and have me shot or arrested. I have to
see the infowar command. I have to win the war.”

“You dream big, little girl. I have been prosecuting the war on our
enemies since before your hero mother was born. The first thing I
learned is that war is the art of the possible. It is possible that
we will win the siege, given enough time and losses. It is not
possible that you will win the war.”

“So you’ll have me shot rather than try.”

“I wouldn’t have you shot if I could help it. I owe your mother
that much.”

“If you keep talking about my mother I will shoot you.” She found
his calm tone calmed her, too. The soldiers still saluted them, the
old people waved, and she supposed that if any of them knew she had
the old hero under the gun, she’d be torn to pieces. But she was
calm and the day was a sunny one.

“My apologies,” the old man said.

“I could have you run away and try to find them on my own.”

“You’d never find them.”

“I found the wizard. I put a weapon under your chin. I’m fifteen
years old and I did that much. I will find them and I’ll---”

“You’ll what? You’ll tell them to go to the wizard’s flat to
retrieve his technology? I assure you, if that was to come to pass,
there would be no technology to get by the time you reached his
flat.”

They were getting closer to the front. The distant gunfire and
zizzing trench-busters were crystal-clear in her amazing new ears.

“He gave some to me,” she said. “My hearing aids failed yesterday
and he got them back online with hardened logic. I have it in my
head.”

“You---” The old man stopped in his tracks. She almost shot him by
accident, ploughing into him. He turned around, much faster than
she’d seen him move to date. “You have it in your head?”

He reached for her and she jerked the sidearm up. Absently he took
it away from her in a single cobra-swift movement and dropped it in
his shirt pocket. He reached for her again with his one hand and
tilted her head, looking for the small scars beneath her jaw.

“He fixed these?”

“Yesterday. I was deaf yesterday morning.”

“You’re not lying? If you’re lying, I will have you shot.”

“She was deaf,” Trover said, very quietly. “Now she can hear again.
My sister isn’t lying.”

They both looked at him.

“Come with me, little girl,” the old hero said, and he struck off.

\tb

Six hours after Valentine left her dead mother behind in their
grimy, bare flat, she came to the infowar command.

It was far back of the lines, near the old woods at the western
side of the city, and the entrance to it was guarded by five
checkpoints. They took away the sidearm from the old hero at the
last one, along with several other small weapons the old hero was
carrying. They searched and wanded Valentine and Trover, and made
Trover turn out his pockets. It turned out that he was carrying
Valentine’s old clasp-knife, which had disappeared some months
before. He handed it to the soldier solemnly, and she kissed his
cheek and tousled his hair and for a moment, she looked just like
their mother and Valentine felt tears behind her eyes.

“We’re here,” the old hero said. “Come with me.”

There were three airlocks to pass through, and then they were put
into airtight suits with breathing bottles. They didn’t have one
that would fit Trover, but the nice soldier who’d kissed his cheek
promised to look after him.

Beyond the last airlock, it was like something from before the
siege, clean and bold and humming with energy.

“We keep everything that works here,” the old hero said. “This is
our last cache of materiel that hasn’t been compromised by the
infowar. It’s a completely sealed space. If a single strand of
malware got in, it would turn epidemic and wipe out everything.”

His voice sounded like it was coming from a million miles away.
Shrouded in her breathing hood, Valentine felt like she was in the
first days of a better nation, a time when everything worked and
smelled of sharp cleanliness, not rot and ruin. Hooded figures
walked past them without a glance.

The old hero led her deep into the maze, then through yet another
airlock.

“Comrade,” the old hero said. “A word, please.”

The hooded figure to whom he spoke looked up from its workbench and
peered through the old hero’s hood. Then it saluted smartly and
hurried to the old hero’s side.

“General---” The hooded figure had a man’s voice, almost as old as
the old hero’s voice.

The old hero\dash{}the general\dash{}touched his hand to his hood and then
pulled a retractable wire out of his helmet and presented it to the
other. The other patched it into his helmet’s collar. Even with her
marvelous new ears, Valentine couldn’t hear what they said.

They released their umbilicus a moment later and the other one
turned to Valentine.

“Is it true?” His voice was choked, like he could barely get the
words out.

“In my ears,” she said. “Hardened logic.”

The other man danced from foot to foot. “It can’t be true,” he
said.

She nodded.

The old man rooted through his workbench and came up with a wand
that he put against her the back of neck. It was similar to the
wand that the doctor/camera-woman at the wizard’s house had used to
figure out her hearing aids.

“You have it?” the general asked.

“I have it,” the other one said. “I have copied it. Whether I can
decompile it, whether I can make anything useful from it\dash{}well,
we’ll see.”

“Tomorrow, then,” the general said.

The other one didn’t answer. He was hunched over a terminal on his
workbench, fingers punishing his keyboard.

“Now where?” Valentine asked as they shucked their isolation suits,
the smell of stink and rot flooding back into her nostrils.

“Now we clean house,” the general said. “Get your brother and your
gun.”

\tb

Twenty four hours after the wizard cured Valentine’s hearing, she
helped arrest him.

The general knocked on the wizard’s door and it swung open. Ana had
her cast off again, had her bad cane.

“Yes, comrade?”

“I have business with you and yours. Bring them out here, please.”

Ana took in the line of soldiers in the road before her, carrying
weapons from knives to old gunpowder weapons to small, floppy
sidearms and she went ashen.

“I knew it would be today,” she said. She turned to Valentine.
“When you came back this morning, I knew it would be today.”

“Call them,” the old hero said.

“They already know you’re here.” Smoke emerged from the doorway
behind her. “It’s all destroying itself. There was never a chance
of you getting access to it.”

The general shrugged with one shoulder. Valentine wondered if his
stump was smooth like a billiard ball or angry and wounded or
shriveled like dried fruit.

She gripped her mother’s sidearm tighter and watched the wizard
emerge. The documentarians. The wizard’s eyes glittered.

“It’s all gone,” he said. “You won’t get a scrap of it. What a
goddamned waste. We were on your side, you know.”

“You were very well-fed,” the general said.

One of the documentarians sobbed.

“What a pointless goddamned waste. Spiteful, stupid, bone\-headed---”
The wizard broke off, looking at Valentine. “Her hearing aids.”

Valentine smiled. “Yes,” she said. “My hearing aids. I’m recording
you now. Do you have any words you’d like to say for the
microphone?”

The wizard’s jaw dropped to his chest and his whole body sort of
crumpled, slumping in the grasp of the soldier who held him.

“You little---”

Valentine put a sarcastic finger to her lips and then made a show
of covering Trover’s ears. She saw Ana smile involuntarily before
the woman turned away.

\tb

Three days after they arrested the wizard, the sky-cars lifted off
again. They roared over the enemy lines, dropping intelligent motes
that zeroed in on enemy soldiers and burrowed up their nostrils and
in their ears and in the corners of their eyes and rattled in their
skulls until their brains were paste and goo.

Four days after they arrested the wizard, the printers started to
supply food and drugs. Clever wormy robots sought out and
inoculated the zombies.

Ten days after they arrested the wizard, the buildings started to
repair themselves. The lifts all worked again, all at once, in a
synchronized citywide \emph{whirrr} of convenience and
civilization.

Fourteen days after they arrested the wizard, the siege ended.

Valentine and Trover were in the civil defense bunkhouse. They’d
buried their mother that morning, in the woods, in a perfectly
square grave that the robots had excavated for them, amid the
ranked hundreds of thousands that the robots were digging through
the woods, marking each with a small plaque inset to the soil,
bearing a name and a date of birth, and sometimes a day of death,
and the legend, HERO OF THE SIEGE.

Trover hadn’t spoken all that day, but he had tossed in the first
shovel-full of dirt at their mother’s grave. Around them, the
survivors had wailed and torn their clothes and shoveled at the
massed dead.

The soldiers laughed and sang around them, drinking champagne and
eating chocolate. The men hugged them and the women kissed them,
even the sour woman from the city.

The general saw them sitting in their corner, Trover’s hand in
Valentine’s, and he got them and brought them back into the cells.
He handed Valentine a key and gestured toward the wing.

“Go and get them. They’re free to go now. Tell them to go far.”

Ana and the wizard were sharing a tiny cell, the documentarians
were in three other cells. Valentine turned the old metal key in
each lock in turn.

“It’s over,” she said. “Victory. The general says to go far.”

Ana hugged her so long Valentine thought she’d never let go, but
when she did let go, Valentine wished she’d come back.

Valentine never saw them again.

\tb

Ten years after the siege, Valentine got her medal.

The ceremony was a small one. They had almost run out of special
medals to bestow on the living heroes of the siege, and children
came last. The only times she saw Trover these days was at a
friend’s ceremony. The rest of the time, he was preoccupied with
his studies. He was training to be a diplomat. He still had a
terrible temper. Apparently this was an asset at the System Trade
Union.

Valentine walked there, but she was just about the only one. Others
flew, either in sky-cars or on invisible ground-effect cushions.
There were a thousand of them getting their medals today, and she
and Trover were placed next to each other in the long queue, which
was alphabetical by surname.

“They should have given you the biggest and first medal, Vale,”
Trover said. His hands were in white fists. “You! You won the war!
And \emph{he} knows it!”

On stage, the general shook hands with another medal-reci\-pient. He
was up to the C’s, and Valentine and Trover’s last name started
with an X. It would be a while yet.

“His other arm is very convincing,” she said.

Trover just fumed.

When they took the stage, the general looked at them and winked. He
gave them each a medal, then took her by the shoulders and then
hugged her to his breast. He was still thin and fragile, but he was
also still quick and his hug was firm. He pressed his palm to hers
and her body told her he was sending her some data, which she
accepted with surprise but without comment.

Trover led her off of the stage. She examined her new download. An
audio file. She played it, and it played in her cochlea.

\emph{I found the wizard. I put a weapon under your chin. I’m fifteen years old and I did that much. I will find them and I’ll\dash{}}

\emph{You’ll what? You’ll tell them to go to the wizard’s flat to retrieve his technology? I assure you, if that was to come to pass, there would be no technology to get by the time you reached his flat.}

\emph{He gave some to me. My hearing aids failed yesterday and he got them back online with hardened logic. I have it in my head.}

\emph{You\dash{}You have it in your head?}

She’d never forgotten those words, not in ten years, not through
the reconstruction or her years abroad, not in school and not in
work. Not a day had gone by without her thinking of it. Lots of
people had ears that could buffer now, and hers now had a
hundred-year buffer along with all the audio ever recorded on tap
for her pleasure, but she never bothered to rewind her hearing.
Those words, in her mind, were all the rewinding she needed.

She sat down hard, right there, on the sugary grass.

Trover was at her side in a flash, calling her name anxiously. She
was crying uncontrollably, but she was smiling too. Those words,
pulled off of her ears ten years ago, when they’d gone to infowar
command. Oh, God, those old friends, those words. The wizard and
Ana. It had been so long. Where had the time gone?

\tb

The next day, she met an old face.

“You!” he said. He had a thick accent\dash{}the kind of accent that said
he’d learned her language the hard way; that he hadn’t just
installed it.

She looked at him. He was very familiar, but she couldn’t place
him. Maybe if he didn’t have that silly beard, forked into two
theatrical points, the way they were wearing them in Catalan that
year. She tried to picture him without it. He was grinning like a
fool and laughing.

“I can’t believe it’s you!”

She shook her head slowly. Where the hell did she know this guy
from? She was supposed to be going to the cine with friends that
night\dash{}the new show screened between the trees in the western woods
and you walked around through it and drank fizzy elderflower and
talked with your friends as the story unfolded around you. It was a
warm night and perfect for such things.

“You don’t remember me?”

Her tooth tingled. The one that had been knocked out in the trench
and re-sprouted after the siege. Then she recognized him.

“\emph{Withnail?}”

He hopped in place. “Valentine! You remembered!”

She put her hand to her breast and staggered back dramatically,
hamming it up. He was still very handsome, and she’d never
forgotten her first kiss.

“What the hell are you doing here?”

“I have a layover,” he said. “Tokyo tomorrow. But I wanted to stop
and see the place---”

“Remember the dead?” she said. He had been the enemy, after all.
How many of her countrymen had he shot?

“Remember,” he said. “Remember everything.”

How many of his comrades had died on the day the death rained from
the sky? Surely they had died in great number on that day.

The woods were full of her dead. Mata was there. And there was the
movie tonight. She touched the medal on her lapel. He had no medal.
The soldiers who’d prosecuted the siege received no medals.

“You’re here until when?”

“Tomorrow,” he said, “first thing.”

“First thing tomorrow. Come and see a movie tonight,” she said.

He looked at her and cocked his head. She wasn’t beautiful, she
knew, but sometimes men looked at her that way. Something about
what she’d done, they could see it.

“I’d like that very much,” he said.

She played back a little audio as they walked together, for a
terrible silence descended on them as they walked, awkward and
oppressive.

\emph{Would you have turned us in if we didn’t help you?}

\emph{No one would have believed me anyway.}

“Valentine?”

“Yes, Withnail?”

“Thank you,” he said. “For the food. And the kiss. It was my very
first.”

“Mine too.”

“The finest one, too.”

She snorted and punched him in the shoulder.

“Shut up, Withnail,” she said.

“Yes, Comrade Hero,” he said.

She let him kiss her, but only once.

That night, anyway.


\heading{Creative Commons License Deed}

Attribution-NonCommercial-ShareAlike 2.5

You are free:

* to Share\dash{}to copy, distribute, display, and perform the work

* to Remix\dash{}to make derivative works

Under the following conditions:

* Attribution. You must attribute the work in the manner specified
by the author or licensor.

* Noncommercial. You may not use this work for commercial
purposes.

* Share Alike. If you alter, transform, or build upon this work,
you may distribute the resulting work only under a license
identical to this one.



* For any reuse or distribution, you must make clear to others the
license terms of this work.

* Any of these conditions can be waived if you get permission from
the copyright holder.

Disclaimer: Your fair use and other rights are in no way affected
by the above.

This is a human-readable summary of the Legal Code (the full
license):

\url{http://creativecommons.org/licenses/by-nc-sa/2.5/legalcode}

\end{document}
