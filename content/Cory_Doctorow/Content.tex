\section{CONTENT: Selected Essays on Technology, Creativity, Copyright and the Future of the Future}

By Cory Doctorow, doctorow@craphound.com

\subsection{A word about this downloadable file:}

I've been releasing my books online for free since my first novel,
Down and Out in the Magic Kingdom, came out in 2003, and with every
one of those books, I've included a little essay explaining why I
do this sort of thing.

I was tempted to write another one of these essays for this
collection, but then it hit me:
\textbf{this is a collection of essays that are largely concerned with exactly this subject}.

You see, I don't just write essays about copyright to serve as
forewards to my books: I write them for magazine,s, newspapers, and
websites -- I write speeches on the subject for audiences of every
description and in every nation. And finally, here, I've collected
my favorites, the closest I've ever come to a Comprehensive
Doctorow Manifesto.

So I'm going to skip the foreword this time around: the
\textbf{whole book} is my explanation for why I'm giving it away
for free online.

If you like this book and you want to thank me, here's what I'd ask
you to do, in order of preference:

\begin{itemize}
\item
  Buy a copy:
  \href{http://craphound.com/content/buy}{http://craphound.com/content/buy}
\item
  Donate a copy to a school or library:
  \href{http://craphound.com/content/donate}{http://craphound.com/content/donate}
\item
  Send the ebook to five friends and tell them why you liked it
\item
  Convert the ebook to a new file-format (see the download page for
  more)
\end{itemize}
Now, on to the book!

\subsection{Copyright notice:}

This entire work (with the exception of the introduction by John
Perry Barlow) is copyright 2008 by Cory Doctorow and released under
the terms of a Creative Commons US
Attribution-NonCommercial-ShareAlike license
(\href{http://creativecommons.org/licenses/by-nc-sa/3.0/us}{http://creativecommons.org/licenses/by-nc-sa/3.0/us}/).
Some Rights Reserved.

The introduction is copyright 2008 by John Perry Barlow and
released under the terms of a Creative Commons US
Attribution-NonCommercial-ShareAlike license
(\href{http://creativecommons.org/licenses/by-nc-sa/3.0/us}{http://creativecommons.org/licenses/by-nc-sa/3.0/us}/).
Some Rights Reserved.

\subsection{Publication history and acknowledgments:}

Introduction: 2008, John Perry Barlow

Microsoft Research DRM Talk (This talk was originally given to
Microsoft's Research Group and other interested parties from within
the company at their Redmond offices on June 17, 2004.)

The DRM Sausage Factory (Originally published as "A
Behind-The-Scenes Look At How DRM Becomes Law," InformationWeek,
July 11, 2007)

Happy Meal Toys versus Copyright: How America chose Hollywood and
Wal-Mart, and why it's doomed us, and how we might survive anyway
(Originally published as "How Hollywood, Congress, And DRM Are
Beating Up The American Economy," InformationWeek, June 11, 2007)

Why Is Hollywood Making A Sequel To The Napster Wars? (Originally
published in InformationWeek, August 14, 2007)

You DO Like Reading Off a Computer Screen (Originally published in
Locus Magazine, March 2007)

How Do You Protect Artists? (Originally published in The Guardian
as "Online censorship hurts us all," Tuesday, Oct 2, 2007)

It's the Information Economy, Stupid (Originally published in The
Guardian as "Free data sharing is here to stay," September 18,
2007)

Downloads Give Amazon Jungle Fever (Originally published in The
Guardian, December 11, 2007)

What's the Most Important Right Creators Have? (Originally
published as "How Big Media's Copyright Campaigns Threaten Internet
Free Expression," InformationWeek, November 5, 2007)

Giving it Away (Originally published on Forbes.com, December 2006)

Science Fiction is the Only Literature People Care Enough About to
Steal on the Internet (Originally published in Locus Magazine, July
2006)

How Copyright Broke (Originally published in Locus Magazine,
September, 2006)

In Praise of Fanfic (Originally published in Locus Magazine, May
2007)

Metacrap: Putting the torch to seven straw-men of the meta-utopia
(Self-published, 26 August 2001)

Amish for QWERTY (Originally published on the O'Reilly Network,
07/09/2003,
\href{http://www.oreillynet.com/pub/a/wireless/2003/07/09/amish_qwerty.html}{http://www.oreillynet.com/pub/a/wireless/2003/07/09/amish\_qwerty.html})

Ebooks: Neither E, Nor Books (Paper for the O'Reilly Emerging
Technologies Conference, San Diego, February 12, 2004)

Free(konomic) E-books (Originally published in Locus Magazine,
September 2007)

The Progressive Apocalypse and Other Futurismic Delights
(Originally published in Locus Magazine, July 2007)

When the Singularity is More Than a Literary Device: An Interview
with Futurist-Inventor Ray Kurzweil (Originally published in
Asimov's Science Fiction Magazine, June 2005)

Wikipedia: a genuine Hitchhikers' Guide to the Galaxy -- minus the
editors (Originally published in The Anthology at the End of the
Universe, April 2005)

Warhol is Turning in His Grave (Originally published in The
Guardian, November 13, 2007)

The Future of Ignoring Things (Originally published on
InformationWeek's Internet Evolution, October 3, 2007)

Facebook's Faceplant (Originally published as "How Your Creepy
Ex-Co-Workers Will Kill Facebook," in InformationWeek, November 26,
2007)

The Future of Internet Immune Systems (Originally published on
InformationWeek's Internet Evolution, November 19, 2007)

All Complex Ecosystems Have Parasites (Paper delivered at the
O'Reilly Emerging Technology Conference, San Diego, California, 16
March 2005)

READ CAREFULLY (Originally published as "Shrinkwrap Licenses: An
Epidemic Of Lawsuits Waiting To Happen" in InformationWeek,
February 3, 2007)

World of Democracycraft (Originally published as "Why Online Games
Are Dictatorships," InformationWeek, April 16, 2007)

Snitchtown (Originally published in Forbes.com, June 2007)

\subsection{Dedication:}

For the founders of the Electronic Frontier Foundation: John Perry
Barlow, Mitch Kapor and John Gilmore

For the staff -- past and present -- of the Electronic Frontier
Foundation

For the supporters of the Electronic Frontier Foundation

\subsection{Introduction by John Perry Barlow}

San Francisco - Seattle - Vancouver - San Francisco

Tuesday, April 1, 2008

"Content," huh? Ha! Where's the container?

Perhaps these words appear to you on the pages of a book, a
physical object that might be said to have "contained" the thoughts
of my friend and co-conspirator Cory Doctorow as they were
transported in boxes and trucks all the way from his marvelous mind
into yours. If that is so, I will concede that you might be
encountering "content". (Actually, if that's the case, I'm
delighted on Cory's behalf, since that means that you have also
paid him for these thoughts. We still know how to pay creators
directly for the works they embed in stuff.)

But the chances are excellent that you're reading these liquid
words as bit-states of light on a computer screen, having taken
advantage of his willingness to let you have them in that form for
free. In such an instance, what "contains" them? Your hard disk?
His? The Internet and all the servers and routers in whose caches
the ghosts of their passage might still remain? Your mind? Cory's?

To me, it doesn't matter. Even if you're reading this from a book,
I'm still not convinced that what you have in your hands is its
container, or that, even if we agreed on that point, that a little
ink in the shape of, say, the visual pattern you're trained to
interpret as meaning "a little ink" in whatever font the publisher
chooses, is not, as Magritte would remind us, the same thing as a
little ink, even though it is.

Meaning is the issue. If you couldn't read English, this whole book
would obviously contain nothing as far as you were concerned. Given
that Cory is really cool and interesting, you might be motivated to
learn English so that you could read this book, but even then it
wouldn't be a container so much as a conduit.

The real "container" would be process of thought that began when I
compressed my notion of what is meant by the word "ink" - which,
when it comes to the substances that can be used to make marks on
paper, is rather more variable than you might think - and would
kind of end when you decompressed it in your own mind as whatever
you think it is.

I know this is getting a bit discursive, but I do have a point. Let
me just make it so we can move on.

I believe, as I've stated before, that information is
simultaneously a relationship, an action, and an area of shared
mind. What it isn't is a noun.

Information is not a thing. It isn't an object. It isn't something
that, when you sell it or have it stolen, ceases to remain in your
possession. It doesn't have a market value that can be objectively
determined. It is not, for example, much like a 2004 Ducati ST4S
motorcycle, for which I'm presently in the market, and which seems
- despite variabilities based on, I must admit, informationally-
based conditions like mileage and whether it's been dropped - to
have a value that is pretty consistent among the specimens I can
find for a sale on the Web.

Such economic clarity could not be established for anything "in"
this book, which you either obtained for free or for whatever price
the publisher eventually puts on it. If it's a book you're reading
from, then presumably Cory will get paid some percentage of
whatever you, or the person who gave it to you, paid for it.

But I won't. I'm not getting paid to write this forward, neither in
royalties nor upfront. I am, however, getting some intangible
value, as one generally does whenever he does a favor for a friend.
For me, the value being retrieved from going to the trouble of
writing these words is not so different from the value you retrieve
from reading them. We are both mining a deeply intangible "good,"
which lies in interacting with The Mind of Cory Doctorow. I mention
this because it demonstrates the immeasurable role of relationship
as the driving force in an information economy.

But neither am I creating content at the moment nor are you
"consuming" it (since, unlike a hamburger, these words will remain
after you're done with them, and, also unlike a hamburger you won't
subsequently, wellŠ never mind.) Unlike real content, like the
stuff in a shipping container, these words have neither grams nor
liters by which one might measure their value. Unlike gasoline, ten
bucks worth of this stuff will get some people a lot further than
others, depending on their interest and my eloquence, neither of
which can be quantified.

It's this simple: the new meaning of the word "content," is plain
wrong. In fact, it is intentionally wrong. It's a usage that only
arose when the institutions that had fattened on their ability to
bottle and distribute the genius of human expression began to
realize that their containers were melting away, along with their
reason to be in business. They started calling it content at
exactly the time it ceased to be. Previously they had sold books
and records and films, all nouns to be sure. They didn't know what
to call the mysterious ghosts of thought that were attached to
them.

Thus, when not applied to something you can put in a bucket (of
whatever size), "content" actually represents a plot to make you
think that meaning is a thing. It isn't. The only reason they want
you to think that it is because they know how to own things, how to
give them a value based on weight or quantity, and, more to the
point, how to make them artificially scarce in order to increase
their value.

That, and the fact that after a good 25 years of advance warning,
they still haven't done much about the Economy of Ideas besides
trying to stop it from happening.

As I get older, I become less and less interested in saying "I told
you so." But in this case, I find it hard to resist. Back during
the Internet equivalent of the Pleistocene. I wrote a piece for an
ancestor of Wired magazine called Wired magazine that was titled,
variously, "The Economy of Ideas" or "Wine without Bottles." In
this essay, I argued that it would be deucedly difficult to
continue to apply the Adam Smithian economic principles regarding
the relationship between scarcity and value to any products that
could be reproduced and distributed infinitely at zero cost.

I proposed, moreover, that, to the extent that anything might be
scarce in such an economy, it would be attention, and that
invisibility would be a bad strategy for increasing attention.
That, in other words, familiarity might convey more value to
information that scarcity would.

I did my best to tell the folks in what is now called "The Content
Industry" - the institutions that once arose for the useful purpose
of conveying creative expression from one mind to many - that this
would be a good time to change their economic model. I proposed
that copyright had worked largely because it had been difficult, as
a practical matter, to make a book or a record or motion picture
film spool.

It was my theory that as soon as all human expression could be
reduced into ones and zeros, people would begin to realize what
this "stuff" really was and come up with an economic paradigm for
rewarding its sources that didn't seem as futile as claiming to own
the wind. Organizations would adapt. The law would change. The
notion of "intellectual property," itself only about 35 years old,
would be chucked immediately onto the magnificent ash-heap of
Civilization's idiotic experiments.

Of course, as we now know, I was wrong. Really wrong.

As is my almost pathological inclination, I extended them too much
credit. I imputed to institutions the same capacities for
adaptability and recognition of the obvious that I assume for
humans. But institutions, having the legal system a fundamental
part of their genetic code, are not so readily ductile.

This is particularly true in America, where some combination of
certainty and control is the actual "deity" before whose altar we
worship, and where we have a regular practice of spawning large and
inhuman collective organisms that are a kind of meta-parasite.
These critters - let's call them publicly-held corporations - may
be made out of humans, but they are not human. Given human folly,
that characteristic might be semi-ok if they were actually as
cold-bloodedly expedient as I once fancied them - yielding only to
the will of the markets and the raw self-interest of their
shareholders. But no. They are also symbiotically subject to the
"religious beliefs" of those humans who feed in their upper
elevations.

Unfortunately, the guys (and they mostly are guys) who've been
running The Content Industry since it started to die share
something like a doctrinal fundamentalism that has led them to such
beliefs as the conviction that there's no difference between
listening to a song and shop-lifting a toaster.

Moreover, they dwell in such a sublime state of denial that they
think they are stewarding the creative process as it arises in the
creative humans they exploit savagely - knowing, as they do, that a
creative human would rather be heard than paid - and that they, a
bunch of sated old scoundrels nearing retirement would be able to
find technological means for wrapping "containers" around "their"
"content" that the adolescent electronic Hezbollah they've inspired
by suing their own customers will neither be smart nor motivated
enough to shred whatever pathetic digital bottles their lackeys
design.

And so it has been for the last 13 years. The companies that claim
the ability to regulate humanity's Right to Know have been tireless
in their endeavors to prevent the inevitable. The won most of the
legislative battles in the U.S. and abroad, having purchased all
the government money could buy. They even won most of the contests
in court. They created digital rights management software schemes
that behaved rather like computer viruses.

Indeed, they did about everything they could short of seriously
examining the actual economics of the situation - it has never been
proven to me that illegal downloads are more like shoplifted goods
than viral marketing - or trying to come up with a business model
that the market might embrace.

Had it been left to the stewardship of the usual suspects, there
would scarcely be a word or a note online that you didn't have to
pay to experience. There would be increasingly little free speech
or any consequence, since free speech is not something anyone can
own.

Fortunately there were countervailing forces of all sorts,
beginning with the wise folks who designed the Internet in the
first place. Then there was something called the Electronic
Frontier Foundation which I co-founded, along with Mitch Kapor and
John Gilmore, back in 1990. Dedicated to the free exchange of
useful information in cyberspace, it seemed at times that I had
been right in suggesting then that practically every institution of
the Industrial Period would try to crush, or at least own, the
Internet. That's a lot of lawyers to have stacked against your
cause.

But we had Cory Doctorow.

Had nature not provided us with a Cory Doctorow when we needed one,
it would have been necessary for us to invent a time machine and go
into the future to fetch another like him. That would be about the
only place I can imagine finding such a creature. Cory, as you will
learn from his various rants "contained" herein was perfectly
suited to the task of subduing the dinosaurs of content.

He's a little like the guerilla plumber Tuttle in the movie Brazil.
Armed with a utility belt of improbable gizmos, a wildly
over-clocked mind, a keyboard he uses like a verbal machine gun,
and, best of all, a dark sense of humor, he'd go forth against
massive industrial forces and return grinning, if a little beat
up.

Indeed, many of the essays collected under this dubious title are
not only memoirs of his various campaigns but are themselves the
very weapons he used in them. Fortunately, he has spared you some
of the more sophisticated utilities he employed. He is not
battering you with the nerdy technolingo he commands when stacked
up against various minutiacrats, but I assure you that he can speak
geek with people who, unlike Cory, think they're being pretty
social when they're staring at the other person's shoes.

This was a necessary ability. One of the problems that EFF has to
contend with is that even though most of our yet-unborn
constituency would agree heartily with our central mission - giving
everybody everywhere the right to both address and hear everybody
everywhere else - the decisions that will determine the eventual
viability of that right are being made now and generally in
gatherings invisible to the general public, using terminology,
whether technical or legal, that would be the verbal equivalent of
chloroform to anyone not conversant with such arcana.

I've often repeated my belief that the first responsibility of a
human being is to be a better ancestor. Thus, it seems fitting that
the appearance of this book, which details much of Cory's time with
the EFF, coincides with the appearance of his first-born child,
about whom he is a shameless sentimental gusher.

I would like to think that by the time this newest prodigy, Poesy
Emmeline Fibonacci Nautilus Taylor Doctorow - you see what I mean
about paternal enthusiasm - has reached Cory's age of truly
advanced adolescence, the world will have recognized that there are
better ways to regulate the economy of mind than pretending that
its products are something like pig iron. But even if it hasn't, I
am certain that the global human discourse will be less encumbered
than it would have been had not Cory Doctorow blessed our current
little chunk of space/time with his fierce endeavors.

And whatever it is that might be "contained" in the following.

\subsection{Microsoft Research DRM Talk}

(This talk was originally given to Microsoft's Research Group and
other interested parties from within the company at their Redmond
offices on June 17, 2004.)

Greetings fellow pirates! Arrrrr!

I'm here today to talk to you about copyright, technology and DRM,
I work for the Electronic Frontier Foundation on copyright stuff
(mostly), and I live in London. I'm not a lawyer -- I'm a kind of
mouthpiece/activist type, though occasionally they shave me and
stuff me into my Bar Mitzvah suit and send me to a standards body
or the UN to stir up trouble. I spend about three weeks a month on
the road doing completely weird stuff like going to Microsoft to
talk about DRM.

I lead a double life: I'm also a science fiction writer. That means
I've got a dog in this fight, because I've been dreaming of making
my living from writing since I was 12 years old. Admittedly, my
IP-based biz isn't as big as yours, but I guarantee you that it's
every bit as important to me as yours is to you.

Here's what I'm here to convince you of:

\begin{enumerate}
\item
  That DRM systems don't work
\item
  That DRM systems are bad for society
\item
  That DRM systems are bad for business
\item
  That DRM systems are bad for artists
\item
  That DRM is a bad business-move for MSFT
\end{enumerate}
It's a big brief, this talk. Microsoft has sunk a lot of capital
into DRM systems, and spent a lot of time sending folks like Martha
and Brian and Peter around to various smoke-filled rooms to make
sure that Microsoft DRM finds a hospitable home in the future
world. Companies like Microsoft steer like old Buicks, and this
issue has a lot of forward momentum that will be hard to soak up
without driving the engine block back into the driver's
compartment. At best I think that Microsoft might convert some of
that momentum on DRM into angular momentum, and in so doing, save
all our asses.

Let's dive into it.

\subsubsection{1. DRM systems don't work}

This bit breaks down into two parts:

\begin{enumerate}
\item
  A quick refresher course in crypto theory
\item
  Applying that to DRM
\end{enumerate}
Cryptography -- secret writing -- is the practice of keeping
secrets. It involves three parties: a sender, a receiver and an
attacker (actually, there can be more attackers, senders and
recipients, but let's keep this simple). We usually call these
people Alice, Bob and Carol.

Let's say we're in the days of the Caesar, the Gallic War. You need
to send messages back and forth to your generals, and you'd prefer
that the enemy doesn't get hold of them. You can rely on the idea
that anyone who intercepts your message is probably illiterate, but
that's a tough bet to stake your empire on. You can put your
messages into the hands of reliable messengers who'll chew them up
and swallow them if captured -- but that doesn't help you if Brad
Pitt and his men in skirts skewer him with an arrow before he knows
what's hit him.

So you encipher your message with something like ROT-13, where
every character is rotated halfway through the alphabet. They used
to do this with non-worksafe material on Usenet, back when anyone
on Usenet cared about work-safe-ness -- A would become N, B is O, C
is P, and so forth. To decipher, you just add 13 more, so N goes to
A, O to B yadda yadda.

Well, this is pretty lame: as soon as anyone figures out your
algorithm, your secret is g0nez0red.

So if you're Caesar, you spend a lot of time worrying about keeping
the existence of your messengers and their payloads secret. Get
that? You're Augustus and you need to send a message to Brad
without Caceous (a word I'm reliably informed means "cheese-like,
or pertaining to cheese") getting his hands on it. You give the
message to Diatomaceous, the fleetest runner in the empire, and you
encipher it with ROT-13 and send him out of the garrison in the
pitchest hour of the night, making sure no one knows that you've
sent it out. Caceous has spies everywhere, in the garrison and
staked out on the road, and if one of them puts an arrow through
Diatomaceous, they'll have their hands on the message, and then if
they figure out the cipher, you're b0rked. So the existence of the
message is a secret. The cipher is a secret. The ciphertext is a
secret. That's a lot of secrets, and the more secrets you've got,
the less secure you are, especially if any of those secrets are
shared. Shared secrets aren't really all that secret any longer.

Time passes, stuff happens, and then Tesla invents the radio and
Marconi takes credit for it. This is both good news and bad news
for crypto: on the one hand, your messages can get to anywhere with
a receiver and an antenna, which is great for the brave fifth
columnists working behind the enemy lines. On the other hand,
anyone with an antenna can listen in on the message, which means
that it's no longer practical to keep the existence of the message
a secret. Any time Adolf sends a message to Berlin, he can assume
Churchill overhears it.

Which is OK, because now we have computers -- big, bulky primitive
mechanical computers, but computers still. Computers are machines
for rearranging numbers, and so scientists on both sides engage in
a fiendish competition to invent the most cleverest method they can
for rearranging numerically represented text so that the other side
can't unscramble it. The existence of the message isn't a secret
anymore, but the cipher is.

But this is still too many secrets. If Bobby intercepts one of
Adolf's Enigma machines, he can give Churchill all kinds of
intelligence. I mean, this was good news for Churchill and us, but
bad news for Adolf. And at the end of the day, it's bad news for
anyone who wants to keep a secret.

Enter keys: a cipher that uses a key is still more secure. Even if
the cipher is disclosed, even if the ciphertext is intercepted,
without the key (or a break), the message is secret. Post-war, this
is doubly important as we begin to realize what I think of as
Schneier's Law: "any person can invent a security system so clever
that she or he can't think of how to break it." This means that the
only experimental methodology for discovering if you've made
mistakes in your cipher is to tell all the smart people you can
about it and ask them to think of ways to break it. Without this
critical step, you'll eventually end up living in a fool's
paradise, where your attacker has broken your cipher ages ago and
is quietly decrypting all her intercepts of your messages,
snickering at you.

Best of all, there's only one secret: the key. And with dual-key
crypto it becomes a lot easier for Alice and Bob to keep their keys
secret from Carol, even if they've never met. So long as Alice and
Bob can keep their keys secret, they can assume that Carol won't
gain access to their cleartext messages, even though she has access
to the cipher and the ciphertext. Conveniently enough, the keys are
the shortest and simplest of the secrets, too: hence even easier to
keep away from Carol. Hooray for Bob and Alice.

Now, let's apply this to DRM.

In DRM, the attacker is \textbf{also the recipient}. It's not Alice
and Bob and Carol, it's just Alice and Bob. Alice sells Bob a DVD.
She sells Bob a DVD player. The DVD has a movie on it -- say,
Pirates of the Caribbean -- and it's enciphered with an algorithm
called CSS -- Content Scrambling System. The DVD player has a CSS
un-scrambler.

Now, let's take stock of what's a secret here: the cipher is
well-known. The ciphertext is most assuredly in enemy hands, arrr.
So what? As long as the key is secret from the attacker, we're
golden.

But there's the rub. Alice wants Bob to buy Pirates of the
Caribbean from her. Bob will only buy Pirates of the Caribbean if
he can descramble the CSS-encrypted VOB -- video object -- on his
DVD player. Otherwise, the disc is only useful to Bob as a
drinks-coaster. So Alice has to provide Bob -- the attacker -- with
the key, the cipher and the ciphertext.

Hilarity ensues.

DRM systems are usually broken in minutes, sometimes days. Rarely,
months. It's not because the people who think them up are stupid.
It's not because the people who break them are smart. It's not
because there's a flaw in the algorithms. At the end of the day,
all DRM systems share a common vulnerability: they provide their
attackers with ciphertext, the cipher and the key. At this point,
the secret isn't a secret anymore.

\subsubsection{2. DRM systems are bad for society}

Raise your hand if you're thinking something like, "But DRM doesn't
have to be proof against smart attackers, only average individuals!
It's like a speedbump!"

Put your hand down.

This is a fallacy for two reasons: one technical, and one social.
They're both bad for society, though.

Here's the technical reason: I don't need to be a cracker to break
your DRM. I only need to know how to search Google, or Kazaa, or
any of the other general-purpose search tools for the cleartext
that someone smarter than me has extracted.

Raise your hand if you're thinking something like, "But NGSCB can
solve this problem: we'll lock the secrets up on the logic board
and goop it all up with epoxy."

Put your hand down.

Raise your hand if you're a co-author of the Darknet paper.

Everyone in the first group, meet the co-authors of the Darknet
paper. This is a paper that says, among other things, that DRM will
fail for this very reason. Put your hands down, guys.

Here's the social reason that DRM fails: keeping an honest user
honest is like keeping a tall user tall. DRM vendors tell us that
their technology is meant to be proof against average users, not
organized criminal gangs like the Ukrainian pirates who stamp out
millions of high-quality counterfeits. It's not meant to be proof
against sophisticated college kids. It's not meant to be proof
against anyone who knows how to edit her registry, or hold down the
shift key at the right moment, or use a search engine. At the end
of the day, the user DRM is meant to defend against is the most
unsophisticated and least capable among us.

Here's a true story about a user I know who was stopped by DRM.
She's smart, college educated, and knows nothing about electronics.
She has three kids. She has a DVD in the living room and an old VHS
deck in the kids' playroom. One day, she brought home the Toy Story
DVD for the kids. That's a substantial investment, and given the
generally jam-smeared character of everything the kids get their
paws on, she decided to tape the DVD off to VHS and give that to
the kids -- that way she could make a fresh VHS copy when the first
one went south. She cabled her DVD into her VHS and pressed play on
the DVD and record on the VCR and waited.

Before I go farther, I want us all to stop a moment and marvel at
this. Here is someone who is practically technophobic, but who was
able to construct a mental model of sufficient accuracy that she
figured out that she could connect her cables in the right order
and dub her digital disc off to analog tape. I imagine that
everyone in this room is the front-line tech support for someone in
her or his family: wouldn't it be great if all our non-geek friends
and relatives were this clever and imaginative?

I also want to point out that this is the proverbial honest user.
She's not making a copy for the next door neighbors. She's not
making a copy and selling it on a blanket on Canal Street. She's
not ripping it to her hard-drive, DivX encoding it and putting it
in her Kazaa sharepoint. She's doing something \textbf{honest} --
moving it from one format to another. She's home taping.

Except she fails. There's a DRM system called Macrovision embedded
-- by law -- in every VHS that messes with the vertical blanking
interval in the signal and causes any tape made in this fashion to
fail. Macrovision can be defeated for about \$10 with a gadget
readily available on eBay. But our infringer doesn't know that.
She's "honest." Technically unsophisticated. Not stupid, mind you
-- just naive.

The Darknet paper addresses this possibility: it even predicts what
this person will do in the long run: she'll find out about Kazaa
and the next time she wants to get a movie for the kids, she'll
download it from the net and burn it for them.

In order to delay that day for as long as possible, our lawmakers
and big rightsholder interests have come up with a disastrous
policy called anticircumvention.

Here's how anticircumvention works: if you put a lock -- an access
control -- around a copyrighted work, it is illegal to break that
lock. It's illegal to make a tool that breaks that lock. It's
illegal to tell someone how to make that tool. One court even held
it illegal to tell someone where she can find out how to make that
tool.

Remember Schneier's Law? Anyone can come up with a security system
so clever that he can't see its flaws. The only way to find the
flaws in security is to disclose the system's workings and invite
public feedback. But now we live in a world where any cipher used
to fence off a copyrighted work is off-limits to that kind of
feedback. That's something that a Princeton engineering prof named
Ed Felten and his team discovered when he submitted a paper to an
academic conference on the failings in the Secure Digital Music
Initiative, a watermarking scheme proposed by the recording
industry. The RIAA responded by threatening to sue his ass if he
tried it. We fought them because Ed is the kind of client that
impact litigators love: unimpeachable and clean-cut and the RIAA
folded. Lucky Ed. Maybe the next guy isn't so lucky.

Matter of fact, the next guy wasn't. Dmitry Sklyarov is a Russian
programmer who gave a talk at a hacker con in Vegas on the failings
in Adobe's e-book locks. The FBI threw him in the slam for 30 days.
He copped a plea, went home to Russia, and the Russian equivalent
of the State Department issued a blanket warning to its researchers
to stay away from American conferences, since we'd apparently
turned into the kind of country where certain equations are
illegal.

Anticircumvention is a powerful tool for people who want to exclude
competitors. If you claim that your car engine firmware is a
"copyrighted work," you can sue anyone who makes a tool for
interfacing with it. That's not just bad news for mechanics --
think of the hotrodders who want to chip their cars to tweak the
performance settings. We have companies like Lexmark claiming that
their printer cartridges contain copyrighted works -- software that
trips an "I am empty" flag when the toner runs out, and have sued a
competitor who made a remanufactured cartridge that reset the flag.
Even garage-door opener companies have gotten in on the act,
claiming that their receivers' firmware are copyrighted works.
Copyrighted cars, print carts and garage-door openers: what's next,
copyrighted light-fixtures?

Even in the context of legitimate -- excuse me, "traditional" --
copyrighted works like movies on DVDs, anticircumvention is bad
news. Copyright is a delicate balance. It gives creators and their
assignees some rights, but it also reserves some rights to the
public. For example, an author has no right to prohibit anyone from
transcoding his books into assistive formats for the blind. More
importantly, though, a creator has a very limited say over what you
can do once you lawfully acquire her works. If I buy your book,
your painting, or your DVD, it belongs to me. It's my property. Not
my "intellectual property" -- a whacky kind of pseudo-property
that's swiss-cheesed with exceptions, easements and limitations --
but real, no-fooling, actual tangible \textbf{property} -- the kind
of thing that courts have been managing through property law for
centuries.

But anticirumvention lets rightsholders invent new and exciting
copyrights for themselves -- to write private laws without
accountability or deliberation -- that expropriate your interest in
your physical property to their favor. Region-coded DVDs are an
example of this: there's no copyright here or in anywhere I know of
that says that an author should be able to control where you enjoy
her creative works, once you've paid for them. I can buy a book and
throw it in my bag and take it anywhere from Toronto to Timbuktu,
and read it wherever I am: I can even buy books in America and
bring them to the UK, where the author may have an exclusive
distribution deal with a local publisher who sells them for double
the US shelf-price. When I'm done with it, I can sell it on or give
it away in the UK. Copyright lawyers call this "First Sale," but it
may be simpler to think of it as "Capitalism."

The keys to decrypt a DVD are controlled by an org called DVD-CCA,
and they have a bunch of licensing requirements for anyone who gets
a key from them. Among these is something called region-coding: if
you buy a DVD in France, it'll have a flag set that says, "I am a
European DVD." Bring that DVD to America and your DVD player will
compare the flag to its list of permitted regions, and if they
don't match, it will tell you that it's not allowed to play your
disc.

Remember: there is no copyright that says that an author gets to do
this. When we wrote the copyright statutes and granted authors the
right to control display, performance, duplication, derivative
works, and so forth, we didn't leave out "geography" by accident.
That was on-purpose.

So when your French DVD won't play in America, that's not because
it'd be illegal to do so: it's because the studios have invented a
business-model and then invented a copyright law to prop it up. The
DVD is your property and so is the DVD player, but if you break the
region-coding on your disc, you're going to run afoul of
anticircumvention.

That's what happened to Jon Johansen, a Norwegian teenager who
wanted to watch French DVDs on his Norwegian DVD player. He and
some pals wrote some code to break the CSS so that he could do so.
He's a wanted man here in America; in Norway the studios put the
local fuzz up to bringing him up on charges of
\textbf{unlawfully trespassing upon a computer system.} When his
defense asked, "Which computer has Jon trespassed upon?" the answer
was: "His own."

His no-fooling, real and physical property has been expropriated by
the weird, notional, metaphorical intellectual property on his DVD:
DRM only works if your record player becomes the property of
whomever's records you're playing.

\subsubsection{3. DRM systems are bad for biz}

This is the worst of all the ideas embodied by DRM: that people who
make record-players should be able to spec whose records you can
listen to, and that people who make records should have a veto over
the design of record-players.

We've never had this principle: in fact, we've always had just the
reverse. Think about all the things that can be plugged into a
parallel or serial interface, which were never envisioned by their
inventors. Our strong economy and rapid innovation are byproducts
of the ability of anyone to make anything that plugs into anything
else: from the Flo-bee electric razor that snaps onto the end of
your vacuum-hose to the octopus spilling out of your car's
dashboard lighter socket, standard interfaces that anyone can build
for are what makes billionaires out of nerds.

The courts affirm this again and again. It used to be illegal to
plug anything that didn't come from AT\&T into your phone-jack.
They claimed that this was for the safety of the network, but
really it was about propping up this little penny-ante racket that
AT\&T had in charging you a rental fee for your phone until you'd
paid for it a thousand times over.

When that ban was struck down, it created the market for
third-party phone equipment, from talking novelty phones to
answering machines to cordless handsets to headsets -- billions of
dollars of economic activity that had been suppressed by the closed
interface. Note that AT\&T was one of the big beneficiaries of
this: they \textbf{also} got into the business of making
phone-kit.

DRM is the software equivalent of these closed hardware interfaces.
Robert Scoble is a Softie who has an excellent blog, where he wrote
an essay about the best way to protect your investment in the
digital music you buy. Should you buy Apple iTunes music, or
Microsoft DRM music? Scoble argued that Microsoft's music was a
sounder investment, because Microsoft would have more downstream
licensees for its proprietary format and therefore you'd have a
richer ecosystem of devices to choose from when you were shopping
for gizmos to play your virtual records on.

What a weird idea: that we should evaluate our record-purchases on
the basis of which recording company will allow the greatest
diversity of record-players to play its discs! That's like telling
someone to buy the Betamax instead of the Edison Kinetoscope
because Thomas Edison is a crank about licensing his patents; all
the while ignoring the world's relentless march to the more open
VHS format.

It's a bad business. DVD is a format where the guy who makes the
records gets to design the record players. Ask yourself: how much
innovation has there been over the past decade of DVD players?
They've gotten cheaper and smaller, but where are the weird and
amazing new markets for DVD that were opened up by the VCR? There's
a company that's manufacturing the world's first HDD-based DVD
jukebox, a thing that holds 100 movies, and they're charging
\textbf{\$27,000} for this thing. We're talking about a few
thousand dollars' worth of components -- all that other cost is the
cost of anticompetition.

\subsubsection{4. DRM systems are bad for artists}

But what of the artist? The hardworking filmmaker, the ink-stained
scribbler, the heroin-cured leathery rock-star? We poor slobs of
the creative class are everyone's favorite poster-children here:
the RIAA and MPAA hold us up and say, "Won't someone please think
of the children?" File-sharers say, "Yeah, we're thinking about the
artists, but the labels are The Man, who cares what happens to
you?"

To understand what DRM does to artists, you need to understand how
copyright and technology interact. Copyright is inherently
technological, since the things it addresses -- copying,
transmitting, and so on -- are inherently technological.

The piano roll was the first system for cheaply copying music. It
was invented at a time when the dominant form of entertainment in
America was getting a talented pianist to come into your living
room and pound out some tunes while you sang along. The music
industry consisted mostly of sheet-music publishers.

The player piano was a digital recording and playback system.
Piano-roll companies bought sheet music and ripped the notes
printed on it into 0s and 1s on a long roll of computer tape, which
they sold by the thousands -- the hundreds of thousands -- the
millions. They did this without a penny's compensation to the
publishers. They were digital music pirates. Arrrr!

Predictably, the composers and music publishers went nutso. Sousa
showed up in Congress to say that:

These talking machines are going to ruin the artistic development
of music in this country. When I was a boy...in front of every
house in the summer evenings, you would find young people together
singing the songs of the day or old songs. Today you hear these
infernal machines going night and day. We will not have a vocal
chord left. The vocal chord will be eliminated by a process of
evolution, as was the tail of man when he came from the ape.

The publishers asked Congress to ban the piano roll and to create a
law that said that any new system for reproducing music should be
subject to a veto from their industry association. Lucky for us,
Congress realized what side of their bread had butter on it and
decided not to criminalize the dominant form of entertainment in
America.

But there was the problem of paying artists. The Constitution sets
out the purpose of American copyright: to promote the useful arts
and sciences. The composers had a credible story that they'd do
less composing if they weren't paid for it, so Congress needed a
fix. Here's what they came up with: anyone who paid a music
publisher two cents would have the right to make one piano roll of
any song that publisher published. The publisher couldn't say no,
and no one had to hire a lawyer at \$200 an hour to argue about
whether the payment should be two cents or a nickel.

This compulsory license is still in place today: when Joe Cocker
sings "With a Little Help from My Friends," he pays a fixed fee to
the Beatles' publisher and away he goes -- even if Ringo hates the
idea. If you ever wondered how Sid Vicious talked Anka into letting
him get a crack at "My Way," well, now you know.

That compulsory license created a world where a thousand times more
money was made by a thousand times more creators who made a
thousand times more music that reached a thousand times more
people.

This story repeats itself throughout the technological century,
every ten or fifteen years. Radio was enabled by a voluntary
blanket license -- the music companies got together and asked for a
consent decree so that they could offer all their music for a flat
fee. Cable TV took a compulsory: the only way cable operators could
get their hands on broadcasts was to pirate them and shove them
down the wire, and Congress saw fit to legalize this practice
rather than screw around with their constituents' TVs.

Sometimes, the courts and Congress decided to simply take away a
copyright -- that's what happened with the VCR. When Sony brought
out the VCR in 1976, the studios had already decided what the
experience of watching a movie in your living room would look like:
they'd licensed out their programming for use on a machine called a
Discovision, which played big LP-sized discs that were read-only.
Proto-DRM.

The copyright scholars of the day didn't give the VCR very good
odds. Sony argued that their box allowed for a fair use, which is
defined as a use that a court rules is a defense against
infringement based on four factors: whether the use transforms the
work into something new, like a collage; whether it uses all or
some of the work; whether the work is artistic or mainly factual;
and whether the use undercuts the creator's business-model.

The Betamax failed on all four fronts: when you time-shifted or
duplicated a Hollywood movie off the air, you made a
non-transformative use of 100 percent of a creative work in a way
that directly undercut the Discovision licensing stream.

Jack Valenti, the mouthpiece for the motion-picture industry, told
Congress in 1982 that the VCR was to the American film industry "as
the Boston Strangler is to a woman home alone."

But the Supreme Court ruled against Hollywood in 1984, when it
determined that any device capable of a substantial non-infringing
use was legal. In other words, "We don't buy this Boston Strangler
business: if your business model can't survive the emergence of
this general-purpose tool, it's time to get another business-model
or go broke."

Hollywood found another business model, as the broadcasters had, as
the Vaudeville artists had, as the music publishers had, and they
made more art that paid more artists and reached a wider audience.

There's one thing that every new art business-model had in common:
it embraced the medium it lived in.

This is the overweening characteristic of every single successful
new medium: it is true to itself. The Luther Bible didn't succeed
on the axes that made a hand-copied monk Bible valuable: they were
ugly, they weren't in Church Latin, they weren't read aloud by
someone who could interpret it for his lay audience, they didn't
represent years of devoted-with-a-capital-D labor by someone who
had given his life over to God. The thing that made the Luther
Bible a success was its scalability: it was more popular because it
was more proliferate: all success factors for a new medium pale
beside its profligacy. The most successful organisms on earth are
those that reproduce the most: bugs and bacteria, nematodes and
virii. Reproduction is the best of all survival strategies.

Piano rolls didn't sound as good as the music of a skilled pianist:
but they \textbf{scaled better}. Radio lacked the social elements
of live performance, but more people could build a crystal set and
get it aimed correctly than could pack into even the largest
Vaudeville house. MP3s don't come with liner notes, they aren't
sold to you by a hipper-than-thou record store clerk who can help
you make your choice, bad rips and truncated files abound: I once
downloaded a twelve-second copy of "Hey Jude" from the original
Napster. Yet MP3 is outcompeting the CD. I don't know what to do
with CDs anymore: I get them, and they're like the especially nice
garment bag they give you at the fancy suit shop: it's nice and you
feel like a goof for throwing it out, but Christ, how many of these
things can you usefully own? I can put ten thousand songs on my
laptop, but a comparable pile of discs, with liner notes and so
forth -- that's a liability: it's a piece of my monthly
storage-locker costs.

Here are the two most important things to know about computers and
the Internet:

\begin{enumerate}
\item
  A computer is a machine for rearranging bits
\item
  The Internet is a machine for moving bits from one place to another
  very cheaply and quickly
\end{enumerate}
Any new medium that takes hold on the Internet and with computers
will embrace these two facts, not regret them. A newspaper press is
a machine for spitting out cheap and smeary newsprint at speed: if
you try to make it output fine art lithos, you'll get junk. If you
try to make it output newspapers, you'll get the basis for a free
society.

And so it is with the Internet. At the heyday of Napster, record
execs used to show up at conferences and tell everyone that Napster
was doomed because no one wanted lossily compressed MP3s with no
liner notes and truncated files and misspelled metadata.

Today we hear ebook publishers tell each other and anyone who'll
listen that the barrier to ebooks is screen resolution. It's
bollocks, and so is the whole sermonette about how nice a book
looks on your bookcase and how nice it smells and how easy it is to
slip into the tub. These are obvious and untrue things, like the
idea that radio will catch on once they figure out how to sell you
hotdogs during the intermission, or that movies will really hit
their stride when we can figure out how to bring the actors out for
an encore when the film's run out. Or that what the Protestant
Reformation really needs is Luther Bibles with facsimile
illumination in the margin and a rent-a-priest to read aloud from
your personal Word of God.

New media don't succeed because they're like the old media, only
better: they succeed because they're worse than the old media at
the stuff the old media is good at, and better at the stuff the old
media are bad at. Books are good at being paperwhite,
high-resolution, low-infrastructure, cheap and disposable. Ebooks
are good at being everywhere in the world at the same time for free
in a form that is so malleable that you can just pastebomb it into
your IM session or turn it into a page-a-day mailing list.

The only really successful epublishing -- I mean, hundreds of
thousands, millions of copies distributed and read -- is the
bookwarez scene, where scanned-and-OCR'd books are distributed on
the darknet. The only legit publishers with any success at
epublishing are the ones whose books cross the Internet without
technological fetter: publishers like Baen Books and my own, Tor,
who are making some or all of their catalogs available in ASCII and
HTML and PDF.

The hardware-dependent ebooks, the DRM use-and-copy-restricted
ebooks, they're cratering. Sales measured in the tens, sometimes
the hundreds. Science fiction is a niche business, but when you're
selling copies by the ten, that's not even a business, it's a
hobby.

Every one of you has been riding a curve where you read more and
more words off of more and more screens every day through most of
your professional careers. It's zero-sum: you've also been reading
fewer words off of fewer pages as time went by: the dinosauric
executive who prints his email and dictates a reply to his
secretary is info-roadkill.

Today, at this very second, people read words off of screens for
every hour that they can find. Your kids stare at their Game Boys
until their eyes fall out. Euroteens ring doorbells with their
hypertrophied, SMS-twitching thumbs instead of their index
fingers.

Paper books are the packaging that books come in. Cheap
printer-binderies like the Internet Bookmobile that can produce a
full bleed, four color, glossy cover, printed spine, perfect-bound
book in ten minutes for a dollar are the future of paper books:
when you need an instance of a paper book, you generate one, or
part of one, and pitch it out when you're done. I landed at SEA-TAC
on Monday and burned a couple CDs from my music collection to
listen to in the rental car. When I drop the car off, I'll leave
them behind. Who needs 'em?

Whenever a new technology has disrupted copyright, we've changed
copyright. Copyright isn't an ethical proposition, it's a
utilitarian one. There's nothing \textbf{moral} about paying a
composer tuppence for the piano-roll rights, there's nothing
\textbf{immoral} about not paying Hollywood for the right to
videotape a movie off your TV. They're just the best way of
balancing out so that people's physical property rights in their
VCRs and phonographs are respected and so that creators get enough
of a dangling carrot to go on making shows and music and books and
paintings.

Technology that disrupts copyright does so because it simplifies
and cheapens creation, reproduction and distribution. The existing
copyright businesses exploit inefficiencies in the old production,
reproduction and distribution system, and they'll be weakened by
the new technology. But new technology always gives us more art
with a wider reach: that's what tech is \textbf{for}.

Tech gives us bigger pies that more artists can get a bite out of.
That's been tacitly acknowledged at every stage of the copyfight
since the piano roll. When copyright and technology collide, it's
copyright that changes.

Which means that today's copyright -- the thing that DRM nominally
props up -- didn't come down off the mountain on two stone tablets.
It was created in living memory to accommodate the technical
reality created by the inventors of the previous generation. To
abandon invention now robs tomorrow's artists of the new businesses
and new reach and new audiences that the Internet and the PC can
give them.

\subsubsection{5. DRM is a bad business-move for MSFT}

When Sony brought out the VCR, it made a record player that could
play Hollywood's records, even if Hollywood didn't like the idea.
The industries that grew up on the back of the VCR -- movie
rentals, home taping, camcorders, even Bar Mitzvah videographers --
made billions for Sony and its cohort.

That was good business -- even if Sony lost the Betamax-VHS format
wars, the money on the world-with-VCRs table was enough to make up
for it.

But then Sony acquired a relatively tiny entertainment company and
it started to massively screw up. When MP3 rolled around and Sony's
walkman customers were clamoring for a solid-state MP3 player, Sony
let its music business-unit run its show: instead of making a
high-capacity MP3 walkman, Sony shipped its Music Clips,
low-capacity devices that played brain-damaged DRM formats like
Real and OpenMG. They spent good money engineering "features" into
these devices that kept their customers from freely moving their
music back and forth between their devices. Customers stayed away
in droves.

Today, Sony is dead in the water when it comes to walkmen. The
market leaders are poky Singaporean outfits like Creative Labs --
the kind of company that Sony used to crush like a bug, back before
it got borged by its entertainment unit -- and PC companies like
Apple.

That's because Sony shipped a product that there was no market
demand for. No Sony customer woke up one morning and said, "Damn, I
wish Sony would devote some expensive engineering effort in order
that I may do less with my music." Presented with an alternative,
Sony's customers enthusiastically jumped ship.

The same thing happened to a lot of people I know who used to rip
their CDs to WMA. You guys sold them software that produced
smaller, better-sounding rips than the MP3 rippers, but you also
fixed it so that the songs you ripped were device-locked to their
PCs. What that meant is that when they backed up their music to
another hard-drive and reinstalled their OS (something that the
spyware and malware wars has made more common than ever), they
discovered that after they restored their music that they could no
longer play it. The player saw the new OS as a different machine,
and locked them out of their own music.

There is no market demand for this "feature." None of your
customers want you to make expensive modifications to your products
that make backing up and restoring even harder. And there is no
moment when your customers will be less forgiving than the moment
that they are recovering from catastrophic technology failures.

I speak from experience. Because I buy a new Powerbook every ten
months, and because I always order the new models the day they're
announced, I get a lot of lemons from Apple. That means that I hit
Apple's three-iTunes-authorized-computers limit pretty early on and
found myself unable to play the hundreds of dollars' worth of
iTunes songs I'd bought because one of my authorized machines was a
lemon that Apple had broken up for parts, one was in the shop
getting fixed by Apple, and one was my mom's computer, 3,000 miles
away in Toronto.

If I had been a less good customer for Apple's hardware, I would
have been fine. If I had been a less enthusiastic evangelist for
Apple's products -- if I hadn't shown my mom how iTunes Music Store
worked -- I would have been fine. If I hadn't bought so much iTunes
music that burning it to CD and re-ripping it and re-keying all my
metadata was too daunting a task to consider, I would have been
fine.

As it was Apple rewarded my trust, evangelism and out-of-control
spending by treating me like a crook and locking me out of my own
music, at a time when my Powerbook was in the shop -- i.e., at a
time when I was hardly disposed to feel charitable to Apple.

I'm an edge case here, but I'm a \textbf{leading edge} case. If
Apple succeeds in its business plans, it will only be a matter of
time until even average customers have upgraded enough hardware and
bought enough music to end up where I am.

You know what I would totally buy? A record player that let me play
everybody's records. Right now, the closest I can come to that is
an open source app called VLC, but it's clunky and buggy and it
didn't come pre-installed on my computer.

Sony didn't make a Betamax that only played the movies that
Hollywood was willing to permit -- Hollywood asked them to do it,
they proposed an early, analog broadcast flag that VCRs could hunt
for and respond to by disabling recording. Sony ignored them and
made the product they thought their customers wanted.

I'm a Microsoft customer. Like millions of other Microsoft
customers, I want a player that plays anything I throw at it, and I
think that you are just the company to give it to me.

Yes, this would violate copyright law as it stands, but Microsoft
has been making tools of piracy that change copyright law for
decades now. Outlook, Exchange and MSN are tools that abet
widescale digital infringement.

More significantly, IIS and your caching proxies all make and serve
copies of documents without their authors' consent, something that,
if it is legal today, is only legal because companies like
Microsoft went ahead and did it and dared lawmakers to prosecute.

Microsoft stood up for its customers and for progress, and won so
decisively that most people never even realized that there was a
fight.

Do it again! This is a company that looks the world's roughest,
toughest anti-trust regulators in the eye and laughs. Compared to
anti-trust people, copyright lawmakers are pantywaists. You can
take them with your arm behind your back.

In Siva Vaidhyanathan's book The Anarchist in the Library, he talks
about why the studios are so blind to their customers' desires.
It's because people like you and me spent the 80s and the 90s
telling them bad science fiction stories about impossible DRM
technology that would let them charge a small sum of money every
time someone looked at a movie -- want to fast-forward? That
feature costs another penny. Pausing is two cents an hour. The mute
button will cost you a quarter.

When Mako Analysis issued their report last month advising phone
companies to stop supporting Symbian phones, they were just writing
the latest installment in this story. Mako says that phones like my
P900, which can play MP3s as ringtones, are bad for the cellphone
economy, because it'll put the extortionate ringtone sellers out of
business. What Mako is saying is that just because you bought the
CD doesn't mean that you should expect to have the ability to
listen to it on your MP3 player, and just because it plays on your
MP3 player is no reason to expect it to run as a ringtone. I wonder
how they feel about alarm clocks that will play a CD to wake you up
in the morning? Is that strangling the nascent "alarm tone"
market?

The phone companies' customers want Symbian phones and for now, at
least, the phone companies understand that if they don't sell them,
someone else will.

The market opportunity for a truly capable devices is enormous.
There's a company out there charging \textbf{\$27,000} for a DVD
jukebox -- go and eat their lunch! Steve Jobs isn't going to do it:
he's off at the D conference telling studio execs not to release
hi-def movies until they're sure no one will make a hi-def DVD
burner that works with a PC.

Maybe they won't buy into his BS, but they're also not much
interested in what you have to sell. At the Broadcast Protection
Discussion Group meetings where the Broadcast Flag was hammered
out, the studios' position was, "We'll take anyone's DRM except
Microsoft's and Philips'." When I met with UK broadcast wonks about
the European version of the Broadcast Flag underway at the Digital
Video Broadcasters' forum, they told me, "Well, it's different in
Europe: mostly they're worried that some American company like
Microsoft will get their claws into European television."

American film studios didn't want the Japanese electronics
companies to get a piece of the movie pie, so they fought the VCR.
Today, everyone who makes movies agrees that they don't want to let
you guys get between them and their customers.

Sony didn't get permission. Neither should you. Go build the record
player that can play everyone's records.

Because if you don't do it, someone else will.

\subsection{The DRM Sausage Factory}

(Originally published as "A Behind-The-Scenes Look At How DRM
Becomes Law," InformationWeek, July 11, 2007)

Otto von Bismarck quipped, "Laws are like sausages, it is better
not to see them being made." I've seen sausages made. I've seen
laws made. Both pale in comparison to the process by which
anti-copying technology agreements are made.

This technology, usually called "Digital Rights Management" (DRM)
proposes to make your computer worse at copying some of the files
on its hard-drive or on other media. Since all computer operations
involve copying, this is a daunting task -- as security expert
Bruce Schneier has said, "Making bits harder to copy is like making
water that's less wet."

At root, DRMs are technologies that treat the owner of a computer
or other device as an attacker, someone against whom the system
must be armored. Like the electrical meter on the side of your
house, a DRM is a technology that you possess, but that you are
never supposed to be able to manipulate or modify. Unlike the your
meter, though, a DRM that is defeated in one place is defeated in
all places, nearly simultaneously. That is to say, once someone
takes the DRM off a song or movie or ebook, that freed collection
of bits can be sent to anyone else, anywhere the network reaches,
in an eyeblink. DRM crackers need cunning: those who receive the
fruits of their labor need only know how to download files from the
Internet.

Why manufacture a device that attacks its owner? A priori, one
would assume that such a device would cost more to make than a
friendlier one, and that customers would prefer not to buy devices
that treat them as presumptive criminals. DRM technologies limit
more than copying: they limit ranges of uses, such as viewing a
movie in a different country, copying a song to a different
manufacturer's player, or even pausing a movie for too long.
Surely, this stuff hurts sales: who goes into a store and asks, "Do
you have any music that's locked to just one company's player? I'm
in the market for some lock-in."

So why do manufacturers do it? As with many strange behaviors,
there's a carrot at play here, and a stick.

The carrot is the entertainment industries' promise of access to
their copyrighted works. Add DRM to your iPhone and we'll supply
music for it. Add DRM to your TiVo and we'll let you plug it into
our satellite receivers. Add DRM to your Zune and we'll let you
retail our music in your Zune store.

The stick is the entertainment industries' threat of lawsuits for
companies that don't comply. In the last century, entertainment
companies fought over the creation of records, radios, jukeboxes,
cable TV, VCRs, MP3 players and other technologies that made it
possible to experience a copyrighted work in a new way without
permission. There's one battle that serves as the archetype for the
rest: the fight over the VCR.

The film studios were outraged by Sony's creation of the VCR. They
had found a DRM supplier they preferred, a company called
Discovision that made non-recordable optical discs. Discovision was
the only company authorized to play back movies in your living
room. The only way to get a copyrighted work onto a VCR cassette
was to record it off the TV, without permission. The studios argued
that Sony -- whose Betamax was the canary in this legal coalmine --
was breaking the law by unjustly endangering their revenue from
Discovision royalties. Sure, they \textbf{could} just sell
pre-recorded Betamax tapes, but Betamax was a read-write medium:
they could be \textbf{copied}. Moreover, your personal library of
Betamax recordings of the Sunday night movie would eat into the
market for Discovision discs: why would anyone buy a pre-recorded
video cassette when they could amass all the video they needed with
a home recorder and a set of rabbit-ears?

The Supreme Court threw out these arguments in a 1984 5-4 decision,
the "Betamax Decision." This decision held that the VCR was legal
because it was "capable of sustaining a substantially
non-infringing use." That means that if you make a technology that
your customers \textbf{can} use legally, you're not on the hook for
the illegal stuff they do.

This principle guided the creation of virtually every piece of IT
invented since: the Web, search engines, YouTube, Blogger, Skype,
ICQ, AOL, MySpace... You name it, if it's possible to violate
copyright with it, the thing that made it possible is the Betamax
principle.

Unfortunately, the Supremes shot the Betamax principle in the gut
two years ago, with the Grokster decision. This decision says that
a company can be found liable for its customers' bad acts if they
can be shown to have "induced" copyright infringement. So, if your
company advertises your product for an infringing use, or if it can
be shown that you had infringement in mind at the design stage, you
can be found liable for your customers' copying. The studios and
record labels and broadcasters \textbf{love} this ruling, and they
like to think that it's even broader than what the courts set out.
For example, Viacom is suing Google for inducing copyright
infringement by allowing YouTube users to flag some of their videos
as private. Private videos can't be found by Viacom's
copyright-enforcement bots, so Viacom says that privacy should be
illegal, and that companies that give you the option of privacy
should be sued for anything you do behind closed doors.

The gutshot Betamax doctrine will bleed out all over the industry
for decades (or until the courts or Congress restore it to health),
providing a grisly reminder of what happens to companies that try
to pour the entertainment companies' old wine into new digital
bottles without permission. The tape-recorder was legal, but the
digital tape-recorder is an inducement to infringement, and must be
stopped.

The promise of access to content and the threat of legal execution
for non-compliance is enough to lure technology's biggest players
to the DRM table.

I started attending DRM meetings in March, 2002, on behalf of my
former employers, the Electronic Frontier Foundation. My first
meeting was the one where Broadcast Flag was born. The Broadcast
Flag was weird even by DRM standards. Broadcasters are required, by
law, to deliver TV and radio without DRM, so that any
standards-compliant receiver can receive them. The airwaves belong
to the public, and are loaned to broadcasters who have to promise
to serve the public interest in exchange. But the MPAA and the
broadcasters wanted to add DRM to digital TV, and so they proposed
that a law should be passed that would make all manufacturers
promise to \textbf{pretend} that there was DRM on broadcast
signals, receiving them and immediately squirreling them away in
encrypted form.

The Broadcast Flag was hammered out in a group called the Broadcast
Protection Discussion Group (BPDG) a sub-group from the MPAA's
"Content Protection Technology Working Group," which also included
reps from all the big IT companies (Microsoft, Apple, Intel, and so
on), consumer electronics companies (Panasonic, Philips, Zenith),
cable companies, satellite companies, and anyone else who wanted to
pay \$100 to attend the "public" meetings, held every six weeks or
so (you can attend these meetings yourself if you find yourself
near LAX on one of the upcoming dates).

CPTWG (pronounced Cee-Pee-Twig) is a venerable presence in the DRM
world. It was at CPTWG that the DRM for DVDs was hammered out.
CPTWG meetings open with a "benediction," delivered by a lawyer,
who reminds everyone there that what they say might be quoted "on
the front page of the New York Times," (though journalists are
barred from attending CPTWG meetings and no minutes are published
by the organization) and reminding all present not to do anything
that would raise eyebrows at the FTC's anti-trust division (I could
swear I've seen the Microsoft people giggling during this part,
though that may have been my imagination).

The first part of the meeting is usually taken up with
administrative business and presentations from DRM vendors, who
come out to promise that this time they've really, really figured
out how to make computers worse at copying. The real meat comes
after the lunch, when the group splits into a series of smaller
meetings, many of them closed-door and private (the representatives
of the organizations responsible for managing DRM on DVDs splinter
off at this point).

Then comes the working group meetings, like the BPDG. The BPDG was
nominally set up to set up the rules for the Broadcast Flag. Under
the Flag, manufacturers would be required to limit their "outputs
and recording methods" to a set of "approved technologies."
Naturally, every manufacturer in the room showed up with a
technology to add to the list of approved technologies -- and the
sneakier ones showed up with reasons why their competitors'
technologies \textbf{shouldn't} be approved. If the Broadcast Flag
became law, a spot on the "approved technologies" list would be a
license to print money: everyone who built a next-gen digital TV
would be required, by law, to buy only approved technologies for
their gear.

The CPTWG determined that there would be three "chairmen" of the
meetings: a representative from the broadcasters, a representative
from the studios, and a representative from the IT industry (note
that no "consumer rights" chair was contemplated -- we proposed one
and got laughed off the agenda). The IT chair was filled by an
Intel representative, who seemed pleased that the MPAA chair, Fox
Studios's Andy Setos, began the process by proposing that the
approved technologies should include only two technologies, both of
which Intel partially owned.

Intel's presence on the committee was both reassurance and threat:
reassurance because Intel signaled the fundamental reasonableness
of the MPAA's requirements -- why would a company with a bigger
turnover than the whole movie industry show up if the negotiations
weren't worth having? Threat because Intel was poised to gain an
advantage that might be denied to its competitors.

We settled in for a long negotiation. The discussions were drawn
out and heated. At regular intervals, the MPAA reps told us that we
were wasting time -- if we didn't hurry things along, the world
would move on and consumers would grow accustomed to un-crippled
digital TVs. Moreover, Rep Billy Tauzin, the lawmaker who'd
evidently promised to enact the Broadcast Flag into law, was
growing impatient. The warnings were delivered in quackspeak,
urgent and crackling, whenever the discussions dragged, like the
crack of the commissars' pistols, urging us forward.

You'd think that a "technology working group" would concern itself
with technology, but there was precious little discussion of bits
and bytes, ciphers and keys. Instead, we focused on what amounted
to contractual terms: if your technology got approved as a DTV
"output," what obligations would you have to assume? If a TiVo
could serve as an "output" for a receiver, what outputs would the
TiVo be allowed to have?

The longer we sat there, the more snarled these contractual terms
became: winning a coveted spot on the "approved technology" list
would be quite a burden! Once you were in the club, there were all
sorts of rules about whom you could associate with, how you had to
comport yourself and so on.

One of these rules of conduct was "robustness." As a condition of
approval, manufacturers would have to harden their technologies so
that their customers wouldn't be able to modify, improve upon, or
even understand their workings. As you might imagine, the people
who made open source TV tuners were not thrilled about this, as
"open source" and "non-user-modifiable" are polar opposites.

Another was "renewability:" the ability of the studios to revoke
outputs that had been compromised in the field. The studios
expected the manufacturers to make products with remote "kill
switches" that could be used to shut down part or all of their
device if someone, somewhere had figured out how to do something
naughty with it. They promised that we'd establish criteria for
renewability later, and that it would all be "fair."

But we soldiered on. The MPAA had a gift for resolving the worst
snarls: when shouting failed, they'd lead any recalcitrant player
out of the room and negotiate in secret with them, leaving the rest
of us to cool our heels. Once, they took the Microsoft team out of
the room for \textbf{six hours}, then came back and announced that
digital video would be allowed to output on non-DRM monitors at a
greatly reduced resolution (this "feature" appears in Vista as
"fuzzing").

The further we went, the more nervous everyone became. We were
headed for the real meat of the negotiations: the \textbf{criteria}
by which approved technology would be evaluated: how many bits of
crypto would you need? Which ciphers would be permissible? Which
features would and wouldn't be allowed?

Then the MPAA dropped the other shoe: the sole criteria for
inclusion on the list would be the approval of one of its
member-companies, or a quorum of broadcasters. In other words, the
Broadcast Flag wouldn't be an "objective standard," describing the
technical means by which video would be locked away -- it would be
purely subjective, up to the whim of the studios. You could have
the best product in the world, and they wouldn't approve it if your
business-development guys hadn't bought enough drinks for their
business-development guys at a CES party.

To add insult to injury, the only technologies that the MPAA were
willing to consider for initial inclusion as "approved" were the
two that Intel was involved with. The Intel co-chairman had a hard
time hiding his grin. He'd acted as Judas goat, luring in Apple,
Microsoft, and the rest, to legitimize a process that would force
them to license Intel's patents for every TV technology they
shipped until the end of time.

Why did the MPAA give Intel such a sweetheart deal? At the time, I
figured that this was just straight quid pro quo, like Hannibal
said to Clarice. But over the years, I started to see a larger
pattern: Hollywood likes DRM consortia, and they hate individual
DRM vendors. (I've written an entire article about this, but here's
the gist: a single vendor who succeeds can name their price and
terms -- think of Apple or Macrovision -- while a consortium is a
more easily divided rabble, susceptible to co-option in order to
produce ever-worsening technologies -- think of Blu-Ray and
HD-DVD). Intel's technologies were held through two consortia, the
5C and 4C groups.

The single-vendor manufacturers were livid at being locked out of
the digital TV market. The final report of the consortium reflected
this -- a few sheets written by the chairmen describing the
"consensus" and hundreds of pages of angry invective from
manufacturers and consumer groups decrying it as a sham.

Tauzin washed his hands of the process: a canny, sleazy Hill
operator, he had the political instincts to get his name off any
proposal that could be shown to be a plot to break voters'
televisions (Tauzin found a better industry to shill for, the
pharmaceutical firms, who rewarded him with a \$2,000,000/year job
as chief of PHARMA, the pharmaceutical lobby).

Even Representative Ernest "Fritz" Hollings ("The Senator from
Disney," who once proposed a bill requiring entertainment industry
oversight of all technologies capable of copying) backed away from
proposing a bill that would turn the Broadcast Flag into law.
Instead, Hollings sent a memo to Michael Powell, then-head of the
FCC, telling him that the FCC already had jurisdiction to enact a
Broadcast Flag regulation, without Congressional oversight.

Powell's staff put Hollings's letter online, as they are required
to do by federal sunshine laws. The memo arrived as a Microsoft
Word file -- which EFF then downloaded and analyzed. Word stashes
the identity of a document's author in the file metadata, which is
how EFF discovered that the document had been written by a staffer
at the MPAA.

This was truly remarkable. Hollings was a powerful committee
chairman, one who had taken immense sums of money from the
industries he was supposed to be regulating. It's easy to be
cynical about this kind of thing, but it's genuinely unforgivable:
politicians draw a public salary to sit in public office and work
for the public good. They're supposed to be working for us, not
their donors.

But we all know that this isn't true. Politicians are happy to give
special favors to their pals in industry. However, the Hollings
memo was beyond the pale. Staffers for the MPAA were writing
Hollings's memos, memos that Hollings then signed and mailed off to
the heads of major governmental agencies.

The best part was that the legal eagles at the MPAA were wrong. The
FCC took "Hollings's" advice and enacted a Broadcast Flag
regulation that was almost identical to the proposal from the BPDG,
turning themselves into America's "device czars," able to burden
any digital technology with "robustness," "compliance" and
"revocation rules." The rule lasted just long enough for the DC
Circuit Court of Appeals to strike it down and slap the FCC for
grabbing unprecedented jurisdiction over the devices in our living
rooms.

So ended the saga of the Broadcast Flag. More or less. In the years
since the Flag was proposed, there have been several attempts to
reintroduce it through legislation, all failed. And as more and
more innovative, open devices like the Neuros OSD enter the market,
it gets harder and harder to imagine that Americans will accept a
mandate that takes away all that functionality.

But the spirit of the Broadcast Flag lives on. DRM consortia are
all the rage now -- outfits like AACS LA, the folks who control the
DRM in Blu-Ray and HD-DVD, are thriving and making headlines by
issuing fatwas against people who publish their secret integers. In
Europe, a DRM consortium working under the auspices of the Digital
Video Broadcasters Forum (DVB) has just shipped a proposed standard
for digital TV DRM that makes the Broadcast Flag look like the work
of patchouli-scented infohippies. The DVB proposal would give DRM
consortium the ability to define what is and isn't a valid
"household" for the purposes of sharing your video within your
"household's devices." It limits how long you're allowed to pause a
video for, and allows for restrictions to be put in place for
hundreds of years, longer than any copyright system in the world
would protect any work for.

If all this stuff seems a little sneaky, underhanded and even
illegal to you, you're not alone. When representatives of nearly
all the world's entertainment, technology, broadcast, satellite and
cable companies gather in a room to collude to cripple their
offerings, limit their innovation, and restrict the market,
regulators take notice.

That's why the EU is taking a hard look at HD-DVD and Blu-Ray.
These systems aren't designed: they're governed, and the governors
are shadowy group of offshore giants who answer to no one -- not
even their own members! I once called the DVD-Copy Control
Association (DVD-CCA) on behalf of a Time-Warner magazine, Popular
Science, for a comment about their DRM. Not only wouldn't they
allow me to speak to a spokesman, the person who denied my request
also refused to be identified.

The sausage factory grinds away, but today, more activists than
ever are finding ways to participate in the negotiations, slowing
them up, making them account for themselves to the public. And so
long as you, the technology-buying public, pay attention to what's
going on, the activists will continue to hold back the tide.

\subsection{Happy Meal Toys versus Copyright: How America chose Hollywood and Wal-Mart, and why it's doomed us, and how we might survive anyway}

(Originally published as "How Hollywood, Congress, And DRM Are
Beating Up The American Economy," InformationWeek, June 11, 2007)

Back in 1985, the Senate was ready to clobber the music industry
for exposing America's impressionable youngsters to sex, drugs and
rock-and-roll. Today, the the Attorney General is proposing to give
the RIAA legal tools to attack people who attempt infringement.

Through most of America's history, the US government has been at
odds with the entertainment giants, treating them as purveyors of
filth. But not anymore: today, the US Trade Rep using America's
political clout to force Russia to institute police inspections of
its CD presses (savor the irony: post-Soviet Russia forgoes its
hard-won freedom of the press to protect Disney and Universal!).

How did entertainment go from trenchcoat pervert to top trade
priority? I blame the "Information Economy."

No one really knows what "Information Economy" means, but by the
early 90s, we knew it was coming. America deployed her least
reliable strategic resource to puzzle out what an "information
economy" was and to figure out how to ensure America stayed atop
the "new economy" -- America sent in the futurists.

We make the future in much the same way as we make the past. We
don't remember everything that happened to us, just selective
details. We weave our memories together on demand, filling in any
empty spaces with the present, which is lying around in great
abundance. In Stumbling on Happiness, Harvard psych prof Daniel
Gilbert describes an experiment in which people with delicious
lunches in front of them are asked to remember their breakfast:
overwhelmingly, the people with good lunches have more positive
memories of breakfast than those who have bad lunches. We don't
remember breakfast -- we look at lunch and superimpose it on
breakfast.

We make the future in the same way: we extrapolate as much as we
can, and whenever we run out of imagination, we just shovel the
present into the holes. That's why our pictures of the future
always seem to resemble the present, only moreso.

So the futurists told us about the Information Economy: they took
all the "information-based" businesses (music, movies and
microcode, in the neat coinage of Neal Stephenson's 1992 novel Snow
Crash) and projected a future in which these would grow to dominate
the world's economies.

There was only one fly in the ointment: most of the world's
economies consist of poor people who have more time than money, and
if there's any lesson to learn from American college kids, it's
that people with more time than money would rather copy information
than pay for it.

Of course they would! Why, when America was a-borning, she was a
pirate nation, cheerfully copying the inventions of European
authors and inventors. Why not? The fledgling revolutionary
republic could copy without paying, keep the money on her shores,
and enrich herself with the products and ideas of imperial Europe.
Of course, once the US became a global hitter in the creative
industries, out came the international copyright agreements: the US
signed agreements to protect British authors in exchange for
reciprocal agreements from the Brits to protect American authors.

It's hard to see why a developing country would opt to export its
GDP to a rich country when it could get the same benefit by mere
copying. The US would have to sweeten the pot.

The pot-sweetener is the elimination of international
trade-barriers. Historically, the US has used tariffs to limit the
import of manufactured goods from abroad, and to encourage the
import of raw materials from abroad. Generally speaking, rich
countries import poor countries' raw materials, process them into
manufactured goods, and export them again. Globally speaking, if
your country imports sugar and exports sugar cane, chances are
you're poor. If your country imports wood and sells paper, chances
are you're rich.

In 1995, the US signed onto the World Trade Organization and its
associated copyright and patent agreement, the TRIPS Agreement, and
the American economy was transformed.

Any fellow signatory to the WTO/TRIPS can export manufactured goods
to the USA without any tariffs. If it costs you \$5 to manufacture
and ship a plastic bucket from your factory in Shenjin Province to
the USA, you can sell it for \$6 and turn a \$1 profit. And if it
costs an American manufacturer \$10 to make the same bucket, the
American manufacturer is out of luck.

The kicker is this: if you want to export your finished goods to
America, you have to sign up to protect American copyrights in your
own country. Quid pro quo.

The practical upshot, 12 years later, is that most American
manufacturing has gone belly up, Wal-Mart is filled with Happy Meal
toys and other cheaply manufactured plastic goods, and the whole
world has signed onto US copyright laws.

But signing onto those laws doesn't mean you'll enforce them. Sure,
where a country is really over a barrel (cough, Russia, cough),
they'll take the occasional pro forma step to enforce US
copyrights, no matter how ridiculous and totalitarian it makes them
appear. But with the monthly Russian per-capita GDP hovering at
\$200, it's just not plausible that Russians are going to start
paying \$15 for a CD, nor is it likely that they'll stop listening
to music until their economy picks up.

But the real action is in China, where pressing bootleg media is a
national sport. China keeps promising that it will do something
about this, but it's not like the US has any recourse if China
drags its heels. Trade courts may find against China, but China
holds all the cards. The US can't afford to abandon Chinese
manufacturing (and no one will vote for the politician who
hextuples the cost of WiFi cards, brassieres, iPods, staplers, yoga
mats, and spatulas by cutting off trade with China). The Chinese
can just sit tight.

The futurists were just plain wrong. An "information economy" can't
be based on selling information. Information technology makes
copying information easier and easier. The more IT you have, the
less control you have over the bits you send out into the world. It
will never, ever, EVER get any harder to copy information from here
on in. The information economy is about selling everything except
information.

The US traded its manufacturing sector's health for its
entertainment industry, hoping that Police Academy sequels could
take the place of the rustbelt. The US bet wrong.

But like a losing gambler who keeps on doubling down, the US
doesn't know when to quit. It keeps meeting with its entertainment
giants, asking how US foreign and domestic policy can preserve its
business-model. Criminalize 70 million American file-sharers?
Check. Turn the world's copyright laws upside down? Check. Cream
the IT industry by criminalizing attempted infringement? Check.

It'll never work. It can never work. There will always be an
entertainment industry, but not one based on excluding access to
published digital works. Once it's in the world, it'll be copied.
This is why I give away digital copies of my books and make money
on the printed editions: I'm not going to stop people from copying
the electronic editions, so I might as well treat them as an
enticement to buy the printed objects.

But there is an information economy. You don't even need a computer
to participate. My barber, an avowed technophobe who rebuilds
antique motorcycles and doesn't own a PC, benefited from the
information economy when I found him by googling for barbershops in
my neighborhood.

Teachers benefit from the information economy when they share
lesson plans with their colleagues around the world by email.
Doctors benefit from the information economy when they move their
patient files to efficient digital formats. Insurance companies
benefit from the information economy through better access to fresh
data used in the preparation of actuarial tables. Marinas benefit
from the information economy when office-slaves look up the
weekend's weather online and decide to skip out on Friday for a
weekend's sailing. Families of migrant workers benefit from the
information economy when their sons and daughters wire cash home
from a convenience store Western Union terminal.

This stuff generates wealth for those who practice it. It enriches
the country and improves our lives.

And it can peacefully co-exist with movies, music and microcode,
but not if Hollywood gets to call the shots. Where IT managers are
expected to police their networks and systems for unauthorized
copying -- no matter what that does to productivity -- they cannot
co-exist. Where our operating systems are rendered inoperable by
"copy protection," they cannot co-exist. Where our educational
institutions are turned into conscript enforcers for the record
industry, they cannot co-exist.

The information economy is all around us. The countries that
embrace it will emerge as global economic superpowers. The
countries that stubbornly hold to the simplistic idea that the
information economy is about selling information will end up at the
bottom of the pile.

What country do you want to live in?

\subsection{Why Is Hollywood Making A Sequel To The Napster Wars?}

(Originally published in InformationWeek, August 14, 2007)

Hollywood loves sequels -- they're generally a safe bet, provided
that you're continuing an already successful franchise. But you'd
have to be nuts to shoot a sequel to a disastrous flop -- say, The
Adventures of Pluto Nash or Town and Country.

As disastrous as Pluto Nash was, it was practically painless when
compared to the Napster debacle. That shipwreck took place six
years ago, when the record industry succeeded in shutting down the
pioneering file-sharing service, and they show no signs of
recovery.

\textbf{The disastrous thing about Napster wasn't that it it existed, but rather that the record industry managed to kill it.}

Napster had an industry-friendly business-model: raise venture
capital, start charging for access to the service, and then pay
billions of dollars to the record companies in exchange for
licenses to their works. Yes, they kicked this plan off without
getting permission from the record companies, but that's not so
unusual. The record companies followed the same business plan a
hundred years ago, when they started recording sheet music without
permission, raising capital and garnering profits, and
\textbf{then} working out a deal to pay the composers for the works
they'd built their fortunes on.

Napster's plan was plausible. They had the fastest-adopted
technology in the history of the world, garnering 52,000,000 users
in 18 months -- more than had voted for either candidate in the
preceding US election! -- and discovering, via surveys, that a
sizable portion would happily pay between \$10 and \$15 a month for
the service. What's more, Napster's architecture included a
gatekeeper that could be used to lock-out non-paying users.

The record industry refused to deal. Instead, they sued, bringing
Napster to its knees. Bertelsmann bought Napster out of the ensuing
bankruptcy, a pattern that was followed by other music giants, like
Universal, who slayed MP3.com in the courts, then brought home the
corpse on the cheap, running it as an internal project.

After that, the record companies had a field day: practically every
venture-funded P2P company went down, and millions of dollars were
funneled from the tech venture capital firms to Sand Hill Road to
the RIAA's members, using P2P companies and the courts as
conduits.

But the record companies weren't ready to replace these services
with equally compelling alternatives. Instead, they fielded
inferior replacements like PressPlay, with limited catalog, high
prices, and anti-copying technology (digital rights management, or
DRM) that alienated users by the millions by treating them like
crooks instead of customers. These half-baked ventures did untold
damage to the record companies and their parent firms.

Just look at Sony: they should have been at the top of the heap.
They produce some of the world's finest, best-designed electronics.
They own one of the largest record labels in the world. The synergy
should have been incredible. Electronics would design the walkmen,
music would take care of catalog, and marketing would sell it all.

You know the joke about European hell? The English do the cooking,
the Germans are the lovers, the Italians are the police and the
French run the government. With Sony, it seemed like music was
designing the walkmen, marketing was doing the catalog, and
electronic was in charge of selling. Sony's portable players -- the
MusicClip and others -- were so crippled by anti-copying technology
that they couldn't even play MP3s, and the music selection at Sony
services like PressPlay was anemic, expensive, and equally hobbled.
Sony isn't even a name in the portable audio market anymore --
today's walkman is an iPod.

Of course, Sony still has a record-label -- for now. But sales are
falling, and the company is reeling from the 2005 "rootkit"
debacle, where in deliberately infected eight million music CDs
with a hacker tool called a rootkit, compromising over 500,000 US
computer networks, including military and government networks, all
in a (failed) bid to stop copying of its CDs.

The public wasn't willing to wait for Sony and the rest to wake up
and offer a service that was as compelling, exciting and versatile
as Napster. Instead, they flocked to a new generation of services
like Kazaa and the various Gnutella networks. Kazaa's business
model was to set up offshore, on the tiny Polynesian island of
Vanuatu, and bundle spyware with its software, making its profits
off of fees from spyware crooks. Kazaa didn't want to pay billions
for record industry licenses -- they used the international legal
and finance system to hopelessly snarl the RIAA's members through
half a decade of wild profitability. The company was eventually
brought to ground, but the founders walked away and started Skype
and then Joost.

Meantime, dozens of other services had sprung up to fill Kazaa's
niche -- AllofMP3, the notorious Russian site, was eventually
killed through intervention of the US Trade Representative and the
WTO, and was reborn practically the next day under a new name.

It's been eight years since Sean Fanning created Napster in his
college dorm-room. Eight years later, there isn't a single
authorized music service that can compete with the original
Napster. Record sales are down every year, and digital music sales
aren't filling in the crater. The record industry has contracted to
four companies, and it may soon be three if EMI can get regulatory
permission to put itself on the block.

The sue-em-all-and-let-God-sort-em-out plan was a flop in the box
office, a flop in home video, and a flop overseas. So why is
Hollywood shooting a remake?

\begin{center}\rule{3in}{0.4pt}\end{center}

YouTube, 2007, bears some passing similarity to Napster, 2001.
Founded by a couple guys in a garage, rocketed to popular success,
heavily capitalized by a deep-pocketed giant. Its business model?
Turn popularity into dollars and offer a share to the rightsholders
whose works they're using. This is an historically sound plan:
cable operators got rich by retransmitting broadcasts without
permission, and once they were commercial successes, they sat down
to negotiate to pay for those copyrights (just as the record
companies negotiated with composers \textbf{after} they'd gotten
rich selling records bearing those compositions).

YouTube 07 has another similarity to Napster 01: it is being sued
by entertainment companies.

Only this time, it's not (just) the record industry. Broadcasters,
movie studios, anyone who makes video or audio is getting in on the
act. I recently met an NBC employee who told me that he thought
that a severe, punishing legal judgment would send a message to the
tech industry not to field this kind of service anymore.

Let's hope he's wrong. Google -- YouTube's owners -- is a grown-up
of a company, unusual in a tech industry populated by corporate
adolescents. They have lots of money and a sober interest in
keeping it. They want to sit down with A/V rightsholders and do a
deal. Six years after the Napster verdict, that kind of willingness
is in short supply.

Most of the tech "companies" with an interest in commercializing
Internet AV have no interest in sitting down with the studios.
They're either nebulous open source projects (like mythtv, a free
hyper-TiVo that skips commercials, downloads and shares videos and
is wide open to anyone who wants to modify and improve it),
politically motivated anarchists (like ThePirateBay, a Swedish
BitTorrent tracker site that has mirrors in three countries with
non-interoperable legal systems, where they respond to legal
notices by writing sarcastic and profane letters and putting them
online), or out-and-out crooks like the bootleggers who use P2P to
seed their DVD counterfeiting operations.

It's not just YouTube. TiVo, who pioneered the personal video
recorder, is feeling the squeeze, being systematically locked out
of the digital cable and satellite market. Their efforts to add a
managed TiVoToGo service were attacked by the rightsholders who
fought at the FCC to block them. Cable/satellite operators and the
studios would much prefer the public to transition to "bundled"
PVRs that come with your TV service.

These boxes are owned by the cable/satellite companies, who have
absolute control over them. Time-Warner has been known to remotely
delete stored episodes of shows just before the DVD ships, and many
operators have started using "flags" that tell recorders not to
allow fast-forwarding, or to prevent recording altogether.

The reason that YouTube and TiVo are more popular than ThePirateBay
and mythtv is that they're the easiest way for the public to get
what it wants -- the video we want, the way we want it. We use
these services because they're like the original Napster: easy,
well-designed, functional.

But if the entertainment industry squeezes these players out,
ThePirateBay and mythtv are right there, waiting to welcome us in
with open arms. ThePirateBay has already announced that it is
launching a YouTube competitor with no-plugin, in-browser viewing.
Plenty of entrepreneurs are looking at easing the pain and cast of
setting up your own mythtv box. The only reason that the barriers
to BitTorrent and mythtv exist is that it hasn't been worth
anyone's while to capitalize projects to bring them down. But once
the legit competitors of these services are killed, look out.

The thing is, the public doesn't want managed services with limited
rights. We don't want to be stuck using approved devices in
approved ways. We never have -- we are the spiritual descendants of
the customers for "illegal" record albums and "illegal" cable TV.
The demand signal won't go away.

There's no good excuse for going into production on a sequel to The
Napster Wars. We saw that movie. We know how it turns out. Every
Christmas, we get articles about how this was the worst Christmas
ever for CDs. You know what? CD sales are \textbf{never} going to
improve. CDs have been rendered obsolete by Internet distribution
-- and the record industry has locked itself out of the only
profitable, popular music distribution systems yet invented.

Companies like Google/YouTube and TiVo are rarities: tech companies
that want to do deals. They need to be cherished by entertainment
companies, not sued.

(Thanks to Bruce Nash and The-Numbers.com for research assistance
with this article)

\subsection{You DO Like Reading Off a Computer Screen}

(Originally published in Locus Magazine, March 2007)

"I don't like reading off a computer screen" -- it's a cliché of
the e-book world. It means "I don't read novels off of computer
screens" (or phones, or PDAs, or dedicated e-book readers), and
often as not the person who says it is someone who, in fact, spends
every hour that Cthulhu sends reading off a computer screen. It's
like watching someone shovel Mars Bars into his gob while telling
you how much he hates chocolate.

But I know what you mean. You don't like reading long-form works
off of a computer screen. I understand perfectly -- in the ten
minutes since I typed the first word in the paragraph above, I've
checked my mail, deleted two spams, checked an image-sharing
community I like, downloaded a YouTube clip of Stephen Colbert
complaining about the iPhone (pausing my MP3 player first), cleared
out my RSS reader, and then returned to write this paragraph.

This is not an ideal environment in which to concentrate on
long-form narrative (sorry, one sec, gotta blog this guy who's made
cardboard furniture) (wait, the Colbert clip's done, gotta start
the music up) (19 more RSS items). But that's not to say that it's
not an entertainment medium -- indeed, practically everything I do
on the computer entertains the hell out of me. It's nearly all
text-based, too. Basically, what I do on the computer is
pleasure-reading. But it's a fundamentally more scattered,
splintered kind of pleasure. Computers have their own cognitive
style, and it's not much like the cognitive style invented with the
first modern novel (one sec, let me google that and confirm it),
Don Quixote, some 400 years ago.

The novel is an invention, one that was engendered by technological
changes in information display, reproduction, and distribution. The
cognitive style of the novel is different from the cognitive style
of the legend. The cognitive style of the computer is different
from the cognitive style of the novel.

Computers want you to do lots of things with them. Networked
computers doubly so -- they (another RSS item) have a million ways
of asking for your attention, and just as many ways of rewarding
it.

There's a persistent fantasy/nightmare in the publishing world of
the advent of very sharp, very portable computer screens. In the
fantasy version, this creates an infinite new market for electronic
books, and we all get to sell the rights to our work all over
again. In the nightmare version, this leads to runaway piracy, and
no one ever gets to sell a novel again.

I think they're both wrong. The infinitely divisible copyright
ignores the "decision cost" borne by users who have to decide, over
and over again, whether they want to spend a millionth of a cent on
a millionth of a word -- no one buys newspapers by the paragraph,
even though most of us only read a slim fraction of any given
paper. A super-sharp, super-portable screen would be used to read
all day long, but most of us won't spend most of our time reading
anything recognizable as a book on them.

Take the record album. Everything about it is technologically
pre-determined. The technology of the LP demanded artwork to
differentiate one package from the next. The length was set by the
groove density of the pressing plants and playback apparatus. The
dynamic range likewise. These factors gave us the idea of the
40-to-60-minute package, split into two acts, with accompanying
artwork. Musicians were encouraged to create works that would be
enjoyed as a unitary whole for a protracted period -- think of Dark
Side of the Moon, or Sgt. Pepper's.

No one thinks about albums today. Music is now divisible to the
single, as represented by an individual MP3, and then subdivisible
into snippets like ringtones and samples. When recording artists
demand that their works be considered as a whole -- like when
Radiohead insisted that the iTunes Music Store sell their whole
album as a single, indivisible file that you would have to listen
to all the way through -- they sound like cranky throwbacks.

The idea of a 60-minute album is as weird in the Internet era as
the idea of sitting through 15 hours of Der Ring des Nibelungen was
20 years ago. There are some anachronisms who love their long-form
opera, but the real action is in the more fluid stuff that can
slither around on hot wax -- and now the superfluid droplets of
MP3s and samples. Opera survives, but it is a tiny sliver of a much
bigger, looser music market. The future composts the past: old
operas get mounted for living anachronisms; Andrew Lloyd Webber
picks up the rest of the business.

Or look at digital video. We're watching more digital video,
sooner, than anyone imagined. But we're watching it in three-minute
chunks from YouTube. The video's got a pause button so you can stop
it when the phone rings and a scrubber to go back and forth when
you miss something while answering an IM.

And attention spans don't increase when you move from the PC to a
handheld device. These things have less capacity for multitasking
than real PCs, and the network connections are slower and more
expensive. But they are fundamentally multitasking devices -- you
can always stop reading an e-book to play a hand of solitaire that
is interrupted by a phone call -- and their social context is that
they are used in public places, with a million distractions. It is
socially acceptable to interrupt someone who is looking at a PDA
screen. By contrast, the TV room -- a whole room for TV! -- is a
shrine where none may speak until the commercial airs.

The problem, then, isn't that screens aren't sharp enough to read
novels off of. The problem is that novels aren't screeny enough to
warrant protracted, regular reading on screens.

Electronic books are a wonderful adjunct to print books. It's great
to have a couple hundred novels in your pocket when the plane
doesn't take off or the line is too long at the post office. It's
cool to be able to search the text of a novel to find a beloved
passage. It's excellent to use a novel socially, sending it to your
friends, pasting it into your sig file.

But the numbers tell their own story -- people who read off of
screens all day long buy lots of print books and read them
primarily on paper. There are some who prefer an all-electronic
existence (I'd like to be able to get rid of the objects after my
first reading, but keep the e-books around for reference), but
they're in a tiny minority.

There's a generation of web writers who produce "pleasure reading"
on the web. Some are funny. Some are touching. Some are enraging.
Most dwell in Sturgeon's 90th percentile and below. They're not
writing novels. If they were, they wouldn't be web writers.

Mostly, we can read just enough of a free e-book to decide whether
to buy it in hardcopy -- but not enough to substitute the e-book
for the hardcopy. Like practically everything in marketing and
promotion, the trick is to find the form of the work that serves as
enticement, not replacement.

Sorry, got to go -- eight more e-mails.

\subsection{How Do You Protect Artists?}

(Originally published in The Guardian as "Online censorship hurts
us all," Tuesday, Oct 2, 2007)

Artists have lots of problems. We get plagiarized, ripped off by
publishers, savaged by critics, counterfeited -- and we even get
our works copied by "pirates" who give our stuff away for free
online.

But no matter how bad these problems get, they're a distant second
to the gravest, most terrifying problem an artist can face:
censorship.

It's one thing to be denied your credit or compensation, but it's
another thing entirely to have your work suppressed, burned or
banned. You'd never know it, however, judging from the state of the
law surrounding the creation and use of internet publishing tools.

Since 1995, every single legislative initiative on this subject in
the UK's parliament, the European parliament and the US Congress
has focused on making it easier to suppress "illegitimate" material
online. From libel to copyright infringement, from child porn to
anti-terror laws, our legislators have approached the internet with
a single-minded focus on seeing to it that bad material is
expeditiously removed.

And that's the rub. I'm certainly no fan of child porn or hate
speech, but every time a law is passed that reduces the burden of
proof on those who would remove material from the internet,
artists' fortunes everywhere are endangered.

Take the US's 1998 Digital Millennium Copyright Act, which has
equivalents in every European state that has implemented the 2001
European Union Copyright Directive. The DMCA allows anyone to have
any document on the internet removed, simply by contacting its
publisher and asserting that the work infringes his copyright.

The potential for abuse is obvious, and the abuse has been
widespread: from the Church of Scientology to companies that don't
like what reporters write about them, DMCA takedown notices have
fast become the favorite weapon in the cowardly bully's arsenal.

But takedown notices are just the start. While they can help
silence critics and suppress timely information, they're not
actually very effective at stopping widespread copyright
infringement. Viacom sent over 100,000 takedown notices to YouTube
last February, but seconds after it was all removed, new users
uploaded it again.

Even these takedown notices were sloppily constructed: they
included videos of friends eating at barbecue restaurants and
videos of independent bands performing their own work. As a
Recording Industry Association of America spokesman quipped, "When
you go trawling with a net, you catch a few dolphins."

Viacom and others want hosting companies and online service
providers to preemptively evaluate all the material that their
users put online, holding it to ensure that it doesn't infringe
copyright before they release it.

This notion is impractical in the extreme, for at least two
reasons. First, an exhaustive list of copyrighted works would be
unimaginably huge, as every single creative work is copyrighted
from the instant that it is created and "fixed in a tangible
medium".

Second, even if such a list did exist, it would be trivial to
defeat, simply by introducing small changes to the infringing
copies, as spammers do with the text of their messages in order to
evade spam filters.

In fact, the spam wars have some important lessons to teach us
here. Like copyrighted works, spams are infinitely varied and more
are being created every second. Any company that could identify
spam messages -- including permutations and variations on existing
spams -- could write its own ticket to untold billions.

Some of the smartest, most dedicated engineers on the planet devote
every waking hour to figuring out how to spot spam before it gets
delivered. If your inbox is anything like mine, you'll agree that
the war is far from won.

If the YouTubes of the world are going to prevent infringement,
they're going to have to accomplish this by hand-inspecting every
one of the tens of billions of blog posts, videos, text-files,
music files and software uploads made to every single server on the
internet.

And not just cursory inspections, either -- these inspections will
have to be undertaken by skilled, trained specialists (who'd better
be talented linguists, too -- how many English speakers can spot an
infringement in Urdu?).

Such experts don't come cheap, which means that you can anticipate
a terrible denuding of the fertile jungle of internet hosting
companies that are primary means by which tens of millions of
creative people share the fruits of their labor with their fans and
colleagues.

It would be a great Sovietisation of the world's digital printing
presses, a contraction of a glorious anarchy of expression into a
regimented world of expensive and narrow venues for art.

It would be a death knell for the kind of focused, non-commercial
material whose authors couldn't fit the bill for a "managed"
service's legion of lawyers, who would be replaced by more of the
same -- the kind of lowest common denominator rubbish that fills
the cable channels today.

And the worst of it is, we're marching toward this "solution" in
the name of protecting artists. Gee, thanks.

\subsection{It's the Information Economy, Stupid}

(Originally published in The Guardian as "Free data sharing is here
to stay," September 18, 2007)

Since the 1970s, pundits have predicted a transition to an
"information economy." The vision of an economy based on
information seized the imaginations of the world's governments. For
decades now, they have been creating policies to "protect"
information -- stronger copyright laws, international treaties on
patents and trademarks, treaties to protect anti-copying
technology.

The thinking is simple: an information economy must be based on
buying and selling information. Therefore, we need policies to make
it harder to get access to information unless you've paid for it.
That means that we have to make it harder for you to share
information, even after you've paid for it. Without the ability to
fence off your information property, you can't have an information
market to fuel the information economy.

But this is a tragic case of misunderstanding a metaphor. Just as
the industrial economy wasn't based on making it harder to get
access to machines, the information economy won't be based on
making it harder to get access to information. Indeed, the opposite
seems to be true: the more IT we have, the easier it is to access
any given piece of information -- for better or for worse.

It used to be that copy-prevention companies' strategies went like
this: "We'll make it easier to buy a copy of this data than to make
an unauthorized copy of it. That way, only the uber-nerds and the
cash-poor/time-rich classes will bother to copy instead of buy."
But every time a PC is connected to the Internet and its owner is
taught to use search tools like Google (or The Pirate Bay), a third
option appears: you can just download a copy from the Internet.
Every techno-literate participant in the information economy can
choose to access any data, without having to break the anti-copying
technology, just by searching for the cracked copy on the public
Internet. If there's one thing we can be sure of, it's that an
information economy will increase the technological literacy of its
participants.

As I write this, I am sitting in a hotel room in Shanghai, behind
the Great Firewall of China. Theoretically, I can't access blogging
services that carry negative accounts of Beijing's doings, like
Wordpress, Blogspot and Livejournal, nor the image-sharing site
Flickr, nor Wikipedia. The (theoretically) omnipotent bureaucrats
of the local Minitrue have deployed their finest engineering talent
to stop me. Well, these cats may be able to order political
prisoners executed and their organs harvested for Party members,
but they've totally failed to keep Chinese people (and big-nose
tourists like me) off the world's Internet. The WTO is rattling its
sabers at China today, demanding that they figure out how to stop
Chinese people from looking at Bruce Willis movies without
permission -- but the Chinese government can't even figure out how
to stop Chinese people from looking at seditious revolutionary
tracts online.

And, of course, as Paris Hilton, the Church of Scientology and the
King of Thailand have discovered, taking a piece of information off
the Internet is like getting food coloring out of a swimming pool.
Good luck with that.

To see the evidence of the real information economy, look to all
the economic activity that the Internet enables -- not the stuff
that it impedes. All the commerce conducted by salarymen who can
book their own flights with Expedia instead of playing blind-man's
bluff with a travel agent ("Got any flights after 4PM to
Frankfurt?"). All the garage crafters selling their goods on
Etsy.com. All the publishers selling obscure books through Amazon
that no physical bookstore was willing to carry. The salwar kameez
tailors in India selling bespoke clothes to westerners via eBay,
without intervention by a series of skimming intermediaries. The
Internet-era musicians who use the net to pack venues all over the
world by giving away their recordings on social services like
MySpace. Hell, look at my last barber, in Los Angeles: the man
doesn't use a PC, but I found him by googling for "barbers" with my
postcode -- the information economy is driving his cost of customer
acquisition to zero, and he doesn't even have to actively
participate in it.

Better access to more information is the hallmark of the
information economy. The more IT we have, the more skill we have,
the faster our networks get and the better our search tools get,
the more economic activity the information economy generates. Many
of us sell information in the information economy -- I sell my
printed books by giving away electronic books, lawyers and
architects and consultants are in the information business and they
drum up trade with Google ads, and Google is nothing but an
info-broker -- but none of us rely on curtailing access to
information. Like a bottled water company, we compete with free by
supplying a superior service, not by eliminating the competition.

The world's governments might have bought into the old myth of the
information economy, but not so much that they're willing to ban
the PC and the Internet.

\subsection{Downloads Give Amazon Jungle Fever}

(Originally published in The Guardian, December 11, 2007)

Let me start by saying that I love Amazon. I buy everything from
books to clothes to electronics to medication to food to batteries
to toys to furniture to baby supplies from the company. I once even
bought an ironing board on Amazon. No company can top them for ease
of use or for respecting consumer rights when it comes to refunds,
ensuring satisfaction, and taking good care of loyal customers.

As a novelist, I couldn't be happier about Amazon's existence. Not
only does Amazon have a set of superb recommendation tools that
help me sell books, but it also has an affiliate program that lets
me get up to 8.5\% in commissions for sales of my books through the
site - nearly doubling my royalty rate.

As a consumer advocate and activist, I'm delighted by almost every
public policy initiative from Amazon. When the Author's Guild tried
to get Amazon to curtail its used-book market, the company refused
to back down. Founder Jeff Bezos (who is a friend of mine) even
wrote, "when someone buys a book, they are also buying the right to
resell that book, to loan it out, or to even give it away if they
want. Everyone understands this."

More recently, Amazon stood up to the US government, who'd gone on
an illegal fishing expedition for terrorists (TERRORISTS!
TERRORISTS! TERRORISTS!) and asked Amazon to turn over the
purchasing history of 24,000 Amazon customers. The company spent a
fortune fighting for our rights, and won.

It also has a well-deserved reputation for taking care over
copyright "takedown" notices for the material that its customers
post on its site, discarding ridiculous claims rather than blindly
acting on every single notice, no matter how frivolous.

But for all that, it has to be said: Whenever Amazon tries to sell
a digital download, it turns into one of the dumbest companies on
the web.

Take the Kindle, the \$400 handheld ebook reader that Amazon
shipped recently, to vast, ringing indifference.

The device is cute enough - in a clumsy, overpriced, generation-one
kind of way - but the early adopter community recoiled in horror at
the terms of service and anti-copying technology that infected it.
Ebooks that you buy through the Kindle can't be lent or resold
(remember, "when someone buys a book, they are also buying the
right to resell that book...Everyone understands this.")

Mark Pilgrim's "The Future of Reading" enumerates five other Kindle
showstoppers: Amazon can change your ebooks without notifying you
or getting your permission; and if you violate any of the
"agreement", it can delete your ebooks, even if you've paid for
them, and you get no appeal.

It's not just the Kindle, either. Amazon Unbox, the semi-abortive
video download service, shipped with terms of service that included
your granting permission for Amazon to install any software on your
computer, to spy on you, to delete your videos, to delete any other
file on your hard drive, to deny you access to your movies if you
lose them in a crash. This comes from the company that will
cheerfully ship you a replacement DVD if you email them and tell
them that the one you just bought never turned up in the post.

Even Amazon's much-vaunted MP3 store comes with terms of service
that prevent lending and reselling.

I am mystified by this. Amazon is the kind of company that every
etailer should study and copy - the gold standard for e-commerce.
You'd think that if there was any company that would intuitively
get the web, it would be Amazon.

What's more, this is a company that stands up to rightsholder
groups, publishers and the US government - but only when it comes
to physical goods. Why is it that whenever a digital sale is in the
offing, Amazon rolls over on its back and wets itself?

\subsection{What's the Most Important Right Creators Have?}

(Originally published as "How Big Media's Copyright Campaigns
Threaten Internet Free Expression," InformationWeek, November 5,
2007)

Any discussion of "creator's rights" is likely to be limited to
talk about copyright, but copyright is just a side-dish for
creators: the most important right we have is the right to free
expression. And these two rights are always in tension.

Take Viacom's claims against YouTube. The entertainment giant says
that YouTube has been profiting from the fact that YouTube users
upload clips from Viacom shows, and they demand that YouTube take
steps to prevent this from happening in the future. YouTube
actually offered to do something very like this: they invited
Viacom and other rightsholders to send them all the clips they
wanted kept offline, and promised to programatically detect these
clips and interdict them.

But Viacom rejected this offer. Rather, the company wants YouTube
to just figure it out, determine a priori which video clips are
being presented with permission and which ones are not. After all,
Viacom does the very same thing: it won't air clips until a
battalion of lawyers have investigated them and determined whether
they are lawful.

But the Internet is not cable television. Net-based hosting outfits
-- including YouTube, Flickr, Blogger, Scribd, and the Internet
Archive -- offer free publication venues to all comers, enabling
anyone to publish anything. In 1998's Digital Millennium Copyright
Act, Congress considered the question of liability for these
companies and decided to offer them a mixed deal: hosting companies
don't need to hire a million lawyers to review every blog-post
before it goes live, but rightsholders can order them to remove any
infringing material from the net just by sending them a notice that
the material infringes.

This deal enabled hosting companies to offer free platforms for
publication and expression to everyone. But it also allowed anyone
to censor the Internet, just by making claims of infringement,
without offering any evidence to support those claims, without
having to go to court to prove their claims (this has proven to be
an attractive nuisance, presenting an irresistible lure to anyone
with a beef against an online critic, from the Church of
Scientology to Diebold's voting machines division).

The proposal for online hosts to figure out what infringes and what
doesn't is wildly impractical. Under most countries' copyright
laws, creative works receive a copyright from the moment that they
are "fixed in a tangible medium" (hard drives count), and this
means that the pool of copyrighted works is so large as to be
practically speaking infinite. Knowing whether a work is
copyrighted, who holds the copyright, and whether a posting is made
with the rightsholder's permission (or in accord with each nation's
varying ideas about fair use) is impossible. The only way to be
sure is to start from the presumption that each creative work is
infringing, and then make each Internet user prove, to some
lawyer's satisfaction, that she has the right to post each drib of
content that appears on the Web.

Imagine that such a system were the law of the land. There's no way
Blogger or YouTube or Flickr could afford to offer free hosting to
their users. Rather, all these hosted services would have to charge
enough for access to cover the scorching legal bills associated
with checking all material. And not just the freebies, either: your
local ISP, the servers hosting your company's website or your page
for family genealogy: they'd all have to do the same kind of
continuous checking and re-checking of every file you publish with
them.

It would be the end of any publication that couldn't foot the legal
bills to get off the ground. The multi-billion-page Internet would
collapse into the homogeneous world of cable TV (remember when we
thought that a "500-channel universe" would be unimaginably broad?
Imagine an Internet with only 500 "channels!"). From Amazon to Ask
A Ninja, from Blogger to The Everlasting Blort, every bit of online
content is made possible by removing the cost of paying lawyers to
act as the Internet's gatekeepers.

This is great news for artists. The traditional artist's lament is
that our publishers have us over a barrel, controlling the narrow
and vital channels for making works available -- from big gallery
owners to movie studios to record labels to New York publishers.
That's why artists have such a hard time negotiating a decent deal
for themselves (for example, most beginning recording artists have
to agree to have money deducted from their royalty statements for
"breakage" of records en route to stores -- and these deductions
are also levied against digital sales through the iTunes Store!).

But, thanks to the web, artists have more options than ever. The
Internet's most popular video podcasts aren't associated with TV
networks (with all the terrible, one-sided deals that would
entail), rather, they're independent programs like RocketBoom,
Homestar Runner, or the late, lamented Ze Frank Show. These
creators -- along with all the musicians, writers, and other
artists using the net to earn their living -- were able to write
their own ticket. Today, major artists like Radiohead and Madonna
are leaving the record labels behind and trying novel, net-based
ways of promoting their work.

And it's not just the indies who benefit: the existence of
successful independent artists creates fantastic leverage for
artists who negotiate with the majors. More and more, the big media
companies' "like it or leave it" bargaining stance is being
undermined by the possibility that the next big star will shrug,
turn on her heel, and make her fortune without the big companies'
help. This has humbled the bigs, making their deals better and more
artist-friendly.

Bargaining leverage is just for starters. The greatest threat that
art faces is suppression. Historically, artists have struggled just
to make themselves heard, just to safeguard the right to express
themselves. Censorship is history's greatest enemy of art. A
limited-liability Web is a Web where anyone can post anything and
reach \textbf{everyone}.

What's more, this privilege isn't limited to artists. All manner of
communication, from the personal introspection in public "diaries"
to social chatter on MySpace and Facebook, are now possible. Some
artists have taken the bizarre stance that this "trivial" matter is
unimportant and thus a poor excuse for allowing hosted services to
exist in the first place. This is pretty arrogant: a society where
only artists are allowed to impart "important" messages and where
the rest of us are supposed to shut up about our loves, hopes,
aspirations, jokes, family and wants is hardly a democratic
paradise.

Artists are in the free expression business, and technology that
helps free expression helps artists. When lowering the cost of
copyright enforcement raises the cost of free speech, every artist
has a duty to speak out. Our ability to make our art is
inextricably linked with the billions of Internet users who use the
network to talk about their lives.

\subsection{Giving it Away}

(Originally published in Forbes.com, December 2006)

I've been giving away my books ever since my first novel came out,
and boy has it ever made me a bunch of money.

When my first novel, Down and Out in the Magic Kingdom, was
published by Tor Books in January 2003, I also put the entire
electronic text of the novel on the Internet under a Creative
Commons License that encouraged my readers to copy it far and wide.
Within a day, there were 30,000 downloads from my site (and those
downloaders were in turn free to make more copies). Three years and
six printings later, more than 700,000 copies of the book have been
downloaded from my site. The book's been translated into more
languages than I can keep track of, key concepts from it have been
adopted for software projects and there are two competing fan audio
adaptations online.

Most people who download the book don't end up buying it, but they
wouldn't have bought it in any event, so I haven't lost any sales,
I've just won an audience. A tiny minority of downloaders treat the
free e-book as a substitute for the printed book--those are the
lost sales. But a much larger minority treat the e-book as an
enticement to buy the printed book. They're gained sales. As long
as gained sales outnumber lost sales, I'm ahead of the game. After
all, distributing nearly a million copies of my book has cost me
nothing.

The thing about an e-book is that it's a social object. It wants to
be copied from friend to friend, beamed from a Palm device, pasted
into a mailing list. It begs to be converted to witty signatures at
the bottom of e-mails. It is so fluid and intangible that it can
spread itself over your whole life. Nothing sells books like a
personal recommendation--when I worked in a bookstore, the sweetest
words we could hear were "My friend suggested I pick up...." The
friend had made the sale for us, we just had to consummate it. In
an age of online friendship, e-books trump dead trees for word of
mouth.

There are two things that writers ask me about this arrangement:
First, does it sell more books, and second, how did you talk your
publisher into going for this mad scheme?

There's no empirical way to prove that giving away books sells more
books--but I've done this with three novels and a short story
collection (and I'll be doing it with two more novels and another
collection in the next year), and my books have consistently
outperformed my publisher's expectations. Comparing their sales to
the numbers provided by colleagues suggests that they perform
somewhat better than other books from similar writers at similar
stages in their careers. But short of going back in time and
re-releasing the same books under the same circumstances without
the free e-book program, there's no way to be sure.

What is certain is that every writer who's tried giving away
e-books to sell books has come away satisfied and ready to do it
some more.

How did I talk Tor Books into letting me do this? It's not as if
Tor is a spunky dotcom upstart. They're the largest science fiction
publisher in the world, and they're a division of the German
publishing giant Holtzbrinck. They're not patchouli-scented
info-hippies who believe that information wants to be free. Rather,
they're canny assessors of the world of science fiction, perhaps
the most social of all literary genres. Science fiction is driven
by organized fandom, volunteers who put on hundreds of literary
conventions in every corner of the globe, every weekend of the
year. These intrepid promoters treat books as markers of identity
and as cultural artifacts of great import. They evangelize the
books they love, form subcultures around them, cite them in
political arguments, sometimes they even rearrange their lives and
jobs around them.

What's more, science fiction's early adopters defined the social
character of the Internet itself. Given the high correlation
between technical employment and science fiction reading, it was
inevitable that the first nontechnical discussion on the Internet
would be about science fiction. The online norms of idle chatter,
fannish organizing, publishing and leisure are descended from SF
fandom, and if any literature has a natural home in cyberspace,
it's science fiction, the literature that coined the very word
"cyberspace."

Indeed, science fiction was the first form of widely pirated
literature online, through "bookwarez" channels that contained
books that had been hand-scanned, a page at a time, converted to
digital text and proof-read. Even today, the mostly widely pirated
literature online is SF.

Nothing could make me more sanguine about the future. As publisher
Tim O'Reilly wrote in his seminal essay, Piracy is Progressive
Taxation, "being well-enough known to be pirated [is] a crowning
achievement." I'd rather stake my future on a literature that
people care about enough to steal than devote my life to a form
that has no home in the dominant medium of the century.

What about that future? Many writers fear that in the future,
electronic books will come to substitute more readily for print
books, due to changing audiences and improved technology. I am
skeptical of this--the codex format has endured for centuries as a
simple and elegant answer to the affordances demanded by print,
albeit for a relatively small fraction of the population. Most
people aren't and will never be readers--but the people who are
readers will be readers forever, and they are positively pervy for
paper.

But say it does come to pass that electronic books are all anyone
wants.

I don't think it's practical to charge for copies of electronic
works. Bits aren't ever going to get harder to copy. So we'll have
to figure out how to charge for something else. That's not to say
you can't charge for a copy-able bit, but you sure can't force a
reader to pay for access to information anymore.

This isn't the first time creative entrepreneurs have gone through
one of these transitions. Vaudeville performers had to transition
to radio, an abrupt shift from having perfect control over who
could hear a performance (if they don't buy a ticket, you throw
them out) to no control whatsoever (any family whose 12-year-old
could build a crystal set, the day's equivalent of installing
file-sharing software, could tune in). There were business models
for radio, but predicting them a priori wasn't easy. Who could have
foreseen that radio's great fortunes would be had through creating
a blanket license, securing a Congressional consent decree,
chartering a collecting society and inventing a new form of
statistical mathematics to fund it?

Predicting the future of publishing--should the wind change and
printed books become obsolete--is just as hard. I don't know how
writers would earn their living in such a world, but I do know that
I'll never find out by turning my back on the Internet. By being in
the middle of electronic publishing, by watching what hundreds of
thousands of my readers do with my e-books, I get better market
intelligence than I could through any other means. As does my
publisher. As serious as I am about continuing to work as a writer
for the foreseeable future, Tor Books and Holtzbrinck are just as
serious. They've got even more riding on the future of publishing
than me. So when I approached my publisher with this plan to give
away books to sell books, it was a no-brainer for them.

It's good business for me, too. This "market research" of giving
away e-books sells printed books. What's more, having my books more
widely read opens many other opportunities for me to earn a living
from activities around my writing, such as the Fulbright Chair I
got at USC this year, this high-paying article in Forbes, speaking
engagements and other opportunities to teach, write and license my
work for translation and adaptation. My fans' tireless evangelism
for my work doesn't just sell books--it sells me.

The golden age of hundreds of writers who lived off of nothing but
their royalties is bunkum. Throughout history, writers have relied
on day jobs, teaching, grants, inheritances, translation, licensing
and other varied sources to make ends meet. The Internet not only
sells more books for me, it also gives me more opportunities to
earn my keep through writing-related activities.

There has never been a time when more people were reading more
words by more authors. The Internet is a literary world of written
words. What a fine thing that is for writers.

\subsection{Science Fiction is the Only Literature People Care Enough About to Steal on the Internet}

(Originally published in Locus Magazine, July 2006)

As a science fiction writer, no piece of news could make me more
hopeful. It beats the hell out of the alternative -- a future where
the dominant, pluripotent, ubiquitous medium has no place for
science fiction literature.

When radio and records were invented, they were pretty bad news for
the performers of the day. Live performance demanded charisma, the
ability to really put on a magnetic show in front of a crowd. It
didn't matter how technically accomplished you were: if you stood
like a statue on stage, no one wanted to see you do your thing. On
the other hand, you succeeded as a mediocre player, provided you
attacked your performance with a lot of brio.

Radio was clearly good news for musicians -- lots more musicians
were able to make lots more music, reaching lots more people and
making lots more money. It turned performance into an industry,
which is what happens when you add technology to art. But it was
terrible news for charismatics. It put them out on the street,
stuck them with flipping burgers and driving taxis. They knew it,
too. Performers lobbied to have the Marconi radio banned, to send
Marconi back to the drawing board, charged with inventing a radio
they could charge admission to. "We're charismatics, we do
something as old and holy as the first story told before the first
fire in the first cave. What right have you to insist that we
should become mere clerks, working in an obscure back-room, leaving
you to commune with our audiences on our behalf?"

Technology giveth and technology taketh away. Seventy years later,
Napster showed us that, as William Gibson noted, "We may be at the
end of the brief period during which it is possible to charge for
recorded music." Surely we're at the end of the period where it's
possible to exclude those who don't wish to pay. Every song
released can be downloaded gratis from a peer-to-peer network (and
will shortly get easier to download, as hard-drive
price/performance curves take us to a place where all the music
ever recorded will fit on a disposable pocket-drive that you can
just walk over to a friend's place and copy).

But have no fear: the Internet makes it possible for recording
artists to reach a wider audience than ever dreamt of before. Your
potential fans may be spread in a thin, even coat over the world,
in a configuration that could never be cost-effective to reach with
traditional marketing. But the Internet's ability to lower the
costs for artists to reach their audiences and for audiences to
find artists suddenly renders possible more variety in music than
ever before.

Those artists can use the Internet to bring people back to the live
performances that characterized the heyday of Vaudeville. Use your
recordings -- which you can't control -- to drive admissions to
your performances, which you can control. It's a model that's
worked great for jam bands like the Grateful Dead and Phish. It's
also a model that won't work for many of today's artists; 70 years
of evolutionary pressure has selected for artists who are more
virtuoso than charismatic, artists optimized for recording-based
income instead of performance-based income. "How dare you tell us
that we are to be trained monkeys, capering on a stage for your
amusement? We're not charismatics, we're white-collar workers. We
commune with our muses behind closed doors and deliver up our work
product when it's done, through plastic, laser-etched discs. You
have no right to demand that we convert to a live-performance
economy."

Technology giveth and technology taketh away. As bands on MySpace
-- who can fill houses and sell hundreds of thousands of discs
without a record deal, by connecting individually with fans -- have
shown, there's a new market aborning on the Internet for music, one
with fewer gatekeepers to creativity than ever before.

That's the purpose of copyright, after all: to decentralize who
gets to make art. Before copyright, we had patronage: you could
make art if the Pope or the king liked the sound of it. That
produced some damned pretty ceilings and frescos, but it wasn't
until control of art was given over to the market -- by giving
publishers a monopoly over the works they printed, starting with
the Statute of Anne in 1710 -- that we saw the explosion of
creativity that investment-based art could create. Industrialists
weren't great arbiters of who could and couldn't make art, but they
were better than the Pope.

The Internet is enabling a further decentralization in who gets to
make art, and like each of the technological shifts in cultural
production, it's good for some artists and bad for others. The
important question is: will it let more people participate in
cultural production? Will it further decentralize decision-making
for artists?

And for SF writers and fans, the further question is, "Will it be
any good to our chosen medium?" Like I said, science fiction is the
only literature people care enough about to steal on the Internet.
It's the only literature that regularly shows up, scanned and run
through optical character recognition software and lovingly
hand-edited on darknet newsgroups, Russian websites, IRC channels
and elsewhere (yes, there's also a brisk trade in comics and
technical books, but I'm talking about prose fiction here -- though
this is clearly a sign of hope for our friends in tech publishing
and funnybooks).

Some writers are using the Internet's affinity for SF to great
effect. I've released every one of my novels under Creative Commons
licenses that encourage fans to share them freely and widely --
even, in some cases, to remix them and to make new editions of them
for use in the developing world. My first novel, Down and Out in
the Magic Kingdom, is in its sixth printing from Tor, and has been
downloaded more than 650,000 times from my website, and an untold
number of times from others' websites.

I've discovered what many authors have also discovered: releasing
electronic texts of books drives sales of the print editions. An SF
writer's biggest problem is obscurity, not piracy. Of all the
people who chose not to spend their discretionary time and cash on
our works today, the great bulk of them did so because they didn't
know they existed, not because someone handed them a free e-book
version.

But what kind of artist thrives on the Internet? Those who can
establish a personal relationship with their readers -- something
science fiction has been doing for as long as pros have been
hanging out in the con suite instead of the green room. These
conversational artists come from all fields, and they combine the
best aspects of charisma and virtuosity with charm -- the ability
to conduct their online selves as part of a friendly salon that
establishes a non-substitutable relationship with their audiences.
You might find a film, a game, and a book to be equally useful
diversions on a slow afternoon, but if the novel's author is a pal
of yours, that's the one you'll pick. It's a competitive advantage
that can't be beat.

See Neil Gaiman's blog, where he manages the trick of carrying on a
conversation with millions. Or Charlie Stross's Usenet posts.
Scalzi's blogs. J. Michael Straczynski's presence on Usenet --
while in production on Babylon 5, no less -- breeding an army of
rabid fans ready to fax-bomb recalcitrant TV execs into submission
and syndication. See also the MySpace bands selling a million units
of their CDs by adding each buyer to their "friends lists." Watch
Eric Flint manage the Baen Bar, and Warren Ellis's good-natured
growling on his sites, lists, and so forth.

Not all artists have in them to conduct an online salon with their
audiences. Not all Vaudevillians had it in them to transition to
radio. Technology giveth and technology taketh away. SF writers are
supposed to be soaked in the future, ready to come to grips with
it. The future is conversational: when there's more good stuff that
you know about that's one click away or closer than you will ever
click on, it's not enough to know that some book is good. The least
substitutable good in the Internet era is the personal
relationship.

Conversation, not content, is king. If you were stranded on a
desert island and you opted to bring your records instead of your
friends, we'd call you a sociopath. Science fiction writers who can
insert themselves into their readers' conversations will be set for
life.

\subsection{How Copyright Broke}

(Originally published in Locus Magazine, September, 2006)

The theory is that if the Internet can't be controlled, then
copyright is dead. The thing is, the Internet is a machine for
copying things cheaply, quickly, and with as little control as
possible, while copyright is the right to control who gets to make
copies, so these two abstractions seem destined for a fatal
collision, right?

Wrong.

The idea that copyright confers the exclusive right to control
copying, performance, adaptation, and general use of a creative
work is a polite fiction that has been mostly harmless throughout
its brief history, but which has been laid bare by the Internet,
and the disjoint is showing.

Theoretically, if I sell you a copy of one of my novels, I'm
conferring upon you a property interest in a lump of atoms -- the
pages of the book -- as well as a license to make some reasonable
use of the ethereal ideas embedded upon the page, the copyrighted
work.

Copyright started with a dispute between Scottish and English
publishers, and the first copyright law, 1709's Statute of Anne,
conferred the exclusive right to publish new editions of a book on
the copyright holder. It was a fair competition statute, and it was
silent on the rights that the copyright holder had in respect of
his customers: the readers. Publishers got a legal tool to fight
their competitors, a legal tool that made a distinction between the
corpus -- a physical book -- and the spirit -- the novel writ on
its pages. But this legal nicety was not "customer-facing." As far
as a reader was concerned, once she bought a book, she got the same
rights to it as she got to any other physical object, like a potato
or a shovel. Of course, the reader couldn't print a new edition,
but this had as much to do with the realities of technology as it
did with the law. Printing presses were rare and expensive: telling
a 17th-century reader that he wasn't allowed to print a new edition
of a book you sold him was about as meaningful as telling him he
wasn't allowed to have it laser-etched on the surface of the moon.
Publishing books wasn't something readers did.

Indeed, until the photocopier came along, it was practically
impossible for a member of the audience to infringe copyright in a
way that would rise to legal notice. Copyright was like a
tank-mine, designed only to go off when a publisher or record
company or radio station rolled over it. We civilians couldn't
infringe copyright (many thanks to Jamie Boyle for this useful
analogy).

It wasn't the same for commercial users of copyrighted works. For
the most part, a radio station that played a record was expected to
secure permission to do so (though this permission usually comes in
the form of a government-sanctioned blanket license that cuts
through all the expense of negotiating in favor of a single monthly
payment that covers all radio play). If you shot a movie, you were
expected to get permission for the music you put in it. Critically,
there are many uses that commercial users never paid for. Most
workplaces don't pay for the music their employees enjoy while they
work. An ad agency that produces a demo reel of recent commercials
to use as part of a creative briefing to a designer doesn't pay for
this extremely commercial use. A film company whose set-designer
clips and copies from magazines and movies to produce a "mood book"
never secures permission nor offers compensation for these uses.

Theoretically, the contours of what you may and may not do without
permission are covered under a legal doctrine called "fair use,"
which sets out the factors a judge can use to weigh the question of
whether an infringement should be punished. While fair use is a
vital part of the way that works get made and used, it's very rare
for an unauthorized use to get adjudicated on this basis.

No, the realpolitik of unauthorized use is that users are not
required to secure permission for uses that the rights holder will
never discover. If you put some magazine clippings in your mood
book, the magazine publisher will never find out you did so. If you
stick a Dilbert cartoon on your office-door, Scott Adams will never
know about it.

So while technically the law has allowed rights holders to
infinitely discriminate among the offerings they want to make --
Special discounts on this book, which may only be read on
Wednesdays! This film half-price, if you agree only to show it to
people whose names start with D! -- practicality has dictated that
licenses could only be offered on enforceable terms.

When it comes to retail customers for information goods -- readers,
listeners, watchers -- this whole license abstraction falls flat.
No one wants to believe that the book he's brought home is only
partly his, and subject to the terms of a license set out on the
flyleaf. You'd be a flaming jackass if you showed up at a con and
insisted that your book may not be read aloud, nor photocopied in
part and marked up for a writers' workshop, nor made the subject of
a piece of fan-fiction.

At the office, you might get a sweet deal on a coffee machine on
the promise that you'll use a certain brand of coffee, and even
sign off on a deal to let the coffee company check in on this from
time to time. But no one does this at home. We instinctively and
rightly recoil from the idea that our personal, private dealings in
our homes should be subject to oversight from some company from
whom we've bought something. We bought it. It's ours. Even when we
rent things, like cars, we recoil from the idea that Hertz might
track our movements, or stick a camera in the steering wheel.

When the Internet and the PC made it possible to sell a lot of
purely digital "goods" -- software, music, movies and books
delivered as pure digits over the wire, without a physical good
changing hands, the copyright lawyers groped about for a way to
take account of this. It's in the nature of a computer that it
copies what you put on it. A computer is said to be working, and of
high quality, in direct proportion to the degree to which it
swiftly and accurately copies the information that it is presented
with.

The copyright lawyers had a versatile hammer in their toolbox: the
copyright license. These licenses had been presented to
corporations for years. Frustratingly (for the lawyers), these
corporate customers had their own counsel, and real bargaining
power, which made it impossible to impose really interesting
conditions on them, like limiting the use of a movie such that it
couldn't be fast-forwarded, or preventing the company from letting
more than one employee review a journal at a time.

Regular customers didn't have lawyers or negotiating leverage. They
were a natural for licensing regimes. Have a look at the next
click-through "agreement" you're provided with on purchasing a
piece of software or an electronic book or song. The terms set out
in those agreements are positively Dickensian in their marvelous
idiocy. Sony BMG recently shipped over eight million music CDs with
an "agreement" that bound its purchasers to destroy their music if
they left the country or had a house-fire, and to promise not to
listen to their tunes while at work.

But customers understand property -- you bought it, you own it --
and they don't understand copyright. Practically no one understands
copyright. I know editors at multibillion-dollar publishing houses
who don't know the difference between copyright and trademark (if
you've ever heard someone say, "You need to defend a copyright or
you lose it," you've found one of these people who confuse
copyright and trademark; what's more, this statement isn't
particularly true of trademark, either). I once got into an
argument with a senior Disney TV exec who truly believed that if
you re-broadcasted an old program, it was automatically
re-copyrighted and got another 95 years of exclusive use (that's
wrong).

So this is where copyright breaks: When copyright lawyers try to
treat readers and listeners and viewers as if they were (weak and
unlucky) corporations who could be strong-armed into license
agreements you wouldn't wish on a dog. There's no conceivable world
in which people are going to tiptoe around the property they've
bought and paid for, re-checking their licenses to make sure that
they're abiding by the terms of an agreement they doubtless never
read. Why read something if it's non-negotiable, anyway?

The answer is simple: treat your readers' property as property.
What readers do with their own equipment, as private, noncommercial
actors, is not a fit subject for copyright regulation or oversight.
The Securities Exchange Commission doesn't impose rules on you when
you loan a friend five bucks for lunch. Anti-gambling laws aren't
triggered when you bet your kids an ice-cream cone that you'll
bicycle home before them. Copyright shouldn't come between an
end-user of a creative work and her property.

Of course, this approach is made even simpler by the fact that
practically every customer for copyrighted works already operates
on this assumption. Which is not to say that this might make some
business-models more difficult to pursue. Obviously, if there was
some way to ensure that a given publisher was the only source for a
copyrighted work, that publisher could hike up its prices, devote
less money to service, and still sell its wares. Having to compete
with free copies handed from user to user makes life harder --
hasn't it always?

But it is most assuredly possible. Look at Apple's wildly popular
iTunes Music Store, which has sold over one billion tracks since
2003. Every song on iTunes is available as a free download from
user-to-user, peer-to-peer networks like Kazaa. Indeed, the P2P
monitoring company Big Champagne reports that the average
time-lapse between a iTunes-exclusive song being offered by Apple
and that same song being offered on P2P networks is 180 seconds.

Every iTunes customer could readily acquire every iTunes song for
free, using the fastest-adopted technology in history. Many of them
do (just as many fans photocopy their favorite stories from
magazines and pass them around to friends). But Apple has figured
out how to compete well enough by offering a better service and a
better experience to realize a good business out of this. (Apple
also imposes ridiculous licensing restrictions, but that's a
subject for a future column).

Science fiction is a genre of clear-eyed speculation about the
future. It should have no place for wishful thinking about a world
where readers willingly put up with the indignity of being treated
as "licensees" instead of customers.

\subsection{And now a brief commercial interlude:}

If you're enjoying this book and have been thinking of buying a
copy, here's a chance to do so:

\href{http://craphound.com/content/buy}{http://craphound.com/content/buy}

\subsection{In Praise of Fanfic}

(Originally published in Locus Magazine, May 2007)

I wrote my first story when I was six. It was 1977, and I had just
had my mind blown clean out of my skull by a new movie called Star
Wars (the golden age of science fiction is 12; the golden age of
cinematic science fiction is six). I rushed home and stapled a
bunch of paper together, trimmed the sides down so that it
approximated the size and shape of a mass-market paperback, and set
to work. I wrote an elaborate, incoherent ramble about Star Wars,
in which the events of the film replayed themselves, tweaked to
suit my tastes.

I wrote a lot of Star Wars fanfic that year. By the age of 12, I'd
graduated to Conan. By the age of 18, it was Harlan Ellison. By the
age of 26, it was Bradbury, by way of Gibson. Today, I hope I write
more or less like myself.

Walk the streets of Florence and you'll find a copy of the David on
practically every corner. For centuries, the way to become a
Florentine sculptor has been to copy Michelangelo, to learn from
the master. Not just the great Florentine sculptors, either --
great or terrible, they all start with the master; it can be the
start of a lifelong passion, or a mere fling. The copy can be art,
or it can be crap -- the best way to find out which kind you've got
inside you is to try.

Science fiction has the incredible good fortune to have attracted
huge, social groups of fan-fiction writers. Many pros got their
start with fanfic (and many of them still work at it in secret),
and many fanfic writers are happy to scratch their itch by working
only with others' universes, for the sheer joy of it. Some fanfic
is great -- there's plenty of Buffy fanfic that trumps the
official, licensed tie-in novels -- and some is purely dreadful.

Two things are sure about all fanfic, though: first, that people
who write and read fanfic are already avid readers of writers whose
work they're paying homage to; and second, that the people who
write and read fanfic derive fantastic satisfaction from their
labors. This is great news for writers.

Great because fans who are so bought into your fiction that they'll
make it their own are fans forever, fans who'll evangelize your
work to their friends, fans who'll seek out your work however you
publish it.

Great because fans who use your work therapeutically, to work out
their own creative urges, are fans who have a damned good reason to
stick with the field, to keep on reading even as our numbers
dwindle. Even when the fandom revolves around movies or TV shows,
fanfic is itself a literary pursuit, something undertaken in the
world of words. The fanfic habit is a literary habit.

In Japan, comic book fanfic writers publish fanfic manga called
dojinshi -- some of these titles dwarf the circulation of the work
they pay tribute to, and many of them are sold commercially.
Japanese comic publishers know a good thing when they see it, and
these fanficcers get left alone by the commercial giants they
attach themselves to.

And yet for all this, there are many writers who hate fanfic. Some
argue that fans have no business appropriating their characters and
situations, that it's disrespectful to imagine your precious
fictional people into sexual scenarios, or to retell their stories
from a different point of view, or to snatch a victorious happy
ending from the tragic defeat the writer ended her book with.

Other writers insist that fans who take without asking -- or
against the writer's wishes -- are part of an "entitlement culture"
that has decided that it has the moral right to lift scenarios and
characters without permission, that this is part of our larger
postmodern moral crisis that is making the world a worse place.

Some writers dismiss all fanfic as bad art and therefore unworthy
of appropriation. Some call it copyright infringement or trademark
infringement, and every now and again, some loony will actually
threaten to sue his readers for having had the gall to tell his
stories to each other.

I'm frankly flabbergasted by these attitudes. Culture is a lot
older than art -- that is, we have had social storytelling for a
lot longer than we've had a notional class of artistes whose
creativity is privileged and elevated to the numinous, far above
the everyday creativity of a kid who knows that she can paint and
draw, tell a story and sing a song, sculpt and invent a game.

To call this a moral failing -- and a new moral failing at that! --
is to turn your back on millions of years of human history. It's no
failing that we internalize the stories we love, that we rework
them to suit our minds better. The Pygmalion story didn't start
with Shaw or the Greeks, nor did it end with My Fair Lady.
Pygmalion is at least thousands of years old -- think of Moses
passing for the Pharaoh's son! -- and has been reworked in a
billion bedtime stories, novels, D\&D games, movies, fanfic
stories, songs, and legends.

Each person who retold Pygmalion did something both original -- no
two tellings are just alike -- and derivative, for there are no new
ideas under the sun. Ideas are easy. Execution is hard. That's why
writers don't really get excited when they're approached by people
with great ideas for novels. We've all got more ideas than we can
use -- what we lack is the cohesive whole.

Much fanfic -- the stuff written for personal consumption or for a
small social group -- isn't bad art. It's just not art. It's not
written to make a contribution to the aesthetic development of
humanity. It's created to satisfy the deeply human need to play
with the stories that constitute our world. There's nothing trivial
about telling stories with your friends -- even if the stories
themselves are trivial. The act of telling stories to one another
is practically sacred -- and it's unquestionably profound. What's
more, lots of retellings are art: witness Pat Murphy's wonderful
There and Back Again (Tolkien) and Geoff Ryman's brilliant World
Fantasy Award-winning Was (L. Frank Baum).

The question of respect is, perhaps, a little thornier. The
dominant mode of criticism in fanfic circles is to compare a work
to the canon -- "Would Spock ever say that, in 'real' life?" What's
more, fanfic writers will sometimes apply this test to works that
are of the canon, as in "Spock never would have said that, and Gene
Roddenberry has no business telling me otherwise."

This is a curious mix of respect and disrespect. Respect because
it's hard to imagine a more respectful stance than the one that
says that your work is the yardstick against which all other work
is to be measured -- what could be more respectful than having your
work made into the gold standard? On the other hand, this business
of telling writers that they've given their characters the wrong
words and deeds can feel obnoxious or insulting.

Writers sometimes speak of their characters running away from them,
taking on a life of their own. They say that these characters --
drawn from real people in our lives and mixed up with our own
imagination -- are autonomous pieces of themselves. It's a short
leap from there to mystical nonsense about protecting our notional,
fictional children from grubby fans who'd set them to screwing each
other or bowing and scraping before some thinly veiled version of
the fanfic writer herself.

There's something to the idea of the autonomous character. Big
chunks of our wetware are devoted to simulating other people,
trying to figure out if we are likely to fight or fondle them. It's
unsurprising that when you ask your brain to model some other
person, it rises to the task. But that's exactly what happens to a
reader when you hand your book over to him: he simulates your
characters in his head, trying to interpret that character's
actions through his own lens.

Writers can't ask readers not to interpret their work. You can't
enjoy a novel that you haven't interpreted -- unless you model the
author's characters in your head, you can't care about what they do
and why they do it. And once readers model a character, it's only
natural that readers will take pleasure in imagining what that
character might do offstage, to noodle around with it. This isn't
disrespect: it's active reading.

Our field is incredibly privileged to have such an active fanfic
writing practice. Let's stop treating them like thieves and start
treating them like honored guests at a table that we laid just for
them.

\subsection{Metacrap: Putting the torch to seven straw-men of the meta-utopia}

(Self-published, 26 August 2001)

\subsubsection{1. Introduction}

Metadata is "data about data" -- information like keywords,
page-length, title, word-count, abstract, location, SKU, ISBN, and
so on. Explicit, human-generated metadata has enjoyed recent
trendiness, especially in the world of XML. A typical scenario goes
like this: a number of suppliers get together and agree on a
metadata standard -- a Document Type Definition or scheme -- for a
given subject area, say washing machines. They agree to a common
vocabulary for describing washing machines: size, capacity, energy
consumption, water consumption, price. They create machine-readable
databases of their inventory, which are available in whole or part
to search agents and other databases, so that a consumer can enter
the parameters of the washing machine he's seeking and query
multiple sites simultaneously for an exhaustive list of the
available washing machines that meet his criteria.

If everyone would subscribe to such a system and create good
metadata for the purposes of describing their goods, services and
information, it would be a trivial matter to search the Internet
for highly qualified, context-sensitive results: a fan could find
all the downloadable music in a given genre, a manufacturer could
efficiently discover suppliers, travelers could easily choose a
hotel room for an upcoming trip.

A world of exhaustive, reliable metadata would be a utopia. It's
also a pipe-dream, founded on self-delusion, nerd hubris and
hysterically inflated market opportunities.

\subsubsection{2. The problems}

There are at least seven insurmountable obstacles between the world
as we know it and meta-utopia. I'll enumerate them below:.

\subsubsection{2.1 People lie}

Metadata exists in a competitive world. Suppliers compete to sell
their goods, cranks compete to convey their crackpot theories (mea
culpa), artists compete for audience. Attention-spans and wallets
may not be zero-sum, but they're damned close.

That's why:

\begin{itemize}
\item
  A search for any commonly referenced term at a search-engine like
  Altavista will often turn up at least one porn link in the first
  ten results.
\item
  Your mailbox is full of spam with subject lines like "Re: The
  information you requested."
\item
  Publisher's Clearing House sends out advertisements that holler
  "You may already be a winner!"
\item
  Press-releases have gargantuan lists of empty buzzwords attached to
  them.
\end{itemize}
Meta-utopia is a world of reliable metadata. When poisoning the
well confers benefits to the poisoners, the meta-waters get awfully
toxic in short order.

\subsubsection{2.2 People are lazy}

You and me are engaged in the incredibly serious business of
creating information. Here in the Info-Ivory-Tower, we understand
the importance of creating and maintaining excellent metadata for
our information.

But info-civilians are remarkably cavalier about their information.
Your clueless aunt sends you email with no subject line, half the
pages on Geocities are called "Please title this page" and your
boss stores all of his files on his desktop with helpful titles
like "UNTITLED.DOC."

This laziness is bottomless. No amount of ease-of-use will end it.
To understand the true depths of meta-laziness, download ten random
MP3 files from Napster. Chances are, at least one will have no
title, artist or track information -- this despite the fact that
adding in this info merely requires clicking the "Fetch Track Info
from CDDB" button on every MP3-ripping application.

Short of breaking fingers or sending out squads of vengeful
info-ninjas to add metadata to the average user's files, we're
never gonna get there.

\subsubsection{2.3 People are stupid}

Even when there's a positive benefit to creating good metadata,
people steadfastly refuse to exercise care and diligence in their
metadata creation.

Take eBay: every seller there has a damned good reason for
double-checking their listings for typos and misspellings. Try
searching for "plam" on eBay. Right now, that turns up nine typoed
listings for "Plam Pilots." Misspelled listings don't show up in
correctly-spelled searches and hence garner fewer bids and lower
sale-prices. You can almost always get a bargain on a Plam Pilot at
eBay.

The fine (and gross) points of literacy -- spelling, punctuation,
grammar -- elude the vast majority of the Internet's users. To
believe that J. Random Users will suddenly and en masse learn to
spell and punctuate -- let alone accurately categorize their
information according to whatever hierarchy they're supposed to be
using -- is self-delusion of the first water.

\subsubsection{2.4 Mission: Impossible -- know thyself}

In meta-utopia, everyone engaged in the heady business of
describing stuff carefully weighs the stuff in the balance and
accurately divines the stuff's properties, noting those results.

Simple observation demonstrates the fallacy of this assumption.
When Nielsen used log-books to gather information on the viewing
habits of their sample families, the results were heavily skewed to
Masterpiece Theater and Sesame Street. Replacing the journals with
set-top boxes that reported what the set was actually tuned to
showed what the average American family was really watching: naked
midget wrestling, America's Funniest Botched Cosmetic Surgeries and
Jerry Springer presents: "My daughter dresses like a slut!"

Ask a programmer how long it'll take to write a given module, or a
contractor how long it'll take to fix your roof. Ask a laconic
Southerner how far it is to the creek. Better yet, throw darts --
the answer's likely to be just as reliable.

People are lousy observers of their own behaviors. Entire religions
are formed with the goal of helping people understand themselves
better; therapists rake in billions working for this very end.

Why should we believe that using metadata will help J. Random User
get in touch with her Buddha nature?

\subsubsection{2.5 Schemas aren't neutral}

In meta-utopia, the lab-coated guardians of epistemology sit down
and rationally map out a hierarchy of ideas, something like this:

Nothing:

Black holes

Everything:

Matter:

Earth:

Planets

Washing Machines

Wind:

Oxygen

Poo-gas

Fire:

Nuclear fission

Nuclear fusion

"Mean Devil Woman" Louisiana Hot-Sauce

In a given sub-domain, say, Washing Machines, experts agree on
sub-hierarchies, with classes for reliability, energy consumption,
color, size, etc.

This presumes that there is a "correct" way of categorizing ideas,
and that reasonable people, given enough time and incentive, can
agree on the proper means for building a hierarchy.

Nothing could be farther from the truth. Any hierarchy of ideas
necessarily implies the importance of some axes over others. A
manufacturer of small, environmentally conscious washing machines
would draw a hierarchy that looks like this:

Energy consumption:

Water consumption:

Size:

Capacity:

Reliability

While a manufacturer of glitzy, feature-laden washing machines
would want something like this:

Color:

Size:

Programmability:

Reliability

The conceit that competing interests can come to easy accord on a
common vocabulary totally ignores the power of organizing
principles in a marketplace.

\subsubsection{2.6 Metrics influence results}

Agreeing to a common yardstick for measuring the important stuff in
any domain necessarily privileges the items that score high on that
metric, regardless of those items' overall suitability. IQ tests
privilege people who are good at IQ tests, Nielsen Ratings
privilege 30- and 60-minute TV shows (which is why MTV doesn't show
videos any more -- Nielsen couldn't generate ratings for
three-minute mini-programs, and so MTV couldn't demonstrate the
value of advertising on its network), raw megahertz scores
privilege Intel's CISC chips over Motorola's RISC chips.

Ranking axes are mutually exclusive: software that scores high for
security scores low for convenience, desserts that score high for
decadence score low for healthiness. Every player in a metadata
standards body wants to emphasize their high-scoring axes and
de-emphasize (or, if possible, ignore altogether) their low-scoring
axes.

It's wishful thinking to believe that a group of people competing
to advance their agendas will be universally pleased with any
hierarchy of knowledge. The best that we can hope for is a detente
in which everyone is equally miserable.

\subsubsection{2.7 There's more than one way to describe something}

"No, I'm not watching cartoons! It's cultural anthropology."

"This isn't smut, it's art."

"It's not a bald spot, it's a solar panel for a sex-machine."

Reasonable people can disagree forever on how to describe
something. Arguably, your Self is the collection of associations
and descriptors you ascribe to ideas. Requiring everyone to use the
same vocabulary to describe their material denudes the cognitive
landscape, enforces homogeneity in ideas.

And that's just not right.

\subsubsection{3. Reliable metadata}

Do we throw out metadata, then?

Of course not. Metadata can be quite useful, if taken with a
sufficiently large pinch of salt. The meta-utopia will never come
into being, but metadata is often a good means of making rough
assumptions about the information that floats through the
Internet.

Certain kinds of implicit metadata is awfully useful, in fact.
Google exploits metadata about the structure of the World Wide Web:
by examining the number of links pointing at a page (and the number
of links pointing at each linker), Google can derive statistics
about the number of Web-authors who believe that that page is
important enough to link to, and hence make extremely reliable
guesses about how reputable the information on that page is.

This sort of observational metadata is far more reliable than the
stuff that human beings create for the purposes of having their
documents found. It cuts through the marketing bullshit, the
self-delusion, and the vocabulary collisions.

Taken more broadly, this kind of metadata can be thought of as a
pedigree: who thinks that this document is valuable? How closely
correlated have this person's value judgments been with mine in
times gone by? This kind of implicit endorsement of information is
a far better candidate for an information-retrieval panacea than
all the world's schema combined.

\subsection{Amish for QWERTY}

(Originally published on the O'Reilly Network, 07/09/2003)

I learned to type before I learned to write. The QWERTY keyboard
layout is hard-wired to my brain, such that I can't write anything
of significance without that I have a 101-key keyboard in front of
me. This has always been a badge of geek pride: unlike the creaking
pen-and-ink dinosaurs that I grew up reading, I'm well adapted to
the modern reality of technology. There's a secret elitist pride in
touch-typing on a laptop while staring off into space, fingers
flourishing and caressing the keys.

But last week, my pride got pricked. I was brung low by a phone.
Some very nice people from Nokia loaned me a very
latest-and-greatest camera-phone, the kind of gadget I've described
in my science fiction stories. As I prodded at the little 12-key
interface, I felt like my father, a 60s-vintage computer scientist
who can't get his wireless network to work, must feel. Like a
creaking dino. Like history was passing me by. I'm 31, and I'm
obsolete. Or at least Amish.

People think the Amish are technophobes. Far from it. They're
ideologues. They have a concept of what right-living consists of,
and they'll use any technology that serves that ideal -- and
mercilessly eschew any technology that would subvert it. There's
nothing wrong with driving the wagon to the next farm when you want
to hear from your son, so there's no need to put a phone in the
kitchen. On the other hand, there's nothing right about your
livestock dying for lack of care, so a cellphone that can call the
veterinarian can certainly find a home in the horse barn.

For me, right-living is the 101-key, QWERTY, computer-centric
mediated lifestyle. It's having a bulky laptop in my bag, crouching
by the toilets at a strange airport with my AC adapter plugged into
the always-awkwardly-placed power source, running software that I
chose and installed, communicating over the wireless network. I use
a network that has no incremental cost for communication, and a
device that lets me install any software without permission from
anyone else. Right-living is the highly mutated,
commodity-hardware- based, public and free Internet. I'm
QWERTY-Amish, in other words.

I'm the kind of perennial early adopter who would gladly volunteer
to beta test a neural interface, but I find myself in a moral panic
when confronted with the 12-button keypad on a cellie, even though
that interface is one that has been greedily adopted by billions of
people worldwide, from strap-hanging Japanese schoolgirls to Kenyan
electoral scrutineers to Filipino guerrillas in the bush. The idea
of paying for every message makes my hackles tumesce and evokes a
reflexive moral conviction that text-messaging is inherently
undemocratic, at least compared to free-as-air email. The idea of
only running the software that big-brother telco has permitted me
on my handset makes me want to run for the hills.

The thumb-generation who can tap out a text-message under their
desks while taking notes with the other hand -- they're in for it,
too. The pace of accelerated change means that we're all of us
becoming wed to interfaces -- ways of communicating with our tools
and our world -- that are doomed, doomed, doomed. The 12-buttoners
are marrying the phone company, marrying a centrally controlled
network that requires permission to use and improve, a Stalinist
technology whose centralized choke points are subject to regulation
and the vagaries of the telcos. Long after the phone companies have
been out-competed by the pure and open Internet (if such a glorious
day comes to pass), the kids of today will be bound by its
interface and its conventions.

The sole certainty about the future is its Amishness. We will all
bend our brains to suit an interface that we will either have to
abandon or be left behind. Choose your interface -- and the values
it implies -- carefully, then, before you wed your thought
processes to your fingers' dance. It may be the one you're stuck
with.

\subsection{Ebooks: Neither E, Nor Books}

(Paper for the O'Reilly Emerging Technologies Conference, San
Diego, February 12, 2004)

Forematter:

This talk was initially given at the O'Reilly Emerging Technology
Conference [
\href{http://conferences.oreillynet.com/et2004}{http://conferences.oreillynet.com/et2004}/
], along with a set of slides that, for copyright reasons (ironic!)
can't be released alongside of this file. However, you will find,
interspersed in this text, notations describing the places where
new slides should be loaded, in [square-brackets].

For starters, let me try to summarize the lessons and intuitions
I've had about ebooks from my release of two novels and most of a
short story collection online under a Creative Commons license. A
parodist who published a list of alternate titles for the
presentations at this event called this talk, "eBooks Suck Right
Now," [eBooks suck right now] and as funny as that is, I don't
think it's true.

No, if I had to come up with another title for this talk, I'd call
it: "Ebooks: You're Soaking in Them." [Ebooks: You're Soaking in
Them] That's because I think that the shape of ebooks to come is
almost visible in the way that people interact with text today, and
that the job of authors who want to become rich and famous is to
come to a better understanding of that shape.

I haven't come to a perfect understanding. I don't know what the
future of the book looks like. But I have ideas, and I'll share
them with you:

\begin{enumerate}
\item
  Ebooks aren't marketing. [Ebooks aren't marketing] OK, so ebooks
  *are* marketing: that is to say that giving away ebooks sells more
  books. Baen Books, who do a lot of series publishing, have found
  that giving away electronic editions of the previous installments
  in their series to coincide with the release of a new volume sells
  the hell out of the new book -- and the backlist. And the number of
  people who wrote to me to tell me about how much they dug the ebook
  and so bought the paper-book far exceeds the number of people who
  wrote to me and said, "Ha, ha, you hippie, I read your book for
  free and now I'm not gonna buy it." But ebooks *shouldn't* be just
  about marketing: ebooks are a goal unto themselves. In the final
  analysis, more people will read more words off more screens and
  fewer words off fewer pages and when those two lines cross, ebooks
  are gonna have to be the way that writers earn their keep, not the
  way that they promote the dead-tree editions.
\item
  Ebooks complement paper books. [Ebooks complement paper books].
  Having an ebook is good. Having a paper book is good. Having both
  is even better. One reader wrote to me and said that he read half
  my first novel from the bound book, and printed the other half on
  scrap-paper to read at the beach. Students write to me to say that
  it's easier to do their term papers if they can copy and paste
  their quotations into their word-processors. Baen readers use the
  electronic editions of their favorite series to build concordances
  of characters, places and events.
\item
  Unless you own the ebook, you don't 0wn the book [Unless you own
  the ebook, you don't 0wn the book]. I take the view that the book
  is a "practice" -- a collection of social and economic and artistic
  activities -- and not an "object." Viewing the book as a "practice"
  instead of an object is a pretty radical notion, and it begs the
  question: just what the hell is a book? Good question. I write all
  of my books in a text-editor [TEXT EDITOR SCREENGRAB] (BBEdit, from
  Barebones Software -- as fine a text-editor as I could hope for).
  From there, I can convert them into a formatted two-column PDF
  [TWO-UP SCREENGRAB]. I can turn them into an HTML file [BROWSER
  SCREENGRAB]. I can turn them over to my publisher, who can turn
  them into galleys, advanced review copies, hardcovers and
  paperbacks. I can turn them over to my readers, who can convert
  them to a bewildering array of formats [DOWNLOAD PAGE SCREENGRAB].
  Brewster Kahle's Internet Bookmobile can convert a digital book
  into a four-color, full-bleed, perfect-bound, laminated-cover,
  printed-spine paper book in ten minutes, for about a dollar. Try
  converting a paper book to a PDF or an html file or a text file or
  a RocketBook or a printout for a buck in ten minutes! It's ironic,
  because one of the frequently cited reasons for preferring paper to
  ebooks is that paper books confer a sense of ownership of a
  physical object. Before the dust settles on this ebook thing,
  owning a paper book is going to feel less like ownership than
  having an open digital edition of the text.
\item
  Ebooks are a better deal for writers. [Ebooks are a better deal for
  writers] The compensation for writers is pretty thin on the ground.
  *Amazing Stories,* Hugo Gernsback's original science fiction
  magazine, paid a couple cents a word. Today, science fiction
  magazines pay...a couple cents a word. The sums involved are so
  minuscule, they're not even insulting: they're *quaint* and
  *historical*, like the WHISKEY 5 CENTS sign over the bar at a
  pioneer village. Some writers do make it big, but they're *rounding
  errors* as compared to the total population of sf writers earning
  some of their living at the trade. Almost all of us could be making
  more money elsewhere (though we may dream of earning a
  stephenkingload of money, and of course, no one would play the
  lotto if there were no winners). The primary incentive for writing
  has to be artistic satisfaction, egoboo, and a desire for
  posterity. Ebooks get you that. Ebooks become a part of the corpus
  of human knowledge because they get indexed by search engines and
  replicated by the hundreds, thousands or millions. They can be
  googled.
\end{enumerate}
Even better: they level the playing field between writers and
trolls. When Amazon kicked off, many writers got their knickers in
a tight and powerful knot at the idea that axe-grinding yahoos were
filling the Amazon message-boards with ill-considered slams at
their work -- for, if a personal recommendation is the best way to
sell a book, then certainly a personal condemnation is the best way
to \textbf{not} sell a book. Today, the trolls are still with us,
but now, the readers get to decide for themselves. Here's a bit of
a review of Down and Out in the Magic Kingdom that was recently
posted to Amazon by "A reader from Redwood City, CA":

[QUOTED TEXT]

\begin{quote}
I am really not sure what kind of drugs critics are smoking, or
what kind of payola may be involved. But regardless of what
Entertainment Weekly says, whatever this newspaper or that magazine
says, you shouldn't waste your money. Download it for free from
Corey's (sic) site, read the first page, and look away in disgust
-- this book is for people who think Dan Brown's Da Vinci Code is
great writing.

\end{quote}
Back in the old days, this kind of thing would have really pissed
me off. Axe-grinding, mouth-breathing yahoos, defaming my good
name! My stars and mittens! But take a closer look at that damning
passage:

[PULL-QUOTE]

\begin{quote}
Download it for free from Corey's site, read the first page

\end{quote}
You see that? Hell, this guy is \textbf{working for me}!
[ADDITIONAL PULL QUOTES] Someone accuses a writer I'm thinking of
reading of paying off Entertainment Weekly to say nice things about
his novel, "a surprisingly bad writer," no less, whose writing is
"stiff, amateurish, and uninspired!" I wanna check that writer out.
And I can. In one click. And then I can make up my own mind.

You don't get far in the arts without healthy doses of both ego and
insecurity, and the downside of being able to google up all the
things that people are saying about your book is that it can play
right into your insecurities -- "all these people will have it in
their minds not to bother with my book because they've read the
negative interweb reviews!" But the flipside of that is the ego:
"If only they'd give it a shot, they'd see how good it is." And the
more scathing the review is, the more likely they are to give it a
shot. Any press is good press, so long as they spell your URL right
(and even if they spell your name wrong!).

\begin{enumerate}
\item
  Ebooks need to embrace their nature. [Ebooks need to embrace their
  nature.] The distinctive value of ebooks is orthogonal to the value
  of paper books, and it revolves around the mix-ability and
  send-ability of electronic text. The more you constrain an ebook's
  distinctive value propositions -- that is, the more you restrict a
  reader's ability to copy, transport or transform an ebook -- the
  more it has to be valued on the same axes as a paper-book. Ebooks
  *fail* on those axes. Ebooks don't beat paper-books for
  sophisticated typography, they can't match them for quality of
  paper or the smell of the glue. But just try sending a paper book
  to a friend in Brazil, for free, in less than a second. Or loading
  a thousand paper books into a little stick of flash-memory dangling
  from your keychain. Or searching a paper book for every instance of
  a character's name to find a beloved passage. Hell, try clipping a
  pithy passage out of a paper book and pasting it into your
  sig-file.
\item
  Ebooks demand a different attention span (but not a shorter one).
  [Ebooks demand a different attention span (but not a shorter one).]
  Artists are always disappointed by their audience's
  attention-spans. Go back far enough and you'll find cuneiform
  etchings bemoaning the current Sumerian go-go lifestyle with its
  insistence on myths with plotlines and characters and action, not
  like we had in the old days. As artists, it would be a hell of a
  lot easier if our audiences were more tolerant of our penchant for
  boring them. We'd get to explore a lot more ideas without worrying
  about tarting them up with easy-to-swallow chocolate coatings of
  entertainment. We like to think of shortened attention spans as a
  product of the information age, but check this out:
\end{enumerate}
[Nietzsche quote]

\begin{quote}
To be sure one thing necessary above all: if one is to practice
reading as an \textbf{art} in this way, something needs to be
un-learned most thoroughly in these days.

\end{quote}
In other words, if my book is too boring, it's because you're not
paying enough attention. Writers say this stuff all the time, but
this quote isn't from this century or the last. [Nietzsche quote
with attribution] It's from the preface to Nietzsche's "Genealogy
of Morals," published in \textbf{1887.}

Yeah, our attention-spans are \textbf{different} today, but they
aren't necessarily \textbf{shorter}. Warren Ellis's fans managed to
hold the storyline for Transmetropolitan [Transmet cover] in their
minds for \textbf{five years} while the story trickled out in
monthly funnybook installments. JK Rowlings's installments on the
Harry Potter series get fatter and fatter with each new volume.
Entire forests are sacrificed to long-running series fiction like
Robert Jordan's Wheel of Time books, each of which is approximately
20,000 pages long (I may be off by an order of magnitude one way or
another here). Sure, presidential debates are conducted in
soundbites today and not the days-long oratory extravaganzas of the
Lincoln-Douglas debates, but people manage to pay attention to the
24-month-long presidential campaigns from start to finish.

\begin{enumerate}
\item
  We need *all* the ebooks. [We need *all* the ebooks] The vast
  majority of the words ever penned are lost to posterity. No one
  library collects all the still-extant books ever written and no one
  person could hope to make a dent in that corpus of written work.
  None of us will ever read more than the tiniest sliver of human
  literature. But that doesn't mean that we can stick with just the
  most popular texts and get a proper ebook revolution.
\end{enumerate}
For starters, we're all edge-cases. Sure, we all have the shared
desire for the core canon of literature, but each of us want to
complete that collection with different texts that are as
distinctive and individualistic as fingerprints. If we all look
like we're doing the same thing when we read, or listen to music,
or hang out in a chatroom, that's because we're not looking closely
enough. The shared-ness of our experience is only present at a
coarse level of measurement: once you get into really granular
observation, there are as many differences in our "shared"
experience as there are similarities.

More than that, though, is the way that a large collection of
electronic text differs from a small one: it's the difference
between a single book, a shelf full of books and a library of
books. Scale makes things different. Take the Web: none of us can
hope to read even a fraction of all the pages on the Web, but by
analyzing the link structures that bind all those pages together,
Google is able to actually tease out machine-generated conclusions
about the relative relevance of different pages to different
queries. None of us will ever eat the whole corpus, but Google can
digest it for us and excrete the steaming nuggets of goodness that
make it the search-engine miracle it is today.

\begin{enumerate}
\item
  Ebooks are like paper books. [Ebooks are like paper books]. To
  round out this talk, I'd like to go over the ways that ebooks are
  more like paper books than you'd expect. One of the truisms of
  retail theory is that purchasers need to come into contact with a
  good several times before they buy -- seven contacts is tossed
  around as the magic number. That means that my readers have to hear
  the title, see the cover, pick up the book, read a review, and so
  forth, seven times, on average, before they're ready to buy.
\end{enumerate}
There's a temptation to view downloading a book as comparable to
bringing it home from the store, but that's the wrong metaphor.
Some of the time, maybe most of the time, downloading the text of
the book is like taking it off the shelf at the store and looking
at the cover and reading the blurbs (with the advantage of not
having to come into contact with the residual DNA and burger king
left behind by everyone else who browsed the book before you). Some
writers are horrified at the idea that three hundred thousand
copies of my first novel were downloaded and "only" ten thousand or
so were sold so far. If it were the case that for ever copy sold,
thirty were taken home from the store, that would be a horrifying
outcome, for sure. But look at it another way: if one out of every
thirty people who glanced at the cover of my book bought it, I'd be
a happy author. And I am. Those downloads cost me no more than
glances at the cover in a bookstore, and the sales are healthy.

We also like to think of physical books as being inherently
\textbf{countable} in a way that digital books aren't (an irony,
since computers are damned good at counting things!). This is
important, because writers get paid on the basis of the number of
copies of their books that sell, so having a good count makes a
difference. And indeed, my royalty statements contain precise
numbers for copies printed, shipped, returned and sold.

But that's a false precision. When the printer does a run of a
book, it always runs a few extra at the start and finish of the run
to make sure that the setup is right and to account for the
occasional rip, drop, or spill. The actual total number of books
printed is approximately the number of books ordered, but never
exactly -- if you've ever ordered 500 wedding invitations, chances
are you received 500-and-a-few back from the printer and that's
why.

And the numbers just get fuzzier from there. Copies are stolen.
Copies are dropped. Shipping people get the count wrong. Some
copies end up in the wrong box and go to a bookstore that didn't
order them and isn't invoiced for them and end up on a sale table
or in the trash. Some copies are returned as damaged. Some are
returned as unsold. Some come back to the store the next morning
accompanied by a whack of buyer's remorse. Some go to the place
where the spare sock in the dryer ends up.

The numbers on a royalty statement are actuarial, not actual. They
represent a kind of best-guess approximation of the copies shipped,
sold, returned and so forth. Actuarial accounting works pretty
well: well enough to run the juggernaut banking, insurance, and
gambling industries on. It's good enough for divvying up the
royalties paid by musical rights societies for radio airplay and
live performance. And it's good enough for counting how many copies
of a book are distributed online or off.

Counts of paper books are differently precise from counts of
electronic books, sure: but neither one is inherently countable.

And finally, of course, there's the matter of selling books.
However an author earns her living from her words, printed or
encoded, she has as her first and hardest task to find her
audience. There are more competitors for our attention than we can
possibly reconcile, prioritize or make sense of. Getting a book
under the right person's nose, with the right pitch, is the hardest
and most important task any writer faces.

\begin{center}\rule{3in}{0.4pt}\end{center}

I care about books, a lot. I started working in libraries and
bookstores at the age of 12 and kept at it for a decade, until I
was lured away by the siren song of the tech world. I knew I wanted
to be a writer at the age of 12, and now, 20 years later, I have
three novels, a short story collection and a nonfiction book out,
two more novels under contract, and another book in the works.
[BOOK COVERS] I've won a major award in my genre, science fiction,
[CAMPBELL AWARD] and I'm nominated for another one, the 2003 Nebula
Award for best novelette. [NEBULA]

I own a \textbf{lot} of books. Easily more than 10,000 of them, in
storage on both coasts of the North American continent [LIBRARY
LADDER]. I have to own them, since they're the tools of my trade:
the reference works I refer to as a novelist and writer today. Most
of the literature I dig is very short-lived, it disappears from the
shelf after just a few months, usually for good. Science fiction is
inherently ephemeral. [ACE DOUBLES]

Now, as much as I love books, I love computers, too. Computers are
fundamentally different from modern books in the same way that
printed books are different from monastic Bibles: they are
malleable. Time was, a "book" was something produced by many
months' labor by a scribe, usually a monk, on some kind of durable
and sexy substrate like foetal lambskin. [ILLUMINATED BIBLE]
Gutenberg's xerox machine changed all that, changed a book into
something that could be simply run off a press in a few minutes'
time, on substrate more suitable to ass-wiping than exaltation in a
place of honor in the cathedral. The Gutenberg press meant that
rather than owning one or two books, a member of the ruling class
could amass a library, and that rather than picking only a few
subjects from enshrinement in print, a huge variety of subjects
could be addressed on paper and handed from person to person.
[KAPITAL/TIJUANA BIBLE]

Most new ideas start with a precious few certainties and a lot of
speculation. I've been doing a bunch of digging for certainties and
a lot of speculating lately, and the purpose of this talk is to lay
out both categories of ideas.

This all starts with my first novel, Down and Out in the Magic
Kingdom [COVER], which came out on January 9, 2003. At that time,
there was a lot of talk in my professional circles about, on the
one hand, the dismal failure of ebooks, and, on the other, the new
and scary practice of ebook "piracy." [alt.binaries.e-books
screengrab] It was strikingly weird that no one seemed to notice
that the idea of ebooks as a "failure" was at strong odds with the
notion that electronic book "piracy" was worth worrying about: I
mean, if ebooks are a failure, then who gives a rats if intarweb
dweebs are trading them on Usenet?

A brief digression here, on the double meaning of "ebooks." One
meaning for that word is "legitimate" ebook ventures, that is to
say, rightsholder-authorized editions of the texts of books,
released in a proprietary, use-restricted format, sometimes for use
on a general-purpose PC and sometimes for use on a special-purpose
hardware device like the nuvoMedia Rocketbook [ROCKETBOOK]. The
other meaning for ebook is a "pirate" or unauthorized electronic
edition of a book, usually made by cutting the binding off of a
book and scanning it a page at a time, then running the resulting
bitmaps through an optical character recognition app to convert
them into ASCII text, to be cleaned up by hand. These books are
pretty buggy, full of errors introduced by the OCR. A lot of my
colleagues worry that these books also have deliberate errors,
created by mischievous book-rippers who cut, add or change text in
order to "improve" the work. Frankly, I have never seen any
evidence that any book-ripper is interested in doing this, and
until I do, I think that this is the last thing anyone should be
worrying about.

Back to Down and Out in the Magic Kingdom [COVER]. Well, not yet. I
want to convey to you the depth of the panic in my field over ebook
piracy, or "bookwarez" as it is known in book-ripper circles.
Writers were joining the discussion on alt.binaries.ebooks using
assumed names, claiming fear of retaliation from scary hax0r kids
who would presumably screw up their credit-ratings in retaliation
for being called thieves. My editor, a blogger, hacker and
guy-in-charge-of-the-largest-sf-line-in-the-world named Patrick
Nielsen Hayden posted to one of the threads in the newsgroup,
saying, in part [SCREENGRAB]:

\begin{quote}
Pirating copyrighted etext on Usenet and elsewhere is going to
happen more and more, for the same reasons that everyday folks make
audio cassettes from vinyl LPs and audio CDs, and videocassette
copies of store-bought videotapes. Partly it's greed; partly it's
annoyance over retail prices; partly it's the desire to Share Cool
Stuff (a motivation usually underrated by the victims of this kind
of small-time hand-level piracy). Instantly going to Defcon One
over it and claiming it's morally tantamount to mugging little old
ladies in the street will make it kind of difficult to move forward
from that position when it doesn't work. In the 1970s, the record
industry shrieked that "home taping is killing music." It's hard
for ordinary folks to avoid noticing that music didn't die. But the
record industry's credibility on the subject wasn't exactly
enhanced.

\end{quote}
Patrick and I have a long relationship, starting when I was 18
years old and he kicked in toward a scholarship fund to send me to
a writers' workshop, continuing to a fateful lunch in New York in
the mid-Nineties when I showed him a bunch of Project Gutenberg
texts on my Palm Pilot and inspired him to start licensing Tor's
titles for PDAs [PEANUTPRESS SCREENGRAB], to the
turn-of-the-millennium when he bought and then published my first
novel (he's bought three more since -- I really like Patrick!).

Right as bookwarez newsgroups were taking off, I was shocked silly
by legal action by one of my colleagues against AOL/Time-Warner for
carrying the alt.binaries.ebooks newsgroup. This writer alleged
that AOL should have a duty to remove this newsgroup, since it
carried so many infringing files, and that its failure to do so
made it a contributory infringer, and so liable for the incredibly
stiff penalties afforded by our newly minted copyright laws like
the No Electronic Theft Act and the loathsome Digital Millennium
Copyright Act or DMCA.

Now there was a scary thought: there were people out there who
thought the world would be a better place if ISPs were given the
duty of actively policing and censoring the websites and newsfeeds
their customers had access to, including a requirement that ISPs
needed to determine, all on their own, what was an unlawful
copyright infringement -- something more usually left up to judges
in the light of extensive amicus briefings from esteemed copyright
scholars [WIND DONE GONE GRAPHIC].

This was a stupendously dumb idea, and it offended me down to my
boots. Writers are supposed to be advocates of free expression, not
censorship. It seemed that some of my colleagues loved the First
Amendment, but they were reluctant to share it with the rest of the
world.

Well, dammit, I had a book coming out, and it seemed to be an
opportunity to try to figure out a little more about this ebook
stuff. On the one hand, ebooks were a dismal failure. On the other
hand, there were more books posted to alt.binaries.ebooks every
day.

This leads me into the two certainties I have about ebooks:

\begin{enumerate}
\item
  More people are reading more words off more screens every day
  [GRAPHIC]
\item
  Fewer people are reading fewer words off fewer pages every day
  [GRAPHIC]
\end{enumerate}
These two certainties begged a lot of questions.

[CHART: EBOOK FAILINGS]

\begin{itemize}
\item
  Screen resolutions are too low to effectively replace paper
\item
  People want to own physical books because of their visceral appeal
  (often this is accompanied by a little sermonette on how good books
  smell, or how good they look on a bookshelf, or how evocative an
  old curry stain in the margin can be)
\item
  You can't take your ebook into the tub
\item
  You can't read an ebook without power and a computer
\item
  File-formats go obsolete, paper has lasted for a long time
\end{itemize}
None of these seemed like very good explanations for the "failure"
of ebooks to me. If screen resolutions are too low to replace
paper, then how come everyone I know spends more time reading off a
screen every year, up to and including my sainted grandmother
(geeks have a really crappy tendency to argue that certain
technologies aren't ready for primetime because their grandmothers
won't use them -- well, my grandmother sends me email all the time.
She types 70 words per minute, and loves to show off grandsonular
email to her pals around the pool at her Florida retirement
condo)?

The other arguments were a lot more interesting, though. It seemed
to me that electronic books are \textbf{different} from paper
books, and have different virtues and failings. Let's think a
little about what the book has gone through in years gone by. This
is interesting because the history of the book is the history of
the Enlightenment, the Reformation, the Pilgrims, and, ultimately
the colonizing of the Americas and the American Revolution.

Broadly speaking, there was a time when books were hand-printed on
rare leather by monks. The only people who could read them were
priests, who got a regular eyeful of the really cool cartoons the
monks drew in the margins. The priests read the books aloud, in
Latin [LATIN BIBLE] (to a predominantly non-Latin-speaking
audience) in cathedrals, wreathed in pricey incense that rose from
censers swung by altar boys.

Then Johannes Gutenberg invented the printing press. Martin Luther
turned that press into a revolution. [LUTHER BIBLE] He printed
Bibles in languages that non-priests could read, and distributed
them to normal people who got to read the word of God all on their
own. The rest, as they say, is history.

Here are some interesting things to note about the advent of the
printing press:

[CHART: LUTHER VERSUS THE MONKS]

\begin{itemize}
\item
  Luther Bibles lacked the manufacturing quality of the illuminated
  Bibles. They were comparatively cheap and lacked the typographical
  expressiveness that a really talented monk could bring to bear when
  writing out the word of God
\item
  Luther Bibles were utterly unsuited to the traditional use-case for
  Bibles. A good Bible was supposed to reinforce the authority of the
  man at the pulpit. It needed heft, it needed impressiveness, and
  most of all, it needed rarity.
\item
  The user-experience of Luther Bibles sucked. There was no incense,
  no altar boys, and who (apart from the priesthood) knew that
  reading was so friggin' hard on the eyes?
\item
  Luther Bibles were a lot less trustworthy than the illuminated
  numbers. Anyone with a press could run one off, subbing in any
  apocryphal text he wanted -- and who knew how accurate that
  translation was? Monks had an entire Papacy behind them, running a
  quality-assurance operation that had stood Europe in good stead for
  centuries.
\end{itemize}
In the late nineties, I went to conferences where music execs
patiently explained that Napster was doomed, because you didn't get
any cover-art or liner-notes with it, you couldn't know if the rip
was any good, and sometimes the connection would drop mid-download.
I'm sure that many Cardinals espoused the points raised above with
equal certainty.

What the record execs and the cardinals missed was all the ways
that Luther Bibles kicked ass:

[CHART: WHY LUTHER BIBLES KICKED ASS]

\begin{itemize}
\item
  They were cheap and fast. Loads of people could acquire them
  without having to subject themselves to the authority and approval
  of the Church
\item
  They were in languages that non-priests could read. You no longer
  had to take the Church's word for it when its priests explained
  what God really meant
\item
  They birthed a printing-press ecosystem in which lots of books
  flourished. New kinds of fiction, poetry, politics, scholarship and
  so on were all enabled by the printing presses whose initial
  popularity was spurred by Luther's ideas about religion.
\end{itemize}
Note that all of these virtues are orthogonal to the virtues of a
monkish Bible. That is, none of the things that made the Gutenberg
press a success were the things that made monk-Bibles a success.

By the same token, the reasons to love ebooks have precious little
to do with the reasons to love paper books.

[CHART: WHY EBOOKS KICK ASS]

\begin{itemize}
\item
  They are easy to share. Secrets of Ya-Ya Sisterhood went from a
  midlist title to a bestseller by being passed from hand to hand by
  women in reading circles. Slashdorks and other netizens have social
  life as rich as reading-circlites, but they don't ever get to see
  each other face to face; the only kind of book they can pass from
  hand to hand is an ebook. What's more, the single factor most
  correlated with a purchase is a recommendation from a friend --
  getting a book recommended by a pal is more likely to sell you on
  it than having read and enjoyed the preceding volume in a series!
\item
  They are easy to slice and dice. This is where the Mac evangelist
  in me comes out -- minority platforms matter. It's a truism of the
  Napsterverse that most of the files downloaded are bog-standard
  top-40 tracks, like 90 percent or so, and I believe it. We all want
  to popular music. That's why it's popular. But the interesting
  thing is the other ten percent. Bill Gates told the New York Times
  that Microsoft lost the search wars by doing "a good job on the 80
  percent of common queries and ignor[ing] the other stuff. But it's
  the remaining 20 percent that counts, because that's where the
  quality perception is." Why did Napster captivate so many of us?
  Not because it could get us the top-40 tracks that we could hear
  just by snapping on the radio: it was because 80 percent of the
  music ever recorded wasn't available for sale anywhere in the
  world, and in that 80 percent were all the songs that had ever
  touched us, all the earworms that had been lodged in our
  hindbrains, all the stuff that made us smile when we heard it.
  Those songs are different for all of us, but they share the trait
  of making the difference between a compelling service and, well,
  top-40 Clearchannel radio programming. It was the minority of
  tracks that appealed to the majority of us. By the same token, the
  malleability of electronic text means that it can be readily
  repurposed: you can throw it on a webserver or convert it to a
  format for your favorite PDA; you can ask your computer to read it
  aloud or you can search the text for a quotation to cite in a book
  report or to use in your sig. In other words, most people who
  download the book do so for the predictable reason, and in a
  predictable format -- say, to sample a chapter in the HTML format
  before deciding whether to buy the book -- but the thing that
  differentiates a boring e-text experience from an exciting one is
  the minority use -- printing out a couple chapters of the book to
  bring to the beach rather than risk getting the hardcopy wet and
  salty.
\end{itemize}
Tool-makers and software designers are increasingly aware of the
notion of "affordances" in design. You can bash a nail into the
wall with any heavy, heftable object from a rock to a hammer to a
cast-iron skillet. However, there's something about a hammer that
cries out for nail-bashing, it has affordances that tilt its holder
towards swinging it. And, as we all know, when all you have is a
hammer, everything starts to look like a nail.

The affordance of a computer -- the thing it's designed to do -- is
to slice-and-dice collections of bits. The affordance of the
Internet is to move bits at very high speed around the world at
little-to-no cost. It follows from this that the center of the
ebook experience is going to involve slicing and dicing text and
sending it around.

Copyright lawyers have a word for these activities: infringement.
That's because copyright gives creators a near-total monopoly over
copying and remixing of their work, pretty much forever
(theoretically, copyright expires, but in actual practice,
copyright gets extended every time the early Mickey Mouse cartoons
are about to enter the public domain, because Disney swings a very
big stick on the Hill).

This is a huge problem. The biggest possible problem. Here's why:

[CHART: HOW BROKEN COPYRIGHT SCREWS EVERYONE]

\begin{itemize}
\item
  Authors freak out. Authors have been schooled by their peers that
  strong copyright is the only thing that keeps them from getting
  savagely rogered in the marketplace. This is pretty much true: it's
  strong copyright that often defends authors from their publishers'
  worst excesses. However, it doesn't follow that strong copyright
  protects you from your *readers*.
\item
  Readers get indignant over being called crooks. Seriously. You're a
  small businessperson. Readers are your customers. Calling them
  crooks is bad for business.
\item
  Publishers freak out. Publishers freak out, because they're in the
  business of grabbing as much copyright as they can and hanging onto
  it for dear life because, dammit, you never know. This is why
  science fiction magazines try to trick writers into signing over
  improbable rights for things like theme park rides and action
  figures based on their work -- it's also why literary agents are
  now asking for copyright-long commissions on the books they
  represent: copyright covers so much ground and takes to long to
  shake off, who wouldn't want a piece of it?
\item
  Liability goes through the roof. Copyright infringement, especially
  on the Net, is a supercrime. It carries penalties of \$150,000 per
  infringement, and aggrieved rights-holders and their
  representatives have all kinds of special powers, like the ability
  to force an ISP to turn over your personal information before
  showing evidence of your alleged infringement to a judge. This
  means that anyone who suspects that he might be on the wrong side
  of copyright law is going to be terribly risk-averse: publishers
  non-negotiably force their authors to indemnify them from
  infringement claims and go one better, forcing writers to prove
  that they have "cleared" any material they quote, even in the case
  of brief fair-use quotations, like song-titles at the opening of
  chapters. The result is that authors end up assuming potentially
  life-destroying liability, are chilled from quoting material around
  them, and are scared off of public domain texts because an honest
  mistake about the public-domain status of a work carries such a
  terrible price.
\item
  Posterity vanishes. In the Eldred v. Ashcroft Supreme Court hearing
  last year, the court found that 98 percent of the works in
  copyright are no longer earning money for anyone, but that figuring
  out who these old works belong to with the degree of certainty that
  you'd want when one mistake means total economic apocalypse would
  cost more than you could ever possibly earn on them. That means
  that 98 percent of works will largely expire long before the
  copyright on them does. Today, the names of science fiction's
  ancestral founders -- Mary Shelley, Arthur Conan Doyle, Edgar Allan
  Poe, Jules Verne, HG Wells -- are still known, their work still a
  part of the discourse. Their spiritual descendants from Hugo
  Gernsback onward may not be so lucky -- if their work continues to
  be "protected" by copyright, it might just vanish from the face of
  the earth before it reverts to the public domain.
\end{itemize}
This isn't to say that copyright is bad, but that there's such a
thing as good copyright and bad copyright, and that sometimes, too
much good copyright is a bad thing. It's like chilis in soup: a
little goes a long way, and too much spoils the broth.

From the Luther Bible to the first phonorecords, from radio to the
pulps, from cable to MP3, the world has shown that its first
preference for new media is its "democratic-ness" -- the ease with
which it can reproduced.

(And please, before we get any farther, forget all that business
about how the Internet's copying model is more disruptive than the
technologies that proceeded it. For Christ's sake, the Vaudeville
performers who sued Marconi for inventing the radio had to go from
a regime where they had \textbf{one hundred percent} control over
who could get into the theater and hear them perform to a regime
where they had \textbf{zero} percent control over who could build
or acquire a radio and tune into a recording of them performing.
For that matter, look at the difference between a monkish Bible and
a Luther Bible -- next to that phase-change, Napster is peanuts)

Back to democratic-ness. Every successful new medium has traded off
its artifact-ness -- the degree to which it was populated by
bespoke hunks of atoms, cleverly nailed together by master
craftspeople -- for ease of reproduction. Piano rolls weren't as
expressive as good piano players, but they scaled better -- as did
radio broadcasts, pulp magazines, and MP3s. Liner notes, hand
illumination and leather bindings are nice, but they pale in
comparison to the ability of an individual to actually get a copy
of her own.

Which isn't to say that old media die. Artists still
hand-illuminate books; master pianists still stride the boards at
Carnegie Hall, and the shelves burst with tell-all biographies of
musicians that are richer in detail than any liner-notes booklet.
The thing is, when all you've got is monks, every book takes on the
character of a monkish Bible. Once you invent the printing press,
all the books that are better-suited to movable type migrate into
that new form. What's left behind are those items that are best
suited to the old production scheme: the plays that \textbf{need}
to be plays, the books that are especially lovely on creamy paper
stitched between covers, the music that is most enjoyable performed
live and experienced in a throng of humanity.

Increased democratic-ness translates into decreased control: it's a
lot harder to control who can copy a book once there's a
photocopier on every corner than it is when you need a monastery
and several years to copy a Bible. And that decreased control
demands a new copyright regime that rebalances the rights of
creators with their audiences.

For example, when the VCR was invented, the courts affirmed a new
copyright exemption for time-shifting; when the radio was invented,
the Congress granted an anti-trust exemption to the record labels
in order to secure a blanket license; when cable TV was invented,
the government just ordered the broadcasters to sell the
cable-operators access to programming at a fixed rate.

Copyright is perennially out of date, because its latest rev was
generated in response to the last generation of technology. The
temptation to treat copyright as though it came down off the
mountain on two stone tablets (or worse, as "just like" real
property) is deeply flawed, since, by definition, current copyright
only considers the last generation of tech.

So, are bookwarez in violation of copyright law? Duh. Is this the
end of the world? \textbf{Duh}. If the Catholic church can survive
the printing press, science fiction will certainly weather the
advent of bookwarez.

\begin{center}\rule{3in}{0.4pt}\end{center}

Lagniappe [Lagniappe]

We're almost done here, but there's one more thing I'd like to do
before I get off the stage. [Lagniappe: an unexpected bonus or
extra] Think of it as a "lagniappe" -- a little something extra to
thank you for your patience.

About a year ago, I released my first novel, Down and Out in the
Magic Kingdom, on the net, under the terms of the most restrictive
Creative Commons license available. All it allowed my readers to do
was send around copies of the book. I was cautiously dipping my toe
into the water, though at the time, it felt like I was taking a
plunge.

Now I'm going to take a plunge. Today, I will re-license the text
of Down and Out in the Magic Kingdom under a Creative Commons
"Attribution-ShareAlike-Derivs-Noncommercial" license [HUMAN
READABLE LICENSE], which means that as of today, you have my
blessing to create derivative works from my first book. You can
make movies, audiobooks, translations, fan-fiction, slash fiction
(God help us) [GEEK HIERARCHY], furry slash fiction [GEEK HIERARCHY
DETAIL], poetry, translations, t-shirts, you name it, with two
provisos: that one, you have to allow everyone else to rip, mix and
burn your creations in the same way you're hacking mine; and on the
other hand, you've got to do it noncommercially.

The sky didn't fall when I dipped my toe in. Let's see what happens
when I get in up to my knees.

The text with the new license will be online before the end of the
day. Check craphound.com/down for details.

Oh, and I'm also releasing the text of this speech under a Creative
Commons Public Domain dedication, [Public domain dedication] giving
it away to the world to do with as it see fits. It'll be linked off
my blog, Boing Boing, before the day is through.

\subsection{Free(konomic) E-books}

(Originally published in Locus Magazine, September 2007)

Can giving away free electronic books really sell printed books? I
think so. As I explained in my March column ("You Do Like Reading
Off a Computer Screen"), I don't believe that most readers want to
read long-form works off a screen, and I don't believe that they
will ever want to read long-form works off a screen. As I say in
the column, the problem with reading off a screen isn't resolution,
eyestrain, or compatibility with reading in the bathtub: it's that
computers are seductive, they tempt us to do other things, making
concentrating on a long-form work impractical.

Sure, some readers have the cognitive quirk necessary to read
full-length works off screens, or are motivated to do so by other
circumstances (such as being so broke that they could never hope to
buy the printed work). The rational question isn't, "Will giving
away free e-books cost me sales?" but rather, "Will giving away
free e-books win me more sales than it costs me?"

This is a very hard proposition to evaluate in a quantitative way.
Books aren't lattes or cable-knit sweaters: each book sells (or
doesn't) due to factors that are unique to that title. It's hard to
imagine an empirical, controlled study in which two "equivalent"
books are published, and one is also available as a free download,
the other not, and the difference calculated as a means of
"proving" whether e-books hurt or help sales in the long run.

I've released all of my novels as free downloads simultaneous with
their print publication. If I had a time machine, I could
re-release them without the free downloads and compare the royalty
statements. Lacking such a device, I'm forced to draw conclusions
from qualitative, anecdotal evidence, and I've collected plenty of
that:

\begin{itemize}
\item
  Many writers have tried free e-book releases to tie in with the
  print release of their works. To the best of my knowledge, every
  writer who's tried this has repeated the experiment with future
  works, suggesting a high degree of satisfaction with the outcomes
\item
  A writer friend of mine had his first novel come out at the same
  time as mine. We write similar material and are often compared to
  one another by critics and reviewers. My first novel had a free
  download, his didn't. We compared sales figures and I was doing
  substantially better than him -- he subsequently convinced his
  publisher to let him follow suit
\item
  Baen Books has a pretty good handle on expected sales for new
  volumes in long-running series; having sold many such series, they
  have lots of data to use in sales estimates. If Volume N sells X
  copies, we expect Volume N+1 to sell Y copies. They report that
  they have seen a measurable uptick in sales following from free
  e-book releases of previous and current volumes
\item
  David Blackburn, a Harvard PhD candidate in economics, published a
  paper in 2004 in which he calculated that, for music, "piracy"
  results in a net increase in sales for all titles in the 75th
  percentile and lower; negligible change in sales for the "middle
  class" of titles between the 75th percentile and the 97th
  percentile; and a small drag on the "super-rich" in the 97th
  percentile and higher. Publisher Tim O'Reilly describes this as
  "piracy's progressive taxation," apportioning a small
  wealth-redistribution to the vast majority of works, no net change
  to the middle, and a small cost on the richest few
\item
  Speaking of Tim O'Reilly, he has just published a detailed,
  quantitative study of the effect of free downloads on a single
  title. O'Reilly Media published Asterisk: The Future of Telephony,
  in November 2005, simultaneously releasing the book as a free
  download. By March 2007, they had a pretty detailed picture of the
  sales-cycle of this book -- and, thanks to industry standard
  metrics like those provided by Bookscan, they could compare it,
  apples-to-apples style, against the performance of competing books
  treating with the same subject. O'Reilly's conclusion: downloads
  didn't cause a decline in sales, and appears to have resulted in a
  lift in sales. This is particularly noteworthy because the book in
  question is a technical reference work, exclusively consumed by
  computer programmers who are by definition disposed to read off
  screens. Also, this is a reference work and therefore is more
  likely to be useful in electronic form, where it can be easily
  searched
\item
  In my case, my publishers have gone back to press repeatedly for my
  books. The print runs for each edition are modest -- I'm a midlist
  writer in a world with a shrinking midlist -- but publishers print
  what they think they can sell, and they're outselling their
  expectations
\item
  The new opportunities arising from my free downloads are so
  numerous as to be uncountable -- foreign rights deals, comic book
  licenses, speaking engagements, article commissions -- I've made
  more money in these secondary markets than I have in royalties
\item
  More anecdotes: I've had literally thousands of people approach me
  by e-mail and at signings and cons to say, "I found your work
  online for free, got hooked, and started buying it." By contrast,
  I've had all of five e-mails from people saying, "Hey, idiot,
  thanks for the free book, now I don't have to buy the print
  edition, ha ha!"
\end{itemize}
Many of us have assumed, a priori, that electronic books substitute
for print books. While I don't have controlled, quantitative data
to refute the proposition, I do have plenty of experience with this
stuff, and all that experience leads me to believe that giving away
my books is selling the hell out of them.

More importantly, the free e-book skeptics have no evidence to
offer in support of their position -- just hand-waving and dark
muttering about a mythological future when book-lovers give up
their printed books for electronic book-readers (as opposed to the
much more plausible future where book lovers go on buying their
fetish objects and carry books around on their electronic
devices).

I started giving away e-books after I witnessed the early days of
the "bookwarez" scene, wherein fans cut the binding off their
favorite books, scanned them, ran them through optical character
recognition software, and manually proofread them to eliminate the
digitization errors. These fans were easily spending 80 hours to
rip their favorite books, and they were only ripping their favorite
books, books they loved and wanted to share. (The 80-hour figure
comes from my own attempt to do this -- I'm sure that rippers get
faster with practice.)

I thought to myself that 80 hours' free promotional effort would be
a good thing to have at my disposal when my books entered the
market. What if I gave my readers clean, canonical electronic
editions of my works, saving them the bother of ripping them, and
so freed them up to promote my work to their friends?

After all, it's not like there's any conceivable way to stop people
from putting books on scanners if they really want to. Scanners
aren't going to get more expensive or slower. The Internet isn't
going to get harder to use. Better to confront this challenge head
on, turn it into an opportunity, than to rail against the future
(I'm a science fiction writer -- tuning into the future is supposed
to be my metier).

The timing couldn't have been better. Just as my first novel was
being published, a new, high-tech project for promoting sharing of
creative works launched: the Creative Commons project (CC). CC
offers a set of tools that make it easy to mark works with whatever
freedoms the author wants to give away. CC launched in 2003 and
today, more than 160,000,000 works have been released under its
licenses.

My next column will go into more detail on what CC is, what
licenses it offers, and how to use them -- but for now, check them
out online at creativecommons.org.

\subsection{The Progressive Apocalypse and Other Futurismic Delights}

(Originally published in Locus Magazine, July 2007)

Of course, science fiction is a literature of the present. Many's
the science fiction writer who uses the future as a warped mirror
for reflecting back the present day, angled to illustrate the
hidden strangeness buried by our invisible assumptions: Orwell
turned 1948 into Nineteen Eighty-Four. But even when the fictional
future isn't a parable about the present day, it is necessarily a
creation of the present day, since it reflects the present day
biases that infuse the author. Hence Asimov's Foundation, a New
Deal-esque project to think humanity out of its tribulations though
social interventionism.

Bold SF writers eschew the future altogether, embracing a
futuristic account of the present day. William Gibson's forthcoming
Spook Country is an act of "speculative presentism," a book so
futuristic it could only have been set in 2006, a book that
exploits retrospective historical distance to let us glimpse just
how alien and futuristic our present day is.

Science fiction writers aren't the only people in the business of
predicting the future. Futurists -- consultants, technology
columnists, analysts, venture capitalists, and entrepreneurial
pitchmen -- spill a lot of ink, phosphors, and caffeinated hot air
in describing a vision for a future where we'll get more and more
of whatever it is they want to sell us or warn us away from.
Tomorrow will feature faster, cheaper processors, more Internet
users, ubiquitous RFID tags, radically democratic political
processes dominated by bloggers, massively multiplayer games whose
virtual economies dwarf the physical economy.

There's a lovely neologism to describe these visions: "futurismic."
Futurismic media is that which depicts futurism, not the future. It
is often self-serving -- think of the antigrav Nikes in Back to the
Future III -- and it generally doesn't hold up well to scrutiny.

SF films and TV are great fonts of futurismic imagery: R2D2 is a
fully conscious AI, can hack the firewall of the Death Star, and is
equipped with a range of holographic projectors and antipersonnel
devices -- but no one has installed a \$15 sound card and some
text-to-speech software on him, so he has to whistle like Harpo
Marx. Or take the Starship Enterprise, with a transporter capable
of constituting matter from digitally stored plans, and radios that
can breach the speed of light.

The non-futurismic version of NCC-1701 would be the size of a
softball (or whatever the minimum size for a warp drive,
transporter, and subspace radio would be). It would zip around the
galaxy at FTL speeds under remote control. When it reached an
interesting planet, it would beam a stored copy of a landing party
onto the surface, and when their mission was over, it would beam
them back into storage, annihilating their physical selves until
they reached the next stopping point. If a member of the landing
party were eaten by a green-skinned interspatial hippie or giant
toga-wearing galactic tyrant, that member would be recovered from
backup by the transporter beam. Hell, the entire landing party
could consist of multiple copies of the most effective crewmember
onboard: no redshirts, just a half-dozen instances of Kirk
operating in clonal harmony.

Futurism has a psychological explanation, as recounted in Harvard
clinical psych prof Daniel Gilbert's 2006 book, Stumbling on
Happiness. Our memories and our projections of the future are
necessarily imperfect. Our memories consist of those observations
our brains have bothered to keep records of, woven together with
inference and whatever else is lying around handy when we try to
remember something. Ask someone who's eating a great lunch how
breakfast was, and odds are she'll tell you it was delicious. Ask
the same question of someone eating rubbery airplane food, and
he'll tell you his breakfast was awful. We weave the past out of
our imperfect memories and our observable present.

We make the future in much the same way: we use reasoning and
evidence to predict what we can, and whenever we bump up against
uncertainty, we fill the void with the present day. Hence the
injunction on women soldiers in the future of Starship Troopers, or
the bizarre, glassed-over "Progressland" city diorama at the end of
the 1964 World's Fair exhibit The Carousel of Progress, which
Disney built for GE.

Lapsarianism -- the idea of a paradise lost, a fall from grace that
makes each year worse than the last -- is the predominant future
feeling for many people. It's easy to see why: an imperfectly
remembered golden childhood gives way to the worries of adulthood
and physical senescence. Surely the world is getting worse: nothing
tastes as good as it did when we were six, everything hurts all the
time, and our matured gonads drive us into frenzies of bizarre,
self-destructive behavior.

Lapsarianism dominates the Abrahamic faiths. I have an Orthodox
Jewish friend whose tradition holds that each generation of rabbis
is necessarily less perfect than the rabbis that came before, since
each generation is more removed from the perfection of the Garden.
Therefore, no rabbi is allowed to overturn any of his forebears'
wisdom, since they are all, by definition, smarter than him.

The natural endpoint of Lapsarianism is apocalypse. If things get
worse, and worse, and worse, eventually they'll just run out of
worseness. Eventually, they'll bottom out, a kind of rotten death
of the universe when Lapsarian entropy hits the nadir and takes us
all with it.

Running counter to Lapsarianism is progressivism: the Enlightenment
ideal of a world of great people standing on the shoulders of
giants. Each of us contributes to improving the world's storehouse
of knowledge (and thus its capacity for bringing joy to all of us),
and our descendants and proteges take our work and improve on it.
The very idea of "progress" runs counter to the idea of
Lapsarianism and the fall: it is the idea that we, as a species,
are falling in reverse, combing back the wild tangle of entropy
into a neat, tidy braid.

Of course, progress must also have a boundary condition -- if only
because we eventually run out of imaginary ways that the human
condition can improve. And science fiction has a name for the upper
bound of progress, a name for the progressive apocalypse:

We call it the Singularity.

Vernor Vinge's Singularity takes place when our technology reaches
a stage that allows us to "upload" our minds into software, run
them at faster, hotter speeds than our neurological wetware
substrate allows for, and create multiple, parallel instances of
ourselves. After the Singularity, nothing is predictable because
everything is possible. We will cease to be human and become (as
the title of Rudy Rucker's next novel would have it) Postsingular.

The Singularity is what happens when we have so much progress that
we run out of progress. It's the apocalypse that ends the human
race in rapture and joy. Indeed, Ken MacLeod calls the Singularity
"the rapture of the nerds," an apt description for the mirror-world
progressive version of the Lapsarian apocalypse.

At the end of the day, both progress and the fall from grace are
illusions. The central thesis of Stumbling on Happiness is that
human beings are remarkably bad at predicting what will make us
happy. Our predictions are skewed by our imperfect memories and our
capacity for filling the future with the present day.

The future is gnarlier than futurism. NCC-1701 probably wouldn't
send out transporter-equipped drones -- instead, it would likely
find itself on missions whose ethos, mores, and rationale are
largely incomprehensible to us, and so obvious to its crew that
they couldn't hope to explain them.

Science fiction is the literature of the present, and the present
is the only era that we can hope to understand, because it's the
only era that lets us check our observations and predictions
against reality.

\subsection{When the Singularity is More Than a Literary Device: An Interview with Futurist-Inventor Ray Kurzweil}

(Originally published in Asimov's Science Fiction Magazine, June
2005)

It's not clear to me whether the Singularity is a technical belief
system or a spiritual one.

The Singularity -- a notion that's crept into a lot of skiffy, and
whose most articulate in-genre spokesmodel is Vernor Vinge --
describes the black hole in history that will be created at the
moment when human intelligence can be digitized. When the speed and
scope of our cognition is hitched to the price-performance curve of
microprocessors, our "progress" will double every eighteen months,
and then every twelve months, and then every ten, and eventually,
every five seconds.

Singularities are, literally, holes in space from whence no
information can emerge, and so SF writers occasionally mutter about
how hard it is to tell a story set after the information
Singularity. Everything will be different. What it means to be
human will be so different that what it means to be in danger, or
happy, or sad, or any of the other elements that make up the
squeeze-and-release tension in a good yarn will be unrecognizable
to us pre-Singletons.

It's a neat conceit to write around. I've committed Singularity a
couple of times, usually in collaboration with gonzo Singleton
Charlie Stross, the mad antipope of the Singularity. But those
stories have the same relation to futurism as romance novels do to
love: a shared jumping-off point, but radically different
morphologies.

Of course, the Singularity isn't just a conceit for noodling with
in the pages of the pulps: it's the subject of serious-minded
punditry, futurism, and even science.

Ray Kurzweil is one such pundit-futurist-scientist. He's a serial
entrepreneur who founded successful businesses that advanced the
fields of optical character recognition (machine-reading) software,
text-to-speech synthesis, synthetic musical instrument simulation,
computer-based speech recognition, and stock-market analysis. He
cured his own Type-II diabetes through a careful review of the
literature and the judicious application of first principles and
reason. To a casual observer, Kurzweil appears to be the star of
some kind of Heinlein novel, stealing fire from the gods and
embarking on a quest to bring his maverick ideas to the public
despite the dismissals of the establishment, getting rich in the
process.

Kurzweil believes in the Singularity. In his 1990 manifesto, "The
Age of Intelligent Machines," Kurzweil persuasively argued that we
were on the brink of meaningful machine intelligence. A decade
later, he continued the argument in a book called The Age of
Spiritual Machines, whose most audacious claim is that the world's
computational capacity has been slowly doubling since the crust
first cooled (and before!), and that the doubling interval has been
growing shorter and shorter with each passing year, so that now we
see it reflected in the computer industry's Moore's Law, which
predicts that microprocessors will get twice as powerful for half
the cost about every eighteen months. The breathtaking sweep of
this trend has an obvious conclusion: computers more powerful than
people; more powerful than we can comprehend.

Now Kurzweil has published two more books, The Singularity Is Near,
When Humans Transcend Biology (Viking, Spring 2005) and Fantastic
Voyage: Live Long Enough to Live Forever (with Terry Grossman,
Rodale, November 2004). The former is a technological roadmap for
creating the conditions necessary for ascent into Singularity; the
latter is a book about life-prolonging technologies that will
assist baby-boomers in living long enough to see the day when
technological immortality is achieved.

See what I meant about his being a Heinlein hero?

I still don't know if the Singularity is a spiritual or a
technological belief system. It has all the trappings of
spirituality, to be sure. If you are pure and kosher, if you live
right and if your society is just, then you will live to see a
moment of Rapture when your flesh will slough away leaving nothing
behind but your ka, your soul, your consciousness, to ascend to an
immortal and pure state.

I wrote a novel called Down and Out in the Magic Kingdom where
characters could make backups of themselves and recover from them
if something bad happened, like catching a cold or being
assassinated. It raises a lot of existential questions: most
prominently: are you still you when you've been restored from
backup?

The traditional AI answer is the Turing Test, invented by Alan
Turing, the gay pioneer of cryptography and artificial intelligence
who was forced by the British government to take hormone treatments
to "cure" him of his homosexuality, culminating in his suicide in
1954. Turing cut through the existentialism about measuring whether
a machine is intelligent by proposing a parlor game: a computer
sits behind a locked door with a chat program, and a person sits
behind another locked door with his own chat program, and they both
try to convince a judge that they are real people. If the computer
fools a human judge into thinking that it's a person, then to all
intents and purposes, it's a person.

So how do you know if the backed-up you that you've restored into a
new body -- or a jar with a speaker attached to it -- is really
you? Well, you can ask it some questions, and if it answers the
same way that you do, you're talking to a faithful copy of
yourself.

Sounds good. But the me who sent his first story into Asimov's
seventeen years ago couldn't answer the question, "Write a story
for Asimov's" the same way the me of today could. Does that mean
I'm not me anymore?

Kurzweil has the answer.

"If you follow that logic, then if you were to take me ten years
ago, I could not pass for myself in a Ray Kurzweil Turing Test. But
once the requisite uploading technology becomes available a few
decades hence, you could make a perfect-enough copy of me, and it
would pass the Ray Kurzweil Turing Test. The copy doesn't have to
match the quantum state of my every neuron, either: if you meet me
the next day, I'd pass the Ray Kurzweil Turing Test. Nevertheless,
none of the quantum states in my brain would be the same. There are
quite a few changes that each of us undergo from day to day, we
don't examine the assumption that we are the same person closely.

"We gradually change our pattern of atoms and neurons but we very
rapidly change the particles the pattern is made up of. We used to
think that in the brain -- the physical part of us most closely
associated with our identity -- cells change very slowly, but it
turns out that the components of the neurons, the tubules and so
forth, turn over in only days. I'm a completely different set of
particles from what I was a week ago.

"Consciousness is a difficult subject, and I'm always surprised by
how many people talk about consciousness routinely as if it could
be easily and readily tested scientifically. But we can't postulate
a consciousness detector that does not have some assumptions about
consciousness built into it.

"Science is about objective third party observations and logical
deductions from them. Consciousness is about first-person,
subjective experience, and there's a fundamental gap there. We live
in a world of assumptions about consciousness. We share the
assumption that other human beings are conscious, for example. But
that breaks down when we go outside of humans, when we consider,
for example, animals. Some say only humans are conscious and
animals are instinctive and machinelike. Others see humanlike
behavior in an animal and consider the animal conscious, but even
these observers don't generally attribute consciousness to animals
that aren't humanlike.

"When machines are complex enough to have responses recognizable as
emotions, those machines will be more humanlike than animals."

The Kurzweil Singularity goes like this: computers get better and
smaller. Our ability to measure the world gains precision and grows
ever cheaper. Eventually, we can measure the world inside the brain
and make a copy of it in a computer that's as fast and complex as a
brain, and voila, intelligence.

Here in the twenty-first century we like to view ourselves as
ambulatory brains, plugged into meat-puppets that lug our precious
grey matter from place to place. We tend to think of that grey
matter as transcendently complex, and we think of it as being the
bit that makes us us.

But brains aren't that complex, Kurzweil says. Already, we're
starting to unravel their mysteries.

"We seem to have found one area of the brain closely associated
with higher-level emotions, the spindle cells, deeply embedded in
the brain. There are tens of thousands of them, spanning the whole
brain (maybe eighty thousand in total), which is an incredibly
small number. Babies don't have any, most animals don't have any,
and they likely only evolved over the last million years or so.
Some of the high-level emotions that are deeply human come from
these.

"Turing had the right insight: base the test for intelligence on
written language. Turing Tests really work. A novel is based on
language: with language you can conjure up any reality, much more
so than with images. Turing almost lived to see computers doing a
good job of performing in fields like math, medical diagnosis and
so on, but those tasks were easier for a machine than demonstrating
even a child's mastery of language. Language is the true embodiment
of human intelligence."

If we're not so complex, then it's only a matter of time until
computers are more complex than us. When that comes, our brains
will be model-able in a computer and that's when the fun begins.
That's the thesis of Spiritual Machines, which even includes a
(Heinlein-style) timeline leading up to this day.

Now, it may be that a human brain contains n logic-gates and runs
at x cycles per second and stores z petabytes, and that n and x and
z are all within reach. It may be that we can take a brain apart
and record the position and relationships of all the neurons and
sub-neuronal elements that constitute a brain.

But there are also a nearly infinite number of ways of modeling a
brain in a computer, and only a finite (or possibly nonexistent)
fraction of that space will yield a conscious copy of the original
meat-brain. Science fiction writers usually hand-wave this step: in
Heinlein's "Man Who Sold the Moon," the gimmick is that once the
computer becomes complex enough, with enough "random numbers," it
just wakes up.

Computer programmers are a little more skeptical. Computers have
never been known for their skill at programming themselves -- they
tend to be no smarter than the people who write their software.

But there are techniques for getting computers to program
themselves, based on evolution and natural selection. A programmer
creates a system that spits out lots -- thousands or even millions
-- of randomly generated programs. Each one is given the
opportunity to perform a computational task (say, sorting a list of
numbers from greatest to least) and the ones that solve the problem
best are kept aside while the others are erased. Now the survivors
are used as the basis for a new generation of randomly mutated
descendants, each based on elements of the code that preceded them.
By running many instances of a randomly varied program at once, and
by culling the least successful and regenerating the population
from the winners very quickly, it is possible to evolve effective
software that performs as well or better than the code written by
human authors.

Indeed, evolutionary computing is a promising and exciting field
that's realizing real returns through cool offshoots like "ant
colony optimization" and similar approaches that are showing good
results in fields as diverse as piloting military UAVs and
efficiently provisioning car-painting robots at automotive plants.

So if you buy Kurzweil's premise that computation is getting
cheaper and more plentiful than ever, then why not just use
evolutionary algorithms to evolve the best way to model a
scanned-in human brain such that it "wakes up" like Heinlein's Mike
computer?

Indeed, this is the crux of Kurzweil's argument in Spiritual
Machines: if we have computation to spare and a detailed model of a
human brain, we need only combine them and out will pop the
mechanism whereby we may upload our consciousness to digital
storage media and transcend our weak and bothersome meat
forever.Indeed, this is the crux of Kurzweil's argument in
Spiritual Machines: if we have computation to spare and a detailed
model of a human brain, we need only combine them and out will pop
the mechanism whereby we may upload our consciousness to digital
storage media and transcend our weak and bothersome meat forever.

But it's a cheat. Evolutionary algorithms depend on the same
mechanisms as real-world evolution: heritable variation of
candidates and a system that culls the least-suitable candidates.
This latter -- the fitness-factor that determines which individuals
in a cohort breed and which vanish -- is the key to a successful
evolutionary system. Without it, there's no pressure for the system
to achieve the desired goal: merely mutation and more mutation.

But how can a machine evaluate which of a trillion models of a
human brain is "most like" a conscious mind? Or better still: which
one is most like the individual whose brain is being modeled?

"It is a sleight of hand in Spiritual Machines," Kurzweil admits.
"But in The Singularity Is Near, I have an in-depth discussion
about what we know about the brain and how to model it. Our tools
for understanding the brain are subject to the Law of Accelerating
Returns, and we've made more progress in reverse-engineering the
human brain than most people realize." This is a tasty Kurzweilism
that observes that improvements in technology yield tools for
improving technology, round and round, so that the thing that
progress begets more than anything is more and yet faster
progress.

"Scanning resolution of human tissue -- both spatial and temporal
-- is doubling every year, and so is our knowledge of the workings
of the brain. The brain is not one big neural net, the brain is
several hundred different regions, and we can understand each
region, we can model the regions with mathematics, most of which
have some nexus with chaos and self-organizing systems. This has
already been done for a couple dozen regions out of the several
hundred.

"We have a good model of a dozen or so regions of the auditory and
visual cortex, how we strip images down to very low-resolution
movies based on pattern recognition. Interestingly, we don't
actually see things, we essentially hallucinate them in detail from
what we see from these low resolution cues. Past the early phases
of the visual cortex, detail doesn't reach the brain.

"We are getting exponentially more knowledge. We can get detailed
scans of neurons working in vivo, and are beginning to understand
the chaotic algorithms underlying human intelligence. In some
cases, we are getting comparable performance of brain regions in
simulation. These tools will continue to grow in detail and
sophistication.

"We can have confidence of reverse-engineering the brain in twenty
years or so. The reason that brain reverse engineering has not
contributed much to artificial intelligence is that up until
recently we didn't have the right tools. If I gave you a computer
and a few magnetic sensors and asked you to reverse-engineer it,
you might figure out that there's a magnetic device spinning when a
file is saved, but you'd never get at the instruction set. Once you
reverse-engineer the computer fully, however, you can express its
principles of operation in just a few dozen pages.

"Now there are new tools that let us see the interneuronal
connections and their signaling, in vivo, and in real-time. We're
just now getting these tools and there's very rapid application of
the tools to obtain the data.

"Twenty years from now we will have realistic simulations and
models of all the regions of the brain and [we will] understand how
they work. We won't blindly or mindlessly copy those methods, we
will understand them and use them to improve our AI toolkit. So
we'll learn how the brain works and then apply the sophisticated
tools that we will obtain, as we discover how the brain works.

"Once we understand a subtle science principle, we can isolate,
amplify, and expand it. Air goes faster over a curved surface: from
that insight we isolated, amplified, and expanded the idea and
invented air travel. We'll do the same with intelligence.

"Progress is exponential -- not just a measure of power of
computation, number of Internet nodes, and magnetic spots on a hard
disk -- the rate of paradigm shift is itself accelerating, doubling
every decade. Scientists look at a problem and they intuitively
conclude that since we've solved 1 percent over the last year,
it'll therefore be one hundred years until the problem is
exhausted: but the rate of progress doubles every decade, and the
power of the information tools (in price-performance, resolution,
bandwidth, and so on) doubles every year. People, even scientists,
don't grasp exponential growth. During the first decade of the
human genome project, we only solved 2 percent of the problem, but
we solved the remaining 98 percent in five years."

But Kurzweil doesn't think that the future will arrive in a rush.
As William Gibson observed, "The future is here, it's just not
evenly distributed."

"Sure, it'd be interesting to take a human brain, scan it,
reinstantiate the brain, and run it on another substrate. That will
ultimately happen."

"But the most salient scenario is that we'll gradually merge with
our technology. We'll use nanobots to kill pathogens, then to kill
cancer cells, and then they'll go into our brain and do benign
things there like augment our memory, and very gradually they'll
get more and more sophisticated. There's no single great leap, but
there is ultimately a great leap comprised of many small steps.

"In The Singularity Is Near, I describe the radically different
world of 2040, and how we'll get there one benign change at a time.
The Singularity will be gradual, smooth.

"Really, this is about augmenting our biological thinking with
nonbiological thinking. We have a capacity of 1026 to 1029
calculations per second (cps) in the approximately 1010 biological
human brains on Earth and that number won't change much in fifty
years, but nonbiological thinking will just crash through that. By
2049, nonbiological thinking capacity will be on the order of a
billion times that. We'll get to the point where bio thinking is
relatively insignificant.

"People didn't throw their typewriters away when word-processing
started. There's always an overlap -- it'll take time before we
realize how much more powerful nonbiological thinking will
ultimately be."

It's well and good to talk about all the stuff we can do with
technology, but it's a lot more important to talk about the stuff
we'll be allowed to do with technology. Think of the global
freak-out caused by the relatively trivial advent of peer-to-peer
file-sharing tools: Universities are wiretapping their campuses and
disciplining computer science students for writing legitimate,
general purpose software; grandmothers and twelve-year-olds are
losing their life savings; privacy and due process have sailed out
the window without so much as a by-your-leave.

Even P2P's worst enemies admit that this is a general-purpose
technology with good and bad uses, but when new tech comes along it
often engenders a response that countenances punishing an infinite
number of innocent people to get at the guilty.

What's going to happen when the new technology paradigm isn't
song-swapping, but transcendent super-intelligence? Will the
reactionary forces be justified in razing the whole ecosystem to
eliminate a few parasites who are doing negative things with the
new tools?

"Complex ecosystems will always have parasites. Malware [malicious
software] is the most important battlefield today.

"Everything will become software -- objects will be malleable,
we'll spend lots of time in VR, and computhought will be orders of
magnitude more important than biothought.

"Software is already complex enough that we have an ecological
terrain that has emerged just as it did in the bioworld.

"That's partly because technology is unregulated and people have
access to the tools to create malware and the medicine to treat it.
Today's software viruses are clever and stealthy and not
simpleminded. Very clever.

"But here's the thing: you don't see people advocating shutting
down the Internet because malware is so destructive. I mean,
malware is potentially more than a nuisance -- emergency systems,
air traffic control, and nuclear reactors all run on vulnerable
software. It's an important issue, but the potential damage is
still a tiny fraction of the benefit we get from the Internet.

"I hope it'll remain that way -- that the Internet won't become a
regulated space like medicine. Malware's not the most important
issue facing human society today. Designer bioviruses are. People
are concerted about WMDs, but the most daunting WMD would be a
designed biological virus. The means exist in college labs to
create destructive viruses that erupt and spread silently with long
incubation periods.

"Importantly, a would-be bio-terrorist doesn't have to put malware
through the FDA's regulatory approval process, but scientists
working to fix bio-malware do.

"In Huxley's Brave New World, the rationale for the totalitarian
system was that technology was too dangerous and needed to be
controlled. But that just pushes technology underground where it
becomes less stable. Regulation gives the edge of power to the
irresponsible who won't listen to the regulators anyway.

"The way to put more stones on the defense side of the scale is to
put more resources into defensive technologies, not create a
totalitarian regime of Draconian control.

"I advocate a one hundred billion dollar program to accelerate the
development of anti-biological virus technology. The way to combat
this is to develop broad tools to destroy viruses. We have tools
like RNA interference, just discovered in the past two years to
block gene expression. We could develop means to sequence the genes
of a new virus (SARS only took thirty-one days) and respond to it
in a matter of days.

"Think about it. There's no FDA for software, no certification for
programmers. The government is thinking about it, though! The
reason the FCC is contemplating Trusted Computing mandates," -- a
system to restrict what a computer can do by means of hardware
locks embedded on the motherboard -- "is that computing technology
is broadening to cover everything. So now you have communications
bureaucrats, biology bureaucrats, all wanting to regulate
computers.

"Biology would be a lot more stable if we moved away from
regulation -- which is extremely irrational and onerous and doesn't
appropriately balance risks. Many medications are not available
today even though they should be. The FDA always wants to know what
happens if we approve this and will it turn into a thalidomide
situation that embarrasses us on CNN?

"Nobody asks about the harm that will certainly accrue from
delaying a treatment for one or more years. There's no political
weight at all, people have been dying from diseases like heart
disease and cancer for as long as we've been alive. Attributable
risks get 100-1000 times more weight than unattributable risks."

Is this spirituality or science? Perhaps it is the melding of both
-- more shades of Heinlein, this time the weird religions founded
by people who took Stranger in a Strange Land way too seriously.

After all, this is a system of belief that dictates a means by
which we can care for our bodies virtuously and live long enough to
transcend them. It is a system of belief that concerns itself with
the meddling of non-believers, who work to undermine its goals
through irrational systems predicated on their disbelief. It is a
system of belief that asks and answers the question of what it
means to be human.

It's no wonder that the Singularity has come to occupy so much of
the science fiction narrative in these years. Science or
spirituality, you could hardly ask for a subject better tailored to
technological speculation and drama.

\subsection{Wikipedia: a genuine Hitchhikers' Guide to the Galaxy -- minus the editors}

(Originally published in The Anthology at the End of the Universe,
April 2005)

"Mostly Harmless" -- a phrase so funny that Adams actually titled a
book after it. Not that there's a lot of comedy inherent in those
two words: rather, they're the punchline to a joke that anyone
who's ever written for publication can really get behind.

Ford Prefect, a researcher for the Hitchhiker's Guide to the
Galaxy, has been stationed on Earth for years, painstakingly
compiling an authoritative, insightful entry on Terran geography,
science and culture, excerpts from which appear throughout the H2G2
books. His entry improved upon the old one, which noted that Earth
was, simply, "Harmless."

However, the Guide has limited space, and when Ford submits his
entry to his editors, it is trimmed to fit:

"What? Harmless? Is that all it's got to say? Harmless! One word!"

Ford shrugged. "Well, there are a hundred billion stars in the
Galaxy, and only a limited amount of space in the book's
microprocessors," he said, "and no one knew much about the Earth of
course."

"Well for God's sake I hope you managed to rectify that a bit."

"Oh yes, well I managed to transmit a new entry off to the editor.
He had to trim it a bit, but it's still an improvement."

"And what does it say now?" asked Arthur.

"Mostly harmless," admitted Ford with a slightly embarrassed
cough.

[fn: My lifestyle is as gypsy and fancy-free as the characters in
H2G2, and as a result my copies of the Adams books are thousands of
miles away in storages in other countries, and this essay was
penned on public transit and cheap hotel rooms in Chile, Boston,
London, Geneva, Brussels, Bergen, Geneva (again), Toronto,
Edinburgh, and Helsinki. Luckily, I was able to download a dodgy,
re-keyed version of the Adams books from a peer-to-peer network,
which network I accessed via an open wireless network on a random
street-corner in an anonymous city, a fact that I note here as
testimony to the power of the Internet to do what the Guide does
for Ford and Arthur: put all the information I need at my
fingertips, wherever I am. However, these texts \textbf{are} a
little on the dodgy side, as noted, so you might want to confirm
these quotes before, say, uttering them before an Adams truefan.]

And there's the humor: every writer knows the pain of laboring over
a piece for days, infusing it with diverse interesting factoids and
insights, only to have it cut to ribbons by some distant editor (I
once wrote thirty drafts of a 5,000-word article for an editor who
ended up running it in three paragraphs as accompaniment for what
he decided should be a photo essay with minimal verbiage.)

Since the dawn of the Internet, H2G2 geeks have taken it upon
themselves to attempt to make a Guide on the Internet. Volunteers
wrote and submitted essays on various subjects as would be likely
to appear in a good encyclopedia, infusing them with equal measures
of humor and thoughtfulness, and they were edited together by the
collective effort of the contributors. These projects --
Everything2, H2G2 (which was overseen by Adams himself), and others
-- are like a barn-raising in which a team of dedicated volunteers
organize the labors of casual contributors, piecing together a free
and open user-generated encyclopedia.

These encyclopedias have one up on Adams's Guide: they have no
shortage of space on their "microprocessors" (the first volume of
the Guide was clearly written before Adams became conversant with
PCs!). The ability of humans to generate verbiage is far
outstripped by the ability of technologists to generate low-cost,
reliable storage to contain it. For example, Brewster Kahle's
Internet Archive project (archive.org) has been making a copy of
the Web -- the \textbf{whole} Web, give or take -- every couple of
days since 1996. Using the Archive's Wayback Machine, you can now
go and see what any page looked like on a given day.

The Archive doesn't even bother throwing away copies of pages that
haven't changed since the last time they were scraped: with storage
as cheap as it is -- and it is \textbf{very} cheap for the Archive,
which runs the largest database in the history of the universe off
of a collection of white-box commodity PCs stacked up on packing
skids in the basement of a disused armory in San Francisco's
Presidio -- there's no reason not to just keep them around. In
fact, the Archive has just spawned two "mirror" Archives, one
located under the rebuilt Library of Alexandria and the other in
Amsterdam. [fn: Brewster Kahle says that he was nervous about
keeping his only copy of the "repository of all human knowledge" on
the San Andreas fault, but keeping your backups in a
censorship-happy Amnesty International watchlist state and/or in a
floodplain below sea level is probably not such a good idea
either!]

So these systems did not see articles trimmed for lack of space;
for on the Internet, the idea of "running out of space" is
meaningless. But they \textbf{were} trimmed, by editorial cliques,
and rewritten for clarity and style. Some entries were rejected as
being too thin, while others were sent back to the author for
extensive rewrites.

This traditional separation of editor and writer mirrors the
creative process itself, in which authors are exhorted to
concentrate on \textbf{either} composing \textbf{or} revising, but
not both at the same time, for the application of the critical mind
to the creative process strangles it. So you write, and then you
edit. Even when you write for your own consumption, it seems you
have to answer to an editor.

The early experimental days of the Internet saw much
experimentation with alternatives to traditional editor/author
divisions. Slashdot, a nerdy news-site of surpassing popularity
[fn: Having a link to one's website posted to Slashdot will almost
inevitably overwhelm your server with traffic, knocking all but the
best-provisioned hosts offline within minutes; this is commonly
referred to as "the Slashdot Effect."], has a baroque system for
"community moderation" of the responses to the articles that are
posted to its front pages. Readers, chosen at random, are given
five "moderator points" that they can use to raise or lower the
score of posts on the Slashdot message boards. Subsequent readers
can filter their views of these boards to show only highly ranked
posts. Other readers are randomly presented with posts and their
rankings and are asked to rate the fairness of each moderator's
moderation. Moderators who moderate fairly are given more
opportunities to moderate; likewise message-board posters whose
messages are consistently highly rated.

It is thought that this system rewards good "citizenship" on the
Slashdot boards through checks and balances that reward good
messages and fair editorial practices. And in the main, the
Slashdot moderation system works [fn: as do variants on it, like
the system in place at Kur5hin.org (pronounced "corrosion")]. If
you dial your filter up to show you highly scored messages, you
will generally get well-reasoned, or funny, or genuinely useful
posts in your browser.

This community moderation scheme and ones like it have been
heralded as a good alternative to traditional editorship. The
importance of the Internet to "edit itself" is best understood in
relation to the old shibboleth, "On the Internet, everyone is a
slushreader." [fn: "Slush" is the term for generally execrable
unsolicited manuscripts that fetch up in publishers' offices --
these are typically so bad that the most junior people on staff are
drafted into reading (and, usually, rejecting) them]. When the
Internet's radical transformative properties were first bandied
about in publishing circles, many reassured themselves that even if
printing's importance was de-emphasized, that good editors would
always been needed, and doubly so online, where any mouth-breather
with a modem could publish his words. Someone would need to
separate the wheat from the chaff and help keep us from drowning in
information.

One of the best-capitalized businesses in the history of the world,
Yahoo!, went public on the strength of this notion, proposing to
use an army of researchers to catalog every single page on the Web
even as it was created, serving as a comprehensive guide to all
human knowledge. Less than a decade later, Yahoo! is all but out of
that business: the ability of the human race to generate new pages
far outstrips Yahoo!'s ability to read, review, rank and categorize
them.

Hence Slashdot, a system of distributed slushreading. Rather than
professionalizing the editorship role, Slashdot invites
contributors to identify good stuff when they see it, turning
editorship into a reward for good behavior.

But as well as Slashdot works, it has this signal failing: nearly
every conversation that takes place on Slashdot is shot through
with discussion, griping and gaming
\textbf{on the moderation system itself}. The core task of Slashdot
has \textbf{become} editorship, not the putative subjects of
Slashdot posts. The fact that the central task of Slashdot is to
rate other Slashdotters creates a tenor of meanness in the
discussion. Imagine if the subtext of every discussion you had in
the real world was a kind of running, pedantic nitpickery in which
every point was explicitly weighed and judged and commented upon.
You'd be an unpleasant, unlikable jerk, the kind of person that is
sometimes referred to as a "slashdork."

As radical as Yahoo!'s conceit was, Slashdot's was more radical.
But as radical as Slashdot's is, it is still inherently
conservative in that it presumes that editorship is necessary, and
that it further requires human judgment and intervention.

Google's a lot more radical. Instead of editors, it has an
algorithm. Not the kind of algorithm that dominated the early
search engines like Altavista, in which laughably bad artificial
intelligence engines attempted to automatically understand the
content, context and value of every page on the Web so that a
search for "Dog" would turn up the page more relevant to the
query.

Google's algorithm is predicated on the idea that people are good
at understanding things and computers are good at counting things.
Google counts up all the links on the Web and affords more
authority to those pages that have been linked to by the most other
pages. The rationale is that if a page has been linked to by many
web-authors, then they must have seen some merit in that page. This
system works remarkably well -- so well that it's nearly
inconceivable that any search-engine would order its rankings by
any other means. What's more, it doesn't pervert the tenor of the
discussions and pages that it catalogs by turning each one into a
performance for a group of ranking peers. [fn: Or at least, it
\textbf{didn't}. Today, dedicated web-writers, such as bloggers,
are keenly aware of the way that Google will interpret their
choices about linking and page-structure. One popular sport is
"googlebombing," in which web-writers collude to link to a given
page using a humorous keyword so that the page becomes the top
result for that word -- which is why, for a time, the top result
for "more evil than Satan" was Microsoft.com. Likewise, the
practice of "blogspamming," in which unscrupulous spammers post
links to their webpages in the message boards on various blogs, so
that Google will be tricked into thinking that a wide variety of
sites have conferred some authority onto their penis-enlargement
page.]

But even Google is conservative in assuming that there is a need
for editorship as distinct from composition. Is there a way we can
dispense with editorship altogether and just use composition to
refine our ideas? Can we merge composition and editorship into a
single role, fusing our creative and critical selves?

You betcha.

"Wikis" [fn: Hawai'ian for "fast"] are websites that can be edited
by anyone. They were invented by Ward Cunningham in 1995, and they
have become one of the dominant tools for Internet collaboration in
the present day. Indeed, there is a sort of Internet geek who
throws up a Wiki in the same way that ants make anthills:
reflexively, unconsciously.

Here's how a Wiki works. You put up a page:

Welcome to my Wiki. It is rad.

There are OtherWikis that inspired me.

Click "publish" and bam, the page is live. The word "OtherWikis"
will be underlined, having automatically been turned into a link to
a blank page titled "OtherWikis" (Wiki software recognizes words
with capital letters in the middle of them as links to other pages.
Wiki people call this "camel-case," because the capital letters in
the middle of words make them look like humped camels.) At the
bottom of it appears this legend: "Edit this page."

Click on "Edit this page" and the text appears in an editable
field. Revise the text to your heart's content and click "Publish"
and your revisions are live. Anyone who visits a Wiki can edit any
of its pages, adding to it, improving on it, adding camel-cased
links to new subjects, or even defacing or deleting it.

It is authorship without editorship. Or authorship fused with
editorship. Whichever, it works, though it requires effort. The
Internet, like all human places and things, is fraught with
spoilers and vandals who deface whatever they can. Wiki pages are
routinely replaced with obscenities, with links to spammers'
websites, with junk and crap and flames.

But Wikis have self-defense mechanisms, too. Anyone can "subscribe"
to a Wiki page, and be notified when it is updated. Those who
create Wiki pages generally opt to act as "gardeners" for them,
ensuring that they are on hand to undo the work of the spoilers.

In this labor, they are aided by another useful Wiki feature: the
"history" link. Every change to every Wiki page is logged and
recorded. Anyone can page back through every revision, and anyone
can revert the current version to a previous one. That means that
vandalism only lasts as long as it takes for a gardener to come by
and, with one or two clicks, set things to right.

This is a powerful and wildly successful model for collaboration,
and there is no better example of this than the Wikipedia, a free,
Wiki-based encyclopedia with more than one million entries, which
has been translated into 198 languages [fn: That is, one or more
Wikipedia entries have been translated into 198 languages; more
than 15 languages have 10,000 or more entries translated]

Wikipedia is built entirely out of Wiki pages created by
self-appointed experts. Contributors research and write up
subjects, or produce articles on subjects that they are familiar
with.

This is authorship, but what of editorship? For if there is one
thing a Guide or an encyclopedia must have, it is authority. It
must be vetted by trustworthy, neutral parties, who present
something that is either The Truth or simply A Truth, but truth
nevertheless.

The Wikipedia has its skeptics. Al Fasoldt, a writer for the
Syracuse Post-Standard, apologized to his readers for having
recommended that they consult Wikipedia. A reader of his, a
librarian, wrote in and told him that his recommendation had been
irresponsible, for Wikipedia articles are often defaced or worse
still, rewritten with incorrect information. When another
journalist from the Techdirt website wrote to Fasoldt to correct
this impression, Fasoldt responded with an increasingly patronizing
and hysterical series of messages in which he described Wikipedia
as "outrageous," "repugnant" and "dangerous," insulting the
Techdirt writer and storming off in a huff. [fn: see
\href{http://techdirt.com/articles/20040827/0132238_F.shtml}{http://techdirt.com/articles/20040827/0132238\_F.shtml}
for more]

Spurred on by this exchange, many of Wikipedia's supporters decided
to empirically investigate the accuracy and resilience of the
system. Alex Halavais made changes to 13 different pages, ranging
from obvious to subtle. Every single change was found and corrected
within hours. [fn: see
\href{http://alex.halavais.net/news/index.php?p=794}{http://alex.halavais.net/news/index.php?p=794}
for more] Then legendary Princeton engineer Ed Felten ran
side-by-side comparisons of Wikipedia entries on areas in which he
had deep expertise with their counterparts in the current
electronic edition of the Encyclopedia Britannica. His conclusion?
"Wikipedia's advantage is in having more, longer, and more current
entries. If it weren't for the Microsoft-case entry, Wikipedia
would have been the winner hands down. Britannica's advantage is in
having lower variance in the quality of its entries." [fn: see
\href{http://www.freedom-to-tinker.com/archives/000675.html}{http://www.freedom-to-tinker.com/archives/000675.html}
for more] Not a complete win for Wikipedia, but hardly
"outrageous," "repugnant" and "dangerous." (Poor Fasoldt -- his
idiotic hyperbole will surely haunt him through the whole of his
career -- I mean, "repugnant?!")

There has been one very damning and even frightening indictment of
Wikipedia, which came from Ethan Zuckerman, the founder of the
GeekCorps group, which sends volunteers to poor countries to help
establish Internet Service Providers and do other good works
through technology.

Zuckerman, a Harvard Berkman Center Fellow, is concerned with the
"systemic bias" in a collaborative encyclopedia whose contributors
must be conversant with technology and in possession of same in
order to improve on the work there. Zuckerman reasonably observes
that Internet users skew towards wealth, residence in the world's
richest countries, and a technological bent. This means that the
Wikipedia, too, is skewed to subjects of interest to that group --
subjects where that group already has expertise and interest.

The result is tragicomical. The entry on the Congo Civil War, the
largest military conflict the world has seen since WWII, which has
claimed over three million lives, has only a fraction of the
verbiage devoted to the War of the Ents, a fictional war fought
between sentient trees in JRR Tolkien's
\textbf{Lord of the Rings}.

Zuckerman issued a public call to arms to rectify this, challenging
Wikipedia contributors to seek out information on subjects like
Africa's military conflicts, nursing and agriculture and write
these subjects up in the same loving detail given over to science
fiction novels and contemporary youth culture. His call has been
answered well. What remains is to infiltrate the Wikipedia into the
academe so that term papers, Masters and Doctoral theses on these
subjects find themselves in whole or in part on the Wikipedia. [fn
See
\href{http://en.wikipedia.org/wiki/User:Xed/CROSSBOW}{http://en.wikipedia.org/wiki/User:Xed/CROSSBOW}
for more on this]

But if Wikipedia is authoritative, how does it get there? What
alchemy turns the maunderings of "mouth-breathers with modems" into
valid, useful encyclopedia entries?

It all comes down to the way that disputes are deliberated over and
resolved. Take the entry on Israel. At one point, it characterized
Israel as a beleaguered state set upon by terrorists who would
drive its citizens into the sea. Not long after, the entry was
deleted holus-bolus and replaced with one that described Israel as
an illegal state practicing Apartheid on an oppressed ethnic
minority.

Back and forth the editors went, each overwriting the other's with
his or her own doctrine. But eventually, one of them blinked. An
editor moderated the doctrine just a little, conceding a single
point to the other. And the other responded in kind. In this way,
turn by turn, all those with a strong opinion on the matter
negotiated a kind of Truth, a collection of statements that
everyone could agree represented as neutral a depiction of Israel
as was likely to emerge. Whereupon, the joint authors of this
marvelous document joined forces and fought back to back to resist
the revisions of other doctrinaires who came later, preserving
their hard-won peace. [fn: This process was just repeated in
microcosm in the Wikipedia entry on the author of this paper, which
was replaced by a rather disparaging and untrue entry that
characterized his books as critical and commercial failures --
there ensued several editorial volleys, culminating in an uneasy
peace that couches the anonymous detractor's skepticism in context
and qualifiers that make it clear what the facts are and what is
speculation]

What's most fascinating about these entries isn't their "final"
text as currently present on Wikipedia. It is the history page for
each, blow-by-blow revision lists that make it utterly transparent
where the bodies were buried on the way to arriving at whatever
Truth has emerged. This is a neat solution to the problem of
authority -- if you want to know what the fully rounded view of
opinions on any controversial subject look like, you need only
consult its entry's history page for a blistering eyeful of
thorough debate on the subject.

And here, finally, is the answer to the "Mostly harmless" problem.
Ford's editor can trim his verbiage to two words, but they need not
stay there -- Arthur, or any other user of the Guide as we know it
today [fn: that is, in the era where we understand enough about
technology to know the difference between a microprocessor and a
hard-drive] can revert to Ford's glorious and exhaustive version.

Think of it: a Guide without space restrictions and without
editors, where any Vogon can publish to his heart's content.

Lovely.

\subsection{Warhol is Turning in His Grave}

(Originally published in The Guardian, November 13, 2007)

The excellent little programmer book for the National Portrait
Gallery's current show POPARTPORTRAITS has a lot to say about the
pictures hung on the walls, about the diverse source material the
artists drew from in producing their provocative works. They cut up
magazines, copied comic books, drew in trademarked cartoon
characters like Minnie Mouse, reproduced covers from \textbf{Time}
magazine, made ironic use of the cartoon figure of Charles Atlas,
painted over an iconic photo of James Dean or Elvis Presley -- and
that's just in the first room of seven.

The programmer book describes the aesthetic experience of seeing
these repositioned icons of culture high and low, the art created
by the celebrated artists Poons, Rauschenberg, Warhol, et al by
nicking the work of others, without permission, and remaking it to
make statements and evoke emotions never countenanced by the
original creators.

However, the book does not say a word about copyright. Can you
blame it? A treatise on the way that copyright and trademark were
-- \textbf{had to be} -- trammeled to make these works could fill
volumes. Reading the programmer book, you have to assume that the
curators' only message about copyright is that where free
expression is concerned, the rights of the creators of the original
source material appropriated by the pop school take a back seat.

There is, however, another message about copyright in the National
Portrait Gallery: it's implicit in the "No Photography" signs
prominently placed throughout the halls, including one right by the
entrance of the POPARTPORTRAITS exhibition. This isn't intended to
protect the works from the depredations of camera-flashes (it would
read NO FLASH PHOTOGRAPHY if this were so). No, the ban on pictures
is in place to safeguard the copyright in the works hung on the
walls -- a fact that every gallery staffer I spoke to instantly
affirmed when I asked about the policy.

Indeed, it seems that every square centimeter of the Portrait
Gallery is under some form of copyright. I wasn't even allowed to
photograph the NO PHOTOGRAPHS sign. A museum staffer explained that
she'd been told that the typography and layout of the NO
PHOTOGRAPHS legend was, itself, copyrighted. If this is true, then
presumably, the same rules would prevent anyone from taking any
pictures in any public place -- unless you could somehow contrive
to get a shot of Leicester Square without any writing, logos,
architectural facades, or images in it. I doubt Warhol could have
done it.

What's the message of the show, then? Is it a celebration of remix
culture, reveling in the endless possibilities opened up by
appropriating and re-using without permission?

Or is it the epitaph on the tombstone of the sweet days before the
UN's chartering of the World Intellectual Property Organization and
the ensuing mania for turning everything that can be sensed and
recorded into someone's property?

Does this show -- paid for with public money, with some works that
are themselves owned by public institutions -- seek to inspire us
to become 21st century pops, armed with cameraphones, websites and
mixers, or is it supposed to inform us that our chance has passed,
and we'd best settle for a life as information serfs, who can't
even make free use of what our eyes see, our ears hear, of the
streets we walk upon?

Perhaps, just perhaps, it's actually a Dadaist show
\textbf{masquerading} as a pop art show! Perhaps the point is to
titillate us with the delicious irony of celebrating copyright
infringement while simultaneously taking the view that even the NO
PHOTOGRAPHY sign is a form of property, not to be reproduced
without the permission that can never be had.

\subsection{The Future of Ignoring Things}

(Originally published on InformationWeek's Internet Evolution,
October 3, 2007)

For decades, computers have been helping us to remember, but now
it's time for them to help us to ignore.

Take email: Endless engineer-hours are poured into stopping spam,
but virtually no attention is paid to our interaction with our
non-spam messages. Our mailer may strive to learn from our ratings
what is and is not spam, but it expends practically no effort on
figuring out which of the non-spam emails are important and which
ones can be safely ignored, dropped into archival folders, or
deleted unread.

For example, I'm forever getting cc'd on busy threads by
well-meaning colleagues who want to loop me in on some discussion
in which I have little interest. Maybe the initial group invitation
to a dinner (that I'll be out of town for) was something I needed
to see, but now that I've declined, I really don't need to read the
300+ messages that follow debating the best place to eat.

I could write a mail-rule to ignore the thread, of course. But
mail-rule editors are clunky, and once your rule-list grows very
long, it becomes increasingly unmanageable. Mail-rules are where
bookmarks were before the bookmark site del.icio.us showed up --
built for people who might want to ensure that messages from the
boss show up in red, but not intended to be used as a gigantic
storehouse of a million filters, a crude means for telling the
computers what we don't want to see.

Rael Dornfest, the former chairman of the O'Reilly Emerging Tech
conference and founder of the startup IWantSandy, once proposed an
"ignore thread" feature for mailers: Flag a thread as
uninteresting, and your mailer will start to hide messages with
that subject-line or thread-ID for a week, unless those messages
contain your name. The problem is that threads mutate. Last week's
dinner plans become this week's discussion of next year's group
holiday. If the thread is still going after a week, the messages
flow back into your inbox -- and a single click takes you back
through all the messages you missed.

We need a million measures like this, adaptive systems that create
a gray zone between "delete on sight" and "show this to me right
away."

RSS readers are a great way to keep up with the torrent of new
items posted on high-turnover sites like Digg, but they're even
better at keeping up with sites that are sporadic, like your
friend's brilliant journal that she only updates twice a year. But
RSS readers don't distinguish between the rare and miraculous
appearance of a new item in an occasional journal and the latest
click-fodder from Slashdot. They don't even sort your RSS feeds
according to the sites that you click-through the most.

There was a time when I could read the whole of Usenet -- not just
because I was a student looking for an excuse to avoid my
assignments, but because Usenet was once tractable, readable by a
single determined person. Today, I can't even keep up with a single
high-traffic message-board. I can't read all my email. I can't read
every item posted to every site I like. I certainly can't plough
through the entire edit-history of every Wikipedia entry I read.
I've come to grips with this -- with acquiring information on a
probabilistic basis, instead of the old, deterministic,
cover-to-cover approach I learned in the offline world.

It's as though there's a cognitive style built into TCP/IP. Just as
the network only does best-effort delivery of packets, not worrying
so much about the bits that fall on the floor, TCP/IP users also do
best-effort sweeps of the Internet, focusing on learning from the
good stuff they find, rather than lamenting the stuff they don't
have time to see.

The network won't ever become more tractable. There will never be
fewer things vying for our online attention. The only answer is
better ways and new technology to ignore stuff -- a field that's
just being born, with plenty of room to grow.

\subsection{Facebook's Faceplant}

(Originally published as "How Your Creepy Ex-Co-Workers Will Kill
Facebook," in InformationWeek, November 26, 2007)

Facebook's "platform" strategy has sparked much online debate and
controversy. No one wants to see a return to the miserable days of
walled gardens, when you couldn't send a message to an AOL
subscriber unless you, too, were a subscriber, and when the only
services that made it were the ones that AOL management approved.
Those of us on the "real" Internet regarded AOL with a species of
superstitious dread, a hive of clueless noobs waiting to swamp our
beloved Usenet with dumb flamewars (we fiercely guarded our erudite
flamewars as being of a palpably superior grade), the wellspring of
an

Facebook is no paragon of virtue. It bears the hallmarks of the
kind of pump-and-dump service that sees us as sticky, monetizable
eyeballs in need of pimping. The clue is in the steady stream of
emails you get from Facebook: "So-and-so has sent you a message."
Yeah, what is it? Facebook isn't telling -- you have to visit
Facebook to find out, generate a banner impression, and read and
write your messages using the halt-and-lame Facebook interface,
which lags even end-of-lifed email clients like Eudora for
composing, reading, filtering, archiving and searching. Emails from
Facebook aren't helpful messages, they're eyeball bait, intended to
send you off to the Facebook site, only to discover that Fred wrote
"Hi again!" on your "wall." Like other "social" apps (cough eVite
cough), Facebook has all the social graces of a nose-picking,
hyperactive six-year-old, standing at the threshold of your
attention and chanting, "I know something, I know something, I know
something, won't tell you what it is!"

If there was any doubt about Facebook's lack of qualification to
displace the Internet with a benevolent dictatorship/walled garden,
it was removed when Facebook unveiled its new advertising campaign.
Now, Facebook will allow its advertisers use the profile pictures
of Facebook users to advertise their products, without permission
or compensation. Even if you're the kind of person who likes the
sound of a "benevolent dictatorship," this clearly isn't one.

Many of my colleagues wonder if Facebook can be redeemed by opening
up the platform, letting anyone write any app for the service,
easily exporting and importing their data, and so on (this is the
kind of thing Google is doing with its OpenSocial Alliance).
Perhaps if Facebook takes on some of the characteristics that made
the Web work -- openness, decentralization, standardization -- it
will become like the Web itself, but with the added pixie dust of
"social," the indefinable characteristic that makes Facebook into
pure crack for a significant proportion of Internet users.

The debate about redeeming Facebook starts from the assumption that
Facebook is snowballing toward critical mass, the point at which it
begins to define "the Internet" for a large slice of the world's
netizens, growing steadily every day. But I think that this is far
from a sure thing. Sure, networks generally follow Metcalfe's Law:
"the value of a telecommunications network is proportional to the
square of the number of users of the system." This law is best
understood through the analogy of the fax machine: a world with one
fax machine has no use for faxes, but every time you add a fax, you
square the number of possible send/receive combinations (Alice can
fax Bob or Carol or Don; Bob can fax Alice, Carol and Don; Carol
can fax Alice, Bob and Don, etc).

But Metcalfe's law presumes that creating more communications
pathways increases the value of the system, and that's not always
true (see Brook's Law: "Adding manpower to a late softer project
makes it later").

Having watched the rise and fall of SixDegrees, Friendster, and the
many other proto-hominids that make up the evolutionary chain
leading to Facebook, MySpace, et al, I'm inclined to think that
these systems are subject to a Brook's-law parallel: "Adding more
users to a social network increases the probability that it will
put you in an awkward social circumstance." Perhaps we can call
this "boyd's Law" [NOTE TO EDITOR: "boyd" is always lower-case] for
danah [TO EDITOR: "danah" too!] boyd, the social scientist who has
studied many of these networks from the inside as a keen-eyed
net-anthropologist and who has described the many ways in which
social software does violence to sociability in a series of sharp
papers.

Here's one of boyd's examples, a true story: a young woman, an
elementary school teacher, joins Friendster after some of her
Burning Man buddies send her an invite. All is well until her
students sign up and notice that all the friends in her profile are
sunburnt, drug-addled techno-pagans whose own profiles are adorned
with digital photos of their painted genitals flapping over the
Playa. The teacher inveigles her friends to clean up their
profiles, and all is well again until her boss, the school
principal, signs up to the service and demands to be added to her
friends list. The fact that she doesn't like her boss doesn't
really matter: in the social world of Friendster and its progeny,
it's perfectly valid to demand to be "friended" in an explicit
fashion that most of us left behind in the fourth grade. Now that
her boss is on her friends list, our teacher-friend's buddies
naturally assume that she is one of the tribe and begin to send her
lascivious Friendster-grams, inviting her to all sorts of dirty
funtimes.

In the real world, we don't articulate our social networks. Imagine
how creepy it would be to wander into a co-worker's cubicle and
discover the wall covered with tiny photos of everyone in the
office, ranked by "friend" and "foe," with the top eight friends
elevated to a small shrine decorated with Post-It roses and hearts.
And yet, there's an undeniable attraction to corralling all your
friends and friendly acquaintances, charting them and their
relationship to you. Maybe it's evolutionary, some quirk of the
neocortex dating from our evolution into social animals who gained
advantage by dividing up the work of survival but acquired the
tricky job of watching all the other monkeys so as to be sure that
everyone was pulling their weight and not, e.g., napping in the
treetops instead of watching for predators, emerging only to eat
the fruit the rest of us have foraged.

Keeping track of our social relationships is a serious piece of
work that runs a heavy cognitive load. It's natural to seek out
some neural prosthesis for assistance in this chore. My fiancee
once proposed a "social scheduling" application that would watch
your phone and email and IM to figure out who your pals were and
give you a little alert if too much time passed without your
reaching out to say hello and keep the coals of your relationship
aglow. By the time you've reached your forties, chances are you're
out-of-touch with more friends than you're in-touch with, old
summer-camp chums, high-school mates, ex-spouses and their
families, former co-workers, college roomies, dot-com veterans...
Getting all those people back into your life is a full-time job and
then some.

You'd think that Facebook would be the perfect tool for handling
all this. It's not. For every long-lost chum who reaches out to me
on Facebook, there's a guy who beat me up on a weekly basis through
the whole seventh grade but now wants to be my buddy; or the crazy
person who was fun in college but is now kind of sad; or the creepy
ex-co-worker who I'd cross the street to avoid but who now wants to
know, "Am I your friend?" yes or no, this instant, please.

It's not just Facebook and it's not just me. Every "social
networking service" has had this problem and every user I've spoken
to has been frustrated by it. I think that's why these services are
so volatile: why we're so willing to flee from Friendster and into
MySpace's loving arms; from MySpace to Facebook. It's socially
awkward to refuse to add someone to your friends list -- but
\textbf{removing} someone from your friend-list is practically a
declaration of war. The least-awkward way to get back to a friends
list with nothing but friends on it is to reboot: create a new
identity on a new system and send out some invites (of course,
chances are at least one of those invites will go to someone who'll
groan and wonder why we're dumb enough to think that we're pals).

That's why I don't worry about Facebook taking over the net. As
more users flock to it, the chances that the person who
precipitates your exodus will find you increases. Once that
happens, poof, away you go -- and Facebook joins SixDegrees,
Friendster and their pals on the scrapheap of net.history.

\subsection{The Future of Internet Immune Systems}

(Originally published on InformationWeek's Internet Evolution,
November 19, 2007)

Bunhill Cemetery is just down the road from my flat in London. It’s
a handsome old boneyard, a former plague pit (“Bone hill” -- as in,
there are so many bones under there that the ground is actually
kind of humped up into a hill). There are plenty of luminaries
buried there -- John “Pilgrim’s Progress” Bunyan, William Blake,
Daniel Defoe, and assorted Cromwells. But my favorite tomb is that
of Thomas Bayes, the 18th-century statistician for whom Bayesian
filtering is named.

Bayesian filtering is plenty useful. Here’s a simple example of how
you might use a Bayesian filter. First, get a giant load of
non-spam emails and feed them into a Bayesian program that counts
how many times each word in their vocabulary appears, producing a
statistical breakdown of the word-frequency in good emails.

Then, point the filter at a giant load of spam (if you’re having a
hard time getting a hold of one, I have plenty to spare), and count
the words in it. Now, for each new message that arrives in your
inbox, have the filter count the relative word-frequencies and make
a statistical prediction about whether the new message is spam or
not (there are plenty of wrinkles in this formula, but this is the
general idea).

The beauty of this approach is that you needn’t dream up “The Big
Exhaustive List of Words and Phrases That Indicate a Message Is/Is
Not Spam.” The filter naively calculates a statistical fingerprint
for spam and not-spam, and checks the new messages against them.

This approach -- and similar ones -- are evolving into an immune
system for the Internet, and like all immune systems, a little bit
goes a long way, and too much makes you break out in hives.

ISPs are loading up their network centers with intrusion detection
systems and tripwires that are supposed to stop attacks before they
happen. For example, there’s the filter at the hotel I once stayed
at in Jacksonville, Fla. Five minutes after I logged in, the
network locked me out again. After an hour on the phone with tech
support, it transpired that the network had noticed that the
videogame I was playing systematically polled the other hosts on
the network to check if they were running servers that I could join
and play on. The network decided that this was a malicious
port-scan and that it had better kick me off before I did anything
naughty.

It only took five minutes for the software to lock me out, but it
took well over an hour to find someone in tech support who
understood what had happened and could reset the router so that I
could get back online.

And right there is an example of the autoimmune disorder. Our
network defenses are automated, instantaneous, and sweeping. But
our fallback and oversight systems are slow, understaffed, and
unresponsive. It takes a millionth of a second for the
Transportation Security Administration’s body-cavity-search
roulette wheel to decide that you’re a potential terrorist and
stick you on a no-fly list, but getting un-Tuttle-Buttled is a
nightmarish, months-long procedure that makes Orwell look like an
optimist.

The tripwire that locks you out was fired-and-forgotten two years
ago by an anonymous sysadmin with root access on the whole network.
The outsourced help-desk schlub who unlocks your account can’t even
spell "tripwire." The same goes for the algorithm that cut off your
credit card because you got on an airplane to a different part of
the world and then had the audacity to spend your money. (I’ve
resigned myself to spending \$50 on long-distance calls with
Citibank every time I cross a border if I want to use my debit card
while abroad.)

This problem exists in macro- and microcosm across the whole of our
technologically mediated society. The “spamigation bots” run by the
Business Software Alliance and the Music and Film Industry
Association of America (MAFIAA) entertainment groups send out tens
of thousands of automated copyright takedown notices to ISPs at a
cost of pennies, with little or no human oversight. The people who
get erroneously fingered as pirates (as a Recording Industry
Association of America (RIAA) spokesperson charmingly puts it,
“When you go fishing with a dragnet, sometimes you catch a
dolphin.”) spend days or weeks convincing their ISPs that they had
the right to post their videos, music, and text-files.

We need an immune system. There are plenty of bad guys out there,
and technology gives them force-multipliers (like the hackers who
run 250,000-PC botnets). Still, there’s a terrible asymmetry in a
world where defensive takedowns are automatic, but correcting
mistaken takedowns is done by hand.

\subsection{All Complex Ecosystems Have Parasites}

(Paper delivered at the O'Reilly Emerging Technology Conference,
San Diego, California, 16 March 2005)

AOL hates spam. AOL could eliminate nearly 100 percent of its
subscribers' spam with one easy change: it could simply shut off
its internet gateway. Then, as of yore, the only email an AOL
subscriber could receive would come from another AOL subscriber. If
an AOL subscriber sent a spam to another AOL subscriber and AOL
found out about it, they could terminate the spammer's account.
Spam costs AOL millions, and represents a substantial disincentive
for AOL customers to remain with the service, and yet AOL chooses
to permit virtually anyone who can connect to the Internet,
anywhere in the world, to send email to its customers, with any
software at all.

Email is a sloppy, complicated ecosystem. It has organisms of
sufficient diversity and sheer number as to beggar the imagination:
thousands of SMTP agents, millions of mail-servers, hundreds of
millions of users. That richness and diversity lets all kinds of
innovative stuff happen: if you go to nytimes.com and "send a story
to a friend," the NYT can convincingly spoof your return address on
the email it sends to your friend, so that it appears that the
email originated on your computer. Also: a spammer can harvest your
email and use it as a fake return address on the spam he sends to
your friend. Sysadmins have server processes that send them mail to
secret pager-addresses when something goes wrong, and GPLed
mailing-list software gets used by spammers and people running
high-volume mailing lists alike.

You could stop spam by simplifying email: centralize functions like
identity verification, limit the number of authorized mail agents
and refuse service to unauthorized agents, even set up tollbooths
where small sums of money are collected for every email, ensuring
that sending ten million messages was too expensive to contemplate
without a damned high expectation of return on investment. If you
did all these things, you'd solve spam.

By breaking email.

Small server processes that mail a logfile to five sysadmins every
hour just in case would be prohibitively expensive. Convincing the
soviet that your bulk-mailer was only useful to legit mailing lists
and not spammers could take months, and there's no guarantee that
it would get their stamp of approval at all. With verified
identity, the NYTimes couldn't impersonate you when it forwarded
stories on your behalf -- and Chinese dissidents couldn't send out
their samizdata via disposable gmail accounts.

An email system that can be controlled is an email system without
complexity. Complex ecosystems are influenced, not controlled.

The Hollywood studios are conniving to create a global network of
regulatory mandates over entertainment devices. Here they call it
the Broadcast Flag; in Europe, Asia, Australia and Latinamerica
it's called DVB Copy Protection Content Management. These systems
purport to solve the problem of indiscriminate redistribution of
broadcast programming via the Internet, but their answer to the
problem, such as it is, is to require that everyone who wants to
build a device that touches video has to first get permission.

If you want to make a TV, a screen, a video-card, a high-speed bus,
an analog-to-digital converter, a tuner card, a DVD burner -- any
tool that you hope to be lawful for use in connection with digital
TV signals -- you'll have to go on bended knee to get permission to
deploy it. You'll have to convince FCC bureaucrats or a panel of
Hollywood companies and their sellout IT and consumer electronics
toadies that the thing you're going to bring to market will not
disrupt their business models.

That's how DVD works today: if you want to make a DVD player, you
need to ask permission from a shadowy organization called the
DVD-CCA. They don't give permission if you plan on adding new
features -- that's why they're suing Kaleidascape for building a
DVD jukebox that can play back your movies from a hard-drive
archive instead of the original discs.

CD has a rich ecosystem, filled with parasites -- entrepreneurial
organisms that move to fill every available niche. If you spent a
thousand bucks on CDs ten years ago, the ecosystem for CDs would
reward you handsomely. In the intervening decade, parasites who
have found an opportunity to suck value out of the products on
offer from the labels and the dupe houses by offering you the tools
to convert your CDs to ring-tones, karaoke, MP3s, MP3s on iPods and
other players, MP3s on CDs that hold a thousand percent more music
-- and on and on.

DVDs live in a simpler, slower ecosystem, like a terrarium in a
bottle where a million species have been pared away to a manageable
handful. DVDs pay no such dividend. A thousand dollars' worth of
ten-year old DVDs are good for just what they were good for ten
years ago: watching. You can't put your kid into her favorite
cartoon, you can't downsample the video to something that plays on
your phone, and you certainly can't lawfully make a
hard-drive-based jukebox from your discs.

The yearning for simple ecosystems is endemic among people who want
to "fix" some problem of bad actors on the networks.

Take interoperability: you might sell me a database in the
expectation that I'll only communicate with it using your
authorized database agents. That way you can charge vendors a
license fee in exchange for permission to make a client, and you
can ensure that the clients are well-behaved and don't trigger any
of your nasty bugs.

But you can't meaningfully enforce that. EDS and other titanic
software companies earn their bread and butter by producing fake
database clients that impersonate the real thing as they iterate
through every record and write it to a text file -- or simply
provide a compatibility layer through systems provided by two
different vendors. These companies produce software that lies --
parasite software that fills niches left behind by other organisms,
sometimes to those organisms' detriment.

So we have "Trusted Computing," a system that's supposed to let
software detect other programs' lies and refuse to play with them
if they get caught out fibbing. It's a system that's based on
torching the rainforest with all its glorious anarchy of tools and
systems and replacing it with neat rows of tame and planted trees,
each one approved by The Man as safe for use with his products.

For Trusted Computing to accomplish this, everyone who makes a
video-card, keyboard, or logic-board must receive a key from some
certifying body that will see to it that the key is stored in a way
that prevents end-users from extracting it and using it to fake
signatures.

But if one keyboard vendor doesn't store his keys securely, the
system will be useless for fighting keyloggers. If one video-card
vendor lets a key leak, the system will be no good for stopping
screenlogging. If one logic-board vendor lets a key slip, the whole
thing goes out the window. That's how DVD DRM got hacked: one
vendor, Xing, left its keys in a place where users could get at
them, and then anyone could break the DRM on any DVD.

Not only is the Trusted Computing advocates' goal -- producing a
simpler software ecosystem -- wrongheaded, but the methodology is
doomed. Fly-by-night keyboard vendors in distant free trade zones
just won't be 100 percent compliant, and Trusted Computing requires
no less than perfect compliance.

The whole of DRM is a macrocosm for Trusted Computing. The DVB Copy
Protection system relies on a set of rules for translating every
one of its restriction states -- such as "copy once" and "copy
never" -- to states in other DRM systems that are licensed to
receive its output. That means that they're signing up to review,
approve and write special rules for every single entertainment
technology now invented and every technology that will be invented
in the future.

Madness: shrinking the ecosystem of everything you can plug into
your TV down to the subset that these self-appointed arbiters of
technology approve is a recipe for turning the electronics, IT and
telecoms industries into something as small and unimportant as
Hollywood. Hollywood -- which is a tenth the size of IT, itself a
tenth the size of telecoms.

In Hollywood, your ability to make a movie depends on the approval
of a few power-brokers who have signing authority over the
two-hundred-million-dollar budgets for making films. As far as
Hollywood is concerned, this is a feature, not a bug. Two weeks
ago, I heard the VP of Technology for Warners give a presentation
in Dublin on the need to adopt DRM for digital TV, and his
money-shot, his big convincer of a slide went like this:

"With advances in processing power, storage capacity and broadband
access... EVERYBODY BECOMES A BROADCASTER!"

Heaven forfend.

Simple ecosystems are the goal of proceedings like CARP, the panel
that set out the ruinously high royalties for webcasters. The
recording industry set the rates as high as they did so that the
teeming millions of webcasters would be rendered economically
extinct, leaving behind a tiny handful of giant companies that
could be negotiated with around a board room table, rather than
dealt with by blanket legislation.

The razing of the rainforest has a cost. It's harder to send a
legitimate email today than it ever was -- thanks to a world of
closed SMTP relays. The cries for a mail-server monoculture grow
more shrill with every passing moment. Just last week, it was a
call for every mail-administrator to ban the "vacation" program
that sends out automatic responses informing senders that the
recipient is away from email for a few days, because mailboxes that
run vacation can cause "spam blowback" where accounts send their
vacation notices to the hapless individuals whose email addresses
the spammers have substituted on the email's Reply-To line.

And yet there is more spam than there ever was. All the costs we've
paid for fighting spam have added up to no benefit: the network is
still overrun and sometimes even overwhelmed by spam. We've let the
network's neutrality and diversity be compromised, without
receiving the promised benefit of spam-free inboxes.

Likewise, DRM has exacted a punishing toll wherever it has come
into play, costing us innovation, free speech, research and the
public's rights in copyright. And likewise, DRM has not stopped
infringement: today, infringement is more widespread than ever. All
those costs borne by society in the name of protecting artists and
stopping infringement, and not a penny put into an artist's pocket,
not a single DRM-restricted file that can't be downloaded for free
and without encumbrance from a P2P network.

Everywhere we look, we find people who should know better calling
for a parasite-free Internet. Science fiction writers are supposed
to be forward looking, but they're wasting their time demanding
that Amazon and Google make it harder to piece together whole books
from the page-previews one can get via the look-inside-the-book
programs. They're even cooking up programs to spoof deliberately
corrupted ebooks into the P2P networks, presumably to assure the
few readers the field has left that reading science fiction is a
mug's game.

The amazing thing about the failure of parasite-elimination
programs is that their proponents have concluded that the problem
is that they haven't tried hard enough -- with just a few more
species eliminated, a few more policies imposed, paradise will
spring into being. Their answer to an unsuccessful strategy for
fixing the Internet is to try the same strategy, only moreso --
only fill those niches in the ecology that you can sanction. Hunt
and kill more parasites, no matter what the cost.

We are proud parasites, we Emerging Techers. We're engaged in perl
whirling, pythoneering, lightweight javarey -- we hack our cars and
we hack our PCs. We're the rich humus carpeting the jungle floor
and the tiny frogs living in the bromeliads.

The long tail -- Chris Anderson's name for the 95\% of media that
isn't top sellers, but which, in aggregate, accounts for more than
half the money on the table for media vendors -- is the tail of
bottom-feeders and improbable denizens of the ocean's thermal
vents. We're unexpected guests at the dinner table and we have the
nerve to demand a full helping.

Your ideas are cool and you should go and make them real, even if
they demand that the kind of ecological diversity that seems to be
disappearing around us.

You may succeed -- provided that your plans don't call for a simple
ecosystem where only you get to provide value and no one else gets
to play.

\subsection{READ CAREFULLY}

(Originally published as "Shrinkwrap Licenses: An Epidemic Of
Lawsuits Waiting To Happen" in InformationWeek, February 3, 2007)
\textbf{READ CAREFULLY. By reading this article, you agree, on behalf of your employer, to release me from all obligations and waivers arising from any and all NON-NEGOTIATED agreements, licenses, terms-of-service, shrinkwrap, clickwrap, browsewrap, confidentiality, non-disclosure, non-compete and acceptable use policies ("BOGUS AGREEMENTS") that I have entered into with your employer, its partners, licensors, agents and assigns, in perpetuity, without prejudice to my ongoing rights and privileges. You further represent that you have the authority to release me from any BOGUS AGREEMENTS on behalf of your employer.}

READ CAREFULLY -- all in caps, and what it means is, "IGNORE THIS."
That's because the small print in the clickwrap, shrinkwrap,
browsewrap and other non-negotiated agreements is both immutable
and outrageous.

Why read the "agreement" if you know that:

1) No sane person would agree to its text, and

2) Even if you disagree, no one will negotiate a better agreement
with you?

We seem to have sunk to a kind of playground system of forming
contracts. There are those who will tell you that you can form a
binding agreement just by following a link, stepping into a store,
buying a product, or receiving an email. By standing there, shaking
your head, shouting "NO NO NO I DO NOT AGREE," you agree to let me
come over to your house, clean out your fridge, wear your underwear
and make some long-distance calls.

If you buy a downloadable movie from Amazon Unbox, you agree to let
them install spyware on your computer, delete any file they don't
like on your hard-drive, and cancel your viewing privileges for any
reason. Of course, it goes without saying that Amazon reserves the
right to modify the agreement at any time.

The worst offenders are people who sell you movies and music.
They're a close second to people who sell you software, or provide
services over the Internet. There's a rubric to this -- you're
getting a discount in exchange for signing onto an abusive
agreement, but just try and find the software that \textbf{doesn't}
come with one of these "agreements" -- at any price.

For example, Vista, Microsoft's new operating system, comes in a
rainbow of flavors varying in price from \$99 to \$399, but all of
them come with the same crummy terms of service, which state that
"you may not work around any technical limitations in the
software," and that Windows Defender, the bundled anti-malware
program, can delete any program from your hard drive that Microsoft
doesn't like, even if it breaks your computer.

It's bad enough when this stuff comes to us through deliberate
malice, but it seems that bogus agreements can spread almost
without human intervention. Google any obnoxious term or phrase
from a EULA, and you'll find that the same phrase appears in a
dozens -- perhaps thousands -- of EULAs around the Internet. Like
snippets of DNA being passed from one virus to another as they
infect the world's corporations in a pandemic of idiocy, terms of
service are semi-autonomous entities.

Indeed, when rocker Billy Bragg read the fine print on the MySpace
user agreement, he discovered that it appeared that site owner
Rupert Murdoch was laying claim to copyrights in every song
uploaded to the site, in a silent, sinister land-grab that turned
the media baron into the world's most prolific and indiscriminate
hoarder of garage-band tunes.

However, the EULA that got Bragg upset wasn't a Murdoch innovation
-- it dates back to the earliest days of the service. It seems to
have been posted at a time when the garage entrepreneurs who built
MySpace were in no position to hire pricey counsel -- something
borne out by the fact that the old MySpace EULA appears nearly
verbatim on many other services around the Internet. It's not going
out very far on a limb to speculate that MySpace's founders merely
copied a EULA they found somewhere else, without even reading it,
and that when Murdoch's due diligence attorneys were preparing to
give these lucky fellows \$600,000,000, that they couldn't be
bothered to read the terms of service anyway.

In their defense, EULAese is so mind-numbingly boring that it's a
kind of torture to read these things. You can hardly blame them.

But it does raise the question -- why are we playing host to these
infectious agents? If they're not read by customers \textbf{or}
companies, why bother with them?

If you wanted to really be careful about this stuff, you'd prohibit
every employee at your office from clicking on any link, installing
any program, creating accounts, signing for parcels -- even doing a
run to Best Buy for some CD blanks, have you \textbf{seen} the
fine-print on their credit-card slips? After all, these people are
entering into "agreements" on behalf of their employer --
agreements to allow spyware onto your network, to not "work around
any technical limitations in their software," to let malicious
software delete arbitrary files from their systems.

So far, very few of us have been really bitten in the ass by EULAs,
but that's because EULAs are generally associated with companies
who have products or services they're hoping you'll use, and
enforcing their EULAs could cost them business.

But that was the theory with patents, too. So long as everyone with
a huge portfolio of unexamined, overlapping, generous patents was
competing with similarly situated manufacturers, there was a
mutually assured destruction -- a kind of detente represented by
cross-licensing deals for patent portfolios.

But the rise of the patent troll changed all that. Patent trolls
don't make products. They make lawsuits. They buy up the ridiculous
patents of failed companies and sue the everloving hell out of
everyone they can find, building up a war-chest from easy victories
against little guys that can be used to fund more serious campaigns
against larger organizations. Since there are no products to
disrupt with a countersuit, there's no mutually assured
destruction.

If a shakedown artist can buy up some bogus patents and use them to
put the screws to you, then it's only a matter of time until the
same grifters latch onto the innumerable "agreements" that your
company has formed with a desperate dot-bomb looking for an exit
strategy.

More importantly, these "agreements" make a mockery of the law and
of the very \textbf{idea} of forming agreements. Civilization
starts with the idea of a real agreement -- for example, "We crap
\textbf{here} and we sleep \textbf{there}, OK?" -- and if we reduce
the noble agreement to a schoolyard game of no-takebacks, we erode
the bedrock of civilization itself.

\subsection{World of Democracycraft}

(Originally published as "Why Online Games Are Dictatorships,"
InformationWeek, April 16, 2007)

Can you be a citizen of a virtual world? That's the question that I
keep asking myself, whenever anyone tells me about the wonder of
multiplayer online games, especially Second Life, the virtual world
that is more creative playground than game.

These worlds invite us to take up residence in them, to invest time
(and sometimes money) in them. Second Life encourages you to make
stuff using their scripting engine and sell it in the game. You Own
Your Own Mods -- it's the rallying cry of the new generation of
virtual worlds, an updated version of the old BBS adage from the
WELL: You Own Your Own Words.

I spend a lot of time in Disney parks. I even own a share of Disney
stock. But I don't flatter myself that I'm a citizen of Disney
World. I know that when I go to Orlando, the Mouse is going to
fingerprint me and search my bags, because the Fourth Amendment
isn't a "Disney value."

Disney even has its own virtual currency, symbolic tokens called
Disney Dollars that you can spend or exchange at any Disney park.
I'm reasonably confident that if Disney refused to turn my
Mickeybucks back into US Treasury Department-issue greenbacks that
I could make life unpleasant for them in a court of law.

But is the same true of a game? The money in your real-world
bank-account and in your in-game bank-account is really just a
pointer in a database. But if the bank moves the pointer around
arbitrarily (depositing a billion dollars in your account, or
wiping you out), they face a regulator. If a game wants to wipe you
out, well, you probably agreed to let them do that when you signed
up.

Can you amass wealth in such a world? Well, sure. There are rich
people in dictatorships all over the world. Stalin's favorites had
great big dachas and drove fancy cars. You don't need democratic
rights to get rich.

But you \textbf{do} need democratic freedoms to \textbf{stay} rich.
In-world wealth is like a Stalin-era dacha, or the diamond fortunes
of Apartheid South Africa: valuable, even portable (to a limited
extent), but not really \textbf{yours}, not in any stable,
long-term sense.

Here are some examples of the difference between being a citizen
and a customer:

In January, 2006 a World of Warcraft moderator shut down an
advertisement for a "GBLT-friendly" guild. This was a virtual club
that players could join, whose mission was to be "friendly" to
"Gay/Bi/Lesbian/Transgendered" players. The WoW moderator -- and
Blizzard management -- cited a bizarre reason for the shut-down:

"While we appreciate and understand your point of view, we do feel
that the advertisement of a 'GLBT friendly' guild is very likely to
result in harassment for players that may not have existed
otherwise. If you will look at our policy, you will notice the
suggested penalty for violating the Sexual Orientation Harassment
Policy is to 'be temporarily suspended from the game.' However, as
there was clearly no malicious intent on your part, this penalty
was reduced to a warning."

Sara Andrews, the guild's creator, made a stink and embarrassed
Blizzard (the game's parent company) into reversing the decision.

In 2004, a player in the MMO EVE Online declared that the game's
creators had stacked the deck against him, called EVE, "a poorly
designed game which rewards the greedy and violent, and punishes
the hardworking and honest." He was upset over a change in the game
dynamics which made it easier to play a pirate and harder to play a
merchant.

The player, "Dentara Rask," wrote those words in the preamble to a
tell-all memoir detailing an elaborate Ponzi scheme that he and an
accomplice had perpetrated in EVE. The two of them had bilked EVE's
merchants out of a substantial fraction of the game's total GDP and
then resigned their accounts. The objective was to punish the
game's owners for their gameplay decisions by crashing the game's
economy.

In both of these instances, players -- residents of virtual worlds
-- resolved their conflicts with game management through customer
activism. That works in the real world, too, but when it fails, we
have the rule of law. We can sue. We can elect new leaders. When
all else fails, we can withdraw all our money from the bank, sell
our houses, and move to a different country.

But in virtual worlds, these recourses are off-limits. Virtual
worlds can and do freeze players' wealth for "cheating" (amassing
gold by exploiting loopholes in the system), for participating in
real-world gold-for-cash exchanges (eBay recently put an end to
this practice on its service), or for violating some other rule.
The rules of virtual worlds are embodied in EULAs, not
Constitutions, and are always "subject to change without notice."

So what does it mean to be "rich" in Second Life? Sure, you can
have a thriving virtual penis business in game, one that returns a
healthy sum of cash every month. You can even protect your profits
by regularly converting them to real money. But if you lose an
argument with Second Life's parent company, your business vanishes.
In other worlds, the only stable in-game wealth is the wealth you
take out of the game. Your virtual capital investments are totally
contingent. Piss off the wrong exec at Linden Labs, Blizzard, Sony
Online Entertainment, or Sularke and your little in-world business
could disappear for good.

Well, what of it? Why not just create a "democratic" game that has
a constitution, full citizenship for players, and all the
prerequisites for stable wealth? Such a game would be open source
(so that other, interoperable "nations" could be established for
you to emigrate to if you don't like the will of the majority in
one game-world), and run by elected representatives who would
instruct the administrators and programmers as to how to run the
virtual world. In the real world, the TSA sets the rules for
aviation -- in a virtual world, the equivalent agency would
determine the physics of flight.

The question is, would this game be any \textbf{fun}? Well,
democracy itself is pretty fun -- where "fun" means "engrossing and
engaging." Lots of people like to play the democracy game, whether
by voting every four years or by moving to K Street and setting up
a lobbying operation.

But video games aren't quite the same thing. Gameplay conventions
like "grinding" (repeating a task), "leveling up" (attaining a
higher level of accomplishment), "questing" and so on are functions
of artificial scarcity. The difference between a character with
10,000,000 gold pieces and a giant, rare, terrifying crossbow and a
newbie player is which pointers are associated with each
character's database entry. If the elected representatives direct
that every player should have the shiniest armor, best space-ships,
and largest bank-balances possible (this sounds like a pretty good
election platform to me!), then what's left to do?

Oh sure, in Second Life they have an interesting crafting economy
based on creating and exchanging virtual objects. But these objects
are \textbf{also} artificially scarce -- that is, the ability of
these objects to propagate freely throughout the world is limited
only by the software that supports them. It's basically the same
economics of the music industry, but applied to every field of
human endeavor in the entire (virtual) world.

Fun matters. Real world currencies rise and fall based, in part, by
the economic might of the nations that issue them. Virtual world
currencies are more strongly tied to whether there's any reason to
spend the virtual currency on the objects that are denominated in
it. 10,000 EverQuest golds might trade for \$100 on a day when that
same sum will buy you a magic EQ sword that enables you to play
alongside the most interesting people online, running the most fun
missions online. But if all those players out-migrate to World of
Warcraft, and word gets around that Warlord's Command is way more
fun than anything in poor old creaky EverQuest, your EverQuest gold
turns into Weimar Deutschemarks, a devalued currency that you can't
even give away.

This is where the plausibility of my democratic, co-operative, open
source virtual world starts to break down. Elected governments can
field armies, run schools, provide health care (I'm a Canadian),
and bring acid lakes back to health. But I've never done anything
run by a government agency that was a lot of \textbf{fun}. It's my
sneaking suspicion that the only people who'd enjoy playing World
of Democracycraft would be the people running for office there. The
players would soon find themselves playing IRSQuest, Second Notice
of Proposed Rulemaking Life, and Caves of 27 Stroke B.

Maybe I'm wrong. Maybe customership is enough of a rock to build a
platform of sustainable industry upon. It's not like entrepreneurs
in Dubai have a lot of recourse if they get on the wrong side of
the Emir; or like Singaporeans get to appeal the decisions of
President Nathan, and there's plenty of industry there.

And hell, maybe bureaucracies have hidden reserves of fun that have
been lurking there, waiting for the chance to bust out and surprise
us all.

I sure hope so. These online worlds are endlessly diverting places.
It'd be a shame if it turned out that cyberspace was a dictatorship
-- benevolent or otherwise.

\subsection{Snitchtown}

(Originally published in Forbes.com, June 2007)

The 12-story Hotel Torni was the tallest building in central
Helsinki during the Soviet occupation of Finland, making it a
natural choice to serve as KGB headquarters. Today, it bears a
plaque testifying to its checkered past, and also noting the
curious fact that the Finns pulled 40 kilometers of wiretap cable
out of the walls after the KGB left. The wire was solid evidence of
each operative's mistrustful surveillance of his fellow agents.

The East German Stasi also engaged in rampant surveillance, using a
network of snitches to assemble secret files on every resident of
East Berlin. They knew who was telling subversive jokes--but missed
the fact that the Wall was about to come down.

When you watch everyone, you watch no one.

This seems to have escaped the operators of the digital
surveillance technologies that are taking over our cities. In the
brave new world of doorbell cams, wi-fi sniffers, RFID passes, bag
searches at the subway and photo lookups at office security desks,
universal surveillance is seen as the universal solution to all
urban ills. But the truth is that ubiquitous cameras only serve to
violate the social contract that makes cities work.

The key to living in a city and peacefully co-existing as a social
animal in tight quarters is to set a delicate balance of seeing and
not seeing. You take care not to step on the heels of the woman in
front of you on the way out of the subway, and you might take
passing note of her most excellent handbag. But you don't make eye
contact and exchange a nod. Or even if you do, you make sure that
it's as fleeting as it can be.

Checking your mirrors is good practice even in stopped traffic, but
staring and pointing at the schmuck next to you who's got his
finger so far up his nostril he's in danger of lobotomizing himself
is bad form--worse form than picking your nose, even.

I once asked a Japanese friend to explain why so many people on the
Tokyo subway wore surgical masks. Are they extreme germophobes?
Conscientious folks getting over a cold? Oh, yes, he said, yes, of
course, but that's only the rubric. The real reason to wear the
mask is to spare others the discomfort of seeing your facial
expression, to make your face into a disengaged, unreadable
blank--to spare others the discomfort of firing up their mirror
neurons in order to model your mood based on your outward
expression. To make it possible to see without seeing.

There is one city dweller that doesn't respect this delicate social
contract: the closed-circuit television camera. Ubiquitous and
demanding, CCTVs don't have any visible owners. They ... occur.
They exist in the passive voice, the "mistakes were made" voice:
"The camera recorded you."

They are like an emergent property of the system, of being afraid
and looking for cheap answers. And they are everywhere: In London,
residents are photographed more than 300 times a day.

The irony of security cameras is that they watch, but nobody cares
that they're looking. Junkies don't worry about CCTVs. Crazed
rapists and other purveyors of sudden, senseless violence aren't
deterred. I was mugged twice on my old block in San Francisco by
the crack dealers on my corner, within sight of two CCTVs and a
police station. My rental car was robbed by a junkie in a Gastown
garage in Vancouver in sight of a CCTV.

Three mad kids followed my friend out of the Tube in London last
year and murdered him on his doorstep.

Crazy, desperate, violent people don't make rational calculus in
regards to their lives. Anyone who becomes a junkie, crack dealer,
or cellphone-stealing stickup artist is obviously bad at making
life decisions. They're not deterred by surveillance.

Yet the cameras proliferate, and replace human eyes. The cops on my
block in San Francisco stayed in their cars and let the cameras do
the watching. The Tube station didn't have any human guards after
dark, just a CCTV to record the fare evaders.

Now London city councils are installing new CCTVs with
loudspeakers, operated by remote coppers who can lean in and make a
speaker bark at you, "Citizen, pick up your litter." "Stop leering
at that woman." "Move along."

Yeah, that'll work.

Every day the glass-domed cameras proliferate, and the gate-guarded
mentality of the deep suburbs threatens to invade our cities. More
doorbell webcams, more mailbox cams, more cams in our cars.

The city of the future is shaping up to be a neighborly Panopticon,
leeched of the cosmopolitan ability to see, and not be seen, where
every nose pick is noted and logged and uploaded to the Internet.
You don't have anything to hide, sure, but there's a reason we
close the door to the bathroom before we drop our drawers. Everyone
poops, but it takes a special kind of person to want to do it in
public.

The trick now is to contain the creeping cameras of the law. When
the city surveils its citizens, it legitimizes our mutual
surveillance--what's the difference between the cops watching your
every move, or the mall owners watching you, or you doing it to the
guy next door?

I'm an optimist. I think our social contracts are stronger than our
technology. They're the strongest bonds we have. We don't aim
telescopes through each others' windows, because only creeps do
that.

But we need to reclaim the right to record our own lives as they
proceed. We need to reverse decisions like the one that allowed the
New York Metropolitan Transit Authority to line subway platforms
with terrorism cameras, but said riders may not take snapshots in
the station. We need to win back the right to photograph our human
heritage in museums and galleries, and we need to beat back the
snitch-cams rent-a-cops use to make our cameras stay in our
pockets.

They're our cities and our institutions. And we choose the future
we want to live in.

\subsection{Hope you enjoyed it! The actual, physical object that corresponds to this book is superbly designed, portable, and makes a great gift:}

\href{http://craphound.com/content/buy}{http://craphound.com/content/buy}

If you would rather make a donation, you can buy a copy of the book
for a worthy school, library or other institution of your
choosing:

\href{http://craphound.com/content/donate}{http://craphound.com/content/donate}

\subsection{About the Author}

Cory Doctorow (craphound.com) is an award-winning novelist,
activist, blogger and journalist. He is the co-editor of Boing
Boing (boingboing.net), one of the most popular blogs in the world,
and has contributed to The New York Times Sunday Magazine, The
Economist, Forbes, Popular Science, Wired, Make, InformationWeek,
Locus, Salon, Radar, and many other magazines, newspapers and
websites.

His novels and short story collections include
\textbf{Someone Comes to Town, Someone Leaves Town},
\textbf{Down and Out in the Magic Kingdom},
\textbf{Overclocked: Stories of the Future Present} and his most
recent novel, a political thriller for young adults called
\textbf{Little Brother}, published by Tor Books in May, 2008. All
of his novels and short story collections are available as free
downloads under the terms of various Creative Commons licenses.

Doctorow is the former European Director of the Electronic Frontier
Foundation (eff.org) and has participated in many treaty-making,
standards-setting and regulatory and legal battles in countries all
over the world. In 2006/2007, he was the inaugural Canada/US
Fulbright Chair in Public Diplomacy at the Annenberg Center at the
University of Southern California. In 2007, he was also named one
of the World Economic Forum's "Young Global Leaders" and one of
Forbes Magazine's top 25 "Web Celebrities."

Born in Toronto, Canada in 1971, he is a four-time university
dropout. He now resides in London, England with his wife and baby
daughter, where he does his best to avoid the ubiquitous
surveillance cameras while roaming the world, speaking on
copyright, freedom and the future.
