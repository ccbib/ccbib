\hyphenation{mo-no-poly car-ne-gie pro-ject pro-gress mo-dem rou-lette
  browse-wrap Use-net mon-as-tery mo-dems}
\hyphenation{co-me-dic polt-roon stove-pipe Ma-dame scru-ta-ble star-tling}
\hyphenation{heal-thily lim-ou-sines wrest-lers tan-trum push-over un-asked
  bras-siere bro-th-er}
\hyphenation{Can-a-da Fred-rick teen-agers wrest-ler Cha-vez Tho-mas 
  a-nom-a-lies sur-veil-lance ar-mies ref-u-gee ref-u-gees bris-tling
  eve-ning man-chu-ria man-chu-ri-an mid-terms me-di-um jap-a-nese}
\hyphenation{spend-ers googl-ing tour-ist tour-ists leg-end-ary}
\hyphenation{Dan-iel Van-essa Doc-to-row Ste-phen-son}
\hyphenation{de-cade sur-veilled rout-ers Wol-fen-stein teen-ager to-night}
\hyphenation{his-to-gram an-o-nym-ize Ga-la-xy sym-pa-the-tic}
\hyphenation{ar-phid ar-phids Found-ers}
\hyphenation{stran-ger stran-gers shoul-der-blades dump-ling dump-lings}
\hyphenation{ice-pack guard-rail Sep-tem-ber boot-able e-co-nom-ist}
\hyphenation{grown-ups roos-ter shoe-laces li-quid-i-ty}
\hyphenation{side-arm}
\hyphenation{wo-man wo-men tan-trum tan-trums Le-nin-grad zom-bie bunk-house}
\hyphenation{up-tick bio-mass}
\hyphenation{of-fi-cial of-fi-cial-ly gov-ern-ment}
\hyphenation{heal-thy Or-ville spark-ling}
\hyphenation{ves-ti-bule Law-rence au-to-no-mous}
\hyphenation{sau-sage door-step staf-fer}
\hyphenation{tree-trunk}
\hyphenation{to-ron-to}
\hyphenation{qua-dril-lion-aire qua-dril-lion-aires}
\hyphenation{sports-jack-et sports-jack-ets}
\hyphenation{work-space skunk-works}
\hyphenation{kings-ton}


\newenvironment{sign}{\begin{center}\scshape}{\end{center}}
\newenvironment{authorof}{\begin{flushright}\sffamily}{\end{flushright}}

\begin{document}
\begin{center}
\textbf{\huge\textsf{{Craphound}}}
\end{center}

%\setlength{\emergencystretch}{1ex}

From ``A Place So Foreign and Eight More,'' a short story
collection published in September, 2003 by Four Walls Eight Windows
Press (ISBN 1568582862). See http://craphound.com/place for more.

Originally Published in Science Fiction Age, March 1998

Reprinted in:

\begin{itemize}
\item Northern Suns 
  (Tor, 1999, David Hartwell and Glenn Grant, editors)

\item Year's Best Science Fiction XVI 
  (Morrow, 1999, Gardner Dozois, editor)

\item Hayakawa Science Fiction Magazine (Japan)
  September 2001
\end{itemize}

``Like most aliens-mingling-with-human-society stories, Doctorow's 
story serves mostly to hold a mirror up to human nature, but the odd 
corner of human nature it examines is fascinating, and the story is 
smoothly and expertly written, with some good detail and local color 
and some shrewd insights into human nature and human culture, and an 
almost Bradburian vein of rich nostalgia running through it (although 
the nostalgia is quirky enough that perhaps it might more usefully be 
compared to R.A. Lafferty or Terry Bisson than to Bradbury).''

\begin{flushright}
\textsf{-- Gardner Dozois\\ Editor, Asimov's Science Fiction Magazine}
\end{flushright}


\section{Blurbs and quotes:}

\begin{itemize}
\item
  Cory Doctorow straps on his miner's helmet and takes you deep into
  the caverns and underground rivers of Pop Culture, here filtered
  through SF-coloured glasses. Enjoy.

  \begin{authorof}
    Neil Gaiman Author of American Gods and Sandman
  \end{authorof}
\item
  Few writers boggle my sense of reality as much as Cory Doctorow.
  His vision is so far out there, you'll need your GPS to find your
  way back.

  \begin{authorof}
    David Marusek Winner of the Theodore Sturgeon Award, Nebula Award
    nominee
  \end{authorof}
\item
  Cory Doctorow is one of our best new writers: smart, daring, savvy,
  entertaining, ambitious, plugged-in, and as good a guide to the
  wired world of the twenty-first century that stretches out before
  us as you're going to find.

  \begin{authorof}
    Gardner Dozois Editor, Asimov's SF
  \end{authorof}
\item
  He sparkles! He fizzes! He does backflips and breaks the furniture!
  Science fiction needs Cory Doctorow!

  \begin{authorof}
    Bruce Sterling Author of The Hacker Crackdown and Distraction
  \end{authorof}
\item
  Cory Doctorow strafes the senses with a geekspeedfreak explosion of
  gomi kings with heart, weirdass shapeshifters from Pleasure Island
  and jumping automotive jazz joints. If this is Canadian science
  fiction, give me more.

  \begin{authorof}
    Nalo Hopkinson Author of Midnight Robber and Brown Girl in the Ring
  \end{authorof}
\item
  Cory Doctorow is the future of science fiction. An nth-generation
  hybrid of the best of Greg Bear, Rudy Rucker, Bruce Sterling and
  Groucho Marx, Doctorow composes stories that are as BPM-stuffed as
  techno music, as idea-rich as the latest issue of NEW SCIENTIST,
  and as funny as humanity's efforts to improve itself. Utopian,
  insightful, somehow simultaneously ironic and heartfelt, these nine
  tales will upgrade your basal metabolism, overwrite your cortex
  with new and efficient subroutines and generally improve your life
  to the point where you'll wonder how you ever got along with them.
  Really, you should need a prescription to ingest this book. Out of
  all the glittering crap life and our society hands us, craphound
  supreme Doctorow has managed to fashion some industrial-grade
  art."

  \begin{authorof}
    Paul Di Filippo Author of The Steampunk Trilogy
  \end{authorof}
\item
  As scary as the future, and twice as funny. In this eclectic and
  electric collection Doctorow strikes sparks off today to illuminate
  tomorrow, which is what SF is supposed to do. And nobody does it
  better.

  \begin{authorof}
    Terry Bisson Author of Bears Discover Fire
  \end{authorof}
\end{itemize}


\section{A note about this story}

This story is from my collection,
``A Place So Foreign and Eight More,'' published by Four Walls
Eight Windows Press in September, 2003, ISBN 1568582862. I've
released this story, along with five others, under the terms of a
Creative Commons license that gives you, the reader, a bunch of
rights that copyright normally reserves for me, the creator.

I recently did the same thing with the entire text of my novel,
\href{http://craphound.com/down}{``Down and Out in the Magic Kingdom''},
and it was an unmitigated success. Hundreds of thousands of people
downloaded the book --- good news --- and thousands of people
bought the book --- also good news. It turns out that, as near as
anyone can tell, distributing free electronic versions of books is
a great way to sell more of the paper editions, while
simultaneously getting the book into the hands of readers who would
otherwise not be exposed to my work.

I still don't know how it is artists will earn a living in the age
of the Internet, but I remain convinced that the way to find out is
to do basic science: that is, to do stuff and observe the outcome.
That's what I'm doing here. The thing to remember is that the very
\emph{worst} thing you can do to me as an artist is to not read my
work --- to let it languish in obscurity and disappear from
posterity. Most of the fiction I grew up on is out-of-print, and
this is doubly true for the short stories. Losing a couple bucks to
people who would have bought the book save for the availability of
the free electronic text is no big deal, at least when compared to
the horror that is being irrelevant and unread. And luckily for me,
it appears that giving away the text for free gets me more paying
customers than it loses me.

You can find the canonical version of this file at\\
http://craphound.com/place/download.php

If you'd like to convert this file to some other format and
distribute it, you have my permission, provided that:

\begin{itemize}
\item
  You don't charge money for the distribution

\item
  You keep the entire text intact, including this notice, the license
  below, and the metadata at the end of the file

\item
  You don't use a file-format that has ``DRM'' or ``copy-protection''
  or any other form of use-restriction turned on

\end{itemize}
If you'd like, you can advertise the existence of your edition by
posting a link to it at http://craphound.com/place/000012.php


\subsection{Here's a summary of the license:}

\begin{verbatim}
http://creativecommons.org/licenses/by-nd-nc/1.0

Attribution. The licensor permits others to copy, distribute,
display, and perform the work. In return, licensees must give the
original author credit.

No Derivative Works. The licensor permits others to copy,
distribute, display and perform only unaltered copies of the work
-- not derivative works based on it.

Noncommercial. The licensor permits others to copy, distribute,
display, and perform the work. In return, licensees may not use
the work for commercial purposes -- unless they get the
licensor's permission.
\end{verbatim}

\subsection{And here's the license itself:}

\begin{verbatim}
http://creativecommons.org/licenses/by-nd-nc/1.0-legalcode

THE WORK (AS DEFINED BELOW) IS PROVIDED UNDER THE TERMS OF THIS
CREATIVE COMMONS PUBLIC LICENSE ("CCPL" OR "LICENSE"). THE WORK
IS PROTECTED BY COPYRIGHT AND/OR OTHER APPLICABLE LAW. ANY USE OF
THE WORK OTHER THAN AS AUTHORIZED UNDER THIS LICENSE IS
PROHIBITED.

BY EXERCISING ANY RIGHTS TO THE WORK PROVIDED HERE, YOU ACCEPT
AND AGREE TO BE BOUND BY THE TERMS OF THIS LICENSE. THE LICENSOR
GRANTS YOU THE RIGHTS CONTAINED HERE IN CONSIDERATION OF YOUR
ACCEPTANCE OF SUCH TERMS AND CONDITIONS.

1. Definitions

    a. "Collective Work" means a work, such as a periodical issue,
    anthology or encyclopedia, in which the Work in its entirety in
    unmodified form, along with a number of other contributions,
    constituting separate and independent works in themselves, are
    assembled into a collective whole. A work that constitutes a
    Collective Work will not be considered a Derivative Work (as
    defined below) for the purposes of this License.

    b. "Derivative Work" means a work based upon the Work or upon the
    Work and other pre-existing works, such as a translation, musical
    arrangement, dramatization, fictionalization, motion picture
    version, sound recording, art reproduction, abridgment,
    condensation, or any other form in which the Work may be recast,
    transformed, or adapted, except that a work that constitutes a
    Collective Work will not be considered a Derivative Work for the
    purpose of this License.

    c. "Licensor" means the individual or entity that offers the Work
    under the terms of this License.

    d. "Original Author" means the individual or entity who created
    the Work.

    e. "Work" means the copyrightable work of authorship offered
    under the terms of this License.

    f. "You" means an individual or entity exercising rights under
    this License who has not previously violated the terms of this
    License with respect to the Work, or who has received express
    permission from the Licensor to exercise rights under this
    License despite a previous violation.

2. Fair Use Rights. Nothing in this license is intended to
reduce, limit, or restrict any rights arising from fair use,
first sale or other limitations on the exclusive rights of the
copyright owner under copyright law or other applicable laws.

3. License Grant. Subject to the terms and conditions of this
License, Licensor hereby grants You a worldwide, royalty-free,
non-exclusive, perpetual (for the duration of the applicable
copyright) license to exercise the rights in the Work as stated
below:

    a. to reproduce the Work, to incorporate the Work into one or
    more Collective Works, and to reproduce the Work as incorporated
    in the Collective Works;

    b. to distribute copies or phonorecords of, display publicly,
    perform publicly, and perform publicly by means of a digital
    audio transmission the Work including as incorporated in
    Collective Works;

The above rights may be exercised in all media and formats
whether now known or hereafter devised. The above rights include
the right to make such modifications as are technically necessary
to exercise the rights in other media and formats. All rights not
expressly granted by Licensor are hereby reserved.

4. Restrictions. The license granted in Section 3 above is
expressly made subject to and limited by the following
restrictions:

    a. You may distribute, publicly display, publicly perform, or
    publicly digitally perform the Work only under the terms of this
    License, and You must include a copy of, or the Uniform Resource
    Identifier for, this License with every copy or phonorecord of
    the Work You distribute, publicly display, publicly perform, or
    publicly digitally perform. You may not offer or impose any terms
    on the Work that alter or restrict the terms of this License or
    the recipients' exercise of the rights granted hereunder. You may
    not sublicense the Work. You must keep intact all notices that
    refer to this License and to the disclaimer of warranties. You
    may not distribute, publicly display, publicly perform, or
    publicly digitally perform the Work with any technological
    measures that control access or use of the Work in a manner
    inconsistent with the terms of this License Agreement. The above
    applies to the Work as incorporated in a Collective Work, but
    this does not require the Collective Work apart from the Work
    itself to be made subject to the terms of this License. If You
    create a Collective Work, upon notice from any Licensor You must,
    to the extent practicable, remove from the Collective Work any
    reference to such Licensor or the Original Author, as requested.

    b. You may not exercise any of the rights granted to You in
    Section 3 above in any manner that is primarily intended for or
    directed toward commercial advantage or private monetary
    compensation. The exchange of the Work for other copyrighted
    works by means of digital file-sharing or otherwise shall not be
    considered to be intended for or directed toward commercial
    advantage or private monetary compensation, provided there is no
    payment of any monetary compensation in connection with the
    exchange of copyrighted works.

    c. If you distribute, publicly display, publicly perform, or
    publicly digitally perform the Work or any Collective Works, You
    must keep intact all copyright notices for the Work and give the
    Original Author credit reasonable to the medium or means You are
    utilizing by conveying the name (or pseudonym if applicable) of
    the Original Author if supplied; the title of the Work if
    supplied. Such credit may be implemented in any reasonable
    manner; provided, however, that in the case of a Collective Work,
    at a minimum such credit will appear where any other comparable
    authorship credit appears and in a manner at least as prominent
    as such other comparable authorship credit.

5. Representations, Warranties and Disclaimer

    a. By offering the Work for public release under this License,
    Licensor represents and warrants that, to the best of Licensor's
    knowledge after reasonable inquiry:

        i. Licensor has secured all rights in the Work necessary to grant
        the license rights hereunder and to permit the lawful exercise of
        the rights granted hereunder without You having any obligation to
        pay any royalties, compulsory license fees, residuals or any
        other payments;

        ii. The Work does not infringe the copyright, trademark,
        publicity rights, common law rights or any other right of any
        third party or constitute defamation, invasion of privacy or
        other tortious injury to any third party.

    b. EXCEPT AS EXPRESSLY STATED IN THIS LICENSE OR OTHERWISE AGREED
    IN WRITING OR REQUIRED BY APPLICABLE LAW, THE WORK IS LICENSED ON
    AN "AS IS" BASIS, WITHOUT WARRANTIES OF ANY KIND, EITHER EXPRESS
    OR IMPLIED INCLUDING, WITHOUT LIMITATION, ANY WARRANTIES
    REGARDING THE CONTENTS OR ACCURACY OF THE WORK.

6. Limitation on Liability. EXCEPT TO THE EXTENT REQUIRED BY
APPLICABLE LAW, AND EXCEPT FOR DAMAGES ARISING FROM LIABILITY TO
A THIRD PARTY RESULTING FROM BREACH OF THE WARRANTIES IN SECTION
5, IN NO EVENT WILL LICENSOR BE LIABLE TO YOU ON ANY LEGAL THEORY
FOR ANY SPECIAL, INCIDENTAL, CONSEQUENTIAL, PUNITIVE OR EXEMPLARY
DAMAGES ARISING OUT OF THIS LICENSE OR THE USE OF THE WORK, EVEN
IF LICENSOR HAS BEEN ADVISED OF THE POSSIBILITY OF SUCH DAMAGES.

7. Termination

    a. This License and the rights granted hereunder will terminate
    automatically upon any breach by You of the terms of this
    License. Individuals or entities who have received Collective
    Works from You under this License, however, will not have their
    licenses terminated provided such individuals or entities remain
    in full compliance with those licenses. Sections 1, 2, 5, 6, 7,
    and 8 will survive any termination of this License.

    b. Subject to the above terms and conditions, the license granted
    here is perpetual (for the duration of the applicable copyright
    in the Work). Notwithstanding the above, Licensor reserves the
    right to release the Work under different license terms or to
    stop distributing the Work at any time; provided, however that
    any such election will not serve to withdraw this License (or any
    other license that has been, or is required to be, granted under
    the terms of this License), and this License will continue in
    full force and effect unless terminated as stated above.

8. Miscellaneous

    a. Each time You distribute or publicly digitally perform the
    Work or a Collective Work, the Licensor offers to the recipient a
    license to the Work on the same terms and conditions as the
    license granted to You under this License.

    b. If any provision of this License is invalid or unenforceable
    under applicable law, it shall not affect the validity or
    enforceability of the remainder of the terms of this License, and
    without further action by the parties to this agreement, such
    provision shall be reformed to the minimum extent necessary to
    make such provision valid and enforceable.

    c. No term or provision of this License shall be deemed waived
    and no breach consented to unless such waiver or consent shall be
    in writing and signed by the party to be charged with such waiver
    or consent.

    d. This License constitutes the entire agreement between the
    parties with respect to the Work licensed here. There are no
    understandings, agreements or representations with respect to the
    Work not specified here. Licensor shall not be bound by any
    additional provisions that may appear in any communication from
    You. This License may not be modified without the mutual written
    agreement of the Licensor and You.
\end{verbatim}

\section{Craphound}

Craphound had wicked yard-sale karma, for a rotten, filthy alien
bastard. He was too good at panning out the single grain of gold in
a raging river of uselessness for me not to like him --- respect
him, anyway. But then he found the cowboy trunk. It was two months'
rent to me and nothing but some squirrelly alien kitsch-fetish to
Craphound.

So I did the unthinkable. I violated the Code. I got into a bidding
war with a buddy. Never let them tell you that women poison
friendships: in my experience, wounds from women-fights heal
quickly; fights over garbage leave nothing behind but scorched
earth.

Craphound spotted the sign --- his karma, plus the goggles in his
exoskeleton, gave him the advantage when we were doing 80 kmh on
some stretch of back-highway in cottage country. He was riding
shotgun while I drove, and we had the radio on to the CBC's
summer-Saturday programming: eight weekends with eight hours of old
radio dramas: ``The Shadow,'' ``Quiet Please,'' ``Tom Mix,''
``The Crypt-Keeper'' with Bela Lugosi. It was hour three, and Bogey
was phoning in his performance on a radio adaptation of
\emph{The African Queen}. I had the windows of the old truck rolled
down so that I could smoke without fouling Craphound's breather. My
arm was hanging out the window, the radio was booming, and
Craphound said
``Turn around! Turn around, now, Jerry, now, turn around!''

When Craphound gets that excited, it's a sign that he's spotted a
rich vein. I checked the side-mirror quickly, pounded the brakes
and spun around. The transmission creaked, the wheels squealed, and
then we were creeping along the way we'd come.

``There,'' Craphound said, gesturing with his long, skinny arm. I
saw it. A wooden A-frame real-estate sign, a piece of hand-lettered
cardboard stuck overtop of the realtor's name:

\begin{sign}
EAST MUSKOKA VOLUNTEER FIRE-DEPT

LADIES AUXILIARY RUMMAGE SALE

SAT 25 JUNE
\end{sign}

``Hoo-eee!'' I hollered, and spun the truck onto the dirt road. I
gunned the engine as we cruised along the tree-lined road, trusting
Craphound to spot any deer, signs, or hikers in time to avert
disaster. The sky was a perfect blue and the smells of summer were
all around us. I snapped off the radio and listened to the wind
rushing through the truck. Ontario is \emph{beautiful} in the
summer.

``There!'' Craphound shouted. I hit the turn-off and down-shifted
and then we were back on a paved road. Soon, we were rolling into a
country fire-station, an ugly brick barn. The hall was lined with
long, folding tables, stacked high. The mother lode!

Craphound beat me out the door, as usual. His exoskeleton is
programmable, so he can record little scripts for it like: move
left arm to door handle, pop it, swing legs out to running-board,
jump to ground, close door, move forward. Meanwhile, I'm still
making sure I've switched off the headlights and that I've got my
wallet.

Two blue-haired grannies had a card-table set up out front of the
hall, with a big tin pitcher of lemonade and three boxes of Tim
Horton assorted donuts. That stopped us both, since we share the
superstition that you \emph{always} buy food from old ladies and
little kids, as a sacrifice to the crap-gods. One of the old ladies
poured out the lemonade while the other smiled and greeted us.

``Welcome, welcome! My, you've come a long way for us!''

``Just up from Toronto, ma'am,'' I said. It's an old joke, but it's
also part of the ritual, and it's got to be done.

``I meant your friend, sir. This gentleman.''

Craphound smiled without baring his gums and sipped his lemonade.
``Of course I came, dear lady. I wouldn't miss it for the worlds!''
His accent is pretty good, but when it comes to stock phrases like
this, he's got so much polish you'd think he was reading the news.

The biddie \emph{blushed} and \emph{giggled}, and I felt faintly
sick. I walked off to the tables, trying not to hurry. I chose my
first spot, about halfway down, where things wouldn't be quite so
picked-over. I grabbed an empty box from underneath and started
putting stuff into it: four matched highball glasses with gold
crossed bowling-pins and a line of black around the rim; an Expo
'67 wall-hanging that wasn't even a little faded; a shoebox full of
late sixties O-Pee-Chee hockey cards; a worn, wooden-handled steel
cleaver that you could butcher a steer with.

I picked up my box and moved on: a deck of playing cards
copyrighted
'57, with the logo for the Royal Canadian Dairy, Bala Ontario printed 
on the backs; a fireman's cap with a brass badge so tarnished I couldn't 
read it; a three-story wedding-cake trophy for the 1974 Eastern Region 
Curling Championships. The cash-register in my mind was ringing, ringing, 
ringing. God bless the East Muskoka Volunteer Fire Department Ladies'
Auxiliary.

I'd mined that table long enough. I moved to the other end of the
hall. Time was, I'd start at the beginning and turn over each item,
build one pile of maybes and another pile of definites, try to
strategise. In time, I came to rely on instinct and on the fates,
to whom I make my obeisances at every opportunity.

Let's hear it for the fates: a genuine collapsible top-hat; a
white-tipped evening cane; a hand-carved cherry-wood walking stick;
a beautiful black lace parasol; a wrought-iron lightning rod with a
rooster on top; all of it in an elephant-leg umbrella-stand. I
filled the box, folded it over, and started on another.

I collided with Craphound. He grinned his natural grin, the one
that showed row on row of wet, slimy gums, tipped with writhing,
poisonous suckers. ``Gold! Gold!'' he said, and moved along. I
turned my head after him, just as he bent over the cowboy trunk.

I sucked air between my teeth. It was magnificent: a leather-bound
miniature steamer trunk, the leather worked with lariats, Stetson
hats, war-bonnets and six-guns. I moved toward him, and he popped
the latch. I caught my breath.

On top, there was a kid's cowboy costume: miniature leather chaps,
a tiny Stetson, a pair of scuffed white-leather cowboy boots with
long, worn spurs affixed to the heels. Craphound moved it
reverently to the table and continued to pull more magic from the
trunk's depths: a stack of cardboard-bound Hopalong Cassidy 78s; a
pair of tin six-guns with gunbelt and holsters; a silver star that
said Sheriff; a bundle of Roy Rogers comics tied with twine, in
mint condition; and a leather satchel filled with plastic cowboys
and Indians, enough to re-enact the Alamo.

``Oh, my God,'' I breathed, as he spread the loot out on the
table.

``What are these, Jerry?'' Craphound asked, holding up the 78s.

``Old records, like LPs, but you need a special record player to listen 
to them.''
I took one out of its sleeve. It gleamed, scratch-free, in the
overhead fluorescents.

``I got a 78 player here,'' said a member of the East Muskoka
Volunteer Fire Department Ladies' Auxiliary. She was short enough
to look Craphound in the eye, a hair under five feet, and had a
skinny, rawboned look to her.
``That's my Billy's things, Billy the Kid we called him. He was dotty 
for cowboys when he was a boy. Couldn't get him to take off that fool 
outfit --- nearly got him thrown out of school. He's a lawyer now, in 
Toronto, got a fancy office on Bay Street. I called him to ask if he 
minded my putting his cowboy things in the sale, and you know what? 
He didn't know what I was talking about! Doesn't that beat everything? 
He was dotty for cowboys when he was a boy.''

It's another of my rituals to smile and nod and be as polite as
possible to the erstwhile owners of crap that I'm trying to buy, so
I smiled and nodded and examined the 78 player she had produced. In
lariat script, on the top, it said,
``Official Bob Wills Little Record Player,'' and had a crude
watercolour of Bob Wills and His Texas Playboys grinning on the
front. It was the kind of record player that folded up like a
suitcase when you weren't using it. I'd had one as a kid, with Yogi
Bear silkscreened on the front.

Billy's mom plugged the yellowed cord into a wall jack and took the
78 from me, touched the stylus to the record. A tinny ukelele
played, accompanied by horse-clops, and then a narrator with a
deep, whisky voice said,
``Howdy, Pardners! I was just settin' down by the ole campfire. Why 
don't you stay an' have some beans, an' I'll tell y'all the story of 
how Hopalong Cassidy beat the Duke Gang when they come to rob the Santa Fe.''

In my head, I was already breaking down the cowboy trunk and its
contents, thinking about the minimum bid
I`d place on each item at Sotheby's. Sold individually, I figured I could 
get over two grand for the contents. Then I thought about putting ads in 
some of the Japanese collectors'
magazines, just for a lark, before I sent the lot to the auction
house. You never can tell. A buddy I knew had sold a complete
packaged set of Welcome Back, Kotter action figures for nearly
eight grand that way. Maybe I could buy a new truck\ldots{}

``This is wonderful,'' Craphound said, interrupting my reverie.
``How much would you like for the collection?''

I felt a knife in my guts. Craphound had found the cowboy trunk, so
that meant it was his. But he usually let me take the stuff with
street-value --- he was interested in \emph{everything}, so it
hardly mattered if I picked up a few scraps with which to eke out a
living.

Billy's mom looked over the stuff.
``I was hoping to get twenty dollars for the lot, but if that's too much, 
I'm willing to come down.''

``I'll give you thirty,'' my mouth said, without intervention from
my brain.

They both turned and stared at me. Craphound was unreadable behind
his goggles.

Billy's mom broke the silence.
``Oh, my! Thirty dollars for this old mess?''

``I will pay fifty,'' Craphound said.

``Seventy-five,'' I said.

``Oh, my,'' Billy's mom said.

``Five hundred,'' Craphound said.

I opened my mouth, and shut it. Craphound had built his stake on
Earth by selling a complicated biochemical process for
non-chlorophyll photosynthesis to a Saudi banker. I wouldn't ever
beat him in a bidding war. ``A thousand dollars,'' my mouth said.

``Ten thousand,'' Craphound said, and extruded a roll of hundreds
from somewhere in his exoskeleton.

``My Lord!'' Billy's mom said. ``Ten thousand dollars!''

The other pickers, the firemen, the blue haired ladies all looked
up at that and stared at us, their mouths open.

``It is for a good cause.'' Craphound said.

``Ten thousand dollars!'' Billy's mom said again.

Craphound's digits ruffled through the roll as fast as a croupier's
counter, separated off a large chunk of the brown bills, and handed
them to Billy's mom.

One of the firemen, a middle-aged paunchy man with a comb-over
appeared at Billy's mom's shoulder.

``What's going on, Eva?'' he said.

``This\ldots{}gentleman is going to pay ten thousand dollars for Billy's 
old cowboy things, Tom.''

The fireman took the money from Billy's mom and stared at it. He
held up the top note under the light and turned it this way and
that, watching the holographic stamp change from green to gold,
then green again. He looked at the serial number, then the serial
number of the next bill. He licked his forefinger and started
counting off the bills in piles of ten. Once he had ten piles, he
counted them again.
``That's ten thousand dollars, all right. Thank you very much, mister. 
Can I give you a hand getting this to your car?''

Craphound, meanwhile, had re-packed the trunk and balanced the 78
player on top of it. He looked at me, then at the fireman.

``I wonder if I could impose on you to take me to the nearest bus station. 
I think I'm going to be making my own way home.''

The fireman and Billy's mom both stared at me. My cheeks flushed.
``Aw, c'mon,'' I said. ``I'll drive you home.''

``I think I prefer the bus,'' Craphound said.

``It's no trouble at all to give you a lift, friend,'' the fireman
said.

I called it quits for the day, and drove home alone with the truck
only half-filled. I pulled it into the coach-house and threw a tarp
over the load and went inside and cracked a beer and sat on the
sofa, watching a nature show on a desert reclamation project in
Arizona, where the state legislature had traded a derelict
mega-mall and a custom-built habitat to an alien for a local-area
weather control machine.

\tb

The following Thursday, I went to the little crap-auction house on
King Street. I'd put my finds from the weekend in the sale: lower
minimum bid, and they took a smaller commission than Sotheby's.
Fine for moving the small stuff.

Craphound was there, of course. I knew he'd be. It was where we
met, when he bid on a case of Lincoln Logs I'd found at a
fire-sale.

I'd known him for a kindred spirit when he bought them, and we'd
talked afterwards, at his place, a sprawling, two-storey warehouse
amid a cluster of auto-wrecking yards where the junkyard dogs
barked, barked, barked.

Inside was paradise. His taste ran to shrines --- a collection of
fifties bar kitsch that was a shrine to liquor; a circular waterbed
on a raised podium that was nearly buried under seventies bachelor
pad-inalia; a kitchen that was nearly unusable, so packed it was
with old barn-board furniture and rural memorabilia; a
leather-appointed library straight out of a Victorian gentlemen's
club; a solarium dressed in wicker and bamboo and tiki-idols. It
was a hell of a place.

Craphound had known all about the Goodwills and the Sally Anns, and
the auction houses, and the kitsch boutiques on Queen Street, but
he still hadn't figured out where it all came from.

``Yard sales, rummage sales, garage sales,'' I said, reclining in a
vibrating naughahyde easy-chair, drinking a glass of his pricey
single-malt that he'd bought for the beautiful bottle it came in.

``But where are these? Who is allowed to make them?'' Craphound
hunched opposite me, his exoskeleton locked into a coiled,
semi-seated position.

``Who? Well, anyone. You just one day decide that you need to clean out 
the basement, you put an ad in the \emph{Star}, tape up a few signs, 
and voila, yard sale. Sometimes, a school or a church will get 
donations of old junk and sell it all at one time, as a fundraiser.''

``And how do you locate these?'' he asked, bobbing up and down
slightly with excitement.

``Well, there're amateurs who just read the ads in the weekend 
papers, or just pick a neighbourhood and wander around, but that's 
no way to go about it. What I do is, I get in a truck, and I sniff the 
air, catch the scent of crap and \emph{vroom!}, I'm off like a 
bloodhound on a trail. You learn things over time: like stay away 
from Yuppie yard sales, they never have anything worth buying, just 
the same crap you can buy in any mall.''

``Do you think I might accompany you some day?''

``Hell, sure. Next Saturday? We'll head over to Cabbagetown --- those 
old coach houses, you'd be amazed what people get rid of. It's 
practically criminal.''

``I would like to go with you on next Saturday very much Mr Jerry Abington.''
He used to talk like that, without commas or question marks. Later,
he got better, but then, it was all one big sentence.

``Call me Jerry. It's a date, then. Tell you what, though: there's a 
Code you got to learn before we go out. The Craphound's Code.''

``What is a craphound?''

``You're lookin' at one. You're one, too, unless I miss my guess. You'll 
get to know some of the local craphounds, you hang around with me long 
enough. They're the competition, but they're also your buddies, and 
there're certain rules we have.''

And then I explained to him all about how you never bid against a
craphound at a yard-sale, how you get to know the other fellows'
tastes, and when you see something they might like, you haul it out
for them, and they'll do the same for you, and how you never buy
something that another craphound might be looking for, if all
you're buying it for is to sell it back to him. Just good form and
common sense, really, but you'd be surprised how many amateurs just
fail to make the jump to pro because they can't grasp it.

\tb

There was a bunch of other stuff at the auction, other craphounds'
weekend treasures. This was high season, when the sun comes out and
people start to clean out the cottage, the basement, the garage.
There were some collectors in the crowd, and a whole whack of
antique and junk dealers, and a few pickers, and me, and Craphound.
I watched the bidding listlessly, waiting for my things to come up
and sneaking out for smokes between lots. Craphound never once
looked at me or acknowledged my presence, and I became perversely
obsessed with catching his eye, so I coughed and shifted and walked
past him several times, until the auctioneer glared at me, and one
of the attendants asked if I needed a throat lozenge.

My lot came up. The bowling glasses went for five bucks to one of
the Queen Street junk dealers; the elephant-foot fetched \$350
after a spirited bidding war between an antique dealer and a
collector --- the collector won; the dealer took the top-hat for
\$100. The rest of it came up and sold, or didn't, and at end of
the lot, I'd made over \$800, which was rent for the month plus
beer for the weekend plus gas for the truck.

Craphound bid on and bought more cowboy things --- a box of
super-eight cowboy movies, the boxes mouldy, the stock itself
running to slime; a Navajo blanket; a plastic donkey that dispensed
cigarettes out of its ass; a big neon armadillo sign.

One of the other nice things about that place over Sotheby's, there
was none of this waiting thirty days to get a cheque. I queued up
with the other pickers after the bidding was through, collected a
wad of bills, and headed for my truck.

I spotted Craphound loading his haul into a minivan with
handicapped plates. It looked like some kind of fungus was growing
over the hood and side-panels. On closer inspection, I saw that the
body had been covered in closely glued Lego.

Craphound popped the hatchback and threw his gear in, then opened
the driver's side door, and I saw that his van had been fitted out
for a legless driver, with brake and accelerator levers. A
paraplegic I knew drove one just like it. Craphound's exoskeleton
levered him into the seat, and I watched the eerily precise way it
executed the macro that started the car, pulled the shoulder-belt,
put it into drive and switched on the stereo. I heard tape-hiss,
then, loud as a b-boy cruising Yonge Street, an old-timey cowboy
voice: ``Howdy pardners! Saddle up, we're ridin'!'' Then the van
backed up and sped out of the lot.

I get into the truck and drove home. Truth be told, I missed the
little bastard.

\tb

Some people said that we should have run Craphound and his kin off
the planet, out of the Solar System. They said that it wasn't fair
for the aliens to keep us in the dark about their technologies.
They say that we should have captured a ship and reverse-engineered
it, built our own and kicked ass.

Some people!

First of all, nobody with human DNA could survive a trip in one of
those ships. They're part of Craphound's people's bodies, as I
understand it, and we just don't have the right parts. Second of
all, they \emph{were} sharing their tech with us --- they just
weren't giving it away. Fair trades every time.

It's not as if space was off-limits to us. We can any one of us
visit their homeworld, just as soon as we figure out how. Only they
wouldn't hold our hands along the way.

\tb

I spent the week haunting the ``Secret Boutique,'' AKA the Goodwill
As-Is Centre on Jarvis. It's all there is to do between yard sales,
and sometimes it makes for good finds. Part of my theory of
yard-sale karma holds that if I miss one day at the thrift shops,
that'll be the day they put out the big score. So I hit the stores
diligently and came up with crapola. I had offended the fates, I
knew, and wouldn't make another score until I placated them. It was
lonely work, still and all, and I missed Craphound's good eye and
obsessive delight.

I was at the cash-register with a few items at the Goodwill when a
guy in a suit behind me tapped me on the shoulder.

``Sorry to bother you,'' he said. His suit looked expensive, as did
his manicure and his haircut and his wire-rimmed glasses.
``I was just wondering where you found that.'' He gestured at a
rhinestone-studded ukelele, with a cowboy hat wood-burned into the
body. I had picked it up with a guilty little thrill, thinking that
Craphound might buy it at the next auction.

``Second floor, in the toy section.''

``There wasn't anything else like it, was there?''

``\,'Fraid not,'' I said, and the cashier picked it up and started
wrapping it in newspaper.

``Ah,'' he said, and he looked like a little kid who'd just been
told that he couldn't have a puppy.
``I don't suppose you'd want to sell it, would you?''

I held up a hand and waited while the cashier bagged it with the
rest of my stuff, a few old clothbound novels I thought I could
sell at a used book-store, and a Grease belt-buckle with Olivia
Newton John on it. I led him out the door by the elbow of his
expensive suit.

``How much?'' I had paid a dollar.

``Ten bucks?''

I nearly said, ``Sold!'' but I caught myself. ``Twenty.''

``Twenty dollars?''

``That's what they'd charge at a boutique on Queen Street.''

He took out a slim leather wallet and produced a twenty. I handed
him the uke. His face lit up like a lightbulb.

\tb

It's not that my adulthood is particularly unhappy. Likewise, it's
not that my childhood was particularly happy.

There are memories I have, though, that are like a cool drink of
water. My grandfather's place near Milton, an old Victorian
farmhouse, where the cat drank out of a milk-glass bowl; and where
we sat around a rough pine table as big as my whole apartment; and
where my playroom was the draughty barn with hay-filled lofts
bulging with farm junk and Tarzan-ropes.

There was Grampa's friend Fyodor, and we spent every evening at his
wrecking-yard, he and Grampa talking and smoking while I scampered
in the twilight, scaling mountains of auto-junk. The glove-boxes
yielded treasures: crumpled photos of college boys mugging in front
of signs, roadmaps of far-away places. I found a guidebook from the
1964 New York World's Fair once, and a lipstick like a chrome
bullet, and a pair of white leather ladies' gloves.

Fyodor dealt in scrap, too, and once, he had half of a carny
carousel, a few horses and part of the canopy, paint flaking and
sharp torn edges protruding; next to it, a Korean-war tank minus
its turret and treads, and inside the tank were peeling old pinup
girls and a rotation schedule and a crude Kilroy. The control-room
in the middle of the carousel had a stack of paperback sci-fi
novels, Ace Doubles that had two books bound back-to-back, and when
you finished the first, you turned it over and read the other.
Fyodor let me keep them, and there was a pawn-ticket in one from
Macon, Georgia, for a transistor radio.

My parents started leaving me alone when I was fourteen and I
couldn't keep from sneaking into their room and snooping. Mom's
jewelry box had books of matches from their honeymoon in Acapulco,
printed with bad palm-trees. My Dad kept an old photo in his sock
drawer, of himself on muscle-beach, shirtless, flexing his biceps.

My grandmother saved every scrap of my mother's life in her
basement, in dusty Army trunks. I entertained myself by pulling it
out and taking it in: her Mouse Ears from the big family train-trip
to Disneyland in '57, and her records, and the glittery pasteboard
sign from her sweet sixteen. There were well-chewed stuffed
animals, and school exercise books in which she'd practiced
variations on her signature for page after page.

It all told a story. The penciled Kilroy in the tank made me see
one of those Canadian soldiers in Korea, unshaven and crew-cut like
an extra on M\emph{A}S*H, sitting for bored hour after hour,
staring at the pinup girls, fiddling with a crossword, finally
laying it down and sketching his Kilroy quickly, before anyone
saw.

The photo of my Dad posing sent me whirling through time to
Toronto's Muscle Beach in the east end, and hearing the tinny AM
radios playing weird psychedelic rock while teenagers lounged on
their Mustangs and the girls sunbathed in bikinis that made their
tits into torpedoes.

It all made poems. The old pulp novels and the pawn ticket, when I
spread them out in front of the TV, and arranged them just so, they
made up a poem that took my breath away.

\tb

After the cowboy trunk episode, I didn't run into Craphound again
until the annual Rotary Club charity rummage sale at the Upper
Canada Brewing Company. He was wearing the cowboy hat, sixguns and
the silver star from the cowboy trunk. It should have looked
ridiculous, but the net effect was naive and somehow charming, like
he was a little boy whose hair you wanted to muss.

I found a box of nice old melamine dishes, in various shades of
green --- four square plates, bowls, salad-plates, and a serving
tray. I threw them in the duffel-bag I'd brought and kept browsing,
ignoring Craphound as he charmed a salty old Rotarian while
fondling a box of leather-bound books.

I browsed a stack of old Ministry of Labour licenses --- barber,
chiropodist, bartender, watchmaker. They all had pretty seals and
were framed in stark green institutional metal. They all had
different names, but all from one family, and I made up a little
story to entertain myself, about the proud mother saving her sons'
accreditations and framing hanging them in the spare room with
their diplomas.
``Oh, George Junior's just opened his own barbershop, and little 
Jimmy's still fixing watches\ldots{}''

I bought them.

In a box of crappy plastic Little Ponies and Barbies and Care
Bears, I found a leather Indian headdress, a wooden bow-and-arrow
set, and a fringed buckskin vest. Craphound was still buttering up
the leather books' owner. I bought them quick, for five bucks.

``Those are beautiful,'' a voice said at my elbow. I turned around
and smiled at the snappy dresser who'd bought the uke at the Secret
Boutique. He'd gone casual for the weekend, in an expensive, L.L.
Bean button-down way.

``Aren't they, though.''

``You sell them on Queen Street? Your finds, I mean?''

``Sometimes. Sometimes at auction. How's the uke?''

``Oh, I got it all tuned up,'' he said, and smiled the same smile
he'd given me when he'd taken hold of it at Goodwill.
``I can play `Don't Fence Me In' on it.'' He looked at his feet.
``Silly, huh?''

``Not at all. You're into cowboy things, huh?'' As I said it, I was
overcome with the knowledge that this was ``Billy the Kid,'' the
original owner of the cowboy trunk. I don't know why I felt that
way, but I did, with utter certainty.

``Just trying to re-live a piece of my childhood, I guess. I'm Scott,''
he said, extending his hand.

\emph{Scott?} I thought wildly. \emph{Maybe it's his middle name?}
``I'm Jerry.''

The Upper Canada Brewery sale has many things going for it,
including a beer garden where you can sample their wares and get a
good BBQ burger. We gently gravitated to it, looking over the
tables as we went.

``You're a pro, right?'' he asked after we had plastic cups of
beer.

``You could say that.''

``I'm an amateur. A rank amateur. Any words of wisdom?''

I laughed and drank some beer, lit a cigarette.
``There's no secret to it, I think. Just diligence: you've got to go 
out every chance you get, or you'll miss the big score.''

He chuckled.
``I hear that. Sometimes, I'll be sitting in my office, and I'll 
just \emph{know} that they're putting out a piece of pure gold at 
the Goodwill and that someone else will get to it before my lunch. 
I get so wound up, I'm no good until I go down there and hunt for it. 
I guess I'm hooked, eh?''

``Cheaper than some other kinds of addictions.''

``I guess so. About that Indian stuff --- what do you figure you'd 
get for it at a Queen Street boutique?''

I looked him in the eye. He may have been something high-powered
and cool and collected in his natural environment, but just then,
he was as eager and nervous as a kitchen-table poker-player at a
high-stakes game.

``Maybe fifty bucks,'' I said.

``Fifty, huh?'' he asked.

``About that,'' I said.

``Once it sold,'' he said.

``There is that,'' I said.

``Might take a month, might take a year,'' he said.

``Might take a day,'' I said.

``It might, it might.'' He finished his beer.
``I don't suppose you'd take forty?''

I'd paid five for it, not ten minutes before. It looked like it
would fit Craphound, who, after all, was wearing Scott/Billy's own
boyhood treasures as we spoke. You don't make a living by feeling
guilty over eight hundred percent markups. Still, I'd angered the
fates, and needed to redeem myself.

``Make it five,'' I said.

He started to say something, then closed his mouth and gave me a
look of thanks. He took a five out of his wallet and handed it to
me. I pulled the vest and bow and headdress out my duffel.

He walked back to a shiny black Jeep with gold detail work, parked
next to Craphound's van. Craphound was building onto the Lego body,
and the hood had a miniature Lego town attached to it.

Craphound looked around as he passed, and leaned forward with
undisguised interest at the booty. I grimaced and finished my
beer.

\tb

I met Scott/Billy three times more at the Secret Boutique that
week.

He was a lawyer, who specialised in alien-technology patents. He
had a practice on Bay Street, with two partners, and despite his
youth, he was the senior man.

I didn't let on that I knew about Billy the Kid and his mother in
the East Muskoka Volunteer Fire Department Ladies' Auxiliary. But I
felt a bond with him, as though we shared an unspoken secret. I
pulled any cowboy finds for him, and he developed a pretty good eye
for what I was after and returned the favour.

The fates were with me again, and no two ways about it. I took home
a ratty old Oriental rug that on closer inspection was a 19th
century hand-knotted Persian; an upholstered Turkish footstool; a
collection of hand-painted silk Hawaiiana pillows and a carved
Meerschaum pipe. Scott/Billy found the last for me, and it cost me
two dollars. I knew a collector who would pay thirty in an
eye-blink, and from then on, as far as I was concerned, Scott/Billy
was a fellow craphound.

``You going to the auction tomorrow night?'' I asked him at the
checkout line.

``Wouldn't miss it,'' he said. He'd barely been able to contain his
excitement when I told him about the Thursday night auctions and
the bargains to be had there. He sure had the bug.

``Want to get together for dinner beforehand? 
The Rotterdam's got a good patio.''

He did, and we did, and I had a glass of framboise that packed a
hell of a kick and tasted like fizzy raspberry lemonade; and
doorstopper fries and a club sandwich.

I had my nose in my glass when he kicked my ankle under the table.
``Look at that!''

It was Craphound in his van, cruising for a parking spot. The Lego
village had been joined by a whole postmodern spaceport on the
roof, with a red-and-blue castle, a football-sized flying saucer,
and a clown's head with blinking eyes.

I went back to my drink and tried to get my appetite back.

``Was that an extee driving?''

``Yeah. Used to be a friend of mine.''

``He's a picker?''

``Uh-huh.'' I turned back to my fries and tried to kill the
subject.

``Do you know how he made his stake?''

``The chlorophyll thing, in Saudi Arabia.''

``Sweet!'' he said.
``Very sweet. I've got a client who's got some secondary patents 
from that one. What's he go after?''

``Oh, pretty much everything,'' I said, resigning myself to
discussing the topic after all.
``But lately, the same as you --- cowboys and Injuns.''

He laughed and smacked his knee.
``Well, what do you know? What could he possibly want with the stuff?''

``What do they want with any of it? He got started one day when 
we were cruising the Muskokas,''
I said carefully, watching his face.
``Found a trunk of old cowboy things at a rummage sale. East 
Muskoka Volunteer Fire Department Ladies' Auxiliary.''
I waited for him to shout or startle. He didn't.

``Yeah? A good find, I guess. Wish I'd made it.''

I didn't know what to say to that, so I took a bite of my
sandwich.

Scott continued.
``I think about what they get out of it a lot. There's nothing we have 
here that they couldn't make for themselves. I mean, if they picked 
up and left today, we'd still be making sense of everything they 
gave us in a hundred years. You know, I just closed a deal for a 
biochemical computer that's no-shit 10,000 times faster than 
anything we've built out of silicon. You know what the extee took 
in trade? Title to a defunct fairground outside of Calgary --- they 
shut it down ten years ago because the midway was too unsafe to ride. 
Doesn't that beat all? This thing is worth a billion dollars right 
out of the gate, I mean, within twenty-four hours of the deal closing, 
the seller can turn it into the GDP of Bolivia. For a crummy 
real-estate dog that you couldn't get five grand for!''

It always shocked me when Billy/Scott talked about his job --- it
was easy to forget that he was a high-powered lawyer when we were
jawing and fooling around like old craphounds. I wondered if maybe
he \emph{wasn't} Billy the Kid; I couldn't think of any reason for
him to be playing it all so close to his chest.

``What the hell is some extee going to do with a fairground?''

\tb

Craphound got a free Coke from Lisa at the check-in when he made
his appearance. He bid high, but shrewdly, and never pulled
ten-thousand-dollar stunts. The bidders were wandering the floor,
previewing that week's stock, and making notes to themselves.

I rooted through a box-lot full of old tins, and found one with a
buckaroo at the Calgary Stampede, riding a bucking bronc. I picked
it up and stood to inspect it. Craphound was behind me.

``Nice piece, huh?'' I said to him.

``I like it very much,'' Craphound said, and I felt my cheeks
flush.

``You're going to have some competition tonight, I think,'' I said,
and nodded at Scott/Billy.
``I think he's Billy; the one whose mother sold us --- you --- the 
cowboy trunk.''

``Really?'' Craphound said, and it felt like we were partners
again, scoping out the competition. Suddenly I felt a knife of
shame, like I was betraying Scott/Billy somehow. I took a step
back.

``Jerry, I am very sorry that we argued.''

I sighed out a breath I hadn't known I was holding in.
``Me, too.''

``They're starting the bidding. May I sit with you?''

And so the three of us sat together, and Craphound shook
Scott/Billy's hand and the auctioneer started into his harangue.

It was a night for unusual occurrences. I bid on a piece, something
I told myself I'd never do. It was a set of four matched Li'l
Orphan Annie Ovaltine glasses, like Grandma's had been, and seeing
them in the auctioneer's hand took me right back to her kitchen,
and endless afternoons passed with my colouring books and weird
old-lady hard candies and Liberace albums playing in the living
room.

``Ten,'' I said, opening the bidding.

``I got ten, ten,ten, I got ten, who'll say twenty, who'll say twenty, 
twenty for the four.''

Craphound waved his bidding card, and I jumped as if I'd been
stung.

``I got twenty from the space cowboy, I got twenty, sir will you say thirty?''

I waved my card.

``That's thirty to you sir.''

``Forty,'' Craphound said.

``Fifty,'' I said even before the auctioneer could point back to
me. An old pro, he settled back and let us do the work.

``One hundred,'' Craphound said.

``One fifty,'' I said.

The room was perfectly silent. I thought about my overextended
MasterCard, and wondered if Scott/Billy would give me a loan.

``Two hundred,'' Craphound said.

Fine, I thought. Pay two hundred for those. I can get a set on
Queen Street for thirty bucks.

The auctioneer turned to me.
``The bidding stands at two. Will you say two-ten, sir?''

I shook my head. The auctioneer paused a long moment, letting me
sweat over the decision to bow out.

``I have two --- do I have any other bids from the floor? Any other bids? 
Sold, \$200, to number 57.''
An attendant brought Craphound the glasses. He took them and tucked
them under his seat.

\tb

I was fuming when we left. Craphound was at my elbow. I wanted to
punch him --- I'd never punched anyone in my life, but I wanted to
punch him.

We entered the cool night air and I sucked in several lungfuls
before lighting a cigarette.

``Jerry,'' Craphound said.

I stopped, but didn't look at him. I watched the taxis pull in and
out of the garage next door instead.

``Jerry, my friend,'' Craphound said.

``\emph{What}?'' I said, loud enough to startle myself. Scott,
beside me, jerked as well.

``We're going. I wanted to say goodbye, and to give you some things 
that I won't be taking with me.''

``What?'' I said again, Scott just a beat behind me.

``My people --- we're going. 
It has been decided. We've gotten what we came for.''

Without another word, he set off towards his van. We followed along
behind, shell-shocked.

Craphound's exoskeleton executed another macro and slid the
panel-door aside, revealing the cowboy trunk.

``I wanted to give you this. I will keep the glasses.''

``I don't understand,'' I said.

``You're all leaving?'' Scott asked, with a note of urgency.

``It has been decided. We'll go over the next twenty-four hours.''

``But \emph{why}?'' Scott said, sounding almost petulant.

``It's not something that I can easily explain. As you must know, 
the things we gave you were trinkets to us --- almost worthless. 
We traded them for something that was almost worthless to 
you --- a fair trade, you'll agree --- but it's time to move on.''

Craphound handed me the cowboy trunk. Holding it, I smelled the
lubricant from his exoskeleton and the smell of the attic it had
been mummified in before making its way into his hands. I felt like
I almost understood.

``This is for me,'' I said slowly, and Craphound nodded
encouragingly.
``This is for me, and you're keeping the glasses. And I'll 
look at this and feel\ldots{}''

``You understand,'' Craphound said, looking somehow relieved.

And I \emph{did}. I understood that an alien wearing a cowboy hat
and sixguns and giving them away was a poem and a story, and a
thirtyish bachelor trying to spend half a month's rent on four
glasses so that he could remember his Grandma's kitchen was a story
and a poem, and that the disused fairground outside Calgary was a
story and a poem, too.

``You're craphounds!'' I said. ``All of you!''

Craphound smiled so I could see his gums and I put down the cowboy
trunk and clapped my hands.

\tb

Scott recovered from his shock by spending the night at his office,
crunching numbers talking on the phone, and generally getting while
the getting was good. He had an edge --- no one else knew that they
were going.

He went pro later that week, opened a chi-chi boutique on Queen
Street, and hired me on as chief picker and factum factotum.

Scott was not Billy the Kid. Just another Bay Street shyster with a
cowboy jones. From the way they come down and spend, there must be
a million of them.

Our draw in the window is a beautiful mannequin I found, straight
out of the Fifties, a little boy we call The Beaver. He dresses in
chaps and a Sheriff's badge and sixguns and a miniature Stetson and
cowboy boots with worn spurs, and rests one foot on a beautiful
miniature steamer trunk whose leather is worked with cowboy
motifs.

He's not for sale at any price.

\end{document}
