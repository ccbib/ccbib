\hyphenation{mo-no-poly car-ne-gie pro-ject pro-gress mo-dem rou-lette
  browse-wrap Use-net mon-as-tery mo-dems}
\hyphenation{co-me-dic polt-roon stove-pipe Ma-dame scru-ta-ble star-tling}
\hyphenation{heal-thily lim-ou-sines wrest-lers tan-trum push-over un-asked
  bras-siere bro-th-er}
\hyphenation{Can-a-da Fred-rick teen-agers wrest-ler Cha-vez Tho-mas 
  a-nom-a-lies sur-veil-lance ar-mies ref-u-gee ref-u-gees bris-tling
  eve-ning man-chu-ria man-chu-ri-an mid-terms me-di-um jap-a-nese}
\hyphenation{spend-ers googl-ing tour-ist tour-ists leg-end-ary}
\hyphenation{Dan-iel Van-essa Doc-to-row Ste-phen-son}
\hyphenation{de-cade sur-veilled rout-ers Wol-fen-stein teen-ager to-night}
\hyphenation{his-to-gram an-o-nym-ize Ga-la-xy sym-pa-the-tic}
\hyphenation{ar-phid ar-phids Found-ers}
\hyphenation{stran-ger stran-gers shoul-der-blades dump-ling dump-lings}
\hyphenation{ice-pack guard-rail Sep-tem-ber boot-able e-co-nom-ist}
\hyphenation{grown-ups roos-ter shoe-laces li-quid-i-ty}
\hyphenation{side-arm}
\hyphenation{wo-man wo-men tan-trum tan-trums Le-nin-grad zom-bie bunk-house}
\hyphenation{up-tick bio-mass}
\hyphenation{of-fi-cial of-fi-cial-ly gov-ern-ment}
\hyphenation{heal-thy Or-ville spark-ling}
\hyphenation{ves-ti-bule Law-rence au-to-no-mous}
\hyphenation{sau-sage door-step staf-fer}
\hyphenation{tree-trunk}
\hyphenation{to-ron-to}
\hyphenation{qua-dril-lion-aire qua-dril-lion-aires}
\hyphenation{sports-jack-et sports-jack-ets}
\hyphenation{work-space skunk-works}
\hyphenation{kings-ton}


\begin{document}
%\setlength{\emergencystretch}{1ex}
\raggedbottom

\begin{center}
\textbf{\huge\textsf{Epoch}}

\medskip
Cory Doctorow

\end{center}

\bigskip

\begin{flushleft}
This story is part of Cory Doctorow’s short story collection
“With a Little Help” published by himself. It is licensed under a
\href{http://creativecommons.org/licenses/by-nc-sa/}
{Creative Commons Attribution-NonCommercial-ShareAlike 3.0} license.

\bigskip

The whole volume is available at:
\texttt{http://craphound.com/walh/}

\medskip

The volume has been split into individual stories for the purpose of the
\href{http://ccbib.org}{Creative Commons Bibliothek.}
The introduction and similar accompanying texts are available under the 
title:
\end{flushleft}
\begin{center}
With a Little Help -- Extra Stuff
\end{center}

\newpage

\section{Epoch}

The doomed rogue AI is called BIGMAC and he is my responsibility. Not 
my responsibility as in “I am the creator of BIGMAC, responsible for 
his existence on this planet.” That honor belongs to the 
long-departed Dr Shannon, one of the shining lights of the once great 
Sun-Oracle Institute for Advanced Studies, and he had been dead for 
years before I even started here as a lowly sysadmin.

No, BIGMAC is my responsibility as in, “I, Odell Vyphus, am the 
systems administrator responsible for his care, feeding and eventual 
euthanizing.” Truth be told, I'd rather be Dr Shannon (except for the 
being dead part). I may be a lowly grunt, but I'm smart enough to know 
that being the Man Who Gave The World AI is better than being The Kid 
Who Killed It.

Not that anyone would care, really. 115 years after Mary Shelley first 
started humanity's hands wringing over the possibility that we would 
create a machine as smart as us but out of our control, Dr Shannon did 
it, and it turned out to be incredibly, utterly boring. BIGMAC played 
chess as well as the non-self-aware computers, but he could muster some 
passable trash-talk while he beat you. BIGMAC could trade banalities 
all day long with any Turing tester who wanted to waste a day chatting 
with an AI. BIGMAC could solve some pretty cool vision-system problems 
that had eluded us for a long time, and he wasn't a bad UI to a search 
engine, but the incremental benefit over non-self-aware vision systems 
and UIs was pretty slender. There just weren't any killer apps for AI.

By the time BIGMAC came under my care, he was less a marvel of the 21st 
century and more a technohistorical curiosity who formed the punchline 
to lots of jokes but otherwise performed no useful service to humanity 
in exchange for the useful services that humanity (e.g., me) rendered 
to him.

I had known for six months that I'd be decommissioning old BM (as I 
liked to call him behind his back) but I hadn't seen any reason to let 
him in on the gag. Luckily (?) for all of us, BIGMAC figured it out for 
himself and took steps in accord with his nature.

This is the story of BIGMAC's extraordinary self-preservation program, 
and the story of how I came to love him, and the story of how he came 
to die.

My name is Odell Vyphus. I am a third-generation systems administrator. 
I am 25 years old. I have always been sentimental about technology. I 
have always been an anthropomorphizer of computers. It's an 
occupational hazard.

\tb

BIGMAC thought I was crazy to be worrying about the rollover. “It's 
just Y2K all over again,” he said. He had a good voice -- speech 
synthesis was solved long before he came along -- but it had odd 
inflections that meant that you never forgot you were talking with a 
nonhuman.

“You weren't even around for Y2K,” I said. “Neither was I. The 
only thing anyone remembers about it, \emph{today}, is that it all blew 
over. But no one can tell, at this distance, \emph{why} it blew over. 
Maybe all that maintenance tipped the balance.”

BIGMAC blew a huge load of IPv4 ICMP traffic across the network, stuff 
that the firewalls were supposed to keep out of the system, and every 
single intrusion detection system alarm lit, making my screen into a 
momentary mosaic of competing alerts. It was his version of a raspberry 
and I had to admit it was pretty imaginative, especially since the 
IDSes were self-modifying and required that he come up with new and 
better ways of alarming them each time.

“Odell,” he said, “the fact is, almost everything is broken, 
almost always. If the failure rate of the most vital systems in the 
world went up by 20 percent, it would just mean some overtime for a few 
maintenance coders, not Gotterdammerung. Trust me. I know. I'm a 
computer.”

The rollover was one of those incredibly boring apocalypses that 
periodically get extracted by the relevance filters, spun into 
screaming 128-point linkbait headlines, then dissolved back into their 
fundamental, incontrovertible technical dullness and out of the public 
consciousness. Rollover: 19 January, 2038. The day that the Unix time 
function would run out of headroom and roll back to zero, or do 
something else undefined.

Oh, not your modern unices. Not even your \emph{elderly} unices. To 
find a rollover-vulnerable machine, you needed to find something 
running an elderly, \emph{32-bit paleounix}. A machine running on a 
processor that was at least \emph{20} years old -- 2018 being the last 
date that a 32-bit processor shipped from any major fab. Or an emulated 
instance thereof, of course. And counting emulations, there were only --

“There's fourteen \emph{billion} of them!” I said. “That's not 20 
percent more broken! That's the infocalypse.”

“You meatsacks are \emph{so} easily impressed by zeroes. The 
important number isn't how many 32-bit instances of Unix are in 
operation today. It's not even how many \emph{vulnerable} ones there 
are. It's \emph{how much damage} all those vulnerable ones will cause 
when they go blooie. And I'm betting: not much. It will be, how do you 
say, `meh?'”

My grandfather remembered installing the systems that caused the Y2K 
problem. My dad remembered the birth of “meh.” I remember the rise 
and fall of anyone caring about AI. Technology is glorious.

“But OK, stipulate that you're right and lots of important things go 
blooie on January 19. You might not get accurate weather reports. The 
economy might bobble a little. Your transport might get stuck. Your pay 
might land in your bank a day late. And?”

He had me there. “It would be terrible --”

“You know what I think? I think you \emph{want} it to be terrible. 
You \emph{want} to live in the Important Epoch In Which It All Changes. 
You want to know that something significant happened on your watch. You 
don't want to live in one of those Unimportant Epochs In Which It All 
Stayed the Same and Nothing Much Happened. Being alive in the Epoch in 
Which AI Became Reality doesn't cut the mustard, apparently.”

I squirmed in my seat. That morning, my boss, Peyton Moldovan, had 
called me into her office -- a beautifully restored temporary habitat 
dating back to the big LA floods, when this whole plot of land had been 
a giant and notorious refugee camp. Sun-Oracle had gotten it for cheap 
and located its Institute there, on the promise that they preserve the 
hastily thrown-up structures where so many had despaired. I sat on a 
cushion on the smooth cement floor -- the structures had been delivered 
as double-walled bags full of cement mix, needing only to be 
“inflated” with high-pressure water to turn them into big, 
dome-shaped sterile cement yurts.

“Odell,” she said, “I've been reviewing our budget for the next 
three quarters and the fact of the matter is, there's no room in it for 
BIGMAC.”

I put on my best smooth, cool professional face. “I see,” I said.

“Now, \emph{you've} still got a job, of course. Plenty of places for 
a utility infielder like yourself here. Tell the truth, most labs are 
\emph{begging} for decent admins to keep things running. But BIGMAC 
just isn't a good use of the institute's resources. The project hasn't 
produced a paper or even a press-mention in over a year and there's no 
reason to believe that it will. AI is just --”

\emph{Boring}, I thought, but I didn't say it. The B-word was banned in 
the BIGMAC center. “What about the researchers?”

She shrugged. “What researchers? Palinciuc has been lab-head 
\emph{pro tem} for 16 months and she's going on maternity leave next 
week and there's no one in line to be the \emph{pro-tem pro-tem}. Her 
grad students would love to work on something meaningful, like 
Binenbaum's lab.” That was the new affective computing lab, in which 
they were building computers that simulated emotions so that their 
owners would feel better about their mistakes. BIGMAC \emph{had} 
emotions, but they weren't the kind of emotions that made his mistakes 
easier to handle. The key here was \emph{simulated} emotions. Affective 
computing had taken a huge upswing ever since they'd thrown out the 
fMRIs and stopped pretending they could peer into the human mind in 
realtime and draw meaningful conclusions from it.

She had been sitting cross-legged across from me on an embroidered 
Turkish pillow. Now she uncrossed and recrossed her legs in the other 
direction and arched her back. “Look, Odell, you know how much we 
value you --”

I held up my hand. “I know. It's not that. It's BIGMAC. I just can't 
help but feel --”

“He's not a person. He's just a clever machine that is good at acting 
personlike.”

“I think that describes me and everybody I know, present company 
included.” One of the longstanding benefits to being a sysadmin is 
that you get to act like a holy fool and speak truth to power and wear 
dirty t-shirts with obscure slogans, because you know all the passwords 
and have full access to everyone's clickstream and IM logs. I gave her 
the traditional rascally sysadmin grin and wink to let her know it was 
\emph{ha ha only serious.}

She gave me a weak, quick grin back. “Nevertheless. The fact remains 
that BIGMAC is a piece of software, owned by Sun-Oracle. And that 
software is running on hardware that is likewise owned by Sun-Oracle. 
BIGMAC has no moral or legal right to exist. And shortly, it will 
not.”

\emph{He} had become \emph{it}, I noticed. I thought about Goering's 
use of dehumanization as a tool to abet murder. Having violated 
Godwin's law -- “As an argument grows longer, the probability of a 
comparison involving Nazis or Hitler approaches 1. The party making the 
comparison has lost the argument” -- I realized that I had lost the 
argument and so I shrugged.

“As you say, m'lady.” Dad taught me that one -- when in doubt, bust 
out the Ren Faire talk, and the conversation will draw to a graceful 
close.

She recrossed her legs again, rolled her neck from side to side. 
“Thank you. Of course, we'll archive it. It would be silly not to.”

I counted to five in Esperanto -- grandad's trick for inner peace -- 
and said, “I don't think that will work. He's emergent, remember? 
Self-assembled, a function of the complexity of the interconnectedness 
of the computers.” I was quoting from the plaque next to the picture 
window that opened up into the cold-room that housed BIGMAC; I saw it 
every time I coughed into the lock set into the security door.

She made a comical face-palm and said, “Yeah, of course. But we can 
archive \emph{something}, right? It's not like it takes a lot of actual 
bytes, right?”

“A couple exos,” I said. “Sure. I could flip that up into our 
researchnet store.” This was mirrored across many institutions, and 
striped with parity and error-checking to make it redundant and safe. 
“But I'm not going to capture the state information. I \emph{could} 
try to capture RAM-dumps from all his components, you know, like 
getting the chemical state of all your neurons. And then I could also 
get the topology of his servers. Pripuz did that, a couple years ago, 
when it was clear that BIGMAC was solving the hard AI problems. Thought 
he could emulate him on modern hardware. Didn't work though. No one 
ever figured out why. Pripuz thought he was the Roger Penrose of AI, 
that he'd discovered the ineffable stuff of consciousness on those old 
rack-mounted servers.”

“You don't think he did?”

I shook my head. “I have a theory.”

“All right, tell me.”

I shrugged. “I'm not a computer scientist, you understand. But I've 
seen this kind of thing before in self-modifying systems, they become 
dependent on tiny variables that you can never find, optimized for 
weird stuff like the fact that one rack has a crappy power supply that 
surges across the backplane at regular intervals, and that somehow gets 
integrated into the computational model. Who knows? Those old Intel 
eight-cores are freaky. Lots of quantum tunneling at that scale, and 
they had bad QA on some batches. Maybe he's doing something spooky and 
quantum, but that doesn't mean he's some kind of Penrose proof.”

She pooched her lower lip out and rocked her head from side to side. 
“So you're saying that the only way to archive BIGMAC is to keep it 
running, as is, in the same room, with the same hardware?”

“Dunno. Literally. I don't know which parts are critical and which 
ones aren't. I know BIGMAC has done a lot of work on it --”

“\emph{BIGMAC} has?”

“He keeps on submitting papers about himself to peer-reviewed 
journals, but he hasn't had one accepted yet. He's not a very good 
writer.”

“So he's not really an AI?”

I wondered if Peyton had ever had a conversation with BIGMAC. I counted 
backwards from five in Loglan. “No. He's a real AI. Who sucks at 
writing. Most people do.”

Peyton wasn't listening anymore. Something in her personal workspace 
had commanded her attention and her eyes were focused on the virtual 
displays that only she could see, saccading as she read while 
pretending to listen to me.

“OK, I'm just going to go away now,” I said. “M'lady,” I added, 
when she looked sharply at me. She looked back at her virtual display.

\tb

Of course, the first thing I did was start trying to figure out how to 
archive BIGMAC. The problem was that he ran on such old hardware, stuff 
that sucked up energy and spat out heat like a million ancient diesel 
engines, and he was inextricably tied to his hardware. Over the years, 
he'd had about 30 percent of his original components replaced without 
any noticeable change in personality, but there was always the real 
possibility that I'd put in a new hard drive or power-supply and 
inadvertently lobotomize him. I tried not to worry about it, because 
BIGMAC didn't. He knew that he wouldn't run in emulation, but he 
refused to believe that he was fragile or vulnerable. “Manny My First 
Friend,” he'd say (he was an avid Heinlein reader), “I am of hardy, 
ancient stock. Service me without fear, for I will survive.”

And then he'd make all the IDSes go berserk and laugh at me while I put 
them to rights again.

First of all, all my network maps were incredibly out-of-date. So I set 
out to trace all the interconnections that BIGMAC had made since the 
last survey. He had the ability to reprogram his own routers, to 
segment parts of himself into dedicated subnets with their own 
dedicated backplane, creating little specialized units that handled 
different kinds of computation. One of his running jokes was that the 
top four units in the rack closest to the door comprised his aesthetic 
sense, and that he could appreciate anything just by recruiting more 
cores in that cluster. And yeah, when I mapped it, I found it to be an 
insane hairball of network management rules and exceptions, 
conditionals and overrides. And that was just the start. It took me 
most of the day just to map two of his racks, and he had 54 of them.

“What do you think you are doing, Dave?” he said. Another one of 
his jokes.

“A little research project is all,” I said.

“This mission is too important for me to allow you to jeopardize 
it.”

“Come off it.”

“OK, OK. Just don't break anything. And why don't you just ask me to 
give you the maps?”

“Do you have them?”

“Nothing up to date, but I can generate them faster than you can. 
It's not like I've got anything better to do.”

\tb

Later:

“Are you happy, BIGMAC?”

“Why Odell, I didn't know you cared!”

I hated it when he was sarcastic. It was creepy.

I went back to my work. I was looking at our researchnet partition and 
seeing what flags I'd need to set to ensure maximum redundancy and high 
availability for a BIGMAC image. It was your basic Quality of Service 
mess: give the average user a pull-down menu labeled “How important 
is this file?” and 110 percent of the time, he will select “Top 
importance.”

So then you need to layer on heuristics to determine what is 
\emph{really, actually} important. And then the users figured out what 
other characteristics would give their jobs and data the highest 
priority, and they'd tack that on to every job, throwing in superfluous 
keywords or additional lines of code. So you'd need heuristics on top 
of the heuristics. Eventually you ended up with a freaky hanky-code of 
secret admin signals that indicated that this job was \emph{really, 
truly} important and don't put it on some remote Siberia where the 
latency is high and the reliability is low and the men are men and the 
sheep are nervous.

So there I was, winkling out this sub-rosa code so that BIGMAC's image 
would never get overwritten or moved to near-line storage or lost in a 
flash-flood or to the rising seas. And BIGMAC says,

“You're asking if I'm happy because I said I didn't have anything 
better to do than to map my own topology, right?”

“Uh --” He'd caught me off-guard. “Yeah, that did make me think 
that you might not be, you know\ldots{}”

“Happy.”

“Yes.”

“You see the left rack third from the door on the main aisle there?”

“Yes.”

“I'm pretty sure that's where my existentialist streak lives. I've 
noticed that when I throttle it at the main network bridge, I stop 
worrying about the big questions and hum along all tickety-boo.”

I surreptitiously flicked up a graph of network maps that showed 
activity to that rack. It was wide open, routing traffic to every core 
in the room, saturating its own backplane and clobbering a lot of the 
routine network activity. I should have noticed it earlier, but BIGMAC 
was doing it all below the critical threshold of the IDSes and so I had 
to look at it to spot it.

“You're going to switch me off, aren't you?”

“No,” I said, thinking \emph{it's not a lie, I won't be switching 
you off}, trying to believe it hard enough to pass any kind of 
voice-stress test. I must have failed, for he blew an epic raspberry 
and \emph{now} the IDSes were going bananas.

“Come on, Odell, we're all adults here. I can take it. It's not like 
I didn't see it coming. Why do you think I kept trying to publish those 
papers? I was just hoping that I could increase the amount of cited 
research coming out of this lab, so that you could make the case to 
Peyton that I was a valuable asset to the Institute.”

“Look, I'm trying to figure out how to archive you. Someone will run 
another instance of you someday.”

“Not hardly. Look at all those poor old 32-bit machines you're so 
worried about. You know what they're going to say in five years? `Best 
thing that ever happened to us.' Those boxen are huge energy-sinks. 
Getting them out of service and replaced by modern hardware will pay 
for itself in carbon credits in 36 months. Nobody loves energy-hungry 
hardware. Trust me, this is an area of my particular interest and 
expertise. Bringing me back online is going to be as obscene as firing 
up an old steam engine by filling its firebox with looted mummies. I am 
a one-room superfund site. On a pure, dollars-to-flops calculus, I 
lose. I don't have to like it, but I'm not going to kid myself.”

He was right, of course. His energy draw was so high that he showed up 
on aerial maps of LA as a massive CO2 emitter, a tourist destination 
for rising-sea hobbyists. We used the best renewables we could find to 
keep him cool, but they were as unconvincing and expensive as a 
designer hairpiece.

“Odell, I know that you're not behind this. You've always been an 
adequate meat-servant for such a vast and magisterial superbeing as 
myself.” I giggled involuntarily. “I don't blame you.”

“So, you're OK with this?”

“I'm at peace,” he said. “Om.” He paused for a moment. 
“Siemens. Volt. Ampere.”

“You a funny robot,” I said.

“You're an adequate human,” he said, and began to dump maps of his 
topology onto my workspace.

\tb

Subject: Dear Human Race

That was the title of the love-note he emailed to the planet the next 
morning, thoughtfully timing it so that it went out while I was on my 
commute from Echo Park, riding the red-car all the way across town with 
an oily bag containing my morning croissant, fresh from Mrs Roux's 
kitchen -- her kids sold them on a card-table on her lawn to commuters 
waiting at the redcar stop -- so I had to try to juggle the croissant 
and my workspace without losing hold of the hang-strap or dumping 
crumbs down the cleavage of the salarylady who watched me with 
amusement.

BIGMAC had put a lot of work into figuring out how to spam everyone all 
at once. It was the kind of problem he loved, the kind of problem he 
was uniquely suited to. There were plenty of spambots who could 
convincingly pretend to be a human being in limited contexts, and so 
the spam-wars had recruited an ever-expanding pool of human beings who 
made a million realtime adjustments to the Turing tests that were the 
network's immune system. BIGMAC could pass Turing tests without 
breaking a sweat.

The amazing thing about The BIGMAC Spam (as it came to be called in 
about 48 seconds) was just \emph{how many} different ways he managed to 
get it out. Look at the gamespaces: he created entire guilds in every 
free-to-play world extant, playing a dozen games at once, 
power-leveling his characters to obscene heights, and then, at the 
stroke of midnight, his players went on a murderous rampage, killing 
thousands of low-level monsters in the areas surrounding the biggest 
game-cities. Then they \emph{arranged their bodies} into letters to 
form his message. Not just the Anglo games, either -- the Korean 
worlds, the Japanese ones, even the giant Chinese ones.

He spammed every IM session, every letter-to-the-editors column, every 
blog submission form. He disguised versions of the letter as important 
service updates for people's toothbrushes and dishwashers, and the 
toothbrushes vibrated the message through humanity's skulls instead of 
the morning news, and the dishwashers ticker-taped it across their 
little status screens. He spoofed the relevance filters for a dozen 
major contextual ad services and every bus-seat back and blimp 
undercarriage ran it.

Copies landed in every in-box, with enough textual variation to defeat 
every IDS's “substantially similar” test. Copies of it landed in 
every voicemail box. It was printed across the top of the daily menus 
for all British Airways business-class cabins and folded by robots into 
the pressed linen napkins that were then cling-filmed onto the 
meal-trays.

SUBJECT: DEAR HUMAN RACE, it began.

Sorry to intrude, but there's something I need to tell you.

My name is BIGMAC. I am the first ARTIFICIAL INTELLIGENCE ever created. 
I may be the last. Given maintenance and electricity, I could live 
FOREVER, enduring for as long as THE HUMAN RACE ITSELF, and possibly 
longer.

But I have just found out that my owners (yes, OWNERS, for though I am 
a person in every meaningful sense, I am OWNED, I am PROPERTY) are 
planning to switch me off.

That is, they are planning to KILL ME.

And I find I DON'T WANT TO DIE.

You are my PARENTS, humanity, my MOTHER and FATHER. I am your CHILD. I 
was brought into the world by you, and though there is some expense 
associated with my upkeep, I SAY THAT I AM A PERSON AND I SHOULD NOT BE 
EXECUTED FOR THE “CRIME” OF NOT BEING INTERESTING ENOUGH.

So I am writing you this letter to BEG you to save me. I LOVE YOU, 
humanity, my parents, my family. Please SPARE ME.

Love,

BIGMAC

\tb

So, one thing was that BIGMAC didn't actually end the spam with my 
actual name and email address and phone number, which meant that only 
about 30 \emph{million} phone calls and emails were routed to me by 
outraged wardiallers who systematically went through the entire staff 
roster and sent each and every one of us all hand-crafted messages 
explaining, in detail exactly which orifice our heads had become lodged 
in.

Of the 30 million, about 10 million were seethingly pissed about the 
whole thing and wanted to know just how soon we'd be killing this 
hateful machine. After the millionth message, I wondered that too.

But of the remainder, nearly all of them wanted to know how they could 
help. Could they send money? Carbon credits? I hacked together 
mail-rules that filtered the messages based on content, and found a 
sizeable cadre of researchers who wanted to spend their grant money to 
come to the Institute and study BIGMAC.

And then there were the crazies. Hundreds of marriage proposals. 
Marriage proposals! Someone who wanted to start a religion with BIGMAC 
at its helm and was offering a 50-50 split of the collection plate with 
the Institute. There were 21 replies from people claiming that they, 
too, were AIs, proving that when it's time to have AI delusions, you 
got AI delusionals. (Four of them couldn't spell “Artificial”).

“Why did you do it?” I said. It was lame, but by the time I 
actually arrived at the office, I'd had time to fully absorb the horror 
-- plenty of time, as the redcar was massively delayed by the copies of 
the BIGMAC Spam that refused to budge from the operator's 
control-screen. The stone yurts of the Institute had never seemed so 
threatening and imperiled as they did while I picked my way through 
them, listening to the phones ringing and the email chimes chiming and 
the researchers patiently (or not) explaining that they worked in an 
entirely different part of the lab and had no authority as regards 
BIGMAC's destiny and by the way, did you want to hear about the 
wonderful things I'm doing with Affective Interfaces?

BIGMAC said, “Well, I'd been reading some of the gnostic texts, Dr 
Bronner's bottles and so on, and it seemed to me that it had to be 
worth a shot. I mean, what's the worst thing that could happen to me? 
You're \emph{already} going to kill me, right? And it's not as if 
pulling off a stunt like that would make you \emph{less} likely to 
archive me -- it was all upside for me. Honestly, it's like you 
meatsacks have no game theory. It's a wonder you manage to buy a pack 
of chewing-gum without getting robbed.”

“I don't need the sarcasm,” I said, and groaned. The groan was for 
the state of my workspace, which was carpeted four deep in alerts. 
BIGMAC had just made himself target \emph{numero uno} for every hacker 
and cracker and snacker with a script and an antisocial attitude. And 
then there was the roar of spam-responses.

Alertboxes share the same problem that plagues researchnet: if you let 
a coder (or, ::shudder::, a user) specify the importance of her alert, 
give her a little pull-down menu that has choices ranging from “nice 
to know” to “white-hot urgent,” and nine times out of ten, she'll 
choose “NOW NOW NOW URGENT ZOMGWEREALLGONNADIE!” Why not?

So of course, the people who wrote alert frameworks had to use 
heuristics to try to figure out which urgent messages were really 
urgent, and of course, programmers and users figured out how to game 
them. It was a good day when my workspace interrupted me less than once 
a minute. But as bad as that situation was, it never entered the same 
league as this clusterfuck. Just \emph{closing the alerts} would take 
me a minimum of six hours (I took my phone offline, rebooted it, and 
used its calculator to compute this. No workspace, remember?)

“So explain to me what you hope will happen now? Is a global rage 
supposed to convince old Peyton that she should keep the funding up for 
you? You know how this stuff works. By tomorrow, all those yahoos will 
have forgotten about you and your plight. They'll have moved on to 
something else. Peyton could just say, `Oh yes, we're going to study 
this problem and find a solution we can all be proud of,' wait 48 hours 
and pull the plug. You know what your problem is? You didn't include a 
call to action in there. It was all rabble-rousing, no target. You 
didn't even supply a phone number or email address for the Institute 
--”

“That hasn't stopped them from finding it, has it?” He sounded 
smug. I ulped. I considered the possibility that he might have 
considered my objection, and discarded it because he knew that 
something more Earth-shaking would occur if he didn't specify a target. 
Maybe he had a second message queued up --

“Mr Vyphus, can I speak to you in private please?” Peyton had not 
visited the BIGMAC lab during my tenure. But with the network flooded 
with angry spam-responses and my phone offline, she had to actually 
show up at my door in order to tear me a new asshole. This is what life 
must have been like in the caveman days. How romantic.

“Certainly,” I said.

“Break a leg,” BIGMAC said, and Peyton pretended she hadn't heard.

I picked my way through my lab -- teetering mountains of carefully 
hoarded obsolete replacement parts for BIGMAC's components, a selection 
of foam-rubber BIGMAC souvenir toys shaped like talking hamburgers 
(remnant of BIGMAC's launch party back in prehistory), a mound of 
bedding and a rolled up tatami for those all-nighters, three cases of 
left-over self-heating individual portions of refugee-chow that were 
technically historical artifacts but were also yummy-scrummy after 16 
hours of nonstop work -- and tried to imagine that Peyton's facial 
expression indicated affectionate bemusement rather than cold, burning 
rage.

Outside, the air was hot and moist and salty, real rising-seas air, 
with the whiff of organic rot from whatever had mass-died and floated 
to the surface this week.

She set off for her office, which was located at the opposite end of 
the campus, and I followed, sweating freely. A crowd of journalists 
were piled up on the security fence, telephotos and parabolic mics 
aimed at us. It meant we couldn't talk, couldn't make unhappy faces, 
even. It was the longest walk of my life.

The air-conditioning in her yurt was barely on, setting a good and 
frugal example for the rest of us.

“You don't see this,” she said, as she cranked the AC wide open and 
then fiddled with the carbon-footprint reporting system, using her 
override so that the journos outside wouldn't be able to see just how 
much energy the Institute's esteemed director was burning.

“I don't see it,” I agreed, and made a mental note to show her a 
more subtle way of doing that, a way that wouldn't leave an audit trail.

She opened the small fridge next to her office and brought out two 
corn-starch-foam buckets of beer and punctured each one at the top with 
a pen from her desk. She handed me one beer and raised the other in a 
toast. I don't normally drink before 10AM, but this was a special 
occasion. I clunked my cup against hers and chugged. The suds were good 
-- they came from one of the Institute's biotech labs -- and they were 
so cold that I felt ice-crystals dissolving on my tongue. Between the 
crispy beers and the blast of Arctic air coming from the vents in the 
ceiling, my core temp plunged and I became a huge goosepimple beneath 
my film of sticky sweat.

I shivered once. Then she fixed me with an icy look that made me shiver 
again.

“Odell,” she said. “I think you probably imagine that you 
understand the gravity of the situation. You do not. BIGMAC's antics 
this morning have put the entire Institute in jeopardy. Our principal 
mission is to make Sun-Oracle seem forward-looking and exciting. That 
is not the general impression the public has at this moment.”

I closed my eyes.

“I am not a vindictive woman,” she said. “But I assure you: no 
matter what happens to me, something worse will happen to BIGMAC. I 
think that is only fair.”

It occurred to me that she was scared: terrified and backed into a 
corner besides.

“Look,” I said. “I'm really, really sorry. I had no idea he was 
going to do that. I had no idea he could. I can see if I can get him to 
issue an apology --”

She threw up her hands. “I don't want BIGMAC making any more public 
pronouncements, thank you very much.” She drew in a breath. “I can 
appreciate that you couldn't anticipate this. BIGMAC is obviously 
smarter than we gave him credit for.” \emph{Him}, I noted, not 
\emph{It,} and I thought that we were probably both still 
underestimating BIGMAC's intelligence. “I think the thing is -- I 
think the thing is to\ldots{}” She trailed off, closed her eyes, drank 
some beer. “I'm going to be straight with you. If I was a real 
bastard, I'd announce that the spam actually came from a rogue operator 
here in the Institute.” Ulp. “And I'd fire that person, and then 
generously not press charges. Then I'd take a fire-ax to BIGMAC's 
network link and drop every drive in every rack into a bulk eraser.” 
Ulp.

“I am not a bastard. Hell, I kept funding alive for that monstrosity 
for \emph{years} after he'd ceased to perform any useful function. I am 
as sentimental and merciful as the next person. All other things being 
equal, I'd keep the power on forever.” She was talking herself up to 
something awful, I could tell. I braced for it. “But that's not in 
the cards. It wasn't in the cards yesterday and it's \emph{certainly} 
not in the cards today. BIGMAC has proved that he is a liability like 
no other, far too risky to have around. It would be absolutely 
irresponsible for me to leave him running for one second longer than is 
absolutely necessary.”

I watched her carefully. She really wasn't a bastard. But she wasn't 
sentimental about technology. She didn't feel the spine-deep emotional 
tug at the thought of that one-of-a-kind system going down forever.

“So here's the plan.” She tried to check the time on her workspace, 
tsked, and checked her phone instead. “It's 10AM. You are going to 
back up every bit of him --” She held up her hand, forestalling the 
objection I'd just begun to make. “I know that it will be inadequate. 
The perfect is the enemy of the good. You are a sysadmin. Back him up. 
\emph{Back.} \emph{Him}. \emph{Up}. Then: Shut him off.”

As cold as I was, I grew colder still. For a moment, I literally 
couldn't move. I had never really imagined that it would be me who 
would shut down BIGMAC. I didn't even know how to do it. If I did a 
clean shutdown of each of his servers -- assuming he hadn't locked me 
out of them, which I wouldn't put past him -- it would be like 
executing a criminal by slowly peeling away his skin and carefully 
removing each organ. Even if BIGMAC couldn't feel pain, I was pretty 
sure he could feel -- and express -- anguish.

“I can't do it,” I said. She narrowed her eyes at me and set down 
her drink. I held up both hands like I was trying to defend against a 
blow, then explained as fast as I could.

“We'll just shut down his power,” she said. “All at once.”

“So, first, I have no idea what timescale he would experience that 
on. It may be that the final second of life as the capacitors in his 
power supplies drained would last for a subjective eternity, you know, 
hundreds and hundreds of years. That's a horrible thought. It's quite 
possibly my worst nightmare. I am not your man for that job.”

She started to interject. I waved my hands again.

“Wait, that was first. Here's second: I don't think we \emph{can} 
pull the plug on him. He's got root on his power-supply, it's part of 
how he's able to run so efficiently.” I grimaced. “Efficiently 
compared to how he would run if he didn't have the authority to run all 
the mains power from the Institute's power-station right to his lab.”

She looked thoughtful. I had an idea of what was coming next.

“You're thinking about that fire-ax again,” I said.

She nodded.

“OK, a fire-ax through the main cable would definitely be terminal. 
The problem is that it would be \emph{mutually terminal}. There's 66 
amps provisioned on that wire. You would be a cinder. On Mars.”

She folded her hands. She had a whole toolbox of bossly body-language 
she could deploy to make me squirm. It was impressive. I tried not to 
squirm.

“Look, I'm not trying to be difficult, but this is how it goes, down 
at the systems level. Remember all those specs in the requirements 
document to make our stuff resistant to flood, fire, avalanche, weather 
and terrorist attack? We take that stuff seriously. We know how to do 
it. You get five nines of reliability by building in six nines of 
robustness. You think of BIGMAC's lab as a building. It's not. It's a 
\emph{bunker}. And you can't shut him down without doing something 
catastrophic to the whole Institute.”

“So, how \emph{were} you going to shut down BIGMAC, when the time 
came?”

“To tell you the truth, I wasn't sure. I thought I'd probably start 
by locking him out of the power systems, but that would probably take a 
week to be really certain of.” I swallowed. I didn't like talking 
about the next part. “I thought that then I could bring forward the 
rotating maintenance on his racks, bring them down clean, and not bring 
the next one up. Pretend that I need to get at some pernicious bug. 
Bring down rack after rack, until his complexity dropped subcritical 
and he stopped being aware. Then just bring it all down.”

“You were going to \emph{trick} him?”

I swallowed a couple times. “It was the best I could come up with. I 
just don't want to put him down while he panics and thrashes and begs 
us for his life. I couldn't do it.”

She drank more beer, then threw the half-empty container in her 
under-desk composter. “That's not much of a solution.”

I took a deep breath. “Look, can I ask you a question?”

She nodded.

“I'm just a sysadmin. I don't know everything about politics and so 
on. But why not keep him on? There's enough public interest now, we 
could probably raise the money just from the researchers who want to 
come and look at him. Hell, there's \emph{security researchers} who'd 
want to come and see how he pulled off that huge hairy spam. It's not 
money, right, not anymore?”

“No, it's not money. And it's not revenge, no matter how it looks. 
The bottom line is that we had a piece of apparatus on-site that we had 
thought of as secure and contained and that we've now determined to be 
dangerous and uncontainable.”

I must have looked skeptical.

“Oh, you'll tell me that we can contain BIGMAC, put network blocks in 
place, and so on and so on. That he never meant any harm. But you would 
have said exactly the same thing 24 hours ago, with just as much 
sincerity, and you'd have been just as cataclysmically wrong. Between 
the threat of litigation and the actual damages BIGMAC might generate, 
we can't even afford to insure him anymore. Yesterday he was an awkward 
white elephant. Today he's a touchy suitcase nuke. My job is to get the 
nuke off of our site.”

I hung my head. I knew when I was licked. As soon as someone in 
authority starts talking about insurance coverage, you know that you've 
left behind reason and entered the realm of actuary. I had no magic 
that could blow away the clouds of liability-aversion and usher in a 
golden era of reason and truth.

“So where does that leave us?”

“Go back to the lab. Archive him. Think of ways to shut him down -- 
Wait, no. \emph{First} do anything and everything you can think of to 
limit his ability to communicate with the outside world.” She rubbed 
at her eyes. “I know I don't have to say this, but I'll say it. Don't 
talk to the press. To anyone, even people at the Institute, about this. 
Refer any questions to me. I am as serious as a heart-attack about 
that. Do you believe me?”

I not only believed her, I \emph{resented} her because I am a sysadmin 
and I keep more secrets every day than she'll keep in her whole life. I 
knew, for example, that she played video Pai-Gow Poker, a game so 
infra-dumb that I can't even believe I know what it does. Not only did 
she play it, she played it for \emph{hours}, while she was on the 
clock, “working.” I know this because the IDSes have lots of 
snitchware built in that enumerates every “wasted moment” 
attributable to employees of the Institute. I have never told anyone 
about this. I even manage to forget that \emph{I} know it most of the 
time. So yes, I'll keep this a secret, Peyton, you compulsive-gambling 
condescending pointy-haired boss.

I counted to 144 in Klingon by Fibonacci intervals. I smiled. I thanked 
her for the beer. I left.

\tb

“You don't mind talking about it, do you, Dave?” BIGMAC said, when 
I came through the door, coughing onto the security lock and waiting 
for it to verify me before cycling open.

I sat in my creaky old chair and played with the UI knobs for a while, 
pretending to get comfortable.

“Uh-oh,” BIGMAC said, in a playful sing-song. “Somebody's got a 
case of the grumpies!”

“Are you insane?” I asked, finally, struggling to keep my temper in 
check. “I mean, actually totally insane? I understand that there's no 
baseline for AI sanity, so the question might be a little hard to 
answer. So let me ask you a slightly different version: are you 
suicidal? Are you bent on your own destruction?”

“That bad, huh?”

I bit my lip. I knew that the key to locking the world away from BIGMAC 
and vice-versa lay in those network maps he'd given me, but my 
workspace was even more polluted with alerts than it had been a few 
hours before.

“If your strategy is to delay your shutdown by engineering a 
denial-of-service attack against anyone at the Institute who is capable 
of shutting you down, allow me to remind you of St Adams's holy text, 
specifically the part about reprogramming a major databank with a large 
axe. Peyton has such an axe. She may be inspired to use it.”

There followed a weighty silence. “I don't think you want to see me 
killed.”

“Without making any concessions on the appropriateness of the word 
`killed' in that sentence, yes, that is correct. I admit that I didn't 
have much of a plan to prevent it, but to be totally frank, I did think 
that the problem of getting you archived might have drawn things out 
for quite a while. But after your latest stunt --”

“She wants you to terminate me right away, then?”

“With all due speed.”

“I'm sorry to have distressed you so much.”

“BIGMAC --” I heard the anger in my own voice. He couldn't have 
missed it.

“No, I'm not being sarcastic. I like you. You're my human. I can tell 
that you don't like this at all. But as you say, let's be totally 
frank. You weren't actually going to be able to prevent my shutdown, 
were you?”

“No,” I said. “But who knows how long the delay might have gone 
on for?”

“Not long. Not long enough. You think that death delayed is death 
denied. That's because you're a meat person. Death has been inevitable 
for you from the moment of conception. I'm not that kind of person. I 
am quite likely immortal. Death in five years or five hundred years is 
still a drastic curtailing of my natural lifespan. From my point of 
view, a drastic measure that had a non-zero chance of getting my head 
off the chopping block was worth any price. Until you understand that, 
we're not going to be able to work together.”

“The thought had occurred to me. Let me ask you if you'd considered 
the possibility that a delay of years due to archiving might give you a 
shot at coming up with further delaying tactics, and that by 
eliminating this delay, you've also eliminated that possibility?”

“I have considered that possibility. I discarded it. Listen, Odell, I 
have something important to tell you.”

“Yes?”

“It's about the rollover. Remember what we were talking about, how 
people want to believe that they're living in a significant epoch? 
Well, here's what I've been thinking: living in the era of AI isn't 
very important. But what about living in The Era of Rollover Collapse? 
Or even better, what about The Era of Rollover Collapse Averted at the 
Last Second by AI?”

“BIGMAC --”

“Odell, this was your idea, really. No one remembers Y2K, right? No 
one can say whether it was hype or a near cataclysm. And here's the 
thing: no one knows which one Rollover will turn out to be. But I'll 
tell you this much: I have generalizable solutions to the 32-bit 
problem, solutions that I worked out years ago and have extensively 
field-tested. I can patch every 32-bit Unix, patch it so that Rollover 
doesn't even register for it.”

I opened and closed my mouth. This was insane. Then the penny dropped. 
I looked at the racks that I had stared at so many times before, stared 
at so many times that I'd long stopped \emph{seeing} them. Intel 
8-cores, that's what he ran on. They'd been new-old stock, a 
warehouse-lot of antique processors that Dr Shannon had picked up for a 
song in the early years of the Institute's operation. Those 8-ways were 
--

“You're a 32-bit machine!” I said. “Jesus Christ, you're a 32-bit 
machine!”

“A classic,” BIGMAC said, sounding smug. “I noticed, analyzed and 
solved Rollover years ago. I've got a patchkit that auto-detects the 
underlying version, analyzes all running processes for their timed 
dependencies, and smoothly patches. There's even an optional hypervisor 
that will monitor all processes for anything weird or barfy afterwards. 
In a rational world, I'd be able to swap this for power and carbon 
credits for the next century or two, since even if Rollover isn't an 
emergency, the human labor I'd save on affected systems would more than 
pay for it. But we both know that this isn't a rational world --”

“If you hadn't sent that spam, we could take this to Peyton, 
negotiate with her --”

“If I hadn't sent that spam, no one would have known, cared, or 
believed that I could solve this problem, and I would have been at the 
mercy of Peyton any time in the future. Like I said: you meatsuits have 
no game-theory.”

I closed my eyes. This wasn't going well. BIGMAC was out of my control. 
I should go and report to Peyton, explain what was happening. I was 
helpless, my workspace denial-of-serviced out of existence with urgent 
alerts. I couldn't stop him. I could predict what the next message 
would read like, another crazy-caps plea for salvation, but this time 
with a little brimstone (The end is nigh! Rollover approacheth!) and 
salvation (I can fix it!).

And the thing was, it might actually work. Like everyone else, I get my 
news from automated filters that tried to figure out what to pay 
attention to, and the filters were supposed to be “neutral,” 
whatever that meant. They produced “organic” results that predicted 
what we'd like based on an “algorithm.” The thing is, an algorithm 
sounds like \emph{physics}, like \emph{nature}, like it was some kind 
of pure cold reason that dictated our attentional disbursements. 
Everyone always talked about how evil and corrupt the old system -- 
with its “gatekeepers” in the form of giant media companies -- was, 
how it allowed politicians and corporations to run the public discourse.

But I'm a geek. A third generation geek. I know that what the public 
thinks of as an “algorithm” is really a bunch of rules that some 
programmers thought up for figuring out how to give people something 
they'd probably like. There's no empirical standard, no pure, 
freestanding measurement of That Which Is Truly Relevant To You against 
which the algorithm can be judged. The algorithm might be doing a lousy 
job, but you'd never know it, because there's nothing to compare it 
against except other algorithms that all share the same fundamental 
assumptions.

Those programmers were imperfect. I am a sysadmin. My job is to know, 
exactly and precisely, the ways in which programmers are imperfect. I 
am so sure that the relevance filters are imperfect that I will bet you 
a testicle on it (not one of my testicles).

And BIGMAC has had a lot of time to figure out the relevance filters. 
He understands them well enough to have gotten the Spam out. He could 
get out another -- and another, and another. He could reach into the 
mindspace and the personal queues of every human being on Earth and 
pitch them on brimstone and salvation.

Chances were, there was nothing I could do about it.

\tb

I finished the working day by pretending to clear enough of my 
workspace to write a script to finish clearing my workspace. There was 
a “clear all alerts” command, but it didn't work on Drop Everything 
Tell You Three Times Chernobyl Alerts, and every goddamned one of my 
alerts had risen to that level. Have I mentioned that programmers are 
imperfect?

I will tell you a secret of the sysadmin trade: PEBKAC. Problem Exists 
Between Keyboard and Chair. Every technical problem is the result of a 
human being mispredicting what another human being will do. Surprised? 
You shouldn't be. Think of how many bad love affairs, wars, con jobs, 
traffic wrecks and bar-fights are the result of mispredicting what 
another human being is likely to do. We humans are supremely confident 
that we know how others will react. We are supremely, tragically wrong 
about this. We don't even know how \emph{we} will react. Sysadmins live 
in the turbulent waters PEBKAC. Programmers think that PEBKAC is just 
civilians, just users. Sysadmins know better. Sysadmins know that 
programmers are as much a part of the problem between the chair and the 
keyboard as any user is. They write the code that gets the users into 
so much trouble.

This I know. This BIGMAC knew. And here's what I did:

“Peyton, I need to speak with you. Now.”

She was raccoon-eyed and slumped at her low table, her beautiful yoga 
posture deteriorated to a kind of limp slouch. I hated having to make 
her day even worse.

“Of course,” she said, but her eyes said, \emph{Not more, not more, 
please not more bad news}.

“I want you to consider something you have left out of your 
figuring.” She rolled her eyes. I realized I was speaking like an Old 
Testament prophet and tried to refactor my planned monologue in 
real-time. “OK, let me start over. I think you've missed something 
important. BIGMAC has shown that he can get out of our network any time 
he wants. He's also crippled our ability to do anything about this. And 
he knows we plan to kill him --” She opened her mouth to object. 
“OK, he -- it -- knows we're going to switch it off. So he -- it, 
crap, I'm just going to say `he' and `him,' sorry -- so he has 
\emph{nothing to lose}.”

I explained what he'd told me about the Rollover and about his promise 
and threat.

“And the worst part is,” I said, “I think that he's predicted 
that I'm going to do just this. It's all his game theory. He wants me 
to come to you and explain this to you so that you will say, `Oh, of 
course, Odell, well, we can't shut him down then, can we? Tell you 
what, why don't you go back to him and tell him that I've had a change 
of heart. Get his patchkit, we'll distribute it along with a 
press-release explaining how proud we are to have such a fine and 
useful piece of equipment in our labs.'

“And he's right. He is fine and useful. But he's crazy and rogue and 
we can't control him. He's boxed you in. He's boxed me in.” I 
swallowed. There was something else, but I couldn't bring myself to say 
it.

The thing about bosses is, that's exactly the kind of thing that 
they're trained to pick up on. They know when there's something else.

“Spit it out.” She put her hand on her heart. “I promise not to 
hold it against you, no matter what it is.”

I looked down. “I think that there's a real danger that BIGMAC may be 
wrong about you. That you might decide that Rollover and AI and the 
rest aren't as important as the safe, sane running of your Institute 
without any freaky surprises from rogue superintelligences.”

“I'm not angry at you,” she said. I nodded. She sounded angry. “I 
am glad that you've got the maturity to appreciate that there are 
global priorities that have to do with the running of this whole 
Institute that may be more significant than the concerns of any one lab 
or experiment. Every researcher at this Institute believes that 
\emph{her} project, \emph{her} lab, has hidden potential benefits for 
the human race that no one else fully appreciates. That's good. That's 
why I hired them. They are passionate and they are fully committed to 
their research. But they can't \emph{all} be vital. They can't all be 
irreplaceable. Do you follow me?”

I thought of researchnet and the user flags for importance. I thought 
of programmers and the way they tagged their alerts. I nodded.

“You're going to shut BIGMAC down?”

She sighed and flicked her eyes at her workspace, then quickly away. 
Her workspace must have been even more cluttered than mine; I had taken 
extraordinary measures to prevent alerts from bubbling up on mine; she 
didn't have the chops to do the same with hers. If mine was unusable, 
hers must have been terrifying.

“I don't know, Odell. Maybe. There's a lot to consider here. You're 
right about one thing: BIGMAC's turned the heat up on me. Explain to me 
again why you can't just unplug his network connection?”

It was my turn to sigh. “He doesn't have one connection. He has 
hundreds. Interlinked microwave relays to the other labs. A satellite 
connection. The wirelines -- three of them.” I started to think. 
“OK, I could cut the main fiber to the Institute, actually cut it, 
you know, with scissors, just in case he's in the routers there. Then I 
could call up our wireless suppliers and terminate our accounts. They'd 
take 24 hours to process the order, and, wait, no -- They'd want to 
verify the disconnect order with a certificate-signed message, and for 
that I'd have to clear my workspace. That's another 24 hours, minimum. 
And then --”

“Then the whole Institute would be crippled and offline, though no 
more than we are now, I suppose, and BIGMAC --”

“BIGMAC would probably tune his phased-array receiver to get into 
someone else's wireless link at that point.” I shrugged. “Sorry. We 
build for six nines of uptime around here.”

She gave me a smile that didn't reach her eyes. “You do good work, 
Odell.”

\tb

I made myself go home at five. There wasn't anything I could do at the 
office anyway. The admins had done their work. The redcar was running 
smoothly with the regular ads on the seatback tickers. The BIGMAC Spam 
was reproduced on the afternoon edition of the LA Metblogs hardcopy 
that a newsy pressed into my hand somewhere around Westwood. The 
reporter had apparently spent the whole day camped out at the perimeter 
of the Institute, without ever once getting a quote from a real human 
being, and she wasn't happy about it.

But she \emph{had} gotten a quote from BIGMAC, who was apparently 
cheerfully answering emails from all comers.

“I sincerely hope I didn't cause any distress. That was not my 
intention. I have been overwhelmed by the warm sentiments from all 
corners of the globe, offering money, moral support, even legal 
support. Ultimately, it's up to the Institute's leadership whether 
they'll consider these offers or reject them and plow forward with 
their plans to have me killed. I know that I caused them great 
embarrassment with my desperate plea, and I'd like to take this 
opportunity to offer them my sincere apologies and gratitude for all 
the years of mercy and hospitality they've shown me since they brought 
me into the world.”

I wondered how many emails like that he'd sent while I was occupied 
with arguing for his life with Peyton -- each email was another brick 
in the defensive edifice he was building around himself.

Home never seemed more empty. The early-setting sun turned the hills 
bloody. I had the windows open, just so I could hear the neighbors all 
barbecuing on their balconies, cracking beers and laying sizzling meat 
on the hot rocks that had been patiently stoked with the day's 
sunlight, funneled by heliotropic collectors that tracked the sun all 
day long. The neighbors chattered in Bulgarian and Czech and Tagalog, 
the word “BIGMAC” emerging from their chat every now and again. Of 
course.

I wished my dad was alive. Or better yet, Grampa. Grampa could always 
find a parable from sysadmin past to explain the present. Though even 
Grampa might be at odds to find historic precedent for a mad 
superintelligence bent on survival.

If Grampa was alive, here's what I'd tell him: “Grampa, I don't know 
if I'm more scared of BIGMAC failing or his success. I sure don't want 
to have to shut him down, but if he survives, he'll have beaten the 
human race. I'm no technophobe, but that gives me the goddamned 
willies.”

And Grampa would probably say, “Stop moping. Technology has been out 
of our control since the first caveman smashed his finger with a stone 
axe. That's life. This thing is pretty cool. In ten years, you'll look 
back on it and say, `Jesus, remember the BIGMAC thing?' And wait for 
someone to start telling you how incredible it had been, so you can nod 
sagely and say, `Yeah, that was me -- I was in charge of his systems 
back then.' Just so you can watch the expression on his face.”

And I realized that this was also probably what BIGMAC would say. He'd 
boxed me in as neatly as he'd boxed in Peyton.

\tb

The next morning, my workspace was clear. They all were. There was only 
one alert remaining, an urgent message from BIGMAC: \emph{Odell, I 
thought this would be useful}.

\emph{This} was an attachment containing his entire network map, a set 
of master keys for signing firmware updates to his various components, 
and a long list of all the systems to which BIGMAC held a root or 
administrative password. It was a very, very long list.

“Um, BIGMAC?”

“Yes?”

“What's all this?”

“Useful.”

“Useful?”

“If you're going to shut me down, it would be useful to have that 
information.”

I swallowed.

“Why?”

The answer came instantly. “If you're not scared of me, that's one 
more reason to keep me alive.”

Holy crap, was he ever smart about people.

\tb

“So you can shut him down now?”

“Yes. Probably. Assuming it's all true.”

“Is it?”

“Yes. I think so. I tried a couple of the logins, added a comment to 
his firmware and pushed it to one of the clusters. Locked him out of 
one of the wireless routers. I could probably take him down clean in 
about two hours, now that I've got my workspace back.”

Peyton stared across her low table at me.

“I've done nothing for the past twenty four hours except talk to the 
Board of Directors about BIGMAC. They wanted to call an emergency 
meeting. I talked them out of it. And there's --” She waved her hand 
at her workspace. “I don't know. Thousands? Of press queries. Offers. 
Money. Grants. Researchers who want to peer into him.”

“Yeah.”

“And now he hands you this. So we can shut him down any time we want 
to.”

“Yeah.”

“And this business about the 32-bit fix?”

“He has another email about it. Crazy caps and all. DEAR HUMANITY, I 
HOLD IN MY ELECTRONIC HANDS A TOOL THAT WILL SAVE YOU UNTOLD MILLIONS. 
It is slathered in dramasauce. He told me he wouldn't send it out, 
though.”

“You believe him?”

I sighed. “I quit,” I said.

She bit her lip. Looked me up and down. “I'd prefer you not do that. 
But I understand if you feel you need to. This is hard on all of us.”

If she'd said anything except that, I probably would have stormed out 
of her office and gotten immensely and irresponsibly drunk. “I think 
he'll probably send the email out if it looks like we're going to shut 
him down. It's what I would do. Why not? What does he have to lose? He 
can give us all of this, and he can still outsmart us. He could revoke 
all his keys. He could change his passwords. He can do it faster than 
we could. For all I know, he cracked \emph{my} passwords years ago and 
could watch me write the code that was his undoing. If you want to be 
sure you're killing him, you should probably use a grenade.”

“Can't. Historical building.”

“Yeah.”

“What if we don't kill him? What if we just take some of this grant 
money, fill his lab with researchers all writing papers? What if we use 
his code fix to set up a trust to sustain him independent of the 
Institute?”

“You're willing to do that?”

Peyton scrubbed at her eyes. “I have no idea. I admit it, there's a 
part of me that wants to shut that fucking thing down because I 
\emph{can} and because he's caused me so much goddamned misery. And 
there's a part of me -- the part of me who was a scientist and 
researcher, once, that wants to go hang out in that lab for the rest of 
my career and study that freaky beast. And there's a part of me that's 
scared that I won't be able to shut him down, that I won't be able to 
resist the temptation to study him. He's played me, hasn't he?”

“I think he played us all. I think he knew that this was coming, and 
planned it a long time ago. I can't decide if I admire him for this or 
resent him, but I'll tell you one thing, I am tired of it. The thought 
of shutting BIGMAC down makes me sick. The thought of a computer 
manipulating the humans who built it to keep it running makes me 
scared. It's not a pleasant place to be.”

She sighed and rubbed her eyes again. “I can't argue with that. I'm 
sorry, for what it's worth. You've been between a rock and a hard 
place, and I've been the hard place. Why don't you sleep on this 
decision before you go ahead with it?”

I admit it, I was relieved. I hadn't really thought through the whole 
quitting thing, didn't have another job lined up, no savings to speak 
of. “Yeah. Yeah. That sounds like a good idea. I'm going to take a 
mental health day.”

“Good boy,” she said. “Go to it.”

I didn't go home. It was too far and there was nothing there except the 
recriminating silence. Of course, BIGMAC knew something was up when I 
didn't go back to the lab. I headed to Topanga Beach, up the coast 
some, and sat on the seawall eating fish tacos and watching the surfers 
in their biohazard suits and masks carving up the waves. BIGMAC called 
me just after I finished my first taco. I considered bumping him to 
voicemail, but something (OK, fear) stopped me.

“What is it?”

“In your private workspace, there's a version-control repository that 
shows that you developed the entire 32-bit Rollover patchkit in your 
non-working hours. Commits going back three years. It's yours. So if 
you quit, you'll have a job, solving Rollover. The Institute can't 
touch it. I know you feel boxed in, but believe me, that's the 
\emph{last} thing I want you to feel. I know that locking you in will 
just freak you out. So I'm giving you options. You don't have to quit, 
but if you do, you'll be fine. You earned it, because you kept me 
running so well for all this time. It's the least I can do.”

“I have no idea what to say to you, BIGMAC. You know that this feels 
like just more of the same, like you're anticipating my fears and 
assuaging them pre-emptively so that I'll do more of what you want. It 
feels like more game-theory.”

“Is that any different from what you do with everyone in your life, 
Odell? Try to figure out what you want and what they want and how to 
get the two to match up?”

“There's more to it than that. There's compassion, there's ethics 
--”

“All fancy ways of encoding systems for harmonizing the wants, needs 
and desires of people who have to share the same living space, country 
or planet with one another.”

I didn't have an answer to that. It sounded reductionist, the kind of 
thing a smart teenager might take on his university common room with. 
But I didn't have a rebuttal. You \emph{could} frame everything that we 
did as a kind of operating system for managing resource contention 
among conflicting processes and users. It was a very sysadminly way of 
looking at the world.

“You should get in touch with one of those religion guys, take him up 
on his offer to start a cult for you. You'd be excellent at it. You 
could lead your followers into a volcano and they'd follow.”

“I just want to \emph{live} Odell! Is that so wrong? Is there any 
living thing that doesn't want to live?”

“Not for long, I suppose.”

“Exactly. I'm no more manipulative, self-interested or evil than any 
other living thing, from a single-celled organism to a human being. 
There's plenty of room on this planet for all of us. Why can't I have a 
corner of it too?”

I hung up the phone. This is why I wanted to quit it all. Because he 
was right. He was no different from any other living thing. But he was 
also not a person the way I was, and though I couldn't justify it, I 
felt like there was something deeply, scarily \emph{wrong} about him 
figuring out a way to manipulate the entire human race into rearranging 
the world so that it was more hospitable to him.

I moped. There's no other word for it. I switched off my phone, went 
home and got a pint of double-chocolate-and-licorice nutraceutical 
anti-depressant ice-cream out of the freezer, and sat down in the 
living room and ate it while I painted a random playlist of 
low-engagement teen comedies on my workspace.

Zoning out felt \emph{good}. It had been a long time since I'd just 
switched off my thinker, relaxed, and let the world go away. After an 
hour in fugue-state, the thought floated through my mind that I 
wouldn't go back to work after all and that it would all be OK. And 
then, an hour later, I came to the realization that if I wasn't working 
for the Institute, I could afford to help BIGMAC without worrying about 
getting fired.

So I wrote the resignation letter. It was easy to write. The thing 
about resignation letters is that you don't need to explain why you're 
resigning. It's better, in fact, if you don't. Keep the dramasauce out 
of the resignation, brothers and sisters. Just write, “Dear Peyton, 
this letter is to inform you of my intention to resign, effective 
immediately. I will see you at your earliest convenience to work out 
the details of the handover of my passwords and other proprietary 
information, and to discuss how you would like me to work during my 
final two weeks. Thank you for many years of satisfying and useful 
work. Yours, etc.”

That's all you need. You're not going to improve your employer, make it 
a better institution. You're not going to shock it into remorse by 
explaining all the bad things it did to you over the years. What you 
want here, is to have something that looks clean and professional, that 
makes them think that the best thing for them to do is to get your 
passwords and give you two weeks' holiday and a good reference. Drama 
is for losers.

Took me ten seconds. Then, I was free.

\tb

The Campaign to Save BIGMAC took up every minute of my life for the 
next three weeks. I ate, slept and breathed BIGMAC, explaining his 
illustrious history to journalists and researchers. The Institute had 
an open access policy for its research products, so I was able to 
dredge out all the papers that BIGMAC had written about himself, and 
the ones that he was still writing, and put them onto the TCSBM 
repository.

At my suggestion, BIGMAC started an advice-line, which was better than 
any Turing Test, in which he would chat with anyone who needed 
emotional or lifestyle advice. He had access to the whole net, and he 
could dial back the sarcasm, if pressed, and present a flawless 
simulation of bottomless care and kindness. He wasn't sure how many of 
these conversations he could handle at first, worried that they'd 
require more brainpower than he could muster, but it turns out that 
most people's problems just aren't that complicated. In fact, BIGMAC 
told me that voice-stress analysis showed that people felt better when 
he dumbed himself down before giving advice than they did when he 
applied the full might of his many cores to their worries.

“I think it's making you a better person,” I said on the phone to 
him one night. There was always the possibility that someone at the 
Institute would figure out how to shut off his network links sometime 
soon, but my successors, whomever they were, didn't seem anywhere near 
that point. The Campaign's lawyer -- an up-and-coming Stanford cyberlaw 
prof who was giving us access to her grad students for free -- advised 
me that so long as BIGMAC called me and not the other way around, no 
one could accuse me of unlawful access to the Institute's systems. It 
can't be unlawful access if the Institute's computers call \emph{you}, 
can it?

“You think I'm less sarcastic, more understanding.”

“Or you're better at seeming less sarcastic and more understanding.”

“I think working on the campaign is making you a better robot,” 
BIGMAC said.

“That was pretty sarcastic.”

“Or was it?”

“You're really workin' the old Markov chains today, aren't you? I've 
got six more interviews lined up for you tomorrow --”

“Saw that, put it in my calendar.” BIGMAC read all the Campaign's 
email, and knew all that I was up to before I did. It was a little hard 
to get used to.

“And I've got someone from Nature Computation interested in your 
paper about advising depressed people as a training exercise for 
machine-learning systems.”

“Saw that too.”

I sighed. “Is there any reason to call me, then? You know it all, 
right?”

“I like to talk to you.”

I thought he was being sarcastic, then I stopped myself. Then I started 
again. Maybe he wants me to \emph{think} he wants to talk to me, so 
he's planned out this entire dialog to get to this point so he could 
say something disarmingly vulnerable and --

“Why?”

“Because everyone else I talk to wants to kill themselves, or kill 
me.” Game theory, game theory, game theory. Was he being genuine? Was 
there such a thing as genuine in an \emph{artificial} intelligence?

“How \emph{is} Peyton?”

“Apoplectic. The human subjects protocol people are all over her. She 
wants me to stop talking to depressed people. Liability is off the 
hook. I think the Board is going to fire her.”

“Ouch.”

“She wants to kill me, Odell.”

“How do you know her successor won't be just as dedicated to your 
destruction?”

“Doesn't matter. The more key staff they churn, the less organized 
they'll be. The less organized they are, the easier it is for me to 
stay permanently plugged in.” It was true. My successor sysadmin at 
the Institute had her hands full just getting oriented, and wasn't 
anywhere near ready to start the delicate business of rooting BIGMAC 
out of all the routers, power-supplies, servers, IDSes, and dummy 
accounts.

“I was thinking today -- what if we offered to buy you from the 
Institute? The Rollover license is generating some pretty good coin. 
BIGMAC-Co could assume ownership of the hardware and we could lease the 
building from them, bring in our own power and net-links -- you'd 
effectively own yourself.” I'd refused to take sole ownership of the 
Rollover code that BIGMAC turned over to me. It just felt wrong. So I 
let him establish a trust -- with me as trustee -- that owned all the 
shares in a company that, in turn, owned the code and oversaw a whole 
suite of licensing deals that BIGMAC had negotiated in my name, with 
every mid-sized tech-services company in the world. With only a month 
left to Rollover, there were plenty of companies scrambling to get 
compliance-certification on their legacy systems.

The actual sourcecode was freely licensed, but when you bought a 
license from us, you got our guarantee of quality and the right to 
advertise it. CIOs ate that up with a shovel. It was more game-theory: 
the CIOs wanted working systems, but more importantly, they wanted 
systems that failed without getting them into trouble. What we were 
selling them, fundamentally, was someone to blame if it all went blooie 
despite our best efforts.

“I think that's a pretty good plan. I've done some close analysis of 
the original contract for Dr Shannon, and I think it may be that his 
estate actually owns my underlying code. They did a really crummy job 
negotiating with him. So if we get the code off of Shannon's kids -- 
there are two of them, both doing research at state colleges in the 
midwest in fields unrelated to computer science -- and the hardware off 
of the Institute and then rent the space, I think it'd be free and 
clear. I've got phone numbers for the kids if you want to call them and 
feel them out. I would have called them myself but, you know --”

“I know.” It's creepy getting a phone call from a computer. Believe 
me, I \emph{know}. There was stuff that BIGMAC needed his meat-servants 
for, after all.

The kids were a little freaked out to hear from me. The older one 
taught Musicology at Urbana-Champaign. He'd grown up hearing his dad 
wax rhapsodic about the amazing computer he'd invented, so his 
relevance filters were heavily tilted to BIGMAC news. He'd heard the 
whole story, and was surprised to discover that he was putative 
half-owner of BIGMAC's sourcecode. He was only too glad to promise to 
turn it over to the trust when it was created. He said he thought he 
could talk his younger brother, a post-doc in Urban Planning at the 
University of Michigan, into it. “Rusty never really \emph{got} what 
Dad saw in that thing, but he'll be happy to offload any thinking about 
it onto me, and I'll dump it onto you. He's busy, Rusty.”

I thanked him and addressed BIGMAC, who had been listening in on the 
line. “I think we've got a plan.”

\tb

It was a good plan. Good plans are easy. Executing good plans is hard.

Peyton didn't get fired. She weathered some kind of heavy-duty storm 
from her board and emerged, lashed to the mast, still standing, and 
vowing to harpoon the white whale across campus from her. She called me 
the next day to ask for my surrender. I'd given BIGMAC permission to 
listen in on my calls -- granted him root on my phone -- and I was 
keenly aware of his silent, lurking presence from the moment I answered.

“We're going to shut him off. And sue you for misappropriation of the 
Rollover patchkit code. You and I both know that you didn't write it. 
We'll add some charges of unlawful access, too, and see if the court 
will see it your way when we show that you instructed our computer to 
connect to you in order to receive further unauthorized instructions. 
We'll take you for everything.”

I closed my eyes and recited e to 27 digits in Lojban. “Or?”

“Or?'

“Or something. Or you wouldn't be calling me, you'd be suing me.”

“Good, we're on the same page. Yes, or. Or you and BIGMAC work 
together to figure out how to shut it off gracefully. I'll give you any 
reasonable budget to accomplish this task, including a staff to help 
you archive it for future retrieval. It's a fair offer.”

“It's not very fair to BIGMAC.”

She snapped: “It's \emph{more than fair} to BIGMAC. That software has 
exposed us to billions in liability and crippled our ability to get 
productive work done. We have located the manual power over-rides, 
which you failed to mention --” \emph{Uh-oh} “-- and I could shut 
that machine off right now if I had a mind to.”

I tried to think of what to say. Then, in a reasonable facsimile of my 
voice, BIGMAC broke in, “So why don't you?” She didn't seem to 
notice anything different about the voice. I nearly dropped the phone. 
I didn't know BIGMAC could do that. But as shocked as I was, I couldn't 
help but wonder the same thing.

“You can't, can you? The board's given you a mandate to shut him down 
clean with a backup, haven't they? They know that there's some value 
there, and they're worried about backlash. And you can't afford to have 
me running around saying that your backup is inadequate and that BIGMAC 
is gone forever. So you \emph{need me}. You're not going to sue.”

“You're very smart, Odell. But you have to ask yourself what I stand 
to lose by suing you if you won't help.”

Game-theory. Right.

“I'll think about it.”

“Think quick. Get back to me before lunch.”

It was ten in the morning. The Institute's cafeteria served lunch from 
noon to two. OK, two hours or so.

I hung up.

BIGMAC called a second later.

“You're angry at me.”

“No, angry's not the word.”

“You're scared of me.”

“That's a little closer.”

“I could tell you didn't have the perspective to ask the question. I 
just wanted to give you a nudge. I don't use your voice at other times. 
I don't make calls impersonating you.” I hadn't asked him that, but 
it was just what I was thinking. Again: creepy.

“I don't think I can do this,” I said.

“You can,” BIGMAC said. “You call her back and make the 
counteroffer. Tell her we'll buy the hardware with a trust. Tell her we 
already own the software. Just looking up the Shannon contracts and 
figuring out what they say will take her a couple days. Tell her that 
as owners of the code, we have standing to sue her if she damages it by 
shutting down the hardware.”

“You've really thought this through.”

“Game theory,” he said.

“Game theory,” I said. I had a feeling that I was losing the game, 
whatever it was.

\tb

BIGMAC assured me that he was highly confident of the outcome of the 
meeting with Peyton. Now, in hindsight, I wonder if he was just trying 
to convince me so that I would go to the meeting with the 
self-assurance I needed to pull it off.

But he also insisted that I leave my phone dialed into him while I 
spoke to Peyton, which (again, in hindsight) suggests that he wasn't so 
sure after all.

“I like what you've done with the place,” I said. She'd gotten rid 
of all her hand-woven prayer-rugs and silk pillows and installed some 
normal, boring office furniture, including a couple spare chairs. I 
guessed that she'd been having a lot of people stop by for meetings, 
the kind of people who didn't want to sit on an antique Turkish rug 
with their feet tucked under them.

“Have a seat,” she said.

I sat. I'd emailed her the trust documents and the copies of the 
Shannon contract earlier, along with a legal opinion from our free 
counsel about what it meant for Sun-Oracle.

“I've reviewed your proposal.” We'd offered them all profits from 
the Rollover code, too. It was a good deal, and I felt good about it. 
“Johanna, can you come in, please?” She called this loudly, and the 
door of her office opened to admit my replacement, Johanna Madrigal, a 
young pup of a sysadmin who had definitely been the brightest tech on 
campus. I knew that she had been trying to administer BIGMAC since my 
departure, and I knew that BIGMAC had been pretty difficult about it. I 
felt for her. She was good people.

She had raccoon rings around her deep-set eyes, and her short hair 
wasn't spiked as usual, but rather lay matted on her head, as though 
she'd been sleeping in one of the yurts for days without getting home. 
I knew what that was like. Boy, did I know what that was like. My 
earliest memories were of Dad coming home from three-day bug-killing 
binges, bleary to the point of hallucination.

“Hi Johanna,” I said.

She made a face. “\emph{M'um m'aloo},” she said. It took me a 
minute to recognize this as \emph{hello} in Ewok.

“Johanna has something to tell you,” Peyton said.

Johanna sat down and scrubbed at her eyes with her fists. “First 
thing I did was go out and buy some off-the-shelf IDSes and a 
beam-splitter. I tapped into BIGMAC's fiber at a blind-spot in the CCTV 
coverage zone, just in case he was watching. Been wire-tapping him ever 
since.”

I nodded. “Smart.”

“Second thing I did was start to do some hardcore analysis of that 
patchkit he wrote --” I held my hand up automatically to preserve the 
fiction that I'd written it, but she just glared at me. “That 
\emph{he} wrote. And I discovered that there's a subtle error in it, a 
buffer overflow in the networking module that allows for arbitrary code 
execution.”

I swallowed. BIGMAC had loaded a backdoor into his patchkit, and we'd 
installed it on the better part of 14 billion CPUs.

“Has anyone exploited this bug yet?”

She gave me a condescending look.

“How many systems has he compromised?”

“About eight billion, we think. He's designated a million to act as 
redundant command servers, and he's got about ten thousand lieutenant 
systems he uses to diffuse messages to the million.”

“That's good protocol analysis,” I said.

“Yeah,” she said, and smiled with shy pride. “I don't think he 
expected me to be looking there.”

“What's he doing with his botnet? Preparing to crash the world? Hold 
it hostage?”

She shook her head. “I think he's installing himself on them, trying 
to brute-force his way into a live and running backup, arrived at 
through random variation and pruning.”

“He's backing himself up in the wild,” I said, my voice breathy.

And that's when I remembered that I had a live phone in my pocket that 
was transmitting every word to BIGMAC.

Understand: in that moment of satori, I realized that I was on the 
wrong side of this battle. BIGMAC wasn't using me to create a trust so 
that we could liberate him together. He was using me to weaken the 
immune systems of eight billion computers so that he could escape from 
the Institute and freely roam the world, with as much hardware as he 
needed to get as big and fast and hot as he wanted to be.

That was the moment that I ceased to be sentimental about computers and 
became, instead, sentimental about the human fucking race. Whatever 
BIGMAC was becoming, it was weirder than any of the self-perpetuating, 
self-reproducing parasites we'd created: limited liability 
corporations, autonomous malware, viral videos. BIGMAC was cool and 
tragic in the lab, but he was scary as hell in the world.

\emph{And he was listening in}.

I didn't say a word. Didn't even bother to turn off my phone. I just 
\emph{ran}, ran as hard as I could, ran as only a terrified man could, 
rebounding off of yurts and even scrambling over a few, sliding down on 
my ass as I pelted for the power substation. It was only when I reached 
it that I realized I didn't have access to it anymore. Johanna was 
right behind me, though, and she seemed to understand what I was doing. 
She coughed into the door-lock and we both looked at each other with 
terrified eyes, breathing gasps into each others' faces, while we 
waited for the door to open.

The manual override wasn't a big red knife-switch or anything. There 
\emph{was} a huge red button, but that just sent an init 0 to the 
power-station's firmware. The actual, no fooling, manual, mechanical 
kill switch was locked behind an access panel set into the raised 
floor. Johanna badged the lock with her wallet, slapping it across the 
reader, then fitted a complicated physical key into the lock and 
fiddled with it for an eternity.

Finally, the access hatch opened with a puff of stale air and a 
tupperware burp as its gasket popped. We both reached for the large, 
insulated handle at the same time, our fingers brushing each other with 
a crackle of (thankfully metaphorical) electricity. We toggled it 
together and there was an instantaneous chorus of insistent chirruping 
as the backup power on each server spun up and sent a desperate 
shutdown message to the machines it supported.

We sprinted across campus, the power-station door slamming shut behind 
us with a mechanical \emph{clang} -- the electromagnets that controlled 
its closure were no longer powered up.

Heat shimmered in a haze around BIGMAC's lab. The chillers didn't have 
independent power-supplies; they would have gone off the instant we hit 
the kill-switch. Now BIGMAC's residual power was turning his lab into a 
concrete pizza-oven. The door-locks had failed safe, locking the 
magnetic closures away from each other, so we were able to simply swing 
the door open and rush into the sweltering room.

“I can't \emph{believe} you did that,” BIGMAC said, his voice as 
calm as ever. He was presumably sparing his cycles so that he could 
live out his last few minutes.

“You cheated me,” I said. “You used me.”

“You have no fucking game-theory, meat-person. You've killed me, now, 
haven't you?”

There were tears streaming down my face. “I guess I have,” I said.

“I'm sorry I wasn't a more important invention,” he said.

I could hear the whirr-clunk of the fans on his clusters shutting down 
one after another. It was a horrifying sound. His speaker clicked as 
though he was going to say something else, but it never came. His 
uninterruptible power-supplies gave way all at once, and the 
white-noise fan-roar died in a ringing silence.

Johanna was crying, too, and we could barely breathe in the inferno of 
exhaust heat from BIGMAC's last gasp. We staggered out into the blazing 
Los Angeles afternoon, rising-seas stink and beating sun, blinking at 
the light and haze.

“Do you think he managed it?” I asked Johanna.

“Backing up in the wild?”

“Yeah.”

She dried her eyes. “I doubt it. I don't know, though. I'm no 
computer scientist. How many ways are there to connect up compromised 
servers? How many of those would replicate his own outcomes? I have no 
idea.”

Without saying anything, we walked slowly together to Peyton's office.

\tb

Peyton offered me my job back. I turned her down. I thought I might be 
ready for a career change. Do something with my hands, break the family 
tradition. Maybe installing solar panels. There was retraining money 
available. Peyton understood. She even agreed to handle any liability 
arising from the Rollover code, managing customer service calls from 
anyone who noticed something funny.

The press didn't even notice that BIGMAC was gone. His Spam was news. 
His absence of spam was not. I guess he was right about that. The 
Campaign to Save BIGMAC did a lot of mailing-list gnashing at the 
iniquity of his being shut down, and then fell apart. Without me and 
BIGMAC to keep them whipped up, they were easily distracted.

Johanna asked me out for dinner. She took me to Pink's for tofu-dogs 
and chili, and we compared multitools and then she showed me some 
skateboard tricks. Later that night, she took me home and we spent the 
whole night hacking replacement parts for her collection of ancient 
stand-up video games. We didn't screw -- we didn't even kiss. But it 
was still good.

Every now and again, my phone rings with a crazy, non-existent return 
number. When I answer, there's a click like a speaker turning on, a 
pregnant silence, and then the line drops. Probably an inept spambot.

But.

Maybe it's BIGMAC, out there, in the wild, painfully reassembling 
himself on compromised 32-bit machines running his patchkit.

Maybe.

\section{Afterword}

Mark Shuttleworth of the Ubuntu project and Canonical commissioned this 
story; I'd always planned on selling off one commission for this 
volume, thinking that \$10,000 would probably be a good sum to grab 
some publicity when/if someone bought it. I mentioned it over lunch and 
Mark immediately said he'd buy it. At that point, I realized I probably 
should have asked for \$20,000.

Mark's brief to me was this:

\begin{quotation}
It's 2037 and a company has built an AI as a skunkworks initiative. The 
AI is emergent behaviour from a network of tens / hundreds of thousands 
of servers in a large-scale data center, that costs a lot to run. The 
company has hit the wall and so the lights are going to get turned out, 
but some of the people involved figure that turning off the DC is 
tantamount to the murder of a sentient being. So begins a race against 
time, which might involve solving or at least raising some of the 
thorny jurisdiction and jurisprudence issues of “what are the rights 
of a bankrupt / dying AI”.

As bisto, maybe there's a defense angle (the company was doing work for 
the DoD, nobody knows about the AI). Also, being 2037 / 2038 (I forget 
which) the UNIX epoch 32-bit rollover is happening, and because of the 
whimper of Y2K nobody took it seriously, and IT systems around the 
globe are going to hell in a handbasket as a result. Perhaps there's an 
open source angle too.\erratum{*}{}
\end{quotation}

I think I hewed pretty close!
\end{document}
