\usepackage{url}
\DeclareUrlCommand\url{\def\UrlLeft{\hspace{0pt plus 2ex}}%
        \def\UrlRight{\hspace{0pt plus 2ex}}%
        \urlstyle{tt}}


\hyphenation{mo-no-poly car-ne-gie pro-ject pro-gress mo-dem rou-lette
  browse-wrap Use-net mon-as-tery mo-dems}
\hyphenation{co-me-dic polt-roon stove-pipe Ma-dame scru-ta-ble star-tling}
\hyphenation{heal-thily lim-ou-sines wrest-lers tan-trum push-over un-asked
  bras-siere bro-th-er}
\hyphenation{Can-a-da Fred-rick teen-agers wrest-ler Cha-vez Tho-mas 
  a-nom-a-lies sur-veil-lance ar-mies ref-u-gee ref-u-gees bris-tling
  eve-ning man-chu-ria man-chu-ri-an mid-terms me-di-um jap-a-nese}
\hyphenation{spend-ers googl-ing tour-ist tour-ists leg-end-ary}
\hyphenation{Dan-iel Van-essa Doc-to-row Ste-phen-son}
\hyphenation{de-cade sur-veilled rout-ers Wol-fen-stein teen-ager to-night}
\hyphenation{his-to-gram an-o-nym-ize Ga-la-xy sym-pa-the-tic}
\hyphenation{ar-phid ar-phids Found-ers}
\hyphenation{stran-ger stran-gers shoul-der-blades dump-ling dump-lings}
\hyphenation{ice-pack guard-rail Sep-tem-ber boot-able e-co-nom-ist}
\hyphenation{grown-ups roos-ter shoe-laces li-quid-i-ty}
\hyphenation{side-arm}
\hyphenation{wo-man wo-men tan-trum tan-trums Le-nin-grad zom-bie bunk-house}
\hyphenation{up-tick bio-mass}
\hyphenation{of-fi-cial of-fi-cial-ly gov-ern-ment}
\hyphenation{heal-thy Or-ville spark-ling}
\hyphenation{ves-ti-bule Law-rence au-to-no-mous}
\hyphenation{sau-sage door-step staf-fer}
\hyphenation{tree-trunk}
\hyphenation{to-ron-to}
\hyphenation{qua-dril-lion-aire qua-dril-lion-aires}
\hyphenation{sports-jack-et sports-jack-ets}
\hyphenation{work-space skunk-works}
\hyphenation{kings-ton}


\newcommand{\shopad}[2]{
  \emph{#1}
  \par\smallskip\noindent
  \emph{#2}
  \par\medskip\noindent}

\newcommand\edialog[1]{
{
  \setlength\parindent{0pt}
  \setlength\hangindent{10pt}
  \raggedright
  \textgreater\ \texttt{#1}
  \par
}}

\hyphenation{fin-ger-spit-zen-ge-fuhl Mat-thew Dib-yen-du}

\begin{document}
\raggedbottom
\frontmatter

\title{For the Win}
\author{Cory Doctorow
\thanks{\texttt{doctorow@craphound.com}}}
\date{Last updated 16 Sept 2010}

\maketitle

\section{READ THIS FIRST}

This book is distributed under a Creative Commons
Attribution-NonCommercial-ShareAlike 3.0 license. That means:

You are free:

\begin{itemize}
\item 
  to Share -- to copy, distribute and transmit the work
\item 
  to Remix -- to adapt the work
\end{itemize}
Under the following conditions:
\begin{itemize}
\item 
  Attribution. You must attribute the work in the manner specified by
  the author or licensor (but not in any way that suggests that they
  endorse you or your use of the work).
\item 
  Noncommercial. You may not use this work for commercial purposes.
\item 
  Share Alike. If you alter, transform, or build upon this work, you
  may distribute the resulting work only under the same or similar
  license to this one.
\item 
  For any reuse or distribution, you must make clear to others the
  license terms of this work. The best way to do this is with a link
  \url{http://craphound.com/ftw}
\item 
  Any of the above conditions can be waived if you get my permission
\end{itemize}
More info here:
\url{http://creativecommons.org/licenses/by-nc-sa/3.0/}
See the end of this file for the complete legalese.

\section{INTRODUCTION}

\emph{For the Win} is my second young adult novel, and, like my
2008 book \emph{Little Brother}, it is meant to do more than tell a
story. \emph{For the Win} is a book about economics (a subject that
suddenly got a lot more relevant about halfway through the writing
of this book, when the world's economy slid unceremoniously into
the toilet and got stuck there), justice, politics, games and
labor. \emph{For the Win} connects the dots between the way we
shop, the way we organize, and the way we play, and why some people
are rich, some are poor, and how we seemed to get stuck there.

I hope that readers of this book will be inspired to dig deeper
into the subjects of ``behavioral economics'' (and related subjects
like ``neuroeconomics'') and to start asking hard questions about how
we end up with the stuff we own, and what it costs our human
brothers and sisters to make those goods, and why we think we need
them.

But it's a poor politics that can only express itself by choosing
to buy or not buy something. Sometimes (often!), you need to
organize to make a difference.

This is the golden age of organizing. If there's one thing the
Internet's changed forever, it's the relative difficulty and cost
of getting a bunch of people in the same place, working for the
same goal. That's not always good (thugs, bullies, racists and
loonies never had it so good), but it is fundamentally
\emph{game-changing}.

It's hard to remember just how difficult this organizing stuff used
to be: how hard it was to do something as trivial as getting ten
friends to agree on dinner and a movie, let alone getting millions
of people together to raise money for a political candidate, get
the vote out, protest corruption, or save an endangered and beloved
institution.

The net doesn't solve the problem of injustice, but it solves the
first hard problem of righting wrongs: getting everyone together
and keeping them together. You still have to do the even
\emph{harder} work of risking life, limb, personal fortune,
reputation,

Every wonderful thing in our world has fight in its history. Our
rights, our good fortune, our happiness and all that is sweet was
paid for, once upon a time, by principled people who risked
everything to change the world for the better. Those risks are not
diminished one iota by the net. But the rewards are every bit as
sweet.

\section{AUDIOBOOK}

The good folks at Random House Audio produced a \emph{fantastic}
audio edition of this book. You can buy it on CD, or you can buy
the MP3 version from a variety of online booksellers.
\href{http://craphound.com/?cat=10}{I also sell it myself on my site}

Unfortunately, you can't buy this book from the world's most
popular audiobook vendors: Apple's iTunes and Amazon's Audible.
That's because neither store would allow me to sell the audiobook
on terms that I believe are fair and just.

Specifically, Apple refused to carry the book unless it had
``digital rights management'' on it. This is the technology that
locks music to Apple's devices. It's illegal to move DRM-crippled
files to devices that Apple hasn't blessed, which means that if I
encourage you to buy my works through Apple, I lose the ability to
choose to continue to sell to you from Apple's competition at some
later date in the future. That seems like a bad deal for both of
us.

To its credit, Audible (which supplies all of the audiobooks on
iTunes) \emph{was} willing to sell this book without DRM, but they
insisted on including their extremely onerous ``end user license
agreement,'' which \emph{also} prohibits moving my book to a device
that Audible hasn't approved. To make it easy for them, I offered
to simply record a little intro that said, ``Cory Doctorow and
Random House Audio grant you permission to use this book in any way
that does not violate copyright law.'' That way, they wouldn't have
to make \emph{any} changes to their site or the agreements you have
to click through to use it. But Audible refused.

I wouldn't sell this book through Wal-Mart if they insisted that
you could only shelve it on a Wal-Mart bookcase and I won't sell it
through any online retailer that imposes the same requirement on
your virtual bookshelves. That's also why you won't find my books
for sale for the Kindle or iPad stores -- both stores insist on the
right to lock you into terms that I believe are unfair and bad for
both of us.

I'm pretty bummed about this. For the record, I would gladly sell
through both Apple and Audible if they'd let me sell it without
DRM, and under the world's shortest EULA (``Don't violate copyright
law.'') In the meantime, I thank you in advance for patronizing
online audiobook sellers who respect the rights of both authors and
audiences. And I am especially grateful to Random House Audio for
backing me in this fight to get a fair deal for all of us.

\section{THE COPYRIGHT THING}

The Creative Commons license at the top of this file probably
tipped you off to the fact that I've got some pretty unorthodox
views about copyright. Here's what I think of it, in a nutshell: a
little goes a long way, and more than that is too much.

I like the fact that copyright lets me sell rights to my publishers
and film studios and so on. It's nice that they can't just take my
stuff without permission and get rich on it without cutting me in
for a piece of the action. I'm in a pretty good position when it
comes to negotiating with these companies: I've got a great agent
and a decade's experience with copyright law and licensing
(including a stint as a delegate at WIPO, the UN agency that makes
the world's copyright treaties). What's more, there's just not that
many of these negotiations -- even if I sell fifty or a hundred
different editions of \emph{For the Win} (which would put it in top
millionth of a percentile for fiction), that's still only a hundred
negotiations, which I could just about manage.

I \emph{hate} the fact that fans who want to do what readers have
always done are expected to play in the same system as all these
hotshot agents and lawyers. It's just \emph{stupid} to say that an
elementary school classroom should have to talk to a lawyer at a
giant global publisher before they put on a play based on one of my
books. It's ridiculous to say that people who want to ``loan'' their
electronic copy of my book to a friend need to get a \emph{license}
to do so. Loaning books has been around longer than any publisher
on Earth, and it's a fine thing.

Copyright laws are increasingly passed without democratic debate or
scrutiny. In Great Britain, where I live, Parliament has just
passed the Digital Economy Act, a complex copyright law that allows
corporate giants to disconnect whole families from the Internet if
anyone in the house is accused (without proof) of copyright
infringement; it also creates a ``Great Firewall of Britain'' that is
used to censor any site that record companies and movie studios
don't like. This law was passed without any serious public debate
in Parliament, rushed through using a dirty process through which
our elected representatives betrayed the public to give a huge,
gift-wrapped present to their corporate pals.

It gets worse: around the world, rich countries like the US, the EU
and Canada have been negotiating a secret copyright treaty called
``The Anti-Counterfeiting Trade Agreement'' (ACTA) that has all the
problems that the Digital Economy Act had and then some. The plan
is to agree to this in secret, without public debate, and then
force the world's poorest countries to sign up for it by refusing
to allow them to sell goods to rich countries unless the do. In
America, the plan is to pass it without Congressional debate, using
the executive power of the President. Though this began under Bush,
the Obama administration has pursued it with great enthusiasm.

So if you're not violating copyright law right now, you will be
soon. And the penalties are about to get a lot worse. As someone
who relies on copyright to earn my living, this makes me sick. If
the big entertainment companies set out to destroy copyright's
mission, they couldn't do any better than they're doing now.

So, basically, \emph{screw that}. Or, as the singer, Wobbly and
union organizer Woody Guthrie so eloquently put it:

``This song is Copyrighted in U.S., under Seal of Copyright
\#154085, for a period of 28 years, and anybody caught singin' it
without our permission, will be mighty good friends of ourn, cause
we don't give a dern. Publish it. Write it. Sing it. Swing to it.
Yodel it. We wrote it, that's all we wanted to do.''

\section{DONATIONS AND A WORD TO TEACHERS AND LIBRARIANS}

Every time I put a book online for free, I get emails from readers
who want to send me donations for the book. I appreciate their
generous spirit, but I'm not interested in cash donations, because
my publishers are really important to me. They contribute
immeasurably to the book, improving it, introducing it to audiences
I could never reach, helping me do more with my work. I have no
desire to cut them out of the loop.

But there has to be some good way to turn that generosity to good
use, and I think I've found it.

Here's the deal: there are lots of teachers and librarians who'd
love to get hard-copies of this book into their kids' hands, but
don't have the budget for it (teachers in the US spend around
\$1,200 out of pocket each on classroom supplies that their budgets
won't stretch to cover, which is why I sponsor a classroom at
Ivanhoe Elementary in my old neighborhood in Los Angeles; you can
adopt a class yourself
at \url{http://www.adoptaclassroom.org/}).

There are generous people who want to send some cash my way to
thank me for the free ebooks.

I'm proposing that we put them together.

If you're a teacher or librarian and you want a free copy of
\emph{For the Win}, email
\url{freeftwbook@gmail.com} with
your name and the name and address of your school. It'll be posted
to
\url{http://craphound.com/ftw/donate/}
by my fantastic helper, Olga Nunes, so that potential donors can
see it.

If you enjoyed the electronic edition of \emph{For the Win} and you
want to donate something to say thanks, go to
\url{http://craphound.com/ftw/donate/}
and find a teacher or librarian you want to support. Then go to
Amazon, BN.com, or your favorite electronic bookseller and order a
copy to the classroom, then email a copy of the receipt (feel free
to delete your address and other personal info first!) to
\url{freeftwbook@gmail.com} so that
Olga can mark that copy as sent. If you don't want to be publicly
acknowledged for your generosity, let us know and we'll keep you
anonymous, otherwise we'll thank you on the donate page.

I've done this with three of my titles now, and gotten more than a
thousand books into the hands of readers through your generosity. I
am more grateful than words can express for this -- one of my
readers called it ``paying your debts forward with instant
gratification.'' That's a heck of a thing, isn't it?

\section{ABOUT THE BOOKSTORE DEDICATIONS}

Many scenes in this file have been dedicated to bookstores: stores
that I love, stores that have helped me discover books that opened
my mind, stores that have helped my career along. The stores didn't
pay me anything for this -- I haven't even told them about it --
but it seems like the right thing to do. After all, I'm hoping that
you'll read this ebook and decide to buy the paper book, so it only
makes sense to suggest a few places you can pick it up!

\section{Dedication:}

For Poesy: Live as though it were the early days of a better
nation.

\mainmatter

\chapter*{Part I: The gamers and their games, the workers at their work}

\shopad{This scene is dedicated to BakkaPhoenix Books in Toronto, Canada. Bakka is the oldest science fiction bookstore in the world, and it made me the mutant I am today. I wandered in for the first time around the age of 10 and asked for some recommendations. Tanya Huff (yes,
\emph{the} Tanya Huff, but she wasn't a famous writer back then!) took me back into the used section and pressed a copy of H. Beam Piper's ``Little Fuzzy'' into my hands, and changed my life forever. By the time I was 18, I was working at Bakka -- I took over from Tanya when she retired to write full time -- and I learned life-long lessons about how and why people buy books. I think every writer should work at a bookstore (and plenty of writers have worked at Bakka over the years! For the 30th anniversary of the store, they put together an anthology of stories by Bakka writers that included work by Michelle Sagara (AKA Michelle West), Tanya Huff, Nalo Hopkinson, Tara Tallan --and me!)}
{\href{http://www.bakkaphoenixbooks.com/}{BakkaPhoenix Books}: 697 Queen Street West, Toronto ON Canada M6J1E6, +1 416 963 9993}

In the game, Matthew's characters killed monsters, as they did
every single night. But tonight, as Matthew thoughtfully
chopsticked a dumpling out of the styrofoam clamshell, dipped it in
the red hot sauce and popped it into his mouth, his little squadron
did something extraordinary: they began to \emph{win}.
There were eight monitors on his desk, arranged in two ranks of
four, the top row supported on a shelf he'd bought from an old lady
scrap dealer in front of the Dongmen market. She'd also sold him
the monitors, shaking her head at his idiocy: at a time when
everyone wanted giant, 30'' screens, why did he want this collection
of dinky little 9'' displays?

\emph{So they'd all fit on his desk}.

Not many people could play eight simultaneous games of
Svartalfaheim Warriors. For one thing, Coca Cola (who owned the
game), had devoted a lot of programmer time to preventing you from
playing more than one game on a single PC, so you had to somehow
get eight PCs onto one desk, with eight keyboards and eight mice on
the desk, too, and room enough for your dumplings and an ashtray
and a stack of Indian comic books and that stupid war-axe that Ping
gave him and his notebooks and his sketchbook and his laptop and
--

It was a crowded desk.

And it was noisy. He'd set up eight pairs of cheap speakers, each
glued to the appropriate monitor, turned down low to the normal hum
of Svartalfaheim -- the clash of axes, the roar of ice-giants, the
eldritch music of black elves (which sounded a lot like the demo
programs on the electric keyboards his mother had spent half her
life manufacturing). Now they were all making casino noise,
\emph{pay off} noises, as his raiding party began to clean up. The
gold rolled into their accounts. He was playing trolls -- it was
trolls versus elves in Svartalfaheim, though there was an expansion
module with light elves and some kind of walking tree -- and he'd
come through an instanced dungeon that was the underground lair of
a minor dark elvish princeling. The lair was only medium hard, with
a lot of crappy little monsters early on, then a bunch of dark elf
cannon-fodder to be mown down, some traps, and then the level-boss,
a wizard who had to be taken out by the spell-casters in Matthew's
party while the healers healed them and the tanks killed anything
that tried to attack them.

So far, so good. Matthew had run and mapped the dungeon on his
second night in-world, a quick reccy that showed that he could
expect to do about 400 gold's worth of business there in about 20
minutes, which made it a pretty poor way to earn a living. But
Matthew kept \emph{very} good notes, and among his notes was the
fact that the very last set of guards had dropped some
mareridtbane, which was part of the powerful Living Nightmare spell
in the new expansion module. There were players all over Germany,
Switzerland and Denmark who were buying mareridtbane for 800 gold
per plant. His initial reccy had netted him \emph{five} plants.
That brought the total expected take from the dungeon up to 4,400
gold for 20 minutes, or 13,200 gold per hour -- which, at the day's
exchange, was worth about \$30, or 285 Renminbi.

Which was -- he thought for a second -- more than 71 bowls of
dumplings.

\emph{Jackpot.}

His hands flew over the mice, taking direct control over the squad.
He'd work out the optimal path through the dungeon now, then head
out to the Huoda internet cafe and see who he could find to do runs
with him at this. With any luck, they could take -- his eyes rolled
up as he thought again -- a \emph{million} gold out of the dungeon
if they could get the whole cafe working on it. They'd dump the
gold as they went, and by the time Coca Cola's systems
administrators figured out anything was wrong, they'd have pulled
almost \$3000 out of the game. That was a year's rent, for one
night's work. His hands trembled as he flipped open a notebook to a
new page and began to take notes with his left hand while his right
hand worked the game.

He was just about to close his notebook and head for the cafe -- he
needed more dumplings on the way, could he stop for them? Could he
afford to? But he needed to eat. And coffee. Lots of coffee -- when
the door splintered and smashed against the wall bouncing back
before it was kicked open again, admitting the cold fluorescent
light from outside into his tiny cave of a room. Three men entered
his room and closed the door behind them, restoring the dark. One
of them found the lightswitch and clicked it a few times without
effect, then cursed in Mandarin and punched Matthew in the ear so
hard his head spun around on his neck, contriving to bounce off the
desk. The pain was blinding, searing, sudden.

``Light,'' one of the men commanded, his voice reaching Matthew
through the high-pitched whine of his ringing ear. Clumsily, he
fumbled for the desk-lamp behind the Indian comics, knocked it
over, and then one of the men seized it roughly and turned it on,
shining it full on Matthew's face, making him squint his watering
eyes.

``You have been warned,'' the man who'd hit him said. Matthew
couldn't see him, but he didn't need to. He knew the voice, the
unmistakable Wenjhou accent, almost impossible to understand. ``Now,
another warning.'' There was a \emph{snick} of a telescoping baton
being unfurled and Matthew flinched and tried to bring his arms up
to shield his head before the weapon swung. But the other two had
him by the arms now, and the baton whistled past his ear.

But it didn't smash his cheekbone, nor his collarbone. Rather, it
was the screen before him that smashed, sending tiny, sharp
fragments of glass out in a cloud that seemed to expand in slow
motion, peppering his face and hands. Then another screen went. And
another. And another. One by one, the man dispassionately smashed
all eight screens, letting out little smoker's grunts as he worked.
Then, with a much bigger, guttier grunt, he took hold of one end of
the shelf and tipped it on its edge, sending the smashed monitors
on it sliding onto the floor, taking the comics, the clamshell, the
ashtray, all of it sliding to the narrow bed that was jammed up
against the desk, then onto the floor in a crash as loud as a
basketball match in a glass factory.

Matthew felt the hands on his shoulders tighten and he was lifted
out of his chair and turned to face the man with the accent, the
man who had worked as the supervisor in Mr Wing's factory, almost
always silent. But when he spoke, they all jumped in their seat,
never sure of whether his barely contained rage would break,
whether someone would be taken off the factory floor and then
returned to the dorm that night, bruised, cut, sometimes crying in
the night for parents left behind back in the provinces.

The man's face was calm now, as though the violence against the
machines had scratched the unscratchable itch that made him clench
and unclench his fists at all times. ``Matthew, Mr Wing wants you to
know that he thinks of you as a wayward son, and bears you no ill
will. You are always welcome in his home. All you need to do is ask
for his forgiveness, and it will be given.'' It was the longest
speech Matthew had ever heard the man give, and it was delivered
with surprising tenderness, so it was quite a surprise when the man
brought his knee up into Matthew's balls, hard enough that he saw
stars.

The hands released him and he slumped to the floor, a strange sound
in his ears that he realized after a moment must have been his
voice. He was barely aware of the men moving around his tiny room
as he gasped like fish, trying to get air into his lungs, air
enough to scream at the incredible, radiant pain in his groin.

But he did hear the horrible electrical noise as they tasered the
box that held his computers, eight PCs on eight individual boards,
stuck in a dented sheet-metal case he'd bought from the same old
lady. The ozone smell afterwards sent him whirling back to his
grandfather's little flat, the smell of the dust crisping on the
heating coil that the old man only turned on when he came to visit.
He did hear them gather up his notebooks and tread heavily on the
PC case, and pull the shattered door shut behind them. The light
from the desklamp painted a crazy oval on the ceiling that he
stared at for a long time before he got to his feet, whimpering at
the pain in his balls.

The night guard was standing at the end of the corridor when he
limped out into the night. He was only a boy, even younger than
Matthew -- sixteen, in a uniform that was two sizes too big for his
skinny chest, a hat that was always slipping down over his eyes, so
he had to look up from under the brim like a boy wearing his
father's hat.

``You OK?'' the boy said. His eyes were wide, his face pale.

Matthew patted himself down, wincing at the pain in his ear, the
shooting stabbing feeling in his neck.

``I think so,'' he said.

``You'll have to pay for the door,'' the guard said.

``Thanks,'' Matthew said. ``Thanks so much.''

``It's OK,'' the boy said. ``It's my job.''

Matthew clenched and unclenched his fists and headed out into the
Shenzhen night, limping down the stairs and into the neon glow. It
was nearly midnight, but Jiabin Road was still throbbing with
music, food and hawkers and touts, old ladies chasing foreigners
down the street, tugging at their sleeves and offering them
``beautiful young girls'' in English. He didn't know where he was
going, so he just walked, fast, fast as he could, trying to walk
off the pain and the enormity of his loss. The computers in his
room hadn't cost much to build, but he hadn't had much to begin
with. They'd been nearly everything he owned, save for his comics,
a few clothes -- and the war-axe. Oh, the war-axe. That was an
entertaining vision, picking it up and swinging it over his head
like a dark elf, the whistle of its blade slicing the air, the
meaty \emph{thunk} as it hit the men.

He knew it was ridiculous. He hadn't been in a fight since he was
ten years old. He'd been a \emph{vegetarian} until last year! He
wasn't going to hit anyone with a war axe. It was as useless as his
smashed computers.

Gradually, he slowed his pace. He was out of the central area
around the train station now, in the outer ring of the town center,
where it was dark and as quiet as it ever got. He leaned against
the steel shutters over a grocery market and put his hands on his
thighs and let his sore head droop.

Matthew's father had been unusual among their friends -- a
Cantonese who succeeded in the new Shenzhen. When Premier Deng
changed the rules so that the Pearl River Delta became the world's
factory, his family's ancestral province had filled overnight with
people from the provinces. They'd ``jumped into the sea'' -- left
safe government factory jobs to seek their fortune here on the
south Chinese coast -- and everything had changed for Matthew's
family. His grandfather, a Christian minister who'd been sent to a
labor camp during the Cultural Revolution -- had never made the
adjustment, a problem that struck many of the native Cantonese, who
seemed to stand still as the outsiders raced past them to become
rich and powerful.

But not Matthew's father. The old man had started off as a driver
for a shoe-factory boss -- learning to drive on the job, nearly
cracking up the car more than once, though the owner didn't seem to
mind. After all, he'd never ridden in a car before he'd made it big
in Shenzhen. But he got his break one day when the pattern-maker
was too sick to work and all production ceased while the girls who
worked on the line argued about the best way to cut the leather for
a new order that had come in.

Matthew's father loved to tell this story. He'd heard the argument
go back and forth for a day as the line jerked along slowly, and
he'd sat on his chair and thought, and thought, and then he'd stood
up and closed his eyes and pictured the calm ocean until the
thunder of his heartbeat slowed to a normal beat. Then he'd walked
into the owner's office and said, ``Boss, I can show you how to cut
those hides.''

It was no easy task. The hides were all slightly different shapes
-- cows weren't identical, after all -- and parts of them were
higher grade than others. The shoe itself, an Italian men's loafer,
needed six different pieces for each side, and only some of them
were visible. The parts that were inside the shoe didn't need to
come from the finest leather, but the parts outside did. All this
Matthew's father had absorbed while sitting in his chair and
listening to the arguments. He'd always loved to draw, always had a
good head for space and design.

And before his boss could throw him out of the office, he'd plucked
up his courage and seized a pen off the desk and rooted a crumpled
cigarette package out of the trash -- expensive foreign cigarettes,
affected by all the factory owners as a show of wealth -- torn it
open and drawn a neat cowhide, and quickly shown how the shoes
could be fit to the hide with a minimum of wastage, a design that
would get ten pairs of shoes per hide.

``Ten?'' the boss said.

``Ten,'' Matthew's father said, proudly. He knew that the most that
Master Yu, the regular cutter, ever got out of a hide was nine.
``Eleven, if you use a big hide, or if you're making small shoes.''

``You can cut this?''

Now, before that day, Matthew's father had never cut a hide in his
life, had no idea how to slice the supple leather that came back
from the tanner. But that morning he'd risen two hours early,
before anyone else was awake, and he'd taken his leather jacket, a
graduation present from his own father that he'd owned and
treasured for ten years, and he'd taken the sharpest knife in the
kitchen, and he'd sliced the jacket to ribbons, practicing until he
could make the knife slice the leather in the same reliable,
efficient arcs that his eyes and mind could trace over them.

``I can try,'' he said, with modesty. He was nervous about his
boldness. His boss wasn't a nice man, and he'd fired many employees
for insubordination. If he fired Matthew's father, he would be out
a job and a jacket. And the rent was due, and the family had no
savings.

The boss looked at him, looked at the sketch. ``OK, you try.''

And that was the day that Matthew's father stopped being Driver
Fong and became Master Fong, the junior cutter at the Infinite
Quality Shoe Factory. Less than a year later, he was the head
cutter, and the family thrived.

Matthew had heard this story so many times growing up that he could
recite it word-for-word with his father. It was more than a story:
it was the family legend, more important than any of the history
he'd learned in school. As stories went, it was a good one, but
Matthew was determined that his own life would have an even better
story still. Matthew would not be the second Master Fong. He would
be Boss Fong, the first -- a man with his own factory, his own
fortune.

And like his father, Matthew had a gift.

Like his father, Matthew could look at a certain kind of problem
and \emph{see} the solution. And the problems Matthew could solve
involved killing monsters and harvesting their gold and prestige
items, better and more efficiently than anyone else he'd ever met
or heard of.

Matthew was a gold farmer, but not just one of those guys who found
themselves being approached by an Internet cafe owner and offered
seven or eight RMB to keep right on playing, turning over all the
gold they won to the boss, who'd sell it on by some mysterious
process. Matthew was Master Fong, the gold farmer who could run a
dungeon once and tell you exactly the right way to run it again to
get the maximum gold in the minimum time. Where a normal farmer
might make 50 gold in an hour, Matthew could make 500. And if you
watched Matthew play, you could do it too.

Mr Wing had quickly noticed Matthew's talent. Mr Wing didn't like
games, didn't care about the legends of Iceland or England or India
or Japan. But Mr Wing understood how to make boys work. He
displayed their day's take on big boards at both ends of his
factory, treated the top performers to lavish meals and baijiu
parties in private rooms at his karaoke club where there were
beautiful girls. Matthew remembered these evenings through a bleary
haze: a girl on either side of him on a sofa, pressed against him,
their perfume in his nose, refilling his glass as Mr Wing toasted
him for a hero, extolling his achievements. The girls oohed and
aahed and pressed harder against him. Mr Wing always laughed at him
the next day, because he'd pass out before he could go with one of
the girls into an even \emph{more} private room.

Mr Wing made sure all the other boys knew about this failing, made
sure that they teased ``Master Fong'' about his inability to hold his
liquor, his shyness around girls. And Matthew saw exactly what Boss
Wing was doing: setting Matthew up as a hero, above his friends,
then making sure that his friends knew that he wasn't \emph{that}
much of a hero, that he could be toppled. And so they all farmed
gold harder, for longer hours, eating dumplings at their computers
and shouting at each other over their screens late into the night
and the cigarette haze.

The hours had stretched into days, the days had stretched into
months, and one day Matthew woke up in the dorm room filled with
farts and snores and the smell of 20 young men in a too-small room,
and realized that he'd had enough of working for Boss Wing. That
was when he decided that he would become his own man. That was when
he set out to be Boss Fong.

\tb

\shopad{This scene is dedicated to Amazon.com, the largest Internet bookseller in the world. Amazon is
\emph{amazing} -- a ``store'' where you can get practically any book ever published (along with practically everything else, from laptops to cheese-graters), where they've elevated recommendations to a high art, where they allow customers to directly communicate with each other, where they are constantly inventing new and better ways of connecting books with readers. Amazon has always treated me like gold -- the founder, Jeff Bezos, even posted a reader-review for my first novel! -- and I shop there like crazy (looking at my spreadsheets, it appears that I buy something from Amazon approximately every
\emph{six days}). Amazon's in the process of reinventing what it means to be a bookstore in the twenty-first century and I can't think of a better group of people to be facing down that thorny set of problems.}
{\href{http://www.amazon.com/exec/obidos/ASIN/0765322161/downandoutint-20}{Amazon}}

Wei-Dong Goldberg woke one minute before his alarm rang, the
glowing numbers showing 12:59. 1AM in Los Angeles, 6PM in China,
and it was time to go raiding.

He wiped the sleep out of his eyes and climbed out of his narrow
bed -- his mom still put his goddamned Spongebob sheets on it, so
he'd drawn beards and horns and cigarettes on all the faces in
permanent marker -- and crossed silently to his school-bag and
retrieved his laptop, then felt around on his desk for the little
Bluetooth earwig, screwing it into his ear.

He made a pile of pillows against the headboard and sat
cross-legged against them, lifting the lid and firing up his
gamespy, looking for his buds, all the way over there in Shenzhen.
As the screen filled with names and the games they could be found
in, he smiled to himself. It was time to play.

Three clicks later and he was in Savage Wonderland, spawning on his
clockwork horse with his sword in his hand, amid the garden of
talking, hissing flowers, ready to do battle. And there were his
boys, riding up alongside of him, their clockwork mounts snorting
and champing for battle.

``Ni hao!'' he said into his headset, in as loud a whisper as he
dared. His father had a bladder problem and he got up all night
long and never slept very deeply. Wei-Dong couldn't afford that. If
his parents caught him at it one more time, they'd take away his
computer. They'd ground him. They'd send him to a military academy
where they shaved your head and you got beaten up in the shower
because it built character. He'd been treated to all these threats
and more, and they'd made an impression on him.

Not enough of an impression to get him to stop playing games in the
middle of the night, of course.

``Ni hao!'' he said again. There was laughter, distant and flanged by
network churn.

``Hello, Leonard,'' Ping said. ``You are learning your Chinese well, I
see.'' Ping still called him \emph{Leonard}, but at least he was
talking in Mandarin to him now, which was a big improvement. The
guys normally liked to practice their English on him, which meant
he couldn't practice his Chinese on \emph{them}.

``I practice,'' he said.

They laughed again and he knew that he'd gotten something wrong.
The intonation. He was always getting it wrong. He'd say, ``I'll go
aggro those demons and you buff the cleric,'' and it would come out,
``I am a bowl of noodles, I have beautiful eyelashes.'' But he was
getting better. By the time he got to China, he'd have it nailed.

``Are we raiding?'' he said.

``Yes!'' Ping said, and the others agreed. ``We just need to wait for
the gweilo.'' Wei-Dong loved that he wasn't the gweilo anymore.
Gweilo meant ``foreign devil,'' and technically, he qualified. But he
was one of the raiders now, and the gweilos were the paying
customers who shelled out good dollars or euros or rupees or pounds
to play alongside of them.

Here was the gweilo now. You could tell because he frequently
steered his horse off the path and into the writhing grasp of the
living plants, having to stop over and over to hack away their
grasping vines. After watching this show for a minute or two, he
rode out and cast a protection spell around them both, and the
vines sizzled on the glowing red bubble that surrounded them both.

``Thanks,'' the gweilo said.

``No problem,'' he said.

``Woah, you speak English?'' The gweilo had a strong New Jersey
accent.

``A little,'' Wei-Dong said, with a smile.
\emph{Better than you, dummy}, he thought.

``OK, let's do this thing,'' the gweilo said, and the rest of the
party caught up with them.

The gweilo had paid them to raid an instance of The Walrus's
Garden, a pretty hard underwater dungeon that had some really good
drops in it -- ingredients for potions, some pretty good weapons,
and, of course, lots of gold. There were a couple prestige items
that dropped there, albeit rarely -- you could get a vorpal blade
and helmet if you were very lucky. The deal was, the gweilo paid
them to run the instance with him, and he could just hang back and
let the raiders do all the heavy lifting, but he'd come forward to
deal the coup de grace to any big bosses they beat down, so he'd
get the experience points. He got to keep the gold, the weapons,
the prestige items, all of it -- and all for the low, low cost of
\$75. The raiders got the cash, the gweilo got to level up fast and
pick up a ton of treasure.

Wei-Dong often wondered what kind of person would pay strangers to
help them get ahead in a game? The usual reason that gweilos gave
for hiring raiders was that they wanted to play with their friends,
and their friends were all more advanced than them. But Wei-Dong
had joined games after his friends and being the noob in his little
group, he'd just asked his buds to take him raiding with them,
twinking him until his character was up to their level. So if this
gweilo had so many pals in this game that he wanted to level up to
meet them, why couldn't he get them to power-level his character up
with them? Why was he paying the raiders?

Wei-Dong suspected that it was because the guy had no friends.

``God\emph{damn} would you look at that?'' It was at least the tenth
time the guy had said it in ten minutes as they rode to the
seashore. This time it was the tea-party, a perpetual melee that
was a blur of cutlery whistling through the air, savage chairs
roaming in packs, chasing luckless players who happened to aggro
them, and a crazy-hard puzzle in which you had to collect and
arrange the crockery just so, stunning each piece so that it
wouldn't crawl away before you were done with it. It was pretty
cool, Wei-Dong had to admit (he'd solved the puzzle in two days of
hard play, and gotten the teapot for his trouble, which he could
use to summon genies in moments of dire need). But the gweilo was
acting like he'd never seen computer graphics, ever.

They rode on, chattering in Chinese on a private channel. Mostly,
it was too fast for Wei-Dong to follow, but he caught the gist of
it. They were talking about work -- the raids they had set up for
the rest of the night, the boss and his stupid rules, the money and
what they'd do with it. Girls. They were always talking about
girls.

At last they were at the seaside, and Wei-Dong cast the Red Queen's
Air Pocket, using up the last of his oyster shells to do so. They
all dismounted, flapping their gills comically as they sloshed into
the water (``God\emph{damn},'' breathed the gweilo).

The Walrus's Garden was a tricky raid, because it was different
every time you ran it, the terrain regenerating for each party. As
the spellcaster, Wei-Dong's job was to keep the lights on and the
air flowing so that no matter what came, they'd see it in time to
prepare and vanquish it. First came the octopuses, rising from the
bottom with a puff of sand, sailing through the water toward them.
Lu, the tank, positioned himself between the party and the
octopuses, and, after thrashing around and firing a couple of
missiles at them to aggro them, went totally still as, one after
another, they wrapped themselves around him, crushing him with
their long tentacles, their faces crazed masks of pure
malevolence.

Once they were all engrossed in the tank, the rest of the party
swarmed them, the four of them drawing their edged weapons with a
watery \emph{clang} and going to work in a writhing knot. Wei-Dong
kept a close eye on the tank's health and cast his healing spells
as needed. As each octopus was reduced to near death, the raiders
pulled away and Wei-Dong hissed into his mic, ``Finish him!'' The
gweilo fumbled around for the first two beasts, but by the end, he
was moving efficiently to dispatch them.

``That was \emph{sick},'' the gweilo said. ``Totally badass! How'd
that guy absorb all that damage, anyway?''

``He's a tank,'' Wei-Dong said. ``Fighter class, heavy armor. Lots of
buffs. And I was keeping up the healing spells the whole time.''

``I'm fighter class, aren't I?''

\emph{You don't know?} This guy had a \emph{lot} more money than
brains, that was for sure.

``I just started playing. I'm not much of a gamer. But you know, all
my friends --''

\emph{I know}, Wei-Dong thought.
\emph{All the cool kids you knew were doing it, so you decided you had to keep up with them. You don't have any friends -- yet. But you think you will, if you play.}
``Sure,'' he said. ``Just stick close, you're doing fine. You'll be
leveled up by breakfast time.'' That was another mark against the
gweilo: he had the money to pay for a power-levelling session with
their raiding guild, but he wasn't willing to pay the premium to do
it in a decent American timezone. That was good news for the rest
of the guild, sure -- it saved them having to find somewhere to do
the run during daylight hours in China, when the Internet cafes
were filled with straights -- but it meant that Wei-Dong had to be
up in the middle of the night and then drag his butt around school
all the next day.

Not that it wasn't worth it.

Now they were into the crags and caves of the garden, dodging the
eels and giant lobsters that surged out of their holes as they
passed. Wei-Dong found some more oyster shells and surreptitiously
picked them up. Technically, they were the gweilo's to have first
refusal over, but they were needed if he was going to keep on
casting the Air Pocket, which he might have to do if they kept up
at this slow pace. And the gweilo didn't notice, anyway.

``You're not in China, are you?'' the gweilo asked.

``Not exactly,'' he said, looking out the window at the sky over
Orange County, the most boring ZIP code in California.

``Where are you guys?''

``They're in China. Where I live, you can see the Disneyland
fireworks show every night.''

``God\emph{damn},'' the gweilo said. ``Ain't you got better things to
do than help some idiot level up in the middle of the night?''

``I guess I don't,'' he said. Mixed in behind were the guys laughing
and catcalling in Chinese on their channel. He grinned to hear
them.

``I mean, hell, I can see why someone in China'd do a crappy job for
a rotten 75 bucks, but if you're in America, dude, you should have
some \emph{pride}, get some real work!''

``And why would someone in China want to do a crappy job?'' The guys
were listening in now. They didn't have great English, but they
spoke enough to get by.

``You know, it's \emph{China}. There's \emph{billions} of 'em. Poor
as dirt and ignorant. I don't blame 'em. You can't blame 'em. It's
not their fault. But hell, once you get out of China and get to
America, you should \emph{act} like an American. We don't do that
kind of work.''

``What makes you think I 'got out of China'?''

``Didn't you?''

``I was born here. My parents were born here. Their parents were
born here. Their parents came here from Russia.''

``I didn't know they had Chinese in Russia.''

Wei-Dong laughed. ``I'm not Chinese, dude.''

``You aren't? Well, god\emph{damn} then, I'm sorry. I figured you
were. What are you, then, the boss or something?''

Wei-Dong closed his eyes and counted to ten. When he opened them
again, the carpenters had swum out of the wrecked galleon before
them, their T-squares and saws at the ready. They moved by building
wooden boxes and gates around themselves, which acted as
barricades, and they worked \emph{fast}. On the land, you could
burn their timbers, but that didn't work under the sea. Once they
had you boxed in, they drove long nails through boards around you.
It was a grisly, slow way to die.

Of course, they had the gweilo surrounded in a flash, and they all
had to pile on to fight them free. Xiang summoned his familiar, a
boar, and Wei-Dong spelled it its own air bubble and it set to
work, tearing up the planks with its tusks. When at last the
carpenters managed to kill it, it turned into a baby and floated,
lifeless, to the ocean's surface, accompanied by a ghostly weeping.
Savage Wonderland \emph{looked} like it was all laughs, but it was
really grim when you got down to it, and the puzzles were hard and
the big bosses were \emph{really} hard.

Speaking of bosses: they put down the last of the carpenters and as
they did, a swirling current disturbed the sea-bottom, kicking up
sand that settled slowly, revealing the vorpal blade and armor,
encrusted in barnacles. And the gweilo gave a whoop and a holler
and dove for it clumsily, as they all shouted at once for him to
stop, to wait, and then --

And then he triggered the trap that they all knew was there.

And then there was \emph{trouble.}

The Jabberwock did indeed have eyes of flame, and it did make a
``burbling'' sound, just like it said in the poem. But the Jabberwock
did a lot more than give you dirty looks and belch. The Jabberwock
was \emph{mean}, it soaked up a lot of damage, and it gave as good
as it got. It was fast, too, faster than the carpenters, so one
minute you could be behind it and then it would do a barrel roll --
its tail like a whip, cracking and knocking back anything that got
in its way -- and it would be facing you, rearing up with its
spindly claws splayed, its narrow chest heaving. The jaws that
bite, the claws that catch -- and once they'd caught you, the
Jabberwock would beat you against the hardest surface in reach,
doing insane damage while you squirmed to get free. And the
burbling? Not so much like burping, really: more like the sound of
meat going through a grinder, a nasty sound. A \emph{bloody}
sound.

The first time Wei-Dong had managed to kill a Jabberwock -- after a
weekend's continuous play -- he'd crashed hard and had nightmares
about that sound.

``Nice going, jackass,'' Wei-Dong said as he hammered on his
keyboard, trying to get all his spells up and running without
getting disemboweled by the nightmare beast before them. It had Lu
and was beating the everloving piss out of him, but that was OK, it
was just Lu, his job was to get beaten up. Wei-Dong cast his
healing spells at Lu while he swam back as fast as he could.

``Now, that's not nice,'' the gweilo said. ``How the hell was I
supposed to know --''

``You weren't. You didn't know. You don't know. That's the
\emph{point}. That's why you hired \emph{us}. Now we're going to
use up all our spells and potions fighting this thing --'' he broke
off for a second and hit some more keys ``-- and it's going to take
\emph{days} to get it all back, just because you couldn't wait at
the back like you were \emph{supposed} to.''

``I don't have to take this,'' the gweilo said. ``I'm a customer,
dammit.''

``You want to be a \emph{dead} customer, buddy?'' Wei-Dong said. He'd
barely had any time to talk with his guildies on the whole raid,
he'd been stuck talking to this dumb English speaker. Now the guy
was mouthing off to him. It made him want to throw his computer
against the wall. See what being nice gets you?

If the gweilo replied, Wei-Dong didn't hear it, because the
Jabberwock was really pouring on the heat. He was out of potions
and healing spells and Lu wasn't going to last much longer. Oh,
\emph{crap}. It had Ping in its other claw now, and it was worrying
at his armor with a long fang, trying to peel him like a grape. He
tabbed over to his voice-chat controller and dialled up the Chinese
channel to full, tuning out the gweilo.

It was a chaos of fast, profane dialect, slangy Chinese that mixed
in curse-words from Japanese comics and Indian movies. The boys
were all hollering, too fast for him to get more than the sense of
things.

There was Ping, though, calling for him. ``Leonard! Healing!''

``I'm out!'' he said, hating how this was all going. ``I'm totally
empty. Used it all up on Lu!''

``That's it, then,'' Ping said. ``We're dead.'' They all howled with
disappointment. In spite of himself, Wei-Dong grinned. ``You think
he'll reschedule, or are we going to have to give him his money
back?''

Wei-Dong didn't know, but he had a feeling that this goober wasn't
going to be very cooperative if they told him that he'd gotten up
in the middle of the night for nothing. Even if it was his fault.

He sucked in some whistling breaths through his nose and tried to
calm down. It was almost 2AM now. In the house around him, all was
silent. A car revved its engine somewhere far away, but the night
was so quiet the sound carried into his bedroom.

``OK,'' he said. ``OK, let me do something about this.''

Every game had a couple of BFGs, Big Friendly Guns (or at least
\emph{some} kind of Big Gun), that were nearly impossible to get
and nearly impossible to resist. In Savage Wonderland, they were
also nearly impossible to re-load: the rare monster blunderbuss
that you had to spend \emph{months} gathering parts for fired huge
loads of sharpened cutlery from the Tea Party, and just collecting
enough for a single load took eight or nine hours of gameplay.
Impossible to get -- impossible to load. Practically no one had
one.

But Wei-Dong did. Ignoring the shouting in his headset, he backed
off to the edge of the blunderbuss's range and began to arm it, a
laborious process of dumping all that cutlery into the muzzle. ``Get
in front of it,'' he said. ``In front of it, now!''

His guildies could see what he was doing now and they were whooping
triumphantly, arraying their toons around its front, occupying its
attention, clearing his line of fire. All he needed was
one\ldots{}more\ldots{}second.

He pulled the trigger. There was a snap and a hiss as the powder in
the pan began to burn. The sound made the Jabberwock turn its head
on its long, serpentine neck. It regarded him with its burning eyes
and it dropped Ping and Lu to the oceanbed. The powder in the pan
flared -- and died.

\emph{Misfire}!

\emph{Ohcrapohcrapohcrap,} he muttered, hammering, \emph{hammering}
on the re-arm sequence, his fingers a blur on the mouse-buttons.
``Crapcrapcrapcrap.''

The Jabberwock smiled, and made that wet meaty sound again.
\emph{Burble burble, little boy, I'm coming for you}. It was the
sound from his nightmare, the sound of his dream of heroism dying.
The sound of a waste of a day's worth of ammo and a night's worth
of play. He was a dead man.

The Jabberwock did one of those whipping, rippling barrel-rolls
that were its trademark. The currents buffeted him, sending him
rocking from side to side. He corrected, overcorrected, corrected
again, hit the re-arm button, the fire button, the re-arm button,
the fire button --

The Jabberwock was facing him now. It reared back, flexing its
claws, clicking its jaws together. In a second it would be on him,
it would open him from crotch to throat and eat his guts, any
second now --

\emph{Crash!} The sound of the blunderbuss was like an explosion in
a pots-and-pans drawer, a million metallic clangs and bangs as the
sea was sliced by a rapidly expanding cone of lethal, screaming
metal tableware.

The Jabberwock \emph{dissolved}, ripped into a slowly rising
mushroom of meat and claws and leathery scales. The left side of
its head ripped toward him and bounced off him, settling in the
sand. The water turned pink, then red, and the death-screech of the
Jabberwock seemed to carom off the water and lap back over him
again and again. It was a \emph{fantastic} sound.

His guildies were going nuts, seven thousand miles away, screaming
his name, and not \emph{Leonard,} but \emph{Wei-Dong}, chanting it
in their Internet Cafe off Jiabin Road in Shenzhen. Wei-Dong was
grinning ferociously in his bedroom, basking in it.

And when the water cleared, there again were the vorpal blade and
helmet in their crust of barnacles, sitting innocently on the ocean
floor. The gweilo -- the gweilo, he'd forgotten all about the
gweilo! -- moved clumsily toward it.

``I don't think so,'' said Ping, in pretty good English. His toon
moved so fast that the gweilo probably didn't even see him coming.
Ping's sword went snicker-snack, and the gweilo's head fell to the
sand, a dumb, betrayed expression on its face.

``What the --''

Wei-Dong dropped him from the chat.

``That's your treasure, brother,'' Ping said. ``You earned it.''

``But the money --''

``We can make the money tomorrow night. That was
\emph{killer, dude}!'' It was one of Ping's favorite English
phrases, and it was the highest praise in their guild. And now he
had a vorpal blade and helmet. It was a good night.

They surfaced and paddled to shore and conjured up their mounts
again and rode back to the guild-hall, chatting all the way,
dispatching the occasional minor beast without much fuss. The guys
weren't too put out at being 75 bucks' poorer than they'd expected.
They were players first, business people second. And that had been
\emph{fun}.

And now it was 2:30 and he'd have to be up for school in four
hours, and at this rate, he was going to be lying awake for a
\emph{long} time. ``OK, I'm going to go guys,'' he said, in his best
Chinese. They bade him farewell, and the chat channel went dead. In
the sudden silence of his room, he could hear his pulse pounding in
his ears. And another sound -- a tread on the floor outside his
door. A hand on the doorknob --

\emph{Crapcrapcrap}

He manged to get the lid of the laptop down and his covers pulled
up before the door opened, but he was still holding the machine
under the sheets, and his father's glare from the doorway told him
that he wasn't fooling anyone. Wordlessly, still glaring, his
father crossed the room and delicately removed the earwig from
Wei-Dong's ear. It glowed telltale blue, blinking, looking for the
laptop that was now sleeping under Wei-Dong's artistically
redecorate Spongebob sheets.

``Dad --'' he began.

``Leonard, it's 2:30 in the morning. I'm not going to discuss this
with you right now. But we're going to talk about it in the
morning. And you're going to have a long, long time to think about
it afterward.'' He yanked back the sheet and took the laptop out of
Wei-Dong's now-limp hand.

``Dad!'' he said, as his father turned and left the room, but his
father gave no indication he'd heard before he pulled the bedroom
door firmly and authoritatively shut.

\tb

\shopad{This scene is dedicated to Borderlands Books, San Francisco's magnificent independent science fiction bookstore. Borderlands is not just notorious for its brilliant events, signings, book clubs and such, but also for its amazing hairless Egyptian cat, Ripley, who likes to perch like a buzzing gargoyle on the computer at the front of the store. Borderlands is about the friendliest bookstore you could ask for, filled with comfy places to sit and read, and staffed by incredibly knowledgeable clerks who know everything there is to know about science fiction. Even better, they've always been willing to take orders for my book (by net or phone) and hold them for me to sign when I drop into the store, then they ship them within the US for free!}
{\href{http://www.borderlands-books.com/}{Borderlands Books}: 866 Valencia Ave, San Francisco CA USA 94110 +1 888 893 4008}

Mala missed the birdcalls. When they'd lived in the village,
there'd been birdsong every morning, breaking the perfect peace of
the night to let them know that the sun was rising and the day was
beginning. That was when she'd been a little girl. Here in Mumbai,
there were some sickly rooster calls at dawn, but they were nearly
drowned out by the neverending trafficsong: the horns, the engines
revving, the calls late in the night.

In the village, there'd been the birdcalls, the silence, and peace,
times when everyone wasn't always watching. In Mumbai, there was
nothing but the people, the people everywhere, so that every breath
you breathed tasted of the mouth that had exhaled it before you got
it.

She and her mother and her brother slept together in a tiny room
over Mr Kunal's plastic-recycling factory in Dharavi, the huge
squatter's slum at the north end of the city. During the day, the
room was used to sort plastic into a dozen tubs -- the plastic
coming from an endless procession of huge rice-sacks that were
filled at the shipyards. The ships went to America and Europe and
Asia filled with goods made in India and came back filled with
garbage, plastic that the pickers of Dharavi sorted, cleaned,
melted and reformed into pellets and shipped to the factories so
that they could be turned into manufactured goods and shipped back
to America, Europe and Asia.

When they'd arrived at Dharavi, Mala had found it terrifying: the
narrow shacks growing up to blot out the sky, the dirt lanes
between them with gutters running in iridescent blue and red from
the dye-shops, the choking always-smell of burning plastic, the
roar of motorbikes racing between the buildings. And the eyes, eyes
from every window and roof, all watching them as mamaji led her and
her little brother to the factory of Mr Kunal, where they were to
live now and forevermore.

But barely a year had gone by and the smell had disappeared. The
eyes had become friendly. She could hop from one lane to another
with perfect confidence, never getting lost on her way to do the
marketing or to attend the afternoon classes at the little
school-room over the restaurant. The sorting work had been boring,
but never hard, and there was always food, and there were other
girls to play with, and mamaji had made friends who helped them
out. Piece by piece, she'd become a Dharavi girl, and now she
looked on the newcomers with a mixture of generosity and pity.

And the work -- well, the work had gotten a lot better, just
lately.

It started when she was in the games-cafe with Yasmin, stealing an
hour after lessons to spend a few Rupees of the money she'd saved
from her pay-packet (almost all of it went to the family, of
course, but mamaji sometimes let her keep some back and advised her
to spend it on a treat at the cornershop). Yasmin had never played
Zombie Mecha, but of course they'd both seen the movies at the
little filmi house on the road that separated the Muslim and the
Hindu sections of Dharavi. Mala \emph{loved} Zombie Mecha, and she
was good at it, too. She preferred the PvP servers where players
could hunt other players, trying to topple their giant mecha-suits
so that the zombies around them could swarm over it, crack open its
cockpit cowl and feast on the av within.

Most of the girls at the game cafe came in and played little games
with cute animals and trading for hearts and jewels. But for Mala,
the action was in the awesome carnage of the multiplayer war games.
It only took a few minutes to get Yasmin through the basics of
piloting her little squadron and then she could get down to
\emph{tactics}.

That was it, that was what none of the other players seemed to
understand: \emph{tactics} were \emph{everything}. They treated the
game like it was a random chaos of screeching rockets and
explosions, a confusion to be waded into and survived, as best as
you could.

But for Mala, the confusion was something that happened to other
people. For Mala, the explosions and camera-shake and the screech
of the zombies were just minor details, to be noted among the Big
Picture, the armies arrayed on the battlefield in her mind. On that
battlefield, the massed forces took on a density and a color that
showed where their strengths and weaknesses were, how they were
joined to each other and how pushing one this one, over here, would
topple that one over there. You could face down your enemies head
on, rockets against rockets, guns against guns, and then the winner
would be the luckier one, or the one with the most ammo, or the one
with the best shields.

But if you were \emph{smart}, you didn't have to be lucky, or
tougher. Mala liked to lob rockets and grenades \emph{over} the
opposing armies, to their left and right, creating box-canyons of
rubble and debris that blocked their escape. Meanwhile, a few of
her harriers would be off in the weeds aggroing huge herds of
zombies, getting them \emph{really} mad, gathering them up until
they were like locusts, blotting out the ground in all directions,
leading them ever closer to that box canyon.

Just before they'd come into view, her frontal force would peel
off, running away in a seeming act of cowardice. Her enemies would
be buoyed up by false confidence and give chase -- until they saw
the harriers coming straight for them, with an unstoppable,
torrential pestilence of zombies hot on their heels. Most times,
they were too shocked to do \emph{anything}, not even fire at the
harriers as they ran straight for their lines and \emph{through}
them, into the one escape left behind in the box-canyon, blowing
the crack shut as they left. Then it was just a matter of waiting
for the zombies to overwhelm and devour your opponents, while you
snickered and ate a sweet and drank a little tea from the urn by
the cashier's counter. The sounds of the zombies rending the armies
of her enemies and gnawing their bones was \emph{particularly}
satisfying.

Yasmin had been distracted by the zombies, the disgusting entrails,
the shining rockets. But she'd seen, oh yes, she'd \emph{seen} how
Mala's strategies were able to demolish much larger opposing armies
and she got over her squeamishness.

And so on they played, drawing an audience: first the hooting
derisive boys (who fell silent when they watched the armies fall
before her, and who started to call her ``General Robotwalla''
without even a hint of mockery), and then the girls, shy at first,
peeking over the boys' shoulders, then shoving forward and cheering
and beating their fists on the walls and stamping their feet for
each dramatic victory.

It wasn't cheap, though. Mala's carefully hoarded store of Rupees
shrank, buffered somewhat by a few coins from other players who
paid her a little here and there to teach them how to really play.
She knew she could have borrowed the money, or let some boy spend
it on her -- there was already fierce competition for the right to
go over the road to the drinkswalla and buy her a masala Coke, a
fizzing, foaming spicy explosion of Coke and masala spice and
crushed ice that soothed the rawness at the back of her throat that
had been her constant companion since they'd come to Dharavi.

But nice girls from the village didn't let boys buy them things.
Boys wanted something in return. She knew that, knew it from the
movies and from the life around her. She knew what happened to
girls who let boys take care of their needs. There was always a
reckoning.

When the strange man first approached her, she thought about nice
girls and boys and what they expected, and she wouldn't talk to him
or meet his eye. She didn't know what he wanted, but he wasn't
going to get it from her. So when he got up from his chair by the
cashier as she came into the cafe, rose and crossed to intercept
her with his smart linen suit and good shoes and short, neatly
oiled hair, and small moustache, she'd stepped around him, stepped
past him, pretended she didn't hear him say, ``Excuse me, miss,'' and
``Miss? Miss? Please, just a moment of your time.''

But Mrs Dibyendu, the owner of the cafe, shouted at her, ``Mala, you
listen to this man, you listen to what he has to say to you. You
don't be rude in my shop, no you don't!'' And because Mrs Dibyendu
was also from a village, and because her mother had said that Mala
could play games but only in Mrs Dibyendu's cafe, Mrs Dibyendu
being the sort of person you could trust not to allow improper
doings, or drugs, or violence, or criminality, Mala stopped and
turned to the man, silent, expecting.

``Ah,'' he said. ``Thank you.'' He nodded to Mrs Dibyendu. ``Thank you.''
He turned back to her, and to the army of boys and girls who'd
gathered around her, \emph{her} army, the ones who called her
General Robotwallah and meant it.

``I hear that you are a very good player,'' he said. Mala waggled her
chin back and forth, half-closing her eyes, letting her chin say,
\emph{Yes, I'm a good player, and I'm good enough that I don't need to boast about it.}

``Is she a good player?''

Mala turned to her army, who had the discipline to remain silent
until she gave them the nod. She waggled her chin at them:
\emph{go on}.

And they erupted in an enthused babble, extolling the virtues of
their General Robotwallah, the epic battles they'd fought and won
against impossible odds.

``I have some work for good players.''

Mala had heard rumors of this. ``You represent a league?''

The man smiled a little smile and shook his head. He smelled of
citrusy cologne and betel, a sweet combination of smells she'd
never smelled before. ``No, not a league. You know that in the game,
there are players who don't play for fun? Players who play to make
money?''

``The kind of money you're offering to us?''

His chin waggled and he chuckled. ``No, not exactly. There are
players who play to build up game-money, which they sell on to
other players who are too lazy to do the playing for themselves.''

Mala thought about this for a moment. The containers went out of
India filled with goods and came back filled with garbage for
Dharavi. Somewhere out there, in the America of the filmi shows,
there was a world of people with unimaginable wealth. ``We'll do
it,'' she said. ``I've already got more credits than I can spend. How
much do they pay for them?''

Again, the chuckle. ``Actually,'' he said, then stopped. Her army was
absolutely silent now, hanging on his every word. From the machines
came the soft crashing of the wars, taking place in the world
inside the network, all day and all night long. ``Actually, that's
not exactly it. We want you and your friends to destroy them, kill
their avs, take their fortunes.''

Mala thought for another instant, puzzled. Who would want to kill
these other players? ``You're a rival?''

The man waggled his chin. \emph{Maybe yes, maybe no.}

She thought some more. ``You work for the game!'' she said. ``You work
for the game and you don't want --''

``Who I work for isn't important,'' the man said, holding up his
fingers. He wore a wedding ring on one hand, and two gold rings on
the other. He was missing the top joints on three of his fingers,
she saw. That was common in the village, where farmers were always
getting caught in the machines. Here was a man from a village, a
man who'd come to Mumbai and become a man in a neat suit with a
neat mustache and gold rings glinting on what remained of his
fingers. Here was the reason her mother had brought them to
Dharavi, the reason for the sore throat and the burning eyes and
the endless work over the plastic-sorting tubs.

``What's important is that we would pay you and your friends --''

``My army,'' she said, interrupting him without thinking. For a
moment his eyes flashed dangerously and she sensed that he was
about to slap her, but she stood her ground. She'd been slapped
plenty before. He snorted once through his nose, then went on.

``Yes, Mala, your army. We would pay you to destroy these players.
You'd be told what sort of mecha they were piloting, what their
player-names were, and you'd have to root them out and destroy
them. You'd keep all their wealth, and you'd get Rupees, too.''

``How much?''

He made a pained expression, like he had a little gas. ``Perhaps we
should discuss that in private, later? With your mother present?''

Mala noticed that he didn't say, ``Your parents,'' but rather, ``Your
mother.'' Mrs Dibyendu and he had been talking, then. He knew about
Mala, and she didn't know about him. She was just a girl from the
village, after all, and this was the world, where she was still
trying to understand it all. She was a general, but she was also a
girl from the village. General Girl From the Village.

So he'd come that night to Mr Kunal's factory, and Mala's mother
had fed him thali and papadams from the women's papadam collective,
and they'd boiled chai in the electric kettle and the man had
pretended that his fine clothes and gold belonged here, and had
squatted back on his heels like a man in the village, his hairy
ankles peeking out over his socks. No one Mala knew wore socks.

``Mr Banerjee,'' mamaji said, ``I don't understand this, but I know
Mrs Dibyendu. If she says you can be trusted\ldots{}'' She trailed off,
because really, she didn't know Mrs Dibyendu. In Dharavi, there
were many hazards for a young girl. Mamaji would fret over them
endlessly while she brushed out Mala's hair at night, all the ways
a girl could find herself ruined or hurt here. But the money.

``A lakh of rupees every month,'' he said. ``Plus a bonus. Of course,
she'll have to pay her 'army' --'' he'd given Mala a little chin
waggle at that, \emph{see, I remember} ``-- out of that. But how
much would be up to her.''

``These children wouldn't have any money if it wasn't for my Mala!''
mamaji said, affronted at their imaginary grasping hands. ``They're
only playing a game! They should be glad just to play with her!''
Mamaji had been furious when she discovered that Mala had been
playing at the cafe all these afternoons. She thought that Mala
only played once in a while, not with every rupee and moment she
had spare. But when the man -- Mr Banerjee -- had mentioned her
talent and the money it could earn for the family, suddenly mamaji
had become her daughter's business manager.

Mala saw that Mr Banerjee had known this would happen and wondered
what else Mrs Dibyendu had told him about their family.

``Mamaji,'' she said, quietly, keeping her eyes down in the way they
did in the village. ``They're my army, and they need paying if they
play well. Otherwise they won't be my army for long.''

Mamaji looked hard at her. Beside them, Mala's little brother Gopal
took advantage of their distraction to sneak the last bit of
eggplant off Mala's plate. Mala noticed, but pretended she hadn't,
and concentrated on keeping her eyes down.

Mamaji said, ``Now, Mala, I know you want to be good to your
friends, but you have to think of your family first. We will find a
fair way to compensate them -- maybe we could prepare a weekly
feast for them here, using some of the money. I'm sure they could
all use a good meal.''

Mala didn't like to disagree with her mother, and she'd never done
so in front of strangers, but --

But this was her army, and she was their general. She knew what
made them tick, and they'd heard Mr Banerjee announce that she
would be paid in cash for their services. They believed in
fairness. They wouldn't work for food while she worked for a lakh
(a \emph{lakh} -- \emph{100,000} rupees! The whole family lived on
200 rupees a day!) of cash.

``Mamaji,'' she said, ``it wouldn't be right or fair.'' It occurred to
Mala that Mr Banerjee had mentioned the money in front of the army.
He could have been more discreet. Perhaps it was deliberate. ``And
they'd know it. I can't earn this money for the family on my own,
Mamaji.''

Her mother closed her eyes and breathed through her nose, a sign
that she was trying to keep hold of her temper. If Mr Banerjee
hadn't been present, Mala was sure she would have gotten a proper
beating, the kind she'd gotten from her father before he left them,
when she was a naughty little girl in the village. But if Mr
Banerjee wasn't here, she wouldn't have to talk back to her mother,
either.

``I'm sorry for this, Mr Banerjee,'' Mamaji said, not looking at
Mala. ``Girls of this age, they become rebellious -- impossible.''

Mala thought about a future in which instead of being General
Robotwallah, she had to devote her life to begging and bullying her
army into playing with her so that she could keep all the money
they made for her family, while their families went hungry and
their mothers demanded that they come home straight from school.
When Mr Banerjee mentioned his gigantic sum, it had conjured up a
vision of untold wealth, a real house, lovely clothes for all of
them, Mamaji free to spend her afternoons cooking for the family
and resting out of the heat, a life away from Dharavi and the smoke
and the stinging eyes and sore throats.

``I think your little girl is right,'' Mr Banerjee said, with quiet
authority, and Mala's entire family stared at him, speechless. An
adult, taking Mala's side over her mother? ``She is a very good
leader, from what I can see. If she says her people need paying, I
believe that she is correct.'' He wiped at his mouth with a
handkerchief. ``With all due respect, of course. I wouldn't dream of
telling you how to raise your children, of course.''

``Of course\ldots{}'' Mamaji said, as if in a dream. Her eyes were
downcast, her shoulders slumped. To be spoken to this way, in her
own home, by a stranger, in front of her children! Mala felt
terrible. Her poor mother. And it was all Mr Banerjee's fault: he'd
mentioned the money in front of her army, and then he'd brought her
mother to this point --

``I will find a way to get them to fight without payment, Mamaji --''
But she was cut short by her mother's hand, coming up, palm out to
her.

``Quiet, daughter,'' she said. ``If this man, this \emph{gentleman},
says you know what you're doing, well, then I can't contradict him,
can I? I'm just a simple woman from the village. I don't understand
these things. You must do what this gentleman says, of course.''

Mr Banerjee stood and smoothed his suit back into place with the
palms of his hands. Mala saw that he'd gotten some chana on his
shirt and lapel, and that made her feel better somehow, like he was
a mortal and not some terrible force of nature who'd come to
destroy their little lives.

He made a little namaste at Mamaji, hands pressed together at his
chest, a small hint of a bow. ``Good night, Mrs Vajpayee. That was a
lovely supper. Thank you.'' he said. ``Good night, General
Robotwallah. I will come to the cafe tomorrow at three o'clock to
talk more about your missions. Good night, Gopal,'' he said, and her
brother looked up at him, guiltily, eggplant still poking out of
the corner of his mouth.

Mala thought that Mamaji might slap her once the man had left, but
they all went to bed together without another word, and Mala
snuggled up to her mother the same as she did every night, stroking
her long hair. It had been shining and black when they left the
village, but a year later, it was shot through with grey and it
felt wiry. Mamaji's hand caught hers and stilled it, the callouses
on her fingers rough.

``Sleep, daughter,'' she murmured. ``You have an important job, now.
You need your sleep.''

The next morning, they avoided one another's eyes, and things were
hard for a week, until she brought home her first pay-packet,
folded carefully in the sole of her shoe. Her army had carved
through the enemy forces like the butcher's cleaver parting heads
from chickens. There had been a large bonus in their pay-packet,
and even after she'd paid Mrs Dibyendu and bought everyone masala
Coke at the Hotel Hajj next door, and paid the army their wages,
there was almost 2,000 rupees left, and she took Mamaji into the
smallest sorting room in the loft of the factory, up the ladder.
Mamaji's eyes lit up when she saw the money, and she'd kissed Mala
on the forehead and taken her in the longest, fiercest hug of their
lives together.

And now it was all wonderful between them. Mamaji had begun to look
for a place for them further towards the middle of Dharavi, the old
part where the tin and scrap buildings had been gradually replaced
with brick ones, where the potters' kilns smoked a clean woodsmoke
instead of the dirty, scratchy plastic smoke near Mr Kunal's
factory. Mala had new school-clothes, new shoes, and so did Gopal,
and Mamaji had new brushes for her hair and a new sari that she
wore after her work-day was through, looking pretty and young, the
way Mala remembered her from the village.

And the battles were \emph{glorious}.

She entered the cafe out of the melting, dusty sun of late day and
stood in the doorway. Her army was already assembled, practicing on
their machines, passing gupshup in the shadows of the dark, noisy
room, or making wet eyes at one another through the dim. She barely
had time to grin and then hide the grin before they noticed her and
climbed to their feet, standing straight and proud, saluting her.

She didn't know which one of them had begun the saluting business.
It had started as a joke, but now it was serious. They vibrated at
attention, all eyes on her. They had on better clothes, they looked
well-fed. General Robotwallah was leading her army to victory and
prosperity.

``Let's play,'' she said. In her pocket, her handphone had the latest
message from Mr Banerjee with the location of the day's target.
Yasmin was at her usual place, at Mala's right hand, and at her
left sat Fulmala, who had a bad limp from a leg that she'd broken
and that hadn't healed right. But Fulmala was smart and fast, and
she grasped the tactics better than anyone in the cafe except Mala
herself. And Yasmin, well, Yasmin could make the boys behave, which
was a major accomplishment, since left to their own they liked to
squabble and one-up each other, in a reckless spiral that always
ended badly. But Yasmin could talk to them in a way that was stern
like an older sister, and they'd fall into line.

Mala had her army, her lieutenants, and her mission. She had her
machine, the fastest one in the cafe, with a bigger monitor than
any of the others, and she was ready to go to war.

She touched up her displays, rolled her head from side to side, and
led her army to battle again.

\tb

\shopad{This scene is dedicated to Barnes and Noble, a US national chain of bookstores. As America's mom-and-pop bookstores were vanishing, Barnes and Noble started to build these gigantic temples to reading all across the land. Stocking tens of thousands of titles (the mall bookstores and grocery-store spinner racks had stocked a small fraction of that) and keeping long hours that were convenient to families, working people and others potential readers, the B\&N stores kept the careers of many writers afloat, stocking titles that smaller stores couldn't possibly afford to keep on their limited shelves. B\&N has always had strong community outreach programs, and I've done some of my best-attended, best-organized signings at B\&N stores, including the great events at the (sadly departed) B\&N in Union Square, New York, where the mega-signing after the Nebula Awards took place, and the B\&N in Chicago that hosted the event after the Nebs a few years later. Best of all is that B\&N's ``geeky'' buyers really Get It when it comes to science fiction, comics and manga, games and similar titles. They're passionate and knowledgeable about the field and it shows in the excellent selection on display at the stores.}
{\href{http://search.barnesandnoble.com/Little-Brother/Cory-Doctorow/e/9780765322166/?itm=6}{Barnes and Noble, nationwide}}

Gold. It's all about gold.

But not regular gold, the sort of thing you dig out of the ground.
That stuff was for the last century. There's not enough of it, for
one thing: all the gold ever dug out of the ground in the history
of the world would only amount to a cube whose sides were the
length of a tennis court. And curiously, there's also too much of
it: all the certificates of gold ownership issued into the world
add up to a cube twice that size. Some of those certificates don't
amount to anything -- and no one knows which ones. No one has
independently audited Fort Knox since 1956 FCK. For all we know,
it's empty, the gold smuggled out and sold, put in a vault, sold as
certificates, then stolen again and put into another vault, used as
the basis for more certificates.

Not regular gold.

\emph{Virtual} gold.

Call it what you want: in one game it's called ``Credits,'' in
another, ``Volcano Bucks.'' There are groats, Disney Dollars,
cowries, moolah, and Fool's Gold, and a million other kinds of gold
out there. Unlike real gold, there's no vault of reserves backing
the certificates. Unlike money, there's no government involved in
their issue.

Virtual gold is issued by companies. Game companies. Game companies
who declare, ``So many gold pieces can buy this piece of armor,'' or
``So many credits can buy this space ship'' or ``So much Jools can buy
this zeppelin.'' And because they say it, it is true. Countries and
their banks have to mess around with the ugly business of
convincing citizens to believe what they say: the government may
say, ``This social security check will provide for all your needs in
a month,'' but that doesn't mean that the merchants who supply those
needs will agree.

Companies don't have this problem. When Coca Cola says that 76
groats will buy you one dwarvish axe in Svartalfaheim Warriors,
that's it: the price of an axe is 76 groats. Don't like it? Go play
somewhere else.

Virtual money isn't backed by gold or governments: it's backed by
\emph{fun}. So long as a game is fun, players somewhere will want
to buy into it, because as fun as the game is, it's always more fun
if you're one of the haves, with all the awesome armor and killer
weapons, than if you're some lowly noob have-not with a dagger,
fighting your way up to your first sword.

But where there's money to be spent, there's money to be made. For
some players, the most fun game of all is the game that carves them
out a slice of the pie. Not all the action belongs to the giant
companies up on their tall offices and the games they make. Plenty
of us can get in on the action from down below, where the grubby
little people are.

Of course, this makes the companies \emph{bonkers}. They're big
daddy, they know what's best for their worlds. They are
\emph{in control}. They design the levels and the difficulty to
make it all perfectly balanced. They design the puzzles. They
decree that light elves can't talk to dark elves, that players on
Russian servers can't hop onto the Chinese servers, that it would
take the average player 32 hours to attain the Von Klausewitz drive
and 48 hours to earn the Order of the Armored Penguin. If you don't
like it, you're supposed to \emph{leave}: you're not supposed to
just \emph{buy} your way out of it. Or if you do, you should have
the decency to buy it from \emph{them}.

And here's a little something they won't tell you, these Gods of
the Virtual: they \emph{can't} control it. Kids, crooks, and
weirdos all over the world have riddled their safe little
terrarrium worlds with tunnels leading to the great outdoors. There
are multiple, competing interworld exchanges: want to swap out your
Zombie Mecha wealth for a fully loaded spaceship and a crew of
jolly space-pirates to crew it? Ten different gangs want your
business -- they'll fix you right up with someone else's spaceship
and take your mecha, arms and ammo into inventory for the next
person who wants to immigrate to Zombie Mecha from some other
magical world.

And the Gods are powerless to stop it. For every barrier they put
up, there are hundreds of smart, motivated players of the Big Game
who will knock it down.

You'd think it'd be impossible, wouldn't you? After all, these
aren't mere games of cops and robbers, played out in real cities
filled with real people. They don't need an all-points bulletin to
find a fugitive at large: every person in the world is in the
database, and they own the database. They don't need a search
warrant to find the contraband hiding under your floorboards: the
floorboards, the contraband, the house and you are all in the
database -- and they own the database.

It should be impossible, but it isn't, and here's why: the biggest
sellers of gold and treasure, levels and experience in the worlds
\emph{are the game companies themselves}. Oh, they don't
\emph{call} it power-levelling and gold-farming -- they package it
with prettier, more palatable names, like ``accelerated progress
bonus pack'' and ``All Together Now(TM)'' and lots of other
redonkulous names that don't fool anyone.

But the Gods aren't happy with merely turning a buck on players who
are too lazy to work their way up through the game. They've got a
much, much weirder game in play. They sell gold to people
\emph{who don't even play the game}. That's right: if you're a
bigshot finance guy and you're looking for somewhere to stash a
million bucks where it will do some good, you can buy a million
dollars' worth of virtual gold, hang onto it as the game grows and
becomes more and more fun, as the value of the gold rises and
rises, and then you can sell it back for real money through the
official in-game banks, pocketing a chunky profit for your
trouble.

So while you're piloting your mecha, swinging your axe or
commanding your space fleet, there's a group of weird old grownups
in suits in fancy offices all over the world watching your play
eagerly, trying to figure out if the value of in-game gold is going
to go up or down. When a game starts to suck, everyone rushes to
sell out their holdings, getting rid of the gold as fast as they
can before its value it obliterated by bored gamers switching to a
competing service. And when the game gets \emph{more} fun, well,
that's an even bigger frenzy, as the bidding wars kick up to high
gear, every banker in the world trying to buy the same gold for the
same world.

Is it any wonder that eight of the 20 largest economies in the
world are in virtual countries? And is it any wonder that playing
has become such a serious business?

\tb

\shopad{This scene is dedicated to Secret Headquarters in Los Angeles, my drop-dead all-time favorite comic store in the world. It's small and selective about what it stocks, and every time I walk in, I walk out with three or four collections I'd never heard of under my arm. It's like the owners, Dave and David, have the uncanny ability to predict exactly what I'm looking for, and they lay it out for me seconds before I walk into the store. I discovered about three quarters of my favorite comics by wandering into SHQ, grabbing something interesting, sinking into one of the comfy chairs, and finding myself transported to another world. When my second story-collection, OVERCLOCKED, came out, they worked with local illustrator Martin Cenreda to do a free mini-comic based on Printcrime, the first story in the book. I left LA about a year ago, and of all the things I miss about it, Secret Headquarters is right at the top of the list.}
{\href{http://www.thesecretheadquarters.com/}{Secret Headquarters}: 3817 W. Sunset Boulevard, Los Angeles, CA 90026 +1 323 666 2228}

Matthew stood outside the door of the Internet cafe, breathing
deeply. On the walk over, he'd managed to calm down a little, but
as he drew closer, he became more and more convinced that Boss
Wing's boys would be waiting for him there, and all his friends
would be curled up on the ground, beaten unconscious. He'd brought
four of the best players with him out of Boss Wing's factory, and
he knew that Boss Wing wasn't happy about that \emph{at all}.

He was hyperventilating, his head swimming. He still hurt. It felt
like he had a soccer ball-sized red sun of pain burning in his
underwear and one of the things he wanted most and least to do was
to find a private spot to have a look in there. There was a
bathroom in the cafe, so that was that, it was time to go inside.

He walked up the four flights of stairs painfully, passing under
the gigantic murals from gamespace, avoiding the plastic plants on
each landing that reeked of piss from players who didn't want to
wait for the bathroom. From the third floor up, he was enveloped in
the familiar cloud of body odor, cigarette smoke and cursing that
told him he was on his way to his true home.

In the doorway, he paused and peered around, looking for any sign
of Boss Wing's goons, but it was business as usual: rows and rows
of tables with PCs on them, a few couples sharing machines, but
mostly, it was boys playing, skinny, with their shirts rolled up
over their bellies to catch any breeze that might happen through
the room. There were no breezes, just the eddies in the smoke
caused by the growl of all those PC fans whining as they sucked
particulate-laden smoky air over the superheated motherboards and
monster video cards.

He slunk past the sign-in desk, staffed tonight by a new kid,
someone else just arrived from the provinces to find his fortune
here in bad old Shenzhen. Matthew wanted to grab the kid and carry
him to the city limits, explaining all the way that there was no
fortune to be found here anymore, it all belonged to men like Boss
Wing. \emph{Go home,} he thought at the boy,
\emph{Go home, this place is done.}

His boys were playing at their usual table. They had made a pyramid
from alternating layers of Double Happiness cigarette packs and
empty coffee cups. They looked up as he neared them, smiling and
laughing at some joke. Then they saw the look on his face and they
fell silent.

He sat down at a vacant chair and stared at their screens. They'd
been playing, of course. They were always playing. When they worked
in Boss Wing's factory, they'd pull an 18 hour shift and then
they'd relax by playing some more, running their own characters
through the dungeons they'd been farming all day long. It's why
Boss Wing had such an easy time recruiting for his factory: the
pitch was seductive. ``Get paid to play!''

But it wasn't the same when you worked for someone else.

He tried to find the words to start and couldn't.

``Matthew?'' It was Yo, the oldest of them. Yo actually had a family,
a wife and a young daughter. He'd left Boss Wing's factory and
followed Matthew.

Matthew stared at his hands, took a deep breath, and made a
decision: ``Sorry, I just had a little fight on the way over here.
I've got good news, though: I've got a way to make us all very rich
in a very short time.'' And, from memory, Master Fong described the
way he'd found into the rich dungeon of Svartalfaheim Warriors. He
commandeered a computer and showed them, showed them how to shave
the seconds off the run, where to make sure to stop and grab and
pick up. And then they each took up a machine and went to work.

In time, the ache in his pants faded. Someone gave him a cigarette,
then another. Someone brought him some dumplings. Master Fong ate
them without tasting them. He and his team were at work, and they
were making money, and someday soon, they'd have a fortune that
would make Boss Wing look like a small-timer.

Sometime during the shift, his phone rang. It was his mother. She
wanted to wish him a happy birthday. He had just turned 17.

\tb

\shopad{This scene is dedicated to Powell's Books, the legendary ``City of Books'' in Portland, Oregon. Powell's is the largest bookstore in the world, an endless, multi-storey universe of papery smells and towering shelves. They stock new and used books on the same shelves -- something I've always loved -- and every time I've stopped in, they've had a veritable mountain of my books, and they've been incredibly gracious about asking me to sign the store-stock. The clerks are friendly, the stock is fabulous, and there's even a Powell's at the Portland airport, making it just about the best airport bookstore in the world for my money!}
{\href{http://www.powells.com/cgi-bin/biblio?isbn=9780765322166}{Powell's Books}: 1005 W Burnside, Portland, OR 97209 USA +1 800 878 7323}

Wei-Dong's game-suspension lasted all of 20 minutes. That's how
long it took him to fake a migraine, get a study-pass, sneak into
the resource center, beat the network filter and log on. It was
getting very late back in China, but that was OK, the boys stayed
up late when they were working, and they were glad to have him.

Wei-Dong's real name wasn't Wei-Dong, of course. His real name was
Leonard Goldberg. He'd chosen Wei-Dong after looking up the
meanings of Chinese names and coming up with Strength of the East,
which he liked the sound of. This system for picking names worked
well for the Chinese kids he knew -- when their parents immigrated
to the States, they'd just pick some English name and that was it.
Why not? Why was it better to pick a name because your grandfather
had it than because you liked the sound of it?

He'd tried to explain this to his parents, but it didn't make much
of an impression on them. They were cool with him being interested
in other cultures, but that didn't mean he could get out of having
a Bar-Mitzvah or that they would call him Wei-Dong. And it didn't
mean that they approved of him being up all night with his buds in
China, making money.

Wei-Dong knew that this could all be seen as very lame, an outcast
kid so desperate to make friends that he abandoned his high school
altogether and sucked up to someone in another hemisphere with free
labor instead. But it wasn't like that. Wei-Dong had plenty of
friends at Ronald Regan Secondary School. Plenty of kids thought
that China was the most interesting place in the world, loved the
movies and the food and the comics and the games. And there were
lots of Chinese kids in school too and while a couple clearly
thought he was weird, lots more got it. After all, most of them
were into India the way he was into China, so they had that in
common.

And so what if he was skipping a class? It was Social Studies,
ferchrissakes! They were supposed to be studying China, but
Wei-Dong knew about ten times more about the subject than the
teacher did. As he whispered in Mandarin into his earwig, he
thought that this was like an independent study project. His
teachers should be giving him bonus marks.

``Now what?'' he said. ``What's the mission?''

``We were thinking of running the Walrus's Garden a few more times,
now that we've got it fresh in our heads. Maybe we could pick up
another vorpal blade.'' That's what the guys did when there weren't
any paying gweilos -- they went raiding for prestige items. It
wasn't the most exciting thing of all, but you never knew what
might happen.

``I'm into it,'' he said. He had a free period after this one, then
lunch, so technically he could play for three hours solid. They'd
all be ready to log off and go to bed by then, anyway.

``You're a good gweilo, you know?'' Wei-Dong knew Ping was kidding.
He didn't care if the guys called him gweilo. It wasn't a racist
term, not really, not like ``chink'' or ``slant-eye.'' Just a term of
affection. And as nicknames went, ``Foreign ghost'' was actually kind
of cool.

So they hit the Garden and ran it and they did pretty well, and
they went and put the money in the guild bank and went back for
more. Then they did it again. Somewhere in there, the bell rang.
Somewhere in there, some of his friends came and talked to him and
he muted the earwig and said some things back to them, but he
didn't really know what he'd said. Something.

Then, on the third run, the bad thing happened. They were almost to
the shore, and they'd banished their mounts. Wei-Dong was prepping
the Queen's Air Pocket, dipping into the monster supply of oyster
shells he'd built up on the previous runs.

And out they came, a dozen knights on huge, fearsome black steeds,
rising out of the water in unison, rending the air with the angry
chorus of their mounts and their battle-cries. The water fountained
up around them and they fell upon Wei-Dong and his guildies.

He shouted something into his earwig, a warning, and all around him
in the resource center, kids looked up from their conversations to
stare at him. He'd become a dervish, hammering away at his keyboard
and mousing furiously, his eyes fixed on the screen.

The black riders moved with eerie synchrony. Either they were
monsters -- monsters such as Wei-Dong had never encountered -- or
they were the most practiced, cooperative raiding party he'd ever
seen. He had his vorpal blade out now, and his guildies were all
fighting as well. In his earwig, they cursed in the Chinese
dialects of six different provinces. Under other circumstances,
Wei-Dong would have taken notes, but now he was fighting for his
life.

Lu had bravely taken the point between the riders and the party,
the huge tank standing fast with his mace and broadsword, engaging
all twelve of the knights without regard for his own safety.
Wei-Dong poured healing spells on him as he attempted to make his
own mark on the riders with the vorpal blade, three times as long
as he was.

The vorpal blade could do incredible damage, but it wasn't easy to
use. Twice, Wei-Dong accidentally sliced into members of his own
party, though not badly -- thank God, or he'd never hear the end of
it -- but he couldn't get a cut in on the black knights, who were
too fast for him.

Then Lu fell, going down on one knee, pierced through the throat by
a pike wielded by a rider whose steed's eyes were the icy blue of
the Caterpillar's mist. The rider lifted Lu into the air, his feet
kicking limply, and another knight beheaded him with a contemptuous
swing of his sword. Lu fell in two pieces to the gritty beach sand
and in the earwig, he cursed them, using an expression that
Wei-Dong had painstakingly translated into ``Screw eight generations
of your ancestors.''

With Lu down, the rest of them were practically helpless. They
fought valiantly, coordinating their attacks, pouring on fire from
their magic items and best spells, but the black knights were
unbeatable. Before he died, Wei-Dong managed to hit one with the
vorpal blade and had the momentary satisfaction of watching the
knight stagger and clutch at his chest, but then the fighter closed
with him, drawing a pair of short swords that he spun like a
magician doing knife tricks. There was no question of parrying him,
and seconds later, Wei-Dong was in the sand, watching the knight's
spiked boot descend on his face, hearing the crunch of his
cheekbones and nose shattering under the weight. Then he was
respawning in the distant Lake of Tears, naked and unarmed, and he
had to corpse-run to the body of his toon before the bastards got
his vorpal blade.

He heard his guildies dying in the earwig, one after another, as he
ran, ghostly and ethereal, across the hills and dales of
Wonderland. He reached his corpse just in time to watch the knights
loot the body, and the bodies of his teammates. He rose up again,
helpless and unarmed and made flesh by the body of his toon,
vulnerable.

One of the knights sent him a chat-request. He clicked it,
silencing the background noises from Shenzhen.

``You farmers aren't welcome here anymore, Comrade,'' the voice said.
It had an accent he didn't recognize. Maybe Russian? And the
speaker was just a kid! ``We're patrolling now. You come back again,
we'll hunt and kill you again, and again, and again. You understand
me, Chinee?'' Not just a kid: a \emph{girl} -- a little girl,
threatening him from somewhere in the world.

``Who put you in charge, \emph{missy}?'' he said. ``And what makes you
think I'm Chinese, anyway?''

There was a nasty laugh. ``Missy, huh? I'm in charge because I just
kicked your ass, and because I can kick it again, as many times as
I need to. And I don't care if you're in China, Vietnam, Indonesia
-- it doesn't make a difference. We'll kill you and all the farmers
in Wonderland. This game isn't farmable anymore. I'm done talking
to you now.'' And the black knight decapitated him with contemptuous
ease.

He flipped back to the guild channel, ready to tell them about what
had just happened, his mind reeling, and that's when he looked up
into the face of his father, standing over him, with a look on his
face that could curdle milk.

``Get up, Leonard,'' he said. ``And come with me.''

He wasn't alone. There was Mr Adams, the vice-principal, and the
school's rent-a-cop, Officer Turner, and the guidance counsellor,
Ms Ramirez. They presented him with the stony faces of Mount
Rushmore, faces without a hint of mercy. His father reached over
and took the earwig out of his ear, gently, carefully. Then, with
exactly the same care, he dropped the earwig to the polished
concrete floor of the resource centre and brought his heel down on
it, the \emph{crunch} loud in the perfectly silent room.

Leonard stood up. The room was full of kids pretending not to look
at him. They were all looking at him. He followed his father into
the hallway and as the door swung shut, he heard, unmistakably, the
sound of a hundred giggles in unison.

They boxed him in on the walk to the vice-principal's office,
trapping him. Not that he'd run -- he had nowhere to run \emph{to},
but it still made him feel claustrophobic. This was not good. This
was very, very bad.

Here's how bad it was: ``You're going to send me to
\emph{military school}?''

``Not military school,'' Ms Ramirez said. She said it with that
maddening, patronizing guidance-counsellor tone. ``The Martindale
Academy has no military or martial component. It's merely a very
structured, supervised environment. They have a fantastic track
record in helping students like you concentrate on grades and pull
themselves out of academic troubles. They've got a beautiful campus
in a beautiful location, and Martindale boys go on to fill many
important --''

And on and on. She'd swallowed the sales brochure like a burrito
and now it was rebounding on her. He tuned her out and looked at
his father. Benny Rosenbaum wasn't the sort of person you could
read easily. The people who worked for him at Rosenbaum Shipping
and Logistics called him The Wall, because you couldn't get
anything past him, under him, through him, or over him. Not that he
was a hardcase, but he couldn't be swayed by emotional arguments:
if you tried to approach him with anything less than fully
computerized logic, you might as well forget it.

But there were little tells, little ways you could figure out what
the weather was like in old Benny. That thing he was doing with his
watch strap, working at the catch, that was one of them. So was the
little jump in the hinge of his jaw, like he was chewing an
invisible wad of gum. Combine those with the fact that he was away
from his work in the middle of the day, when he should be making
sure that giant steel containers were humming around the globe --
well, for Leonard, it meant that the lava was pretty close to the
surface of Mount Benny this afternoon.

He turned to his dad. ``Shouldn't we be talking about this as a
family, Dad? Why are we doing this here?''

Benny regarded him, fiddled with his watch strap, nodded at the
guidance counsellor and made a little ``go-on'' gesture that betrayed
nothing.

``Leonard,'' she said. ``Leonard, you need to understand just how
serious this has become. You're one term paper away from flunking
two of your subjects: history and biology. You've gone from being
an A student in math, English and social studies to a C-minus. At
this rate, you'll have blown the semester by Thanksgiving. Put it
this way: you've gone from being in the ninetieth percentile of
Ronald Regan Secondary School Sophomores to the \emph{twelfth}.
This is a signal, Leonard, from you to us, and it's signalling,
S-O-S, S-O-S.''

``We thought you were on drugs,'' his father said, absolutely calm.
``We actually tested a hair follicle from your pillow. I had a guy
follow you around. Near as I can tell, you smoke a little pot with
your friends, but you don't actually see your friends anymore, do
you?''

``You tested my hair?''

His father made that go-on gesture of his, an old favorite of his.
``And had you followed. Of course we did. We're in charge of you.
We're responsible for you. We don't own you, but if you screw up so
bad that you end up spending the rest of your life as a bum, it'll
be down to us, and we'll have to bail you out. You understand that,
Leonard? We're responsible for you, and we'll do whatever we have
to in order to make sure you don't screw up your life.''

Leonard bit back a retort. The sinking feeling that had started
with the crushing of his earwig had sunk as low as it would go. Now
his palms were sweating, his heart was racing, and he had no idea
what would come out of his mouth the next time we spoke.

``We used to call this an intervention, when I was your age,'' the
vice-principal said. He still looked like the real-estate agent
he'd been before he switched to teaching, the last time the market
had crashed. He was affable, inoffensive, his eyes wide and
trustworthy. They called him Babyface Adams in the halls. But
Leonard knew about salesmen, knew that no matter how friendly they
appeared, they were always on the lookout for weaknesses to
exploit. ``And we'd do it for drug addicts. But I don't think you're
addicted to drugs. I think you're addicted to games.''

``Oh come \emph{on},'' Leonard said. ``There's no such thing. I can
show you the research papers. Game addiction? That's just something
they thought up to sell newspapers. Dad, come on, you don't really
believe this stuff, do you?''

His dad pointedly refused to meet his gaze, directing his attention
to the Vice-Principal.

``Leonard, we know you're a very smart young man, but no one is so
smart as to never need help. I don't want to argue definitions of
addictions with you --''

``\emph{Because you'll lose.}'' Leonard spat it out, surprising
himself with the vehemence. Old Babyface smiled his affable,
salesman's smile:
\emph{Oh yes, good sir, you're certainly right there, very clever of you. Now, may I show you something in a mock-Tudor split-level with a three-car garage and an above-ground pool?}

``You're a very smart young man, Leonard. It doesn't matter if
you're medically addicted, psychologically dependent, or just --''
he waved his hands, looking for the right words -- ``or if you just
spend too darn much time playing games and not enough time in the
real world. None of that matters. What matters is that you're in
trouble. And we're going to help you with that. Because we care
about you and we want to see you succeed.''

It suddenly sank in. Leonard knew how these things went. Somewhere,
right now, Officer Turner was cleaning out his locker and loading
its contents into a couple of paper Trader Joe's grocery sacks.
Somewhere, some secretary was taking his name off of the rolls of
each of his classes. Right now, his mother was packing his suitcase
back at home, filling it with three or four changes of clothes, a
fresh toothbrush -- and nothing else. When he left this room, he'd
disappear from Orange County as thoroughly as if he'd been snatched
off the street by serial killers.

Only it wouldn't be his mutilated body that would surface in a few
months time, decomposed and grisly, an object lesson to all the
kiddies of Ronald Reagan High to be on the alert for dangerous
strangers. It would be his mutilated \emph{personality} that would
surface, a slack-jawed pod-person who'd been crammed into the
happy-well-adjusted-citizen mold that would carry him through an
adulthood as a good, trouble-free worker-bee in the hive.

``Dad, come \emph{on}. You can't just do this to me! I'm your son! I
deserve a chance to pull my grades up, don't I? Before you send me
off to some brainwashing center?''

``You had your chance to pull your grades up, Leonard,'' Ms Ramirez
said, and the Vice-Principal nodded vigorously. ``You've had all
semester. If you plan on graduating and going on to university,
this is the time to do something drastic to make sure that
happens.''

``It's time to go,'' his father said, ostentatiously checking his
watch. Honestly, who still wore a watch? He had a phone, Leonard
knew, just like all normal people. An old-fashioned wind-up watch
was about as useful in this day and age as an ear-trumpet or a suit
of chain-mail. He had a whole case full of them -- dozens of them.
His father could have all the ridiculous affectations and hobbies
he wanted, spend a small fortune on them, and no one wanted to send
\emph{him} off to the nuthouse.

It was so goddamned \emph{unfair}. He wanted to shout it as they
led him out to his father's impeccable little Huawei Darter. He
bought new one every year, getting a chunky discount straight from
the factory, who loaded his personal car into its own container and
craned it into one of Dad's big ships in port in Guangzhou. The car
smelled of the black licorice sweets that Dad sucked on, and of the
giant steel thermos-cup of coffee that Dad slipped into the
cup-holder every morning, refilling through the day at a bunch of
diners where they called him by his first name and let him run a
tab.

And outside the windows, through the subtle grey tint, the streets
of Anaheim whipped past, rows of identical houses branching off of
a huge, divided arterial eight-lane road. He'd known these streets
all his life, he'd walked them, met the panhandlers that worked the
tourist trade, the footsore Disney employees who'd missed the
shuttle, hiking the mile to the cast-member parking, the retired
weirdos walking their dogs, the other larval Orange County
pod-people who were still too young or poor or unlucky to have a
car.

The sky was that pure blue that you got in OC, no clouds, a
postcard smiley-face sun nearly at noontime high, perfect for
tourist shots. Leonard saw it all for the first time, really
\emph{saw} it, because he knew he was seeing it for the last time.

``It's not so bad,'' his dad said. ``Stop acting like you're going to
prison. It's a swanky boarding school, for chrissakes. And not one
of those schools where they beat you down in the bathroom or
anything. They're practically hippies up there. Your mother and I
aren't sending you to the gulag, kid.''

``It doesn't matter what you say, Dad. Just forget it. Here's the
facts: you've kidnapped me from my school and you're sending me
away to some place where they're supposed to 'fix' me. You haven't
given me any say in this. You haven't consulted me. You can say how
much you love me, how much it's for my own good, talk and talk and
talk, but it won't change those facts. I'm sixteen years old, Dad.
I'm as old as Zaidy Shmuel was when he married Bubbie and came to
America, you know that?''

``That was during the war --''

``Who cares? He was your grandfather, and he was old enough to start
a family. You can bet your ass he wouldn't have stood still for
being kidnapped --'' His father snorted. ``\emph{Kidnapped} because
his hobbies weren't his parents' idea of a good time. God! What the
hell is the matter with you? I always knew you were kind of a
prick, but --''

His father calmly steered the car to the curb and pulled over,
changing three lanes smoothly, with a shoulder-check before each,
weaving through the tourist traffic and gardeners' pickup trucks
without raising a single horn. He popped the emergency brake with
one hand and his seatbelt with the other, twisting in his seat to
bring his face right up to Leonard's.

``You are on thin goddamned ice, kid. You can make me the villain if
you want to, if you need to, but you know, somewhere in that
hormone-addled teenaged brain of yours, that this was \emph{your}
doing. How many times, Leonard? How many times have we talked to
you about balance, about keeping your grades up, taking a little
time out of your game? How many chances did you get before this?''

Leonard laughed hotly. There were tears of rage behind his eyes,
trying to get out. He swallowed hard. ``Kidnapped,'' he said.
``Kidnapped and shipped away because you don't think I'm getting
good enough grades in math and English. Like any of it matters --
when was the last time you solved a quadratic equation Dad? Who
\emph{cares} if I get into a good university? What am I going to
get a degree in that will help me survive the next twenty years?
What did you get your degree in, again, Dad? Oh, that's right,
\emph{Ancient Languages.} Bet \emph{that} comes up a lot when
you're shipping giant containers of plastic garbage from China,
huh?''

His father shook his head. Behind them, cars were braking and
honking at each other as they maneuvered around the stopped Huawei.
``This isn't about me, son. This is about you -- about pissing away
your life on some stupid game. At least speaking Latin helps me
understand Spanish. What are you going to make of all your hours
and years of killing dragons?''

Leonard fumed. He knew the answer to this, somewhere. The games
were taking over the world. There was money to be made there. He
was learning to work on teams. All this and more, these were the
reasons for playing, and none of them were as important as the most
important reason: it just \emph{felt right}, adventuring in-world
--

There was a particularly loud shriek of brakes from behind them,
and it kept coming, getting louder and louder, and there was a
blare of horns, too, and the sound didn't stop, got louder than you
could have imagined it getting. He turned his head to look over his
shoulder and --

\emph{Crash}

The car seemed to leap into the air, rising up first on its front
tires in a reverse-wheelie and then the front wheels spun and the
car shot forward ten yards in a second. There was the sound of
crumpling metal, his father's curse, and then a clang like temple
bells as his head bounced off the dashboard. The world went dark.

\tb

\shopad{This scene is dedicated to New York City's Books of Wonder, the oldest and largest kids' bookstore in Manhattan. They're located just a few blocks away from Tor Books' offices in the Flatiron Building and every time I drop in to meet with the Tor people, I always sneak away to Books of Wonder to peruse their stock of new, used and rare kids' books. I'm a heavy collector of rare editions of Alice in Wonderland, and Books of Wonder never fails to excite me with some beautiful, limited-edition Alice. They have tons of events for kids and one of the most inviting atmospheres I've ever experienced at a bookstore.}
{\href{http://www.booksofwonder.com/}{Books of Wonder}: 18 West 18th St, New York, NY 10011 USA +1 212 989 3270}

Mala was in the world with a small raiding party, just a few of her
army. It was late -- after midnight -- and Mrs Dibyendu had turned
the cafe over to her idiot nephew to run things. These days, the
cafe stayed open when Mala and her army wanted to use it, day or
night, and there were always soldiers who'd vie for the honor of
escorting General Robotwallah home afterwards. Mamaji -- Mamaji had
a new fine flat, with two complete rooms, and one of them was all
for Mamaji alone, hers to sleep in without the snuffling and
gruffling of her two children. There were places in Dharavi where
ten or fifteen might have shared that room, sleeping on coats -- or
each other. Mamaji had a mattress, brought to her by a strong young
man from Chor Bazaar, carried with him on the roof of the Marine
Line train through the rush hour heat and press of bodies.

Mamaji didn't complain when Mala played after midnight.

``More, just there,'' Sushant said. He was two years older than her,
the tallest of them all, with short hair and a crazy smile that
reminded her of the face of a dog that has had its stomach rubbed
into ecstasy.

And there they were, three mecha in a triangle, methodically
clubbing zombies in the head, spattering their rotten brains and
dropping them into increasing piles. Eventually, the game would
send out ghouls to drag away the bodies, but for now, they piled
waist deep around the level one mechas.

``I have them,'' Yasmin said, her scopes locking on. This was a new
kind of mission for them, wiping out these little trios of mecha
who were grinding endlessly against the zombies. Mr Banerjee had
tasked them to this after the more aggressive warriors had been
hunted to extinction by their army. According to Mr Banerjee, these
were each played by a single person, someone who was getting paid
to level up basic mecha to level four or five, to be sold at
auction to rich players. Always in threes, always grinding the
zombies, always in this part of the world, like vermin.

``Fire,'' she said, and the pulse weapons fired concentric rings of
force into the trio. They froze, systems cooked, and as Mala
watched, the zombies swarmed over the mechas, toppling them,
working relentlessly at them, until they had found their way
inside. A red mist fountained into the sky as they dismembered the
pilots.

``Nice one,'' she said, arching her back over her chair, slurping the
dregs of a cup of chai that had grown cold at her side. Mrs
Dibyendu's idiot nephew was standing barefoot in the doorway of the
cafe, spitting betel into the street, the sweet smell wafting back
to her. The sleep was gathering in her mind, waiting to pounce on
her, so it was time to go. She turned to tell her army so when her
headphones filled with the thunder of incoming mechas, and
\emph{lots} of them.

She slammed her bottom down into the seat and spun around, fingers
flying to the keyboard, eyes on the screen. The enemy mecha were
coming in locked in a megamecha configurations, fifteen -- no
\emph{twenty} -- of them joined together to form a bot so huge that
she looked like a gnat next to it.

``To me!'' she cried, and ``Formation,'' and her soldiers came to their
keyboard, her army initiating their own megamecha sequence, but it
took too long and there weren't enough of them, and though they
fought bravely, the giant enemy craft tore them to pieces, lifting
each warbot and peering inside its cowl as it ripped open the armor
and dropped the squirming pilot to the surging zombie tide at its
feet. Too late, Mala remembered her strategy, remembered what it
had been like when she had \emph{always} commanded the weaker
force, the defensive footing she should have put her army on as
soon as she saw how she was outmatched.

Too late. An instant later, her own mecha was in the enemy's
clutches, lifted to its face, and as she neared it, the lights on
her console changed and a soft klaxon sounded: the bot was
attempting to infiltrate her own craft's systems, to interface with
them, to pwn them. That was another game within this game, the
hack-and-be-hacked game, and she was very good at it. It involved
solving a series of logic puzzles, solving them faster than the
foe, and she clicked and typed as she figured out how to build a
bridge using blocks of irregular size, as she figured out how to
open a lock whose tumblers had to be clicked just so to make the
mechanism work, as she figured out --

She wasn't fast enough. Her army gathered around her as her console
locked up, the enemy inside her mecha now, running it from
bootloader to flamethrower.

``Hello,'' a voice said in her headphones. That was something you
could do, when you controlled another player's armor -- you could
take over its comms. She thought of yanking out the headphones and
switching to speaker so that her army could listen in too, but some
premonition stayed her hand. This enemy had gone to some trouble to
talk to her, personally, so she would hear what it had to say.

``My name is Big Sister Nor,'' she said, and it \emph{was} a she, a
woman's voice, no, a \emph{girl's} voice -- maybe something in
between. Her Hindi was strangely accented, like the Chinese actors
in the filmi shows she'd seen. ``It's been a pleasure to fight you.
Your guild did very well. Of course, we did better.'' Mala heard a
ragged cheer and realized that there were dozens of enemies on the
chat channel, all listening in. What she had mistaken for static on
the channel was, in fact, dozens of enemies, somewhere in the
world, all breathing into their microphones as this woman spoke.

``You are very good players,'' Mala said, whispering it so that only
her mic heard.

``I'm not just a player, and neither are you, my dear.'' There was
something sisterly in that voice, none of the gloating
competitiveness that Mala felt for the players she'd bested in the
game before. In spite of herself, Mala found she was smiling a
little. She rocked her chin from side to side --
\emph{Oh, you're a clever one, do go on} -- and her soldiers around
her made the same gesture.

``I know why you fight. You think you're doing an honest job of
work, but have you ever stopped to consider why someone would pay
you to attack other workers in the game?''

Mala shooed away her army, making a pointed gesture toward the
door. When she was alone, she said, ``Because they muck up the game
for the real players. They interfere.''

The giant mecha shook its head slowly. ``Are you really so blind? Do
you think the syndicate that pays you does so because they care
about whether the game is \emph{fun}? Oh, dear.''

Mala's mind whirred. It was like solving one of those puzzles. Of
course Mr Banerjee didn't care about the other players. Of course
he didn't work for the game. If he worked for the game, he could
just suspend the accounts of the players Mala fought. Cleaner and
neater. The solution loomed in her mind's eye. ``They're business
rivals, then?''

``Oh yes, you are as clever as I thought you must be. Yes indeed.
They are business rivals. Somewhere, there is a group of players
just like them, being paid to level up mecha, or farm gold, or
acquire land, or do any of the other things that can turn labor
into money. And who do you suppose the money goes to?''

``To my boss,'' she said. ``And his bosses. That's how it goes.''
Everyone worked for someone.

``Does that sound fair to you?''

``Why not?'' Mala said. ``You work, you make something or do
something, and the person you do it for pays you something for your
work. That's the world, that's how it works.''

``What does the person who pays you do to earn his piece of your
labor?''

Mala thought. ``He figures out how to turn the labor into money. He
pays me for what I do. These are stupid questions, you know.''

``I know,'' Big Sister Nor said. ``It's the stupid questions that have
some of the most surprising and interesting answers. Most people
never think to ask the stupid questions. Do you know what a union
is?''

Mala thought. There were unions all over Mumbai, but none in
Dharavi. She'd heard many people speak of them, though. ``A group of
workers,'' she said. ``Who make their bosses pay them more.'' She
thought about all she'd heard. ``They stop other workers from taking
their jobs. They go on strike.''

``That's what unions \emph{do}, all right. But it's not much of a
sense of what they are. Tell me this: if you went to your boss and
asked for more money, shorter hours, and better working conditions,
what do you think he'd say?''

``He'd laugh at me and send me away,'' Mala said. It was an
unbelievably stupid question.

``You're almost certainly right. But what if all the workers he went
to said the same thing? What if, everywhere he went, there were
workers saying, 'We are worth so much,' and 'We will not be treated
this way,' and 'You cannot take away our jobs unless there is a
just reason for doing so'? What if all workers, everywhere,
demanded this treatment?''

Mala found she was shaking her head. ``It's a ridiculous idea.
There's always someone poor who'll take the job. It doesn't matter.
It won't work.'' She found that she was furious. ``Stupid!''

``I admit that it's all rather improbable,'' the woman said, and
there was an unmistakable tone of amusement in her voice. ``But
think for a moment about your employer. Do you know where his
employers are? Do you know where the players you're fighting are?
Where their customers are? Do you know where I am?''

``I don't see why that matters --''

``Oh, it matters. It matters because although all these people are
all over the world, there's no real distance between them. We chat
here like neighbors, but I am in Singapore, and you are in India.
Where? Delhi? Kolkata? Mumbai?''

``Mumbai,'' she admitted.

``You don't sound like Mumbai,'' she said. ``You have a lovely accent.
Uttar Pradesh?''

Mala was surprised to hear the state of her birth and her village
guessed so easily. ``Yes,'' she said. She was a girl from the
village, she was General Robotwallah and this woman had taken the
measure of her very quickly.

``This game is headquartered in America, in a city called Atlanta.
The corporation is registered in Cyprus, in Europe. The players are
all over the world. These ones that you've been fighting are in
Vietnam. We'd been having a lovely conversation before you came and
blew them all to pieces. We are everywhere, but we are all here.
Anyone your boss ever hired to do your job would end up here, and
we could find that worker and talk to them. Wherever your boss
goes, his workers will all come and work here. And we will have a
chat like this with them, and talk to them about what a world we
could have, if all workers cooperated to protect each others'
interests.''

Mala was still shaking her head. ``They'd just blow you away. Hire
an army like me. It's a stupid idea.''

The giant metamecha lifted her up to its face, where its giant
teeth champed and clanged. ``Do you think there's an army that could
best us?''

Mala thought that maybe her army could, if they were in force, if
they were prepared. Then she thought of how much successful war
you'd have to persecute to win one of these giant beasts. ``Maybe
not. Maybe you can do what you say you can do.'' She thought some
more. ``But in the meantime, we wouldn't have any work.''

The giant metal face nodded. ``Yes, that's true. At first you may
not find yourself with your wages. And maybe your fellow workers
would contribute a little to help you out. That's another thing
unions do -- it's called strike pay. But eventually, you, and me,
and all of us, would enjoy a world where we are paid a living wage,
and where we labor under livable conditions, and where our
workplaces are fair and decent. Isn't that worth a little
sacrifice?''

There it was, ``You ask me to make a sacrifice. Why should I
sacrifice? We are poor. We fight for a very little, because we have
even less. Why do you think that we should sacrifice? Why don't
\emph{you} sacrifice?''

``Oh, sister, we've all sacrificed. I understand that this is all
very new to you, and that it will take some getting used to. I'm
sure we'll see each other again, someday. After all, we all play in
the same world here, don't we?''

Mala realized that the breathing she'd heard, the other voices on
the chat channel, had all fallen silent. For a short time, it had
just been Mala and this woman who called her ``sister.''

``What is your name?''

``I'm Nor-Ayu,'' she said. ``But they call me 'Big Sister Nor.' All
over the world, they call me this. What do I call you?''

Mala's name was on the tip of her tongue, but she did not say it.
Instead, she said, ``General Robotwallah.''

``A very good name,'' Big Sister Nor said. ``It was my pleasure to
meet you.'' With that, the giant mecha dropped her and turned and
lumbered away, crushing zombies under its feet.

Mala stood up and felt the many pops and snaps of her spine and
muscles. She had been sitting for, oh, hours and hours.

She rolled her head from side to side on her neck, working out the
stiffness there and she saw Mrs Dibyendu's idiot nephew watching
her. His lip was pouched with reeking betel saliva, and he was
staring at her with a frankness that made her squirm right to the
pit of her stomach.

``You stayed behind for me,'' he said, a huge grin on his face. His
teeth were brown. He wasn't really an idiot -- not soft in the
head, anyway. But he was very thick and very slow, with a brutal
strength that Mrs Dibyendu always described as his ``special
fortitude.'' Mala thought he was just a thug. She'd seen him walking
in the narrow streets of Dharavi. He never shifted for women or old
people, making them go around him even when it meant stepping into
mud or worse. And he chewed betel all the time. Lots of people
chewed betel, it was like smoking, but her mother detested the
habit and had told her so many times that it was a ``low'' habit and
dirty that she couldn't help but think less of betel chewers.

He regarded her with his bloodshot eyes. She suddenly felt very
vulnerable, the way she'd felt all the time, when they'd first come
to Dharavi. She took a step to the right and he took a step to the
right as well. That was a line crossed: once he blocked her exit,
he'd announced his intention to hurt her. That was basic military
strategy. He had made the first move, so he had the initiative, but
he'd also showed his hand quickly, so --

She feinted left and he fell for it. She lowered her head like a
bull and butted it into the middle of his chest. Already
off-balance, he went down on his back. She didn't stop moving,
didn't look back, just kept going, thinking of that charging bull,
running over him as she made for the doorway without stopping. One
heel came down on his ribcage, the next on his face, mashing his
lips and nose. She wished that something had gone \emph{crunch} but
nothing did.

She was out the door in an instant and into the cool air of the
dark, dark Dharavi night. Around her, the sound of rats running
over the roofs, the distant sounds of the roads, snoring. And many
other, less identifiable sounds, sounds that might have been
lurkers hiding in the shadows around them. Muffled speech. A
distant train.

Suddenly, sending her army away didn't seem like such a good idea.

Behind her, she heard a much clearer sound of menace. The idiot
nephew crashing through the door, his shoes on the packed earth
road. She slipped back into an alley between two buildings, barely
wider than her, her feet splashing through some kind of warm liquid
that wafted an evil stench up to her nose. The idiot nephew
lumbered past into the night. She stayed put. He lumbered back,
looking in all directions for her.

There she stood, waiting for him to give up, but he would not. Back
and forth he charged. He'd become the bull, enraged, tireless,
stupid. She heard his voice rasping in his chest. She had her
mobile phone in her hand, her other hand cupped over it, shielding
the treacherous light it gave off from its tiny screen. It was
12:47 now, and she had never been alone at this hour in all her 14
years.

She could text someone in her army -- they would come to get her,
wouldn't they? If they were awake, or if their phones' chirps woke
them. No one was awake at this hour, though. And how to explain?
What to say?

She felt like an idiot. She felt ashamed. She should have predicted
this, should have been the general, should have employed strategy.
Instead, she'd gotten boxed in.

She could wait. All night, if necessary. No need to let her army
know of her weakness. Idiot nephew would tire or the sun would
rise, it was all the same to her.

Through the thin walls of the houses on either side of her, the
sound of snoring. The evil smell rose up from the liquid below her
in the ditch, and something slimy was squishing between her toes.
It burned at her skin. The rats scampered overhead, sounding like
rain on the tin roofs. Stupid, stupid, stupid, it was her mantra,
over and over in her mind.

The bull was tiring. The next time he passed, his breath came in
terrible wheezes that blew the stink of betel before him like sweet
rot. She could wait for his next pass, then run.

It was a good plan. She hated it. He had -- He'd threatened her.
He'd scared her. He should \emph{pay}. She was the General
Robotwallah, not merely some girl from the village. She was from
Dharavi, tough. Smart.

He wheezed past and she slipped out of the alley, her feet coming
free of the muck with audible \emph{plops}. He was facing away from
her still, hadn't heard her yet, and he had his back to her. The
stupid boys in her army only fought face to face, talked about the
``honor'' of hitting from behind. Honor was just stupid boy-things.
Victory beat honor.

She braced herself and ran toward him, both arms stiff, hands at
shoulder-height. She hit him high and kept moving, the way he had
before, and down he fell again, totally unprepared for the assault
from the rear. The sound he made on the dirt was like the sound of
a goat dropping at the butcher's block. He was trying to roll over
and she turned around and ran at him, jumping up in the air and
landing with both muddy feet on his head, driving his face into the
mud. He shouted in pain, the sound muffled by the dirt, and then
lay, stunned.

She went back to him then, and knelt at his head, his hairy earlobe
inches from her lips.

``I wasn't waiting for you at the cafe. I was minding my own
business,'' she said. ``I don't like you. You shouldn't chase girls
or the girls might turn around and catch you. Do you understand me?
Tell me you understand me before I rip out your tongue and wipe
your ass with it.'' They talked like this on the chat-channels for
the games all the time, the boys did, and she'd always disapproved
of it. But the words had power, she could feel it in her mouth, hot
as blood from a bit tongue.

``Tell me you understand me, idiot!'' she hissed.

``I understand,'' he said, and the words came mashed, from mashed
lips and a mashed nose.

She turned on her heel and began to walk away. He groaned behind
her, then called out, ``Whore! Stupid whore!''

She didn't think, she just acted. Turned around, ran at his
still-prone body, indistinct in the dusk, one step, two step, like
a champion footballer coming in for a penalty kick and then she
\emph{did} kick him, the foetid water spraying off her shoe's
saturated toe as it connected with his big, stupid ribcage.
Something snapped in there -- maybe several somethings, and oh,
didn't that feel \emph{wonderful}?

He was every man who'd scared her, who'd shouted filthy things
after her, who'd terrorized her mother. He was the bus driver who'd
threatened to put them out on the roadside when they wouldn't pay
him a bribe. Everything and everyone that had ever made her feel
small and afraid, a girl from the village. All of them.

She turned around. He was clutching at his side and blubbering now,
crying stupid tears on his stupid cheeks, luminous in the smudgy
moonlight that filtered through the haze of plastic smoke that hung
over Dharavi. She would up and took another pass at him, one step,
two step, \emph{kick}, and \emph{crunch}, that satisfying sound
from his ribs again. His sobs caught in his chest and then he took
a huge, shuddering breath and \emph{howled} like a wounded cat in
the night, screamed so loud that here in Dharavi, the lights came
on and voices came to the windows.

It was as though a spell had been broken. She was shaking and
drenched in sweat, and there were people peering at her in the
dark. Suddenly she wanted to be home as fast as possible, if not
faster. Time to go.

She ran. Mala had loved to run through the fields as a little girl,
hair flying behind her, knees and arms pumping, down the dirt
roads. Now she ran in the night, the reek of the ditch water
smacking her in the nose with each squelching step. Voices chased
her through the night, though they came filtered through the hammer
of her pulse in her ears and later she could not say whether they
were real or imagined.

But finally she was home and pelting up the steps to the
third-floor flat she had rented for her family. Her thundering
footsteps raised cries from the downstairs neighbors, but she
ignored them, fumbled with her key, let herself in.

Her brother Gopal looked up at her from his mat, blinking in the
dark, his skinny chest bare. ``Mala?''

``It's OK,'' she said. ``Nothing. Sleep, Gopal.''

He slumped back down. Mala's shoes stank. She peeled them off,
using just the tips of her fingers, and left them outside the door.
Perhaps they would be stolen -- though you would have to be
desperate indeed to steal those shoes. Now her feet stank. There
was a large jug of water in the corner, and a dipper. Carefully,
she carried the dipper to the window, opened the squealing shutter,
and poured the water slowly over the her feet, propping first one
and then the other on the windowsill. Gopal stirred again. ``Be
quiet,'' he said, ``it's sleep-time.''

She ignored him. She was still out of breath, and the reality of
what she'd done was setting in for her. She had kicked the idiot
nephew -- how many times? Two? Three? And something in his body had
gone \emph{crack} each time. Why had he blocked her? Why had he
followed her into the night? What was it that made the big and the
strong take such sport in terrorizing the weak? Whole groups of
boys would do this to girls and even grown women sometimes --
follow them, calling after them, touching them, sometimes it even
led to rape. They called it ``Eve-teasing'' and they treated it like
a game. It wasn't a game, not if you were the victim.

Why did they make her do it? Why did all of them make her do it?
The sound of the crack had been so satisfying then, and it was so
sickening now. She was shaking, though the night was so hot, one of
those steaming nights where everything was slimy with the
low-hanging, soupy moisture.

And she was crying, too, the crying coming out without her being
able to control it, and she was ashamed of that, too, because
that's what a girl from the village would do, not brave General
Robotwallah.

Calloused hands touched her shoulders, squeezed them. The smell of
her mother in her nose: clean sweat, cooking spice, soap. Strong,
thin arms encircled her from behind.

``Daughter, oh daughter, what happened to you?''

And she wanted to tell Mamaji everything, but all that came out
were cries. She turned her head to her mother's bosom and heaved
with the sobs that came and came and came in waves, feeling like
they'd turn her inside out. Gopal got up and moved into the next
room, silent and scared. She noticed this, noticed all of it as
from a great distance, her body sobbing, her mind away somewhere,
cool and remote.

``Mamaji,'' she said at last. ``There was a boy.''

Her mother squeezed her harder. ``Oh, Mala, sweet girl --''

``No, Mamaji, he didn't touch me. He tried to. I knocked him down.
Twice. And I kicked him and kicked him until I heard things
breaking, and then I ran home.''

``Mala!'' her mother held her at arm's length. ``Who was he?'' Meaning,
\emph{Was he someone who can come after us, who can make trouble for us, who could ruin us here in Dharavi?}

``He was Mrs Dibyendu's nephew, the big one, the one who makes
trouble all the time.''

Her mothers fingers tightened on her arms and her eyes went wide.

``Oh, Mala, Mala -- oh, no.''

And Mala knew exactly what her mother meant by this, why she was
consumed with horror. Her relationship with Mr Banerjee came from
Mrs Dibyendu. And the flat, their lives, the phone and the clothes
they wore -- they all came from Mr Banerjee. They balanced on a
shaky pillar of relationships, and Mrs Dibyendu was at the bottom
of it, all resting on her shoulders. And the idiot nephew could
convince her to shrug her shoulders and all would come tumbling
down -- the money, the security, all of it.

That was the biggest injustice of all, the injustice that had
driven her to kick and kick and kick -- this oaf of a boy knew that
he could get away with his grabbing and intimidation because she
couldn't afford to stop him. But she had stopped him and she could
not -- would not -- be sorry.

``I can talk with Mr Banerjee,'' she said. ``I have his phone number.
He knows that I'm a good worker -- he'll make it all better. You'll
see, Mamaji, don't worry.''

``Why, Mala, why? Couldn't you have just run away? Why did you have
to hurt this boy?''

Mala felt some of the anger flood back into her. Her mother, her
own mother --

But she understood. Her mother wanted to protect her, but her
mother wasn't a general. She was just a girl from the village, all
grown up. She had been beaten down by too many boys and men, too
much hurt and poverty and fear. This was what Mala was destined to
become, someone who ran from her attackers because she couldn't
afford to anger them.

She wouldn't do it.

No matter what happened with Mr Banerjee and Mrs Dibyendu and her
stupid idiot nephew, she was not going to become that person.

\tb

\shopad{This scene is dedicated to Borders, the global bookselling giant that you can find in cities all over the world -- I'll never forget walking into the gigantic Borders on Orchard Road in Singapore and discovering a shelf loaded with my novels! For many years, the Borders in Oxford Street in London hosted Pat Cadigan's monthly science fiction evenings, where local and visiting authors would read their work, speak about science fiction and meet their fans. When I'm in a strange city (which happens a lot) and I need a great book for my next flight, there always seems to be a Borders brimming with great choices -- I'm especially partial to the Borders on Union Square in San Francisco.}
{\href{http://www.bordersstores.com/locator/locator.jsp}{Borders worldwide}}

If you want to get rich without making anything or doing anything
that anyone needs or wants, you need to be \emph{fast}.

The technical term for this is \emph{arbitrage}. Imagine that you
live in an apartment block and it's snowing so hard out that no one
wants to dash out to the convenience store. Your neighbor to the
right, Mrs Hungry, wants a banana and she's willing to pay \$0.50
for it. Your neighbor to the left, Mr Full, has a whole cupboard
full of bananas, but he's having a hard time paying his phone bill
this month, so he'll sell as many bananas as you want to buy for
\$0.30 apiece.

You might think that the neighborly thing to do here would be to
call up Mrs Hungry and tell her about Mr Full, letting them
consummate the deal. If you think that, forget getting rich without
doing useful work.

If you're an arbitrageur, then you think of your neighbors'
regrettable ignorance as an opportunity. You snap up all of Mr
Full's bananas, then scurry over to Mrs Hungry's place with your
hand out. For every banana she buys, you pocket \$0.20. This is
called arbitrage.

Arbitrage is a high-risk way to earn a living. What happens if Mrs
Hungry changes her mind? You're stuck holding the bananas, that's
what.

Or what happens if some other arbitrageur beats you to Mrs Hungry's
door, filling her apartment with all the bananas she could ever
need? Once again, you're stuck with a bunch of bananas and nowhere
to put them (though a few choice orifices do suggest themselves
here).

In the real world, arbitrageurs don't drag around bananas -- they
buy and sell using networked computers, surveying all the
outstanding orders (``bids'') and asks, and when they find someone
willing to pay more for something than someone else is paying for
it, they snap up that underpriced item, mark it up, and sell it.

And this happens very, very quickly. If you're going to beat the
other arbitrageurs with the goods, if you're going to get there
before the buyer changes her mind, you've got to move faster than
the speed of thought. Literally. Arbitrage isn't a matter of a
human being vigilantly watching the screens for price-differences.

No, arbitrage is all done by automated systems. These little
traderbots rove the world's networked marketplaces, looking for
arbitrage opportunities, buying something and selling it in less
than a microsecond. A good arbitrage house conducts a
\emph{billion} or more trades every day, squeezing a few cents out
of each one. A billion times a few cents is a lot of money -- if
you've got a fast computer cluster, a good software engineer, and a
blazing network connection, you can turn out
\emph{ten or twenty million} dollars a day.

Not bad, considering that all you're doing is exploiting the fact
that there's a person over here who wants to buy something and a
person over there who wants to sell it. Not bad, considering that
if you and all your arbitraging buddies were to vanish tomorrow,
the economy and the world wouldn't even notice. No one needs or
wants your ``service'' but it's still a sweet way to get rich.

The best thing about arbitrage is that you don't need to know a
single, solitary thing about the stuff you're buying and selling in
order to get rich off of it. Whether it's bananas or a vorpal
blade, all you need to know about the things you're buying is that
someone over \emph{here} wants to buy them for more than someone
over \emph{there} wants to sell them for. Good thing, too -- if
you're closing the deal in less than a microsecond, there's no time
to sit down and google up a bunch of factoids about the
merchandise.

And the merchandise is pretty weird. Start with the fact that a lot
of this stuff doesn't even exist -- vorpal blades, grabthar's
hammers, the gold of a thousand imaginary lands.

Now consider that people trade more than gold: the game Gods sell
all kinds of funny money. How about this one:

Offered: Svartalfaheim Warriors bonds, worth 100,000 gold, payable
six months from now. This isn't even \emph{real} fake gold -- it's
the promise of real fake gold at some time in the future. Stick
that into the market for a couple months, baby, and watch it go.
Here's a trader who'll pay five percent more than it was worth
yesterday -- he's betting that the game will get more popular some
time between now and six months from now, and so the value of goods
in the game will go up at the same time.

Or maybe he's betting that the game Gods will just raise the price
on everything and make it harder to clobber enough monsters to
raise the gold to get it, driving away all but the hardest-core
players, who'll pay anything to get their hands on the dough.

Or maybe he's an idiot.

Or maybe he thinks \emph{you're} an idiot and you'll give him ten
percent tomorrow, figuring that he knows something you don't.

And if you think that's weird, here's an even better one!

Coca-Cola sells you a six-month Svartalfaheim Warriors 100,000 gold
bond, but you're worried that it's going to fall in value between
now and D-Day, when the bond matures. So you find another trader
and you ask him for some insurance: you offer him \$1.50 to insure
your bond. If the bond goes up in value, he gets to keep the \$1.50
and you get to keep the profits from the bond. If the bond goes
down in value, he has to pay you the difference. If that's more
than \$1.50, he's losing money.

This is basically an insurance policy. If you go to a
life-insurance company and ask them for a policy on your life,
they'll make a bet on how likely it is that you're going to croak,
and charge you enough that, on average, they make a profit
(providing they're guessing accurately at your chances of dying).
So if the trader you're talking to thinks that Svartalfaheim
Warriors is going to tank, he might charge you \$10, or \$100.

So far, so good, right?

Now, here's where it gets even weirder. Follow along.

Imagine that there's a third party to this transaction, some guy
sitting on the sidelines, holding onto a pot of money, trying to
figure out what to do with it. He watches you go to the trader and
buy an insurance policy for \$1.50 -- if Svartalfaheim Warriors
gets better, you're out \$1.50, if it gets worse, the trader has to
make up the difference.

After you've sealed your deal, this third party, being something of
a ghoul, goes up to the same trader and says, ``Hey, how about this?
I want to place the same bet you've just placed with that guy. I'll
give you \$1.50 and if his bond goes up, you keep it. If his bond
goes down, you pay me \emph{and} him the difference.'' Essentially,
this guy is betting that your bond is junk, and so maybe he finds a
taker.

Now he's got this bet, which is worth nothing if your bond goes up,
and worth some unknown amount if your bond craters. And you know
what he does with it?

\emph{He sells it}.

He packages it up and finds some sucker who wants to buy his \$1.50
bet on your bond for more than the \$1.50 he'll have to cough up if
your bond goes up. And the sucker buys it and then \emph{he} sells
it. And then another sucker buys it and \emph{he} sells it. And
before you know it, the 100,000 gold-piece bond you bought for \$15
has \$1,000 worth of bets hanging off of it.

And \emph{this} is the kind of thing an arbitrageur is buying and
selling. He's not carrying bananas from Mr Full to Mrs Hungry --
he's buying and selling bets on insurance policies on promises of
imaginary gold.

And this is what he calls an honest day's work.

Nice work if you can get it.

\tb

\shopad{This scene is dedicated to Compass Books/Books Inc, the oldest independent bookstore in the western USA. They've got stores up and down California, in San Francisco, Burlingame, Mountain View and Palo Alto, but coolest of all is that they run a killer bookstore in the middle of Disneyland's Downtown Disney in Anaheim. I'm a stone Disney park freak (see my first novel, Down and Out in the Magic Kingdom if you don't believe it), and every time I've lived in California, I've bought myself an annual Disneyland pass, and on practically every visit, I drop by Compass Books in Downtown Disney. They stock a brilliant selection of unauthorized (and even critical) books about Disney, as well as a great variety of kids books and science fiction, and the cafe next door makes a mean cappuccino.}
{\href{http://www.booksinc.net/NASApp/store/Product;jsessionid=abcF-ch09-pbU6m7ZRrLr?s=showproduct\&isbn=0765322166}{Compass Books/Books Inc}}

Matthew Fong and his employees raided through the night and into
the next day, farming as much gold as they could get out of their
level while the getting was good. They slept in shifts, and they
co-opted anyone who made the mistake of asking what they were up
to, dragooning them into mining the dungeon with them.

All the while, Master Fong was getting the gold out of their
accounts as fast as it landed in them. He knew that once the game
Gods got wind of his operation, they'd swoop in, suspend everyone's
accounts, and seize any gold they had in their inventory. The trick
was to be sure that there wasn't anything for them to seize.

So he hopped online and hit the big brokerage message-boards. These
weren't just grey-market, they were blackest black, and you needed
to know someone heavy to get in on them. Matthew's heavy was a guy
from Sichuan, skinny and shaky, with several missing teeth. He
called himself ``Cobra,'' and he'd been the one who'd introduced
Matthew to Boss Wing all those months before. Cobra worked for
someone who worked for someone who worked for one of the big
cartels, tough criminal organizations that had all the markets for
turning game-gold into cash sewn up.

Cobra had given him a login and a briefing on how to do deals on
the brokernet. Now as the night wore on, he picked his way through
the interface, listing his gold and setting an asking price that
was half of the selling price listed on the white, above-ground
gold-store that gweilos used to buy the game gold from the
brokers.

He waited, and waited, and waited, but no one bought his gold.
Every game world was divided into local servers and shards, and
when you signed up, you needed to set which server you wanted to
play on. Once you'd picked a server, you were stuck there -- your
toon couldn't just wander between the parallel universes. This made
buying and selling gold all the more difficult: if a gweilo wanted
to buy gold for his toon on server A, he needed to find a farmer
who had mined his gold on server A. If you mined all your gold on
server B, you were out of luck.

That's where the brokers came in. They bought gold from everyone,
and held it in an ever-shifting network of accounts, millions of
toons who fanned out all over the worlds and exchanged small
amounts of gold at irregular intervals, to fool the anti-laundering
snoops in the game logic that relentlessly hunted for farmers and
brokers to bust.

Avoiding those filters was a science, one that had been hammered
together over decades in the real world before it migrated to the
games. If a big pension fund in the real world wanted to buy half a
billion dollars' worth of stock in Google, the last thing they want
to do is tip off everyone else that they're about to sink that much
cash into Google. If they did, everyone else would snap up Google
stock before they could get to it, mark it up, and gouge them on
it.

So anyone who wants to buy a lot of anything -- who wants to move a
lot of money around -- has to know how to do it in a way that's
invisible to snoops. They have to be statistically insignificant,
which means that a single big trade has to be broken up into
millions of little trades that look like ordinary suckers buying
and selling a little stock for the hell of it.

No matter what secrets you're trying to keep and no matter who
you're trying to keep them from, the techniques are the same. In
every game world there were thousands of seemingly normal
characters doing seemingly normal things, giving each other
seemingly normal sums of money, but at the end of the day, it all
added up to millions of gold in trade, taking place right under the
noses of the game Gods.

Matthew down-priced his gold, seeking the price at which a broker
would deign to notice him and take it off of him. All the trading
took place in slangy, rapid Chinese -- that was one of the ways the
brokers kept their hold on the market, since there weren't that
many Russians and Indonesians and Indians who could follow it and
play along -- replete with insults and wheedles. Eventually,
Matthew found the magic price. It was lower than he'd hoped for,
but not by much, and now that he'd found it, he was able to move
the team's gold as fast as they could accumulate it, shuttling
dummy players in and out of the dungeon they were working to take
the cash to bots run by the brokers.

Finally, it dried up. First, the amount of gold in the dungeon
sharply decreased, with the gold dropping from 12,000 per hour to
8,000, then 2,000, then a paltry 100. The mareridtbane disappeared
next, which was a pity, because he was able to sell that directly,
hawking it in the big towns, pasting and pasting and pasting his
offer into the chat where the real players could see it. And then
in came the cops, moderators with special halos around them who
dropped canned lectures into the chat, stern warnings about having
violated the game's terms of service.

And then the account suspensions, the games vanishing from one
screen after another, popping like soap bubbles. They were all
dropped back to the login screens and they slumped, grinning crazy
and exhausted, in their seats, looking at each other in exhausted
relief. It was over, at last.

``How much?'' Lu asked, flung backwards over his chair, not opening
his eyes or lifting his head. ``How much, Master Fong?''

Matthew didn't have his notebooks anymore, so he'd been keeping
track on the insides of Double Happiness cigarette packages, long,
neat tallies of numbers. His pen flickered from sheet to sheet,
checking the math one final time, then, quietly, ``\$3,400.''

There was a stunned silence. ``How much?'' Lu had his eyes open now.

Matthew made a show of checking the figures again, but that's all
it was, a show. He knew that the numbers were right. ``Three
thousand, four hundred and two dollars and fourteen cents.'' It was
double the biggest score they'd ever made for Boss Wing. It was the
most money any of them had ever made. His share of it was more than
his father made in a month. And he'd made it in one night.

``Sorry, \emph{how much}?''

``8,080 bowls of dumplings, Lu. That much.''

The silence was even thicker. That was a lot of dumplings. That was
enough to rent their own place to use as a factory, a place with
computers and a fast internet connection and bedrooms to sleep in,
a place where they could earn and earn, where they could grow rich
as any boss.

Lu leapt out of his chair and whooped, a sound so loud that the
entire cafe turned to look at them, but they didn't care, they were
all out of their seats now, whooping and dancing around and hugging
each other.

And now it was the day, a new day, the sun had come up and gone
down and risen in their long labor in the cafe, and they had won.
It was a new day for them and for everyone around them.

They stepped out into the sun and there were people on the streets,
throngs buying and selling, touts hustling, pretty girls in good
clothes walking arm in arm under a single parasol. The heat of the
day was like a blast furnace after the air-conditioned cool of the
cafe, but that was good, too -- it baked out the funk of
cigarette-mouth, coffee-mouth, no-food-mouth. Suddenly, none of
them were sleepy. They all wanted to eat.

So Matthew took them out for breakfast. They were his team, after
all. They took over the back table at an Indian restaurant near the
train station, a place he'd overheard his uncle Yiu-Yu telling his
parents about, bragging about some business associate who took him
there. Very sophisticated. And he'd read so much about Indian food
in his comics, he couldn't wait to try some.

All the other customers in there were either foreigners or Hong
Kong people, but they didn't let that get to them. The boys sat at
their back table and played with their forks and ate plate after
plate of curry and fresh hot flatbreads called naan, and it was
delicious and strange and the perfect end to what had turned out to
be the perfect night.

Halfway through the dessert -- delicious mango ice-cream -- the
sleeplessness finally caught up with them all. They sat on their
seats in their torpor, hands over their bellies, eyes half-open,
and Matthew called for the check.

They stepped out again into the light. Matthew had decided to go to
his parents' place, to sleep on the sofa for a little while, before
figuring out what to do about his smashed room with its smashed
door.

As they blinked in the light, a familiar Wenjhou accented voice
said, ``You aren't a very smart boy, are you?''

Matthew turned. Boss Wing's man was there, and three of his
friends. They rushed forward and grabbed the boys before they could
react, one of them so big that he grabbed a boy in each hand and
nearly lifted them off their feet.

His friends struggled to get free, but Boss Wing's man methodically
slapped them until they stopped.

Matthew couldn't believe that this was happening -- in broad
daylight, right here next to the train station! People crossed the
street to avoid them. Matthew supposed he would have done so too.

Boss Wing's man leaned in so close Matthew could smell the fish
he'd had for lunch on his breath. ``Why are you a stupid boy,
Matthew? You didn't seem stupid when you worked for Boss Wing. You
always seemed smarter than these children.'' He flapped his hand
disparagingly at the boys. ``But Boss Wing, he trained you,
sheltered you, fed you, paid you -- do you think it's honorable or
fair for you to take all that investment and run out the door with
it?''

``We don't owe Boss Wing anything!'' Lu shouted. ``You think you can
make us work for him?''

Boss Wing's man shook his head. ``What a little hothead. No one
wants to force you to do anything, child. We just don't think it's
fair for you to take all the training and investment we made in you
and run across the street and start up a competing business. It's
not right, and Boss Wing won't stand for it.''

The curry churned in Matthew's stomach. ``We have the right to start
our own business.'' The words were braver than he felt, but these
were \emph{his} boys, and they gave him bravery. ``If Boss Wing
doesn't like the competition, let him find another line of work.''

Boss Wing's man didn't give him any forewarning before he slapped
Matthew so hard his head rang like a gong. He stumbled back two
steps, then tripped over his heels and fell on his ass, landing on
the filthy sidewalk. Boss Wing's man put a foot on his chest and
looked down at him.

``Little boy, it doesn't work like that. Here's the deal -- Boss
Wing understands if you don't want to work at his factory, that's
fine. He's willing to sell you the franchise to set up your own
branch operation of his firm. All you have to do is pay him a
franchise fee of 60 percent of your gross earnings. We watched your
gold-sales from Svartalfaheim. You can do as much of that kind of
work as you like, and Boss Wing will even take care of the sales
end of things for you, so you'll be free to concentrate on your
work. And because it's your firm, you get to decide how you divide
the money -- you can pay yourself anything you like out of it.''

Matthew burned with shame. His friends were all looking at him,
goggle eyed, scared. The weight from the foot on his chest
increased until he couldn't draw a breath.

Finally, he gasped out, ``\emph{Fine},'' and the pressure went away.
Boss Wing's man extended a hand, helped him to his feet.

``Smart,'' he said. ``I knew you were a smart boy.'' He turned to
Matthew's friends. ``Your little boss here is a smart man. He'll
take you places. You listen to him now.''

Then, without another word, he turned on his heel and walked away,
his men following him.

\tb

\shopad{This scene is dedicated to Anderson's Bookshops, Chicago's legendary kids' bookstore. Anderson's is an old, old family-run business, which started out as an old-timey drug-store selling some books on the side. Today, it's a booming, multi-location kids' book empire, with some incredibly innovative bookselling practices that get books and kids together in really exciting ways. The best of these is the store's mobile book-fairs, in which they ship huge, rolling bookcases, already stocked with excellent kids' books, direct to schools on trucks -- voila, instant book-fair!}
{\href{http://site.booksite.com/5156/search/?q=for\%20the\%20win\%20doctorow\&search=yes\&custcat=}{Anderson's Bookshops}: 123 West Jefferson, Naperville, IL 60540 USA +1 630 355 2665}

The car that had plowed into Wei-Dong's father's car was driven by
a very exasperated, very tired British man, fat and bald, with two
angry kids in the back seat and an angry wife in the front seat.

He was steadily, quietly cursing in British, which was a lot like
cursing in American, but with a lot more ``bloodies'' in it. He paced
the sidewalk beside the wrecked Huawei, his wife calling at him
from inside the car to get back in the bloody car, Ronald, but
Ronald wasn't having any of it.

Wei-Dong sat on the narrow strip of grass between the road and the
sidewalk, dazed in the noon sun, waiting for his vision to stop
swimming. Benny sat next to him, holding a wad of kleenex to
staunch the bleeding from his broken nose, which he'd bounced off
of the dashboard. Wei-Dong brought his hands up to his forehead to
finger the lump there again. His hands smelled of new plastic, the
smell of the airbag that he'd had to punch his way out of.

The fat man crouched next to him. ``Christ, son, you look like
you've been to the wars. But you'll be all right, right? Could have
been much worse.''

``Sir,'' Benny Rosenbaum said, in a quiet voice muffled by the
kleenex. ``Please leave us alone now. When the police come, we can
all talk, all right?''

``'Course, 'course.'' His kids were screaming now, hollering from the
back seat about getting to Disneyland, when were they getting to
Disneyland? ``Shut it, you monsters,'' he roared. The sound made
Wei-Dong flinch back. He wobbled to his feet.

``Sit down, Leonard,'' his father said. ``You shouldn't have gotten
out of the car, and you certainly shouldn't be walking around now.
You could have a concussion or a spinal injury. Sit down,'' he
repeated, but Wei-Dong needed to get off the grass, needed to walk
off the sick feeling in his stomach.

Uh-oh. He barely made it to the curb, hands braced on the crumpled,
flaking rear section of the Huawei, before he started to barf, a
geyser of used food that shot straight out of his guts and flew all
over the wreck of the car. A moment later, his father's hands were
on his shoulders, steadying him. Angrily, he shook them off.

There were sirens coming now, and the fat man was talking intensely
to old Benny, though it was quiet enough that Wei-Dong could only
make out a few words -- \emph{insurance, fault, vacation} -- all in
a wheedling tone. His father kept trying to get a word in, but the
guy was talking over him. Wei-Dong could have told him that this
wasn't a good strategy. Nothing was surer to make Volcano Benny
blow. And here it came.

``\emph{Shut your mouth for a second, all right? Just SHUT IT.}''

The shout was so loud that even the kids in the back seat went
silent.

``YOU HIT US, you goddamned idiot! We're not going to go halves on
the damage. We're not going to settle this for cash. I don't
\emph{care} if you're jetlagged, I don't \emph{care} if you didn't
buy the extra insurance on your rental car, I don't \emph{care} if
this will ruin your vacation. You could have killed us, you
understand that, moron?''

The man held up his hands and cringed behind them. ``You were parked
in the middle of the road, mate,'' he said, a note of pleading in
his voice.

Everyone was watching them, the kids and the guy's wife, the
rubberneckers who slowed down to see the accident. The two men were
totally focused on each other.

In other words, no one was watching Wei-Dong.

He thought about the sound his earwig made, crunching under his
father's steel-toed shoe, heard the sirens getting closer, and\ldots{}

He\ldots{}

Left.

He sidled away toward the shrubs that surrounded a mini-mall and
gas-station, nonchalant, clutching his school-bag, like he was just
getting his bearings, but he was headed toward a gap there, a
narrow one that he just barely managed to squeeze through. He
popped through into the parking lot around the mini-mall, filled
with stores selling \$3 t-shirts and snow-globes and large bottles
of filtered water. On this side of the shrubs, the world was normal
and busy, filled with tourists on their way to or from Disneyland.

He picked up his pace, keeping his face turned away from the stores
and the CCTV cameras outside of them. He felt in his pocket, felt
the few dollars there. He had to get away, far away, fast, if he
was going to get away at all.

And there was his salvation, the tourist bus that rolled through
the streets of the Anaheim Resort District, shuttling people from
hotels to restaurants to the parks, crowded with sugared-up kids
and conventioneers with badges hanging around their necks, and it
was trundling to the stop just a few yards away. He broke into a
run, stumbled from the pain that seared through his head like a
lightning bolt, then settled for walking as quickly as he could.
The sirens were very, very loud now, right there on the other side
of the shrubs, and he was almost at the bus and there was his
father's voice, calling his name and there was the bus and --

-- his foot came down on the bottom step, his back foot came up to
join it, and the impatient driver closed the doors behind him and
released the air-brake with a huge sigh and the bus lurched
forward.

``Wei-Dong Rosenbaum,'' he whispered to himself, ``you've just escaped
a parental kidnapping to a military school, what are you going to
do now?'' He grinned. ``I'm going to Disneyland!''

The bus trundled down Katella, heading for the bus-entrance, and
then it disgorged its load of frenetic tourists. Wei-Dong mingled
with them, invisible in the mass of humanity skipping past the
huge, primary-colored traffic pylons. He was on autopilot, remained
on autopilot as he unslung his school-bag to let the bored security
goon paw through it.

He'd had a Disneyland annual pass since he was old enough to ride
the bus. All the kids he knew had them too -- it beat going to the
mall after school, and even though it got boring after a while, he
could think of no better place to disappear into while thinking
through his next steps.

He walked down Main Street, heading for the little pink castle at
the end of the road. He knew that there were secluded benches on
the walkways around the castle, places where he could sit down and
think for a moment. His head felt like it was full of candy floss.

First thing he did after sitting down was check his phone. The
ringer had been off -- school rules -- but he'd felt it vibrating
continuously in his pocket. Fifteen missed calls from his father.
He dialled up his voicemail and listened to his dad rant about
coming back \emph{right now} and all the dire things that would
happen to him if he didn't.

``Kid, whatever you think you're doing, you're wrong about it.
You're going to come home eventually. The sooner you call me back,
the less trouble we're going to have. And the longer you wait --
\emph{you listen to this, Leonard} -- the longer you wait,
\emph{the worse it's going to be}. There are worse things than
boarding school, kid. Much, much worse.''

He stared vacantly at the sky, listening to this, and then he
dropped the phone as though he'd been scorched by it.

\emph{It had a GPS in it}. They were always using phones to find
runaways and bad guys and lost hikers. He picked the phone up off
the pavement and slid the back out and removed the battery, then
put it in his jacket pocket, returning the phone to his jeans. He
wasn't much of a fugitive.

The police had been on the way to the accident when he left. They'd
arrived minutes later. The old man had decided that he'd run away,
so he'd be telling the cops that. He was a minor, and truant, and
he'd been in a car accident, and hell, face it, his family was
rich. That meant that the police would pay attention to his dad,
which meant that they'd be doing everything they could to locate
him. If they hadn't yet figured out where his phone was, they'd
know soon enough -- they'd run the logs and find the call from
Disneyland to his voicemail.

He started moving, shoving his way through the crowds, heading back
up Main Street. He ducked around behind a barbershop quartet and
realized that he was standing in front of an ATM. They'd be
shutting down his card any second, too -- or, if they were smart,
they'd leave the card live and use it to track him. He needed cash.
He waited while a pair of German tourists fumbled with the machine
and then jammed his card into it and withdrew \$500, the most the
machine would dispense. He hit it again for another \$500,
self-conscious now of the inch-thick wad of twenties in his hand.
He tried for a third withdrawal, but the machine told him he'd gone
to his daily limit. He didn't think he had much more than \$1,000
in the bank, anyway -- that was several years' worth of birthday
money, plus a little from his summer job working at a Chinese PC
repair shop at a mini-mall in Irvine.

He folded the wad and stuck it in his pocket and headed out of the
park, not bothering with the hand-stamp. He started to head for the
street, but then he turned on his heel and headed toward the
Downtown Disney shopping complex and the hotels that attached to
it. There were cheap tour-buses that went from there up to LA, down
to San Diego, to all the airports. There was no easier, cheaper way
to get far from here.

The lobby of the Grand Californian Hotel soared to unimaginable
heights, giant beams criss-crossing through the cavernous space.
Wei-Dong had always liked this place. It always seemed so
\emph{rendered}, like an imaginary place, with the intricate marble
inlays on the floor, the ten-foot-high stained-glass panels set
into the sliding doors, the embroidered upholstery on the sofas.
Now, though, he just wanted to get through it and onto a bus to --

Where?

Anywhere.

He didn't know what he was going to do next, but one thing he did
know, he wasn't going to be sent away to some school for screwups,
kicked off the Internet, kicked off the games. His father wouldn't
have allowed anyone to do this to \emph{him}, no matter what
problems he was having. The old man would never let himself be
pushed around and shaken up like this.

His mother would worry -- but she always worried, didn't she? He'd
send her email once he got somewhere, an email every day, let her
know that he was OK. She was good to him. Hell, the old man was
good to him, come to that. Mostly. But he was seventeen now, he
wasn't a kid, he wasn't a broken toy to be shipped back to the
manufacturer.

The man behind the concierge desk didn't bat an eye when Wei-Dong
asked for the schedule for the airport shuttles, just handed it
over. Wei-Dong sat down in the darkest corner by the stone
fireplace, the most inconspicuous place in the whole hotel. He was
starting to get paranoid now, he could recognize the feeling, but
it didn't help soothe him as he jumped and stared at every Disney
cop who strolled through the lobby, doubtless he was looking as
guilty as a mass-murderer.

The next bus was headed for LAX, and the one after, for the Santa
Monica airport. Wei-Dong decided that LAX was the right place to
go. Not so he could get on a plane -- if his dad had called the
cops, he was sure they'd have some kind of trace on at the
ticket-sales windows. He didn't know exactly how that worked, but
he understood how bottlenecks worked, thanks to gaming. Right now,
he could be anywhere in LA, which meant that they'd have to devote
a gigantic amount of effort in order to find him. But if he tried
to leave by airplane, there'd be a much smaller number of places
they'd have to check to catch him -- the airline counters at four
or five airports in town -- and that was a lot more practical.

But LAX also had cheap buses to \emph{everywhere} in LA, buses that
went to every hotel and neighborhood. It would take a long time,
sure -- an hour and a half from Disneyland to LAX, another hour or
two to get back to LA, but that was fine. He needed time -- time to
figure out what he was going to do next.

Because when he was totally honest with himself, he had to admit
that he had no freaking idea.

\tb

\shopad{This scene is dedicated to the University Bookstore at the University of Washington, whose science fiction section rivals many specialty stores, thanks to the sharp-eyed, dedicated science fiction buyer, Duane Wilkins. Duane's a real science fiction fan -- I first met him at the World Science Fiction Convention in Toronto in 2003 -- and it shows in the eclectic and informed choices on display at the store. One great predictor of a great bookstore is the quality of the ``shelf review'' -- the little bits of cardboard stuck to the shelves with (generally hand-lettered) staff-reviews extolling the virtues of books you might otherwise miss. The staff at the University Bookstore have clearly benefited from Duane's tutelage, as the shelf reviews at the University Bookstore are second to none.}
{\href{http://www4.bookstore.washington.edu/\_trade/ShowTitleUBS.taf?ActionArg=Title\&ISBN=9780765322166}{The University Bookstore} 4326 University Way NE, Seattle, WA 98105 USA +1 800 335 READ}

Mala woke early, after a troubled sleep. In the village, she'd
often risen early, and listened to the birds. But there was no
birdsong when her eyes fluttered open, only the sussuration of
Dharavi -- cars, rats, people, distant factory noises, goats. A
rooster. Well, that was a kind of bird. A little smile touched her
lips, and she felt slightly better.

Not much, though. She sat up and rubbed her eyes, stretched her
arms. Gopal still slept, snoring softly, lying on his stomach the
way he had when he was a baby. She needed the toilet, and, as it
was light out, she decided that she would go out to the communal
one a little ways away, rather than using the covered bucket in the
room. In the village, they'd had a proper latrine, deep dug, with a
pot of clean water outside of it that the women kept filled all the
time. Here in Dharavi, the communal toilet was a much more
closed-in, reeking place, never very clean. The established
families in Dharavi had their own private toilets, so the public
ones were only used by newcomers.

It wasn't so bad this morning. There were ladies who got up even
earlier than her to slosh it out with water hauled from the nearby
communal tap. By nightfall, the reek would be eye-watering.

She loitered in the street in front of the house. It wasn't too hot
yet, or too crowded, or too noisy. She wished it was. Maybe the
noise and the crowds would drown out the worry racing through her
mind. Maybe the heat would bake it out.

She'd brought her mobile out with her. It danced with notifiers
about new things she could pay to see -- shows and cartoons and
political messages, sent in the night. She flicked them away
impatiently and scrolled through her address-book, stopping at Mr
Banerjee's name and staring at it. Her finger poised over the send
button.

It was too early, she thought. He'd be asleep. But he never was,
was he? Mr Banerjee seemed to be awake at all hours, messaging her
with new targets to take her army to. He'd be awake. He'd have been
up all night, talking to Mrs Dibyendu.

Her finger hovered over the Send button.

The phone rang.

She nearly dropped it in surprise, but she managed to settle it in
her hand and switch off the ringer, peer at the face. Mr Banerjee,
of course, as though he'd been conjured into her phone by her
thoughts and her staring anxiety.

``Hello?'' she said.

``Mala,'' he said. He sounded grave.

``Mr Banerjee.'' It came out in a squeak.

He didn't say anything else. She knew this trick. She used it with
her army, especially on the boys. Saying nothing made a balloon of
silence in your opponent's head, one that swelled to fill it, until
it began to echo with their anxieties and doubts. It worked very
well. It worked very well, even if you knew how it worked. It was
working well on her.

She bit her lip. Otherwise she would have blurted something, maybe
\emph{He was going to hurt me} or \emph{He had it coming} or
\emph{I did nothing wrong}.

Or, \emph{I am a warrior and I am not ashamed}.

\emph{There}. There was the thought, though it wanted to slip away
and hide behind \emph{He was going to hurt me}, that was the
thought she needed, the platoon she needed to bring to the fore.
She marshalled the thought, chivvied it, turned it into an orderly
skirmish line and marched it forward.

``Mrs Dibyendu's idiot nephew tried to assault me last night, in
case you haven't heard.'' She waited a beat. ``I didn't let him do
it. I don't think he'll try it again.''

There was a snort, very faint, down the phone line. A suppressed
laugh? Barely contained anger? ``I heard about it, Mala. The boy is
in the hospital.''

``Good,'' she said, before she could stop herself.

``One of his ribs broke and punctured his lung. But they say he'll
live. Still, it was quite close.''

She felt sick. Why? Why did it have to be this way? Why couldn't he
have left her alone? ``I'm glad he'll live.''

``Mrs Dibyendu called me in the night to tell me that her sister's
only son had been attacked. That he'd been attacked by a vicious
gang of your friends. Your 'army'.''

Now \emph{she} snorted. ``He says it because he's embarrassed to
have been so badly beaten by me, just me, just a girl.''

Again, the silence ballooned in the conversation.
\emph{He's waiting for me to say I'm sorry, that I'll make it up somehow, that he can take it from my wages.}
She swallowed.
\emph{I won't do it. The idiot made me attack him, and he deserved what he got.}

``Mrs Dibyendu,'' he began, then stopped. ``There are expenses that
come from something like this, Mala. Everything has a cost. You
know that. It costs you to play at Mrs Dibyendu's cafe. It costs me
to have you do it. Well, this has a cost, too.''

Now it was her turn to be quiet, and to think at him, as hard as
she can,
\emph{Oh yes, well, I think I already exacted payment from idiot nephew. I think he's paid the cost.}

``Are you listening to me?''

She made a grunt of assent, not trusting herself to open her
mouth.

``Good. Listen carefully. The next month, you work for \emph{me}.
Every rupee is mine, and I make this bad thing that you've brought
down on yourself go away.''

She pulled the phone away from her head as if it had gone red hot
and burned her. She stared at the faceplate. From very far away, Mr
Banerjee said, ``\emph{Mala?} \emph{Mala?}'' She put the phone back
to her head.

She was breathing hard now. ``It's impossible,'' she said, trying to
stay calm. ``The army won't fight without pay. My mother can't live
without my pay. We'll lose our home. No,'' she repeated, ``it's not
possible.''

``Not possible? Mala, it had better be possible. Whether or not you
work for me, I will have to make this right with Mrs Dibyendu. It's
my duty, as your employer, to do this. And that will cost money.
You have incurred a debt that I must settle for you, and that means
that you have to be prepared to settle with \emph{me}.''

``Then don't settle it,'' she said. ``Don't give her one rupee. There
are other places we can play. Her nephew brought it on himself. We
can play somewhere else.''

``Mala, did anyone \emph{see} this boy lay his hands on you?''

``No,'' she said. ``He waited until we were alone.''

``And why were you alone with him? Where was your army?''

``They'd already gone home. I'd stayed late.'' She thought of Big
Sister Nor and her metamecha, of the union. Mr Banerjee would be
even angrier if she told him about Big Sister Nor. ``I was studying
tactics,'' she said. ``Practicing on my own.''

``You stayed alone with this boy, in the middle of the night. What
happened, really, Mala? Did you want to see what it was like to
kiss him like a fillum star, and then it got out of control? Is
that how it happened?''

``\emph{No!}'' She shouted it so loud that she heard people groaning
in their beds, calling sleepily out from behind their open windows.
``I stayed late to practice, he tried to stop me. I knocked him down
and he chased me. I knocked him down and then I taught him why he
shouldn't have chased me.''

``Mala,'' he said, and she thought he was trying to sound fatherly
now, stern and old and masculine. ``You should have known better
than to put yourself in that position. A general knows that you win
some fights by not getting into them at all. Now, I'm not an
unreasonable man. Of course, you and your mother and your army all
need my money if you're going to keep fighting. You can borrow a
wage-packet from me during this month, something to pay everyone
with, and then you can pay it back, little by little, over the next
year or so. I'll take five in twenty rupees for 12 months, and
we'll call it even.''

It was hope, terrible, awful hope. A chance to keep her army, her
flat, her respect. All it would cost her was one quarter of her
earnings. She'd have three quarters left. Three quarters was better
than nothing. It was better than telling Mamaji that it was all
over.

``Yes,'' she said. ``All right, fine. But we don't play at Mrs
Dibyendu's cafe anymore.''

``Oh, no,'' he said. ``I won't hear of it. Mrs Dibyendu will be glad
to have you back. You'll have to apologize to her, of course. You
can bring her the money for her nephew. That will make her feel
better, I'm sure, and heal any wounds in your friendship.''

``Why?'' There were tears on her cheeks now. ``Why not let us go
somewhere else? Why does it matter?''

``Because, Mala, I am the boss and you are the worker and that is
the factory you work in. That's why.'' His voice was hard now, all
the lilt of false concern gone away, leaving behind a grinding like
rock on rock.

She wanted to put the phone down on him, the way they did in the
movies when they had their giant screaming rows, and threw their
phones into the well or smashed them on the wall. But she couldn't
afford to destroy her phone and she couldn't afford to make Mr
Banerjee angry.

So she said, ``All right,'' in a quiet little voice that sounded like
a mouse trying not to be noticed.

``Good girl, Mala. Smart girl. Now, I've got your next mission for
you. Are you ready?''

Numbly, she memorized the details of the mission, who she was going
to kill and where. She thought that if she did this job quickly,
she could ask him for another one, and then another -- work longer
hours, pay off the debt more quickly.

``Smart girl, good girl,'' he said again, once she'd repeated the
details back to him, and then he put the phone down.

She pocketed her phone. Around her, Dharavi had woken, passing by
her like she was a rock in a river, pressing past her on either
side. Men with shovels and wheelbarrows, boys with enormous
rice-sacks on each shoulder, filled with grimy plastic bottles on
their way to some sorting house, a man with a long beard and kufi
skullcap and kurta shirt hanging down to his knees leading a goat
with a piece of rope. A trio of women in saris, their midriffs
stretched and striated with the marks of the babies they'd borne,
carrying heavy buckets of water from the communal tap. There were
cooking smells in the air, a sizzle of dhal on the grill and the
fragrant smell of chai. A boy passed by her, younger than Gopal,
wearing flapping sandals and short pants, and he spat a stream of
sickly sweet betel at her feet.

The smell made her remember where she was and what had happened and
what she had to do now.

She went past the Das family on the ground floor and trudged up the
stairs to their flat. Mamaji and Gopal were awake and bustling.
Mamaji had fetched the water and was making the breakfast over the
propane burner, and Gopal had his school uniform shirt and
knee-trousers on. The Dharavi school he attended lasted for half
the day, which gave him a little time to play and do homework and
then a few more hours to work alongside of Mamaji in the factory.

``Where have you been?'' Mamaji said.

``On the phone,'' she said, patting the little pocket sewn of her
tunic. ``With Mr Banerjee.'' She waggled her chin from side to side,
saying \emph{I've had business}.

``What did he say?'' Mamaji's voice was quiet and full of false
nonchalance.

Mamaji didn't need to know what transpired between Mr Banerjee and
her. Mala was the general and she could manage her own affairs.

``He said that all was forgiven. The boy deserved it. He'll make it
fine with Mrs Dibyendu, and it will be fine.'' She waggled her chin
from side to side again --
\emph{It's all fine. I've taken care of it}.

Mamaji stared into the pan and the food sizzling in it and nodded
to herself. Though she couldn't see, Mala nodded back. She was
General Robotwallah and she could make it all good.

\tb

\shopad{This scene is dedicated to Forbidden Planet, the British chain of science fiction and fantasy book, comic, toy and video stores. Forbidden Planet has stores up and down the UK, and also sports outposts in Manhattan and Dublin, Ireland. It's dangerous to set foot in a Forbidden Planet -- rarely do I escape with my wallet intact. Forbidden Planet really leads the pack in bringing the gigantic audience for TV and movie science fiction into contact with science fiction books -- something that's absolutely critical to the future of the field.}
{\href{http://www.forbiddenplanet.co.uk}{Forbidden Planet, UK, Dublin and New York City}}

Wei-Dong had been to downtown LA once, on a class trip to the
Disney Concert Hall, but then they'd driven in, parked, and marched
like ducklings into the hall and then out again, without spending
any time actually wandering around. He remembered watching the
streets go by from the bus window, faded store windows and
slow-moving people, check-cashing places and liquor stores. And
Internet cafes. Lots and lots of Internet cafes, especially in
Koreatown, where every strip mall had a garish sign advertising ``PC
Baang'' -- Korean for net-cafe.

But he didn't know exactly where Koreatown was, and he needed an
Internet cafe to google it, and so he caught the LAX bus to the
Disney Concert Hall, thinking he could retrace the bus-route and
find his way to those shops, get online, talk to his homies in
Guangzhou, figure out the next thing.

But Koreatown turned out to be harder to find and farther than he'd
thought. He asked the bus-driver for directions, who looked at him
like he was crazy and pointed downhill. And so he started walking,
and walking, and walking for block after dusty block. From the
window of the school-bus, downtown LA had looked slow-moving and
faded, like a photo left too long in a window.

On foot, it was frenetic, the movement of the buses, the homeless
people walking or wheeling or hobbling past him, asking him for
money. He had \$1000 in his front jeans pocket, and it seemed to
him that the bulge must be as obvious as a boner at the blackboard
in class. He was sweating, and not just from the heat, which seemed
ten degrees hotter than it had been in Disneyland.

And now he wasn't anywhere near Koreatown, but had rather found his
way to Santee Alley, the huge, open-air pirate market in the middle
of LA. He'd heard about the place before, you saw it all the time
in news-specials about counterfeit goods busts, pictures of Mexican
guys being led away while grimly satisfied cops in suits or uniform
baled up mountains of fake shirts, fake DVDs, fake jeans, fake
games.

Santee Alley was a welcome relief from the streets around it. He
wandered deep into the market, the storefronts all blaring their
technobrega and reggaton at him, the hawkers calling out their
wares. It was like the real market on which all the hundreds of
in-game markets he'd visited had been based upon and he found
himself slowing down and looking in at the gangster clothes and the
bad souvenir junk and the fake electronics. He bought a big cup of
watermelon drink and a couple of empanadas from a stall, carefully
drawing a single twenty from his pocket without bringing out the
whole thing.

Then he'd found an Internet cafe, filled with Guatemalans chatting
with their families back home, wearing slick and tiny earwigs. The
girl behind the counter -- barely older than him -- sold him one
that claimed to be a Samsung for \$18, and then rented him a PC to
use it with. The fake earwig fit as well as his real one had,
though it had a rough seam of plastic running around its length
while his had been as smooth as beach-glass.

But it didn't matter. He had his network connection, he had his
earwig, and he had his game. What more could he need?

Well, his posse, for starters. They were nowhere to be found. He
checked his new watch and pressed the button that flipped it to the
Chinese timezone. 5AM. Well, that explained it.

He checked his inventory, checked the guild-bank. He hadn't been
able to do the corpse run after he'd been snatched out of the game
by his father and the Ronald Reagan High Thought Police, so he
didn't expect to have his vorpal blade still, but he did, which
meant that one of the gang had rescued it for him, which was
awfully thoughtful. But that was just what guildies did for each
other, after all.

It was coming up to dinner-time on the east coast, which meant that
Savage Wonderland was starting to fill up with people getting home
from work. He thought about the black riders who slaughtered them
that morning and wondered who they'd been. There were plenty of
people who hunted gold farmers, either because they worked for the
game or for a rival gold-farm clan, or because they were bored rich
players who hated the idea of poor people invading ``their'' space
and working where they played.

He knew he should flip to his email and check for messages from his
parents. He didn't like using email, but his parents were addicted
to it. No doubt they were freaking out by now, calling out the army
and navy and the national guard to find their wayward son. Well,
they could freak out all they wanted. He wasn't going to go back
and he didn't need to go back.

He had \$1000 in his pocket, he was nearly 18 years old, and there
were lots of ways to get by in the big city that didn't involve
selling drugs or your body. His guildies had shown him that. All
you needed to earn a living was a connection to the net and a brain
in your head. He looked around the cafe at the dozens of
Guatemalans talking to home on their earwigs, many not much older
than him. If they could earn a living -- not speaking the language,
not legal to work, no formal education, hardly any idea of how to
use technology beyond the little bit of knowledge necessary to call
home on the cheap -- then surely he could. His grandfather had come
to America and found a job when he was Wei-Dong's age. It was a
family tradition, practically.

It wasn't that he didn't love his parents. He did. They were good
people. They loved him in their way. But they lived in a bubble of
unreality, a bubble called Orange County, where they still had rows
of neat identical houses and neat identical lives, while around
them, everything was collapsing. His father couldn't see it, even
though hardly a day went by that he didn't come home and complain
bitterly about the containers that had fallen off his ship in yet
another monster storm, about the price of diesel sailing through
the stratosphere, about the plummeting dollar and the skyrocketing
Renminbi and the ever-tightening belts of Americans whose orders
for goods from South China were clobbering his business.

Wei-Dong had figured all this out because he paid attention and he
saw things as they were. Because he talked to China, and China
talked back to him. The fat and comfortable world he'd grown up in
was not permanent; scratched in the sand, not carved in stone. His
friends in China could see it better than anyone else could. Lu had
worked as a security guard in a factory in Shilong New Town, a city
that made appliances for sale in Britain. It had taken Wei-Dong
some time to understand this: the entire city, four million people,
did nothing but make appliances for sale in Britain, a country with
eighty million people.

Then, one day, the factories on either side of Lu's had closed.
They had all made goods for a few different companies, employing
armies of young women to run the machines and assemble the pieces
that came out of them. Young women always got the best jobs. Bosses
liked them because they worked hard and didn't argue so much -- at
least, that's what everyone said. When Lu left his village in
Sichuan province to come to south China, he'd talked to one of the
girls who had come home from the factories for the Mid-Autumn
Festival, a girl who'd left a few years before and found wealth in
Dongguan, who'd bought her parents a fine new two-storey house with
her money, who came home every year for the Festival in fine
clothes with a new mobile phone in a designer bag, looking like an
alien or a model stepped fresh out of a magazine ad.

``If you go to a factory and it's not full of young girls, don't
take a job there,'' was her advice. ``Any place that can't attract a
lot of young girls, there's something wrong with it.'' But the
factory that Lu worked at -- all the factories in Shilong New Town
-- were filled with young girls. The only jobs for men were as
drivers, security guards, cleaners and cooks. The factories boomed,
each one a small city itself, with its own kitchens, its own
dormitories, its own infirmary and its own customs checkpoint where
every vehicle and visitor going in or out of the wall got checked
and inspected.

And these indomitable cities had crumbled. The Highest Quality
Dishwasher Company factory closed on Monday. The Boundless Energy
Enterprises hot-water heater plant went on Wednesday. Every day, Lu
saw the bosses come in and out in their cars, waving them through
after they'd flicked their IDs at him. One day, he steeled his
nerve and leaned in the window, his face only inches from that of
the man who paid his wages every month.

``We're doing better than the neighbors, eh, Boss?'' He tried for a
jovial smile, the best he could muster, but he knew it wasn't very
good.

``We do fine,'' the boss had barked. He had very smooth skin and a
smart sport-coat, but his shoulders were dusted with dandruff. ``And
no one says otherwise!''

``Just as you say, boss,'' Lu said, and leaned out of the window,
trying to keep his smile in place. But he'd seen it in the boss's
face -- the factory would close.

The next day, no bus came to the bus-stop. Normally, there would
have been fifty or sixty people waiting for the bus, mostly young
men, the women mostly lived in the dorms. Security guards and
janitors didn't rate dorm rooms. That morning, there were eight
people waiting when he arrived at the bus-stop. Ten minutes went by
and a few more trickled to the stop, and still no bus came. Thirty
minutes passed -- Lu was now officially late for work -- and still
no bus came. He canvassed his fellow waiters to see if anyone was
going near his factory and might want to share a taxi -- an
otherwise unthinkable luxury, but losing his job even was more
unthinkable.

One other guy, with a Shaanxi accent, was willing, and that's when
they noticed that there didn't seem to be any taxis cruising on the
road either. So Lu, being Lu, walked to work, fifteen kilometers in
the scorching, melting, dripping heat, his security guard's shirt
and coat over his arm, his undershirt rolled up to bare his belly,
the dust caking up on his shoes. And when he arrived at the Miracle
Spirit condenser dryer factory and found himself in a mob of
thousands of screeching young women in factory-issue smocks,
crowded around the fence and the double-padlocked rattling it and
shouting at the factory's darkened doors. Many of the girls had
small backpacks or duffel-bags, overstuffed and leaking underwear
and makeup on the ground.

``What's going on?'' he shouted at one, pulling her out of the mob.

``The bastards shut the factory and put us out. They did it at
shift-change. Pulled the fire-alarm and screamed 'Fire' and 'Smoke'
and when we were all out here, they ran out and padlocked the
gate!''

``Who?'' He'd always thought that if the factory were going to shut
down, they'd use the security guards to do it. He'd always thought
that he, at least, would get one last paycheck out of the company.

``The bosses, six of them. Mr Dai and five of his supervisors. They
locked the front gate and then they drove off through the back
gate, locking it behind them. We're all locked out. All my things
are in there! My phone, my money, my clothes --''

Her last paycheck. It was only three days to payday, and, of
course, the company had kept their first eight weeks' wages when
they all started working. You had to ask your boss's permission if
you wanted to change jobs and keep the money -- otherwise you'd
have to abandon two months' pay.

Around Lu, the screams rose in pitch and small, feminine fists
flailed at the air. Who were they shouting at? The factory was
empty. The factory was empty. If they climbed the fence, cutting
the barbed wire at the top, and then broke the locks on the factory
doors, they'd have the run of the place. They couldn't carry out a
condenser dryer -- not easily, anyway -- but there were plenty of
small things: tools, chairs, things from the kitchen, the personal
belongings of the girls who hadn't thought to bring them with when
the fire alarm sounded. Lu knew about all the things that could be
smuggled out of the factory. He was a security guard. Or had been.
Part of his job had been to search the other employees when they
left to make sure they weren't stealing. His supervisor, Mr Chu,
had searched \emph{him} at the end of each shift, in turn. He
wasn't sure who, if anyone, searched Mr Chu.

He had a small multitool that he clipped to his belt every morning.
Having a set of pliers, a knife, and a screwdriver on you all the
time changed the way you saw the world -- it became a place to be
cut, sliced, pried and unscrewed.

``Is that your only jacket?'' he shouted into the ear of the girl
he'd been talking to. She was a little shorter than him, with a
large mole on her cheek that he rather liked.

``Of course not!'' she said. ``I have three others inside.''

``If I get you those three, can I use this one?'' He unfolded the
pliers on his multitool. They were joined by a set of cogs that
compounded the leverage of a squeezing palm, and the jaws of the
plier were inset with a pair of wicked-sharp wire-cutters. The girl
in his village had worked for a time in the SOG factory in Dongguan
and she'd given him a pair and wished him good luck in South
China.

The girl with three more jackets looked up at the barbed wire.
``You'll be cut to ribbons,'' she said.

He grinned. ``Maybe,'' he said. ``I think I can do it, though.''

``Boys,'' she hollered in his ear. He could smell her breakfast
congee on her breath, mixed with toothpaste. It made him homesick.
``All right. But be careful!'' She shrugged out of the jacket,
revealing a set of densely muscled arms, worked to lean strength on
the line. He wrapped it around his left hand, then wrapped his own
coat around that, so that his hand looked like a cartoon
boxing-glove, trailing sleeves flapping down beneath it.

It wasn't easy to climb the fence with one hand wrapped in a dozen
thicknesses of fabric, but he'd always been a great climber, even
in the village, a daring boy who'd gotten a reputation for climbing
anything that stood still: trees, houses, even factories. He had
one good hand, two feet, and one bandaged hand, and that was enough
to get up the fifteen feet to the top. Once there, he gingerly
wrapped his left hand around the razorwire, careful to pull
straight down on it and not to saw from side to side. He had a
vision of himself slipping and falling, the razorwire slicing his
fingers from his hand so that they fell to the other side of the
fence, wriggling like worms in the dust as he clutched his mangled
hand and screamed, geysering blood over the girls around him.

\emph{Well, you'd better not slip, then}, he thought grimly,
carefully unfolding the multitool with his other hand, flipping it
around like a butterfly knife (a move he'd often practiced, playing
gunfighter in his room or when no one else was around at the gate).
He gingerly slid it around the first coil of wire and squeezed
down, watching the teeth on the gears mesh and strain at one
another, turning the leverage of his right hand into hundreds of
pounds of pressure bearing down right at the cutting edge of the
pliers. They bit into the wire, caught, and then parted it.

The coil of wire sprang free with a \emph{twoingggg} sound, and he
ducked away just in time to avoid having his nose -- and maybe his
ear and eye -- sliced off by the wire.

But now he could transfer his left hand to the top of the fence,
and put more weight on it, and reach for the second coil of wire
with the cutters, hanging way out from the fence, as far as he
could, to avoid the coil when it sprang free. Which it did, parting
just as easily as the other coil had, and flying directly at him,
and it was only by releasing his feet and dangling one-handed from
the fence, slamming his body into it, that he avoided having his
throat cut. As it was, the wire made a long scratch in the back of
his scalp, which began to bleed freely down his back. He ignored
it. Either it was shallow and would stop on its own, or it was deep
and he'd need medical attention, but either way, he was going to
clear the fencetop.

All that remained now were three strands of barbed wire, and they
were tougher to cut than the razorwire had been, but the barbs were
widely spaced and the wire itself was less prone to crazy twanging
whipsaws than the coiled razorwire. As each one parted, there was a
roar of approval from the girls below him, and even though his
scalp was stinging fiercely, he thought this might just be his
finest hour, the first time in his life that he'd been something
more than a security guard who'd left his backwards town to find
insignificance in Guandong province.

And now he was able to unwind the jackets from around his hand and
simply hop over the fence and clamber down the other side like a
monkey, grinning all the way at the horde of young girls who were
coming up the other side in a great wave. It wasn't long before the
girl with three more jackets caught him up. He shook out her jacket
-- sliced through in four or five places -- like a waiter offering
a lady her coat, and she delicately slid those muscular arms into
it and then she turned him around and poked at his scalp.

``Shallow,'' she said. ``It'll bleed a lot, but you'll be OK.'' She
planted a sisterly kiss on his cheek. ``You're a good boy,'' she
said, and then ran off to join the stream of girls who were
entering the factory through a smashed door.

Shortly, he found himself alone in the factory yard, amid the neat
gravel pathways and the trimmed lawns. He let himself into the
factory but he couldn't actually bring himself to take anything,
though they owed him nearly three months' wages. Somehow, it seemed
to him that the girls who'd used the tools should have their pick
of the tools, that the men who'd cooked the meals should have their
pick of the things from the kitchens.

Finally, he settled on one of the communal bicycles that were
neatly parked near the factory gates. These were used by all the
employees equally, and besides, he needed to get home and walking
back with a scalp wound in the mid-day heat didn't sound like much
of a plan.

On the way home, the world seemed much changed. He'd become a
criminal, for one thing, which seemed to him to be quite a distance
from a security guard. But it was more than that: the air seemed
clearer (later, he read that the air \emph{was} clearer, thanks to
all the factories that had shut down and the buses that had stayed
parked). Most of the shops seemed closed and the remainder were
tended by listless storekeepers who sat on their stoops or played
Mah-Jongg on them, though it was the middle of the day. All the
restaurants and cafes were shut. At a train-crossing, he watched an
intercity train shoot past, every car jammed with young women and
their bags, leaving Shilong New Town to find their way somewhere
else where there was still work.

Just like that, in the space of just a week or two, this giant city
had died. It had all seemed so incredibly powerful when he'd
arrived, new paved roads and new stores and new buildings, and the
factories soaring against the sky wherever you looked.

By the time he reached home -- dizzy from the aching cut on his
scalp, sweaty, hungry -- he knew that the magical city was just a
pile of concrete and a mountain of workers' sweat, and that it had
all the permanence of a dream. Somewhere, in a distant land he
barely knew the name of, people had stopped buying washing
machines, and so his city had died.

He thought he'd lie down for just the briefest of naps, but by the
time he got up and gathered a few things into a duffel-bag and got
back on his bike, not bothering to lock the door of his apartment
behind him, the train station was barricaded, and there was a long
line of refugees slogging down the road to Shenzhen, two days' walk
away at least. He was glad he'd taken the bicycle then. Later, he
found a working ATM and drew out some cash, which was more
reassuring than he'd anticipated. For a while there, it had seemed
like the world had come to an end. It was a relief to find out that
it was just his little corner.

In Shenzhen, he'd started hanging out in Internet cafes, because
they were the cheapest places to sit indoors, out of the heat, and
because they were filled with young men like him, scraping by. And
because he could talk to his parents from there, telling them
made-up stories about his non-existent job-search, promising that
he'd start sending money home soon.

And that was where the guild found him, Ping and his friends, and
they had this buddy on the other side of the planet, this Wei-Dong
character who'd hung rapt on every turn of his tale, who'd told him
that he'd written it up for a social studies report at school,
which made them all laugh. And he'd found happiness and work, and
he'd found a truth, too: the world wasn't built on rock, but rather
on sand, and it would shift forever.

Wei-Dong didn't know how much longer his father's business would
last. Maybe thirty years -- but he thought it would be a lot less
than that. Every day, he woke in his bedroom under his Spongebob
sheets and thought about which of these things he could live
without, just how \emph{basic} his life could get.

And here it was, the chance to find out. When his
great-grandparents had been his age, they'd been war-refugees,
crossing the ocean on a crowded boat, travelling on stolen papers,
an infant in his great-grandmother's arms and another in her belly.
If they could do it, Wei-Dong could do it.

He'd need a place to stay, which meant money, which meant a job.
The guild would cut him in for his share of the money from the
raids, but that wasn't enough to survive in America. Or was it? He
wondered how much the Guatemalans around him earned at their
illegal dishwashing and cleaning and gardening jobs.

In any event, he wouldn't have to find out, because he had
something they didn't have: a Social Security Number. And yes, that
meant that eventually his parents would be able to find him, but in
another month, he'd be 18 and it'd be too late for them to do
anything about it if he didn't want to cooperate.

In those hours where he'd planned for the demise of his family's
fortune, he'd settled quickly on the easiest job he could step
into: Mechanical Turk.

The Turks were an army of workers in gamespace. All you had to do
was prove that you were a decent player -- the game had the stats
to know it -- and sign up, and then log in whenever you wanted a
shift. The game would ping you any time a player did something the
game didn't know how to interpret -- talked too intensely to a
non-player character, stuck a sword where it didn't belong, climbed
a tree that no one had bothered to add any details too -- and you'd
have to play spot-referee. You'd play the non-player character,
choose a behavior for the stabbed object, or make a decision from a
menu of possible things you might find in a tree.

It didn't pay much, but it didn't take much time, either. Wei-Dong
had calculated that if he played two computers -- something he was
sure he could keep up -- and did a new job every twenty seconds on
each, he could make as much as the senior managers at his father's
company. He'd have to do it for ten hours a day, but he'd spent
plenty of weekends playing for 12 or even 14 hours a day, so hell,
it was practically money in the bank.

So he used the rented PC to sign onto his account and started
filling in the paperwork to apply for the job. All the while, he
was conscious of his rarely-used email account and of the messages
from his parents that surely awaited him. The forms were long and
boring, but easy enough, even the little essay questions where you
had to answer a bunch of hypothetical questions about what you'd do
if a player did this or said that. And that email from his parents
was lurking, demanding that he download it and read it --

He flipped to a browser and brought up his email. It had been weeks
since he'd last checked it and it was choked with hundreds of
spams, but there, at the top:

RACHEL ROSENBAUM -- WHERE ARE YOU???

Of course his mother was the one to send the email. It was always
her on email, sending him little encouraging notes through the
school day, reminding him of his grandparents' and cousins' and
father's birthdays. His father used email when he had to, usually
at two in the morning when he couldn't sleep for worry about work
and he needed to bawl out his managers without waking them up on
the phone. But if the phone was an option, Dad would take it.

WHERE ARE YOU???

The subject-line said it all, didn't it?

\emph{Leonard, this is crazy. If you want to be treated like an adult, start acting like one. Don't sneak around behind our backs, playing games in the middle of the night. Don't run off to God-knows-where to sulk.}

\emph{We can negotiate this like family, like grownups, but first you'll have to COME HOME and stop behaving like a SPOILED BRAT. We love you, Leonard, and we're worried about you, and we want to help you. I know when you're 17 it's easy to feel like you have all the answers --}

He stopped reading and blew hot air out his nostrils. He hated it
when adults told him he only felt the way he did because he was
\emph{young}. As if being young was like being insane or drunk,
like the convictions he held were hallucinations caused by a mental
illness that could only be cured by waiting five years. Why not
just stick him in a box and lock it until he turned 22?

He began to hit reply, then realized that he was logged in without
going through an anonymizer. His guildies were big into these --
they were servers that relayed your traffic, obscuring your
identity and the addresses you were trying to avoid. The best ones
came from Falun Gong, the weird religious cult that the Chinese
government was bent on stamping out. Falun Gong put new relays
online every hour or so, staying a hop ahead of the Great Firewall
of China, the all-seeing, all-knowing, all-controlling server-farm
that was supposed to keep 1.6 billion Chinese people from looking
at the wrong kind of information.

No one in the guild had much time for Falun Gong or its quirky
beliefs, but everyone agreed that they ran a tight ship when it
came to punching holes in the Great Firewall. A quick troll through
the ever-rotating index-pages for Falun Gong relays found Wei-Dong
a machine that would take his traffic. \emph{Then} he replied to
his Mom. Let her try to run his backtrail -- it would dead-end with
a notorious Chinese religious cult. That'd give her something to
worry about all right!

\emph{Mom, I'm fine. I'm acting like an adult (taking care of myself, making my own decisions). It might have been wrong to lie to you guys about what I was doing with my time, but kidnapping your son to military school is about as non-adult as you can get. I'll be in touch when I get a chance. I love you two. Don't worry, I'm safe.}

Was he, really? As safe as his great-grandparents had been,
stepping off the ship in New York. As safe as Lu had been,
bicycling the cracked road to Shenzhen.

He'd find a place to stay -- he could google ``cheap hotel downtown
los angeles'' as well as the next kid. He had money. He had a SSN.
He had a job -- two jobs, counting the guild work -- and he had
plenty of practice missions he'd have to run before he'd start
earning. And it was time to get down to it.

\chapter*{Part II: Hard work at play}

\shopad{This scene is dedicated to the incomparable Mysterious Galaxy in San Diego, California. The Mysterious Galaxy folks have had me in to sign books every time I've been in San Diego for a conference or to teach (the Clarion Writers' Workshop is based at UC San Diego in nearby La Jolla, CA), and every time I show up, they pack the house. This is a store with a loyal following of die-hard fans who know that they'll always be able to get great recommendations and great ideas at the store. In summer 2007, I took my writing class from Clarion down to the store for the midnight launch of the final Harry Potter book and I've never seen such a rollicking, awesomely fun party at a store.}
{\href{http://www.mystgalaxy.com/book/9780765322166}{Mysterious Galaxy}: 7051 Clairemont Mesa Blvd., Suite \#302 San Diego, CA USA 92111 +1 858 268 4747}

They came for the workers in the game and in the real world, a
coordinated assault that left Big Sister Nor's organization in
tatters.
On that fateful night, she'd taken up the back room of Headshot, a
PC Baang in the Geylang district in Singapore, a neighborhood that
throbbed all night long from the roaring sex-trade from the legal
brothels and the illegal street-hookers. Any time after dark, the
Geylang's streets were choked with people, from adventurous diners
eating in the excellent all-night restaurants (almost all of them
halal, which always made her smile) to guest workers and
Singaporeans on the prowl for illicit thrills to the girls dashing
out on their breaks to the all-night supermarkets to do their
shopping.

The Geylang was as unbuttoned as Singapore got, one of the few
places where you could be ``out of bounds'' -- doing something that
was illegal, immoral, unmentionable, or bad for social harmony --
without attracting too much attention. Headshot strobed all night
long with networked poker games, big shoot-em-up tournaments,
guestworkers phoning home on the cheap, shouting over the
noise-salad of all those games, and, on that night, Big Sister Nor
and her clan.

They called themselves the Webblies, which was an obscure little
joke that pleased Big Sister Nor an awful lot. Nearly a century
ago, a group of workers had formed a union called the Industrial
Workers of the World, the first union that said that all workers
needed to stick up for each other, that every worker was welcome no
matter the color of his skin, no matter if the worker was a woman,
no matter if the worker did ``skilled'' or ``unskilled'' work. They
called themselves the Wobblies.

Information about the Wobblies was just one of the many ``out of
bounds'' subjects that were blocked on the Singaporean Internet, and
so of course Big Sister Nor had made it her business to find out
more about them. The more she read, the more sense this group from
out of history made for the world of \emph{right now} -- everything
that the IWW had done needed doing \emph{today}, and what's more,
it would be easier today than it had been.

Take organizing workers. Back then, you'd have to actually get into
the factory or at least stand at its gates to talk to workers about
signing a union card and demanding better conditions, higher wages
and shorter hours. Now you could reach those same people online,
from anywhere in the world. Once they were members, they could talk
to all the other members, using the same tools.

She'd decided to call her little group the Industrial Workers of
the World Wide Web, the IWWWW, and that was another of those jokes
that pleased her an awful lot. And the IWWWW had grown and grown
and grown. Gold farmers were easy pickings: working in terrible
conditions all over the world, for terrible wages, hated by the
game-runners and the rich players alike. They already understood
about working in teams, they'd already formed their own little
guilds -- and they were better at using the Internet than their
bosses would ever be.

Now, a year later, the IWWWW had over 20,000 members signed up in
six countries, paying dues and filling up a fat strike fund that
had finally been called into use, in Shenzhen, the last place Big
Sister Nor had ever expected to see a walkout.

But they had, they had! The boss, some character named Wing, had
declared a lock-in at three of his ``factories'' -- Internet cafes
that he'd taken over to support his burgeoning army of workers --
in order to take advantage of a sploit in Mushroom Kingdom, a
Mario-based MMO that had a huge following in Brazil. One of his
workers had found a way to triple the gold they took out of one of
the dungeons, and he wanted to extract every penny he could before
Nintendo-Sun caught on to it.

The next thing she knew, her phone was rattling with urgent
messages relayed from her various in-game identities to tell her
that the workers had knocked aside the factory management and
guards and stormed out, climbing the sides of the buildings or the
utility poles and cutting the cafes' network links. They'd formed
up out front and begun to chant impromptu slogans -- mostly adapted
from their in-game battle-cries. And now they wanted to know what
to do.

``It's a wildcat strike,'' Big Sister Nor said to her lieutenants,
The Mighty Krang and Justbob, the former a small Chinese guy with
frosted purple tips in his hair, the latter a Tamil girl in a
beautiful, immaculate sari and silk slippers -- a girl who had
previously run with one of the most notorious girl-gangs in Asia
and spent three years in prison for her trouble. ``They've walked
out in Shenzhen.'' She forwarded the tweets and blips and alerts off
her phone, then showed them her screen while they waited for the
forwards to land on their devices.

``It's crazy,'' The Mighty Krang said, dancing from foot to foot,
excitedly. ``It's crazy, it's crazy, it's --''

``Wonderful,'' Justbob said, planting her palms on his shoulders and
bringing him back to the earth. ``And overdue. I predicted this. I
predicted it from the start. As soon as you start collecting dues
for a 'strike fund,' someone's going to go on strike. And la-la,
here we are, wildcatting the night away.''

The next step was to head for headquarters, the back room at
Headshot, to slam themselves into their chairs and to hit the
worlds, spreading the word to all 20,000 members about the
first-ever strike. Big Sister Nor went to work on a plan:

1. Spread the word to the rank-and-file

2. Recruit in-world pickets to block the work-site so that Boss
Wing couldn't bring in scabs -- replacement workers -- to get the
job done

3. Get the strike-leaders on the phone and talk about human-rights
lawyers, strike-pay, sleeping quarters for any workers who relied
on the factory for dorm-beds

4. Get footage and real-time reports from the strikers out to the
human rights wires, get the strike-leaders on interviews with the
press

She'd done this before, in real life, on the other side of things,
as a wildcat strike leader walking off the line when the bosses at
her weaving factory in Taman Makmur announced pay cuts because
their big European distributor had cut its orders. It happened
every year, but it made her so angry -- the workers didn't get
bonuses, sharing in the good fortune when distributors increased
their orders, but they were made to share the burden when orders
went down. Well, forget it, enough was enough. She'd stood up in
the middle of the factory floor and denounced the bosses for the
greedy, immoral bastards they were, and when the security moved in
to take her, she'd stood proud and strong, ready to be beaten for
her insolence.

Instead, her fellow workers had risen to her defense, the young
women around her getting to their feet and surrounding her,
cheering her, ululating cries shouting around waggling tongues that
bounced off the ceiling and filled the room and her heart, making
them all brave, so that the security men moved back, and they'd
taken over the factory, blocking the gates, shutting it down, and
then someone from the Malaysian Union of Textile Employees had been
there to get them to sign cards, and someone had made her picket
captain and then --

And then it had all come crashing down around them, police vans
moving in, the police forming a line and ordering them to disperse,
to get back to work, to stop this foolishness before someone got
hurt, barking the orders through a bullhorn, glaring at them from
beneath their riot helmets, banging their truncheons on their
shields, spraying them with teargas.

Their line wavered, disintegrated, retreated. But they reformed in
an alley near the factory, amid a gang of staring children, and the
women from the MUTE collared the children and sent them running to
get milk -- cow's milk, goat's milk, anything they could find, and
the MUTE organizers had rinsed their eyes with the milk, holding
their faces still while they coughed and gagged. The fat-soluble CS
gas rinsed away, leaving them teary but able to see, and the coughs
dispersed, and someone produced a bag of charcoal-filter cycling
masks, and someone else had a bag of swimming goggles, and the
women put them on and pulled their hijabs over their noses, over
the masks, so that they looked like some species of snouted animal,
and they reformed their line and marched back, chanting their
slogans.

The police gassed them again, but this time, the picket captains
were able to hold the line, to send brave women forward to grab the
smoking cannisters and throw them back over police lines. For a
moment, it looked like the police would charge, but the strikers
and the organizers had been feeding a photostream to the Internet
using mobile phones that tunneled through the national firewall,
getting them up on the human rights wires, and so the Ministry of
Labour was getting phone calls from the foreign press, and they
were on the phone to the Ministry of Justice, and the police
withdrew.

The first skirmish was over, and the strikers settled in for a long
siege. No one got in or out of the factory without being harangued
by hundreds of young women, shoving literature detailing their
working conditions and grievances and demands through the windows
of their cars and buses. Some replacement workers got in, some
picked fights, some turned around and left. A unionized trucker
refused to cross their line, and wouldn't take away the load he'd
been charged with picking up, so it just sat there on the docks.

The days turned into weeks, and they fed their families as best as
they could with the strike pay, which came to a third of what
they'd earned in the plant, but the factory owners -- a subsidiary
of a Dutch company -- were hurting too. The MUTE organizers
explained that the parent company had to release its quarterly
statement to its shareholders, who would demand to know why this
major factory was sitting idle instead of making money. The
organizers offered confident reassurances that when this happened,
the workers' demands would be met, the strike settled, and they
could get back to work.

So they hung in there, keeping their spirits up on the line, and
then --

The factory closed.

Big Sister Nor found out about it one night as she was playing
Theater of War VII, a game she'd played since she was a little
girl. One of her guildies was a girl whose brother had passed by
the factory on his way home from school, and he'd seen them moving
the machines out of the plant, driving away in huge lorries.

She'd texted everyone she knew, \emph{Get to the factory now}, but
by the time they got there, the factory was dead, empty, the gates
chained shut. No one from the union met them. None of them answered
her calls.

And the women she'd called sister, the women who'd saved her when
she'd said \emph{enough}, they all looked to her and said,
\emph{What do we do now?}

And she hadn't known. She'd managed to hold the tears in until she
got home, but then they'd flowed, and her parents -- who'd doubted
her and harangued her every step of the way -- scolded her for her
foolishness, told her it was her fault that all her friends were
jobless.

She'd lain in bed that night, miserable, and had been woken by the
soft chirp of her phone.

\emph{I'm outside.} It was Affendi, the MUTE organizer she'd been
closest to. \emph{Come to the door}.

She'd crept outside on cat's feet and barely had time to make out
Affendi's outline before she collapsed into Nor's arms. She had
been beaten bloody, her eyes blacked, two of her fingers broken,
her lips mashed and one of her teeth missing. She managed a mangled
smile and whispered, ``It's all part of the job.''

The cheap hotel where the four organizers had shared a room was
raided just after dinner, the police taking them away. They'd been
prepared for this, had lawyers standing by to help them when it
happened, but they didn't get to call lawyers. They didn't go to
the jailhouse. Instead, they'd been taken to a shantytown behind
the main train-station and three policemen had stood guard while a
group of private security forces from the plant had taken turns
beating them with truncheons and fists and boots, screaming insults
at them, calling them whores, tearing at their clothes, beating
their breasts and thighs.

It only stopped when one of the women fell unconscious, bleeding
from a head-wound, eyelids fluttering. The men had fled then, after
taking their money and identity papers, leaving them weeping and
hurt. Affendi had managed to hide her spare mobile phone -- a tiny
thing the size of a matchbook -- in the elastic of her underpants,
and that had enabled her to call the MUTE headquarters for help.
Once the ambulance was on its way, she'd come to get Nor.

``They'll probably come for you, too,'' she said. ``They usually try
to make an example of the workers who start trouble.''

``But you told me that they were going to have to give in because of
their shareholders --''

Affendi held up a broken hand. ``I thought they would. But they
decided to leave. We think they're probably going to Indonesia. The
new laws there make it much harder to organize the workers. That's
how it goes, sometimes.'' She shrugged, then winced and sucked air
over her teeth. ``We thought they'd want to stay put here. The
provincial government gave them too much to come here -- tax
breaks, new roads, free utilities for five years. But there are new
Special Economic Zones in Indonesia that have even better deals.''
She shrugged again, winced again. ``You may be all right here, of
course. Maybe they'll just move on. But I thought you should be
given the chance to get somewhere safe with us, if you wanted to.''

Nor shook her head. ``I don't understand. Somewhere safe?''

``The union has a safe-house across the provincial line. We can take
you there tonight. We can help you find work, get set up. You can
help us unionize another factory.''

A light rain fell, pattering off the palms that lined her street
and splashing down in wet, fat drops, bringing an earthy smell up
from the soil. A fat drop slid off an unseen leaf overhead and
spattered on Nor's neck, reminding her that she'd gone out of the
house without her hijab, something she almost never did. It seemed
to her an omen, like her life was changing in every single way.

``Where are we going?''

``You find out when we get there. I don't know either. That's why
it's a safe house -- no one knows where it is unless they have to.
MUTE organizers have been murdered, you understand.''

\emph{Why didn't you tell me this when all this started?} She
wanted to say. But her parents \emph{had} told her. Management had
warned them, through bullhorns, that they were risking everything.
She'd laughed at them, filled with the feeling of sisterhood and
safety, of \emph{power}. That feeling was gone now.

And she'd gone with Affendi, and she'd worked in a factory that was
much like the factory she'd left, and there had been a union fight
much like the one she'd fought, but this time, they were better
prepared and the workers had called Nor ``Big Sister,'' a term of
endearment that had scared her a little, coming from the mouths of
women much older than her, coming from young girls who could never
appreciate the danger.

And this time, the owners hadn't fled, the workers had won better
conditions, and Big Sister Nor found that she didn't want to make
textiles anymore. She found that she had a taste for the fight.

Now there was a young man, someone called Matthew Fong, in
Shenzhen, and he was relying on her to help him win his dignity,
fair wages, and a safe and secure workplace. And he was doing it in
China, where unofficial unions were illegal and where labor
organizers sometimes disappeared into prison for years.

The Mighty Krang could speak a beautiful Mandarin as well as his
native Cantonese, so he was in charge of giving soundbites to the
foreign Chinese press, that network of news-resources serving the
hundreds of millions of people of Chinese ancestry living abroad.
They were key, because they were intimately connected to the whole
sprawling enterprise of imports and exports, and when they spoke,
the bureaucrats in Beijing listened. And The Mighty Krang could put
on a voice that was so smoothly convincing you'd swear it was a
newscaster.

Justbob was in charge of moral support for the strikers, talking to
them in broken Cantonese and Singlish and gamer-speak on conference
calls, keeping their morale up. She could work three phones and two
computers like a human octopus, her attention split across a dozen
conversations without losing the thread in any of them.

And Big Sister Nor? She was in-world, in several worlds, rallying
Webblies to the site of the Mushroom Kingdom, finding gamers
converging from all over Asia -- where it was night -- and from
Europe -- where it was day -- and America -- where it was morning.
Management had wasted no time moving replacement workers in. There
were always desperate subcontractors out in the provinces of China,
ten kids in a dead industrial town in Dongbei who'd been lured to
computers with pretty talk about getting paid to play. Across a
dozen different shards of the same Mushroom Kingdom world, a dozen
alternate realities, they came, and Big Sister Nor played general
in a skirmish against them, as strikers blocked the entrance to the
dungeon and sent a stream of pro-union chats and URLs to them even
as they fought them to keep them out of the dungeon.

The battle wasn't much of a fight, not at first. The replacement
workers were there to kill dumb non-player characters in a boring,
predictable way that wouldn't trigger the Mechanical Turks and
bring their operation to the attention of Nintendo-Sun. They were
all seasoned gamers, and they were used to teamplay, and many of
the Webblies had never fought side-by-side before. But the Webblies
were fighting for the movement, and the replacement workers -- they
called them ``scabs,'' another old word from out of history -- were
fighting because they didn't know what else to do.

It was a rout. The scabs were sent back to their respawn points by
the thousand, unable to return to work until they'd done their
corpse runs, and the Webblies raised their swords and shot
fireballs into the sky and cheered in a dozen languages.

The news was good from Shenzhen, too, judging from what Justbob was
saying into her headsets and typing onto her screens. The
strike-line was holding, and while the police were there, they
hadn't moved in -- in fact, it sounded like they'd moved to hold
back the private factory security!

Silently, Big Sister Nor thanked Matthew Fong for picking a fight
that -- seemingly -- they'd be able to win. She shouted up to Ezhil
in the front of Headshot, calling for ginseng bubble-tea all
around, the ginseng root would give them all a little shot of
energy. Couldn't live on caffeine and taurine alone!

``Ezhil!'' she shouted a minute later, looking up from her mouse.
``Bubble tea!'' If she'd been paying attention, she would have
noticed the squeak in his voice as he promised right away, right
away.

But her attention was fixed on her screens, because that's where it
was all suddenly going very wrong indeed. What she'd taken for
strikers' victorious fireballs launched into the sky were landing
among the players now, inflicting major damage. Just as she was
noticing this, a volley of skidding, spiked turtle-shells came
sliding in from offscreen, in twelve worlds at once.

\emph{Ambush!}

She barked the word into her headset in Mandarin, then Cantonese,
then Hindi, then English. The cry was taken up by the players and
they rallied, forming battle-squares, healers in the middle, tanks
on the outside, nimble thieves and scouts spreading out into the
mushroom forests, looking for the ambush.

This would work much better if they were a regular guild, all
playing on the side of the evil Bowser or of the valiant Princess
Peach, because if you were all on the same side, the game would
coordinate your movements for you, give you radar for where and how
all the other players were moving. But the strikers were from both
sides of Mushroom Kingdom's moral coin, and as far as the game was
concerned, they were sworn enemies. Their IMs were unintelligible
to one another, and the default option for any ``opposing'' av you
clicked on was ATTACK, leading to a lot of accidental skirmishes.

But gold farmers knew all about playing their own game, one that
lived on top of the game that the companies wanted them to play.
The game's communications tools were powerful and easy, but nothing
(apart from the ridiculous ``agreement'' you had to click every time
you started up the game) kept you from using anything you wanted.
They favored free chat systems developed to help corporate
work-groups collaborate; since these services always had free
demo-versions available, hoping to snag some office-person into
buying 30,000 licenses for their mega-corp. These systems even
allowed them to stream screen-caps from their own computers, and
Big Sister Nor saw to it that these were arranged sequentially,
forming a huge, panoramic view of the entire battlefield.

She flicked through the battlescenes and the communications hub,
fingers flying on the keyboard. They had a Koopa Turbo Hammer in
seven of the worlds, a huge, whirling god-hammer that could clobber
a score of attackers on a single throw, and she had it brought
forward, using the scouts' screencaps to pinpoint the enemies'
positions, conferring them to the hammer-throwers, a passel of
hulking Kongs with protruding fangs and enormous, hairy chests.

That was seven battles down; in the remaining five, she ordered the
Peaches to form up with their umbrellas at the ready, then had two
Bowsers ``bounce'' each of them, sticking to them while doing minimum
damage. The Peaches unfurled their umbrellas and sailed into the
air, taking their Bowsers with them, to drop behind enemy lines,
ready to breathe fire and stomp the opposing forces. This was a
devastating attack, one that was only possible if you played the
farmers' game, cooperating through a side-channel -- normally,
Bowsers and Princess Peaches were on the opposite sides of the
Great War that was at the center of the Mushroom Kingdom story.

It should have worked -- the hammers, the Bowsers, the skilled
players of a dozen guilds, bristling with armament and armor,
spelling and firing and skirmishing.

It should have worked -- but it hadn't.

The mysterious attackers -- she'd branded them ``Pinkertons'' in her
mind, after the strike-breaking goons from the Pinkerton Detective
Agency who'd been the old Wobblies' worst enemies -- had seemingly
endless numbers, and every attack they launched seemed to do
maximum damage. Meanwhile, they were able to pull off incredible
dodges and defenses against the strikers' attacks. And their aim!
Every fireball, every turtle, every sound-bomb, every flung axe
found its target with perfect accuracy.

It was almost as though they were --

-- Cheating!

That had to be it. They were using aimhacks, dodgehacks, all the
prohibited add-on software that the game was supposed to be able to
spot and disable. Somehow, they'd gotten past the game's defenses.
It didn't matter. The game was always stacked against gold
farmers.

``Pull back!'' she shouted. ``Retreat!'' This was going to have to be
guerrilla war, jungle war, hiding in the bushes and sniping at them
as they'd sniped at her. She'd lure them into the clearing that
marked the dungeon's entrance and then they'd slip around them into
the mushroom forest, using their superior coordination to trump the
hacks and numbers the Pinkertons had on their side. In her headset,
she heard the ragged breathing, the curses in six languages, the
laughter and shouting of players all over the world, listening to
her rap out commands in all the different versions of Mushroom
Kingdom that they were fighting in.

She found that she was grinning. This was \emph{fun.} This was a
\emph{lot} more fun than being tear-gassed.

It had been Big Sister Nor's idea to use the games for organizing.
Why risk your neck in the factory or standing at its gates when you
could slip right in among the workers, no matter where they were in
the world, and talk to them about joining up? Plenty of the MUTE
old guard had thought she was crazy, but there was lots of support,
too -- especially when Nor showed them that they could reach the
Indonesian textile workers who'd inherited her job when her factory
had closed up and moved on, simply by logging into Spirals of the
Golden Snail, a game that had taken the whole Malay peninsula by
storm.

It didn't matter where you fought, it mattered whether you won. And
the more she thought about it, the more she realized that they
could win in-game. The bosses were better at firing teargas at
them, but they were better at lobbing fireballs, pulsed energy
weapons, photon torpedoes and savage flying fish -- and they always
would be. What's more, a striker who lost a skirmish in-game merely
had to re-spawn and do a corpse-run, possibly losing a little
inventory in the process. A striker who lost a skirmish AFK -- away
from keyboard -- might end up dead.

Big Sister Nor lived in perpetual fear of having someone's death on
her hands.

The battle was turning again. The Pinkertons had all fallen for her
gambit, letting them rush past and back into the mushroom forest,
effectively trading places. Now they were digging in the woods,
laying little ambushes, fortifying positions and laying down
withering fire from all directions. The breathing, gasping,
triumphant muttering voices in her head and the hastily clattered
in-game chat gave her a feeling like the battle was resting
delicately balanced on her fingertips, every shift and change
dancing felt as a tremor against the sensitive pads of her
fingers.

Big Sister Nor called for her bubble tea again, realizing that a
very long time indeed had gone by since she'd first ordered it.
This time, no one answered. The skin on the back of her neck
prickled and she slipped her headphones off her head. Justbob and
The Mighty Krang caught on a second later, removing their earwigs.
There was no noise at all from the front of Headshot, none of the
normal hyperactive calling of gamer-kids, or the shouts of
guestworkers phoning home on cheap earwigs.

Big Sister Nor stood up quietly and quickly and backed up against
the wall, motioning to the others to do the same. On her screen,
she saw another rally by the Pinkertons, who'd taken advantage of
the sudden lack of strategic leadership to capture several of the
small striker strongholds. She inched her way toward the door and
very, very, \emph{very} slowly tilted her head to see around the
frame, then whipped it back as quick as she could.

\emph{RUN}, she mouthed to her lieutenants, and they broke for the
rear entrance, the escape hatch that Big Sister Nor always made
sure of before she holed up to do union work.

On their heels came the Pinkertons, the real world Pinkertons,
Malay men in workers' clothes, poor men, men armed with stout
sticks and a few chains, men who'd been making their way to the
door when Big Sister Nor chanced to look around it.

They shouted after them now, excited and tight voices, like the
catcalls of drunken boys on streetcorners when they were feeling
the bravery of numbers and hormones and liquor. That was a
dangerous sound. It was the sound of fools egging each other on.

Big Sister Nor hit the crashbar on the rear door with both palms,
slamming into it with the full weight of her body. The door's
gas-lift was broken, so it swung back like a mousetrap, and it was
a good thing it did, because it moved so fast that the two
Pinkertons waiting to bar their exit didn't have time to get out of
the way. One was knocked over on his ass, the other was slammed
into the cinderblock wall with a jarring thud that Big Sister Nor
felt in her palms.

The door rebounded into her, knocking her back into The Mighty
Krang, who caught her, pushed her on, hands on her shoulderblades,
breath ragged in her ears.

They were in a dark, narrow, stinking alley behind that connected
two of the Lorangs, the small streets that ran off Geylang Road,
and it was time to R and G -- to run and gun, what you did when all
your other plans collapsed. Big Sister Nor had thought this through
far enough to make sure they had a back door, but no farther than
that.

The Pinkertons were close behind, but they were all squeezed down
into the incredibly narrow confines of the alleyway, and no one
could really run or move faster than a desperate shuffle.

But then they broke free into the next Lorang, and Big Sister Nor
broke left, hoping to make it far enough up the road to get into
sight of the diners at the all-night restaurants.

She didn't make it.

One of the men threw his truncheon at her and it hit her square
between her shoulders, knocking the breath from her and causing her
to go down on one knee. Justbob twined one hand in her blouse and
hauled her to her feet with a sound of tearing cloth, and dragged
her on, but they'd lost a step to her fall, and now the men were on
them.

Justbob whirled around, snarling, shouting a worldless cry, using
the movement as inertia for a wild roundhouse kick that connected
with one of the Pinkertons, a man with sleepy eyes and a thick
mustache. Justbob's foot caught him in the side, and they all heard
the sound of his ribs breaking under the toe of her demure sandal
with its fake jewels. The sandal flew on and clattered to the road
with the cheap sound of paste gems.

The men hadn't expected that, and there was a moment when they
stopped in their tracks, staring at their fallen comrade, and in
that instant, Big Sister Nor thought that -- just maybe -- they
could get away. But Justbob's chest was heaving, her face contorted
in rage, and she \emph{leapt} at the next man, a fat man in a
sweaty sportcoat, thumbs aiming at his eyes, and as she reached
him, the man beside him lifted his truncheon and brought it down,
glancing off her high, fine cheekbone and then smashing against her
collarbone.

Justbob howled like a wounded dog and fell back, landing a hard
punch in her attacker's groin as she fell back.

But now the Pinkertons were on them, and their arms were raised,
their truncheons held high, and as the first one swung into Big
Sister Nor's left breast, she cried out and her mind was filled
with Affendi and her broken fingers, her unrecognizably bruised
face. Somewhere, just a few tantalizing meters up the Lorang, night
people were eating a huge feast of fish and goat in curry, the
smells in the air. But that was there. Here, Big Siter Nor was
infinitely far from them, and the truncheons rose and fell and she
curled up to protect her head, her breasts, her stomach, and in so
doing exposed her tender kidneys, her delicate short-ribs, and
there she lay, enduring a season in hell that went on for an
eternity and a half.

\tb

\shopad{This scene is dedicated to Chapters/Indigo, the national Canadian megachain. I was working at Bakka, the independent science fiction bookstore, when Chapters opened its first store in Toronto and I knew that something big was going on right away, because two of our smartest, best-informed customers stopped in to tell me that they'd been hired to run the science fiction section. From the start, Chapters raised the bar on what a big corporate bookstore could be, extending its hours, adding a friendly cafe and lots of seating, installing in-store self-service terminals and stocking the most amazing variety of titles.}
{\href{http://www.chapters.indigo.ca/books/For-The-Win-Cory-Doctorow/9780765322166-item.html}{Chapters/Indigo}}

Connor Prikkel sometimes thought of math as a beautiful girl, the
kind of girl that he'd dreamt of wooing, dating, even marrying,
while sitting in the back of any class that wasn't related to math,
daydreaming. A beautiful girl like Jenny Rosen, who'd had classes
with him all through high-school, who always seemed to know the
answer no matter what the subject, who had a light dusting of
freckles around her nose and a quirky half-smile. Who dressed in
jeans that she'd tailored herself, in t-shirts she'd modded,
stitching multiple shirts together to make tight little
half-shirts, elaborate shawls, mock turtelnecks.

Jenny Rosen had seemed to have it all: beauty and brains and, above
all, rationality: she didn't like the way that store-bought jeans
fit, so she hacked her own. She didn't like the t-shirts that
everyone wore, so she changed the shirts to suit her taste. She was
funny, she was clever, and he'd been completely, head-over-heels in
love with her from sophomore English right through to senior
American History.

They'd been friendly through that time, though not really friends.
Connor's friends were into gaming and computers, Jenny's friends
were jocks and school-paper kids. But friendly, sure, enough to say
hello in the hallway, enough to become lab partners in sophomore
physics (she was a careful taker of notes, and her hair-stuff
smelled \emph{amazing}, and their hands brushed against each other
a hundred times that semester).

And then, in senior year, he'd asked her out to a movie. Then she'd
asked him to a track rally. Then he'd asked her to work with him on
an American History project on Chinese railway workers that
involved going to Chinatown after school, and there they'd had a
giant dim sum meal and then sat in a park and talked for hours, and
then they'd stopped talking and started kissing.

And one thing led to another, and the kissing led to more kissing,
and then their friends all started to whisper, ``Did you hear about
Connor and Jenny?'' and she met his parents and he met hers. And it
had all seemed perfect.

But it wasn't perfect. Anything but.

In the four months, two weeks and three days that they were
officially a couple, they had approximately 2,453,212 arguments,
each more blazing than the last. Theoretically, he understood
everything he needed to about her. She loved sports. She loved to
use her mind. She loved humor. She loved silly comedies and slow
music without words.

And so he would go away and plan out exactly how to deliver all
these things to her, plugging in her loves like variables into an
equation, working out elaborate schemes to deliver them to her.

But it never worked. He'd work it out so that they could go to a
ball game at AT\&T Park and she'd want to go see a concert at Cow
Palace instead. He'd take her to see a new wacky comedy and she'd
want to go home and work on an overdue assignment. No matter how
hard he tried to get her reality and his theory to match up, he
always failed.

In his heart of hearts, he knew it wasn't her fault. He knew that
he had some deficiency that caused him to live in the imaginary
world he sometimes thought of as ``theory-land,'' the country where
everything behaved as it was supposed to.

After graduation, through his bachelor's degree in pure math at
Berkeley, his Masters in Signal Processing at Caltech, and the
first year of a PhD in economics at Stanford, he had occasion to
date lots of beautiful women, and every time, he found himself
ground to pulp between the gears of real-world and theory-land. He
gave up on women and his PhD on a fine day in October, telling the
prof who was supposed to be his advisor that he could find someone
else to teach his freshman math courses, grade his papers, and
answer his email.

He walked off the Stanford campus and into the monied streets of
Palo Alto, and he packed up his car and drove to his new job, as
chief economist for Coca Cola's games division, and finally, he
found a real world that matched the beautiful elegance of
theory-land.

Coca Cola ran or franchised anywhere from a dozen to thirty
game-worlds at any given time. The number of games went up or down
according to the brutal, elegant logic of the economics of fun:

a certain amount of difficulty

plus

a certain amount of your friends

plus

a certain amount of interesting strangers

plus

a certain amount of reward

plus

a certain amount of opportunity

equalled

fun

.

That was the equation that had come to him one day early in his
second semester of the PhD grind, a bolt of inspiration like the
finger of god reaching down into his brain. The magic was that
equals sign, just before the fun, because once you could express
fun as a function of other variables, you could establish its
relationship to those variables -- if we reduce the difficulty and
the number of your friends playing, can we increase the reward and
make the fun stay the same?

This line of thought drove him to phone in a sick-call to his
advisor and head straight home, where he typed and drew and
scribbled and thought and thought and thought, and he phoned in
sick the next day, and the next -- and then it was the weekend, and
he let his phone run down, shut off his email and IM, and worked,
eating when he had to.

By the time he found himself shoving fingerloads of butter into his
mouth, having emptied the fridge of all else, he knew he was onto
something.

He called them the Prikkel equations, and they described in
elegant, pure, abstract math the relationship between all the
variables that went into fun, and how fun equalled money, inasmuch
as people would pay to play fun games, and would pay more for items
that had value in those games.

Technically, he should have sent the paper to his advisor. He'd
signed a contract when he was accepted to the University giving
ownership of all his ideas to the school forever, in exchange for
the promise of someday adding ``PhD'' to his name. It hadn't seemed
like a good idea at the time, but the alternative was the awesomely
craptacular job-market, and so he'd signed it.

But he wasn't going to give this to Stanford. He wasn't going to
\emph{give} it to anybody. He was going to \emph{sell} it.

He didn't go back to campus after that, but rather plunged into a
succession of virtual worlds, plotting the time in hours it took
him to achieve different tasks, and comparing that to the price of
gold in the black-, grey- and white-market exchanges for in-game
wealth.

Each number slotted in perfectly, just where he'd expected it to
go. His equations \emph{fit}, and the world fit his equations. He'd
finally found a place where the irrational was rendered
comprehensible. And what's more, he could \emph{manipulate} the
world using his equations.

He decided to do a little fantasy trading: working from his
equations, he'd predicted that the gold in MAD Magazine's
Shlabotnik's Curse was wildly undervalued. It was an incredibly fun
game -- or at least, it satisfied the fun equation -- but for some
reason, game money and elite items were going for peanuts. Sure
enough, in 36 hours, his imaginary MAD Money was worth \$130 in
imaginary real money.

Then he took his \$130 stake and sank it into four other game
currencies, spreading out his bets. Three of the four hit the
jackpot, bringing his total up to \$200 in imaginary dollars. Now
he decided to spend some real money -- he already knew that he
wasn't going back to campus, so that meant his grad student grant
would vanish shortly. He'd need to pay the rent while he searched
for a buyer for his equations.

He'd already proven to his own satisfaction that he could predict
the movement of game currencies, but now he wanted to branch out
into the weirder areas of game economics: elite items, the rare
prestige items that were insanely difficult to acquire in-game.
Some of them had a certain innate value -- powerful weapons and
armor, ingredients for useful spells -- but others seemed to hold
value by sheer rarity or novelty. Why should a purple suit of armor
cost ten times as much as the red one, given that both suits of
armor had exactly the same play value?

Of course, the purple one was much harder to come by. You had to
either buy it with unimaginable mountains of gold -- so players who
saw your av sporting it would assume that you had played your ass
off to earn for it -- or pull off some fantastic stunt to get it,
like doing a 60-player raid on a nigh-unkillable boss. Like a
designer label on an otherwise unimpressive article of clothing,
these items were valuable because people who saw them assumed they
had to cost a lot or be hard to get, and thought more of the owner
for having them. In other words, they cost a lot because\ldots{}they
cost a lot!

So far, so good -- but could you use Prikkel's Equations to predict
\emph{how much} they'd cost? Connor thought so. He thought you
could use a formula that combined the fun quotient of the game and
the number of hours needed to get the item, and derive the ``value''
of any elite item from purple armor to gold pinstripes on your
spaceship to a banana-cream pie the size of an apartment block.

Yes, it would work. Connor was sure of it. He started to calculate
the true value of various elite items, casting about for
undervalued items. What he discovered surprised him: while virtual
currency tended to rest pretty close to its real value, plus or
minus five percent, the value-gap in elite items was
\emph{gigantic}. Some items routinely traded for two or three
hundred percent of their real value -- as predicted by his
Equations, anyway -- and some traded at a pittance.

Never for a moment did he doubt his equations, though a more humble
or more cautious person might have. No, Connor looked at this
paradoxical picture and the first thing that came into his head
wasn't ``Oops.'' It was \emph{BUY}!

And he bought. Anything that was undervalued, he bought, in great
storehouses, so much that he had to create alts and secondaries in
many worlds, because his primary characters couldn't \emph{carry}
all the undervalued junk he was buying. He spent a hundred dollars
-- two hundred -- three hundred, snapping up assets, spreadsheeting
their nominal value. On paper, he was incredibly, unspeakably rich.
On paper, he could afford to move out of his one-bedroom apartment
that was a little too close to the poor and scary East Palo Alto
for his suburban tastes, buy a McMansion somewhere on the
peninsula, and go into business full time, spending his days buying
magic armor and zeppelins and flaming hamburgers, and his evening
opening checks.

In reality, he was going broke. The theory said that these assets
were wildly undervalued. The marketplace said otherwise. He'd
cornered the market on several kinds of marvellous gew-gaws, but no
one seemed to actually want to buy them from him. He remembered
Jenny Rosen, and all the crushing ways that theory and reality
could sometimes stop communicating with one another.

When the first red bills came in, he stuck them under his keyboard
and kept buying. He didn't need to pay his cell phone bill. He
didn't need his cell-phone to buy magic lizards. His student loans?
He wasn't a student anymore, so he didn't see why he should worry
about them -- they couldn't kick him out of school. Car payments?
Let them repo it (and they did, one night, at 2AM, and he waved
goodbye to the little hunk of junk as the repo man drove it away,
then turned back to his keyboard). Credit card bills? So long as
there was one card that was still good, one card he could use to
pay the subscription fees for his games, that was all that
mattered.

Living close to East Palo Alto had its advantages: for one thing,
there were food-banks there, places where he could line up with
other poor people to get giant bricks of government cheese, bags of
day-old bread, boxes of irregular and unlovely root-vegetables. He
fried all the latter in an all-day starch festival and froze them,
and then he proceeded to live off of cheese and potato sandwiches,
and one morning, he realized that his entire body and everything
that came out of it -- breath, burps, farts, even his urine --
smelled of cheese sandwiches. He didn't care. There were ostrich
plumes to buy.

Disaster struck: he lost track of which credit card he was ignoring
and had half of his accounts suspended when his monthly
subscription fees bounced. Half his wealth, wiped out. And the
other card wasn't far behind.

He thought he could probably call his parents and grovel a bit and
get a bus ticket to Petaluma, hole up in his folks' basement and
lick his wounds and be yet another small-town failure who came home
with his tail between his legs. He'd need a roll of quarters and a
payphone, of course, because his cellphone was now an inert,
unpaid, debt-haunted brick. Lucky for him, East Palo Alto was the
kind of place where you got lots of people who were too poor even
to go into debt with a cell-phone, people who also needed to use
payphones.

He tucked himself into his grimy bed on a Wednesday morning and
thought, \emph{Tomorrow, tomorrow I will call them.}

But tomorrow he didn't. And Friday he didn't, though he was now out
of government cheese and wasn't eligible for more until Monday. He
could eat potato sandwiches. He couldn't buy assets anymore, but he
was still tracking them, watching them trade and identifying the
bargains he \emph{would} buy, if only he had a little more
liquidity, a little more cashish.

Saturday, he brushed his teeth, because he remembered to do that
sometimes, and his gums bled and there were sores on the insides of
his mouth and \emph{now} he was ready to call his parents, but it
was 11PM somehow, how did the day shoot past, and they went to bed
at 9 every night. He'd call them on Sunday.

And on Sunday -- on Sunday -- on that magical, wonderful Sunday, on
Sunday --

THE MARKET MOVED!

There he was, pricing assets, recording their values in his
spreadsheet, and he realized that the asset he was booking -- a
steampunk leather gasmask adorned with a cluster of huge leathery
ear-trumpets and brass cogs and rivets (no better than a standard
gasmask in the blighted ecotastrophe world that was Rising Seas,
but infinitely cooler) -- had already been entered onto his sheet,
weeks before. Indeed, he'd booked the mask when its real world cash
value was about \$0.18, against the \$4.54 the Equations predicted.
And now he was booking it at \$1.24, which meant that the 750 of
them he had in inventory had just jumped from \$135 to \$930, a
profit of \$795.

There was a strange sound. He realized after a moment that it was
his stomach, growling for food. He could flip his gasmasks now,
take the \$795 onto one of his PayPal debit cards, and eat like a
king. He might even be able to buy back some of his lost accounts
and recover his assets.

But Connor did not consider doing this, even for a second. He
dashed to the sink and filled up three cooking pots with water and
brought them back to his desk, along with a cup. He filled the cup
and drank it, filled it and drank it, filling his stomach with
water until it stopped demanding to be filled. This was California,
after all, where people paid good money to go to ``retreats'' for
``liquid fasting'' and ``detox.'' So he could wait out food for a day
or two\ldots{} After all, his Equations predicted that these things
should go to \$3,405. He was just getting started.

And now the gasmasks were rising. He'd get up, go to the bathroom
-- his kidneys were certainly getting a workout! -- and return to
check the listings on the official exchange sites and the
black-market ones where the gold-farmers hung out. He had a little
formula for calculating the real price, using these two prices as a
kind of beacon. No matter how he calculated it, his gasmasks were
rising.

And yes, some of his other assets were rising, too. A robot dog, up
from \$1.32 to \$1.54, still pretty far off from the \$8.17 he'd
predicted, but he owned a thousand of the things, which meant that
he'd just made \$1,318.46 here, and he was just getting started.

Up and up the prices went, as asset after asset attained liftoff,
and he began to suspect that his asset-buying spree had coincided
with an inter-world depression across all virtual economies, which
accounted for the huge quantities of undervalued assets he'd found
lying around. There was probably an interesting cause for all those
virtual economies slumping at once, but that was something to study
another day. As it was, he was more interested in the fact that the
economies were bouncing back while he was sitting on mountains of
dirt-cheap imaginary gewgaws, knickknacks, tchotchkes and
white-elephants, and that their values were taking off like crazy.

And now it was time to convert some of those assets to money and
some of that money to food, rent, and paid-off bills. His
collection of articulated tentacles from Nemo's Adventures on the
Ocean Floor were maturing nicely -- he'd bought them at \$0.22,
priced them at \$3.21, and now they were trading at \$3.27 -- so he
dumped them, and regretted that he'd only bought 400 of them.
Still, he managed to dump them for a handy \$1150 profit (by the
time he'd sold 300 of them, the price had started to tip down
again, as the supply of tentacles increased and the demand
diminished).

The money dribbled into his PayPal account and he used that to
order three pizzas, a gallon of orange juice and ten boxes of
salad, paid off his suspended accounts, and sent \$400 to his
landlord against the \$3500 he owed for two months' rent, along
with a begging letter promising to pay the rest off within a day or
two.

While he waited for the pizzas to arrive, he decided he'd better
shower and shave and try to do something about his hair, which had
started to go into dreadlocks from a month without seeing a
hairbrush. In the end, he just cut the tangles out, and got dressed
in something other than his filthy housecoat for the first time in
a week -- marvelling at how his jeans hungoff his prominent hips,
how his t-shirt clung to his wasted chest, his ribs like a
xylophone through the pale skin. He opened all the windows, aware
of the funk of body-odor and stale computer-filtered air in his
apartment, and realized as he did that it was morning, and thanked
his lucky stars that he lived in a college town, where you could
get a pizza delivered at 8:30AM.

He barfed after eating the first pizza, getting most of it into the
big pot he'd used to hold his drinking water, big chunks of crust
and pepperoni, reeking of sour stomach-acid. He didn't let that put
him off. His PayPal account was now bulging, up to \$50,000, and he
was just getting started. He switched to salads and juice, figuring
it would take a little while to get used to food again, and not
having the time just now to take a long bio-break. His body would
have to wait. He ordered an urn of coffee from a place that catered
corporate meetings, the kind of thing that held 80 cups' worth, and
threw in a plate of sliced veggie and some pastries.

Selling was getting easier now. The economies were bouncing back,
and from the tone of the thank-you messages he got from his buyers,
he understood that there was a kind of reverse-panic in the air, a
sense that players all over the world were starting to worry that
if they didn't buy this junk now, they'd never be able to buy it,
because the prices would go up and up and up forever.

And it was then that he had his second great flash, the second time
that the finger of God reached down and touched his mind, with a
force that shook him out of his chair and set him to pacing his
living room like a tiger, muttering to himself.

Once, when he'd been working on his Masters, he'd participated in a
study for a pal in the economics department. They'd locked twenty
five grad students into a room and given each of them a poker chip.
``You can do whatever you want with those chips,'' the experimenter
had said. ``But you might want to hang onto them. Every hour, on the
hour, I'm going to unlock this door and give you twenty dollars for
each poker chip you're holding. I'll do this eight times, for the
next eight hours. Then I'll unlock the door for a final time and
you can go home and your poker chips will be worthless -- though
you'll be able to keep all the money you've acquired over the
course of the experiment.''

He'd snorted and rolled his eyes at the other grad students, who
were mostly doing the same. It was going to be a loooong eight
hours. After all, everyone knew what the value of the poker chips
were: \$160 in the first hour, \$140 in the next, \$120 in the next
and so on. What would be the point of trading a poker chip to
anyone else for anything less than it was worth?

For the first hour, they all sat around and griped about how boring
it all was. Then, the experimenter walked back into the room with a
tray of sandwiches and 25 \$20 bills. ``Poker chips, please,'' he
said, and they dutifully held out their chips, and one by one, each
received a crisp new \$20 bill.

``One down, seven to go,'' someone said, once the experimenter had
left. The sandwiches were largely untouched. They waited. They
flirted in a bored way, or made small talk. The hour ticked past.

Then, at 55 minutes past the hour, one guy, a real joker with red
hair and mischievous freckles, got out of the beat-up old orange
sofa turned to the prettiest girl in the room, a lovely Chinese
girl with short hair and homemade clothes that reminded Connor of
Jenny's fashion, and said, ``Rent me your poker chip for five
minutes? I'll pay you \$20.''

That cracked the entire room up. It was the perfect demonstration
of the absurdity of sitting around, waiting for the \$20 hour. The
Chinese girl laughed, too, and they solemnly traded. In came the
grad student, five minutes later, with another wad of twenties and
a cooler filled with smoothies in tetrapaks. ``Poker chips, please,''
he said, and the joker held up his two chips. They all grinned at
one another, like they'd gotten one over on the student, and he
grinned a little too and handed two twenties to the redhead. The
Chinese girl held up her extra twenty, showing that she had the
same as everyone else. Once he'd gone, Red gave her back her chip.
She pocketed it and went back to sitting in one of the dusty old
armchairs.

They drank their smoothies. There were murmured conversations, and
it seemed like a lot of people were trading their chips back and
forth. Connor laughed to see this, and he wasn't the only one, but
it was all in fun. Twenty dollars was the going rate for an hour's
rental, after all -- the exactly and perfectly rational sum.

``Give me your poker-chip for 20 minutes for \$5?'' The asker was at
the young end of the room, about 22, with a soft, cultured southern
accent. She was also very pretty. He checked the clock on the wall:
``It's only half past,'' he said. ``What's the point?''

She grinned at him. ``You'll see.''

A five dollar bill was produced and the poker-chip left his
custody. The pretty southern girl talked with another girl, and
after a moment, \$10 traded hands, rather conspicuously. ``Hey,'' he
began, but the southern girl tipped him a wink, and he fell
silent.

Anxiously, he watched the clock, waiting for the 20 minutes to tick
past. ``I need the chip back,'' he said, to the southern girl.

She shrugged. ``You need to talk to her,'' she said, jerking her
thumb over her shoulder, then she ostentatiously pulled a paperback
novel -- \emph{The Fountainhead} -- out of her backpack and buried
her nose in it. He felt a complicated emotion: he wanted to laugh,
and he wanted to shout at the girl. He chose laughter, conscious of
all the people watching him, and approached the other girl, who was
tall and solidly built, with a no-nonsense look that went perfectly
with her no-nonsense clothes and haircut.

``Yes?'' she said, when he approached her.

``You've got my chip,'' he said.

``No,'' she said. ``I do not.''

``But the chip she sold you, I'd only rented it to her.''

``You need to take it up with her,'' the girl who had his chip said.

``But it's my chip,'' he said. ``It wasn't hers to sell to you.'' He
didn't want to say,
\emph{I'm also pretty intimidated by anyone who has the gall to pull a stunt like that.}
Was it his imagination, or was the southern girl smiling to
herself, a smug little smile?

``Not my problem, I'm afraid,'' she said. ``Too bad.''

Now \emph{everyone} was watching very closely and he felt himself
blushing, losing his cool. He swallowed and tried to put on a
convincing smile. ``Yeah, I guess I really should be more careful
who I trust. Will you sell me my chip?''

``My chip,'' she said, flipping it in the air. He was tempted to try
and grab it out of the air, but that might have led to a wrestling
match right here, in front of everyone. How embarrassing!

``Yeah,'' he said. ``Your chip.''

``OK,'' she said. ``\$15.''

``Deal,'' he said, thinking,
\emph{I've already earned \$45 here, I can afford to let go of \$15.}

``In seven minutes,'' she said. He looked at the clock: it was 11:54.
In seven minutes, she'd have gotten his \$20. Correction:
\emph{her} \$20.

``That's not fair,'' he said.

She raised one eyebrow at him, hoisting it so high it seemed like
it'd touch her hairline. ``Oh really? I think that this chip is
worth \$120. \$15 seems like a bargain to you.''

``I'll give you \$20,'' the redhead said.

``\$25,'' said someone else, laughing.

``Fine, fine,'' Connor said, hastily, now blushing so hard he
actually felt light-headed. ``\$15.''

``Too late,'' she said. ``The price is now \$25.''

He heard the room chuckle, felt it preparing to holler out a new
price -- \$40? \$60? -- and he quickly snapped, ``\$25'' and dug out
his wallet.

The girl took his money -- how did he know she would give him the
chip? He felt like an idiot as soon as it had left his hand -- and
then the experimenter came in. ``Lunch!'' he called out, wheeling in
a cart laden with boxed salads, vegetarian sushi, and a couple
buckets of fried chicken. ``Poker chips!'' The twenties were handed
around.

The girl with his money spent an inordinate amount of time picking
out her lunch, then, finally, turned to him with a look of fakey
surprise, and said, ``Oh right, here,'' and handed him his chip. The
guy with the red hair snickered.

Well, that was the beginning of the game, the thing that turned the
next five hours into one of the most intense, emotional experiences
he'd ever taken part in. Players formed buying factions, bought out
other players, pooled their wealth. Someone changed the wall clock,
sneakily, and then they all spent 30 minutes arguing about who's
watch or phone was more accurate, until the researcher came back in
with a handful of twenties.

In the sixth hour of the experiment, Connor suddenly realized that
he was in the minority, an outlier among two great factions: one of
which controlled nearly all the poker chips, the other of which
controlled nearly all the cash. And there was only two hours left,
which meant that his single chip was worth \$40.

And something began to gnaw at his belly. Fear. Envy. Panic. The
certainty that, when the experiment ended, he'd be the only poor
one, the only one without a huge wad of cash. The savvy traders
around them had somehow worked themselves into positions of power
and wealth, while he'd been made tentative by his bad early
experience and had stood pat while everyone else created the
market.

So he set out to buy more chips. Or to sell his chip. He didn't
care which -- he just wanted to be rich.

He wasn't the only one: after the seventh hour, the entire
marketplace erupted in a fury of buying and selling, which made
\emph{no damned sense} because now, \emph{now} the chips were all
worth exactly \$20 each, and in just a few minutes, they'd be
absolutely worthless. He kept telling himself this, but he also
found himself bidding, harder and harder, for chips. Luckily, he
wasn't the most frightened person in the room. That turned out to
be the redhead, who went after chips like a crackhead chasing a
rock, losing all the casual cool he'd started with and chasing
chips with money, IOUs.

Here's the thing, cash should have been \emph{king}. The cash would
still be worth something in an hour. The poker chips were like soap
bubbles, about to pop. But those holding the chips were the kings
and queens of the game, of the market. In seven short hours, they'd
been conditioned to think of the chips as ATMs that spat out
twenties, and even though their rational minds knew better, their
hearts were all telling them to corner the chip.

At 4:53, seven minutes before his chip would have its final payout,
he sold it to the Fountainhead lady for \$35, smirking at her until
she turned around and sold it to the redhead for \$50. The
researcher came into the room, handed out his twenties, thanked
them for their time, and sent them on their way.

No one met anyone else's eye as they departed. No one offered
anyone else a phone number or email address or IM. It was as if
they'd all just done something they were ashamed of, like they'd
all taken part in a mob beating or a witch-burning, and now they
just wanted to get away. Far away.

For years, Connor had puzzled over the mania that had seized that
room full of otherwise sane people, that had found a home in his
own heart, had driven him like an addiction. What had brought him
to that shameful place?

Now, as he watched the value of his virtual assets climb and climb
and climb, climb higher than his Equations predicted, higher than
any sane person should be willing to spend on them, he
\emph{understood}.

The emotion that had driven them in that experimenter's lab, that
was driving the unseen bidders around the world: it wasn't greed.

It was \emph{envy}.

Greed was predictable: if one slice of pizza is good, it makes
sense that your intuition will tell you that five or ten slices
would be even better.

But envy wasn't about what was good: it was about what someone else
thought was good. It was the devil who whispered in your ear about
your neighbor's car, his salary, his clothes, his girlfriend --
better than yours, more expensive than yours, more beautiful than
yours. It was the dagger through your heart that could drive you
from happiness to misery in a second without changing a single
thing about your circumstances. It could turn your perfect life
into a perfect mess, just by comparing it to someone who had
more/better/prettier.

Envy is what drove that flurry of buying and selling in the lab.
The redhead, writing IOUs and emptying his wallet: he'd been driven
by the fear that he was missing out on what the rest of them were
getting. Connor had sold his chip in the last hour because everyone
else seemed to have gotten rich selling theirs. He could have kept
his chip to himself for eight hours and walked out \$160 richer,
and used the time to study, or snooze, or do yoga in the back. But
he'd felt that siren call:
\emph{Someone else is getting rich, why aren't you?}

And now the markets were running and \emph{everything} was shooting
up in value: his collection of red oxtails (useful in the
preparation of the Revelations spell in Endtimes) should have been
selling at \$4.21 each. He'd bought them for \$2.10 each. They were
presently priced at \emph{\$14.51 each}.

It was insane.

It was wonderful.

Connor knew it couldn't last. Eventually, there would be a
marketwide realization that these were overpriced -- just as the
market had recently realized that they had been underpriced.
Bidding would cease. The last, most scared person who bought an
overpriced game asset would be unable to flip it, would have to pay
for it.

Rationally, he supposed he should sell at his Equation-predicted
number. Anything higher was just a bet on someone else's
irrationality. But still -- would he really be better off flipping
his 50 oxtails for \$200, when he could wait a few minutes and sell
them for \$700? It didn't have to be all or nothing. He divided his
assets up into two groups; the ones he'd bought most cheaply, he
set aside to allow to rise as far as they could. They represented
his lowest-risk inventory, the cheapest losses to absorb. The
remaining assets, he flipped at the second they reached the value
predicted by his Equations.

He quickly sold out of the second group, leaving him to watch the
speculative assets climb higher and higher. He had a dozen games
open on his computer, flipping from one to the next, monitoring the
chatter and their associated websites and marketplaces, getting a
sense for where they were going. Filtering the tweets and the
status messages on the social networks, he felt a curious sense of
familiarity: they were going nuts out there in a way that was
almost identical to the craziness that had swept over the group in
the poker-chip experiment. In their hearts, everyone knew that
peacock plumes and purple armor were vastly overvalued, but they
also knew that some people were getting rich off of them, and that
if the prices kept climbing that they'd never be able to own one
themselves.

Nevermind that they never wanted to own one \emph{before}, of
course! The important thing wasn't what they needed or loved, it
was the idea that someone else would have something that they
couldn't have.

Connor had made his second great discovery: Envy, not greed, was
the most powerful force in any economy.

(Later, when Connor was writing articles about this for glossy
magazines and travelling all over the world to talk about it,
plenty of people from marketing departments would point out that
they'd known this for generations had spent centuries producing ads
that were aimed squarely at envy's solar plexus. It was true, he
had to admit -- but it was also true that practically every
economist he'd ever met had considered marketing people to be a
bunch of shallow, foolish court jesters with poor math skills and
had therefore largely ignored them)

He watched the envy mount, and tried to get a feel for it all, to
track the sentiments as they bubbled up. It was hard -- practically
impossible, honestly -- because it was all spread out and no one
had written the chat programs and the games and the social networks
and the twitsites to track this kind of thing. He ended up with a
dozen browsers open, each with dozens of tabs, flipping through
them in a high speed blur, not reading exactly, but skimming,
absorbing the \emph{sense} of how things were going. He could feel
the money and the thoughts and the goods all balanced on his
fingertips, feel their weight shifting back and forth.

And so he felt it when things started to go wrong. It was a bunch
of subtle indicators, a blip in prices in this market, a joyous
tweet from a player who'd just discovered an easy-to-kill miniboss
with a huge storehouse stuffed with peacock feathers. The envy
bubble was collapsing. Someone had popped it and the air was
whooshing out.

SELL!

At that moment, his speculative assets were theoretically worth
over \emph{four hundred thousand dollars}, but ten minutes later,
it was \$250,000 and falling like a rock. He knew this one too --
fear -- fear that everyone else got out while the getting was good,
that the musical chairs had all been filled, that you were the most
scared person in a chain of terrorized people who bought overpriced
junk because someone even more scared would buy it off of you.

But Connor could rise above the fear, fly over it, flip his assets
in a methodical, rapidfire way. He got out with over \$120,000 in
cash, plus the \$80,000 he'd gotten from his ``rationally priced''
assets, and now his PayPal accounts were bulging with profits and
it was all over.

Except it wasn't.

One by one, his game accounts began to shut down, his characters
kicked out, his passwords changed. He was limp with exhaustion, his
hands trembling as he typed and re-typed his passwords. And then he
noticed the new email, from the four companies that controlled the
twelve games he'd been playing: they'd all cut him off for
violating their Terms of Service. Specifically, he'd ``Interfered
with the game economy by engaging in play that was apt to cause
financial panic.''

``What the hell does that mean?'' he shouted at his computer,
resisting the urge to hurl his mouse at the wall. He'd been awake
for over 48 hours now, had made hundreds of thousands of dollars in
a mere weekend, and had been graced with a thunderbolt of
realization about the way that the world's economy ran. Oh, and
he'd validated his Equations.

He could solve this problem later.

He didn't even make it into bed. He curled up on the floor, in a
nest of pizza boxes and blankets, and slept for 18 hours, until he
was awoken by the bailiff who came to evict him for being three
months behind on the rent.

\tb

\shopad{This scene is dedicated to San Francisco's Booksmith, ensconced in the storied Haight-Ashbury neighborhood, just a few doors down from the Ben and Jerry's at the exact corner of Haight and Ashbury. The Booksmith folks really know how to run an author event -- when I lived in San Francisco, I used to go down all the time to hear incredible writers speak (William Gibson was unforgettable). They also produce little baseball-card-style trading cards for each author -- I have two from my own appearances there.}
{\href{http://thebooksmith.booksense.com}{Booksmith}: 1644 Haight St. San Francisco CA 94117 USA +1 415 863 8688}

Yasmin didn't see Mala anymore. If you weren't in the gang,
``General Robotwallah'' didn't want to talk to you.

And Yasmin didn't want to be in the gang.

She, too, had had a visit from Big Sister Nor. The woman had made
sense. They did all the work, they made almost none of the money.
Not just in games, either -- her parents had spent their whole
lives toiling for others, and those others had gotten wealthier and
wealthier, and they'd stayed in Dharavi.

Mr Banerjee had paid Mala's army more than any other slum-child
could earn, it was true, and they were getting paid for playing
their game, which had felt like a miracle -- at first. But the more
Yasmin thought about it, the less miraculous it became. Big Sister
Nor showed her pictures, in-game, of the workers whose jobs they'd
been disrupting. Some had been in Indonesia, some had been in
Thailand, some had been in Malaysia, some had been in China. And
lots of them had been in India, in Sri Lanka, in Pakistan, and in
Bangladesh, where her parents had come from. They looked like her.
They looked like her friends.

And \emph{they} were just trying to earn money, too. They were just
trying to help their families, the way Mala's army had. ``You don't
have to hurt other workers to survive,'' Big Sister Nor told her.
``We can all thrive together.''

Day after day, Yasmin had snuck into Mrs Dibyendu's Internet cafe
before the Army met -- not at Mrs Dibyendu's, but at a new Internet
shop a little further down the road, near the women's papadam
collective -- and chatted with Big Sister Nor and listened to her
stories of how it could be.

She'd never talked about it with anyone else in the army. As far as
they knew, she was Mala's loyal lieutenant, sturdy and dependable.
She had to enforce discipline in the ranks, which meant keeping the
boys from fighting too much and keeping the girls from ganging up
on one another with hissing, whispered rumors. To them, she was a
stern, formidable fighter, someone to obey unconditionally in
battle. She couldn't approach them to say, ``Have you ever thought
about fighting for workers instead of fighting against them?''

No matter how much Big Sister Nor wanted her to.

``Yasmin, they listen to you, la, they love you and look up to you.
You say it yourself.'' Her Hindi was strangely accented and peppered
with English and Chinese words. But there were lots of funny
accents in Dharavi, dialects and languages from across Mother
India.

Finally, she agreed to do it. Not to talk to the soldiers, but to
talk to Mala, who had been her friend since Yasmin had found her
carrying a huge sack of rice home from Mr Bhatt's shop with her
little brother, looking lost and scared in the alleys of Dharavi.
She and Mala had been inseparable since then, and Yasmin had always
been able to tell her anything.

``Good morning, General,'' she said, falling into step beside Mala as
she trekked to the community tap with a water-can in each hand. She
took one can from Mala and took her now free hand and gave it a
sisterly squeeze.

Mala grinned at her and squeezed back, and the smile was like the
old Mala, the Mala from before General Robotwallah had come into
being. ``Good morning, Lieutenant.'' Mala was pretty when she smiled,
her serious eyes filled with mischief, her square small teeth all
on display. When she smiled like this, Yasmin felt like she had a
sister.

They talked in low voices as they waited for the tap, passing
gupshup about their families. Mala's mother had met a man at Mr
Bhatt's factory, a man whose parents had come to Mumbai a
generation before, but from the same village. He'd grown up on
stories about life in the village, and he could listen to Mala's
mamaji tell stories of that promised land all day long. He was
gentle and had a big laugh, and Mala approved. Yasmin's Nani, her
grandmother, had been in touch with a matchmaker in London, and she
was threatening to find Yasmin a husband there, though her parents
were having none of it.

Once they had the water, Yasmin helped Mala carry it back to her
building, but stopped her before they got there, in the lee of an
overhanging chute that workers used to dump bundled cardboard from
a second-story factory down to carriers on the ground. The factory
hadn't started up yet, so it was quiet now.

``Big Sister Nor asked me to talk to you, Mala.''

Mala stiffened and her smile faded. They weren't talking as sisters
anymore. The hard look, the General Robotwallah look, was in her
eyes. ``What did she say to you?''

``The same she said to you, I imagine. That the people we fight
against are also workers, like us. Children, like us. That we can
live without hurting others. That we can work with them, with
workers everywhere --''

Mala held up her hand, the General's command for silence in the
war-room. ``I've heard it, I've heard it. And what, you think she's
right? You want to give it all up and go back to how we were
before? Back to school, back to work, back to no money and no food
and being afraid all the time?''

Yasmin didn't remember being afraid all the time, and school hadn't
been that bad, had it? ``Mala,'' she said, placatingly. ``I just
wanted to talk about this with you. You've saved us, all of us in
the Army, brought us out of misery and into riches and work. But we
work and work for Mr Banerjee, for his bosses, and our parents work
for bosses, and the children we fight in the game work for bosses,
and I just think --'' She drew in a breath. ``I think I have more in
common with the workers than I do with the bosses. That maybe, if
we all come together, we can demand a better deal from all of them
--''

Mala's eyes blazed. ``You want to lead the Army, is that it? You
want to take us on this mission of yours to make \emph{friends}
with everyone, to join with them to fight Mr Banerjee and the
bosses, Mr Bhatt who owns the factory and the people who own the
game? And how will you fight, little Yasmin? Are you going to upset
the entire world so that it's finally \emph{fair} and \emph{kind}
to everyone?''

Yasmin shrank back, but she took a deep breath and looked into the
General's terrible eyes. ``What's so wrong with kindness, Mala?
What's so terrible about surviving without harming other people?''

Mala's lip curled up in a snarl of pure disgust. ``Don't you know by
now, Yasmin? Haven't you figured it out yet? Look around us!'' She
waved her water can wildly, nearly clubbing an old woman who was
inching past, bearing her own water cans. ``Look around! You know
that there are people all over the world who have fine cars and
fine meals, servants and maids? There are people all over the world
who have \emph{toilets}, Yasmin, and \emph{running water}, and who
get to each have their own bedroom with a fine bed to sleep in! Do
you think those people are going to give up their fine beds and
their fine houses and cars for \emph{you}? And if they don't give
it up, where will it come from? How many beds and cars are there?
Are there enough for all of us? In this world, Yasmin, there just
isn't enough. That means that there are going to be losers and
winners, just like in any game, and you get to decide if you want
to be a winner or a loser.''

Yasmin mumbled something under her breath.

``What?'' Mala shouted at her. ``What are you saying, girl? Speak up
so I can hear you!''

``I don't think it's like that. I think we can be kind to other
people and that they will be kind to us. I think that we can stick
together, like a team, like the army, and we can all work together
to make the world a better place.''

Mala laughed, but it sounded forced, and Yasmin thought she saw
tears starting in her friend's eyes. ``You know what happens when
you act like that, Yasmin? They find a way to destroy you. To force
you to become an animal. Because \emph{they're} animals. They want
to win, and if you offer them your hand, they'll slice off your
fingers. You have to be an animal to survive.''

Yasmin shook her head, negating everything. ``It's not true, Mala!
Our neighbors here, they're not animals. They're people. They're
good people. We have nothing and yet we all cooperate. We help each
other --''

``Oh fine, maybe you can make a little group of friends here, people
who would have to look you in the eye if they did you a dirty
trick. But it's a big world. Do you think that Big Sister Nor's
friends in Singapore, in China, in America, in Russia -- do you
think \emph{they'll} think twice before they destroy you? In
Africa, in --'' She waved her arm, taking in all the countries she
didn't know the names of, filled with teeming masses of predatory
workers, ready to take their jobs from them. ``Listen: do you really
care so much for Chinese and Russians and all those other people?
Will you take bread out of your mouth to give it to them? For a
bunch of \emph{foreigners} who wouldn't spit on you if you were on
fire?''

Yasmin thought she knew her friend, but this was like nothing she'd
ever heard from Mala before. Where had all this Indian patriotism
come from? ``Mala, it's foreigners who own all the games we're
playing. Who cares if they're foreigners? Isn't the fact that
they're people enough? Didn't you used to rage about the stupid
caste system and say that everyone deserved equality?''

``Deserved!'' Mala spat the word out like a curse. ``Who cares what
you deserve, if you don't get it. Fill your belly with deserve.
Sleep on a bed of deserve. See what you get from deserve!''

``So your army is about taking whatever they can get, even if it
hurts someone else?''

Mala stood up very straight. ``That's right, it's \emph{my} army,
Yasmin. My army! And you're not a part of it anymore. Don't bother
coming around again, because, because --''

``Because I'm not your friend or your lieutenant anymore,'' Yasmin
said. ``I understand, General Mala Robotwallah. But your army won't
last forever and our sisterhood might have, if you'd only valued it
more. I'm sorry you are making this decision, General Robotwallah,
but it's yours to make. Your karma.'' She set down the water-can and
turned on her heel and started away, back stiff, waiting for Mala
to jump on her back and wrestle her into the mud, waiting for her
to run up and hug her and beg her for forgiveness. She got to the
next corner, a narrow laneway between more plastic recycling
factories, and contrived to look back over her shoulder as she
turned, pretending to be dodging to avoid a pair of goats being led
by an old Tamil man.

Mala was standing tall as a soldier, eyes burning into her, and
they transfixed her for a moment, froze her in her tracks, so that
she really \emph{did} have to dodge around the goats. When she
looked back again, the General had departed, her skinny arms
straining with her water-cans.

Big Sister Nor told her to be understanding.

``She's still your friend,'' the woman said, her voice emanating from
the gigantic robot that stood guard over a group of Webbly
gold-farmers who were methodically raiding an old armory, clearing
out the zombies and picking up the cash and weapon-drops that
appeared every time they ran the dungeon. ``She may not know it, but
she's on the side of workers. The other side -- the boss's side --
they'll use her services, but they'll never let her into their
camp. The best she can hope for is to be a cherished pet, a
valuable bit of hired muscle. I don't think she'll stay put for
that, do you?''

But it wasn't much comfort. In one morning, Yasmin had lost her
best friend and her occupation. She started going to school again,
but she'd fallen behind in the work in the six months she'd been
away and now the master wanted her to stay back a year and sit with
the grade four students, which was embarrassing. She'd always been
a good student and it galled her to sit with the younger kids --
and to make things worse, she was tall for her age and towered over
them. Gradually, she stopped attending the school.

Her parents were outraged, of course. But they'd been outraged when
Yasmin had joined the army, too, and her father had beaten her for
ten days running, while she refused to cry, refused to have her
will broken. In the end, they'd been won over by her stubbornness.
And, of course, by the money she brought home.

Yasmin could handle her parents.

Mrs Dibyendu's Internet Cafe was a sad place now that the Army had
moved on. Mala had forced that on Mr Banerjee, and had counted it
as a great show of her strength when she prevailed. But Yasmin
thought she never would have won the argument if Mrs Dibyendu
hadn't been so eager to get rid of the Army.

Yasmin doubted that Mrs Dibyendu had anticipated the effect that
the Army's departure would have on her little shop, though. Once
the Army had gone, every kid in Dharavi had moved with them -- no
one under the age of 30 would set foot in the cafe. No one except
Yasmin, who now sat there all day long, fighting for the workers.

``You are very good at this,'' Justbob told her. She was Big Sister
Nor's lieutenant, and her Hindi was terrible, so they got by in a
broken English that each could barely understand. Nevertheless,
Justbob's play was aggressive and just this side of reckless,
utterly fearless, and she screamed out fearsome battle-cries in
Tamil and Chinese when she played, which made Yasmin laugh even as
the hairs on her arms stood up. Justbob liked to put Yasmin in
charge of strategy while she led the armies of defenders from
around the world who played on their side, defending workers from
people like Mala.

``Thank you,'' Yasmin said, and dispatched a squadron to feint at the
left flank of a twenty-cruiser unit of rusting battle-cars that
bristled with bolted-on machine-guns and grenade launchers. She
mostly played Mad Max: Autoduel and Civilization these days,
avoiding Zombie Mecha and the other games that Mala and her Army
ruled in. Autoduel was huge now, linked to a reality TV show in
which crazy white people fought each other in the deserts in
Australia with killer cars just like the ones in the game.

The opposing army bought the feint, turning in a wide arc to
present their forward guns to her zippy little motorcycle scouts
who must have looked like easy pickings -- the fast dirt-bikes
couldn't support any real arms or armor, so each driver was limited
to hand-weapons, mostly Uzis on full auto, spraying steel-jacketed
rounds toward the heavily armored snouts of the enemy, who returned
withering fire with tripod-mounted machine-guns and grenades.

But as they turned, they rolled into a double-row of mines Yasmin
had laid by stealth at the start of the battle, and then, as the
cars rocked and slammed into each other and spun out of control,
Justbob's dragoons swept in from the left, and their splendid
battle-wagon came in from the right -- a lumbering two-storey RV
plated with triple-thick armor, pierced with gun-slits for a
battery of flame-throwers and automatic ballistic weapons, mostly
firing depleted uranium rounds that cut through the enemy cars like
butter. It wasn't hard to outrun the battle-wagon, but there was
nowhere for the enemy to go, and a few minutes later, all that was
left of the enemy were oily petrol fires and horribly mutilated
bodies.

Yasmin zoomed out and booted her command-trike around a dune to
where the work-party continued to labor, doing their job,
excavating a buried city full of feral mutants and harvesting its
rich ammo-dumps and art-treasures for the tenth time that day.
Yasmin couldn't really talk to them -- they were from somewhere in
China called Fujian, and besides, they were busy. They'd left their
boss and formed a worker's co-op that split the earnings evenly,
but they'd had to go heavily into debt to buy the computers to do
it, and from what Yasmin understood, their families could be hurt
or even killed if they missed a payment, since they'd had to borrow
the money from gangsters.

It would have been nice if they'd had access to a better source of
money, but it certainly wouldn't be Yasmin. Her Army money had run
out a few weeks after she'd left Mala, and though the IWWWW paid
her a little money to guard union shops, it didn't come to much,
especially compared to the money Mr Banerjee had to throw around.

At least she wasn't hurting other poor people to survive. The goons
she'd just wiped out would get paid even though they'd lost. And
she had to admit it: this was \emph{fun}. There was a real thrill
in playing the game, playing it well, getting this army of people
to follow her lead to cooperate and become an unstoppable weapon.

Then, Justbob was gone. Not even a hastily typed ``gtg,'' she just
wasn't on the end of her mic. And there were crashing sounds,
shouts in a language Yasmin didn't speak. Distant screaming.

Yasmin flipped over to Minerva, the social networking site that the
Webblies favored, as she did a thousand times a day. Minerva had
been developed for gamers, and it had all kinds of nice dashboards
that showed you what worlds all your friends were in, what kind of
battles they were fighting and so on. It was easy to get lost in
Minerva, falling into a clicktrance of screencaps of famous
battles, trash-talking between guilds, furious arguments about the
best way to run a level -- and the endless rounds of gold-farmer
bashing. One thing she loved about Minerva was the auto-translate
feature, whose database included all kinds of international gamer
shorthands and slangs, knowing that Kekekekeke was Korean for LOL
and a million other bits of vital dialects. This made Minerva
especially useful for the Webblies' global network of guilds,
worker co-ops, locals and clans.

Her dashboard was going \emph{crazy}. Webblies from all over the
world were tweeting about something happening in China, a big
strike from a group of gold-farmers who'd walked out on their boss,
and were now picketing outside of their factories. Players from all
over the world were rushing to a site in Mushroom Kingdom to
blockade some sploit that they'd been mining before they walked
out. Yasmin hadn't ever played Mushroom Kingdom and she wouldn't be
any use there -- you had to know a lot about a world's weapons and
physics and player-types before you could do any damage. But
judging from the status ticker zipping past, there were plenty of
Webblies available on every shard to fill the gap.

She followed the messages as they went by, watched the rallies and
the retreats, the victories and defeats, and waited on tenterhooks
for the battle to end when the GMs discovered what they were up to
and banned everyones' accounts. That was the secret weapon in all
these battles: anyone who snitched to the employees of the
companies that ran the worlds could destroy both teams, wiping out
their accounts and loot in an instant. No one could afford that --
and no one could afford to fight in battles that were so massive
that they caught the eye of the GMs, either.

And yet, here were the Webblies, hundreds of them, all risking
their accounts and their livelihoods to beat back goons who were
trying to break a strike. Yasmin's blood sang -- this was it, this
was what Big Sister Nor was always talking about: Solidarity! An
injury to one is an injury to all! We're all on the same team --
and we stay together.

There were videos and pictures streaming from the strike, too --
skinny Chinese boys blinking owlishly in the daylight, on busy
streets in a distant land, standing with arms linked in front of
glass doorways, chanting slogans in Chinese. Passers-by goggled at
them, or pointed, or laughed. Mostly they were girls, older than
Yasmin, in their late teens and early twenties, very well-dressed,
with fashionable haircuts and short skirts and ironed blouses and
shining hair. They stared and some of them talked with the boys,
who basked in the attention. Yasmin knew about boys and girls and
the way they made each other act -- hadn't she seen and used that
knowledge when she was Mala's lieutenant?

And now more and more of the girls were joining the boys -- not
exactly joining, but crowding around them, standing in clumps,
talking amongst themselves. And there were police coming in too,
lots of pictures of the police filling in and Yasmin's heart sank.
She could see, with her strategist's eye, how the police positions
would work in planning a rush at the strikers, shutting off their
escape routes, boxing them in and trapping them when the police
swept in.

Now the photos slowed, now the videos stopped. Gloved hands reached
up and snatched away cameras, covering lenses. The last audiofeed
was shouts, angry, scared, hurt --

And now the ticker at the bottom of her screen was going even
crazier, messages from the pickets in China about the police rush,
and there was a moment of unreality as Yasmin felt that she was
reading about an in-game battle again, set in some gameworld
modelled on industrial China, a place that seemed as foreign to her
as Zombie Mecha or Mad Max. But these were real people, skirmishing
with real police, being clubbed with real truncheons. Yasmin's
imagination supplied images of people screaming, writhing,
trampling each other with all the vividness of one of her games. It
was a familiar scene, but instead of zombies, it was young, pale
Chinese boys and beautiful, fashionable Chinese girls caught in the
crush, falling beneath the truncheons.

And then the messages died away, as everyone on the scene fell
silent. The ticker still crawled with other Webblies around the
world, someone saying that the Chinese police could shut down all
the mobile devices in a city or a local area if they wanted. So
maybe the people were still there, still recording and writing it
down. Maybe they hadn't all been arrested and taken away.

Yasmin buried her face in her hands and breathed heavily. Mrs
Dibyendu shouted something at her, maybe concerned. It was
impossible to tell over the song of the blood in her ears and the
hammer of the blood in her chest.

Out there, Webblies all over the world were fighting for a better
deal for poor people, and what did it matter? How could her
solidarity help those people in China? How could they help
\emph{her} when she needed it? Where were Big Sister Nor and
Justbob and The Mighty Krang now that she needed them?

She stumbled out into the light, blinking, thinking of those skinny
Chinese boys and the police in their strategic positions around
them. Suddenly, the familiar alleys and lanes of Dharavi felt
sinister and claustrophobic, as though people were watching her
from every angle, getting ready to attack her. And after all, she
was just a girl, a little girl, and not a mighty warrior or a
general.

Her treacherous feet had led her down the road, around a corner,
behind the yard where the women's baking co-op set out their
papadams in the sun, and past the new cafe where Mala and her army
fought. They were in there now, the sound of their boisterous play
floating out on the air like smoke, like the mouthwatering
temptation smells of cooking food.

What were they shouting about? Some battle they'd fought -- a
battle in Mushroom Kingdom. A battle against the Webblies. Of
course. They were the best. Who else would you hire to fight the
armies of the Webblies? She felt a sick lurch in her gut, a feeling
of the earth dropping away from beneath her feet. She was alone
now, truly alone, the enemy of her former friends. There was no one
on her side except for some distant people in a distant land whom
she'd never met -- whom she'd probably never meet.

Dispirited, she turned away and headed for home. Her father was
away for a few days, travelling to Pune to install a floor for
work. He worked in an adhesive tile plant where they printed out
fake stone designs on adhesive-backed squares of durable vinyl that
could be easily laid in the office towers of Pune's industrial
parks. There were always tiles around their home, and Yasmin had
never paid them much attention until she started to game with Mala,
and then she'd noticed with a shock one day that the strange,
angular blurring around the edges of the fine ``marble'' veins in the
tiles were the same compression smears you got when the game's
graphics started to choke, ``JPEG artifacts,'' they called them in
the message boards. It was as though the little imperfections that
make the games slightly unreal were creeping into the real world.

That feeling was with her now as she ghosted away from the cafe,
but she was brought back to reality by a tap on her shoulder. She
whirled around, startled, feeling, for some reason, like she was
about to be punched.

But it was Sushant, the tallest boy in Mala's army, who had never
blustered and fought like the other boys, but had stared intently
at his screen as though he wished he could escape into it. Yasmin
found herself staring straight down his eyes, and he waggled his
chin apologetically and smiled shyly at her.

``I thought I saw you passing by,'' he said. ``And I thought --'' He
dropped his eyes.

``You thought what?'' she said. It came out harshly, an anger she
hadn't known she'd been feeling.

``I thought I'd come out and\ldots{}'' He trailed off.

``What? What did you think, Sushant?'' Her own chin was wagging from
side to side now, and she leaned her face down toward his, noses
just barely apart. She could smell his lunch of spinach bahji on
his breath.

He shrank back, winced. Yasmin realized that he was terrified.
Realized that he had probably risked quite a lot just by coming out
to talk to her. Discipline was everything in Mala's army. Hadn't
Yasmin been in charge of enforcing discipline?

``I'm sorry,'' she said, backing away. ``It's nice to see you again,
Sushant. Have you eaten?'' It was a formality, because she knew he
had, but it was what one friend said to another in Dharavi, in
Mumbai -- maybe in all of India, for all Yasmin knew.

He smiled again, a faltering little shy smile. It was heartbreaking
to see. Yasmin realized that she'd never said much to him when she
was Mala's lieutenant. He'd never needed cajoling or harsh words to
get down to work, so she'd practically ignored him. ``I thought I'd
come out and say hello because we've all missed you. I hoped that
maybe you and Mala could --'' Again he faltered, and Yasmin felt her
own chin jutting out involuntarily in a stubborn, angry way.

``Mala and I have chosen different roads,'' she said, making a
conscious effort to sound calm. ``That's final. Does it go well for
her and you?''

He nodded. ``We win every battle.''

``Congratulations.''

``But now -- lately -- I've been thinking --''

She waited for him to say more. The moment stretched. Grownups
bumped past them and she realized that they probably thought they
were courting, being a boy and a girl together. If news of that got
back to her father --

But it didn't matter to her anymore. Her father was off installing
JPEG artifacts in an IT park in Pune. She was out of the army and
out of friends and out of school. What could anything matter.

``I talk to your friends,'' he said at last.

``My friends?'' She didn't know she had any.

``The Webblies. Your new army. They come to me while I fight, send
me private messages. At first I ignored them, but lately I've been
on drogue, and I have a lot of time to think. And they sent me
pictures -- the people I was hurting. Kids like you and me, all
over the world. And it made me think.'' He paused, licked his lips.
``About karma. About hurting people to live. About all the things
that they say. I don't think I want to do this forever. Or that I
can do it forever.''

Yasmin was at a loss for words. Were there really other people,
right here in Dharavi, right here in Mala's army, who felt as she
did? She'd never imagined such a thing, somehow. But here he was.

``You know that Mala's army pays ten times what you can get with the
Webblies, right?''

``For now,'' he said. ``That's the point, right? Chee! If we fight
now, we can raise the wages of everyone who works for a living
instead of owning things for a living, right?''

``I never thought of the division that way. Owning things for a
living, I mean.''

His shyness receded. He was clearly enjoying having someone to talk
to about this. ``It all comes down to owning versus working. Someone
has to do the organizing, I guess -- there wouldn't be a Zombie
Mecha if someone didn't get a lot of people together, working to
make all that code. Someone has to pay the game-masters and do all
of that. I understand that part. It makes sense to me. My mother
works in Mrs Dotta's fabric-dyeing shop. Someone has to buy the
dyes, get the cloth, buy the vats and the tools, arrange to sell it
once it's done, otherwise, my mother wouldn't have a job. I always
stopped there, thinking, all right, if Mrs Dotta does all that
work, and makes a job for my mother, why shouldn't she get paid for
it?

``But now I think that there's no reason that Mrs Dotta's job is
more important than my mother's job. Mamaji wouldn't have a job
without Mrs Dotta's factory, but Mrs Dotta wouldn't have a factory
without mamaji's work, right?'' He waggled his chin defiantly.

``That's right,'' Yasmin said. She was nervous about being in public
with this boy, but she had to admit that it was exciting to hear
this all from him.

``So why should Mrs Dotta have the right to fire my mother, but my
mother not have the right to fire Mrs Dotta? If they depend on each
other, why should one of them always have the power to demand and
the other one always have to ask for favors?''

Yasmin felt his excitement, but she knew that there had to be more
to it than this. ``Isn't Mrs Dotta taking all the risk? Doesn't she
have to find the money to start the factory, and doesn't she lose
it if the factory closes?''

``Doesn't mamaji risk losing her job? Doesn't Mamaji risk growing
sick from the fumes and the chemicals in the dyes? There's nothing
eternal or perfect or natural about it! It's just something we all
agreed to -- bosses get to be in charge, instead of just being
another kind of worker who contributes a different kind of work!''

``And that's what you think you'll get from the Webblies? An end to
bosses?''

He looked down, blushing. ``No,'' he said. ``No, I don't think so. I
think that it's too much to ask for. But maybe the workers can get
a better deal. That's what Big Sister Nor talks about, isn't it?
Good pay, good places to work, fairness? Not being fired just
because you disagree with the boss?''

\emph{Or the general}, Yasmin thought. Aloud, she said, ``So you'll
leave the army? You want to be a Webbly?''

Now he looked down further. ``Yes,'' he said, at last. ``Eventually.
It all keeps going around and around in my mind. I don't know if
I'm ready yet.'' He risked a look up at her. ``I don't know if I'm as
brave as you.''

Anger surged through her, hot and irrational. How \emph{dare} he
talk about her ``bravery''? He was just using that as an excuse to go
on getting rich in Mala's army. He understood \emph{so well} what
was wrong and what needed to be done. Understood it better than
Yasmin! But he didn't want to give up his comfort and friendships.
That wasn't cowardice, it was \emph{greed}. He was too greedy to
give it up.

He must have seen this in her face, because he took a step back and
held up his hands. ``It's not that I won't do it someday -- but I
don't know what good it would do for me to do this today, on my
own. What would change if I stopped fighting for Mala's army? She's
just one general with one army among hundreds all over the world,
and I'm just one fighter in the army. I --'' He faltered. ``What's
the sense in giving up so much if it won't make a difference?''

Yasmin's anger boiled in her, ate at her like acid, but she bit her
tongue, because that little voice inside her was saying, ``You're
mostly angry because you thought you had a comrade, someone who'd
keep you company, and it turned out that all he wanted to do was
confess to you and have you forgive him. And it was true. She was
far more upset by her loneliness than by his cowardice, or greed,
or whatever it was.

``I. Need. To. Go. Now,'' she said, biting on the words, keeping the
anger out of her voice by sheer force of will.

She didn't wait for him to raise his eyes, just turned on her heel
and walked and walked and walked, through the familiar alleys of
Dharavi, not going anywhere but trying to escape anyway, like a
chained animal pacing off its patch. She was chained -- chained by
birth and by circumstance. Her family might have been rich. They
might have been high-caste. She might be in another country -- in
America, in China, in Singapore, all the distant lands. But she was
here, and she had no control over that. There was a whole world out
there and this was where fate had put her.

She wouldn't be changing the world. She wouldn't be going to any of
those places. She hadn't even left Dharavi, except once with her
mother, when she took Yasmin and her brothers on a train to see a
beach where it had been hot and sandy and the water had been too
dangerous to swim in, so they'd stood on the shore and then walked
down a road of smart shops where they couldn't afford to shop, and
then they'd waited for the bus again and gone home. Yasmin had seen
the multiverses of the games, but she hadn't even seen Mumbai.

Now where? She was tired and hungry, angry and exhausted. Home? It
was still afternoon, so her mother and brothers were all out
working or in school. That emptiness\ldots{} It scared her. She wasn't
used to being alone. It wasn't a natural state in Dharavi. She was
very thirsty, the wind was blowing plastic smoke into her eyes and
face, making her nostrils and sinuses and throat raw. Mrs
Dibyendu's cafe would have chai, and Mrs Dibyendu would give her a
cup of it and some computer time on credit, because Mrs Dibyendu
was desperate to save her cafe from bankruptcy now that the army
had abandoned it.

Mrs Dibyendu's idiot nephew doled her out a cup of chai grudgingly.
He hadn't learned a thing from the savage beating that Mala had
laid on him. He still stood too close, still went out Eve teasing
with his gang of badmashes. Yasmin knew that he would have loved to
take revenge on Mala, and that Mala never went out after dark
without three or four of the biggest boys from the army. It made
her furious. No matter how much Mala had hurt her, she had the
right to go around her home without fearing this idiot. His upper
lip was curled in a permanent sneer, thanks to the scar Mala's feet
had left behind.

She sat down to a computer, logged in. She was sure that the idiot
nephew used all kinds of badware to spy on what they did on the
computers, but she'd bought a login fob from one of the shops at
the edge of Dharavi, and it did magic, logging her in with a
different password every time she sat down, so that her PayPal and
game accounts were all safe.

Mindlessly, she plunged back into her usual routine. Login to
Minerva, check for Webbly protection missions in the worlds she
played. But there were no missions waiting. The Webbly feeds were
all afire with chatter about the strike in Shenzhen, rumors of the
numbers arrested, rumors of shootings. She watched it tick past
helplessly, wondering where all these rumors came from. Everyone
seemed to know something that she didn't know. How did they know?

A direct message popped up on her screen. It was from a stranger,
but it was someone in the inner Webbly affinity group, which meant
that Big Sister Nor, The Mighty Krang, or Justbob had manually
approved her. Anyone could join the outer Webblies, but there were
very few inner Webblies.

\edialog{Hello, can you read this?}

It was a full sentence, with punctuation, and the question was as
daft as you could imagine. It was the kind of message her father
might send. She knew immediately that she was communicating with an
adult, and one who didn't game.

\edialog{yes}

\edialog{Our mutual friend B.S.N. has asked me to contact
you. You are in Mumbai, correct?}

She had a moment's hesitation. This was a very grownup, very
non-gamer way to type. Maybe this was someone working for the other
side? But Mumbai was as huge as the world. ``In Mumbai'' was only
slightly more specific than ``In India'' or ``On Earth.''

\edialog{yes}

\edialog{Where are you? Can I come and get you? I must talk
with you.}

\edialog{talking now lol}

\edialog{What? Oh, I see. No, I must TALK with you. This is
official business. B.S.N. specifically said I must make contact
with you.}

She swallowed a couple times, drained the dregs of her chai.

\edialog{ok}

\edialog{Splendid. Where shall I come and get you from?}

She swallowed again. When they'd gone to the beach, her mother had
been very clear on this:
\emph{Don't tell anyone you are from Dharavi. For Mumbaikars, Dharavi is like Hell, the place of eternal torment, and those who dwell here are monsters.}
This grown up sounded very proper indeed. Perhaps he would think
that Dharavi was Hell and would leave her be.

\edialog{dharavi girl}

\edialog{One moment.}

There was a long pause. She wondered if he was trying to get in
touch with Big Sister Nor, to tell her that her warrior was a
slum-child, to find someone better to help.

\edialog{You know this place?}

It was a picture of the Dharavi Mosque, tall and imposing, looming
over the whole Muslim quarter.

\edialog{course!!}

\edialog{I'll be there in about an hour. This is me.}

Another picture. It wasn't the middle-aged man in a suit she'd been
expecting, but a young man, barely older than a teenager, short
gelled hair and a leather jacket, stylish blue-jeans and black
motorcycle boots.

\edialog{Can you give me your phone number? I will call you
when I'm close.}

\edialog{lol}

\edialog{I'm sorry?}

\edialog{dharavi girl -- no phone for me}

She'd had a phone, when she was in Mala's army. They all had
phones. But it was the first thing to go when she quit the army.
She still had it in a drawer, couldn't bear to sell it, but it
didn't work as a phone anymore, though she sometimes used it as a
calculator (all the games had turned themselves off right after the
service was disconnected, to her disappointment).

\edialog{Sorry, sorry. Of course. Meet you there in about an
hour then.}

Her heart thudded in her chest. Meeting a strange man, going on a
secret errand -- it was the sort of thing that always ended in
terrible tragedy, defilement and murder, in the stories. And an
hour from now would be --

\edialog{cant meet at the mosque}

It would be right in the middle of 'Asr, afternoon prayers, and the
Mosque would be mobbed by her father's friends. All it would take
would be for one of them to see her with a strange man, with gelled
hair, a Hindu judging from the rakhi on his wrist, poking free of
the leather jacket. Her father would go \emph{insane}.

\edialog{meet me at mahim junction station instead by the
crash barriers}

It would take her an hour to walk there, but it would be safe.

There was a pause. Then another picture: two boys straddling one of
the huge cement barriers in front of the station. It was where she
and her brothers had waited while their mother queued up for the
tickets.

\edialog{Here?}

\edialog{yes}

\edialog{OK then. I'll be on a Tata 620 scooter.}

Another picture of a lovingly polished little bike, a proud purple
gas-tank on its skeletal chromed frame. There were thousands of
these in Dharavi, driven by would-be badmashes who'd saved up a
little money for a pair of wheels.

\edialog{ill be there}

She handed her cup to idiot nephew, not even seeing the grimace on
his face as she dashed past him, out into the roadway, back home to
change and put some few things in a bag before her mother or
brothers came home. She didn't know where she was going or how long
she'd be away, and the last thing she wanted was to have to explain
this to her mother. She would leave a note, one of her brothers
would read it to her mother. She'd just say, ``Away on union
business. Back soon. Love you.'' And that would have to be enough --
because, after all, it was all she knew.

On the long walk to Mahim Junction station, she alternated between
nervous excitement and nervous dread. This was foolish, to be sure,
but it was also all she had left. If Big Sister Nor vouched for
this man -- chee! she didn't even know his name! -- then who was
Yasmin to doubt him?

As she got closer to the edge of Dharavi, the laneways widened to
streets, wide enough for skinny, shoeless boys to play
ditch-cricket in. They shouted things at her, ``offending decency,''
as the schoolteacher, Mr Hossain, had always said when the
badmashes gathered outside the school to call things to the girls
as they left the classroom. But she knew how to ignore them, and
besides, she had picked up her brother Abdur's lathi, using it as a
walking stick, having tied a spare hijab underscarf to the top to
make it seem more innocuous. They'd played gymnastics games in the
schoolyard with sticks like lathis, but without the iron binding on
the tip. Still, she felt sure she could swing it fearsomely enough
to scare off any badmash who got in her way on this fateful day. It
was only at the station that she realized she had no idea how they
would carry it on the little scooter.

She'd brought her phone along, just to tell the time with, and now
an hour had gone by and there was no sign of the man with the short
gelled hair. Another twenty minutes ticked past. She was used to
this: nothing in Dharavi ran on precise time except for the calls
to prayer from the mosque, the rooster crows in the morning, and
the calls to muster in Mala's army, which were always precisely
timed, with fierce discipline for stragglers who showed up late for
battle.

Trains came in and trains came out. She saw some men she
recognized: friends of her father who worked in Mumbai proper, who
would have recognized her if she hadn't been wearing her hijab
pulled up to her nose and pinned there. She was acutely aware of
the Hindu boys' stare. Hindus and Muslims didn't get along,
officially. Unofficially, of course, she knew as many Hindus as
Muslims in Dharavi, in the army, in school. But on the impersonal,
grand scale, she was always \emph{other}. They were ``Mumbaikars'' --
``real'' people from Mumbai. Her parents insisted on calling the city
``Bombay,'' the old name of the city from before the fierce Hindu
nationalists had changed it, proclaiming that India was for Hindus
and Hindus alone. She and her people could go back to Bangladesh,
to Pakistan, to one of the Muslim strongholds where they were in
the majority, and leave India to the real Indians.

Mostly, it didn't touch her, because mostly, she only met people
who knew her and whom she knew -- or people who were entirely
virtual and who cared more about whether she was an Orc or a Fire
Elf than if she was a Muslim. But here, on the edge of the known
world, she was a girl in a hijab, an eye-slit and a long, modest
dress and a stout stick, and they were all \emph{staring} at her.

She kept herself amused by thinking about how she would attack or
defend the station using a variety of games' weapons-systems. If
they were all zombies, she'd array the mechas here, here and here,
using the railway bed as a channel to lure combatants into
flamethrower range. If they were fighting on motorcycles, she'd
circle that way with her cars, this way with her motorcycles, and
pull the death-lorry in there. It brought a smile to her face,
safely hidden behind the hijab.

And here was the man, pulling into the lot on his green motorcycle,
wiping the road dust off his glasses with his shirt-tail before
tucking it back into his jacket. He looked around nervously at the
people outside the station -- working people streaming back and
forth, badmashes and beggars loitering and sauntering and getting
in everyone's way. Several beggars were headed toward him now,
children with their hands outstretched, some of them carrying
smaller children on their hips. Even over the crowd noises, Yasmin
could hear their sad, practiced cries.

She reached under her chin and checked the pin holding her hijab in
place, then approached the rider, moving through the beggars as
though they weren't there. They shied away from her lathi like
flies dodging a raised hand. He was so disconcerted by the beggars
that it took him a minute to notice the veiled young girl standing
in front of him, clutching a meter-and-a-half long stick bound in
iron.

``Yasmin?'' His Hindi was like a fillum star's. Up close, he was very
handsome, with straight teeth and a neatly trimmed little mustache
and a strong nose and chin.

She nodded.

He looked at her lathi. ``I have some bungee cables,'' he said. ``I
think we can attach that to the side of the bike. And I brought you
a helmet.''

She nodded again. She didn't know what to say. He moved to the
locked carrier-box on the back of his bike, pushing away a little
beggar-boy who'd been fingering the lock, and pushed his thumb into
the locking mechanism's print-reader. It sprang open and he fished
inside, coming up with a helmet that looked like something out of a
manga cartoon, streamlined, with intricate designs etched into its
surface in hot yellow and pink. On the front of the helmet was a
sticker depicting Sai Baba, the saint that both Muslims and Hindus
agreed upon. Yasmin thought this was a good omen -- even if he was
a Hindu boy, he'd brought her a helmet that she could wear without
defiling Islam.

She took the manga Sai Baba helmet from him, noting that the
sticker was holographic and that Sai Baba turned to look her
straight in the eye as she hefted it. It was heavier than it
looked, with thick padding inside. No one in Dharavi wore
crash-helmets on motorcycles -- and the boy wasn't wearing one,
either. But as she contemplated the narrow saddle, she thought
about falling off at 70 kilometers per hour on some Mumbai road and
decided that she was glad he'd brought it. So she nodded a third
time and lifted it over her head. It went on slowly, her head
pushing its way in like a hand caught in a tangled sleeve, pushing
to displace the fabric, which slowly gave way. Then she was inside
it, and the sounds around her were dead and distant, the sights all
tinted yellow through the one-way mirrored eye-visor. She felt
tentatively at her head -- which felt like it would loll forward
under the helmet's weight if she turned her face too quickly -- and
found the visor's catch and lifted it up. The sound got a little
brighter and sharper.

Meanwhile, the boy had been affixing the lathi along the bike's
length, to the amusement of the beggar children, who offered
laughing advice and mockery. He had a handful of bungee cords that
he'd extracted from the bike's box, and he wrapped them again and
again around the pole, finding places on the bike's skeletal chrome
to fix the hooks, testing the handlebars to ensure that he could
still steer. At last he grunted, stood, dusted his hands off on his
jeans and turned to her.

``Ready?''

She drew in a deep breath, spoke at last. ``Where are we going?''

``Andheri,'' he said. ``Near the film studios.''

She nodded as though she knew where that was. In a way, of course,
she did: there were plenty of movies about, well, the golden age of
making movies, when Andheri had been \emph{the} place to be,
glamorous and bustling. But most of those movies had been about how
Andheri's sun had set, with all the big filmi production places
moving away. What would it be like today?

``And when will we come back?''

He waggled his chin, thinking. ``Tonight, certainly. I'll make sure
of that. And some union people can come back with us and make sure
you get to your door safely. I've thought of everything.''

``And what is your name?''

He stared at her for a moment, his jaw hanging open in surprise.
``OK, I didn't think of everything! I'm Ashok. Do you know how to
ride a scooter?''

She shook her head. She'd seen plenty of people riding on
motorcycles and scooters, in twos and even in threes and fours --
sometimes a whole family, with children on mothers' laps on the
back -- but she'd never gotten on one. Standing next to it now, it
seemed insubstantial and well, \emph{slippery}, the kind of thing
that was easier to fall off of than to stay on.

``OK,'' he said, waggling his chin, considering her clothing. ``It's
harder with the dress,'' he said. ``You'll have to sit side-saddle.''
He climbed up on the bike's saddle and demonstrated, keeping his
knees together and pressed against the bike's side, twisting his
body around. ``You'll have to hold onto me very tight.'' He grinned
his movie-star grin.

Yasmin realized what a mistake this had all been. This strange man.
His motorcycle. Going off to Mumbai, away from Dharavi, to a
strange place, for a strange reason. And he had her lathi, which
wasn't even hers, and if she turned on her heel and went back into
Dharavi, she'd still have to explain the missing lathi to her
brother, and the note to her mother. And now she was going to get
killed in Mumbai traffic with a total stranger on the way to
Bollywood's favorite ghost-town.

But as hopeless as it was, it wasn't as hopeless as being alone,
not in the army, not in school, not in the Webblies. Not as
hopeless as being poor Yasmin, the Dharavi girl, born in Dharavi,
bred in Dharavi.

She levered herself sidesaddle onto the bike and Ashok climbed over
the saddle and sat down, his leather jacket pressed up against her
side. She tried to square her hips to face forward, and found
herself in such a precarious position that she nearly tipped over
backwards.

``You have to hold on,'' Ashok said, and the beggar children jeered
and made rude gestures. Shutting her eyes, she put her arms around
his waist, feeling how skinny he was under that fancy jacket, and
interlaced her fingers around his stomach. It was less precarious
now, but she still felt as though she would fall at any second --
and they weren't even moving yet!

Ashok kicked back the bike's stand and revved the engine. A cloud
of biodiesel exhaust escaped from the tailpipe, smelling like old
cooking oil -- it probably started out as old cooking oil, of
course -- spicy and stale. Yasmin's stomach gurgled and she blushed
beneath her hijab, sure he could feel the churning of her empty
stomach. But he just turned his head and said, ``Ready?''

``Yes,'' she said, but her voice came out in a squeak.

They barely made it fifty meters before she shouted ``Stop! Stop!''
in his ear. She had never been more afraid in all her life. She
forced her fingers to unlace themselves and drew her trembling
hands back into her lap.

``What's wrong?''

``I don't want to die!'' she shouted. ``I don't want to die on your
maniac bike in this maniac traffic!''

He waggled his chin. ``It's the dress,'' he said. ``If you could only
straddle the seat.''

Yasmin patted her thighs miserably, then she hiked up her dress,
revealing the salwar -- loose trousers -- she wore beneath it.
Ashok nodded. ``That'll do,'' he said. ``But you need to tie up the
legs, so they don't get caught in the wheel. He flipped open his
cargo box again and passed her two plastic zip-strips which she
used to tie up each ankle.

``Right, off we go,'' he said, and she straddled the bike, putting
her arms around his waist again. He smelled of his hair gel and of
leather, and of sweat from the road. She felt like she'd gone to
another planet now, even though she could still see Mahim Junction
behind her. She squeezed his waist for dear life as he revved the
engine and maneuvered the bike back into traffic.

She realized that he'd been taking it easy for her sake before,
driving relatively slowly and evenly in deference to her precarious
position. Now that she was more secure, he drove like the baddest
badmash she'd ever seen in any action film. He gunned the little
bike up the edge of the ditch, beside the jerky, slow traffic,
always on the brink of tipping into the stinking ditch, being
killed by a swerving driver or a door opening suddenly so the
driver could spit out a stream of betel; or running over one of the
beggars who lined the road's edge, tapping on the windows and
making sad faces at the trapped motorists.

She'd piloted a million virtual vehicles in her career as a gamer,
at high speeds, through dangerous terrain. It wasn't remotely the
same, even with the helmet's reality-filtering padding and visor.
She could hear her own whimpering in her head. Every nerve in her
body was screaming \emph{Get off this thing while you can}! But her
rational mind kept on insisting that this boy clearly rode his bike
through Mumbai every day and managed to survive.

And besides, there was so much Mumbai to see as they sped down the
road, and that was much more interesting than worrying about
imminent death. As they sped down the causeway, they neared a huge
suspension bridge, eight lanes wide, all white concrete and steel
cables, proudly proclaimed to be the Bandra-Worli Sea Link by an
intricate sign in Hindi and English. They sped up the ramp to it,
riding close to the steel girders that lined the bridge's edge, and
beneath them, the sea sparkled blue and seemed so close that she
could reach down and skim her fingertips in the waves. The air
smelled of salt and the sea, the choking traffic fumes whipped away
by a wind that ruffled her dress and trousers, pasting them to her
body. Her fear ebbed away as they crossed the bridge, and did not
come back as they rolled off of it, back into Mumbai, back into the
streets all choked with traffic and people. They swerved around
saddhus, naked holy men covered in paint. They swerved around
dabbahwallahs, men who delivered home-cooked lunches from wives to
husbands all over the city, in tiffin pails arranged in huge wooden
frames, balanced upon their heads.

She knew they were almost at Andheri when they passed the gigantic
Infinity Mall, and then turned alongside a high, ancient brick wall
that ran for hundreds of meters, fencing in a huge estate that had
to be one of the film studios. Outside the wall, along the drainage
ditch, was a bustling market of hawkers, open-air restaurants,
beggars, craftsmen, and, among them, film-makers in smart suits
with dark glasses, clutching mobile phones as they picked their way
along. The bike swerved through all this, avoiding a long line of
expensive, spotless dark cars that ran the length of the wall in an
endless queue to pass through the security checkpoint at the
gatehouse.

She took all this in as they sped down the length of the wall,
cornering sharply at the end, following it along to a much narrower
gate. Two guards with rifles attached to their belts by chains
stood before it, and they hefted their guns as Ashok drew nearer.
Then he drew closer still and the guards recognized him and stepped
away, revealing the narrow gap in the wall that was barely wide
enough for the bike to pass through, though Ashok took it at speed,
and Yasmin gasped when her billowing sleeves rasped against the
ancient, pitted brick.

Passing through the gate was like passing into another world.
Before them, the studios spread forever, the farthest edge lost in
the pollution haze. Roads and pathways mazed the grounds, detouring
around the biggest buildings Yasmin had ever seen, huge buildings
that looked like train stations or airplane hangars from war films.
The grounds were all manicured grass, orderly fruit trees, and
workmen going back and forth on mysterious errands with toolbelts
jangling around their waists, carrying huge bundles of pipe and
lumber and cloth.

Ashok drove them past the hangars -- those must be the sound-stages
where they shot the movies, there was a good studio-map in Zombie
Mecha where you could fight zombies through a series of wood-backed
film scenery -- and toward a series of low-slung trailers that
hugged the wall to their left. Each one had a miniature fence in
front of it, and a small flower-garden, so neat and tidy that at
first she thought the flowers must be fake.

Finally, Ashok slowed the bike and then coasted to a stop, killing
the engine. The engine noise still hummed in her ears, though, and
she continued to feel the thrum of the bike in her legs and bum.
She unlocked her hands from around Ashok's waist, prying her
fingers apart, and stepped off the bike, catching her toe on the
lathi and falling to the grass. Blushing, she got to her feet,
unsteady but upright.

Ashok grinned at her. ``You all right there, sister?''

She wanted to say something sharp and cutting in response, but
nothing came. The words had been beaten out of her by the ride.
Suddenly, she felt as though she could hardly breathe, and the
fabric of her hijab seemed filled with road dust that it released
into her nose and mouth with every inhalation. She carefully undid
the pin and moved her hijab so that it no longer covered her face.

Ashok stared at her in horror. ``You -- you're just a little girl!''

She bridled and the words came to her again. ``I am \emph{14} --
there were girls my age with husbands and babies in Dharavi! I'm a
skilled fighter and commander. I'm no little girl!''

He blushed a purple color and clasped his hands at his chest
apologetically. ``Forgive me,'' he said. ``But -- Well, I assumed you
were 18 or 19. You're tall. I've brought you all this way and
you're, well, you're a child! Your parents will be mad with
worry!''

She gave him her best steely glare, the one she used to make the
boys in the Army behave when they were getting too, well,
\emph{boyish}. ``I left them a note. And I'll be back tonight. And
I'm old enough to worry about this sort of thing on my own account,
thank you very much. Now, you've dragged me halfway across India
for some mysterious purpose, and I'm sure that it wasn't just to
have me stand around here talking about my family life.''

He recovered himself and grinned again. ``Sorry, sorry. Right, we're
here for a meeting. It's important. The Webblies have never had
much contact with real unions, but now that Nor is in trouble,
she's asked me to take up her cause with the unions here. There's
meetings like this happening all over the world today -- in China
and Indonesia, in Pakistan and Mexico and Guatemala. The people
waiting for us inside -- they're labor leaders, representatives of
the garment-workers' union, the steelworkers' union, even the
Transport and Dock Workers' union -- the biggest unions in Mumbai.
With their support, the Webblies can have access to money, warm
bodies for picket lines, influence and power. But they don't know
anything about what you do -- they've never played a game. They
think that the Internet is for email and pornography. So you're
here -- \emph{we're} here -- to explain this to them.''

She swallowed a few times. There was so much in all that she didn't
understand -- and what she \emph{did} understand, she wasn't very
happy about. For example, this \emph{real} union business -- the
Webblies were a real union! But there was more pressing business
than her irritation, for example: ``What do you mean
\emph{we're here to explain}? Are you a gamer?''

He shook his head ruefully. ``Haven't got the patience for it. I'm
an economist. Labor economist. I've spent a lot of time with BSN,
working out strategy with her.''

She wasn't exactly certain what an economist was, but she also felt
that admitting this might further undermine her credibility with
this man who had called her a child. ``I need my lathi,'' she said.

``You don't need a lathi in this meeting,'' he said. ``No one will
attack us.''

``Someone will steal it,'' she said.

``This isn't Dharavi,'' he said. ``No one will steal it.''

That did it. She could talk about the problems in Dharavi.
\emph{She} was a Dharavi girl. But this stranger had no business
saying bad things about her home. ``I need my lathi in case I have
to beat your brains out with it for rubbishing my home,'' she said,
between gritted teeth.

``Sorry, sorry.'' He squatted down beside the bike and began to
unravel the bungee cords from around the lathi. She also went down
on one knee and began to worry at the zipstraps that tied up her
trouser legs at the ankles, but they only went in one direction,
and once they'd locked tight, they wouldn't loosen. Ashok looked up
from the bungee cords.

``You need to cut them off,'' he said. ``Here, one moment.'' He fished
in his trouser-pocket and came up with a wicked flick-knife that he
snapped open. He took gentle hold of the strap on her right ankle
and slid the blade between it and her leg. She held her breath as
he sliced through the strap, then flicked the knife closed, turned
to her other leg, and, grasping her ankle, cut away the other
strap. He looked up at her. Their eyes met, then she looked away.

``Be careful,'' she said, though he'd finished. He handed her the
lathi. She gripped it with numb fingers, nearly dropped it, gripped
it.

``OK,'' he said. ``OK.'' He shook his head. ``The people in there don't
know anything about you or what you do. They are a little, you
know, old fashioned.'' He smiled and seemed to be remembering
something. ``Very old fashioned, in some cases. And they're not very
good with children. Young people, I mean.'' He held up his hands as
she raised her lathi. ``I only mean to warn you.'' He considered her.
``Maybe you could cover your face again?''

Yasmin considered this for a moment. Of course, she didn't want to
cover her face. She wanted to just go in as herself. Why shouldn't
she be able to? But wearing the hijab had some advantages, and one
was that no one would ask you why you were covering your face.
Ashok had clearly believed she was much older until she'd undraped
it.

Wordlessly, she unpinned the fabric, brought it across her face,
and repinned it. He gave her a happy thumbs up and said, ``All
right! They're good people, you know. Very good people. They want
to be on our side.'' He swallowed, thought some, rocked his chin
from side to side. ``But perhaps they don't know that yet.''

He marched to the door, which was made of heavy metal screen over
glass, and opened it, then gestured inside with a grand sweep of
his arm. Trying to look as dignified as possible, she stepped into
the gloom of the trailer, where it was cool and smelled of betel
and chai and bleach, and where a lazy ceiling fan beat the air,
trailing long snot-trails of dust.

This was what she noticed first, and not the people sitting around
the room on sofas and easy-chairs. Those people were sunk deep into
their chairs and sitting silently, their eyes lost in shadow. But
after a moment, they began to shift minutely, staring at her. Ashok
entered behind her and said, ``Hello! Hello! I'm glad you could all
make it!''

And then they stood, and they were all much older than her, much
older than Ashok. The youngest was her mother's age, and he was fat
and sleek and had great jowls and short hair in a fringe around his
ears. There were three others, another man in kurta pyjamas with a
Muslim skull cap and two very old women in sarees that showed the
wrinkled skin on their bellies.

Ashok introduced them around, Mr Phadkar of the steelworkers'
union, Mr Honnenahalli of the transport and dock workers' union,
and Mrs Rukmini and Mrs Muthappa, both from the garment workers'
union. ``These good people are interested in Big Sister Nor's work
and so she asked me to bring you round to talk to them. Ladies and
gentlemen, this is Yasmin, a trusted activist within the IWWWW
organization. She is here to answer your questions.''

They all greeted her politely, but their smiles never reached their
eyes. Ashok busied himself in a corner where there was a chai pot
and cups, pouring out masala chai for everyone and bringing it
around on a tray. ``I will be your chaiwallah,'' he said. ``You just
all talk.''

Yasmin's throat was terribly dry, but she was veiled, and so she
passed on the chai, but quickly regretted it as the talk began.

``I understand that your 'work' is just playing games, is that
right?'' said Mr Honnenahalli, the fat man who worked with the
Transport and Dock Workers' union.

``We work in the games, yes,'' Yasmin said.

``And so you organize people who play games. How are they workers?
They sound like players to me. In the transport trade, we work.''

Yasmin rocked her chin from side to side and was glad of her veil.
She remembered her talk with Sushant. ``We work the way anyone
works, I suppose. We have a boss who asks us to do work, and he
gets rich from our work.''

That made the two old aunties smile, and though it was dark in the
room, she thought it was a genuine one.

``Sister,'' said Mr Phadkar, he in the skullcap, ``tell us about these
games. How are they played?''

So she told them, starting with Zombie Mecha, aided by the fact
that Mr Phadkar had actually seen one of the many films based on
the game. But as she delved into character classes, leveling up,
unlocking achievements, and so on, she saw that she was losing
them.

``It all sounds very complicated,'' Mr Honnenahalli said, after she
had spoken for a good thirty minutes, and her throat was so dry it
felt like she had eaten a mouthful of sand and salt. ``Who plays
these games? Who has time?''

This was something she often heard from her father, and so she told
Mr Honnenahalli what she always told him. ``Millions of people, rich
and poor, men and women, boys and girls, all over the world. They
spend crores and crores of rupees, and thousands of hours. It's a
game, yes, but it's also as complicated as life in some ways.''

Mr Honnenahalli twisted his face up into a sour lemon expression.
``People in life \emph{make} things that matter. They don't just --''
He flapped a hand, miming some kind of pointless labor. ``They don't
just press buttons and play make believe.''

She felt her cheeks coloring and was glad again of the veil. Ashok
held up a hand. ``If a humble chai-wallah may intervene here.'' Mr
Honnenahalli gave him a hostile look, but he nodded. ``'Pressing
buttons and playing make believe' describes several important
sectors of the economy, not least the entire financial industry.
What is banking, if not pressing buttons and asking everyone to
make believe that the outcomes have value?''

The old aunties smiled and Mr Honnenahalli grunted. ``You're a
clever bugger, Ashok. You can always be clever, but clever doesn't
feed people or get them a fair deal from their employers.''

Ashok nodded as though this point had never occurred to him, though
Yasmin was pretty certain from his smile that he'd expected this,
too. ``Mr Honnenahalli, there are over 9,000,000 people working in
this industry, and it turns over 500 crore rupees every year. It's
averaging six percent quarterly growth. And eight of the 20 largest
economies in the world are not countries, they're games, issuing
their own currency, running their own fiscal policies, and setting
their own labor laws.''

Mr Honnenahalli scowled, making his jowls wobble, and raised his
eyebrows. ``They have labor policies in these games?''

``Oh yes,'' Ashok said. ``Their policy is that no one may work in
their worlds without their permission, that they have absolute
power to set wages, hire and fire, that they can exile you if they
don't like you or for any other reason, and that anyone caught
violating the rules can be stripped of all virtual property and
expelled without access to a trial, a judge, or elected
officials.''

That got their attention. Yasmin filed away that description. She'd
heard Big Sister Nor say similar things, but this was better put
than any previous rendition. And there was no denying its effect on
the room -- they jolted as if they'd been shocked and all opened
their mouths to say something, then closed them.

Finally, one of the aunties said, ``Tell me, you say that nine
million people work in these places: where? Bangalore? Pune?
Kolkata?'' These were the old IT cities, where the phone banks and
the technology companies were.

Ashok nodded, ``Some of them there. Some right here in Mumbai.'' He
looked at Yasmin, clearly waiting for her to say something.

``I work in Dharavi,'' she said. And did she imagine it, or did their
noses all wrinkle up a little, did they all subtly shift their
weight away from her, as though to escape the shit-smell of a
Dharavi girl?

``She works in Dharavi,'' Ashok said. ``But only a million or two work
here in India. The majority are in China, or Indonesia, or Vietnam.
Some are in South America, some are in the United States. Wherever
there is IT, there are people who work in the games.''

Now the auntie sat back. ``I see,'' she said. ``Well, that's very
interesting, Ashok, but what do we have to do with China? We're not
in China.''

Yasmin shook her head. ``The game isn't in China,'' she said, as
though explaining something to a child. ``The game is everywhere.
The players are all in the same place.''

Mr Phadkar said, ``You don't understand, sister. Workers in these
places compete with our workers. The big companies go wherever the
work is cheapest and most unorganized. Our members lose jobs to
these people, because they don't have the self-respect to stand up
for a fair wage. We can't compete with the Chinese or the
Indonesians or the Vietnamese -- even the beggars here expect
better wages than they command!''

Mr Honnenahalli patted his belly and nodded. ``We are Indian
workers. We represent them. These workers, what happens to them --
it's none of our affair.''

Ashok nodded. ``Well, that's fine for your unions and your members.
But the union that Yasmin works for --''

Mr Honnenahalli snorted, and his jowls shook. ``It's not a union,''
he said. ``It's a gang of kids playing games!''

``It's tens of thousands of organized workers in solidarity with one
another,'' Ashok said, mildly, as though he was a teacher correcting
a student. ``In 14 countries. Look, these players, they're already
organized in guilds. That's practically unions already. You worry
that union jobs in India might become non-union jobs in Vietnam --
well, here's how you can organize the workers in Vietnam, too! The
companies are multinational -- why should labor still stick to
borders? What does a border mean, anyway?''

``Plenty, if the border is with Pakistan. People \emph{die} for
borders, sonny. You can sit there, with your college education, and
talk about how borders don't matter, but all that means is that
you're totally out of touch with the average Indian worker. Indian
workers want Indian jobs, not jobs for Chinese or what-have-you.
Let the Chinese organize the Chinese.''

``They \emph{are},'' Yasmin broke in. ``They're striking in China
right now! A whole factory walked out, and the police beat them
down. And I helped them with their picket line!''

Mr Honnenahalli prepared to bluster some more, but one of the old
aunties laid a frail hand on his forearm. ``How did you help with a
picket-line in China from Dharavi, daughter?''

And so Yasmin told them the story of the battle of Mushroom
Kingdom, and the story of the battle of Shenzhen, and what she'd
seen and heard.

``Wildcat strikes,'' Mr Honnenahalli said. ``Craziness. No strategy,
no organization. Doomed. Those workers may never see the light of
day again.''

``Not unless their comrades rally to them,'' Ashok said. ``Comrades
like Yasmin and her group. You want to see something workers are
prepared to fight for? You need to get to an internet cafe and see.
See who is out of touch with workers. You can talk all you want
about 'Indian workers,' but until you find solidarity with
\emph{all} workers, you'll never be able to protect your precious
\emph{Indian workers}.'' He was losing his temper now, losing that
schoolmasterish cool. ``Those workers got bad treatment from their
employer so they went out. Their jobs can just be moved -- to
Vietnam, to Cambodia, to Dharavi -- and their strike broken. Can't
you \emph{see it}?
\emph{We finally have the same tools as the bosses}! For a factory
owner, all places are the same, and it's no difference whether the
shirts are sewn here or there, so long as they can be loaded onto a
shipping container when it's done. But now, for us, all places are
the same too! We can go anywhere just by sitting down at a
computer. For forty years, things have gotten harder and harder for
workers -- now it's time to change that.''

Yasmin felt herself grinning beneath the veil. That's it, Ashok,
give it to him! But then she saw the faces of the old people in the
room: stony and heartless.

``Those are nice words,'' one of the aunties said. ``Honestly. It's a
beautiful vision. But my workers don't have computers. They don't
go to Internet cafes. They dye clothing all day. When their jobs go
abroad, they can't chase them with your computers.''

``They can be part of the Webblies too!'' Yasmin said. ``That's the
beauty of it. The ones who work in games, we can go anywhere,
organize anywhere, and wherever your workers are, we are too! We
can go anywhere, no one can keep us out. We can organize dyers
anywhere, through the gamers.''

Mr Honnenahalli nodded. ``I thought so. And when this is all done,
the Webblies organize all the workers in the world, and our unions,
what happens to them? They melt away? Or they're absorbed by you?
Oh yes, I understand very well. A very neat deal all around. You
certainly do play games over there at the Webblies.''

Ashok and Yasmin both started to speak at once, then both stopped,
then exchanged glances. ``It's not like that,'' Yasmin said. ``We're
offering to help. We don't want to take over.''

Mr Honnenahalli said, ``Perhaps you don't, but perhaps someone else
does. Can you speak for everyone? You say you've never met this Big
Sister Nor of yours, nor her lieutenants, the Mighty Whatever and
Justbob.''

``I've met them dozens of times,'' Yasmin said quietly.

``Oh, certainly. In the game. What is the old joke from America? On
the Internet, nobody knows you're a dog. Perhaps these friends of
yours are old men or little children. Perhaps they're in the next
Internet cafe in Dharavi. The Internet is full of lies and tricks
and filth, little sister --'' Her back stiffened. It was one thing
to be called 'sister,' but 'little sister' wasn't friendly. It was
a dismissal. ``And who's to say you haven't fallen for one of these
tricks?''

Ashok held up a hand. ``Perhaps this is all a dream, then. Perhaps
you are all figments of my imagination. Why should we believe in
anything, if this is the standard all must rise to? I've spoken to
Big Sister Nor many times, and to many other members of the IWWWW
around the world. You represent two million construction workers --
how many of them have \emph{you} met? How are we to know that
\emph{they} are real?''

``This is getting us all nowhere,'' one of the aunties said. ``You
were very kind to come and visit with us, Ashok, and you, too,
Yasmin. It was very courteous for you to tell us what you were up
to. Thank you.''

``Wait,'' Ashok said. ``That can't be all! We came here to ask you for
help -- for \emph{solidarity}. We've just had our first strike, and
our executive cell is offline and missing --'' Yasmin turned her
head at this. What did that mean? ``And we need help: a strike fund,
administrative support, legal assistance --''

``Out of the question,'' Mr Honnenahalli said.

``I'm afraid so,'' said Mr Phadkar. ``I'm sorry, brother. Our charter
doesn't allow us to intervene with other unions -- especially not
the sort of organization you represent.''

``It's impossible,'' said one of the aunties, her mouth tight and
sorry. ``This just isn't the sort of thing we do.''

Ashok went to the kettle and set about making more chai. ``Well, I'm
sorry to have wasted your time,'' he said. ``I'm sure we'll figure
something out.''

They all stared at one another, then Mr Honnenahalli stood with a
wheeze, picking up an overstuffed briefcase at his feet and leaving
the little building. Mr Phadkar followed, smiling softly at the
aunties and waving tentatively at Yasmin. She didn't meet his eye.
One of the aunties got up and tried to say something to Ashok, but
he shrugged her off. She went back to her partner and helped her to
her old, uncertain feet. The pair of them squeezed Yasmin's
shoulders before departing.

Once the door had banged shut behind them, Ashok turned and hissed
\emph{bainchoad} at the room. Yasmin had heard worse words than
this every day in the alleys of Dharavi and in the game-room when
the army was fighting, and hearing it from this soft boy almost
made her giggle. But she heard the choke in his voice, like he was
holding back tears, and she didn't want to smile anymore. She
reached up and unhooked her hijab, repinning it around her neck,
freeing her face to cool in the sultry air the fan whipped around
them. She crossed to Ashok and took a cup of tea from him and
sipped it as quickly as she could, relishing the warm wet against
her dry, scratchy throat. Now that her face was clear of hijab, she
could smell the strong reek of old betel spit, and saw that the
baseboards of the scuffed walls were stained pink with old
spittle.

``Ashok,'' she said, using the voice she'd used to enforce discipline
in the army. ``Ashok, look at me. What was that -- that
\emph{meeting} about? Why was I here?''

He sat down in the chair that Mr Phadkar had just vacated and
sipped at his chai.

``Oh, I've made a bloody mess of it all, I have,'' he said.

``Ashok,'' she said, that stern note in her voice. ``Complain later.
Talk now. What did you just drag me halfway across Mumbai for?''

``I've been working on this meeting for months, ever since Big
Sister Nor asked me to. I told her that I thought the trade unions
here would embrace the Webblies, would see the power of a global
labor movement that could organize everywhere all at once. She
loved the idea, and ever since then, I've been sweet-talking the
union execs here, trying to get them to see the potential. With
their members helping us -- and with our members helping them -- we
could change the world. Change it like that!'' He snapped his
fingers. ``But then the strike broke out, and Big Sister Nor told me
she needed help \emph{right now}, otherwise those comrades would
end up in jail forever, or worse. She said she thought you'd be
able to help, and we were all going to talk about it before we came
down, but then, when I was riding to get you --'' He broke off,
drank chai, stared out the grimy, screened in windows at the
manicured grounds of the film studio. ``I got a call from The Mighty
Krang. They were beaten. Badly. All three of them, though Krang
managed to escape. Big Sister Nor is in hospital, unconscious. The
Mighty Krang said he thought it was one of the Chinese factory
owners -- they've been getting meaner, sending in threats. And
they've got lots of contacts in Singapore.''

Yasmin finished her chai. Her hair itched with dust and sweat, and
she slid a finger up underneath it and scratched at a bead of sweat
that was trickling down her head. ``All right,'' she said. ``What had
you hoped for from those old people?''

``Money,'' he said. ``Support. They have the ear of the press. If
their members demanded justice for the workers in Shenzhen, rallied
at the Chinese consulates all around India\ldots{}'' He waved his hands.
``I'm not sure, to be honest. It was supposed to happen weeks from
now, after I'd done a lot more whispering in their ears, finding
out what they wanted, what they could give, what we could give
them. It wasn't supposed to happen in the middle of a strike.'' He
stared miserably at the floor.

Yasmin thought about Sushant, about his fear of leaving Mala's
army. As long as soldiers like him fought for the other side, the
Webblies wouldn't be able to blockade the strikes in-game. So. So
she'd have to stop Mala's army. Stop all the armies. The soldiers
who fought for the bosses were on the wrong side. They'd see that.

``What if we helped ourselves?'' she said. ``What if we got so big
that the unions had to join us?''

``Yes, what if, what if. It's so easy to play what if. But I can't
see how this will happen.''

``I think I can get more fighters in the games. We can protect any
strike.''

``Well, that's fine for the games, but it doesn't help the players.
Big Sister Nor is still in hospital. The Webblies in Shenzhen are
still in jail.''

``All I can do is what I can do,'' Yasmin said. ``What can you do?
What do economists do?''

He looked rueful. ``We go to university and learn a lot of maths. We
use the maths to try to predict what large numbers of people will
do with their money and labor. Then we try to come up with
recommendations for influencing it.''

``And this is what you do with your life?''

``Yes, I suppose it all sounds bloody pointless, doesn't it? Maybe
that's why I'm willing to take the games so seriously -- they're no
less imaginary than anything else I do. But I became an economist
because nothing made sense without it. Why were my parents poor?
Why were our cousins in America so rich? Why would America send its
garbage to India? Why would India send its wood to America? Why
does anyone care about gold?

``That was the really strange one. Gold is such a useless thing, you
know? It's heavy, it's not much good for making things out of --
too soft for really long-wearing jewelry. Stainless steel is much
better for rings.'' He tapped an intricate ring on his right hand on
the arm of the chair. ``There's not much of it, of course. All the
gold we've ever dug out of the ground would form a cube with sides
the length of a tennis court.'' Yasmin had seen pictures of tennis
courts, but she wasn't clear how big this actually was. Not very
large, she supposed. ``We dig it out of one hole in the ground and
then put it in another hole in the ground, some vault somewhere,
and call it money. It seemed ridiculous.

``But everyone \emph{knows} gold is valuable. How did they all agree
on this? That's where I started to get really fascinated. Because
gold and money are really closely related. It used to be that money
was just an easy way of carrying around gold. The government would
fill a hole in the ground with gold, and then print notes saying,
'This note is worth so many grams of gold.' So rather than carrying
heavy gold around to buy things, we could carry around easy paper
money.

``It's funny, isn't it? We dig gold out of holes in the ground,
weigh it, and then put it in another hole in the ground! What good
is gold? Well, it puts a limit on how much money a government can
make. If they want to make more money, they have to get more gold
from somewhere. ''

``Why does it matter how much money a country prints?''

``Well, imagine that the government decided to print a crore of
rupees for every person in India. We'd all be rich, right?''

Yasmin thought for a moment. ``No, of course not. Everything would
get more expensive, right?''

He waggled his chin. He was sounding like a schoolteacher again.
``Very good,'' he said. ``That's inflation: more money makes
everything more expensive. If inflation happened evenly, it
wouldn't be so bad. Say your pay doubled overnight, and so did all
the prices -- you'd be all right, because you could just buy as
much as you could the day before, though it 'cost' twice as much.
But there's a problem with this. Do you know what it is?''

Yasmin thought. ``I don't know.'' She thought some more. Ashok was
nodding at her, and she felt like it was something obvious, almost
visible. ``I just don't know.''

``A hint,'' he said. ``Savings.''

She thought about this some more. ``Savings. If you had money saved,
it wouldn't double along with wages, right?'' She shook her head. ``I
don't see why that's such a problem, though. We've got some money
saved, but it's just a few thousand rupees. If wages doubled, we'd
get that back quickly from the new money coming in.''

He looked surprised, then laughed. ``I'm sorry,'' he said. ``Of
course. But there are some people and companies and governments
that have a \emph{lot} of savings. Rich people might save crores of
rupees -- those savings would be cut in half overnight. Or a
hospital might have many crores saved for a new wing. Or the
government or a union might have crores in savings for pensions.
What if you work all your life for a pension of two thousand rupees
a month, and then, a year before you're supposed to start
collecting it, it gets cut in half?''

Yasmin didn't know anyone who had a pension, though she'd heard of
them. ``I don't know,'' she said. ``You'd work, I suppose.''

``You're not making this easy,'' Ashok said. ``Let me put it this way:
there are a lot of powerful, rich people who would be very upset if
inflation wiped out their savings. But governments are very tempted
by inflation. Say you're fighting an expensive war, and you need to
buy tanks and pay the soldiers and put airplanes in the sky and
keep the missiles rolling out of the factories. That's expensive
stuff. You have to pay for it somehow. You could borrow the money
--''

``Governments borrow money?''

``Oh yes, they're shocking beggars! They borrow it from other
governments, from companies -- even from their own people. But if
you're not likely to win the war -- or if victory will wipe you out
-- then it's unlikely anyone will voluntarily lend you the money to
fight it. But governments don't have to rely on voluntary payments,
do they?''

Yasmin could see where this was going. ``They can just tax people.''

``Correct,'' he said. ``If you weren't such a clearly sensible girl,
I'd suggest you try a career as an economist, Yasmin! OK, so
governments can just raise taxes. But people who have to pay too
much tax are unlikely to vote for you the next time around. And if
you're a dictator, nothing gets the revolutionaries out in the
street faster than runaway taxation. So taxes are only of limited
use in paying for a war.''

``Which is why governments like inflation, right?''

``Correct again! First, governments can print a lot of money that
they can use to buy missiles and tanks and so on, all the while
borrowing even more, as fast as they can. Then, when prices and
wages all go up and up -- say, a hundred times -- then suddenly
it's very easy to repay all that money they borrowed. Maybe it took
a thousand workers' tax to add up to a crore of rupees before
inflation, and now it just takes one. Of course, the person who
loaned you the money is in trouble, but by that time, you've won
the war, gotten reelected, and all without crippling your country
with debt. Bravo.''

Yasmin turned this over. She found it surprisingly easy to follow
-- all she had to do was think of what happened to the price of
goods in the different games she played, going up and down, and she
could easily see how inflation would work to some players' benefit
and not others. ``But governments don't have to use inflation just
to win wars, do they?'' She thought of the politicians who came
through Dharavi, grubbing for the votes the people there might
deliver. She thought of their promises. ``You could use inflation to
build schools, hospitals, that sort of thing. Then, when the debt
caught up with you, you could just use inflation to wipe it out.
You'd get a lot of votes that way, I'm quite sure.''

``Oh yes, that's the other side of the equation. Governments are
always trying to get re-elected with guns or butter -- or both. You
can certainly get a lot of votes by buying a lot of inflationary
hospitals and schools, but inflation is like fatty food -- you
always pay the price for it eventually. Once hyperinflation sets
in, no one can pay the teachers or nurses or doctors, so the next
election is likely to end your career.

``But the temptation is powerful, very powerful. And that's where
gold comes in. Can you think of how?''

Yasmin thought some more. Gold, inflation; inflation, gold. They
danced in her head. Then she had it. ``You can't make more money
unless you have more gold, right?''

He beamed at her. ``Gold star!'' he said. ``That's it exactly. That's
what rich people like about gold. It is a disciplinarian, a
policeman in the treasury, and it stops government from being
tempted into funding their folly with fake money. If you have a lot
of savings, you want to discipline the government's money-printing
habits, because every rupee they print devalues your own wealth.
But no government has enough gold to cover the money they've
printed. Some governments fill their vaults with other valuable
things, like other dollars or euros.''

``So dollars and euros are based on gold, then?''

``Not at all!'' No, they're backed by other currencies, and by little
bits of metal, and by dreams and boasts. So at the end of the day,
it's all based on nothing!''

``Just like game-gold!'' she said.

``Another gold star! Even \emph{gold} isn't based on gold! Most of
the time, if you buy gold in the real world, you just buy a
certificate saying that you own some bar of gold in some vault
somewhere in the world. The postman doesn't deliver a gold-brick
through your mail-slot. And here's the dirty secret about gold:
there is more gold available through certificates of deposit than
has ever been dug out of the ground.''

``How is that possible?''

``How do you think it's possible?''

``Someone's printing certificates without having the gold to back
them up?''

``That's a good theory. Here's what I think happens. Say you have a
vault full of gold in Hong Kong. Call it a thousand bars. You sell
the thousand bars' worth of gold through the certificate market,
and lock the door. Now, some time later, someone -- a security
guard, an executive at the bank -- walks into the vault and walks
out again with ten gold bars from the middle of the pile. These ten
bars of gold are sold at a metals market, and they end up in a
vault in Switzerland, which prints certificates for \emph{its} gold
holdings and sells them on. Then, one day, an executive at the
Swiss bank helps himself to ten bars from \emph{that} vault and
they get sold on the metals market. Before you know it, your ten
bars of gold have been sold to a hundred different people.''

``It's inflation!''

He clapped. ``Top pupil! Correct. There's a saying from physics,
'It's turtles all the way down.' Do you know it? It comes from a
story about a British physicist, Bertrand Russell, who gave a
lecture about the universe, how the Earth goes around the Sun and
so on. And a little old granny in the audience says, 'It's all
rubbish! The world is flat and rests on the back of a turtle!' And
Russell says, 'If that's so, what does the turtle stand on?' And
the granny says, 'You can't fool me, sonny, it's turtles all the
way down!''' In other words, what lives under the illusion is yet
another illusion, and under that one is another illusion again.
Supposedly good currency is backed by gold, but the gold itself
doesn't exist. Bad currency isn't backed by gold, it's backed by
other currencies, and \emph{they} don't exist. At the end of the
day, all that any of this is based on is, what, can you tell me?''

``Belief,'' Yasmin said. ``Or fear, yes? Fear that if you stop
believing in the money, you won't be able to buy anything. It
\emph{is} just like game-gold! I remember one time when Zombie
Mecha started charging for buffs that used to be free and
overnight, all the players left. The people who were left behind
were so desperate, walking around, trying to hawk their gold and
weapons, offering prices that were tiny compared to just a few days
before. It was like everyone had stopped believing in Zombie Mecha
and then it stopped existing! And then the game dropped its prices
and people came back and the prices shot back up again.''

``We call it 'confidence','' Ashok said. ``If you have 'confidence' in
the economy, you can use its money. If you don't have confidence in
the economy, you want to get away from it and get it away from you.
And it's turtles all the way down. There's almost nothing that's
worth \emph{anything}, except for confidence. Go to a steel foundry
here in Mumbai and you'll find men risking their lives, working in
the fires of hell in their bare feet without helmets or gloves,
casting steel to make huge round metal plates to cover the sewer
entrances in America. Why do they do it? Because they are given
rupees -- which are worth nothing unless you have confidence in
them. And why are they given rupees? Because someone -- the boss --
thinks that he'll get dollars for his steel discs. What are dollars
worth?''

``Nothing?''

``\emph{Nothing!} Unless you believe in them. And what about the
discs -- what good are they? They're the wrong size for the sewer
openings in Mumbai. You could melt them down and do something else
with them, but apart from that, they're just bloody heavy biscuits
that serve no useful purpose. So why does any of this happen?''

Yasmin said, ``Oh, that's simple. You really don't know?''

``It's easy? Please, tell me. It's not easy for me and I've been
studying it all my life.''

``It all happens because it's a \emph{game}!''

He looked offended. ``Maybe it's a game for the rich and powerful --
but it's not any fun for the poor and the workers and the savers
who get the wrong end of it.''

``Games don't need to be \emph{fun}, they only have to be, I don't
know, \emph{interesting}? No, \emph{captivating}! There are so many
times when I find myself playing and playing and playing, and I
can't stop even though it's all gotten very boring and repetitive.
'One more quest,' I tell myself. 'One more kill.' And then again,
'One more, one more, one more.' The important thing about a game
isn't how fun it is, it's how easy it is to start playing and how
hard it is to stop.''

``Aha. OK, that makes sense. What, specifically, makes it hard to
stop?''

``Oh, many little things. For example, in Zombie Mecha, if you stop
playing without going to a mecha-base, you get 'fatigued.' So when
you come back to the game, you play worse and earn fewer points for
making the same kills and running the same dungeons. So you think,
'OK, I'm done for today, time to go back to a base.' And you run
for a base, which is never very close to the quests, and on the
way, you get a new quest, a short one that has a lot of good
rewards. You do the quest. Now you head for the base again, but
again, you find yourself on a quest, but this one is a little
longer than it seemed, and now even more time has gone by. Finally,
you reach the base, but you've played so much that you've almost
levelled up, and it would be a pity to stop playing now when just a
few random kills would get you to the next level and then you can
buy some very good new weapons and training at the base, so you
hunt down some of the biters around the base-entrance, and now you
level up, and you get some good new weapons, and you've also just
unlocked many new quests. These quests are given to you when you
reach the base, and some of them look very interesting, and now
some of your friends have joined you, so you can group with them
and run the quests together, which will be much quicker and a lot
more fun. And by the time you stop, it's been three, sometimes four
hours more play than you thought you'd do.''

``This happens a lot?''

``Oh yes. Many times a week for me. And I don't even play for points
-- I play to help the union! The more play you do, the more sense
it makes to keep on playing. All this business with gold and rupees
and dollars and steel plates -- we play that game all the time,
don't we? So of course it works. Everyone plays it because everyone
has played it all their lives.''

``I can see why Big Sister Nor told me I must talk with you,'' he
said. ``You're a very clever girl.''

She looked down.

``What do we do about Big Sister Nor?''

``She thinks we need to find money and support for the strikers. I
think she needs money and support for \emph{herself}. She says
she's fine, but she's in hospital and it sounds like she was badly
beaten.''

``How do we get her support from here? They're so far away.''
Thinking:
\emph{Mumbai's opposite corner is far away for me -- China might as well be the moon or the Mushroom Kingdom.}
``And how do we know that Big Sister Nor will be safe where she
is?''

``Both good questions,'' he said. ``It's frustrating. They're so close
when we're all online, but so far when we need to do something that
involves the physical world.'' He began to pace. ``This is Big Sister
Nor's department. She sees a way to tie up the virtual world and
the real world, to move work and ideas and money from one to the
other.''

``Maybe we should just concentrate on the games, then? They're the
part we know how to use.''

``But these people are in trouble in the real world,'' Ashok said,
balling his hands into fists.

And Yasmin found herself giggling, and then laughing, really
laughing. It was so obvious!

``Oh, Ashok,'' she said, ``oh, yes, they certainly are.''

And she knew just what to do about it.

\tb

\shopad{This scene is dedicated to Waterstone's, the national UK bookselling chain. Waterstone's is a chain of stores, but each one has the feel of a great independent store, with tons of personality, great stock (especially audiobooks!), and knowledgeable staff. Of particular note is the Manchester Deansgate store, which has an \emph{outstanding} sf section.}
{\href{http://www.waterstones.com}{Waterstones}}

Lu didn't know where to go. Boss Wing's dormitories were out of the
question, of course. And while he knew a dozen Internet cafes in
Shenzhen where he could sit and log on to the game, he didn't
really want to be playing just then. Not with everyone else in
jail.

But he had to sit down. He'd been hit hard in the head and on the
shoulder and he was very dizzy. He'd thrown up once already,
holding onto a bus-stop pole and leaning over the gutter, earning a
disapproving cluck from an old woman who walked past hauling a huge
barrow full of electronic waste.

He had thought of texting Matthew and the others, to find out if
the police had them in custody, but he was afraid that the police
would track him back if he did, using the phone network to locate
him and pick him up.

It had all felt so \emph{wonderful}. They'd stood up from their
computers, chanting angrily, the war-chants from the games, which
Boss Wing and his goons never played, and so it had all been
totally perplexing to them. Their faces had gone from puzzlement to
anger to fear as all the boys in the room stood together and
marched out of the cafe, blocking the doorways so that no one could
come in.

And there had been girls, and old grannies, and young men stopping
to admire them as they stood, shoulder to shoulder, chanting
bravely at the cowardly goons from Boss Wing's factory, goons who'd
been so tough just a few minutes before, willing to slap you in the
head if you talked too much, ready to dock your pay, too. Ever
since they'd tried to go out on their own, life had gotten steadily
worse. Boss Wing had a huge operation, with plenty of in-game
muscle to stand guard against rich players who hunted the gold
farmers for sport, but he was cruel and cheap and you were lucky if
you saw half the wages you'd earned after all the fines for
``breaking rules'' had been charged against your salary.

Their phones rang and buzzed with photos from other Boss Wing
factories where the workers had gone out too, and there were wars
in Mushroom Kingdom as the Webblies kept anyone else from working
their zone. And the police came and they'd stayed brave, Matthew
and Ping and all his friends. They were workers, they were
warriors, they were an army and their cause was just. They would
not be intimidated.

And then the gas came. And then the clubs started swinging. And
then the screams had started. And then Lu had run, run through the
stinging clouds of gas and the chaos of battle -- so like and so
unlike the million battles he'd fought in the games -- and he'd
thrown up and now --

Now he had no idea where to go.

And then his phone rang. The number was blanked out, which made his
pulse hammer in his throat. Did the secret police blank out the
number when they called you? But if the secret police knew he
existed and had his phone number, they could just pick him up where
he stood, using the phone's damned tracking function.

It wasn't the police. With trepidation, he slid his finger over the
talk button on the screen.

``Wei?'' he said, cautiously.

``Lu? Is that you?'' The call had the weird, echoey sound of a cheap
net-calling service, the digital fuzz of packets that travelled
third class on the global network. The accent was difficult, too,
thick-tongued and off-kilter. He knew the sound and he knew the
voice.

``\emph{Wei-Dong}?''

``Yes!''

``Wei-Dong in \emph{America}?'' He hadn't heard from the strange
gweilo since they'd gone to Boss Wing and Ping had had to kick him
out of the guild. Boss Wing didn't allow them to raid with outside
people, or even talk to them in game. He had spyware on all his PCs
that told him when you broke those rules, and you lost a day's
wages for the first offense, a week's wages for the second.

``Lu, it's me! Look, did I just see you and Ping getting beaten up
by the cops?''

``I don't know, did you?'' The disorientation from his head wound was
fierce, and he wondered if he was really having this conversation.
It was very strange.

``I -- I just saw you getting beaten up on a video from Shenzhen. I
think I did. Was it you?''

``We just got beaten up,'' he said. ``I'm hurt.''

``Are you badly hurt? I couldn't reach Ping, so I tried you.'' He was
excited, his voice tight. ``What happened?''

Lu was still grappling with the idea that the gweilo had just
called him from thousands of kilometers away. ``You saw me on the
Internet in America?''

``Every gamer in the world saw you, Lu! You couldn't have timed it
better! After dinner is the busiest time on the servers, and the
word went around like nothing I've ever seen before. Everyone in
every game was chatting about it, passing around links to the video
streams and the photos. It was even on the real news! My neighbor
banged on my wall and asked me if I knew anything about it. It was
incredible!''

``You saw me getting beaten up on the Internet?''

``Dude, \emph{everyone} saw you getting beaten up on the Internet.''

Lu didn't know what to say. ``Did I look good?''

Wei-Dong laughed like a hyena. ``You looked \emph{great}!''

A dam broke, Lu laughed and laughed and laughed, as all the tension
flooded out of him. He finally stopped, knowing that if he didn't
he'd throw up again. He was by the train station now, in the heavy
foot-traffic, all kinds of people moving purposefully around him as
he stood still, a woozy island in the rushing stream. He backed up
to a stairwell in front of a beauty parlor and sank to his
haunches, squatting and holding the phone to his head.

``Wei-Dong?''

``Yes.''

``Why are you calling me?''

There was an uncomfortable silence on the line, broken by soft
digital flanging. ``I wanted to help you,'' he said at last. ``Help
the Webblies.''

``You know about the Webblies?'' Lu had half-believed that Matthew
had made them up, a fantasy army of thousands of imaginary friends
who would fight for them.

``Know about them? Lu, they're the ass-kickingest guild in the
world! No one can beat them! Coca-Cola Games is sending us three
memos a day about them!''

``Why does Coca-Cola send you memos?''

``Oh.'' More silence. ``Didn't I tell you? I'm working for them now.
I'm a Turk.''

``Oh,'' said Lu. He knew about the Turks, but he never really thought
about what kind of people would work in ten second increments
making up dialog for non-player characters or figuring out what
happened when you shot an office chair with a blunderbuss. ``That
must be interesting.''

Wei-Dong made a wet noise. ``It's miserable,'' he said. ``I run four
different sessions at once, and I'm barely earning enough to pay
the rent. And they make so much money off of us! Last month, they
announced quarterly profits and games with Turks are earning 30
percent more than the ones without. They're hiring more Turks as
fast as they can -- it's all over the board here. But our wages
aren't going up. So I've been thinking of the Webblies, you
know\ldots{}'' He trailed off. ``Like maybe you guys can help us if we
help you? We all play for our money, right? So why shouldn't we be
on the same side.''

``Sounds right to me,'' Lu said. He was still trying to comprehend
the fact that the Webblies were apparently famous with American
teenagers. ``Wait,'' he said, playing back Wei-Dong's accented,
ungrammatical speech. ``You're paying rent?''

``Yeah,'' Wei-Dong said. ``Yeah! Living on my own now. It's great! I
have a crappy room in a, not sure what you call it, a hotel, kind
of. But for people who don't have any money. But I can get wireless
here and I've got four machines and there's plenty of stuff I can
walk to, at least compared to home --'' He began to babble about his
favorite restaurants and the clubs that had all-ages nights and a
million tiny irrelevant details about Los Angeles, which might as
well have been the Mushroom Kingdom for all that it mattered to Lu.
He let it wash over him and tried to think of places he could go to
recuperate. He fleetingly wished for his mother, who always knew
some kind of traditional Chinese remedy for his ailments. They
often didn't work, but sometimes they did, and his mother's gentle
application of them worked their own magic.

He was suddenly, nauseously, overwhelmingly homesick. ``Wei-Dong,''
he said, interrupting the virtual tour of Los Angeles. ``I need to
think now. I don't know what to do. I'm hurt, I'm on the street,
and I can't call anyone in case the police trace the call. What do
I do?''

``Oh. Well. I don't know exactly. I was hoping that you'd know what
\emph{I} should do, to tell you the truth. I want to get
involved!''

``I think I want to get \emph{uninvolved}.'' Lu's homesickness was
turning to anger. Who was this \emph{boy} to call him from the
other side of the world, demanding to ``get involved?'' Didn't he
have enough problems of his own? ``What can you do for me from
there? What is any of this -- this \emph{garbage} worth? How will
everyone going to jail make my life better? How will having my head
beaten in help make things better? How?''

``I don't know.'' Wei-Dong's voice was small and hurt. Lu struggled
to control his anger. The gweilo wanted to help. It wasn't his
fault he didn't know how to help. Lu didn't know how to help,
either.

``I don't know either,'' Lu said. ``Why don't you think about how to
help and call me back. I need to find somewhere to rest, maybe a
nurse or a doctor. OK?''

``Sure,'' the gweilo said. ``Sure. Of course. I'll call you back soon,
don't worry.''

Every time a Hong Kong train came into the Shenzhen Railway
Station, it disgorged a massive crowd of people: Hong Kong people
in sharp business styles, rich kids, foreigners, and workers from
Shenzhen returning from contracts abroad, clutching backpacks. The
dense group got broken up by the taxi-rank and the shopping mall,
and emerged as a diffuse cloud onto the street where Lu had been
talking. Now he worked his way back through this crowd, listening
to snatches of hundreds of conversations about business,
manufacturing -- and gold farming.

It was on everyone's lips, talk about the strike, about the police
action, about the farmers. Of course most people in China had heard
of gold farming and all the stories about the money you could make
by just playing video games, but you never heard this kind of
business-person talking about it. Not smart, fancy people with
obvious wealth and power, the kind of people who skipped back and
forth between Hong Kong and Shenzhen, talking rapidly into their
earwigs, telling other people what to do.

What had the gweilo said?
\emph{Everyone saw you getting beaten up on the Internet!} Were
these people looking closely at him? Now it seemed they were. Of
course, he was bloody, staring, red-eyed. Why wouldn't they stare
at him? But maybe --

``You're one of them, aren't you?'' She was 22 or 23, with perfect
fingernails on the hand she rested on his arm, coming on him from
behind. He gave an involuntary squeak and jump, and she giggled a
little. ``You must be,'' she said. She held up her phone. ``I watched
the video five times on the train. You should see the commentary.
So ugly!''

He knew about this. Any time something that made the government
look bad managed to find its way online, there was an army of
commenters who'd tweet and post and comment about how the
government was in the right, how the story was all wrong, how the
people in it were guilty of all kinds of terrible things. Lu knew
he shouldn't believe any of it, but it was impossible to read it
all without feeling a little niggle of doubt, then a little more,
and then, like an ice-cube on a bruise, the outrage he'd felt at
first would go numb.

The thought that he, himself, was at the center of one of these
smear-storms made him feel like he was going to throw up again. The
girl must have seen this, for she gave his arm a little squeeze.
``Oh, don't look so serious. You looked great on the video. I'm sure
no one believes all that rubbish!'' She pursed her lips. ``Well, of
course, that's not true. I'm sure lots of people believe it. But
they're fools. And so many more were inspired, I'm sure. I'm Jie.''

``Lu,'' Lu said, after trying and failing to come up with an alias.
He was not cut out to be a fugitive. ``It was nice to meet you,'' he
said, and shrugged her hand off and set off deeper into the crowd.

She grabbed his arm again. ``Oh, please stop. We need to talk.
Please?''

He stopped. He didn't have much experience with girls, but
something about her voice made him want to stay. ``Why do we need to
talk?''

``I want to get your story,'' she said. ``For my show.''

``Your \emph{show}?''

She leaned in close -- so close he could smell her perfume -- and
whispered, ``I'm Jiandi,'' she said.

He looked at her blankly.

She shook her head. ``Jiandi,'' she hissed. ``Jiandi! From the Factory
Girl Show!''

He shrugged. ``What kind of show?''

``Every night!'' she said. ``At 9PM! Twelve million factory workers
listen to me! They phone me with their problems. We go out over the
net, audio, through the, uh,'' she dropped her voice, ``the Falun
Gong proxies.''

``Oh,'' he said, and began to move away.

``It's not religious,'' she said. ``I just help them with their
problems. The --'' she dropped her voice ``\emph{proxies} are just
how we get the show into the factories. They try to block me
because we tell the truth about the work conditions -- the girls
who are sexually pressured by their bosses, the marketing rip-offs,
the wage rip offs, lock-ins --''

``OK,'' he said. ``I get the picture. Thank you but no.''

``Come \emph{on},'' she said and looked deep into his eyes. Hers were
dark and lined with thin, precise green eye-pencil, and her
eyebrows were shaped into surprised, sophisticated arches. ``You
look like you need a place to clean up, and maybe a meal. I can get
that for you.''

``You can?''

``Lu, I'm \emph{famous}! I have advertisers who pay a \emph{lot} to
sponsor my show. I have millions of supporters all over Shenzhen,
even in Guangzhou and Dongguan. Even in Shanghai and Beijing! I'm a
hero to them, Lu. I can put your story into the ears of every
worker in the Pearl River Delta like \emph{that}!'' She snapped her
fingers in front of his nose, making him blink and start back
again. She laughed. ``You're cute,'' she said. ``Come on, it'll be
wonderful.''

``Where do we go?'' he said, cautiously.

``Oh, I have a place,'' she said.

She grabbed his hand -- her fingers were dry and cool, and touched
with cold spots where the rings she wore met his skin. She led him
away through the crowd, which seemed to part magically before her.
It had all become like a dream now, with the pain crowding Lu's
vision into a hazy-edged tunnel. He wondered if she'd have
something for the pain. He wondered if she knew any traditional
medicine, if she'd mix him up a bitter tea with complicated scents
and small bits of hard things floating in it. All this he wondered,
and the streets and sidewalks slipped past beneath their feet like
magic. You could automatically follow your guildies in game, just
click on them and select follow, and the whole guild could do that
when there was a lot of distance to cover, so that only one player
had to pay attention on the long march across the world, while the
others relaxed and smoked and ate and used the toilet, while their
characters trailed like a string of pack-animals behind the
leader.

That's what this felt like, like he was a character whose player
had stepped out for a cigarette and a piss-break and the character
bumped along mindlessly behind the leader.

``Do you live here?'' he said as they reached the lobby of a tall
apartment building. It was a ``handshake building,'' so close to the
building next to it that the tenants could lean out their windows
and shake hands with their neighbors across the lane. The lobby
smelled of cooking and sweat, but it was clean and there was a
working intercom and lock at the door.

``No,'' she said. ``I do some of my shows from here. There are two or
three of them, to confuse the jingcha.'' He thought it was funny to
hear her use the gamer clan term for police. She saw it, and said,
``Oh yes, the zengfu think I'm very biantai and they'd PK me if they
could.'' He laughed at this, because it was nearly impenetrable
slang -- the government think I'm a pervert so they want to
``player-kill'' -- destroy -- me if they can. It was one thing to
hear a boy with his shirt rolled up over his belly and a cigarette
hanging out of his face saying this, another to hear this delicate,
preciously made-up girl.

The elevator was broken, so she led him up five flights of stairs,
the walls decorated with lavish graffiti: murals of curse-words,
scenes of factory life, phone numbers you could call to buy fake
identity papers, degrees, certificates. Lu's own dorm room was in a
building that Boss Wing rented, and he climbed twice this many
stairs every day, but this climb felt like it was going to kill
him. On Jie's floor, there was an old lady squatting by the
stairway door, in the hall. She nodded at the two of them.

``Mrs Yun,'' Jie said, ``I would like you to meet Hui. He is a
mechanic who has come to repair my air-conditioner.'' The old lady
nodded curtly and looked away.

Jie attacked one of the apartment doors with a key ring, opening
four different locks with large, elaborate, thick keys and then
putting her shoulder into the door, which swung heavily back,
clanging against a door-stop with a metallic sound. She motioned
him inside and closed the door, shooting the four bolts from the
inside and slapping at several light-switches.

The apartment had two big rooms, the living room in which they
stood, and a connecting bedroom that he could see from the doorway.
There was a little kitchen area against the wall beside them, and
the rest of the room was taken up with a sofa and a large desk with
chairs on either side of it, covered in a litter of recording gear:
a mixer, several large sets of headphones, and a couple of skinny
mics on stands. Every centimeter of wall-space was \emph{covered}
in paper: newspaper clippings, letters, drawings -- all liberally
sprinkled with stickers, hearts, cute animal doodles.

Jie waved her hand at it: ``My studio!'' she said, and twirled
around. ``All my fan-mail and my press.'' She ran her fingers lightly
over a wall. Peering more closely at it, Lu saw that every letter
began ``Dear Jiani'' and that they were all written in neat, girlish
hands. ``I have a post-box in Macau. My friends send the letters
there and they scan them and email them to me. All right under the
zengfu's nose!''

``And the old lady in the hall?''

She flopped down on the sofa, her skirt riding up around her
thighs, and kicked her shoes in expert arcs to the mat by the door.
``Our building's answer to the bound-foot grannies' detective
squad,'' she said, and he laughed again at the slang. Back in
Nanjing, they'd used this term to talk about the little old ladies
who were always snooping around, gossiping about who was doing
something evil or wicked. They didn't really have bound feet -- the
practice of binding little girls' feet to the point where they grew
up unable to walk properly was dead, and he'd never seen a real
bound foot outside of a museum, though the grannies would always
exclaim over the girls' feet, passing evil remarks if a girl had
large feet, cooing if she had small ones -- but they acted all
pinched anyway.

``And she'll believe that I'm a repairman? I don't have any tools!''

``Oh, no,'' Jie laughed again. It was a pretty sound. Lu could see
how she'd be a very popular netshow host. That laugh was
infectious. ``No, she'll think we're having sex!''

He felt himself turning red and stammering. ``Oh -- Uh --''

Now she was howling with laughter, head flung back, hair fanned out
over the sofa-cushions. ``You should see your face! Look, so long as
Grandma Mao out there thinks I'm just a garden-variety slut, she
won't suspect that I'm really Jiandi, Scourge of the Politburo and
Voice of the Pearl River Delta, all right? Now, get your shoes off
and let's have a look at that head-wound.''

He did as he was bade, neatly lining his shoes up by the doorway
and stepping gingerly onto the dusty wooden floor. Jia stood and
led him by the shoulders to one of the rolling chairs by the desk
and pushed him down on it, then leaned over him and stared intently
at his scalp. ``OK,'' she said. ``First of all, you need to switch
shampoo, you have very greasy hair, it's shameful. Second of all,
you appear to have a pigeon's egg growing out of your head, which
has got to sting a little. I'll tell you what, I'll get you
something cold to hold on it for a few moments, then I want you to
go have a shower and clean it out well. It looks like it bled a
little, but not much, which is lucky for you, since scalp wounds
usually bleed like crazy. Then, once we've got you into a more
civilized state, I'll put you on the Internet and make you even
more famous. Sound good?''

He opened his mouth to object, but she was already spinning away
and digging through the small fridge, crouching, hair falling over
her shoulders in a way that Lu couldn't stop staring it. Now she
had a bag of frozen Hahaomai chicken dumplings -- he recognized the
packaging, it was what they ate for dinner most nights in Boss
Wing's dormitory -- and was wrapping it in a tea-towel, and
pressing it to his head. It felt like it weighed 500 kilos and had
been cooled to absolute zero, but it also made his head stop
throbbing almost immediately. He slumped in the chair and closed
his eyes and held the dumplings to the spot where the zengfu -- the
slang was infectious -- had given him a love-tap. He tracked Jia's
movements around him by the sounds she made and the puffs of
perfume and hair stuff whenever she passed close. This was not bad,
he thought -- a lot better than things had been an hour ago when
he'd been crouching in front of the station talking to the gweilo.

``Right,'' she said, ``take these.'' He opened his eyes and saw that
she was holding out two chalky pills and a glass of water for him.

``What are they?'' he said, narrowing his eyes at the glare of the
sunset light streaming in the window. He'd been nearly asleep.

``Poison,'' she said. ``I've decided to put you out of your misery.
Take them.''

He took them.

``The shower's through there,'' she said, pointing toward the
bedroom. ``There's a towel on the toilet-seat, and I found some
pajamas that should fit you. We'll rinse out your clothes and put
them on the heater to dry while we talk. No offense, Mr Labor Hero,
but you smell like something long dead.''

He was blushing again, he could tell, and there was nothing for it
but to duck and scurry through the bedroom -- he had a jumbled
impression of a narrow bed with a thin blanket crumbled at the
bottom, a litter of stuffed animals, and mounds of fake handbags
overflowing with clothing and toiletries. Then he was in the
bathroom, the sink-lip covered in mysterious pots and potions, all
the oddments of a girl which a million billboards hinted at, but
which he'd never seen in place, lids askew, powder spilling out. It
was all so much less glamorous than it appeared on the billboards,
where everything looked like it was slightly wet and glistening,
but it was much more exciting.

Every horizontal space in the shower seemed to support some kind of
bottle. Lu bought big two liter jugs of shower gel that he could
use as shampoo, too, but after squinting at the labels, he found
one that appeared to be for bodies and another for hair, and made
use of both. The water on his head felt like little sharp stones
beating against it, and his shoulder began to throb as he rubbed
the shampoo in. After the shower, he cleared the steam off the
mirror and craned around to get a look at it, and could just make
out the huge, raised bruise there, a club-shaped purple bruised
line surrounded by a halo of greeny-yellow swelling.

``There's something you can wear on the bed,'' Jia yelled from the
other side of the door. He cautiously turned the knob and found
that she'd drawn a curtain across the door to the bedroom, leaving
him alone in naked semi-darkness. On the bed, neatly folded, a pair
of track pants and a t-shirt for an employment bureau, the kind of
thing they gave out to the people who stood in front of them all
day long, paid for every person they brought in to apply for a job.
It was a tight fit, but he got it on, and balled up his clothes,
which really did stink, and peeked around the curtain.

``Hello?''

``Come on out here, beautiful!'' she said, as he stepped out, his
bare feet on the dusty tile. She leaned in and sniffed at him with
a delicate little sniffle. ``Mmmm, you chose the dang-gui shampoo.
Very good. Very good for ladies' reproductive issues.'' She patted
his stomach. ``You'll have a little baby there in no time!''

He now felt like he would faint from embarrassment, literally, the
room spinning around him.

She must have seen it in his face, for she stopped laughing and
gave his hand a squeeze. ``Don't worry,'' she said. ``It's only
teasing. Dang-gui is good for everything. Your mother must have
given it to you.'' And yes, he realized now, that was where he knew
that smell from -- he remembered wishing that his mother was there
to give him some herbs, and that wish must have guided his hand
among the many bottles in her shower.

``Do you live here?'' he said.

``In this pit?'' She made a face. ``No, no! This is just one of my
studios. It helps to have a lot of places where I can work. Makes
life harder for the zengfu.''

``But the clothes, the bed?''

``Just a few things I leave for the nights when I work late. My show
can go all night, sometimes, depending on how many callers I have.''
She smiled again. She had dimples. He hadn't ever noticed a girl's
dimples before. The head injury was making him feel woozy. Or maybe
it was love.

``And now?''

``And now we talk to you about what you've seen,'' she said. ``My show
starts in --'' she looked at the face of her phone -- ``12 minutes.
Just enough time for you to have a drink and get comfortable.'' She
fished in her fridge and brought out a water filter jug and filled
a glass from a small rack next to the tiny sink. He took it and
drank it greedily and she fetched him the filter, setting it down
on one side of the desk before settling into the chair on the other
side.

She began to click and type and furrow her brow in an adorable way,
slipping on a set of huge headphones, positioning a mic. She waved
to him and he settled into the opposite chair, refilling his
glass.

``What kind of show is this again?''

``You are such a \emph{boy}!'' she said, looking up from her screen,
fingers still punishing her keyboard with insectile clicks from her
manicured fingernails.

He looked down at himself. ``I suppose I am,'' he said.

``What I mean is, if you were a girl, you'd know all about this.
Every factory girl listens to me, believe it. I start broadcasting
after dinner, and they all log in and call in and chat and phone
and tell me all their troubles and I tell them what they need to
hear. Mostly, it comes down to this: if your boss wants to screw
you, find another job, or be prepared to be screwed in more ways
than one. If your boyfriend is a deadbeat who won't work and
borrows money from you, get a new boyfriend, even if he is the
'love of your life.' If your girlfriends are talking trash about
you, confront them, have a good cry, and start over. If your
girlfriend is screwing your boyfriend, get rid of both of them. If
you are screwing your girlfriend's boyfriend, stop -- dump him,
confess to her, and don't do it again.'' She was ticking these off
on her fingers like a shopping list.

``It sounds a little repetitive,'' he said. He wondered if she was
making it up, or possibly delusional. Could there really be a show
that every factory girl listened to that he'd never heard of? He
thought of how little the factory girls in Shilong New Town had
talked to him when he worked as a security guard and decided that
yes, it was totally possible.

``It's very repetitive, but we all like it that way, my girls and
me. Some problems are universal. Some things you just can't say too
often. Anyway, that's not all there is to it. We have variety! We
have you!''

``Me,'' he said. ``You're going to put me on a show with all these
girls on it? Why? Won't that make the police want to get me even
more?''

``Darling, the police already want you. Remember the video. Your
face is everywhere. The more famous you are, the harder it will be
for them to arrest you. Trust me.''

``How can you be sure? Have you ever done this before?''

``Every day,'' she said, eyes wide. ``I'm my own case study. The
police have been after me for two years now, and I've stayed out of
their clutches. I do it by being too popular to catch!''

``I don't think I understand how that works,'' he said.

She looked at the face of her phone. ``We've only got a minute.
Here, quickly, I'll explain: if you're a fugitive, being poor is
hard. Even harder than for non-fugitives. It's expensive being on
the run. You need lots of places to live. Lots of different phones
that you can abandon. You need to be able to pay li --'' bribes --
``and you need to be able to move fast. Being famous means that you
have access to money and favors from a lot of different people. My
listeners keep me going, either through direct donations or through
my advertisers.''

``You have ads? Who would buy an ad on a fugitive's radio show?''

She shrugged. ``The Taiwanese,'' she said. The island of Taiwan had
considered itself separate from China since 1949 but China had
never stopped laying claim to it -- without much success. ``Falun
Gong, sometimes.'' She saw the look of shock on his face. ``Don't
worry, \emph{I'm} not religious. But I'll take their money. They
don't care if I make fun of them on the show, so long as I run
their ads, too.''

He shook his head. ``It's all too strange,'' he said.

She held up her hand for silence and swung down a little mic from
one of the headphones' earpieces. ``Hello, girls!'' she called into
the mic, clicking her mouse. ``It's your best friend here, Sister
Jiandi, the friend you can always rely on, the friend who will
never let you down, the friend you can confide all your secrets in
-- provided you don't mind eight million factory girls finding out
about it!'' She giggled at her own joke. ``Oh, sisters, it's going to
be a good night, I can tell! I have a special surprise for you a
little later, but first, let's talk! Tonight I'm using Amazon
France chat, chat.amazon.fr, so go and sign up now. You'll get me
at jiandi88888. Remember to use a couple of the latest FLG proxies
before you make the call -- and it looks like the translation
services at Yahoo.ru and 123india.in are both unblocked at the
moment, which should make it easier to sign up. Well, what are you
waiting for? Get signed up!''

She clicked something and he heard a blaring ad for Falun Gong
start in his headphone and he slipped one off the side of his head.
Jie swung her mic away and pointed a finger at him. ``Feeling the
magic yet?''

``This is it? This is your big show?''

``Oh yes,'' she said. ``We'll probably have to switch chats three or
four times tonight, as they update the firewall. It's fun! Wait,
you'll see.'' In his ear, the ad was wrapping up and he slipped the
other headphone back into place.

``Talk to me,'' Jie said, her voice full of warmth. It took him a
moment to realize she was talking into her mic, to her audience,
not to him. Her fingers were working the keyboard and mouse.

``Hello?''

``Yes, darling, hello. You're live. Talk, talk! We've only got all
night!''

``Oh, um --'' The voice was female, with a strong Henan accent, and
it was scared.

``It's OK, sweetie, my heart, it's OK. Tell me.'' Jie's voice was a
coo, a purr, a seduction. Her eyes were moist, her lips pursed in a
gesture of pure caring. Lu wanted to tell her \emph{his} secrets.

``It's just that --'' The voice stopped. Crying noises. In the
background, the sounds of a busy factory dorm, girlish calls and
laughter and conversation. Jie made soothing shhh shhh sounds.
``It's my boss,'' the girl said. ``He was so \emph{nice} to me at
first. He said he was taking an interest in me because we are both
from Henan. He said that he would protect me. Show me around the
city. We went to nice places. A restaurant in the stock exchange.
He took me to the Windows on the World park and we dressed up like
ancient warriors.''

``And he wanted something in return, didn't he?''

``I knew he would. I listen to your show. But I thought it would be
different for me. I thought he was different. But he --'' She broke
off. ``After he kissed me, he told me he wanted to do more.
Everything. He told me I owed it to him. That I'd understood that
when I accepted his invitation, and that I would be cheating him if
I didn't --'' She began to cry.

Jie made a face, twirled her finger in an impatient gesture. Lu was
horrified by her callousness. But when the crying stopped, her
voice was again full of compassion and understanding.

``Oh, sweet child, you've been done badly, haven't you? Well, of
course you knew it would happen, but the heart and the head don't
always agree with each other, do they? The question isn't whether
you acted like a fool -- because you did, you acted like a perfect
fool -- the question is what you can do about it now. Am I right?''

``Yes.'' The voice was so tiny and soft he could barely hear it. He
pictured a girl shrunk to the size of a mouse, trembling in fear.

``Well, that's simple. Not easy, but simple. Forfeit your last eight
weeks' wages and walk out of the factory first thing tomorrow
morning. Go down to a job-broker on Xi Li street and find something
-- anything -- that can get you started again. Then you call your
boss's wife -- is he married?''

``Yes.'' The voice was a little bigger now.

``Call his wife and tell her everything. Tell her what he did, what
he said, what you said back. Tell her you're sorry, and tell her
you're sorry her husband is such a sack of rotten, stinking
garbage. Tell her you walked away on the pay he was holding back,
and that you've left your job. And then you start to work again.
And no matter what your new boss says or does, don't go out with
him. Do you understand?''

``Call his wife --''

``Call his wife, walk away from your pay, and start over. There's
nothing else that will work. You can't talk to this man. He has
raped you -- that's what it is, you know, when someone in power
coerces you into sex, it's rape, just rape -- and he'll do it again
and again and again. He'll do it to the other girls in the factory.
You tell as many as you can why you're leaving. In fact, you tell
me what factory you work in and the name of your boss, right now,
and then millions and millions of girls will know about it, too.
They'll steer clear of this dog, and maybe you'll save a few souls
with your bravery. What do you say?''

``You want me to name my boss? Now? But I thought this was
confidential --''

``You don't \emph{have} to. But do you want another girl to go
through what you just went through? What do you think would have
happened if you had heard another girl speak his name on this show,
last month, before you went out with him. What will you do? Will
you save your sisters from the pain you're in? Or will you protect
your bruised ego and let the next girl suffer, and the next?'' She
waited a moment. The girl on the phone said nothing, though the
sounds of people moving around the dorm could still be faintly
heard. Lu imagined her under her blanket on her bunk, hand over the
mouthpiece of her phone, whispering her secrets to millions of
girls. What a strange world. ``Well?''

``I'll do it,'' the girl said.

``What's that? Say it loud!''

``I'll do it!'' the girl said, and let out a little laugh, and the
laugh was echoed by the girlish voices near her, as the girls in
her dorm realized that the confession they'd been listening into on
their computers and phones and radios had been emanating from a
bunk in their midst. There was a squeal of feedback as one of the
radios got too close to the phone, and Jie's fingers clicked at the
keyboard, squelching the feedback but somehow leaving the other
squeals, the girlish squeals. They were cheering her, the girls in
the dorm, cheering her and chanting her name, her real name, now on
the radio, but it didn't matter, because the girl was laughing
harder than ever.

``It's Bau Peixiong,'' she said, laughing. ``Bau Peixiong at the
HuaXia sports factory.'' She laughed, a liberated sound.

``OK, OK, girls,'' Jie said into her mic, in a commanding tone. The
voices quieted. ``Now, your sister has just made a sacrifice for all
of you, so you need to help her. She needs money -- your pig of a
boss won't give her the eight weeks' pay he's holding onto,
especially not after she calls his wife. She needs help packing,
help finding a job. Someone there is thinking of changing jobs,
someone there knows where there's a job for this girl. Tell her.
Help her move out. Help her find the new job. This is your duty to
your sister. Promise me!''

From the phone, a babble of girls saying, ``I promise! I promise!''

``Very good,'' Jie said. ``Now, stay tuned friends, for soon I will be
unveiling a wonderful surprise!'' A mouseclick and then there was
another ad, this time for a company that provided fake credentials
for people looking for work, guaranteed to pass database lookups.
Both of them slipped their headphones off and Jie drained her
water-glass, a little trickle sliding down her chin and throat. Lu
suppressed a groan. She was \emph{so} beautiful, and all that power
and confidence --

``That was a pretty good opener, wasn't it?'' she said, raising her
eyebrows at him.

``Is it like this all the time?''

``Oh, that was a particularly good one. But yes, most nights it goes
like that. Six or seven hours' worth of it. You still think it'd
get repetitious?''

``I can see how that would stay interesting.''

``After all, you kill the same monsters over and over again all
night long, don't you? That must be pretty dull.''

He considered this. ``Not really,'' he said. ``It's the teamwork, I
guess. All of us working together, and it's not really the same
every time -- the games vary the monster-spawning a lot. Sometimes
you get really good drops, too -- that can be very exciting! You're
going down a corridor you've cleared a dozen times, and you
discover that this time it's filled with 200 vampires and then one
of them drops an epic sword, and it's not boring at all anymore.''
He shrugged. ``My guildie Matthew says it's intermittent
reinforcement.''

She held up a finger and said, ``Hold on to that,'' and clicked and
started talking into her mic again, taking a call from another
factory girl, this one more angry than sad. ``I had a friend who was
selling franchises for a line of herbal remedies,'' she said, and
Jie rolled her eyes.

``Go on,'' she said. ``Sounds like a great opportunity.'' The sarcasm
in her voice was unmistakable.

``That's what I thought,'' the girl said. She sounded like she wanted
to punch something. ``At first I thought it was about selling the
herbal remedies, and I liked that, because my mother always gave me
herbs when I was sick as a girl, and I thought that a lot of the
girls here would want to buy the remedies too because they missed
home.''

``Yes,'' Jie said. ``Who wouldn't want to remember her mommy?''

``Exactly! Just what I thought. And my friend told me about how much
money I could make, but not from selling the herbs! She said that
selling the herbs would be my 'downliners' job, and that I would
manage them. I would be a boss!''

``Who wouldn't want to be a boss?''

``Right! She said that she was recruiting me to be in the top layer
of the organization, and that I would then go and recruit two of my
friends to be my salespeople. They'd each pay me for the right to
sign up more downliners, and that all the downliners would buy
herbs from me and then I would get a share of all their profits.
She showed me how if my two downliners signed up two more, and each
of \emph{them} signed up two more, and so on, that I would have
hundreds of downliners working for me in just a few days! And if I
only got a few RMB from each one, I'd be making thousands every
month, just for signing up two people.''

``A very generous friend,'' Jie said, and though she sounded like she
was joking, she wasn't smiling.

``Yes, yes! That's what I thought. And all I needed to do was pay
her one small fee for the right to sell downline, and she would
supply me with herbs and sales kits and everything else I needed.
She said that she was signing me up because I was Fujianese, like
her, and she wanted to take care of me. She said I should find
girls who were still back in the village, girls I'd gone to school
with, and call them and sign them up, because they needed to make
money.''

``Why would girls in the village need herbal remedies? Wouldn't they
have their mothers?''

That stopped the angry, fast-talking girl. ``I didn't think of
that,'' she said, at last. ``It seemed like I was going to be a hero
for everyone, and like I would escape from the factory and get
rich. My friend said she was going to quit in a few weeks and get
her own apartment. I thought about moving out of the dorm, having
money to send home --''

``You dreamed about money and all that it could buy you, but you
didn't devote the same attention to figuring out whether this thing
could possibly work, right?''

Another silence. ``Yes,'' she said. ``I have to say that this is
true.''

``And then?''

``It started OK. I sold a few downlines, but they were having
trouble making their downline commitments. And then my friend, she
started to ask me for her percentage of my income. When I told her
I wasn't receiving the income my downliners owed me, she changed.''

``Go on.'' Jie's eyes were fixed on the wall behind Lu's head. She
was in another world, it seemed, picturing the girl and her
problem.

``She got angry. She said that I had made a commitment to her, and
that she had made commitments to her uplines based on this, and
that I would have to pay her so that she could pay the people she
owed. She made me feel like I'd betrayed her, betrayed the
incredible opportunity. She said I was just a simple girl from a
village, not fit to be a business-woman. She called me all day,
over and over, screaming, 'Where's my money?'''

``So what did you do?''

``I finally went to her. I cried. I told her I didn't know what to
do. And she told me that I knew, but that I didn't have the courage
to do it. She told me I had to go to my downliners, get tough on
them, get the money out of them. And if they wouldn't pay, I'd have
to get the money some other way: from my parents, my friends, my
savings. I could get new downliners next month.''

``And so you called up your downliners?''

``I did.'' She drew in a heaving breath. ``At first, I was gentle and
kind to them, but my friend called me over and over again, and I
got angry. Angry at them, not at her. It was their fault that I was
having to spend all this time and energy, that I couldn't sleep or
eat. And so I got meaner. I threatened them, begged them, shouted
at them. These two girls, they were my old friends. I'd known them
since we were little babies. I knew their secrets. I threatened to
call my friend's father and tell him that she had let a boy take
naked pictures of her when she was 15. I threatened to tell my
other friend's sister that she had kissed her boyfriend.''

``Did they pay what they owed you?''

``At first. The first month, they paid. The next month, though, I
had to call them and shout at them some more. It was like I was
sitting above myself, watching a crazy stranger say these terrible
things to my old, old friends. But they paid again. And then, in
the third month --'' She stopped abruptly. The silence swelled. Lu
felt it getting thicker, staticky.

``What happened?''

``Then one friend ate rat poison.'' Her voice was a tiny, far-away
whisper. More silence. ``I had told her that I would go to her
father and -- and --'' Silence. ``It was how her mother had committed
suicide when we were both small. The same kind of poison. Her
father was a hard man, an Old One Hundred Names who had lived
through the Cultural Revolution. He has no mercy on him. When she
couldn't get the money, she stole it. Got caught. He was going to
find out. And if he didn't, I would tell him about the photos she'd
taken. And she couldn't face that. I drove her to kill herself. It
was me. I killed her.''

``She killed herself,'' Jie said, her voice full of compassion. ``It's
the women's disease in China. We're the only country in the world
where more women commit suicide than men. You can't take the blame
for this.'' She paused. ``Not all of it.''

``That's not all,'' the girl said, all the anger gone out of her
voice now, nothing left behind but distilled despair.

``Of course not,'' Jie said. ``You still owe for this month. And next
month, and the month after.''

``My friend, the one who brought me into this, she knows\ldots{}
\emph{things}\ldots{} about me. The kind of things I knew about my
friends. Things that could cost me my job, my home, my
boyfriend\ldots{}''

``Of course. That's how cuanxiao works.'' Lu had heard the term
before. ``Network sales,'' is what it meant. There was always someone
trying to sell you something as part of a cuanxiao scheme. He used
to laugh at it. Now it seemed a lot more serious. ``And somewhere,
upline from here, there's someone else in the cuanxiao, who has
something on her. And there are preachers who can convince you that
you'll make a fortune with cuanxiao, and that you just need to
inspire your family and friends.''

``You know him? Mr Lee. My friend took me to a meeting. Mr Lee
seemed like he was on fire, and he made me so sure that I would
become rich if only --''

``I don't know Mr Lee. But there are hundreds of Mr Lees in Guandong
province. You know what we call them? Pharoahs, like the Egyptian
kings they buried in pyramids. That's because they sit on top of a
pyramid of fools like you. Beneath the pharoah, there's a pair of
downliners, and beneath them, two pairs, and beneath them, two more
pairs, and so on, all passing money up the power to some feudal
idiot from the countryside who knows how to talk a good line and
has never worked a day in his life. Did you ever study math?''

``I got a gold medal in our canton's Math Olympiad!''

``That's very good! Math is useful in this world. Let's do a little
math. If each level of the pyramid has double the number of members
of the previous level, how many members are there on the 10th level
of the pyramid?''

``What? Oh. Um. 2 to the 10. That's --'' \emph{1024} Lu thought to
himself. ``1024, right?''

``Exactly. How many on the 30th level?''

``Um\ldots{}''

Lu pulled out his phone, used the calculator, did some figuring.

``Um\ldots{}.''

``Oh, just guess.''

``It's big. A hundred thousand? No! About five hundred thousand.''

``You should give your medal back, sister. It's over a billion.'' Jie
tapped some numbers into her keyboard. ``1,073,741,824 to be
precise. There's 1.6 billion people in China. Your herb salespeople
were supposed to recruit new downliners every two weeks. At that
rate --'' She typed some more. ``It would be just over a year before
every person in China was working in your pyramid, even the tiny
babies and the oldest grannies.''

``Oh.''

``You know about network selling, you must have. What year are you?''
Meaning, how many years since you left the village?

``Four,'' the girl admitted. ``I did know it. Of course. But I thought
this was different. I thought because there was a real product and
because it was only two people at a time --''

``I don't think you thought about any of that, sister. I think you
thought about having a big apartment and a lot of money. Isn't that
right?''

``There was money, though! It was working for weeks! My friend had
made so much --''

``What level of the pyramid was she on? 10? 20? When you're stealing
from the new people to pay the old people, it's a good deal for the
old people. Not so good for the new people. People like you or your
downliners.''

``I'm a fool,'' the girl said. ``I'm a monster! I destroyed my
friends' lives!'' She was sobbing now, screaming out the confession
for millions of people to hear.

``It's true,'' Jie said, mildly. ``You're a fool and a monster, just
like thousands of other people. Now what are you going to do about
it?''

``\emph{What can I do?}''

``You can stop snivelling and pull yourself together. Your friend,
the one who recruited you? Someone's holding something over her,
the way that she was holding something over you. Sit down with her,
and do whatever it takes to get her out. The most evil thing about
these pyramids is that they turn friend against friend, make us
betray the people we love to keep from being betrayed ourselves.
Even if you're one of the lucky few at the top who makes some money
from it, you pay the price of your integrity, your friendships and
your soul. The only way to win is not to play.''

``But --''

``But, but, but! Listen, foolish girl! You called me tonight because
your soul is stained with the evil that you did. Did you think I
would just tell you that it's all right, you did what you had to
do, no blame on you? No! You know me, I'm Jiandi. I don't grant
absolution. I tell you what you must do to pay for your crimes. You
don't get to confess, feel better and walk away. You have to do the
hard work now -- you have to set things to right, help your
friends, restore your integrity and conscience. Do you hear me?''

``I hear you.'' Quiet, meek.

``Say it louder.'' She snapped it like a general giving an order.

``I hear you!''

``LOUDER!''

``I HEAR YOU!''

``Good!'' She laughed and rubbed at one ear. ``I think they heard you
in Macau! Good girl. Go and do right now!''

And she clicked something and another ad rolled in Lu's headphones.
He took them off, found that his eyes were moist with tears. ``That
poor girl,'' he said.

``There's thousands more like her,'' Jie said. ``It's a sickness, like
gambling. It comes from not understanding numbers. They all win
their little math medals, but they don't believe in the numbers.
Now, you were about to tell me about some kind of reinforcement.''

``Intermittent reinforcement,'' he said. ``My friend Matthew, he leads
our guild, he told me about it. It comes from experiments with
rats. Imagine that you have a rat who gets some food every time he
pushes a lever. How often do you think he pushes the lever?''

``As often as he's hungry, I suppose. I kept mice once -- they knew
when it was time for food and they'd rush over to the corner of the
cage that I dropped their seeds and cheese into.''

``Right. Now, what about a lever that gives food every fifth time
they press the lever?''

``I don't know -- less?''

``About the same, actually, After a while, the rats figure out that
they need five presses for a food pellet and every time they want
feeding, they wander over and hit it five times. Now, what about a
lever that gives food out at random? Sometimes one press, sometimes
one hundred presses?''

``They'd give up, right?''

``Wrong! They press it like crazy, All day and all night. It's like
someone who wins a little money in the lottery one week and then
plays every week afterward, forever. The uncertainty drives them
crazy, it's the most addictive system of all. Matthew says it's the
most important part of game design -- one day you manage to kill a
really hard NPC with a lucky swing, and it drops some incredibly
epic item, and you make more money in ten seconds than you made all
week, and you have to keep going back to that spot, looking for a
monster like it, thinking it'll happen again.''

``But it's random, right?''

``I'm not sure,'' he said. ``Matthew says it is. I sometimes think
that the game company deliberately messes up the odds so that when
you're just about to quit, you get another jackpot.'' He shrugged.
``That's what I'd do, anyway.''

``If it's random, it shouldn't make any difference what you do and
where you play. If you flip a coin ten times and it comes up heads
ten times in a row, you've got exactly the same chance of it coming
up heads an eleventh time than if it had come up all tails, or half
and half.''

``Matthew says stuff like that all the time. He says that although
it may be unlikely that you'll get ten heads in a row, each flip
has exactly the same chance.''

``Matthew sounds like he knows his math.''

``He does. You should meet him sometime.'' He swallowed. ``If he ever
gets out of jail, that is.''

``Oh, we'll have to do something about that.''

She handled six more calls, running the show for another two hours,
breaking for commercials and promising all her listeners the most
exciting event of their lifetime if they just hung in. At first, Lu
listened attentively, but his head hurt and he was so tired, and
eventually he slumped in his seat and dozed, drifting in and out of
dreams as he listened to Jie berating the foolish factory girls of
South China.

He woke to a sprinkle of ice-water on his face, gasped and sat up,
opening his eyes just in time to see Jie dancing back away from
him, laughing, her face glowing with excitement. ``I \emph{love}
doing this show!'' she said. ``You're up next, handsome!''

He looked at his phone and realized that he'd dozed for an hour
more, and that it was well past supper time. His stomach rumbled.
Jie had taken off her shoes and socks and unbuttoned the top two
buttons on her red blouse. Her hair was down and her makeup was
smudged. She looked like she was having the time of her life.

``Wha?'' his head throbbed and it tasted like something had used his
mouth for a toilet.

``Come \emph{on},'' she said, and moved close again, snapping his
headphones on. ``It's coming up on 8PM. This is when my listenership
peaks. They're back from dinner, they're finished gossiping, and
they're all sitting on their beds, tuning in on their computers and
phones and radios. And I've been hyping you for \emph{hours}. Every
pretty girl in the Pearl River Delta is waiting to meet you, are
you ready?''

``I -- I --'' He suddenly couldn't find his tongue. ``Yes!'' he
managed.

``Get your headset on,'' she called, dashing around to her side of
the desk and pouncing on her seat. ``We're live in 10, 9, 8\ldots{}''

He fumbled with his headset, swung the mic down, reached for the
water glass and gulped down too much, choked, tried to keep it in,
choked more, spilled water all down his front. Jie laughed aloud,
gulping it down as she spoke into her mic.

``We're back, we're back, we're back, and now sisters, I have the
special surprise I've been promising you all night! A knight of the
people, a hero of the factory, a killer who has hunted pirates in
space and dragons in the hills, a professional gold-farmer named
--'' She broke off. ``What name shall I call you by, hero?''

``Oh!'' He thought for a second. ``Tank,'' he said. ``It's the kind of
player I am, the tank.''

``A tank!'' She giggled. ``That's just perfect. Oh, sisters, if only
you could see this big, muscled tank I have sitting here in my
studio. Let me tell you about Tank. I was watching a little video
this afternoon, and like many of you, I found myself watching
something amazing: dozens of boys, lined up outside an Internet
cafe, blinking and pale as newborn mice in the daylight. It seemed
that they were a different kind of factory boy, the legendary gold
farmers of Shenzhen, and they were demanding a better job, better
pay, better conditions, and an end to their vicious, greedy bosses.
Does that sound familiar, sisters?

``The police arrived, the dirty jingcha, with their helmets and
clubs and gas, cowards with their faces hidden and their brutal
weapons in hand to fight these boys who only wanted justice. But
did the boys flee? No! Did they go back to their jobs and apologize
to their bosses? No! The mouse army stood its ground, claimed their
workplace as their rightful home, the place their work paid for.
And what did the jingcha do? Tell me, Tank, what did they do?''

Lu looked at her like she was crazy. She made urgent hand-gestures
at him as the silence stretched. ``I, that is, they beat us up!''

``They certainly did! Sisters, download this video now, please!
Watch as the jingcha charge the boys of Shenzhen, breaking their
heads, gassing them, clubbing them. And now, focus on one brave lad
off to the left, right at the 14:22 mark. Strong chin, wide eyes, a
little freckles over his nose, hair in disarray. See him stand his
ground through the charge with his comrades by his side? See the
jingcha with his club who comes upon the boy from behind and hits
him in the shoulder, knocking him down? See the club come up again
and land on the poor boy's head, the blood that flies from the
wound?

``That, sisters, is Tank, the boy sitting across from me, bloodied
but unbowed, brave and strong, standing up for the rights of
workers --'' She dissolved into giggles. Lu giggled too, he couldn't
help it. ``Oh, sorry, sorry. Look, he's a very nice boy, and not bad
to look at, and the jingcha laid into his head and shoulder like
they were tenderizing a steak, and all he was doing was insisting
that he had the right to work like a person and not an animal. And
he's not alone. They call it 'The People's Republic of China,' but
the people don't get any say in the way it's run. It's all
corruption and exploitation.

``I thought the video was amazing, a real inspiration. And then I
saw him, our Tank, wandering dazed and bloody through --'' she broke
off. ``Through a location I will not disclose, so that the jingcha
won't know which video footage they need to review. I saw him and I
told him I wanted to introduce him to you, my friends, and then he
told me the most amazing story I've heard, and you \emph{know} I
hear a lot of amazing stories here every night. A story about a
global movement to improve the lot of workers everywhere, and I
hope that's the story he'll tell us tonight. So, Tank, darling,
start with your injuries. Could you describe them to our friends
out there?''

And Lu did, and then he found himself going from there into the
story of how he came to be a gold farmer, what life was like for
him, the stories Matthew had told him about how Boss Wing had
forced him and his friends to go back to work in his factory,
talking and talking until the water was gone and his mouth was dry,
and mercifully, she called for another commercial.

He sagged into his chair while she got him some more water. ``You
should see the chat rooms,'' she said. ``They're all in love with
you, 'Tank'. The way you rescued those girls' belongings in Shilong
New Town! You're their hero. There are dozens of them who claim
that they were there on that day, that they saw you climbing the
fence. Listen to this, 'His muscles rippled like iron bands as he
clambered up the fence like a mighty jungle creature\ldots{}''' He
snorted water up his sinuses, and Jie gave his bicep a squeeze.
``You need to work out some more, Jungle Creature, your muscles have
gone all soft!''

``How do you have message boards? Don't they block them?''

``Oh, that's easy,'' she said. ``We just pick a random blog out there
on the net, usually one that no one has posted to in a year or two,
and we take over the comment board on one of its posts. Once they
block it -- or the server crashes -- we switch to another one. It's
easy -- and fun!''

He laughed and shook his head, which set his headache going again.
He winced and squeezed his head between his hands. ``Sheer genius!''

Now the commercial was ending, and they both sat down quickly in
their chairs and swung their mics into place. Lu was getting good
at this now, the talk coming to him the way it did when he was
chatting with his guildies. He'd always been the storyteller of the
bunch.

And the story went on -- he told of how the Webblies had come to
him and his guildies in game, had talked to them about the need for
solidarity and mutual aid to protect themselves from bosses, from
players who hunted gold-farmers, from the game company.

``They want to unite Chinese workers,'' Jie said, nodding sagely.

``No!'' He surprised himself with his vehemence. ``Uniting Chinese
workers would be useless. With gold farming, the work can just move
to Indonesia, Vietnam, Cambodia, India -- anywhere workers aren't
organized. It's the same with all work now -- your job can move in
no time at all to anywhere you can build a factory and dock a
container ship. There's no such thing as 'Chinese' workers anymore.
Just workers! And so the Webblies organize all of us, everywhere!''

``That's a lot of workers,'' she said. ``How many have you got?''

He hung his head. ``Jiandi,'' he said. ``We can all see the counter,
and we all cheer when it goes up by a few hundred, but we're a long
way off.''

``Oh, Tank,'' she said. ``Don't be discouraged. Tens of thousands of
people! That's fantastic -- and I'm sure we can get a few members
for you. How can my listeners join up?''

``Eh? Oh!'' He struggled to remember the procedure for this. ``You
need to get at least 50 percent of your co-workers to agree to sign
up, and then we certify the union for your whole factory.''

``Ay-yah! 50 percent! The big factories have 50,000 workers! How do
you do that?''

He shrugged. ``I'm not sure,'' he said. ``We've been mostly signing up
small game-factories, there's not many bigger than 200 workers. It
has to be possible, though. Trade unions all over the world have
organized factories of every size.'' He swallowed, understanding how
lame he sounded. ``Look, this is usually Matthew's side of things.
He understands all of it. I'm just the tank, you understand? I
stand in the front and soak up all the damage. And you can't talk
to Matthew because he's in jail.''

``Ah yes, jail. Tell us about what happened today.''

So he told them the story of the battle, all those millions of
girls out there in the towns of Guangdong, and he found
himself\ldots{}transported. Taken away back to the cafe, the shouting,
the police and the screams, his voice drifting to his ears from a
long way off through the remembered shouts in his ears. When he
stopped, he snapped back to reality and found Jie staring at him
with wet eyes and parted lips. He looked at his phone. It was
nearly midnight.

He shrugged, dry mouthed. ``I -- Well, that's it, I suppose.''

``Wow,'' Jie breathed, and cued up another commercial. ``Are you OK?''

``My head feels like it's being crushed between two heavy rocks,'' he
said. He shifted his butt in his chair and winced. ``And my
shoulder's on fire.''

``I've really kept you up,'' she said. ``We're almost done here,
though. You're a really tough bastard, you know that?''

He didn't feel tough. Truth be told, he felt pretty terrible about
the fact that he'd gotten away while his guildies had all been
locked up. Logically he knew that they wouldn't benefit from him
being jailed alongside of them, but that was logic, not feelings.

``OK,'' she said. ``We're back. What a \emph{story}! Sisters, didn't I
tell you I had something special tonight? Alas, it's nearly time to
go -- we all need some sleep before we go back to work in the
morning, don't we? Just one more thing:
\emph{what are we going to do about this?}''

Suddenly, she wasn't sleepy and soothing. Her eyes were wide, and
she was gripping the edge of her desk tightly. ``We come here from
our villages looking to do an honest job for decent pay so that we
can help our families, so that we can live and survive. What do we
get? Slimy perverts who screw us on the job and off! Bastard
criminals who destroy anyone who challenges their rackets! Cops who
beat us and put us in jail if we dare to challenge the status
quo!''

``Sisters, it \emph{can't go on}! Tank here said there's no such
thing as a Chinese worker anymore, just a worker. I hadn't heard of
these Webblies of his before tonight, and I don't know if they're
any better than your boss or the thief running the network sales
rip-off next door, and I don't care. If there are workers around
the world organizing for a better deal, I want to be a part of it,
and so do you!

``I'll tell you what's going to happen next. Tank and I are going to
find the Webblies and we're going to plan something big. Something
\emph{huge}! I don't know what it will be, but it's going to change
things. There's \emph{millions} of us! Anything we do is
\emph{big}.

``I have a confession to make.'' Her voice got quieter. ``A sin to
confess. I do this show because it makes me money. A lot of money.
I have to spend a lot to stay ahead of the zengfu, but there's
plenty left over. More than you make, I have to confess. It's been
a long time since I was as poor as a factory girl. I'm practically
rich. Not boss-rich, but rich, you understand?

``But I'm with you. I didn't start this show to get rich. I started
it because I was a factory girl and I cared about my sisters. We've
been coming to Guangdong Province since Deng Xiaoping changed the
rules and made the factories here grow. It's been generations,
sisters, and we come, we poor mice from the country, and we are
ground up by the factories we slave in. For every Yuan we send
home, our bosses put a hundred in their pockets. And when we're
done, then what? We become one of the old grannies begging by the
road.

``So listen in tomorrow. We're going to find out more about these
Webblies, we're going to make a plan, and we're going to bring it
to you. In the meantime, don't take any crap off your bosses. Don't
let the cops push you or your sisters and brothers around. And be
good to each other -- we're all on the same side.''

She clicked her mouse and flipped the lid down on her laptop.

``Whew!'' she said. ``What a \emph{night}!''

``Is your show like this every night?''

``Not this good, Tank. You certainly improved things. I'm glad I
kidnapped you from the train station.''

``I am too,'' he said. He was so tired. ``I guess I'll call you
tomorrow about the next show? Maybe we could meet in the morning
and try to reach the Webblies or find a way to try to call my
guildies and see if they're all still in jail?''

``Call me? Don't be stupid, Tank. I'm not letting you out of my
sight.''

``It's OK,'' he said. ``I can find somewhere to sleep.'' When he'd
first arrived in Shenzhen, he'd spent a couple nights sleeping in
parks. He could do that again. It wasn't so bad, if it didn't rain
in the night. Had there been clouds that day? He couldn't
remember.

``You certainly can -- right through that doorway, right there.'' She
pointed to the bedroom.

He was suddenly wide awake. ``Oh, I couldn't --''

``Shut up and go to bed. You've got a head injury, stupid. And
you've just given me hours of great radio show. So you need it and
you've earned it. Bed. Now.''

He was too tired to argue. He stumbled a little on the way to bed,
and she swept the clothes and toys and handbags from the bed onto
the floor just ahead of him. She pulled the sheet over him and
kissed him on the forehead as he settled in. ``Sleep, Tank,'' she
whispered in his ear.

He wondered dimly where she would sleep, as she left the room and
he heard her typing on her computer again. He fell asleep with the
sound of the keys in his ears.

He barely woke when she slid under the covers with him, snuggled up
to him and began to snore softly in his ear.

But he was wide awake an hour later when ten police cars pulled up
out front of Houhai's buildings, sirens blaring, and a helicopter
spotlight bathed the entire building in light as white as daylight.
She went rigid beside him under the covers and then practically
levitated out of the bed.

``Twenty seconds,'' she barked. ``Shoes, your phone, anything else you
need. We won't come back here.''

Lu felt obscurely proud of how calm he felt as he stood up and, in
an unhurried, calm fashion, picked up his shoes -- factory workers'
tennis shoes, cheap and ubiquitous -- and laced them up, then
pulled on his jacket, then moved efficiently into the living room,
where Jie was hosing solvent over all the flat surfaces in the
room. The smell was as sharp as his headache, and intensified it.

She nodded once at him, and then nodded at another pressure-bottle
of solvent and said, ``You do the bathroom and the bedroom.'' He did,
working quickly. He guessed that this would wipe away anything like
a fingerprint or a distinctive kind of dirt. He was done in a
minute, or maybe, less, and she was at his elbow with a ziploc
baggie full of dust. ``Vacuumed out of the seas of the Hong
Kong-Shenzhen train,'' she said. ``Skin cells from a good million
people. Spread it evenly, please. Quickly now.''

The dust got up his nose and made him sneeze, and sunk into the
creases of his palms, and it was all a little icky, but his head
was clear and full of the sirens and the helicopter's thunder. As
he scattered the genetic material throughout, he watched Jie
popping the drive out of her computer and dropping the slender
stick down her cleavage, and \emph{that} finally broke through his
cool. Suddenly, he realized that he'd spent the night sleeping next
to this beautiful girl, and he hadn't even \emph{kissed} her, much
less touched those mysterious and intriguing breasts that now
warmly embraced an extremely compromising piece of storage media, a
sliver of magnetic media that could put them both in jail forever.

She looked around and ticked off a mental checklist on her finger.
Then she snapped a decisive nod and said, ``All right, let's go.''
She led him out into the corridor, which was brightly lit and
empty, leaving him feeling very exposed. She pulled a short prybar
out of her purse and expertly pried open the steel door on a
fuse-panel by the elevators, revealing neat rows of black plastic
breaker switches. She fished in her handbag again and came out with
a disposable butane lighter, which she lit, applying the flame to a
little twist of white vinyl or shiny paper protruding like a
pull-tab from an unobtrusive seam in the panel. It sizzled and
flashed and a twist of black smoke rose from it and then the paper
burned away, the spark disappearing into the panel.

A second later, the entire panel-face erupted in a shower of
sparks, smoke and flame. Jie regarded it with satisfaction as black
smoke poured out of the plate. Then all the lights went out and the
smoke alarms began to toll, a bone-deep dee-dah dee-dah that
drowned out the helicopter, the sirens.

She clicked a little red LED light to life and it bathed her face
in demonic light. She looked very satisfied with herself. It made
Lu feel calm.

``Now what?'' he said.

``Now we stroll out with everyone else who'se running away from the
fire alarms.''

All through the building, doors were opening, bleary families were
emerging, and smoke was billowing, black and acrid. They headed for
the staircase, just behind the Bound-Foot Granny who they'd met the
day before. In the stairwell, they met hundreds, then thousands
more refugees from the building, all carrying armloads of precious
possessions, babies, elderly family members.

At the bottom, the police tried to corral them into an orderly
group in front of the building, but there were too many people, too
much confusion. In the end, it was simple to slip through the
police lines and mingle with the crowd of gawkers from nearby
buildings who'd turned out to watch.

\tb

\shopad{This scene is dedicated to Vancouver's multilingual Sophia Books, a diverse and exciting store filled with the best of the strange and exciting pop culture worlds of many lands. Sophia was around the corner from my hotel when I went to Van to give a talk at Simon Fraser University, and the Sophia folks emailed me in advance to ask me to drop in and sign their stock while I was in the neighborhood. When I got there, I discovered a treasure-trove of never-before-seen works in a dizzying array of languages, from graphic novels to thick academic treatises, presided over by good-natured (even slapstick) staff who so palpably enjoyed their jobs that it spread to every customer who stepped through the door.}
{\href{http://www.sophiabooks.com/}{Sophia Books}: 450 West Hastings St., Vancouver, BC Canada V6B1L1 +1 604 684 0484}

Whether you're a revolutionary, a factory owner, or a little-league
hockey organizer, there's one factor you can't afford to ignore:
the CoaseCost.

Ronald Coase was an American economist who changed everything with
a paper he published in 1937 called ``The Theory of the Firm.''
Coase's paper argued that the real business of \emph{any}
organization was getting people organized. A religion is a system
for organizing people to pray and give money to build churches and
pay priests or ministers or rabbis; a shoe factory is a system for
organizing people to make shoes. A revolutionary conspiracy is a
system for organizing people to overthrow the government.

Organizing is a kind of tax on human activity. For every minute you
spend \emph{doing stuff}, you have to spend a few seconds making
sure that you're not getting ahead or behind or to one side of the
other people you're doing stuff with. The seconds you tithe to an
organization is the CoaseCost, the tax on your work that you pay
for the fact that we're human beings and not ants or bees or some
other species that manages to all march in unison by sheer
instinct.

Oh, you can beat the CoaseCost: just stick to doing projects that
you don't need anyone else's help with. Like, um\ldots{}Tying your
shoes? (Nope, not unless you're braiding your own shoelaces).
Toasting your own sandwich (not unless you gathered the wood for
the fire and the wheat for the bread and the milk for the cheese on
your own).

The fact is, everything you do is collaborative -- somewhere out
there, someone else had a hand in it. And part of the cost of what
you're doing is spent on making sure that you're coordinating
right, that the cheese gets to your fridge and that the electricity
hums through its wires.

You can't eliminate Coase costs, but you can lower it. There's two
ways of doing this: get better organizational techniques (say,
``double-entry book-keeping,'' an Earth-shattering 13th-century
invention that is at the heart of every money-making organization
in the world, from churches to corporations to governments), or get
better technology.

Take going out to the movies. It's Friday night, and you're
thinking of seeing a movie, but you don't want to go alone. Imagine
that the year was 1950 -- how would you solve this problem?

Well, you'd have to find a newspaper and see what's playing. Then
you'd have to call all your friends' houses (no cellular phones,
remember!) and leave messages for them. Then you'd have to wait for
some or all of them to call you back and report on their movie
preferences. Then you'd have to call them back in ones and twos and
see if you could convince a critical mass of them to see the same
movie. Then you'd have to get to the theater and locate each other
and hope that the show wasn't sold out.

How much does this cost? Well, first, let's see how much the movie
is worth: one way to do that is to look at how much someone would
have to pay you to convince you to give up on going to the movies.
Another is to raise the price of the tickets steadily until you
decide not to see a movie after all.

Once you have that number, you can calculate your CoaseCost: you
could ask how much it would cost you to pay someone else to make
the arrangements for you, or how much you could earn at an
after-school job if you weren't playing phone tag with your
friends.

You end up with an equation that looks like this:

[Value of the movie] - [Cost of getting your friends together to
see it] = [Net value of an evening out]

That's why you'll do something less fun (stay in and watch TV) but
simple, rather than going out and doing something more fun but more
complicated. It's not that movies aren't fun -- but if it's too
much of a pain in the ass to get your friends out to see them, then
the number of movies you go to see goes way down.

Now think of an evening out at the movies these days. It's 6:45PM
on a Friday night and the movies are going to all start in the next
20-50 minutes. You pull out your phone and google the listings,
sorted by proximity to you. Then you send out a broadcast
text-message to your friends -- if your phone's very smart, you can
send it to just those friends who are in the neighborhood --
listing the movies and the films. They reply-all to one another,
and after a couple volleys, you've found a bunch of people to see a
flick with. You buy your tickets on the phone.

But then you get there and discover that the crowds are so huge you
can't find each other. So you call one another and arrange to meet
by the snack bar and moments later, you're in your seats, eating
popcorn.

So what? Why should anyone care how much it costs to get stuff
done? Because the CoaseCost is the price of being
\emph{superhuman}.

Back in the old days -- the very, very old days -- your ancestors
were solitary monkeys. They worked in singles or couples to do
everything a monkey needed, from gathering food to taking care of
kids to watching for predators to building nests. This had its
limitations: if you're babysitting the kids, you can't gather food.
If you're gathering food, you might miss the tiger -- and lose the
kids.

Enter the tribe: a group of monkeys that work together, dividing up
the labor. Now they're not just solitary monkeys, they're groups of
monkeys, and they can do more than a single monkey could do. They
have transcended monkeyness. They are \emph{supermonkeys}.

Being a supermonkey isn't easy. If you're an individual
supermonkey, there are two ways to prosper: you can play along with
all your monkey pals to get the kids fed and keep an eye out for
tigers, or you can hide in the bushes and nap, pretending to work,
only showing up at mealtimes.

From an individual perspective, it makes sense to be the
lazy-jerk-monkey. In a big tribe of monkeys, one or two goof-offs
aren't going to bankrupt the group. If you can get away with
napping instead of working, and still get fed, why not do it?

But if \emph{everyone} does it, so much for supermonkeys. Now no
one's getting the fruit, no one's taking care of the kids, and
damn, I thought \emph{you} were looking out for the tigers! Too
many lazy monkeys plus tigers equals lunch.

So monkeys -- and their hairless descendants like you -- need some
specialized hardware to detect cheaters and punish them before the
idea catches on and the tigers show up. That specialized hardware
is a layer of tissue wrapped around the top of your brain called
the neo-cortex -- the ``new bark.'' The neo-cortex is in charge of
keeping track of the monkeys. It's the part of your brain that
organizes people, checks in on them, falls in love with them,
establishes enmity with them. It's the part of your brain that gets
thoroughly lit up when you play with Facebook or other social
networking sites, and it's the part of your brain that houses the
local copies of the people in your life. It's where the voice of
your mother telling you to brush your teeth emanates from.

The neocortex is the CoaseCost as applied to the brain. Every sip
of air you breathe, every calorie you ingest, every lubdub of your
heart goes to feed this new bark that keeps track of the other
people in your group and what they're doing, whether they're in
line or off the reservation.

The CoaseCost is the limit of your ability to be superhuman. If the
CoaseCost of some activity is lower than the value that you'd get
out of it, you can get some friends together and \emph{do it},
transcend the limitations that nature has set on lone hairless
monkeys and \emph{become a superhuman}.

So it follows that high Coase costs make you less powerful and low
Coase costs make you more powerful. What's more, big institutions
with a lot of money and power can overcome high Coase costs: a
government can put 10,000 soldiers onto the battlefield with tanks
and food and medics; you and your buddies cannot. So high Coase
costs can limit \emph{your} ability to be superhuman while leaving
the rich and powerful in possession of super-powers that you could
never attain.

And that's the real reason the powerful fear open systems and
networks. If anyone can set up a free voicecall to anyone else in
the world, using the net, then we can all communicate with the same
ease that's standard for the high and mighty. If anyone can create
and sell virtual wealth in a game, then we're all in the same
economic shoes as the multinational megacorps that start the
games.

And if any worker, anywhere, can communicate with any other worker,
anywhere, for free, instantaneously, without her boss's permission,
then, brother, look out, because the CoaseCost of demanding better
pay, better working conditions and a slice of the pie just got a
\emph{lot} cheaper. And the people who have the power aren't going
to sit still and let a bunch of grunts take it away from them.

\tb

\shopad{This scene is dedicated to the MIT Press Bookshop, a store I've visited on every single trip to Boston over the past ten years. MIT, of course, is one of the legendary origin nodes for global nerd culture, and the campus bookstore lives up to the incredible expectations I had when I first set foot in it. In addition to the wonderful titles published by the MIT press, the bookshop is a tour through the most exciting high-tech publications in the world, from hacker zines like 2600 to fat academic anthologies on video-game design. This is one of those stores where I have to ask them to ship my purchases home because they don't fit in my suitcase.}
{\href{http://web.mit.edu/bookstore/www/}{MIT Press Bookstore}: Building E38, 77 Massachusetts Ave., Cambridge, MA USA 02139-4307 +1 617 253 5249}

Coca Cola Games Command Central had been designed by one of the
world's leading film-set designers. The brief had called for a room
that looked like you could use it to run an evil empire, launch an
intergalactic explorer vessel, or command a high-tech mercenary
army. Everything was curved and brushed steel and spotlit and what
wasn't chrome was black, except for accents of cracked, worn-out
black leather harvested from vintage motorcycle jackets. There were
screens everywhere, built into the tables, rolled up in the ceiling
or floor, even one on the back of the door. Any wall could be drawn
on with special pens that used RFIDs and accelerometers to track
their motions and transmit them to a computer that recorded it all
and splashed it across wireless multitouch screens that were
velcroed up all around the room.

Slick photos of Command Central graced the Coca Cola Games
recruiting site and featured in a series of vanity documentaries
CCG had commissioned about itself, looking designer-fresh, filled
with fit, intense, laughing young people in smart clothes doing
intelligent things.

Coca Cola Games Command Central was a lie.

Ten seconds after the game-runners moved into Command Central,
every multitouch had been broken or stolen. The recessed terminals
set into the tables were obsolete before they were installed and
now they suffered an ignominious fate: serving as stands for
cutting-edge laptops equipped with graphics cards that ran so hot,
their fans sounded like jet-engines.

Fifteen seconds later, every flat surface had been covered with
junk-food wrappers, pizza boxes, energy-drink cans, vintage sci-fi
novels, used kleenexes, origami orc-helmets folded out of post-it
notes, snappy hats, and the infinitely varied junky licensed crap
that CCG made from the game, from Pez dispensers to bicycle
valve-caps to trading cards to flick-knives.

Twenty seconds after that, the room acquired the game-runner funk,
a heady mix of pizza-grease strained through armpit pores, cheap
cologne, unwashed hair, vintage Japanese denim, and motor oil.

And now the sleek supergenius lair had become the exclusive
meeting-cave for a tribe of savage, hyper-competitive, extremely
well-paid game-runners, who holed up in there, gnashing their teeth
and shouting at each other for every hour that God sent. No cleaner
would enter the room, and even the personal assistants would only
go so far as the doorway, where they plaintively called out their
bosses' names and dodged the disgusting food-wrappers that were
hurled at their heads by the game-runners, who did not take kindly
to having their work interrupted.

Connor Prikkel had found His People. Technically he was a
vice-president, but no one reported to him, except for a PA whose
job it was to fish him out of Command Central a couple times a
month, steam-clean him in the corporate gym, stick him in the
corporate jet, and fire him into crowds of players and press around
the world to explain -- with a superior smirk -- just how Coca Cola
Games managed to oversee three of the twenty largest economies in
the world.

The rest of the time, Connor's job was to work on his
fingerspitzengefuhl. That was a useful word. It was a German word,
of course. The Germans had words for \emph{everything}, created by
the simple expedient of bashing as many smaller words as you needed
together until you got one monster mouth-murderer like
fingerspitzengefuhl that exactly and precisely conveyed something
no other language could even get close to.

Fingerspitzengefuhl means ``fingertip feel'' -- that feeling you get
when you've got the world resting against the thick cushion of
nerve-endings on the tips of your fingers. That feeling when you've
got a basketball held lightly in your hands, and you know precisely
where the next bounce will take it when you let it go. That feeling
you get when you're holding onto a baby and you can feel whether
she's falling asleep now, or waking up. That feeling you get when
your hands are resting lightly on the handlebars of your bike,
bouncing down a steep hillside, gentle pressure on the brakes,
riding the razor-edged line between doing an end-over and reaching
the bottom safely.

Proprioception is your ability to sense where your body is in space
relative to everything else. It's a sixth sense, and you don't even
know you have it until you lose it -- like when you intertwine your
fingers and thread your hands through your arms and find that you
wiggle your left finger when you mean to move your right; or when
you step on a ghost step at the top of a staircase and your foot
lands on nothing.

Fingerspitzengefuhl is proprioception for the world, an extension
of your sixth sense into everything around you. You have
fingerspitzengefuhl when you can tell, just by the way the air
feels, that your class is in a bad mood, or that your teammate is
upcourt and waiting for you to pass the ball.

Connor's fingerspitzengefuhl meant that he could feel
\emph{everything} that was happening in the games he ran. He could
tell when there was a run on gold in Svartalfaheim Warriors, or
when Zombie Mecha's credits take a dive. He could tell when there
was a huge raiding guild making a run at Odin's Fortress, six
hundred humans embodied in six hundred avs, coordinated by generals
and captains and lieutenants. He could tell when there was a
traffic jam on the Brooklyn Bridge in Zombie Mecha as too many
ronin tried to enter Manhattan to clear out the Flatiron Building
and complete the Publishing Quest.

All this knowledge came to him through his ever-rotating,
ever-changing feeds -- charts, chat-transcripts, server logs, bars
representing load and memory and failover and rate of subscriber
churn and every other bit of changing information from in the game.
They flickered past in a colorful roll, on the display of his
monster widescreen laptop, opacity dialled down to 10 percent in
the windows that sat over his playscreens in which he ran four avs
in both games.

Every gamerunner had a different way of attaining
fingerspitzengefuhl, as personal as the thought you follow to go to
sleep or the reason you fall in love. Some like a \emph{lot} of
screens -- four or five. Some listened to a lot of read-aloud text
and eavesdropped gamechat. Some only watched charts, some only
logs, some only game-screens. Coca Cola Games had hired some
industrial psychologists to try to come and unpick the
game-runners' methods, try to create a system for reproducing and
refining it. They'd lasted a day before being tossed out of Command
Central amid a torrent of abuse and profanities.

The game-runners didn't want to be systematized. They didn't want
to be studied. To be a game-runner was to attain
fingerspitzengefuhl and vice-versa. Game-runners didn't need
shrinks to tell them when they had fingerspitzengefuhl. When you
had fingerspitzengefuhl, you fell into a warm bath, a kind of
hyper-alert coma, in which knowledge flowed in and out of every
orifice at maximum speed. Fingerspitzengefuhl needed coffee and
energy drinks, junk food and loud goddamned music, grunts of your
co-workers. Fingerspitzengefuhl didn't need industrial psychology.

Connor's fingerspitzengefuhl was the best. It guided the
unconscious dance of his fingers on his laptop, guided him to
eavesdrop on the right conversations, to monitor the right action,
to spot the Webblies' fight with the Pinkertons as it began. He
grunted that special grunt that alerted the rest of his tribe to
danger, and stabbed at his screen with a fat finger greased with
pizza-oil. The knowledge rippled through the room like a wave,
bellies and chins wobbling as the whole tribe tuned into the
fight.

``We should pull the plug on this,'' said Fairfax, a designer who'd
worked her way up to Command Central.

``Forget it,'' said Kaden. ``Twenty thousand gold on the Webblies.''

``Two-to-one?'' said Palmer, the number two economist, who had earned
his PhD but hadn't invented the Prikkel Equations.

``No bets,'' Connor said. ``Just watch the play.''

``You're such a combat freak,'' said Kaden. ``You chose the wrong
specialty. You should have been a military strategist.''

``Bad pay, stupid clothes, and you have to work for the government,''
Connor snapped, noting the stiffened spines of Kaden and Bill, both
recruited out of the Pentagon's anti-terror Delta Force command to
help analyze the big guilds' command-structures and figure out how
to get more money out of them.

``Look at 'em go!'' Fairfax said. Connor had a lot of time for her,
even though they often disagreed. She'd run big teams of
level-designers, graphic artists, AI specialists, programmers, the
whole thing, and she had a good top-down and bottom-up view of
things.

``They're good,'' Connor said. He clicked a little and colored each
of the avs with a national flag representing the country the IP
address of the player was registered to. ``And it's a goddamned
United Nations of players, look at that. What language are they
speaking?'' He clicked some more and took over the room's speakers,
cleverly recessed into walls and floors, now buried under mountains
of pizza-cardboard. The room filled with a gabble of heavily
accented English mixed with Mandarin. His ear picked out Indian
accents, Chinese, something else -- Malay? Indonesian? There were
players from the whole Malay Peninsula in that mob.

``And look at the Pinkertons,'' Fairfax said. She had a background in
programming artificial intelligences, a trade that had changed an
awful lot since the Mechanical Turks stepped in to backstop the AIs
in game. But she had invented the idea of giving the game's
soundtrack its own AI, capable of upping the drama-quotient in the
music when momentous things were afoot, and that holistic view of
gameplay had landed her a seat in Command Central. She was the one
who ordered out for health food and giant salads instead of burgers
by the sack and pints of icecream. ``They're nearly in the same
distribution as the Webblies! Look at this --'' she zoomed in on a
scrolling list of IP addresses, then pulled up another table,
fiddled with their sort order. ``Look! These Pinkertons are fighting
from a netblock that's within 200 meters of these Webblies! They're
neighbors! Oh, this is \emph{hella weird}.''

It was true. Connor banged out a quick script to find and pair any
players who were physically proximate to one another and to try for
maps where they were available. Mostly they weren't -- he'd tried
tracking down these rats before, tried to see where they lived, but
ended up with a dead end. They didn't live on roads -- they lived
in illegal squats, shantytowns in the world's slumzones. The best
he could do was month-old sat photos of these mazes, revealing
mountains of smoldering garbage, toxic open sewers, livestock
pens\ldots{} Connor felt like he should visit one of these places, fly a
team of rats out to Command Central in the company jet, stick them
in a lab and study them and learn how to exterminate them.

Because there was one chart Connor didn't need to load, the chart
showing overall stability of the game economy: his
fingerspitzengefuhl was filling him in just fine. The game economy
was \emph{hosed}.

``OK people, there's plenty to do here. No one else respawns on that
shard. Create a new instance for the Caverns so any real players
who hit them don't have to wade through that mess. Get every one of
those accounts and freeze their assets.'' Esteban, who headed up
customer service, groaned.

``You \emph{know} they're mostly hacked,'' he said. ``There's hundreds
of them! We're going to be untangling the assets for
\emph{months}.''

Connor knew it. The legit players whose accounts had been stolen by
the warring clans of third-world rip-off artists didn't deserve to
have their assets frozen. What's more, there'd be plenty of them
whose assets were part of a larger guild bank that might have the
wealth of dozens or hundreds of players. Of course the Bad Guys
knew this and depended on it, knew it would make the game-runners
cautious and slow when it came time to shut down the accounts they
were using to smuggle around their illicit wealth.

He made eye-contact with Bill, head of security. They'd been going
back and forth over whether it would be worth sucking some of
Connor's budget into the security department to develop some
forensic software that would ferret out the transaction histories
of stolen accounts and figure out what assets the original player
legitimately owned and where the dirty money ended up after it left
his account. Connor hated to part with budget, especially when it
involved Bill, who was a pompous ass who liked to act like he was
some kind of super-cybercop rather than a glorified systems
administrator.

But sometimes you had to bite the bullet. ``We'll handle it,'' he
said. ``Right, Bill?'' The head of security nodded, and began to
pound at his keyboard, no doubt hiring a bunch of his old hacker
buddies to come on board for top dollar and write the code.

``Yeah,'' Bill added. ``Don't worry about it, we've got it covered.''

One by one, the combatants vanished as their accounts were shut
down and frozen out. Some of the soldiers reappeared in the new
instance -- a parallel universe containing an identical dungeon,
but none of the same players -- using new avs, but they could tell
who they were because they originated from the same IP addresses as
the kicked accounts. ``This is great,'' Connor said. ``If they keep
this up, we'll have all their accounts nuked by the end of the
day.''

But the Pinkertons and Webblies must have had the same thought,
because the logins dropped off to near-zero, then zero. The screens
shifted, the eating sounds began anew, and Connor went back to his
economic charts. As he'd felt, the price of assets, currency and
derivatives had gone bonkers. The market somehow knew when there
was trouble in Gold Farmer Land, and began to see-saw with the
expectation that the price of goods was about to change.

Connor's own holdings had dropped by 18 percent in 25 minutes,
costing him a cool \$321,498.18.

He popped open a chat to Bill.

\edialog{This stuff you're commissioning with my budget}

\edialog{Yeah?}

\edialog{I want to use it to run every gold farmer to ground
and throw him out of the game}

\edialog{What?}

\edialog{It'll be there, in the transaction history. Some
kind of fingerprint in play-style and spending that'll let us
auto-detect farmers and toss them out. We're going to have a
perfect, controlled, farmer-free economy. The first of its kind}

\edialog{Connor every complex ecosystem has parasites.}

\edialog{Not this one}

\edialog{It won't work}

\edialog{Wanna bet? Let's make it \$10K. I'll give you 2-1}

\tb

\shopad{This scene is dedicated to The Tattered Cover, Denver's legendary independent bookstore. I happened upon The Tattered Cover quite by accident: Alice and I had just landed in Denver, coming in from London, and it was early and cold and we needed coffee. We drove in aimless rental-car circles, and that's when I spotted it, the Tattered Cover's sign. Something about it tingled in my hindbrain -- I knew I'd heard of this place. We pulled in (got a coffee) and stepped into the store -- a wonderland of dark wood, homey reading nooks, and miles and miles of bookshelves.}
{\href{http://www.tatteredcover.com/book/9780765322166}{The Tattered Cover} 1628 16th St., Denver, CO USA 80202 +1 303 436 1070}

Ashok wove his pretty bike through the narrow alleys of Dharavi,
his headlamp slicing through the night. Yasmin's mother would be
rigid with worry and anger, and would probably beat her, but it was
OK. She and Ashok had sat in that studio shed for hours, talking it
through, getting meat on the bones of her idea, and he had left
long, detailed messages for Big Sister Nor before getting them back
on his bike.

Yasmin tapped him on the shoulder at each junction, showing him
which way to turn. Soon they were nearly at her family's house and
shouted at him to stop, hollering through the helmet. He killed the
engine and the headlight and her bum finally stopped vibrating, her
legs complaining about the hours she'd spent gripping the bike with
the insides of her thighs. She swung unsteadily off her bike and
brought her hands up to her helmet.

Her hands were on her helmet when she heard the voices.

``Is that her?''

``I can't tell.''

They were whispering loudly, and a trick of the grilles over the
helmet's ear-coverings let her hear the sound as though it was
originating from right beside her. She put a firm hand on Ashok's
shoulder and squeezed.

``It's her.'' The voice was Mala's, hard.

Yasmin let go of Ashok's shoulder and brought her hand down to the
cables tying the lathi to the bike, while her free hand moved to
the helmet's visor, swinging it up. She'd repinned her hijab around
her neck and now she was glad she had, as she had pretty good
visibility. It had been a long time since she'd been in a physical
fight, but she understood the principles of it well, knew her
tactics.

The lathi was really well anchored -- Ashok hadn't wanted it to go
flying off while they were running down the motorway -- and now she
brought her other hand down to work at it blind, keeping her eyes
on the shadows around her, listening for the footsteps.

``What about the man?''

``Him too,'' Mala said.

And then they charged, an army of them, coming from the shadows all
around them. ``GO!'' she said to Ashok, trying to keep him from
dismounting the bike, but he got to his feet, squared his
shoulders, and faced away from her, to the soldiers who were
charging him. A rock or lump of cement clanged off her helmet,
making a sound like a cooking pot falling to the floor, and now she
tugged as hard as she could at the lathi and at last it sprang
free, the steel hooks on the tips of the bunjee cables whipping
around and smacking painfully into her hands. She barely noticed,
whirling with the two-meter stick held overhead like a
cricket-bat.

And pulled up short.

The boy closest to her was Sushant. Sushant, who, that afternoon,
had spoken of how he'd longed to join her cause. His face was a
mask of terror in the weak light leaking out of the homes around
them. The steel tip trembled over her shoulder as her wrists
twitched. All she would need to do is unwind the swing, let the
long pole and its steel end whistle through the air with all the
whip-crack force penned up at the lathi's end and she would bash
poor Sushant's head in.

And why not? After all, that's what Mala's army was here for.

All this thought in the blink of an eye, so fast she didn't even
register that she'd thought it, but she did not swing the lathi
through the air at Sushant's head. Instead, she swept it at his
feet, pulling the swing so that it just knocked him backwards,
flying into two more soldiers behind him, boys who had once taken
orders from her.

``Stand down!'' she barked, in the voice of command, and swung the
lathi back, sweeping it toward the army's feet like a broom. They
took a giant step back in unison, eyes crazed and rolling in the
weak light. Sushant was weeping. She'd heard bone break when the
lathi's tip met his ankle. He was holding onto the shoulders of the
two soldiers he'd knocked over, and they were struggling to keep
him upright.

No one said anything and there was just the collective breath of
Dharavi, thousands and thousands of chests rising and falling in
unison, breathing in each others' air, breathing in the stink of
the tanners and the burning reek from the dye factories and the
sting of the plastic smoke.

Then Mala stepped forward. In her hand, she held -- what? A
bottle?

A bottle. With an oily rag hanging out of the end. A petrol bomb.

``Mala!'' she said, and she heard the shock in her own voice. ``You'll
burn the whole of Dharavi down!'' It was the tone of voice you use
when shouting into your headset at a guildie who was about to get
the party killed by accidentally aggroing some giant boss. The tone
that said, \emph{You're being an idiot, cut it out.}

It was the wrong tone to use with Mala. She stiffened up and her
other hand worked at the wheel of a disposable lighter --
\emph{snzz} \emph{snzz}.

Again, she moved before she thought, two running steps while she
brought the lathi up over her shoulder, feeling it thunk against
something behind her as it sliced up, then slicing it back down
again, in that savage, cutting arc, down at Mala's skinny legs,
sweeping them with the whole force of her body, and Mala skipped
backwards, away from the lathi, stumbled, went over backwards --

-- and the lathi \emph{connected}, a solid blow that made a sound
like the butcher's knife parting a goat's head from its neck, and
Mala's scream was so terrible that it actually brought people to
their windows (normally a scream in the night would make them stay
back from it). There was bone sticking out of her leg, glinting
amid the blood that fountained from the wound.

And still she had the petrol bomb, and still she had the lighter,
and now the lighter was lit. Yasmin drew back her foot for a
footballer's kick, knowing as she wound up that she could cripple
Mala's hand with a good kick, ending her career as General
Robotwallah.

Afterwards, she remembered the voice that had chased itself around
her head as she drew back for that kick:

\emph{Do it, do it and end your troubles. Do it because she would do it to you. Do it because it will scare her army out of fighting you and the Webblies. Do it because she betrayed you. Do it because it will keep you safe.}

And she lowered her foot and instead \emph{leapt} on Mala, pinning
her arms with her body. The lighter's flame licked at her arm,
burning her, and she ground it out. She could feel Mala's breath,
snorting and pained, on her throat. She grabbed Mala's left wrist,
shook the hand that held the bomb, smashed it against the ground
until it broke and spilled out the stinking petrol into the ditch
that ran alongside the shacks. She stood up.

Mala's face was ashen, even in the bad light. The blood smell and
the petrol smell were everywhere.

Yasmin looked to Ashok. ``You need to take her to the hospital,'' she
said.

``Yes,'' he said. He was holding onto the side of his head, eye
squeezed shut. ``Yes, of course.''

``What happened to you?''

He shrugged. ``Got too close to your lathi,'' he said and tried for a
brave smile. She remembered the \emph{thunk} as she'd drawn back
for her swing.

``Sorry,'' she said.

Mala's army stood at a distance, staring.

``Go!'' Yasmin said. ``Go. This was a disaster. It was stupid and evil
and wrong. I'm not your enemy, you idiots. GO!''

They went.

``We have to splint her,'' Ashok said. ``Make a stretcher, too. Can't
move her like that.''

Yasmin looked at him, raised an eyebrow.

``My father's a doctor,'' he said.

Yasmin went into the flat, climbed the stairs. Her mother sat up as
she entered the room and opened her mouth to say something, but
Yasmin raised on hand to her and, miraculously, she shut up. Yasmin
looked around the room, took the chair that sat in one corner, an
armload of rags from the bundle they used to keep the room clean,
and left, without saying a word.

Ashok broke the chair into splints by smashing it against a nearby
wall. It was a cheap thing and went to pieces quickly. Yasmin knelt
by Mala and took her hand. Her breathing was shallow, labored.

Mala squeezed her hand weakly. Then she opened her eyes and looked
around, confused. Her eyes settled on Yasmin. They looked at each
other. Mala tried to pull her hand away. Yasmin didn't let go. The
hand was strong, nimble. It had dispatched innumerable zombies and
monsters.

Mala stopped struggling, closed her eyes. Ashok brought over the
splints and rags and hunkered down beside them.

Just before he began to work on her, Mala said something. Yasmin
couldn't quite make it out, but she thought it might be,
\emph{Forgive me.}

\tb

\shopad{This scene is dedicated to Hudson Booksellers, the booksellers that are in practically every airport in the USA. Most of the Hudson stands have just a few titles (though those are often surprisingly diverse), but the big ones, like the one in the AA terminal at Chicago's O'Hare, are as good as any neighborhood store. It takes something special to bring a personal touch to an airport, and Hudson's has saved my mind on more than one long Chicago layover.}
{\href{http://www.hudsongroup.com/HudsonBooksellers\_s.html}{Hudson Booksellers}}

Wei-Dong couldn't get Lu off his mind. A barbarian stabbed a
pumpkin and he decided that the sword would be stuck for three
seconds and then play a standard squashing sound from his
soundboard. He couldn't get Lu off his mind. A pickpocket tried to
steal a phoenix's tailfeather, and he made the phoenix turn around
and curse the player out, spitting flames, shouting at him in
Mandarin, his voice filtered through a gobble-phaser so that it
sounded birdy. He couldn't get Lu off his mind. A zombie
horde-leader tried to batter his way into a barricaded mini-mall,
attempting to go through a ``Going out of business'' signboard that
was only a texture mapped onto an exterior surface that had no
interior. Wei-Dong liked the guy's ingenuity, so he decided that it
would take 3,000 zombie-minutes to break it down, and when it fell,
it would map to the interior of the sporting-goods store where
there were some nice clubs, crossbows and machetes.

And he couldn't get Lu off his mind.

He'd always liked Lu. Of all the guys, Lu was the one who really
got \emph{into} the games. He didn't just love the money, or the
friendship: he loved to \emph{play}. He loved to solve puzzles, to
take down the big bosses on a huge raid, to unlock new lands and
achievements for his avs. Sometimes, as Wei-Dong worked his long
shifts making tiny decisions for the game, he thought about how
much better it would be to play, thanks to the work he was doing,
and imagined the Lu would approve of the artistry. It was nice to
be on the other side of the game, making the fun instead of just
consuming it. The job was long, it was hard, it didn't pay well,
but he was \emph{part of the show}.

But this wasn't a show anymore.

His phone started vibrating in his pocket. He took it out, looked
at the face, put it on his desk. It was his mom. He'd relented and
given her his new number once he turned 18, justifying it to
himself on the ground that he was an adult now and she couldn't
have him tracked down and dragged back. But really, it was because
he couldn't face spending his 18th birthday alone. But he didn't
want to talk to her now. He bumped her to voicemail.

She called back. The phone buzzed. He bumped it to voicemail. A
second later, the phone buzzed again. He reached to turn it off and
then he stopped and answered it.

``Hi, Mom?''

``Leonard,'' she said. ``It's your father.''

``What?''

She took a deep breath, let it out. ``A heart attack. A big one.
They took him to --'' She stopped, took in a deep breath. ``They took
him to the Hoag Center. He's in the ICU. They say it's the best --''
Another breath. ``It's supposed to be the best.''

Wei-Dong's stomach dropped away from him, sinking to a spot
somewhere beneath his chair. His head felt like it might fly away.
``When?''

``Yesterday,'' she said.

He didn't say anything. \emph{Yesterday?} He wanted to shriek it.
His father had been in the hospital since \emph{yesterday} and no
one had told him?

``Oh, Leonard,'' she said. ``I didn't know what to do. You haven't
spoken to him since you left. And --''

\emph{And}?

``I'll come and see him,'' he said. ``I can get a taxi. It'll take
about an hour, I guess.''

``Visiting hours are over,'' she said. ``I've been with him all day.
He isn't conscious very much. I\ldots{} They don't let you use your
phone there. Not in the ICU.''

For months, Wei-Dong had been living as an adult, living a life he
would have described as ideal, before the phone rang. He knew
interesting people, went to exciting places. He
\emph{played games all day}, for a living. He knew the secrets of
gamespace.

Now he understood that a feeling of intense loneliness had been
lurking beneath his satisfaction all along, a bubbling pit of
despair that stank of failure and misery. Wei-Dong loved his
parents. He wanted their approval. He trusted their judgment. That
was why he'd been so freaked out when he discovered that they'd
been plotting to send him away. If he hadn't cared about them, none
of it would have mattered. Somewhere in his mind, he'd had a
cut-scene for his reunion with his parents, inviting them to a
fancy, urban restaurant, maybe one of those raw food places in Echo
Park that he read about all the time in Metroblogs. They'd have a
cultured, sophisticated conversation about the many amazing things
he'd learned on his own, and his father would have to scrape his
jaw off his plate to keep up his end of the conversation.
Afterwards, he'd get on his slick Tata scooter, all tricked out
with about a thousand coats of lacquer over thin bamboo strips, and
cruise away while his parents looked at each other, marvelling at
the amazing son they'd spawned.

It was stupid, he knew it. But the point was, he'd always treated
this time as a holiday, a little interlude in his family life. His
vision quest, when he went off to become a man. A real Bar-Mitzvah,
one that meant something.

The thought that he might never see his father again, never make up
with him -- it hit him like a a blow, like he'd swung a hammer at a
nail and smashed his hand instead.

``Mom --'' His voice came out in a croak. He cleared his throat.
``Mom, I'm going to come down tomorrow and see you both. I'll get a
taxi.''

``OK, Leonard. I think your father would like to see you.''

He wanted her to say something about how selfish he'd been to leave
them behind, what a bad son he'd been. He wanted her to say
something \emph{unfair} so that he could be angry instead of
feeling this terrible, awful guilt.

But she said, ``I love you, Leonard. I can't wait to see you. I've
missed you.''

And so he went to bed with a million self-hating thoughts chanting
in unison in his mind, and he lay there in his bed in the flophouse
hotel for hours, listening to the thoughts and the shouting bums
and clubgoers and the people having sex in other rooms and the
music floating up from car windows, for hours and hours, and he'd
barely fallen asleep when his alarm woke him up. He showered and
scraped off his little butt-fluff mustache with a disposable razor
and ate a peanut butter sandwich and made himself a quadruple
espresso using the nitrous-powered hand-press he'd bought with his
first paycheck and called a cab and brushed his teeth while he
waited for it.

The cabbie was Chinese, and Wei-Dong asked him, in his best
Mandarin, to take him down to Orange County, to his parents' place.
The man was clearly amused by the young white boy who spoke
Chinese, and they talked a little about the weather and the traffic
and then Wei-Dong slept, dozing with his rolled-up jacket for a
pillow, sleeping through the caffeine jitter of the quad-shot as
the early morning LA traffic crawled down the 5.

And he paid the cabbie nearly a day's wages and took his keys out
of his jacket pocket and walked up the walk to his house and let
himself in and his mother was sitting at the kitchen table in her
housecoat, eyes red and puffy, just staring into space.

He stood in the doorway and looked at her and she looked back at
him, then stood uncertainly and crossed to him and gave him a hug
that was tight and trembling and there was wetness on his neck
where her tears streaked it.

``He went,'' she breathed into his ear. ``This morning, about 3 AM.
Another heart attack. Very fast. They said it was practically
instant.'' She cried some more.

And Wei-Dong knew that he would be moving home again.

\tb

The hospital discharged Big Sister Nor and The Mighty Krang and
Justbob two days early, just to be rid of them. For one thing, they
wouldn't stay in their rooms -- instead, they kept sneaking down to
the hospital's cafeteria where they'd commandeer three or four
tables, laboriously pushing them together, moving on crutches and
wheelchairs, then spreading out computers, phones, notepads,
macrame projects, tiny lead miniatures that The Mighty Krang was
always painting with fine camel-hair brushes, cards, flowers,
chocolates and shortbread sent by Webbly supporters.

To top it off, Big Sister Nor had discovered that three of the
women on her ward were Filipina maids who'd been beaten by their
employers, and was holding consciousness-raising meetings where she
taught them how to write official letters of complaint to the
Ministry of Manpower. The nurses loved them -- they'd voted in a
union the year before -- and the hospital administration
\emph{hated} them with the white-hot heat of a thousand suns.

So less than two weeks after being beaten within an inch of their
lives, Big Sister Nor, The Mighty Krang, and Justbob stepped,
blinking, into the choking heat of mid-day in Singapore, wrapped in
bandages, splints and casts. Their bodies were broken, but their
spirits were high. The beating had been, well, \emph{liberating}.
After years of living in fear of being jumped and kicked
half-to-death by goons working for the bosses, they'd been through
it and survived. They'd thrived. Their fear had been burned out.

As they looked at one another, hair sticky and faces flushed from
the steaming heat, they began to smile. Then to giggle. Then to
laugh, as loud and as deep as their injuries would allow.

Justbob swept her hair away from the eyepatch that covered the ruin
of her left eye, scratched under the cast on her arm, and said,
``They should have killed us.''

\chapter*{Part III: Ponzi}

\shopad{This scene is dedicated to the Harvard Bookstore, a wonderful and eclectic bookshop in the heart of one of the all-time kick-ass world-class bookshopping neighborhoods, the stretch of Mass Ave that runs between Harvard and MIT. The last time I visited the store, they'd just gotten in an Espresso print-on-demand book machine that was hooked up to Google's astonishing library of scanned public-domain books and they could print and bind practically any out of print book from the whole of human history for a few dollars in a few minutes. To plumb the unimaginable depths of human creativity this represented, the store had someone whose job it was to just mouse around and find wild titles from out of history to print and stick on the shelves around the machine. I have rarely felt the presence of the future so strongly as I did that night.}
{\href{http://www.harvard.com/}{Harvard Bookstore}: 1256 Massachusetts Avenue, Cambridge MA 02138 USA, +1 (617) 661-1515}

The inside of the shipping container was a lot worse than Wei-Dong
had anticipated. When he'd decided to smuggle himself into China,
he'd done a lot of reading on the subject, starting with searches
on human trafficking -- which was all horror stories about 130
degree noontimes in a roasting box, crammed in with thirty others
-- and then into the sustainable housing movement, where architects
were vying to outdo one another in their simple and elegant
retrofits of containers into cute little apartments.

Why no one had thought to merge the two disciplines was beyond him.
If you're going to smuggle people across the ocean, why not avail
yourself of a cute little kit to transform their steel box into a
cozy little camper? Was he missing something?

Nope. Other than the fact that people-smugglers were all criminal
dirtbags, he couldn't find any reason why a smuggle-ee couldn't
enjoy the ten days at sea in high style. Especially if the
smuggle-ee was now co-owner of a huge shipping and logistics
company based in Los Angeles, with the run of the warehouse and a
Homeland Security all-access pass for the port.

It had taken Wei-Dong three weeks to do the work on the container.
The mail-order conversion kit said that it could be field-assembled
by two unskilled laborers in a disaster area with hand tools in two
days. It took him two weeks, which was a little embarrassing, as
he'd always classed himself as ``skilled'' (but there you go).

And he had special needs, after all. He'd read up on port security
and knew that there'd be sensors looking for the telltale cocktail
of gasses given off by humans: acetone, isoprene, alpha pinene and
lots of other exotic exhaust given off with every breath in a
specific ratio. So he built a little container inside the
container, an airtight box that would hold his gasses in until they
were at sea -- he figured he could survive in it for a good ten
hours before he used up all the air, provided he didn't exercise
too much. The port cops could probe his container all they wanted,
and they'd get the normal mix of volatiles boiling off of the paint
on the inside of the shipping container, untainted by human
exhaust. Provided they didn't actually open his container and then
get too curious about the hermetically sealed box inside, he'd be
golden.

Anyway, by the time he was done, he had a genuinely kick-ass little
nest. He'd loaded up his Dad's Huawei with an entire apartment's
worth of IKEA furniture and then he'd hacked it and nailed it and
screwed it and glued it into the container's interior, making a
cozy ship's cabin with a king-sized bed, a chemical toilet, a
microwave, a desk, and a play area. Once they were at sea, he could
open his little hatch and string out his WiFi receiver -- tapping
into the on-board WiFi used by the crew would be simple, as they
didn't devote a lot of energy to keeping out freeloaders while they
were in the middle of the ocean -- and his solar panel. He had some
very long wires for both, because he'd fixed the waybills so that
his container would be deep in the middle of the stack alongside
one of the gaps that ran between them, rather than on the outside
edge: one percent of shipping containers ended up at the bottom of
the sea, tossed overboard in rough waters, and he wanted to
minimize the chance of dying when his container imploded from the
pressure of hundreds of atmospheres' worth of deep ocean.

Inheritances were handier than he'd suspected. He was able to click
onto Huawei's website and order ten power-packs for their
all-electric runabouts, each one rated for 80 miles' drive. They
were delivered directly to the pier his shipping container was
waiting on (he considered the possibility that the power-packs had
been shipped to America in the same container he was installing
them in, but he knew the odds against it were astronomical -- there
were a \emph{lot} of shipping containers arriving on America's
shores every second). They stacked neatly at one end of the
container, with a barcoded waybill pasted to them that said they
were being returned as defective. They arrived charged, and he was
pretty sure that he'd be able to keep them charged between the Port
of Los Angeles and Shenzhen, using the solar sheets he was going to
deploy on the top of the container stack. He'd tested the
photovoltaic sheets on his father's Huawei and found that he could
fully charge it in six hours, and he'd calculated that he should be
able to run his laptop, air conditioner, and water pumps for four
days on each stack. 16 days' power would be more than enough to
complete the crossing, even if they got hit by bad weather, but it
was good to know that recharging was an option.

Water had given him some pause. Humans consume a \emph{lot} of
water, and while there was plenty of room in his space capsule --
as he'd come to think of the container -- he thought there had to
be a better way to manage his liquid needs on the voyage than
simply moving three or four tons of water into the box. He was deep
in thought when he realized that the solar sheets were all
water-proof and could be easily turned into a funnel that would
feed a length of PVC pipe that he could snake from the top of the
container stack into the space-capsule, where a couple of sterile
hollow drums would hold the water until he was ready to drink it or
shower in it. Afterwards, his waste water could just be pumped out
onto the ship's deck, where it would wash overboard with all the
other water that fell on the ship. If he packed enough water to
keep him going on minimal showers and cooking for a week, the odds
were good that they'd hit a rainstorm and he'd be topped up -- and
if they didn't he could ration his remaining water and arrive in
China a little smellier than he'd started.

He loved this stuff. The planning was exquisite fun, a real
googlefest of interesting HOWTOs and advice. Lots of parts of the
problem of self-sufficiency at sea had been considered before this,
though no one had given much thought to the problem of travelling
in style and secrecy in a container. He was a pioneer. He was
making notes and planning to publish them when the adventure was
over.

Of course, he wouldn't mention the \emph{reason} he needed to
smuggle himself into China, rather than just applying for a tourist
visa.

Wei-Dong's mother didn't know what to make of her son. His father's
death had shattered her, and half the time she seemed to be
speaking to him from behind a curtain of gauze. He found the
anti-depressants her doctor had prescribed and looked up the
side-effects and decided that his mother probably wouldn't be in
any shape to notice that he was up to something weird. Mostly she
just seemed relieved to have him home, and industriously involved
in the family business. She hadn't even blinked when he told her he
was going to take a road trip up the coast, a nice long drive up to
Alaska with minimal net-access, phone activity and so on.

The last cargo to go into the space-capsule was three cardboard
boxes, small enough to load into the trunk of the Huawei, which he
put in long-term parking and double-locked after he'd loaded them
up. Each one was triple-wrapped in water-proof plastic, and inside
them were twenty-five thousand-odd prepaid game-cards for various
MMOs. The face-value of these cards was in excess of \$200,000,
though no money changed hands when he collected them, in lots of a
few hundred, from Chinese convenience stores all over Los Angeles
and Orange County. It had taken three days to get the whole load,
and it had been the hairiest part of the gig so far. The cards were
part of a regular deal whereby the big gold-farmers used networks
of overseas retailers to snaffle up US playtime and ship it back to
China, so that their employees could get online using the US
servers.

Technically, that meant that all the convenience store clerks he
visited were part of a vast criminal underground, but none of them
seemed all that dangerous. Still, if any one of them had been
suspicious about the white kid with the bad Mandarin accent who was
doing the regular pickup, who knew what might happen?

It hadn't, though. Now he had the precious cargo, the boxes of
untraceable, non-sequential game-credit that would let him earn
game-gold. It was all so weird, now that he sat there on his red
leather Ikea sofa, sipping an iced tea and munching a power bar and
contemplating his booty.

Under their scratch-off strips, these cards contained unique
numbers produced by a big random-number generator on a server in
America, then printed in China, then shipped back to America, now
destined for China again. He thought about how much simpler it
would have been to come up with the random numbers in China in the
first place, and chuckled and put his feet up on the boxes.

Of course, if they'd done that, he wouldn't have had any excuse to
build the space-capsule and smuggle himself into China.

\tb

\shopad{This scene is dedicated to London's Clerkenwell Tales, located around the corner from my office in Clerkenwell, a wonderful and eclectic neighborhood in central London. Peter Ho, the owner, is a veteran of Waterstone's, and has opened up exactly the kind of small, expertly curated neighborhood store that every bookish person yearns to have in the vicinity. Peter makes a point of stocking small handmade editions from local printers, and as a result, I'm forever dropping in to say hello over my lunch break and leaving with an armload of exquisite and gorgeous books. It's lethal. In a good way.}
{\href{http://www.clerkenwell-tales.co.uk/}{Clerkenwell Tales}: 30 Exmouth Market EC1R 4QE London +44 (0)20 7713 8135}

Ashok did his best thinking on paper, big sheets of it. He knew
that it was ridiculous. The smart thing to do would be to keep all
the files digital, encrypted on a shared drive on the net where all
the Webblies could get at it. But the numbers made so much more
sense when they were written neatly on flip-chart paper and tacked
up all around the walls of his ``war-room'' -- the back room at Mrs
Dibyendu's cafe, rented by Mala out of the army's wages from Mr
Bannerjee.

Oh yes, Mala was still drawing wages from Mr Bannerjee and her
soldiers were still fighting the missions he sent them on. But
afterwards, in their own time, they fought their own missions, in
Mrs Dibyendu's shop. Mrs Dibyendu was lavishly welcoming to them,
grateful for the business in her shop, which had been in danger of
drying up and blowing away. Idiot nephew had been sent back to
Uttar Pradesh to live with his parents, limping home with his tail
between his legs and leaving Mrs Dibyendu to tend her increasingly
empty shop on her own.

Mrs Dibyendu didn't mind the big sheets of paper. She \emph{loved}
Ashok, smartly dressed and well turned out, and clearly thought
that he and Yasmin had something going on. Ashok tried gently to
disabuse her of this, but she wasn't having any of it. She brought
him sweet chai all day and all night, as he labored over his
sheets.

``Ashok,'' Mala called, limping toward him through the empty cafe,
leaning on the trestle-tables that supported the long rows of
gasping PCs.

He stood up from the table, wiping the chai from his chin with his
hand, wiping his hand on his trousers. Mala made him nervous. He'd
visited her in the hospital, with Yasmin, and sat by her bed while
she refused to look at either of them. He'd picked her up when she
was discharged, and she'd fixed him with that burning look, like a
holy woman, and she'd nodded once at him, and asked him how her
Army could help.

``Mala,'' he said. ``You're early.''

``Not much fighting today,'' she said, shrugging. ``Fighting Webblies
is like fighting children. Badly organized children. We knocked
over twenty jobsites before lunch and I had to call a break. The
Army was getting bored. I've got them on training exercises,
fighting battles against each other.''

``You're the commander, General Robotwallah, I'm sure you know
best.''

She had a very pretty smile, Mala did, though you rarely got to see
it. Mostly you saw her ugly smiles, smiles that seemed to have too
many sharp teeth in them. But her pretty smile was like the sun. It
changed the whole room, made your heart glow. He understood how a
girl like this could command an Army. He stared at the pretty smile
for a minute and his tongue went dry and thick in his mouth.

``I want to talk to you, Ashok. You're sitting here with your paper
and your figures, and you keep telling us to wait, wait a little,
and you'll explain everything. It's been months, Ashok, and still
you say wait, explain. I'm tired of waiting. The Army is tired of
waiting. Being double agents was amusing for a little while, and
it's fun to fight real Pinkertons at night, but they're not going
to wait around forever.''

Ashok held his hands out in a placating gesture that often worked
on Mala. She needed to know that she was the boss. ``Look, it's not
a simple matter. If we're going to take on four virtual worlds at
once, everything has to run like clockwork, each piece firing after
the other. In the meantime --''

She waved at him dismissively. ``In the meantime, Bannerjee grows
more and more suspicious. The man is an idiot, not a moron. He will
eventually figure out that something is going wrong. Or his masters
will. And then --''

``And then we'll have to placate him, or misdirect him. General,
this is a confidence game, a scam, running on four virtual worlds
and twenty real nations, with hundreds of confederates. Confidence
games require planning and cunning. It's not enough to go in, guns
blazing --''

``You think we don't understand planning? You think we don't
understand \emph{cunning}? Ashok, you have never fought. You should
fight. It would help you understand this business you've gotten
into. You think that we're thugs, idiot muscle. Running a battle
requires as much skill as anything you do -- I don't have a fine
education, I am just a girl from the village, I am just a Dharavi
rat, but I am \emph{smart} Ashok, and don't you ever forget it.''

The worst part was, she was right. He \emph{did} often think of her
as a thug. ``Mala, I want to play, but playing would take me away
from planning.''

``You can't plan if you don't play. I'm the general, and I'm
ordering it. You'll join the junior platoon on maneuvers tomorrow
at 10AM. There's skirmishing, then theory, then a couple of battles
overseen by the senior platoon when they arrive. It will be good
for you. They will rag you some, because you are new, but that will
be good for you, too.''

That look in her eyes, the fiery one, told him that he didn't dare
disagree. ``Yes, General,'' he said.

``And you will explain this business to me, now. You will learn my
world, I will learn yours.''

``Mala --''

``I know, I know. I came in and shouted at you because you were
taking too long and now I insist that you take longer.'' She gave
him that smile. She wasn't pretty -- her features were too sharp
for pretty -- but she was beautiful when she smiled. She was going
to be a heart-breaker when she grew up. \emph{If} she grew up.

``Yes, General.''

``Chai!'' she called to Mrs Dibyendu, who brought it round quickly,
averting her eyes from Mala.

``All right, let's start with the basic theory of the scam. Who is
easiest to trick?''

``A fool,'' she said at once.

``Wrong,'' he said. ``Fools are often suspicious, because they've been
taken advantage of. The easiest person to trick is a successful
person, the more successful the better. Why is that?''

Mala thought. ``They have more money, so it's worth tricking them?''

Ashok waggled his chin. ``No, sorry -- by that reasoning, they
should be \emph{more} suspicious, not less.''

Mala scraped a chair over the floor and sat down and made a face at
him. ``I give up, tell me.''

``It's because if a man is successful at doing one thing, he's apt
to assume that he'll be successful at anything. He believes he's a
Brahmin, divinely gifted with the wisdom and strength of character
to succeed. He can't bear the thought that he just got lucky, or
that his parents just got lucky and left him a pile of Rupees. He
can't stand the thought that understanding physics or computers or
cameras doesn't make him an expert on economics or beekeeping or
cookery.

``And his intelligence and his pride work together to make him
\emph{easier} to trick. His pride, naturally, but his intelligence,
too: he's smart enough to understand that there are lots of ways to
get rich. If you tell him a complex tale about how some market
works and can be tricked, he can follow along over rough territory
that would lose a dumber man.

``And there's a third reason that successful men are easier to trick
than fools: they dread being shown up as a fool. When you trick
them, you can trick them again, make them believe that the scheme
fell through. They don't want to go to the police or tell their
friends, because if word gets out that some mighty and powerful man
was tricked, he stands to lose his reputation, without which he
cannot recover his fortune.''

Mala waggled her chin. ``It all makes sense, I suppose.''

``It does,'' Ashok said.

``I am a successful and powerful person,'' she said. Her eyes were
cat-slits.

``You are,'' Ashok said, more cautiously.

``So I would be easier to fool than any of the fools in my army?''

Ashok laughed. ``You are so sharp, General, it's a wonder you don't
cut yourself. Yes, it's possible that all of this is a giant
triple-twist bluff, aimed at fooling you. But what would I want to
fool you for? As rich as your Army has made you, you must know that
I could be just as rich by working as a junior lecturer in
economics at IIT. But General, at the end of the day, you either
trust me or you don't. I can't prove to you that you're inside the
scheme rather than its target. If you want out, that's fine. It
will hurt the plan, but it won't be its death. There's a lot of
people involved here.''

Mala smiled her sunny smile. ``You are a clever man,'' she said. ``And
for now, I will trust you. Go on.''

``Let's step back a little. Do you want to learn some history?''

``Will it help me understand why you're taking so long?''

``I think so,'' he said. ``I think it's a bloody good story, in any
case.''

She made a go-on gesture and sipped her chai, her back very erect,
her bearing regal.

``Back in the 1930s, the biggest confidence jobs were called 'The
Big Store.' They were little stage plays in which there was only
one audience-member, the 'mark' or victim. \emph{Everyone else} was
in the play. The mark would meet a 'roper' on a train, who would
feel him out to see if he had any money. He'd sometimes give him a
little taste of the money to be made -- maybe they'd share some
mysterious 'found' money that he'd planted. That sort of thing
makes the mark trust you more, and also puts him in your power,
because now you know that he's willing to cheat a little.

``Once the train pulled into the strange city and the mark got off,
every single person he met or talked with would be part of the
trick. If the mark was good at finance, the roper would hand him
off to a partner, the 'inside man' who would tell him about a scam
he had for winning horse races; if the mark was good at horse
races, the scam would be about fixing the stock market -- in other
words, whatever the mark knew the least about, that was the center
of the game.

``The mark would be shown a betting parlor or a stock-broker's
office filled with bustling, active people -- so many people that
it was impossible to believe that they could \emph{all} be part of
a scam. Then he'd have the deal explained to him: the brokerage
house or betting parlor got its figures from a telegraph office --
this was before computers -- that would phone in the results. The
mark would then be shown the 'telegraph office' -- another totally
fake business -- and meet a 'friend' of the inside man who was
willing to delay the results by a few minutes, giving them to the
roper and the market just quick enough to let them get their bets
or buys down. They'd know the winners before the office did, so
they'd be betting on a sure thing.

``And they'd try it -- and it would work! The mark could put a few
dollars down and walk away with a few hundred. It was an
eye-popping experience, a real thrill. The mark's imagination would
start to work on him. If he could turn a few dollars into hundreds,
imagine what he could do if he could put down \emph{all} his money,
along with whatever money he could steal from his business, his
family, his friends -- everyone. It wouldn't even be stealing,
because he'd be able to pay everyone back once he won big. And he'd
go and get all the money he could lay hands on, and he'd lay his
bet and he'd lose!

``And it would be his fault. The inside man wouldn't be able to
believe it, he'd said, 'Bet on this horse in the first race,' not
'Bet on this horse for first place' or some similar
misunderstanding. The mark's bad hearing had cost them everything,
all of them. There is a giant scene, and before you know it, the
police are there, ready to arrest everyone. Someone shoots the
policeman, there's blood and screaming, the place empties out, and
the mark counts himself lucky to have escaped with his life. Of
course, all the blood and shooting are fakes, too -- so is the
policeman. He's got a little blood in a bag in his mouth; they
called it a 'cackle-bladder': a fine word, no?

``Now, at this stage, it may be that the mark is completely, totally
broke, not one paisa to his name. If that's the case, he gets away
and never hears from the roper or the inside man again. He spends
the rest of his life broke and broken, hating himself for having
misheard the instruction at the critical moment. And he never, ever
tells anyone, because if he did, it would expose this great man for
a fool.

``But if there's any chance he can get more money -- a friend he
hasn't cleaned out, a company bank account he can access -- they
may contact him \emph{again} and offer him the chance to 'get
even'. You can bet he will -- after all, he's a king among men,
destined to rule, who made his fortune because he's better than
everyone else. Why wouldn't he play again, since the only reason he
lost last time was that he misheard an instruction. Surely that
won't happen again!''

``But it does,'' she said. Her eyes were shining.

``Oh yes, indeed. And again, and again --''

``And again. until he's been bled dry.''

``You've learned the first lesson,'' Ashok said. ``Now, onto advanced
subjects. You know how a pyramid scheme works, yes?''

She waved dismissively. ``Of course.''

``Now, the pyramid scheme is just a kind of skeleton, and like a
skeleton, you can hang a lot of different bodies off of it. It can
look like a plan to sell soap, or a plan to sell vitamins, or
something else altogether. But the important thing is, whatever
it's selling, it has to seem like a good deal. Think back on the
big store -- how do you make something seem like a good deal?''

Mala thought carefully. Ashok could practically see the gears
spinning in her head. Wah! She was \emph{smart}, this Dharavi
girl!

``OK,'' she said. ``OK -- it should be something the mark doesn't know
much about.''

``Got it in one!'' Ashok said. ``If the mark is smart and
accomplished, she'll assume that she knows everything about
everything. Dangle some bait for her that she doesn't really
understand and she'll come along. But there's a way to make even
familiar subjects unfamiliar. Here, look at this.'' He typed at the
disused computer on a corner of his desk, googled an image of a
craps table at a casino.

``This is a gambling game, craps. They play it with dice.''

``I've seen men playing it in the street,'' Mala said.

``This is the casino version. See all the lines and markings?''

She nodded.

``These marks represent different bets -- double if it comes up this
way, triple if it comes up that way. The bets can get very, very
complicated.

``Now, dice aren't that complicated. There are only 36 ways that a
roll can come up: one-one, one-two, one-three, and so on, all that
way up to six-six. It should be easy to tell whether a bet is any
good: take the chance of rolling two sixes, twice in a row: the
odds are 36 times 36 to one. If the bet pays less than those odds,
then you will eventually lose money. If the bet pays more than
those odds, then you will eventually win money.''

Mala shook her head. ``I don't really understand.''

``Imagine flipping a coin.'' He took out his wallet and opened a flap
and pulled out an old brass Chinese coin, pierced in the center
with a square. ``One side is heads, one side is tails. Assuming the
coin is 'fair' -- that is, assuming that both sides of the coin
weigh the same and have the same wind resistance, then the chances
of a coin landing with either face showing are 50-50, or 1-in-1, or
just 'even'.

``Now we play a fair game. I toss the coin, you call out which side
you think it'll land on. If you guess right, you double your bet;
if not, I take your money. If we play this game long enough, we'll
both have the same amount of money as we started with -- it's a
boring game.

``But what if instead I paid you triple if it landed on heads,
provided you took the heads-bet? All you need to do is keep putting
money on heads, and eventually you'll end up with all my money:
when it comes up tails, I win a little; when it comes up heads, you
win a lot. Over time, you'll take it all. So if I offered you this
proposition, you should take it.''

``All right,'' Mala said.

``But what if it was a very complicated bet? What if there were two
coins, and the payout depended on a long list of factors; I'll pay
you triple for any double-head or double-tails, provided that it
isn't the same outcome as the last time, unless it is the
\emph{third} duplicate outcome. Is that a good bet or a bad one?''

Mala shrugged.

``I don't know either -- I'd have to calculate the odds with pen and
paper. But what about this: what if I'll pay you \emph{300 to one}
if you win according to the rules I just set up. You lay down ten
rupees and win, I'll give you \emph{3,000} back?''

Mala cocked her head. ``I'd probably take the bet.''

``Most people would. It's a fantastic cocktail: mix one part
confusing rules and one part high odds, and people will lay down
their money all day. Now, tell me this: would you bet ten rupees on
rolling the dice double-sixes, thirty times in a row?''

``No!'' Mala said. ``That's practically impossible.''

Ashok spread his hands. ``And now you have the second lesson:
everyone has some intuition about odds, even if they are, excuse
me, a girl who has never studied statistics.'' Mala colored, but she
held her tongue. It was true, after all. ``Most people won't bet on
nearly impossible things, not even if you give brilliant odds. But
you can disguise the nearly impossible by making it do a lot of
acrobatics -- making the rules of the game very complicated -- and
then lots of people, even smart people, will place bets on
propositions that are every bit as unlikely as thirty double-sixes
in a row. In fact, smart people are \emph{especially} likely to
place those bets --''

Mala held up her hand. ``Because they're so smart they think they
know everything.''

Ashok clapped. ``Star pupil! You should have been a con-artist or an
economist, if only you weren't such a fine General, General.'' She
grinned. Ashok knew that she loved to hear how good a general she
was. He didn't blame her: if he was a Dharavi girl who'd outsmarted
the slum and made a life, he'd be a little insecure too. It was
just one more thing to like about Mala and her scowling, hard
brilliance. ``Now, my star pupil, put it all together for me.''

She began to recite, counting off on her fingers, like a schoolgirl
recounting a lesson. ``To make a Ponzi scheme that works, that
really works, you need to have

smart people

who are surrounded by con-artists

who are given a chance to bet on something complicated

in a way that they're not good at understanding.''

Ashok clapped and Mala gave a small, ironic bow from her seat.

``So that is what I am doing back here. Devising the scheme that
will take the economies of four entire worlds hostage, make them
ours to smash as we see fit. In order to do that, I need to do some
very fine work.''

Mala pointed at a chart that was dense with scribbled equations and
notations. ``Explain,'' she commanded.

``That is an entirely different sort of lesson,'' Ashok said. ``For a
different day. Or perhaps a year.''

Mala's eyes narrowed.

``My dear general,'' Ashok said, laying it on so thick that they both
knew he was doing it, and he saw the corners of Mala's lips tremble
as they tried to hold back her smile, ``If I asked you to explain
the order of battle to me, you could do two things: either you
could confer some useful, philosophical principles for commanding a
force; or you could vomit up a lifetime's statistics and specifics
about every weapon, every character class, every technique and tip.
The chances are that I'd never memorize a tenth of what you had to
tell me. I don't have the background for it. And, having memorized
it, I would never be able to put it to use because I wouldn't have
had the hard labor that you've put in -- jai ho! -- and so I won't
have the skeleton in my mind on which I might lay the flesh of your
teaching, my guru.'' He checked to see if he'd laid it on too
thickly, decided he hadn't, grinned and namasted to her, just to
ice the biscuit.

Mala nodded regally, keeping her straight face on for as long as
she could, but as she left the room, hobbling on her cane, he was
sure he heard a girlish peal of giggles from her.

\tb

Matthew's first plate of dumplings tasted so good he almost choked
on the saliva that flooded his mouth. After two months in the labor
camp, eating chicken's feet and rice and never enough of either,
freezing at night and broiling during the day, he thought that he
had perfectly reconstructed the taste of dumplings in his mind. On
days when he was digging, each bite of the shovel's tip into the
earth was like the moment that his teeth pierced a dumpling's skin,
letting the steam and oil escape, the meat inside releasing an
aroma that wafted up into his nostrils. On days when he was
hammering, the round stones were the tender dumplings in a
mountain, the worn ground was the squeaking styrofoam tray.
Dumplings danced in his thoughts as he lay on the floor between two
other prisoners; they were in his mind when he rose in the morning.
The only time he didn't think about dumplings was when he was
eating chicken's feet and rice, because they were so awful that
they alone had the power to drive the ghost of dumplings from his
imagination.

Those were the times he thought about what he was going to do when
he got out of jail. What he was going to do in the game. What the
Webblies were planning, and how he would play his part in that
plan.

The prison official that released him assumed that he was one of
the millions of illegal workers with forged papers who'd gone to
Canton, to the Pearl River Delta, to seek his fortune. He was
half-way through a stern, barked lecture about staying out of
trouble and going back to his village in Gui-Zhou or Sichuan or
whatever impoverished backwater he hailed from, before the man
actually looked down at his records and saw that Matthew was,
indeed, Cantonese -- and that he would shortly be transported, at
government expense, back to Shenzhen. The man had fallen silent,
and Matthew, overcome with the comedy of the moment, couldn't help
but thank him profusely -- in Cantonese.

There were dumplings on the train, sold by grim men and women with
deep lines cut into their faces by years and worry and hunger and
misery. This was the provinces, the outer territories, the
mysterious China that had sent millions of girls and boys to Canton
to earn their fortunes in the Pearl River Delta. Matthew knew all
their strange accents, he spoke their strange Mandarin language,
but he was Cantonese, and this was not his people.

Those were not his dumplings.

It wasn't until he debarked at the outskirts of Shenzhen and
transferred to a metro subway that he started to feel at home. It
wasn't until then that he started to think about dumplings. The
girls on the metro were as he remembered them, beautiful and
polished and laughing and well fed. Skulking in the doorway of the
train, watching his reflection in the dark glass, he saw what an
awful skeleton-person he'd become. He had been a young man when he
went in, a boy, really. Now he looked five years older, and he was
shifty and sunken, and there was a scrub of wispy beard on his
cheeks, accentuating their hollowness. He looked like one of the
mass of criminals and grifters and scumbags who hung around the
train station and the street corners -- tough and desperate as a
sewer rat. Unpredictable.

Why not? Sewer rats got lots of dumplings. They had sharp teeth and
sharp wits. They were \emph{fast}. Matthew grinned at his
reflection and the girls on the train gave him a wide berth when
they pulled into the next station.

Lu met him at Guo Mao station, up on the street level, where the
men and women in brisk suits with brisk walks came and went from
the stock exchange, a perfect crowd of people to get lost in. Lu
took both of his hands in a long, soulful, silent shake and led
them away toward the stock exchange, where the identity
counterfeiters were.

These people kept Shenzhen and all of Guandong province running.
They could make you any papers you needed: working permits allowing
a farm girl to move from Xi'an to Shenzhen and make iPods; papers
saying you were a lawyer, a doctor, an engineer; driver's licenses,
vendor's licenses -- even pilot's licenses, according to the card
one of them gave him. They were old ladies, the friendly face of
criminal empires run by hard men with perpetual cigarettes and
dandruff on the shoulders of their dark suits.

They walked in silence through the shouting grabbing crowds, the
flurries of cards advertising fake documents shoved in their hands
by grannies on all sides of them. Lu stopped in front of one granny
and bent and whispered in her ear. She nodded once and went back to
waving her cards, but she must have signalled a confederate
somehow, because a moment later, a young man got up off a bench and
wandered into a gigantic electronics mall and they followed him,
threading their way through stall after stall of parts for mobile
phones -- keyboards, screens, dialpads, diodes -- up an escalator
to another floor of parts, up another escalator and another floor,
and one more to a floor that was completely deserted. Even the
electrical outlets were empty, bare wires dangling from the
receptacles, waiting to be hooked up to plugs.

The boy was 100 meters ahead of them, and they trailed after him,
slipping into a hallway that led toward the emergency stairs. A
little side door was slightly ajar and Lu pushed it open. The boy
wasn't there -- he must have taken the stairs -- but there was
another boy, younger than Lu or Matthew, sitting in front of a
computer, intently playing Mushroom Kingdom. Matthew smiled -- it
was always so strange to see a Chinese person playing a game just
for the fun of it, rather than as a job. He looked up and nodded at
the two of them. Wordlessly, Lu passed him a bundle that the boy
counted carefully, mixed Hong Kong dollars and Chinese renminbi. He
made the money disappear with a nimble-fingered gesture, then
pointed at a stool in a corner of the room with a white screen
behind it. Matthew sat -- still without a word -- and saw that
there was a little webcam positioned on the boy's desk, pointing at
him. He composed his features in an expression of embarrassed
seriousness, the kind of horrible facial expression that all ID
carried, and the boy clicked his mouse and gestured at the door.
``One hour,'' he said.

Lu held the door for Matthew and led him down the fire-stairs, back
into the mall, back onto the street, back among the counterfeiters,
and a short way to a noodle stall that was thronged with people,
and that's when Matthew's mouth began to generate so much saliva
that he had to surreptitiously blot the corners of his lips on the
sleeve of his cheap cotton jacket.

Moment later, he was eating. And eating. And eating. The first bowl
was pork. Then beef. Then prawn. Then some Shanghai dumplings,
filled with pork. And still he ate. His stomach stretched and the
waistband of his jeans pinched him, and he undid the top button and
ate some more. Lu goggled at him all the while, fetching more bowls
of dumplings as needed, bringing back chili sauce and napkins. He
sent and received some texts, and Matthew looked up from his work
of eating at those moments to watch Lu's fierce concentration as he
tapped on his phone's keypad.

``Who is she?'' Matthew asked, as he leaned back and allowed the
latest layer of dumplings to settle in his stomach.

Lu ducked his head and blushed. ``A friend. She's great. She
organized, you know --'' He waved his chopsticks in the direction of
the counterfeiters' market. ``She's -- I don't know what I would
have done without her. She's why I'm not in jail.''

Matthew smiled wryly. ``You'd have gotten out by now.'' He plucked at
his loose shirt. ``Though you might be a few sizes smaller.''

Lu showed Matthew a picture of a South China girl on his phone. She
looked like the perfect model of South China womanhood --
fashionable clothes and hair, a carefully made up double-eyelid, an
expression of mischief and, what, power? That sense of being on top
of her world and the world in general. Matthew nodded
appreciatively. ``Lucky Lu,'' he said.

Lu dropped his voice. ``She's amazing,'' he whispered. ``She got me
papers, cancelled my phone, let the number go dead, then scooped it
up again with a different identity, then forwarded it through a --''
he looked around dramatically and pitched his voice even lower --
``Falun Gong switchboard in Macau, then back to this phone. That's
why you were able to call me. It's incredible -- I'm still in touch
with everyone, but it's all through so many blinds that the zengfu
have no idea where I am or how to trace me.''

``How does she know all this?'' Matthew asked, gently, the dumplings
settling like rocks in his stomach. He was a dead man. ``How do you
know she isn't police herself?''

``She can't be,'' Lu said. ``You'll see why, once we meet up with her.
This much I'm sure of.''

But Matthew couldn't shake the knowledge that this girl would be
taking him back to prison. In prison, everyone had been an
informant. If you informed on your fellow prisoners, you got more
food, more sleep, lighter duty. The best informants were like
little bosses, and the other prisoners courted their favor like
they were on the outside, giving them the equivalent of the ``3 Gs''
-- golf, girls and gambling -- with whatever they could scrape up
from the prison's walls. Matthew had never informed and had never
been informed upon. He always chose the games he played, and he
never played a game he couldn't win.

And so he was numb when he met Jie, who smelled wonderful and had
fantastic manners and a twinkling smile. She had his new identity
papers, with the right picture, but a different name and identity
number, and a fingerprint that he was sure wasn't his own on the
back. She chatted amiably as they walked, about
inconsequentialities, the weather and the food, football scores and
gossip about celebrities, a too-perfect empty-head that made him
even more suspicious of this girl and her impeccable acting.

She led them to a small, run-down handshake building in the old
Cantonese part of town. This was where Matthew had grown up, the
``city-within-a-city'' that the Cantonese had been squeezed into as
South China ceased to be merely a place and had become a symbol for
the New China, the world's factory. Being back in these familiar
streets made him even more prickly, giving him the creeping
certainty that he would be recognized any second, that some poor
boyhood friend of his would be marked by this secret policewoman
and sent to prison with him. He steeled himself to keep walking,
though with each step he wanted to turn and bolt.

The flat she led them to had once been half of a tiny apartment;
now it was reduced to a single, tiny room with piles of girly
clothes and shoes, several computers perched on cheap desks, a sink
whose rim was covered in cosmetics, and a screened-off area that
presumably hid the toilet. The shower was next to the stove and
sink, a tiled square in the corner with a drain set into the floor,
a shower-head anchored to the wall, a curtain rail bolted to the
ceiling.

Once the door was closed, Lu's girlfriend changed demeanour so
abruptly, it was as though she had removed a mask. Her face was now
animated with keen intelligence, her bearing aggressive and keen.
``We need to get you new clothes,'' she said. ``A shave, a haircut,
some money --''

One thing Matthew had learned in prison was the importance of not
getting carried along by other people's scripts. A forceful person
could do that: write a script, spin it out for you, put you in a
role, and before you knew it, you were smuggling sealed packages
from one part of the prison to another. Once someone else was
writing the script, you were all but helpless.

``Wait,'' he said. ``Just stop.'' She looked at him mildly. Lu was less
calm -- Matthew could tell at a glance that he was completely in
this woman's power. ``Madame, I don't mean to be rude, but who the
hell are you, and why should I trust you?''

She laughed. ``You want to know if I'm zengfu,'' she said. Lu looked
scandalized, but she was taking it well. ``Of course you do. I've
got money, apartments, I know where to get good ID papers --''

``And you're very bossy,'' Matthew said.

``I certainly am!'' she said. ``Now, have you ever heard of Jiandi?''

He \emph{had} heard that name. He thought about it for a moment,
casting his mind back to the distant, dreamlike time before prison.
``The radio lady?'' he said, slowly. ``The one who talks to the
factory girls?''

``Yes,'' she said. ``That's the one.''

``OK,'' he said. ``I've heard of her.''

Lu grinned. ``And now you've met her!''

Matthew thought about this for a moment, staring into the girl's
carefully made-up eyes, fringed with long, dark lashes. Finally he
said, ``No offense, but anyone can claim to be someone who no one
has ever seen.''

Lu started to speak, but she held her hand up and silenced him.
``He's right,'' she said. ``Tank, the only reason I'm walking around
free, still broadcasting, is that I am a very paranoid lady. Your
friend's paranoia is just good sense. Have you ever considered that
you've never \emph{listened} to me broadcasting, Tank? You've been
here plenty for the broadcasts, but you've never tuned in. For all
you know, I \emph{am} zengfu, infiltrating your ranks with a giant,
elaborate counterfeit that has other cops calling in, pretending to
be listeners to a show that never goes any farther than the room
I'm sitting in.'' Lu's mouth opened and shut, opened and shut. She
laughed at him. ``Don't worry, I'm no cop. I'm just pointing out
that you're a very trusting sort of boy. Maybe too trusting. Your
friend here is a little more cautious, that's all. I thoroughly
approve.''

Matthew found himself hoping that this girl wasn't a cop for the
simple reason that he was starting to like her. Not to mention that
if she was a cop, he'd go straight back to jail, but now that his
panic was receding, he was able to consider what she would be like
as a comrade. He liked the idea.

``OK,'' he said. ``So, if you're Jiandi, then it should be easy for
you to prove it. Just do a show, and I'll tune in and listen to
it.''

``How do you know Jiandi isn't a cop?'' She had a twinkle in her
eye.

``Not even the cops are that devious,'' he said. ``They couldn't stand
to have all those Falun Gong ads and all that seditious talk about
the party -- it wouldn't last a week, let alone years and years.''

She nodded. ``I think so, too. Lu, do you agree?''

Lu, still miserable looking, nodded glumly.

``Cheer up,'' she said. ``You get to have a little solo time with your
friend!''

They ended up at a new game cafe, far off on the metro line, by the
Windows on the World theme-park. Matthew's father had taken him
there once, and he'd gotten to dress up in ancient battle-armor,
fire arrows at targets while a man with a Cantonese accent dressed
like an American Indian gave him pointers. It had been fun, but
nothing so nice as the games that Matthew was already playing.

The metro let them off just around the corner from it, in front of
a giant, run-down hotel that had been closed the last time Matthew
came through here. The game cafe was in the former restaurant,
something pirate themed with a huge fake pirate ship on the roof.
Inside, it was choked with smoke and the tables had been formed
into the usual long stretches with a PC every meter or so. About
half of them were occupied, and in one corner of the restaurant
there were fifty or sixty gamers who were clearly gold-farmers,
working under the watchful eye of an older goon with a hard face
and a cigarette in one corner of his mouth. It was incredibly hot
inside the cafe, twenty degrees hotter than outside, and it was as
dark and dank as a cave. Matthew felt instantly at home.

Lu shoved some folded up bills at the old man behind the counter,
an evil-looking, toothless grandfather with a pronounced hump and
two missing fingers on one hand. Lu looked back at Matthew, then
ordered a plate of dumplings as well. The man drew a styrofoam tray
out of a chest freezer, punctured the film on top, and put it in
the microwave beside him at the reception desk. ``Go,'' he croaked,
``I'll bring them to you.''

Matthew and Lu sat down at adjacent PCs far from the rest of the
crowd, next to a picture window that had been covered over with
newspapers. Matthew put his eye up to a rip in the paper and peeked
out at the ruins of an elaborate, nautical-themed swimming pool
outside, complete with twisting water-slides and fountains, now
gone green and scummy. ``Nice hotel,'' he said.

Lu was mousing his way over to Jiandi's web-page, weaving the
connection through a series of proxies, looking up the latest
addresses for her stream mirrors, finding one that worked. ``I think
we'll have 45 minutes at least before anyone notices that this PC
is doing something out-of-bounds. I trust that will be plenty of
time for you to satisfy your suspicious mind.''

Matthew saw that Lu was really angry, and he swallowed his own
anger -- something else he'd had plenty of practice at in prison.
``I just want to be safe, Lu. This isn't a game.'' Then he heard his
own words and grinned. ``OK, it \emph{is} a game. But it's also real
life. It has consequences.'' He plucked at the shirt that hung loose
on his skinny body. ``It wouldn't hurt you to be more careful.''

Lu said nothing, but his lips were pursed and white. The old man
brought them their dumplings and they ate them in silence. They
were miserable dumplings, filled with something that tasted like
shredded paper, but they were still better than prison chicken's
feet.

Matthew looked at the boy. He was always thoughtful -- a strange
thing for a tank to be -- and considerate, and brave. He hadn't
been in Matthew's original guild, but when Boss Wing had put him in
charge of the whole elite squad, they'd come willingly, seeing in
Matthew a strategist who could lead them to victory. And when
Matthrew had started whispering to them about the Webblies, Lu had
been as excited as anyone. All that seemed so long ago, a different
life and different time, before a policeman's baton had knocked him
down, before he had gone to prison, before he'd turned into the man
he was now. But Matthew was back in the world now, and Lu had been
living on his wits for months, and --

``I owe you an apology,'' he said, setting down hs chopsticks. ``I
still don't know if I can trust your friend, but I could have been
a little smarter about how I said it. It's been a strange day -- 36
hours ago, I was wearing a prison uniform.''

Lu stared at him, and then a little smile snuck into the corners of
his mouth. ``It's all right,'' he said. ``Here, she's starting.'' He
popped out his earwig, already paired with the computer's
sound-system, wiped it on his sleeve, and handed it to Matthew.
Matthew screwed it into his ear.

``Hello, sisters,'' came the familiar voice. ``It's a little early, I
know, but this is a short and special broadcast for you lucky
ladies who have the day off, are sick in the infirmary, or happen
to have snuck headphones into the factory. Hello, hello, hello.
Shall we take a phone call or two?''

Lu grinned at Matthew and stood and walked out of the cafe. Matthew
touched the earwig, thought about going after him, decided not to.
A moment later, Jiandi said, ``There we go, hello, hello.''

``Hello Jiandi,'' said Lu. Matthew put his eye back up to the gap in
the newspaper-covered glass and found himself staring at a grinning
Lu, standing behind the building, phone to his head.

``Tank!'' she squealed. ``How fantastic to hear from you again. It's
been ages since you came on my show! Tell me, Tank, what's on your
mind today?''

``Justice,'' Lu/Tank said. Matthew found himself laughing quietly,
and he ducked his head so as not to draw attention. ``Justice for
working people. We come to Guanddong Province because they say that
we will be rich. But when we get here, we have bad working
conditions, bad pay, and everything is stacked against us. No one
can get real papers to live here, so we all buy fakes, and the
police know they can stop us at any time and put us in jail or send
us away because we don't have real documents. Our bosses know it,
so they lock us in, or beat us, or steal our pay. I have been here
for five years now, and I see how it works: the rich get richer,
the poor get used up and sent back to the village, ruined. The
corrupt government runs on bribes, not justice, and any attempt by
working people to organize for a better deal is met with violence
and war. The corrupt businessmen buy corrupt policemen who work for
corrupt government.

``I've had enough! It's time for working people to organize -- one
of us is nothing. Together, we can't be stopped. China's
revolutions have come and gone, and still, the few are rich and the
many are poor. It's time for a worldwide revolution: workers in
China, India, America -- all over -- have to fight together. We
will use the Internet because we are better at the Internet than
our bosses are. The Internet is shaped like a worker's
organization: chaotic, spread out, without a few leaders making all
the decisions. We know how to interface with it. Our bosses only
understand the Internet when they can make it shaped like them,
forcing all our clicks through a few bottlenecks that they can own
and control. We can't be controlled. We can't be stopped. We will
win!''

Jiandi laughed into the mic, a throaty, sexy sound. ``Oh, Tank! So
serious! You make us all feel like silly children with your talk!

``But he's right sisters, you know he is. We worry about our little
problems, our bosses trying to screw us or cheat us; police chasing
us, our networks infected and spied on, but we never ask
\emph{why}, what's the system \emph{for}?'' She drew in a deep
breath. ``We never ask what we can do.''

A long silence. Matthew clicked on the computer, verified that he
was indeed tuned into the Factory Girl Show. He felt an unnameable
emotion inside his chest, in his belly. She was what she said she
was. Not a cop. Not a spy.

Well, either that or the whole thing was a huge setup, and the
police had been running this woman's operation for years now,
deceiving millions, just to have this insider. That was an
incredibly weird idea. But sometimes the politburo was incredibly
weird.

``We'll know what to do. Soon enough, sisters, have no fear. Keep
listening -- tune in tonight for our regular show -- and someday
\emph{very soon} we'll tell you what you can do. Wait and wait.

``And you policemen and government bureaucrats and bosses listening
now? Be afraid.''

Her voice clicked off, and a cheerful lunatic started saying crazy
things about how great Falun Gong was, the traditional junk
advertising he'd heard on Jiandi's show before.

He thoughtfully chewed another newspaper dumpling and waited for Lu
to make his way back into the cafe. He'd been out of prison for
less than two days and his life was a million times more
interesting than it had been just a few hours before. And he had
dumplings. Things were happening -- big things.

Lu shook his hand again, and the two of them left quickly, heading
for the metro entrance. As they ran down the stairs, Lu leaned over
and said, quietly, ``Wait until you hear what we've got planned.''
His voice was tight, excited. Almost gleeful.

``I can't wait,'' Matthew said. There was a hopeful feeling bubbling
up inside him now. When was the last time he'd felt hopeful? Oh
yes. It was when he quit Boss Wing's gold farm, taking his guildies
with him, and set up his own business. That hadn't ended well, of
course. But the hope had been \emph{delicious}. It was delicious
now.

\tb

Justbob had her whole network online. These were the best fighters
in the IWWWW, passionate and committed. They'd been fighting off
Pinkertons and dodging game-security for a year, and it had made
them hard. Some of them had been beaten in real life, just like
Justbob and Krang and BSN, and it was quite a badge of honor to
replace your user-icon with a picture of your injuries -- an x-ray
full of shattered bones, a close up of a grisly row of stitches.

She loved her fighters. And they loved her.

``Hello, pretties,'' she cooed into her earwig, adjusting the icepack
she'd wedged between her tailbone and the chair. They were
operating out of a new cafe now, still in the Geylang, which was
the best place to be in Singapore if you wanted to be a little out
of bounds without attracting too much police attention. ``Ready for
the latest word?''

There was a chorus of cheers from all around the world. Justbob
spoke Malay, Indonesian, English, Tamil, and a little Mandarin and
Hindi, but they tended to do things in English, which everyone
spoke a little of. There was a back-channel, of course, a text-chat
where people helped out with translations. They had to speak slow,
but it worked.

``We are going to take on four worlds, all at the same time:
Mushroom Kingdom, Zombie Mecha, Svartalfaheim Warriors, and Magic
of Hogwarts.'' She watched the backchannel, waited until the
translations were all sorted out. ``What do I mean by 'take on?' I
mean \emph{take over}. We're going to seize control of the
economies of all four worlds: the majority of the gold, prestige
items, and power. We're going to do it fast. We're going to be
unstoppable: whenever an operation is disrupted, we will have three
more standing by. We're going to control the destiny of every boss
whose workers toil in those worlds. We're going to rock their
corporate masters. We're going to fight off every Pinkerton, either
converting them to our cause or beating them so badly that they
change careers.

``To do this, we're going to need many thousands of players working
in coordination. Mostly that means doing what they do best: making
gold. But we also expect heavy resistance once word gets out about
what we're up to. We'll need fighters to defend our lines from
Pinkertons, of course, but we also need a lot of distraction and
interference, all over, including -- no, \emph{especially} -- in
worlds where we're \emph{not} going for it. We want game management
thoroughly confused until its too late. You will need proxies,
\emph{lots of them}, and as many avs as you can level up. That's
your number one task right now -- level as many avs as you can, so
that you can switch accounts and jump into a new fighter the second
an old one gets disconnected.'' She watched the backchatter for a
second, then added, ``Yes, of course, we're working on that now. In
a day or so, we'll have prepaid account cards for all of you.
They'll need US proxies to run, so make sure you've got a good list
of them.''

She watched the chatter for another moment. ``Of course, yes, they
will try to shut down the proxies, but if they do, there will be
\emph{howls} from their American players. Do you know how many
Americans sneak out of their work networks to play during the day
using those proxies? If they start blocking proxies, they'll be
blocking some of their best customers. And of course, many
Mechanical Turks are on school networks, using proxies to log in to
their jobs. They can't afford to block all those proxies -- not for
long!''

The back-channel erupted. They liked that. It was good strategy,
like when you aggroed a boss and then found a shelter that put some
low-level baddies between you and it, and provoked a fight where
they all fought each other instead of you. Justbob wished she could
say more about this, because the deviousness of it all had given
her an all-day, all-week, all-month smile when they'd worked it out
in one of the high-level cell meetings. But she understood the need
for secrecy. It was a sure bet that some of the fighters on this
conference were working for the other side; after all, some of
\emph{their} spies were inside the companies, weren't they?

``All right,'' she said, ``all right. Enough talk-talk. Let's kill
something.'' Her headphone erupted in ragged cheering and she
skirmished with her commanders for a happy hour until The Mighty
Krang came and dragged her away so that she could eat dinner.

Big Sister Nor waited until she was seated, with food on her plate
-- sizzling cha kway teow and fried Hokkien noodles, smelling like
heaven-- before she started speaking. ``All right,'' she said. ``Our
man's landing in Shenzhen tomorrow. We've got people who'll help
get him out of the port safely, and he says he's got our cargo, no
problems there. He's been logging in on the voyage, he says he can
get us hundreds of Turks.''

The Mighty Krang waved his chopsticks at her. ``Do you believe
him?''

Big Sister Nor chewed and swallowed thoughtfully. ``I think I do,''
she said. ``He's all enthusiasm, that one. He's one of those kids
who absolutely \emph{loves} gaming and wanted to be part of the
'magic,' but discovered that he was working every hour God sent,
and there were always hidden rules that ended up docking his pay.''
The other two nodded vigorously -- they recognized the pattern, it
was the template for sweatshops all over the world. ``His employers
told him to be grateful to have such a wonderful opportunity and
didn't he know that there were plenty more who'd have his job if he
didn't want it?''

``OK, so he's upset -- what makes you think he can deliver lots of
other upset people?''

She shrugged and speared a prawn. ``He's a natural networker, a real
do-er. You should hear him talk about that shipping container of
his! It's a real hotel on the high seas. Very ingenious. And his
guildies say he's bloody sociable. A nice guy. The kind of guy you
listen to.''

``The kind of guy you follow?'' asked Justbob, scratching at her
scarred eye-socket. She could forget about the itch and the ache
from the side of her face when she was in conference with her
warriors, but she lost that precious distraction the rest of the
time. And her dreams were full of phantom aches from the ruined
socket, and she sometimes woke with tears on her face.

Big Sister Nor said, ``That's what I think.''

The Mighty Krang drank some watermelon juice and drew glyphs in the
table with the condensation. The waitress -- a pretty Tamil girl --
scowled at him with mock theatricality and wiped it away. All the
waitresses had crushes on The Mighty Krang. Even Justbob had to
admit that he was pretty. ``I don't like the idea,'' he said. ``This
is about, you know, \emph{workers}.''

Big Sister Nor fixed him with a level stare. ``You mean 'he's white,
I don't trust him.' He's a worker, too -- even though he works for
the game. We're \emph{all} workers. That's the point of the
Webblies. All workers in one big union -- solidarity. Start making
differences between workers who deserve the union and workers who
don't and the next thing you know, your job will be handed over to
the workers you left out of your little private clubhouse. Krang,
if you're not clear on this, you're in the wrong place. Absolutely
the wrong place. Do I make myself clear?''

This was a different Big Sister Nor than the one they usually knew,
the motherly, patient, understanding one. Her voice was brittle and
stern, her stare piercing. Krang visibly wilted under its glare.
``Fine,'' he said, without much conviction. ``Sorry.'' Justbob felt
embarrassed for him, but not sympathetic. He knew better.

They finished the meal in silence. Big Sister Nor's phone buzzed at
her. She looked at the face, saw the number, put it back down
again. There was a rule: no taking calls during ``family dinners''
between the three of them. But BSN was visibly anxious to get to
this one. She began to eat faster, as fast as she could with her
twisted hand.

``Who was it?'' Justbob asked.

``China,'' she said. ``Urgent. Our boy from America.''

\tb

Ping didn't like the port. Too many cops. He had good papers, but
not even the best papers would stand up long to a cop who actually
radioed in the ID and asked about it. The counterfeiters claimed
that they used good identities for the fakes, real people who
weren't in any kind of trouble, but who knew whether to believe
them?

Anyway, it was just crazy. The gweilo was supposed to wait until
the ship came into dock, change into a set of clean clothes, pin on
ID from his father's company, and just \emph{walk out} of the port,
flashing his identification at anyone who bothered to ask the
skinny white kid what he was doing, carrying two heavy cardboard
boxes out of the secure region. Once he made it clear of the port,
Ping could take him away, make him disappear into the mix of
foreigners, merchants, and business-people thronging the region.

Ping had asked around, found a Webbly who's brother had worked as a
hauler the year before, gotten information about where Leonard
would most likely emerge, and had emailed all that info to Leonard
as he trundled across the ocean.

But there weren't supposed to be \emph{this many} cops, were there?
There were hundreds of them, it seemed like, and not just uniforms.
There were plenty of especially tall men with brush-cuts and
earpieces, dressed like civilians, but moving with far too much
coordination and purpose. Ping walked past the entrance twice, the
first time conducting an imaginary argument with someone over his
phone, trying to exude an aura of distraction that would make him
seem harmless. The second time he walked past while staring
intently at a tourist map, trying to maintain the show of
helplessness. In between, he checked his watch, saw that Leonard
was an hour late, sent a message back to Lu and asked him to see if
he could email Big Sister Nor and find out what was going on. This
was the trickiest moment, since the ship's satellite link was down
while it was in dock, and so Leonard's stolen network connection
was down with it. Once he was clear of the port, they'd give him a
prepaid phone, get him back on the grid, but until then\ldots{}

He nearly dropped the tourist map when his phone went off. A nearby
cop, the tallest man he'd ever seen, looked hard at him and he
smiled sheepishly and withdrew his phone and tried to control the
shaking in his hands as he touched it to life, hoping the noise
hadn't aggroed him.

``Is he with you?'' Big Sister Nor's Mandarin was heavily accented,
but good. He recognized the voice instantly from many late-night
chat sessions and raids.

``Hi!'' he said, in a bright, brittle voice, trying to sound like he
was talking to a girlfriend or sister. ``It's great to hear from
you!''

``You haven't seen him yet?''

``That's right!'' he said, pasting a fake grin on his face for the
benefit of the security man.

``Shit. He was due out hours ago.'' Big Sister Nor went quiet. ``OK,
here's the thing. Whatever happened to him, we need those boxes.''
She cursed in some other language. ``I should have just had him put
the boxes in the container. He wanted to come see you all so badly,
though --'' She broke off.

``OK!'' he said, walking as casually as he could away from the cop.
There was a spot, a doorway in front of a closed grocery store down
the road. He could go there, sit down, talk this through.

``A lot of cops where you are, huh? Don't answer. Listen, Ping, I
need to know -- can you get into the port? If he doesn't make it
out?''

He swallowed. ``I don't think so,'' he whispered. He was almost to
his doorway now.

``What if you have to?''

He was a raid leader, a master strategist. He was no Matthew, but
still, he understood how to get in and out of tight places. And
he'd been a pretty good climber a few years ago, before he'd found
gold-farming. Maybe he could go over the fence? He felt like
throwing up at the thought. There were so many cameras, so many
cops, the fence was \emph{so high}.

``I'd try,'' he said. ``But I would almost certainly go to jail.'' He'd
been held for three days in the local lockup along with most of the
strikers and then released. It had been bad enough -- not as bad as
Matthew's stories -- and he never wanted to go back. ``You have to
see this place, Nor, it's like a fortress.''

She sighed. ``I know what ports look like,'' she said. ``OK, tell you
what -- you wait another hour, see if you can find him. I'll work
on something else here, and call you.''

``OK,'' he said.

Casually, he drifted back along the length of the high fence that
guarded the port, keenly aware of the cameras drilling into the
back of his neck. How many times could he pass by before someone
decided to figure out what he was doing there? They should have
brought a whole party, half a dozen of the gang who could trade off
looking for the stupid gweilo. Ping shook his head in disgust. It
had been fun to know Leonard when he was a kid in California and
they were five kids in China -- exotic, even. No one else partied
with exotic foreigners with bad accents.

It was even exciting when the gweilo had turned into a smuggler for
the cause, crossing the ocean with his booty of hard-earned prepaid
game-cards that would let them all fly under the game companies'
radar.

But it was no longer exciting now that he was about to go to jail
because some dumb kid from across the ocean couldn't figure out how
to get his ass out of the port of Shenzhen.

\tb

It had gone better than Wei-Dong had any right to expect. After
they took to the sea, he'd cut the freighter's WiFi like butter and
hopped onto their satellite link. It was slow -- too slow for
gaming -- but it was OK for messaging and staying in touch with
both the Webblies and the cell of Turks he'd pieced together from
the best people he knew. He'd let himself out of the container on
the first night and climbed up to the top of the stack, trailing
his solar rig and water collector behind him, and affixed both to
an inconspicuous spot on the outside face of the topmost
containers, where no crewmember could spot them. Again, the
operation went off without a hitch.

By day three, he was wishing for some trouble. There was only so
much time he could spend watching the planning emerge on the Webbly
boards, especially since so many of the pieces of the plan were
closely guarded secrets, visible only as blank spots in his
understanding of where he was going and why he was going there. A
thousand times a day, he was struck with the absolute madness of
his position -- a smuggler on the high seas, going to make
revolution in Asia, at the tender age of 18! It was fabulous and
terrifying, depending on what mood he was in.

Mostly that mood was \emph{bored}.

There was nothing to do, and by day five, he was snaffling up all
the traffic on the boat, watching the lovesick crew of six Filipino
sailors sending long-distance romantic notes to their pining
girlfriends. It was entertaining enough downloading a Tagalog
dictionary so he could look up some of the phrases they dropped
into the letters, but after a while, that paled too.

And there were still \emph{days} to go, and the rains had come and
filled up his reservoirs, and so he had water to drink and cook
with, and so he didn't even have itchy skin or malnutrition to keep
him distracted, and so he'd started to do stupid things.

He'd started to sneak around.

Oh, only at night, of course, and at first, only among the
containers, where the crew rarely ventured. But there wasn't much
to see in the container spaces, just the unbroken, ribbed expanses
of containers, radio tagged and painted with huge numbers,
stickered over and locked tight.

So then he started to sneak over to the crew's quarters.

He knew what they'd look like. You can book passage on a freighter,
take a long, weird holiday drifting from port to port around the
world. The travel agents who sell these lonely, no-frills cruises
had plenty of online photos and videos and panoramas of the
accommodations and common rooms. They looked like institutional
rooms everywhere, with big scratched flat-panel displays, worn and
stained carpet, sagging sofas, scuffed tables and chairs. The
difference being that shipside, all that stuff was bolted down.

But after days stuck inside his little secret fortress of solitude,
any change of scenery sounded like a trip to Disneyland and a half.
And so that's how he found himself strolling into the ship's
kitchen at 2AM ship's time -- they were living on Pacific time, and
he'd shifted to Chinese time after they put to sea, so this wasn't
much of a hardship. In the fridge, sandwich fixings, Filipino
single-serving ice cream cones, pre-made boba tea with huge pearls
of tapioca in it, and cans of Starbucks frappucino. He helped
himself, snitching it all into a shoulder-bag he'd brought along,
scurrying back to his den to scarf it down.

That was the first night. The second night, he ate his snack in the
TV room, watching a bootleg DVD of a current-release comedy movie
that opened the day he left LA. He kept the sound low, and even
used the bathroom outside the common room on the corridor that led
to the crew's quarters. He crept around on tiptoe, and muted the TV
every time the ship creaked, his heart thundered as his eyes darted
to each corner of the room, seeking out a nonexistent hiding spot
among the bolted-down furniture.

It was the best night of the trip so far.

So the next night, he had to go further. After having a third pig
out and watching a Bollywood science fiction comedy movie about a
turbanned robot that attacked Bangalore, only to be vanquished by
IT nerds, he snuck down into the engine rooms.

Now \emph{this} was a change of scenery. The door to the engine
room was bolted but not locked, just like all the other doors on
the ship that he'd tried. After all, they were in the middle of the
damned ocean -- it wasn't like they had to worry about
cat-burglars, right? (Present company excepted, of course!).

The big diesel engines were as loud as jets. He found a pair of
greasy soundproof earmuffs and slipped them over his ears, cutting
the noise down somewhat, but it still vibrated up through the soles
of his sneakers, making his bones shake. Everything down here was
fresh and gleaming, polished, oiled and painted. He trailed his
fingers over the control panels, gauges, shut-off valves, raised
his arms to tickle the flexi-hoses that coiled overhead. He'd gamed
a couple of maps set in rooms like this, but the experience in real
life was something else. He was actually \emph{inside} the machine,
inside an engine so powerful it could move thousands of tons of
steel and cargo halfway around the world.

Cool.

As he slipped his muffs off and carefully re-hung them, he noticed
something he really should have spotted on the way in: a little
optical sensor by the engine-room door at the top of the steel
crinkle-cut nonskid stairs, and beside it, a pin-sized camera
ringed with infrared LEDs. Which meant\ldots{}

Which meant that he had tripped an invisible alarm when he entered
the room and broke the beam, and that he'd been recorded ever since
he arrived. Which meant\ldots{}

Which meant he was \emph{doomed}.

His fingers trembled as he worked the catch on the door and slipped
out into the steel shed that guarded the engine-room entrance at
the crew end of the deck. He looked left and right, waiting for a
spotlight to slice through the pitchy night, waiting for a siren to
cut through the roar of the ocean as they sliced it in two with the
boat's mighty prow.

It was quiet. It was dark. For now. The ship only had one night
watch-officer and one night-pilot, and from his network spying, he
knew the duty was an excuse to send email and download pornography,
so it may have been that neither of them had noticed the alert --
yet.

He crept back among the containers, moving as fast as he dared,
painfully aware of how vividly he would stand out to anyone who
even casually glanced down from the ship's bridge atop the
superstructure. Once he reached the containers, he slipped onto the
narrow walkway that ringed the outside of the ship and took off
running, racing for his nest. As he went, he made a mental
checklist of the things he would have to do once he got there,
reeling in his solar panels and antennas, his water collectors.
He'd button down his container as tight as a frog's ass, and they
could search for months before they'd get to his -- meanwhile, he'd
be in Shenzhen in a couple days. Then it would just be a matter of
evading the port security -- who'd be on high alert, once the crew
alerted them to the stowaway. Argh. He was \emph{such} an idiot. It
was all going to crash and burn, just because he got \emph{bored}.

Cursing himself, hyperventilating, running, he skidded out on the
deck and faceplanted into the painted, bird-streaked steel. The
pain was insane. Blood poured from his nose, which he was sure he'd
broken. And now the ship was rocking and pitching hard, and holy
crap, look at those clouds streaking across the sky!

This was not going well. He cornered wobbily around the container
stack, had a hairy, one-foot-in-the-sky moment as the huge ship
rolled beneath him and his hand flailed wildly for the guardrail,
then he caught himself and finished the turn, racing to his
container. Once there, he scrambled along the runs that marked the
course of the life-support tentacles trailing from his box, and he
disconnected each one, working with shaking hands. Hugging the
flexi-hose, cabling, solar cells and antenna to his chest, he
spidered down the container-faces and slipped inside just as
another roll sent him sprawling on his ass.

He undogged the hatches on his airtight inner sanctum and let
himself in. The ship was rocking hard now, and his kitchen stuff,
carelessly left lying around, was rattling back and forth. He
ignored it at first, diving for his laptop and punching up the
traffic-logs from the ship's network, but after a can of tuna
beaned him in the cheek, raising a welt, he set the computer down
and velcroed it into place, then gathered up everything that was
loose and dumped it into his bolted-down chests. Then he went back
to his traffic dumps, looking for anything that sounded like an
official notice of his discovery.

The night-time traffic was always light, some telemetry, the flirty
emails from the skeleton crew. Tonight was no exception. The file
stopped dead at the point that he'd reeled in his antenna, but it
probably wouldn't have lasted much longer anyway. The rain was
pounding down now, a real frog-strangler, sounding like a barrage
of gravel on the steel containers all around him. After a few
minutes of this, he found himself wishing he'd taken the earmuffs.
A few minutes later and he'd forgotten all about the earmuffs, and
he was grabbing for a bag to heave up his stolen food into. The
barfing and the rolling didn't stop, just kept going on and on, his
stomach empty, trying to turn itself inside-out, slimy puke-smears
everywhere in the tiny cabin. He tried to remember what you were
supposed to do for sea-sickness. Watch the horizon, right? No
horizon in the container, just pitching walls and floor and
unsteady light from the battery-powered LED fixtures he'd glued to
the ceiling. The shadows jumped and loomed, increasing the
disorientation.

It was the most miserable he'd ever been. It seemed like it would
never end. At a certain point, he found himself thinking of what it
would be like to be crammed in with 10 or 20 other people, in the
pitch dark, with no chemical toilet, just a bucket that might
overturn on the first pitch and roll. Crammed in and locked in, the
door not due to be opened for days yet, and no way to know what
might greet you at the other side --

Suddenly, he didn't feel nearly so miserable. He roused himself to
look at his computer a little more, but staring at the screen
instantly brought back his sea-sickness. He remembered packing some
ginger tablets that were supposed to be good for calming the
stomach -- he'd read about them on a FAQ page for people going on
their first ocean cruise -- and searching for them in the rocking
box distracted him for a while. He gobbled two of them with water,
noting that the tank was only half full and resolving to save every
drop now that his collector was shut down.

He wasn't sure, but it seemed like the storm was letting up. He
drank a little more water, checked in with his nausea -- a little
better -- and got back to the screen. It was a minor miracle, but
there was no report at all of him being spotted, no urgent
communique back to corporate HQ about the stowaway. Maybe they
hadn't noticed? Maybe they had been focused on the storm?

And there the storm was again, back and even more fierce than it
had been. The rocking built, and built, and built. It wasn't
sickening anymore -- it was \emph{violent}. At one point, Wei-Dong
found himself hanging on to his bed with both hands and feet, his
laptop clamped between his chest and the mattress, as the entire
ship rolled to port and hung there, teetering at an angle that felt
nearly horizontal, before crashing back and rocking in the
\emph{other} direction. Once, twice more the ship rolled, and
Wei-Dong clenched his teeth and fists and eyes and prayed to a
nameless god that they wouldn't tip right over and sink to the
bottom of the ocean. Container ships didn't go down very often, but
they \emph{did} go down. And not only that -- about half a percent
of containers were lost at sea, gone over the side in rough water.
His father always took that personally. One percent didn't sound
like a lot, but, as Wei-Dong's father liked to remind him, that was
20,000 containers, enough to build a high-rise out of. And the
number went up every year, as the seas got rougher and the weather
got harder to predict.

All this went through Wei-Dong's head as he clung for dear life to
his bolted-down bed, battered from head to toe by loose items that
he'd missed when he'd packed everything into his chest. The ship
groaned and strained and then there was a deep metallic grating
noise that he felt all the way to his balls, and then --

-- the container \emph{moved}.

It was a long moment and it seemed like everything had gone silent,
as the sensation of sliding across the massive deck tunneled
through his inner ear and straight into the fear center of his
brain. In that moment, he knew that he was about to die. About to
sink and sink and sink in a weightless eternity as the pressure of
the ocean all around him mounted, until the container imploded and
smeared him across its crumpled walls, dissipating in red streamers
as the container fell to the bottom of the sea.

And then, the ship righted itself. There were tears in his eyes,
and a dampness from his crotch. He'd pissed himself. The rocking
slowed, slowed. Stopped. Now the ship was bobbing as normal, and
Wei-Dong knew that he would live.

His hidey-hole was a wreck. His clothes, his toys, his survival
gear -- all tossed to the four corners. Thankfully, the chemical
toilet had stayed put, with its lid dogged down tight. That would
have been \emph{messy}. Puke, water, other spills slicked every
available surface. According to his watch, it was 4AM on his
personal clock. That made it, uh, 11AM ship's time, which was set
to Los Angeles. If he'd done the math right, it was about 6AM in
their latitude, which should be just about directly in line with
New Zealand. Which meant the sun would be up, and the crew would no
doubt be swarming on deck, surveying the damage and securing the
remaining containers as best as they could with the ship's little
crane and tractors. And \emph{that} meant that he'd have to stay
put, amid the sick and the bad air and the mess, wait until that
ship's night or maybe even the next night. And he had no WiFi,
either.

Shit.

He'd brought along some sleeping pills, just in case, as part of
his everything-and-the-kitchen-sink first-aid box. He found the
sealed plastic chest still bungied to one of the wire shelving
units, beside the precious two boxes of prepaid cards, still
securely lashed to the frame. As he broke the blisterpack and
poured a stingy sip of water into his tin cup, he had a moment's
pause: what if they discovered his container while he was drugged
senseless?

Well, what if they discovered it while he was wide awake? It's not
like he could \emph{run away}.

What an idiot he was.

He ate the pills, then set about cleaning up his place as best as
he could, using old t-shirts as rags. He flipped over the mattress
to expose the unpissed-upon side, and wondered when the pills would
take effect. And then he found that he was too tired to do another
thing except for lying down with his cheek on the bare mattress and
falling into a deep and dreamless sleep.

The pills were supposed to be a ``non-drowsy'' formula, but he woke
feeling like his head was wrapped in foam rubber. Maybe that was
the near-death experience. It was now the middle of ship's night,
and real night. Theoretically, it would be dark outside, and he
could sneak out, survey the damage, maybe rig up his WiFi antenna
and find out whether he was about to be arrested when they made
port. But when he climbed gingerly out of his inner box and tried
to open the door of his container, he discovered that it had been
wedged shut. Not just sticky, or bent at the hinge, but properly
jammed up against the next container, with several tons of cargo on
the other side of the door for him to muscle out of the way. Or
not.

He sat down. He had his headlamp on, as the inside of the container
was dark as the inside of a can of Coke. It splashed crazy shadows
on the walls, the stack of batteries, (he praised his own foresight
at using triple layers of steel strapping to keep them in place)
the hatch leading to his inner sanctum.

By his reckoning, they were only three days out of Shenzhen, plus
or minus whatever course-corrections they'd have to make now that
the storm had passed. Theoretically, he could make it. He had the
water, the food, the electricity, provided that he rationed all
three. But the Webblies would be expecting him to check in before
then, and the boredom would drive him loopy.

He thought about trying to saw through the steel container. It was
possible -- the container-converter message boards were full of
talk about what it took to cut up a container and use it for other
purposes. But nothing in his toolkit could manage it. The closest
he could come would be to drill a hole in the skin with his
cordless drill. He'd used it to assemble his nest, he had a couple
spare boxes of high-speed bits in his toolchest. His biggest bit, a
small circular saw, would punch a hole as big as his thumb, but
only after he'd drilled a guide-hole through the steel. 14 gauge
steel, several times thicker than the support-struts he'd drilled
out when doing his interior work.

It would make an unholy racket, but he was on the cargo deck, well
away from the deckhouse. Assuming no one was patrolling the deck,
there was no way he'd be heard over the sound of the sea and the
rumble of the diesels. He told himself that it was worth the risk
of discovery, since getting a hole would mean getting an antenna
out, and therefore getting onto the network and finding out whether
he'd be safe once they got to China.

No time like the present. He found the toolchest, inside a bigger,
bolted-down box, and recovered the drill. He had a spare charger
for it, with an inverter that would run off the battery stack, and
he plugged it in and got it charging. He'd need a lot of batteries
to get through the ceiling.

Several hours later, he realized that the ceiling might have been a
mistake. His shoulders, arms, and chest all burned and ached. He
found himself taking more and more frequent breaks, windmilling his
arms, but the ache wouldn't subside. His ears hurt too, from the
echoey whining racket of the drill, a hundred nightmares of the
dentist's chair. He kept an eye on his watch, telling himself he'd
just work until the morning shift came on duty, to reduce the risk
that the sound would be heard. But it was still an hour away from
shift change when the battery on his drill died, and he discovered
that the last time he'd switched batteries, he'd neglected to push
the dead one all the way into the charger, and now both his
batteries were dead.

That was as good an excuse as any to stop. He fingered the dent
he'd made in the sheet steel through all his hours of drilling. His
fingertip probed it, but barely seemed to sink in at all. He
detached a chair from its anchors and dragged it over, stood on it,
and put an eye to it, and saw a pinprick of dirty grey light, the
first light of dawn, glimmering at the bottom of his drill-hole.

Sleep did not help his arms. If anything, it just made them worse.
It took him five minutes just to get to the point where he could
lift his arms over his face, working them back and forth. He had a
little pot of Tiger Balm, the red, smelly Chinese muscle rub, in
his first-aid box, and he worked it into his arms, shoulders, chest
and neck, thinking, as he did,
\emph{This stuff isn't doing anything}. A few minutes later, a new
burning spread across his skin, a fiery, minty feeling, hot and
cold at the same time. It was alarming at first, but a few seconds
later, it was \emph{incredible}, like his muscles were all letting
go of their tension at once. He took up his drill, checked his
watch -- middle of the first shift, but screw it, the engines were
groaning, no one would hear it -- and went to work.

He punched through five minutes later. Five minutes! He'd been so
close! He put his eye to the hole again, saw sky, clouds, the
shadows of other containers nearby. His wireless antenna awaited.
It had a big heavy magnetic base, powerful rare-earth magnets that
he'd used to attach it to its earlier spot. They'd worked so well
that he'd had to plant both feet on either side of it and heave,
like he was pulling up a stubborn carrot. Now he didn't need the
base, just the willowy wand of the antenna itself. He disassembled
the antenna, reattached it to the bare wire-ends, and then gently,
gingerly, fed it through his dime-sized hole.

He had a moment's pause as he fed it up, picturing it sticking up
among the even, smooth surfaces of the container-tops, as obvious
as a boner at the chalkboard, but he'd been drilling for so long,
it seemed crazy to stop now. A voice in his head told him that
getting caught was even crazier, but he shut that voice up by
telling it to shut up, since getting information on the ship's
status would be vital to completing his mission. And then the
antenna was up.

He grabbed his laptop and logged into the network and began
snaffling up traffic. He could watch it in realtime -- his sniffer
would helpfully group intercepted emails, clicks, pages, search
terms and IMs into their own reporting panels -- but that was just
frustrating, like watching a progress bar creep across the screen.

Instead he went inside his sanctum and made himself a cup of
instant ramen noodles, using a little more of his precious
electricity and water, and then opened up a can of green tea with
soymilk to wash it down. He ate as slowly as he could, trying to
savor every bite and tell his stomach that food was OK, despite the
rock and roll of the past day. During the meal, he heard footsteps
near his container, the grumble of heavy machinery working at the
containers, and his mouth went dry at the thought of his antenna
sticking up there.

Why had he put it there? Because he couldn't bear the thought of
sitting, bored and restless, in his box for days more. Why was he
doing any of it? Why was he on his way to China? Why had he left
home to be a gamer? Why had he learned Chinese in the first place?
Trapped with his own thoughts, he found himself confronting some
pretty ugly answers. He hadn't wanted to be like all the other
kids. He'd wanted to stand out, be special. Different. To know and
understand and be skilled at things that his father didn't know
anything about. To triumph. To be a part of something bigger than
himself, but to be an \emph{important} part. To be romantic and
special. To care about a justice that his friends didn't even know
existed.

It made him all feel sad and pathetic and needy. It made him want
to go plug into his laptop and get away from his thoughts.

It worked. What he found on his laptop was nothing short of
amazing. First there was a haul of photos emailed from the captain
back to the shipping company, showing the cargo deck of the ship
looking like a tumbled Jenga tower, containers scattered
everywhere, on their sides, on their backs, at crazy angles. It
looked as if the entire top layer of boxes had slipped into the
ocean, and then several more layers' worth on the port side. He
looked more closely. His container was on the starboard side, and
the container from the corresponding position on the other side
appeared to be gone. He looked up the ship's manifest, found the
serial number of the container, matched it to a list of overboard
boxes, swallowed. It had been pure random chance that put his box
on the starboard side. If he'd gone the other way, he'd be
raspberry jam in a crushed tin can at the bottom of the ocean.

He scanned the email traffic for information about the mysterious
stowaway, but it looked as though the storm had literally blown any
concern over him overboard. The manifest he had listed the value
for customs of all the containers on the ship. Most of them were
empty, or at least partially empty, as there wasn't much that
America had that China needed, except empty containers to fill with
more goods to ship to America. Still, the total value of the
missing containers went into the hundreds of thousands of dollars.
He winced. That was going to be a huge insurance bill.

Now it was time to get \emph{his} email, something that he'd been
putting off, because that was even riskier; if the ship's own
administrators were wiretapping their own network, they'd see his
traffic. Oh, it wouldn't look like email from him to Big Sister Nor
and his guildies and the Turks back in America. It'd look like
gigantic amounts of random junk, originating on an internal address
that didn't correspond to any known machine on the ship. Its
destination was unclear -- it hopped immediately into TOR, The
Onion Router, which bounced it like a pea in a maraca around the
globe's open relays. He was counting on the ship's lax IT security
and the fact that the crew were always connecting up new devices
like phones and handheld games they picked up in port to help him
slide past the eyes of the network. Still, if they were looking for
a stowaway, they might think of looking at the network traffic.

He sat at his keyboard, fingers poised, and debated with himself.
Deep down, he knew how this debate would end. He could no more stay
off the network and away from his friends than he could stay cooped
up in the tin can without poking his antenna off the ship.

So he did it. Sent emails, watched the network traffic, held his
breath. So far, so good. Then: a rumble and a clatter and a pair of
thunderous \emph{clangs} from above. His heart thudded in his ears
and more metallic sounds crashed through the confined space. What
was it? He placed the noises, connected them to the pictures he'd
seen earlier. The crew had the forklift and tractor out, and the
crane swinging, and they were rearranging the containers for
stability and trim. He yanked his antenna in and dove for the inner
sanctum, dogging his hatch and throwing all loose objects into the
lockers before flinging himself over the bed and grabbing hold of
the post and clinging to it with fingers and toes as the container
rocked and rolled for the second time in 24 hours.

\tb

``So where'd you end up?'' Ping asked, passing Wei-Dong another
parcel of longzai rice and chicken folded in a lotus leaf. Ping had
wanted to go to the Pizza Hut, but Wei-Dong had looked so hurt and
offended at the suggestion, and had been so insistent on eating
something ``real'' that he'd taken the gweilo to a cafe in the
Cantonese quarter, near the handshake buildings. Wei-Dong had loved
it from the moment they'd sat down, and had ordered confidently,
impressing both Ping and the waiter with his knowledge of South
Chinese food.

Wei-Dong chewed, made a face. ``On the bloody top of the stack,
three high!'' he said. ``With more containers sandwiched in on every
side of me, except the door side, thankfully! But I couldn't climb
down the stack with these.'' He thumped the dirty, beat up cardboard
boxes beside the table. ``So I had to transfer the cards to my
backpack and then climb up and down that stack, over and over
again, until I had it all on the ground. Then I threw down the
collapsed cardboard boxes, climbed to the bottom, and boxed
everything up again.''

Ping's jaw dropped. ``You did all that in the \emph{port}?'' He
thought of all the guards he'd seen, all the cameras.

Wei-Dong shook his head. ``No,'' he said. ``I couldn't take the
chance. I did it at night, in relays, the night before we got in.
And I covered it all in some plastic sheeting I had, which is a
good thing because it rained yesterday. There was a lot of water on
the deck and some of it leaked through the plastic, but the boxes
seem OK. Let's hope the cards are still readable. I figure they
must be -- they're in plastic-wrapped boxes inside.''

``But what about the crew seeing you?''

Wei-Dong laughed. ``Oh, I was shitting bricks the whole time over
that, I promise! I was in full sight of the wheelhouse most of the
time, though thankfully there wasn't any moon out. But yeah, that
was pretty freaky.''

Ping looked at the gweilo, his skinny arms, the fuzz of pubescent
moustache, the shaggy hair, the bad smell. When the boy had finally
emerged from the gate, confidently flashing some kind of badge at
the guard, Ping had wanted to strangle him for being so late and
for looking so \emph{relaxed} about it. Now, though, he couldn't
help but admire his old guildie. He said so.

Wei-Dong actually blushed, and his chest inflated, and he looked so
proud that Ping had to say it again. ``I'm in awe,'' he said. ``What a
story!''

``I just did what I had to do,'' Wei-Dong said with an unconvincing,
nonchalant shrug. His Mandarin was better than Ping remembered it.
Maybe it was just being face to face rather than over a fuzzy,
unreliable net-link, the ability to see the whole body, the whole
face.

All of Ping's earlier worry and irritation melted away. He was
overcome by a wave of affection for this kid who had travelled
thousands of kilometers to be part of the same big guild. ``Don't
take this the wrong way,'' he said, ``but I have to tell you this. A
few hours ago, I was very upset with you. I thought it was just ego
or stupidity, your coming all this way with the boxes. I wanted to
strangle you. I thought you were a stupid, spoiled --'' He saw the
look on Wei-Dong's face, pure heartbreak and stopped, held up his
hands. ``Wait! What I'm trying to say is, I thought all this, but
then I met you and heard your story, and I realized that you want
this just as much as I do, and have as much at stake now. That
you're a real, a real \emph{comrade}.'' The word was funny, an old
communist word that had been leached of color and meaning by ten
million hours of revolutionary song-singing in school. But it fit.

And it worked. Wei-Dong's chest swelled up even bigger, like a
balloon about to sail away, and his cheeks glowed like red coals.
He fumbled for words, but his Chinese seemed to have fled him, so
Ping laughed and handed him another lotus leaf, this one filled
with seafood.

``Eat!'' he said. ``Eat!'' He checked the time on his phone, read the
coded messages there from Big Sister Nor. ``You've got 10 minutes to
finish and then we have to get to the guild-house for the big
call!''

\tb

You're in a strange town, or a strange part of town. A little
disoriented already, that's key. Maybe it's just a strange time to
be out, first thing in the morning in the business district, or
very late at night in clubland, or the middle of the day in the
suburbs, and no one else is around.

A stranger approaches you. He's well-dressed, smiling. His
body-language says,
\emph{I am a friend, and I'm slightly out of place, too.} He's
holding something. It's a pane of glass, large, fragile, the size
of a road atlas or a Monopoly board. He's struggling with it. It's
heavy? Slippery? As he gets closer, he says, with a note of
self-awareness at the absurdity of this all, ``Can you please hold
this for a second?'' He sounds a little desperate too, like he's
about to drop it.

You take hold of it. Fragile. Large. Heavy. Very awkward.

And, still smiling, the stranger methodically and quickly plunges
his hands into your pockets and begins to transfer your keys,
wallet and cash into his own pockets. He never breaks eye-contact
in the ten or 15 seconds it takes him to accomplish the task, and
then he turns on his heel and walks away (he doesn't run, that's
important) very quickly, for a dozen steps, and \emph{then} he
breaks into a wind-sprint of a run, powering up like Daffy Duck
splitting on Elmer Fudd.

You're still holding onto the pane of glass.

Why are you holding onto that pane of glass?

What else are you going to do with it? Drop it and let it break on
the strange pavement? Set it down carefully?

Tell you one thing you're not going to do. You're not going to run
with it. Running with a ten kilo slab of sharp-edged glass in your
hands is even dumber than taking hold of it in the first place.

\tb

``What's at work here?'' Big Sister Nor was on the video-conference
window, with The Mighty Krang and Justbob to either side of her,
heads down on their screens, keeping the back-channel text-chat
running while Big Sister Nor lectured. She was speaking Mandarin,
then Hindi. The text-chat was alive in three alphabets and five
languages, and machine-translations appeared beneath the words.
English for Wei-Dong, Chinese for his guildies. There were a couple
thousand people logged in direct, and tens of thousands due to
check in later when they finished their shifts.

``Dingleberry in K-L says 'Disorientation,''' The Mighty Krang said,
without looking up.

Big Sister Nor nodded. ``And?''

``'Social Contract,''' said Justbob. ``That's MrGreen in Singapore.''

BSN showed her teeth in a hard grin. ``Singapore, where they know
all about the social contract! Yes, yes! That's just it. A person
comes up to you and asks you for help, you help; it's in our
instincts, it's in our upbringing. It's what keeps us all
civilized.''

And then she told them a story of a group of workers in Phenom
Penh, gold farmers who worked for someone who was supposed to be
very kindly and good to them, took them out for lunch once a week,
brought in good dinners and movies to show when they worked late,
but who always seemed to make small\ldots{} \emph{mistakes}\ldots{} in their
pay-packets. Not much, and he was always embarrassed when it
happened and paid up, and he was even more embarrassed when he
``forgot'' that it was pay day and was a day, two days, three days
late paying them. But he was their friend, their good friend, and
they had an unwritten contract with him that said that they were
all good friends and you don't call your good friend a thief.

And then he disappeared.

They came to work one day -- three days after pay-day, and they
hadn't been paid yet, of course -- and the man who ran the Internet
cafe had simply shrugged and said he had no idea where this boss
had gone. A few of the workers had even worked through the day, and
even the next, because their good friend must be about to show up
someday soon! And then their accounts stopped working; all the
accounts, all the characters they'd been levelling, the personal
characters they used for the big rare-drop raids, everything.

Some of them went home, some of them found other jobs. And
eventually, some of them ran into their old boss again. He was
running a new gold farm, with new young men working for him. The
boss was so apologetic, he even cried and begged their forgiveness;
his creditors had called in their loans and he'd had to flee to
escape them, but he wanted to make it up to the workers, his
friends, whom he'd loved as sons. He'd put them to work as senior
members of his new farm, at double their old wages, just give him
another chance.

The first pay-day was late. One day. Two days. Three days. Then,
the boss didn't come to work at all. Some of the younger, newer
workers wanted to work some more, because, after all, the boss was
their dear friend. And the old hands, the ones who'd just been
taken for a second time, they finally admitted to their fellow
workers what they'd known all along: the boss was a crook, and he'd
just robbed them all.

``That's how it works. You violate the social contract, the other
person doesn't know what to do about it. There's no script for it.
There's a moment where time stands still, and in that moment, you
can empty out his pockets.''

There were more stories like this, and they made everyone laugh,
sprinkles of ``kekekekeke'' in the chat, but when it was over,
Wei-Dong felt his first tremor of doubt.

``What is it?'' Jie asked him. She was very beautiful, and from what
he could understand, she was a very famous radio person, some kind
of local hero for the factory girls. It was clear that Lu was
head-over-heels in love with her, and everyone else deferred to her
as well. When she turned her attention on him, the whole room
turned with her. The room -- a flat in a strange old part of town
-- was crowded with people, hot and loud with the fans from the
computers.

``It's just,'' he said, waved his hands. He was suddenly very tired.
He hadn't had a nap or even a shower since sneaking out of the
port, and meeting all these people, having the videoconference with
Big Sister Nor, it was all so much. His Chinese fled him and he
found himself fumbling for the words. He swallowed, thought it
through. ``Look,'' he said. ``I want to help all the workers get a
better deal, the Turks, the farmers, the factory girls.'' They all
nodded cautiously. ``But is that what we're doing here? Are we going
to win any rights by, you know, by being crooks? By ripping people
off?''

The group erupted into speech. Apparently he'd opened up an old
debate, and the room was breaking into its traditional sides. The
Chinese was fast and slangy, and he lost track of it very quickly,
and then the magnitude of what he'd done finally, really
\emph{hit him}. Here he was, thousands of miles from home, an
illegal immigrant in a country where he stood out like a sore
thumb. He was about to get involved in a criminal enterprise --
hell he was \emph{already} involved in it -- that was supposed to
rock the world to its foundations. And he was only 18. He felt two
inches tall and as flat as a pancake.

``Wei-Dong,'' one of the boys said, in his ear. It was Matthew, who
had a funny, leathery, worn look to him, but whose eyes twinkled
with intelligence. ``Come on, let's get you out of here. They'll be
at this for hours.''

He looked Matthew up and down. Technically, they were guildies, but
who knew what that meant anymore? What sort of social contract did
they \emph{really} have, these strangers and him?

``Come on,'' Matthew said, and his face was kind and caring. ``We'll
get you somewhere to sleep, find you some clothes.''

That offer was too good to pass up. Matthew led him out of the
apartment, out of the building, and out in the streets. The sun had
set while they were conferenced in, and the heat had gone out of
the air. Matthew led him up and down several maze-like alleys,
through some giant housing blocks, and then into another building,
this one even more run-down than the last one. They went up nine
flights of stairs, and by the time they reached the right floor,
Wei-Dong felt like he would collapse. His thighs burned, his chest
heaved and ached, and the sweat was coursing down his face and neck
and back and butt and thighs.

``I had the same question as you,'' Matthew said. ``When I got out of
jail.''

Wei-Dong willed himself not to edge away from Matthew. The
apartment was filled with thin mattresses, covering nearly the
entire floor like some kind of crazy, thick carpet. They sat on
adjacent beds, shoes off. Wei-Dong must have made some sign of his
surprise, because Matthew smiled a sad smile. ``I went to jail for
going on strike with other Webblies. I'm not a murderer,
Wei-Dong.''

Wei-Dong felt himself blushing. He mumbled and apology.

``I had a long talk with Big Sister Nor. Here's what she told me:
she said that a traditional strike, where you take your labor away
from the bosses and demand a better deal, that it wouldn't work
here. That we needed to do that, but that we also needed to be able
to show everyone who has us at their mercy that they've overrated
their power. When the bosses say, 'We'll beat you up,' or when the
police say, 'We'll put you in jail,' or when the game companies
say, 'We'll throw you out,'' we need to be able to say, 'Oh no you
won't!'''

The sheer delight he put into this last phrase made Wei-Dong smile,
even though he was so tired he could barely move his face.

He scrubbed at his eyes with the backs of his hands and said,
``Look, I think my emotions are on trampolines today. It's been a
very big day.'' Matthew chuckled. ``You understand.''

``I understand. I just wanted to let you know that this isn't just
about being a crook. It's about changing the power dynamics in the
battle. You're a fighter, you understand that, don't you? I hear
you play healers. You know what a raid is like with and without a
healer?''

Wei-Dong nodded. ``It's a very different fight,'' he said. ``Different
tactics, different feel.''

``A different dynamic. There's math to describe it, you know? I
found a research paper on it. It's fascinating. I'll email you a
copy. What we're doing here, we're changing the dynamic, the
balance of power, for workers everywhere. You'll see.''

Wei-Dong yawned and waved his fist over his mouth weakly.

``You need to sleep,'' Matthew said. ``Good night, comrade.''

Wei-Dong woke once in the night, and every mattress was filled, and
everyone was snoring and breathing and snuffling and scratching.
There must have been twenty guys in the room with him, a human
carpet of restless energy, cigarette-and-garlic breath, foot-odor,
body-odor, and muffled grumbles. It was so utterly unlike the ship,
unlike his room in the Cecil Hotel in LA, unlike his parents' home
in Orange County\ldots{} The ground actually felt like it was sloping
away for a minute, like the storm-tossed deck of a container ship,
and he thought for a wild, disoriented minute that there was an
earthquake, and pictured the highrise buildings he'd seen clustered
together on the way over crashing into one another like dominoes.
Then the land righted itself again and the panic dissipated.

He thought of his mother and knew that he'd have to find a PC and
give her a call the next day. They'd exchanged a lot of email while
he was on the ship, a lot of reminisces about his dad, and he'd
felt closer to her than he had in years.

Thinking of his mother gave him an odd feeling of peace, not the
homesick he'd half-expected, and he drifted off again amid the
farts and the grunts and the human sounds of the human people he'd
put himself among.

\tb

Connor's fingerspitzengefuhl was going crazy. Like all the
game-runners, he had a sizeable portfolio of game assets and
derivatives. It wasn't exactly fair -- betting on the future of
game-gold when you got a say in that future put you at a sizeable
advantage over the people on the other side of the bets. But screw
'em if they can't take a joke.

Besides, his portfolio was so big and complex that he couldn't
manage it himself. Like everyone else, he had a broker, a guy who
worked for one of the big houses, a company that had once been an
auto-manufacturer before it went bankrupt, got bailed out, wrung
out, twisted and financialized until the only thing left of any
value in it was the part of the company that had packaged up and
sold off the car-loans suckers had taken out on its
clunkermobiles.

And his broker \emph{loved} him, because whenever Connor phoned in
an order for a certain complex derivative -- say, a buy-order for
\$300,000 worth of insurance policies on six-month gatling gun
futures from Zombie Mecha -- then it was a good bet that there were
going to be a lot fewer gatling guns in Zombie Mecha in six months
(or that the gatling gun would get a power-up, maybe depleted
uranium ammo that could rip through ten zombies before stopping),
driving the price of the guns way, way up. The broker, in turn,
could make money on that prediction by letting his best clients in
on the deal, buying gatling gun insurance policies, or even gatling
gun futures, or futures on gatling gun insurance, raking in fat
commissions and getting everyone else rich at the same time.

So Connor had an advantage. So who was complaining? Who did it
hurt?

And in turn, Connor's broker liked to call him up with hot tips on
other financial instruments he might want to consider, financial
instruments that came to him from his other clients, a diverse
group of highly placed people who were privy to all sorts of
secrets and insider knowledge. Every day this week, the broker,
Ira, had called up Connor and had a conversation that went like
this:

Ira: ``Hey, man, is this a good time?''

Connor (distractedly, locked in battle with his many screens and
their many feeds): ``I've always got time for you, buddy. You've got
my money.''

Ira: ``Well, I appreciate it. I'll try to be quick. We've got a new
product we're getting behind this week, something that kinda took
us by surprise. It's from Mushroom Kingdom, which is weird for us,
because Nintendo tends to play all that stuff very close and tight,
leaving nothing on the table for the rest of us. But we've got a
line on a fully hedged, no-risk package that I wanted to give you
first crack at, because we're in limited supply\ldots{}''

And from there it descended into an indecipherable babble of
banker-ese, like a bunch of automated text generated by searching
the web for ``fully hedged'' (meaning, we've got a bet that pays out
if you win and another that pays out if you lose, so no matter
what, you come out ahead, something that everyone promised and no
one ever delivered) and blowing around the text that came up in the
search-result snippets, like a verbal whirlwind with ``fully hedged''
in the middle of it.

The thing was, Connor was \emph{really good} at speaking
banker-ese, and this just didn't add up. The payoff was gigantic,
15 percent in a single quarter, up to 45 percent in the ideal
scenario, and that was in a tight market where most people were
happy to be taking in one or two percent. This was the kind of
promise he associated with crazy, high-risk ventures, not anything
``fully hedged.''

He stopped Ira's enthusiastically sputtering explanation, said,
``You said no-risk there, buddy?''

Ira drew in a breath. ``Did I say that?''

``Yup.''

``Well, you know, \emph{everything}'s got a risk. But yeah, I'm
putting my own money into this.'' He swallowed. ``I don't want to
pressure you --''

Connor couldn't help himself, he snorted. Ira had many things going
for him, but he was a pushy son of a bitch.

``Really!'' But he sounded contrite. ``OK, let me be straight with
you. I didn't believe it myself, either. None of us did. You know
what bond salesmen are like, we've seen it all. But there were kids
in the office, straight out of school. These kids, they have a lot
more time to play than we do --'' Connor repressed the snort, but
just barely. The last time Ira played a game, it had been World of
Warcraft, in the dawn of time. He was a competent, if unimaginative
broker, but he was no gamer. That's OK, he also wasn't a
pork-farmer, but he could still buy pork-futures. ``-- and they were
hearing about this stuff from other players. They'd started buying
in for themselves, using their monthly bonuses, you know, it's kind
of a tradition to treat that bonus money as pennies from heaven and
spend it on long-shot bets. Anyway, they started to clean up, and
clean up, and clean up.''

``So how do you know it's not tapped out?''

``That's the thing. A couple of the old timers bought into it and
you know, they started to clean up too. And then I got in on it
--''

``How long ago?''

``Two months ago,'' he said, sheepishly. ``It's paying a monthly
coupon of 16 percent on average. I've started to move my long-term
savings into it too.''

``Two months? How many of your other clients have you brought in on
this deal?'' He felt a curious mixture of anger and elation -- how
dare Ira keep this to himself, and how fine that he was about to
share it!

``None!'' Ira was speaking quickly now. ``Look, Connor, all my cards
on the table now. You're the best customer I got. Without you,
hell, my take home pay'd probably be cut in half. The only reason I
haven't brought this to you before now is, you know, there wasn't
any more to go around! Any time there was an offer on these things,
they'd be snapped up in a second.''

``So what happened? Did all your greedy pals get their fill?''

Ira laughed. ``Not hardly! But you know how it goes, as soon as
something takes off like these vouchers, there's a lot of people
trying to figure out how to make more of them. Turns out there's a
bank, one of these offshore ones that's some Dubai prince's private
fortune, and the Prince is a doubter. The bank's selling very long
bets against these bonds on great terms. They're one-year coupons
and they pay off \emph{big} if the bonds don't crash. So now
there's some uncertainty in the pool and some people are flipping,
betting that the Prince knows something they don't, buying his
paper and selling their bonds. We've gone one better: we've got a
floating pool of hedged-off packages that balance out the Prince's
bets and these bonds, so no matter what happens, you're in the
green. We buy or sell every day based on the rates on each. It's
--''

``Risk free?''

``Virtually risk free. Absolutely.''

Connor's mouth was dry. There was something going on here,
something big. His mind was at war with itself. Finance was a game,
the biggest game, and the rules were set by the players, not by a
designer. Sometimes the rules went crazy and you got a little
pocket of insanity, where a small bet could give you unimaginable
wins. He knew how this worked. Of course he did. Hadn't he been
chasing gold farmers up and down nine worlds, trying to find their
own little high-return pockets and turn them inside out? At the
same time, there was just no such thing as a free lunch. Something
that looked too good to be true probably was too good to be true.
All that and all the other sayings he'd grown up with, all that
commonsense that his simple parents had gifted him with, them with
their small-town house and no mortgage and sensible retirement
funds that would have them clipping coupons and going to
two-for-one sales for the rest of their lives.

``Twenty grand,'' he blurted. It was a lot, but he could handle it.
He'd made more than that on his investments in the past 90 days. He
could make it up in the next 90 days if --

``\emph{Twenty}? Are you kidding? Connor, look, this is the kind of
thing comes along once in a lifetime! I came to you \emph{first},
buddy, so you could get in big. Shit, buddy, I'll sell you twenty
grand's worth of these things, but I tell you what --''

It made him feel small, even though he knew it was \emph{supposed}
to make him feel small. It was like there were two Connors, a cool,
rational one and an emotional one, bitterly fighting over control
of his body. Rational won, though it was a hard-fought thing.

``Twenty's all I've got in cash right now,'' he lied, emotional
Connor winning this small concession. ``If I could afford more --''

``Oh!'' Ira said, and Connor could hear the toothy smile in his
voice. ``Connor, pal, I don't do this very often, and I'd appreciate
it if you'd keep this to yourself, but how about if I promise you
that your normal trades for today will pick up an extra, uh, make
it 20 more, for a total of 40 thousand. Would you want to plow that
profit into these puppies?''

Connor's mouth went dry. He knew how this worked, but he'd long ago
given up on being a part of it. It was the oldest broker-scam in
the world: every day, brokers made a number of ``off-book'' trades,
buying stocks and bonds and derivatives on the hunch that they'd go
up. Being ``off-book'' meant that these trades weren't assigned to
any particular client's account; the money to buy them came out of
the general account for the brokerage house.

At the end of the day, some -- maybe all -- of those trades would
have come out ahead. Some -- maybe all -- would have come out
behind. And that's when the magic began. By back-dating the books,
the broker could assign the shitty trades to shitty customers,
cheapskates, or big, locked-in, slow-moving customers, like
loosely-managed estates for long-dead people whose wealth was held
in trust. The gains could be written to the broker's best
customers, like some billionaire that the broker was hoping to do
more business with. In this way, every broker got a certain amount
of discretion every day in choosing who would make money and who
would lose it. It was just a larger version of the barista at the
coffee shop slipping her regulars a large instead of a medium every
now and again, without charging for the upgrade. The partners who
ran the brokerages knew that this was going on, and so did many of
the customers. It was impossible to prove that you'd lost money or
gained money this way -- unless your broker told you at 9:15 on a
Tuesday morning that your account would have an extra \$20,000 in
it by 5PM.

Ira had just taken a big risk in telling Connor what he was going
to do for him. Now that he had this admission, he could,
theoretically, have Ira arrested for securities fraud. That is,
until and unless he gave Ira the go-ahead, at which point they'd
\emph{both} be guilty, in on it together.

And there rational and emotional Connor wrestled, on the knife-edge
between wealth and conspiracy and pointless, gainless honesty. They
tumbled onto the conspiracy side. After all, Connor and the broker
bent the rules every time Connor ordered a trade on one of Coca
Cola Games's futures. This was just the same thing, only moreso.

``Do it,'' he said. ``Thanks, Ira.''

Ira's breath whooshed out over the phone, and Connor realized that
the broker had been holding his breath and waiting on his reply,
waiting to find out if he'd gone too far. The salesman really
wanted to sell him this package.

Later, in Command Central, Connor watched his feeds and thought
about it, and something felt\ldots{}\emph{hinky}. Why had Ira been so
eager? Because Connor was such a great customer and Ira thought if
he made Connor a ton of money, Connor would give it back to him to
continue investing, making more and more money for him, and more
and more commissions for the broker?

And now that his antennae were up, he started to see all kinds of
ghosts in his feeds, little hints of gold and elite items changing
hands in funny ways, valued too high or not high enough, all out of
whack with the actual value in-game. Of course, who knew what the
in-game value of anything could really be? Say the game-runners
decided to make the Zombie Mecha gatling guns fire depleted uranium
ammo, starting six months from now. The easy calculation had
gatling guns shooting up in value in six months, because it would
make it possible for the Mechas to wade through giant hordes of
zombies without being overpowered. But what if that made the game
\emph{too} easy, and lots of players left? Once your buddies went
over to Anthills and Hives and started team-playing huge, warring
hive-intelligences, would you want to hang around Zombie Mecha,
alone and forlorn, firing your gatling gun at the zombies? Would
the zombies stop being fun objectives and start being mere
collections of growling pixels?

It took the subtle fingerspitzengefuhl of a fortune-teller to
really predict what would happen to the game when you nerfed or
buffed one character class or weapon or monster. Every change like
this was watched closely by game-runners for weeks, around the
clock, and they'd tweak the characteristics of the change from
minute to minute, trying to get the game into balance.

The feeds told the story. Out there in gameland, there was a hell
of a lot of activity, trades back and forth, and it worried him. He
started to ask the other game-runners if they noticed anything out
of the ordinary but then something else leapt out of his feeds:
there! Gold-farmers!

He'd been looking for them everywhere, and finding them. Gold
farming had a number of signatures that you could spot with the
right feed. Any time someone logged in from a mysterious Asian IP
address, walked to the nearest trading post, stripped off every
scrap of armor and bling and sold it, then took all the resulting
cash and the entire contents of her guild bank and turned it over
to some level one noob on a free trial account that had only
started an hour before, who, in turn, turned the money over to a
series of several hundred more noobs who quickly scattered and
deposited it in their own guild banks, well, that was a sure bet
you'd found some gold farmer who was hacking accounts. Hell, half
the time you could tell who the farmers were just by looking at the
names they gave their guilds: real players either went for the
heroic (``Savage Thunder'') or the ironic (``The Nerf Herders'') or the
eponymous (``Jim's Raiders'') but they rarely went by
``asdf\-a\-sd\-f\-a\-sd\-f\-a\-s\-d\-f\-a\-sd\-f\-a\-sd\-f\-a\-sd\-f\-a\-sd\-f2329'' or, God help him,
707A\-55D\-F\-0D\-7E\-15B\-B\-B\-9F\-B3B\-E\-1\-6\-5\-6\-2\-F\-2\-2%
C\-0\-2\-6\-A\-8\-8\-2\-E\-4\-0\-1\-6\-4\-C\-7\-B\-1\-4\-9\-B\-1\-5%
D\-E\-7\-1\-3\-7\-E\-D\-1\-A.

But as soon as he tweaked his feeds to catch them, the farmers
figured out how to dodge them. The guilds got good names, the
hacked players started behaving more plausibly -- having half-assed
dialogue with the toons they were buffing with all their goods --
and the gangs that converged on any accidental motherlode in the
game did a lot of realistic milling about and chatting in broken
English. Increasingly, the players were logging in with prepaid
cards diverted from the US over American proxies, making them
indistinguishable from the lucrative American kid trade, who were
apt to start playing by buying some prepaid cards along with their
Cokes and gum at the convenience store. Those kids had the
attention spans of gnats, and if you knocked them offline after
mistaking them for a gold farmer, they left and went straight to a
competing world and never again showed up in your game or on your
balance sheet.

It was amazing how fast information spread among these creeps.
Well, not amazing. After all, information spread among normal
players faster than you'd believe too -- it was great, you hardly
had to lift a finger or spend a penny on marketing when you
released some new elite items or unveiled a new world. The players
would talk it up for you, spreading the word at the speed of
gossip. And the same jungle telegraph ran through the farmers'
underground, he could see it at work.

And there were more of them, a little guild of twenty, all grinding
and grinding the same campaign. They were fresh characters, created
two days before, and they'd been created by players who knew what
they were doing -- it was just the perfect balance between rezzers
and tanks and casters, a good mix of AOE and melee weapons. They'd
levelled damned fast -- he pulled up some forensics on some of the
toons, felt his fingerspitzengefuhl tingle as the game guttered
like a flame in a breeze. He'd installed the forensics packages
over the howls of protest from the admin team who'd shown him chart
after chart about what running the kind of history he wanted to see
would do to server performance. He'd gotten his forensics, but only
after promising to use them sparingly.

And there it was: the players had levelled each other by going into
a PvP -- Player versus Player -- tournament area and repeatedly
killing one another. As soon as one of them dinged up a level, he
would stand undefended and let the other player kill him quickly.
The game gave megapoints for killing a higher level player. Once
player two dinged, they switched places, and laddered, one after
the other, up to heights that normal players would take forever to
attain.

The campaign they were running was simple: scrounging a mix of
earth-fairy wings and certain mushroom caps, giving them over to a
potion-master who would pay them in gold. It wasn't anything
special and it was a little below their levels, but when he charted
out the returns in gold and experience per hour, he saw that
someone had carelessly created a mission that would pay out nearly
triple what the regular campaign was supposed to deliver. He shook
his head. \emph{How the hell did they figure this stuff out?} You'd
need to chart every single little finicky mission in the game and
there were \emph{tens of thousands} of missions, created by
designers who used software algorithms to spin a basic scenario
into hundreds of variants.

And there they were, happily collecting their mushroom caps and
killing the brown fairies and plucking their wings. Every now and
again they'd happen on a bigger monster that wandered into their
aggro zone and they'd dispatch it with cool ease.

His finger trembled over the macro that would suspend their
accounts and boot them off the server. It didn't move.

He admired them, that was the problem. They were doing something
efficiently, quietly and well, with a minimum of fuss. They
understood the game nearly as well as he did, without the benefit
of Command Central and its many feeds. He --

He logged in.

He picked an av he'd buffed up to level 43, halfway up the ladder
to the maximum, which was 90. Regulus was an elf healer, tall and
whip-thin, with a huge rucksack bulging with herbs and potions. He
was a nominal member of one of the mid-sized player guilds, one of
the ones that would accept even any player for a small fee, which
offered training courses, guild-banking, scheduled events, all with
the glad sanction of Coca Cola. The right sort of people.

\edialog{Hello}

Two months before, the players would have kept on running their
mission, blithely ignoring him. But that was one of the tell-tales
his feeds looked for to pick out the farmers. Instead, these toons
all waved at him and did little emotes, some of which were quite
good custom jobs including dance-moves, elaborate mime and other
gestures. If his feeds hadn't picked these jokers out as farmers,
he'd have pegged them as hardcore players. But they hadn't actually
spoken or chatted him anything. They were almost certainly Chinese
and English would be hard for them.

\edialog{Wanna group?}

He offered them a really plum quest, one that had a crazy-high gold
and experience reward for a relatively nearby objective: retrieving
Dvalinn's runes from a deep cave that they'd have to fight their
way into, killing a bunch of gimpy dwarves and a couple of decent
bosses on the way. The quest was chained to one that led to a fight
with Fenrisulfr, one of the biggest bosses in Svartalfaheim
Warriors, a megaboss that you needed a huge party to take down, but
which rewarded you with enormous treasure. The whole thing was
farmer-bait he'd cooked up specifically for this kind of mission.

After a decent interval -- short, but long enough for the players
to be puzzling through a machine-translation of the quest-text --
they gladly joined, sending simple thanks over text.

He pretended he saw nothing weird about their silence as they
progressed toward the objective, but in the meantime, he
concentrated on observing them closely, trying to picture them
around a table in a smoky cafe in China or Vietnam or Cambodia or
Malaysia, twenty skinny boys with oily hair and zits, cigarettes in
the corners of their mouths, squinting around the curl of smoke.
Maybe they were in more than one place, two or even three groups.
They almost certainly had some kind of back-channel, be it voice,
text, or simply shouting at each other over the table, because they
moved with good coordination, but with enough individualism that it
seemed unlikely that this was all one guy running twenty bots.

\edialog{Where you from?}

He had to be aware that they were probably trying to figure out if
he was from the game, and if he made things too easy for them, he
might tip them off.

One player, an ogre caster with a huge club and a bandoleer of
mystic skulls etched with runes, replied

\edialog{We're Chinese, hope that's OK with you}

This was more frank than he'd expected. Other groups he'd
approached with the same gimmick had been much more close-lipped,
claiming to come from unlikely places in the midwest like Sioux
Falls, places that seemed to have been chosen by randomly clicking
on a map of the USA.

\edialog{China!}

he typed,

\edialog{You seem pretty good with English then!}

The ogre -- Prince Simon, according to his stats -- emoted a little
bow.

\edialog{I studied in school. My guildies aren't same good.}

Connor thought about who he was pretending to be: a young player in
a big American city like LA. What would he say to these people?

\edialog{Is it late there?}

\edialog{Yes, after dinner. We always play after dinner.}

\edialog{Sounds like a lot of fun! I wish I had a big group
of friends who were free after dinner. It's always homework
homework homework}

Connor's fictional persona was sharpening up for him now, a lonely
high-school kid in La Jolla or San Deigo, somewhere on the ocean,
somewhere white and middle class and isolated. Somewhere without
sidewalks. The kind of kid who might come across a plum quest live
Dvalinn's runes and have to go and round up a group of strangers to
run it with him.

\edialog{It's a good time}

the ogre said. A pause.

\edialog{My friend wants to know what you're studying?}

His persona floated an answer into his head.

\edialog{I'm about to graduate. I've applied for civil
engineering at a couple of schools. Hope I get in!}

The ogre said,

\edialog{I was a civil engineer before I left home. I
designed bridges, five bridges. For a high-speed train system.}

Connor mentally revised his image of the boys into young men,
adults.

\edialog{When did you leave home?}

\edialog{2 years. No more work. I will go home soon though I
think. I have a family there. A little son, only 3}

The ogre messaged him an image. A grinning Chinese boy in a sailor
suit, toothy, holding a drippy ice cream cone like a baton, waving
it like a conductor.

Connor's fictional 17 year old didn't have any reaction to the
picture, but his 36-year-old self did. A father leaving his son
behind, plunging off to find work. Connor hadn't ever had to
support someone, but he'd thought about it a lot. In Connor's
world, where people's motives were governed by envy and fear, the
picture of this baby was seismic, an earthquake shaking things up
and making the furnishings fall to the floor and shatter. He
struggled to find his character.

\edialog{Cute! You must miss him}

\edialog{A lot. It's like being in the army. I will do this
for a few years, then go home.}

What a world! Here was this civil engineer, accomplished, in love,
a father, living far away, working all day to amass virtual
treasures, playing cat-and-mouse with Connor and his people.

\edialog{So what advice do you have for someone going into
civil engineering?}

The ogre emoted a big laugh.

\edialog{Don't try to find work in China}

Connor emoted a big laugh too -- and led the party to Dvalinn's
runes, losing himself in the play even as he struggled to remain
clinical and observant. Some of his fellow gamerunners looked over
his shoulder now and again, watched them run the mission, made
little cutting remarks. Among the gamerunners, the actual game
itself was slightly looked down upon, something for the marks to
play. The real game, the big game was the game of designing the
game, the game of tweaking all the variables in the giant hamster
cage that all the suckers were paying to run through.

But Connor never forgot how he came to the game, where his
equations had come from: from \emph{play}, thousands of hours in
the worlds, absorbing their physics and reality through his fingers
and ears and eyes. As far as he was concerned, you couldn't do your
job in the game unless you played it too. He marked the snotty
words, noticed who delivered them, and took down his mental
estimation of each one by a few pegs.

Now they were in the dungeon, which he'd just slapped together, but
which he nevertheless found himself really enjoying. As a raiding
guild, the Chinese were superb: coordinated, slick, smart. He had a
tendency to think of gold farmers as mindless droids, repeating a
task set for them by some boss who showed them how to use the mouse
and walked away. But of course the gold farmers played all day,
every day, even more than the most hardcore players. They
\emph{were} hardcore players. Hardcore players he'd sworn to
eliminate, but he couldn't let himself forget that they \emph{were}
hardcore.

They fought their way through to the big boss, and the team were so
good that Connor couldn't help himself -- he reached into the
game's guts and buffed the hell out of the boss, upping his level
substantially and equipping him with a bunch of special attacks
from the library of Nasties that he kept in his private workspace.
Now the boss was incredibly intimidating, a challenge that would
require flawless play from the whole team.

\edialog{Oh no}

he typed.

\edialog{What are we going to do?}

And the ogre sprang into action, and the players formed two ranks,
those with melee attacks in the vanguard, spellcasters, healers,
ranged attackers and AOE attackers in the back, seeking out ledges
and other high places out of range of the boss, a huge dire wolf
with many ranged spells as well as a vicious bite and powerful paws
that could lash out and pin a player until the wolf could bring its
jaws to bear on him.

The boss had a bunch of smaller fighters, dwarves, who streamed out
of the caves leading to the central cavern in great profusion,
harassing the back rank and intercepting the major attacks the
forward guard assembled. As a healer and rezzer, Connor ran to and
fro, looking for safe spots to sit down, meditate, and cast healing
energy at the fighters in the fore who were soaking up incredible
damage from the big boss and his minions. He lost concentration for
a second and two of the dwarves hit him with thrown axes, high and
low, and he found himself incapped, sprawled on the cave floor,
with more bad guys on the way.

His heart was thundering, that old feeling that reminded him that
his body couldn't tell the difference between excitement on screen
and danger in the real world, and when another player, one of the
Chinese whom he had not spoken with at all, rescued him, he felt a
surge of gratitude that was totally genuine, originating in his
spine and stomach, not his head.

In the end, 12 of the 20 players were irreversibly killed in the
battle, respawned at some distant point too far away to reach them
before the battle ended. The boss finally howled, a mighty sound
that made stalactites thunder down from the ceiling and shatter
into sprays of sharp rock that dealt minor damage to the survivors
of their party, damage that they flinched away from anyway, as they
were all running in the red. The experience points were incredible
-- he dinged up a full level -- and there were several very good
drops. He almost reached for his workspace to add a few more to
reward his comrades for their skill and bravery, forcibly reminding
himself that he was \emph{not on their side}, that this was
research and infiltration.

\edialog{You guys are great!}

The ogre emoted a bow and a little victory dance, another custom
number that was graceful and funny at once.

\edialog{You play well. Good luck with your studies.}

Connor's fingers hovered over the keys.

\edialog{I hope you get to see your family soon}

The ogre emoted a quick hug, and it made Connor feel momentarily
ashamed of what he did next. But he did it. He added the entire
guild to his watchlist, so that every message and move would be
logged, machine-translated into English. Every transaction they
made -- all the gold they sold or gave away -- would be traced and
traced again as part of Connor's efforts to unravel the complex,
multi-thousand-party networks that were used to warehouse, convert
and distribute game-goods. He had hundreds of accounts in the
database already, and at the rate he was going, he'd have thousands
by the end of the week -- and it was already Wednesday.

\tb

The police raided Jie's studio while she and Lu were out eating
dumplings and staring into each others' eyes. It was one of her
backup studios, but they'd worked out of it two days in a row, and
had been about to work out of it for a third. This was a violation
of basic security, but Jie's many apartments were fast filling up
with Webblies who had quit their farming jobs in frustration and
joined the full-time effort to amass gold and treasure for the
plan.

The dumpling shop was run by a young woman who looked after her two
year old son and her sister's four year old daughter, but she was
nevertheless always cheerful when they came in, if prone to making
suggestive remarks about young love and the dangers of early
parenthood.

She was just handing them the bill -- Lu once again made a show of
reaching for it, though not so fast that Jie coudn't snatch it from
him and pay it herself, as she was the one with all the money in
the relationship -- when his phone went crazy.

He pulled it out, looked at its face, saw that it was Big Sister
Nor, calling from a number that she wasn't supposed to be using for
another 24 hours according to protocol. That means that she worried
her old number had been compromised, which meant that things were
bad. Turning to the wall and covering the receiver with his hand,
he answered.

``Wei?''

``You've been burned.'' It was The Mighty Krang, whose Taiwanese
accent was instantly recognizable. ``We're watching the webcams in
the studio now. Ten cops, tearing the place apart.''

``Shit!'' he said it so loudly that the four year old cackled with
laughter and dumpling lady scowled at him. Jie slid close to him
and put her cheek next to his -- he instantly felt a little better
for her company -- and whispered, ``What is it?''

``You're all secure, right?''

He thought about it for a second. All their disks were encrypted,
and they self-locked after ten minutes of idle time. The police
wouldn't be able to read anything off any of the machines. He had
two sets of IDs on him, the current one, which was due to be
flushed later that day according to normal procedure, and the next
set, hidden in a pocket sewn into the inside of his pants-leg.
Ditto for his current and next SIMs, one loaded in his current
phone and a pouch of new ones in order of planned usage inserted
into a slit in his belt. He covered the mouthpiece and whispered to
Jie: ``The studio's gone.'' She sucked air past her teeth. ``Are you
all buttoned-up?''

She clicked her tongue. ``Don't worry about me, I've been doing this
for a lot longer than you.'' She began to methodically curse under
her breath, digging through her purse and switching out IDs and
cracking open her phone to swap the SIM. ``I had really nice stuff
in that place,'' she said. ``Good clothes. My favorite mic. We are
such idiots. Never should have recorded there twice in a row.''

The Mighty Krang must have heard, because he chuckled. ``Sounds like
you're both OK?''

``Well, Jiandi won't be able to go on the air tonight,'' he said.

``Screw that,'' Jie said. She took the phone from him. ``Tell Big
Sister Nor that we're going on air at the usual time tonight.
Normal service, no interruptions.''

Lu didn't hear the reply, but he could see from Jie's grimly
satisfied expression that The Mighty Krang had praised her. It had
been Big Sister Nor's idea to rig all the studios with webcams all
the Webblies could access, just in the front rooms. It was a little
weird, trying to ignore the all-seeing eye of the webcam screwed in
over the door. But when you're sleeping 20 to a room, it's easy to
let go of your ideas about privacy -- but all the same, Lu and Jie
now sat far apart when broadcasting, and snuck into the bathroom to
make out afterward.

And now the webcams had paid off. He took the phone back and
listened as The Mighty Krang narrated a play-back of the video,
cops breaking the door down, securing the space. Then an evidence
team that spliced batteries into the computers' power cables so
they could be unplugged without shutting down (Lu was grateful that
Big Sister Nor had decreed that all their hardware had to be
configured to unmount and re-encrypt the drives when they were
idle), took prints and DNA. They already had Lu's DNA, of course,
because they'd sniffed out one of Jie's other apartments. But Jie
had been way ahead of this: she had a little pocket vacuum cleaner,
intended for clearing crumbs and gunk out of keyboards, and she
surreptitiously vacuumed out the seats whenever she took a train or
a bus, sucking up the random DNA of thousands of people, which she
carefully scattered around her apartments when she got in. He'd
laughed at the ingenuity of this, and she told him she'd read about
it in a novel.

The evidence team brought in a panoramic camera and set it in the
middle of the room and the police cleared out momentarily as it
swept around in a tight, precise mechanical circle, producing a
wraparound high-resolution image of the room. Then the cops swept
back in, minus their paper overshoes, and put every scrap of paper
and every piece of optical and magnetic media into more bags, and
then they destroyed the place.

Working with wrecking bars and wicked little knifes, and starting
from the corner under the front door, they methodically smashed
every single stick of furniture, every floor tile, every gyprock
wall, turning it all into pieces no bigger than playing-cards,
heaping it behind them as they went. They worked in near silence,
without rushing, and didn't appear to relish the task. This wasn't
vandalism, it was absolute annihilation. The policemen had the
regulation brushcut short hair, identical blue uniforms, paper
face-masks, kevlar gloves. One drew closer and closer to the
webcam, spotted it -- a little pinhead with a peel-away adhesive
backing stuck up in a dusty corner -- and peeled it away. His face
loomed large in it for a moment, his pores, a stray hair poking out
of his nostrils, his eyes dead and predatory. Then chaos, and
nothing.

``He stamped on it, we think,'' The Mighty Krang said. ``So much for
the webcams. It'll be the first thing they look for next time.
Still, saved your ass, didn't it?''

The description had momentarily taken away Lu's breath. All his
things, his spare clothes, the comics he'd been reading, a
half-chewed pack of energy gum he'd bought the day before,
disappeared into the bowels of the implacable authoritarian state.
It could have been him.

``We're going to move on to the next safe-house,'' he said. ``We'll
find somewhere to broadcast from tonight.''

``You're bloody right we will,'' said Jie, from his side.

They gave the old building a wide berth as they made their way down
into the Metro, and consciously forced themselves not to flinch
every time a police siren wailed past them. When they came back up
to street level, Jie took Lu's hand and said, out of the corner of
her mouth, ``All right, Tank, what do we do now?''

He shrugged. ``I don't know. That was, uh, \emph{close}.'' He
swallowed. ``Don't be mad if I say something?''

She squeezed his fingers. ``Say it.''

``You don't need to do this,'' he said. She stopped and looked at
him, her face white. Before they'd ever kissed, he always felt a
void between them, an invisible force-field he had to push his way
through in order to tell her how he felt. Once they'd become a
couple, the force-field had thinned, but not vanished, and every
time he said or did something stupid, he felt it pushing him away.
It was back in force now. He spoke quickly, hoping his words would
batter their way through it: ``I mean, this is \emph{crazy}. We're
probably all going to go to jail or get killed.'' She was still
staring at him. ``You're just --'' He swallowed. ``You're \emph{good}
at this stuff, is what I'm trying to say. You could probably
broadcast your show for ten more years without getting caught and
retire a rich woman. You don't need to throw it away on us.''

Her eyes narrowed. ``Did I promise not to get mad?''

He tried a little nervous smile. ``Sort of?''

She looked back and forth. ``Let's walk,'' she said. ``We stand out
here.'' They walked. Her fingers were limp in his hand, and then
slipped out. The force-field grew stronger. He felt more afraid
than he had when The Mighty Krang had described the action from the
studio camera. ``You think I'm doing this all for money? I could
have more money if I wanted to. I could take dirtier advertisers. I
could start a marketing scheme for my girls and ask them to send me
money -- there's millions of them, if each one only sent me a few
RMB, I'd be so rich I could retire.''

The handshake buildings loomed around them, and she broke off as
they found themselves walking single file down a narrow alley
between two buildings. She caught up with him and leaned in close,
speaking so softly it was almost a whisper. ``I could just be
another dirty con-artist who comes to South China, steals all she
can, and goes back home to the countryside. I'm \emph{not} doing
that. Do you know why?''

He fumbled for the words and she dug her fingernails into his palm.
He fell silent.

``It's a rhetorical question,'' she said. ``I'm doing it because
\emph{I believe in this}. I was telling my girls to fight back
against their bosses before you ever played your first game. With
or without you, I'll be telling them to fight back. I like your
group, I like the way they cross borders so easily, even more
easily than I get back and forth from Hong Kong. So I'm supporting
your friends, and telling my girls to support them too. The problem
you have is a \emph{worker's} problem, not a Chinese problem, not a
gamer's problem. The factory girls are workers and they want a good
deal just as much as you and your gamer friends do.''

She was breathing heavily, Lu noticed, angry little snorts through
her nose.

He tried to say something, but all that came out was a mumble.

``What?'' she said, her fingernails digging in again.

``I'm sorry,'' he said. ``I just didn't want you to get hurt.''

``Oh, Tank,'' she said. ``You don't need to be my big, strong
protector. I've been taking care of myself since I left home and
came to South China. It may come as a huge surprise to you, but
girls don't need big, strong boys to look after them.''

He was silent for a moment. They were almost at the entrance of the
safe house. ``Can I just admit that I'm an idiot and we'll leave it
at that?''

She pretended to think it over for a moment. ``That sounds OK to
me,'' she said. And she kissed him, a warm, soft kiss that made his
feet sweaty and the hairs on his neck stand up. She chewed his
lower lip for a moment before letting go, then made a rude gesture
at the boys who were calling down at them from a high balcony
overhead.

``OK,'' she said, ``Let's go do a broadcast.''

\tb

It had all been so neatly planned. They would wait until after
monsoon season with its torrential rains; after Diwali with its
religious observances and firecrackers; after Mid-Autumn Festival
when so many workers would be back in their villages, where the
surveillance was so much less intense. They would wait until the
big orders came in for the US Thanksgiving season, when
sweaty-palmed retailers hoped to make their years profitable with
huge sales on goods made and shipped from the whole Pacific Rim.

That had been a good plan. Everyone liked it. Wei-Dong, the boy
who'd crossed the ocean with their prepaid game-cards, had just
about wet his pants at the brilliance of it. ``You'll have them over
a barrel,'' he kept repeating. ``They'll \emph{have} to give in, and
\emph{fast}.''

The in-game project was running very well. That Ashok fellow in
Mumbai had worked out a very clever plan for signalling the vigor
of their various ``investment vehicles'' and the analysts who watched
this were eating it up. They were selling more bad paper than they
could print. It had surprised everyone, even Ashok, and they'd
actually had to pull some Webblies off sales-duty: it turned out
that a surprising number of people would believe any rumor they
heard on an investment board or in-game canteen.

The Mighty Krang and Big Sister Nor were likewise very happy with
the date and had stuck a metaphorical pin in it, and began to plan.
Justbob was fine with this, but she was a warrior and so she
understood that
\emph{the first casualty of any battle is the plan of attack.} So
while Big Sister Nor and Krang and the other lieutenants in China
and Indonesia and Singapore and Vietnam and Cambodia were beavering
away making plans for the future, Justbob was leading skirmishers
in exercises, huge, world-spanning battles where her warriors ran
their armies up against one another by the thousand.

Big Sister Nor hated it, said it was too high-profile, that it
would tip off the game-runners that there were armies massing in
gamespace, and then they would naturally wonder what the players
were massing \emph{for} and it would all unravel. Justbob thought
it was a lot more likely that the gold-farmers and the elaborate
cons would tip them off, seeing as how armies were about as common
in gamespace as onions were in a stir-fry. She didn't try to tell
this to Big Sister Nor, who hardly played games at all any more.
Instead, she obediently agreed to take it easy, to be careful, and
so on.

And then she sent her armies against one another again.

It wasn't like any other game anyone had ever played. The armies
were vast, running to the thousands and growing every day. She
drilled them for hours, and the generals and leaders and
commandants and whatever they called themselves dreamt up their
best strategy and tactics, devised nightmare ambushes and sneaky
guerilla wars, and they sharpened their antlers against one
another.

As Big Sister Nor's complaints grew more serious, Justbob presented
her with statistics on the number of high-level characters the
Webblies now had at their disposal, as the skirmishing was a fast
way to level up. She had players who controlled five or six
absolute top-level toons, each associated with its own prepaid
account, each accessed via a different proxy and untraceable to the
others. Big Sister Nor warned her again to be careful, and The
Mighty Krang took her aside and told her how irresponsible she was
to endanger the whole effort with her warring. She took off her
eyepatch and scratched at the oozing scars over the ruined socket,
a disconcerting trick that never failed to send The Mighty Krang
packing with a greenish face.

Justbob tried to keep the smile off her face when Big Sister Nor
woke her in the middle of the night to tell her that the plan was
dead, and the action had started, right then, in the middle of
monsoon season, in the middle of Diwali, with only weeks to go
before Mid-Autumn Festival.

``What did it?'' she said, as she pulled on a long dress and wound
her hijab around her head. She'd spent most of her life in western
dress, dressing to shock and for easy getaways, but since she'd
gone straight, she'd opted for the more traditional dress. What it
lacked in mobility it made up for in coolness, anonymity, and the
disorienting effect it had on the men who had once threatened her
(though it hadn't stopped the thugs who'd cost her her eye).

``Another strike in Dongguan. This time in Guangzhou. It's big.''

\tb

The room was stuffy. These rooms always were. But the September
heat had pushed the temperature up to stratospheric heights, so
that the cafe smouldered like the caldera of a dyspeptic volcano.
The cafe's owner, a scarred old man whom everyone knew to be a
front for some heavy gangsters, had sent a technician around with a
screwdriver to remove all the cases from the PCs so that the heat
could dissipate more readily from the sweating motherboards and
those monster-huge graphics cards that bristled with additional
fans and glinted with copper heatsinks. This might have been better
for the computers, but it made the room even hotter and filled it
with a jet-engine roar that was so loud the players couldn't even
use noise-cancelling headsets to chat: they had to confine all
their communications to text.

The cafe had once catered to gamers from off the street, along with
love-sick factory girls who spent long nights chatting with their
virtual boyfriends, homesick workers who logged in to spin lies
about their wonderful lives in South China for the people back
home, as well as the occasional lost tourist who was hoping to get
a little online time to keep up with friends and find cheap hotel
rooms. But for the past two years, it had exclusively housed an
ever-growing cadre of gold-farmers sent there by their bosses, who
oversaw a dozen shifting, interlocked businesses that formed and
dissolved overnight, every time a little trouble blew their way and
it became convenient to roll up the store and disappear like a
genie.

The boys in the cafe that night were all young, not a one over 17.
All the older boys had been purged the month before, when they'd
demanded a break after a 22-hour lock-in to meet a huge order from
an upstream supplier. Getting rid of those troublemakers had two
nice effects for their bosses: it let them move in a cheaper
workforce and it let them avoid paying for all those locked-in
hours. There were always more boys who'd play games for a living.

And these boys could \emph{play}. After a 12-hour shift, they'd
hang around and do four or five more hours' worth of raiding
\emph{for fun}. The room was a cauldron in which boys, heat, noise,
dumplings and network connections were combined to make a
neverending supply of stew of wealth for some mostly invisible
older men.

Ruiling knew that there had been some other boys working there
before, older boys who'd had some kind of dispute with the bosses.
He didn't think about them much but when he did, he pictured slow,
greedy fools who didn't want to really work for a living. Lamers
whose asses he could kick back to Sichuan province or whatever
distant place they'd snuck to the Pearl River Delta from.

Ruiling was a hell of a player. His speciality was PvP -- player
versus player -- because he had the knack of watching another
player's movements for a few seconds and then building up a
near-complete view of that player's idiosyncracies and weak spots.
He couldn't explain it -- the knowledge simply shone through at
him, like an arrow in the eye-socket. The upshot of this was that
no one could level a character faster than Ruiling. He'd simply
wander around a game with a Chinese name, talking in Chinese to the
players he met. Eventually, one of them -- some rich, fat, stupid
westerner who wanted to play vigilante -- would start calling him
names and challenge him to a fight. He'd accept. He would kick ass.
He'd gain points.

It was amazing how satisfying this was.

Ruiling had just finished twelve hours of this and had ordered in a
tray of pork dumplings and doused them in hot Vietnamese rooster
red sauce and chopsticked them into his mouth as fast as he could
chew, and now he was ready to relax with some after-work play. For
this, he always used his own toon, a char he'd started playing with
when he was a boy in Gansu. In some ways, this toon was \emph{him},
so long had he lived with it, lovingly buffing it, training it,
dressing it in the rarest of treasures. He had trained up
innumerable toons and seen them sold off, but Ruiling was
\emph{his}.

Tonight, Ruiling partied with some other farmers he knew from other
parts of China, some of whom he'd known back in his village, some
of whom he'd never met. They were a ferocious nightly raiding guild
that pulled off the hardest missions in the worlds, the cream of
the crop. Word had gotten round and now every night he had an
audience of players who'd just been hired on, watching in awe as he
kicked fantastic quantities of ass. He loved that, loved answering
their questions after he was done playing, helping the whole team
get better. And you know, they loved him too, and that was just as
great.

They ran Buri's fortress, the palace of a long-departed god, the
father of gods, the powerful, elemental force that had birthed
Svartalfaheim and the universe in which it lay. It had fearsome
guardians, required powerful spells just to reach, and had never
been fully run in the history of Svartalfaheim. Just the kind of
mission Ruiling loved to try. This would be his sixth crack at it,
and he was prepared to raid for six hours straight if that's what
it took, and so was the rest of his party.

And then he got Fenrir's Tooth. It was the rarest and most
legendary drop in all of Svartalfaheim Warriors, a powerful
talisman that would turn any wolf-pack or enthral them to the
Tooth's holder. The message boards had been full of talk about it,
and several times there'd been fraudulent auctions for it, but no
one had ever seen it before.

After Ruiling picked it up -- it had come from an epic battle with
an army of Sky Giants, in which the entire raiding party had been
killed -- he was so stunned by it that he couldn't speak for a
moment. He just pointed at the screen while his mouth opened and
shut for a moment.

The players watching him fell silent, too, following his gaze and
his finger, slowly realizing what had just happened. A murmur built
through the crowd, picking up steam, picking up volume, turning
into a \emph{roar}, a triumphant shout that brought the entire cafe
over to see. Over the fans' noise they buzzed excitedly, a
hormone-drenched triumphant tribal chest-beating exercise that
swept them all up. Every boy imagined what it would be like to go
questing with Fenrir's Tooth, able to defeat any force with a flick
of the mouse that would send the wolf packs against your enemies.
Every boy's heart thudded in his chest.

But there was another sound, getting louder and more insistent. An
older voice, raspy with a million cigarettes, a hard voice. ``Sit
down! Sit down! Back to work! Everyone back to work!''

It was Huang the foreman, shouting with a fearsome Fujianese
accent. He was rumored to be an ex-Snakehead, thrown out of the
human smuggling gang for killing too many migrants with rough
treatment. Usually, he sat lizardlike and motionless in the corner,
smoking a succession of cheap Chinese Class-D fake Marlboros, harsh
and unfiltered, a lazy curl of smoke giving him a permanent squint
on one side of his face. Sometimes players would forget he was
there and their shouting and horseplay would get a little out of
control and then he would steal up behind them on cat-silent feet
and deliver a hard blow to the ear that would send them reeling. It
was enough of an object lesson -- ``Don't make the Snakehead mad or
he'll lay a beating on you that you won't forget'' -- that he hardly
ever had to repeat it.

Now, though, he was clouting boys left and right, bellowing orders
in a loud, hoarse voice. The boys retreated to their computers in a
shoving rush, leaving Ruiling alone in his seat, an uncertain smile
on his face.

``Boss,'' he said, ``you see what I've done?'' He pointed to his
screen.

Huang's face was as impassive as ever. He put a hard, heavy hand on
Ruiling's shoulder and leaned in to read the screen, his head
wreathed in smoke. Finally, he straightened. ``Fenrir's Tooth,'' he
said. He nodded. ``A bonus for you, Ruiling. Very good.''

Ruiling shrank back. ``Boss,'' he said, respectfully, speaking loudly
to be heard over the computer fans. ``Boss, that is my character. I
am not working now. It's my personal character.''

Huang turned to look at him, his eyes hard and his expression flat.
``A bonus,'' he said again. ``Well done.''

``It's \emph{my} character,'' Ruiling said, speaking more loudly. ``No
bonus. It's \emph{mine}! \emph{I} earned it, personally, on my own
time.''

He didn't even see the blow, it was that fast. One minute he was
hotly declaring that Fenrir's Tooth was his, the next he was
sprawled on his ass on the floor, his head ringing like a gong. The
foreman put one foot on his throat.

The man said, ``No bonus,'' clearly and distinctly, so that everyone
around could hear. Then he hawked up a huge mouthful of poisonous
green spit from the tar-soaked depths of his blackened lungs and
carefully spat in Ruiling's face.

From the age of four, Ruiling had practised wushu, training with a
man in the village whom all the adults deferred to. The man had
been sent north during the Cultural Revolution, denounced and
beaten and starved, but he never broke. He was as gentle and
patient as a grandmother, and he was as old as the hills, and he
could send an attacker flying through the air with a flick of the
wrist; break a board with his old hands, kick you into the next
life with one old, gnarled foot. For 12 years, Ruiling had gone
three times a week to train with the old man. All the boys had. It
was just part of life in the village. He hadn't practised since he
came to South China, had all but forgotten that relic of a
different China.

But now he remembered every lesson, remembered it deep in his
muscles. He gripped the ankle of the foot that was on his throat,
twisted just \emph{slightly} to gain maximum leverage, and applied
a small, controlled bit of pressure and \emph{threw} the foreman
into the air, sending him sailing in a perfect, graceful arc that
terminated when his head \emph{cracked} against the side of one of
the long trestle-tables, knocking it over and sending a dozen
flatscreens tumbling to the ground, the crash audible over the
computer fans.

Ruiling stood, carefully, and faced the foreman. The man was
groaning on the ground, and Ruiling couldn't keep the small grin
off his face. That had felt \emph{good}. He found that he was
standing in a ready stance, weight balanced evenly on each foot,
feet spread for stability, body side-on to the man on the ground,
presenting a smaller target. His hands were loosely held up, one
before the other, ready to catch a punch and lock the arm and throw
the attacker, ready to counterstrike high or low. The boys around
him were cheering, chanting his name, and Ruiling smiled more
broadly.

The foreman picked himself up off the floor, no expression at all
on his face, a terrible blankness, and Ruiling felt his first
inkling of fear. Something about how the man held himself as he
stood, not anything like the stance in the martial arts games he'd
played in the village. Something altogether more serious. Ruiling
heard a high whining noise and realized it was coming from his own
throat.

He lowered his hands slightly, extended one in a friendly, palm up
way. ``Come on now,'' he said. ``Let's be adults about this.''

And that's when the foreman reached under the shoulder of his
ill-fitting, rumpled, dandruff-speckled suit-jacket and pulled out
a cheap little pistol, pointed it at Ruiling, and shot him square
in the forehead.

Even before Ruiling hit the ground, one eye open, the other shut,
the boys around him began to roar. The foreman had one second to
register the sound of a hundred voices rising in anger before the
boys boiled over, clambering over one another to reach him. Too
late, he tried to tighten his finger on the trigger of the gun he'd
carried ever since leaving behind Fujian province all those years
before. By then, three boys had fastened themselves to his arm and
forced it down so that the gun was aiming into the meat of his old
thigh, and the .22 slug he squeezed off drilled itself into the big
femur before flattening on the shattered bone, spreading out like a
lead coin.

When he opened his mouth to scream, fingers found their way into
his cheeks, viciously tearing at them even as other hands twined
themselves in his hair, fastened themselves to his feet and his
arms, even yanked at his ears. Someone punched him hard in the
balls, twice, and he couldn't breathe around the hands in his
mouth, couldn't scream as he tumbled down. The gun was wrenched
from his hand at the same instant that two fists drilled into his
eyes, and then it was dark and painful and infinite, a moment that
stretched off into his unconsciousness and then into --
annihilation.

\tb

``So now what?'' Justbob slurped at her congee, which they'd sent out
for, along with strong coffee and a plate of fresh rolls. At 3AM in
the Geylang, food choices were slightly limited, but they never
went away altogether.

The Mighty Krang pulled up a video, waited for it to buffer, then
scrolled it past, fast. ``Three of the boys caught the shooting --
the \emph{execution} -- on their phones. The goon who went down,
well, he doesn't look so good.'' A shot from inside the dark room,
now abandoned, the foreman on his back amid a wreck of broken
computers and monitors, motionless, both arms broken at the elbows,
face a ruin of jelly and blood. ``We assume he's dead, but the
strikers aren't letting anyone in.''

``Strikers,'' Justbob said, and The Mighty Krang clicked another
video. This one took longer to load, some server somewhere groaning
under the weight of all the people trying to access it at once.
That never happened any more, it had been years since it had
happened, and it made Justbob realize how fast this thing must be
spreading. The realization scythed through her grogginess, made her
eye spring open, the other ruin work behind its patch.

The video loaded. Hundreds of boys, gathered in front of an
anonymous multi-story building, the kind of place you pass by the
thousand. They'd tied their shirts around their faces, and they
were pumping their fists in the air and more people were coming out
to join them. Boys, old people, girls --

``Girls?''

``Factory girls. Jiandi. She did a special broadcast. Stupid. She
nearly got caught, chased out of another safe house. She's running
out of bolt holes. But she got the word out.''

``Did we know?''

Big Sister Nor's face was a thundercloud, ominous and dark. ``Of
course not. If we'd known, we would have told her not to do it.
Chill out. Hold off. We have a schedule, lots of moving parts.''

``The dead boy?''

``There --'' Krang said, and pointed his mouse at the edge of the
video. A trestle table, set up beside the boys, with the dead boy
draped on it. Looking closely, she could see the bullet hole in his
forehead, the streak of blood running down the side of his face.

``Aha,'' Justbob said. ``Well, we're not going to cool anything out
now.''

Big Sister Nor said, ``We don't know that. There's still a chance
--''

``There's no chance,'' Justbob said, and her finger stabbed at the
screen. ``There are \emph{thousands} of them out there. What's
happening in world?''

``It's a disaster,'' Krang said. ``Every gold-farming operation is in
chaos. Webblies are attacking them by the thousands. And it gets
worse as the day goes by. They're just waking up in China, so fresh
forces should be coming in --''

Justbob swallowed. ``That's not a disaster,'' she said. ``That's
battle. And they'll win. And they'll keep on winning. From this
moment forward, I'd be surprised to see if \emph{any} new gold
comes onto the markets, in any game. We can change logins as fast
as the gamerunners shut down accounts, and what's more, there are
plenty of regular players who've been skirmishing with us for the
fun of it who'll shout bloody murder if they lose their accounts.
We've got the games sewn up.'' She kept her face impassive, reached
for a cup of tea, sipped it, set it down.

Big Sister Nor stared at her for a long time. They had been friends
for a long time, but unlike Krang, Justbob wasn't in worshipful
love with Nor. She knew just how human Big Sister Nor could be, had
seen her screw up in small and big ways. Big Sister Nor knew it,
too and had the strength of character to listen to Justbob even
when she was saying things that Nor didn't want to hear.

Krang looked back and forth between the two young women, feeling
shut out as always, trying not to let it show, failing. He got up
from the table, muttering something about going out for more
coffee, and neither woman took any notice.

``You think that we're ready?'' Big Sister Nor said after the
safe-house door clicked shut.

``I think we have to be,'' said Justbob. ``The first casualty of any
battle\ldots{}''

``I know, I know,'' Big Sister Nor said. ``You can stop saying that
now.''

When The Mighty Krang came back, he saw immediately how things had
gone. He distributed the coffee and got to work.

\tb

Mrs Dibyendu's cafe was locked up tight, shutters drawn over the
windows and doors.

``Hey!'' called Ashok, rapping on the door. ``Hey, Mrs Dibyendu! It's
Ashok! Hey!'' It was nearly 7AM, and Mrs Dibyendu always had the
cafe open by 6:30, catching some of the early morning trade as the
workers who had jobs outside of Dharavi walked to their bus-stops
or the train station. It was unheard of for her to be this late.
``Hey!'' he called again and used his key-ring to rap on the metal
shutter, the sound echoing through the tin frame of the building.

``Go away!'' called a male voice. At first Ashok assumed it came from
one of the two rooms above the cafe, where Mrs Dibyendu rented to a
dozen boarders -- two big families crammed into the small spaces.
He craned his neck up, but the windows there were shuttered too.

``Hey!'' he banged on the door again, loud in the early morning
street.

Someone threw the bolts on the other side of the door and pushed it
open so hard it bounced off his toe and the tip of his nose, making
both sting. He jumped back out of the way and the door opened
again. There was a boy, 17 or 18, with a huge, pitted machete the
length of his forearm. The boy was skinny to the point of
starvation, bare-chested with ribs that stood out like a xylophone.
He stared at Ashok from red-rimmed, stoned eyes, pushed lanky,
greasy hair off his forehead with the back of the hand that wasn't
holding the machete. He brandished it in Ashok's face.

``Didn't you hear me?'' he said. ``Are you deaf? Go away!'' The machete
wobbled in his hand, dancing in the air before his face, so close
it made him cross his eyes.

He stepped back and the boy held his arm out further, keeping the
machete close to his face.

``Where's Mrs Dibyendu?'' Ashok said, keeping his voice as calm as he
could, which wasn't very. It cracked.

``She's gone. Back to the village.'' The boy smiled a crazy, evil
smile. ``Cafe is closed.''

``But --'' he started. The boy took another step forward, and a wave
of alcohol and sweat-smell came with him, a strong smell even amid
Dharavi's stew of smells. ``I have papers in there,'' Ashok said.
``They're mine. In the back room.''

There were other stirring sounds from the cafe now, more skinny
boys showing up in the doorway. More machetes. ``You go now,'' the
lead boy said, and he spat a stream of pink betel-stained saliva at
Ashok's feet, staining the cuffs of his jeans. ``You go while you
can go.''

Ashok took another step back. ``I want to speak to Mrs Dibyendu. I
want to speak to the owner!'' he said, mustering all the courage he
could not to turn on his heel and run. The boys were filing out
into the little sheltered area in front of the doorway now. They
were smiling.

``The owner?'' the boy said. ``I'm his representative. You can tell
me.''

``I want my papers.''

``My papers,'' the boy said. ``You want to buy them?''

The other boys were chuckling now, hyena sounds. Predator sounds.
All those machetes. Every nerve in Ashok's body screaming
\emph{go}. ``I want to speak with the owner. You tell him. I'll be
back this afternoon. To talk with him.''

The bravado was unconvincing even to him and to these street hoods
it must have sounded like a fart in a windstorm. They laughed
louder, and louder still when the boy took another rushing step
toward him, swinging the machete, just missing him, blade whistling
past him with a terrifying whoosh as he backpedaled another step,
bumped into a man carrying a home-made sledgehammer on his way to
work, squeaked, actually \emph{squeaked}, and ran.

Mala's mother answered his knock after a long delay, eyeing him
suspiciously. She'd met him on two other occasions, when he'd
walked ``the General'' home from a late battle, and she hadn't liked
him either time. Now she glared openly and blocked the doorway.
``She's not dressed,'' she said. ``Give her a moment.''

Mala pushed past her, hair caught in a loose ponytail, her gait an
assertive, angry limp. She aimed a perfunctory kiss at her mother's
cheek, missing by several centimeters, and gestured brusquely down
the stairs. Ashok hurried down, through the lower room with its own
family, bustling about and getting ready for work, then down
another flight to the factory floor, and then out into the stinging
Dharavi air. Someone was burning plastic nearby, the stench
stronger than usual, an instant headache of a smell.

``What?'' she said, all business.

He told her about the cafe.

``Bannerjee,'' she said. ``I wondered if he'd try this.'' She got out
her phone and began sending out texts. Ashok stood beside her, a
head taller than her, but feeling somehow smaller than this girl,
this ball of talent and anger in girl form. Dharavi was waking now,
and the muzzein's call to prayer from the big mosque wafted over
the shacks and factories. Livestock sounds -- roosters, goats, a
cowbell and a big bovine sneeze. Babies crying. Women struggled
past with their water jugs.

He thought about how unreal all this was for most of the people he
knew, the union leaders he'd grown up with, his own family. When he
talked with them about Webbly business, they mocked the unreality
of life in games, but what about the unreality of life in Dharavi?
Here were a million people living a life that many others couldn't
even conceive of.

``Come on,'' she said. ``We're meeting at the Hotel U.P..''

When he'd come to Dharavi, the ``hotels'' on the main road in the
Kumbharwada neighborhood had puzzled him, until he found out that
``hotel'' was just another word for restaurant. The Webblies liked
the Hotel U.P., a workers' co-op staffed entirely by women who'd
come from villages in the poor state of Uttar Pradesh. It was
mutual, the women enjoying the chance to mother these serious
children while they spoke in their impenetrable jargon, a blend of
Indian English, gamerspeak, Chinese curses, and Hindi, the curious
dialect that he thought of as \emph{Webbli}, as in \emph{Hindi}.

The Webblies, roused from their beds early in the morning, crowded
in sleepily, demanding chai and masala Cokes and dhosas and aloo
poories. The ladies who owned the restaurant shuttled pancakes and
fried potato popovers to them in great heaps, Mala paying for them
from a wad of greasy rupees she kept in a small purse she kept
before her. Ashok sat beside her on her left hand, and Yasmin sat
on her right, eyes half-lidded. The army had been out late the
night before, on a group trip to a little filmi palace in the heart
of Dharavi, to see three movies in a row as a reward for a run of
genuinely excellent play. Ashok had begged off, even though he'd
been training with the army on Mala's orders. He liked the
Webblies, but he wasn't quite like them. He wasn't a gamer, and it
would ever be thus, no matter how much fighting he did.

``OK,'' Mala said. ``Options. We can find another cafe. There is the
1000 Palms, where we used to fight --'' she nodded at Yasmin,
leaving the rest unsaid,
\emph{when we were still Pinkertons, still against the Webblies}.
``But Bannerjee has something on the owner there, I've seen it with
my own eyes.''

``Bannerjee has something on every cafe in Dharavi,'' Sushant said.
He had been very adventurous in scouting around for other places
for them to play, on Yasmin's orders. Everyone in the army knew
that he had a crush on Yasmin, except Yasmin, who was seemingly
oblivious to it.

``And what about Mrs Dibyendu?'' Yasmin said. ``What about her
business, all the work she put into it?''

Mala nodded. ``I've called her three times. She doesn't answer.
Perhaps they scared her, or took her phone off of her. Or\ldots{}''
Again, she didn't need to say it, \emph{or she is dead.} The stakes
were high, Ashok knew. Very high. ``And there's something else. The
strike has started.''

Ashok jumped a little. \emph{What?} It was too early -- weeks too
early! There was still so much planning to do! He pulled out his
phone, realized that he'd left it switched off, powered it up,
stared impatiently at the boot-screen, listening to the hubub of
soldiers around him. There were \emph{dozens} of messages waiting
for him, from Big Sister Nor and her lieutenants, from the special
operatives who'd been working on the scam with him, from the
American boy who'd been coordinating with the Mechanical Turks.
There had been fighting online and off, through the night, and the
Chinese were thronging the streets, running from cops, regrouping.
Gamespace was in chaos. And he'd been arguing with drunken
thug-boys at the cafe, eating aloo poories and guzzling chai as
though it was just another day. His heart began to race.

``We need to get online,'' he said. ``Urgently.''

Mala broke off an intense discussion of the possibility of getting
PCs into a flat somewhere and bringing in a network link to look at
him. ``Bad as that?''

He held up his phone. ``You've seen, you know.''

``I haven't looked since you came to my place. I knew that there was
nothing we could do until we found a place to work. It is bad,
then.'' It wasn't a question.

They were all hanging on him. ``They need our help,'' he said.

``All right,'' Mala said. ``All right. So. We go and we take over Mrs
Dibyendu's place again. Bannerjee doesn't own it. Everyone in her
road knows that. They will take our side. They must.''

Ashok gulped. ``Force?'' He remembered the boy: drunk, fearless, eyes
flat, the sharp machete trembling.

The gaze Mala turned on him was every bit as flat. She could
transform like that, in a second, in an \emph{instant}. She could
go from pretty young girl, charismatic, open, clever and laughing
to stone-faced General Robotwallah, ferocious and uncompromising.
Her flat eyes glittered.

``Force if necessary, always,'' she said. ``Force. Enough force that
they go away and don't come back. Hit them hard, scare them back to
their holes.'' Around the table, thirty-some Webblies stared at her,
their expressions mirrors of hers. She was their general, and
before she came into their lives, they had been Dharavi rats,
working in factories sorting plastic, going to school for a few
hours every day to share books with four other students. Now they
were royalty, with more money than their parents earned, jobs and
respect. They'd follow her off a cliff. They'd follow her
\emph{into the Sun}.

But Yasmin cleared her throat. ``Force if we must,'' she said. ``But
surely no more than is necessary, and not even that if we can help
it.''

Mala turned to her, back rigid, neck corded, jaw set. Yasmin met
her gaze with calm eyes and then\ldots{}\emph{smiled}, a small and sweet
and genuine smile. ``If the General agrees, of course.''

And Mala melted, the tension going out of her, and she returned
Yasmin's smile. Something had changed between them since the night
Mala had attacked them, something had changed for the better. Now
Yasmin could defuse Mala with a look, a smile, a touch, and the
army respected it, treating Yasmin with reverence, sometimes going
to her with their grievances.

``Of course,'' Mala said. ``No more force than is absolutely
necessary.'' She picked up her cane -- topped with a silver skull, a
gift from her troops -- and made a few vicious swipes in the air,
executed with the grace of a fencer. He knew that there was a lead
weight in the foot of the cane, and he'd seen her knock holes in
brick with a swing. Her densely muscled forearms hardly trembled as
she wielded the cane. Behind her, one of the ladies who ran the
restaurant looked on with heartbreaking sorrow, and Ashok wondered
how many young people she'd seen ruined in her village and here in
the city.

``We go,'' Mala said, and scraped her chair back. Ashok fell in
beside her and the army marched down the main road three abreast,
causing scooters and motorcycles and goats and three-wheeled
auto-rickshaws to part around them. Many times Ashok had seen
swaggering gangs of badmashes on the street, had gotten out of
their way. Now he was in one, a collection of kids, just kids, the
youngest a mere 13, the eldest not yet 20, led by a limping girl
with a long neck and hair in a loose ponytail, and around them,
people reacted with just the same fear. It swelled Ashok's heart,
the power and the fear, and he felt ashamed and exhilarated.

Before the door of Mrs Dibyendu, Mala stooped and pried a rock from
the crumbling pavement with her fingers, unmindful of the filth
that slimed it. She threw it with incredible accuracy, bowling it
like a cricket ball, \emph{crash}, into the sheet-tin door of the
cafe. Immediately, she bent to pick up another rock, prying it
loose before the echoes of the first one had died down. Around
them, in the narrow street, heads appeared from windows and
doorways, and curious pedestrians stopped to look on.

The door banged open and there was the boy who had threatened Ashok
earlier, eyes bloodshot and pink even from a safe distance. He held
his machete up like a sword, a snarl on his lips. It died as he
contemplated the 30 soldiers arrayed before him. Many had produced
lengths of wood or iron, or picked up rocks of their own. They
stared, unwavering, at the boy.

``What is it?'' He was trying for bravado, but it came out with a
squeak at the end. The machete trembled.

``Careful,'' whispered Ashok, to himself, to Mala, to anyone who
would listen. A scared bully was even less predictable than a
confident one.

``Mrs Dibyendu asked us to come re-open her cafe for her,'' Mala
said, gesturing with her phone, held in her free hand. ``You can go
now.''

``The new owner asked us to watch \emph{his} cafe,'' the boy said,
and everyone on the street heard both lies, Mala's and the boy's.
Ashok tried to figure out how old the boy was. 14? 15? Young, dumb,
drunk and angry and armed.

``Careful,'' he whispered again.

Mala pocketed her phone and hefted her rock, eyes never leaving the
boy.

``Five,'' she said.

He grinned at her and spat a stream of pink, betel saliva toward
her feet. She didn't move. No one moved.

``Four.''

He raised the machete, point aimed straight at her. She didn't seem
to notice.

``Three.''

Silence rang over the alley. Someone on a motorbike tried to push
through the crowd, then stopped, cutting the engine.

``Two.''

The boy's eyes cut left, right, left again. He whistled then, hard
and loud, and there was a scrabble of bare feet from the cafe
behind him.

``One,'' Mala said. and raised the rock, winding up like a cricket
bowler again, whole body coiled, and Ashok thought,
\emph{I have to do something. Have to stop them. It's insane.} But
his mouth and his hands and his feet had other ideas. He remained
frozen in place.

The boy raised his machete across his chest, and the hand that held
it trembled even more. Abruptly, Mala threw. The rock flew so fast
it made a sizzling sound in the hot, wet morning air, but it didn't
smash the boy's head in, but rather dashed itself to pieces against
the door-frame behind him, visibly denting it. The boy flinched as
shattered rock bounced off his bare face and chest and arm and
back, a few stray pieces pinging off the machete.

``Leave,'' Mala said. Behind the boy, five more boys, crowding out of
the doorway, each with his machete. They raised their arms.

``Fight!'' hissed one of the boys, the smallest one. There was
something wrong with his head, a web of scar and patchy hair
running down the left side as though he'd had his head bashed in or
been dragged. Ashok couldn't look away from this little boy. He had
a cousin that size, a little boy who liked to play games in the
living room and run around with his friends. A little boy with
shoes and clear eyes and three meals a day and a mother who would
tuck him up every night with a kiss on the forehead.

Mala fixed the boy with her gaze. ``Don't fight,'' she said. ``If you
fight, you lose. Get hurt. Run.'' The army raised their weapons,
made a low rumbling sound that raised to a growl. One of the boys
was on his phone, whispering urgently into it. Ashok saw their fear
and felt a featherweight of relief, these ones would go, not fight.
``Run!'' Mala said, and stamped forward. The boys all flinched.

And some of the army snickered at them, a hateful sound that he'd
heard a thousand times while in-game, a taunting sound that spread
through the ranks like a snake slithering around their feet, and
the fear in the boys' faces changed. Became anger.

The moment balanced on a thread as fine as spider's silk, the
snickering soldiers, the boiling boys, the machetes, the clubs and
sticks, the rocks --

The moment broke. The smallest boy held his machete over his head
and charged them, screaming something wordless, howling, really, a
sound Ashok had never heard a boy make. He got three steps before
two rocks caught him, one in the arm and the second in the face, a
spray of blood and a crunch of bone and a tooth that flew high in
the air as the boy fell backwards as if poleaxed.

And the moment shattered. Machetes raised, the remaining five boys
ran for the army, a crazy look in their faces. Ashok had time to
wonder if the little boy lying motionless on the ground was the
smaller brother of one of the remaining badmashes and then the
fight was joined. The tallest boy, the one who'd answered the door
that morning and spat at him, hacked his way through two soldiers,
dealing out deep cuts to their chests and arms -- Ashok's face
coated with a fine mist of geysering arterial blood -- face
contorted with rage. He was coming for Mala, standing centimeters
from Ashok, and the blood ran off his machete and down his arm.

Mala seemed frozen in place, and Ashok thought that he was about to
die, to watch her die first, and he tensed, blood roaring in his
ears so loudly it drowned out the terrible screams of the fighters
around him, desperate and about to grab for the boy. But as he
shifted his weight, Mala barked ``NO!'' at him, never shifting her
eyes from the leader, and he checked himself, stumbling a half-step
forward. The boy with the machete looked at him for the briefest of
instants and Mala \emph{whirled}, uncoiling herself, using the
weighted skull-tipped cane to push herself off, then whipping out
the arm, the gesture he'd seen her mime countless times in battle
lessons, and the weighted tip crashed into the boy's forearm with a
crack he heard over the battle-sounds, a crack that he'd last heard
that night so many months before, when Mala and her army had come
for him and Yasmin in the night. Ashok the doctor's son knew
exactly what that crack meant.

A blur of fabric as Yasmin danced before him, stooping gracefully
to take the machete up, and the boy just watched, eyes glazed,
shock setting in already. Yasmin delicately and deliberately kicked
him in the kneecap, a well-aimed kick with the toe of her sandal,
coming in from the side, and the boy went down, crying in a little
boy's voice, calling out for his mother with a sound as plaintive
as a baby bird that's fallen from the nest.

It had been mere seconds, but it was already over. Two of the boys
were running away, one was sobbing through a bloody mouth, two were
unconscious. Ashok looked for wounded soldiers. Three had been cut
with machetes, including the two he'd seen hurt by the leader as he
ran for Mala. Remembering the arterial blood, red and rich, Ashok
found its owner first, lying on the ground, eyes half open, breath
labored. He pushed his hands over the injury, a deep cut on the
left arm that spurted with each of the hammering beats of the boy's
chest and he shouted, ``A shirt, anything, a bandage,'' and someone
pressed a shirt into his bloody hands and he applied hard pressure,
staunching the blood. ``Someone call for a doctor,'' he said, making
eye-contact with Anam, a soldier he had hardly spoken to before.
``You have a phone?'' The girl was shivering slightly, but she nodded
and patted a handbag at her side, absentmindedly swinging the
length of iron in her hand. She dropped it. ``You call the doctor,
you understand?'' She nodded. ``What will you do?''

``Call the doctor,'' she said, dreamily, but she began to dial. He
turned and grabbed the hand that had passed him the shirt, and he
saw that it was attached to Mala, who had stripped it off of
another boy in her army. Her chest was heaving, but her gaze was
calm.

``Hold here,'' he ordered, without a moment's scruple about dictating
to the general. This was first aid, it was what he had been trained
for by his father, long before he studied economics, and it brooked
no argument. He pressed her hand against the bloody rag and stood,
not hearing the crackle of his joints. He turned and found the next
injured person, and the next.

And then he came to the boy, the little boy whose misshapen head
had caught his attention. The boy who'd been hit high and low with
two hard-flung rocks. The whole front of his jaw was crushed, a
nightmare of whitish bone and tooth fragments swimming in a jelly
of semi-clotted blood. When Ashok peeled back each eyelid, he saw
that the left pupil was as wide as a sewer entrance, and did not
contract when he moved away and let the sun shine full on it.
``Concussion,'' he muttered to the air, and Yasmin answered, ``Is that
bad?''

``His brain is bleeding,'' Ashok said. ``If it bleeds too much, he
will die.'' He said it simply, as if reading from a textbook. The
boy smelled terrible, and there were sores on his arms and chest
and ankles, swollen, overscratched and infected insect-bites and
boils. ``He has to see a doctor.'' He looked back to the bleeding
soldier. ``Him too.''

He found the girl who'd promised to call a doctor. ``Where is the
doctor?'' He had no idea how much time had passed since he'd told
her to call. It could have been ten minutes or two hours.

She looked confused. ``The ambulance,'' she began. She looked around
helplessly. ``It will come, they said.''

And now that he listened for it, he heard it, a distant dee-dah,
dee-dah. The narrow lane that housed Mrs Dibyendu's cafe would
never admit an ambulance. Without speaking, Yasmin ran for the main
road, to hail it. And now that Ashok was listening, he could hear:
neighbors with their heads stuck out of their windows and doorways,
passing furious opinion and gupshup. They cheered on Mala's army,
rained curses down on the badmashes with their machetes, lamented
Mrs Dibyendu's departure, chattered like tropical birds about how
she had been forced out, weeping, and chased down the road in the
dark of night.

Ashok was covered in blood. It covered his hands, his arms, his
chest, his face. His lips were covered in dried blood, and there
was a coppery taste in his mouth. His shirt and trousers -- soaked.
He straightened and looked around the crowded lane, up at the
chatterers, blinking owlishly. Around him, the soldiers and the
wounded.

Mala was whispering urgently in Sushant's ear, the boy listening
intently. Then he began to move among the soldiers, urging them
inside. The Webblies had work to do. The police would come soon,
and the people inside the building would have the moral authority
to claim it was theirs. The boys with their machetes, injured or
gone, would have no claim. Ashok wondered if he would be arrested,
and, if he was, whether he'd be able to get out. Maybe his father
could take care of it. An important man, a doctor, he could take
care --

Two ambulance technicians arrived, bearing heavy bags and collapsed
stretchers. They were locals, with Dharavi accents, sent from the
Lokmanya Tilak hospital, a huge pile with a good reputation.
Quickly, he described the injuries to the men, and they split up to
look at the most serious cases, the deep arterial cut and the
concussion. Ashok stayed near the small boy, feeling somehow
responsible for him, more responsible than for his own teammate,
watched as the technician fitted the boy with a neck-brace and then
triggered the air-cannister that filled it, immobilizing his head.
Carefully, the technician seated a plastic ring in the donut-hole
center of the brace, over the boy's ruined jaw and nose, so that
the plastic wouldn't interfere with his breathing. He unfurled his
stretcher, snapped its braces to rigidity and looked at Ashok.

``You know the procedure?''

Instead of answering, Ashok positioned himself at the boy's skinny
hips, putting a hand on each, ready to roll him up at the same time
as the medic, keeping his whole body in line to avoid worsening any
spinal injuries. The medic slid the stretcher in place, and Ashok
rolled the boy back. For one brief moment, he was supporting nearly
all the boy's weight in his hands and the child seemed to weigh
nothing, nothing at all, as though he was hollow. Ashok found that
he was crying, silent tears that slid down his face, collecting
blood, slipping into his mouth, doubly salty blood and tear
mixture.

Mala silently slipped her arm in his. She was very warm in the
oppressive heat of the morning. There would be a rain soon, the
humidity couldn't stay this high all day, the water would come
together soon and then the blood would wash away into the rough
gutters that ran the laneway's length.

``He was a brave kid,'' Mala said.

Ashok couldn't find a reply.

``I think he thought that if he charged us with that knife, sliced
one of us up, we'd be so scared we'd go away forever.''

``You really understand him, then?'' Ashok saw Yasmin steal over to
them, slip her fingers into Mala's.

Mala didn't answer.

Yasmin said, ``Everyone thinks that you can win the fight by
striking first.'' Mala's arm tightened on Ashok's arm. ``But
sometimes you win the fight by not fighting.''

Mala said, ``We should call you General Gandhiji.''

``It'd be an honor, but I couldn't live up to Gandhi. He was a great
man.''

Ashok said, ``Gandhi admitted to beating his wife. He was a great
man, but not a saint.'' He swallowed. ``No one mentions that Gandhi
had all that violence inside him. I think it makes him better,
because it means that his way wasn't just some natural instinct he
was born with. It was something he battled for, in his own mind,
every day.'' He looked down at the top of Mala's head, startled for
a moment to realize that she was shorter than him. He had a
tendency to think of her as towering, larger than life.

Mala looked up at him and it seemed that her dark eyes were glowing
in the hot, steamy air, staring out from under her long lashes.
``Controlling yourself is overrated,'' she said. ``There's plenty to
be said for letting go.''

There were so many eyes on them, so many people watching from every
corner of the road, and Ashok felt suddenly very self-conscious.

Inside, the cafe was hardly recognizable. It stank like the den of
some sick animal that had gone to ground, and one corner had been
used as a toilet. Many of the computers had been carelessly moved,
disconnecting their wires, and one screen was in fragments on the
floor. There were betel-spit streaks around the floor, and empty
bottles of cheap, fiery booze so awful even the old drunks in the
streets wouldn't drink it.

But there was also a photo, much-creased and folded, of a worn but
still pretty woman, formally posed, holding a baby and a slightly
larger boy, whom Ashok remembered from the melee. The baby, he
thought, must have been that younger boy, and he wondered what had
become of the woman, and how she was separated from the sons she
held with so much love. And the more he wondered, the more numb and
sorrowful he felt, until the sorrow welled over him in black waves,
like a tide coming in, until he buckled at the knees and went down
to the floor, and if any of the soldiers saw him hold himself and
cry, no one said a word.

His papers were intact, mostly, in the back room where he'd worked,
and the network connection was still up, and the garbage was all
swept out the door and the windows were flung open and soon the
sound of joyous combat and soldierly high spirits filled Mrs
Dibyendu's, as it had for so many days before. Ashok fell into the
numbers and the sheets, seeing how he could work them with the new
dates, and he was so engrossed that he didn't even notice the
sudden silence in the cafe that marked the arrival of a policeman.

The policeman -- fat, corrupt, an old Dharavi rat himself, and more
a creature of the slum than the children -- had already gotten an
account from the neighbors, heard that the machete-wielding
badmashes had been the invaders here, and he wasn't about to get
exercised on behalf of six little nobodies like them. But when
there was a death, there had to be paperwork\ldots{}

``Death?'' Ashok said.

``The small one. Dead by the time he reached the hospital.''

Ashok felt as though the floor was dropping away from him and the
only thing that distracted him and kept him from falling with it
was the gasp of dismay from Yasmin behind him, a sound that started
off as an exhalation of breath but turned into a drawn out whimper.
He turned and saw that she had gone so pale that she was actually
green, and the doctor's son in him noticed that her pupils had
shrunk to pinpricks.

The fat policeman looked at her, and his lips twisted into a wet,
sarcastic smile. ``Everything all right, miss?''

``She's fine,'' Mala said, flatly. She was standing closer to the
policeman than was strictly necessary, too short to stare him in
the eye, but still she seemed to be looking down. Unconsciously,
the policeman shifted his weight back, then took a step back, then
turned.

``Good bye, then,'' he said, brandishing his notebook, containing
Ashok's identity card number; all the soldiers had claimed that
they were never registered for the card, which Ashok really
doubted, but which the policeman didn't question, as the air
whistled out of his nostrils and he sweated in his uniform. The
rains had finally come, the skies opening like floodgates, the rain
falling in sheets the color of the pollution they absorbed on their
fall from the heavens. The clatter on the tin walls and roof was
like a firefight in some cheap game where the guns all made
metallic \emph{pong} and \emph{ping} sounds.

Ashok watched as Yasmin drifted away into Mrs Dibyendu's little
``office,'' the room where she made the chai over a small gas burner;
watched as Mala followed her. He tried to work on his calculations,
but he couldn't concentrate until he saw Mala emerge, face slammed
shut into her General Robotwallah expression, but there were still
tracks from the tears on her cheeks. She looked straight through
him and started to bark orders to her soldiers, who had been
setting the cafe to rights and getting all the systems running
again. A moment later, they were all clicking, shouting, headsets
on, shoulders tight, in another world, and the battle was joined.

Ashok found his way into Mrs Dibyendu's office, found Yasmin
squatting by the wall, heels flat on the ground, hands before her.
She stared silently into those hands, twining them around each
other like snakes.

``Yasmin,'' he whispered. ``Yasmin?''

She looked at him. There were no tears in her eyes, only an
expression of bottomless sorrow. ``I threw the rock,'' she said. ``The
rock that hit that little boy. I threw it. The one that hit him in
the mouth. He was\ldots{}'' She swallowed.

``He was running at us with a machete,'' Ashok said. ``He would have
killed us --''

She chopped her hand through the air, a gesture full of
uncharacteristic violence. ``We
\emph{put ourselves in that position}, in the position where we'd
have to kill him! It was Mala. Mala, she always wants to win before
the battle is fought, win by \emph{annihilating the enemy.} And
then to talk of \emph{Gandhi}?'' She looked like she was going to
punch something, small hands balled in fists and then, abruptly,
she pitched forward and threw up, copiously, a complete ejection of
the entire contents of her stomach, more vomit than Ashok had ever
seen emerge from a human throat. In between convulsions, he
half-led, half-carried her out of the cafe, into the all-pounding
rain, and let her throw up into the laneway, which had become a
rushing river, the rain overflowing the narrow ditches on either
side of it. The water ran right up to the cracked slab of cement
that served as Mrs Dibyendu's doorstep, and Yasmin's hijab was
instantly soaked as she leaned out to spatter the water's turbulent
surface with poories and chai and bile. Her long dress clung to her
narrow back and shoulders, and it heaved with them as she labored
for breath. Ashok was soaked too, the blood-taste in his mouth
again as the water washed the dried blood down his face. The rain
made talking impossible so he didn't have to worry about soothing
words.

At last Yasmin straightened and then sagged against him. He put his
arm around her, grateful for the feeling of another human being,
that contact that penetrated his numbness. Something passed between
them, carried on the thudding of their hearts, transmitted by their
skin, and for a moment, he felt as though here, here at last, was
someone who understood everything about him and here was someone he
understood. The moment ended, ebbing away, until they were standing
in an embarrassed, awkward half-hug, and they wordlessly
disentangled and went back in. Someone had mopped up the vomit,
using the rags that the badmashes had left behind and then kicking
them in a reeking ball in the corner. Yasmin sat down at a computer
and logged in, listening intently to the chatter around her,
catching the order of battle, while Ashok went to his computer and
got ready to talk to Big Sister Nor.

\tb

The day the strike started, Wei-Dong was in the midst of his second
special assignment -- the first one had been to bring over the box
of prepaid cards, which had been handed off into the Webbly network
to be scratched off and then keyed in and sent to Big Sister Nor so
she could portion them out to the fighters.

The second assignment was harder in some ways: he was charged with
finding other Mechanical Turks who might be sympathetic to the
strikers' cause and recruit them. Wei-Dong had never thought of
himself as much of a leader -- he'd always been a loner in school
-- but Big Sister Nor had talked to him at length about all the
ways in which he might convince his fellow Turks to consider
joining this strange enterprise.

Technically, it was simple enough to accomplish. As a Turk, he had
access to the leaderboards of Turk activity, which Coca-Cola Online
made a big deal out of, updating them every ten minutes. The
leaderboards listed each Turk by name and showed which parts of the
game he or she hung out in, how many queries she or he handled per
hour, how highly rated the Turk's rulings and role-play were rated
by the players who were randomly surveyed by a satisfaction-bot
that gave out rare badges to any player who would fill in an
in-game questionnaire. The idea was to inspire the Turks by showing
them how much better their peers were doing. It worked, too --
Wei-Dong had spent many a night trying to pump his stats so that he
could get ahead of the other Turks, scaling to the highest heights
before being knocked down by someone else's all-night run. And, of
course, when you pulled ahead of another Turk, you got to leave a
public ``message of encouragement'' for them, no more than 140
characters so that it could be tweeted and texted straight to them,
and these messages had pushed the boundaries of extremely terse
profanity and boasting.

Wei-Dong had a new use for the boards: he was using them to figure
out which players were likely to switch sides. The game-runners had
created a facility for bulk-downloading historical data from them,
and Turks were encouraged to make crazy mash-ups and visualizations
showing whose play was the best. Wei-Dong had a different idea.

For weeks now, he'd been downloading gigantic amounts of data from
the boards, piping it all into a database that Matthew had helped
him build and now he could run some very specialized queries on it,
queries like, ``Show me Turks who used to lead the pack but have
fallen off, despite long hours of work.'' Or ``Show me Turks who use
a lot of profanity when they're filling in the dialog for
non-player characters.'' And especially, ``Show me Turks who have a
below-average level of ratting out gold-farmers to the bosses.''
This last one was a major enterprise among Turks, who got a big
bonus every time they busted a farmer. Most of the Turks went
``de-lousing'' pretty often, looking to rack up the extra cash. But a
significant minority never, ever hunted the farmers, and these were
Wei-Dong's natural starting point.

He had a long list of leads, and for each one, he had a timetable
of the Turk's habitual login hours and the parts of the world that
the Turk worked most often. Then it was only a matter of logging in
using one of the Webblies many, many toons, heading to that part of
the world, and invoking the Turk and hoping the right person showed
up. It would be easier to just use the Turk message boards, but if
he did, he'd be busted and fired in seconds. This way was less
efficient but it was a lot safer.

Now he was in the Goomba's Star-Fields, a cloudscape in Mushroom
Kingdom where the power-up stars were cultivated in endless rows.
Players could quest here, taking jobs with comical farmers who'd
put them to work weeding the star patches and pulling up the ripe
ones. It was good for training up your abilities; a highly ranked
Star Farmer could get more power-up out of his stars.

And here was the farmer, chewing a corn-stalk and puttering around
his barn, which was also made from clouds. He offered Wei-Dong a
quest -- low-level, just pulling up weeds from some of the
easier-to-reach clouds, the ones that weren't patrolled by hostile
Lakitus. Wei-Dong accepted the quest, and then opened a chat with
the farmer: ``How long have you owned this farm?''

``Oh, youngster, I've been working this farm since I was but a boy
-- and my pappy worked it before me and his pappy before him. Yep,
I guess you could say that we're a farming family, hee hee!''

This was canned dialog, of course. No Turk could ever bring himself
to type anything that hokey. The farmer NPC had a whole range of
snappy answers to stupid questions. The trick to invoking a Turk
was to get outside the box.

``Do you like farming?''

``Ay-yuh, you might say I do. It's a good living -- when the sun
shines! Hee hee!''

Wei-Dong rolled his eyes. Who \emph{wrote} this stuff? ``What
problems do you have as a farmer?''

``Oh, it's a good living -- when the sun shines! Hee hee!''

Wei-Dong smiled a little. Once the NPC started repeating itself, a
Turk would be summoned. The farmer seemed to twitch a little.

``Do you have any problems apart from lack of sunshine?''

``Oh, youngster, you don't want to hear an old farmer's complaints.
Many and many a day I have toiled in these fields and my hands are
tired. Let's speak of more pleasant things, if you please.'' That
was more like it. The dialog was the kind of thing an enthusiastic
role-playing Turk would come up with, and that fit the profile of
the Turk he was after.

``Is your name Jake Snider?'' he typed.

The character didn't move for a second. ``I ken not this Jake
Snider, youngster. You'd best be on with your chores, now.''

``I think you \emph{are} Jake Snider and I think you know that
you're not getting a fair deal out of Coke. You're pulling down
more hours than ever, but your pay is way down. Why do you suppose
that is? Did you know that Coca-Cola Games just had its best
quarter, ever? And that the entire executive group got a 20 percent
raise? Did you know that Coke systematically rotates Turks who make
too much money out of duty, replacing them with newbies who don't
know how to maximize their revenue?''

The farmer started to walk away, rake over his shoulder. Wei-Dong
followed.

``Wait! Here's the thing. It \emph{doesn't have to be this way}!
Workers can organize and demand a better deal from their bosses.
Workers \emph{are} organizing. You give it two more months and
you'll be out on the street. Isn't your pay and your dignity worth
fighting for?

The farmer was headed into his house. Wei-Dong thought for a second
that he was talking to the NPC again, that the Turk had logged out.
But no, there was a little clumsiness in the farmer's movements, a
little hesitation. There was still someone home. ``I know you can't
talk to me in-game. Here's an email address --
D9FA754516116E89833A5B92CE055E19BCD2FA7@gmail.com. Send me a
message and we'll talk in private.''

He held his breath. The Turk could have been ratting him out to
game management, in which case his toon would be nuked in a matter
of minutes and the Webblies would be out one more character and one
more prepaid card. But the NPC went into his house and nothing
happened. Wei-Dong felt a flutter in his chest, and then another, a
few minutes later, when his email pinged.

\edialog{Tell me more}

It was unsigned, but he knew who it came from.

\tb

``You should go to Hong Kong,'' Lu said to Jie, holding her hand
tightly and staring into her eyes. ``You can do the show from there.
It's safer.''

Jie turned her head and blew out a stream of air. She squeezed his
hand. ``I know that you mean the best, Tank, but I won't do it and I
want you to stop talking about it. I'm a Webbly, just like you,
just like everyone here. Sure, I can broadcast from Hong Kong,
\emph{technically}, but what would I broadcast \emph{about}? I'm a
journalist, Tank. I need to be here to see what's going on, to
report on it. I can't do that from HK.''

``But it's not safe --''

She cut him off with a chopping gesture. ``Of course it's not safe!
I haven't been interested in safety since the day I went on the
air. You're not safe. My factory girls aren't safe. The Webblies on
the picket lines aren't safe. Why should I be safe?''

Lu bit down on the words: \emph{because I love you}. Secretly, he
was relieved. He didn't know what he'd do if Jie was in Hong Kong
and he was in Shenzhen. The last of her safe-houses, another flat
in a handshake building, was crowded with Webblies, forty boys all
studiously ignoring them, but he knew they were listening in. They
slept in shifts here, forty at a time, while eighty more went out
to work at friendly net-cafes, taking care never to send more than
two or three into any one cafe lest they draw attention to
themselves. Just the day before, two boys had been followed out of
a cafe by a couple of anonymous hard men who methodically kicked
the everloving crap out of them, right on the public street,
sending one to the hospital.

``You know it's only a matter of time until this place is blown,'' is
what Lu said. ``Someone will get careless and be followed home, or
one of the neighbors will start to talk about all the boys who trek
in and out of the flat at all hours, and then --''

``And then we'll move to another one,'' she said. ``I have been
renting and blowing off apartments for longer than you've been
killing trolls. So long as the advertising keeps on paying, I'll
keep on earning, and if I keep on earning, I can keep on renting.''

``How long will the advertisers pay for you to spend three hours
every night telling factory girls to fight back against their
bosses?''

A smile played over her lips, the secret, confident smile that
always melted his heart. ``Oh, Tank,'' she said. ``The advertisers
don't care what I talk about, so long as the factory girls are
listening, and they are \emph{listening}.''

She patted his hands. ``Now, I want you to go and find me a Webbly
to interview tonight, someone who can tell me how it's all going.
Any more protests?''

He shook his head. ``Not the noisy kind. Too many arrests.'' There
were over a hundred Webblies in jail, all over south China. ``But
you heard about Dongguan?''

She shook her head.

``The Webblies there have a new kind of demonstration. Instead of
making a lot of noise and shouting slogans, they all walk very
slowly around the bus-station, right in the middle of town, eating
ice cream.''

``Ice-cream?''

He grinned. ``Ice-cream. After the jingcha started to arrest anyone
who even \emph{looked} like he was going to protest, they started
posting these very public notices: 'show up at such-and-such a
place and buy an ice-cream.' Dozens, then hundreds of them, eating
ice-cream, grinning like maniacs, and the police were there,
staring at each other like mannequins, like,
\emph{Are we going to arrest these boys for eating ice-cream?} And
then someone got the bright idea of buying \emph{two} ice-creams
and giving one away to someone random passing by. It's the easiest
recruitment tool you can imagine!''

She laughed so long and hard that tears ran down her face. ``I love
you guys,'' she said. ``I can't \emph{wait} to talk about this on
tonight's show.''

``If they get arrested for eating ice-cream, they're going to switch
to getting together and \emph{smiling} at each other. Can you
imagine? \emph{Are we going to arrest these boys for smiling?}''

Her laughter broke through the invisible wall that separated them
from the lounging, off-shift Webblies, who demanded to know what
was so funny. Not all of them knew about the ice-cream -- they were
too busy patrolling the worlds, keeping the gold-farms from being
run with replacement workers -- but everyone agreed that it was
pure genius.

Soon they were downloading videos of the ice-cream eating, and then
another shift of boys trickled in and wanted to be let in on the
joke, and before they knew it, they were planning their own
ice-cream eating festival, and the general hilarity continued until
Jie and Lu slipped away to 'cast her show for the night, grabbing a
couple of hysterical Webblies to interview in between the calls
from the factory girls.

As Lu put his head down on his pillow and draped his arm around
Jie's narrow shoulders and put his face in her thick, fragrant
hair, he had a moment's peace and joy, real joy, knowing that they
couldn't possibly lose.

\tb

The strike was entering its second week when the empire struck
back. Connor had known about the strike for days, but he hadn't
taken action right away. At first he wasn't sure he \emph{wanted}
to take action. The parasites were keeping each other busy, after
all, and the strikers were doing a better job of shutting down the
gold markets than he ever had (much as it hurt to admit it). Plus
there was something \emph{fascinating} about the organization of
these characters -- they all came in through proxies, but by
watching their sleep schedules and sniffing their chatter he knew
that they were scattered all across the Pacific Rim and the
subcontinent. Sitting there in his god's eye, in Command Central,
he felt like he had a front-row seat to an amazing and savage flea
circus in which exotic, armored insects fought each other
endlessly, moving in precise regimented lines that spoke of
military discipline.

But he couldn't leave them to do this forever. He wasn't the only
one in Command Central who'd noticed that this was going on, and
the derivative markets were starting to pick up on the news,
yo-yo-ing so crazily that even the mainstream press had begun to
sniff around. Game-gold markets had been an exotic, silly-season
news-story a couple years back but these days the only people who
paid attention to them were players: high-volume traders
controlling huge fortunes that bought and sold game gold and its
many sub-species in a too-fast-to-follow blur. Until, of course,
word started to leak out about these Webblies and their pitched
battles, their ice-cream socials, their global span -- and now
corporate PR was calling Command Central five times a day, trying
to get a meeting so they could agree on what to tell the press.

So first thing on Monday morning, he gathered all of Command
Central, along with some of the cooler -- that is, less
neurotically paranoid -- lawyers and a couple of the senior PR
people in one of Coke's secure board-rooms for a long session with
the white-board.

``We should just exterminate these parasites,'' Bill said. ``You can
have the ten grand.'' Connor and Bill's bet had become a running
joke in Command Central, but Connor and Bill knew that it was
deadly serious. They were both part of the financial markets, and
they knew that a bet was just another kind of financial
transaction, and had to be honored.

Connor's smile was grim. He hadn't known whether the security chief
would come over to his side; he was such a pragmatist about these
things. Maybe they'd get something done after all. ``You know I'm
with you, but the question is, how high a price are we prepared to
pay to get rid of these people?''

``No price is too high,'' said Kaden, who prided himself on being the
most macho guy in Command Central -- the kind of guy who won't shut
up about his gun collection and his karate prowess. Kaden might
have been a black belt 20 years ago, but five years in Command
Central had made him lavishly, necklessly fat, and unable to go up
a flight of stairs without losing his breath.

Bill -- no lightweight himself -- craned his head around to stare
fishily at Kaden. He made a dismissive grunt and said, ``Oh,
really?''

Kaden -- called out in front of a room full of people -- colored,
dug in. ``Goddamned right. These crooks are in \emph{our} worlds. We
can outspend and outmanoeuvre them. We just have to have the balls
to do what it takes, instead of pussying out the way we always
do.''

Bill grunted again, a sound like a cement-mixer with indigestion.
``No price is too high?''

``Nope.''

``How about shutting down the game? Is that price too high?''

``Don't be stupid.''

``I don't think I'm the one being stupid. There's an upper limit on
how much this company can afford to spend on these jerks. If
removing them from the game costs us more than leaving them there,
we're just shooting ourselves in the head. So let's stop talking
about 'pussying out' and 'no cost is too high' and set some
parameters that we can turn into action, all right?''

``I just mean to say --''

Bill got out of his seat and turned all the way around to face
Kaden, fixing him with a withering stare. ``Go,'' he said. ``Just go.
You're a pretty good level designer, but I've seen better. And as a
person, you're a total waste. You've got nothing useful to add to
this discussion except for stupid slogans. We've heard the stupid
slogans. Go buff your paladin or something and let the grownups get
on with it.''

Silence descended on the meeting room. Connor, standing at the
front of the room, thought about telling Bill to back off, but the
thing was, he was right, Kaden was a total ass, and letting him
talk would just distract them all from getting the job done.

Kaden sat, mouth open and fishlike, for a moment, then looked
around for support. He found none. Bill made a condescending little
shooing gesture. Kaden's face went from red to purple.

``Just go,'' Connor said, and that broke the moment. Kaden slunk out
of the room like a whipped dog and they all turned back to Connor.

``OK,'' Connor said. ``Here's the thing: this has to be about solving
the problem, not posturing or thumping our chests. So let's stick
to the problem.'' He nodded at Bill.

Bill stood, turned around to face the audience. ``Here's what
doesn't work: IP addresses. They're coming in from proxies all over
the US, and they can find proxies faster than we can blacklist
them. Plus we've got tons of legit customers -- expats, mostly --
who live in China and around Asia and use these proxies to escape
their local network blocks. But even if we were willing to throw
those customers under a bus to stop the gold-farmers, we couldn't.

``Also doesn't work: payment tracing. These accounts are bought on
legit prepaid cards. The farmers are all paying customers, in other
words. We could shut off the prepaid cards and insist on credit
cards, but they'd just get prepaid credit cards. And every kid in
America and Canada and Europe who pays for her account with prepaid
cards from the corner store would be out of luck. That's a lot of
customers to throw under the bus -- and they'll just move on to one
of our competitors. Plus, those prepaid cards are \emph{gold}. Kids
buy them and half the time they don't use them -- they're free
money for us.

``Finally doesn't work: Behavioral profiling. Yes, these characters
have some stereotypical behaviors, like running the same grinding
tasks for hours, or engaging in these giant, epic battles. But this
is also characteristic of a huge number of normal players -- again,
these are people we don't want to throw under the bus.

``So what will work?''

Connor nodded. ``One thing I know we can do is get more mileage out
of the busts we make. Once we positively identify a farmer, we
should be able to take out his whole network by backtracking the
people he's chatted with, the ones he's partied with, his
guildies.''

Bill was shaking his head and made a rumbling sound. ``That's the
sound of your bus running over more legit players. These cats can
easily blow that strategy just by recruiting normal players for
their raids and fights. Hell, we \emph{designed it} that way.''

``The money'll be easier to trace,'' said Fairfax, interrupting them.
She looked from one to the other. ``I mean, these farmer types have
to dispose of their gold, and if we take it back from any player
that bought it --''

``They'd go crazy,'' Connor said.

``It's against the terms of service,'' she said. ``They know they're
cheating. It'd be justice. On what basis could they complain? They
agree to the terms every time they log on.''

Connor sighed. The terms of service were 18 screens long and
required a law degree to understand. They prohibited every
conceivable in-game activity, up to and including having fun.
Technically, every player violated the terms every day, which meant
that if they wanted to, they could kick off anyone at any time (of
course, this too was allowed in the terms: ``Coca-Cola Games, Ltd
reserves the right to terminate your account at any time, for any
reason''). ``The problem is that too many players think that buying
gold is all right. We sell gold, after all, on our own exchanges,
all the time. If you nuked every account involved in a gold-farming
buy, we'd depopulate the world by something like 80 percent. We
can't afford it.''

``80 percent? No way --''

``Look,'' he said. ``I've been going after the farmers now for months.
It's the first time we've ever tried to be systematic about them,
instead of just slapping them down when the activity gets a little
too intense. I can show you the numbers if you want, show you how I
worked this out, but for now, let's just say that I'm the expert on
this subject and I'm not making this up.''

Fairfax looked chastened. ``Fine,'' she said. ``So you want to go
after the known associates of the farmers we bust, even though we
can all see how easy it will be to defeat.''

Connor shrugged. ``OK, sure. They'll get around it, eventually. But
we'll have some time to get on them.''

Bill cleared his throat, shook his head again. ``You have any idea
how much transactional data we're going to have to store to keep a
record of every person every player has ever talked to or fought
with? And then someone will have to go over all those transactions,
one by one, every time we bust a player, to make sure we're getting
real confederates and not innocent by-standers. Where are all those
people going to come from?''

Someone in the audience -- it was Baird, the lawyer Connor hated
the least -- said, ``What about the Mechanical Turks?''

Connor and Bill stared at each other, mouths open. The lawyer
looked slightly nervous. ``I mean --''

``Of \emph{course},'' Connor said. ``And we could do it for free. Just
let the Turks keep any gold from the accounts of busted players.''

One of the other economists was young Palmer, and he reminded
Connor of himself a few years back. Connor hated him. His eager
hand shot up. ``I thought the point was to keep all that gold out of
the market,'' he said. ``How can we control the monetary supply if
these goombas are allowed to flood the market with cheap money?''

Connor waved his hands. ``Yes, theoretically these cats are outside
our monetary planning, but even going flat out, they just don't
move the market that much. And if they do, we can restrict the
supply at our side, or adjust the basic in-game costs up or down\ldots{}
And it's not as if the Turks will turn around and spend the gold
right away, or dump it through one of the official exchanges,
especially if we keep the exchange rate low through that period.''

Young Palmer opened his mouth again and Connor stopped him. ``Look,
this is all model-able. Let's stipulate that we can take care of
the monetary supply and move on.'' In the back of his mind, he knew
that he was dismissing a potentially explosive issue with a lot
more cavalier abandon than was really warranted, but the fact was
this was his chance to take care of the gold farmers once and for
all, with the full weight of the company behind him, and if that
screwed up the economy a little, well, they'd fix it later. They
controlled the economy, after all.

Later, at his desk in Command Central, he looked up from his feeds
and saw a room full of the smartest, toughest people in the company
-- in the world -- bent to the same task, ferreting out the
parasites that he'd been chasing for months. And if he himself had
once been a kind of gold-farmer, a speculator of in-game assets,
well, so what? He graduated to something better.

The fact was, there wasn't room on earth for a couple million
gold-farmers to turn into high-paid video-game executives. The fact
was, if you had to slice the pie into enough pieces to give one to
everyone, you'd end up slicing them so thin you could see through
them. ``When 30,000 people share an apple, no one benefits --
especially not the apple.'' It was a quote one of his economics
profs had kept written in the corner of his white-board, and any
time a student started droning on about compassion for the poor,
the old prof would just tap the board and say, ``Are you willing to
share your lunch with 30,000 people?''

And hell, there were at least three million gold-farmers in the
world. Let them get their own goddamned apples.

\tb

``Sea-level'' is a term that refers to the average level of all the
world's oceans. Think of the world as a giant bed-pan, filled
halfway with water. You can blow on one part of the surface and
induce some tiny waves whose crests are higher than the rest of the
water. You can tip the bed-pan from side to side and cause the
water to slosh around, making it higher at one end than another.
But overall, there's a single level to that water, a surface height
that you can easily discern.

Same with the oceans. Though the tides may drag the water from one
edge of the sea to the other -- and really, there's only one sea, a
single, continuous jigsaw-puzzle-piece-shaped body of water that
wraps around all the continents -- though the storms may blow up
waves here and there, in the end, there's only so much water in the
ocean, and it more or less comes to an easily agreed-upon height.
Sea level.

Same with money. There's only so much value in the world: only so
much stuff to buy. If you got all the money in the world, you could
exchange it for all the stuff on earth (at least all the stuff
there is for sale). It doesn't matter, really, whether the money is
in dollars or gold pieces or mushrooms or ringgits or euros or yen.
Add it all together and what you've got is the ocean. What you've
got is sea level.

So what happens if someone just prints a lot more money? What
happens if you just double the amount of money in circulation? Will
the monetary seas rise, drowning the land?

No.

Printing more money doesn't make more money. Printing more money is
like measuring the ocean in liters instead of gallons. Converting
343 quintillion gallons of ocean into 1.6 sextillion liters (give
or take) doesn't give you any more water. Gallons and liters are
measurements of water, not water itself.

And dollars are measures of value, not value itself. If you double
the amount of currency in circulation, you double the price of
everything on Earth. The amount of stuff is fixed, the amount of
currency isn't. That's called inflation, and it can be savage.

Say you're a dictator of a tin-pot republic. For decades, you've
lined your pockets at the peoples' expenses, taxing the crap out of
everyone and embezzling it into your secret off-shore bank-account
in Honduras. Eventually, you've moved so much wealth out of the
country that people are ready to eat their shoes. They start to get
angry. At you.

Normally, you'd just have your soldiers go and make examples of a
few hundred dissidents and leave their grisly, carved up remains by
the roadside in shallow graves as a means of informing your loyal
subjects of what they can expect if they keep this kind of thing
up.

But soldiers -- even the real retarded sadists -- don't work for
free. They want paying. And if you've taken all the money out of
the country and put it in your bank account, you need something to
pay them with.

No problem. You're a dictator. Just call up the treasury department
and order them to print up a couple trillion ducats or gold
certificates or wahoonies or whatever you call your money, and you
start paying the troops. It works -- for a while. The troops take
their dough into town and use it to buy drinks and snazzy clothes
and big meals. They send it home to their families, who use it to
buy lumber and tile and steel and cement to improve their houses,
or to buy farm implements and pay the hired hands to help them
bring up the next crop.

But as the amount of money in circulation grows, it gradually
becomes worth less. The bar raises its drink prices because the
landlord has raised the rent. The landlord has raised the rent
because the cost of feeding his family has gone up, because the
farmer isn't willing to sell his crops for the old prices, because
she's paying double for diesel for the tractor and triple for
water.

And then the soldiers show up at the dictator's palace and explain,
pointedly, with bayonets (if necessary), why their old wages are no
longer sufficient.

No problem. Just call up the treasury and order up another trillion
wahoonies. And watch it all happen again.

This is called inflation, and it's the cheap sugar high of
governments. Like a cramming student sucking down energy beverages,
a government can only print money for so long before they have to
pay the price. It's not pretty, either. Families that carefully
saved all their lives for their retirement suddenly find their tidy
nest-egg is insufficient to cover the price of a dinner out. Every
penny of savings is wiped out in the blink of an eye, and suddenly
you need a lot more soldiers on the job to keep your loyal subject
from gutting you like a fish and hanging you upside down from your
own palace's tallest chimney.

If you're a \emph{very} cheeky dictator, you'll go one further:
take all the savings in the banks that are denominated in real
money -- euros or dollars or yen -- and convert them into wahoonies
at today's exchange rate. Use all that real money to pay the army
for a day or two more, but you'd better save enough to pay for
airfare to some place very, very far away.

If you think inflation is scary, try \emph{deflation}. As people
get poorer -- as less and less money is in circulation -- the value
of money goes up. This is good news for savers: the wahoonie you
banked last year is worth twice as much this year. But it's bad
news for everyone else: only an idiot borrows money in deflationary
times, since the wahoonie you borrow today will be worth twice as
much next year when you repay it. Deflation is uneven, too: the
cost of food may crash because of some amazing new fertilizer,
which means you can buy twice as much cassava per wahoonie. But
this means that farmers are only earning half as much, and won't
pay as much for cable TV. The cable company hasn't had \emph{its}
costs go down, though, so the reduced payment means less profits.
Businesses start to fail, which means more people have less money,
which drives prices down and down and down. Before long, no one can
afford to make or buy \emph{anything}.

In other words, the amount of money in circulation is a big deal.
Theoretically, this amount is watched carefully by clever, serious
economists. In practice, all the world's money is in one big
swirling, whirling pool. Dollars and ducats and wahoonies and
euros, blended together willy nilly, and when one government goes
to the press and starts to churn out bales of bank-notes, everyone
gets the sugar high. And when things crash, and peoples' savings go
up in smoke, the deflationary death-spiral kicks in, and prices
sink, and more companies fail -- and governments go back to the
printing press.

So in practice, this big engine that determines how much food is
grown, whether you'll have to sell your kidneys to feed your
family, whether the factory down the road will make Zeppelins,
whether the restaurant on the corner can afford the coffee beans,
all this important stuff has \emph{no one in charge of it}. It is a
runaway train, the driver dead at the switch, the passengers
clinging on for dear life as their possessions go flying off the
freight-cars and out the windows, and each curve in the tracks
threatens to take it off the rails altogether.

There is a small number of people in the back of the train who
fiercely argue about when it will go off the rails, and whether the
driver is really dead, and whether the train can be slowed down by
everyone just calming down and acting as though everything was all
right. These people are the economists, and some of the first-class
passengers pay them very well for their predictions about whether
the train is doing all right and which side of the car they should
lean into to prevent their hats from falling off on the next
corner.

Everyone else ignores them.

\tb

``Hey, Connor!'' his broker said, his voice tight and nervous, his
cheer transparently false.

``What's wrong?''

``Cut to the chase, huh, man?'' Ira's voice was so tight it twanged.
``You're such a straight-shooter. It's why you're my favorite
customer.''

``What's \emph{wrong}, Ira?'' Command Central roared around him, a
buzz of shouts and conversations and profanity.

``So, you remember those bonds we took you into?''

Connor's chest tightened. He forced himself to stay calm. ``I
remember them.''

``Well, they were paying out really well -- you saw the statements.
Eight percent last month --''

``I saw the statements.''

``Well.''

``Ira,'' Connor said. ``Stop being such a goddamned salesman and tell
me what the hell is going on, or I'm going to hang up this phone
and call your boss.''

``Connor,'' Ira said, his voice hurt. ``Look, we're buddies --''

``We're not buddies. You're a salesman. I'm your customer. I'm
hanging up now.''

``Wait! Come on, wait! OK, here it is. There's a little\ldots{}liquidity
crisis in the underlying assets.''

Connor translated the broker-speak into English. ``They don't have
any money.''

``They don't have any money \emph{this month},'' he said. ``Look, the
coupon on this contract has been through the roof for more than a
year. Ultimately, it can't lose, either, because of how we've
packaged it with a credit-default swap. But right now, this
instant, they're having a tough one-time-only squeeze.''

After the first month's interest had paid out, Connor had
liquidated several other holdings and bought more of the bonds,
bought big. So big that the brokerage had FedExxed him a bottle of
Champagne. He'd lost track of how much he had tied up with Ira's
``fully hedged'' scheme, but he knew it was at least \$150,000. That
had seemed like such a good bet --

``What kind of one-time-only squeeze?''

``Nintendo,'' the broker said. ``They've loosened up their monetary
policy lately. The star-farmers in Mushroom Kingdom are bringing up
huge crops, and so Mario coins are dropping off in cost. But the
word is that this is just a temporary gambit because they've had
such a huge rush of new players who can't afford to keep up with
the old-timers, so they're trying to lower commodity prices to keep
those players onboard. But once those players catch up and start
demanding more power-ups, the prices'll bounce back.''

It sounded plausible to Connor. After all, they'd done similar
things in their own games. The experienced players howled as
inflation lowered the value of their savings, but a player who'd
been honing his toon for two years wasn't going to quit over
something like that. The new blood was vital to keeping the game on
track, replacements for the players who got old, or bored, or poor
-- any of the reasons behind the churn that caused some players to
resign every month.

Churn was one of his biggest economic problems. You could minimize
it in lots of sneaky ways: email a former player to tell him that
you were about to delete the toon he hadn't touched in a year and
there was a one-in-three chance that he'd sign up to play again,
rather than doom this forgotten avatar to the bit-bucket. But
ultimately some players would leave, and the only thing for it was
to bring new players in.

The broker was still droning on. `` -- so really, we expect a huge
surge in four to eight weeks, more than enough to make up for the
drop. And if things go bad enough, there's always the prince and
his bets --''

``What's the bottom line?'' Connor said.

``Bottom line,'' Ira said. ``Bottom line is that there's no coupon
this month. The underlying bonds are selling at a 20 percent
discount off face value.'' He swallowed audibly. ``That's sixty
percent off what you paid for them in this package. But if things
get bad enough, you'll recoup with the insurance --''

Connor tried to keep listening, but his breath was coming in tight
little gasps. Sixty percent! He'd just had more than half his net
worth vanish into thin air. The worst part was that he had other
obligations -- a mortgage, payments due on some of the little
startups he'd bought into, money to pay the contractors who were
fixing up the holiday cottage he'd bought as a rental property in
Bermuda. Without the cash he'd been expecting from these
investments, he could lose it all.

Oblivious, the broker kept talking. ``-- which is why our
recommendation today is to buy. Double down.''

``Excuse me?'' Connor said, loud enough that the people closest to
him in Command Central looked up from their feeds to stare at him.
He scowled at them until they looked away. ``Did you say
\emph{buy}?''

``There's never been a better time,'' the salesman said. Connor
pictured him in his cubicle, a short-haired middle-aged guy in an
old suit that had once been tailor made, a collection of bad habits
glued to a phone, chewed-down fingernails and twitching knees, a
trashcan beside him filled with empty coffee cups, screens
everywhere around him flickering like old silent films. ``Look, any
idiot can buy when the market is up, but how much higher does the
market go when it's already at the top? The only way to make real
money, big money, is to bet against the herd. When everyone else is
dumping their holdings, that's the time to buy, when it's all down
in the basement.''

Connor knew that this made sense. It was the basis of his Prikkel
equations, it was the basis of all the fortunes he'd amassed to
date. Buying stuff that everyone else wanted was a safe,
uninteresting bet that paid practically nothing. Buying into the
things that everyone else was too dumb to want -- that was how you
got \emph{rich}.

``Ira,'' Connor said, ``I hear what you're saying, but you've seen my
accounts. I can't afford to double down. I'm maxxed out.''

``Connor, pal,'' he said, and Connor heard the smile in his voice and
he smiled himself, a reflex he couldn't tamp down even if he'd
wanted to. ``You're not tapped out. You've got a liquidity problem.
You have a relationship with this brokerage. That's worth
something. Hell, that's worth \emph{everything}. We got you into
this problem, and we'll get you out of it. If you need some credit,
that's absolutely do-able. Let me talk to our credit department and
get back to you. I'm sure we can make it all work.''

Connor was overcome by an eerie, schizophrenic sensation. It was as
if his brain had split into two pieces. One piece was shaking its
head vigorously, saying
\emph{Oh no, you're out of your mind, there's no way I'm putting more money into this thing. No, no, no, Christ, no!}

But there was another part of his mind that was saying
\emph{He's right, the best time to buy is at the bottom of the market. These things have been paying out big-time. The explanation makes sense. Just think of how you'll feel when you don't buy in and the security bounces back, all that money you'll miss out on. Think of how you'll feel if you clean up and can buy a bigger house, another income property, a new car. Think of how all these jerks will drool with envy when you make a killing}.

And his mouth opened and the words that came out of it were, ``All
right, that sounds great. I'll take as much as you can sell me on
margin.'' On margin: that was when you bought securities with
borrowed money, because you were sure that the bets would pay off
before you had to pay the money back. It was a dangerous game: if
the margin call came before the bets paid off -- or if they never
paid off -- it could wipe you out.

But these were not bets, really. The way that the brokerage had
packaged them, they were fully hedged. The worse the underlying
bonds did, the more the bets against them from the Prince paid off.
There might be some minor monthly variations, but when it was all
said and done, he just couldn't lose.

``Buy,'' he said. ``Buy, buy, buy.''

Through the rest of the day, he was so preoccupied with worry over
his precarious position that he didn't even notice when every other
executive in Command Central had a nearly identical conversation
with \emph{their} brokers.

\tb

Wei-Dong's mother was the perfect reality check when it came to
games and the Webblies. He'd never appreciated it before he left
home, but once he'd gone to work as a Turk, his mom had tried to
re-establish contact by clipping stories about games and gamers and
emailing them to him. It was always stuff he'd absorbed through his
pores months before, being reported to outsiders with big screaming
OMG WTF headlines that made him snicker.

But he came to appreciate his mom's clippings as a glimpse into a
parallel universe of non-gamers, people who just didn't get how
important all this had become. The best ones were from the
financial press, trying to explain to weirdos who invested in
game-gold exactly what they had bought.

And those clippings were even more important now that he'd come to
China. Mom still thought he was in Alaska, and he made sure to
pepper his occasional emails to her with references to the long
nights and short days, the wilderness, the people -- a lot of it
cut-and-pasted verbatim from the tweets of actual Alaska tourists.

Today, three weeks into the strike, she sent him this:

A UNION FOR VIDEO-GAMERS?

They call themselves the Industrial Workers of the World Wide Web,
and they claim that there are over 100,000 of them today, up from
20,000 just a few weeks ago. They spend their days and nights in
multiplayer video-games, toiling to extract wealth from the
game-engines, violating the game companies' exclusive monopoly over
game-value. The crops these ``gold farmers'' raise are sold on to
rich players in America, Europe and the rest of the developed
world, and the companies that control the games say that this has
the potential to disrupt the carefully balanced internal economies
--

Wei-Dong spacebarred through the article, skimming down. It was
interesting to see one of his mother's feeds talking about
Webblies, but they were so\ldots{} \emph{old school} about it.
Explaining everything.

Then he stopped, scrolled back up.

\ldots{}mysterious, influential pirate radio host who calls herself
Jiandi, whose audience is rumored to be in the tens of millions,
creating a rare and improbable alliance between traditional factory
workers and the gamers. This phenomenon is reportedly repeating
itself around the Pacific Rim, in Indonesia, Malaysia, Cambodia and
Vietnam, though it's unclear whether the ``IWWWW'' chapters in these
countries are mere copycats or whether they're formally affiliated,
under a single command.

Wei-Dong looked up from his screen at the mattress where Lu and Jie
had collapsed after staggering in from the latest broadcast, Jie's
face so much younger in repose. Could she really be this famous DJ
that Mom -- \emph{Mom}, all the way across the world in Los Angeles
-- was reading about?

There was more, screens and screens more, but what really caught
his attention was the mention of the ``market turmoil'' that was
sending bond and stock prices skittering up and down. He didn't
understand that stuff very well -- every time someone had attempted
to explain it to him, his eyes had glazed over -- but it was clear
that the things that they were doing here were having an effect, a
\emph{massive} effect, all over the world.

He almost laughed aloud, but caught himself. Matthew was sleeping
all of six inches from where he sat, and he'd run the
picket-skirmishes for 22 hours straight before keeling over.
Wei-Dong had fought too, but he'd been mostly tasked to recruiting
more Turks to his little list of friendly operatives, a much less
intense kind of game. Still, he should be sleeping, not pecking at
his laptop. In six hours, he'd be back on shift, with only a bowl
of congee and a plate of dumplings to start the day.

He folded down his laptop's lid and stretched his arms over his
head, noting as he did the rank smell of his armpits. The single
shower -- ringed with a scary-looking electrical heater that warmed
up the water as it passed through the showerhead -- wasn't
sufficient for all the Webblies who slept in the flat, and he'd
skipped bathing for two days in a row. He wasn't the only one. The
apartment smelled like the locker rooms at school or like the
homeless shelter near Santee Alley that he used to pass when he
went out for groceries.

He heard a little chirp from somewhere nearby, the cricket-soft
buzz of a mobile phone ringing. He watched as Jie sleepily pawed at
the little purse by her pillow, its strap already looped around her
arm, and extracted a phone, blearily answered it: ``Wei?''

Her sleepy eyes sprang open with such force that he actually heard
her eyelids crinkling. Her bloodshot eyes showed her whole iris,
and she leapt up, shouting in slangy Chinese that came so fast he
couldn't understand her at first.

But then he caught it: ``Police! Outside! GO GO GO!''

There were 58 Webblies sleeping in the safe-house, and in an
instant they all shot out of their blankets, most of them already
dressed, and jammed their toes into their shoes and grabbed little
shoulder-bags containing their data and personal possessions and
crowded into the doorway. They worked in near-silence, the only
sound urgent whispers and curses as they stepped on each others'
shoes. Some made for the window, leaping out to grab the balcony of
the opposite handshake building, and now there was shouting from
the street as the oncoming police spotted them.

He joined the crush of bodies, pushing grimly into the narrow
hallway, then sprinting in the opposite direction to most of the
Webblies, for he had seen Jie running that way, holding tight to
Lu's hand, and Jie seemed to have the survival instincts of a city
rat. If she was running that way, he'd run that way too.

But she'd gotten ahead of him, and when he skidded around the
corner and found himself looking at a short length of corridor
ending with an unmarked door, neither she nor Lu were anywhere to
be seen. He paused for a second, then the unmistakable sound of a
gunshot and a rising wave of panicked screams drove him forward,
hurtling for the unmarked door, hand stretched out to turn the knob
--

-- which was locked!

He bounced off the door, stunned, and went on his ass, and shouted
a single, panicked ``Shit!'' as he cracked his head on the dirty tile
floor. As he struggled back into a seated position, he saw the door
crack open. Jie's bloodshot eye peeked out at him, and she swore in
imaginative, slangy Chinese. ``Gweilo,'' she hissed, ``quickly!''

He got to his feet quickly and reached the door in two quick steps.
Her long fingernails dug into his arm as she dragged him inside the
dimly lit space, which he saw now was a kind of supply closet that
someone had converted into sleeping quarters, with a rolled up bed
in one corner and a corner of one shelf cleared of cleaning
products and disinfectant and piled with a meager stack of clothes
and collection of toiletries and a small vanity mirror.

``The matron,'' Jie said, whispering so quietly that Wei-Dong could
barely hear her. ``She gets to live in here for free. She and I have
an arrangement.'' Lu was on his hands and knees behind her, silently
rearranging the crowded space, working with a small LED flashlight
clamped between his teeth. He was breathing heavily, his skinny
arms trembling as he hefted the giant bottles of bleach and
strained to set them down without making a sound.

``Can I help?'' Wei-Dong whispered.

Jie rolled her eyes. ``Does it look like there's room to help?'' she
said. She was so close to him that he could see her individual
eyelashes, the downy hair on her earlobes. If he took a deep
breath, he'd probably crush her.

He shook his head minutely. ``Sorry.''

Lu made a satisfied grunt and detached the entire bottom shelf from
its bracket. Wei-Dong could see that he'd uncovered an access-hatch
set into the wall, and it showered dust and paint-chips onto the
floor in a cockroach-wing patter as he worked it loose. He passed
it back and Jie tried to grab it, but there wasn't room to maneuver
it in the small space.

From the other side of the door, he heard the tromp, tromp, tromp
of heavy boots, heard the thudding and pounding on the doors, the
muffled and frightened conversations of people roused from their
beds in the middle of the night.

With a low, frustrated, frightened sound Jie grabbed the hatch
cover and moved it out of the way, bashing him so hard in the nose
that he had to stuff his fist in his mouth to stop from crying out.
She gave him a contemptuous look and shoved the hatch into his
hands. It was about 30 inches square, filthy, awkward, made from
age-softened plywood.

Lu had passed through the hatch already, and now Jie was following,
her bare legs flashing in the half-light of the room, and then
Wei-Dong was alone, and the tromp of the boots was louder. Someone
was scuffling in the hallway, a man, shouting in outrage; a woman,
screaming in terror; a baby, howling.

Wei-Dong knelt down and peered into the tiny opening. It was pitch
dark in there. He carefully leaned the cover up against the wall
beside the opening and then climbed in. The floor on the other side
was unfinished concrete, gritty and dusty. He couldn't see a thing
as he pulled himself forward on his elbows, commando-style, his
breath rasping in his ears. He inched forward, feeling cautiously
ahead for obstructions and then discovered that he was holding
something soft and pliant and warm. Jie's breast.

She hissed like a snake and swiped his hand away with sudden
violence. He began to stammer an apology, but she hissed again:
``Shhh!''

He bit back the words.

``Close up the grating,'' she said. He cautiously began to turn
around. The little space was a mere meter high and he repeatedly
smashed his head into the ceiling, which had several unforgiving
metal pipes running along it that bristled with vicious joints and
tees. And he kicked both Jie and Lu several times.

But he eventually found himself with his head and arms outside the
hatch, and he desperately fitted his fingers to the inside of the
grill and inched it into place. It was nearly impossible to
manoeuvre it into the tight space, but he managed, his fingers
white -- and all the while, the sounds from the corridor grew
louder and louder.

``Got it,'' he gasped and slithered away. There were voices from just
outside the door now, deep, impatient male voices and an angry,
shrill woman's voice telling them that this was the stupid broom
closet and to stop being so stupid. Someone shook the doorknob and
then put a shoulder into the door, which shuddered.

Wei-Dong bit his tongue to hold in the squeak and pushed back even
more, the fear on him know, a live thing in his chest. Jie and Lu
pushed at him as he collided with them, but he barely felt it. All
he felt was the fear, fear of the armed men on the other side of
the door, about to come through and see the closet and the obvious
gap on the bottom shelf where things had been shoved aside.
Wei-Dong was suddenly and painfully aware of how far he was from
home, an illegal immigrant with no rights in a country where no one
else had rights, either. He would have cried if he hadn't been
scared to make a sound.

``Come on,'' Jie whispered, a sound barely audible as another crash
rocked the door. Someone had a key in the lock now, jiggling it.
She clicked a tiny red LED to life and it showed him the shape of
the space: a long, low plumbing maintenance area. The pipes above
them gurgled and whooshed softly as the water sluiced through
them.

Lu was beside him, Jie ahead of them, and she was arm-crawling to
the opposite side of the area. He followed as quickly as he could,
ears straining for any sound from behind him.

Jie swore under her breath.

``What?'' Lu said.

``I can't find the other grating,'' she said. ``I thought it was right
here, but --''

Wei-Dong understood now. The maintenance area occupied a dead-space
between their building and the one behind it, and somewhere around
here, there was a grating like the one they'd come through, a
little wormhole into another level of the game. Jie's survival
instincts were incredibly sharp, that much had been obvious, so he
wasn't altogether surprised to discover that she had a back door
prepared.

He peered into the darkness, his whole body slicked with sweat and
grimed with the ancient dust covering the floor.

``The last time, there was a light on the other side. It was easy to
find,'' she said, her voice near panic. He heard the unmistakable
sound of the police entering the utility closet behind them, then
voices.

``We need to search the whole wall,'' Lu said. ``Split up.''

So Wei-Dong found himself squirming over Jie's bare calves, tearing
his jeans on one of the low pipes as he did so. He patted the wall
blindly, feeling around. Away from the small red light, it was
pitch black, disorienting, frightening. Nearby, he heard the sounds
of Jie and Lu searching too.

And then, he found it, his baby fingertip slipping into a grating
hole, then he patted around it, felt its full extent. ``Here, here!''
he whispered loudly, and the other two began to struggle his way.
He jiggled the grating, trying to find the trick that would make it
come away, but it appeared to be screwed in. Increasingly
desperate, he shook the grating, causing a rain of dust and dried
paint to fall on his hands. He was gripping the metal so hard he
could feel it cutting into one finger, a trickle of blood turning
into mud as it mixed with the dirt.

``Light,'' he said. ``Can't see anything.''

A hand patted the length of his leg, feeling its way up his body,
to his arm, then pressed the little light into his hand. Jie's
hand, slim and girlish. He clicked the red light to life and peered
intently at the grating. It wasn't screwed in, but it needed to be
pushed slightly forward before it would lift out. He stuck the
light's handle between his teeth and \emph{pushed} and
\emph{lifted} and the grating popped free.

Just as it did, a long cone of light sliced through the crawlspace,
and then a martial voice demanded ``Halt!'' The light bathed him,
making him squint, and Jie thumped him in the thigh and said,
``GO!''

He went, commando crawling again, Jie's slim hands pushing him to
hurry him along. He emerged into a tiled space, dirty and dark, the
floor wet and slimy. He stood up cautiously, worried about hitting
his head again, then stooped to help Jie through. There were more
shouts coming from the other side of the grating now, and the light
spilled out of it and painted the greenish scum on the old, cracked
grey tile floor. ``Halt!'' again, and ``Halt'' once more, as Jie
finished wriggling through and he bent to grab Lu, peering into the
now-brilliantly-lit crawlspace. Lu had been searching for the
grating at the other end of the crawlspace and he was going as fast
as he could, his face a mask of determination and fear, lips
skinned back from his teeth, blood flowing freely from a scalp
wound.

``Halt!'' again, and Lu put on a burst of speed, and there was the
unmistakable sound of a gun being cocked. Lu's eyes grew wide and
he flung his arms out before him and dug his hands into the ground
and pulled himself along, scrambling with his toes.

``Come on,'' Wei-Dong begged, practically in tears. ``Come on, Lu!''

A gunshot, that flat sound he'd heard in the distance when he was
living in downtown LA, but with an alarming set of whining
aftertones as the bullet bounced from one pipe to another. Water
began to gush onto the floor, and Lu was still too far away.
Wei-Dong went down on his belly and crawled halfway into the space,
holding his arms out: ``Come on, come on,'' crooning it now, not sure
if he was speaking English or Chinese.

And Lu came, and: ``HALT!'' and another gunshot, then two more, and
the water was everywhere, and the whining ricochets were everywhere
and then --

Lu \emph{screamed}, a sound like nothing Wei-Dong had ever heard.
The closest he'd heard was the wail of a cat that he'd once seen
hit by a car in front of his house, a cat that had lain in the
street with its spine broken for an eternity, screaming almost like
a human, a wail that made his skin prickle from his ankles to his
earlobes. Then, Lu \emph{stopped}. Lay stock still. Wei-Dong bit
his tongue so hard he felt blood fill his mouth. Lu's eyes
narrowed, the pupils contracting. He opened his mouth as though he
had just had the most profound insight of his life, and then blood
sloshed out of his mouth, over his lips, and down his chin.

``Lu!'' Wei-Dong called, and was torn between the impulse to go
forward and get him and the impulse to back out and run as fast as
he could, all the way to California if he could --

And then, ``STAY WHERE YOU ARE,'' in that barking, brutal Chinese,
and the gun was cocked again. He smelled the blood from his own
mouth and from Lu, and Lu slumped forward. Then a gunpowder smell.
Then --

-- another shot, which whined and bounced with a deadly sound that
left his ears ringing.

``STAY WHERE YOU ARE,'' the voice said, and Wei-Dong scrambled
backwards as fast as he could.

Jie yanked him to his feet, her face grimed with dust and streaked
with tears. ``Lu?'' she said.

He shook his head, all his Chinese gone for a moment, no words at
all available to him.

Then Jie did an extraordinary thing. She closed her eyes, drew in a
deep breath, drew it in and in, squeezed her fists and her arms and
her neck muscles so that they all stood out, corded and taut.

And then she blew it all out, unclenched her fists, relaxed her
neck, and opened her eyes.

``Let's go,'' she said, and, with a single smooth motion, turned to
the door behind her and shot the bolt, turned the knob and opened
it into another apartment-building corridor, smelling of cooking
spices and ancient, ground-in body-odor and mold. The dim light
from the hallway felt bright compared to the twilight he'd been in
since diving through the bolt-hole, and he saw that he was in a
disused communal shower, the walls green with old mold and slime.

Jie dug a pair of strappy sandals out of her purse and calmly and
efficiently slipped them on. She produced two sealed packets of
wet-wipes, handed one to Wei-Dong and used the other's contents to
wipe her face, her hands, her bare legs, working with brisk
strokes. Though Wei-Dong's heart was hammering and the adrenalin
was surging through his body, he forced himself to do the same,
shoving the dirty wipes in his pocket until there were no more.
There were more shouts from the grating behind them, and distant
sounds from the street below, and Wei-Dong knew it was hopeless,
knew that they were cornered.

But if Jie was going to march on, he would too. Lu was behind him,
with the coppery blood smell, the bonfire smell of the gunpowder.
Ahead of him was China, all of China, the country he'd dreamed of
for years, not a dream anymore, but a brutal reality.

Jie began to walk briskly, her arm waving back and forth like a
metronome as she crossed the length of the building and opened the
door to the stairway without breaking stride. Wei-Dong struggled to
keep up. They pelted down three flights of stairs, the grimy,
barred windows allowing only a grey wash of light. It was dawn
outside.

Only one flight remained, and Jie pulled up abruptly, wheeled on
her heel and looked him in the eye. Her eyes were limned with red,
but her face was composed. ``Why do you have to be white?'' she said.
``You stand out so much. Walk five paces behind me, three paces to
the side, and if they catch you, I won't stop.''

He swallowed. Tried to swallow. His mouth was too dry. Lu was dead
upstairs. The police were outside the door -- he heard calls,
radio-chatter, engines, sirens, shouts -- and they were murderous.

He wanted to say,
\emph{Wait, don't, don't open the door, let's hide here.} But he
didn't say it. They were doomed in here. The police knew which
building they'd entered. The longer they waited, the sooner it
would be before they sealed the exits and searched every corner and
nook.

``Understood,'' he managed, and made his face into a smooth mask.

One more flight.

Jie cracked the door and the dawn light was rosy on her face. She
put her eye to the crack for a moment, then opened it a little
wider and slipped out. Wei-Dong counted to three, slowly, making
his breath as slow as the count, then went out the door himself.

Chaos.

The street was a little wider than most of the lanes near the
handshake buildings, a main road that was just big enough to admit
a car. A car idled at one end of it, two policemen outside it.
Three more police were just entering the building he'd come out of,
using a glass door a few yards away. The blue police-car
bubble-lights painted the walls around them with repeating patterns
of blue and black. Somewhere nearby, shouting. Lots of shouting.
Boyish yells of terror and agony, the thud of clubs, screaming from
the balconies, no words, just the wordless slaughterhouse
soundtrack of dozens of Webblies being beaten. Beaten, while Lu lay
dead or dying in the crawlspace.

He turned left, the direction that Jie had gone, just in time to
see her disappearing down a narrow laneway, turning sideways to
pass into it. He wasn't sure how he could follow her injunction to
stay to one side of her in a space that narrow, but he decided he
didn't care. He wasn't going to try to make his own way out of the
labyrinth of Cantonese-town.

As soon as he entered the alley, though, he regretted it. A
policeman who happened to look down the alley would see him
instantly and he'd be a sitting target, impossible to miss. He
looked over his shoulder so much as he inched along that he tripped
and nearly went over, only stopping himself from falling to the
wet, stinking concrete between the buildings by digging his hands
into the walls on either side of him. Ahead of him, Jie cleared the
other end of the alley and cut right. He hurried to catch her.

Just as he cleared the alley-mouth himself, he heard three more
gunshots, then a barrage of shots, so many he couldn't count them.
He froze, but the sounds had been further away, back where the
Webblies had emerged from their safe house. It could only mean one
thing. He bit his cheek and swallowed the sick feeling rising in
his throat and scrambled to keep up with Jie.

Jie walked quickly -- too quickly; he almost lost her more than
once. But eventually she turned into a metro station and he
followed her down. He'd used the ticket-buying machines before --
they were labelled in Chinese and English -- and he bought a fare
to take him to the end of the line, feeding in some RMB notes from
his wallet. The machine dropped a plastic coin like a poker chip
into its hopper and he took it and rubbed it on the turnstile's
contact-point and clattered down the stairs with the sparse crowd
of workers headed for early shifts.

He positioned himself by one of the doors and reached into his
pocket for a worn tourist guide to Shenzhen, taken from the free
stack at the info-booth at the train-station. It was perfect
camouflage, a kind of invisibility. There was always a gweilo or
two puzzling over a tourist map on the metro, being studiously
ignored by the flocks of perfectly turned-out factory girls who
avoided them as probable perverts and definite sources of
embarrassment.

Jie got off four stops later, and he jumped off at the last minute.
As he did, he caught a glimpse of his reflection in the glass of
the car-doors and saw that one side of his hair was matted with
dried blood which had also run down his neck and dried there. He
cursed himself for his smugness. Invisible! He was probably the
most memorable thing the metro riders saw all that day, a grimy,
bloody gweilo on the train.

He followed Jie up the escalator and saw her pointedly nod toward a
toilet door. He went and jiggled the handle, but it was locked. He
turned to go, and the door opened. Behind it was an ancient
grandmother, with a terrible hump that bent her nearly double.

She gave him a milky stare, pursed her lips and began to close the
door.

``Wait!'' he said in urgent, low Chinese.

``You speak Chinese?''

He nodded. ``Some,'' he said. ``I need to use the bathroom.''

``10 RMB,'' she said. He was pretty sure that she wasn't the official
bathroom-minder, but he wasn't going to argue with her. He dug in
his pocket and found two crumpled fives and passed them to her. It
came to \$1.25 and he was pretty sure it was an insane amount of
money to pay for the use of the bathroom, but he didn't care at
this point.

The bathroom was tiny and cramped with the old woman's possessions
bundled into huge vinyl shopping bags. He positioned himself by the
sink and stared at his reflection in the scratched mirror. He
looked like he'd been through a blender, head-first. He ran the
water and used his cupped hands to splash it ineffectually on his
hair and neck, soaking his t-shirt in the process.

``That's no way to do it,'' the old woman shouted from behind him.
She twisted off the faucet with her arthritic hand. He looked
silently at her. He didn't want to get into an argument with this
weird old crone.

``Shirt off,'' she said, in a stern voice. When he hesitated, she
gave his wrist an impatient slap. ``Off!'' she said. ``Shirt off, lean
forward, hair under the tap. Honestly!''

He did as he was bade, bending deeply at the waist to get his hair
under the faucet in the small, dirty sink. She cranked the tap full
open and used her trembling hands to wash out his hair and scrub at
his bloody neck. When he made to stand up, she slapped his back and
said, ``Stay!''

He stayed. Eventually, she let him up, and dug through her bags
until she found a tattered old men's shirt that she handed to him.
``Dry,'' she said.

The shirt smelled of must and city, but was cleaner than anything
he was wearing. He towelled at his hair, careful of the tender cut
on his scalp.

``It's not deep,'' she said. ``I was a nurse, you'll be OK. A stitch
or two, if you don't want the scar.''

``Thank you,'' Wei-Dong managed. ``Thank you very, very much.''

``Ten RMB,'' she said, and smiled at him, practically toothless. He
gave her two more fives and put his t-shirt on. It smelled
terrible, a thick reek of BO and blood, but it was a black tee with
a picture of a charging orc and it didn't show the blood.

``Go,'' she said. ``No more fighting.''

He left, dazed, and found his way into the station, looking for
Jie. She was waiting by the escalator to the surface, fixing her
makeup in a small mirror that just happened to give her a view of
the bathroom door. She snapped the compact shut and ascended to the
surface. He followed.

\tb

``Forty two dead,'' Big Sister Nor said to Justbob and The Mighty
Krang. ``Forty two dead in Shenzhen. A bloodbath.''

``War,'' Justbob said.

``War,'' The Mighty Krang said, with a viciousness that neither of
them had ever heard from him before. He saw their looks, balled his
fists, glared. ``War,'' he said, again.

``Not a war,'' Big Sister Nor said. ``A strike.''

\tb

``A strike,'' General Robotwallah announced to her troops. ``No more
gold gets in or out of any of our games.''

``Forty two dead,'' Yasmin said, in a voice leaden with sorrow.

\emph{Forty three}, Ashok thought, remembering the boy, and sure
enough, Yasmin mouthed \emph{Forty three} as she sat down.

``We'll need defense here,'' General Robotwallah said. ``Bannerjee
will find more badmashes to try to take us out of this place.''

Sushant stood up and held up a machete that the boys had left
behind. ``We took this place. We'll hold it,'' he said, all teen
bravado. Ashok felt like he would be sick.

Yasmin and the General looked intensely at one another, a silent
conversation taking place.

``No more violence,'' the General said, in the voice of command.

Sushant deflated, looked humiliated. ``But what if they come for us
with knives and clubs and guns?'' he said, defiant.

Yasmin stood up and walked to stand next to her general. ``We make
sure they don't,'' she said.

Ashok stood and went to his little back room and began to place
phone calls.

\tb

``Sisters!'' Jie said, throwing her head back and clenching her
fists. She'd been calm enough as she sat down in the basement of
the Internet cafe, a private room the owner rented out discreetly
to porno freelancers who needed a network connection away from the
public eye. But now it seemed as if all the sorrow and pain she had
shoved down into herself when Lu was shot was pouring out.

``\emph{SISTERS!}'' she said again, and it was a howl, as horrible as
the noise Lu had made, as horrible as the noise that half-dead cat
had made in the street in front of Wei-Dong's house.

The cafe was in the shuttered Intercontinental hotel, in the
theme-restaurant that sported a full-size pirate ship sticking out
of the roof, its sails in tatters. The man behind the desk had
negotiated briskly with Jie for the space, studiously ignoring
Wei-Dong lurking a few steps behind her. She'd motioned him along
with a jerk of her head and led him to the private room, which had
once been a restaurant store-room.

Once the door clicked shut behind them, she produced a bootable USB
stick and restarted the computer from it, fitted an elegant,
slender earwig to her ear and passed one to Wei-Dong, which he
screwed into his own ear. After some futzing with the computer she
signalled to him that they were live and commenced to howl like a
wounded thing.

``\emph{Sisters! My sisters!}'' she said, and tears coursed down her
face. ``They killed him tonight. Poor Tank, my Tank. His name, his
real name was Zha Yue Lu, and I loved him and he never harmed
another human being and the only thing he was guilty of was
demanding decent pay, decent working conditions, vacation time, job
security -- the things we all want from our jobs. The things our
\emph{bosses} take for granted.

``They raided us last night, the vicious jingcha, working for the
bosses as they always have and always will. They beat down the door
and the boys ran like the wind, but they caught them and they
caught them and they caught them. Lu and I tried to escape through
the back way and they --'' She broke then, tears coursing down her
face, a sob bigger than the room itself escaping her chest. The
mixer-readouts on the computer screen spiked red from the burst of
sound. ``They shot him like a dog, shot him dead.''

She sobbed again, and the sobs didn't stop coming. She beat her
fists on the table, tore at her hair, screamed like she was being
cut with knives, screamed until Wei-Dong was sure that someone
would burst the door down expecting to find a murder in progress.

Tentatively, he uncrossed his legs and got to his feet and crossed
to her and caught her beating fists in his hands. She looked at
him, unseeing, and stuck her face into his chest, the hot tears
soaking through his t-shirt, the cries coming and coming. She
pulled away for a moment, gasped, ``I'm sorry, I'll be back in a few
minutes,'' and clicked something, and the mixer levels on the screen
flatlined.

On and on she cried, and soon Wei-Dong was crying too -- crying for
his father, crying for Lu, crying for all the gunshots he'd heard
on the way out of the handshake buildings. They rocked and cried
together like that for what seemed like an eternity, and then Jie
gently disengaged herself and turned back to her computer and
clicked some more.

``Sisters,'' she said, ``for years now I've sat at this mic, talking
to you about love and family and dreams and work. So many of us
came here looking to get away from poverty, looking to find a
decent wage for a decent day's work, and instead found ourselves
beating off perverted bosses, being robbed by marketing schemes,
losing our wages and being tossed out into the street when the
market shifts.

``No more,'' she said, breathing it so low that Wei-Dong had to
strain to hear it. ``No more,'' she said, louder. ``NO MORE!'' she
shouted and stood up and began to pace, gesturing as she did.

``No more asking permission to go to the bathroom! No more losing
our pay because we get sick! No more lock-ins when the big orders
come in. No more overtime without pay. No more burns on our arms
and hands from working the rubber-molding machinery -- how many of
you have the idiotic logo of some stupid company branded into your
flesh from an accident that could have been prevented with decent
safety clothes?

``No more missing eyes. No more lost fingers. No more scalps torn
away from a screaming girl's head as her hair is sucked into some
giant machine with the strength of an ox and the brains of an ant.
NO MORE!''

``Tomorrow, no one works. No one. Sisters, it's time. If one of you
refuses to work, they just fire you and the machines grind on. If
you all refuse to work, \emph{the machines stop}.

``If one factory shuts down, they send the police to open it again,
soldiers with guns and clubs and gas. If \emph{all the factories}
shut down, there aren't enough police in the world to open them
again.''

She looked at her screen. It was going crazy. She clicked in a
call. Wei-Dong heard it in his earpiece.

``Jiandi,'' a breathy, girly voice said. ``Is this Jiandi?''

``Yes, sister, it is,'' she said. ``Who else?'' She smiled a thin
smile.

``Have you heard about the other deaths, in the Cantonese quarter in
Shenzhen? The boys they shot?''

Wei-Dong felt like he was falling. The girl was still speaking.

``-- 42 of them, is what we heard. There were pictures, sent from
phone to phone. Google for 'the fallen 42' and you'll find them.
The police said it was lies, and just now, they said that they were
a criminal gang, but I recognized some of those boys from the
strike before, the one you told us about --''

Wei-Dong dug out his phone and began to google, typing so quickly
he mashed the keys and had to retype the query three times, a
process made all the more cumbersome by the need to use proxies to
get around the blocks on his phone's network connections. But then
he got it, and the photos dribbled into his phone's browser as slow
as glaciers, and soon he was looking at shot after shot of fallen
boys, lying in the narrow lanes, arms thrown out or held up around
their faces, legs limp. The cam-phone photos were a little out of
focus, and the phone's small screen made them even less distinct,
but the sight still hit him like a hammerblow.

The girl was still speaking. ``We've all seen them and the girls in
my dorm are scared, and now you're telling us to walk out of our
jobs. How do you know we won't be shot too?''

Jie's mouth was opening and closing like a fish. She held her hand
out and snapped her fingers at Wei-Dong, who passed her his phone.
Her face was terrible, her lips pulled away from her teeth, which
clicked rhythmically as she looked at the photos.

``Oh,'' she said, as if she hadn't heard the girl's question. ``Oh,''
she said, as if she'd just realized some deep truth that had evaded
her all her life.

``Jiandi?'' the girl said.

``You might be shot,'' Jie said, slowly, as if explaining something
to a child. ``I might be shot. But they can't shoot us all.''

She paused, considering. Tears rolled off her chin, stained the
collar of her shirt.

``Can they?''

She clicked something and a commercial started.

``I can't finish this,'' she said in a dead voice. ``I can't finish
this at all. I should go home.''

Wei-Dong looked down at his hands. ``I don't think that would be
safe.''

She shook her head. ``\emph{Home},'' she said. ``The village. Go back.
There's a little money left. I could go home and my parents could
find some boy for me to marry and I could be just another girl in
the village, growing old. Have my one baby and pray it's a boy.
Swallow pesticide when it gets to be too much.'' She looked into his
eyes and he had to steel himself to keep from flinching away. ``Do
you know that China is the only country where more women commit
suicide than men?''

Wei-Dong spoke, his voice trembling. ``I can't pretend that I know
what your life is like, Jie, but I can't believe that you want to
do that. There are 42 dead. I don't think we can stop here.''
Thinking
\emph{I am so far from home and don't know how I'll get back.}
Thinking, \emph{If she goes, I'll be all alone.} And then thinking,
\emph{Coward} and wanting to hit his head against something until
the thoughts stopped.

She reached for the keyboard and he knew enough about her work
environment to see that she was getting ready to shut down.

``Wait!'' he said. ``Come on, stop.'' He fished for the words. In the
weeks since he'd arrived in China, he'd begun to think in Chinese,
even dream in it sometimes, but now it failed him. ``I --'' He beat
his fists on his thighs in frustration. ``It won't stop now,'' he
said. ``If you go home to the village, it will keep going, but it
won't have you. It won't have Jiandi, the big sister to all the
factory girls. When Lu told me about you, I thought he was crazy,
thought there was no way you could possibly have that many
listeners. He thought you were some kind of god, or a queen, a
leader of an army of millions. He told me he thought you didn't
understand how important you are. How you --'' He paused, gathered
the words. ``You're shiny. That's what he said. You shine, you're
like this bright, shiny thing that people just want to chase after,
to follow. Everyone who meets you, everyone who hears you, they
trust you, they want you to be their friend.

``If you go, the Webblies will still fight, but without you, I think
they'll lose.''

She glared at him. ``They'll probably lose with me, too. Do you have
any idea what a terrible burden you put on me? You \emph{all} put
on me? It's absolutely unfair. I'm not your god, I'm not your
queen. I'm a broadcaster!''

The heat rose in Wei-Dong. ``That's right! You're a broadcaster. You
don't work for some government channel like CCTV, though, do you?
You're underground, criminal. You spent years telling factory girls
to stand up for their rights, years living in safe-houses and
carrying fake IDs. You set yourself up to be where you are now. I
can't believe that you didn't dream about this. Look me in the eye
and tell me that you didn't \emph{dream} about being a leader of
millions, about having them all follow you and look up to you! Tell
me!''

She did something absolutely unexpected. She laughed. A little
laugh, a broken laugh, a laugh with jagged shards of glass in it,
but it was a laugh anyway. ``Yes,'' she said. ``Yes, of course. With a
hairbrush for a microphone, in front of my parents' mirror,
pretending to be the DJ that they all listened to. Of course. What
else?''

Her smile was so sad and radiant it made Wei-Dong weak in the
knees. ``I never thought I'd end up here, though. I thought I'd be a
pretty girl on television, recognized in the street. Not a
fugitive.''

Wei-Dong shrugged, back on familiar territory. ``The future's a
weirder place than we thought it would be when we were little kids.
Look at gold-farming, how weird is that?''

She grinned. ``No weirder than making rubber bananas for Swedish
department-store displays. That was my first job when I came here,
you know?'' She rolled up her sleeves and showed him her arms. They
were crisscrossed with old burn-scars. ``Then making cheap beads for
something called 'Mardi Gras.' Boss Chan liked me, liked how I
worked with the hot plastic. No complaining, even though we didn't
have masks, even though I was burned over and over again.'' She
twisted her forearm and he saw that she had the Nike logo branded
backwards, in bubbled, wrinkled scar there. ``Afterwards, I worked
on the same kind of machine, in a shoe factory. You see the logo?
Many of us have it. It's like we were cattle, and the factory
branded us one at a time.''

``Are you going to talk to the people again?''

She slumped. Slipped in her earwig. Began to prod at the computer.
``Yes,'' she said. ``Yes, I must. As long as they'll listen, I must.''

\tb

Matthew wept as he walked, pacing the streets without seeing. He'd
been one of the first ones out of the building when the police
raided, and he'd slipped through the cordon before they'd tightened
it, slipping into another handshake building, one he'd played in as
a boy, and running up the stairs to the roof, where he'd lain on
his belly amid the broken glass and pebbles, staring down at the
street below as the police chased down and caught his friends, one
after the other, a line of Webblies face-down on the ground,
groaning from the occasional kick or punch when they violated the
silence and tried to speak with one another.

The police began to methodically cuff and hood them, starting at
one end, working in threes -- one to cuff, one to hood, and one to
stand guard with his rifle. It seemed to go on forever, and Matthew
saw that he was far from the only person observing the sick
spectacle: the laundry-hung balconies of the handshake buildings
shivered as people piled out onto them, their mobile phones aimed
at the laneway below. Matthew got out his own phone, zooming in
methodically on each face, trying to get a picture of each Webbly
before he was hooded, thinking vaguely of putting the images on the
big Webbly boards, sending them to the foreign press, the dissident
bloggers who used their offshore servers.

Then, sudden movement. Ping was thrashing on the ground, limbs
flailing, head cracking against the pavement hard enough to be
heard from Matthew's perch six stories up. Matthew knew with
hopeless certainty that it was one of his friend's epileptic
seizures, which didn't come on very often, but which were violent
and terrifying for those around him. The cops tried to grab his
arms and legs, and one of them got a hard kick in the knee for his
trouble, and then Ping's arm cracked the hooded prisoner beside
him, who rolled away, stumbled to his feet, and the cops waded in,
rifle-butts raised and ready.

What happened next seemed to take forever, an eternity during which
Matthew struggled not to scream, struggled on the edge of
indecision, of impotence, of being driven to run to the street
below for his comrades and of being too scared to move from the
spot.

A policeman cracked the hooded Webbly who was on his feet across
the kidneys, and the boy screeched and staggered and happened to
catch hold of the rifle-butt. The two grappled for the gun while
the boys on the pavement shouted, other policemen closing in, and
then one of them unholstered his revolver and calmly shot the
hooded boy in the head, the hood spattered and red as the boy
fell.

That was it. The boys leapt to their feet and \emph{charged},
warriors screaming their battle-cries, unarmed children scared and
brave and stupid, and the police guns fired, and fired, and fired.

The cordite smell overpowered his senses, a smell like the
fireworks he and his friends used to set off on New Year's. Mingled
with it, the blood smell, the shit smell of boys whose bowels had
let go. Matthew cried silently as he aimed his phone at the
carnage, shooting and shooting, and then a policeman looked up at
the crowd observing the massacre and shouted something indistinct,
the camera lens on his helmet glinting in the dawn light, and
Matthew ducked back as the rest of the policemen looked up, and
then he heard the screaming, screaming from all around, from all
the balconies.

He pelted across the roof, headed for the next building, vaulting
the narrow gap between the two with ease. Twice more he leapt from
building to building, running on sheer survival instinct, his mind
a blank. Then he found himself on the street, with no memory of
having descended any stairs, walking briskly, headed for the center
of town, the streets with the fancy shops and the pimps, the
businessmen and the Internet cafes filled with screaming boys
killing orcs and blowing space-pirates out of the sky and
vanquishing evil super-villains.

The tears coursed down his cheeks, and the early morning rush of
people on their way to work gave him a wide berth. He wasn't the
first boy to walk the streets of Shenzhen in tears, and he wouldn't
be the last. He randomly boarded a bus and paid the fare and sat
down, burying his face in his hands, choking back the sobs. He'd
ridden the bus for a full hour before he bothered to look up and
see where he was headed.

Then he had to smile. Somehow, he'd boarded a bus headed for Dafen,
the ``oil painting village,'' where thousands of painters working in
small factories turned out millions of paintings. He'd gone there
once with Ping and the boys, on a rare day off, to wander the
narrow streets and marvel at the canvasses hung everywhere, in
outdoor stalls and in open shops and in huge galleries. The
paintings were mostly in European style, old fashioned, depicting
life in ancient European cities, or the tortured Jesus (these made
Matthew squirm and remember his father's stories of persecution) or
perfect fruit sitting on tables. Some of the shops and stalls had
painters working at them, copying paintings out of books, executing
deft little brushstrokes and closing out the rest of the world. The
books themselves were printed in Dongguan -- Matthew knew a factory
girl who worked at the printer -- and something about the whole
scene had filled Matthew with an unnameable emotion at the thought
of all these painters creating work with their artist's eyes and
hands for use by foreigners who'd never come to China, never
imagine the faces and hands of the painters who made the work.

And here they were, pulling up at the five-meter-tall sculpture of
a hand holding a brush, disgorging dozens of passengers by the side
of the road. All around him rose the tall housing blocks and long
factory buildings, the air scented with breakfast and oil paint and
turpentine.

Matthew came out of his funk enough to notice that many of his
fellow passengers wore paint-stained work-clothes and carried
wooden paint-boxes, and he joined the general throng that snaked
into Dafen, amid the murmur of conversation as workers greeted
friends and passed the gossip.

The time he'd visited Dafen, he'd wandered into a gallery that sold
contemporary paintings by Chinese painters, showing Chinese
settings. He'd never had much use for art, but he'd been poleaxed
by these ones. One showed four factory girls, beautiful and young,
holding mobile phones and designer bags, walking down a rural
village street at Mid-Autumn Festival, the house-fronts and
shop-windows hung with lanterns. The village was old and poor, the
street broken, the people watching from the doorways with seamed
peasant faces, pinched and dried up. The four girls were glamorous
aliens from another world, children who'd been sent away to find
their fortunes, who'd come back changed into a different species
altogether.

And there'd been a picture of an old grandmother sleeping in a
Dongguan bus-shelter, toothless mouth thrown open, huddled under a
fake designer coat that was streaked with grime and torn. And a
picture of a Cantonese man on a ladder between two handshake
buildings, hanging up an illegal cable-wire. The images had been
poignant and painful and beautiful, and he'd stood there looking at
them until the gallery owner chased him out. These were for people
with money, not people like him.

Now, passing by the same shop, he felt a jolt of recognition as he
saw the picture of the four factory girls, arms around each others'
shoulders, in the shop's windows. It hadn't sold -- or maybe the
painter turned them out by the truckload. Maybe there was a factory
full of painters devoted to making copies of this painting.

He became conscious of a distant hubbub, an indistinct roar of
angry voices. He thought he'd been hearing it for some time now,
but it had been subsumed in the sound of the people around him. Now
it was growing louder, and he wasn't the only one who'd noticed it.
It was a chant, thunderous and relentless, with tramping, rhythmic
feet. The crowd craned their necks around to locate the
disturbance, and he joined them.

Then they turned the corner and he saw what it was: a group of
young men and women, paint-stained, holding up sheets of paper with
beautifully calligraphed slogans: ``NON-FORMULA PAINTING FACTORY
UNFAIR!'' ``WE DEMAND WAGES!'' ``BOSS SIU IS CORRUPT!'' The signs were
decorated with artistic flourishes, and he saw that at the far end
of the picket there was a trio of painters crouched over a pile of
paper, brushes working furiously. A new sign went up: ``REMEMBER THE
42!'' and then one that simply said ``IWWWW'' in the funny Western
script, and Matthew felt a surge of elation.

``Who are the 42?'' he asked one of the painters, a pretty young
woman with several prominent moles on her face. She pushed her hair
behind her ears. ``It was three hours ago,'' she said, then looked at
the time on her phone. ``Four hours ago.'' She shook her head,
brought up some pictures on her phone. ``The police executed 42 boys
in Cantonese town. They say that the boys were criminals, but the
neighbors say they were just gold-farmers.'' She showed him the
pictures. His friends, on the ground, heads in hoods, being shot by
policemen, reeling back under the fire. The policemen anonymous
behind their masks. The girl saw the expression on his face and
nodded. ``Terrible, isn't it? Just terrible. And the things the
fifty-cent army have been saying about them --'' The fifty-cent army
was the huge legion of bloggers paid fifty cents -- 4 RMB -- to
write patriotic comments and posts about the government.

He found that he was sitting on the dirty sidewalk, holding the
girl's phone. She knelt down with him and said, ``Hey, mister, are
you all right?''

He nodded his head automatically, then shook it. Because he wasn't
all right. Nothing was all right. ``No,'' he said.

The girl looked at the sign she'd been painting and then at him.
She turned her back on the painting and took his chin, tilted his
face up. ``Are you hurt?''

``Not hurt,'' he said. ``But.'' He shook his head. Pointed at her
phone. Drew out his own. Brought up the photos he'd taken while
trembling on the roof.

``The same photos?'' she said. Then looked closer. ``Different photos.
Where'd you get them?''

He said, ``I took them,'' and it came out in a rasp. ``They were my
friends.''

She jolted as if shocked, then bit her lip and paged through the
photos. She smelled of turpentine and her fingers were very long
and elegant. She reminded Matthew of an elf. ``You were there?'' It
was only half a question, but he nodded anyway. ``Oh, oh, oh,'' she
said, handing him back the phone and giving him a strong, sisterly
hug. ``You poor boy,'' she said.

``We heard about it an hour ago, while we were settling in to work.
We gathered to discuss it, leaving our canvasses, and our boss,
Boss Siu, came by and demanded that we all get back to work. He
wouldn't let us tell him why we were gathered. He never does. It's
like Jiandi says on her radio show -- he controls our bathroom
breaks, docks our wages for talking or sometimes just for looking
up for too long. And when he told us we were all being docked, one
of the girls stood up and shouted a slogan, something like 'Boss
Siu is unfair!' and though it was funny, it was also so
\emph{real}, straight from her heart, and we all stood up too and
then --'' She gestured at the line.

Matthew remembered the day they'd walked out, a million years ago,
remembered the police arriving and taking them to jail, remembered
his vow never to go to jail again. And then he picked up the sign
she'd been making and gripped it by the corners and joined the
line. He wasn't the only one. He shouted the slogans, and his voice
wasn't hoarse anymore, it was strong and loud.

And when the police finally did come, something miraculous
happened: the huge crowd of painters and other workers who'd
gathered at the factory joined ranks with the picketers and picked
up their slogans. They held their phones aloft and photographed the
police as they advanced, with masks and helmets and shields and
batons.

They held their ground.

The police fired gas cannisters.

Painters with big filter masks from the factories seized the
cannisters and calmly threw them through the factory windows,
smoking out the bosses and security men who'd been cowering there,
and they came coughing and weeping and wheezing.

The crowd expanded, moved \emph{toward} the police instead of
\emph{away} from it, and a policeman darted forward out of his
line, club raised, mouth and eyes open very wide behind his
facemask, and three factory girls sidestepped him, tripped him, and
the crowd closed over him. The police line trembled as the man
disappeared from view, and just as it seemed like they would
charge, the mob backed away, and the man was there, moving a little
but painfully, lying on the ground. His helmet, truncheon and
shield were gone, as was his utility belt with its gun and its gas
and its bundle of plastic cuffs.

\emph{Now we have a gun}, Matthew thought, and from a far distance
observed that he was thinking like a tactician again, not like a
terrorized boy, and he knew which way the police should come from
next, that alley over there, if they took it they'd control all the
entrances to the square, trapping the picketers.

``We need people over there,'' he shouted to the painter girl, whose
name was Mei, and who had stood by his side, her fine slender arm
upraised as she called the slogans with him. ``There and there. Lots
of them. If the police seal those areas off --''

She nodded and pushed off through the crowd, tapping people on the
shoulder and shouting in their ears over the roar of the mob and
the police sirens and the oncoming chopper. That chopper made
Matthew's hands sweaty. If it dropped something on them --
\emph{gas, surely, not bombs, surely not bombs} he thought like a
prayer -- there'd be nowhere to hide. Protesters moved off to
defend the alleyways he'd pointed to, armed with bricks and rocks
and cameraphones. The same funnel-shaped alley-mouths that would
make those alleys so deadly in the hands of their enemies would
make them easier to defend.

The chopper was coming on now, and the cameraphones pointed at the
sky, and then the helicopter veered off and headed in a different
direction altogether. As Matthew raised his own phone to photograph
it, he saw that he'd missed several calls. A number he didn't
recognize, overseas. He dialled it back, crouching down low in the
forest of stamping feet to get out of the noise.

``Hello?'' a woman's voice said, in English.

``Do you speak Chinese?'' he said, in Cantonese.

There was a pause, then the phone was handed off to someone else.
``Who is this?'' a man's voice said in Mandarin.

``My name is Matthew,'' he said. ``You called me?''

``You're one of the Shenzhen group?'' the man said.

``Yes,'' he said.

``We've got another survivor!'' he called out and sounded genuinely
elated.

``Who is this?''

``This is The Mighty Krang,'' the man said. ``I work for Big Sister
Nor. We are so happy to hear from you, boy! Are you OK, are you
safe?''

``I'm in the middle of a strike,'' he said. ``Thousands of painters in
Dafen. That's a village in Shenzhen, where they paint --''

``You're in Dafen? We've been seeing pictures out of there, it looks
insane. Tell me what's going on.''

Without thinking, just acting, Matthew scaled a park bench and
stood up very tall and dictated a compact, competent situation
report to the The Mighty Krang, whom he'd seen on plenty of
video-conferences with Big Sister Nor and Justbob, snickering and
clowning in the background. Now he sounded absolutely serious and
intent, asking Matthew to repeat some details to ensure he had them
clear.

``And have you seen the other strikes?''

``Other strikes?''

``All around you,'' he said. ``Lianchuang, Nanling and Jianying
Gongyequ. There's a factory on fire in Jianying Gongyequ. That's
bad business. Wildcatters -- if they'd talked to us first, we would
have told them not to. Still.'' He paused. ``Those photos were
something. The 42.''

``I have more.''

``Where'd you get them?''

``I was there.''

``Oh.''

A long pause.

``Matthew, are you safe where you are?''

Matthew stood up again. The police line had fallen back, the
demonstration had taken on something of a carnival air, the artists
laughing and talking intensely. Some had instruments and were
improvising music.

``Safe,'' he said.

``OK, send me those photos. And stay safe.''

Two more helicopters now, not headed for them. Headed, he guessed,
for the burning factory in Jianying Gongyequ. He hoped no one was
in it.

\tb

Mr Bannerjee came for them that night, with another group of thugs,
but these weren't skinny badmashes, but grown adults, dirty men
with knives and clubs, men who smelled of betel and sweat and smoke
and fiery liquor, a smell that preceded them like a messenger
shouting ``beware, beware.'' They came calling and joking through
Dharavi, a mob that the Webblies heard from a long way off. Mrs
Dibyendu's neighbors came to their windows and clucked worriedly
and sent their children to lie down on the floor.

Mr Bannerjee led the procession, in his pretty suit, the mud
sucking at his fine shoes. He stood in the laneway before the door
to Mrs Dibyendu's cafe and put his hands on his hips and lit a
cigarette, making a show of it, all nonchalance as he puffed it to
life and blew a stream into the hot, wet air.

He waited.

Mala limped to the door and opened it. Behind her, the cafe was
dark and not a thing moved.

Neither said a word. The neighbors looked on in worried silence.

``Mala,'' Mr Bannerjee said, spreading his hands. ``Be reasonable.''

Mala stepped onto the porch of the cafe and sat down, awkwardly
folding her legs beneath her. In a clear, loud voice, she said, ``I
work here. This is my job. I demand the right to safe working
conditions, decent wages, and a just and fair workplace.''

Mr Bannerjee snorted. The men behind him laughed. He took a step
forward, then stopped.

One by one, Mala's army filed out of the cafe, in a disciplined,
military rank. Each one sat down, until the little porch was
crowded with children, sitting down.

Mr Bannerjee snorted again, then laughed. ``You can't be serious,''
he said. ``You want, you want, you want. When I found you, you were
a Dharavi rat, no money, no job, no hope. I gave you a good job,
good wages, and now you want and want and want?'' He made a
dismissive noise and waved his hand at her. ``You will remove
yourself from my cafe and take your schoolchums with you, or you
will be hurt. Very badly.''

The neighbors made scandalized clucking noises at that and Mr
Bannerjee ignored them.

``You won't hurt us,'' Mala said. ``You will go back to your fine
house and your fine friends and you will leave us alone to control
our destiny.''

Mr Bannerjee said nothing, only smoked his cigarette in the night
and stared at them, considering them like a scientist who's
discovered a new species of insects.

``You are making mischief, Mala. I know what you are up to. You are
disrupting things that are bigger than you. I tell you one more
time. Remove yourself from my cafe.''

Mala made a very soft spitting sound, full of contempt.

Mr Bannerjee raised his hand and his mob fell silent, prepared
themselves.

And then there was a sound. A sound of footsteps, hundreds of them.
Thousands of them. An army marching down the laneway from both
sides, and then they were upon them. Ashok leading the column from
the left, old Mrs Rukmini and Mr Phadkar leading the column from
the right.

The columns themselves were composed of union workers -- textile
workers, steel workers, train workers. Ashok's phonecalls and
photos and stories had paid off. Hundreds of text messages were
sent and workers were roused from their beds and they hastily
dressed and gathered to be picked up by union busses and driven all
across Mumbai to Dharavi, guided in to Mrs Dibyendu's shop by
Ashok, who had whispered his thanks to the leaders who had given
him their support.

The workers halted, just a few paces from the gangsters and their
evil smells. Ashok looked at the two groups, the sitting army and
the standing mob, and he deliberately and slowly sat down.

The exquisitely elderly ladies leading the other column did the
same. The sitting spread, moving back through the group, and if any
worker thought of his trousers or her sari before sitting in the
grime of the Dharavi lane, none said a word and none hesitated.

Bannerjee swallowed audibly. One of the neighbors leaning out of a
window snickered. Bannerjee glared up at the windows. ``Houses in
slums like this burn down all the time,'' he said, but his voice
quavered. The neighbor who'd snickered -- a young shirtless man
with burns up and and down his bare chest from some old accident --
closed his shutters. A moment later, he was on the street. He
walked up to Bannerjee, looked him in the eye, and then,
deliberately, folded his legs and sat down before him. Bannerjee
raised his leg as if to kick and the crowd \emph{growled}, a low,
savage sound that made the hair on the back of Mala's neck stand
up, even as she made it herself. It sounded as though all of
Dharavi was an angry dog, straining at its leash, threatening to
lunge.

More neighbors drifted into the street -- old and young, men and
women. They'd known Mrs Dibyendu for years. They'd seen her driven
from her home and business. They were making the same noise. They
sat too.

Mr Bannerjee looked at Mala and opened his mouth as if to say
something, then stopped. She stared at him with utter calm, and
then smiled broadly. ``Boo,'' she said, softly, and he took a step
back.

His own men laughed at this and he went purple in the dim light of
the street. Mala bit her tongue to keep from laughing. He looked so
comical!

He turned with great dignity to look at his men, who were so tense
they practically vibrated. Mala watched in stupefied awe as he
grabbed one at random and slapped him, hard, across the face, a
sound that rang through the narrow laneway. It was the single
dumbest act of leadership she'd ever seen, so perfectly stupid you
could have put it in a jar and displayed it for people to come and
marvel at.

The man regarded Bannerjee for a moment, his eyes furious, his
fists bunched. He was shorter than Bannerjee, but he was carrying a
length of wood and the muscles in his bare forearms jerked and
bunched like a basketful of snakes. Deliberately, the man spat a
glob of evil, pink, betel-stained saliva into Bannerjee's face,
turned on his heel and walked away, delicately picking his way
through the sitting Webblies and workers and neighbors. After a
moment, the rest of Bannerjee's mob followed.

Bannerjee stood alone. The saliva slid down his face. Mala thought
\emph{If he takes out a gun and starts blazing away, it wouldn't surprise me in the least.}
He was totally beaten, humiliated before children and the poor of
Dharavi, and there were so many cameraphone flashes dancing in the
night it was like a disco in a movie.

But perhaps Bannerjee didn't have a gun, or perhaps he had more
self-control than Mala believed. In any case, he, too, turned on
his heel and walked away. At the end of the alley, he turned back
and said, in a voice that could be heard above the buzz of
conversation that sprang up in his wake, ``I know where your parents
live, Mala,'' and then he walked away altogether into the night.

The crowd roared with triumph as he disappeared. Ashok helped her
stand, his hand lingering in hers for longer than was strictly
necessary. She wanted to hug him, but she settled for hugging
Yasmin, who was crying, happy tears like the ones they'd shared so
many times before. Yasmin was as thin as a piece of paper but her
arms were strong, and oh, it did feel good to be held for a moment,
instead of holding everyone else up.

She let go at last and turned to Ashok. ``They came,'' she said.

Instead of answering, he led her to two tiny old ladies, and a man
with a skullcap and a beard. ``Mr Phadkar, Mrs Rukmini and Mrs
Muthappa,'' he said. ``This is Mala. They call her General
Robotwallah. Her workers have been defending the strike. They are
unbeatable, so long as they have a place to work.''

Mr Phadkar looked fierce. ``You will always have a place to work,
General,'' he said, in a voice that was pitched to carry to the
workers who gathered around them, excitedly passing whispered
accounts of the historic meeting back through their ranks.

The old ladies rolled their eyes at one another, which made Mala
smile. They each took one of her hands in their calloused, dry old
hands and squeezed. ``You were very brave,'' one said. ``Please,
introduce us to your comrades.''

They chatted all night, and the women's papadam collective brought
them food, and there was chai, and as there were far too many
people to fit in the little cafe, the party occupied the whole of
the laneway and then out into the street. Mala and her fighters
fought on through the night in shifts, stepping out on their breaks
to mingle, making friends, bringing them into the cafe to explain
what they did and how they did it.

And there were reporters asking questions, and the gupshup flew up
and down the streets and lanes of Dharavi, picking up steam as the
roosters began to call and the first of the early risers walked to
the toilets and the taps and had their ears bent. The bravery of
the children, the valor of the workers, the evil of the sinister
Bannerjee in his suit and the thugs he'd brought with him -- it was
a story straight off the movie screen, and every new ear it entered
was attached to a mouth that was anxious to spread it.

Mala and Yasmin's parents came to see them the next morning, as
they sat groggy after a night like no other night, on the porch of
Mrs Dibyendu's cafe. The parents didn't know what to make of their
strange daughters, but they were visibly proud of them, even
Yasmin's father, which clearly surprised Yasmin, who'd looked like
she expected a beating.

As their mothers gathered them into their bosoms, Mala looked at
Yasmin, and saw the haunted look in Yasmin's eye and knew, just
\emph{knew} that she was thinking of the little boy who'd died.

How did she know? Because Mala herself had never stopped thinking
of him, and thinking of how she'd taken the actions that led to his
death. And because Mala herself knew that no amount of sitting down
peacefully and braving thugs with her moral force instead of her
army would ever wipe the stain of that boy's death off her karma.

And then Mamaji kissed Mala's forehead and murmured many things in
her ear, and her little brother emerged from behind her skirts and
demanded to be shown how it all worked and stared at her with so
much admiration that she thought he'd burst and for a moment, it
was all golden.

Ashok looked on from his little office, meeting with the union
leaders, talking to Big Sister Nor. Something big was brewing with
him, she knew, something even bigger than this miracle that he'd
pulled off. She fobbed her brother off on a group of boys who were
eager to teach him some of the basics and bask in the pure
hero-worship radiating off of him, then slipped back into Ashok's
room and perched at his side on a stool, moving a pile of papers
away first.

``That was incredible,'' she said. ``Absolutely incredible.'' She said
it quietly, with conviction. ``You're our saviour.''

He snorted through his nose, then scrubbed at his eyes with his
fists. ``Mala, my general, you do a hundred incredible things every
day. The only reason all those people came out is because I could
show them what you'd done, explain how you had organized these
children, these slum-rats, into a disciplined force that was
committed to justice.''

She squirmed on her seat. ``I'm just bloodthirsty,'' she said. ``I'm
just one of those people who fights all the time.'' Thinking again
of the boy, the dead boy. His blood was still under Ashok's
fingernails.

He turned and, just for an instant, touched her arm. The gesture
was gentle, tender. No one had ever touched her quite like that. It
broke something in her, some flood-dam that had safely contained
all the pain and fear and shame, and she had to turn and run
blindly out into the lane and around a corner to weep and weep
biting her lip to keep from screaming out her grief. Though she
heard some of the others looking for her, she kept silent and did
not let them find her. Then she realized she was hiding in the same
place in which she'd hidden from Mrs Dibyendu's idiot nephew, and
that broke another dam and it was quite some time before she could
get herself under control and head back into the laneway again.

She didn't get very far. Out front of dozens of businesses, there
were small groups of people boisterously shouting rhymed chants
about working conditions and pay. Crowds gathered to talk to each
other, and there were arguments, laughter, a fistfight. She stood
in the middle of the road and thought, \emph{How can this be?}

And at that moment, she realized that she was not alone. All over
Dharavi, all over the world, there were people like her who wanted
more, \emph{demanded} more, with a yearning that was always just
there, beneath the skin, and it only took the lightest scratch to
let it out.

She didn't go back to Mrs Dibyendu's cafe. Instead, she took her
walking stick and limped all around Dharavi, up and down the
streets where the tiny factories would normally have been hives of
activity. Many of them were, but many were not -- many had workers
and crowds out front, and it was like a virus that was spreading
through the streets and lanes and alleys, and now it was as if all
the crying had lightened her so that her feet barely touched the
ground, as though she might fly away at any instant.

She was just turning to go back to her army and maybe a few hours'
sleep when they grabbed her, hit her very hard on the head, and
dragged her into a tiny, stinking room.

\tb

Confidence is a funny thing. When lots of people believe something
is valuable, it becomes valuable. So if you're selling game-gold
and people think game-gold is valuable, they buy it.

But it's better than that. If there's a wide-spread belief that
Svartalfaheim Warriors swords are valuable, then even people who
\emph{don't} think they're valuable will buy them, because they
believe they can sell them to people who \emph{do} believe that
they're valuable.

And when people who buy to sell to others start to bid on
Svartalfaheim swords, the price of the swords goes up. Of course it
does: the more buyers there are for something, the higher the price
goes. And the higher the price goes, the more buyers there are,
because hey, if the price is high, there must be plenty of suckers
who'll take the swords off your hands in a little while for an even
higher price.

Confidence makes value. Value makes more value, which makes more
confidence. Which makes more value.

But it's not infinite. Think of a cartoon character who runs off a
cliff and keeps running madly in place, able to stay there until
someone points out that he's dancing on air, at which point he
plummets to the sharp rocks beneath him.

For so long as everyone believes in the value of a Svartalfaheim
sword, the sword will be valuable, and get more valuable. As the
pool of people who might buy a Svartalfaheim sword grows -- say,
because they're getting calls from their brokers offering to sell
them elaborate, complex sword futures (a contract to buy a sword at
a later date), or because their smart-ass nieces and nephews are
talking them up -- the likelihood that someone will say, ``Are you
\emph{kidding me?} This is a \emph{sword} in a \emph{video game}!''
goes up.

Indeed, this doubter might have other choice observations, like
this: ``If \emph{everyone} has these swords, doesn't that mean that
there's more swords than anyone could possibly use? Doesn't that
mean that they're not valuable, but \emph{valueless}?''

Or if the doubter is impossibly old fashioned, he might even say:
``What if the people who run this Fartenstein game decide to change
the number of swords available by just \emph{deleting} a ton of
them? Or by printing up a kazillion more? Or change the swords into
toothpicks?''

Oh, the sword-speculators will reply, they'll \emph{never} do that,
it would ruin the game, they can't afford to do that. And here's
the thing: they're half-right. So long as the game-runners believe
that messing around with the swords will piss off all these people
who own, speculate on, buy and sell swords, they can't afford to do
it.

These cartoon characters run in place on air, shouting that the
swords will always go up in value, shouting that the game-runners
will never nerf or otherwise bork them, and they can stay there, up
in the air, waving their swords, being joined by others who are
convinced by their arguments and the incontrovertible fact that
they are indeed not falling, until\ldots{}

Until\ldots{}

Until there's enough widespread confidence in the proposition that
they will fall. Until the press starts to publish wide-eyed stories
about the absurdity of ever believing in the value of these swords,
pointing out that the fall is inevitable, that it was pre-ordained
from the moment the first speculator bought his first sword.

Think of the belief in infallible swords as a solar system. In the
center, there's the sun, gigantic and white-hot, exerting gravity
on the planets and asteroids that spin around and around it. At the
outer edge is the dandruff of planetesimals and asteroids, weakly
caught in the gravity, only halfway committed to being part of the
system. As the sun begins to cool off, begins to shrink with the
force of disbelief, these outer hangers-on fly away. These are the
tasters, the people who bought one or two little swords or
sword-futures or ``fully hedged complex sword derived securities''
because everyone else was doing it. They hear that this thing is
too good to be true and see the prices start to drop and so they
sell off what they've got, take a small loss, and tell their
friends.

Well, now there's a bunch of people saying that swords aren't
really that valuable. Less confidence equals lower prices. And
there's more swords on the market. More swords equals lower prices.
The larger planets, closer in, the investors with a fair bit of
money in imaginary cutlery, these folks see the prices dip and
continue to fall. They hear the brokers and analysts scurrying
around saying, ``No, no, the sun will burn bright forever, the sun
will never dim! Prices will come up again. This is temporary.''

Here's the thing: if the brokers and analysts can convince these
bigger investors that they're right, \emph{they will be right}. If
these bigger investors hold on to their swords, the market will
stay healthy for a while longer.

But if they aren't convincing enough, if these bigger investors
lose confidence and start selling, they'll never stop. That's
because the \emph{first} seller to get out of the sword-market will
get the highest price for his goods. But once he gets out, his
swords will be on the market (remember, more swords equals lower
prices) and everyone else will get a lower price. And when
\emph{they} sell, the prices will go down further, panicking more
investors, putting more swords on the market, forcing the prices
down further.

Somewhere in there, the game-runners are apt to have a minor
freak-out and then a major one. They'll start to mess with the
sword-supply. They'll take swords out of the market, or put swords
in, or nerf swords, or buff the hell out of them, anything to keep
the fun from collapsing out of the game.

And that'll probably make things worse, because this isn't an exact
science, it's a bunch of guesswork, and there are ten zillion ways
to get this wrong and so few ways to get it right.

The sun gets smaller, and dimmer, and the close-in planets are
feeling the tug of oblivion now, the call of deep space that says,
``Spin away, spin away to forever, for the sun is dying!''

They don't want to spin away. They want to hang on. They have so
many swords in the bank, they're practically \emph{made} of swords.
They've made a fortune buying and selling swords. Of course, they
spent the fortune on more swords. Or different swords. Or axes. But
whatever they've spent it on, it's basically the same thing,
because every broker knows that you won't get in trouble for
recommending that people buy things that have always been
profitable.

If the sword market collapses, these planets -- these major,
committed investors -- will die. They will be wiped out. They have
pledged their lives and love and immortal souls to magic swords,
and if the swords break their hearts, they will never recover. So
as the market for swords gets crummier and crummier and crummier
and crummier, they grow more and more insistent that everything is
fine, just fine, it'll all be back to ``normal'' any day now. They
can't afford to lose confidence, because they aren't going to fly
off into space. They're going to fall into the dying sun and will
be incinerated in its glowing heart.

But denial only works for so long. The sun is dying. No one wants
your swords. Your swords are worthless. Even the people who need a
sword to kill some elves or orcs or random wildlife critters are
faintly embarrassed by the fact, because worthless swords are now
the subject of numerous jokes about idiotic investment schemes and
corrupt brokerages and loony investors who got swept up in the heat
of the moment. These people go and kill monsters with bows and
clubs for a while, because everyone knows how much swords suck.

How low can the value of a sword go? Subzero, as it turns out. Not
only can a sword become worthless, it can actually cost you money
to get rid of it. Oh, not the sword itself, of course, but the
\emph{derivatives} of the swords. The bets on swords. Where someone
else has made a bet on whether your sword will go up or down in
value, and then packaged it up with a bunch of other bets, just
figuring out which bets are in which packages can cost so much
money that you end up losing money, even on winning bets.

Confidence is great, but it isn't everything. Reality catches up
with everyone, eventually. All suns extinguish themselves. All
cartoon characters eventually plummet to the bottom of the canyon.
And every sword is eventually worthless.

\tb

Command Central was bedlam. The game-runners snarled at each other
like bad-tempered, huge-bellied dinosaurs, and ate like dinosaurs,
too, sending out for burgers, pizza, buckets of chicken, huge thick
shakes Anything they could scarf down one-handed while they labored
over their screens and shouted insults at one another.

Connor hardly noticed. He was deep in his feeds. Bill's new
security subroutines let him run every player's actions backwards
and forwards like a video, branching off into other players'
timelines every time they crossed paths in a party, a PvP combat
session, a trade, or a conversation. It was an ocean of
information, containing every secret of every player in every game
that Coke ran.

It was too much information. He was looking for something very
precise -- the identities of gold-farmers -- but what he had was
every damned thing ever uttered or done in-game. It was a wondrous
toy and an infinite distraction, and practically every spare moment
Connor could muster was spent writing scripts and filters to help
him make sense of it.

Just now he was watching a feed of every player who had PvP killed
another player, where the dead player's toon had earned more than
1000 Mario coins in the previous hour. This was turning out to be a
rich vein of potential gold-farmers and Webblies. He was just
trying to figure out how to write a script that would also grab the
player IDs of anyone who was \emph{nearby} during one of these
fights, when he realized that Command Central had gotten even
noisier than usual, devolving into raw chaos.

He looked up. ``What's wrong?'' he said, even as his fingers moved to
call up general feeds showing the overall health of the game and
its systems. And even before anyone answered he saw what was wrong.
Server load had spiked across every game-shard, redlining the
server-clusters seated in air-conditioned freight containers all
over the world. It seemed as though every single metric for
server-load was at peak -- calculations per second, memory usage,
disk churn. But on closer examination, he saw that this wasn't
quite true: network load was down. Way down. Somehow, these vast
arrays of computing power were all being made to work so hard they
were in danger of collapsing, but it was all happening without
anyone talking very much to the servers.

Indeed, network load was \emph{so} low that it seemed that hardly
anyone could be logged into these servers -- and yes, there it was,
the number of players logged in was low and falling -- a million
players, then 800,000, then 500,000, then 300,000, and finally the
games stabilized at about 40,000 sessions. Another click revealed
why: the system was kicking off players as the load increased,
trying to make room in memory and on the CPUs for whatever monster
process was tearing through the frigid shipping containers.

``What the hell is going on?'' he said, shouting into the general
din. Kaden was on the phone with ops, shouting at the systems
administrators to get on it, trace every process on the boxes,
identify whatever species of strangler vine was loose in the
machines, choking them to death.

Bill, meanwhile, had set loose \emph{his} special team of grey-hat
hackers to try and figure out if there were any of their black-hat
brethren loose on the systems, crackers who'd broken in to steal
corporate secrets, amass virtual wealth, or simply crash the thing,
either to benefit a competitor, seek ransom or simply destroy for
the pleasure of destruction.

Connor's money was on hackers. Each cluster was built and tested at
Coke Games HQ in Austin, burned in for three solid weeks after it
was all bolted into place in the shipping container. Once it had
been green-lighted, it was loaded onto a flatbed truck and shipped
to a data-center somewhere cold, preferably near a geothermal vent,
tide-farm or wind-farm. There were plenty of sites in Newfoundland
and Alaska, and some very good ones in Iceland and Norway, a few in
Belgium and some in Siberia. The beauty of using standard shipping
containers for their systems is that they were easy to ship (duh).
The beauty of sticking the containers somewhere cold was that the
main cost of running the systems was cooling off the machines as
they relentlessly rubbed electrons against each other, bouncing
them through the pinball-machine guts of the chips within them. On
a cold day when the wind was blowing, they could knock the cost of
running one of those containers in half.

Coke bought their data-center slots in threes, keeping one empty.
When a new container arrived, it was slotted into the empty bay,
run for a week to make sure nothing had been hurt in transit, and
then the oldest container in a Coke-slot was yanked, loaded back
onto a train, or ship, or flatbed truck, and sent back to Austin,
detouring at Mumbai or Shenzhen or Lagos to drop off the computers
within, stripped by work crews who sent them off to the used server
markets to be torn to pieces and salvaged.

The containers were all specialized, only handling local traffic,
to keep down network lag. But if one was overwhelmed, it could
start offloading on its brothers around the planet -- better to
face a laggy play experience than to be knocked off altogether. It
was inconceivable that every server on the planet would suddenly
get a spike in players and hit capacity and not be able to offer
some support to the others. Inconceivable, unless someone had
sabotaged them.

In the meantime, Connor had his feeds, his forensics, his gigantic
haystacks and their hidden needles. Let the others worry about the
downtime. He had bigger fish to fry.

He plunged back in, writing ever-more-refined scripts to try to
catch the bad guys. He had a growing file of suspects to look into
in more depth, using another set of scripts and filters he'd been
drafting in the back of his mind. He already knew how he'd do it:
he'd build his files of bad guys, make it big and deep, follow them
around the game, see who else they knew, get thousands and
thousands of accounts and then:

Destroy them.

In one second, one \emph{instant}, he'd delete every single one of
their accounts, make their gold and elite items vanish, toss every
single one out for terms-of-service violations. That part would be
\emph{easy}. The terms of service were so ridiculously strict and
yet maddeningly vague that simply playing the game necessarily
involved violating them. He'd obliterate them from gamespace and
send them all back to their mommies crying. Thinking this kind of
thing made him feel dirty and good at the same time.

He was deep in meditation when a fat, hairy hand reached over his
shoulder and slammed his laptop lid down so hard he heard the
screen crack, and then the hand reversed its course and slapped him
so hard in the back of the head that his face bounced off the table
in front of him.

Command Central fell perfectly silent as Connor straightened up,
feeling and then tasting the blood pouring out of his nose. His
ears were ringing. He turned his head slowly, because his eyes
wouldn't focus properly and his head felt like it was barely
attached to his neck. Standing over him, snorting like freight
engine, stood Kaden, the head of ops, wearing a two-day beard and
smelling of rancid sweat.

``What --''

The man drew back his beefy fist again, cocking it for another blow
to Connor's head and Connor flinched away involuntarily. He hadn't
been in a fight since his schoolyard days, and he couldn't believe
that this actual adult man had actually hit him with his actual
fists. Something was growing in his chest, bubbling over, headed
into his arms and legs. His breath came in short pants, every
inhale bringing blood into his mouth. His heart thudded. He stood
up abruptly, knocking his chair over backwards and --

Leapt!

He pushed off with both legs, throwing his own considerable bulk
into Kaden's huge, protruding midsection. It was like a medicine
ball, hard and unyielding, and he rebounded off it, just as Kaden's
fist clobbered him again, getting him with a hard hammerblow in the
back of the neck that knocked him to the ground.

He hit the ground with a thud that he felt in every bone in his
body, his head caroming off a table-leg. He got his palms
underneath him and shot to his feet again, coming all the way up,
bringing his knee up into Kaden's balls as he did, doubling the fat
man over. His hands were already in awkward fists and it was
natural as anything to begin to beat the man's head with them,
hitting so hard the skin over his knuckles split.

It had only taken a few seconds, and now the rest of Command
Central reacted. Big hands grabbed his arms, waist, legs, pulled
him away. Across from him, four game-runners had Kaden pinned as
well, shouting at him to calm down, just calm the hell down, all
right?

He did, a little. Someone handed Connor a wad of pizza-parlor
napkins to press against his nose and someone else handed him an
ice-cold can of Coke from the huge cooler at the side of the room
to press against his aching neck.

``What the hell is wrong with you?'' he choked, glaring at Kaden,
still held fast by four beefy game-runners.

``You goddamned \emph{idiot}! You brought down the whole goddamned
network. You and your stupid scripts! Do you have any \emph{idea}
how much you've cost us with your little fishing-expedition?''

Connor's anger and shock morphed into fear.

``What are you talking about?''

``Who ever wrote those damned forensics programs didn't have a
\emph{clue}. They clobbered the servers so hard, taking priority
over every other job, until the system had to kick all the players
off the games so that it could tell \emph{you} what they were
doing. I'll tell you what they were doing, Connor:
\emph{they were trying to connect to the server}.''

Connor shot a look at Bill, who had written the scripts, and saw
that the head of security had gone pale. Connor dimly remembered
him saying that the scripts were experimental and to use them
sparingly, but they had been so \emph{rewarding}, it had given him
such a thrill to sit like a recording angel over the worlds, like
Santa Claus detecting everyone who was naughty and everyone who'd
been nice --

The enormity of what he'd done hit him almost as hard as Kaden's
fist had. He had shut down three of the twenty largest economies in
the world for a period of hours. Coke ran games that turned over
more money than Portugal, Poland or Peru. That was just the P's. If
Coke's games had been real countries, it would have been an act of
war, or treason.

It was easily the biggest screwup of his career. Of his life.
Possibly the biggest screwup
\emph{in the entire history of the Coca Cola corporation}.

Command Central seemed to recede, as if the room was rushing away
from him. Distantly, he heard the game runners hiss explanations to
one another, explaining the magnitude of his all-encompassing
legendary world-beating FAIL.

Connor had never had a failure like this before. He'd screwed up
here and there on the way. But he'd never, ever, never, never --

He shook his head. The hands restraining him loosened. Stiffly, he
bent to pick up his laptop. Slivers of plastic and glass rained
down as he lifted it. He couldn't meet anyone's eyes as he let
himself out of the room.

He wasn't sure how he'd gotten home. His car was in the driveway,
so that implied that he'd driven himself, but he had no
recollection of doing so. And here he was, sitting at his
dining-room table -- grand and dusty, he ate his meals over the
sink when he bothered to eat at home at all -- and his phone was
ringing from a long way off.

Absently, he patted himself down, noticing as he did that he was
holding his car keys, which bolstered his hypothesis that he had
driven himself home. He found his phone and answered it.

``Connor,'' Ira said, ``Connor, I don't know how to tell you this --''

Connor grunted. These were words you never wanted to hear from your
broker.

``Connor are you there?''

He grunted again. Somewhere, his brain was finding some space in
which to be even more alarmed.

``Connor, listen. Are you listening? Connor, it's like this.
Mushroom Kingdom gold is \emph{collapsing,} falling through the
floor. There's no bottom in sight.''

``Oh,'' Connor said. It came out in a breathless squeak.

The broker sighed. He sounded half-hysterical. ``It's worse than
that, though. That Prince in Dubai? Turns out he was writing paper
that he couldn't honor. He's broke, too.''

``He is,'' Connor said. A million miles away, a furious gorilla was
bearing its teeth and beating its hairy fists against the insides
of his skull, screeching something that sounded like
\emph{You said it was risk-free!}

``He isn't saying so, of course.'' Now the broker sounded more than
half-hysterical. He giggled, a laugh that ran up and down several
octaves like a drunk sliding his fingers up and down a piano's
keyboard. ``He's saying things like, 'We are experiencing temporary
cash-flow difficulties that have caused us to defer on some of our
financial obligations, due to overall instability in the market.'
But Connor --'' He giggled again. ``I've been around the block. I
know what financial BS sounds like. The prince is b-r-o-k-e.''

``He is,'' Connor said.
\emph{You said it was risk-free! You said it was risk-free!}

``And there's something else.''

Connor made a tiny sound like a whimper. The broker plunged on.
``This is my last day at Paglia \& Kennedy. Actually, this may be
Paglia \& Kennedy's last day. We just got our notices. Paglia \&
Kennedy sank a \emph{lot} of money into these bonds and their
derivatives.

``Everyone else ran off to steal some office supplies but I thought
I would stand here on the deck of the Titanic and make some phone
calls to my best clients. I put nearly everything into Mushroom
Kingdom gold. Not at first, you understand. But over time, bit by
bit, the returns were just so good --''

``It was risk-free,'' Connor said, louder than he'd planned to.

``Yeah,'' Ira said. ``OK, Connor, buddy, OK. I have other calls to
make.'' Connor could tell the poor guy expected him to be grateful.
He thought he was making up for costing Connor -- how much? A
hundred and eighty thousand? Two hundred thousand? Connor didn't
even know anymore.

``Thanks for calling,'' he said. ``Thanks, Ira. Take care of
yourself.'' He could barely choke the words out, but once he had, he
actually felt a little better.

He hung up the phone and dropped it on the table, letting it
clatter. Somewhere out there, Coke's gameworlds were flickering
back to life, players logging in again, along with gold-farmers,
Webblies, Pinkertons, the whole crew. Not Connor, though. Connor
had lived in a game-world of one kind or another since he was seven
years old, and now he was willing to believe that he'd never visit
one again.

Any second now, he would be fired, he was quite sure. And maybe
arrested. And he was broke. Worse than broke -- he'd bought the
last round of securities from Paglia \& Kennedy on margin, on
borrowed money, and he owed it back. Though with the brokerage
going under they may never come and ask for it.

He drew in a deep breath and closed his eyes. Some smell -- the
sweat that soaked his shirt, the blood that caked his face, the
musty smell of the house -- triggered a strong memory of his place
in Palo Alto, near the Stanford campus, and the long, long time
he'd spent there, buying virtual assets, teetering on the brink of
financial ruin and even starvation. And just like that, he was
free.

Free of the terror of losing his job. Free of the terror of being
broke. Free of the rage at the gold-farmers. Free of the shouting,
roiling anger that was Command Central and free, finally free of
his fingerspitzengefuhl. The world was tumbling free and
uncontrolled and there wasn't a single thing he could do about it
and wasn't that \emph{fine}?

There was an old song that went
\emph{Freedom's just another word for nothing left to lose} and
Connor suddenly understood what it all meant.

When he was eight years old, he'd decided to work on video games.
It was one of those ridiculous kid-things, like deciding to be an
astronaut or a ballerina or a cowboy or a deep-sea diver. Most kids
outgrow their dreams, go on to do something normal and boring. But
Connor had held onto it, finding his way into gamespace through the
most curious of means, and he had trapped himself there. Until
today.

Now the eight-year-old who'd sent him on a quest had finally
released him from it.

He took a shower and iced his nose some more and put on a t-shirt
and a pair of baggy shorts he'd bought on holiday in the Bahamas
the year before (he'd spent most of the trip in his room, online,
logged into gamespace, keeping the fingerspitzengefuhl alive) and
opened his door.

Outside it was Atlanta. He'd lived in the city for seven years,
gone to its movie theaters and eaten at its restaurants, taken his
parents around to its tourist sites when they visited, but he had
never really \emph{lived} there. It was like he'd been on an
extended, seven-year visit. He kicked on a pair of flip-flops he
normally wore when he had to go outside to get the mail and stepped
out his door.

He walked into the baking afternoon sun of Atlanta, breathing in
the humid air that was so wet it seemed like it might condense on
the roof of his mouth and drip onto his tongue. He got to the end
of his walk and looked up and down the street he'd lived on for all
these years, with its giant houses and spreading trees and disused
basketball hoops and he started walking. No one except maids and
gardeners walked anywhere in this neighborhood. Connor couldn't
understand why. The spreading trees smelled great, there were birds
singing, even a snail inching its way across the sidewalk. In half
an hour, Connor saw more interesting new things than he had in a
month.

Oh, the feeling of it all! A lightness in his head, an openness in
his chest. Old pains in his back and shoulders that had been there
so long he'd forgotten about them disappeared, leaving behind a
comfortable feeling as striking as the quiet after a refrigerator's
compressor shuts off, leaving behind unexpected silence.

He was sweating freely, but he didn't mind. It just made the
occasional breath of wind feel that much better.

Eventually, his bladder demanded that he head home, so he ambled
back, waving at the suspicious neighbors who peered at him from
between the curtains of their vast living-room windows. As he
opened his door, he heard his phone ringing. A momentary feeling of
worry arced from his throat to his balls, like a streak of
lightning, but he forced himself to relax again and headed for the
bathroom. Whomever was calling would leave a message. There, the
voicemail had picked it up. He had to pee.

He peed.

The phone started ringing again.

He went into the kitchen and rummaged in his freezer. There was a
loaf of brown bread there -- he never could get through a whole
loaf before it went moldy, so now he bought a dozen loaves at a
time and froze them. He chipped off two slices and put them in the
toaster. There was peanut butter from the health-food store,
crunchy-style, with nothing added. While the bread was toasting, he
stirred the peanut butter with a knife, mixing the oil that was
floating on top with the ground peanuts below. He had honey, but it
had crystallized. No problem -- twenty seconds in the microwave and
it was liquid again. What he really wanted was bananas, but there
weren't any (the phone was ringing again) and he was hungry and
wanted a sandwich now. He'd get bananas later.

The sandwich was (the phone was ringing again) delicious. He needed
fresh bread though, he'd get some of that when he picked up the
bananas. Throw out the frozen (there it was again) bread. He'd eat
fresh from now on, and relish (and again) every bite.

Up until the moment that his finger pressed the green button, he
believed that he was going to switch his phone off. But his finger
came down on the green button and the anxiety sizzled up his arm
and spread out from his shoulder to his whole body as the distant
voice from the phone's earpiece said, ``Hello? Connor?''

Connor watched as his hand wrapped itself around his phone and
lifted it to his ear.

``Yes?'' his mouth said, in the old, tight Connor voice.

``It's Bill,'' the head of security said. ``Can you come into the
office?''

Connor heaved a sigh. ``I'll courier over my badge. You can pack up
my desk and ship it back. If you want to sue me, you'll have to
hire a process server and have him come out here.''

Bill's laugh was bitter and mirthless. ``We're not suing you,
Connor. We're not firing you. We need your help.''

Connor swallowed. This was the one thing he hadn't anticipated:
that his life might come back and suck him into it again. ``What the
hell are you talking about?''

``We think it's your gold-farmers,'' Bill said. ``They've got us by
the balls, and they're squeezing.''

Connor changed into his work clothes like a condemned man dressing
for his own hanging. He prayed that his car wouldn't start, but it
was a new car -- he bought a new one every year, just like everyone
else in Command Central -- and its electric motor hummed to life as
he eyeballed the retina-scanner in the sun-visor.

He drove down his street again, seeing it all through the smoked
glass of his car, the rolled up windows and air-conditioning
drowning out the birdsong and shutting out the smells of the trees
and the nodding flowers. Too fast to spot a snail or a bird.

He headed back to work.

\tb

They came for Big Sister Nor and The Mighty Krang and Justbob in
the dead of night, and this time they brought the police. The three
of them watched the police break down the door, accompanied by a
pair of sour Chinese men with the look of mainland gangsters, the
kind who came to Singapore on easy two-week tourist visas. Nor and
her friends watched the door be broken down from two Lorongs --
side-streets -- down, using a webcam and streaming the video live
to the Webblies' network and a bunch of journalists they'd woken up
as soon as they'd bugged out of the old place, warned by a
sympathetic grocer at the top of Geylang Road.

The fallback house wasn't nearly as nice as the one they'd vacated,
naturally, but the two quickly came into balance as the police
methodically smashed every piece of furniture in the place to
splinters. The Mighty Krang drew real-time annotations on the
screen as the police worked, sometimes writing in the dollar value
of the furniture being smashed, sometimes just drawing mustaches
and eye-patches on the police in the video. When the Chinese men
took out their dicks and began to piss on the wreckage, he leapt to
his trackpad, circled the members in question, drew arrows pointing
to them, and wrote ``TINY!'' in three languages before they'd
finished.

They watched as one of the policemen answered his phone, listened
in as he said, ``Hello?'' and ``What?'' and ``Where?'' and then ``Here?''
``Here?'' feeling around the place where the wall met the ceiling,
until he found the video camera. The look on his face -- a mixture
of horror and fury -- as he disconnected it was priceless.

``Priceless,'' The Mighty Krang said, and turned to his companions,
who were far less amused than he was.

``Oh, do lighten up,'' he said. ``They didn't catch us. The strikers
are striking. Mumbai and Guandong are going crazy. The New York
Times is sending us about ten emails a minute. The Financial Times,
too. And the Times of London. That's just the English papers.
Germans, French\ldots{} And the Times of India, of course, they've got a
reporter in Dharavi, and so do the Mumbai tabloids. We're six of
the top twenty YouTube videos. I've got --'' he looked down, moused
some -- ``82,361 emails from people to the membership address.''

Justbob glowered at him with her good eye. ``Matthew is trapped in
Dafen. 42 are dead. We don't know where Jie and the white boy,
Wei-Dong, are.''

Big Sister Nor reached out her hands and they each took one of
hers. ``Comrades,'' she said, ``comrades. This is the moment, the one
we planned for. We've been hurt. Our friends have been hurt. More
will be hurt when this is over.

``But people like us get hurt \emph{every single day}. We get caught
in machines, we inhale poison vapors, we are beaten or drugged or
raped. Don't forget that. Don't forget what we go through, what
we've been through. We're going to fight this battle with
everything we have, and we will probably lose. But then we will
fight it again, and we will lose a little less, for this battle
will win us many supporters. And then we'll lose \emph{again}. And
\emph{again}. And we will fight on. Because as hard as it is to win
by fighting, it's impossible to win by doing nothing.''

An alert popped up on Krang's screen, reminding him to switch a new
prepaid SIM card into his mobile phone. A second later, the same
alert came up on Big Sister Nor and Justbob's screens.

Big Sister Nor smiled. ``OK,'' she said. ``Back to work.''

They swapped SIMs, pulling new ones out of dated envelopes they
carried in money-belts under their clothes. They powered up their
phones. Both Justbob and The Mighty Krang's phones rang as soon as
they powered up.

The Mighty Krang looked down at the number. ``It's Wei-Dong,'' he
said. ``Told you he was safe.''

Justbob looked at her phone. ``Ashok,'' she said.

They both answered their phones.

\tb

Ashok knew that this time would come. For months, he'd slaved over
models of economic destruction: how much investment in junk
game-securities would it take to put the game-runners into a
position of total vulnerability? He'd modelled it a thousand ways,
tried many variables in his equations, sweated over it, woken in
the night to pace or ride his motorcycle around until the doubts
left his mind.

Somewhere out there, some distant follower of Big Sister Nor's had
convinced the Mechanical Turks to go to work selling his funny
securities. It had been easy enough to package them -- there were
so many companies that would let you roll your own custom security
packages together and market them, and all it took was to figure
out which one was most lax with its verification procedures and
create an account there and invent a ton of virtual wealth through
it. Then he logged into less-sloppy competitors and repackaged the
junk he'd created, making something that seemed a little more
legit. Working his way up the food chain, he'd gone from packager
to packager, steadily accumulating a shellac of respectability
overtop of his financial turds.

Once they had acquired this sheen, brokers came hunting for his
funny money. And since the Webblies were diverting a sizeable chunk
of game-wealth into the underlying pool, he was able to make
everything seem as though it was growing at breakneck speed -- and
it was. After all, all those traders swapping the derivatives were
driving up the prices every time they completed a sale.

Once, at about two in the morning, as Ashok watched the trading
proceed, he realized that he could simply quit the Webblies, sell
the latest batch of funny money, and retire. But he was never
tempted. He'd always known that it was possible to get rich by
trampling on the people around you, by treating them as suckers to
be ripped off. He couldn't do it.

Of course, here he was, \emph{doing it}, but this was different.
His little financial game could end well if all went according to
plan, and now it was time to see if the plan would go the way it
was supposed to.

Justbob took his call in her fractured English, which was better
than her Hindi, limited as it was to orders of battle and military
cursing. He told her that he needed to speak to Big Sister Nor, and
she asked him to wait a moment, as BSN was on the phone with
someone else at the time.

In the background, he heard Big Sister Nor conversing in a mix of
Chinese and English, flipping back and forth in a way that reminded
him of his buddies at university and the way they'd have fun mixing
up English and Hindi words, turning out puns and obscurely dirty
phrases that nevertheless sounded innocent.

He looked at the clock in the corner of his screen. It was 5AM and
outside, he could hear the birds singing. In the next room, Mala's
army fought on in tireless shifts, defending the strike. They slept
in shifts on the floor now, and there were fifty or sixty steel and
garment workers prowling the street out front, visiting other
striking sites around Dharavi with sign-up sheets, trying to
organize the workers of little five- or ten-person shops into their
unions.

He realized he was falling asleep. How long had it been since he'd
last slept for more than an hour or so? Days. He jerked his head up
and forced his eyes open and there was Yasmin before him,
raccoon-eyed beneath the hijab across her forehead. She was
frowning, her mouth bracketed by deep worry lines, another one
above the bridge of her nose. She was holding her lathi.

``Yasmin?'' he said.

She bit her lip. ``Mala is gone,'' she said. ``No one's seen her for
hours. Twelve, maybe fourteen.''

He started to say something but then Big Sister Nor spoke on the
phone, ``Ashok, sorry to keep you waiting.''

He looked to Yasmin, then back at his screen. ``One second,'' he said
to the phone.

``Yasmin, she's probably gone home to sleep --''

Yasmin shook her head once, emphatically. He felt a jolt of fear.

``Ashok?'' Big Sister Nor's voice in his ear.

``Come in,'' he said to Yasmin, ``come here. Close the door.''

He stood up and held his chair out to Yasmin and dropped into a
squat beside her, heels on the ground. He pressed the speaker
button on the phone.

``Nor,'' he said. He always felt faintly ridiculous calling this
woman ``Big Sister,'' though the Webblies seemed to relish it in the
same way they loved saying \emph{General Robotwallah}. ``I have
Yasmin with me here. She tells me that Mala is missing, has been
missing for some hours.''

There was a momentary pause. ``Ashok,'' Nor said, ``that's terrible
news. But I thought you were calling about the other thing --''

He looked at Yasmin, whose eyes were steady on him. He never talked
about the work he did for Big Sister Nor, but everyone knew he was
up to something back here.

``Yes,'' he said. ``The other thing. I need to talk to you about that.
But Yasmin is here and she tells me that Mala is missing.''

Big Sister Nor seemed to hear the gravity in his voice. She took a
deep breath, spoke in a patient voice: ``You know Dharavi better
than I do. What do you think has happened?''

He nodded to Yasmin. ``I think that Bannerjee has her,'' she said. ``I
think that he will hurt her, if he hasn't already.''

From the phone, The Mighty Krang's voice broke in. ``I have
Bannerjee's phone number,'' he said. ``From one of our people in
Guzhen. He emailed us a list of everyone in his boss's address
book.''

Ashok found his hands were in fists. He'd only met Bannerjee once,
but that was enough. The man looked like he was capable of
anything, one of those aliens who could look at a fellow human
being as nothing more than an opportunity to make money. Yasmin's
eyes were wide.

``You want to phone him?''

``Sure,'' The Mighty Krang sounded calm, even flippant, just as he
did in the inspirational videos he posted to the Webbly boards and
YouTube. ``It's worth a try. Maybe he wants to ransom her.''

``Are you joking?''

The light tone left his voice. ``No, Yasmin, I'm not joking. Look,
the Webblies are powerful. Men like Bannerjee understand that. Once
I got Bannerjee's number, I used it to get a full workup on him. We
have some leverage over him. It's possible that we can make him see
reason. And if we can't --'' He trailed off.

``We're no worse off than before,'' Big Sister Nor finished.

``When will we call him?''

``Oh, now would be good. Negotiations are always best in the small
hours. Hang on, I'll get the number.'' The Mighty Krang typed some.
``OK, let's do this.''

``OK,'' Yasmin said in a tiny voice.

``OK,'' Ashok said.

``I'll keep you two muted for him, but live for me. Remember that --
if you talk over him, I'll hear both, which might confuse me.''

``We'll mute our end,'' Ashok said. He saw that his battery was low
and fished around on his desk for a power-cable and plugged it in.
Then he muted the phone. He and Yasmin unconsciously leaned their
heads together over it, so that he could smell his sour breath and
hers, which smelled of vomit. She had been sick. He closed his eyes
and it felt as though there was sandpaper on the insides of his
eyelids.

After a few rings, a sleepy voice mumbled ``Victory to Rama,'' in
Hindi, the traditional phone salutation. It made Ashok snort
derisively. A man like Bannerjee was about as pious as a turnip. As
a jackal.

``Mr Bannerjee,'' Big Sister Nor said in accented Hindi. ``Good
morning.''

``Who is it?'' He had switched to English.

``The Webblies,'' Big Sister Nor said.

``For a Webbly,'' Bannerjee grunted, still sounding half-asleep, ``you
sound an awful lot like an underage Chinese whore. Where are you
calling from, China-Doll? A brothel in Hong Kong?''

``2,500 kilometers from HK, actually. And I'm Indonesian.''

Bannerjee grunted again. ``But you \emph{are} a whore, aren't you?''

``Mr Bannerjee, I am a busy woman --''

``A \emph{popular} whore!''

Yasmin hissed at the phone and Ashok double-checked that the mute
was on. It was.

``-- a busy woman. I've called to make you an offer.''

``I have all the whores I need,'' he said. ``Goodbye.''

``Mr Bannerjee! I'm calling to arrange for the release of Mala,'' Big
Sister Nor spoke quickly. ``And I'm sure if you think about it for
just a moment, you'll realize that there's plenty I can offer you
for her safe return.''

Bannerjee said, ``Mala is missing?'' in a tone that could have won a
medal in the unconvincing Olympics.

``Stop playing games, please. You know that we're not the police.
We're not going to have you arrested. We just want her back.''

``I'm sure you do. She's a delightful girl.''

Yasmin was grasping her opposite elbows so hard her knuckles were
white. Ashok had his fists bunched in the fabric of his
trouser-legs. He made himself loosen them. But Big Sister Nor just
continued on, as though she hadn't heard.

``I'm sure you've seen what's happened to the gold markets. Prices
are on fire. No one can get any gold out of the gold farms, thanks
to my Webblies. If you could promise a farmer access to one spot,
without harassment, just think of what you could charge.''

Bannerjee chuckled. ``And all I have to do is find Mala for you and
give her to you and you will guarantee this to me, is that right?''

``That's the shape and size of it.''

``You will, of course, honor your end of the bargain once I've found
her for you.''

``Of course.''

There was a long silence. Finally, Big Sister Nor spoke again.

``I understand your scepticism. I can give you my word of honor.''

Bannerjee made a rude sound, like a wet fart. ``How about this: I
get the gold out of the game, then I find Mala for you.''

Ashok hated this game he was playing, pretending that he didn't
have Mala, but he could somehow find her. He wanted to crawl
through the phone and strangle the man.

``How about if we just get you some gold?'' It was The Mighty Krang
speaking.

``Oh, there's more of you? Are you also an Indonesian whore 2500
kilometers from Hong Kong, or are you dialled in from some other
exotic locale?''

``We can get the gold out of the game faster than anyone you could
hire. All the best gold farmers are in the union. The scabs they've
got working in the shops right now are so crap they'll probably
screw up and get themselves banned.'' Ashok loved that Krang wasn't
playing Bannerjee's taunting game either.

Bannerjee snorted. ``That's not bad,'' he said.

``We could use an escrow service, one we both agree on.'' The
gold-markets ran on escrow services, trustworthy parties that would
hold gold and cash while a deal was closing, working for a small
percentage.

``And you would return Mala to us?''

``I would do everything I could to find the poor girl and get her
into your hands.'' Gold, silver and bronze medals in the 100-yard
slime.

They dickered over price and timing -- Mala ended up promising him
a 300,000 Svartalfaheim runestones -- and Krang disconnected
Bannerjee.

``Brilliant,'' Ashok said, trying to force some enthusiasm into his
voice, while inside he was quavering at the thought of Mala in the
hands of Bannerjee.

``Very good,'' Yasmin said.

``Yes, yes,'' Big Sister Nor said. ``And your team will get the
runestones for us, and I'm sure you'll do it quickly and well
because she is your general. All our problems should be that easy
to solve. Now, Ashok, how have you done with your complicated
problem?''

Ashok looked at Yasmin, who showed no signs of leaving.

``I think we're there. The trick was to create a situation where
they \emph{can't} put things back together without our help. Our
accounts control the gold underneath so many of these securities
that if they kick us all off, they'll create a massive crash, both
in-game and out-of-game. At the same time, they can't afford to
leave us running around freely, because there's a hundred ways we
could crash the system, too, from resigning in a huge group all at
once to repeating the Mushroom Kingdom job.'' Crashing the Mushroom
Kingdom securities had been easy -- Mushroom Kingdom was already
riddled with scams that had been flying under the radar of
Nintendo's incompetent economist and security teams. Ashok had used
Webblies and some of the Mechanical Turks that Big Sister Nor had
supplied through her mysterious contact on the inside, building up
a catalog of all the other scams and then giving them a nudge here
and a shove there, using Webblies to produce gold on demand when
necessary.

He'd gone into it thinking that he'd never manage to take on the
Mushroom Kingdom economy, believing that the security would be
all-knowing and all-powerful. But in truth, it had all been held
together with twine and wishful thinking, straining at the seams,
and it had only taken a little pushing and pulling to first make it
swell to unheard-of heights, and then to explode gloriously.

``But we couldn't afford to repeat the Mushroom Kingdom job. There
was no way we could have pulled that one out of the nosedive, once
it started. It was doomed from the start. With Coca-Cola's games,
we have to be able to promise to put it all back together again if
they play cricket with us.'' Talking about his work made him forget
momentarily about Mala, let the iron bands around his chest loosen,
just a little.

``If we had kept things on schedule, it would have been much easier.
But you know, with things all chaotic, I had to rush things. I've
been dumping our gold reserves on the market for hours now, which
has sent the market absolutely crazy, especially after they had
that crash. How on Earth did you manage that?''

Big Sister Nor snorted. ``It wasn't me. We're not sure if they got
hacked, or some kind of big crash. It \emph{was} well-timed,
though.''

``Would you tell me if you \emph{had} caused it?''

Yasmin looked faintly shocked.

``Ashok,'' BSN said, with mock sternness, ``I tell everyone anything I
think they need to know, and I usually tell them anything
\emph{they} think they need to know. We're not in the secrets
business around here.''

That made Ashok pause. He'd always thought of the operation as
being shrouded in secrecy. Certainly Big Sister Nor had never
volunteered any details about her contact with the Mechanical Turks
-- but then, he'd never asked, had he? Nor had he ever asked if he
could discuss his project with Mala's army. He shook his head. What
if the secrecy had been all in his mind?

``OK,'' he said. ``Fine. The problem is this: if I had enough time --
if I had the time we'd planned on -- I'd be in a position to take
Svartalfaheim right up to the brink of collapse and then either
save it or let it collapse. It all comes down to how much gold we
had in our reserves, and how much of the trading we controlled.

``But I've had to rush the schedule, which means that I can't give
you both. I can bring the economy to the brink of ruin, but when I
do, I need to know in advance whether we're going to let it blow
up, or whether we're going to let it recover. I can't decide
later.'' He swallowed. ``I think that means we have to destroy it. I
still have Zombie Mecha and Clankers underway. We can show them our
force by taking out Svartalfaheim and then threaten to take out the
other two.''

``Why do you want to do it that way?''

He shook his head, realized she couldn't see him. ``Listen, they're
not going to give in to you. You're going to go in there and start
giving them orders and they're going to assume you're some
ridiculous third-world crook. They're going to tell you to get
lost. If you make a threat and you can't make good on it, that'll
be the last time you hear from them. They'll never take you
seriously after that.''

Big Sister Nor clucked her tongue. ``Are we so easy to dismiss?''

``Yes,'' Ashok said. ``\emph{I} know what the Webblies can do. But
they don't. And they won't, until we show them.''

``We have Mushroom Kingdom for that.''

That stopped him. ``Yes, that's true of course. But that was so
\emph{easy} --''

``They don't know that. They don't know anything about us, as you
point out. So yes, maybe they'll assume we're weak and maybe
they'll assume we're strong. But one thing I know is, if they give
us what we want and \emph{then} we destroy their game, they'll
never trust us again.''

``So you're saying you want me to set this all up so that we can't
make good on our threat?''

``If we have to choose --''

``We do.''

``Then yes, that's just what I want, Ashok. I'll just have to be
sure that whatever happens, we don't need to carry out our
threat.''

``OK,'' Ashok said. ``I can do that.''

``Good. And Ashok?''

``Yes?''

``I need you to speak with them,'' she said. ``With who ever they get
to talk to us. I'll be on the call, too, of course. But you need to
talk to them, to explain to them what we've done and what we can
do.''

Ashok swallowed. ``I'm not good at that sort of talk --''

Yasmin made a rude noise. ``Don't listen to him,'' she said. ``You
talked the steelworkers and the garment-workers into coming to
Dharavi!''

``I did,'' he said. ``I didn't think it would work -- they'd never
listened before. But once I explained what kind of situation you
were all in, the thugs, the violence, told them that all of Dharavi
would know if they came down --''

``Once you really believed in it,'' Big Sister Nor said. ``That's the
difference. I've heard you talk about the things you love, Ashok.
You are very convincing when it comes to that. The difference
between all the conversations you had with them before and the last
one is that you came to them as a Webbly last time, not as someone
who was playing a game to make himself feel like he was doing
something important.'' The criticism took him off guard and pierced
him. He \emph{had} been playing a game at first, taken with his own
cleverness at the vision of kids all over the world running circles
around the tired old unions he'd hung around with all his life. But
now, it wasn't a game anymore. Or rather, it \emph{was} a game, but
it was one that he took deadly serious.

``OK,'' he said. ``I'll talk to them.''

\tb

Now it was Jie's turn to watch Wei-Dong, as he typed furiously at
his keyboard, reaching out to hundreds of Mechanical Turks who'd
said, ``Yes, yes, we're on your side; yes, we're tired of the crummy
pay and of always having the threat of being fired over our heads.''
He reached out to them and what he told them all was:

\emph{Now}

Now it begins, now we are ready, now we move. He sent them links to
the YouTube videos of the protests in China, the picket lines in
India, the workers who'd begun to walk off the job in Indonesia and
Vietnam and Cambodia, saying, ``Us too, us all together, us too.''

Only it wasn't working the way it was supposed to. The Mechanical
Turks had been happy enough to seed a little disinformation, to
pass on some weird-sounding stock-tips or to look the other way
when the Webblies were fighting the Pinkertons, but they balked at
going to Coke and saying, ``We demand, we want, we are all one.''
Just from their typing, he could feel their fear, the terror that
they might find themselves without a job next month, that they
might be the only ones who stood up.

But not all of them. First one, then five, then fifty, and finally
over a hundred of his Turks were with him, ready to put their names
to a list of dues-paying Webblies who wanted to bargain as a group
with Coke for a better deal. That was only 20 percent of what he'd
bargained for, but they still accounted for 35 of the top fifty
performers on the Webbly leaderboards.

He kept up a running account for Jie, muttering in Chinese to her
between messages and quick voice calls.

``Now what?'' she said. She was jammed up in a corner of the room,
resting on her sweater, which she'd spread out over the filthy
mattress, eyes barely open.

``Now I call Coke,'' he said. He had talked this over with Big Sister
Nor a dozen times, iterating through the plan, even role-playing it
with The Mighty Krang playing the management on the other end. But
that didn't mean that he was calm -- anything but, he felt like he
might throw up at any instant.

``How is that supposed to work?''

He closed his eyes, which were burning with exhaustion and dried
tears. ``Are you hungry?''

She nodded. ``I was thinking of going upstairs for some dumplings,''
she said.

``Bring me some?''

She got up and walked unsteadily to the door. She pulled a compact
out of her purse and looked at herself, made a face, then said,
``Tea?''

He'd drunk tea for years, but right now he needed coffee, no matter
how American that made him feel. ``Coffee,'' he said. ``Two coffees.''

She smiled a sad little smile. ``Of course. I'll bring a syringe,
too.''

But he was already back at his computer, screwing in his borrowed
earwig, dialling in on the employee-only emergency number.

``Co' Cola Games level two support, this is Brianna speaking,'' the
voice was flat, American, bored, female, Hispanic.

``I need to speak to someone in operations,'' he said. ``This is
Leonard Goldberg, Turk number 4446E764.''

``Hello, Leonard. Can I have the fifth letter of your security
code?''

He had to think hard for a moment. Like the name Leonard Goldberg,
like his entire American life, the security code he used to
communicate with his employers seemed like it was in a distant
fairytale land. ``K for kilo,'' he said. ``No, wait, Z for Zulu.''

``And the second letter?''

``A for alpha.''

``OK, Leonard, what can I do for you?''

``I need to speak to someone in operations,'' he said. ``Level four,
please.''

``What do you need to speak to operations about, please?'' He could
hear her clicking away at her screen, looking up the escalation
procedures. Technically it wasn't supposed to be possible to go
from level two support to level four without going through level
three. But the entire escalations manual was available in the
private discussion forums on the unofficial Turk groups if you knew
where to look for them.

``I, uh, I think I found someone, who was, like, a pedophile? Like
he might have been trying to get some kids to give him their RL
addresses?'' Kid-diddlers, mafia, terrorists or pirates, the four
express tickets to level four support. Anything that meant calling
in the federal cops or the international ones. He figured that a
potential pedophile would have just the right amount of ick to get
him escalated without the call being sent straight to the cops.

Brianna typed something, read something, muttered ``Just a minute,
hon,'' read some more. ``OK, level four it is.'' She parked him on
hold.

Jie came back with a styrofoam clamshell brimming over with
steaming dumplings and a bottle of nuclear-hot Vietnamese rooster
sauce and a pair of chopsticks. She picked one up, blew on it,
dipped it in the sauce and held it out to him. He popped it into
his mouth and chewed it, blowing out at the same time to try to
cool off the scalding pork inside. They shared a smile, then the
call started up again.

``Hello, Coca Cola Games, level four ops, Gordon speaking, your name
please.''

Leonard went through the authentication routine with Gordon again,
his password coming more easily to him this time.

``All right, Leonard, I hear you found a pedophile? One moment while
I pull up your interaction history --''

``Don't bother,'' Wei-Dong said, his pulse going so fast he felt like
he was going to explode. ``I made that up.''

``Did you.'' It wasn't really a question.

``I need to speak to Command Central,'' he said. ``It's urgent.''

``I see.''

Wei-Dong waited. This Gordon character was supposed to get angry or
sarcastic, not quiet. The pause stretched until he felt he
\emph{had} to fill it. ``It's about the Webblies, I have a message
for Command Central.''

``Uh huh.''

Oh, for Christ's sake. ``Gordon, listen. I know you think I'm just a
kid and you probably think I'm full of crap, but I
\emph{need to speak to Command Central right now.} I promise you,
if you don't connect me with them, you'll regret it.''

``I will, will I? Well, listen, Leonard, I've been looking at your
interaction history and you certainly seem like an efficient
worker, so I'm going to go easy on you. \emph{You} can't talk to
Command Central. Period. Tell me what you want, and I'll see that
someone gets back to you.''

\emph{This} was something Wei-Dong had prepared for. ``Gordon,
please relay the following to Command Central. Do you have a pen?''

``Oh, this is \emph{all} being recorded.'' There was the sarcasm he'd
been waiting for. He was getting under his skin. Right.

``Tell them that I represent the Industrial Workers of the World
Wide Web, Local 56, and that we need to speak with Coca Cola
Games's Chief Economist immediately in order to avert a collapse on
the scale of the Mushroom Kingdom disaster. Tell them that we have
two hours to act before the collapse takes place. Did you get
that?''

``What? You're kidding --''

``I'm serious. I'll hold while you tell them.'' He muted the
connection and immediately dialled back to Singapore and told
Justbob what had happened. She assured him that they'd get their
economist on the line as quickly as possible and put him on hold.
He bridged both calls into his earpiece but isolated them so that
they wouldn't be able to hear him, then told Jie what had just
happened.

``When can I interview you about this for the radio show?''

He swallowed. ``I think maybe never. Part of this story can probably
never be publicly told. We'll ask BSN, OK?''

She made a face, but nodded. And now there was Gordon.

``Leonard, you there, buddy?''

``I'm here,'' he said.

``You're logging in from a lot of proxies lately. Where exactly are
you located? We have you in LA.''

``I'm not in LA,'' Wei-Dong said, grinning. ``I'm a little ways off
from there. You don't need to know where. How's it coming with
Command Central, Gordon? Time's a-wastin'.'' Keep the pressure up,
that was a critical part of the plan. Don't give them time to
think. Get them to run around like headless chickens.

``I'm on it,'' Gordon said. He swallowed audibly. ``Look, you're not
serious, are you?''

``You saw what happened to Mushroom Kingdom, right?''

``I saw.''

``OK then,'' Wei-Dong said. He'd been warned not to admit to any
wrongdoing personally.

``You're serious?''

``You know, 15 minutes have gone by already.''

Another swallow. ``I'll be right back.''

A new line cut in, different background noise, chaotic, lots of
chatter. Gordon had probably been a teleworker sitting in his
underwear in his living room. This was different. This was a room
filled with angry, arguing people who were typing on keyboards like
machineguns.

``This is William Vaughan, head of security for Coca Cola Games.
Hello, Leonard.''

``Hello, Mr Vaughan.'' Leonard said. Be polite. That was part of the
plan, too. Real operators were grownups, polite, businesslike. ``May
I speak with Connor Prikkel, please?'' Prikkel's name had been easy
enough to google. Wei-Dong had spent some time watching videos of
the man at conferences. He seemed like an awkward, super-brainy
academic type run to fat. He typed a quick one-handed message to
Justbob: \emph{Got cmd ctnrl, where r u?}

``Mr Prikkel is away from the office. I have been asked to speak
with you in his stead.''

He had prepped for this, too. ``I'm afraid that I need to talk with
Connor Prikkel personally.''

``That's not possible,'' Vaughan said, sounding like he was barely
holding onto his temper.

``Mr Vaughan,'' Wei-Dong said. He hadn't spoken this much English for
weeks. It was weird. He'd started to think in Chinese, to dream in
it. ``I don't know if uh, Gordon told you what I told him --''

``Yes, he did. That's why you're talking to me now.''

``Mr Prikkel is qualified to evaluate what I have to say to him. I'm
not qualified to understand it. And no offense, I don't think you
are either.''

``I'll be the judge of that.''

Justbob sent him a message back: \emph{5 min}.

``I've got a better idea,'' Wei-Dong said. ``You get Mr Prikkel and
call me back. I'll leave you a voice-chat ID. You can listen in on
the call.''

``How about if I just trace where you're calling \emph{us} from and
we call the police? Leonard, kid, you are working on my last good
nerve and I'm about to lose it with you. Fair warning.''

Wei-Dong tisked. He was starting to enjoy this. ``Mr Vaughan, here's
the thing. In --'' he looked at the clock -- ``about ten minutes,
you're going to see total chaos in your gold markets. All those
contracts that Coke Games has written for gold futures are going to
start to slide into oblivion. You can spend the next ten minutes
trying to trace me, but you're not going to find me, and even if
you do, you're not going to be able to do anything about it,
because I am an ocean away from the nearest police force that will
give you the time of day.'' The security man started to choke out a
response, but Wei-Dong kept talking. ``I'd prefer \emph{not} to
destroy the game. I love it. I love playing all these games. You
have my record there, you know it. We all feel that way, all the
Webblies. It's where we go to work every day. We \emph{want} it to
succeed. But we want that to happen on terms that are fair to us.
So believe me when I tell you that I am calling to strike a bargain
that you can afford, that we can live with and that will save the
game and get everything back on track by the end of the day.'' He
looked at the clock again, did some mental arithmetic. ``By tomorrow
morning, your time, that is.''

He could almost hear the gears turning in Vaughan's head. ``You're
in Asia, somewhere?''

``Is that the only thing that you got from that?''

He made a little conciliatory snort. ``You're a long way from home,
kid. Ten minutes, huh?''

Wei-Dong said, ``Eight, now. Give or take.''

``That's some pretty impressive economic forecasting.''

``When you've got 400,000 gold farmers working with a few thousand
Mechanical Turks, you can do some pretty impressive things.'' The
numbers were all inflated. But Vaughan would assume they were. If
Wei-Dong had given him the real numbers, he'd have underestimated
their strength. He liked how this was going.

\emph{2 min more} from Justbob.

``OK, Vaughan, here's how Mr Prikkel can reach me. Sooner, rather
than later.'' He named the ID and the service, one that was run out
of the Mangalore Special Economic Zone. It was pretty reliable and
easy to sign up for, and they supported strong crypto and didn't
log connections. He'd heard that it was a favorite with diplomats
from poor countries that couldn't run their own servers.

``Wait --''

``Call me!'' he said, and gave him the details once more.

\emph{They'll call me back} he typed to Justbob.
\emph{Our guy wasn't there.}

Justbob called him right away, and he heard The Mighty Krang and
Big Sister Nor holding another conversation in the background. ``You
hung up?''

``It wasn't the right guy. I think he was away, maybe on holidays or
something. They'll get him on the phone. no worries.'' But Justbob
sounded worried, and he didn't like that. He shrugged mentally.
He'd done the best he could, using his best judgement. He'd been
shot at, seen his friend killed. He'd smuggled himself halfway
around the world. He'd earned some autonomy.

He ate some of the now-cold dumplings and tried not to worry as the
time stretched out. Ten minutes, fifteen minutes. Justbob sent more
and more impatient notes. Jie fell asleep on the disgusting
mattress, her sweater spread out beneath her head, her face girlish
and sad in repose.

Then his computer rang.

``Hello?'' Texting, \emph{Phone.}

``This is Connor Prikkel. I understand you needed to speak to me?''

\emph{Now} he texted and clicked the button that pulled Justbob and
her economist onto the call.

\tb

No one in Command Central would meet Connor's eye when he came back
into the office, his nose swollen and his eyes red and puffy. He
grabbed a spare computer from the shelves by the door -- smashed
laptops weren't exactly unheard-of in the high-tension environment
of Command Central -- and plugged it in and powered it up.

``The markets are going crazy,'' Bill said in a low voice, while
around them, Command Central's denizens -- minus Kaden, who seemed
to have been removed for his own good -- made a show of pretending
not to listen in. ``Huge amounts of gold have hit the market in the
past ten minutes, and the price is whipsawing down.''

Connor nodded. ``Sure, our normal monetary policy has had to assume
that a certain amount of gold would be entering the system from
these characters. When they stopped the flow a couple weeks ago, we
had to pick up production to keep inflation down. I had assumed
that they were too busy fighting to mine any more gold, but it
looks like they spent that time building up their reserves. Now
that they're dumping it --''

``Can you do something about it?''

Connor thought. All the peace and serenity he'd attained just an
hour ago, when he was a man with nothing to lose, was melting away.
He had the curious sensation of his muscles returning to their
habitual, knotted states. But a new clarity descended on him. He'd
been thinking of the Webblies as a pack of gang-kids, fighting a
gang-war with their former bosses. This business, though, was
sophisticated beyond anything that some gangsters would kick up. It
was an act of sophisticated economic sabotage.

``I'd better talk to this kid,'' he said, quickly paging through the
data, setting up feeds, feeling the return of his
fingerspitzengefuhl.

Bill made a sour face. ``You think they're for real?''

``I think we can't afford to assume they aren't.'' The voice was
someone else's. He recognized it: the voice of a company man doing
the company's business.

A few minutes later, he said, ``This is Connor Prikkel. I understand
you needed to speak to me?''

``Mr Prikkel, it is very good to speak with you.'' The voice had a
heavy Indian accent, and the background was flavored with the
unmistakable sound of gamers at their games, shooting, shouting.

Bill, listening in with his own earpiece, shook his head. ``That's
not the kid.''

``I'm here too.'' This voice was young, unmistakably American. When
it cut in, the background changed, no gamers, no shouting. These
two were in different rooms. He had an intuition that they might be
in different \emph{countries}, and he remembered all the battles
he'd spied upon in which the sides were from all over Asia and even
Eastern Europe, South America and Africa.

``Mr Prikkel -- Doctor Prikkel,'' Connor supressed a laugh. The PhD
was purely honorary, and he never used it. ``My name is Ashok
Balgangadhar Tilak. Allow me to begin by saying that, having read
your publications and watched dozens of your presentations, I
consider you to be one of the great economics thinkers of our
age.''

``Thank you, Mr Tilak,'' Connor said. ``But --''

``So it is somewhat brash of me to say what I am about to say.
Nevertheless, I will say it: We own your games. We control the
underlying assets against which a critical mass of securities have
been written; further, we control the substantial number of those
securities and can sell them as we see fit, through a very large
number of dummy accounts. Finally, we have orders in ourselves for
many of the sureties that you have used to hedge this deal, orders
that will automatically execute should you try to float more to
absorb the surplus.''

Connor typed furiously. ``You don't expect me to take your word for
this?''

``Naturally not. I expect you to look to the example of Mushroom
Kingdom. And to the turmoil in Svartalfaheim Warriors. Then I'd
suggest that you cautiously audit the books for Zombie Mecha and
Clankers.''

``I will.'' Again, that company man's voice, from so far away. The
feeds were confirming it, though, the trading volume was insane,
but underneath it all there was a sense of \emph{directedness}, as
though someone were making it all happen.

``Very good.''

``Now, I suppose there's something coming here. Blackmail, I'm
guessing. Cash.''

``Nothing of the sort,'' said the Indian man, sounding affronted.
``All we're after is peace.''

``Peace.''

``Exactly. I can undo everything we've done, put the markets back
together again, stop the bleeding by unwinding the trades very
carefully and very gently, working with you to make a soft landing
for everyone. The markets will dip, but they'll recover, especially
when you make the announcement.''

``The announcement that we've made peace with you.''

``Oh yes,'' Ashok said. ``Of course. Your employers expect that you
can run your economy like a toy train set, on neat rails. But we
know better. Gold-farming is an inevitable consequence of your
marketplace, and that pushes the train off the rails. But imagine
this: what if your employer were to recognize the legitimacy of
gold farming as a practice, allowing our workers to participate as
legitimate actors in a large and complex economy. Our exchanges
would move above-ground, where you could monitor them, and we would
meet regularly with you to discuss our membership's concerns and
you would tell us about your employers' concerns. There would still
be underground traders, of course, but they would be pushed off
into the margins. Every decent farmer in the world wants to join
the Webblies, for we represent the best players and everyone knows
it. And we'll be at every non-union farm-site in every game,
talking to the workers about the deal they will get if they band
with us.''

``And all we have to do is\ldots{} what?''

``Cooperate. Union gold that comes out of Coke's games will be
legitimate and freely usable. We'll have a cooperative that buys
and sells, just like today's exchange markets, but it will all be
above-board, transparently governed by elected managers who will be
subject to recall if they behave badly.''

``So we replace one cartel with another one?''

``Dr Prikkel, I wouldn't ever ask such a thing of you. No, of course
not. We don't object to other unionized operations in the space. I
have colleagues here from the Transport and Dock Workers' Union who
are interested in organizing some of these workers. Let there be as
many gold exchanges as the market can bear, all certified by you,
all run by the workers who create them.''

``What about the \emph{players}, Mr Tilak? Do they get a say in
this?''

``Oh, I think the players have already had their say. After all,
whom do you suppose is \emph{buying} all this gold?''

``And you expect me to make all this happen in an hour?''

The American kid broke in. ``45 minutes now.''

``Of course not. Today, all we seek is an agreement
\emph{in principle}. Obviously, this is the kind of thing that Coca
Cola Games's board of directors will have to approve. However, we
are of the impression that the board is likely to pay close
attention to any recommendations brought to it by its chief
economist, especially one of your standing.''

Connor found himself grinning. These kids -- not just kids, he
reminded himself -- were gutsy. And what's more, they were
\emph{gamers}, something that was emphatically \emph{not} true of
CCG's board, who were as boring a bunch of mighty captains of
industry as you could hope to find. ``Is that it?''

``No.'' It was the American kid again. He consulted his notes.
Leonard Goldberg. In LA. Except Bill was pretty sure this kid was
in Asia somewhere. He suspected there was a story in there.

``Hello, Leonard.''

``Hi, Connor. I'm emailing you a list of names right now.''

``I see it.'' The message popped up in his public account, the one
that was usually filtered by an intern before he saw it. He grabbed
it, saw that it had been encrypted to his public key, decrypted it.
It was a list of names, with numbers beside them. ``OK, go ahead.''

``That's the names of Turks who've joined the Webblies.''

``You've got Turks who want to moonlight as gold farmers?''

``No.'' The boy said, speaking as though to an idiot. ``I've got Turks
who want to join a union.''

``The Webblies.''

``The Webblies.''

Connor snorted. ``I see. And is this union certified under US labor
law? Have you considered the fact that you are all independent
contractors and not employees?''

The boy cut in. ``Yes, yes, all of that. But these are your best
Turks, and they're Webblies, and we're all in it together.''

``You know, they'll never go for it.''

``Your teamsters are unionized. Your \emph{janitors} are unionized.
Now your Mechanical Turks are --''

``Son, you're not a union. Under US law, you're nothing.''

The Indian man cleared his voice. ``That is all true, but this is
likewise true of IWWWW members around the world in all their
respective countries. Many countries prohibit \emph{all} unions.
And we ask you to recognize these workers' rights.''

``We're not those workers' employers.''

``You claim you're not \emph{our} employers either,'' said the boy,
with a maddening note of triumph in his voice. ``Remember? We're
'independent contractors', right?''

``Exactly.''

``Dr Prikkel, let me explain. The IWWWW is open to all workers,
regardless of nationality or employment, and it will work for all
those workers' rights, in solidarity. Our gold farmers will stand
up for our Mechanical Turks, and vice versa.''

``Goddamned right,'' said the boy. ``An insult to one --''

``Is an insult to all. The gold farmers have a modest set of
demands: modest benefits, job security, a pension plan. All the
same things that we plan on asking our farmers' employers for.
Nothing your division can't afford.''

``Are you saying that your demands are contingent on recognizing the
demands from Mr Goldberg's friends.''

``Precisely.''

``And you will destroy the economy of Svartalfaheim Warriors in 45
minutes --''

``38 minutes,'' said the kid.

``Unless I agree \emph{in principle} that we will do this?''

``You have summed it all up admirably,'' said the Indian economist.
``Well done.''

``Can you give me a minute?''

``I can give you 38 minutes.''

``37,'' said the kid.

He muted them, and he and Bill stared at each other for a long
time.

``Is this as crazy as it sounds?''

``Actually, the crazy part is that it's not all that crazy.
Impossible, but not crazy. We already let lots of third parties
play with our economies -- independent brokers, the people who buy
and sell their instruments. There's no technical reason these
characters can't be a part of our planning. Hell, if they can do
what they say, we'll be way more profitable than we are now.

``For one thing, we won't need to crash the servers tracking them
all down.''

Connor grimaced. ``Right. But then there's the impossible part.
Leaving out the whole thing about the Turks, which is just
\emph{crazy}, there's the fact that the board will never, ever,
never, never --''

Bill held a hand up. ``Now, that's where I disagree with you. When
you meet with the board, you're always trying to sell them on some
weird-ass egghead financial idea that makes them worry that they're
going to lose their life's savings. When I go to them, it's to ask
them for some leeway to fight scammers and hackers. They understand
scammers and hackers, and they say yes. If we were to ask them
together --''

``You think this is a good idea?''

``It's a better idea than chasing these kids around gamespace like
Captain Ahab chasing the white whale. The formal definition of
insanity is doing the same thing repeatedly but expecting a
different outcome. It's time we tried something different.''

``What about the Turks?''

``What about them?''

``They're looking for --''

``They're looking to take about half a percent out of the company's
bottom line, if that. We spend more on your first-class plane
tickets to economics conferences every year than they want. Big
freakin' deal.''

``But if we give in on this thing, they'll ask for more.''

``And if we don't give in on this, we're going to spend the next
hundred years chasing Chinese and Indian kids around gamespace
instead of devoting our energy to fighting \emph{real} ripoffs and
hacker creeps. Security is always about choosing your battles.
Every complex ecosystem has parasites. You've got ten times more
bacteria cells than blood cells in your body. The trick with
parasites is to figure out how to co-exist with them.''

``I can't believe I'm hearing you say this.''

``That's because I'm not a gamer. I don't care who wins. I don't
care who loses. I'm a security expert. I care about what the costs
are to secure the systems that I'm in charge of. We can let these
kids 'win' some little battles, pay the cost for that, and save ten
times as much by not having to chase 'em.''

Connor shook his head. ``What about them?'' he said, rolling his eyes
around the room to encompass the rest of Command Central, most of
whom were openly eavesdropping now.

Bill turned to them. ``Hands up: who wants to make and run totally
kick-ass games that make us richer than hell?'' Every hand shot up.
``Who wants to spend their time chasing a bunch of skinny poor kids
around instead of just finding a way to neutralize them?'' A few
hands stayed defiantly in the air, among them Kaden, who had come
back into the room while Connor was on the phone and was now
glaring at both of them. Bill turned back to Connor. ``I think we'll
be OK,'' he said. He jerked his head over his shoulder and said,
loudly, ``Those goons are so ornery they'd say no if you asked them
whether they wanted a lifetime's supply of free ice-cream.''

\tb

300,000 runestones hadn't seemed like much when Yasmin started.
After all, the gold was for Mala, and Mala was all she could think
of. And she had Mala's army on her side, all of them working
together.

But it had been days since she'd slept properly, and there were
reporters every few minutes, pushing into Mrs Dibyendu's cafe with
their cameras and recorders and pads and asking her all sorts of
mad questions and she had to keep her temper and speak modestly and
calmly with them when every nerve in her body was shrieking
\emph{Can't you see how busy I am? Can't you see what I have to do?}
But the army covered itself with glory and not one soldier lost his
or her temper, and the press all marvelled at them and their
curious work.

At least the steelworkers and garment workers had the sense not to
interrupt them, and they were mostly busy with their organizing
adventures in Dharavi to bother them anyway. The story of how
they'd saved this gang of Dharavi children from bad men with
weapons had spread to every corner, and the workers they'd inspired
to walk off the job were half in awe of them.

Piece by piece, though, they were able to build the fortune. Yasmin
found them an instanced mission with a decent payoff, one that
three or four players could run at a time, and she directed them
all into it, sending them down the caverns after the dwarves and
ogres below in gangs, prowling up and down the narrow, blisteringly
hot aisles between the machines, pointing out ways of getting the
work done faster, noting each player's total, until, after a
seeming eternity, they had it all.

``Ashok,'' she said, banging unannounced into his office. He was bent
over his keyboard, earwig screwed in, muttering in English to his
Dr Prikkel in America. He held up a hand and asked the man to
excuse him -- she hated how subservient he sounded, but had to
admit that he'd been very cool when the negotiations had been
underway -- and put him on mute.

``Yasmin?''

``We have Mala's ransom,'' she said.

``Yes,'' he said, ``of course.'' He sent a quick message to the central
cell in Singapore and got Bannerjee's number, then quickly dialled
it on speaker. Bannerjee answered, this time in a much less fuzzy
and sleep-addled voice.

``Victory to Rama!''

``We have your money,'' Ashok said. ``Our team are delivering it to
the escrow's hut now. You can check for yourself.''

``So serious, so businesslike. It's only a game, friend -- relax!''

Yasmin felt like she might throw up. The man was so\ldots{}\emph{evil}.
What made a man that bad? She understood, really understood, how
Mala must feel all the time. A feeling like there were people who
\emph{needed} to be \emph{punished} and she was the person who must
do it. She pushed the feeling down.

``All right, good. I see that it is there. I will tell you where to
find your friend when you tell the escrow agent to release the
money, yes?''

Ashok waggled his chin at the phone, thinking hard. Yasmin suddenly
realized something she should have understood from the beginning:
escrow agent or no, either they were going to have to trust
Bannerjee to let Mala go after they released the money, or
Bannerjee would have to trust them to release the money after he
gave them Mala. Escrow services worked for cash trades, not for
ransoms. She felt even sicker.

``You release Mala first and --''

``Oh, come on. Why on Earth would I do that? You hold me in so much
contempt, there's no way you'll give me what you've promised. After
all, you can always spend 300,000 runestones. I, on the other hand,
have no particular use for a disrespectful little girl. Why
wouldn't I tell you where to find her?''

Ashok and Yasmin locked eyes. She remembered the last time she'd
seen Mala, how tired she had been, how thin, how pained her limp.
``Do it,'' she said, covering the mic with her hand.

``The passphrase for the escrow is 'Victory to Rama','' Ashok said,
his tone wooden.

Bannerjee laughed loudly, then put them on hold, cutting them off.
After a moment, Ashok looked at his screen, watching the alerts.
``He's taken the money.'' They waited a minute longer. Another
minute. Ashok redialled Bannerjee.''

``Victory to Rama,'' the man said, with a mocking voice. Right away,
Yasmin knew that he wouldn't give them Mala.

``Mala,'' Ashok said.

``Piss off,'' Bannerjee said.

``Mala,'' Ashok said.

``One million runestones,'' Bannerjee said.

``Mala,'' Ashok said. ``Or else.''

``Or else what?''

``Or else I take everything.''

``Oh yes?''

``I will take 30,000 now. And I will take 30,000 more every five
minutes until you give us Mala.''

Bannerjee began to laugh again, and Ashok cut him off again, then
transferred back to his American at Coca Cola.

``Dr Prikkel,'' he said. ``I know we're busy rescuing the economy from
ruin, but I have a small but important favor to ask of you.''

The American's voice was bemused. ``Go ahead.''

Ashok gave him the name of the toon that Bannerjee had sent to the
escrow house. ``He has kidnapped a friend of ours and won't give her
back.''

``Kidnapped?''

``Taken her into captivity.''

``In the game?''

``In the world.''

``Jesus.''

``And Rama too. We paid the ransom but --''

Yasmin stopped listening. Ashok clearly thought he was the
cleverest man who ever walked God's Earth, but she'd had enough of
games. She sank down on her heels and regarded the dirty floor, her
eyes going in and out of focus from lack of sleep and food.

Gradually, she became aware that Ashok was talking to Bannerjee
again.

``She is at Lokmanya Tilak Municipal General. She was brought to the
casualty ward earlier today, without any name. She should still be
there.''

``How do you know she hasn't gone?''

``She won't have gone,'' Bannerjee said. ``Now get out of my bank
account or I will come down there and blow your balls off.''

It took Yasmin a moment to understand how Bannerjee could be so
sure that Mala hadn't left the hospital -- she must have been so
badly injured that she couldn't leave. She found that she was
wailing, making a sound like a cat in the night, a terrible sound
that she couldn't contain. Mala's army came running and she tried
to stop so that she could explain it to them, but she couldn't.

In the end, they all walked to LT hospital together, a solemn
procession through the streets of Dharavi. A few people scurried
forward to ask what was going on, and once they were told, they
joined. More and more people joined until they arrived at the
hospital in a huge mob of hundreds of silent people. Ashok and
Yasmin and Sushant went to the counter and told the shocked ward
sister why they were there. She paged through her record-book for
an eternity before saying, ``It must be this one.'' She looked at
them sternly. ``But you can't all go. Who is the girl's mother?''

Ashok and Yasmin looked back at the crowd. Neither of them had
thought to fetch Mala's mother. They were Mala's family. She was
their general. ``Take us to her, please,'' Yasmin said. ``We will
bring her mother.''

The sister looked like she would not let them pass, but Ashok
jerked his head over his shoulder. ``They won't leave until we see
her, you know.'' He waggled his chin good-naturedly and smiled and
for a moment Yasmin remembered how handsome he'd been when she'd
first met him on his motorcycle.

The sister blew out an exasperated sigh. ``Come with me,'' she said.

They wouldn't have recognized Mala if she hadn't told them which
bed was hers. Her head had been shaved and bandaged, and one side
of her face was a mass of bruises. Her left arm was in a sling.

Yasmin let out an involuntary groan when she saw her, and the ward
sister beside her squeezed her arm. ``She wasn't raped,'' the woman
whispered in her ear. ``And the doctor says there was no
brain-damage.''

Yasmin cried now, really cried, the way she hadn't let herself cry
before, the cry from her soul and her stomach, the cry that
wouldn't let go, the cry that drove her to her knees as though she
were being beaten with a lathi. She curled up into a ball and cried
and cried, and the ward sister led her to a seat and tried to put a
pill between her lips but she wouldn't let it in. She needed to be
alert and awake, needed to stop crying, needed --

Ashok squatted against the wall beside her, clenching and
unclenching his fists. ``I'll ruin him,'' he muttered over and over
again, ignoring the stares of the other patients on the ward with
their visitors. ``I'll \emph{destroy} him.''

This got through to Yasmin. ``How?''

``Every piaster, ever runestone, every gold piece that man takes out
of a game we will take away from him. He is finished.''

``He'll find some other way to survive, some other way of hurting
people to get by.''

Ashok shook his head. ``Fine. I'll find a way to ruin that, too. He
is powerful and strong and ruthless, but we are smart and fast and
there are \emph{so many} of us.''

\tb

Dafen was full of choking smoke. Matthew pushed his way through the
crowds. He'd tried to bring the painter girl, Mei, with him, but
she had run into a group of her friends and had gone off with them,
stopping to kiss him hard on the lips, then laughing at his
surprised expression and kissing him again. The second time, he had
the presence of mind to kiss her back and for a second he actually
managed to forget he was in the middle of a riot. Mei's friends
hooted and called at them and she gave his bottom a squeeze and
took his phone out of his fingers and typed her number into it, hit
SAVE. The phone network had died an hour before, when the police
retreated from Dafen and fell back to a defensive cordon around the
whole area.

And then he was alone, making his way back toward the huge statue
of the hand holding the brush, the entrance to Dafen. Painters
thronged the streets, carrying beautifully made signs, singing
songs, drinking fiery, cheap baijiu whose smells mixed with the
smoke and the oil paint and the turpentine.

The police line bristled as he peered around the corner of a cafe
at the edge of Dafen. He wasn't the only one eyeing them nervously
-- there was a little group of white tourists cowering in the cafe,
clutching their cameras and staring incredulously at their dead
phones. Matthew listened in on their conversation, straining to
understand the rapid English, and gathered that they'd been brought
here by a driver from their hotel, a Hilton in Jiabin Road.

``Hello,'' he said, trying his English out. He wished that the
gweilo, Wei-Dong, had let him practice more. ``You need help?'' He
was intensely self-conscious about how bad he must sound, his
accent and grammar terrible. Matthew prided himself on how
well-spoken he was in Chinese.

The eldest tourist, a woman with wrinkled arms and neck showing
beneath a top with thin straps, looked hard at him. She removed her
oversized sunglasses and assayed a little Chinese. ``We are fine,''
she said, her accent no better than Matthew's, which he found oddly
comforting. She was with three others, a man he took to be her
husband and two young men, about Matthew's age, who looked like a
cross between her and the husband: sons.

``Please,'' he said. ``I take you out, find taxi. You tell --'' he
tried to find the word for policemen, couldn't remember it, found
himself searching through his game-vocabulary. ``Knights? Paladins?
Soldiers. You tell soldiers I am guide. We all go.''

The boys grinned at him and he thought they must be gamers, because
they'd really perked up at \emph{paladins}, and he tried grinning
back at them, though truth be told he didn't feel like doing
anything. They conferred in hushed voices.

``No thank you,'' the older man said. ``We're all right.''

He squeezed his eyes shut. He had to get somewhere that his phone
would work, had to check in with Big Sister Nor and find out where
the others were, what the plan was. He'd have to get new papers,
maybe go to one of the provinces or try to sneak into Hong Kong.
``You help me,'' he managed. ``I no go without you. Without, uh,
foreigners.'' He gestured at the police, at their shields. ``They not
hurt foreigners.''

The older man's eyes widened in comprehension. They spoke again
among themselves. He caught the word ``criminal.''

``I not criminal,'' he said. But he knew it was a lie and felt like
they must know it too. He was a criminal and a former prisoner, and
he would never be anything but, for his whole life; just like his
grandfather.

They all stared at him, then looked away.

``Please,'' he said, looking at each one in turn. He jerked his head
at the police. ``They hurt people soon.''

The woman drew in a deep breath, turned to the man, said, ``We need
to get out of here anyway. It will be good to have a local.''

The taller of the two boys said, ``What do you play?''

``Svartalfaheim Warriors, Zombie Mecha, Mushroom Kingdom, Clankers,
Big Smoke, Toon,'' he said, ticking them off on his fingers.

``All of them?'' The boys boggled at him.

He nodded. ``All.''

They laughed and he laughed too, small sounds in the roar of the
crowds and the thunder of the choppers overhead.

``You are sure about this?'' the woman said. Adding, ``Certain?'' in
Chinese. He nodded twice.

``Come with me,'' he said and drew in a deep breath and led them out
toward the police lines.

\tb

Wei-Dong didn't want to wake Jie, but he needed to sleep. He
finally curled up on the floor next to the mattress, using his
shoulderbag as a pillow to get his face off of the filthy carpet.
At first he lay rigid in the brightly lit room, his mind swirling
with all he'd seen and done, but then he must have fallen asleep
and fallen hard, because the next thing he knew, he was swimming up
from the depths of total oblivion as Jie shook his shoulder and
called his name. He opened his eyes to slits and peered at her.

``Wha?'' he managed, then realized he was talking English and said,
``What?'' in Chinese.

``Time to go,'' she said. ``Big Sister Nor says we have to move.''

He sat up. His mouth was full of evil-tasting salty paste, a stale
residue of dumplings and sleep. Self consciously, he breathed
through his nose.

``Where?''

``Hong Kong,'' she said. ``Then\ldots{}'' She shrugged. ``Taiwan, maybe?
Somewhere we can tell the story of the dead without being arrested.
That's the most important thing.''

``How are we going to cross the border? I don't have a Chinese visa
in my passport.''

She grinned. ``That part is easy. We go to my counterfeiter.''

It was as good a plan as any. Wei-Dong had watched the Webblies
change papers again and again. Shenzhen was full of counterfeiters.
He rode the Metro apart from her again, staring at his stupid guide
map and trying to look like a stupid tourist, invisible. It was
easier this time around, because there was so much else going on --
factory girls talking about Jie's radio show and ``the 42,''
policemen prowling the cars and demanding the papers of any group
of three or more people, searching bags and, once, confiscating a
banner painted on a bedsheet. Wei-Dong didn't see what it said, but
the police took four screaming, kicking girls off the train at the
next station. Shenzhen was in chaos.

They got off the train at the stock market station, and he followed
Jie, leaving a hundred yards between them. But he came up against
her when they got to the surface. The last time he'd been here, it
had been thronged with counterfeiters and touts handing out fliers
advertising their services, scrap-buyers with scales lining the
sidewalks, hawkers selling fruit and ices. Now it was wall-to-wall
police, a cordon formed around the entrance to the stock-market.
Officers were stationed every few yards on the street, too,
checking papers.

Jie picked up her phone and pretended to talk into it, but Wei-Dong
could see she just didn't want to look suspicious. He got out his
tourist-map and pretended to study it. Gradually, they both made
their way back into the station. She joined him at a large map of
the surrounding area.

``Now what?'' he whispered, trying not to move his mouth.

``How were you going to get out of here?'' she said.

His stomach tightened. ``I hadn't really thought about it much,'' he
said.

She hissed in frustration. ``You must have had some idea. How about
the way you got in?''

He hadn't told anyone the details of his transoceanic voyage. It
would have felt weird to admit that he was part owner of a giant
shipping company. Besides, he didn't really \emph{feel} like it was
his. It was his father's.

Two policemen passed by, grim-faced, moving quickly, an urgent,
insectile buzz coming from their earpieces.

``Really?''

``If we could get into the port,'' he said. ``I think I could get us
anywhere.''

She smiled, and it was the first real smile he'd seen on her face
since -- since before the shooting had started.

``But I need to call my mother.''

\tb

The policemen that questioned Matthew were so tense they
practically vibrated, but the tourist lady put on a big show of
being offended that they were being stopped and demanded that they
be allowed to go, practically shouting in English. Matthew
translated every word, speaking over the policemen as they tried to
ask him more questions about how he'd come to be there and what had
happened to get his clothes so dirty with paint and mud.

The tourist lady took out her camera and aimed it at the policemen,
and that ended the friendly discussion. Before she could bring the
screen up to her face, a policeman's gloved hand had closed around
the lens. The two boys moved forward and it looked like someone
would start shoving soon, and the man was shouting in English, and
all the noise was enough to attract the attention of an officer who
gave the cops a blistering tongue lashing for wasting everyone's
time and waved them on with a stern gesture.

Matthew could hardly believe he was free. The tourists seemed to
think it was all a game as he urged them down the road a way, out
of range of the police cordon and away from the shouting. They
walked up the shoulder of the Shenhui Highway, staying right on the
edge as huge trucks blew past them so fast it sucked the breath out
of their lungs.

``Taxi?'' the woman asked him.

He shook his head. ``I no think taxi today,'' he said. ``Private car,
maybe.''

She seemed to understand. He began to wave at every car that passed
them by, and eventually one stopped, a Chang'an sedan that had seen
better days, its trunk held shut with a bungee cord that allowed
the lid to bang as the car rolled to a stop. It was driven by a man
in a dirty chauffeur's uniform. Matthew leaned in and said, ``100
RMB to take us to Jiabin Road.'' It was high, but he was sure the
tourists could afford it.

``No, too far,'' the man said. ``I have another job --''

``200,'' Matthew said.

The man grinned, showing a mouthfull of steel teeth. ``OK, everyone
in.''

They were on the road for a mere five minutes before his phone
chirped to let him know that he had voicemail waiting for him. It
was Justbob, from Big Sister Nor.

\tb

``Mom?''

``Leonard?''

``Hi, Mom.'' He tried to ignore Jie who was looking at him with an
expression of mingled hilarity and awe. She had an encyclopedic
knowledge of gamer cafes with private rooms, and had brought them
to this one in the ground floor of a youth hostel that catered to
foreigners and had a room set off for karaoke and net-access.

``It's been so long since I've heard your voice, Leonard.''

``I know, Mom.''

``How's your trip?''

``Um, fine.'' He tried to remember where he told her he'd be.
Portland? San Francisco?

``Oh, Leonard,'' she said, and he heard that she was crying. It was
what, 8PM back in LA, and she was crying and alone. He felt so
homesick at that moment he thought he would split in two and he
felt the tears running down his own cheeks.

``I love you, Mom,'' he blubbered.

And they both cried for a long time, and when he risked a look at
Jie, she was crying too.

``Mom,'' he said, choking back snot. ``I have a favor to ask of you. A
big favor.''

``You're in trouble.''

``Yes.'' There was no point in denying it. ``I'm in trouble. And I
can't explain it right now.''

``You're in China, aren't you?''

He didn't know what to say. ``You knew.''

``I suspected. It's that gamer thing, isn't it? I did the math on
when you answered my messages, when you called.''

``You knew?''

``I'm not stupid, Leonard.'' She wasn't crying anymore. ``I thought I
knew, but I didn't want to say anything until you told me.''

``I'm sorry, Mom.''

She didn't say anything.

``Are you coming home?''

He looked at Jie. ``I don't know. Eventually. I have something I
have to do here, first.''

``And you need my help with that.''

``Mom, I need you to order a shipment from Shenzhen to Mumbai.'' Big
Sister Nor had suggested it, and Jie had shrugged and said that it
was fine with her, one place was as good as any other. ``I'll give
you the container number. And you have to have Mr Alford call the
port authority here and tell them that I'm authorized to access
it.''

``No, Leonard. I'll call the embassy, I'll get you home, but this is
--'' He could picture her hand flapping around her head. ``It's
crazy, is what it is.''

``Mom --''

``No.''

``Mom, \emph{listen}. This is about a lot more than just me. There
are people here, friends, whose lives are at stake. You can call
the embassy all you want but I won't go there. If you don't help
me, I'll have to do this on my own, and I have to be honest with
you, Mom, I don't think I'll be able to do it. But I can't abandon
my friends.''

She was crying again.

``I'm going to be at the port in --'' he checked the screen of his
phone -- ``in three hours. I've got my passport with me, that'll get
me inside, \emph{if} you've got it squared away with the port
authority. The container number is WENU432134. It's at the western
port. Do you have that?''

``Leonard, I won't do it.''

``WENU432134,'' he said, very slowly, and hung up.

\tb

There were five of them in all. Matthew, Jie, Wing, Shirong, and
Wei-Dong. They'd stopped at a 7-11 on the way to the train station
and bought as much food as they could carry, asking the bemused
clerk to pack it in boxes and seal them with packing tape.

As they approached the port, they stopped talking, walking slowly
and deliberately. Wei-Dong steeled himself and walked to the
guard's booth. He hadn't called his mother back. There hadn't been
time. Shenzhen was in chaos, police-checks and demonstrations
everywhere, some riots, spirals of black smoke heading into the
sky.

He motioned for Wing to join him. They had agreed that he would
play interpreter, to make Wei-Dong seem like more of a hopeless
gweilo, above suspicion. They'd found him some cheap fake Chinese
Nike gear to wear, a ridiculous track suit that reminded him of the
Russian gangsters he'd see around Santee Alley.

Wordlessly, he handed his passport -- his real passport, held
safely all this time -- to the young man on the gate. ``WENU432134,''
he said. ``Goldberg Logistics container.''

He waited for Wing to translate, watched him sketch out the English
letters on his palm.

The security guard looked over his shoulder at the two policemen in
the booth with him. He picked up a scratched tablet and prodded at
it with a blunt finger, squinting at Wei-Dong's passport. Wei-Dong
hoped that he wouldn't try something clever, like riffling its
pages looking for a Chinese visa.

He began to shake his head, said ``I don't see it --''

Wei-Dong felt sweat run down his butt-crack and over his thighs. He
craned his neck to see the screen. There it was, but the number had
been entered wrong, WENU432144. He pointed to it and said, ``Tell
him that this is the one.'' He sent a silent thanks to his mother.
The guard compared the number to the one he'd entered and then
seemed about to let them pass. Then one of the policeman said,
``Wait.''

The cop shouldered the security guard out of the way, took the
passport from him, examined it closely, holding a page up to the
light to see the watermark. ``What are you bringing?''

Wei-Dong waited for Wing to translate.

``Samples,'' he said. ``Clothes.''

He opened up the box at his feet and pulled out a folded tee-shirt
emblazoned with some Chinese characters that said ``I'm stupid
enough to think that this shirt looks cool.'' Jie had found them
from one of the few stubborn peddlers left on the street outside of
the Metro entrance near the train station. The cop snorted and
said, ``Does he know what this says?''

Wing nodded. ``Yes,'' he said. ``But he thinks that other Americans
won't. If they like it, they will order twenty thousand from us!''
He laughed, and after a moment, the cop and the security guard
joined in. The cop slapped Wei-Dong on the shoulder and Wei-Dong
forced a laugh out as well.

``OK,'' the cop said, handing back his papers. The security guard
gave them directions. ``But you'll have to use the north gate to
leave. We're closing this one for the evening in half an hour.''

Wing made a show of translating for Wei-Dong, who had the presence
of mind to pretend to listen, but he was rocking on his heels,
almost at the point of collapse from lack of sleep and food.

They walked in total silence to the container, and Wei-Dong managed
to only look over his shoulder once. Jie caught his eye when he did
and waggled a finger at him. He smiled wryly and looked ahead,
following the directions.

The container was just as he'd left it, and his key fit the
padlock. The four marvelled at the cleverness of his work inside as
they efficiently unpacked their food.

``Three nights, huh?'' said Jie, as he pulled the door shut behind
them.

``After they load us.''

``When will that be?''

He sighed. ``I need to call my mother to find out.'' He pulled out
his phone and Jie handed him her last SIM and a calling card.

\tb

Big Sister Nor, The Mighty Krang and Justbob had no warning this
time. Three men, small-time crooks working on contract for a man in
Dongguan who owned one of the big gold-exchanges, worked silently
and efficiently. They followed Justbob back from a Malaysian satay
restaurant that they were known to frequent, back to the latest
safe-house, a room over a massage-parlor on Changi Road, where the
Webblies could tap into the wireless from a nearby office building.
They waited patiently outside for all the windows to go dark.

Then they methodically attached bicycle locks to each doorway. It
was nearly 5AM and the few passers-by paid them no particular
attention. Once they had locked each door, they hurled petrol bombs
through windows on the ground floor. They stayed just long enough
to make sure that the fires were burning cheerily before they got
into two cars parked around the corner and sped off. The next
morning, they crossed into Kuala Lumpur and did not return to
Singapore for eight months, drawing a small salary from the man in
Dongguan while they laid low.

Big Sister Nor was the first one awake, roused by the sound of
three windows smashing in close succession. She smelled the greasy
smoke a moment later and began to shout, in her loudest voice,
``Fire! Fire!'' just as she had practiced in a thousand dreams.

Justbob and The Mighty Krang were up an instant later. Justbob went
to the stairs and ventured halfway down toward the massage-parlor
before the flames forced her up again. The Mighty Krang broke out
the window with a chair -- it had been painted shut -- and leaned
way out, far enough to see the lock that had been added to the
door. He breathlessly but calmly reported this to Big Sister Nor,
who had already popped the drives out of their control machines.
She handed them to him, listened to Justbob's assessment of the
staircase and nodded.

They could hear the screams from the floor below them as the girls
from the massage parlor broke out their own windows and called for
help. A girl emerged, legs first, from one of the massage parlor's
small, high windows. She was screaming, on fire, rolling on the
ground. A few people were in the street below, talking into their
phones -- the fire department would be here soon. It wouldn't be
soon enough. Choking smoke was already filling the room, and they
were forced to their knees.

``Out the window,'' Big Sister Nor gasped. ``You'll probably break a
leg, but that's better than staying here.''

``You first,'' The Mighty Krang said.

``Me last,'' she said, in a voice that brooked no argument. ``After
you two are out.'' She managed a small smile. ``Try to catch me,
OK?''

Justbob grabbed The Mighty Krang's arm and pulled him toward the
window. He got as far as the sill, then balked. ``Too far!'' he said,
dropping back to his belly. Justbob gave him a withering look, then
hauled herself over the sill, dropped so she was hanging by her
arms, then allowed herself to drop the rest of the way. If she made
a sound, it was lost in the roar of the flames that were just
outside the door now. The floor was too hot to touch.

``GO!'' Big Sister Nor said.

``You're our leader, our Big Sister Nor,'' he said, and grabbed her
arm. ``We're all nothing without you!'' She shook his hand off.

``No, you idiot,'' she said. ``I am nothing more than the switchboard.
You all lead yourselves. Remember that!'' She grabbed the waistband
of his jeans, just over his butt, and practically threw him out the
window. The air whistled past him for an instant, and then there
was a tremendous, jarring impact, and then blackness.

Big Sister Nor was on fire, her loose Indian cotton trousers, her
long black hair. The room was all smoke now, and every breath was
fire, too. She smelled her own nose-hairs singe as a breath of
scalding air passed into her lungs, which froze and refused to work
anymore. She stood and took one step to the window, standing for a
moment like a flaming avatar of some tragic god in the window
before she faltered, went down on one knee, then the flames
engulfed her.

And below, the crowd on the street began to cry. Justbob cried too,
from the pavement where she was being tended by a passerby who knew
some first aid and was applying pressure to the ruin of her left
leg. The Mighty Krang was unconscious, with a broken arm and three
broken ribs.

But he remembered what Big Sister Nor told him, and he wrote those
words down, typing them with his left hand in English, Malay, Hindi
and Chinese, recording them with his smoke-ruined voice from his
hospital bed.

His words -- Big Sister Nor's words -- went out all over the world,
spreading from phone to message board to site to site. You lead
yourselves.

The words were heard by factory girls all over South China, back on
the job after a few short days of energetic chaos, mass firings and
mass arrests. They were heard by factory boys all over Cambodia and
Vietnam. They were heard in the alleys of Dharavi and in the living
rooms of Mechanical Turks all over Europe, the US and Canada. They
were published in many languages on the cover of many newspapers
and aired on many broadcasts.

These last treated the words as a report from a distant world --
``Did you know that these strange games and the people who played
them took it all so seriously?'' But for the people who needed to
hear them, the words were heard.

They were heard by five friends who downloaded them over the
achingly slow network connection on the container ship, a day out
of Shenzhen port. Five friends who wept to hear them. Five friends
who took strength from them.

\tb

They hid in the inner container when the ship entered the Mumbai
Harbor, heading for the Mumbai Port Trust. Wei-Dong had googled the
security procedures at Mumbai Port, and he didn't think they were
using gas chromatograph to detect smuggled people, but they didn't
want to take any chances. It was crowded, and the toilet had
stopped working, and they had only managed to gather enough water
for one brief shower each on the three day passage.

They fell against one-another, then clung to the floor as the
container was lifted on a crane and set down again. They heard the
outer door open, then shut, and muffled conversations. Then they
were rolling.

Cautiously, they opened the inner door. The smell of Mumbai --
spicy, dusty, hot and wet -- filled the container. Light streamed
in from the little holes Wei-Dong had drilled an eternity ago on
the passage to Shenzhen.

Now they heard the sound of horns, many, many horns. Lots of
motorcycle engines, loud. Diesel exhaust. The huge, bellowing
air-horn of the truck their container had been placed upon. The
truck stopped and started many times, made a few slow, lumbering
turns, then stopped. A moment later, the engines stopped too.

The five of them held their breaths, listened to the footsteps
outside, listened to a conversation in Hindi, adult male voices.
Listened to the scrape of the catch on the container's big rear
doors.

And then sunlight -- dusty, hot, with swirling clouds of dust and
the pong of human urine -- flooded into the container. They
shielded their eyes and looked into the faces of two grinning
Indian men, with fierce mustaches and neatly pressed shirts. The
men held out their hands and helped them down, one at a time, into
a narrow alley that was entirely filled by the truck, which neatly
shielded them from view. Wei-Dong couldn't imagine backing a truck
into a space this narrow.

The men gestured at the interior of the container, miming,
\emph{Do you have everything?} Wei-Dong and Jie made sure everyone
was clear and then nodded. The men waggled their chins at them,
shook Wei-Dong and Jie's hands, brief and dry, and edged their way
back along the space between the truck and the alley's walls. The
engine roared to life, a cloud of diesel blew into their face, and
the truck pulled away, lights glowing over a handpainted sign on
the bumper that read HORN PLEASE.

The truck blew its horn once as it cleared the alley-mouth and
turned an impossibly tight right turn. The alley was flooded with
light and noise from the street, and then they saw a man and a girl
walking down it, toward them.

They drew close. The girl was wearing some kind of headscarf with a
veil that covered most of her face. The man had short, gelled hair
and was dressed in a pressed white shirt tucked into black slacks.
The two groups stood and looked at one another for a long moment,
then the man held his hand out.

``Ashok Balgangadhar Tilak,'' he said.

``Leonard Goldberg,'' Leonard said. They shook. It was another short,
dry handshake.

The girl held her hand out. ``Yasmin Gardez,'' she said.

She barely took his hand, and the shake was brief.

``We all lead ourselves,'' Leonard said. He hadn't planned on saying
it, but it came out just the same, and Wing understood it and
translated it into Chinese, and for a moment, no one needed to say
anything more.

``We have places for you to stay in Dharavi,'' Yasmin said. Leonard
translated. ``We all want to hear what you have to tell us. And we
have work for you, if you want.''

``We want to work,'' Wing said.

``That's good,'' Ashok said, and they struck out.

They emerged beside a hotel. The street before them thronged with
people, more than they could comprehend, and cars, and
three-wheelers, and bicycles, and trucks of all sizes. It was a
hive of activity that made even Shenzhen seem sedate. For a moment
none of them said anything.

``Mumbai is a busy place,'' Yasmin said.

``We have friends in the Transport and Dock Workers' Union,'' Ashok
said, casually, setting off down the crowded pavement, ignoring the
children who approached them, begging, holding their hands out,
tugging at their sleeves. Leonard felt as though he was walking
through an insane dream. ``They were glad to help.''

The street ended at the ocean, a huge, shimmering harbor dotted
with ferries and other craft. Ahead of them spread an enormous
plaza, the size of several football fields stitched together,
covered in gardens, and, where it met the ocean, an enormous
archway topped with minarets and covered with intricate carvings,
and all around them, thousands of people, talking, walking,
selling, begging, sleeping, running, riding.

The five of them stopped and gaped. Three days locked in a
container with nothing to see that was more than a few yards away
had robbed them of the ability to easily focus on large, far-away
objects, and it took a long while to get it all into their heads.
Yasmin and Ashok indulged them, smiling a little.

``The Gateway of India,'' Yasmin said, and Leonard translated
absently.

To one side stood a hotel as big as the giant conference center
hotels near Disneyland, done up like some kind of giant temple,
vast and ungainly. Leonard looked at it for a moment, then shooed
away the beggars that had approached them. Yasmin scolded them in
Hindi and they smiled at her and backed off a few paces, saying
something clearly insulting that Yasmin ignored.

``It's incredible,'' Leonard said.

``Mumbai is\ldots{}'' Ashok waved his hand. ``It's amazing. Even where
we're going -- the other end of the Harbour Line, our humble home,
is incredible. I love it here.''

Wing said, ``I loved it in China.'' He looked grave.

``I hope that you can go back again some day,'' Ashok said. ``All of
you. All of us. Anywhere we want.''

Jie said, ``They put down the strikes in China.'' Leonard
translated.

Yasmin and Ashok nodded solemnly. ``There will be other strikes,''
Yasmin said.

A man was approaching them. A white man, pale and obvious among the
crowds, trailing a comet-tail of beggars. Leonard saw him first,
then Ashok turned to follow his gaze and whispered ``Oh, my, this
\emph{is} interesting.''

The man drew up to them. He was fat, racoon-eyed, hair a wild mess
around his head. He was wearing a polo shirt emblazoned with the
Coca-Cola Games logo and a pair of blue-jeans that didn't fit him
well, and Birkenstocks. He wouldn't have looked more American if he
was holding up the Statue of Liberty's torch and singing ``Star
Spangled Banner.''

Ashok held his hand out. ``Dr Prikkel, I presume.''

``Mr Tilak.'' They shook. He turned to Leonard. ``Leonard, I
believe.''

Leonard gulped and took the man's hand. He had a firm, American
handshake. The four Chinese Webblies were talking among themselves.
Leonard whispered to them, explaining who the man was, explaining
that he had no idea what he was doing there.

``You'll have to forgive me for the dramatics,'' Connor Prikkel said.
``I knew that I would have to come to Mumbai to meet with you and
your extraordinary friends, curiosity demanded it. But once we put
our competitive intelligence people onto your organization, it
wasn't hard to find a hole in your mail server, and from there we
intercepted the details of this meeting. I thought it would make an
impression if I came in person.''

``Are you going to call the police?'' Wing said, in halting English.

Prikkel smiled. ``Shit, no, son. What good would that do? There's
thousands of you Webbly bastards. No, I figure if Coca Cola Games
is going to be doing business with you, it'd be worth sitting down
and chatting. Besides, I had some vacation days I needed to use
before the end of the year, which meant I didn't have to convince
my boss to let me come out here.''

They were blocking the sidewalk and getting jostled every few
seconds as someone pushed past them. One of them nearly knocked
Prikkel into a zippy three-wheeled cab and Ashok caught his arm and
steadied him.

``Are you going to fire me?'' Leonard said.

Prikkel made a face. ``Not my department, but to be totally honest,
I think that's probably a good bet. You and the other ones who
signed your little petition.'' He shrugged. ``I can do stuff like
take money out of that bastard's account when your friend's life is
at stake -- it's not like he's gonna complain, right? But how Coke
Games contracts with its workforce? Not my department.''

Yasmin's eyes blazed. ``You can't -- we won't let you.''

``That's a rather interesting proposition,'' he said, and two men
holding a ten-foot-long tray filled with round tin lunchpails
squeezed past him, knocking him into Jie. ``One I think we could
certainly have a good time discussing.'' He gestured toward the huge
wedding-cake hotel. ``I'm staying at the Taj. Care to join me for
lunch?''

Ashok looked at Yasmin, and something unspoken passed between them.
``Let us take \emph{you} out for lunch,'' Ashok said. ``As our guest.
We know a wonderful place in Dharavi. It's only a short train
journey.''

Prikkel looked at each of them in turn, then shrugged. ``You know
what? I'd be honored.''

They set off for the train station. Jie snorted. ``I can't
\emph{wait} to broadcast this.'' Leonard grinned. He couldn't wait
either.

\backmatter

\section{Acknowledgements:}

Thanks to Russell Galen, Patrick Nielsen Hayden, and my beautiful
and enormously patient wife Alice -- I couldn't have written this
without you three.

Thanks to the Silklisters, Rishab Ghosh, and Ashok Banker and Yoda,
Keyan Bowes, Rajeev Suri, Sachin Janghel, Vishal Gondal, Sushant
Bhalerao and Menyu Singhfor all your assistance in Mumbai.

Thanks to LemonED, Andrew Lih, Paul Denlinger, Bunnie Huang, Kaiser
Kuo, Anne Stevenson-Yang, Leslie Chang, Ethan Zuckerman, John
Kennedy, Marilyn Terrell, Peter Hessler, Christine Lu, Jon
Phillips, Henry Oh, for invaluable aid in China.

Thanks to Julian Dibbell, Ge Jin, Matthew Chew, James Seng, Jonas
Luster, Steven Davis, Dan Kelly and Victor Pineiro for help with
the gold farmers.

Thanks to Max Keiser, Alan Wexelblat and Mark Soderstrom for
economics advice.

Thanks to Thomas ``CmdLn'' Gideon, Dan McDonald, Kurt Von Finck,
Canonical, Inc, and Ken Snider for tech support!

Thanks to MrBrown and the Singapore bloggers for unforgettable
street-dinners.

Thanks also to JP Rangaswami and Marilyn Tyrell.

Many thanks to Ken Macleod for letting me use IWWWW and ``Webbly.''

\begin{center}\rule{3in}{0.4pt}\end{center}

\section{Bio:}

GPG key fingerprint: 0BC4 700A 06E2 072D 3A77 F8E2 9026 DBBE 1FC2
37AF

\href{http://www.flickr.com/photos/doctorow/sets/72157622138315932/}{Gallery of publicity photos}

Cory Doctorow
(\url{doctorow@craphound.com}/\url{http://craphound.com/})
is the author of several science fiction novels. Some are for
adults, others are for young people and adults. He's also the
author of a book of essays (\emph{Content}, Tachyon Books), a
graphic novel
(\emph{Cory Doctorow's Futuristic Tales of the Here and Now}, IDW)
and two collections of short stories, both currently in print from
Thunder's Mouth Press.

Born in 1971 in Toronto, Canada, he now lives in London, England
with his wonderful wife, Alice, and his scrumptious two year old
daughter, Poesy. He formerly served as European Director of the
Electronic Frontier Foundation and is a fellow of that
organization. He is also affiliated with the Open University
Faculty of Computer Science (UK) and the University of Waterloo
Independent Studies Program (Canada).

He is the co-editor and co-owner of the widely read blog Boing
Boing (boingboing.net) and writes columns for \emph{The Guardian}
newspaper, \emph{Publishers Weekly}, \emph{Locus Magazine}, and
\emph{Make Magazine}.

\begin{center}\rule{3in}{0.4pt}\end{center}

\section{Creative Commons}

Creative Commons Legal Code

Attribution-NonCommercial-ShareAlike 3.0 Unported

CREATIVE COMMONS CORPORATION IS NOT A LAW FIRM AND DOES NOT PROVIDE
LEGAL SERVICES. DISTRIBUTION OF THIS LICENSE DOES NOT CREATE AN
ATTORNEY-CLIENT RELATIONSHIP. CREATIVE COMMONS PROVIDES THIS
INFORMATION ON AN ``AS-IS'' BASIS. CREATIVE COMMONS MAKES NO
WARRANTIES REGARDING THE INFORMATION PROVIDED, AND DISCLAIMS
LIABILITY FOR DAMAGES RESULTING FROM ITS USE.

License

THE WORK (AS DEFINED BELOW) IS PROVIDED UNDER THE TERMS OF THIS
CREATIVE COMMONS PUBLIC LICENSE (``CCPL'' OR ``LICENSE''). THE WORK IS
PROTECTED BY COPYRIGHT AND/OR OTHER APPLICABLE LAW. ANY USE OF THE
WORK OTHER THAN AS AUTHORIZED UNDER THIS LICENSE OR COPYRIGHT LAW
IS PROHIBITED.

BY EXERCISING ANY RIGHTS TO THE WORK PROVIDED HERE, YOU ACCEPT AND
AGREE TO BE BOUND BY THE TERMS OF THIS LICENSE. TO THE EXTENT THIS
LICENSE MAY BE CONSIDERED TO BE A CONTRACT, THE LICENSOR GRANTS YOU
THE RIGHTS CONTAINED HERE IN CONSIDERATION OF YOUR ACCEPTANCE OF
SUCH TERMS AND CONDITIONS.

1. Definitions

1. ``Adaptation'' means a work based upon the Work, or upon the Work
and other pre-existing works, such as a translation, adaptation,
derivative work, arrangement of music or other alterations of a
literary or artistic work, or phonogram or performance and includes
cinematographic adaptations or any other form in which the Work may
be recast, transformed, or adapted including in any form
recognizably derived from the original, except that a work that
constitutes a Collection will not be considered an Adaptation for
the purpose of this License. For the avoidance of doubt, where the
Work is a musical work, performance or phonogram, the
synchronization of the Work in timed-relation with a moving image
(``synching'') will be considered an Adaptation for the purpose of
this License.

2. ``Collection'' means a collection of literary or artistic works,
such as encyclopedias and anthologies, or performances, phonograms
or broadcasts, or other works or subject matter other than works
listed in Section 1(g) below, which, by reason of the selection and
arrangement of their contents, constitute intellectual creations,
in which the Work is included in its entirety in unmodified form
along with one or more other contributions, each constituting
separate and independent works in themselves, which together are
assembled into a collective whole. A work that constitutes a
Collection will not be considered an Adaptation (as defined above)
for the purposes of this License.

3. ``Distribute'' means to make available to the public the original
and copies of the Work or Adaptation, as appropriate, through sale
or other transfer of ownership.

4. ``License Elements'' means the following high-level license
attributes as selected by Licensor and indicated in the title of
this License: Attribution, Noncommercial, ShareAlike. 5. ``Licensor''
means the individual, individuals, entity or entities that offer(s)
the Work under the terms of this License.

6. ``Original Author'' means, in the case of a literary or artistic
work, the individual, individuals, entity or entities who created
the Work or if no individual or entity can be identified, the
publisher; and in addition (i) in the case of a performance the
actors, singers, musicians, dancers, and other persons who act,
sing, deliver, declaim, play in, interpret or otherwise perform
literary or artistic works or expressions of folklore; (ii) in the
case of a phonogram the producer being the person or legal entity
who first fixes the sounds of a performance or other sounds; and,
(iii) in the case of broadcasts, the organization that transmits
the broadcast.

7. ``Work'' means the literary and/or artistic work offered under the
terms of this License including without limitation any production
in the literary, scientific and artistic domain, whatever may be
the mode or form of its expression including digital form, such as
a book, pamphlet and other writing; a lecture, address, sermon or
other work of the same nature; a dramatic or dramatico-musical
work; a choreographic work or entertainment in dumb show; a musical
composition with or without words; a cinematographic work to which
are assimilated works expressed by a process analogous to
cinematography; a work of drawing, painting, architecture,
sculpture, engraving or lithography; a photographic work to which
are assimilated works expressed by a process analogous to
photography; a work of applied art; an illustration, map, plan,
sketch or three-dimensional work relative to geography, topography,
architecture or science; a performance; a broadcast; a phonogram; a
compilation of data to the extent it is protected as a
copyrightable work; or a work performed by a variety or circus
performer to the extent it is not otherwise considered a literary
or artistic work.

8. ``You'' means an individual or entity exercising rights under this
License who has not previously violated the terms of this License
with respect to the Work, or who has received express permission
from the Licensor to exercise rights under this License despite a
previous violation. 9. ``Publicly Perform'' means to perform public
recitations of the Work and to communicate to the public those
public recitations, by any means or process, including by wire or
wireless means or public digital performances; to make available to
the public Works in such a way that members of the public may
access these Works from a place and at a place individually chosen
by them; to perform the Work to the public by any means or process
and the communication to the public of the performances of the
Work, including by public digital performance; to broadcast and
rebroadcast the Work by any means including signs, sounds or
images.

10. ``Reproduce'' means to make copies of the Work by any means
including without limitation by sound or visual recordings and the
right of fixation and reproducing fixations of the Work, including
storage of a protected performance or phonogram in digital form or
other electronic medium.

2. Fair Dealing Rights. Nothing in this License is intended to
reduce, limit, or restrict any uses free from copyright or rights
arising from limitations or exceptions that are provided for in
connection with the copyright protection under copyright law or
other applicable laws.

3. License Grant. Subject to the terms and conditions of this
License, Licensor hereby grants You a worldwide, royalty-free,
non-exclusive, perpetual (for the duration of the applicable
copyright) license to exercise the rights in the Work as stated
below:

1. to Reproduce the Work, to incorporate the Work into one or more
Collections, and to Reproduce the Work as incorporated in the
Collections;

2. to create and Reproduce Adaptations provided that any such
Adaptation, including any translation in any medium, takes
reasonable steps to clearly label, demarcate or otherwise identify
that changes were made to the original Work. For example, a
translation could be marked ``The original work was translated from
English to Spanish,'' or a modification could indicate ``The original
work has been modified.'';

3. to Distribute and Publicly Perform the Work including as
incorporated in Collections; and, 4. to Distribute and Publicly
Perform Adaptations.

The above rights may be exercised in all media and formats whether
now known or hereafter devised. The above rights include the right
to make such modifications as are technically necessary to exercise
the rights in other media and formats. Subject to Section 8(f), all
rights not expressly granted by Licensor are hereby reserved,
including but not limited to the rights described in Section 4(e).

4. Restrictions. The license granted in Section 3 above is
expressly made subject to and limited by the following
restrictions:

1. You may Distribute or Publicly Perform the Work only under the
terms of this License. You must include a copy of, or the Uniform
Resource Identifier (URI) for, this License with every copy of the
Work You Distribute or Publicly Perform. You may not offer or
impose any terms on the Work that restrict the terms of this
License or the ability of the recipient of the Work to exercise the
rights granted to that recipient under the terms of the License.
You may not sublicense the Work. You must keep intact all notices
that refer to this License and to the disclaimer of warranties with
every copy of the Work You Distribute or Publicly Perform. When You
Distribute or Publicly Perform the Work, You may not impose any
effective technological measures on the Work that restrict the
ability of a recipient of the Work from You to exercise the rights
granted to that recipient under the terms of the License. This
Section 4(a) applies to the Work as incorporated in a Collection,
but this does not require the Collection apart from the Work itself
to be made subject to the terms of this License. If You create a
Collection, upon notice from any Licensor You must, to the extent
practicable, remove from the Collection any credit as required by
Section 4(d), as requested. If You create an Adaptation, upon
notice from any Licensor You must, to the extent practicable,
remove from the Adaptation any credit as required by Section 4(d),
as requested.

2. You may Distribute or Publicly Perform an Adaptation only under:
(i) the terms of this License; (ii) a later version of this License
with the same License Elements as this License; (iii) a Creative
Commons jurisdiction license (either this or a later license
version) that contains the same License Elements as this License
(e.g., Attribution-NonCommercial-ShareAlike 3.0 US) (``Applicable
License''). You must include a copy of, or the URI, for Applicable
License with every copy of each Adaptation You Distribute or
Publicly Perform. You may not offer or impose any terms on the
Adaptation that restrict the terms of the Applicable License or the
ability of the recipient of the Adaptation to exercise the rights
granted to that recipient under the terms of the Applicable
License. You must keep intact all notices that refer to the
Applicable License and to the disclaimer of warranties with every
copy of the Work as included in the Adaptation You Distribute or
Publicly Perform. When You Distribute or Publicly Perform the
Adaptation, You may not impose any effective technological measures
on the Adaptation that restrict the ability of a recipient of the
Adaptation from You to exercise the rights granted to that
recipient under the terms of the Applicable License. This Section
4(b) applies to the Adaptation as incorporated in a Collection, but
this does not require the Collection apart from the Adaptation
itself to be made subject to the terms of the Applicable License.

3. You may not exercise any of the rights granted to You in Section
3 above in any manner that is primarily intended for or directed
toward commercial advantage or private monetary compensation. The
exchange of the Work for other copyrighted works by means of
digital file-sharing or otherwise shall not be considered to be
intended for or directed toward commercial advantage or private
monetary compensation, provided there is no payment of any monetary
compensation in con-nection with the exchange of copyrighted
works.

4. If You Distribute, or Publicly Perform the Work or any
Adaptations or Collections, You must, unless a request has been
made pursuant to Section 4(a), keep intact all copyright notices
for the Work and provide, reasonable to the medium or means You are
utilizing: (i) the name of the Original Author (or pseudonym, if
applicable) if supplied, and/or if the Original Author and/or
Licensor designate another party or parties (e.g., a sponsor
institute, publishing entity, journal) for attribution
(``Attribution Parties'') in Licensor's copyright notice, terms of
service or by other reasonable means, the name of such party or
parties; (ii) the title of the Work if supplied; (iii) to the
extent reasonably practicable, the URI, if any, that Licensor
specifies to be associated with the Work, unless such URI does not
refer to the copyright notice or licensing information for the
Work; and, (iv) consistent with Section 3(b), in the case of an
Adaptation, a credit identifying the use of the Work in the
Adaptation (e.g., ``French translation of the Work by Original
Author,'' or ``Screenplay based on original Work by Original
Author''). The credit required by this Section 4(d) may be
implemented in any reasonable manner; provided, however, that in
the case of a Adaptation or Collection, at a minimum such credit
will appear, if a credit for all contributing authors of the
Adaptation or Collection appears, then as part of these credits and
in a manner at least as prominent as the credits for the other
contributing authors. For the avoidance of doubt, You may only use
the credit required by this Section for the purpose of attribution
in the manner set out above and, by exercising Your rights under
this License, You may not implicitly or explicitly assert or imply
any connection with, sponsorship or endorsement by the Original
Author, Licensor and/or Attribution Parties, as appropriate, of You
or Your use of the Work, without the separate, express prior
written permission of the Original Author, Licensor and/or
Attribution Parties.

5. For the avoidance of doubt: 1. Non-waivable Compulsory License
Schemes. In those jurisdictions in which the right to collect
royalties through any statutory or compulsory licensing scheme
cannot be waived, the Licensor reserves the exclusive right to
collect such royalties for any exercise by You of the rights
granted under this License; 2. Waivable Compulsory License Schemes.
In those jurisdictions in which the right to collect royalties
through any statutory or compulsory licensing scheme can be waived,
the Licensor reserves the exclusive right to collect such royalties
for any exercise by You of the rights granted under this License if
Your exercise of such rights is for a purpose or use which is
otherwise than noncommercial as permitted under Section 4(c) and
otherwise waives the right to collect royalties through any
statutory or compulsory licensing scheme; and,

3. Voluntary License Schemes. The Licensor reserves the right to
collect royalties, whether individually or, in the event that the
Licensor is a member of a collecting society that administers
voluntary licensing schemes, via that society, from any exercise by
You of the rights granted under this License that is for a purpose
or use which is otherwise than noncommercial as permitted under
Section 4(c). 6. Except as otherwise agreed in writing by the
Licensor or as may be otherwise permitted by applicable law, if You
Reproduce, Distribute or Publicly Perform the Work either by itself
or as part of any Adaptations or Collections, You must not distort,
mutilate, modify or take other derogatory action in relation to the
Work which would be prejudicial to the Original Author's honor or
reputation. Licensor agrees that in those jurisdictions (e.g.
Japan), in which any exercise of the right granted in Section 3(b)
of this License (the right to make Adaptations) would be deemed to
be a distortion, mutilation, modification or other derogatory
action prejudicial to the Original Author's honor and reputation,
the Licensor will waive or not assert, as appropriate, this
Section, to the fullest extent permitted by the applicable national
law, to enable You to reasonably exercise Your right under Section
3(b) of this License (right to make Adaptations) but not
otherwise.

5. Representations, Warranties and Disclaimer

UNLESS OTHERWISE MUTUALLY AGREED TO BY THE PARTIES IN WRITING AND
TO THE FULLEST EXTENT PERMITTED BY APPLICABLE LAW, LICENSOR OFFERS
THE WORK AS-IS AND MAKES NO REPRESENTATIONS OR WARRANTIES OF ANY
KIND CONCERNING THE WORK, EXPRESS, IMPLIED, STATUTORY OR OTHERWISE,
INCLUDING, WITHOUT LIMITATION, WARRANTIES OF TITLE,
MERCHANTABILITY, FITNESS FOR A PARTICULAR PURPOSE, NONINFRINGEMENT,
OR THE ABSENCE OF LATENT OR OTHER DEFECTS, ACCURACY, OR THE
PRESENCE OF ABSENCE OF ERRORS, WHETHER OR NOT DISCOVERABLE. SOME
JURISDICTIONS DO NOT ALLOW THE EXCLUSION OF IMPLIED WARRANTIES, SO
THIS EXCLUSION MAY NOT APPLY TO YOU.

6. Limitation on Liability. EXCEPT TO THE EXTENT REQUIRED BY
APPLICABLE LAW, IN NO EVENT WILL LICENSOR BE LIABLE TO YOU ON ANY
LEGAL THEORY FOR ANY SPECIAL, INCIDENTAL, CONSEQUENTIAL, PUNITIVE
OR EXEMPLARY DAMAGES ARISING OUT OF THIS LICENSE OR THE USE OF THE
WORK, EVEN IF LICENSOR HAS BEEN ADVISED OF THE POSSIBILITY OF SUCH
DAMAGES.

7. Termination

1. This License and the rights granted hereunder will terminate
automatically upon any breach by You of the terms of this License.
Individuals or entities who have received Adaptations or
Collections from You under this License, however, will not have
their licenses terminated provided such individuals or entities
remain in full compliance with those licenses. Sections 1, 2, 5, 6,
7, and 8 will survive any termination of this License.

2. Subject to the above terms and conditions, the license granted
here is perpetual (for the duration of the applicable copyright in
the Work). Notwithstanding the above, Licensor reserves the right
to release the Work under different license terms or to stop
distributing the Work at any time; provided, however that any such
election will not serve to withdraw this License (or any other
license that has been, or is required to be, granted under the
terms of this License), and this License will continue in full
force and effect unless terminated as stated above.

8. Miscellaneous

1. Each time You Distribute or Publicly Perform the Work or a
Collection, the Licensor offers to the recipient a license to the
Work on the same terms and conditions as the license granted to You
under this License. 2. Each time You Distribute or Publicly Perform
an Adaptation, Licensor offers to the recipient a license to the
original Work on the same terms and conditions as the license
granted to You under this License.

3. If any provision of this License is invalid or unenforceable
under applicable law, it shall not affect the validity or
enforceability of the remainder of the terms of this License, and
without further action by the parties to this agreement, such
provision shall be reformed to the minimum extent necessary to make
such provision valid and enforceable.

4. No term or provision of this License shall be deemed waived and
no breach consented to unless such waiver or consent shall be in
writing and signed by the party to be charged with such waiver or
consent. 5. This License constitutes the entire agreement between
the parties with respect to the Work licensed here. There are no
understandings, agreements or representations with respect to the
Work not specified here. Licensor shall not be bound by any
additional provisions that may appear in any communication from
You. This License may not be modified without the mutual written
agreement of the Licensor and You.

6. The rights granted under, and the subject matter referenced, in
this License were drafted utilizing the terminology of the Berne
Convention for the Protection of Literary and Artistic Works (as
amended on September 28, 1979), the Rome Convention of 1961, the
WIPO Copyright Treaty of 1996, the WIPO Performances and Phonograms
Treaty of 1996 and the Universal Copyright Convention (as revised
on July 24, 1971). These rights and subject matter take effect in
the relevant jurisdiction in which the License terms are sought to
be enforced according to the corresponding provisions of the
implementation of those treaty provisions in the applicable
national law. If the standard suite of rights granted under
applicable copyright law includes additional rights not granted
under this License, such additional rights are deemed to be
included in the License; this License is not intended to restrict
the license of any rights under applicable law.

Creative Commons Notice

Creative Commons is not a party to this License, and makes no
warranty whatsoever in connection with the Work. Creative Commons
will not be liable to You or any party on any legal theory for any
damages whatsoever, including without limitation any general,
special, incidental or consequential damages arising in connection
to this license. Notwithstanding the foregoing two (2) sentences,
if Creative Commons has expressly identified itself as the Licensor
hereunder, it shall have all rights and obligations of Licensor.

Except for the limited purpose of indicating to the public that the
Work is licensed under the CCPL, Creative Commons does not
authorize the use by either party of the trademark ``Creative
Commons'' or any related trademark or logo of Creative Commons
without the prior written consent of Creative Commons. Any
permitted use will be in compliance with Creative Commons'
then-current trademark usage guidelines, as may be published on its
website or otherwise made available upon request from time to time.
For the avoidance of doubt, this trademark restriction does not
form part of this License.

Creative Commons may be contacted at
\url{http://creativecommons.org/}.

\textit{For the Win} by
\href{http://craphound.com/ftw}{Cory Doctorow}
is licensed under a
\href{http://creativecommons.org/licenses/by-nc-sa/3.0/us/}{Creative Commons Attribution-Noncommercial-Share Alike 3.0 United States License}.

\end{document}
