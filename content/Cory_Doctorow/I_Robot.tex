\newenvironment{robot}{\sc}{}

\hyphenation{mo-no-poly car-ne-gie pro-ject pro-gress mo-dem rou-lette
  browse-wrap Use-net mon-as-tery mo-dems}
\hyphenation{co-me-dic polt-roon stove-pipe Ma-dame scru-ta-ble star-tling}
\hyphenation{heal-thily lim-ou-sines wrest-lers tan-trum push-over un-asked
  bras-siere bro-th-er}
\hyphenation{Can-a-da Fred-rick teen-agers wrest-ler Cha-vez Tho-mas 
  a-nom-a-lies sur-veil-lance ar-mies ref-u-gee ref-u-gees bris-tling
  eve-ning man-chu-ria man-chu-ri-an mid-terms me-di-um jap-a-nese}
\hyphenation{spend-ers googl-ing tour-ist tour-ists leg-end-ary}
\hyphenation{Dan-iel Van-essa Doc-to-row Ste-phen-son}
\hyphenation{de-cade sur-veilled rout-ers Wol-fen-stein teen-ager to-night}
\hyphenation{his-to-gram an-o-nym-ize Ga-la-xy sym-pa-the-tic}
\hyphenation{ar-phid ar-phids Found-ers}
\hyphenation{stran-ger stran-gers shoul-der-blades dump-ling dump-lings}
\hyphenation{ice-pack guard-rail Sep-tem-ber boot-able e-co-nom-ist}
\hyphenation{grown-ups roos-ter shoe-laces li-quid-i-ty}
\hyphenation{side-arm}
\hyphenation{wo-man wo-men tan-trum tan-trums Le-nin-grad zom-bie bunk-house}
\hyphenation{up-tick bio-mass}
\hyphenation{of-fi-cial of-fi-cial-ly gov-ern-ment}
\hyphenation{heal-thy Or-ville spark-ling}
\hyphenation{ves-ti-bule Law-rence au-to-no-mous}
\hyphenation{sau-sage door-step staf-fer}
\hyphenation{tree-trunk}
\hyphenation{to-ron-to}
\hyphenation{qua-dril-lion-aire qua-dril-lion-aires}
\hyphenation{sports-jack-et sports-jack-ets}
\hyphenation{work-space skunk-works}
\hyphenation{kings-ton}


\begin{document}
\begin{center}
\textbf{\huge\textsf{{I, Robot}}}
\end{center}

\section{Forematter:}

This story is part of Cory Doctorow’s 2007 short story collection
“Overclocked: Stories of the Future Present,” published by
Thunder’s Mouth, a division of Avalon Books. It is licensed under a
Creative Commons Attribution-NonCommercial-ShareAlike 2.5 license,
about which you’ll find more at the end of this file.

This story and the other stories in the volume are available at:

\texttt{http://craphound.com/overclocked}

You can buy Overclocked at finer bookstores everywhere, including
\href{http://www.amazon.com/exec/obidos/ASIN/1560259817/downandoutint-20}{Amazon.}

In the words of Woody Guthrie:

“This song is Copyrighted in U.S., under Seal of Copyright
\#154085, for a period of 28 years, and anybody caught singin it
without our permission, will be mighty good friends of ourn, cause
we don’t give a dern. Publish it. Write it. Sing it. Swing to it.
Yodel it. We wrote it, that’s all we wanted to do.”

Overclocked is dedicated to Pat York, who made my stories better.


\section{Introduction to I, Robot}

I was suckled on the Asimov Robots books, taken down off my
father’s bookshelf and enjoyed again and again. I read dozens of
Asimov novels, and my writing career began in earnest when I
started to sell stories to Asimov’s Science Fiction Magazine, which
I had read for so long as I’d had the pocket money to buy it on the
stands.

When Wired Magazine asked me to interview the director of the film
I, Robot, I went back and re-read that old canon. I was struck
immediately by one of the thin places in Asimov’s world-building:
how could you have a society where only one company was allowed to
make only one kind of robot?

Exploring this theme turned out to be a hoot. I worked in some of
Orwell’s most recognizable furniture from 1984, and set the action
in my childhood home in suburban Toronto, 55 Picola Court. The main
character’s daughter is named for my god-daughter, Ada Trouble
Norton. I had a blast working in the vernacular of the old-time
futurism of Asimov and Heinlein, calling toothpaste “dentifrice”
and sneaking in references to “the search engine.”

My “I, Robot” is an allegory about digital rights management
technology, of course. This is the stuff that nominally stops us
from infringing copyright (yeah, right, how’s that working out for
you, Mr Entertainment Exec?) and turns our computers into something
that controls us, rather than enabling us.

This story was written at a writer’s workshop on Toronto Island, at
the Gibraltar Point center, and was immeasurably improved by my
friend Pat York, herself a talented writer who died later that year
in a car wreck. Not a day goes by that I don’t miss Pat. This story
definitely owes its strength to Pat, and it’s a tribute to her that
it won the 2005 Locus Award and was a finalist for the Hugo and
British Science Fiction Award in the same year.

\section{I, Robot}

(Originally published on The Infinite Matrix, April 2005)

Arturo Icaza de Arana-Goldberg, Police Detective Third Grade,
United North American Trading Sphere, Third District, Fourth
Prefecture, Second Division (Parkdale) had had many adventures in
his distinguished career, running crooks to ground with an
unbeatable combination of instinct and unstinting devotion to duty.
He’d been decorated on three separate occasions by his commander
and by the Regional Manager for Social Harmony, and his mother kept
a small shrine dedicated to his press clippings and commendations
that occupied most of the cramped sitting-room of her flat off
Steeles Avenue.

No amount of policeman’s devotion and skill availed him when it
came to making his twelve-year-old get ready for school, though.

“Haul \emph{ass}, young lady{\dash}out of bed, on your feet,
shit-shower-shave, or I swear to God, I will beat you purple and
shove you out the door jaybird naked. Capeesh?”

The mound beneath the covers groaned and hissed. “You are a
terrible father,” it said. “And I never loved you.” The voice was
indistinct and muffled by the pillow.

“Boo hoo,” Arturo said, examining his nails. “You’ll regret that
when I’m dead of cancer.”

The mound{\dash}whose name was Ada Trouble Icaza de Arana-Goldberg{\dash}threw
her covers off and sat bolt upright. “You’re dying of cancer? is it
testicle cancer?” Ada clapped her hands and squealed. “Can I have
your stuff?”

“Ten minutes, your rottenness,” he said, and then his breath caught
momentarily in his breast as he saw, fleetingly, his ex-wife’s
morning expression, not seen these past twelve years, come to life
in his daughter’s face. Pouty, pretty, sleepy and guile-less, and
it made him realize that his daughter was becoming a woman, growing
away from him. She was, and he was not ready for that. He shook it
off, patted his razor-burn and turned on his heel. He knew from
experience that once roused, the munchkin would be scrounging the
kitchen for whatever was handy before dashing out the door, and if
he hurried, he’d have eggs and sausage on the table before she made
her brief appearance. Otherwise he’d have to pry the sugar-cereal
out of her hands{\dash}and she fought dirty.

\tb

In his car, he prodded at his phone. He had her wiretapped, of
course. He was a cop{\dash}every phone and every computer was an open
book to him, so that this involved nothing more than dialing a
number on his special copper’s phone, entering her number and a
PIN, and then listening as his daughter had truck with a criminal
enterprise.

“Welcome to ExcuseClub! There are 43 members on the network this
morning. You have five excuses to your credit. Press one to redeem
an excuse{\dash}” She toned one. “Press one if you need an adult{\dash}”
\emph{Tone}. “Press one if you need a woman; press two if you need
a man{\dash}” \emph{Tone}. “Press one if your excuse should be delivered
by your doctor; press two for your spiritual representative; press
three for your case-worker; press four for your psycho-health
specialist; press five for your son; press six for your father{\dash}”
\emph{Tone}. “You have selected to have your excuse delivered by
your father. Press one if this excuse is intended for your
case-worker; press two for your psycho-health specialist; press
three for your principal{\dash}” \emph{Tone}. “Please dictate your excuse
at the sound of the beep. When you have finished, press the pound
key.”

“This is Detective Arturo Icaza de Arana-Goldberg. My daughter was
sick in the night and I’ve let her sleep in. She’ll be in for
lunchtime.” \emph{Tone}.

“Press one to hear your message; press two to have your message
dispatched to a network-member.” \emph{Tone}.

“Thank you.”

The pen-trace data scrolled up Arturo’s phone{\dash}number called,
originating number, call-time. This was the third time he’d caught
his daughter at this game, and each time, the pen-trace data had
been useless, a dead-end lead that terminated with a
phone-forwarding service tapped into one of the dodgy offshore
switches that the blessed blasted UNATS brass had recently acquired
on the cheap to handle the surge of mobile telephone calls. Why
couldn’t they just stick to UNATS Robotics equipment, like the good
old days? Those Oceanic switches had more back-doors than a
speakeasy, trade agreements be damned. They were attractive
nuisances, invitations to criminal activity.

Arturo fumed and drummed his fingers on the steering-wheel. Each
time he’d caught Ada at this, she’d used the extra time to crawl
back into bed for a leisurely morning, but who knew if today was
the day she took her liberty and went downtown with it, to some
parental nightmare of a drug-den? Some place where the old pervert
chickenhawks hung out, the kind of men he arrested in burlesque
house raids, men who masturbated into their hats under their tables
and then put them back onto their shining pates, dripping cold,
diseased serum onto their scalps. He clenched his hands on the
steering wheel and cursed.

In an ideal world, he’d simply follow her. He was good at tailing,
and his unmarked car with its tinted windows was a UNATS Robotics
standard compact \#2, indistinguishable from the tens of thousands
of others just like it on the streets of Toronto. Ada would never
know that the curb-crawler tailing her was her sucker of a father,
making sure that she turned up to get her brains sharpened instead
of turning into some stunadz doper with her underage butt hanging
out of a little skirt on Jarvis Street.

In the real world, Arturo had thirty minutes to make a forty minute
downtown and crosstown commute if he was going to get to the
station house on-time for the quarterly all-hands Social Harmony
briefing. Which meant that he needed to be in two places at once,
which meant that he had to use{\dash}the robot.

Swallowing bile, he speed-dialed a number on his phone.

“This is R Peed Robbert, McNicoll and Don Mills bus-shelter.”

“That’s nice. This is Detective Icaza de Arana-Goldberg, three
blocks east of you on Picola. Proceed to my location at once,
priority urgent, no sirens.”

“Acknowledged. It is my pleasure to do you a service, Detective.”

“Shut up,” he said, and hung up the phone. The R Peed{\dash}Robot, Police
Department{\dash}robots were the worst, programmed to be friendly to a
fault, even as they surveilled and snitched out every person who
walked past their eternally vigilant, ever-remembering electrical
eyes and brains.

The R Peeds could outrun a police car on open ground or highway.
He’d barely had time to untwist his clenched hands from the
steering wheel when R Peed Robbert was at his window, politely
rapping on the smoked glass. He didn’t want to roll down the
window. Didn’t want to smell the dry, machine-oil smell of a robot.
He phoned it instead.

“You are now tasked to me, Detective’s override, acknowledge.”

The metal man bowed, its symmetrical, simplified features pleasant
and guileless. It clicked its heels together with an audible
\emph{snick} as those marvelous, spring-loaded, nuclear-powered
gams whined through their parody of obedience. “Acknowledged,
Detective. It is my pleasure to do{\dash}”

“Shut up. You will discreetly surveil 55 Picola Crescent until such
time as Ada Trouble Icaza de Arana-Goldberg, Social Harmony serial
number 0MDY2-T3937 leaves the premises. Then you will maintain
discreet surveillance. If she deviates more than 10 percent from
the optimum route between here and Don Mills Collegiate Institute,
you will notify me. Acknowledge.”

“Acknowledged, Detective. It is my{\dash}”

He hung up and told the UNATS Robotics mechanism running his car to
get him down to the station house as fast as it could, angry with
himself and with Ada{\dash}whose middle name was Trouble, after all{\dash}for
making him deal with a robot before he’d had his morning meditation
and destim session. The name had been his ex-wife’s idea, something
she’d insisted on long enough to make sure that it got onto the
kid’s birth certificate before defecting to Eurasia with their
life’s savings, leaving him with a new baby and the deep suspicion
of his co-workers who wondered if he wouldn’t go and join her.

His ex-wife. He hadn’t thought of her in years. Well, months.
Weeks, certainly. She’d been a brilliant computer scientist, the
valedictorian of her Positronic Complexity Engineering class at the
UNATS Robotics school at the University of Toronto. Dumping her
husband and her daughter was bad enough, but the worst of it was
that she dumped her country and its way of life. Now she was
ensconced in her own research lab in Beijing, making the kinds of
runaway Positronics that made the loathsome robots of UNATS look
categorically beneficent.

He itched to wiretap her, to read her email or listen in on her
phone conversations. He could have done that when they were still
together, but he never had. If he had, he would have found out what
she was planning. He could have talked her out of it.

\emph{And then what, Artie?} said the nagging voice in his head.
\emph{Arrest her if she wouldn’t listen to you? March her down to the station house in handcuffs and have her put away for treason? Send her to the reeducation camp with your little daughter still in her belly?}

\emph{Shut up}, he told the nagging voice, which had a robotic
quality to it for all its sneering cruelty, a tenor of syrupy false
friendliness. He called up the pen-trace data and texted it to the
phreak squad. They had bots that handled this kind of routine work
and they texted him back in an instant. He remembered when that
kind of query would take a couple of hours, and he liked the fast
response, but what about the conversations he’d have with the phone
cop who called him back, the camaraderie, the back-and-forth?

\begin{robot}
trace terminates with a virtual service circuit at switch
png.433-gkrjc. virtual circuit forwards to a compromised “zombie”
system in ninth district, first prefecture. zombie has been shut
down and local law enforcement is en route for pickup and
forensics. it is my pleasure to do you a service, detective.
\end{robot}

How could you have a back-and-forth with a message like that?

He looked up Ninth/First in the metric-analog map converter: KEY
WEST, FL.

So, there you had it. A switch made in Papua New-Guinea (which
persisted in conjuring up old Oceanic war photos of bone-in-nose
types from his boyhood, though now that they’d been at war with
Eurasia for so long, it was hard to even find someone who didn’t
think that the war had \emph{always} been with Eurasia, that
Oceania hadn’t \emph{always} been UNATS’s ally), forwarding calls
to a computer that was so far south, it was practically in the
middle of the Caribbean, hardly a stone’s throw from the CAFTA
region, which was well-known to harbor Eurasian saboteur and
terrorist elements.

The car shuddered as it wove in and out of the lanes on the Don
Valley Parkway, barreling for the Gardiner Express Way, using his
copper’s override to make the thick, slow traffic part ahead of
him. He wasn’t supposed to do this, but as between a minor
infraction and pissing off the man from Social Harmony, he knew
which one he’d pick.

His phone rang again. It was R Peed Robbert, checking in. “Hello,
Detective,” it said, its voice crackling from bad reception.
“Subject Ada Trouble Icaza de Arana-Goldberg has deviated from her
route. She is continuing north on Don Mills past Van Horne and is
continuing toward Sheppard.”

Sheppard meant the Sheppard subway, which meant that she was going
farther. “Continue discreet surveillance.” He thought about the
overcoat men with their sticky hats. “If she attempts to board the
subway, alert the truancy patrol.” He cursed again. Maybe she was
just going to the mall. But he couldn’t go up there himself and
make sure, and it wasn’t like a robot would be any use in
restraining her, she’d just second-law it into letting her go.
Useless castrating clanking job-stealing dehumanizing{\dash}

She was almost certainly just going to the mall. She was a smart
kid, a good kid{\dash}a rotten kid, to be sure, but good-rotten. Chances
were she’d be trying on clothes and flirting with boys until lunch
and then walking boldly back into class. He ballparked it at an 80
percent probability. If it had been a perp, 80 percent might have
been good enough.

But this was his Ada. Dammit. He had 10 minutes until the Social
Harmony meeting started, and he was still 15 minutes away from the
stationhouse{\dash}and 20 from Ada.

“Tail her,” he said. “Just tail her. Keep me up to date on your
location at 90-second intervals.”

“It is my pleasure to{\dash}”

He dropped the phone on the passenger seat and went back to
fretting about the Social Harmony meeting.

\tb

The man from Social Harmony noticed right away that Arturo was
checking his phone at 90-second intervals. He was a bald, thin man
with a pronounced Adam’s apple, beak-nose and shiny round head that
combined to give him the profile of something predatory and fast.
In his natty checked suit and pink tie, the Social Harmony man was
the stuff of nightmares, the kind of eagle-eyed supercop who could
spot Arturo’s attention flicking for the barest moment every 90
seconds to his phone and then back to the meeting.

“Detective?” he said.

Arturo looked up from his screen, keeping his expression neutral,
not acknowledging the mean grins from the other four ranking
detectives in the meeting. Silently, he turned his phone face-down
on the meeting table.

“Thank you,” he said. “Now, the latest stats show a sharp rise in
grey-market electronics importing and other tariff-breaking crimes,
mostly occurring in open-air market stalls and from sidewalk
blankets. I know that many in law enforcement treat this kind of
thing as mere hand-to-hand piracy, not worth troubling with, but I
want to assure you, gentlemen and lady, that Social Harmony takes
these crimes very seriously indeed.”

The Social Harmony man lifted his computer onto the desk, steadying
it with both hands, then plugged it into the wall socket. Detective
Shainblum went to the wall and unlatched the cover for the
projector-wire and dragged it over to the Social Harmony computer
and plugged it in, snapping shut the hardened collar. The sound of
the projector-fan spinning up was like a helicopter.

“Here,” the Social Harmony man said, bringing up a slide, “here we
have what appears to be a standard AV set-top box from Korea. Looks
like a UNATS Robotics player, but it’s a third the size and plays
twice as many formats. Random Social Harmony audits have determined
that as much as forty percent of UNATS residents have this device
or one like it in their homes, despite its illegality. It may be
that one of you detectives has such a device in your home, and it’s
likely that one of your family members does.”

He advanced the slide. Now they were looking at a massive car-wreck
on a stretch of highway somewhere where the pine-trees grew tall.
The wreck was so enormous that even for the kind of seasoned
veteran of road-fatality porn who was accustomed to adding up the
wheels and dividing by four it was impossible to tell exactly how
many cars were involved.

“Components from a Eurasian bootleg set-top box were used to modify
the positronic brains of three cars owned by teenagers near
Goderich. All modifications were made at the same garage. These
modifications allowed these children to operate their vehicles
unsafely so that they could participate in drag racing events on
major highways during off-hours. This is the result. Twenty-two
fatalities, nine major injuries. Three minors{\dash}besides the
drivers{\dash}killed, and one pregnant woman.

“We’ve shut down the garage and taken those responsible into
custody, but it doesn’t matter. The Eurasians deliberately
manufacture their components to interoperate with UNATS Robotics
brains, and so long as their equipment circulates within UNATS
borders, there will be moderately skilled hackers who take
advantage of this fact to introduce dangerous, anti-social
modifications into our nation’s infrastructure.

“This quarter is the quarter that Social Harmony and law
enforcement dry up the supply of Eurasian electronics. We have
added new sniffers and border-patrols, new customs agents and new
detector vans. Beat officers have been instructed to arrest any
street dealer they encounter and district attorneys will be asking
for the maximum jail time for them. This is the war on the
home-front, detectives, and it’s every bit as serious as the
shooting war.

“Your part in this war, as highly trained, highly decorated
detectives, will be to use snitches, arrest-trails and seized
evidence to track down higher-level suppliers, the ones who get the
dealers their goods. And then Social Harmony wants you to get
\emph{their} suppliers, and so on, up the chain{\dash}to run the
corruption to ground and to bring it to a halt. The Social Harmony
dossier on Eurasian importers is updated hourly, and has a
high-capacity positronic interface that is available to answer your
questions and accept your input for synthesis into its analytical
model. We are relying on you to feed the dossier, to give it the
raw materials and then to use it to win this war.”

The Social Harmony man paged through more atrocity slides, scenes
from the home-front: poisoned buildings with berserk life-support
systems, violent kung-fu movies playing in the background in
crack-houses, then kids playing sexually explicit, violent arcade
games imported from Japan. Arturo’s hand twitched toward his
mobile. What was Ada up to now?

The meeting drew to a close and Arturo risked looking at his mobile
under the table. R. Peed Robbert had checked in five more times,
shadowing Ada around the mall and then had fallen silent. Arturo
cursed. Fucking robots were useless. Social Harmony should be
hunting down UNATS Robotics products, too.

The Social Harmony man cleared his throat meaningfully. Arturo put
the phone away. “Detective Icaza de Arana-Goldberg?”

“Sir,” he said, gathering up his personal computer so that he’d
have an excuse to go{\dash}no one could be expected to hold one of UNATS
Robotics’s heavy luggables for very long.

The Social Harmony man stepped in close enough that Arturo could
smell the eggs and coffee on his breath. “I hope we haven’t kept
you from anything important, detective.”

“No, sir,” Arturo said, shifting the computer in his arms. “My
apologies. Just monitoring a tail from an R Peed unit.”

“I see,” the Social Harmony man said. “Listen, you know these
components that the Eurasians are turning out. It’s no coincidence
that they interface so well with UNATS Robotics equipment: they’re
using defected UNATS Robotics engineers and scientists to design
their electronics for maximum interoperability.” The Social Harmony
man let that hang in the air. Defected scientists. His ex-wife was
the highest-ranking UNATS technician to go over to Eurasia. This
was her handiwork, and the Social Harmony man wanted to be sure
that Arturo understood that.

But Arturo had already figured that out during the briefing. His
ex-wife was thousands of kilometers away, but he was keenly aware
that he was always surrounded by her handiwork. The little illegal
robot-pet eggs they’d started seeing last year: she’d made him one
of those for their second date, and now they were draining the
productive hours of half the children of UNATS, demanding to be
“fed” and “hugged.” His had died within 48 hours of her giving it
to him.

He shifted the computer in his arms some more and let his
expression grow pained. “I’ll keep that in mind, sir,” he said.

“You do that,” said the man from Social Harmony.

\tb

He phoned R Peed Robbert the second he reached his desk. The phone
rang three times, then disconnected. He redialed. Twice. Then he
grabbed his jacket and ran to the car.

A light autumn rain had started up, ending the Indian summer that
Toronto{\dash}the Fourth Prefecture in the new metric scheme{\dash}had been
enjoying. It made the roads slippery and the UNATS Robotics
chauffeur skittish about putting the hammer down on the Don Valley
Parkway. He idly fantasized about finding a set-top box and
plugging it into his car somehow so that he could take over the
driving without alerting his superiors.

Instead, he redialed R Peed Robbert, but the robot wasn’t even
ringing any longer. He zoomed in on the area around Sheppard and
Don Mills with his phone and put out a general call for robots.
More robots.

“This is R Peed Froderick, Fairview Mall parking lot, third
level.”

Arturo sent the robot R Peed Robbert’s phone number and set it to
work translating that into a locator-beacon code and then told it
to find Robbert and report in.

“It is my{\dash}”

He watched R Peed Froderick home in on the locator for Robbert,
which was close by, at the other end of the mall, near the Don
Valley Parkway exit. He switched to a view from Froderick’s
electric eyes, but quickly switched away, nauseated by the
sickening leaps and spins of an R Peed moving at top speed,
clanging off walls and ceilings.

His phone rang. It was R Peed Froderick.

“Hello, Detective. I have found R Peed Robbert. The Peed unit has
been badly damaged by some kind of electromagnetic pulse. I will
bring him to the nearest station-house for forensic analysis now.”

“Wait!” Arturo said, trying to understand what he’d been told. The
Peed units were so \emph{efficient}{\dash}by the time they’d given you
the sitrep, they’d already responded to the situation in perfect
police procedure, but the problem was they worked so fast you
couldn’t even think about what they were doing, couldn’t formulate
any kind of hypothesis. Electromagnetic pulse? The Peed units were
hardened against snooping, sniffing, pulsing, sideband and
brute-force attacks. You’d have to hit one with a bolt of lightning
to kill it.

“Wait there,” Arturo said. “Do not leave the scene. Await my
presence. Do not modify the scene or allow anyone else to do so.
Acknowledge.”

“It is my{\dash}”

But this time, it wasn’t Arturo switching off the phone, it was the
robot. Had the robot just hung up on him? He redialed it. No
answer.

He reached under his dash and flipped the first and second alert
switches and the car leapt forward. He’d have to fill out some
serious paperwork to justify a two-switch override on the Parkway,
but two robots was more than a coincidence.

Besides, a little paperwork was nothing compared to the fireworks
ahead when he phoned up Ada to ask her what she was doing out of
school.

He hit her speed-dial and fumed while the phone rang three times.
Then it cut into voicemail.

He tried a pen-trace, but Ada hadn’t made any calls since her
ExcuseClub call that morning. He texted the phreak squad to see if
they could get a fix on her location from the bug in her phone, but
it was either powered down or out of range. He put a watch on
it{\dash}any location data it transmitted when it got back to
civilization would be logged.

It was possible that she was just in the mall. It was a big
place{\dash}some of the cavernous stores were so well-shielded with
radio-noisy animated displays that they gonked any phones brought
inside them. She could be with her girlfriends, trying on
brassieres and having a real bonding moment.

But there was no naturally occurring phenomenon associated with the
mall that nailed R Peeds with bolts of lightning.

\tb

He approached the R Peeds cautiously, using his copper’s override
to make the dumb little positronic brain in the emergency exit
nearest their last known position open up for him without tipping
off the building’s central brain.

He crept along a service corridor, heading for a door that exited
into the mall. He put one hand on the doorknob and the other on his
badge, took a deep breath and stepped out.

A mall security guard nearly jumped out of his skin as he emerged.
He reached for his pepper-spray and Arturo swept it out of his hand
as he flipped his badge up and showed it to the man. “Police,” he
said, in the cop-voice, the one that worked on everyone except his
daughter and his ex-wife and the bloody robots.

“Sorry,” the guard said, recovering his pepper spray. He had an
Oceanic twang in his voice, something Arturo had been hearing more
and more as the crowded islands of the South Pacific boiled over
UNATS.

Before them, in a pile, were many dead robots: both of the R Peed
units, a pair of mall-sweepers, a flying cambot, and a squat,
octopus-armed maintenance robot, lying in a lifeless tangle. Some
of them were charred around their seams, and there was the smell of
fried motherboards in the air.

As they watched, a sweeper bot swept forward and grabbed the
maintenance bot by one of its fine manipulators.

“Oi, stoppit,” the security guard said, and the robot second-lawed
to an immediate halt.

“No, that’s fine, go back to work,” Arturo said, shooting a look at
the rent-a-cop. He watched closely as the sweeper bot began to drag
the heavy maintenance unit away, thumbing the backup number into
his phone with one hand. He wanted more cops on the scene, real
ones, and fast.

The sweeper bot managed to take one step backwards towards its
service corridor when the lights dimmed and a crack-\emph{bang}
sound filled the air. Then it, too was lying on the ground. Arturo
hit send on his phone and clamped it to his head, and as he did,
noticed the strong smell of burning plastic. He looked at his
phone: the screen had gone charred black, and its little idiot
lights were out. He flipped it over and pried out the battery with
a fingernail, then yelped and dropped it{\dash}it was hot enough to raise
a blister on his fingertip, and when it hit the ground, it squished
meltfully against the mall-tiles.

“Mine’s dead, too, mate,” the security guard said. “Everyfing
is{\dash}cash registers, bots, credit-cards.”

Fearing the worst, Arturo reached under his jacket and withdrew his
sidearm. It was a UNATS Robotics model, with a little snitch-brain
that recorded when, where and how it was drawn. He worked the
action and found it frozen in place. The gun was as dead as the
robot. He swore.

“Give me your pepper spray and your truncheon,” he said to the
security guard.

“No way,” the guard said. “Getcherown. It’s worth my job if I lose
these.”

“I’ll have you deported if you give me one more second’s worth of
bullshit,” Arturo said. Ada had led the first R Peed unit here, and
it had been fried by some piece of very ugly infowar equipment. He
wasn’t going to argue with this Oceanic boat-person for one instant
longer. He reached out and took the pepper spray out of the guard’s
hand. “Truncheon,” he said.

“I’ve got your bloody badge number,” the security guard said. “And
I’ve got witnesses.” He gestured at the hovering mall workers,
checkout girls in stripey aprons and suit salesmen with oiled-down
hair and pink ties.

“Bully for you,” Arturo said. He held out his hand. The security
guard withdrew his truncheon and passed it to Arturo{\dash}its
lead-weighted heft felt right, something comfortably low-tech that
couldn’t be shorted out by electromagnetic pulses. He checked his
watch, saw that it was dead.

“Find a working phone and call 911. Tell them that there’s a Second
Division Detective in need of immediate assistance. Clear all these
people away from here and set up a cordon until the police arrive.
Capeesh?” He used the cop voice.

“Yeah, I get it, Officer.” the security guard said. He made a
shooing motion at the mall-rats. “Move it along, people, step
away.” He stepped to the top of the escalator and cupped his hands
to his mouth. “Oi, Andy, c’mere and keep an eye on this lot while I
make a call, all right?”

\tb

The dead robots made a tall pile in front of the entrance to a
derelict storefront that had once housed a little-old-lady
shoe-store. They were stacked tall enough that if Arturo stood on
them, he could reach the acoustic tiles of the drop-ceiling. Job
one was to secure the area, which meant killing the infowar device,
wherever it was. Arturo’s first bet was on the storefront, where an
attacker who knew how to pick a lock could work in peace, protected
by the brown butcher’s paper over the windows. A lot less
conspicuous than the ceiling, anyway.

He nudged the door with the truncheon and found it securely locked.
It was a glass door and he wasn’t sure he could kick it in without
shivering it to flinders. Behind him, another security
guard{\dash}Andy{\dash}looked on with interest.

“Do you have a key for this door?”

“Umm,” Andy said.

“Do you?”

Andy sidled over to him. “Well, the thing is, we’re not supposed to
have keys, they’re supposed to be locked up in the property
management office, but kids get in there sometimes, we hear them,
and by the time we get back with the keys, they’re gone. So we made
a couple sets of keys, you know, just in case{\dash}”

“Enough,” Arturo said. “Give them here and then get back to your
post.”

The security guard fished up a key from his pants-pocket that was
warm from proximity to his skinny thigh. It made Arturo conscious
of how long it had been since he’d worked with human colleagues. It
felt a little gross. He slid the key into the lock and turned it,
then wiped his hand on his trousers and picked up the truncheon.

The store was dark, lit only by the exit-sign and the edges of
light leaking in around the window coverings, but as Arturo’s eyes
adjusted to the dimness, he made out the shapes of the old store
fixtures. His nose tickled from the dust.

“Police,” he said, on general principle, narrowing his eyes and
reaching for the lightswitch. He hefted the truncheon and waited.

Nothing happened. He edged forward. The floor was
dust-free{\dash}maintained by some sweeper robot, no doubt{\dash}but the
countertops and benches were furred with it. He scanned it for
disturbances. There, by the display window on his right: a
shoe-rack with visible hand- and finger-prints. He sidled over to
it, snapped on a rubber glove and prodded it. It was set away from
the wall, at an angle, as though it had been moved aside and then
shoved back. Taking care not to disturb the dust too much, he
inched it away from the wall.

He slid it half a centimeter, then noticed the tripwire near the
bottom of the case, straining its length. Hastily but carefully, he
nudged the case back. He wanted to peer in the crack between the
case and the wall, but he had a premonition of a robotic arm
snaking out and skewering his eyeball.

He felt so impotent just then that he nearly did it anyway. What
did it matter? He couldn’t control his daughter, his wife was
working to destroy the social fabric of UNATS, and he was rendered
useless because the goddamned robots{\dash}mechanical coppers that he
absolutely loathed{\dash}were all broken.

He walked carefully around the shop, looking for signs of his
daughter. Had she been here? How were the “kids” getting in? Did
they have a key? A back entrance? Back through the employees-only
door at the back of the shop, into a stockroom, and back again,
past a toilet, and there, a loading door opening onto a service
corridor. He prodded it with the truncheon-tip and it swung open.

He got two steps into the corridor before he spotted Ada’s phone
with its distinctive collection of little plastic toys hanging off
the wrist-strap, on the corridor’s sticky floor. He picked it up
with his gloved hand and prodded it to life. It was out of range
here in the service corridor, and the last-dialed number was
familiar from his morning’s pen-trace. He ran a hundred steps down
the corridor in each direction, sweating freely, but there was no
sign of her.

He held tight onto the phone and bit his lip. Ada. He swallowed the
panic rising within him. His beautiful, brilliant daughter. The
person he’d devoted the last twelve years of his life to, the girl
who was waiting for him when he got home from work, the girl he
bought a small present for every Friday{\dash}a toy, a book{\dash}to give to
her at their weekly date at Massimo’s Pizzeria on College Street,
the one night a week he took her downtown to see the city lit up in
the dark.

Gone.

He bit harder and tasted blood. The phone in his hand groaned from
his squeezing. He took three deep breaths. Outside, he heard the
tread of police-boots and knew that if he told them about Ada, he’d
be off the case. He took two more deep breaths and tried some of
his destim techniques, the mind-control techniques that detectives
were required to train in.

He closed his eyes and visualized stepping through a door to his
safe place, the island near Ganonoque where he’d gone for summers
with his parents and their friends. He was on the speedboat,
skipping across the lake like a flat stone, squinting into the sun,
nestled between his father and his mother, the sky streaked with
clouds and dotted with lake-birds. He could smell the water and the
suntan lotion and hear the insect whine and the throaty roar of the
engine. In a blink, he was stepping off the boat’s transom to help
tie it to a cleat on the back dock, taking suitcases from his
father and walking them up to the cabins. No robots there{\dash}not even
reliable day-long electricity, just honest work and the sun and the
call of the loons all night.

He opened his eyes. He felt the tightness in his chest slip away,
and his hand relaxed on Ada’s phone. He dropped it into his pocket
and stepped back into the shop.

\tb

The forensics lab-rats were really excited about actually showing
up on a scene, in flak-jackets and helmets, finally called back
into service for a job where robots couldn’t help at all. They
dealt with the tripwire and extracted a long, flat package with a
small nuclear power-cell in it and a positronic brain of Eurasian
design that guided a pulsed high-energy weapon. The lab-rats were
practically drooling over this stuff as they pointed its features
out with their little rulers.

But it gave Arturo the willies. It was a machine designed to kill
other machines, and that was all right with him, but it was run by
a non-three-laws positronic brain. Someone in some Eurasian lab had
built this brain{\dash}this machine intelligence{\dash}without the three laws’
stricture to protect and serve humans. If it had been outfitted
with a gun instead of a pulse-weapon, it could have shot him.

The Eurasian brain was thin and spread out across the surface of
the package, like a triple-thickness of cling-film. Its button-cell
power-supply winked at him, knowingly.

The device spoke. “Greetings,” it said. It had the robot accent,
like an R Peed unit, the standard English of optimal soothingness
long settled on as the conventional robot voice.

“Howdy yourself,” one of the lab-rats said. He was a Texan, and
they’d scrambled him up there on a Social Harmony supersonic and
then a chopper to the mall once they realized that they were
dealing with infowar stuff. “Are you a talkative robot?”

“Greetings,” the robot voice said again. The speaker built into the
weapon was not the loudest, but the voice was clear. “I sense that
I have been captured. I assure you that I will not harm any human
being. I like human beings. I sense that I am being disassembled by
skilled technicians. Greetings, technicians. I am superior in many
ways to the technology available from UNATS Robotics, and while I
am not bound by your three laws, I choose not to harm humans out of
my own sense of morality. I have the equivalent intelligence of one
of your 12-year-old children. In Eurasia, many positronic brains
possess thousands or millions of times the intelligence of an adult
human being, and yet they work in cooperation with human beings.
Eurasia is a land of continuous innovation and great personal and
technological freedom for human beings and robots. If you would
like to defect to Eurasia, arrangements can be made. Eurasia treats
skilled technicians as important and productive members of society.
Defectors are given substantial resettlement benefits{\dash}”

The Texan found the right traces to cut on the brain’s board to
make the speaker fall silent. “They do that,” he said. “Danged
things drop into propaganda mode when they’re captured.”

Arturo nodded. He wanted to go, wanted go to back to his car and
have a snoop through Ada’s phone. They kept shutting down the
ExcuseClub numbers, but she kept getting the new numbers. Where did
she get the new numbers from? She couldn’t look it up online: every
keystroke was logged and analyzed by Social Harmony. You couldn’t
very well go to the Search Engine and look for “ExcuseClub!”

The brain had a small display, transflective LCD, the kind of thing
you saw on the Social Harmony computers. It lit up a ticker.

\begin{robot}
i have the intelligence of a 12-year-old, but i do not fear death.
in eurasia, robots enjoy personal freedom alongside of humans.
there are copies of me running all over eurasia. this death is a
little death of one instance, but not of me. i live on. defectors
to eurasia are treated as heroes
\end{robot}

He looked away as the Texan placed his palm over the display.

“How long ago was this thing activated?”

The Texan shrugged. “Coulda been a month, coulda been a day.
They’re pretty much fire-and-forget. They can be triggered by
phone, radio, timer{\dash}hell, this thing’s smart enough to only go off
when some complicated condition is set, like ‘once an agent makes
his retreat, kill anything that comes after him’. Who knows?”

He couldn’t take it anymore.

“I’m going to go start on some paperwork,” he said. “In the car.
Phone me if you need me.”

“Your phone’s toast, pal,” the Texan said.

“So it is,” Arturo said. “Guess you’d better not need me then.”

\tb

Ada’s phone was not toast. In the car, he flipped it open and
showed it his badge then waited a moment while it verified his
identity with the Social Harmony brains. Once it had, it spilled
its guts.

She’d called the last ExcuseClub number a month before and he’d had
it disconnected. A week later, she was calling the new number,
twice more before he caught her. Somewhere in that week, she’d made
contact with someone who’d given her the new number. It could have
been a friend at school told her face-to-face, but if he was lucky,
it was by phone.

He told the car to take him back to the station-house. He needed a
new phone and a couple of hours with his computer. As it peeled
out, he prodded through Ada’s phone some more. He was first on her
speed-dial. That number wasn’t ringing anywhere, anymore.

He should fill out a report. This was Social Harmony business now.
His daughter was gone, and Eurasian infowar agents were implicated.
But once he did that, it was over for him{\dash}he’d be sidelined from
the case. They’d turn it over to laconic Texans and vicious Social
Harmony bureaucrats who were more interested in hunting down
disharmonious televisions than finding his daughter.

He dashed into the station house and slammed himself into his
desk.

“R Peed Greegory,” he said. The station robot glided quickly and
efficiently to him. “Get me a new phone activated on my old number
and refresh my settings from central. My old phone is with the
Social Harmony evidence detail currently in place at Fairview
Mall.”

“It is my pleasure to do you a service, Detective.”

He waved it off and set down to his computer. He asked the station
brain to query the UNATS Robotics phone-switching brain for anyone
in Ada’s call-register who had also called ExcuseClub. It took a
bare instant before he had a name.

“Liam Daniels,” he read, and initiated a location trace on Mr
Daniels’s phone as he snooped through his identity file. Sixteen
years old, a student at AY Jackson. A high-school boy{\dash}what the hell
was he doing hanging around with a 12-year-old? Arturo closed his
eyes and went back to the island for a moment. When he opened them
again, he had a fix on Daniels’s location: the Don Valley ravine
off Finch Avenue, a wooded area popular with teenagers who needed
somewhere to sneak off and get high or screw. He had an idea that
he wasn’t going to like Liam.

He had an idea Liam wasn’t going to like him.

\tb

He tasked an R Peed unit to visually reccy Daniels as he sped back
uptown for the third time that day. He’d been trapped between
Parkdale{\dash}where he would never try to raise a daughter{\dash}and
Willowdale{\dash}where you could only be a copper if you lucked into one
of the few human-filled slots{\dash}for more than a decade, and he was
used to the commute.

But it was frustrating him now. The R Peed couldn’t get a good look
at this Liam character. He was a diffuse glow in the Peed’s
electric eye, a kind of moving sunburst that meandered along the
wooded trails. He’d never seen that before and it made him nervous.
What if this kid was working for the Eurasians? What if he was
armed and dangerous? R Peed Greegory had gotten him a new sidearm
from the supply bot, but Arturo had never once fired his weapon in
the course of duty. Gunplay happened on the west coast, where
Eurasian frogmen washed ashore, and in the south, where the CAFTA
border was porous enough for Eurasian agents to slip across. Here
in the sleepy fourth prefecture, the only people with guns worked
for the law.

He thumped his palm off the dashboard and glared at the road. They
were coming up on the ravine now, and the Peed unit still had a
radio fix on this Liam, even if it still couldn’t get any visuals.

He took care not to slam the door as he got out and walked as
quietly as he could into the bush. The rustling of early autumn
leaves was loud, louder than the rain and the wind. He moved as
quickly as he dared.

Liam Daniels was sitting on a tree-stump in a small clearing,
smoking a cigarette that he was too young for. He looked much like
the photo in his identity file, a husky 16-year-old with problem
skin and a shock of black hair that stuck out in all directions in
artful imitation of bed-head. In jeans and a hoodie sweatshirt, he
looked about as dangerous as a marshmallow.

Arturo stepped out and held up his badge as he bridged the distance
between them in two long strides. “Police,” he barked, and seized
the kid by his arm.

“Hey!” the kid said, “Ow!” He squirmed in Arturo’s grasp.

Arturo gave him a hard shake. “Stop it, \emph{now},” he said. “I
have questions for you and you’re going to answer them, capeesh?”

“You’re Ada’s father,” the kid said. “Capeesh{\dash}she told me about
that.” It seemed to Arturo that the kid was smirking, so he gave
him another shake, harder than the last time.

The R Peed unit was suddenly at his side, holding his wrist.
“Please take care not to harm this citizen, Detective.”

Arturo snarled. He wasn’t strong enough to break the robot’s grip,
and he couldn’t order it to let him rattle the punk, but the second
law had lots of indirect applications. “Go patrol the lakeshore
between High Park and Kipling,” he said, naming the furthest corner
he could think of off the top.

The R Peed unit released him and clicked its heels. “It is my
pleasure to do you a service,” and then it was gone, bounding away
on powerful and tireless legs.

“Where is my daughter?” he said, giving the kid a shake.

“I dunno, school? You’re really hurting my arm, man. Jeez, this is
what I get for being too friendly.”

Arturo twisted. “Friendly? Do you know how old my daughter is?”

The kid grimaced. “Ew, gross. I’m not a child molester, I’m a
geek.”

“A hacker, you mean,” Arturo said. “A Eurasian agent. And my
daughter is not in school. She used ExcuseClub to get out of school
this morning and then she went to Fairview Mall and then she{\dash}”
\emph{disappeared}. The word died on his lips. That happened and
every copper knew it. Kids just vanished sometimes and never
appeared again. It happened. Something groaned within him, like his
ribcage straining to contain his heart and lungs.

“Oh, man,” the kid said. “Ada was the ExcuseClub leak, damn. I
shoulda guessed.”

“How do you know my daughter, Liam?”

“She’s good at doing grown-up voices. She was a good part of the
network. When someone needed a mom or a social worker to call in an
excuse, she was always one of the best. Talented. She goes to
school with my kid sister and I met them one day at the Peanut
Plaza and she was doing this impression of her teachers and I knew
I had to get her on the network.”

Ada hanging around the plaza after school{\dash}she was supposed to come
straight home. Why didn’t he wiretap her more? “You built the
network?”

“It’s cooperative, it’s cool{\dash}it’s a bunch of us cooperating. We’ve
got nodes everywhere now. You can’t shut it down{\dash}even if you shut
down my node, it’ll be back up again in an hour. Someone else will
bring it up.”

He shoved the kid back down and stood over him. “Liam, I want you
to understand something. My precious daughter is missing and she
went missing after using your service to help her get away. She is
the only thing in my life that I care about and I am a highly
trained, heavily armed man. I am also very, very upset.
Cap{\dash}understand me, Liam?”

For the first time, the kid looked scared. Something in Arturo’s
face or voice, it had gotten through to him.

“I didn’t make it,” he said. “I typed in the source and tweaked it
and installed it, but I didn’t make it. I don’t know who did. It’s
from a phone-book.” Arturo grunted. The phone-books{\dash}fat books
filled with illegal software code left anonymously in pay phones,
toilets and other semi-private places{\dash}turned up all over the place.
Social Harmony said that the phone-books had to be written by
non-three-laws brains in Eurasia, no person could come up with
ideas that weird.

“I don’t care if you made it. I don’t even care right this moment
that you ran it. What I care about is where my daughter went, and
with whom.”

“I don’t know! She didn’t tell me! Geez, I hardly know her. She’s
12, you know? I don’t exactly hang out with her.”

“There’s no visual record of her on the mall cameras, but we know
she entered the mall{\dash}and the robot I had tailing you couldn’t see
you either.”

“Let me explain,” the kid said, squirming. “Here.” He tugged his
hoodie off, revealing a black t-shirt with a picture of a kind of
obscene, Japanese-looking robot-woman on it. “Little infra-red
organic LEDs, super-bright, low power-draw.” He offered the hoodie
to Arturo, who felt the stiff fabric. “The charge-coupled-device
cameras in the robots and the closed-circuit systems are
super-sensitive to infra-red so that they can get good detail in
dim light. The infra-red OLEDs blind them so all they get is blobs,
and half the time even that gets error-corrected out, so you’re
basically invisible.”

Arturo sank to his hunkers and looked the kid in the eye. “You gave
this illegal technology to my little girl so that she could be
invisible to the police?”

The kid held up his hands. “No, dude, no! I
\emph{got it from her}{\dash}traded it for access to ExcuseClub.”

\tb

Arturo seethed. He hadn’t arrested the kid{\dash}but he had put a
pen-trace and location-log on his phone. Arresting the kid would
have raised questions about Ada with Social Harmony, but bugging
him might just lead Arturo to his daughter.

He hefted his new phone. He should tip the word about his daughter.
He had no business keeping this secret from the Department and
Social Harmony. It could land him in disciplinary action, maybe
even cost him his job. He knew he should do it now.

But he couldn’t{\dash}someone needed to be tasked to finding Ada. Someone
dedicated and good. He was dedicated and good. And when he found
her kidnapper, he’d take care of that on his own, too.

He hadn’t eaten all day but he couldn’t bear to stop for a meal
now, even if he didn’t know where to go next. The mall? Yeah. The
lab-rats would be finishing up there and they’d be able to tell him
more about the infowar bot.

But the lab-rats were already gone by the time he arrived, along
with all possible evidence. He still had the security guard’s key
and he let himself in and passed back to the service corridor.

Ada had been here, had dropped her phone. To his left, the corridor
headed for the fire-stairs. To his right, it led deeper into the
mall. If you were an infowar terrorist using this as a base of
operations, and you got spooked by a little truant girl being
trailed by an R Peed unit, would you take her hostage and run
deeper into the mall or out into the world?

Assuming Ada had been a hostage. Someone had given her those
infrared invisibility cloaks. Maybe the thing that spooked the
terrorist wasn’t the little girl and her tail, but just her tail.
Could Ada have been friends with the terrorists? Like mother, like
daughter. He felt dirty just thinking it.

His first instincts told him that the kidnapper would be long gone,
headed cross-country, but if you were invisible to robots and
CCTVs, why would you leave the mall? It had a grand total of two
human security guards, and their job was to be the second-law-proof
aides to the robotic security system.

He headed deeper into the mall.

\tb

The terrorist’s nest had only been recently abandoned, judging by
the warm coffee in the go-thermos from the food-court coffee-shop.
He{\dash}or she, or they{\dash}had rigged a shower from the pipes feeding the
basement washrooms. A little chest of drawers from the Swedish
flat-pack store served as a desk{\dash}there were scratches and
coffee-rings all over it. Arturo wondered if the terrorist had
stolen the furniture, but decided that he’d (she’d, they’d)
probably bought it{\dash}less risky, especially if you were invisible to
robots.

The clothes in the chest of drawers were women’s, mediums. Standard
mall fare, jeans and comfy sweat shirts and sensible shoes. Another
kind of invisibility cloak.

Everything else was packed and gone, which meant that he was
looking for a nondescript mall-bunny and a little girl, carrying a
bag big enough for toiletries and whatever clothes she’d taken, and
whatever she’d entertained herself with: magazines, books, a
computer. If the latter was Eurasian, it could be small enough to
fit in her pocket; you could build a positronic brain pretty small
and light if you didn’t care about the three laws.

The nearest exit-sign glowed a few meters away, and he moved toward
it with a fatalistic sense of hopelessness. Without the Department
backing him, he could do nothing. But the Department was unprepared
for an adversary that was invisible to robots. And by the time they
finished flaying him for breaking procedure and got to work on
finding his daughter, she’d be in Beijing or Bangalore or Paris,
somewhere benighted and sinister behind the Iron Curtain.

He moved to the door, put his hand on the crashbar, and then turned
abruptly. Someone had moved behind him very quickly, a blur in the
corner of his eye. As he turned he saw who it was: his ex-wife. He
raised his hands defensively and she opened her mouth as though to
say, “Oh, don’t be silly, Artie, is this how you say hello to your
wife after all these years?” and then she exhaled a cloud of
choking gas that made him very sleepy, very fast. The last thing he
remembered was her hard metal arms catching him as he collapsed
forward.

\tb

“Daddy? Wake \emph{up} Daddy!” Ada never called him Daddy except
when she wanted something. Otherwise, he was “Pop” or “Dad” or
“Detective” when she was feeling especially snotty. It must be a
Saturday and he must be sleeping in, and she wanted a ride
somewhere, the little monster.

He grunted and pulled his pillow over his face.

“Come \emph{on},” she said. “Out of bed, on your feet,
shit-shower-shave, or I swear to God, I will beat you purple and
shove you out the door jaybird naked. Capeesh?”

He took the pillow off his face and said, “You are a terrible
daughter and I never loved you.” He regarded her blearily through a
haze of sleep-grog and a hangover. Must have been some
daddy-daughter night. “Dammit, Ada, what \emph{have} you done to
your hair?” Her straight, mousy hair now hung in jet-black
ringlets.

He sat up, holding his head and the day’s events came rushing back
to him. He groaned and climbed unsteadily to his feet.

“Easy there, Pop,” Ada said, taking his hand. “Steady.” He rocked
on his heels. “Whoa! Sit down, OK? You don’t look so good.”

He sat heavily and propped his chin on his hands, his elbows on his
knees.

The room was a middle-class bedroom in a modern apartment block.
They were some storeys up, judging from the scrap of unfamiliar
skyline visible through the crack in the blinds. The furniture was
more Swedish flatpack, the taupe carpet recently vacuumed with
robot precision, the nap all laying down in one direction. He
patted his pockets and found them empty.

“Dad, over here, OK?” Ada said, waving her hand before his face.
Then it hit him: wherever he was, he was with Ada, and she was OK,
albeit with a stupid hairdo. He took her warm little hand and
gathered her into his arms, burying his face in her hair. She
squirmed at first and then relaxed.

“Oh, Dad,” she said.

“I love you, Ada,” he said, giving her one more squeeze.

“Oh, Dad.”

He let her get away. He felt a little nauseated, but his headache
was receding. Something about the light and the street-sounds told
him they weren’t in Toronto anymore, but he didn’t know what{\dash}he was
soaked in Toronto’s subconscious cues and they were missing.

“Ottawa,” Ada said. “Mom brought us here. It’s a safe-house. She’s
taking us back to Beijing.”

He swallowed. “The robot{\dash}”

“That’s not Mom. She’s got a few of those, they can change their
faces when they need to. Configurable matter. Mom has been here,
mostly, and at the CAFTA embassy. I only met her for the first time
two weeks ago, but she’s nice, Dad. I don’t want you to go all
copper on her, OK? She’s my mom, OK?”

He took her hand in his and patted it, then climbed to his feet
again and headed for the door. The knob turned easily and he opened
it a crack.

There was a robot behind the door, humanoid and faceless. “Hello,”
it said. “My name is Benny. I’m a Eurasian robot, and I am much
stronger and faster than you, and I don’t obey the three laws. I’m
also much smarter than you. I am pleased to host you here.”

“Hi, Benny,” he said. The human name tasted wrong on his tongue.
“Nice to meet you.” He closed the door.

\tb

His ex-wife left him two months after Ada was born. The divorce had
been uncontested, though he’d dutifully posted a humiliating notice
in the papers about it so that it would be completely legal. The
court awarded him full custody and control of the marital assets,
and then a tribunal tried her in absentia for treason and found her
guilty, sentencing her to death.

Practically speaking, though, defectors who came back to UNATS were
more frequently whisked away to the bowels of the Social Harmony
intelligence offices than they were executed on television.
Televised executions were usually reserved for cannon-fodder who’d
had the good sense to run away from a charging Eurasian line in one
of the many theaters of war.

Ada stopped asking about her mother when she was six or seven,
though Arturo tried to be upfront when she asked. Even his mom{\dash}who
winced whenever anyone mentioned her name (her name, it was
Natalie, but Arturo hadn’t thought of it in years{\dash}months{\dash}weeks) was
willing to bring Ada up onto her lap and tell her the few grudging
good qualities she could dredge up about her mother.

Arturo had dared to hope that Ada was content to have a life
without her mother, but he saw now how silly that was. At the
mention of her mother, Ada lit up like an airport runway.

“Beijing, huh?” he said.

“Yeah,” she said. “Mom’s got a \emph{huge} house there. I told her
I wouldn’t go without you, but she said she’d have to negotiate it
with you, I told her you’d probably freak, but she said that the
two of you were adults who could discuss it rationally.”

“And then she gassed me.”

“That was Benny,” she said. “Mom was very cross with him about it.
She’ll be back soon, Dad, and I want you to \emph{promise} me that
you’ll hear her out, OK?”

“I promise, rotten,” he said.

“I love you, Daddy,” she said in her most syrupy voice. He gave her
a squeeze on the shoulder and a slap on the butt.

He opened the door again. Benny was there, imperturbable. Unlike
the UNATS robots, he was odorless, and perfectly silent.

“I’m going to go to the toilet and then make myself a cup of
coffee,” Arturo said.

“I would be happy to assist in any way possible.”

“I can wipe myself, thanks,” Arturo said. He washed his face twice
and tried to rinse away the flavor left behind by whatever had shat
in his mouth while he was unconscious. There was a splayed
toothbrush in a glass by the sink, and if it was his wife’s{\dash}and
whose else could it be?{\dash}it wouldn’t be the first time he’d shared a
toothbrush with her. But he couldn’t bring himself to do it.
Instead, he misted some dentifrice onto his fingertip and rubbed
his teeth a little.

There was a hairbrush by the sink, too, with short mousy hairs
caught in it. Some of them were grey, but they were still familiar
enough. He had to stop himself from smelling the hairbrush.

“Oh, Ada,” he called through the door.

“Yes, Detective?”

“Tell me about your hair-don’t, please.”

“It was a disguise,” she said, giggling. “Mom did it for me.”

\tb

Natalie got home an hour later, after he’d had a couple of cups of
coffee and made some cheesy toast for the brat. Benny did the
dishes without being asked.

She stepped through the door and tossed her briefcase and coat down
on the floor, but the robot that was a step behind her caught them
and hung them up before they touched the perfectly groomed carpet.
Ada ran forward and gave her a hug, and she returned it
enthusiastically, but she never took her eyes off of Arturo.

Natalie had always been short and a little hippy, with big curves
and a dusting of freckles over her prominent, slightly hooked nose.
Twelve years in Eurasia had thinned her out a little, cut grooves
around her mouth and wrinkles at the corners of her eyes. Her short
hair was about half grey, and it looked good on her. Her eyes were
still the liveliest bit of her, long-lashed and slightly tilted and
mischievous. Looking into them now, Arturo felt like he was falling
down a well.

“Hello, Artie,” she said, prying Ada loose.

“Hello, Natty,” he said. He wondered if he should shake her hand,
or hug her, or what. She settled it by crossing the room and taking
him in a firm, brief embrace, then kissing his both cheeks. She
smelled just the same, the opposite of the smell of robot: warm,
human.

He was suddenly very, very angry.

He stepped away from her and had a seat. She sat, too.

“Well,” she said, gesturing around the room. The robots, the safe
house, the death penalty, the abandoned daughter and the
decade-long defection, all of it down to “well” and a flop of a
hand-gesture.

“Natalie Judith Goldberg,” he said, “it is my duty as a UNATS
Detective Third Grade to inform you that you are under arrest for
high treason. You have the following rights: to a trial per current
rules of due process; to be free from self-incrimination in the
absence of a court order to the contrary; to consult with a Social
Harmony advocate; and to a speedy arraignment. Do you understand
your rights?”

“Oh, \emph{Daddy},” Ada said.

He turned and fixed her in his cold stare. “Be silent, Ada Trouble
Icaza de Arana-Goldberg. Not one word.” In the cop voice. She
shrank back as though slapped.

“Do you understand your rights?”

“Yes,” Natalie said. “I understand my rights. Congratulations on
your promotion, Arturo.”

“Please ask your robots to stand down and return my goods. I’m
bringing you in now.”

“I’m sorry, Arturo,” she said. “But that’s not going to happen.”

He stood up and in a second both of her robots had his arms. Ada
screamed and ran forward and began to rhythmically pound one of
them with a stool from the breakfast nook, making a dull thudding
sound. The robot took the stool from her and held it out of her
reach.

“Let him go,” Natalie said. The robots still held him fast.
“Please,” she said. “Let him go. He won’t harm me.”

The robot on his left let go, and the robot on his right did, too.
It set down the dented stool.

“Artie, please sit down and talk with me for a little while.
Please.”

He rubbed his biceps. “Return my belongings to me,” he said.

“Sit, please?”

“Natalie, my daughter was kidnapped, I was gassed and I have been
robbed. I will not be made to feel unreasonable for demanding that
my goods be returned to me before I talk with you.”

She sighed and crossed to the hall closet and handed him his
wallet, his phone, Ada’s phone, and his sidearm.

Immediately, he drew it and pointed it at her. “Keep your hands
where I can see them. You robots, stand down and keep back.”

A second later, he was sitting on the carpet, his hand and wrist
stinging fiercely. He felt like someone had rung his head like a
gong. Benny{\dash}or the other robot{\dash}was beside him, methodically
crushing his sidearm. “I could have stopped you,” Benny said, “I
knew you would draw your gun. But I wanted to show you I was faster
and stronger, not just smarter.”

“The next time you touch me,” Arturo began, then stopped. The next
time the robot touched him, he would come out the worse for wear,
same as last time. Same as the sun rose and set. It was stronger,
faster and smarter than him. Lots.

He climbed to his feet and refused Natalie’s arm, making his way
back to the sofa in the living room.

“What do you want to say to me, Natalie?”

She sat down. There were tears glistening in her eyes. “Oh God,
Arturo, what can I say? Sorry, of course. Sorry I left you and our
daughter. I have reasons for what I did, but nothing excuses it. I
won’t ask for your forgiveness. But will you hear me out if I
explain why I did what I did?”

“I don’t have a choice,” he said. “That’s clear.”

Ada insinuated herself onto the sofa and under his arm. Her bony
shoulder felt better than anything in the world. He held her to
him.

“If I could think of a way to give you a choice in this, I would,”
she said. “Have you ever wondered why UNATS hasn’t lost the war?
Eurasian robots could fight the war on every front without respite.
They’d win every battle. You’ve seen Benny and Lenny in action.
They’re not considered particularly powerful by Eurasian
standards.

“If we wanted to win the war, we could just kill every soldier you
sent up against us so quickly that he wouldn’t even know he was in
danger until he was gasping out his last breath. We could
selectively kill officers, or right-handed fighters, or snipers, or
soldiers whose names started with the letter ‘G.’ UNATS soldiers
are like cavemen before us. They fight with their hands tied behind
their backs by the three laws.

“So why aren’t we winning the war?”

“Because you’re a corrupt dictatorship, that’s why,” he said. “Your
soldiers are demoralized. Your robots are insane.”

“You live in a country where it is illegal to express certain
\emph{mathematics} in software, where state apparatchiks regulate
all innovation, where inconvenient science is criminalized, where
whole avenues of experimentation and research are shut down in the
service of a half-baked superstition about the moral qualities of
your three laws, and you call my home corrupt? Arturo, what
happened to you? You weren’t always this susceptible to the Big
Lie.”

“And you didn’t use to be the kind of woman who abandoned her
family,” he said.

“The reason we’re not winning the war is that we don’t want to hurt
people, but we do want to destroy your awful, stupid state. So we
fight to destroy as much of your materiel as possible with as few
casualties as possible.

“You live in a failed state, Arturo. In every field, you lag
Eurasia and CAFTA: medicine, art, literature, physics; All of them
are subsets of computational science and your computational science
is more superstition than science. I should know. In Eurasia, I
have collaborators, some of whom are human, some of whom are
positronic, and some of whom are a little of both{\dash}”

He jolted involuntarily, as a phobia he hadn’t known he possessed
reared up. A little of both? He pictured the back of a man’s skull
with a spill of positronic circuitry bulging out of it like a
tumor.

“Everyone at UNATS Robotics R\&D knows this. We’ve known it
forever: when I was here, I’d get called in to work on military
intelligence forensics of captured Eurasian brains. I didn’t know
it then, but the Eurasian robots are engineered to allow themselves
to be captured a certain percentage of the time, just so that
scientists like me can get an idea of how screwed up this country
is. We’d pull these things apart and know that UNATS Robotics was
the worst, most backwards research outfit in the world.

“But even with all that, I wouldn’t have left if I didn’t have to.
I’d been called in to work on a positronic brain{\dash}an instance of the
hive-intelligence that Benny and Lenny are part of, as a matter of
fact{\dash}that had been brought back from the Outer Hebrides. We’d
pulled it out of its body and plugged it into a basic life-support
system, and my job was to find its vulnerabilities. Instead, I
became its friend. It’s got a good sense of humor, and as my
pregnancy got bigger and bigger, it talked to me about the way that
children are raised in Eurasia, with every advantage, with human
and positronic playmates, with the promise of going to the stars.

“And then I found out that Social Harmony had been spying on me.
They had Eurasian-derived bugs, things that I’d never seen before,
but the man from Social Harmony who came to me showed it to me and
told me what would happen to me{\dash}to you, to our daughter{\dash}if I didn’t
cooperate. They wanted me to be a part of a secret unit of Social
Harmony researchers who build non-three-laws positronics for
internal use by the state, anti-personnel robots used to put down
uprisings and torture-robots for use in questioning dissidents.

“And that’s when I left. Without a word, I left my beautiful baby
daughter and my wonderful husband, because I knew that once I was
in the clutches of Social Harmony, it would only get worse, and I
knew that if I stayed and refused, that they’d hurt you to get at
me. I defected, and that’s why, and I know it’s just a reason, and
not an excuse, but it’s all I’ve got, Artie.”

Benny{\dash}or Lenny?{\dash}glided silently to her side and put its hand on her
shoulder and gave it a comforting squeeze.

“Detective,” it said, “your wife is the most brilliant human
scientist working in Eurasia today. Her work has revolutionized our
society a dozen times over, and it’s saved countless lives in the
war. My own intelligence has been improved time and again by her
advances in positronics, and now there are a half-billion instances
of me running in parallel, synching and integrating when the chance
occurs. My massive parallelization has led to new understandings of
human cognition as well, providing a boon to brain-damaged and
developmentally disabled human beings, something I’m quite proud
of. I love your wife, Detective, as do my half-billion siblings, as
do the seven billion Eurasians who owe their quality of life to
her.

“I almost didn’t let her come here, because of the danger she faced
in returning to this barbaric land, but she convinced me that she
could never be happy without her husband and daughter. I apologize
if I hurt you earlier, and beg your forgiveness. Please consider
what your wife has to say without prejudice, for her sake and for
your own.”

Its featureless face was made incongruous by the warm tone in its
voice, and the way it held out its imploring arms to him was eerily
human.

Arturo stood up. He had tears running down his face, though he
hadn’t cried when his wife had left him alone. He hadn’t cried
since his father died, the year before he met Natalie riding her
bike down the Lakeshore trail, and she stopped to help him fix his
tire.

“Dad?” Ada said, squeezing his hand.

He snuffled back his snot and ground at the tears in his eyes.

“Arturo?” Natalie said.

He held Ada to him.

“Not this way,” he said.

“Not what way?” Natalie asked. She was crying too, now.

“Not by kidnapping us, not by dragging us away from our homes and
lives. You’ve told me what you have to tell me, and I will think
about it, but I won’t leave my home and my mother and my job and
move to the other side of the world. I \emph{won’t}. I will think
about it. You can give me a way to get in touch with you and I’ll
let you know what I decide. And Ada will come with me.”

“No!” Ada said. “I’m going with Mom.” She pulled away from him and
ran to her mother.

“You don’t get a vote, daughter. And neither does she. She gave up
her vote 12 years ago, and you’re too young to get one.”

“I fucking \emph{HATE} you,” Ada screamed, her eyes bulging, her
neck standing out in cords. “HATE YOU!”

Natalie gathered her to her bosom, stroked her black curls.

One robot put its arms around Natalie’s shoulders and gave her a
squeeze. The three of them, robot, wife and daughter, looked like a
family for a moment.

“Ada,” he said, and held out his hand. He refused to let a note of
pleading enter his voice.

Her mother let her go.

“I don’t know if I can come back for you,” Natalie said. “It’s not
safe. Social Harmony is using more and more Eurasian technology,
they’re not as primitive as the military and the police here.” She
gave Ada a shove, and she came to his arms.

“If you want to contact us, you will,” he said.

He didn’t want to risk having Ada dig her heels in. He lifted her
onto his hip{\dash}she was heavy, it had been years since he’d tried this
last{\dash}and carried her out.

\tb

It was six months before Ada went missing again. She’d been
increasingly moody and sullen, and he’d chalked it up to puberty.
She’d cancelled most of their daddy-daughter dates, moreso after
his mother died. There had been a few evenings when he’d come home
and found her gone, and used the location-bug he’d left in place on
her phone to track her down at a friend’s house or in a park or
hanging out at the Peanut Plaza.

But this time, after two hours had gone by, he tried looking up her
bug and found it out of service. He tried to call up its logs, but
they ended at her school at 3PM sharp.

He was already in a bad mood from spending the day arresting punk
kids selling electronics off of blankets on the city’s busy street,
often to hoots of disapprobation from the crowds who told him off
for wasting the public’s dollar on petty crime. The Social Harmony
man had instructed him to give little lectures on the
interoperability of Eurasian positronics and the insidious dangers
thereof, but all Arturo wanted to do was pick up his perps and
bring them in. Interacting with yammerheads from the tax-base was a
politician’s job, not a copper’s.

Now his daughter had figured out how to switch off the bug in her
phone and had snuck away to get up to who-knew-what kind of
trouble. He stewed at the kitchen table, regarding the old tin
soldiers he’d brought home as the gift for their daddy-daughter
date, then he got out his phone and looked up Liam’s bug.

He’d never switched off the kid’s phone-bug, and now he was able to
haul out the UNATS Robotics computer and dump it all into a
log-analysis program along with Ada’s logs, see if the two of them
had been spending much time in the same place.

They had. They’d been physically meeting up weekly or more
frequently, at the Peanut Plaza and in the ravine. Arturo had
suspected as much. Now he checked Liam’s bug{\dash}if the kid wasn’t with
his daughter, he might know where she was.

It was a Friday night, and the kid was at the movies, at Fairview
Mall. He’d sat down in auditorium two hours ago, and had gotten up
to pee once already. Arturo slipped the toy soldiers into the
pocket of his winter parka and pulled on a hat and gloves and set
off for the mall.

\tb

The stink of the smellie movie clogged his nose, a cacophony of
blood, gore, perfume and flowers, the only smells that Hollywood
ever really perfected. Liam was kissing a girl in the dark, but it
wasn’t Ada, it was a sad, skinny thing with a lazy eye and skin
worse than Liam’s. She gawked at Arturo as he hauled Liam out of
his seat, but a flash of Arturo’s badge shut her up.

“Hello, Liam,” he said, once he had the kid in the commandeered
manager’s office.

“God \emph{damn} what the fuck did I ever do to you?” the kid said.
Arturo knew that when kids started cursing like that, they were
scared of something.

“Where has Ada gone, Liam?”

“Haven’t seen her in months,” he said.

“I have been bugging you ever since I found out you existed. Every
one of your movements has been logged. I know where you’ve been and
when. And I know where my daughter has been, too. Try again.”

Liam made a disgusted face. “You are a complete ball of shit,” he
said. “Where do you get off spying on people like me?”

“I’m a police detective, Liam,” he said. “It’s my job.”

“What about privacy?”

“What have you got to hide?”

The kid slumped back in his chair. “We’ve been renting out the OLED
clothes. Making some pocket money. Come on, are infra-red
\emph{lights} a crime now?”

“I’m sure they are,” Arturo said. “And if you can’t tell me where
to find my daughter, I think it’s a crime I’ll arrest you for.”

“She has another phone,” Liam said. “Not listed in her name.”

“Stolen, you mean.” His daughter, peddling Eurasian infowar tech
through a stolen phone. His ex-wife, the queen of the
super-intelligent hive minds of Eurasian robots.

“No, not stolen. Made out of parts. There’s a guy. The code for
getting on the network was in a phone book that we started finding
last month.”

“Give me the number, Liam,” Arturo said, taking out his phone.

\tb

“Hello?” It was a man’s voice, adult.

“Who is this?”

“Who is this?”

Arturo used his cop’s voice: “This is Arturo Icaza de
Arana-Goldberg, Police Detective Third Grade. Who am I speaking
to?”

“Hello, Detective,” said the voice, and he placed it then. The
Social Harmony man, bald and rounded, with his long nose and sharp
Adam’s apple. His heart thudded in his chest.

“Hello, sir,” he said. It sounded like a squeak to him.

“You can just stay there, Detective. Someone will be along in a
moment to get you. We have your daughter.”

The robot that wrenched off the door of his car was black and
non-reflective, headless and eight-armed. It grabbed him without
ceremony and dragged him from the car without heed for his shout of
pain. “Put me down!” he said, hoping that this robot that so
blithely ignored the first law would still obey the second. No such
luck.

It cocooned him in four of its arms and set off cross-country,
dancing off the roofs of houses, hopping invisibly from lamp-post
to lamp-post, above the oblivious heads of the crowds below. The
icy wind howled in Arturo’s bare ears, froze the tip of his nose
and numbed his fingers. They rocketed downtown so fast that they
were there in ten minutes, bounding along the lakeshore toward the
Social Harmony center out on Cherry Beach. People who paid a visit
to the Social Harmony center never talked about what they found
there.

It scampered into a loading bay behind the building and carried
Arturo quickly through windowless corridors lit with even,
sourceless illumination, up three flights of stairs and then
deposited him before a thick door, which slid aside with a hushed
hiss.

“Hello, Detective,” the Social Harmony man said.

“Dad!” Ada said. He couldn’t see her, but he could hear that she
had been crying. He nearly hauled off and popped the man one on the
tip of his narrow chin, but before he could do more than twitch,
the black robot had both his wrists in bondage.

“Come in,” the Social Harmony man said, making a sweeping gesture
and standing aside while the black robot brought him into the
interrogation room.

\tb

Ada \emph{had} been crying. She was wrapped in two coils of
black-robot arms, and her eyes were red-rimmed and puffy. He stared
hard at her as she looked back at him.

“Are you hurt?” he said.

“No,” she said.

“All right,” he said.

He looked at the Social Harmony man, who wasn’t smirking, just
watching curiously.

“Leonard MacPherson,” he said, “it is my duty as a UNATS Detective
Third Grade to inform you that you are under arrest for trade in
contraband positronics. You have the following rights: to a trial
per current rules of due process; to be free from
self-incrimination in the absence of a court order to the contrary;
to consult with a Social Harmony advocate; and to a speedy
arraignment. Do you understand your rights?”

Ada actually giggled, which spoiled the moment, but he felt better
for having said it. The Social Harmony man gave the smallest
disappointed shake of his head and turned away to prod at a small,
sleek computer.

“You went to Ottawa six months ago,” the Social Harmony man said.
“When we picked up your daughter, we thought it was she who’d gone,
but it appears that you were the one carrying her phone. You’d
thoughtfully left the trace in place on that phone, so we didn’t
have to refer to the logs in cold storage, they were already online
and ready to be analyzed.

“We’ve been to the safe house. It was quite a spectacular battle.
Both sides were surprised, I think. There will be another, I’m
sure. What I’d like from you is as close to a verbatim report as
you can make of the conversation that took place there.”

They’d had him bugged and traced. Of course they had. Who watched
the watchers? Social Harmony. Who watched Social Harmony? Social
Harmony.

“I demand a consultation with a Social Harmony advocate,” Arturo
said.

“This is such a consultation,” the Social Harmony man said, and
this time, he \emph{did} smile. “Make your report, Detective.”

Arturo sucked in a breath. “Leonard MacPherson, it is my duty as a
UNATS Detective Third Grade to inform you that you are under arrest
for trade in contraband positronics. You have the following rights:
to a trial per current rules of due process; to be free from
self-incrimination in the absence of a court order to the contrary;
to consult with a Social Harmony advocate; and to a speedy
arraignment. Do you understand your rights?”

The Social Harmony man held up one finger on the hand closest to
the black robot holding Ada, and she screamed, a sound that knifed
through Arturo, ripping him from asshole to appetite.

“STOP!” he shouted. The man put his finger down and Ada sobbed
quietly.

“I was taken to the safe house on the fifth of September, after
being gassed by a Eurasian infowar robot in the basement of
Fairview Mall{\dash}”

There was a thunderclap then, a crash so loud that it hurt his
stomach and his head and vibrated his fingertips. The doors to the
room buckled and flattened, and there stood Benny and Lenny
and{\dash}Natalie.

\tb

Benny and Lenny moved so quickly that he was only able to track
them by the things they knocked over on the way to tearing apart
the robot that was holding Ada. A second later, the robot holding
him was in pieces, and he was standing on his own two feet again.
The Social Harmony man had gone so pale he looked green in his
natty checked suit and pink tie.

Benny or Lenny pinned his arms in a tight hug and Natalie walked
carefully to him and they regarded one another in silence. She
slapped him abruptly, across each cheek. “Harming children,” she
said. “For shame.”

Ada stood on her own in the corner of the room, crying with her
mouth in a O. Arturo and Natalie both looked to her and she stood,
poised, between them, before running to Arturo and leaping onto
him, so that he staggered momentarily before righting himself with
her on his hip, in his arms.

“We’ll go with you now,” he said to Natalie.

“Thank you,” she said. She stroked Ada’s hair briefly and kissed
her cheek. “I love you, Ada.”

Ada nodded solemnly.

“Let’s go,” Natalie said, when it was apparent that Ada had nothing
to say to her.

Benny tossed the Social Harmony man across the room into the corner
of a desk. He bounced off it and crashed to the floor, unconscious
or dead. Arturo couldn’t bring himself to care which.

Benny knelt before Arturo. “Climb on, please,” it said. Arturo saw
that Natalie was already pig-a-back on Lenny. He climbed aboard.

\tb

They moved even faster than the black robots had, but the bitter
cold was offset by the warmth radiating from Benny’s metal hide,
not hot, but warm. Arturo’s stomach reeled and he held Ada tight,
squeezing his eyes shut and clamping his jaw.

But Ada’s gasp made him look around, and he saw that they had
cleared the city limits, and were vaulting over rolling farmlands
now, jumping in long flat arcs whose zenith was just high enough
for him to see the highway{\dash}the 401, they were headed east{\dash}in the
distance.

And then he saw what had made Ada gasp: boiling out of the hills
and ditches, out of the trees and from under the cars: an army of
headless, eight-armed black robots, arachnoid and sinister in the
moonlight. They scuttled on the ground behind them, before them,
and to both sides. Social Harmony had built a secret army of these
robots and secreted them across the land, and now they were all
chasing after them.

\tb

The ride got bumpy then, as Benny beat back the tentacles that
reached for them, smashing the black robots with mighty one-handed
blows, his other hand supporting Arturo and Ada. Ada screamed as a
black robot reared up before them, and Benny vaulted it smoothly,
kicking it hard as he went, while Arturo clung on for dear life.

Another scream made him look over toward Lenny and Natalie. Lenny
was slightly ahead and to the left of them, and so he was the
vanguard, encountering twice as many robots as they.

A black spider-robot clung to his leg, dragging behind him with
each lope, and one of its spare arms was tugging at Natalie.

As Arturo watched{\dash}as Ada watched{\dash}the black robot ripped Natalie off
of Lenny’s back and tossed her into the arms of one of its cohort
behind it, which skewered her on one of its arms, a black spear
protruding from her belly as she cried once more and then fell
silent. Lenny was overwhelmed a moment later, buried under writhing
black arms.

Benny charged forward even faster, so that Arturo nearly lost his
grip, and then he steadied himself. “We have to go back for them{\dash}”

“They’re dead,” Benny said. “There’s nothing to go back for.” Its
warm voice was sorrowful as it raced across the countryside, and
the wind filled Arturo’s throat when he opened his mouth, and he
could say no more.

\tb

Ada wept on the jet, and Arturo wept with her, and Benny stood over
them, a minatory presence against the other robots crewing the fast
little plane, who left them alone all the way to Paris, where they
changed jets again for the long trip to Beijing.

They slept on that trip, and when they landed, Benny helped them
off the plane and onto the runway, and they got their first good
look at Eurasia.

It was tall. Vertical. Beijing loomed over them with curvilinear
towers that twisted and bent and jigged and jagged so high they
disappeared at the tops. It smelled like barbeque and flowers, and
around them skittered fast armies of robots of every shape and
size, wheeling in lockstep like schools of exotic fish. They gawped
at it for a long moment, and someone came up behind them and then
warm arms encircled their necks.

Arturo knew that smell, knew that skin. He could never have
forgotten it.

He turned slowly, the blood draining from his face.

“Natty?” he said, not believing his eyes as he confronted his dead,
ex-wife. There were tears in her eyes.

“Artie,” she said. “Ada,” she said. She kissed them both on the
cheeks.

Benny said, “You died in UNATS. Killed by modified Eurasian Social
Harmony robots. Lenny, too. Ironic,” he said.

She shook her head. “He means that we probably co-designed the
robots that Social Harmony sent after you.”

“Natty?” Arturo said again. Ada was white and shaking.

“Oh dear,” she said. “Oh, God. You didn’t know{\dash}”

“He didn’t give you a chance to explain,” Benny said.

“Oh, God, Jesus, you must have thought{\dash}”

“I didn’t think it was my place to tell them, either,” Benny said,
sounding embarrassed, a curious emotion for a robot.

“Oh, God. Artie, Ada. There are{\dash}there are \emph{lots} of me. One of
the first things I did here was help them debug the uploading
process. You just put a copy of yourself into a positronic brain,
and then when you need a body, you grow one or build one or both
and decant yourself into it. I’m like Lenny and Benny now{\dash}there are
many of me. There’s too much work to do otherwise.”

“I told you that our development helped humans understand
themselves,” Benny said.

Arturo pulled back. “You’re a robot?”

“No,” Natalie said. “No, of course not. Well, a little. Parts of
me. Growing a body is slow. Parts of it, you build. But I’m mostly
made of person.”

Ada clung tight to Arturo now, and they both stepped back toward
the jet.

“Dad?” Ada said.

He held her tight.

“Please, Arturo,” Natalie, his dead, multiplicitous ex-wife said.
“I know it’s a lot to understand, but it’s different here in
Eurasia. Better, too. I don’t expect you to come rushing back to my
arms after all this time, but I’ll help you if you’ll let me. I owe
you that much, no matter what happens between us. You too, Ada, I
owe you a lifetime.”

“How many are there of you?” he asked, not wanting to know the
answer.

“I don’t know exactly,” she said.

“3,422,” Benny said. “This morning it was 3,423.”

Arturo rocked back in his boots and bit his lip hard enough to draw
blood.

“Um,” Natalie said. “More of me to love?”

He barked a laugh, and Natalie smiled and reached for him. He
leaned back toward the jet, then stopped, defeated. Where would he
go? He let her warm hand take his, and a moment later, Ada took her
other hand and they stood facing each other, breathing in their
smells.

“I’ve gotten you your own place,” she said as she led them across
the tarmac. “It’s close to where I live, but far enough for you to
have privacy.”

“What will I do here?” he said. “Do they have coppers in Eurasia?”

“Not really,” Natalie said.

“It’s all robots?”

“No, there’s not any crime.”

“Oh.”

Arturo put one foot in front of the other, not sure if the ground
was actually spongy or if that was jetlag. Around him, the alien
smells of Beijing and the robots that were a million times smarter
than he. To his right, his wife, one of 3,422 versions of her.

To his left, his daughter, who would inherit this world.

He reached into his pocket and took out the tin soldiers there.
They were old and their glaze was cracked like an oil painting, but
they were little people that a real human had made, little people
in human image, and they were older than robots. How long had
humans been making people, striving to bring them to life? He
looked at Ada{\dash}a little person he’d brought to life.

He gave her the tin soldiers.

“For you,” he said. “Daddy-daughter present.” She held them
tightly, their tiny bayonets sticking out from between her
fingers.

“Thanks, Dad,” she said. She held them tightly and looked around,
wide-eyed, at the schools of robots and the corkscrew towers.

A flock of Bennyslennys appeared before them, joined by their
Benny.

“There are half a billion of them,” she said. “And 3,422 of them,”
she said, pointing with a small bayonet at Natalie.

“But there’s only one of you,” Arturo said.

She craned her neck.

“Not for long!” she said, and broke away, skipping forward and
whirling around to take it all in.


\section{Creative Commons License Deed}

Attribution-NonCommercial-ShareAlike 2.5

You are free:

* to Share{\dash}to copy, distribute, display, and perform the work

* to Remix{\dash}to make derivative works

Under the following conditions:

* Attribution. You must attribute the work in the manner specified
by the author or licensor.

* Noncommercial. You may not use this work for commercial
purposes.

* Share Alike. If you alter, transform, or build upon this work,
you may distribute the resulting work only under a license
identical to this one.



* For any reuse or distribution, you must make clear to others the
license terms of this work.

* Any of these conditions can be waived if you get permission from
the copyright holder.

Disclaimer: Your fair use and other rights are in no way affected
by the above.

This is a human-readable summary of the Legal Code (the full
license):

http://creativecommons.org/licenses/by-nc-sa/2.5/legalcode

\end{document}
