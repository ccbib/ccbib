%\documentclass{memoir}
%
%\usepackage[latin1]{inputenc}
\usepackage{url}
\DeclareUrlCommand\url{\def\UrlLeft{\hspace{0pt plus 2ex}}%
        \def\UrlRight{\hspace{0pt plus 2ex}}%
        \urlstyle{tt}}
%\usepackage{hyperref}
%\usepackage{palatino}
%
%\maxsecnumdepth{part}
%\chapterstyle{demo}
%
%\setlength\parindent{0pt}
%\setlength\parskip{6pt}
%
%\setlength\epigraphwidth\textwidth
%
%%% Enable hyphenation and neat justification with typewriter fonts
%\renewcommand\texttt[1]{{\ttfamily\hyphenchar\font45%
%\fontdimen3\font=.167em\fontdimen4\font=.056em\relax #1}}
%%% Same, but with ragged justification
%%\renewcommand\texttt[1]{{\ttfamily\hyphenchar\font45\relax #1}}
%
%\makepagestyle{lbrother}
%\makerunningwidth{lbrother}{\textwidth}
%%\makeheadposition{lbrother}{flushright}{flushleft}{}{}
%\makeheadrule{lbrother}{\textwidth}{\normalrulethickness}
%\makeevenhead{lbrother}{\thepage}{}{Cory Doctorow}
%\makeoddhead{lbrother}{Little Brother}{}{\thepage}
%\pagestyle{lbrother}

\newcommand\edialog[1]{
{
  \setlength\parindent{0pt}
  \setlength\hangindent{10pt}
  \raggedright
  \textgreater\ \texttt{#1}
  \par
}}

\newcommand{\fancybreak}[1]{
\begin{center}
#1
\end{center}
}

\newcommand{\epigraph}[2]{
  \emph{#1}
  \par\smallskip\noindent
  \emph{#2}
  \par\medskip\noindent}

\newcommand{\query}[1]{\texttt{\textgreater\textgreater\textgreater\ #1}\\}
\newcommand{\response}[1]{\texttt{#1}}
\newenvironment{serverlog}{
%  \newcommand{\query}[1]{\texttt{\textgreater\textgreater\textgreater\ #1}\\}
%  \newcommand{\response}[1]{\par\texttt{#1}\par}
  \raggedright
  \setlength{\parskip}{1\baselineskip plus 0.5\baselineskip}
  \setlength{\parindent}{0pt}
}{\smallskip}

\newenvironment{news}{
  \begin{center}
  \setlength{\parskip}{1\baselineskip plus 0.5\baselineskip}
}{\end{center}}

\newenvironment{newsquote}{
  \begin{quote}
  \small
%  \sffamily
}{
  \end{quote}
}

\hyphenation{mo-no-poly car-ne-gie pro-ject pro-gress mo-dem rou-lette
  browse-wrap Use-net mon-as-tery mo-dems}
\hyphenation{co-me-dic polt-roon stove-pipe Ma-dame scru-ta-ble star-tling}
\hyphenation{heal-thily lim-ou-sines wrest-lers tan-trum push-over un-asked
  bras-siere bro-th-er}
\hyphenation{Can-a-da Fred-rick teen-agers wrest-ler Cha-vez Tho-mas 
  a-nom-a-lies sur-veil-lance ar-mies ref-u-gee ref-u-gees bris-tling
  eve-ning man-chu-ria man-chu-ri-an mid-terms me-di-um jap-a-nese}
\hyphenation{spend-ers googl-ing tour-ist tour-ists leg-end-ary}
\hyphenation{Dan-iel Van-essa Doc-to-row Ste-phen-son}
\hyphenation{de-cade sur-veilled rout-ers Wol-fen-stein teen-ager to-night}
\hyphenation{his-to-gram an-o-nym-ize Ga-la-xy sym-pa-the-tic}
\hyphenation{ar-phid ar-phids Found-ers}
\hyphenation{stran-ger stran-gers shoul-der-blades dump-ling dump-lings}
\hyphenation{ice-pack guard-rail Sep-tem-ber boot-able e-co-nom-ist}
\hyphenation{grown-ups roos-ter shoe-laces li-quid-i-ty}
\hyphenation{side-arm}
\hyphenation{wo-man wo-men tan-trum tan-trums Le-nin-grad zom-bie bunk-house}
\hyphenation{up-tick bio-mass}
\hyphenation{of-fi-cial of-fi-cial-ly gov-ern-ment}
\hyphenation{heal-thy Or-ville spark-ling}
\hyphenation{ves-ti-bule Law-rence au-to-no-mous}
\hyphenation{sau-sage door-step staf-fer}
\hyphenation{tree-trunk}
\hyphenation{to-ron-to}
\hyphenation{qua-dril-lion-aire qua-dril-lion-aires}
\hyphenation{sports-jack-et sports-jack-ets}
\hyphenation{work-space skunk-works}
\hyphenation{kings-ton}

\newcommand{\winston}{w1n\-5t0n}

\begin{document}
\raggedbottom
\frontmatter

%\begin{titlingpage}

\title{Little Brother}
\author{Cory Doctorow
\thanks{\texttt{doctorow@craphound.com}}}
\date{}

\maketitle
%\sloppy

\section{READ THIS FIRST}

This book is distributed under a Creative Commons
At\-trib\-u\-tion-Non\-Com\-mer\-cial-Share\-Alike 3.0 license. That means:

\noindent
You are free:

\begin{itemize}
\item  to Share -- to copy, distribute and transmit the work
\item  to Remix -- to adapt the work
\end{itemize}

\noindent
Under the following conditions:

\begin{itemize}
\item Attribution. You must attribute the work in the manner specified
  by the author or licensor (but not in any way that suggests that
  they endorse you or your use of the work).
\item Noncommercial. You may not use this work for commercial
  purposes.    
\item Share Alike. If you alter, transform, or build upon this work,
  you may distribute the resulting work only under the same or similar
  license to this one.
\item For any reuse or distribution, you must make clear to others the
  license terms of this work. The best way to do this is with a link
  \hspace{0pt plus 1fil}\mbox{}
  \url{http://craphound.com/littlebrother}\hspace{0pt plus 1fill}\mbox{}
\item Any of the above conditions can be waived if you get my
  permission
\end{itemize}

\noindent
More info here: \\
\url{http://creativecommons.org/licenses/by-nc-sa/3.0/}

\noindent
See the end of this file for the complete legalese.

%\end{titlingpage}

\section{INTRODUCTION}

I wrote Little Brother in a white-hot fury between May 7, 2007 and
July 2, 2007: exactly eight weeks from the day I thought it up to the
day I finished it (Alice, to whom this book is dedicated, had to put
up with me clacking out the final chapter at 5AM in our hotel in Rome,
where we were celebrating our anniversary). I'd always dreamed of
having a book just materialize, fully formed, and come pouring out of
my fingertips, no sweat and fuss -- but it wasn't nearly as much fun
as I'd thought it would be. There were days when I wrote 10,000 words,
hunching over my keyboard in airports, on subways, in taxis --
anywhere I could type. The book was trying to get out of my head, no
matter what, and I missed so much sleep and so many meals that friends
started to ask if I was unwell.

When my dad was a young university student in the 1960s, he was one of
the few ``counterculture'' people who thought computers were a good
thing. For most young people, computers represented the
de-humanization of society. University students were reduced to
numbers on a punchcard, each bearing the legend ``DO NOT BEND, SPINDLE,
FOLD OR MUTILATE,'' prompting some of the students to wear pins that
said, ``I AM A STUDENT: DO NOT BEND, SPINDLE, FOLD OR MUTILATE ME.''
Computers were seen as a means to increase the ability of the
authorities to regiment people and bend them to their will.

When I was a 17, the world seemed like it was just going to get more
free. The Berlin Wall was about to come down. Computers -- which had
been geeky and weird a few years before -- were everywhere, and the
modem I'd used to connect to local bulletin board systems was now
connecting me to the entire world through the Internet and commercial
online services like GEnie. My lifelong fascination with activist
causes went into overdrive as I saw how the main difficulty in
activism -- organizing -- was getting easier by leaps and bounds (I
still remember the first time I switched from mailing out a newsletter
with hand-written addresses to using a database with mail-merge). In
the Soviet Union, communications tools were being used to bring
information -- and revolution -- to the farthest-flung corners of the
largest authoritarian state the Earth had ever seen.

But 17 years later, things are very different. The computers I love
are being co-opted, used to spy on us, control us, snitch on us. The
National Security Agency has illegally wiretapped the entire USA and
gotten away with it. Car rental companies and mass transit and traffic
authorities are watching where we go, sending us automated tickets,
finking us out to busybodies, cops and bad guys who gain illicit
access to their databases. The Transport Security Administration
maintains a ``no-fly'' list of people who'd never been convicted of any
crime, but who are nevertheless considered too dangerous to fly. The
list's contents are secret. The rule that makes it enforceable is
secret. The criteria for being added to the list are secret. It has
four-year-olds on it. And US senators. And decorated veterans --
actual war heroes.

The 17 year olds I know understand to a nicety just how dangerous a
computer can be. The authoritarian nightmare of the 1960s has come
home for them. The seductive little boxes on their desks and in their
pockets watch their every move, corral them in, systematically
depriving them of those new freedoms I had enjoyed and made such good
use of in my young adulthood.

What's more, kids were clearly being used as guinea-pigs for a new
kind of technological state that all of us were on our way to, a world
where taking a picture was either piracy (in a movie theater or museum
or even a Starbucks), or terrorism (in a public place), but where we
could be photographed, tracked and logged hundreds of times a day by
every tin-pot dictator, cop, bureaucrat and shop-keeper. A world where
any measure, including torture, could be justified just by waving your
hands and shouting ``Terrorism! 9/11! Terrorism!'' until all dissent
fell silent.

We don't have to go down that road.

If you love freedom, if you think the human condition is dignified by
privacy, by the right to be left alone, by the right to explore your
weird ideas provided you don't hurt others, then you have common cause
with the kids whose web-browsers and cell phones are being used to
lock them up and follow them around.

If you believe that the answer to bad speech is more speech -- not
censorship -- then you have a dog in the fight.

If you believe in a society of laws, a land where our rulers have to
tell us the rules, and have to follow them too, then you're part of
the same struggle that kids fight when they argue for the right to
live under the same Bill of Rights that adults have.

This book is meant to be part of the conversation about what an
information society means: does it mean total control, or unheard-of
liberty? It's not just a noun, it's a verb, it's something you do.

\section{DO SOMETHING}

This book is meant to be something you do, not just something you
read. The technology in this book is either real or nearly real. You
can build a lot of it. You can share it and remix it (see THE
COPYRIGHT THING, below). You can use the ideas to spark important
discussions with your friends and family. You can use those ideas to
defeat censorship and get onto the free Internet, even if your
government, employer or school doesn't want you to.

Making stuff: The folks at Instructables have put up some killer
HOWTOs for building the technology in this book. It's easy and
incredibly fun. There's nothing so rewarding in this world as making
stuff, especially stuff that makes you more free:
\url{http://www.instructables.com/member/w1n5t0n/}

Discussions: There's an educator's manual for this book that my
publisher, Tor, has put together that has tons of ideas for classroom,
reading group and home discussions of the ideas in it:
\url{http://www.tor-forge.com/static/Little_Brother_Readers_Guide.pdf}

Defeat censorship: The afterword for this book has lots of resources
for increasing your online freedom, blocking the snoops and evading
the censorware blocks. The more people who know about this stuff, the
better.

Your stories: I'm collecting stories of people who've used technology
to get the upper hand when confronted with abusive authority. I'm
going to be including the best of these in a special afterword to the
UK edition (see below) of the book, and I'll be putting them online as
well. Send me your stories at \url{doctorow@craphound.com}, with the subject
line ``Abuses of Authority''.

\section{GREAT BRITAIN}

I'm a Canadian, and I've lived in lots of places (including San
Francisco, the setting for Little Brother), and now I love in London,
England, with my wife Alice and our little daughter, Poesy. I've lived
here (off and on) for five years now, and though I love it to tiny
pieces, there's one thing that's always bugged me: my books aren't
available here. Some stores carried them as special items, imported
from the USA, but it wasn't published by a British publisher.

That's changed! HarperCollins UK has bought the British rights to this
book (along with my next young adult novel, FOR THE WIN), and they're
publishing it just a few months after the US edition, on November 17,
2008 (the day after I get back from my honeymoon!).

I'm so glad about this, I could bust, honestly. Not just because
they're finally selling my books in my adopted homeland, but because
\emph{I'm raising a daughter here, dammit}, and the surveillance and
control mania in this country is starting to scare me bloodless. It
seems like the entire police and governance system in Britain has
fallen in love with DNA-swabbing, fingerprinting and video-recording
everyone, on the off chance that someday you might do something
wrong. In early 2008, the head of Scotland Yard seriously proposed
taking DNA from \emph{five-year-olds} who display ``offending traits''
because they'll probably grow up to be criminals. The next week, the
London police put up posters asking us all to turn in people who seem
to be taking pictures of the ubiquitous CCTV spy-cameras because
anyone who pays too much to the surveillance machine is probably a
terrorist.

America isn't the only country that lost its mind this
decade. Britain's right there in the nuthouse with it, dribbling down
its shirt front and pointing its finger at the invisible bogeymen and
screaming until it gets its meds.

We need to be having this conversation all over the planet.

Like I said, the UK edition goes on sale on November 17 (ISBN:
978-0-00-728842-7). There'll even be a limited edition, signed
hardcover for people who like their books all artifact-y. If you want
to be notified when the book goes on sale, just drop me an email at
\url{doctorow@craphound.com} with the subject line LITTLE BROTHER UK
EDITION.

\section{OTHER EDITIONS}

My agent, Russell Galen (and his sub-agent Danny Baror) did an amazing
job of pre-selling rights to Little Brother in many languages and
formats. Here's the list as of today (May 4, 2008). I'll be updating
it as more editions are sold, so feel free to grab another copy of
this file\footnote{
  \url{http://craphound.com/littlebrother/download}}
 if there's an
edition you're hoping to see, or see
\url{http://craphound.com/littlebrother/buy/} for links to buy all the
currently shipping editions.

\begin{itemize}
\item Audiobook from Random House\footnote{
\url{http://www.randomhouse.com/audio/littlebrotheraudiobook}}

A condition of my deal with Random House is that they're not allowed
to release this on services that use ``DRM'' (Digital Rights Management)
systems intended to control use and copying. That means that you won't
find this book on Audible or iTunes, because Audible refuses to sell
books without DRM (even if the author and publisher don't want DRM),
and iTunes only carries Audible audiobooks. However, you can buy the
MP3 file direct from RandomHouse or many other fine 
\erratum{etailers}{retailers}, or
through this widget:\\
\url{http://www.zipidee.com/zipidAudioPreview.aspx?aid=c5a8e946-fd2c-4b9e-a748-f297bba17de8}
\end{itemize}
My foreign rights agent, Danny Baror, has presold a number of
foreign editions:
\begin{itemize}
\item Greece: Pataki
\item Russia: AST Publishing
\item France: Universe Poche
\item Norway: Det Norske Samlaget
\end{itemize}
      
No publication dates yet for these, but I'll keep updating this file
as more information is available. You can also subscribe to my mailing
list for more info.

\section{THE COPYRIGHT THING}

The Creative Commons license at the top of this file probably tipped
you off to the fact that I've got some pretty unorthodox views about
copyright. Here's what I think of it, in a nutshell: a little goes a
long way, and more than that is too much.

I like the fact that copyright lets me sell rights to my publishers
and film studios and so on. It's nice that they can't just take my
stuff without permission and get rich on it without cutting me in for
a piece of the action. I'm in a pretty good position when it comes to
negotiating with these companies: I've got a great agent and a
decade's experience with copyright law and licensing (including a
stint as a delegate at WIPO, the UN agency that makes the world's
copyright treaties). What's more, there's just not that many of these
negotiations -- even if I sell fifty or a hundred different editions
of Little Brother (which would put it in top millionth of a percentile
for fiction), that's still only a hundred negotiations, which I could
just about manage.

I \emph{hate} the fact that fans who want to do what readers have always
done are expected to play in the same system as all these hotshot
agents and lawyers. It's just \emph{stupid} to say that an elementary
school classroom should have to talk to a lawyer at a giant global
publisher before they put on a play based on one of my books. It's
ridiculous to say that people who want to ``loan'' their electronic copy
of my book to a friend need to get a \emph{license} to do so. Loaning books
has been around longer than any publisher on Earth, and it's a fine
thing.

I recently saw Neil Gaiman give a talk at which someone asked him how
he felt about piracy of his books. He said, ``Hands up in the audience
if you discovered your favorite writer for free -- because someone
loaned you a copy, or because someone gave it to you? Now, hands up if
you found your favorite writer by walking into a store and plunking
down cash.'' Overwhelmingly, the audience said that they'd discovered
their favorite writers for free, on a loan or as a gift. When it comes
to my favorite writers, there's no boundaries: I'll buy every book
they publish, just to own it (sometimes I buy two or three, to give
away to friends who \emph{must} read those books). I pay to see them
live. I buy t-shirts with their book-covers on them. I'm a customer
for life.

Neil went on to say that he was part of the tribe of readers, the tiny
minority of people in the world who read for pleasure, buying books
because they love them. One thing he knows about everyone who
downloads his books on the Internet without permission is that they're
\emph{readers}, they're people who love books.

People who study the habits of music-buyers have discovered something
curious: the biggest pirates are also the biggest spenders. If you
pirate music all night long, chances are you're one of the few people
left who also goes to the record store (remember those?) during the
day. You probably go to concerts on the weekend, and you probably
check music out of the library too. If you're a member of the red-hot
music-fan tribe, you do lots of \emph{everything} that has to do with
music, from singing in the shower to paying for black-market vinyl
bootlegs of rare Eastern European covers of your favorite death-metal
band.

Same with books. I've worked in new bookstores, used bookstores and
libraries. I've hung out in pirate ebook (``bookwarez'') places
online. I'm a stone used bookstore junkie, and I go to book fairs for
fun. And you know what? It's the same people at all those places: book
fans who do lots of everything that has to do with books. I buy weird,
fugly pirate editions of my favorite books in China because they're
weird and fugly and look great next to the eight or nine other
editions that I paid full-freight for of the same books. I check books
out of the library, google them when I need a quote, carry dozens
around on my phone and hundreds on my laptop, and have (at this
writing) more than 10,000 of them in storage lockers in London, Los
Angeles and Toronto.

If I could loan out my physical books without giving up possession of
them, I \emph{would}. The fact that I can do so with digital files is not a
bug, it's a feature, and a damned fine one. It's embarrassing to see
all these writers and musicians and artists bemoaning the fact that
art just got this wicked new feature: the ability to be shared without
losing access to it in the first place. It's like watching restaurant
owners crying down their shirts about the new free lunch machine
that's feeding the world's starving people because it'll force them to
reconsider their business-models. Yes, that's gonna be tricky, but
let's not lose sight of the main attraction: free lunches!

Universal access to human knowledge is in our grasp, for the first
time in the history of the world. This is not a bad thing.

In case that's not enough for you, here's my pitch on why giving away
ebooks makes sense at this time and place:

Giving away ebooks gives me artistic, moral and commercial
satisfaction. The commercial question is the one that comes up most
often: how can you give away free ebooks and still make money?

For me -- for pretty much every writer -- the big problem isn't
piracy, it's obscurity (thanks to Tim O'Reilly for this great
aphorism). Of all the people who failed to buy this book today, the
majority did so because they never heard of it, not because someone
gave them a free copy. Mega-hit best-sellers in science fiction sell
half a million copies -- in a world where 175,000 attend the San Diego
Comic Con alone, you've got to figure that most of the people who
``like science fiction'' (and related geeky stuff like comics, games,
Linux, and so on) just don't really buy books. I'm more interested in
getting more of that wider audience into the tent than making sure
that everyone who's in the tent bought a ticket to be there.

Ebooks are verbs, not nouns. You copy them, it's in their nature. And
many of those copies have a destination, a person they're intended
for, a hand-wrought transfer from one person to another, embodying a
personal recommendation between two people who trust each other enough
to share bits. That's the kind of thing that authors (should) dream
of, the proverbial sealing of the deal. By making my books available
for free pass-along, I make it easy for people who love them to help
other people love them.

What's more, I don't see ebooks as substitute for paper books for most
people. It's not that the screens aren't good enough, either: if
you're anything like me, you already spend every hour you can get in
front of the screen, reading text. But the more computer-literate you
are, the less likely you are to be reading long-form works on those
screens -- that's because computer-literate people do more things with
their computers. We run IM and email and we use the browser in a
million diverse ways. We have games running in the background, and
endless opportunities to tinker with our music libraries. The more you
do with your computer, the more likely it is that you'll be
interrupted after five to seven minutes to do something else. That
makes the computer extremely poorly suited to reading long-form works
off of, unless you have the iron self-discipline of a monk.

The good news (for writers) is that this means that ebooks on
computers are more likely to be an enticement to buy the printed book
(which is, after all, cheap, easily had, and easy to use) than a
substitute for it. You can probably read just enough of the book off
the screen to realize you want to be reading it on paper.

So ebooks sell print books. Every writer I've heard of who's tried
giving away ebooks to promote paper books has come back to do it
again. That's the commercial case for doing free ebooks.
 
Now, onto the artistic case. It's the twenty-first century. Copying
stuff is never, ever going to get any harder than it is today (or if
it does, it'll be because civilization has collapsed, at which point
we'll have other problems). Hard drives aren't going to get bulkier,
more expensive, or less capacious. Networks won't get slower or harder
to access. If you're not making art with the intention of having it
copied, you're not really making art for the twenty-first
century. There's something charming about making work you don't want
to be copied, in the same way that it's nice to go to a Pioneer
Village and see the olde-timey blacksmith shoeing a horse at his
traditional forge. But it's hardly, you know, \emph{contemporary}. I'm a
science fiction writer. It's my job to write about the future (on a
good day) or at least the present. Art that's not supposed to be
copied is from the past.

Finally, let's look at the moral case. Copying stuff is natural. It's
how we learn (copying our parents and the people around us). My first
story, written when I was six, was an excited re-telling of Star Wars,
which I'd just seen in the theater. Now that the Internet -- the
world's most efficient copying machine -- is pretty much everywhere,
our copying instinct is just going to play out more and more. There's
no way I can stop my readers, and if I tried, I'd be a hypocrite: when
I was 17, I was making mix-tapes, photocopying stories, and generally
copying in every way I could imagine. If the Internet had been around
then, I'd have been using it to copy as much as I possibly could.

There's no way to stop it, and the people who try end up doing more
harm than piracy ever did. The record industry's ridiculous holy war
against file-sharers (more than 20,000 music fans sued and counting!)
exemplifies the absurdity of trying to get the food-coloring out of
the swimming pool. If the choice is between allowing copying or being
a frothing bully lashing out at anything he can reach, I choose the
former.

\section{DONATIONS AND A WORD TO TEACHERS AND LIBRARIANS}

Every time I put a book online for free, I get emails from readers who
want to send me donations for the book. I appreciate their generous
spirit, but I'm not interested in cash donations, because my
publishers are really important to me. They contribute immeasurably to
the book, improving it, introducing it to audience I could never
reach, helping me do more with my work. I have no desire to cut them
out of the loop.

But there has to be some good way to turn that generosity to good use,
and I think I've found it.

Here's the deal: there are lots of teachers and librarians who'd love
to get hard-copies of this book into their kids' hands, but don't have
the budget for it (teachers in the US spend around \$1,200 out of
pocket each on classroom supplies that their budgets won't stretch to
cover, which is why I sponsor a classroom at Ivanhoe Elementary in my
old neighborhood in Los Angeles; you can adopt a class yourself here:
\url{http://www.adoptaclassroom.org/}).

There are generous people who want to send some cash my way to thank
me for the free ebooks.

I'm proposing that we put them together. 

If you're a teacher or librarian and you want a free copy of Little
Brother, email \url{freelittlebrother@gmail.com} with your name and
the name and address of your school. It'll be posted to
\url{http://craphound.com/littlebrother/category/donate/} \hspace{0pt plus 10pt}
by my fantastic
helper, Olga Nunes, so that potential donors can see it.

If you enjoyed the electronic edition of Little Brother and you want
to donate something to say thanks, go to
\url{http://craphound.com/littlebrother/category/donate/} and find a
teacher or librarian you want to support. Then go to Amazon, BN.com,
or your favorite electronic bookseller and order a copy to the
classroom, then email a copy of the receipt (feel free to delete your
address and other personal info first!) to
\url{freelittlebrother@gmail.com} so that Olga can mark that copy as
sent. If you don't want to be publicly acknowledged for your
generosity, let us know and we'll keep you anonymous, otherwise we'll
thank you on the donate page.

I have no idea if this will end up with hundreds, dozens or just a few
copies going out -- but I have high hopes!

\section{DEDICATION}

For Alice, who makes me whole

\section{QUOTES}

``A rousing tale of techno-geek rebellion, as necessary and dangerous
as file sharing, free speech, and bottled water on a plane.''

Scott Westerfeld, author of UGLIES and EXTRAS

\fancybreak{\#}

``I can talk about Little Brother in terms of its bravura political
speculation or its brilliant uses of technology -- each of which make
this book a must-read -- but, at the end of it all, I'm haunted by the
universality of Marcus's rite-of-passage and struggle, an experience
any teen today is going to grasp: the moment when you choose what your
life will mean and how to achieve it.''

Steven C Gould, author of JUMPER and REFLEX

\fancybreak{\#}

I'd recommend Little Brother over pretty much any book I've read this
year, and I'd want to get it into the hands of as many smart 13 year
olds, male and female, as I can.

Because I think it'll change lives. Because some kids, maybe just a
few, won't be the same after they've read it. Maybe they'll change
politically, maybe technologically. Maybe it'll just be the first book
they loved or that spoke to their inner geek. Maybe they'll want to
argue about it and disagree with it. Maybe they'll want to open their
computer and see what's in there. I don't know. It made me want to be
13 again right now and reading it for the first time, and then go out
and make the world better or stranger or odder. It's a wonderful,
important book, in a way that renders its flaws pretty much
meaningless.

Neil Gaiman, author of ANASI BOYS

\fancybreak{\#}

Little Brother is a scarily realistic adventure about how homeland
security technology could be abused to wrongfully imprison innocent
Americans. A teenage hacker-turned-hero pits himself against the
government to fight for his basic freedoms. This book is action-packed
with tales of courage, technology, and demonstrations of digital
disobedience as the technophile's civil protest."

Bunnie Huang, author of HACKING THE XBOX

\fancybreak{\#}

Cory Doctorow is a fast and furious storyteller who gets all the
details of alternate reality gaming right, while offering a startling,
new vision of how these games might play out in the high-stakes
context of a terrorist attack. Little Brother is a brilliant novel
with a bold argument: hackers and gamers might just be our country's
best hope for the future.

Jane McGonical, Designer, I Love Bees

\fancybreak{\#}


The right book at the right time from the right author -- and, not
entirely coincidentally, Cory Doctorow's best novel yet.

John Scalzi, author of OLD MAN'S WAR

\fancybreak{\#}

It's about growing up in the near future where things have kept going
on the way they've been going, and it's about hacking as a habit of
mind, but mostly it's about growing up and changing and looking at the
world and asking what you can do about that. The teenage voice is
pitch-perfect. I couldn't put it down, and I loved it.

Jo Walton, author of FARTHING


\fancybreak{\#}

A worthy younger sibling to Orwell's 1984, Cory Doctorow's LITTLE
BROTHER is lively, precocious, and most importantly, a little scary.

Brian K Vaughn, author of Y: THE LAST MAN

\fancybreak{\#}

``Little Brother'' sounds an optimistic warning. It extrapolates from
current events to remind us of the ever-growing threats to
liberty. But it also notes that liberty ultimately resides in our
individual attitudes and actions. In our increasingly authoritarian
world, I especially hope that teenagers and young adults will read it
-- and then persuade their peers, parents and teachers to follow suit.

Dan Gillmor, author of WE, THE MEDIA

\section{ABOUT THE BOOKSTORE DEDICATIONS}

Every chapter of this file has been dedicated to a different
bookstore, and in each case, it's a store that I love, a store that's
helped me discover books that opened my mind, a store that's helped my
career along. The stores didn't pay me anything for this -- I haven't
even told them about it -- but it seems like the right thing to
do. After all, I'm hoping that you'll read this ebook and decide to
buy the paper book, so it only makes sense to suggest a few places you
can pick it up!

\mainmatter

\chapter{Chapter 1}

\epigraph{This chapter is dedicated to BakkaPhoenix Books in Toronto,
Canada. Bakka is the oldest science fiction bookstore in the world,
and it made me the mutant I am today. I wandered in for the first time
around the age of 10 and asked for some recommendations. Tanya Huff
(yes, \emph{the} Tanya Huff, but she wasn't a famous writer back then!)
took me back into the used section and pressed a copy of H. Beam
Piper's ``Little Fuzzy'' into my hands, and changed my life forever. By
the time I was 18, I was working at Bakka -- I took over from Tanya
when she retired to write full time -- and I learned life-long lessons
about how and why people buy books. I think every writer should work
at a bookstore (and plenty of writers have worked at Bakka over the
years! For the 30th anniversary of the store, they put together an
anthology of stories by Bakka writers than included work by Michelle
Sagara (AKA Michelle West), Tanya Huff, Nalo Hopkinson, Tara Tallan
--and me!)}
{BakkaPhoenix Books\footnote{\url{http://www.bakkaphoenixbooks.com/}}:
697 Queen
Street West, Toronto ON Canada M6J1E6, +1 416 963 9993}

I'm a senior at Cesar Chavez high in San Francisco's sunny Mission
district, and that makes me one of the most surveilled people in the
world. My name is Marcus Yallow, but back when this story starts, I
was going by w1n5t0n. Pronounced ``Winston.''

\emph{Not} pronounced ``Double-you-one-enn-five-tee-zero-enn'' -- unless
you're a clueless disciplinary officer who's far enough behind the
curve that you still call the Internet ``the information superhighway.''

I know just such a clueless person, and his name is Fred Benson, one
of three vice-principals at Cesar Chavez. He's a sucking chest wound
of a human being. But if you're going to have a jailer, better a
clueless one than one who's really on the ball.

``Marcus Yallow,'' he said over the PA one Friday morning. The PA isn't
very good to begin with, and when you combine that with Benson's
habitual mumble, you get something that sounds more like someone
struggling to digest a bad burrito than a school announcement. But
human beings are good at picking their names out of audio confusion --
it's a survival trait.

I grabbed my bag and folded my laptop three-quarters shut -- I didn't
want to blow my downloads -- and got ready for the inevitable.

``Report to the administration office immediately.''

My social studies teacher, Ms Galvez, rolled her eyes at me and I
rolled my eyes back at her. The Man was always coming down on me, just
because I go through school firewalls like wet kleenex, spoof the
gait-recognition software, and nuke the snitch chips they track us
with. Galvez is a good type, anyway, never holds that against me
(especially when I'm helping get with her webmail so she can talk to
her brother who's stationed in Iraq).

My boy Darryl gave me a smack on the ass as I walked past. I've known
Darryl since we were still in diapers and escaping from play-school,
and I've been getting him into and out of trouble the whole time. I
raised my arms over my head like a prizefighter and made my exit from
Social Studies and began the perp-walk to the office.

I was halfway there when my phone went. That was another no-no --
phones are muy prohibido at Chavez High -- but why should that stop
me? I ducked into the toilet and shut myself in the middle stall (the
furthest stall is always grossest because so many people head straight
for it, hoping to escape the smell and the squick -- the smart money
and good hygiene is down the middle). I checked the phone -- my home
PC had sent it an email to tell it that there was something new up on
Harajuku Fun Madness, which happens to be the best game ever invented.

I grinned. Spending Fridays at school was teh suck anyway, and I was
glad of the excuse to make my escape.

I ambled the rest of the way to Benson's office and tossed him a wave
as I sailed through the door.

``If it isn't Double-you-one-enn-five-tee-zero-enn,'' he said. Fredrick
Benson -- Social Security number 545-03-2343, date of birth August 15
1962, mother's maiden name Di Bona, hometown Petaluma -- is a lot
taller than me. I'm a runty 5'8", while he stands 6'7", and his
college basketball days are far enough behind him that his chest
muscles have turned into saggy man-boobs that were painfully obvious
through his freebie dot-com polo-shirts. He always looks like he's
about to slam-dunk your ass, and he's really into raising his voice
for dramatic effect. Both these start to lose their efficacy with
repeated application.

``Sorry, nope,'' I said. ``I never heard of this R2D2 character of
yours.''

``W1n5t0n,'' he said, spelling it out again. He gave me a hairy eyeball
and waited for me to wilt. Of course it was my handle, and had been
for years. It was the identity I used when I was posting on
message-boards where I was making my contributions to the field of
applied security research. You know, like sneaking out of school and
disabling the minder-tracer on my phone. But he didn't know that this
was my handle. Only a small number of people did, and I trusted them
all to the end of the earth.

``Um, not ringing any bells,'' I said. I'd done some pretty cool stuff
around school using that handle -- I was very proud of my work on
snitch-tag killers -- and if he could link the two identities, I'd be
in trouble. No one at school ever called me w1n5t0n or even
Winston. Not even my pals. It was Marcus or nothing.

Benson settled down behind his desk and tapped his class-ring
nervously on his blotter. He did this whenever things started to go
bad for him. Poker players call stuff like this a ``tell'' -- something
that let you know what was going on in the other guy's head. I knew
Benson's tells backwards and forwards.

``Marcus, I hope you realize how serious this is.''

``I will just as soon as you explain what this is, sir.'' I always say
``sir'' to authority figures when I'm messing with them. It's my own
tell.

He shook his head at me and looked down, another tell. Any second now,
he was going to start shouting at me. ``Listen, kiddo! It's time you
came to grips with the fact that we know about what you've been doing,
and that we're not going to be lenient about it. You're going to be
lucky if you're not expelled before this meeting is through. Do you
want to graduate?''

``Mr Benson, you still haven't explained what the problem is --''

He slammed his hand down on the desk and then pointed his finger at
me. ``The \emph{problem}, Mr Yallow, is that you've been engaged in criminal
conspiracy to subvert this school's security system, and you have
supplied security countermeasures to your fellow students. You know
that we expelled Graciella Uriarte last week for using one of your
devices.'' Uriarte had gotten a bad rap. She'd bought a radio-jammer
from a head-shop near the 16th Street BART station and it had set off
the countermeasures in the school hallway. Not my doing, but I felt
for her.

``And you think I'm involved in that?''

``We have reliable intelligence indicating that you are \winston'' --
again, he spelled it out, and I began to wonder if he hadn't figured
out that the 1 was an I and the 5 was an S. ``We know that this w1n5t0n
character is reponsible for the theft of last year's standardized
tests.'' That actually hadn't been me, but it was a sweet hack, and it
was kind of flattering to hear it attributed to me. ``And therefore
liable for several years in prison unless you cooperate with me.''

``You have `reliable intelligence'? I'd like to see it.''

He glowered at me. ``Your attitude isn't going to help you.''

``If there's evidence, sir, I think you should call the police and turn
it over to them. It sounds like this is a very serious matter, and I
wouldn't want to stand in the way of a proper investigation by the
duly constituted authorities.''

``You want me to call the police.''

``And my parents, I think. That would be for the best.''

We stared at each other across the desk. He'd clearly expected me to
fold the second he dropped the bomb on me. I don't fold. I have a
trick for staring down people like Benson. I look slightly to the left
of their heads, and think about the lyrics to old Irish folk songs,
the kinds with three hundred verses. It makes me look perfectly
composed and unworried.

\emph{And the wing was on the bird and the bird was on the egg and the egg
was in the nest and the nest was on the leaf and the leaf was on the
twig and the twig was on the branch and the branch was on the limb and
the limb was in the tree and the tree was in the bog -- the bog down
in the valley-oh! High-ho the rattlin' bog, the bog down in the
valley-oh --}

``You can return to class now,'' he said. ``I'll call on you once the
police are ready to speak to you.''

``Are you going to call them now?''

``The procedure for calling in the police is complicated. I'd hoped
that we could settle this fairly and quickly, but since you insist --''

``I can wait while you call them is all,'' I said. ``I don't mind.''

He tapped his ring again and I braced for the blast.

``\emph{Go!}'' he yelled. ``Get the hell out of my office, you miserable
little --''

I got out, keeping my expression neutral. He wasn't going to call the
cops. If he'd had enough evidence to go to the police with, he would
have called them in the first place. He hated my guts. I figured he'd
heard some unverified gossip and hoped to spook me into confirming it.

I moved down the corridor lightly and sprightly, keeping my gait even
and measured for the gait-recognition cameras. These had been
installed only a year before, and I loved them for their sheer
idiocy. Beforehand, we'd had face-recognition cameras covering nearly
every public space in school, but a court ruled that was
unconstitutional. So Benson and a lot of other paranoid school
administrators had spent our textbook dollars on these idiot cameras
that were supposed to be able to tell one person's walk from
another. Yeah, right.

I got back to class and sat down again, Ms Galvez warmly welcoming me
back. I unpacked the school's standard-issue machine and got back into
classroom mode. The SchoolBooks were the snitchiest technology of them
all, logging every keystroke, watching all the network traffic for
suspicious keywords, counting every click, keeping track of every
fleeting thought you put out over the net. We'd gotten them in my
junior year, and it only took a couple months for the shininess to
wear off. Once people figured out that these ``free'' laptops worked for
the man -- and showed a never-ending parade of obnoxious ads to boot
-- they suddenly started to feel very heavy and burdensome.

Cracking my SchoolBook had been easy. The crack was online within a
month of the machine showing up, and there was nothing to it -- just
download a DVD image, burn it, stick it in the SchoolBook, and boot it
while holding down a bunch of different keys at the same time. The DVD
did the rest, installing a whole bunch of hidden programs on the
machine, programs that would stay hidden even when the Board of Ed did
its daily remote integrity checks of the machines. Every now and again
I had to get an update for the software to get around the Board's
latest tests, but it was a small price to pay to get a little control
over the box.

I fired up IMParanoid, the secret instant messenger that I used when I
wanted to have an off-the-record discussion right in the middle of
class. Darryl was already logged in.

\edialog{The game's afoot! Something big is going down with Harajuku Fun
Madness, dude. You in?}

\edialog{No. Freaking. Way. If I get caught ditching a third time, I'm
expelled. Man, you know that. We'll go after school.}

\edialog{You've got lunch and then study-hall, right? That's two
hours. Plenty of time to run down this clue and get back before anyone
misses us. I'll get the whole team out.}

Harajuku Fun Madness is the best game ever made. I know I already said
that, but it bears repeating. It's an ARG, an Alternate Reality Game,
and the story goes that a gang of Japanese fashion-teens discovered a
miraculous healing gem at the temple in Harajuku, which is basically
where cool Japanese teenagers invented every major subculture for the
past ten years. They're being hunted by evil monks, the Yakuza (AKA
the Japanese mafia), aliens, tax-inspectors, parents, and a rogue
artificial intelligence. They slip the players coded messages that we
have to decode and use to track down clues that lead to more coded
messages and more clues.

Imagine the best afternoon you've ever spent prowling the streets of a
city, checking out all the weird people, funny hand-bills,
street-maniacs, and funky shops. Now add a scavenger hunt to that, one
that requires you to research crazy old films and songs and teen
culture from around the world and across time and space. And it's a
competition, with the winning team of four taking a grand prize of ten
days in Tokyo, chilling on Harajuku bridge, geeking out in Akihabara,
and taking home all the Astro Boy merchandise you can eat. Except that
he's called ``Atom Boy'' in Japan.

That's Harajuku Fun Madness, and once you've solved a puzzle or two,
you'll never look back.

\edialog{No man, just no. NO. Don't even ask.}

\edialog{I need you D. You're the best I've got. I swear I'll get us
  in and out without anyone knowing it. You know I can do that, right?}

\edialog{I know you can do it}

\edialog{So you're in?}

\edialog{Hell no}

\edialog{Come on, Darryl. You're not going to your deathbed wishing
  you'd spent more study periods sitting in school}

\edialog{I'm not going to go to my deathbed wishing I'd spent more
  time playing ARGs either}

\edialog{Yeah but don't you think you might go to your death-bed
  wishing you'd spent more time with Vanessa Pak?}

Van was part of my team. She went to a private girl's school in the
East Bay, but I knew she'd ditch to come out and run the mission with
me. Darryl has had a crush on her literally for years -- even before
puberty endowed her with many lavish gifts. Darryl had fallen in love
with her mind. Sad, really.

\edialog{You suck}

\edialog{You're coming?}

He looked at me and shook his head. Then he nodded. I winked at him
and set to work getting in touch with the rest of my team.

\fancybreak{\#}

I wasn't always into ARGing. I have a dark secret: I used to be a
LARPer. LARPing is Live Action Role Playing, and it's just about what
it sounds like: running around in costume, talking in a funny accent,
pretending to be a super-spy or a vampire or a medieval knight. It's
like Capture the Flag in monster-drag, with a bit of Drama Club thrown
in, and the best games were the ones we played in Scout Camps out of
town in Sonoma or down on the Peninsula. Those three-day epics could
get pretty hairy, with all-day hikes, epic battles with
foam-and-bamboo swords, casting spells by throwing beanbags and
shouting ``Fireball!'' and so on. Good fun, if a little goofy. Not
nearly as geeky as talking about what your elf planned on doing as you
sat around a table loaded with Diet Coke cans and painted miniatures,
and more physically active than going into a mouse-coma in front of a
massively multiplayer game at home.

The thing that got me into trouble were the mini-games in the
hotels. Whenever a science fiction convention came to town, some
LARPer would convince them to let us run a couple of six-hour
mini-games at the con, piggybacking on their rental of the
space. Having a bunch of enthusiastic kids running around in costume
lent color to the event, and we got to have a ball among people even
more socially deviant than us.

The problem with hotels is that they have a lot of non-gamers in them,
too -- and not just sci-fi people. Normal people. From states that
begin and end with vowels. On holidays.

And sometimes those people misunderstand the nature of a game.

Let's just leave it at that, OK?

\fancybreak{\#}

Class ended in ten minutes, and that didn't leave me with much time to
prepare. The first order of business were those pesky gait-recognition
cameras. Like I said, they'd started out as face-recognition cameras,
but those had been ruled unconstitutional. As far as I know, no court
has yet determined whether these gait-cams are any more legal, but
until they do, we're stuck with them.

``Gait'' is a fancy word for the way you walk. People are pretty good at
spotting gaits -- next time you're on a camping trip, check out the
bobbing of the flashlight as a distant friend approaches you. Chances
are you can identify him just from the movement of the light, the
characteristic way it bobs up and down that tells our monkey brains
that this is a person approaching us.

Gait recognition software takes pictures of your motion, tries to
isolate you in the pics as a silhouette, and then tries to match the
silhouette to a database to see if it knows who you are. It's a
biometric identifier, like fingerprints or retina-scans, but it's got
a lot more ``collisions'' than either of those. A biometric ``collision''
is when a measurement matches more than one person. Only you have your
fingerprint, but you share your gait with plenty other people.

Not exactly, of course. Your personal, inch-by-inch walk is yours and
yours alone. The problem is your inch-by-inch walk changes based on
how tired you are, what the floor is made of, whether you pulled your
ankle playing basketball, and whether you've changed your shoes
lately. So the system kind of fuzzes-out your profile, looking for
people who walk kind of like you.

There are a lot of people who walk kind of like you. What's more, it's
easy not to walk kind of like you -- just take one shoe off. Of
course, you'll always walk like you-with-one-shoe-off in that case, so
the cameras will eventually figure out that it's still you. Which is
why I prefer to inject a little randomness into my attacks on
gait-recognition: I put a handful of gravel into each shoe. Cheap and
effective, and no two steps are the same. Plus you get a great
reflexology foot massage in the process (I kid. Reflexology is about
as scientifically useful as gait-recognition).

The cameras used to set off an alert every time someone they didn't
recognize stepped onto campus.

This did \emph{not} work.

The alarm went off every ten minutes. When the mailman came by. When a
parent dropped in. When the grounds-people went to work fixing up the
basketball court. When a student showed up wearing new shoes.

So now it just tries to keep track of who's where and when. If someone
leaves by the school-gates during classes, their gait is checked to
see if it kinda-sorta matches any student gait and if it does,
whoop-whoop-whoop, ring the alarm!

Chavez High is ringed with gravel walkways. I like to keep a couple
handsful of rocks in my shoulder-bag, just in case. I silently passed
Darryl ten or fifteen pointy little bastards and we both loaded our
shoes.

Class was about to finish up -- and I realized that I still hadn't
checked the Harajuku Fun Madness site to see where the next clue was!
I'd been a little hyper-focused on the escape, and hadn't bothered to
figure out where we were escaping \emph{to}.

I turned to my SchoolBook and hit the keyboard. The web-browser we
used was supplied with the machine. It was a locked-down spyware
version of Internet Explorer, Microsoft's crashware turd that no one
under the age of 40 used voluntarily.

I had a copy of Firefox on the USB drive built into my watch, but that
wasn't enough -- the SchoolBook ran Windows Vis\-ta4Schools, an antique
operating system designed to give school administrators the illusion
that they controlled the programs their students could run.

But Vista4Schools is its own worst enemy. There are a lot of programs
that Vista4Schools doesn't want you to be able to shut down --
keyloggers, censorware -- and these programs run in a special mode
that makes them invisible to the system. You can't quit them because
you can't even see they're there.

Any program whose name starts with \$SYS\$ is invisible to the operating
system. it doesn't show up on listings of the hard drive, nor in the
process monitor. So my copy of Firefox was called \$SYS\$Firefox -- and
as I launched it, it became invisible to Windows, and so invisible to
the network's snoopware.

Now I had an indie browser running, I needed an indie network
connection. The school's network logged every click in and out of the
system, which was bad news if you were planning on surfing over to the
Harajuku Fun Madness site for some extra-curricular fun.

The answer is something ingenious called TOR -- The Onion Router. An
onion router is an Internet site that takes requests for web-pages and
passes them onto other onion routers, and on to other onion routers,
until one of them finally decides to fetch the page and pass it back
through the layers of the onion until it reaches you. The traffic to
the onion-routers is encrypted, which means that the school can't see
what you're asking for, and the layers of the onion don't know who
they're working for. There are millions of nodes -- the program was
set up by the US Office of Naval Research to help their people get
around the censorware in countries like Syria and China, which means
that it's perfectly designed for operating in the confines of an
average American high school.

TOR works because the school has a finite blacklist of naughty
addresses we aren't allowed to visit, and the addresses of the nodes
change all the time -- no way could the school keep track of them
all. Firefox and TOR together made me into the invisible man,
impervious to Board of Ed snooping, free to check out the Harajuku FM
site and see what was up.

There it was, a new clue. Like all Harajuku Fun Madness clues, it had
a physical, online and mental component. The online component was a
puzzle you had to solve, one that required you to research the answers
to a bunch of obscure questions. This batch included a bunch of
questions on the plots in d\={o}jinshi -- those are comic books drawn
by fans of manga, Japanese comics. They can be as big as the official
comics that inspire them, but they're a lot weirder, with crossover
story-lines and sometimes really silly songs and action. Lots of love
stories, of course. Everyone loves to see their favorite toons hook
up.

I'd have to solve those riddles later, when I got home. They were
easiest to solve with the whole team, downloading tons of d\={o}jinshi
files and scouring them for answers to the puzzles.

I'd just finished scrap-booking all the clues when the bell rang and
we began our escape. I surreptitiously slid the gravel down the side
of my short boots -- ankle-high Blundstones from Australia, great for
running and climbing, and the easy slip-on/slip-off laceless design
makes them convenient at the never-ending metal-detectors that are
everywhere now.

We also had to evade physical surveillance, of course, but that gets
easier every time they add a new layer of physical snoopery -- all the
bells and whistles lull our beloved faculty into a totally false sense
of security. We surfed the crowd down the hallways, heading for my
favorite side-exit. We were halfway along when Darryl hissed, ``Crap! I
forgot, I've got a library book in my bag.''

``You're kidding me,'' I said, and hauled him into the next bathroom we
passed. Library books are bad news. Every one of them has an arphid --
Radio Frequency ID tag -- glued into its binding, which makes it
possible for the librarians to check out the books by waving them over
a reader, and lets a library shelf tell you if any of the books on it
are out of place.

But it also lets the school track where you are at all times. It was
another of those legal loopholes: the courts wouldn't let the schools
track \emph{us} with arphids, but they could track \emph{library books}, and use
the school records to tell them who was likely to be carrying which
library book.

I had a little Faraday pouch in my bag -- these are little wallets
lined with a mesh of copper wires that effectively block radio energy,
silencing arphids. But the pouches were made for neutralizing ID cards
and toll-book transponders, not books like --

``Introduction to Physics?'' I groaned. The book was the size of a
dictionary.

\chapter{Chapter 2}

\epigraph{This chapter is dedicated to Amazon.com, the largest Internet
bookseller in the world. Amazon is \emph{amazing} -- a ``store'' where you
can get practically any book ever published (along with practically
everything else, from laptops to cheese-graters), where
\discretionary{they}{have}{they've}
elevated recommendations to a high art, where they allow customers to
directly communicate with each other, where they are constantly
invented new and better ways of connecting books with readers. Amazon
has always treated me like gold -- the founder, Jeff Bezos, even
posted a reader-review for my first nov\-el! -- and I shop there like
crazy (looking at my spreadsheets, it appears that I buy something
from Amazon approximately every \emph{six days}). Amazon's in the process
of reinventing what it means to be a bookstore in the twenty-first
century and I can't think of a better group of people to be facing
down that thorny set of problems.}
{Amazon: 
\url{http://www.amazon.com/exec/obidos/ASIN/0765319853/downandoutint-20}}

``I'm thinking of majoring in physics when I go to Berkeley,'' Darryl
said. His dad taught at the University of California at Berkeley,
which meant he'd get free tuition when he went. And there'd never been
any question in Darryl's household about whether he'd go.

``Fine, but couldn't you research it online?''

``My dad said I should read it. Besides, I didn't plan on committing
any crimes today.''

``Skipping school isn't a crime. It's an infraction. They're totally
different.''

``What are we going to do, Marcus?''

``Well, I can't hide it, so I'm going to have to nuke it.'' Killing
arphids is a dark art. No merchant wants malicious customers going for
a walk around the shop-floor and leaving behind a bunch of lobotomized
merchandise that is missing its invisible bar-code, so the
manufacturers have refused to implement a ``kill signal'' that you can
radio to an arphid to get it to switch off. You can reprogram arphids
with the right box, but I hate doing that to library books. It's not
exactly tearing pages out of a book, but it's still bad, since a book
with a reprogrammed arphid can't be shelved and can't be found. It
just becomes a needle in a haystack.

That left me with only one option: nuking the thing. Literally. 30
seconds in a microwave will do in pretty much every arphid on the
market. And because the arphid wouldn't answer at all when D checked
it back in at the library, they'd just print a fresh one for it and
recode it with the book's catalog info, and it would end up clean and
neat back on its shelf.

All we needed was a microwave.

``Give it another two minutes and the teacher's lounge will be empty,''
I said.

Darryl grabbed his book at headed for the door. ``Forget it, no
way. I'm going to class.''

I snagged his elbow and dragged him back. ``Come on, D, easy now. It'll
be fine.''

``The \emph{teacher's lounge}? Maybe you weren't listening, Marcus. If I get
busted \emph{just once more}, I am \emph{expelled.} You hear that? \emph{Expelled.}''

``You won't get caught,'' I said. The one place a teacher 
\discretionary{would}{not}{wouldn't} be
after this period was the lounge. ``We'll go in the back way.'' The
lounge had a little kitchenette off to one side, with its own entrance
for teachers who just wanted to pop in and get a cup of joe. The
microwave -- which always reeked of popcorn and spilled soup -- was
right in there, on top of the miniature fridge.

Darryl groaned. I thought fast. ``Look, the bell's \emph{already rung}. if
you go to study hall now, you'll get a late-slip. Better not to show
at all at this point. I can infiltrate and exfiltrate any room on this
campus, D. You've seen me do it. I'll keep you safe, bro.''

He groaned again. That was one of Darryl's tells: once he starts
groaning, he's ready to give in.

``Let's roll,'' I said, and we took off.

It was flawless. We skirted the classrooms, took the back stairs into
the basement, and came up the front stairs right in front of the
teachers' lounge. Not a sound came from the door, and I quietly turned
the knob and dragged Darryl in before silently closing the door.

The book just barely fit in the microwave, which was looking even less
sanitary than it had the last time I'd popped in here to use it. I
conscientiously wrapped it in paper towels before I set it down. ``Man,
teachers are \emph{pigs},'' I hissed. Darryl, white faced and tense, said
nothing.

The arphid died in a shower of sparks, which was really quite lovely
(though not nearly as pretty as the effect you get when you nuke a
frozen grape, which has to be seen to be believed).

Now, to exfiltrate the campus in perfect anonymity and make our
escape.

Darryl opened the door and began to move out, me on his heels. A
second later, he was standing on my toes, elbows jammed into my chest,
as he tried to back-pedal into the closet-sized kitchen we'd just
left.

``Get back,'' he whispered urgently. ``Quick -- it's Charles!''

Charles Walker and I don't get along. We're in the same grade, and
we've known each other as long as I've known Darryl, but that's where
the resemblance ends. Charles has always been big for his age, and now
that he's playing football and on the juice, he's even bigger. He's
got anger management problems -- I lost a milk-tooth to him in the
third grade, and he's managed to keep from getting in trouble over
them by becoming the most active snitch in school.

It's a bad combination, a bully who also snitches, taking great
pleasure in going to the teachers with whatever infractions he's
found. Benson \emph{loved} Charles. Charles liked to let on that he had
some kind of unspecified bladder problem, which gave him a ready-made
excuse to prowl the hallways at Chavez, looking for people to fink on.

The last time Charles had caught some dirt on me, it had ended with me
giving up LARPing. I had no intention of being caught by him again.

``What's he doing?''

``He's coming this way is what he's doing,'' Darryl said. He was
shaking.

``OK,'' I said. ``OK, time for emergency countermeasures.'' I got my phone
out. I'd planned this well in advance. Charles would never get me
again. I emailed my server at home, and it got into motion.

A few seconds later, Charles's phone spazzed out spectacularly. I'd
had tens of thousands of simultaneous random calls and text messages
sent to it, causing every chirp and ring it had to go off and keep on
going off. The attack was accomplished by means of a botnet, and for
that I felt bad, but it was in the service of a good cause.

Botnets are where infected computers spend their afterlives. When you
get a worm or a virus, your computer sends a message to a chat channel
on IRC -- the Internet Relay Chat. That message tells the botmaster --
the guy who deployed the worm -- that the computers in there ready to
do his bidding. Botnets are supremely powerful, since they can
comprise thousands, even hundreds of thousands of computers, scattered
all over the Internet, connected to juicy high-speed connections and
running on fast home PCs. Those PCs normally function on behalf of
their owners, but when the botmaster calls them, they rise like
zombies to do his bidding.

There are so many infected PCs on the Internet that the price of
hiring an hour or two on a botnet has crashed. Mostly these things
work for spammers as cheap, distributed spambots, filling your mailbox
with come-ons for boner-pills or with new viruses that can infect you
and recruit your machine to join the botnet.

I'd just rented 10 seconds' time on three thousand PCs and had each of
them send a text message or voice-over-IP call to Charles's phone,
whose number I'd extracted from a sticky note on Benson's desk during
one fateful office-visit.

Needless to say, Charles's phone was not equipped to handle
this. First the SMSes filled the memory on his phone, causing it to
start choking on the routine operations it needed to do things like
manage the ringer and log all those incoming calls' bogus return
numbers (did you know that it's \emph{really easy} to fake the return
number on a caller ID? There are about fifty ways of doing it -- just
google ``spoof caller id'').

Charles stared at it dumbfounded, and jabbed at it furiously, his
thick eyebrows knotting and wiggling as he struggled with the demons
that had possessed his most personal of devices. The plan was working
so far, but he wasn't doing what he was supposed to be doing next --
he was supposed to go find some place to sit down and try to figure
out how to get his phone back.  Darryl shook me by the shoulder, and I
pulled my eye away from the crack in the door.

``What's he doing?'' Darryl whispered.

``I totaled his phone, but he's just staring at it now instead of
moving on.'' It wasn't going to be easy to reboot that thing. Once the
memory was totally filled, it would have a hard time loading the code
it needed to delete the bogus messages -- and there was no bulk-erase
for texts on his phone, so he'd have to manually delete all of the
thousands of messages.

Darryl shoved me back and stuck his eye up to the door. A moment
later, his shoulders started to shake. I got scared, thinking he was
panicking, but when he pulled back, I saw that he was laughing so hard
that tears were streaming down his cheeks.

``Galvez just totally busted him for being in the halls during class
\emph{and} for having his phone out -- you should have seen her tear into
him. She was really enjoying it.''

We shook hands solemnly and snuck back out of the corridor, down the
stairs, around the back, out the door, past the fence and out into the
glorious sunlight of afternoon in the Mission. Valencia Street had
never looked so good. I checked my watch and yelped.

``Let's move! The rest of the gang is meeting us at the cable-cars in
twenty minutes!''

\fancybreak{\#}

Van spotted us first. She was blending in with a group of Korean
tourists, which is one of her favorite ways of camouflaging herself
when she's ditching school. Ever since the truancy moblog went live,
our world is full of nosy shopkeepers and pecksniffs who take it upon
themselves to snap our piccies and put them on the net where they can
be perused by school administrators.

She came out of the crowd and bounded toward us. Darryl has had a
thing for Van since forever, and she's sweet enough to pretend she
doesn't know it. She gave me a hug and then moved onto Darryl, giving
him a quick sisterly kiss on the cheek that made him go red to the
tops of his ears.

The two of them made a funny pair: Darryl is a little on the heavy
side, though he wears it well, and he's got a kind of pink complexion
that goes red in the cheeks whenever he runs or gets excited. He's
been able to grow a beard since we were 14, but thankfully he started
shaving after a brief period known to our gang as ``the Lincoln years.''
And he's tall. Very, very tall. Like basketball player tall.

Meanwhile, Van is half a head shorter than me, and skinny, with
straight black hair that she wears in crazy, elaborate braids that she
researches on the net. She's got pretty coppery skin and dark eyes,
and she loves big glass rings the size of radishes, which click and
clack together when she dances.

``Where's Jolu?'' she said.

``How are you, Van?'' Darryl asked in a choked voice. He always ran a
step behind the conversation when it came to Van.

``I'm great, D. How's your every little thing?'' Oh, she was a bad, bad
person. Darryl nearly fainted.

Jolu saved him from social disgrace by showing up just then, in an
oversize leather baseball jacket, sharp sneakers, and a meshback cap
advertising our favorite Mexican masked wrestler, El Santo
Junior. Jolu is Jose Luis Torrez, the completing member of our
foursome. He went to a super-strict Catholic school in the Outer
Richmond, so it wasn't easy for him to get out. But he always did: no
one exfiltrated like our Jolu. He liked his jacket because it hung
down low -- which was pretty stylish in parts of the city -- and
covered up all his Catholic school crap, which was like a bulls-eye
for nosy jerks with the truancy moblog bookmarked on their phones.

``Who's ready to go?'' I asked, once we'd all said hello. I pulled out
my phone and showed them the map I'd downloaded to it on the
BART. ``Near as I can work out, we wanna go up to the Nikko again, then
one block past it to O'Farrell, then left up toward Van
Ness. Somewhere in there we should find the wireless signal.''

Van made a face. ``That's a nasty part of the Tenderloin.'' I couldn't
argue with her. That part of San Francisco is one of the weird bits --
you go in through the Hilton's front entrance and it's all touristy
stuff like the cable-car turnaround and family restaurants. Go through
to the other side and you're in the 'Loin, where every tracked out
transvestite hooker, hard-case pimp, hissing drug dealer and cracked
up homeless person in town was concentrated. What they bought and
sold, none of us were old enough to be a part of (though there were
plenty of hookers our age plying their trade in the 'Loin.)

``Look on the bright side,'' I said. ``The only time you want to go up
around there is broad daylight. None of the other players are going to
go near it until tomorrow at the earliest. This is what we in the ARG
business call a \emph{monster head start.}''

Jolu grinned at me. ``You make it sound like a good thing,'' he said.

``Beats eating uni,'' I said.

``We going to talk or we going to win?'' Van said. After me, she was
hands-down the most hardcore player in our group. She took winning
very, very seriously.

We struck out, four good friends, on our way to decode a clue, win the
game -- and lose everything we cared about, forever.

\fancybreak{\#}

The physical component of today's clue was a set of GPS coordinates --
there were coordinates for all the major cities where Harajuku Fun
Madness was played -- where we'd find a WiFi access-point's
signal. That signal was being deliberately jammed by another, nearby
WiFi point that was hidden so that it couldn't be spotted by
conventional wifinders, little key-fobs that told you when you were
within range of someone's open access-point, which you could use for
free.

We'd have to track down the location of the ``hidden'' access point by
measuring the strength of the ``visible'' one, finding the spot where it
was most mysteriously weakest. There we'd find another clue -- last
time it had been in the special of the day at Anzu, the swanky sushi
restaurant in the Nikko hotel in the Tenderloin. The Nikko was owned
by Japan Airlines, one of Harajuku Fun Madness's sponsors, and the
staff had all made a big fuss over us when we finally tracked down the
clue. They'd given us bowls of miso soup and made us try uni, which is
sushi made from sea urchin, with the texture of very runny cheese and
a smell like very runny dog-droppings. But it tasted \emph{really} good. Or
so Darryl told me. I wasn't going to eat that stuff.

I picked up the WiFi signal with my phone's wifinder about three
blocks up O'Farrell, just before Hyde Street, in front of a dodgy
``Asian Massage Parlor'' with a red blinking CLOSED sign in the
window. The network's name was HarajukuFM, so we knew we had the right
spot.

``If it's in there, I'm not going,'' Darryl said.

``You all got your wifinders?'' I said.

Darryl and Van had phones with built-in wifinders, while Jolu, being
too cool to carry a phone bigger than his pinky finger, had a separate
little directional fob.

``OK, fan out and see what we see. You're looking for a sharp drop off
in the signal that gets worse the more you move along it.''

I took a step backward and ended up standing on someone's toes. A
female voice said ``oof'' and I spun around, worried that some crack-ho
was going to stab me for breaking her heels.

Instead, I found myself face to face with another kid my age. She had
a shock of bright pink hair and a sharp, rodent-like face, with big
sunglasses that were practically air-force goggles. She was dressed in
striped tights beneath a black granny dress, with lots of little
Japanese decorer toys safety pinned to it -- anime characters, old
world leaders, emblems from foreign soda-pop.

She held up a camera and snapped a picture of me and my crew.

``Cheese,'' she said. ``You're on candid snitch-cam.''

``No way,'' I said. ``You wouldn't --''

``I will,'' she said. ``I will send this photo to truant watch in thirty
seconds unless you four back off from this clue and let me and my
friends here run it down. You can come back in one hour and it'll be
all yours. I think that's more than fair.''

I looked behind her and noticed three other girls in similar garb --
one with blue hair, one with green, and one with purple. ``Who are you
supposed to be, the Popsicle Squad?''

``We're the team that's going to kick your team's ass at Harajuku Fun
Madness,'' she said. ``And I'm the one who's \emph{right this second} about
to upload your photo and get you in \emph{so much trouble} --''

Behind me I felt Van start forward. Her all-girls school was notorious
for its brawls, and I was pretty sure she was ready to knock this
chick's block off.

Then the world changed forever.

We felt it first, that sickening lurch of the cement under your feet
that every Californian knows instinctively -- \emph{earthquake}. My first
inclination, as always, was to get away: ``when in trouble or in doubt,
run in circles, scream and shout.'' But the fact was, we were already
in the safest place we could be, not in a building that could fall in
on us, not out toward the middle of the road where bits of falling
mortice could brain us.

Earthquakes are eerily quiet -- at first, anyway -- but this wasn't
quiet. This was loud, an incredible roaring sound that was louder than
anything I'd ever heard before. The sound was so punishing it drove me
to my knees, and I wasn't the only one. Darryl shook my arm and
pointed over the buildings and we saw it then: a huge black cloud
rising from the northeast, from the direction of the Bay.

There was another rumble, and the cloud of smoke spread out, that
spreading black shape we'd all grown up seeing in movies. Someone had
just blown up something, in a big way.

There were more rumbles and more tremors. Heads appeared at windows up
and down the street. We all looked at the mushroom cloud in silence.

Then the sirens started.

I'd heard sirens like these before -- they test the civil defense
sirens at noon on Tuesdays. But I'd only heard them go off unscheduled
in old war movies and video games, the kind where someone is bombing
someone else from above. Air raid sirens. The wooooooo sound made it
all less real.

``Report to shelters immediately.'' It was like the voice of God, coming
from all places at once. There were speakers on some of the electric
poles, something I'd never noticed before, and they'd all switched on
at once.

``Report to shelters immediately.'' Shelters? We looked at each other in
confusion. What shelters? The cloud was rising steadily, spreading
out. Was it nuclear? Were we breathing in our last breaths?

The girl with the pink hair grabbed her friends and they tore ass
downhill, back toward the BART station and the foot of the hills.

``REPORT TO SHELTERS IMMEDIATELY.'' There was screaming now, and a lot
of running around. Tourists -- you can always spot the tourists,
they're the ones who think CALIFORNIA = WARM and spend their San
Francisco holidays freezing in shorts and t-shirts -- scattered in
every direction.

``We should go!'' Darryl hollered in my ear, just barely audible over
the shrieking of the sirens, which had been joined by traditional
police sirens. A dozen SFPD cruisers screamed past us.

``REPORT TO SHELTERS IMMEDIATELY.''

``Down to the BART station,'' I hollered. My friends nodded. We closed
ranks and began to move quickly downhill.

\chapter{Chapter 3}

\epigraph{This chapter is dedicated to Borderlands Books, San Francisco's
magnificent independent science fiction bookstore. Borderlands is
basically located across the street from the fictional Cesar Chavez
High depicted in Little Brother, and it's not just notorious for its
brilliant events, signings, book clubs and such, but also for its
amazing hairless Egyptian cat, Ripley, who likes to perch like a
buzzing gargoyle on the computer at the front of the
store. Borderlands is about the friendliest bookstore you could ask
for, filled with comfy places to sit and read, and staffed by
incredibly knowledgeable clerks who know everything there is to know
about science fiction. Even better, they've always been willing to
take orders for my book (by net or phone) and hold them for me to sign
when I drop into the store, then they ship them within the US for
free!}
{Borderland Books: \url{http://www.borderlands-books.com/} 866 Valencia
Ave, San Francisco CA USA 94110 +1 888 893 4008}

We passed a lot of people in the road on the way to the Powell Street
BART. They were running or walking, white-faced and silent or shouting
and panicked. Homeless people cowered in doorways and watched it all,
while a tall black tranny hooker shouted at two mustached young men
about something.

The closer we got to the BART, the worse the press of bodies
became. By the time we reached the stairway down into the station, it
was a mob-scene, a huge brawl of people trying to crowd their way down
a narrow staircase. I had my face crushed up against someone's back,
and someone else was pressed into my back.

Darryl was still beside me -- he was big enough that he was hard to
shove, and Jolu was right behind him, kind of hanging on to his
waist. I spied Vanessa a few yards away, trapped by more people.

``Screw you!'' I heard Van yell behind me. ``Pervert! Get your hands off
of me!''

I strained around against the crowd and saw Van looking with disgust
at an older guy in a nice suit who was kind of smirking at her. She
was digging in her purse and I knew what she was digging for.

``Don't mace him!'' I shouted over the din. ``You'll get us all too.''

At the mention of the word mace, the guy looked scared and kind of
melted back, though the crowd kept him moving forward. Up ahead, I saw
someone, a middle-aged lady in a hippie dress, falter and fall. She
screamed as she went down, and I saw her thrashing to get up, but she
couldn't, the crowd's pressure was too strong. As I neared her, I bent
to help her up, and was nearly knocked over her. I ended up stepping
on her stomach as the crowd pushed me past her, but by then I don't
think she was feeling anything.

I was as scared as I'd ever been. There was screaming everywhere now,
and more bodies on the floor, and the press from behind was as
relentless as a bulldozer. It was all I could do to keep on my feet.

We were in the open concourse where the turnstiles were. It was hardly
any better here -- the enclosed space sent the voices around us
echoing back in a roar that made my head ring, and the smell and
feeling of all those bodies made me feel a claustrophobia I'd never
known I was prone to.

People were still cramming down the stairs, and more were squeezing
past the turnstiles and down the escalators onto the platforms, but it
was clear to me that this wasn't going to have a happy ending.

``Want to take our chances up top?'' I said to Darryl.

``Yes, hell yes,'' he said. ``This is vicious.''

I looked to Vanessa -- there was no way she'd hear me. I managed to
get my phone out and I texted her.

\edialog{We're getting out of here}

I saw her feel the vibe from her phone, then look down at it and then
back at me and nod vigorously. Darryl, meanwhile, had clued Jolu in.

``\emph{What's the plan?}'' Darryl shouted in my ear.

``We're going to have to go back!'' I shouted back, pointing at the
remorseless crush of bodies.

``It's impossible!'' he said.

``It's just going to get more impossible the longer we wait!''

He shrugged. Van worked her way over to me and grabbed hold of my
wrist. I took Darryl and Darryl took Jolu by the other hand and we
pushed out.

It wasn't easy. We moved about three inches a minute at first, then
slowed down even more when we reached the stairway. The people we
passed were none too happy about us shoving them out of the way,
either. A couple people swore at us and there was a guy who looked
like he'd have punched me if he'd been able to get his arms loose. We
passed three more crushed people beneath us, but there was no way I
could have helped them. By that point, I wasn't even thinking of
helping anyone. All I could think of was finding the spaces in front
of us to move into, of Darryl's mighty straining on my wrist, of my
death-grip on Van behind me.

We popped free like Champagne corks an eternity later, blinking in the
grey smoky light. The air raid sirens were still blaring, and the
sound of emergency vehicles' sirens as they tore down Market Street
was even louder. There was almost no one on the streets anymore --
just the people trying hopelessly to get underground. A lot of them
were crying. I spotted a bunch of empty benches -- usually staked out
by skanky winos -- and pointed toward them.

We moved for them, the sirens and the smoke making us duck and hunch
our shoulders. We got as far as the benches before Darryl fell
forward.

We all yelled and Vanessa grabbed him and turned him over. The side of
his shirt was stained red, and the stain was spreading. She tugged his
shirt up and revealed a long, deep cut in his pudgy side.

``Someone freaking \emph{stabbed} him in the crowd,'' Jolu said, his hands
clenching into fists. ``Christ, that's vicious.''

Darryl groaned and looked at us, then down at his side, then he
groaned and his head went back again.

Vanessa took off her jean jacket and then pulled off the cotton hoodie
she was wearing underneath it. She wadded it up and pressed it to
Darryl's side. ``Take his head,'' she said to me. ``Keep it elevated.'' To
Jolu she said, ``Get his feet up -- roll up your coat or something.''
Jolu moved quickly. Vanessa's mother is a nurse and she'd had first
aid training every summer at camp. She loved to watch people in movies
get their first aid wrong and make fun of them. I was so glad to have
her with us.

We sat there for a long time, holding the hoodie to Darryl's side. He
kept insisting that he was fine and that we should let him up, and Van
kept telling him to shut up and lie still before she kicked his ass.

``What about calling 911?'' Jolu said.

I felt like an idiot. I whipped my phone out and punched 911. The
sound I got wasn't even a busy signal -- it was like a whimper of pain
from the phone system. You don't get sounds like that unless there's
three million people all dialing the same number at once. Who needs
botnets when you've got terrorists?

``What about Wikipedia?'' Jolu said.

``No phone, no data,'' I said.

``What about them?'' Darryl said, and pointed at the street. I looked
where he was pointing, thinking I'd see a cop or an paramedic, but
there was no one there.

``It's OK buddy, you just rest,'' I said.

``No, you idiot, what about \emph{them}, the cops in the cars? There!''

He was right. Every five seconds, a cop car, an ambulance or a
firetruck zoomed past. They could get us some help. I was such an
idiot.

``Come on, then,'' I said, ``let's get you where they can see you and
flag one down.''

Vanessa didn't like it, but I figured a cop wasn't going to stop for a
kid waving his hat in the street, not that day. They just might stop
if they saw Darryl bleeding there, though. I argued briefly with her
and Darryl settled it by lurching to his feet and dragging himself
down toward Market Street.

The first vehicle that screamed past -- an ambulance -- didn't even
slow down. Neither did the cop car that went past, nor the firetruck,
nor the next three cop-cars. Darryl wasn't in good shape -- he was
white-faced and panting. Van's sweater was soaked in blood.

I was sick of cars driving right past me. The next time a car appeared
down Market Street, I stepped right out into the road, waving my arms
over my head, shouting ``\emph{STOP}.'' The car slewed to a stop and only
then did I notice that it wasn't a cop car, ambulance or fire-engine.

It was a military-looking Jeep, like an armored Hummer, only it didn't
have any military insignia on it. The car skidded to a stop just in
front of me, and I jumped back and lost my balance and ended up on the
road. I felt the doors open near me, and then saw a confusion of
booted feet moving close by. I looked up and saw a bunch of
military-looking guys in coveralls, holding big, bulky rifles and
wearing hooded gas masks with tinted face-plates.

I barely had time to register them before those rifles were pointed at
me. I'd never looked down the barrel of a gun before, but everything
you've heard about the experience is true. You freeze where you are,
time stops, and your heart thunders in your ears. I opened my mouth,
then shut it, then, very slowly, I held my hands up in front of me.

The faceless, eyeless armed man above me kept his gun very level. I
didn't even breathe. Van was screaming something and Jolu was shouting
and I looked at them for a second and that was when someone put a
coarse sack over my head and cinched it tight around my windpipe, so
quick and so fiercely I barely had time to gasp before it was locked
on me. I was pushed roughly but dispassionately onto my stomach and
something went twice around my wrists and then tightened up as well,
feeling like baling wire and biting cruelly. I cried out and my own
voice was muffled by the hood.

I was in total darkness now and I strained my ears to hear what was
going on with my friends. I heard them shouting through the muffling
canvas of the bag, and then I was being impersonally hauled to my feet
by my wrists, my arms wrenched up behind my back, my shoulders
screaming.

I stumbled some, then a hand pushed my head down and I was inside the
Hummer. More bodies were roughly shoved in beside me.

``Guys?'' I shouted, and earned a hard thump on my head for my
trouble. I heard Jolu respond, then felt the thump he was dealt,
too. My head rang like a gong.

``Hey,'' I said to the soldiers. ``Hey, listen! We're just high school
students. I wanted to flag you down because my friend was
bleeding. Someone stabbed him.'' I had no idea how much of this was
making it through the muffling bag. I kept talking. ``Listen -- this is
some kind of misunderstanding. We've got to get my friend to a
hospital --''

Someone went upside my head again. It felt like they used a baton or
something -- it was harder than anyone had ever hit me in the head
before. My eyes swam and watered and I literally couldn't breathe
through the pain. A moment later, I caught my breath, but I didn't say
anything. I'd learned my lesson.

Who were these clowns? They weren't wearing insignia. Maybe they were
terrorists! I'd never really believed in terrorists before -- I mean,
I knew that in the abstract there were terrorists somewhere in the
world, but they didn't really represent any risk to me. There were
millions of ways that the world could kill me -- starting with getting
run down by a drunk burning his way down Valencia -- that were
infinitely more likely and immediate than terrorists. Terrorists
killed a lot fewer people than bathroom falls and accidental
electrocutions. Worrying about them always struck me as about as
useful as worrying about getting hit by lightning.

Sitting in the back of that Hummer, my head in a hood, my hands lashed
behind my back, lurching back and forth while the bruises swelled up
on my head, terrorism suddenly felt a lot riskier.

The car rocked back and forth and tipped uphill. I gathered we were
headed over Nob Hill, and from the angle, it seemed we were taking one
of the steeper routes -- I guessed Powell Street.

Now we were descending just as steeply. If my mental map was right, we
were heading down to Fisherman's Wharf. You could get on a boat there,
get away. That fit with the terrorism hypothesis. Why the hell would
terrorists kidnap a bunch of high school students?

We rocked to a stop still on a downslope. The engine died and then the
doors swung open. Someone dragged me by my arms out onto the road,
then shoved me, stumbling, down a paved road. A few seconds later, I
tripped over a steel staircase, bashing my shins. The hands behind me
gave me another shove. I went up the stairs cautiously, not able to
use my hands. I got up the third step and reached for the fourth, but
it wasn't there. I nearly fell again, but new hands grabbed me from in
front and dragged me down a steel floor and then forced me to my knees
and locked my hands to something behind me.

More movement, and the sense of bodies being shackled in alongside of
me. Groans and muffled sounds. Laughter. Then a long, timeless
eternity in the muffled gloom, breathing my own breath, hearing my own
breath in my ears.

\fancybreak{\#}

I actually managed a kind of sleep there, kneeling with the
circulation cut off to my legs, my head in canvas twilight. My body
had squirted a year's supply of adrenalin into my bloodstream in the
space of 30 minutes, and while that stuff can give you the strength to
lift cars off your loved ones and leap over tall buildings, the
payback's always a bitch.

I woke up to someone pulling the hood off my head. They were neither
rough nor careful -- just \ldots impersonal. Like someone at McDonald's
putting together burgers.

The light in the room was so bright I had to squeeze my eyes shut, but
slowly I was able to open them to slits, then cracks, then all the way
and look around.

We were all in the back of a truck, a big 16-wheeler. I could see the
wheel-wells at regular intervals down the length. But the back of this
truck had been turned into some kind of mobile
command-post/jail. Steel desks lined the walls with banks of slick
flat-panel displays climbing above them on articulated arms that let
them be repositioned in a halo around the operators. Each desk had a
gorgeous office-chair in front of it, festooned with user-interface
knobs for adjusting every millimeter of the sitting surface, as well
as height, pitch and yaw.

Then there was the jail part -- at the front of the truck, furthest
away from the doors, there were steel rails bolted into the sides of
the vehicle, and attached to these steel rails were the prisoners.

I spotted Van and Jolu right away. Darryl might have been in the
remaining dozen shackled up back here, but it was impossible to say --
many of them were slumped over and blocking my view. It stank of sweat
and fear back there.

Vanessa looked at me and bit her lip. She was scared. So was I. So was
Jolu, his eyes rolling crazily in their sockets, the whites showing. I
was scared. What's more, I had to piss like a \emph{race-horse.}

I looked around for our captors. I'd avoided looking at them up until
now, the same way you don't look into the dark of a closet where your
mind has conjured up a boogey-man. You don't want to know if you're
right.

But I had to get a better look at these jerks who'd kidnapped us. If
they were terrorists, I wanted to know. I didn't know what a terrorist
looked like, though TV shows had done their best to convince me that
they were brown Arabs with big beards and knit caps and loose cotton
dresses that hung down to their ankles.

Not so our captors. They could have been half-time-show cheerleaders
on the Super Bowl. They looked \emph{American} in a way I couldn't exactly
define. Good jaw-lines, short, neat haircuts that weren't quite
military. They came in white and brown, male and female, and smiled
freely at one another as they sat down at the other end of the truck,
joking and drinking coffees out of go-cups. These weren't Ay-rabs from
Afghanistan: they looked like tourists from Nebraska.

I stared at one, a young white woman with brown hair who barely looked
older than me, kind of cute in a scary office-power-suit way. If you
stare at someone long enough, they'll eventually look back at you. She
did, and her face slammed into a totally different configuration,
dispassionate, even robotic. The smile vanished in an instant.

``Hey,'' I said. ``Look, I don't understand what's going on here, but I
really need to take a leak, you know?''

She looked right through me as if she hadn't heard.

``I'm serious, if I don't get to a can soon, I'm going to have an ugly
accident. It's going to get pretty smelly back here, you know?''

She turned to her colleagues, a little huddle of three of them, and
they held a low conversation I couldn't hear over the fans from the
computers.

She turned back to me. ``Hold it for another ten minutes, then you'll
each get a piss-call.''

``I don't think I've got another ten minutes in me,'' I said, letting a
little more urgency than I was really feeling creep into my
voice. ``Seriously, lady, it's now or never.''

She shook her head and looked at me like I was some kind of pathetic
loser. She and her friends conferred some more, then another one came
forward. He was older, in his early thirties, and pretty big across
the shoulders, like he worked out. He looked like he was Chinese or
Korean -- even Van can't tell the difference sometimes -- but with
that bearing that said \emph{American} in a way I couldn't put my finger
on.

He pulled his sports-coat aside to let me see the hardware strapped
there: I recognized a pistol, a tazer and a can of either mace or
pepper-spray before he let it fall again.

``No trouble,'' he said.

``None,'' I agreed.

He touched something at his belt and the shackles behind me let go, my
arms dropping suddenly behind me. It was like he was wearing Batman's
utility belt -- wireless remotes for shackles! I guessed it made
sense, though: you wouldn't want to lean over your prisoners with all
that deadly hardware at their eye-level -- they might grab your gun
with their teeth and pull the trigger with their tongues or something.

My hands were still lashed together behind me by the plastic
strapping, and now that I wasn't supported by the shackles, I found
that my legs had turned into lumps of cork while I was stuck in one
position. Long story short, I basically fell onto my face and kicked
my legs weakly as they went pins-and-needles, trying to get them under
me so I could rock up to my feet.

The guy jerked me to my feet and I clown-walked to the very back of
the truck, to a little boxed-in porta-john there. I tried to spot
Darryl on the way back, but he could have been any of the five or six
slumped people. Or none of them.

``In you go,'' the guy said.

I jerked my wrists. ``Take these off, please?'' My fingers felt like
purple sausages from the hours of bondage in the plastic cuffs.

The guy didn't move.

``Look,'' I said, trying not to sound sarcastic or angry (it wasn't
easy). ``Look. You either cut my wrists free or you're going to have to
aim for me. A toilet visit is not a hands-free experience.'' Someone in
the truck sniggered. The guy didn't like me, I could tell from the way
his jaw muscles ground around. Man, these people were wired tight.

He reached down to his belt and came up with a very nice set of
multi-pliers. He flicked out a wicked-looking knife and sliced through
the plastic cuffs and my hands were my own again.

``Thanks,'' I said.

He shoved me into the bathroom. My hands were useless, like lumps of
clay on the ends of my wrists. As I wiggled my fingers limply, they
tingled, then the tingling turned to a burning feeling that almost
made me cry out. I put the seat down, dropped my pants and sat down. I
didn't trust myself to stay on my feet.

As my bladder cut loose, so did my eyes. I wept, crying silently and
rocking back and forth while the tears and snot ran down my face. It
was all I could do to keep from sobbing -- I covered my mouth and held
the sounds in. I didn't want to give them the satisfaction.

Finally, I was peed out and cried out and the guy was pounding on the
door. I cleaned my face as best as I could with wads of toilet paper,
stuck it all down the john and flushed, then looked around for a sink
but only found a pump-bottle of heavy-duty hand-sanitizer covered in
small-print lists of the bio-agents it worked on. I rubbed some into
my hands and stepped out of the john.

``What were you doing in there?'' the guy said.

``Using the facilities,'' I said. He turned me around and grabbed my
hands and I felt a new pair of plastic cuffs go around them. My wrists
had swollen since the last pair had come off and the new ones bit
cruelly into my tender skin, but I refused to give him the
satisfaction of crying out.

He shackled me back to my spot and grabbed the next person down, who,
I saw now, was Jolu, his face puffy and an ugly bruise on his cheek.

``Are you OK?'' I asked him, and my friend with the utility belt
abruptly put his hand on my forehead and shoved hard, bouncing the
back of my head off the truck's metal wall with a sound like a clock
striking one. ``No talking,'' he said as I struggled to refocus my eyes.

I didn't like these people. I decided right then that they would pay a
price for all this.

One by one, all the prisoners went to the can, and came back, and when
they were done, my guard went back to his friends and had another cup
of coffee -- they were drinking out of a big cardboard urn of
Starbucks, I saw -- and they had an indistinct conversation that
involved a fair bit of laughter.

Then the door at the back of the truck opened and there was fresh air,
not smoky the way it had been before, but tinged with ozone. In the
slice of outdoors I saw before the door closed, I caught that it was
dark out, and raining, with one of those San Francisco drizzles that's
part mist.

The man who came in was wearing a military uniform. A US military
uniform. He saluted the people in the truck and they saluted him back
and that's when I knew that I wasn't a prisoner of some terrorists --
I was a prisoner of the United States of America.

\fancybreak{\#}

They set up a little screen at the end of the truck and then came for
us one at a time, unshackling us and leading us to the back of the
truck. As close as I could work it -- counting seconds off in my head,
one hippopotami, two hippopotami -- the interviews lasted about seven
minutes each. My head throbbed with dehydration and caffeine
withdrawal.

I was third, brought back by the woman with the severe haircut. Up
close, she looked tired, with bags under her eyes and grim lines at
the corners of her mouth.

``Thanks,'' I said, automatically, as she unlocked me with a remote and
then dragged me to my feet. I hated myself for the automatic
politeness, but it had been drilled into me.

She didn't twitch a muscle. I went ahead of her to the back of the
truck and behind the screen. There was a single folding chair and I
sat in it. Two of them -- Severe Haircut woman and utility belt man --
looked at me from their ergonomic super-chairs.

They had a little table between them with the contents of my wallet
and backpack spread out on it.

``Hello, Marcus,'' Severe Haircut woman said. ``We have some questions
for you.''

``Am I under arrest?'' I asked. This wasn't an idle question. If you're
not under arrest, there are limits on what the cops can and can't do
to you. For starters, they can't hold you forever without arresting
you, giving you a phone call, and letting you talk to a lawyer. And
hoo-boy, was I ever going to talk to a lawyer.

``What's this for?'' she said, holding up my phone. The screen was
showing the error message you got if you kept trying to get into its
data without giving the right password. It was a bit of a rude message
-- an animated hand giving a certain universally recognized gesture --
because I liked to customize my gear.

``Am I under arrest?'' I repeated. They can't make you answer any
questions if you're not under arrest, and when you ask if you're under
arrest, they have to answer you. It's the rules.

``You're being detained by the Department of Homeland Security,'' the
woman snapped.

``Am I under arrest?''

``You're going to be more cooperative, Marcus, starting right now.'' She
didn't say, ``or else,'' but it was implied.

``I would like to contact an attorney,'' I said. ``I would like to know
what I've been charged with. I would like to see some form of
identification from both of you.''

The two agents exchanged looks.

``I think you should really reconsider your approach to this
situation,'' Severe Haircut woman said. ``I think you should do that
right now. We found a number of suspicious devices on your person. We
found you and your confederates near the site of the worst terrorist
attack this country has ever seen. Put those two facts together and
things don't look very good for you, Marcus. You can cooperate, or you
can be very, very sorry. Now, what is this for?''

``You think I'm a terrorist? I'm seventeen years old!''

``Just the right age -- Al Qaeda loves recruiting impressionable,
idealistic kids. We googled you, you know. You've posted a lot of very
ugly stuff on the public Internet.''

``I would like to speak to an attorney,'' I said.

Severe haircut lady looked at me like I was a bug. ``You're under the
mistaken impression that you've been picked up by the police for a
crime. You need to get past that. You are being detained as a
potential enemy combatant by the government of the United States. If I
were you, I'd be thinking very hard about how to convince us that you
are not an enemy combatant. Very hard. Because there are dark holes
that enemy combatants can disappear into, very dark deep holes, holes
where you can just vanish. Forever. Are you listening to me young man?
I want you to unlock this phone and then decrypt the files in its
memory. I want you to account for yourself: why were you out on the
street? What do you know about the attack on this city?''

``I'm not going to unlock my phone for you,'' I said, indignant. My
phone's memory had all kinds of private stuff on it: photos, emails,
little hacks and mods I'd installed. ``That's private stuff.''

``What have you got to hide?''

``I've got the right to my privacy,'' I said. ``And I want to speak to an
attorney.''

``This is your last chance, kid. Honest people don't have anything to
hide.''

``I want to speak to an attorney.'' My parents would pay for it. All the
FAQs on getting arrested were clear on this point. Just keep asking to
see an attorney, no matter what they say or do. There's no good that
comes of talking to the cops without your lawyer present. These two
said they weren't cops, but if this wasn't an arrest, what was it?

In hindsight, maybe I should have unlocked my phone for them.

\chapter{Chapter 4}

\epigraph{This chapter is dedicated to Barnes and Noble, a US national
  chain of bookstores. As America's mom-and-pop bookstores were
  vanishing, Barnes and Noble started to build these gigantic temples
  to reading all across the land. Stocking tens of thousands of titles
  (the mall bookstores and grocery-store spinner racks had stocked a
  small fraction of that) and keeping long hours that were convenient
  to families, working people and others potential readers, the B\&N
  stores kept the careers of many writers afloat, stocking titles that
  smaller stores couldn't possibly afford to keep on their limited
  shelves. B\&N has always had strong community outreach programs, and
  I've done some of my best-attended, best-organized signings at B\&N
  stores, including the great events at the (sadly departed) B\&N in
  Union Square, New York, where the mega-signing after the Nebula
  Awards took place, and the B\&N in Chicago that hosted the event
  after the Nebs a few years later. Best of all is that B\&N's ``geeky''
  buyers really Get It when it comes to science fiction, comics and
  manga, games and similar titles. They're passionate and
  knowledgeable about the field and it shows in the excellent
  selection on display at the stores.}
{Barnes and Noble, nationwide:
\url{http://search.barnesandnoble.com/Little-Brother/Cory-Doctorow/e/978076531985/?itm=6}}

They re-shackled and re-hooded me and left me there. A long time
later, the truck started to move, rolling downhill, and then I was
hauled back to my feet. I immediately fell over. My legs were so
asleep they felt like blocks of ice, all except my knees, which were
swollen and tender from all the hours of kneeling.

Hands grabbed my shoulders and feet and I was picked up like a sack of
potatoes. There were indistinct voices around me. Someone
crying. Someone cursing.

I was carried a short distance, then set down and re-shackled to
another railing. My knees wouldn't support me anymore and I pitched
forward, ending up twisted on the ground like a pretzel, straining
against the chains holding my wrists.

Then we were moving again, and this time, it wasn't like driving in a
truck. The floor beneath me rocked gently and vibrated with heavy
diesel engines and I realized I was on a ship! My stomach turned to
ice. I was being taken off America's shores to somewhere \emph{else}, and
who the hell knew where that was? I'd been scared before, but this
thought \emph{terrified} me, left me paralyzed and wordless with fear. I
realized that I might never see my parents again and I actually tasted
a little vomit burn up my throat. The bag over my head closed in on me
and I could barely breathe, something that was compounded by the weird
position I was twisted into.

But mercifully we weren't on the water for very long. It felt like an
hour, but I know now that it was a mere fifteen minutes, and then I
felt us docking, felt footsteps on the decking around me and felt
other prisoners being unshackled and carried or led away. When they
came for me, I tried to stand again, but couldn't, and they carried me
again, impersonally, roughly.

When they took the hood off again, I was in a cell.

The cell was old and crumbled, and smelled of sea air. There was one
window high up, and rusted bars guarded it. It was still dark
outside. There was a blanket on the floor and a little metal toilet
without a seat, set into the wall. The guard who took off my hood
grinned at me and closed the solid steel door behind him.

I gently massaged my legs, hissing as the blood came back into them
and into my hands. Eventually I was able to stand, and then to pace. I
heard other people talking, crying, shouting. I did some shouting too:
``Jolu! Darryl! Vanessa!'' Other voices on the cell-block took up the
cry, shouting out names, too, shouting out obscenities. The nearest
voices sounded like drunks losing their minds on a
street-corner. Maybe I sounded like that too.

Guards shouted at us to be quiet and that just made everyone yell
louder. Eventually we were all howling, screaming our heads off,
screaming our throats raw. Why not? What did we have to lose?

\fancybreak{\#}

The next time they came to question me, I was filthy and tired,
thirsty and hungry. Severe haircut lady was in the new questioning
party, as were three big guys who moved me around like a cut of
meat. One was black, the other two were white, though one might have
been hispanic. They all carried guns. It was like a Benneton's ad
crossed with a game of Counter-Strike.

They'd taken me from my cell and chained my wrists and ankles
together. I paid attention to my surroundings as we went. I heard
water outside and thought that maybe we were on Alcatraz -- it was a
prison, after all, even if it had been a tourist attraction for
generations, the place where you went to see where Al Capone and his
gangster contemporaries did their time. But I'd been to Alcatraz on a
school trip. It was old and rusted, medieval. This place felt like it
dated back to World War Two, not colonial times.

There were bar-codes laser-printed on stickers and placed on each of
the cell-doors, and numbers, but other than that, there was no way to
tell who or what might be behind them.

The interrogation room was modern, with fluorescent lights, ergonomic
chairs -- not for me, though, I got a folding plastic garden-chair --
and a big wooden board-room table. A mirror lined one wall, just like
in the cop shows, and I figured someone or other must be watching from
behind it. Severe haircut lady and her friends helped themselves to
coffees from an urn on a side-table (I could have torn her throat out
with my teeth and taken her coffee just then), and then set a
styrofoam cup of water down next to me -- without unlocking my wrists
from behind my back, so I couldn't reach it. Hardy har har.

``Hello, Marcus,'' Severe Haircut woman said. ``How's your 'tude doing
today?''

I didn't say anything.

``This isn't as bad as it gets you know,'' she said. ``This is as \emph{good}
as it gets from now on. Even once you tell us what we want to know,
even if that convinces us that you were just in the wrong place at the
wrong time, you're a marked man now. We'll be watching you everywhere
you go and everything you do. You've acted like you've got something
to hide, and we don't like that.''

It's pathetic, but all my brain could think about was that phrase,
``convince us that you were in the wrong place at the wrong time.'' This
was the worst thing that had ever happened to me. I had never, ever
felt this bad or this scared before. Those words, ``wrong place at the
wrong time,'' those six words, they were like a lifeline dangling
before me as I thrashed to stay on the surface.

``Hello, Marcus?'' she snapped her fingers in front of my face. ``Over
here, Marcus.'' There was a little smile on her face and I hated myself
for letting her see my fear. ``Marcus, it can be a lot worse than
this. This isn't the worst place we can put you, not by a damned
sight.'' She reached down below the table and came out with a
briefcase, which she snapped open. From it, she withdrew my phone, my
arphid sniper/cloner, my wifinder, and my memory keys. She set them
down on the table one after the other.

``Here's what we want from you. You unlock the phone for us today. If
you do that, you'll get outdoor and bathing privileges. You'll get a
shower and you'll be allowed to walk around in the exercise
yard. Tomorrow, we'll bring you back and ask you to decrypt the data
on these memory sticks. Do that, and you'll get to eat in the mess
hall. The day after, we're going to want your email passwords, and
that will get you library privileges.''

The word ``no'' was on my lips, like a burp trying to come up, but it
wouldn't come. ``Why?'' is what came out instead.

``We want to be sure that you're what you seem to be. This is about
your security, Marcus. Say you're innocent. You might be, though why
an innocent man would act like he's got so much to hide is beyond
me. But say you are: you could have been on that bridge when it
blew. Your parents could have been. Your friends. Don't you want us to
catch the people who attacked your home?''

It's funny, but when she was talking about my getting ``privileges'' it
scared me into submission. I felt like I'd done \emph{something} to end up
where I was, like maybe it was partially my fault, like I could do
something to change it.

But as soon as she switched to this BS about ``safety'' and ``security,''
my spine came back. ``Lady,'' I said, ``you're talking about attacking my
home, but as far as I can tell, you're the only one who's attacked me
lately. I thought I lived in a country with a constitution. I thought
I lived in a country where I had \emph{rights}. You're talking about
defending my freedom by tearing up the Bill of Rights.''

A flicker of annoyance passed over her face, then went away. ``So
melodramatic, Marcus. No one's attacked you. You've been detained by
your country's government while we seek details on the worst terrorist
attack ever perpetrated on our nation's soil. You have it within your
power to help us fight this war on our nation's enemies. You want to
preserve the Bill of Rights? Help us stop bad people from blowing up
your city. Now, you have exactly thirty seconds to unlock that phone
before I send you back to your cell. We have lots of other people to
interview today.''

She looked at her watch. I rattled my wrists, rattled the chains that
kept me from reaching around and unlocking the phone. Yes, I was going
to do it. She'd told me what my path was to freedom -- to the world,
to my parents -- and that had given me hope. Now she'd threatened to
send me away, to take me off that path, and my hope had crashed and
all I could think of was how to get back on it.

So I rattled my wrists, wanting to get to my phone and unlock it for
her, and she just looked at me coldly, checking her watch.

``The password,'' I said, finally understanding what she wanted of
me. She wanted me to say it out loud, here, where she could record it,
where her pals could hear it. She didn't want me to just unlock the
phone. She wanted me to submit to her. To put her in charge of me. To
give up every secret, all my privacy. ``The password,'' I said again,
and then I told her the password. God help me, I submitted to her
will.

She smiled a little prim smile, which had to be her ice-queen
equivalent of a touchdown dance, and the guards led me away. As the
door closed, I saw her bend down over the phone and key the password
in.

I wish I could say that I'd anticipated this possibility in advance
and created a fake password that unlocked a completely innocuous
partition on my phone, but I wasn't nearly that paranoid/clever.

You might be wondering at this point what dark secrets I had locked
away on my phone and memory sticks and email. I'm just a kid, after
all.

The truth is that I had everything to hide, and nothing. Between my
phone and my memory sticks, you could get a pretty good idea of who my
friends were, what I thought of them, all the goofy things we'd
done. You could read the transcripts of the electronic arguments we'd
carried out and the electronic reconciliations we'd arrived at.

You see, I don't delete stuff. Why would I? Storage is cheap, and you
never know when you're going to want to go back to that
stuff. Especially the stupid stuff. You know that feeling you get
sometimes where you're sitting on the subway and there's no one to
talk to and you suddenly remember some bitter fight you had, some
terrible thing you said? Well, it's usually never as bad as you
remember. Being able to go back and see it again is a great way to
remind yourself that you're not as horrible a person as you think you
are. Darryl and I have gotten over more fights that way than I can
count.

And even that's not it. I know my phone is private. I know my memory
sticks are private. That's because of cryptography -- message
scrambling. The math behind crypto is good and solid, and you and me
get access to the same crypto that banks and the National Security
Agency use. There's only one kind of crypto that anyone uses: crypto
that's public, open and can be deployed by anyone. That's how you know
it works.

There's something really liberating about having some corner of your
life that's \emph{yours}, that no one gets to see except you. It's a little
like nudity or taking a dump. Everyone gets naked every once in a
while. Everyone has to squat on the toilet. There's nothing shameful,
deviant or weird about either of them. But what if I decreed that from
now on, every time you went to evacuate some solid waste, you'd have
to do it in a glass room perched in the middle of Times Square, and
you'd be buck naked?

Even if you've got nothing wrong or weird with your body -- and how
many of us can say that? -- you'd have to be pretty strange to like
that idea. Most of us would run screaming. Most of us would hold it in
until we exploded.

It's not about doing something shameful. It's about doing something
\emph{private}. It's about your life belonging to you.

They were taking that from me, piece by piece. As I walked back to my
cell, that feeling of deserving it came back to me. I'd broken a lot
of rules all my life and I'd gotten away with it, by and large. Maybe
this was justice. Maybe this was my past coming back to me. After all,
I had been where I was because I'd snuck out of school.

I got my shower. I got to walk around the yard. There was a patch of
sky overhead, and it smelled like the Bay Area, but beyond that, I had
no clue where I was being held. No other prisoners were visible during
my exercise period, and I got pretty bored with walking in circles. I
strained my ears for any sound that might help me understand what this
place was, but all I heard was the occasional vehicle, some distant
conversations, a plane landing somewhere nearby.

They brought me back to my cell and fed me, a half a pepperoni pie
from Goat Hill Pizza, which I knew well, up on Potrero Hill. The
carton with its familiar graphic and 415 phone number was a reminder
that only a day before, I'd been a free man in a free country and that
now I was a prisoner. I worried constantly about Darryl and fretted
about my other friends. Maybe they'd been more cooperative and had
been released. Maybe they'd told my parents and they were frantically
calling around.

Maybe not.

The cell was fantastically spare, empty as my soul. I fantasized that
the wall opposite my bunk was a screen, that I could be hacking right
now, opening the cell-door. I fantasized about my workbench and the
projects there -- the old cans I was turning into a ghetto
surround-sound rig, the aerial photography kite-cam I was building, my
homebrew laptop.

I wanted to get out of there. I wanted to go home and have my friends
and my school and my parents and my life back. I wanted to be able to
go where I wanted to go, not be stuck pacing and pacing and pacing.

\fancybreak{\#}

They took my passwords for my USB keys next. Those held some
interesting messages I'd downloaded from one online discussion group
or another, some chat transcripts, things where people had helped me
out with some of the knowledge I needed to do the things I did. There
was nothing on there you couldn't find with Google, of course, but I
didn't think that would count in my favor.

I got exercise again that afternoon, and this time there were others
in the yard when I got there, four other guys and two women, of all
ages and racial backgrounds. I guess lots of people were doing things
to earn their ``privileges.''

They gave me half an hour, and I tried to make conversation with the
most normal-seeming of the other prisoners, a black guy about my age
with a short afro. But when I introduced myself and stuck my hand out,
he cut his eyes toward the cameras mounted ominously in the corners of
the yard and kept walking without ever changing his facial expression.

But then, just before they called my name and brought me back into the
building, the door opened and out came -- Vanessa! I'd never been more
glad to see a friendly face. She looked tired and grumpy, but not
hurt, and when she saw me, she shouted my name and ran to me. We
hugged each other hard and I realized I was shaking. Then I realized
she was shaking, too.

``Are you OK?'' she said, holding me at arms' length.

``I'm OK,'' I said. ``They told me they'd let me go if I gave them my
passwords.''

``They keep asking me questions about you and Darryl.''

There was a voice blaring over the loudspeaker, shouting at us to stop
talking, to walk, but we ignored it.

``Answer them,'' I said, instantly. ``Anything they ask, answer them. If
it'll get you out.''

``How are Darryl and Jolu?''

``I haven't seen them.''

The door banged open and four big guards boiled out. Two took me and
two took Vanessa. They forced me to the ground and turned my head away
from Vanessa, though I heard her getting the same treatment. Plastic
cuffs went around my wrists and then I was yanked to my feet and
brought back to my cell.

No dinner came that night. No breakfast came the next morning. No one
came and brought me to the interrogation room to extract more of my
secrets. The plastic cuffs didn't come off, and my shoulders burned,
then ached, then went numb, then burned again. I lost all feeling in
my hands.

I had to pee. I couldn't undo my pants. I really, really had to pee.

I pissed myself.

They came for me after that, once the hot piss had cooled and gone
clammy, making my already filthy jeans stick to my legs. They came for
me and walked me down the long hall lined with doors, each door with
its own bar code, each bar code a prisoner like me. They walked me
down the corridor and brought me to the interrogation room and it was
like a different planet when I entered there, a world where things
were normal, where everything didn't reek of urine. I felt so dirty
and ashamed, and all those feelings of deserving what I got came back
to me.

Severe haircut lady was already sitting. She was perfect: coifed and
with just a little makeup. I smelled her hair stuff. She wrinkled her
nose at me. I felt the shame rise in me.

``Well, you've been a very naughty boy, haven't you? Aren't you a
filthy thing?''

Shame. I looked down at the table. I couldn't bear to look up. I
wanted to tell her my email password and get gone.

``What did you and your friend talk about in the yard?''

I barked a laugh at the table. ``I told her to answer your questions. I
told her to cooperate.''

``So do you give the orders?''

I felt the blood sing in my ears. ``Oh come on,'' I said. ``We play a
\emph{game} together, it's called Harajuku Fun Madness. I'm the \emph{team
captain}. We're not terrorists, we're high school students. I don't
give her orders. I told her that we needed to be \emph{honest} with you so
that we could clear up any suspicion and get out of here.''

She didn't say anything for a moment.

``How is Darryl?'' I said.

``Who?''

``Darryl. You picked us up together. My friend. Someone had stabbed him
in the Powell Street BART. That's why we were up on the surface. To
get him help.''

``I'm sure he's fine, then,'' she said.

My stomach knotted and I almost threw up. ``You don't \emph{know}? You
haven't got him here?''

``Who we have here and who we don't have here is not something we're
going to discuss with you, ever. That's not something you're going to
know. Marcus, you've seen what happens when you don't cooperate with
us. You've seen what happens when you disobey our orders. You've been
a little cooperative, and it's gotten you almost to the point where
you might go free again. If you want to make that possibility into a
reality, you'll stick to answering my questions.''

I didn't say anything.

``You're learning, that's good. Now, your email passwords, please.''

I was ready for this. I gave them everything: server address, login,
password. This didn't matter. I didn't keep any email on my server. I
downloaded it all and kept it on my laptop at home, which downloaded
and deleted my mail from the server every sixty seconds. They wouldn't
get anything out of my mail -- it got cleared off the server and
stored on my laptop at home.

Back to the cell, but they cut loose my hands and they gave me a
shower and a pair of orange prison pants to wear. They were too big
for me and hung down low on my hips, like a Mexican gang-kid in the
Mission. That's where the baggy-pants-down-your-ass look comes from
you know that? From prison. I tell you what, it's less fun when it's
not a fashion statement.

They took away my jeans, and I spent another day in the cell. The
walls were scratched cement over a steel grid. You could tell, because
the steel was rusting in the salt air, and the grid shone through the
green paint in red-orange. My parents were out that window, somewhere.

They came for me again the next day.

``We've been reading your mail for a day now. We changed the password
so that your home computer couldn't fetch it.''

Well, of course they had. I would have done the same, now that I
thought of it.

``We have enough on you now to put you away for a very long time,
Marcus. Your possession of these articles --'' she gestured at all my
little gizmos -- ``and the data we recovered from your phone and memory
sticks, as well as the subversive material we'd no doubt find if we
raided your house and took your computer. It's enough to put you away
until you're an old man. Do you understand that?''

I didn't believe it for a second. There's no way a judge would say
that all this stuff constituted any kind of real crime. It was free
speech, it was technological tinkering. It wasn't a crime.

But who said that these people would ever put me in front of a judge.

``We know where you live, we know who your friends are. We know how you
operate and how you think.''

It dawned on me then. They were about to let me go. The room seemed to
brighten. I heard myself breathing, short little breaths.

``We just want to know one thing: what was the delivery mechanism for
the bombs on the bridge?''

I stopped breathing. The room darkened again.

``What?''

``There were ten charges on the bridge, all along its length. They
weren't in car-trunks. They'd been placed there. Who placed them
there, and how did they get there?''

``What?'' I said it again.

``This is your last chance, Marcus,'' she said. She looked sad. ``You
were doing so well until now. Tell us this and you can go home. You
can get a lawyer and defend yourself in a court of law. There are
doubtless extenuating circumstances that you can use to explain your
actions. Just tell us this thing, and you're gone.''

``I don't know what you're talking about!'' I was crying and I didn't
even care. Sobbing, blubbering. ``I have \emph{no idea what you're talking
about}!''

She shook her head. ``Marcus, please. Let us help you. By now you know
that we always get what we're after.''

There was a gibbering sound in the back of my mind. They were
\emph{insane}. I pulled myself together, working hard to stop the
tears. ``Listen, lady, this is nuts. You've been into my stuff, you've
seen it all. I'm a seventeen year old high school student, not a
terrorist! You can't seriously think --''

``Marcus, haven't you figured out that we're serious yet?'' She shook
her head. ``You get pretty good grades. I thought you'd be smarter than
that.'' She made a flicking gesture and the guards picked me up by the
armpits.

Back in my cell, a hundred little speeches occurred to me. The French
call this ``esprit d'escalier'' -- the spirit of the staircase, the
snappy rebuttals that come to you after you leave the room and slink
down the stairs. In my mind, I stood and delivered, telling her that I
was a citizen who loved my freedom, which made me the patriot and made
her the traitor. In my mind, I shamed her for turning my country into
an armed camp. In my mind, I was eloquent and brilliant and reduced
her to tears.

But you know what? None of those fine words came back to me when they
pulled me out the next day. All I could think of was freedom. My
parents.

``Hello, Marcus,'' she said. ``How are you feeling?''

I looked down at the table. She had a neat pile of documents in front
of her, and her ubiquitous go-cup of Starbucks beside her. I found it
comforting somehow, a reminder that there was a real world out there
somewhere, beyond the walls.

``We're through investigating you, for now.'' She let that hang
there. Maybe it meant that she was letting me go. Maybe it meant that
she was going to throw me in a pit and forget that I existed.

``And?'' I said finally.

``And I want you to impress on you again that we are very serious about
this. Our country has experienced the worst attack ever committed on
its soil. How many 9/11s do you want us to suffer before you're
willing to cooperate? The details of our investigation are secret. We
won't stop at anything in our efforts to bring the perpetrators of
these heinous crimes to justice. Do you understand that?''

``Yes,'' I mumbled.

``We are going to send you home today, but you are a marked man. You
have not been found to be above suspicion -- we're only releasing you
because we're done questioning you for now. But from now on, you
\emph{belong} to us. We will be watching you. We'll be waiting for you to
make a misstep. Do you understand that we can watch you closely, all
the time?''

``Yes,'' I mumbled.

``Good. You will never speak of what happened here to anyone,
ever. This is a matter of national security. Do you know that the
death penalty still holds for treason in time of war?''

``Yes,'' I mumbled.

``Good boy,'' she purred. ``We have some papers here for you to sign.''
She pushed the stack of papers across the table to me. Little post-its
with SIGN HERE printed on them had been stuck throughout them. A guard
undid my cuffs.

I paged through the papers and my eyes watered and my head swam. I
couldn't make sense of them. I tried to decipher the legalese. It
seemed that I was signing a declaration that I had been voluntarily
held and submitted to voluntary questioning, of my own free will.

``What happens if I don't sign this?'' I said.

She snatched the papers back and made that flicking gesture again. The
guards jerked me to my feet.

``Wait!'' I cried. ``Please! I'll sign them!'' They dragged me to the
door. All I could see was that door, all I could think of was it
closing behind me.

I lost it. I wept. I begged to be allowed to sign the papers. To be so
close to freedom and have it snatched away, it made me ready to do
anything. I can't count the number of times I've heard someone say,
``Oh, I'd rather die than do something-or-other'' -- I've said it myself
now and again. But that was the first time I understood what it really
meant. I would have rather died than go back to my cell.

I begged as they took me out into the corridor. I told them I'd sign
anything.

She called out to the guards and they stopped. They brought me
back. They sat me down. One of them put the pen in my hand.

Of course, I signed, and signed and signed.

\fancybreak{\#}

My jeans and t-shirt were back in my cell, laundered and folded. They
smelled of detergent. I put them on and washed my face and sat on my
cot and stared at the wall. They'd taken everything from me. First my
privacy, then my dignity. I'd been ready to sign anything. I would
have signed a confession that said I'd assassinated Abraham Lincoln.

I tried to cry, but it was like my eyes were dry, out of tears.

They got me again. A guard approached me with a hood, like the hood
I'd been put in when they picked us up, whenever that was, days ago,
weeks ago.

The hood went over my head and cinched tight at my neck. I was in
total darkness and the air was stifling and stale. I was raised to my
feet and walked down corridors, up stairs, on gravel. Up a
gangplank. On a ship's steel deck. My hands were chained behind me, to
a railing. I knelt on the deck and listened to the thrum of the diesel
engines.

The ship moved. A hint of salt air made its way into the hood. It was
drizzling and my clothes were heavy with water. I was outside, even if
my head was in a bag. I was outside, in the world, moments from my
freedom.

They came for me and led me off the boat and over uneven ground. Up
three metal stairs. My wrists were unshackled. My hood was removed.

I was back in the truck. Severe haircut woman was there, at the little
desk she'd sat at before. She had a ziploc bag with her, and inside it
were my phone and other little devices, my wallet and the change from
my pockets. She handed them to me wordlessly.

I filled my pockets. It felt so weird to have everything back in its
familiar place, to be wearing my familiar clothes. Outside the truck's
back door, I heard the familiar sounds of my familiar city.

A guard passed me my backpack. The woman extended her hand to me. I
just looked at it. She put it down and gave me a wry smile. Then she
mimed zipping up her lips and pointed to me, and opened the door.

It was daylight outside, gray and drizzling. I was looking down an
alley toward cars and trucks and bikes zipping down the road. I stood
transfixed on the truck's top step, staring at freedom.

My knees shook. I knew now that they were playing with me again. In a
moment, the guards would grab me and drag me back inside, the bag
would go over my head again, and I would be back on the boat and sent
off to the prison again, to the endless, unanswerable questions. I
barely held myself back from stuffing my fist in my mouth.

Then I forced myself to go down one stair. Another stair. The last
stair. My sneakers crunched down on the crap on the alley's floor,
broken glass, a needle, gravel. I took a step. Another. I reached the
mouth of the alley and stepped onto the sidewalk.

No one grabbed me.

I was free.

Then strong arms threw themselves around me. I nearly cried.

\chapter{Chapter 5}

\epigraph{This chapter is dedicated to Secret Headquarters in Los
  Angeles, my drop-dead all-time favorite comic store in the
  world. It's small and selective about what it stocks, and every time
  I walk in, I walk out with three or four collections I'd never heard
  of under my arm. It's like the owners, Dave and David, have the
  uncanny ability to predict exactly what I'm looking for, and they
  lay it out for me seconds before I walk into the store. I discovered
  about three quarters of my favorite comics by wandering into SHQ,
  grabbing something interesting, sinking into one of the comfy
  chairs, and finding myself transported to another world. When my
  second story-collection, OVERCLOCKED, came out, they worked with
  local illustrator Martin Cenreda to do a free mini-comic based on
  Printcrime, the first story in the book. I left LA about a year ago,
  and of all the things I miss about it, Secret Headquarters is right
  at the top of the list.}
{Secret Headquarters\footnote{\url{http://www.thesecretheadquarters.com/}}:
3817 W. Sunset Boulevard, Los Angeles, CA 90026 +1 323 666 2228}

But it was Van, and she \emph{was} crying, and hugging me so hard I
couldn't breathe. I didn't care. I hugged her back, my face buried in
her hair.

``You're OK!'' she said.

``I'm OK,'' I managed.

She finally let go of me and another set of arms wrapped themselves
around me. It was Jolu! They were both there. He whispered, ``You're
safe, bro,'' in my ear and hugged me even tighter than Vanessa had.

When he let go, I looked around. ``Where's Darryl?'' I asked.

They both looked at each other. ``Maybe he's still in the truck,'' Jolu
said.

We turned and looked at the truck at the alley's end. It was a
nondescript white 18-wheeler. Someone had already brought the little
folding staircase inside. The rear lights glowed red, and the truck
rolled backwards towards us, emitting a steady eep, eep, eep.

``Wait!'' I shouted as it accelerated towards us. ``Wait! What about
Darryl?'' The truck drew closer. I kept shouting. ``What about Darryl?''

Jolu and Vanessa each had me by an arm and were dragging me away. I
struggled against them, shouting. The truck pulled out of the alley's
mouth and reversed into the street and pointed itself downhill and
drove away. I tried to run after it, but Van and Jolu wouldn't let me
go.

I sat down on the sidewalk and put my arms around my knees and
cried. I cried and cried and cried, loud sobs of the sort I hadn't
done since I was a little kid. They wouldn't stop coming. I couldn't
stop shaking.

Vanessa and Jolu got me to my feet and moved me a little ways up the
street. There was a Muni bus stop with a bench and they sat me on
it. They were both crying too, and we held each other for a while, and
I knew we were crying for Darryl, whom none of us ever expected to see
again.

\fancybreak{\#}

We were north of Chinatown, at the part where it starts to become
North Beach, a neighborhood with a bunch of neon strip clubs and the
legendary City Lights counterculture bookstore, where the Beat poetry
movement had been founded back in the 1950s.

I knew this part of town well. My parents' favorite Italian restaurant
was here and they liked to take me here for big plates of linguine and
huge Italian ice-cream mountains with candied figs and lethal little
espressos afterward.

Now it was a different place, a place where I was tasting freedom for
the first time in what seemed like an enternity.

We checked our pockets and found enough money to get a table at one of
the Italian restaurants, out on the sidewalk, under an awning. The
pretty waitress lighted a gas-heater with a barbeque lighter, took our
orders and went inside. The sensation of giving orders, of controlling
my destiny, was the most amazing thing I'd ever felt.

``How long were we in there?'' I asked.

``Six days,'' Vanessa said.

``I got five,'' Jolu said.

``I didn't count.''

``What did they do to you?'' Vanessa said. I didn't want to talk about
it, but they were both looking at me. Once I started, I couldn't
stop. I told them everything, even when I'd been forced to piss
myself, and they took it all in silently. I paused when the waitress
delivered our sodas and waited until she got out of earshot, then
finished. In the telling, it receded into the distance. By the end of
it, I couldn't tell if I was embroidering the truth or if I was making
it all seem \emph{less} bad. My memories swam like little fish that I
snatched at, and sometimes they wriggled out of my grasp.

Jolu shook his head. ``They were hard on you, dude,'' he said. He told
us about his stay there. They'd questioned him, mostly about me, and
he'd kept on telling them the truth, sticking to a plain telling of
the facts about that day and about our friendship. They had gotten him
to repeat it over and over again, but they hadn't played games with
his head the way they had with me. He'd eaten his meals in a mess-hall
with a bunch of other people, and been given time in a TV room where
they were shown last year's blockbusters on video.

Vanessa's story was only slightly different. After she'd gotten them
angry by talking to me, they'd taken away her clothes and made her
wear a set of orange prison overalls. She'd been left in her cell for
two days without contact, though she'd been fed regularly. But mostly
it was the same as Jolu: the same questions, repeated again and again.

``They really hated you,'' Jolu said. ``Really had it in for you. Why?''

I couldn't imagine why. Then I remembered.

\emph{You can cooperate, or you can be very, very sorry.}

``It was because I wouldn't unlock my phone for them, that first
night. That's why they singled me out.'' I couldn't believe it, but
there was no other explanation. It had been sheer vindictiveness. My
mind reeled at the thought. They had done all that as a mere
punishment for defying their authority.

I had been scared. Now I was angry. ``Those bastards,'' I said,
softly. ``They did it to get back at me for mouthing off.''

Jolu swore and then Vanessa cut loose in Korean, something she only
did when she was really, really angry.

``I'm going to get them,'' I whispered, staring at my soda. ``I'm going
to get them.''

Jolu shook his head. ``You can't, you know. You can't fight back
against that.''

\fancybreak{\#}

None of us much wanted to talk about revenge then. Instead, we talked
about what we would do next. We had to go home. Our phones' batteries
were dead and it had been years since this neighborhood had any
payphones. We just needed to go home. I even thought about taking a
taxi, but there wasn't enough money between us to make that possible.

So we walked. On the corner, we pumped some quarters into a San
Francisco Chronicle newspaper box and stopped to read the front
section. It had been five days since the bombs went off, but it was
still all over the front cover.

Severe haircut woman had talked about ``the bridge'' blowing up, and I'd
just assumed that she was talking about the Golden Gate bridge, but I
was wrong. The terrorists had blown up the \emph{Bay bridge}.

``Why the hell would they blow up the Bay Bridge?'' I said. ``The Golden
Gate is the one on all the postcards.'' Even if you've never been to
San Francisco, chances are you know what the Golden Gate looks like:
it's that big orange suspension bridge that swoops dramatically from
the old military base called the Presidio to Sausalito, where all the
cutesy wine-country towns are with their scented candle shops and art
galleries. It's picturesque as hell, and it's practically the symbol
for the state of California. If you go to the Disneyland California
Adventure park, there's a replica of it just past the gates, with a
monorail running over it.

So naturally I assumed that if you were going to blow up a bridge in
San Francisco, that's the one you'd blow.

``They probably got scared off by all the cameras and stuff,'' Jolu
said. ``The National Guard's always checking cars at both ends and
there's all those suicide fences and junk all along it.'' People have
been jumping off the Golden Gate since it opened in 1937 -- they
stopped counting after the thousandth suicide in 1995.

``Yeah,'' Vanessa said. ``Plus the Bay Bridge actually goes somewhere.''
The Bay Bridge goes from downtown San Francisco to Oakland and thence
to Berkeley, the East Bay townships that are home to many of the
people who live and work in town. It's one of the only parts of the
Bay Area where a normal person can afford a house big enough to really
stretch out in, and there's also the university and a bunch of light
industry over there. The BART goes under the Bay and connects the two
cities, too, but it's the Bay Bridge that sees most of the
traffic. The Golden Gate was a nice bridge if you were a tourist or a
rich retiree living out in wine country, but it was mostly
ornamental. The Bay Bridge is -- was -- San Francisco's work-horse
bridge.

I thought about it for a minute. ``You guys are right,'' I said. ``But I
don't think that's all of it. We keep acting like terrorists attack
landmarks because they hate landmarks. Terrorists don't hate landmarks
or bridges or airplanes. They just want to screw stuff up and make
people scared. To make terror. So of course they went after the Bay
Bridge after the Golden Gate got all those cameras -- after airplanes
got all metal-detectored and X-rayed.'' I thought about it some more,
staring blankly at the cars rolling down the street, at the people
walking down the sidewalks, at the city all around me. ``Terrorists
don't hate airplanes or bridges. They love terror.'' It was so obvious
I couldn't believe I'd never thought of it before. I guess that being
treated like a terrorist for a few days was enough to clarify my
thinking.

The other two were staring at me. ``I'm right, aren't I? All this crap,
all the X-rays and ID checks, they're all useless, aren't they?''

They nodded slowly.

``Worse than useless,'' I said, my voice going up and cracking. ``Because
they ended up with us in prison, with Darryl --'' I hadn't thought of
Darryl since we sat down and now it came back to me, my friend,
missing, disappeared. I stopped talking and ground my jaws together.

``We have to tell our parents,'' Jolu said.

``We should get a lawyer,'' Vanessa said.

I thought of telling my story. Of telling the world what had become of
me. Of the videos that would no doubt come out, of me weeping, reduced
to a groveling animal.

``We can't tell them anything,'' I said, without thinking.

``What do you mean?'' Van said.

``We can't tell them anything,'' I repeated. ``You heard her. If we talk,
they'll come back for us. They'll do to us what they did to Darryl.''

``You're joking,'' Jolu said. ``You want us to --''

``I want us to fight back,'' I said. ``I want to stay free so that I can
do that. If we go out there and blab, they'll just say that we're
kids, making it up. We don't even know where we were held! No one will
believe us. Then, one day, they'll come for us.

``I'm telling my parents that I was in one of those camps on the other
side of the Bay. I came over to meet you guys there and we got
stranded, and just got loose today. They said in the papers that
people were still wandering home from them.''

``I can't do that,'' Vanessa said. ``After what they did to you, how can
you even think of doing that?''

``It happened to \emph{me}, that's the point. This is me and them, now. I'll
beat them, I'll get Darryl. I'm not going to take this lying down. But
once our parents are involved, that's it for us. No one will believe
us and no one will care. If we do it my way, people will care.''

``What's your way?'' Jolu said. ``What's your plan?''

``I don't know yet,'' I admitted. ``Give me until tomorrow morning, give
me that, at least.'' I knew that once they'd kept it a secret for a
day, it would have to be a secret forever. Our parents would be even
more skeptical if we suddenly ``remembered'' that we'd been held in a
secret prison instead of taken care of in a refugee camp.

Van and Jolu looked at each other.

``I'm just asking for a chance,'' I said. ``We'll work out the story on
the way, get it straight. Give me one day, just one day.''

The other two nodded glumly and we set off downhill again, heading
back towards home. I lived on Potrero Hill, Vanessa lived in the North
Mission and Jolu lived in Noe Valley -- three wildly different
neighborhoods just a few minutes' walk from one another.

We turned onto Market Street and stopped dead. The street was
barricaded at every corner, the cross-streets reduced to a single
lane, and parked down the whole length of Market Street were big,
nondescript 18-wheelers like the one that had carried us, hooded, away
from the ship's docks and to Chinatown.

Each one had three steel steps leading down from the back and they
buzzed with activity as soldiers, people in suits, and cops went in
and out of them. The suits wore little badges on their lapels and the
soldiers scanned them as they went in and out -- wireless
authorization badges. As we walked past one, I got a look at it, and
saw the familiar logo: Department of Homeland Security. The soldier
saw me staring and stared back hard, glaring at me.

I got the message and moved on. I peeled away from the gang at Van
Ness. We clung to each other and cried and promised to call each
other.

The walk back to Potrero Hill has an easy route and a hard route, the
latter taking you over some of the steepest hills in the city, the
kind of thing that you see car chases on in action movies, with cars
catching air as they soar over the zenith. I always take the hard way
home. It's all residential streets, and the old Victorian houses they
call ``painted ladies'' for their gaudy, elaborate paint-jobs, and front
gardens with scented flowers and tall grasses. Housecats stare at you
from hedges, and there are hardly any homeless.

It was so quiet on those streets that it made me wish I'd taken the
\emph{other} route, through the Mission, which is \ldots \emph{raucous} is probably
the best word for it. Loud and vibrant. Lots of rowdy drunks and angry
crack-heads and unconscious junkies, and also lots of families with
strollers, old ladies gossiping on stoops, lowriders with boom-cars
going thumpa-thumpa-thumpa down the streets. There were hipsters and
mopey emo art-students and even a couple old-school punk-rockers, old
guys with pot bellies bulging out beneath their Dead Kennedys
shirts. Also drag queens, angry gang kids, graffiti artists and
bewildered gentrifiers trying not to get killed while their
real-estate investments matured.

I went up Goat Hill and walked past Goat Hill Pizza, which made me
think of the jail I'd been held in, and I had to sit down on the bench
out front of the restaurant until my shakes passed. Then I noticed the
truck up the hill from me, a nondescript 18-wheeler with three metal
steps coming down from the back end. I got up and got moving. I felt
the eyes watching me from all directions.

I hurried the rest of the way home. I didn't look at the painted
ladies or the gardens or the housecats. I kept my eyes down.

Both my parents' cars were in the driveway, even though it was the
middle of the day. Of course. Dad works in the East Bay, so he'd be
stuck at home while they worked on the bridge. Mom -- well, who knew
why Mom was home.

They were home for me.

Even before I'd finished unlocking the door it had been jerked out of
my hand and flung wide. There were both of my parents, looking gray
and haggard, bug-eyed and staring at me. We stood there in frozen
tableau for a moment, then they both rushed forward and dragged me
into the house, nearly tripping me up. They were both talking so loud
and fast all I could hear was a wordless, roaring gabble and they both
hugged me and cried and I cried too and we just stood there like that
in the little foyer, crying and making almost-words until we ran out
of steam and went into the kitchen.

I did what I always did when I came home: got myself a glass of water
from the filter in the fridge and dug a couple cookies out of the
``biscuit barrel'' that mom's sister had sent us from England. The
normalcy of this made my heart stop hammering, my heart catching up
with my brain, and soon we were all sitting at the table.

``Where have you been?'' they both said, more or less in unison.

I had given this some thought on the way home. ``I got trapped,'' I
said. ``In Oakland. I was there with some friends, doing a project, and
we were all quarantined.''

``For five days?''

``Yeah,'' I said. ``Yeah. It was really bad.'' I'd read about the
quarantines in the Chronicle and I cribbed shamelessly from the quotes
they'd published. ``Yeah. Everyone who got caught in the cloud. They
thought we had been attacked with some kind of super-bug and they
packed us into shipping containers in the docklands, like sardines. It
was really hot and sticky. Not much food, either.''

``Christ,'' Dad said, his fists balling up on the table. Dad teaches in
Berkeley three days a week, working with a few grad students in the
library science program. The rest of the time he consults for clients
in city and down the Peninsula, third-wave dotcoms that are doing
various things with archives. He's a mild-mannered librarian by
profession, but he'd been a real radical in the sixties and wrestled a
little in high school. I'd seen him get crazy angry now and again --
I'd even made him that angry now and again -- and he could seriously
lose it when he was Hulking out. He once threw a swing-set from Ikea
across my granddad's whole lawn when it fell apart for the fiftieth
time while he was assembling it.

``Barbarians,'' Mom said. She's been living in America since she was a
teenager, but she still comes over all British when she encounters
American cops, health-care, airport security or homelessness. Then the
word is ``barbarians,'' and her accent comes back strong. We'd been to
London twice to see her family and I can't say as it felt any more
civilized than San Francisco, just more cramped.

``But they let us go, and ferried us over today.'' I was improvising
now.

``Are you hurt?'' Mom said. ``Hungry?''

``Sleepy?''

``Yeah, a little of all that. Also Dopey, Doc, Sneezy and Bashful.'' We
had a family tradition of Seven Dwarfs jokes. They both smiled a
little, but their eyes were still wet. I felt really bad for
them. They must have been out of their minds with worry. I was glad
for a chance to change the subject. ``I'd totally love to eat.''

``I'll order a pizza from Goat Hill,'' Dad said.

``No, not that,'' I said. They both looked at me like I'd sprouted
antennae. I normally have a thing about Goat Hill Pizza -- as in, I
can normally eat it like a goldfish eats his food, gobbling until it
either runs out or I pop. I tried to smile. ``I just don't feel like
pizza,'' I said, lamely. ``Let's order some curry, OK?'' Thank heaven
that San Francisco is take-out central.

Mom went to the drawer of take-out menus (more normalcy, feeling like
a drink of water on a dry, sore throat) and riffled through them. We
spent a couple of distracting minutes going through the menu from the
halal Pakistani place on Valencia. I settled on a mixed tandoori grill
and creamed spinach with farmer's cheese, a salted mango lassi (much
better than it sounds) and little fried pastries in sugar syrup.

Once the food was ordered, the questions started again. 
\discretionary{They}{had}{They'd} heard
from Van's, Jolu's and Darryl's families (of course) and had tried to
report us missing. The police were taking names, but there were so
many ``displaced persons'' that they weren't going to open files on
anyone unless they were still missing after seven days.

Meanwhile, millions of have-you-seen sites had popped up on the net. A
couple of the sites were old MySpace clones that had run out of money
and saw a new lease on life from all the attention. After all, some
venture capitalists had missing family in the Bay Area. Maybe if they
were recovered, the site would attract some new investment. I grabbed
dad's laptop and looked through them. They were plastered with
advertising, of course, and pictures of missing people, mostly grad
photos, wedding pictures and that sort of thing. It was pretty
ghoulish.

I found my pic and saw that it was linked to Van's, Jolu's, and
Darryl's. There was a little form for marking people found and another
one for writing up notes about other missing people. I filled in the
fields for me and Jolu and Van, and left Darryl blank.

``You forgot Darryl,'' Dad said. He didn't like Darryl much -- once he'd
figured out that a couple inches were missing out of one of the
bottles in his liquor cabinet, and to my enduring shame I'd blamed it
on Darryl. In truth, of course, it had been both of us, just fooling
around, trying out vodka-and-Cokes during an all-night gaming session.

``He wasn't with us,'' I said. The lie tasted bitter in my mouth.

``Oh my God,'' my mom said. She squeezed her hands together. ``We just
assumed when you came home that you'd all been together.''

``No,'' I said, the lie growing. ``No, he was supposed to meet us but we
never met up. He's probably just stuck over in Berkeley. He was going
to take the BART over.''

Mom made a whimpering sound. Dad shook his head and closed his
eyes. ``Don't you know about the BART?'' he said.

I shook my head. I could see where this was going. I felt like the
ground was rushing up to me.

``They blew it up,'' Dad said. ``The bastards blew it up at the same time
as the bridge.''

That hadn't been on the front page of the Chronicle, but then, a BART
blowout under the water wouldn't be nearly as picturesque as the
images of the bridge hanging in tatters and pieces over the Bay. The
BART tunnel from the Embarcadero in San Francisco to the West Oakland
station was submerged.

I went back to Dad's computer and surfed the headlines. No one was
sure, but the body count was in the thousands. Between the cars that
plummeted 191 feet to the sea and the people drowned in the trains,
the deaths were mounting. One reporter claimed to have interviewed an
``identity counterfeiter'' who'd helped ``dozens'' of people walk away
from their old lives by simply vanishing after the attacks, getting
new ID made up, and slipping away from bad marriages, bad debts and
bad lives.

Dad actually got tears in his eyes, and Mom was openly crying. They
each hugged me again, patting me with their hands as if to assure
themselves that I was really there. They kept telling me they loved
me. I told them I loved them too.

We had a weepy dinner and Mom and Dad had each had a couple glasses of
wine, which was a lot for them. I told them that I was getting sleepy,
which was true, and mooched up to my room. I wasn't going to bed,
though. I needed to get online and find out what was going on. I
needed to talk to Jolu and Vanessa. I needed to get working on finding
Darryl.

I crept up to my room and opened the door. I hadn't seen my old bed in
what felt like a thousand years. I lay down on it and reached over to
my bedstand to grab my laptop. I must have not plugged it in all the
way -- the electrical adapter needed to be jiggled just right -- so it
had slowly discharged while I was away. I plugged it back in and gave
it a minute or two to charge up before trying to power it up again. I
used the time to get undressed and throw my clothes in the trash -- I
never wanted to see them again -- and put on a clean pair of boxers
and a fresh t-shirt. The fresh-laundered clothes, straight out of my
drawers, felt so familiar and comfortable, like getting hugged by my
parents.

I powered up my laptop and punched a bunch of pillows into place
behind me at the top of the bed. I scooched back and opened my
computer's lid and settled it onto my thighs. It was still booting,
and man, those icons creeping across the screen looked \emph{good}. It came
all the way up and then it started giving me more low-power
warnings. I checked the power-cable again and wiggled it and they went
away. The power-jack was really flaking out.

In fact, it was so bad that I couldn't actually get anything
done. Every time I took my hand off the power-cable it lost contact
and the computer started to complain about its battery. I took a
closer look at it.

The whole case of my computer was slightly misaligned, the seam split
in an angular gape that started narrow and widened toward the back.

Sometimes you look at a piece of equipment and discover something like
this and you wonder, ``Was it always like that?'' Maybe you just never
noticed.

But with my laptop, that wasn't possible. You see, I built it. After
the Board of Ed issued us all with SchoolBooks, there was no way my
parents were going to buy me a computer of my own, even though
technically the SchoolBook didn't belong to me, and I wasn't supposed
to install software on it or mod it.

I had some money saved -- odd jobs, Christmases and birthdays, a
little bit of judicious ebaying. Put it all together and I had enough
money to buy a totally crappy, five-year-old machine.

So Darryl and I built one instead. You can buy laptop cases just like
you can buy cases for desktop PCs, though they're a little more
specialized than plain old PCs. I'd built a couple PCs with Darryl
over the years, scavenging parts from Craigslist and garage sales and
ordering stuff from cheap cheap Taiwanese vendors we found on the
net. I figured that building a laptop would be the best way to get the
power I wanted at the price I could afford.

To build your own laptop, you start by ordering a ``barebook'' -- a
machine with just a little hardware in it and all the right slots. The
good news was, once I was done, I had a machine that was a whole pound
lighter than the Dell I'd had my eye on, ran faster, and cost a third
of what I would have paid Dell. The bad news was that assembling a
laptop is like building one of those ships in a bottle. It's all
finicky work with tweezers and magnifying glasses, trying to get
everything to fit in that little case. Unlike a full-sized PC -- which
is mostly air -- every cubic millimeter of space in a laptop is spoken
for. Every time I thought I had it, I'd go to screw the thing back
together and find that something was keeping the case from closing all
the way, and it'd be back to the drawing board.

So I knew \emph{exactly} how the seam on my laptop was supposed to look
when the thing was closed, and it was \emph{not} supposed to look like
this.

I kept jiggling the power-adapter, but it was hopeless. There was no
way I was going to get the thing to boot without taking it apart. I
groaned and put it beside the bed. I'd deal with it in the morning.

\fancybreak{\#}

That was the theory, anyway. Two hours later, I was still staring at
the ceiling, playing back movies in my head of what they'd done to me,
what I should have done, all regrets and \emph{esprit d'escalier.}

I rolled out of bed. It had gone midnight and I'd heard my parents hit
the sack at eleven. I grabbed the laptop and cleared some space on my
desk and clipped the little LED lamps to the temples of my magnifying
glasses and pulled out a set of little precision screwdrivers. A
minute later, I had the case open and the keyboard removed and I was
staring at the guts of my laptop. I got a can of compressed air and
blew out the dust that the fan had sucked in and looked things over.

Something wasn't right. I couldn't put my finger on it, but then it
had been months since I'd had the lid off this thing. Luckily, the
third time I'd had to open it up and struggle to close it again, I'd
gotten smart: I'd taken a photo of the guts with everything in
place. I hadn't been totally smart: at first, I'd just left that pic
on my hard drive, and naturally I couldn't get to it when I had the
laptop in parts. But then I'd printed it out and stuck it in my messy
drawer of papers, the dead-tree graveyard where I kept all the
warranty cards and pin-out diagrams. I shuffled them -- they seemed
messier than I remembered -- and brought out my photo. I set it down
next to the computer and kind of unfocused my eyes, trying to find
things that looked out of place.

Then I spotted it. The ribbon cable that connected the keyboard to the
logic-board wasn't connected right. That was a weird one. There was no
torque on that part, nothing to dislodge it in the course of normal
operations. I tried to press it back down again and discovered that
the plug wasn't just badly mounted -- there was something between it
and the board. I tweezed it out and shone my light on it.

There was something new in my keyboard. It was a little chunk of
hardware, only a sixteenth of an inch thick, with no markings. The
keyboard was plugged into it, and it was plugged into the board. It
other words, it was perfectly situated to capture all the keystrokes I
made while I typed on my machine.

It was a bug.

My heart thudded in my ears. It was dark and quiet in the house, but
it wasn't a comforting dark. There were eyes out there, eyes and ears,
and they were watching me. Surveilling me. The surveillance I faced at
school had followed me home, but this time, it wasn't just the Board
of Education looking over my shoulder: the Department of Homeland
Security had joined them.

I almost took the bug out. Then I figured that who ever put it there
would know that it was gone. I left it in. It made me sick to do it.

I looked around for more tampering. I couldn't find any, but did that
mean there hadn't been any? Someone had broken into my room and
planted this device -- had disassembled my laptop and reassembled
it. There were lots of other ways to wiretap a computer. I could never
find them all.

I put the machine together with numb fingers. This time, the case
wouldn't snap shut just right, but the power-cable stayed in. I booted
it up and set my fingers on the keyboard, thinking that I would run
some diagnostics and see what was what.

But I couldn't do it.

Hell, maybe my room was wiretapped. Maybe there was a camera spying on
me now.

I'd been feeling paranoid when I got home. Now I was nearly out of my
skin. It felt like I was back in jail, back in the interrogation room,
stalked by entities who had me utterly in their power. It made me want
to cry.

Only one thing for it.

I went into the bathroom and took off the toilet-paper roll and
replaced it with a fresh one. Luckily, it was almost empty already. I
unrolled the rest of the paper and dug through my parts box until I
found a little plastic envelope full of ultra-bright white LEDs I'd
scavenged out of a dead bike-lamp. I punched their leads through the
cardboard tube carefully, using a pin to make the holes, then got out
some wire and connected them all in series with little metal clips. I
twisted the wires into the leads for a nine-volt battery and connected
the battery. Now I had a tube ringed with ultra-bright, directional
LEDs, and I could hold it up to my eye and look through it.

I'd built one of these last year as a science fair project and had
been thrown out of the fair once I showed that there were hidden
cameras in half the classrooms at Chavez High. Pinhead video-cameras
cost less than a good restaurant dinner these days, so they're showing
up everywhere. Sneaky store clerks put them in changing rooms or
tanning salons and get pervy with the hidden footage they get from
their customers -- sometimes they just put it on the net. Knowing how
to turn a toilet-paper roll and three bucks' worth of parts into a
camera-detector is just good sense.

This is the simplest way to catch a spy-cam. They have tiny lenses,
but they reflect light like the dickens. It works best in a dim room:
stare through the tube and slowly scan all the walls and other places
someone might have put a camera until you see the glint of a
reflection. If the reflection stays still as you move around, that's a
lens.

There wasn't a camera in my room -- not one I could detect,
anyway. There might have been audio bugs, of course. Or better
cameras. Or nothing at all. Can you blame me for feeling paranoid?

I loved that laptop. I called it the Salmagundi, which means anything
made out of spare parts.

Once you get to naming your laptop, you know that you're really having
a deep relationship with it. Now, though, I felt like I didn't want to
ever touch it again. I wanted to throw it out the window. Who knew
what they'd done to it? Who knew how it had been tapped?

I put it in a drawer with the lid shut and looked at the ceiling. It
was late and I should be in bed. There was no way I was going to sleep
now, though. I was tapped. Everyone might be tapped. The world had
changed forever.

""I'll find a way to get them,'' I said. It was a vow, I knew it when I
heard it, though I'd never made a vow before.

I couldn't sleep after that. And besides, I had an idea.

Somewhere in my closet was a shrink-wrapped box containing one
still-sealed, mint-in-package Xbox Universal. Every Xbox has been sold
way below cost -- Microsoft makes most of its money charging games
companies money for the right to put out Xbox games -- but the
Universal was the first Xbox that Microsoft decided to give away
entirely for free.

Last Christmas season, there'd been poor losers on every corner
dressed as warriors from the Halo series, handing out bags of these
game-machines as fast as they could. I guess it worked -- everyone
says they sold a whole butt-load of games. Naturally, there were
countermeasures to make sure you only played games from companies that
had bought licenses from Microsoft to make them.

Hackers blow through those countermeasures. The Xbox was cracked by a
kid from MIT who wrote a best-selling book about it, and then the 360
went down, and then the short-lived Xbox Portable (which we all called
the ``luggable'' -- it weighed three pounds!) succumbed. The Universal
was supposed to be totally bulletproof. The high school kids who broke
it were Brazilian Linux hackers who lived in a \emph{favela} -- a kind of
squatter's slum.

Never underestimate the determination of a kid who is time-rich and
cash-poor.

Once the Brazilians published their crack, we all went nuts on
it. Soon there were dozens of alternate operating systems for the Xbox
Universal. My favorite was ParanoidXbox, a flavor of Paranoid
Linux. Paranoid Linux is an operating system that assumes that its
operator is under assault from the government (it was intended for use
by Chinese and Syrian dissidents), and it does everything it can to
keep your communications and documents a secret. It even throws up a
bunch of ``chaff'' communications that are supposed to disguise the fact
that you're doing anything covert. So while you're receiving a
political message one character at a time, ParanoidLinux is pretending
to surf the Web and fill in questionnaires and flirt in
chat-rooms. Meanwhile, one in every five hundred characters you
receive is your real message, a needle buried in a huge haystack.

I'd burned a ParanoidXbox DVD when they first appeared, but I'd never
gotten around to unpacking the Xbox in my closet, finding a TV to hook
it up to and so on. My room is crowded enough as it is without letting
Microsoft crashware eat up valuable workspace.

Tonight, I'd make the sacrifice. It took about twenty minutes to get
up and running. Not having a TV was the hardest part, but eventually I
remembered that I had a little overhead LCD projector that had
standard TV RCA connectors on the back. I connected it to the Xbox and
shone it on the back of my door and got ParanoidLinux installed.

Now I was up and running, and ParanoidLinux was looking for other Xbox
Universals to talk to. Every Xbox Universal comes with built-in
wireless for multiplayer gaming. You can connect to your neighbors on
the wireless link and to the Internet, if you have a wireless Internet
connection. I found three different sets of neighbors in range. Two of
them had their Xbox Universals also connected to the
Internet. ParanoidXbox loved that configuration: it could siphon off
some of my neighbors' Internet connections and use them to get online
through the gaming network. The neighbors would never miss the
packets: they were paying for flat-rate Internet connections, and they
weren't exactly doing a lot of surfing at 2AM.

The best part of all this is how it made me \emph{feel}: in control. My
technology was working for me, serving me, protecting me. It wasn't
spying on me. This is why I loved technology: if you used it right, it
could give you power and privacy.

My brain was really going now, running like 60. There were lots of
reasons to run ParanoidXbox -- the best one was that anyone could
write games for it. Already there was a port of MAME, the Multiple
Arcade Machine Emulator, so you could play practically any game that
had ever been written, all the way back to Pong -- games for the Apple
][+ and games for the Colecovision, games for the NES and the
Dreamcast, and so on.

Even better were all the cool multiplayer games being built
specifically for ParanoidXbox -- totally free hobbyist games that
anyone could run. When you combined it all, you had a free console
full of free games that could get you free Internet access.

And the best part -- as far as I was concerned -- was that
ParanoidXbox was \emph{paranoid}. Every bit that went over the air was
scrambled to within an inch of its life. You could wiretap it all you
wanted, but you'd never figure out who was talking, what they were
talking about, or who they were talking to. Anonymous web, email and
IM. Just what I needed.

All I had to do now was convince everyone I knew to use it too.

\chapter{Chapter 6}

\epigraph{This chapter is dedicated to Powell's Books, the legendary
  ``City of Books'' in Portland, Oregon. Powell's is the largest
  bookstore in the world, an endless, multi-storey universe of papery
  smells and towering shelves. They stock new and used books on the
  same shelves -- something I've always loved -- and every time I've
  stopped in, they've had a veritable mountain of my books, and
  they've been incredibly gracious about asking me to sign the
  store-stock. The clerks are friendly, the stock is fabulous, and
  there's even a Powell's at the Portland airport, making it just
  about the best airport bookstore in the world for my money!}
{Powell's Books\footnote{
\url{http://www.powells.com/cgi-bin/biblio?isbn=9780765319852}}: 1005 W
Burnside, Portland, OR 97209 USA +1 800 878 7323}

Believe it or not, my parents made me go to school the next day. I'd
only fallen into feverish sleep at three in the morning, but at seven
the next day, my Dad was standing at the foot of my bed, threatening
to drag me out by the ankles. I managed to get up -- something had
died in my mouth after painting my eyelids shut -- and into the
shower.

I let my mom force a piece of toast and a banana into me, wishing
fervently that my parents would let me drink coffee at home. I could
sneak one on the way to school, but watching them sip down their black
gold while I was drag-assing around the house, getting dressed and
putting my books in my bag -- it was awful.

I've walked to school a thousand times, but today it was different. I
went up and over the hills to get down into the Mission, and
everywhere there were trucks. I saw new sensors and traffic cameras
installed at many of the stop-signs. Someone had a lot of surveillance
gear lying around, waiting to be installed at the first
opportunity. The attack on the Bay Bridge had been just what they
needed.

It all made the city seem more subdued, like being inside an elevator,
embarrassed by the close scrutiny of your neighbors and the ubiquitous
cameras.

The Turkish coffee shop on 24th Street fixed me up good with a go-cup
of Turkish coffee. Basically, Turkish coffee is mud, pretending to be
coffee. It's thick enough to stand a spoon up in, and it has way more
caffeine than the kiddee-pops like Red Bull. Take it from someone
who's read the Wikipedia entry: this is how the Ottoman Empire was
won: maddened horsemen fueled by lethal jet-black coffee-mud.

I pulled out my debit card to pay and he made a face. ``No more debit,''
he said.

``Huh? Why not?'' I'd paid for my coffee habit on my card for years at
the Turk's. He used to hassle me all the time, telling me I was too
young to drink the stuff, and he still refused to serve me at all
during school hours, convinced that I was skipping class. But over the
years, the Turk and me have developed a kind of gruff understanding.

He shook his head sadly. ``You wouldn't understand. Go to school, kid.''

There's no surer way to make me want to understand than to tell me I
won't. I wheedled him, demanding that he tell me. He looked like he
was going to throw me out, but when I asked him if he thought I wasn't
good enough to shop there, he opened up.

``The security,'' he said, looking around his little shop with its tubs
of dried beans and seeds, its shelves of Turkish groceries. ``The
government. They monitor it all now, it was in the papers. PATRIOT Act
II, the Congress passed it yesterday. Now they can monitor every time
you use your card. I say no. I say my shop will not help them spy on
my customers.''

My jaw dropped.

``You think it's no big deal maybe? What is the problem with government
knowing when you buy coffee? Because it's one way they know where you
are, where you been. Why you think I left Turkey? Where you have
government always spying on the people, is no good. I move here twenty
years ago for freedom -- I no help them take freedom away.''

``You're going to lose so many sales,'' I blurted. I wanted to tell him
he was a hero and shake his hand, but that was what came
out. ``Everyone uses debit cards.''

``Maybe not so much anymore. Maybe my customers come here because they
know I love freedom too. I am making sign for window. Maybe other
stores do the same. I hear the ACLU will sue them for this.''

``You've got all my business from now on,'' I said. I meant it. I
reached into my pocket. ``Um, I don't have any cash, though.''

He pursed his lips and nodded. ``Many peoples say the same thing. Is
OK. You give today's money to the ACLU.''

In two minutes, the Turk and I had exchanged more words than we had in
all the time I'd been coming to his shop. I had no idea he had all
these passions. I just thought of him as my friendly neighborhood
caffeine dealer. Now I shook his hand and when I left his store, I
felt like he and I had joined a team. A secret team.

\fancybreak{\#}

I'd missed two days of school but it seemed like I hadn't missed much
class. They'd shut the school on one of those days while the city
scrambled to recover. The next day had been devoted, it seemed, to
mourning those missing and presumed dead. The newspapers published
biographies of the lost, personal memorials. The Web was filled with
these capsule obituaries, thousands of them.

Embarrassingly, I was one of those people. I stepped into the
schoolyard, not knowing this, and then there was a shout and a moment
later there were a hundred people around me, pounding me on the back,
shaking my hand. A couple girls I didn't even know kissed me, and they
were more than friendly kisses. I felt like a rock star.

My teachers were only a little more subdued. Ms Galvez cried as much
as my mother had and hugged me three times before she let me go to my
desk and sit down. There was something new at the front of the
classroom. A camera. Ms Galvez caught me staring at it and handed me a
permission slip on smeary Xeroxed school letterhead.

The Board of the San Francisco Unified School District had held an
emergency session over the weekend and unanimously voted to ask the
parents of every kid in the city for permission to put closed circuit
television cameras in every classroom and corridor. The law said they
couldn't force us to go to school with cameras all over the place, but
it didn't say anything about us \emph{volunteering} to give up our
Constitutional rights. The letter said that the Board were sure that
they would get complete compliance from the City's parents, but that
they would make arrangements to teach those kids' whose parents
objected in a separate set of ``unprotected'' classrooms.

Why did we have cameras in our classrooms now? Terrorists. Of
course. Because by blowing up a bridge, terrorists had indicated that
schools were next. Somehow that was the conclusion that the Board had
reached anyway.

I read this note three times and then I stuck my hand up.

``Yes, Marcus?''

``Ms Galvez, about this note?''

``Yes, Marcus.''

``Isn't the point of terrorism to make us afraid? That's why it's
called \emph{terror}ism, right?''

``I suppose so.'' The class was staring at me. I wasn't the best student
in school, but I did like a good in-class debate. They were waiting to
hear what I'd say next.

``So aren't we doing what the terrorists want from us? Don't they win
if we act all afraid and put cameras in the classrooms and all of
that?''

There was some nervous tittering. One of the others put his hand
up. It was Charles. Ms Galvez called on him.

``Putting cameras in makes us safe, which makes us less a\-fraid.''

``Safe from what?'' I said, without waiting to be called on.

``Terrorism,'' Charles said. The others were nodding their heads.

``How do they do that? If a suicide bomber rushed in here and blew us
all up --''

``Ms Galvez, Marcus is violating school policy. We're not supposed to
make jokes about terrorist attacks --''

``Who's making jokes?''

``Thank you, both of you,'' Ms Galvez said. She looked really unhappy. I
felt kind of bad for hijacking her class. ``I think that this is a
really interesting discussion, but I'd like to hold it over for a
future class. I think that these issues may be too emotional for us to
have a discussion about them today. Now, let's get back to the
suffragists, shall we?''

So we spent the rest of the hour talking about suffragists and the new
lobbying strategies they'd devised for getting four women into every
congresscritter's office to lean on him and let him know what it would
mean for his political future if he kept on denying women the vote. It
was normally the kind of thing I really liked -- little guys making
the big and powerful be honest. But today I couldn't concentrate. It
must have been Darryl's absence. We both liked Social Studies and we
would have had our SchoolBooks out and an IM session up seconds after
sitting down, a back-channel for talking about the lesson.

I'd burned twenty ParanoidXbox discs the night before and I had them
all in my bag. I handed them out to people I knew were really, really
into gaming. They'd all gotten an Xbox Universal or two the year
before, but most of them had stopped using them. The games were really
expensive and not a lot of fun. I took them aside between periods, at
lunch and study hall, and sang the praises of the ParanoidXbox games
to the sky. Free and fun -- addictive social games with lots of cool
people playing them from all over the world.

Giving away one thing to sell another is what they call a ``razor blade
business'' -- companies like Gillette give you free razor-blade handles
and then stiff you by charging you a small fortune for the
blades. Printer cartridges are the worst for that -- the most
expensive Champagne in the world is cheap when compared with inkjet
ink, which costs all of a penny a gallon to make wholesale.

Razor-blade businesses depend on you not being able to get the
``blades'' from someone else. After all, if Gillette can make nine bucks
on a ten-dollar replacement blade, why not start a competitor that
makes only four bucks selling an identical blade: an 80 percent profit
margin is the kind of thing that makes your average business-guy go
all drooly and round-eyed.

So razor-blade companies like Microsoft pour a lot of effort into
making it hard and/or illegal to compete with them on the blades. In
Microsoft's case, every Xbox has had countermeasures to keep you from
running software that was released by people who didn't pay the
Microsoft blood-money for the right to sell Xbox programs.

The people I met didn't think much about this stuff. They perked up
when I told them that the games were unmonitored. These days, any
online game you play is filled with all kinds of unsavory sorts. First
there are the pervs who try to get you to come out to some remote
location so they can go all weird and Silence of the Lambs on
you. Then there are the cops, who are pretending to be gullible kids
so they can bust the pervs. Worst of all, though, are the monitors who
spend all their time spying on our discussions and snitching on us for
violating their Terms of Service, which say, no flirting, no cussing,
and no ``clear or masked language which insultingly refers to any
aspect of sexual orientation or sexuality.''

I'm no 24/7 horn-dog, but I'm a seventeen year old boy. Sex does come
up in conversation every now and again. But God help you if it came up
in chat while you were gaming. It was a real buzz-kill. No one
monitored the ParanoidXbox games, because they weren't run by a
company: they were just games that hackers had written for the hell of
it.

So these game-kids loved the story. They took the discs greedily, and
promised to burn copies for all of their friends -- after all, games
are most fun when you're playing them with your buddies.

When I got home, I read that a group of parents were suing the school
board over the surveillance cameras in the classrooms, but that they'd
already lost their bid to get a preliminary injunction against them.

\fancybreak{\#}

I don't know who came up with the name Xnet, but it stuck. You'd hear
people talking about it on the Muni. Van called me up to ask me if I'd
heard of it and I nearly choked once I figured out what she was
talking about: the discs I'd started distributing last week had been
sneakernetted and copied all the way to Oakland in the space of two
weeks. It made me look over my shoulder -- like I'd broken a rule and
now the DHS would come and take me away forever.

They'd been hard weeks. The BART had completely abandoned cash fares
now, switching them for arphid ``contactless'' cards that you waved at
the turnstiles to go through. They were cool and convenient, but every
time I used one, I thought about how I was being tracked. Someone on
Xnet posted a link to an Electronic Frontier Foundation white paper on
the ways that these things could be used to track people, and the
paper had tiny stories about little groups of people that had
protested at the BART stations.

I used the Xnet for almost everything now. I'd set up a fake email
address through the Pirate Party, a Swedish political party that hated
Internet surveillance and promised to keep their mail accounts a
secret from everyone, even the cops. I accessed it strictly via Xnet,
hopping from one neighbor's Internet connection to the next, staying
anonymous -- I hoped -- all the way to Sweden. I wasn't using w1n5ton
anymore. If Benson could figure it out, anyone could. My new handle,
come up with on the spur of the moment, was M1k3y, and I got a \emph{lot}
of email from people who heard in chat rooms and message boards that I
could help them troubleshoot their Xnet configurations and
connections.

I missed Harajuku Fun Madness. The company had suspended the game
indefinitely. They said that for ``security reasons'' they didn't think
it would be a good idea to hide things and then send people off to
find them. What if someone thought it was a bomb? What if someone put
a bomb in the same spot?

What if I got hit by lightning while walking with an umbrella? Ban
umbrellas! Fight the menace of lightning!

I kept on using my laptop, though I got a skin-crawly feeling when I
used it. Whoever had wiretapped it would wonder why I didn't use it. I
figured I'd just do some random surfing with it every day, a little
less each day, so that anyone watching would see me slowly changing my
habits, not doing a sudden reversal. Mostly I read those creepy obits
-- all those thousands of my friends and neighbors dead at the bottom
of the Bay.

Truth be told, I \emph{was} doing less and less homework every day. I had
business elsewhere. I burned new stacks of ParanoidXbox every day,
fifty or sixty, and took them around the city to people I'd heard were
willing to burn sixty of their own and hand them out to their friends.

I wasn't too worried about getting caught doing this, because I had
good crypto on my side. Crypto is cryptography, or ``secret writing,''
and it's been around since Roman times (literally: Augustus Caesar was
a big fan and liked to invent his own codes, some of which we use
today for scrambling joke punchlines in email).

Crypto is math. Hard math. I'm not going to try to explain it in
detail because I don't have the math to really get my head around it,
either -- look it up on Wikipedia if you really want.

But here's the Cliff's Notes version: Some kinds of mathematical
functions are really easy to do in one direction and really hard to do
in the other direction. It's easy to multiply two big prime numbers
together and make a giant number. It's really, really hard to take any
given giant number and figure out which primes multiply together to
give you that number.

That means that if you can come up with a way of scrambling something
based on multiplying large primes, unscrambling it without knowing
those primes will be hard. Wicked hard. Like, a trillion years of all
the computers ever invented working 24/7 won't be able to do it.

There are four parts to any crypto message: the original message,
called the ``cleartext.'' The scrambled message, called the
``ciphertext.'' The scrambling system, called the ``cipher.'' And finally
there's the key: secret stuff you feed into the cipher along with the
cleartext to make ciphertext.

It used to be that crypto people tried to keep all of this a
secret. Every agency and government had its own ciphers \emph{and} its own
keys. The Nazis and the Allies didn't want the other guys to know how
they scrambled their messages, let alone the keys that they could use
to descramble them. That sounds like a good idea, right?

Wrong.

The first time anyone told me about all this prime factoring stuff, I
immediately said, ``No way, that's BS. I mean, \emph{sure} it's hard to do
this prime factorization stuff, whatever you say it is. But it used to
be impossible to fly or go to the moon or get a hard-drive with more
than a few kilobytes of storage. Someone \emph{must} have invented a way of
descrambling the messages.'' I had visions of a hollow mountain full of
National Security Agency mathematicians reading every email in the
world and snickering.

In fact, that's pretty much what happened during World War II. That's
the reason that life isn't more like Castle Wolfenstein, where I've
spent many days hunting Nazis.

The thing is, ciphers are hard to keep secret. There's a lot of math
that goes into one, and if they're widely used, then everyone who uses
them has to keep them a secret too, and if someone changes sides, you
have to find a new cipher.

The Nazi cipher was called Enigma, and they used a little mechanical
computer called an Enigma Machine to scramble and unscramble the
messages they got. Every sub and boat and station needed one of these,
so it was inevitable that eventually the Allies would get their hands
on one.

When they did, they cracked it. That work was led by my personal
all-time hero, a guy named Alan Turing, who pretty much invented
computers as we know them today. Unfortunately for him, he was gay, so
after the war ended, the stupid British government forced him to get
shot up with hormones to ``cure'' his homosexuality and he killed
himself. Darryl gave me a biography of Turing for my 14th birthday --
wrapped in twenty layers of paper and in a recycled Batmobile toy, he
was like that with presents -- and I've been a Turing junkie ever
since.

Now the Allies had the Enigma Machine, and they could intercept lots
of Nazi radio-messages, which shouldn't have been that big a deal,
since every captain had his own secret key. Since the Allies didn't
have the keys, having the machine shouldn't have helped.

Here's where secrecy hurts crypto. The Enigma cipher was flawed. Once
Turing looked hard at it, he figured out that the Nazi cryptographers
had made a mathematical mistake. By getting his hands on an Enigma
Machine, Turing could figure out how to crack \emph{any} Nazi message, no
matter what key it used.

That cost the Nazis the war. I mean, don't get me wrong. That's good
news. Take it from a Castle Wolfenstein veteran. You wouldn't want the
Nazis running the country.

After the war, cryptographers spent a lot of time thinking about
this. The problem had been that Turing was smarter than the guy who
thought up Enigma. Any time you had a cipher, you were vulnerable to
someone smarter than you coming up with a way of breaking it.

And the more they thought about it, the more they realized that
\emph{anyone} can come up with a security system that he can't figure out
how to break. But \emph{no one} can figure out what a smarter person might
do.

You have to publish a cipher to know that it works. You have to tell
\emph{as many people as possible} how it works, so that they can thwack on
it with everything they have, testing its security. The longer you go
without anyone finding a flaw, the more secure you are.

Which is how it stands today. If you want to be safe, you don't use
crypto that some genius thought of last week. You use the stuff that
people have been using for as long as possible without anyone figuring
out how to break them. Whether you're a bank, a terrorist, a
government or a teenager, you use the same ciphers.

If you tried to use your own cipher, there'd be the chance that
someone out there had found a flaw you missed and was doing a Turing
on your butt, deciphering all your ``secret'' messages and chuckling at
your dumb gossip, financial transactions and military secrets.

So I knew that crypto would keep me safe from eavesdroppers, but I
wasn't ready to deal with histograms.

\fancybreak{\#}

I got off the BART and waved my card over the turnstile as I headed up
to the 24th Street station. As usual, there were lots of weirdos
hanging out in the station, drunks and Jesus freaks and intense
Mexican men staring at the ground and a few gang kids. I looked
straight past them as I hit the stairs and jogged up to the
surface. My bag was empty now, no longer bulging with the ParanoidXbox
discs I'd been distributing, and it made my shoulders feel light and
put a spring in my step as I came up the street. The preachers were at
work still, exhorting in Spanish and English about Jesus and so on.

The counterfeit sunglass sellers were gone, but they'd been replaced
by guys selling robot dogs that barked the national anthem and would
lift their legs if you showed them a picture of Osama bin Laden. There
was probably some cool stuff going on in their little brains and I
made a mental note to pick a couple of them up and take them apart
later. Face-recognition was pretty new in toys, having only recently
made the leap from the military to casinos trying to find cheats, to
law enforcement.

I started down 24th Street toward Potrero Hill and home, rolling my
shoulders and smelling the burrito smells wafting out of the
restaurants and thinking about dinner.

I don't know why I happened to glance back over my shoulder, but I
did. Maybe it was a little bit of subconscious sixth-sense stuff. I
knew I was being followed.

They were two beefy white guys with little mustaches that made me
think of either cops or the gay bikers who rode up and down the
Castro, but gay guys usually had better haircuts. They had on
windbreakers the color of old cement and blue-jeans, with their
waistbands concealed. I thought of all the things a cop might wear on
his waistband, of the utility-belt that DHS guy in the truck had
worn. Both guys were wearing Bluetooth headsets.

I kept walking, my heart thumping in my chest. I'd been expecting this
since I started. I'd been expecting the DHS to figure out what I was
doing. I took every precaution, but Severe-Haircut woman had told me
that she'd be watching me. She'd told me I was a marked man. I
realized that I'd been waiting to get picked up and taken back to
jail. Why not? Why should Darryl be in jail and not me? What did I
have going for me? I hadn't even had the guts to tell my parents -- or
his -- what had really happened to us.

I quickened my steps and took a mental inventory. I didn't have
anything incriminating in my bag. Not too incriminating, anyway. My
SchoolBook was running the crack that let me IM and stuff, but half
the people in school had that. I'd changed the way I encrypted the
stuff on my phone -- now I \emph{did} have a fake partition that I could
turn back into cleartext with one password, but all the good stuff was
hidden, and needed another password to open up. That hidden section
looked just like random junk -- when you encrypt data, it becomes
indistinguishable from random noise -- and they'd never even know it
was there.

There were no discs in my bag. My laptop was free of incriminating
evidence. Of course, if they thought to look hard at my Xbox, it was
game over. So to speak.

I stopped where I was standing. I'd done as good a job as I could of
covering myself. It was time to face my fate. I stepped into the
nearest burrito joint and ordered one with carnitas -- shredded pork
-- and extra salsa. Might as well go down with a full stomach. I got a
bucket of horchata, too, an ice-cold rice drink that's like watery,
semi-sweet rice-pudding (better than it sounds).

I sat down to eat, and a profound calm fell over me. I was about to go
to jail for my ``crimes,'' or I wasn't. My freedom since they'd taken me
in had been just a temporary holiday. My country was not my friend
anymore: we were now on different sides and I'd known I could never
win.

The two guys came into the restaurant as I was finishing the burrito
and going up to order some churros -- deep-fried dough with cinnamon
sugar -- for dessert. I guess they'd been waiting outside and got
tired of my dawdling.

They stood behind me at the counter, boxing me in. I took my churro
from the pretty granny and paid her, taking a couple of quick bites of
the dough before I turned around. I wanted to eat at least a little of
my dessert. It might be the last dessert I got for a long, long time.

Then I turned around. They were both so close I could see the zit on
the cheek of the one on the left, the little booger up the nose of the
other.

``'Scuse me,'' I said, trying to push past them. The one with the booger
moved to block me.

``Sir,'' he said, ``can you step over here with us?'' He gestured toward
the restaurant's door.

``Sorry, I'm eating,'' I said and moved again. This time he put his hand
on my chest. He was breathing fast through his nose, making the booger
wiggle. I think I was breathing hard too, but it was hard to tell over
the hammering of my heart.

The other one flipped down a flap on the front of his windbreaker to
reveal a SFPD insignia. ``Police,'' he said. ``Please come with us.''

``Let me just get my stuff,'' I said.

``We'll take care of that,'' he said. The booger one stepped right up
close to me, his foot on the inside of mine. You do that in some
martial arts, too. It lets you feel if the other guy is shifting his
weight, getting ready to move.

I wasn't going to run, though. I knew I couldn't outrun fate.

\chapter{Chapter 7}

\epigraph{This chapter is dedicated to New York City's Books of
  Wonder, the oldest and largest kids' bookstore in Manhattan. They're
  located just a few blocks away from Tor Books' offices in the
  Flatiron Building and every time I drop in to meet with the Tor
  people, I always sneak away to Books of Wonder to peruse their stock
  of new, used and rare kids' books. I'm a heavy collector of rare
  editions of Alice in Wonderland, and Books of Wonder never fails to
  excite me with some beautiful, limited-edition Alice. They have tons
  of events for kids and one of the most inviting atmospheres I've
  ever experienced at a bookstore.}  
{Books of Wonder
\url{http://www.booksofwonder.com/} 18 West 18th St, New York, NY
10011 USA +1 212 989 3270}

They took me outside and around the corner, to a waiting unmarked
police car. It wasn't like anyone in that neighborhood would have had
a hard time figuring out that it was a cop-car, though. Only police
drive big Crown Victorias now that gas had hit seven bucks a
gallon. What's more, only cops could double-park in the middle of Van
Ness street without getting towed by the schools of predatory
tow-operators that circled endlessly, ready to enforce San Francisco's
incomprehensible parking regulations and collect a bounty for
kidnapping your car.

Booger blew his nose. I was sitting in the back seat, and so was
he. His partner was sitting in the front, typing with one finger on an
ancient, ruggedized laptop that looked like Fred Flintstone had been
its original owner.

Booger looked closely at my ID again. ``We just want to ask you a few
routine questions.''

``Can I see your badges?'' I said. These guys were clearly cops, but it
couldn't hurt to let them know I knew my rights.

Booger flashed his badge at me too fast for me to get a good look at
it, but Zit in the front seat gave me a long look at his. I got their
division number and memorized the four-digit badge number. It was
easy: 1337 is also the way hackers write ``leet,'' or ``elite.''

They were both being very polite and neither of them was trying to
intimidate me the way that the DHS had done when I was in their
custody.

``Am I under arrest?''

``You've been momentarily detained so that we can ensure your safety
and the general public safety,'' Booger said.

He passed my driver's license up to Zit, who pecked it slowly into his
computer. I saw him make a typo and almost corrected him, but figured
it was better to just keep my mouth shut.

``Is there anything you want to tell me, Marcus? Do they call you
Marc?''

``Marcus is fine,'' I said. Booger looked like he might be a nice
guy. Except for the part about kidnapping me into his car, of course.

``Marcus. Anything you want to tell me?''

``Like what? Am I under arrest?''

``You're not under arrest right now,'' Booger said. ``Would you like to
be?''

``No,'' I said.

``Good. We've been watching you since you left the BART. Your Fast Pass
says that you've been riding to a lot of strange places at a lot of
funny hours.''

I felt something let go inside my chest. This wasn't about the Xnet at
all, then, not really. They'd been watching my subway use and wanted
to know why it had been so freaky lately. How totally stupid.

``So you guys follow everyone who comes out of the BART station with a
funny ride-history? You must be busy.''

``Not everyone, Marcus. We get an alert when anyone with an uncommon
ride profile comes out and that helps us assess whether we want to
investigate. In your case, we came along because we wanted to know why
a smart-looking kid like you had such a funny ride profile?''

Now that I knew I wasn't about to go to jail, I was getting
pissed. These guys had no business spying on me -- Christ, the BART
had no business \emph{helping} them to spy on me. Where the hell did my
subway pass get off on finking me out for having a ``nonstandard ride
pattern?''

``I think I'd like to be arrested now,'' I said.

Booger sat back and raised his eyebrow at me.

``Really? On what charge?''

``Oh, you mean riding public transit in a nonstandard way isn't a
crime?''

Zit closed his eyes and scrubbed them with his thumbs.

Booger sighed a put-upon sigh. ``Look, Marcus, we're on your side
here. We use this system to catch bad guys. To catch terrorists and
drug dealers. Maybe you're a drug dealer yourself. Pretty good way to
get around the city, a Fast Pass. Anonymous.''

``What's wrong with anonymous? It was good enough for Thomas
Jefferson. And by the way, am I under arrest?''

``Let's take him home,'' Zit said. ``We can talk to his parents.''

``I think that's a great idea,'' I said. ``I'm sure my parents will be
anxious to hear how their tax dollars are being spent --''

I'd pushed it too far. Booger had been reaching for the door handle
but now he whirled on me, all Hulked out and throbbing veins. ``Why
don't you shut up right now, while it's still an option? After
everything that's happened in the past two weeks, it wouldn't kill you
to cooperate with us. You know what, maybe we \emph{should} arrest you. You
can spend a day or two in jail while your lawyer looks for you. A lot
can happen in that time. A \emph{lot}. How'd you like that?''

I didn't say anything. I'd been giddy and angry. Now I was scared
witless.

``I'm sorry,'' I managed, hating myself again for saying it.

Booger got in the front seat and Zit put the car in gear, cruising up
24th Street and over Potrero Hill. They had my address from my ID.

Mom answered the door after they rang the bell, leaving the chain
on. She peeked around it, saw me and said, ``Marcus? Who are these
men?''

``Police,'' Booger said. He showed her his badge, letting her get a good
look at it -- not whipping it away the way he had with me. ``Can we
come in?''

Mom closed the door and took the chain off and let them in. They
brought me in and Mom gave the three of us one of her looks.

``What's this about?''

Booger pointed at me. ``We wanted to ask your son some routine
questions about his movements, but he declined to answer them. We felt
it might be best to bring him here.''

``Is he under arrest?'' Mom's accent was coming on strong. Good old Mom.

``Are you a United States citizen, ma'am?'' Zit said.

She gave him a look that could have stripped paint. ``I shore am,
hyuck,'' she said, in a broad southern accent. ``Am \emph{I} under arrest?''

The two cops exchanged a look.

Zit took the fore. ``We seem to have gotten off to a bad start. We
identified your son as someone with a nonstandard public transit usage
pattern, as part of a new pro-active enforcement program. When we spot
people whose travels are unusual, or that match a suspicious profile,
we investigate further.''

``Wait,'' Mom said. ``How do you know how my son uses the Muni?''

``The Fast Pass,'' he said. ``It tracks voyages.''

``I see,'' Mom said, folding her arms. Folding her arms was a bad
sign. It was bad enough she hadn't offered them a cup of tea -- in
Mom-land, that was practically like making them shout through the
mail-slot -- but once she folded her arms, it was not going to end
well for them. At that moment, I wanted to go and buy her a big bunch
of flowers.

``Marcus here declined to tell us why his movements had been what they
were.''

``Are you saying you think my son is a terrorist because of how he
rides the bus?''

``Terrorists aren't the only bad guys we catch this way,'' Zit
said. ``Drug dealers. Gang kids. Even shoplifters smart enough to hit a
different neighborhood with every run.''

``You think my son is a drug dealer?''

``We're not saying that --'' Zit began. Mom clapped her hands at him to
shut him up.

``Marcus, please pass me your backpack.''

I did.

Mom unzipped it and looked through it, turning her back to us first.

``Officers, I can now affirm that there are no narcotics, explosives,
or shoplifted gewgaws in my son's bag. I think we're done here. I
would like your badge numbers before you go, please.''

Booger sneered at her. ``Lady, the ACLU is suing three hundred cops on
the SFPD, you're going to have to get in line.''

\fancybreak{\#}

Mom made me a cup of tea and then chewed me out for eating dinner when
I knew that she'd been making falafel. Dad came home while we were
still at the table and Mom and I took turns telling him the story. He
shook his head.

``Lillian, they were just doing their jobs.'' He was still wearing the
blue blazer and khakis he wore on the days that he was consulting in
Silicon Valley. ``The world isn't the same place it was last week.''

Mom set down her teacup. ``Drew, you're being ridiculous. Your son is
not a terrorist. His use of the public transit system is not cause for
a police investigation.''

Dad took off his blazer. ``We do this all the time at my work. It's how
computers can be used to find all kinds of errors, anomalies and
outcomes. You ask the computer to create a profile of an average
record in a database and then ask it to find out which records in the
database are furthest away from average. It's part of something called
Bayesian analysis and it's been around for centuries now. Without it,
we couldn't do spam-filtering --''

``So you're saying that you think the police should suck as hard as my
spam filter?'' I said.

Dad never got angry at me for arguing with him, but tonight I could
see the strain was running high in him. Still, I couldn't resist. My
own father, taking the police's side!

``I'm saying that it's perfectly reasonable for the police to conduct
their investigations by starting with data-mining, and then following
it up with leg-work where a human being actually intervenes to see why
the abnormality exists. I don't think that a computer should be
telling the police whom to arrest, just helping them sort through the
haystack to find a needle.''

``But by taking in all that data from the transit system, they're
\emph{creating the haystack},'' I said. ``That's a gigantic mountain of data
and there's almost nothing worth looking at there, from the police's
point of view. It's a total waste.''

``I understand that you don't like that this system caused you some
inconvenience, Marcus. But you of all people should appreciate the
gravity of the situation. There was no harm done, was there? They even
gave you a ride home.''

\emph{They threatened to send me to jail,} I thought, but I could see there
was no point in saying it.

``Besides, you still haven't told us where the blazing hells you've
been to create such an unusual traffic pattern.''

That brought me up short.

``I thought you relied on my judgment, that you didn't want to spy on
me.'' He'd said this often enough. ``Do you really want me to account
for every trip I've ever taken?''

\fancybreak{\#}

I hooked up my Xbox as soon as I got to my room. I'd bolted the
projector to the ceiling so that it could shine on the wall over my
bed (I'd had to take down my awesome mural of punk rock handbills I'd
taken down off telephone poles and glued to big sheets of white
paper).

I powered up the Xbox and watched as it came onto the screen. I was
going to email Van and Jolu to tell them about the hassles with the
cops, but as I put my fingers to the keyboard, I stopped again.

A feeling crept over me, one not unlike the feeling I'd had when I
realized that they'd turned poor old Salmagundi into a traitor. This
time, it was the feeling that my beloved Xnet might be broadcasting
the location of every one of its users to the DHS.

It was what Dad had said: \emph{You ask the computer to create a profile of
an average record in a database and then ask it to find out which
records in the database are furthest away from average.}

The Xnet was secure because its users weren't directly connected to
the Internet. They hopped from Xbox to Xbox until they found one that
was connected to the Internet, then they injected their material as
undecipherable, encrypted data. No one could tell which of the
Internet's packets were Xnet and which ones were just plain old
banking and e-commerce and other encrypted communication. You couldn't
find out who was tying the Xnet, let alone who was using the Xnet.

But what about Dad's ``Bayesian statistics?'' I'd played with Bayesian
math before. Darryl and I once tried to write our own better spam
filter and when you filter spam, you need Bayesian math. Thomas Bayes
was an 18th century British mathematician that no one care about until
a couple hundred years after he died, when computer scientists
realized that his technique for statistically analyzing mountains of
data would be super-useful for the modern world's info-Himalayas.

Here's some of how Bayesian stats work. Say you've got a bunch of
spam. You take every word that's in the spam and count how many times
it appears. This is called a ``word frequency histogram'' and it tells
you what the probability is that any bag of words is likely to be
spam. Now, take a ton of email that's not spam -- in the biz, they
call that ``ham'' -- and do the same.

Wait until a new email arrives and count the words that appear in
it. Then use the word-frequency histogram in the candidate message to
calculate the probability that it belongs in the ``spam'' pile or the
``ham'' pile. If it turns out to be spam, you adjust the ``spam''
histogram accordingly. There are lots of ways to refine the technique
-- looking at words in pairs, throwing away old data -- but this is
how it works at core. It's one of those great, simple ideas that seems
obvious after you hear about it.

It's got lots of applications -- you can ask a computer to count the
lines in a picture and see if it's more like a ``dog'' line-frequency
histogram or a ``cat'' line-frequency histogram. It can find porn, bank
fraud, and flamewars. Useful stuff.

And it was bad news for the Xnet. Say you had the whole Internet
wiretapped -- which, of course, the DHS has. You can't tell who's
passing Xnet packets by looking at the contents of those packets,
thanks to crypto.

What you \emph{can} do is find out who is sending way, way more encrypted
traffic out than everyone else. For a normal Internet surfer, a
session online is probably about 95 percent cleartext, five percent
ciphertext. If someone is sending out 95 percent ciphertext, maybe you
could dispatch the computer-savvy equivalents of Booger and Zit to ask
them if they're terrorist drug-dealer Xnet users.

This happens all the time in China. Some smart dissident will get the
idea of getting around the Great Firewall of China, which is used to
censor the whole country's Internet connection, by using an encrypted
connection to a computer in some other country. Now, the Party there
can't tell what the dissident is surfing: maybe it's porn, or
bomb-making instructions, or dirty letters from his girlfriend in the
Philippines, or political material, or good news about
Scientology. They don't have to know. All they have to know is that
this guy gets way more encrypted traffic than his neighbors. At that
point, they send him to a forced labor camp just to set an example so
that everyone can see what happens to smart-asses.

So far, I was willing to bet that the Xnet was under the DHS's radar,
but it wouldn't be the case forever. And after tonight, I wasn't sure
that I was in any better shape than a Chinese dissident. I was putting
all the people who signed onto the Xnet in jeopardy. The law didn't
care if you were actually doing anything bad; they were willing to put
you under the microscope just for being statistically abnormal. And I
couldn't even stop it -- now that the Xnet was running, it had a life
of its own.

I was going to have to fix it some other way.

I wished I could talk to Jolu about this. He worked at an Internet
Service Provider called Pigspleen Net that had hired him when he was
twelve, and he knew way more about the net than I did. If anyone knew
how to keep our butts out of jail, it would be him.

Luckily, Van and Jolu and I were planning to meet for coffee the next
night at our favorite place in the Mission after school. Officially,
it was our weekly Harajuku Fun Madness team meeting, but with the game
canceled and Darryl gone, it was pretty much just a weekly weep-fest,
supplemented by about six phone-calls and IMs a day that went, ``Are
you OK? Did it really happen?'' It would be good to have something else
to talk about.

\fancybreak{\#}

``You're out of your mind,'' Vanessa said. ``Are you actually, totally,
really, for-real crazy or what?''

She had shown up in her girl's school uniform because she'd been stuck
going the long way home, all the way down to the San Mateo bridge then
back up into the city, on a shuttle-bus service that her school was
operating. She hated being seen in public in her gear, which was
totally Sailor Moon -- a pleated skirt and a tunic and
knee-socks. She'd been in a bad mood ever since she turned up at the
cafe, which was full of older, cooler, mopey emo art students who
snickered into their lattes when she turned up.

``What do you want me to do, Van?'' I said. I was getting exasperated
myself. School was unbearable now that the game wasn't on, now that
Darryl was missing. All day long, in my classes, I consoled myself
with the thought of seeing my team, what was left of it. Now we were
fighting.

``I want you to stop putting yourself at risk, M1k3y.'' The hairs on the
back of my neck stood up. Sure, we always used our team handles at
team meetings, but now that my handle was also associated with my Xnet
use, it scared me to hear it said aloud in a public place.

``Don't use that name in public anymore,'' I snapped.

Van shook her head. ``That's just what I'm taking about. You could end
up going to jail for this, Marcus, and not just you. Lots of
people. After what happened to Darryl --''

``I'm doing this for Darryl!'' Art students swiveled to look at us and I
lowered my voice. ``I'm doing this because the alternative is to let
them get away with it all.''

``You think you're going to stop them? You're out of your mind. They're
the government.''

``It's still our country,'' I said. ``We still have the right to do
this.''

Van looked like she was going to cry. She took a couple of deep
breaths and stood up. ``I can't do it, I'm sorry. I can't watch you do
this. It's like watching a car-wreck in slow motion. You're going to
destroy yourself, and I love you too much to watch it happen.''

She bent down and gave me a fierce hug and a hard kiss on the cheek
that caught the edge of my mouth. ``Take care of yourself, Marcus,'' she
said. My mouth burned where her lips had pressed it. She gave Jolu the
same treatment, but square on the cheek. Then she left.

Jolu and I stared at each other after she'd gone.

I put my face in my hands. ``Dammit,'' I said, finally.

Jolu patted me on the back and ordered me another latte. ``It'll be
OK,'' he said.

``You'd think Van, of all people, would understand.'' Half of Van's
family lived in North Korea. Her parents never forgot that they had
all those people living a crazy dictator, not able to escape to
America, the way her parents had.

Jolu shrugged. ``Maybe that's why she's so freaked out. Because she
knows how dangerous it can get.''

I knew what he was talking about. Two of Van's uncles had gone to jail
and had never reappeared.

``Yeah,'' I said.

``So how come you weren't on Xnet last night?''

I was grateful for the distraction. I explained it all to him, the
Bayesian stuff and my fear that we couldn't go on using Xnet the way
we had been without getting nabbed. He listened thoughtfully.

``I see what you're saying. The problem is that if there's too much
crypto in someone's Internet connection, they'll stand out as
unusual. But if you don't encrypt, you'll make it easy for the bad
guys to wiretap you.''

``Yeah,'' I said. ``I've been trying to figure it out all day. Maybe we
could slow the connection down, spread it out over more peoples'
accounts --''

``Won't work,'' he said. ``To get it slow enough to vanish into the
noise, you'd have to basically shut down the network, which isn't an
option.''

``You're right,'' I said. ``But what else can we do?''

``What if we changed the definition of normal?''

And that was why Jolu got hired to work at Pigspleen when he was
12. Give him a problem with two bad solutions and he'd figure out a
third totally different solution based on throwing away all your
assumptions. I nodded vigorously. ``Go on, tell me.''

``What if the average San Francisco Internet user had a \emph{lot} more
crypto in his average day on the Internet? If we could change the
split so it's more like fifty-fifty cleartext to ciphertext, then the
users that supply the Xnet would just look like normal.''

``But how do we do that? People just don't care enough about their
privacy to surf the net through an encrypted link. They don't see why
it matters if eavesdroppers know what they're googling for.''

``Yeah, but web-pages are small amounts of traffic. If we got people to
routinely download a few giant encrypted files every day, that would
create as much ciphertext as thousands of web-pages.''

``You're talking about indienet,'' I said.

``You got it,'' he said.

indienet -- all lower case, always -- was the thing that made
Pigspleen Net into one of the most successful independent ISPs in the
world. Back when the major record labels started suing their fans for
downloading their music, a lot of the independent labels and their
artists were aghast. How can you make money by suing your customers?

Pigspleen's founder had the answer: she opened up a deal for any act
that wanted to work with their fans instead of fighting them. Give
Pigspleen a license to distribute your music to its customers and it
would give you a share of the subscription fees based on how popular
your music was. For an indie artist, the big problem isn't piracy,
it's obscurity: no one even cares enough about your tunes to steal
'em.

It worked. Hundreds of independent acts and labels signed up with
Pigspleen, and the more music there was, the more fans switched to
getting their Internet service from Pigspleen, and the more money
there was for the artists. Inside of a year, the ISP had a hundred
thousand new customers and now it had a million -- more than half the
broadband connections in the city.

``An overhaul of the indienet code has been on my plate for months
now,'' Jolu said. ``The original programs were written really fast and
dirty and they could be made a lot more efficient with a little
work. But I just haven't had the time. One of the high-marked to-do
items has been to encrypt the connections, just because Trudy likes it
that way.'' Trudy Doo was the founder of Pigspleen. She was an old time
San Francisco punk legend, the singer/front-woman of the
anarcho-feminist band Speedwhores, and she was crazy about privacy. I
could totally believe that she'd want her music service encrypted on
general principles.

``Will it be hard? I mean, how long would it take?''

``Well, there's tons of crypto code for free online, of course,'' Jolu
said. He was doing the thing he did when he was digging into a meaty
code problem -- getting that faraway look, drumming his palms on the
table, making the coffee slosh into the saucers. I wanted to laugh --
everything might be destroyed and crap and scary, but Jolu would write
that code.

``Can I help?''

He looked at me. ``What, you don't think I can manage it?''

``What?''

``I mean, you did this whole Xnet thing without even telling
me. Without talking to me. I kind of thought that you didn't need my
help with this stuff.''

I was brought up short. ``What?'' I said again. Jolu was looking really
steamed now. It was clear that this had been eating him for a long
time. ``Jolu --''

He looked at me and I could see that he was furious. How had I missed
this? God, I was such an idiot sometimes. ``Look dude, it's not a big
deal --'' by which he clearly meant that it was a really big deal ``--
it's just that you know, you never even \emph{asked}. I hate the
DHS. Darryl was my friend too. I could have really helped with it.''

I wanted to stick my head between my knees. ``Listen Jolu, that was
really stupid of me. I did it at like two in the morning. I was just
crazy when it was happening. I --'' I couldn't explain it. Yeah, he was
right, and that was the problem. It had been two in the morning but I
could have talked to Jolu about it the next day or the next. I hadn't
because I'd known what he'd say -- that it was an ugly hack, that I
needed to think it through better. Jolu was always figuring out how to
turn my 2 AM ideas into real code, but the stuff that he came out with
was always a little different from what I'd come up with. I'd wanted
the project for myself. I'd gotten totally into being M1k3y.

``I'm sorry,'' I said at last. ``I'm really, really sorry. You're totally
right. I just got freaked out and did something stupid. I really need
your help. I can't make this work without you.''

``You mean it?''

``Of course I mean it,'' I said. ``You're the best coder I know. You're a
goddamned genius, Jolu. I would be honored if you'd help me with
this.''

He drummed his fingers some more. ``It's just -- You know. You're the
leader. Van's the smart one. Darryl was\ldots He was your
second-in-command, the guy who had it all organized, who watched the
details. Being the programmer, that was \emph{my} thing. It felt like you
were saying you didn't need me.''

``Oh man, I am such an idiot. Jolu, you're the best-qual\-i\-fied person I
know to do this. I'm really, really, really --''

``All right, already. Stop. Fine. I believe you. We're all really
screwed up right now. So yeah, of course you can help. We can probably
even pay you -- I've got a little budget for contract programmers.''

``Really?'' No one had ever paid me for writing code.

``Sure. You're probably good enough to be worth it.'' He grinned and
slugged me in the shoulder. Jolu's really easy-going most of the time,
which is why he'd freaked me out so much.

I paid for the coffees and we went out. I called my parents and let
them know what I was doing. Jolu's mom insisted on making us
sandwiches. We locked ourselves in his room with his computer and the
code for indienet and we embarked on one of the great all-time
marathon programming sessions. Once Jolu's family went to bed around
11:30, we were able to kidnap the coffee-machine up to his room and go
IV with our magic coffee bean supply.

If you've never programmed a computer, you should. 
\discretionary{There}{is}{There's} nothing
like it in the whole world. When you program a computer, it does
\emph{exactly} what you tell it to do. It's like designing a machine -- any
machine, like a car, like a faucet, like a gas-hinge for a door --
using math and instructions. It's awesome in the truest sense: it can
fill you with awe.

A computer is the most complicated machine you'll ever use. It's made
of billions of micro-miniaturized transistors that can be configured
to run any program you can imagine. But when you sit down at the
keyboard and write a line of code, those transistors do what you tell
them to.

Most of us will never build a car. Pretty much none of us will ever
create an aviation system. Design a building. Lay out a city.

Those are complicated machines, those things, and they're off-limits
to the likes of you and me. But a computer is like, ten times more
complicated, and it will dance to any tune you play. You can learn to
write simple code in an afternoon. Start with a language like Python,
which was written to give non-programmers an easier way to make the
machine dance to their tune. Even if you only write code for one day,
one afternoon, you have to do it. Computers can control you or they
can lighten your work -- if you want to be in charge of your machines,
you have to learn to write code.

We wrote a lot of code that night.

\chapter{Chapter 8}

\epigraph{This chapter is dedicated to Borders, the global bookselling
  giant that you can find in cities all over the world -- I'll never
  forget walking into the gigantic Borders on Orchard Road in
  Singapore and discovering a shelf loaded with my novels! For many
  years, the Borders in Oxford Street in London hosted Pat Cadigan's
  monthly science fiction evenings, where local and visiting authors
  would read their work, speak about science fiction and meet their
  fans. When I'm in a strange city (which happens a lot) and I need a
  great book for my next flight, there always seems to be a Borders
  brimming with great choices -- I'm especially partial to the Borders
  on Union Square in San Francisco.}
{Borders worldwide\\ \url{http://www.bordersstores.com/locator/locator.jsp}}

I wasn't the only one who got screwed up by the histograms. There are
lots of people who have abnormal traffic patterns, abnormal usage
patterns. Abnormal is so common, it's practically normal.

The Xnet was full of these stories, and so were the newspapers and the
TV news. Husbands were caught cheating on their wives; wives were
caught cheating on their husbands, kids were caught sneaking out with
illicit girlfriends and boyfriends. A kid who hadn't told his parents
he had AIDS got caught going to the clinic for his drugs.

Those were the people with something to hide -- not guilty people, but
people with secrets. There were even more people with nothing to hide
at all, but who nevertheless resented being picked up, and
questioned. Imagine if someone locked you in the back of a police car
and demanded that you prove that you're \emph{not} a terrorist.

It wasn't just public transit. Most drivers in the Bay Area have a
FasTrak pass clipped to their sun-visors. This is a little radio-based
``wallet'' that pays your tolls for you when you cross the bridges,
saving you the hassle of sitting in a line for hours at the
toll-plazas. They'd tripled the cost of using cash to get across the
bridge (though they always fudged this, saying that FasTrak was
cheaper, not that anonymous cash was more expensive). Whatever
holdouts were left afterward disappeared after the number of
cash-lanes was reduced to just one per bridge-head, so that the cash
lines were even longer.

So if you're a local, or if you're driving a rental car from a local
agency, you've got a FasTrak. It turns out that toll-plazas aren't the
only place that your FasTrak gets read, though. The DHS had put
FasTrak readers all over town -- when you drove past them, they logged
the time and your ID number, building an ever-more perfect picture of
who went where, when, in a database that was augmented by ``speeding
cameras,'' ``red light cameras'' and all the other license-plate cameras
that had popped up like mushrooms.

No one had given it much thought. And now that people were paying
attention, we were all starting to notice little things, like the fact
that the FasTrak doesn't have an off-switch.

So if you drove a car, you were just as likely to be pulled over by an
SFPD cruiser that wanted to know why you were taking so many trips to
the Home Depot lately, and what was that midnight drive up to Sonoma
last week about?

The little demonstrations around town on the weekend were
growing. Fifty thousand people marched down Market Street after a week
of this monitoring. I couldn't care less. The people who'd occupied my
city didn't care what the natives wanted. They were a conquering
army. They knew how we felt about that.

One morning I came down to breakfast just in time to hear Dad tell Mom
that the two biggest taxi companies were going to give a ``discount'' to
people who used special cards to pay their fares, supposedly to make
drivers safer by reducing the amount of cash they carried. I wondered
what would happen to the information about who took which cabs where.

I realized how close I'd come. The new indienet client had been pushed
out as an automatic update just as this stuff started to get bad, and
Jolu told me that 80 percent of the traffic he saw at Pigspleen was
now encrypted. The Xnet just might have been saved.

Dad was driving me nuts, though.

``You're being paranoid, Marcus,'' he told me over breakfast one day as
I told him about the guys I'd seen the cops shaking down on BART the
day before.

``Dad, it's ridiculous. They're not catching any terrorists, are they?
It's just making people scared.''

``They may not have caught any terrorists yet, but they're sure getting
a lot of scumbags off the streets. Look at the drug dealers -- it says
they've put dozens of them away since this all started. Remember when
those druggies robbed you? If we don't bust their dealers, it'll only
get worse.'' I'd been mugged the year before. They'd been pretty
civilized about it. One skinny guy who smelled bad told me he had a
gun, the other one asked me for my wallet. They even let me keep my
ID, though they got my debit card and Fast Pass. It had still scared
me witless and left me paranoid and checking my shoulder for weeks.

``But most of the people they hold up aren't doing anything wrong,
Dad,'' I said. This was getting to me. My own father! ``It's crazy. For
every guilty person they catch, they have to punish thousands of
innocent people. That's just not good.''

``Innocent? Guys cheating on their wives? Drug dealers? You're
defending them, but what about all the people who died? If you don't
have anything to hide --''

``So you wouldn't mind if they pulled \emph{you} over?'' My dad's histograms
had proven to be depressingly normal so far.

``I'd consider it my duty,'' he said. ``I'd be proud. It would make me
feel safer.''

Easy for him to say.

\fancybreak{\#}

Vanessa didn't like me talking about this stuff, but she was too smart
about it for me to stay away from the subject for long. We'd get
together all the time, and talk about the weather and school and
stuff, and then, somehow, I'd be back on this subject. Vanessa was
cool when it happened -- she didn't Hulk out on me again -- but I
could see it upset her.

Still.

``So my dad says, 'I'd consider it my duty.' Can you freaking \emph{believe}
it? I mean, God! I almost told him then about going to jail, asking
him if he thought that was our 'duty'!''

We were sitting in the grass in Dolores Park after school, watching
the dogs chase frisbees.

Van had stopped at home and changed into an old t-shirt for one of her
favorite Brazilian tecno-brega bands, Carioca Proi\-bi\-dão -- the
forbidden guy from Rio. She'd gotten the shirt at a live show we'd all
gone to two years before, sneaking out for a grand adventure down at
the Cow Palace, and she'd sprouted an inch or two since, so it was
tight and rode up her tummy, showing her flat little belly button.

She lay back in the weak sun with her eyes closed behind her shades,
her toes wiggling in her flip-flops. I'd known Van since forever, and
when I thought of her, I usually saw the little kid I'd known with
hundreds of jangly bracelets made out of sliced-up soda cans, who
played the piano and couldn't dance to save her life. Sitting out
there in Dolores Park, I suddenly saw her as she was.

She was totally h4wt -- that is to say, hot. It was like looking at
that picture of a vase and noticing that it was also two faces. I
could see that Van was just Van, but I could also see that she was
hella pretty, something I'd never noticed.

Of course, Darryl had known it all along, and don't think that I
wasn't bummed out anew when I realized this.

``You can't tell your dad, you know,'' she said. ``You'd put us all at
risk.'' Her eyes were closed and her chest was rising up and down with
her breath, which was distracting in a really embarrassing way.

``Yeah,'' I said, glumly. ``But the problem is that I know he's just
totally full of it. If you pulled my dad over and made him prove he
wasn't a child-molesting, drug-dealing terrorist, he'd go
berserk. Totally off-the-rails. He hates being put on hold when he
calls about his credit-card bill. Being locked in the back of a car
and questioned for an hour would give him an aneurism.''

``They only get away with it because the normals feel smug compared to
the abnormals. If everyone was getting pulled over, it'd be a
disaster. No one would ever get anywhere, they'd all be waiting to get
questioned by the cops. Total gridlock.''

Woah.

``Van, you are a total genius,'' I said.

``Tell me about it,'' she said. She had a lazy smile and she looked at
me through half-lidded eyes, almost romantic.

``Seriously. We can do this. We can mess up the profiles
easily. Getting people pulled over is easy.''

She sat up and pushed her hair off her face and looked at me. I felt a
little flip in my stomach, thinking that she was really impressed with
me.

``It's the arphid cloners,'' I said. ``They're totally easy to make. Just
flash the firmware on a ten-dollar Radio Shack reader/\discretionary{}{}{}writer and
you're done. What we do is go around and randomly swap the tags on
people, overwriting their Fast Passes and FasTraks with other people's
codes. That'll make \emph{everyone} skew all weird and screwy, and make
everyone look guilty. Then: total gridlock.''

Van pursed her lips and lowered her shades and I realized she was so
angry she couldn't speak.

``Good bye, Marcus,'' she said, and got to her feet. Before I knew it,
she was walking away so fast she was practically running.

``Van!'' I called, getting to my feet and chasing after her. ``Van!
Wait!''

She picked up speed, making me run to catch up with her.

``Van, what the hell,'' I said, catching her arm. She jerked it away so
hard I punched myself in the face.

``You're psycho, Marcus. You're going to put all your little Xnet
buddies in danger for their lives, and on top of it, you're going to
turn the whole city into terrorism suspects. Can't you stop before you
hurt these people?''

I opened and closed my mouth a couple times. ``Van, \emph{I'm} not the
problem, \emph{they} are. I'm not arresting people, jailing them, making
them disappear. The Department of Homeland Security are the ones doing
that. I'm fighting back to make them stop.''

``How, by making it worse?''

``Maybe it has to get worse to get better, Van. Isn't that what you
were saying? If everyone was getting pulled over --''

``That's not what I meant. I didn't mean you should get everyone
arrested. If you want to protest, join the protest movement. Do
something positive. Didn't you learn \emph{anything} from Darryl?
\emph{Anything?}''

``You're damned right I did,'' I said, losing my cool. ``I learned that
they can't be trusted. That if you're not fighting them, you're
helping them. That they'll turn the country into a prison if we let
them. What did you learn, Van? To be scared all the time, to sit tight
and keep your head down and hope you don't get noticed? You think it's
going to get better? If we don't do anything, this is as \emph{good as it's
going to get}. It will only get worse and worse from now on. You want
to help Darryl? Help me bring them down!''

There it was again. My vow. Not to get Darryl free, but to bring down
the entire DHS. That was crazy, even I knew it. But it was what I
planned to do. No question about it.

Van shoved me hard with both hands. She was strong from school
athletics -- fencing, lacrosse, field hockey, all the girls-school
sports -- and I ended up on my ass on the disgusting San Francisco
sidewalk. She took off and I didn't follow.

\fancybreak{\#}

\edialog{The important thing about security systems isn't how they
  work, it's how they fail.}

That was the first line of my first blog post on Open Revolt, my Xnet
site. I was writing as M1k3y, and I was ready to go to war.

\edialog{Maybe all the automatic screening is supposed to catch
  terrorists. Maybe it will catch a terrorist sooner or later. The
  problem is that it catches \emph{us} too, even though we're not doing
  anything wrong.}

\edialog{The more people it catches, the more brittle it gets. If it
  catches too many people, it dies.}

\edialog{Get the idea?}

I pasted in my HOWTO for building a arphid cloner, and some tips for
getting close enough to people to read and write their tags. I put my
own cloner in the pocket of my vintage black leather motocross jacket
with the armored pockets and left for school. I managed to clone six
tags between home and Chavez High.

It was war they wanted. It was war they'd get.

\fancybreak{\#}

If you ever decide to do something as stupid as build an automatic
terrorism detector, here's a math lesson you need to learn first. It's
called ``the paradox of the false positive,'' and it's a doozy.

Say you have a new disease, called Super-AIDS. Only one in a million
people gets Super-AIDS. You develop a test for Super-AIDS that's 99
percent accurate. I mean, 99 percent of the time, it gives the correct
result -- true if the subject is infected, and false if the subject is
healthy. You give the test to a million people.

One in a million people have Super-AIDS. One in a hundred people that
you test will generate a ``false positive'' -- the test will say he has
Super-AIDS even though he doesn't. That's what ``99 percent accurate''
means: one percent wrong.

What's one percent of one million?

1,000,000/100 = 10,000

One in a million people has Super-AIDS. If you test a million random
people, you'll probably only find one case of real Super-AIDS. But
your test won't identify \emph{one} person as having Super-AIDS. It will
identify \emph{10,000} people as having it.

Your 99 percent accurate test will perform with 99.99 percent
\emph{inaccuracy}.

That's the paradox of the false positive. When you try to find
something really rare, your test's accuracy has to match the rarity of
the thing you're looking for. If you're trying to point at a single
pixel on your screen, a sharp pencil is a good pointer: the pencil-tip
is a lot smaller (more accurate) than the pixels. But a pencil-tip is
no good at pointing at a single \emph{atom} in your screen. For that, you
need a pointer -- a test -- that's one atom wide or less at the tip.

This is the paradox of the false positive, and here's how it applies
to terrorism:

Terrorists are really rare. In a city of twenty million like New York,
there might be one or two terrorists. Maybe ten of them at the
outside. 10/20,000,000 = 0.00005 percent. One twenty-thousandth of a
percent.

That's pretty rare all right. Now, say you've got some software that
can sift through all the bank-records, or toll-pass records, or public
transit records, or phone-call records in the city and catch
terrorists 99 percent of the time.

In a pool of twenty million people, a 99 percent accurate test will
identify two hundred thousand people as being terrorists. But only ten
of them are terrorists. To catch ten bad guys, you have to haul in and
investigate two hundred thousand innocent people.

Guess what? Terrorism tests aren't anywhere \emph{close} to 99 percent
accurate. More like 60 percent accurate. Even 40 percent accurate,
sometimes.

What this all meant was that the Department of Homeland Security had
set itself up to fail badly. They were trying to spot incredibly rare
events -- a person is a terrorist -- with inaccurate systems.

Is it any wonder we were able to make such a mess?

\fancybreak{\#}

I stepped out the front door whistling on a Tuesday morning one week
into the Operation False Positive. I was rockin' out to some new music
I'd downloaded from the Xnet the night before -- lots of people sent
M1k3y little digital gifts to say thank you for giving them hope.

I turned onto 23d Street and carefully took the narrow stone steps cut
into the side of the hill. As I descended, I passed Mr Wiener Dog. I
don't know Mr Wiener Dog's real name, but I see him nearly every day,
walking his three panting wiener dogs up the staircase to the little
parkette. Squeezing past them all on the stairs is pretty much
impossible and I always end up tangled in a leash, knocked into
someone's front garden, or perched on the bumper of one of the cars
parked next to the curb.

Mr Wiener Dog is clearly Someone Important, because he has a fancy
watch and always wears a nice suit. I had mentally assumed that he
worked down in the financial district.

Today as I brushed up against him, I triggered my arphid cloner, which
was already loaded in the pocket of my leather jacket. The cloner
sucked down the numbers off his credit-cards and his car-keys, his
passport and the hundred-dollar bills in his wallet.

Even as it was doing that, it was flashing some of them with new
numbers, taken from other people I'd brushed against. It was like
switching the license-plates on a bunch of cars, but invisible and
instantaneous. I smiled apologetically at Mr Wiener Dog and continued
down the stairs. I stopped at three of the cars long enough to swap
their FasTrak tags with numbers taken offall over cars I'd gone past
the day before.

You might think I was being a little aggro here, but I was cautious
and conservative compared to a lot of the Xnetters. A couple girls in
the Chemical Engineering program at UC Berkeley had figured out how to
make a harmless substance out of kitchen products that would trip an
explosive sniffer. They'd had a merry time sprinkling it on their
profs' briefcases and jackets, then hiding out and watching the same
profs try to get into the auditoriums and libraries on campus, only to
get flying-tackled by the new security squads that had sprung up
everywhere.

Other people wanted to figure out how to dust envelopes with
substances that would test positive for anthrax, but everyone else
thought they were out of their minds. Luckily, it didn't seem like
they'd be able to figure it out.

I passed by San Francisco General Hospital and nodded with
satisfaction as I saw the huge lines at the front doors. They had a
police checkpoint too, of course, and there were enough Xnetters
working as interns and cafeteria workers and whatnot there that
everyone's badges had been snarled up and swapped around. I'd read the
security checks had tacked an hour onto everyone's work day, and the
unions were threatening to walk out unless the hospital did something
about it.

A few blocks later, I saw an even longer line for the BART. Cops were
walking up and down the line pointing people out and calling them
aside for questioning, bag-searches and pat-downs. They kept getting
sued for doing this, but it didn't seem to be slowing them down.

I got to school a little ahead of time and decided to walk down to
22nd Street to get a coffee -- and I passed a police checkpoint where
they were pulling over cars for secondary inspection.

School was no less wild -- the security guards on the metal detectors
were also wanding our school IDs and pulling out students with odd
movements for questioning. Needless to say, we all had pretty weird
movements. Needless to say, classes were starting an hour or more
later.

Classes were crazy. I don't think anyone was able to concentrate. I
overheard two teachers talking about how long it had taken them to get
home from work the day before, and planning to sneak out early that
day.

It was all I could do to keep from laughing. The paradox of the false
positive strikes again!

Sure enough, they let us out of class early and I headed home the long
way, circling through the Mission to see the havoc. Long lines of
cars. BART stations lined up around the blocks. People swearing at
ATMs that wouldn't dispense their money because they'd had their
accounts frozen for suspicious activity (that's the danger of wiring
your checking account straight into your FasTrak and Fast Pass!).

I got home and made myself a sandwich and logged into the Xnet. It had
been a good day. People from all over town were crowing about their
successes. We'd brought the city of San Francisco to a standstill. The
news-reports confirmed it -- they were calling it the DHS gone
haywire, blaming it all on the fake-ass ``security'' that was supposed
to be protecting us from terrorism. The Business section of the San
Francisco Chronicle gave its whole front page to an estimate of the
economic cost of the DHS security resulting from missed work hours,
meetings and so on. According to the Chronicle's economist, a week of
this crap would cost the city more than the Bay Bridge bombing had.

Mwa-ha-ha-ha.

The best part: Dad got home that night late. Very late. Three \emph{hours}
late. Why? Because he'd been pulled over, searched, questioned. Then
it happened \emph{again}. Twice.

Twice!

\chapter{Chapter 9}

\epigraph{This chapter is dedicated to Compass Books/Books Inc, the
  oldest independent bookstore in the western USA. They've got stores
  up and down California, in San Francisco, Burlingame, Mountain View
  and Palo Alto, but coolest of all is that they run a killer
  bookstore in the middle of Disneyland's Downtown Disney in
  Anaheim. I'm a stone Disney park freak (see my first novel, Down and
  Out in the Magic Kingdom if you don't believe it), and every time
  I've lived in California, I've bought myself an annual Disneyland
  pass, and on practically every visit, I drop by Compass Books in
  Downtown Disney. They stock a brilliant selection of unauthorized
  (and even critical) books about Disney, as well as a great variety
  of kids books and science fiction, and the cafe next door makes a
  mean cappuccino.}
{Compass Books/Books Inc:
\url{http://www.booksinc.net/NASApp/store/Product;jsessionid=abcF-ch09-pbU6m7ZRrLr?s=showproduct&isbn=0765319853}}

He was so angry I thought he was going to pop. You know I said I'd
only seen him lose his cool rarely? That night, he lost it more than
he ever had.

``You wouldn't believe it. This cop, he was like eighteen years old and
he kept saying, 'But sir, why were you in Berkeley yesterday if your
client is in Mountain View?' I kept explaining to him that I teach at
Berkeley and then he'd say, 'I thought you were a consultant,' and
we'd start over again. It was like some kind of sitcom where the cops
have been taken over by the stupidity ray.

``What's worse was he kept insisting that I'd been in Berkeley today as
well, and I kept saying no, I hadn't been, and he said I had
been. Then he showed me my FasTrak billing and it said I'd driven the
San Mateo bridge three times that day!

``That's not all,'' he said, and drew in a breath that let me know he
was really steamed. ``They had information about where I'd been, places
that \emph{didn't have a toll plaza}. They'd been polling my pass just on
the street, at random. And it was \emph{wrong}! Holy crap, I mean, they're
spying on us all and they're not even competent!''

I'd drifted down into the kitchen as he railed there, and now I was
watching him from the doorway. Mom met my eye and we both raised our
eyebrows as if to say, \emph{Who's going to say 'I told you so' to him?} I
nodded at her. She could use her spousular powers to nullify his rage
in a way that was out of my reach as a mere filial unit.

``Drew,'' she said, and grabbed him by the arm to make him stop stalking
back and forth in the kitchen, waving his arms like a street-preacher.

``What?'' he snapped.

``I think you owe Marcus an apology.'' She kept her voice even and
level. Dad and I are the spazzes in the household -- Mom's a total
rock.

Dad looked at me. His eyes narrowed as he thought for a minute. ``All
right,'' he said at last. ``You're right. I was talking about competent
surveillance. These guys were total amateurs. I'm sorry, son,'' he
said. ``You were right. That was ridiculous.'' He stuck his hand out and
shook my hand, then gave me a firm, unexpected hug.

``God, what are we doing to this country, Marcus? Your generation
deserves to inherit something better than this.'' When he let me go, I
could see the deep wrinkles in his face, lines I'd never noticed.

I went back up to my room and played some Xnet games. There was a good
multiplayer thing, a clockwork pirate game where you had to quest
every day or two to wind up your whole crew's mainsprings before you
could go plundering and pillaging again. It was the kind of game I
hated but couldn't stop playing: lots of repetitive quests that
weren't all that satisfying to complete, a little bit of
player-versus-player combat (scrapping to see who would captain the
ship) and not that many cool puzzles that you had to figure
out. Mostly, playing this kind of game made me homesick for Harajuku
Fun Madness, which balanced out running around in the real world,
figuring out online puzzles, and strategizing with your team.

But today it was just what I needed. Mindless entertainment.

My poor dad.

I'd done that to him. He'd been happy before, confident that his tax
dollars were being spent to keep him safe. I'd destroyed that
confidence. It was false confidence, of course, but it had kept him
going. Seeing him now, miserable and broken, I wondered if it was
better to be clear-eyed and hopeless or to live in a fool's
paradise. That shame -- the shame I'd felt since I gave up my
passwords, since they'd broken me -- returned, leaving me listless and
wanting to just get away from myself.

My character was a swabbie on the pirate ship \emph{Zombie Charger}, and
he'd wound down while I'd been offline. I had to IM all the other
players on my ship until I found one willing to wind me up. That kept
me occupied. I liked it, actually. There was something magic about a
total stranger doing you a favor. And since it was the Xnet, I knew
that all the strangers were friends, in some sense.

\edialog{Where u located?}

The character who wound me up was called Lizanator, and it was female,
though that didn't mean that it was a girl. Guys had some weird
affinity for playing female characters.

\edialog{San Francisco}

I said.

\edialog{No stupe, where you located in San Fran?}

\edialog{Why, you a pervert?}

That usually shut down that line of conversation. Of course every
gamespace was full of pedos and pervs, and cops pretending to be pedo-
and perv-bait (though I sure hoped there weren't any cops on the
Xnet!). An accusation like that was enough to change the subject nine
out of ten times.

\edialog{Mission? Potrero Hill? Noe? East Bay?}

\edialog{Just wind me up k thx?}

She stopped winding.

\edialog{You scared?}

\edialog{Safe -- why do you care?}

\edialog{Just curious}

I was getting a bad vibe off her. She was clearly more than just
curious. Call it paranoia. I logged off and shut down my Xbox.

\fancybreak{\#}

Dad looked at me over the table the next morning and said, ``It looks
like it's going to get better, at least.'' He handed me a copy of the
\emph{Chronicle} open to the third page.

\begin{newsquote}
  A Department of Homeland Security spokesman has confirmed that the
  San Francisco office has requested a 300 percent budget and
  personnel increase from DC
\end{newsquote}


What?

\begin{newsquote}
  Major General Graeme Sutherland, the commanding officer for
  Northern California DHS operations, confirmed the request at a press
  conference yesterday, noting that a spike in suspicious activity in
  the Bay Area prompted the request. ``We are tracking a spike in
  underground chatter and activity and believe that saboteurs are
  deliberately manufacturing false security alerts to undermine our
  efforts.''
\end{newsquote}

My eyes crossed. No freaking way.

\begin{newsquote}
  ``These false alarms are potentially 'radar chaff' intended
    to disguise real attacks. The only effective way of combatting
    them is to step up staffing and analyst levels so that we can
    fully investigate every lead.''
\end{newsquote}

\begin{newsquote}
  Sutherland noted the delays experienced all over the city were
  ``unfortunate'' and committed to eliminating them.
\end{newsquote}

I had a vision of the city with four or five times as many DHS
enforcers, brought in to make up for my own stupid ideas. Van was
right. The more I fought them, the worse it was going to get.

Dad pointed at the paper. ``These guys may be fools, but they're
methodical fools. They'll just keep throwing resources at this problem
until they solve it. It's tractable, you know. Mining all the data in
the city, following up on every lead. They'll catch the terrorists.''

I lost it. ``Dad! Are you \emph{listening to yourself}? They're talking
about investigating practically every person in the city of San
Francisco!''

``Yeah,'' he said, ``that's right. They'll catch every alimony cheat,
every dope dealer, every dirt-bag and every terrorist. You just
wait. This could be the best thing that ever happened to this
country.''

``Tell me you're joking,'' I said. ``I beg you. You think that that's
what they intended when they wrote the Constitution? What about the
Bill of Rights?''

``The Bill of Rights was written before data-mining,'' he said. He was
awesomely serene, convinced of his rightness. ``The right to freedom of
association is fine, but why shouldn't the cops be allowed to mine
your social network to figure out if you're hanging out with
gangbangers and terrorists?''

``Because it's an invasion of my privacy!'' I said.

``What's the big deal? Would you rather have privacy or terrorists?''

Agh. I hated arguing with my dad like this. I needed a coffee. ``Dad,
come on. Taking away our privacy isn't catching terrorists: it's just
inconveniencing normal people.''

``How do you know it's not catching terrorists?''

``Where are the terrorists they've caught?''

``I'm sure we'll see arrests in good time. You just wait.''

``Dad, what the hell has happened to you since last night? You were
ready to go nuclear on the cops for pulling you over --''

``Don't use that tone with me, Marcus. What's happened since last night
is that I've had the chance to think it over and to read \emph{this}.'' He
rattled his paper. ``The reason they caught me is that the bad guys are
actively jamming them. They need to adjust their techniques to
overcome the jamming. But they'll get there. Meanwhile the occasional
road stop is a small price to pay. This isn't the time to be playing
lawyer about the Bill of Rights. This is the time to make some
sacrifices to keep our city safe.''

I couldn't finish my toast. I put the plate in the dishwasher and left
for school. I had to get out of there.

\fancybreak{\#}

The Xnetters weren't happy about the stepped up police surveillance,
but they weren't going to take it lying down. Someone called a
phone-in show on KQED and told them that the police were wasting their
time, that we could monkeywrench the system faster than they could
untangle it. The recording was a top Xnet download that night.

``This is California Live and we're talking to an anonymous caller at a
payphone in San Francisco. He has his own information about the
slowdowns we've been facing around town this week. Caller, you're on
the air.''

``Yeah, yo, this is just the beginning, you know? I mean, like, we're
just getting started. Let them hire a billion pigs and put a
checkpoint on every corner. We'll jam them all! And like, all this
crap about terrorists? We're not terrorists! Give me a break, I mean,
really! We're jamming up the system because we hate the Homeland
Security, and because we love our city. Terrorists? I can't even spell
jihad. Peace out.''

He sounded like an idiot. Not just the incoherent words, but also his
gloating tone. He sounded like a kid who was indecently proud of
himself. He \emph{was} a kid who was indecently proud of himself.

The Xnet flamed out over this. Lots of people thought he was an idiot
for calling in, while others thought he was a hero. I worried that
there was probably a camera aimed at the payphone he'd used. Or an
arphid reader that might have sniffed his Fast Pass. I hoped he'd had
the smarts to wipe his fingerprints off the quarter, keep his hood up,
and leave all his arphids at home. But I doubted it. I wondered if
he'd get a knock on the door sometime soon.

The way I knew when something big had happened on Xnet was that I'd
suddenly get a million emails from people who wanted M1k3y to know
about the latest haps. It was just as I was reading about Mr
Can't-Spell-Jihad that my mailbox went crazy. Everyone had a message
for me -- a link to a livejournal on the Xnet -- one of the many
anonymous blogs that were based on the Freenet document publishing
system that was also used by Chinese democracy advocates.

\edialog{Close call}

\edialog{We were jamming at the Embarcadero tonite and goofing around
  giving everyone a new car key or door key or Fast Pass or FasTrak,
  tossing around a little fake gunpowder. There were cops everywhere
  but we were smarter then them; we're there pretty much every night
  and we never get caught.}

\edialog{So we got caught tonight. It was a stupid mistake we got
  sloppy we got busted. It was an undercover who caught my pal and
  then got the rest of us. They'd been watching the crowd for a long
  time and they had one of those trucks nearby and they took four of
  us in but missed the rest.}

\edialog{The truck was JAMMED like a can of sardines with every kind
  of person, old young black white rich poor all suspects, and there
  were two cops trying to ask us questions and the undercovers kept
  bringing in more of us. Most people were trying to get to the front
  of the line to get through questioning so we kept on moving back and
  it was like hours in there and really hot and it was getting more
  crowded not less.}

\edialog{At like 8PM they changed shifts and two new cops came in and
  bawled out the two cops who were there all like wtf? aren't you
  doing anything here. They had a real fight and then the two old cops
  left and the new cops sat down at their desks and whispered to each
  other for a while.}

\edialog{Then one cop stood up and started shouting EVERYONE JUST GO
  HOME JESUS CHRIST WE'VE GOT BETTER THINGS TO DO THAN BOTHER YOU WITH
  MORE QUESTIONS IF YOU'VE DONE SOMETHING WRONG JUST DON'T DO IT AGAIN
  AND LET THIS BE A WARNING TO YOU ALL.}

\edialog{A bunch of the suits got really pissed which was HILARIOUS
  because I mean ten minutes before they were buggin about being held
  there and now they were wicked pissed about being let go, like make
  up your minds!}

\edialog{We split fast though and got out and came home to write
  this. There are undercovers everywhere, believe. If you're jamming,
  be open-eyed and get ready to run when problems happen. If you get
  caught try to wait it out they're so busy they'll maybe just let you
  go.}

\edialog{We made them that busy! All those people in that truck were
  there because we'd jammed them. So jam on!}

I felt like I was going to throw up. Those four people -- kids I'd
never met -- they nearly went away forever because of something I'd
started.

Because of something I'd told them to do. I was no better than a
terrorist.

\fancybreak{\#}

The DHS got their budget requisition approved. The President went on
TV with the Governor to tell us that no price was too high for
security. We had to watch it the next day in school at assembly. My
Dad cheered. He'd hated the President since the day he was elected,
saying he wasn't any better than the last guy and the last guy had
been a complete disaster, but now all he could do was talk about how
decisive and dynamic the new guy was.

``You have to take it easy on your father,'' Mom said to me one night
after I got home from school. She'd been working from home as much as
possible. Mom's a freelance relocation specialist who helps British
people get settled in in San Francisco. The UK High Commission pays
her to answer emails from mystified British people across the country
who are totally confused by how freaky we Americans are. She explains
Americans for a living, and she said that these days it was better to
do that from home, where she didn't have to actually see any Americans
or talk to them.

I don't have any illusions about Britain. America may be willing to
trash its Constitution every time some Jihadist looks cross-eyed at
us, but as I learned in my ninth-grade Social Studies independent
project, the Brits don't even \emph{have} a Constitution. They've got laws
there that would curl the hair on your toes: they can put you in jail
for an entire year if they're really sure that you're a terrorist but
don't have enough evidence to prove it. Now, how sure can they be if
they don't have enough evidence to prove it? How'd they get that sure?
Did they see you committing terrorist acts in a really vivid dream?

And the surveillance in Britain makes America look like amateur
hour. The average Londoner is photographed 500 times a day, just
walking around the streets. Every license plate is photographed at
every corner in the country. Everyone from the banks to the public
transit company is enthusiastic about tracking you and snitching on
you if they think you're remotely suspicious.

But Mom didn't see it that way. She'd left Britain halfway through
high school and she'd never felt at home here, no matter that she'd
married a boy from Petaluma and raised a son here. To her, this was
always the land of barbarians, and Britain would always be home.

``Mom, he's just wrong. You of all people should know that. Everything
that makes this country great is being flushed down the toilet and
he's going along with it. Have you noticed that they haven't \emph{caught
any terrorists}? Dad's all like, 'We need to be safe,' but he needs to
know that most of us don't feel safe. We feel endangered all the
time.''

``I know this all, Marcus. Believe me, I'm not fan of what's been
happening to this country. But your father is --'' She broke off. ``When
you didn't come home after the attacks, he thought --''

She got up and made herself a cup of tea, something she did whenever
she was uncomfortable or disconcerted.

``Marcus,'' she said. ``Marcus, we thought you were dead. Do you
understand that? We were mourning you for days. We were imagining you
blown to bits, at the bottom of the ocean. Dead because some bastard
decided to kill hundreds of strangers to make some point.''

That sank in slowly. I mean, I understood that they'd been
worried. Lots of people died in the bombings -- four thousand was the
present estimate -- and practically everyone knew someone who didn't
come home that day. There were two people from my school who had
disappeared.

``Your father was ready to kill someone. Anyone. He was out of his
mind. You've never seen him like this. I've never seen him like it
either. He was out of his mind. He'd just sit at this table and curse
and curse and curse. Vile words, words I'd never heard him say. One
day -- the third day -- someone called and he was sure it was you, but
it was a wrong number and he threw the phone so hard it disintegrated
into thousands of pieces.'' I'd wondered about the new kitchen phone.

``Something broke in your father. He loves you. We both love you. You
are the most important thing in our lives. I don't think you realize
that. Do you remember when you were ten, when I went home to London
for all that time? Do you remember?''

I nodded silently.

``We were ready to get a divorce, Marcus. Oh, it doesn't matter why
anymore. It was just a bad patch, the kind of thing that happens when
people who love each other stop paying attention for a few years. He
came and got me and convinced me to come back for you. We couldn't
bear the thought of doing that to you. We fell in love again for
you. We're together today because of you.''

I had a lump in my throat. I'd never known this. No one had ever told
me.

``So your father is having a hard time right now. He's not in his right
mind. It's going to take some time before he comes back to us, before
he's the man I love again. We need to understand him until then.''

She gave me a long hug, and I noticed how thin her arms had gotten,
how saggy the skin on her neck was. I always thought of my mother as
young, pale, rosy-cheeked and cheerful, peering shrewdly through her
metal-rim glasses. Now she looked a little like an old woman. I had
done that to her. The terrorists had done that to her. The Department
of Homeland Security had done that to her. In a weird way, we were all
on the same side, and Mom and Dad and all those people we'd spoofed
were on the other side.

\fancybreak{\#}

I couldn't sleep that night. Mom's words kept running through my
head. Dad had been tense and quiet at dinner and we'd barely spoken,
because I didn't trust myself not to say the wrong thing and because
he was all wound up over the latest news, that Al Qaeda was definitely
responsible for the bombing. Six different terrorist groups had
claimed responsibility for the attack, but only Al Qaeda's Internet
video disclosed information that the DHS said they hadn't disclosed to
anyone.

I lay in bed and listened to a late-night call-in radio show. The
topic was sex problems, with this gay guy who I normally loved to
listen to, he would give people such raw advice, but good advice, and
he was really funny and campy.

Tonight I couldn't laugh. Most of the callers wanted to ask what to do
about the fact that they were having a hard time getting busy with
their partners ever since the attack. Even on sex-talk radio, I
couldn't get away from the topic.

I switched the radio off and heard a purring engine on the street
below.

My bedroom is in the top floor of our house, one of the painted
ladies. I have a sloping attic ceiling and windows on both sides --
one overlooks the whole Mission, the other looks out into the street
in front of our place. There were often cars cruising at all hours of
the night, but there was something different about this engine noise.

I went to the street-window and pulled up my blinds. Down on the
street below me was a white, unmarked van whose roof was festooned
with radio antennas, more antennas than I'd ever seen on a car. It was
cruising very slowly down the street, a little dish on top spinning
around and around.

As I watched, the van stopped and one of the back doors popped open. A
guy in a DHS uniform -- I could spot one from a hundred yards now --
stepped out into the street. He had some kind of handheld device, and
its blue glow lit his face. He paced back and forth, first scouting my
neighbors, making notes on his device, then heading for me. There was
something familiar in the way he walked, looking down --

He was using a wifinder! The DHS was scouting for Xnet nodes. I let go
of the blinds and dove across my room for my Xbox. I'd left it up
while I downloaded some cool animations one of the Xnetters had made
of the President's no-price-too-high speech. I yanked the plug out of
the wall, then scurried back to the window and cracked the blind a
fraction of an inch.

The guy was looking down into his wifinder again, walking back and
forth in front of our house. A moment later, he got back into his van
and drove away.

I got out my camera and took as many pictures as I could of the van
and its antennas. Then I opened them in a free image-editor called The
GIMP and edited out everything from the photo except the van, erasing
my street and anything that might identify me.

I posted them to Xnet and wrote down everything I could about the
vans. These guys were definitely looking for the Xnet, I could tell.

Now I really couldn't sleep.

Nothing for it but to play wind-up pirates. There'd be lots of players
even at this hour. The real name for wind-up pirates was Clockwork
Plunder, and it was a hobbyist project that had been created by
teenaged death-metal freaks from Finland. It was totally free to play,
and offered just as much fun as any of the \$15/month services like
Ender's Universe and Middle Earth Quest and Discworld Dungeons.

I logged back in and there I was, still on the deck of the Zombie
Charger, waiting for someone to wind me up. I hated this part of the
game.

\edialog{Hey you}

I typed to a passing pirate.

\edialog{Wind me up?}

He paused and looked at me.

\edialog{y should i?}

\edialog{We're on the same team. Plus you get experience points.}

What a jerk.

\edialog{Where are you located?}

\edialog{San Francisco}

This was starting to feel familiar.

\edialog{Where in San Francisco?}

I logged out. There was something weird going on in the game. I jumped
onto the livejournals and began to crawl from blog to blog. I got
through half a dozen before I found something that froze my blood.

Livejournallers love quizzes. What kind of hobbit are you? Are you a
great lover? What planet are you most like? Which character from some
movie are you? What's your emotional type? They fill them in and their
friends fill them in and everyone compares their results. Harmless
fun.

But the quiz that had taken over the blogs of the Xnet that night was
what scared me, because it was anything but harmless:

\begin{itemize}
\item What's your sex
\item What grade are you in?
\item What school do you go to?
\item Where in the city do you live?
\end{itemize}

The quizzes plotted the results on a map with colored pushpins for
schools and neighborhoods, and made lame recommendations for places to
buy pizza and stuff.

But look at those questions. Think about my answers:

\begin{itemize}
\item Male 
\item 17
\item Chavez High
\item Potrero Hill
\end{itemize}
There were only two people in my whole school who matched that
profile. Most schools it would be the same. If you wanted to figure
out who the Xnetters were, you could use these quizzes to find them
all.

That was bad enough, but what was worse what what it implied: someone
from the DHS was using the Xnet to get at us. The Xnet was compromised
by the DHS.

We had spies in our midst.

\fancybreak{\#}

I'd given Xnet discs to hundreds of people, and they'd done the
same. I knew the people I gave the discs to pretty well. Some of them
I knew very well. I've lived in the same house all my life and I've
made hundreds and hundreds of friends over the years, from people who
went to daycare with me to people I played soccer with, people who
LARPed with me, people I met clubbing, people I knew from school. My
ARG team were my closest friends, but there were plenty of people I
knew and trusted enough to hand an Xnet disc to.

I needed them now.

I woke Jolu up by ringing his cell phone and hanging up after the
first ring, three times in a row. A minute later, he was up on Xnet
and we were able to have a secure chat. I pointed him to my blog-post
on the radio vans and he came back a minute later all freaked out.

\edialog{You sure they're looking for us?}

In response I sent him to the quiz.

\edialog{OMG we're doomed}

\edialog{No it's not that bad but we need to figure out who we can trust}

\edialog{How?}

\edialog{That's what I wanted to ask you -- how many people can you
  totally vouch for like trust them to the ends of the earth?}

\edialog{Um 20 or 30 or so}

\edialog{I want to get a bunch of really trustworthy people together
  and do a key-exchange web of trust thing}

Web of trust is one of those cool crypto things that I'd read about
but never tried. It was a nearly foolproof way to make sure that you
could talk to the people you trusted, but that no one else could
listen in. The problem is that it requires you to physically meet with
the people in the web at least once, just to get started.

\edialog{I get it sure. That's not bad. But how you going to get
  everyone together for the key-signing?}

\edialog{That's what I wanted to ask you about -- how can we do it
  without getting busted?}

Jolu typed some words and erased them, typed more and erased them.

\edialog{Darryl would know}

I typed.

\edialog{God, this was the stuff he was great at.}

Jolu didn't type anything. Then,

\edialog{How about a party?}

he typed.

\edialog{How about if we all get together somewhere like we're
  teenagers having a party and that way we'll have a ready-made excuse
  if anyone shows up asking us what we're doing there?}

\edialog{That would totally work! You're a genius, Jolu.}

\edialog{I know it. And you're going to love this: I know just where
  to do it, too}

\edialog{Where?}

\edialog{Sutro baths!}

\chapter{Chapter 10}

\epigraph{This chapter is dedicated to Anderson's Bookshops, Chicago's
  legendary kids' bookstore. Anderson's is an old, old family-run
  business, which started out as an old-timey drug-store selling some
  books on the side. Today, it's a booming, multi-location kids' book
  empire, with some incredibly innovative bookselling practices that
  get books and kids together in really exciting ways. The best of
  these is the store's mobile book-fairs, in which they ship huge,
  rolling bookcases, already stocked with excellent kids' books,
  direct to schools on trucks -- voila, instant book-fair!}
{Anderson's Bookshops \url{http://www.andersonsbookshop.com/} 123 West
Jefferson, Naperville, IL 60540 USA +1 630 355 2665}

What would you do if you found out you had a spy in your midst? You
could denounce him, put him up against the wall and take him out. But
then you might end up with another spy in your midst, and the new spy
would be more careful than the last one and maybe not get caught quite
so readily.

Here's a better idea: start intercepting the spy's communications and
feed him and his masters misinformation. Say his masters instruct him
to gather information on your movements. Let him follow you around and
take all the notes he wants, but steam open the envelopes that he
sends back to HQ and replace his account of your movements with a
fictitious one. If you want, you can make him seem erratic and
unreliable so they get rid of him. You can manufacture crises that
might make one side or the other reveal the identities of other
spies. In short, you own them.

This is called the man-in-the-middle attack and if you think about it,
it's pretty scary. Someone who man-in-the-middles your communications
can trick you in any of a thousand ways.

Of course, there's a great way to get around the man-in-the-middle
attack: use crypto. With crypto, it doesn't matter if the enemy can
see your messages, because he can't decipher them, change them, and
re-send them. That's one of the main reasons to use crypto.

But remember: for crypto to work, you need to have keys for the people
you want to talk to. You and your partner need to share a secret or
two, some keys that you can use to encrypt and decrypt your messages
so that men-in-the-middle get locked out.

That's where the idea of public keys comes in. This is a little hairy,
but it's so unbelievably elegant too.

In public key crypto, each user gets two keys. They're long strings of
mathematical gibberish, and they have an almost magic
property. Whatever you scramble with one key, the other will unlock,
and vice-versa. What's more, they're the \emph{only} keys that can do this
-- if you can unscramble a message with one key, you \emph{know} it was
scrambled with the other (and vice-versa).

So you take either one of these keys (it doesn't matter which one) and
you just \emph{publish} it. You make it a total \emph{non-secret}. You want
anyone in the world to know what it is. For obvious reasons, they call
this your ``public key.''

The other key, you hide in the darkest reaches of your mind. You
protect it with your life. You never let anyone ever know what it
is. That's called your ``private key.'' (Duh.)

Now say you're a spy and you want to talk with your bosses. Their
public key is known by everyone. Your public key is known by
everyone. No one knows your private key but you. No one knows their
private key but them.

You want to send them a message. First, you encrypt it with your
private key. You could just send that message along, and it would work
pretty well, since they would know when the message arrived that it
came from you. How? Because if they can decrypt it with your public
key, it can \emph{only} have been encrypted with your private key. This is
the equivalent of putting your seal or signature on the bottom of a
message. It says, ``I wrote this, and no one else. No one could have
tampered with it or changed it.''

Unfortunately, this won't actually keep your message a
\emph{secret}. That's because your public key is really well known (it has
to be, or you'll be limited to sending messages to those few people
who have your public key). Anyone who intercepts the message can read
it. They can't change it and make it seem like it came from you, but
if you don't want people to know what you're saying, you need a better
solution.

So instead of just encrypting the message with your private key, you
\emph{also} encrypt it with your boss's public key. Now it's been locked
twice. The first lock -- the boss's public key -- only comes off when
combined with your boss's private key. The second lock -- your private
key -- only comes off with your public key. When your bosses receive
the message, they unlock it with both keys and now they know for sure
that: a) you wrote it and b) that only they can read it.

It's very cool. The day I discovered it, Darryl and I immediately
exchanged keys and spent months cackling and rubbing our hands as we
exchanged our military-grade secret messages about where to meet after
school and whether Van would ever notice him.

But if you want to understand security, you need to consider the most
paranoid possibilities. Like, what if I tricked you into thinking that
\emph{my} public key was your boss's public key? You'd encrypt the message
with your private key and my public key. I'd decrypt it, read it,
re-encrypt it with your boss's \emph{real} public key and send it on. As
far as your boss knows, no one but you could have written the message
and no one but him could have read it.

And I get to sit in the middle, like a fat spider in a web, and all
your secrets belong to me.

Now, the easiest way to fix this is to really widely advertise your
public key. If it's \emph{really} easy for anyone to know what your real
key is, man-in-the-middle gets harder and harder. But you know what?
Making things well-known is just as hard as keeping them secret. Think
about it -- how many billions of dollars are spent on shampoo ads and
other crap, just to make sure that as many people know about something
that some advertiser wants them to know?

There's a cheaper way of fixing man-in-the-middle: the web of
trust. Say that before you leave HQ, you and your bosses sit down over
coffee and actually tell each other your keys. No more
man-in-the-middle! You're absolutely certain whose keys you have,
because they were put into your own hands.

So far, so good. But there's a natural limit to this: how many people
can you physically meet with and swap keys? How many hours in the day
do you want to devote to the equivalent of writing your own phone
book? How many of those people are willing to devote that kind of time
to you?

Thinking about this like a phonebook helps. The world was once a place
with a lot of phonebooks, and when you needed a number, you could look
it up in the book. But for many of the numbers that you wanted to
refer to on a given day, you would either know it by heart, or you'd
be able to ask someone else. Even today, when I'm out with my
cell-phone, I'll ask Jolu or Darryl if they have a number I'm looking
for. It's faster and easier than looking it up online and they're more
reliable, too. If Jolu has a number, I trust him, so I trust the
number, too. That's called ``transitive trust'' -- trust that moves
across the web of our relationships.

A web of trust is a bigger version of this. Say I meet Jolu and get
his key. I can put it on my ``keyring'' -- a list of keys that I've
signed with my private key. That means you can unlock it with my
public key and know for sure that me -- or someone with my key, anyway
-- says that ``this key belongs to this guy.''

So I hand you my keyring and provided that you trust me to have
actually met and verified all the keys on it, you can take it and add
it to your keyring. Now, you meet someone else and you hand the whole
ring to him. Bigger and bigger the ring grows, and provided that you
trust the next guy in the chain, and he trusts the next guy in his
chain and so on, you're pretty secure.

Which brings me to keysigning parties. These are \emph{exactly} what they
sound like: a party where everyone gets together and signs everyone
else's keys. Darryl and I, when we traded keys, that was kind of a
mini-keysigning party, one with only two sad and geeky attendees. But
with more people, you create the seed of the web of trust, and the web
can expand from there. As everyone on your keyring goes out into the
world and meets more people, they can add more and more names to the
ring. You don't have to meet the new people, just trust that the
signed key you get from the people in your web is valid.

So that's why web of trust and parties go together like peanut butter
and chocolate.

\fancybreak{\#}

``Just tell them it's a super-private party, invitational only,'' I
said. ``Tell them not to bring anyone along or they won't be admitted.''

Jolu looked at me over his coffee. ``You're joking, right? You tell
people that, and they'll bring \emph{extra} friends.''

``Argh,'' I said. I spent a night a week at Jolu's these days, keeping
the code up to date on indienet. Pigspleen actually paid me a non-zero
sum of money to do this, which was really weird. I never thought I'd
be paid to write code.

``So what do we do? We only want people we really trust there, and we
don't want to mention why until we've got everyone's keys and can send
them messages in secret.''

Jolu debugged and I watched over his shoulder. This used to be called
``extreme programming,'' which was a little embarrassing. Now we just
call it ``programming.'' Two people are much better at spotting bugs
than one. As the cliche goes, ``With enough eyeballs, all bugs are
shallow.''

We were working our way through the bug reports and getting ready to
push out the new rev. It all auto-updated in the background, so our
users didn't really need to do anything, they just woke up once a week
or so with a better program. It was pretty freaky to know that the
code I wrote would be used by hundreds of thousands of people,
\emph{tomorrow}!

``What do we do? Man, I don't know. I think we just have to live with
it.''

I thought back to our Harajuku Fun Madness days. There were lots of
social challenges involving large groups of people as part of that
game.

``OK, you're right. But let's at least try to keep this secret. Tell
them that they can bring a maximum of one person, and it has to be
someone they've known personally for a minimum of five years.''

Jolu looked up from the screen. ``Hey,'' he said. ``Hey, that would
totally work. I can really see it. I mean, if you told me not to bring
anyone, I'd be all, 'Who the hell does he think he is?' But when you
put it that way, it sounds like some awesome 007 stuff.''

I found a bug. We drank some coffee. I went home and played a little
Clockwork Plunder, trying not to think about key-winders with nosy
questions, and slept like a baby.

\fancybreak{\#}

Sutro baths are San Francisco's authentic fake Roman ruins. When it
opened in 1896, it was the largest indoor bathing house in the world,
a huge Victorian glass solarium filled with pools and tubs and even an
early water slide. It went downhill by the fifties, and the owners
torched it for the insurance in 1966. All that's left is a labyrinth
of weathered stone set into the sere cliff-face at Ocean Beach. It
looks for all the world like a Roman ruin, crumbled and mysterious,
and just beyond them is a set of caves that let out into the sea. In
rough tides, the waves rush through the caves and over the ruins --
they've even been known to suck in and drown the occasional tourist.

Ocean Beach is way out past Golden Gate park, a stark cliff lined with
expensive, doomed houses, plunging down to a narrow beach studded with
jellyfish and brave (insane) surfers. There's a giant white rock that
juts out of the shallows off the shore. That's called Seal Rock, and
it used to be the place where the sea lions congregated until they
were relocated to the more tourist-friendly environs of Fisherman's
Wharf.

After dark, there's hardly anyone out there. It gets very cold, with a
salt spray that'll soak you to your bones if you let it. The rocks are
sharp and there's broken glass and the occasional junkie needle.

It is an awesome place for a party.

Bringing along the tarpaulins and chemical glove-warmers was my
idea. Jolu figured out where to get the beer -- his older brother,
Javier, had a buddy who actually operated a whole underage drinking
service: pay him enough and he'd back up to your secluded party spot
with ice-chests and as many brews as you wanted. I blew a bunch of my
indienet programming money, and the guy showed up right on time: 8PM,
a good hour after sunset, and lugged the six foam ice-chests out of
his pickup truck and down into the ruins of the baths. He even brought
a spare chest for the empties.

``You kids play safe now,'' he said, tipping his cowboy hat. He was a
fat Samoan guy with a huge smile, and a scary tank-top that you could
see his armpit- and belly- and shoul\-der-hair escaping from. I peeled
twenties off my roll and handed them to him -- his markup was 150
percent. Not a bad racket.

He looked at my roll. ``You know, I could just take that from you,'' he
said, still smiling. ``I'm a criminal, after all.''

I put my roll in my pocket and looked him levelly in the eye. I'd been
stupid to show him what I was carrying, but I knew that there were
times when you should just stand your ground.

``I'm just messing with you,'' he said, at last. ``But you be careful
with that money. Don't go showing it around.''

``Thanks,'' I said. ``Homeland Security'll get my back though.''

His smile got even bigger. ``Ha! They're not even real five-oh. Those
peckerwoods don't know nothin'.''

I looked over at his truck. Prominently displayed in his windscreen
was a FasTrak. I wondered how long it would be until he got busted.

``You got girls coming tonight? That why you got all the beer?''

I smiled and waved at him as though he was walking back to his truck,
which he should have been doing. He eventually got the hint and drove
away. His smile never faltered.

Jolu helped me hide the coolers in the rubble, working with little
white LED torches on headbands. Once the coolers were in place, we
threw little white LED keychains into each one, so it would glow when
you took the styrofoam lids off, making it easier to see what you were
doing.

It was a moonless night and overcast, and the distant streetlights
barely illuminated us. I knew we'd stand out like blazes on an
infrared scope, but there was no chance that we'd be able to get a
bunch of people together without being observed. I'd settle for being
dismissed as a little drunken beach-party.

I don't really drink much. There's been beer and pot and ecstasy at
the parties I've been going to since I was 14, but I hated smoking
(though I'm quite partial to a hash brownie every now and again),
ecstasy took too long -- who's got a whole weekend to get high and
come down -- and beer, well, it was all right, but I didn't see what
the big deal was. My favorite was big, elaborate cocktails, the kind
of thing served in a ceramic volcano, with six layers, on fire, and a
plastic monkey on the rim, but that was mostly for the theater of it
all.

I actually like being drunk. I just don't like being hungover, and
boy, do I ever get hungover. Though again, that might have to do with
the kind of drinks that come in a ceramic volcano.

But you can't throw a party without putting a case or two of beer on
ice. It's expected. It loosens things up. People do stupid things
after too many beers, but it's not like my friends are the kind of
people who have cars. And people do stupid things no matter what --
beer or grass or whatever are all incidental to that central fact.

Jolu and I each cracked beers -- Anchor Steam for him, a Bud Lite for
me -- and clinked the bottles together, sitting down on a rock.

``You told them 9PM?''

``Yeah,'' he said.

``Me too.''

We drank in silence. The Bud Lite was the least alcoholic thing in the
ice-chest. I'd need a clear head later.

``You ever get scared?'' I said, finally.

He turned to me. ``No man, I don't get scared. I'm always scared. I've
been scared since the minute the explosions happened. I'm so scared
sometimes, I don't want to get out of bed.''

``Then why do you do it?''

He smiled. ``About that,'' he said. ``Maybe I won't, not for much
longer. I mean, it's been great helping you. Great. Really
excellent. I don't know when I've done anything so important. But
Marcus, bro, I have to say. . .'' He trailed off.

``What?'' I said, though I knew what was coming next.

``I can't do it forever,'' he said at last. ``Maybe not even for another
month. I think I'm through. It's too much risk. The DHS, you can't go
to war on them. It's crazy. Really actually crazy.''

``You sound like Van,'' I said. My voice was much more bitter than I'd
intended.

``I'm not criticizing you, man. I think it's great that you've got the
bravery to do this all the time. But I haven't got it. I can't live my
life in perpetual terror.''

``What are you saying?''

``I'm saying I'm out. I'm going to be one of those people who acts like
it's all OK, like it'll all go back to normal some day. I'm going to
use the Internet like I always did, and only use the Xnet to play
games. I'm going to get out is what I'm saying. I won't be a part of
your plans anymore.''

I didn't say anything.

``I know that's leaving you on your own. I don't want that, believe
me. I'd much rather you give up with me. You can't declare war on the
government of the USA. It's not a fight you're going to win. Watching
you try is like watching a bird fly into a window again and again.''

He wanted me to say something. What \emph{I} wanted to say was, \emph{Jesus
Jolu, thanks so very much for abandoning me! Do you forget what it was
like when they took us away? Do you forget what the country used to be
like before they took it over?} But that's not what he wanted me to
say. What he wanted me to say was:

``I understand, Jolu. I respect your choice.''

He drank the rest of his bottle and pulled out another one and twisted
off the cap.

``There's something else,'' he said.

``What?''

``I wasn't going to mention it, but I want you to understand why I have
to do this.''

``Jesus, Jolu, \emph{what}?''

``I hate to say it, but you're \emph{white}. I'm not. White people get
caught with cocaine and do a little rehab time. Brown people get
caught with crack and go to prison for twenty years. White people see
cops on the street and feel safer. Brown people see cops on the street
and wonder if they're about to get searched. The way the DHS is
treating you? The law in this country has always been like that for
us.''

It was so unfair. I didn't ask to be white. I didn't think I was being
braver just because I'm white. But I knew what Jolu was saying. If the
cops stopped someone in the Mission and asked to see some ID, chances
were that person wasn't white. Whatever risk I ran, Jolu ran
more. Whatever penalty I'd pay, Jolu would pay more.

``I don't know what to say,'' I said.

``You don't have to say anything,'' he said. ``I just wanted you to know,
so you could understand.''

I could see people walking down the side trail toward us. They were
friends of Jolu's, two Mexican guys and a girl I knew from around,
short and geeky, always wearing cute black Buddy Holly glasses that
made her look like the outcast art-student in a teen movie who comes
back as the big success.

Jolu introduced me and gave them beers. The girl didn't take one, but
instead produced a small silver flask of vodka from her purse and
offered me a drink. I took a swallow -- warm vodka must be an acquired
taste -- and complimented her on the flask, which was embossed with a
repeating motif of Parappa the Rapper characters.

``It's Japanese,'' she said as I played another LED keyring over
it. ``They have all these great booze-toys based on kids'
games. Totally twisted.''

I introduced myself and she introduced herself. ``Ange,'' she said, and
shook my hand with hers -- dry, warm, with short nails. Jolu
introduced me to his pals, whom he'd known since computer camp in the
fourth grade. More people showed up -- five, then ten, then twenty. It
was a seriously big group now.

We'd told people to arrive by 9:30 sharp, and we gave it until 9:45 to
see who all would show up. About three quarters were Jolu's
friends. I'd invited all the people I really trusted. Either I was
more discriminating than Jolu or less popular. Now that he'd told me
he was quitting, it made me think that he was less discriminating. I
was really pissed at him, but trying not to let it show by
concentrating on socializing with other people. But he wasn't
stupid. He knew what was going on. I could see that he was really
bummed. Good.

``OK,'' I said, climbing up on a ruin, ``OK, hey, hello?'' A few people
nearby paid attention to me, but the ones in the back kept on
chatting. I put my arms in the air like a referee, but it was too
dark. Eventually I hit on the idea of turning my LED keychain on and
pointing it at each of the talkers in turn, then at me. Gradually, the
crowd fell quiet.

I welcomed them and thanked them all for coming, then asked them to
close in so I could explain why we were there. I could tell they were
into the secrecy of it all, intrigued and a little warmed up by the
beer.

``So here it is. You all use the Xnet. It's no coincidence that the
Xnet was created right after the DHS took over the city. The people
who did that are an organization devoted to personal liberty, who
created the network to keep us safe from DHS spooks and enforcers.''
Jolu and I had worked this out in advance. We weren't going to cop to
being behind it all, not to anyone. It was way too risky. Instead,
we'd put it out that we were merely lieutenants in ``M1k3y'''s army,
acting to organize the local resistance.

``The Xnet isn't pure,'' I said. ``It can be used by the other side just
as readily as by us. We know that there are DHS spies who use it
now. They use social engineering hacks to try to get us to reveal
ourselves so that they can bust us. If the Xnet is going to succeed,
we need to figure out how to keep them from spying on us. We need a
network within the network.''

I paused and let this sink in. Jolu had suggested that this might be a
little heavy -- learning that you're about to be brought into a
revolutionary cell.

``Now, I'm not here to ask you to do anything active. You don't have to
go out jamming or anything. You've been brought here because we know
you're cool, we know you're trustworthy. It's that trustworthiness I
want to get you to contribute tonight. Some of you will already be
familiar with the web of trust and keysigning parties, but for the
rest of you, I'll run it down quickly --'' Which I did.

``Now what I want from you tonight is to meet the people here and
figure out how much you can trust them. We're going to help you
generate key-pairs and share them with each other.''

This part was tricky. Asking people to bring their own laptops
wouldn't have worked out, but we still needed to do something hella
complicated that wouldn't exactly work with paper and pencil.

I held up a laptop Jolu and I had rebuilt the night before, from the
ground up. ``I trust this machine. Every component in it was laid by
our own hands. It's running a fresh out-of-the-box version of
ParanoidLinux, booted off of the DVD. If there's a trustworthy
computer left anywhere in the world, this might well be it.

``I've got a key-generator loaded here. You come up here and give it
some random input -- mash the keys, wiggle the mouse -- and it will
use that as the seed to create a random public- and private key for
you, which it will display on the screen. You can take a picture of
the private key with your phone, and hit any key to make it go away
forever -- it's not stored on the disk at all. Then it will show you
your public key. At that point, you call over all the people here you
trust and who trust you, and \emph{they} take a picture of the screen with
you standing next to it, so they know whose key it is.

``When you get home, you have to convert the photos to keys. This is
going to be a lot of work, I'm afraid, but you'll only have to do it
once. You have to be super-careful about typing these in -- one
mistake and you're screwed. Luckily, we've got a way to tell if you've
got it right: beneath the key will be a much shorter number, called
the 'fingerprint'. Once you've typed in the key, you can generate a
fingerprint from it and compare it to the fingerprint, and if they
match, you've got it right.''

They all boggled at me. OK, so I'd asked them to do something pretty
weird, it's true, but still.


\chapter{Chapter 11}

\epigraph{This chapter is dedicated to the University Bookstore at the
  University of Washington, whose science fiction section rivals many
  specialty stores, thanks to the sharp-eyed, dedicated science
  fiction buyer, Duane Wilkins. Duance's a real science fiction fan --
  I first met him at the World Science Fiction Convention in Toronto
  in 2003 -- and it shows in the eclectic and informed choices on
  display at the store. One great predictor of a great bookstore is
  the quality of the ``shelf review'' -- the little bits of cardboard
  stuck to the shelves with (generally hand-lettered) staff-reviews
  extolling the virtues of books you might otherwise miss. The staff
  at the University Bookstore have clearly benefited from Duane's
  tutelage, as the shelf reviews at the University Bookstore are
  second to none.}
{The University Bookstore\footnote{
\url{http://www4.bookstore.washington.edu/_trade/ShowTitleUBS.taf?ActionArg=Title&ISBN=9780765319852}}:
4326 University Way NE, Seattle, WA 98105 USA +1 800 335 READ}

Jolu stood up.

``This is where it starts, guys. This is how we know which side you're
on. You might not be willing to take to the streets and get busted for
your beliefs, but if you \emph{have} beliefs, this will let us know
it. This will create the web of trust that tells us who's in and who's
out. If we're ever going to get our country back, we need to do
this. We need to do something like this.''

Someone in the audience -- it was Ange -- had a hand up, holding a
beer bottle.

``So call me stupid but I don't understand this at all. Why do you want
us to do this?''

Jolu looked at me, and I looked back at him. It had all seemed so
obvious when we were organizing it. ``The Xnet isn't just a way to play
free games. It's the last open communications network in America. It's
the last way to communicate without being snooped on by the DHS. For
it to work we need to know that the person we're talking to isn't a
snoop. That means that we need to know that the people we're sending
messages to are the people we think they are.

``That's where you come in. You're all here because we trust you. I
mean, really trust you. Trust you with our lives.''

Some of the people groaned. It sounded melodramatic and stupid.

I got back to my feet.

``When the bombs went off,'' I said, then something welled up in my
chest, something painful. ``When the bombs went off, there were four of
us caught up by Market Street. For whatever reason, the DHS decided
that made us suspicious. They put bags over our heads, put us on a
ship and interrogated us for days. They humiliated us. Played games
with our minds. Then they let us go.

``All except one person. My best friend. He was with us when they
picked us up. He'd been hurt and he needed medical care. He never came
out again. They say they never saw him. They say that if we ever tell
anyone about this, they'll arrest us and make us disappear.

``Forever.''

I was shaking. The shame. The goddamned shame. Jolu had the light on
me.

``Oh Christ,'' I said. ``You people are the first ones I've told. If this
story gets around, you can bet they'll know who leaked it. You can bet
they'll come knocking on my door.'' I took some more deep
breaths. ``That's why I volunteered on the Xnet. That's why my life,
from now on, is about fighting the DHS. With every breath. Every
day. Until we're free again. Any one of you could put me in jail now,
if you wanted to.''

Ange put her hand up again. ``We're not going to rat on you,'' she
said. ``No way. I know pretty much everyone here and I can promise you
that. I don't know how to know who to trust, but I know who \emph{not} to
trust: old people. Our parents. Grownups. When they think of someone
being spied on, they think of someone \emph{else}, a bad guy. When they
think of someone being caught and sent to a secret prison, it's
someone \emph{else} -- someone brown, someone young, someone foreign.

``They forget what it's like to be our age. To be the object of
suspicion \emph{all the time}! How many times have you gotten on the bus
and had every person on it give you a look like you'd been gargling
turds and skinning puppies?

``What's worse, they're turning into adults younger and younger out
there. Back in the day, they used to say 'Never trust anyone over 30.'
I say, 'Don't trust any bastard over 25!'''

That got a laugh, and she laughed too. She was pretty, in a weird,
horsey way, with a long face and a long jaw. ``I'm not really kidding,
you know? I mean, think about it. Who elected these ass-clowns? Who
let them invade our city? Who voted to put the cameras in our
classrooms and follow us around with creepy spyware chips in our
transit passes and cars? It wasn't a 16-year-old. We may be dumb, we
may be young, but we're not scum.''

``I want that on a t-shirt,'' I said.

``It would be a good one,'' she said. We smiled at each other.

``Where do I go to get my keys?'' she said, and pulled out her phone.

``We'll do it over there, in the secluded spot by the caves. I'll take
you in there and set you up, then you do your thing and take the
machine around to your friends to get photos of your public key so
they can sign it when they get home.''

I raised my voice. ``Oh! One more thing! Jesus, I can't believe I
forgot this. \emph{Delete those photos once you've typed in the keys}! The
last thing we want is a Flickr stream full of pictures of all of us
conspiring together.''

There was some good-natured, nervous chuckling, then Jolu turned out
the light and in the sudden darkness I could see nothing. Gradually,
my eyes adjusted and I set off for the cave. Someone was walking
behind me. Ange. I turned and smiled at her, and she smiled back,
luminous teeth in the dark.

``Thanks for that,'' I said. ``You were great.''

``You mean what you said about the bag on your head and everything?''

``I meant it,'' I said. ``It happened. I never told anyone, but it
happened.'' I thought about it for a moment. ``You know, with all the
time that went by since, without saying anything, it started to feel
like a bad dream. It was real though.'' I stopped and climbed up into
the cave. ``I'm glad I finally told people. Any longer and I might have
started to doubt my own sanity.''

I set up the laptop on a dry bit of rock and booted it from the DVD
with her watching. ``I'm going to reboot it for every person. This is a
standard ParanoidLinux disc, though I guess you'd have to take my word
for it.''

``Hell,'' she said. ``This is all about trust, right?''

``Yeah,'' I said. ``Trust.''

I retreated some distance as she ran the key-generator, listening to
her typing and mousing to create randomness, listening to the crash of
the surf, listening to the party noises from over where the beer was.

She stepped out of the cave, carrying the laptop. On it, in huge white
luminous letters, were her public key and her fingerprint and email
address. She held the screen up beside her face and waited while I got
my phone out.

``Cheese,'' she said. I snapped her pic and dropped the camera back in
my pocket. She wandered off to the revelers and let them each get pics
of her and the screen. It was festive. Fun. She really had a lot of
charisma -- you didn't want to laugh at her, you just wanted to laugh
\emph{with} her. And hell, it \emph{was} funny! We were declaring a secret war
on the secret police. Who the hell did we think we were?

So it went, through the next hour or so, everyone taking pictures and
making keys. I got to meet everyone there. I knew a lot of them --
some were my invitees -- and the others were friends of my pals or my
pals' pals. We should all be buddies. We were, by the time the night
was out. They were all good people.

Once everyone was done, Jolu went to make a key, and then turned away,
giving me a sheepish grin. I was past my anger with him, though. He
was doing what he had to do. I knew that no matter what he said, he'd
always be there for me. And we'd been through the DHS jail
together. Van too. No matter what, that would bind us together
forever.

I did my key and did the perp-walk around the gang, letting everyone
snap a pic. Then I climbed up on the high spot I'd spoken from earlier
and called for everyone's attention.

``So a lot of you have noted that there's a vital flaw in this
procedure: what if this laptop can't be trusted? What if it's secretly
recording our instructions? What if it's spying on us? What if
Jose-Luis and I can't be trusted?''

More good-natured chuckles. A little warmer than before, more beery.

``I mean it,'' I said. ``If we were on the wrong side, this could get all
of us -- all of \emph{you} -- into a heap of trouble. Jail, maybe.''

The chuckles turned more nervous.

``So that's why I'm going to do this,'' I said, and picked up a hammer
I'd brought from my Dad's toolkit. I set the laptop down beside me on
the rock and swung the hammer, Jolu following the swing with his
keychain light. Crash -- I'd always dreamt of killing a laptop with a
hammer, and here I was doing it. It felt pornographically good. And
bad.

Smash! The screen-panel fell off, shattered into millions of pieces,
exposing the keyboard. I kept hitting it, until the keyboard fell off,
exposing the motherboard and the hard-drive. Crash! I aimed square for
the hard-drive, hitting it with everything I had. It took three blows
before the case split, exposing the fragile media inside. I kept
hitting it until there was nothing bigger than a cigarette lighter,
then I put it all in a garbage bag. The crowd was cheering wildly --
loud enough that I actually got worried that someone far above us
might hear over the surf and call the law.

``All right!'' I called. ``Now, if you'd like to accompany me, I'm going
to march this down to the sea and soak it in salt water for ten
minutes.''

I didn't have any takers at first, but then Ange came forward and took
my arm in her warm hand and said, ``That was beautiful,'' in my ear and
we marched down to the sea together.

It was perfectly dark by the sea, and treacherous, even with our
keychain lights. Slippery, sharp rocks that were difficult enough to
walk on even without trying to balance six pounds of smashed
electronics in a plastic bag. I slipped once and thought I was going
to cut myself up, but she caught me with a surprisingly strong grip
and kept me upright. I was pulled in right close to her, close enough
to smell her perfume, which smelled like new cars. I love that smell.

``Thanks,'' I managed, looking into the big eyes that were further
magnified by her mannish, black-rimmed glasses. I 
\discretionary{could}{not}{couldn't} tell what
color they were in the dark, but I guessed something dark, based on
her dark hair and olive complexion. She looked Mediterranean, maybe
Greek or Spanish or Italian.

I crouched down and dipped the bag in the sea, letting it fill with
salt water. I managed to slip a little and soak my shoe, and I swore
and she laughed. We'd hardly said a word since we lit out for the
ocean. There was something magical in our wordless silence.

At that point, I had kissed a total of three girls in my life, not
counting that moment when I went back to school and got a hero's
welcome. That's not a gigantic number, but it's not a minuscule one,
either. I have reasonable girl radar, and I think I could have kissed
her. She wasn't h4wt in the traditional sense, but there's something
about a girl and a night and a beach, plus she was smart and
passionate and committed.

But I didn't kiss her, or take her hand. Instead we had a moment that
I can only describe as spiritual. The surf, the night, the sea and the
rocks, and our breathing. The moment stretched. I sighed. This had
been quite a ride. I had a lot of typing to do tonight, putting all
those keys into my keychain, signing them and publishing the signed
keys. Starting the web of trust.

She sighed too.

``Let's go,'' I said.

``Yeah,'' she said.

Back we went. It was a good night, that night.

\fancybreak{\#}

Jolu waited after for his brother's friend to come by and pick up his
coolers. I walked with everyone else up the road to the nearest Muni
stop and got on board. Of course, none of us was using an issued Muni
pass. By that point, Xnetters habitually cloned someone else's Muni
pass three or four times a day, assuming a new identity for every
ride.

It was hard to stay cool on the bus. We were all a little drunk, and
looking at our faces under the bright bus lights was kind of
hilarious. We got pretty loud and the driver used his intercom to tell
us to keep it down twice, then told us to shut up right now or he'd
call the cops.

That set us to giggling again and we disembarked in a mass before he
did call the cops. We were in North Beach now, and there were lots of
buses, taxis, the BART at Market Street, neon-lit clubs and cafes to
pull apart our grouping, so we drifted away.

I got home and fired up my Xbox and started typing in keys from my
phone's screen. It was dull, hypnotic work. I was a little drunk, and
it lulled me into a half-sleep.

I was about ready to nod off when a new IM window popped up.

\edialog{herro!}

I didn't recognize the handle -- spexgril -- but I had an idea who
might be behind it.

\edialog{hi}

I typed, cautiously.

\edialog{it's me, from tonight}

Then she paste-bombed a block of crypto. I'd already entered her
public key into my keychain, so I told the IM client to try decrypting
the code with the key.

\edialog{it's me, from tonight}

It was her!

\edialog{Fancy meeting you here}

I typed, then encrypted it to my public key and mailed it off.

\edialog{It was great meeting you}

I typed.

\edialog{You too. I don't meet too many smart guys who are also cute
  and also socially aware. Good god, man, you don't give a girl much
  of a chance.}

My heart hammered in my chest.

\edialog{Hello? Tap tap? This thing on? I wasn't born here folks, but
  I'm sure dying here. Don't forget to tip your waitresses, they work
  hard. I'm here all week.}

I laughed aloud.

\edialog{I'm here, I'm here. Laughing too hard to type is all}

\edialog{Well at least my IM comedy-fu is still mighty}

Um.

\edialog{It was really great to meet you too}

\edialog{Yeah, it usually is. Where are you taking me?}

\edialog{Taking you?}

\edialog{On our next adventure?}

\edialog{I didn't really have anything planned}

\edialog{Oki -- then I'll take YOU. Friday. Dolores Park. Illegal
  open air concert. Be there or be a dodecahedron}

\edialog{Wait what?}

\edialog{Don't you even read Xnet? It's all over the place. You ever
  hear of the Speedwhores?}

I nearly choked. That was Trudy Doo's band -- as in Trudy Doo, the
woman who had paid me and Jolu to update the indienet code.

\edialog{Yeah I've heard of them}

\edialog{They're putting on a huge show and they've got like fifty
  bands signed to play the bill, going to set up on the tennis courts
  and bring out their own amp trucks and rock out all night}

I felt like I'd been living under a rock. How had I missed that? There
was an anarchist bookstore on Valencia that I sometimes passed on the
way to school that had a poster of an old revolutionary named Emma
Goldman with the caption ``If I can't dance, I don't want to be a part
of your revolution.'' I'd been spending all my energies on figuring out
how to use the Xnet to organize dedicated fighters so they could jam
the DHS, but this was so much cooler. A big concert -- I had no idea
how to do one of those, but I was glad someone did.

And now that I thought of it, I was damned proud that they were using
the Xnet to do it.

\fancybreak{\#}

The next day I was a zombie. Ange and I had chatted -- flirted --
until 4AM. Lucky for me, it was a Saturday and I was able to sleep in,
but between the hangover and the sleep-dep, I could barely put two
thoughts together.

By lunchtime, I managed to get up and get my ass out onto the
streets. I staggered down toward the Turk's to buy my coffee -- these
days, if I was alone, I always bought my coffee there, like the Turk
and I were part of a secret club.

On the way, I passed a lot of fresh graffiti. I liked Mission
graffiti; a lot of the times, it came in huge, luscious murals, or
sarcastic art-student stencils. I liked that the Mission's taggers
kept right on going, under the nose of the DHS. Another kind of Xnet,
I supposed -- they must have all kinds of ways of knowing what was
going on, where to get paint, what cameras worked. Some of the cameras
had been spray-painted over, I noticed.

Maybe they used Xnet!

Painted in ten-foot-high letters on the side of an auto-yard's fence
were the drippy words: DON'T TRUST ANYONE OVER 25.

I stopped. Had someone left my ``party'' last night and come here with a
can of paint? A lot of those people lived in the neighborhood.

I got my coffee and had a little wander around town. I kept thinking I
should be calling someone, seeing if they wanted to get a movie or
something. That's how it used to be on a lazy Saturday like this. But
who was I going to call? Van wasn't talking to me, I didn't think I
was ready to talk to Jolu, and Darryl --

Well, I couldn't call Darryl.

I got my coffee and went home and did a little searching around on the
Xnet's blogs. These anonablogs were untraceable to any author --
unless that author was stupid enough to put her name on it -- and
there were a lot of them. Most of them were apolitical, but a lot of
them weren't. They talked about schools and the unfairness there. They
talked about the cops. Tagging.

Turned out there'd been plans for the concert in the park for
weeks. It had hopped from blog to blog, turning into a full-blown
movement without my noticing. And the concert was called Don't Trust
Anyone Over 25.

Well, that explained where Ange got it. It was a good slogan.

\fancybreak{\#}

Monday morning, I decided I wanted to check out that anarchist
bookstore again, see about getting one of those Emma Goldman
posters. I needed the reminder.

I detoured down to 16th and Mission on my way to school, then up to
Valencia and across. The store was shut, but I got the hours off the
door and made sure they still had that poster up.

As I walked down Valencia, I was amazed to see how much of the DON'T
TRUST ANYONE OVER 25 stuff there was. Half the shops had DON'T TRUST
merch in the windows: lunchboxes, babydoll tees, pencil-boxes, trucker
hats. The hipster stores have been getting faster and faster, of
course. As new memes sweep the net in the course of a day or two,
stores have gotten better at putting merch in the windows to
match. Some funny little youtube of a guy launching himself with
jet-packs made of carbonated water would land in your inbox on Monday
and by Tuesday you'd be able to buy t-shirts with stills from the
video on it.

But it was amazing to see something make the leap from Xnet to the
head shops. Distressed designer jeans with the slogan written in
careful high school ball-point ink. Embroidered patches.

Good news travels fast.

It was written on the black-board when I got to Ms Gal\-vez's Social
Studies class. We all sat at our desks, smiling at it. It seemed to
smile back. There was something profoundly cheering about the idea
that we could all trust each other, that the enemy could be
identified. I knew it wasn't entirely true, but it wasn't entirely
false either.

Ms Galvez came in and patted her hair and set down her SchoolBook on
her desk and powered it up. She picked up her chalk and turned around
to face the board. We all laughed. Good-naturedly, but we laughed.

She turned around and was laughing too. ``Inflation has hit the
nation's slogan-writers, it seems. How many of you know where this
phrase comes from?''

We looked at each other. ``Hippies?'' someone said, and we
laughed. Hippies are all over San Francisco, both the old stoner kinds
with giant skanky beards and tie-dyes, and the new kind, who are more
into dress-up and maybe playing hacky-sack than protesting anything.

``Well, yes, hippies. But when we think of hippies these days, we just
think of the clothes and the music. Clothes and music were incidental
to the main part of what made that era, the sixties, important.

``You've heard about the civil rights movement to end segregation,
white and black kids like you riding buses into the South to sign up
black voters and protest against official state racism. California was
one of the main places where the civil rights leaders came from. We've
always been a little more political than the rest of the country, and
this is also a part of the country where black people have been able
to get the same union factory jobs as white people, so they were a
little better off than their cousins in the southland.

``The students at Berkeley sent a steady stream of freedom riders
south, and they recruited them from information tables on campus, at
Bancroft and Telegraph Avenue. You've probably seen that there are
still tables there to this day.

``Well, the campus tried to shut them down. The president of the
university banned political organizing on campus, but the civil rights
kids wouldn't stop. The police tried to arrest a guy who was handing
out literature from one of these tables, and they put him in a van,
but 3,000 students surrounded the van and refused to let it
budge. They wouldn't let them take this kid to jail. They stood on top
of the van and gave speeches about the First Amendment and Free
Speech.

``That galvanized the Free Speech Movement. That was the start of the
hippies, but it was also where more radical student movements came
from. Black power groups like the Black Panthers -- and later gay
rights groups like the Pink Panthers, too. Radical women's groups,
even 'lesbian separatists' who wanted to abolish men altogether! And
the Yippies. Anyone ever hear of the Yippies?''

``Didn't they levitate the Pentagon?'' I said. I'd once seen a
documentary about this.

She laughed. ``I forgot about that, but yes, that was them! Yippies
were like very political hippies, but they weren't serious the way we
think of politics these days. They were very playful. Pranksters. They
threw money into the New York Stock Exchange. They circled the
Pentagon with hundreds of protestors and said a magic spell that was
supposed to levitate it. They invented a fictional kind of LSD that
you could spray onto people with squirt-guns and shot each other with
it and pretended to be stoned. They were funny and they made great TV
-- one Yippie, a clown called Wavy Gravy, used to get hundreds of
protestors to dress up like Santa Claus so that the cameras would show
police officers arresting and dragging away Santa on the news that
night -- and they mobilized a lot of people.

``Their big moment was the Democratic National Convention in 1968,
where they called for demonstrations to protest the Vietnam
War. Thousands of demonstrators poured into Chicago, slept in the
parks, and picketed every day. They had lots of bizarre stunts that
year, like running a pig called Pigasus for the presidential
nomination. The police and the demonstrators fought in the streets --
they'd done that many times before, but the Chicago cops didn't have
the smarts to leave the reporters alone. They beat up the reporters,
and the reporters retaliated by finally showing what really went on at
these demonstrations, so the whole country watched their kids being
really savagely beaten down by the Chicago police. They called it a
'police riot.'

``The Yippies loved to say, 'Never trust anyone over 30.' They meant
that people who were born before a certain time, when America had been
fighting enemies like the Nazis, could never understand what it meant
to love your country enough to refuse to fight the Vietnamese. They
thought that by the time you hit 30, your attitudes would be frozen
and you couldn't ever understand why the kids of the day were taking
to the streets, dropping out, freaking out.

``San Francisco was ground zero for this. Revolutionary armies were
founded here. Some of them blew up buildings or robbed banks for their
cause. A lot of those kids grew up to be more or less normal, while
others ended up in jail. Some of the university dropouts did amazing
things -- for example, Steve Jobs and Steve Wozniak, who founded Apple
Computers and invented the PC.''

I was really getting into this. I knew a little of it, but I'd never
heard it told like this. Or maybe it had never mattered as much as it
did now. Suddenly, those lame, solemn, grown-up street demonstrations
didn't seem so lame after all. Maybe there was room for that kind of
action in the Xnet movement.

I put my hand up. ``Did they win? Did the Yippies win?''

She gave me a long look, like she was thinking it over. No one said a
word. We all wanted to hear the answer.

``They didn't lose,'' she said. ``They kind of imploded a little. Some of
them went to jail for drugs or other things. Some of them changed
their tunes and became yuppies and went on the lecture circuit telling
everyone how stupid they'd been, talking about how good greed was and
how dumb they'd been.

``But they did change the world. The war in Vietnam ended, and the kind
of conformity and unquestioning obedience that people had called
patriotism went out of style in a big way. Black rights, women's
rights and gay rights came a long way. Chicano rights, rights for
disabled people, the whole tradition of civil liberties was created or
strengthened by these people. Today's protest movement is the direct
descendant of those struggles.''

``I can't believe you're talking about them like this,'' Charles
said. He was leaning so far in his seat he was half standing, and his
sharp, skinny face had gone red. He had wet, large eyes and big lips,
and when he got excited he looked a little like a fish.

Ms Galvez stiffened a little, then said, ``Go on, Charles.''

``You've just described terrorists. Actual terrorists. They blew up
buildings, you said. They tried to destroy the stock exchange. They
beat up cops, and stopped cops from arresting people who were breaking
the law. They attacked us!''

Ms Galvez nodded slowly. I could tell she was trying to figure out how
to handle Charles, who really seemed like he was ready to
pop. ``Charles raises a good point. The Yippies weren't foreign agents,
they were American citizens. When you say 'They attacked us,' you need
to figure out who 'they' and 'us' are. When it's your fellow
countrymen --''

``Crap!'' he shouted. He was on his feet now. ``We were at war
then. These guys were giving aid and comfort to the enemy. It's easy
to tell who's us and who's them: if you support America, you're us. If
you support the people who are shooting at Americans, you're \emph{them}.''

``Does anyone else want to comment on this?''

Several hands shot up. Ms Galvez called on them. Some people pointed
out that the reason that the Vietnamese were shooting at Americans is
that the Americans had flown to Vietnam and started running around the
jungle with guns. Others thought that Charles had a point, that people
\discretionary{should}{not}{shouldn't} be allowed to do illegal things.

Everyone had a good debate except Charles, who just shouted at people,
interrupting them when they tried to get their points out. Ms Galvez
tried to get him to wait for his turn a couple times, but he wasn't
having any of it.

I was looking something up on my SchoolBook, something I knew I'd
read.

I found it. I stood up. Ms Galvez looked expectantly at me. The other
people followed her gaze and went quiet. Even Charles looked at me
after a while, his big wet eyes burning with hatred for me.

``I wanted to read something,'' I said. ``It's short. 'Governments are
instituted among men, deriving their just powers from the consent of
the governed, that whenever any form of government becomes destructive
of these ends, it is the right of the people to alter or abolish it,
and to institute new government, laying its foundation on such
principles, and organizing its powers in such form, as to them shall
seem most likely to effect their safety and happiness.'''

\chapter{Chapter 12}

\epigraph{This chapter is dedicated to Forbidden Planet, the British
  chain of science fiction and fantasy books, comics, toys and
  videos. Forbidden Planet has stores up and down the UK, and also
  sports outposts in Manhattan and Dublin, Ireland. It's dangerous to
  set foot in a Forbidden Planet -- rarely do I escape with my wallet
  intact. Forbidden Planet really leads the pack in bringing the
  gigantic audience for TV and movie science fiction into contact with
  science fiction books -- something that's absolutely critical to the
  future of the field.}
{Forbidden Planet, UK, Dublin and New York City:
\url{http://www.forbiddenplanet.co.uk}}

Ms Galvez's smile was wide.

``Does anyone know what that comes from?''

A bunch of people chorused, ``The Declaration of Independence.''

I nodded.

``Why did you read that to us, Marcus?''

``Because it seems to me that the founders of this country said that
governments should only last for so long as we believe that they're
working for us, and if we stop believing in them, we should overthrow
them. That's what it says, right?''

Charles shook his head. ``That was hundreds of years ago!'' he
said. ``Things are different now!''

``What's different?''

``Well, for one thing, we don't have a king anymore. They were talking
about a government that existed because some old jerk's
great-great-great-grandfather believed that God put him in charge and
killed everyone who disagreed with him. We have a democratically
elected government --''

``I didn't vote for them,'' I said.

``So that gives you the right to blow up a building?''

``What? Who said anything about blowing up a building? The Yippies and
hippies and all those people believed that the government no longer
listened to them -- look at the way people who tried to sign up voters
in the South were treated! They were beaten up, arrested --''

``Some of them were killed,'' Ms Galvez said. She held up her hands and
waited for Charles and me to sit down. ``We're almost out of time for
today, but I want to commend you all on one of the most interesting
classes I've ever taught. This has been an excellent discussion and
I've learned much from you all. I hope you've learned from each other,
too. Thank you all for your contributions.

``I have an extra-credit assignment for those of you who want a little
challenge. I'd like you to write up a paper comparing the political
response to the anti-war and civil rights movements in the Bay Area to
the present day civil rights responses to the War on Terror. Three
pages minimum, but take as long as you'd like. I'm interested to see
what you come up with.''

The bell rang a moment later and everyone filed out of the class. I
hung back and waited for Ms Galvez to notice me.

``Yes, Marcus?''

``That was amazing,'' I said. ``I never knew all that stuff about the
sixties.''

``The seventies, too. This place has always been an exciting place to
live in politically charged times. I really liked your reference to
the Declaration -- that was very clever.''

``Thanks,'' I said. ``It just came to me. I never really appreciated what
those words all meant before today.''

``Well, those are the words every teacher loves to hear, Marcus,'' she
said, and shook my hand. ``I can't wait to read your paper.''

\fancybreak{\#}

I bought the Emma Goldman poster on the way home and stuck it up over
my desk, tacked over a vintage black-light poster. I also bought a
NEVER TRUST t-shirt that had a photoshop of Grover and Elmo kicking
the grownups Gordon and Susan off Sesame Street. It made me laugh. I
later found out that there had already been about six photoshop
contests for the slogan online in places like Fark and Worth1000 and
B3ta and there were hundreds of ready-made pics floating around to go
on whatever merch someone churned out.

Mom raised an eyebrow at the shirt, and Dad shook his head and
lectured me about not looking for trouble. I felt a little vindicated
by his reaction.

Ange found me online again and we IM-flirted until late at night
again. The white van with the antennas came back and I switched off my
Xbox until it had passed. We'd all gotten used to doing that.

Ange was really excited by this party. It looked like it was going to
be monster. There were so many bands signed up they were talking about
setting up a B-stage for the secondary acts.

\edialog{How'd they get a permit to blast sound all night in that
  park?  There's houses all around there}

\edialog{Per-mit? What is ``per-mit''? Tell me more of your hu-man
  per-mit.}

\edialog{Woah, it's illegal?}

\edialog{Um, hello? \emph{You're} worried about breaking the law?}

\edialog{Fair point}

\edialog{LOL}

I felt a little premonition of nervousness though. I mean, I was
taking this perfectly awesome girl out on a date that weekend -- well,
she was taking me, technically -- to an illegal rave being held in the
middle of a busy neighborhood.

It was bound to be interesting at least.

\fancybreak{\#}

Interesting.

People started to drift into Dolores Park through the long Saturday
afternoon, showing up among the ultimate frisbee players and the
dog-walkers. Some of them played frisbee or walked dogs. It wasn't
really clear how the concert was going to work, but there were a lot
of cops and undercovers hanging around. You could tell the undercovers
because, like Zit and Booger, they had Castro haircuts and Nebraska
physiques: tubby guys with short hair and untidy mustaches. They
drifted around, looking awkward and uncomfortable in their giant
shorts and loose-fitting shirts that no-doubt hung down to cover the
chandelier of gear hung around their midriffs.

Dolores Park is pretty and sunny, with palm trees, tennis courts, and
lots of hills and regular trees to run around on, or hang out
on. Homeless people sleep there at night, but that's true everywhere
in San Francisco.

I met Ange down the street, at the anarchist bookstore. That had been
my suggestion. In hindsight, it was a totally transparent move to seem
cool and edgy to this girl, but at the time I would have sworn that I
picked it because it was a convenient place to meet up. She was
reading a book called \emph{Up Against the Wall Motherf\_\_\_\_\_r} when I got
there.

``Nice,'' I said. ``You kiss your mother with that mouth?''

``Your mama don't complain,'' she said. ``Actually, it's a history of a
group of people like the Yippies, but from New York. They all used
that word as their last names, like 'Ben M-F.' The idea was to have a
group out there, making news, but with a totally unprintable
name. Just to screw around with the news-media. Pretty funny, really.''
She put the book back on the shelf and now I wondered if I should hug
her. People in California hug to say hello and goodbye all the
time. Except when they don't. And sometimes they kiss on the
cheek. It's all very confusing.

She settled it for me by grabbing me in a hug and tugging my head down
to her, kissing me hard on the cheek, then blowing a fart on my
neck. I laughed and pushed her away.

``You want a burrito?'' I asked.

``Is that a question or a statement of the obvious?''

``Neither. It's an order.''

I bought some funny stickers that said THIS PHONE IS TAPPED which were
the right size to put on the receivers on the pay phones that still
lined the streets of the Mission, it being the kind of neighborhood
where you got people who couldn't necessarily afford a cellphone.

We walked out into the night air. I told Ange about the scene at the
park when I left.

``I bet they have a hundred of those trucks parked around the block,''
she said. ``The better to bust you with.''

``Um.'' I looked around. ``I sort of hoped that you would say something
like, 'Aw, there's no chance they'll do anything about it.'''

``I don't think that's really the idea. The idea is to put a lot of
civilians in a position where the cops have to decide, are we going to
treat these ordinary people like terrorists? It's a little like the
jamming, but with music instead of gadgets. You jam, right?''

Sometimes I forget that all my friends don't know that Marcus and
M1k3y are the same person. ``Yeah, a little,'' I said.

``This is like jamming with a bunch of awesome bands.''

``I see.''

Mission burritos are an institution. They are cheap, giant and
delicious. Imagine a tube the size of a bazooka shell, filled with
spicy grilled meat, guacamole, salsa, tomatoes, refried beans, rice,
onions and cilantro. It has the same relationship to Taco Bell that a
Lamborghini has to a Hot Wheels car.

There are about two hundred Mission burrito joints. 
\discretionary{They}{are}{They're} all
heroically ugly, with uncomfortable seats, minimal decor -- faded
Mexican tourist office posters and electrified framed Jesus and Mary
holograms -- and loud mariachi music. The thing that distinguishes
them, mostly, is what kind of exotic meat they fill their wares
with. The really authentic places have brains and tongue, which I
never order, but it's nice to know it's there.

The place we went to had both brains and tongue, which we didn't
order. I got carne asada and she got shredded chicken and we each got
a big cup of horchata.

As soon as we sat down, she unrolled her burrito and took a little
bottle out of her purse. It was a little stainless-steel aerosol
canister that looked for all the world like a pepper-spray
self-defense unit. She aimed it at her burrito's exposed guts and
misted them with a fine red oily spray. I caught a whiff of it and my
throat closed and my eyes watered.

``What the hell are you doing to that poor, defenseless burrito?''

She gave me a wicked smile. ``I'm a spicy food addict,'' she said. ``This
is capsaicin oil in a mister.''

``Capsaicin --''

``Yeah, the stuff in pepper spray. This is like pepper spray but
slightly more dilute. And way more delicious. Think of it as Spicy
Cajun Visine if it helps.''

My eyes burned just thinking of it.

``You're kidding,'' I said. ``You are so not going to eat that.''

Her eyebrows shot up. ``That sounds like a challenge, sonny. You just
watch me.''

She rolled the burrito up as carefully as a stoner rolling up a joint,
tucking the ends in, then re-wrapping it in tinfoil. She peeled off
one end and brought it up to her mouth, poised with it just before her
lips.

Right up to the time she bit into it, I couldn't believe that she was
going to do it. I mean, that was basically an anti-personnel weapon
she'd just slathered on her dinner.

She bit into it. Chewed. Swallowed. Gave every impression of having a
delicious dinner.

``Want a bite?'' she said, innocently.

``Yeah,'' I said. I like spicy food. I always order the curries with
four chilies next to them on the menu at the Pakistani places.

I peeled back more foil and took a big bite.

Big mistake.

You know that feeling you get when you take a big bite of horseradish
or wasabi or whatever, and it feels like your sinuses are closing at
the same time as your windpipe, filling your head with trapped,
nuclear-hot air that tries to batter its way out through your watering
eyes and nostrils? That feeling like steam is about to pour out of
your ears like a cartoon character?

This was a lot worse.

This was like putting your hand on a hot stove, only it's not your
hand, it's the entire inside of your head, and your esophagus all the
way down to your stomach. My entire body sprang out in a sweat and I
choked and choked.

Wordlessly, she passed me my horchata and I managed to get the straw
into my mouth and suck hard on it, gulping down half of it in one go.

``So there's a scale, the Scoville scale, that we chili-fanciers use to
talk about how spicy a pepper is. Pure capsaicin is about 15 million
Scovilles. Tabasco is about 2,500. Pepper spray is a healthy three
million. This stuff is a puny 100,000, about as hot as a mild Scotch
Bonnet Pepper. I worked up to it in about a year. Some of the real
hardcore can get up to a half million or so, two hundred times hotter
than Tabasco. That's pretty freaking hot. At Scoville temperatures
like that, your brain gets totally awash in endorphins. It's a better
body-stone than hash. And it's good for you.''

I was getting my sinuses back now, able to breathe without gasping.

``Of course, you get a ferocious ring of fire when you go to the john,''
she said, winking at me.

Yowch.

``You are insane,'' I said.

``Fine talk from a man whose hobby is building and smashing laptops,''
she said.

``Touche,'' I said and touched my forehead.

``Want some?'' She held out her mister.

``Pass,'' I said, quickly enough that we both laughed.

When we left the restaurant and headed for Dolores park, she put her
arm around my waist and I found that she was just the right height for
me to put my arm around her shoulders. That was new. I'd never been a
tall guy, and the girls I'd dated had all been my height -- teenaged
girls grow faster than guys, which is a cruel trick of nature. It was
nice. It felt nice.

We turned the corner on 20th Street and walked up toward
Dolores. Before we'd taken a single step, we could feel the buzz. It
was like the hum of a million bees. There were lots of people
streaming toward the park, and when I looked toward it, I saw that it
was about a hundred times more crowded than it had been when I went to
meet Ange.

That sight made my blood run hot. It was a beautiful cool night and we
were about to party, really party, party like there was no
tomorrow. ``Eat drink and be merry, for tomorrow we die.''

Without saying anything we both broke into a trot. There were lots of
cops, with tense faces, but what the hell were they going to do? There
were a \emph{lot} of people in the park. I'm not so good at counting
crowds. The papers later quoted organizers as saying there were 20,000
people; the cops said 5,000. Maybe that means there were 12,500.

Whatever. It was more people than I'd ever stood among, as part of an
unscheduled, unsanctioned, \emph{illegal} event.

We were among them in an instant. I can't swear to it, but I don't
think there was anyone over 25 in that press of bodies. Everyone was
smiling. Some young kids were there, 10 or 12, and that made me feel
better. No one would do anything too stupid with kids that little in
the crowd. No one wanted to see little kids get hurt. This was just
going to be a glorious spring night of celebration.

I figured the thing to do was push in towards the tennis courts. We
threaded our way through the crowd, and to stay together we took each
other's hands. Only staying together didn't require us to intertwine
fingers. That was strictly for pleasure. It was very pleasurable.

The bands were all inside the tennis courts, with their guitars and
mixers and keyboards and even a drum kit. Later, on Xnet, I found a
Flickr stream of them smuggling all this stuff in, piece by piece, in
gym bags and under their coats. Along with it all were huge speakers,
the kind you see in automotive supply places, and among them, a stack
of \ldots car batteries. I laughed. Genius! That was how they were going to
power their stacks. From where I stood, I could see that they were
cells from a hybrid car, a Prius. Someone had gutted an eco-mobile to
power the night's entertainment. The batteries continued outside the
courts, stacked up against the fence, tethered to the main stack by
wires threaded through the chain-link. I counted -- 200 batteries!
Christ! Those things weighed a ton, too.

There's no way they organized this without email and wikis and mailing
lists. And there's no way people this smart would have done that on
the public Internet. This had all taken place on the Xnet, I'd bet my
boots on it.

We just kind of bounced around in the crowd for a while as the bands
tuned up and conferred with one another. I saw Trudy Doo from a
distance, in the tennis courts. She looked like she was in a cage,
like a pro wrestler. She was wearing a torn wife-beater and her hair
was in long, fluorescent pink dreads down to her waist. She was
wearing army camouflage pants and giant gothy boots with steel
over-toes. As I watched, she picked up a heavy motorcycle jacket, worn
as a catcher's mitt, and put it on like armor. It probably was armor,
I realized.

I tried to wave to her, to impress Ange I guess, but she didn't see me
and I kind of looked like a spazz so I stopped. The energy in the
crowd was amazing. You hear people talk about ``vibes'' and ``energy'' for
big groups of people, but until you've experienced it, you probably
think it's just a figure of speech.

It's not. It's the smiles, infectious and big as watermelons, on every
face. Everyone bopping a little to an unheard rhythm, shoulders
rocking. Rolling walks. Jokes and laughs. The tone of every voice
tight and excited, like a firework about to go off. And you can't help
but be a part of it. Because you are.

By the time the bands kicked off, I was utterly stoned on
crowd-vibe. The opening act was some kind of Serbian turbo-folk, which
I couldn't figure out how to dance to. I know how to dance to exactly
two kinds of music: trance (shuffle around and let the music move you)
and punk (bash around and mosh until you get hurt or exhausted or
both). The next act was Oakland hip-hoppers, backed by a thrash metal
band, which is better than it sounds. Then some bubble-gum pop. Then
Speedwhores took the stage, and Trudy Doo stepped up to the mic.

``My name is Trudy Doo and you're an idiot if you trust me. I'm thirty
two and it's too late for me. I'm lost. I'm stuck in the old way of
thinking. I still take my freedom for granted and let other people
take it away from me. You're the first generation to grow up in Gulag
America, and you know what your freedom is worth to the last goddamned
cent!''

The crowd roared. She was playing fast little skittery nervous chords
on her guitar and her bass player, a huge fat girl with a dykey
haircut and even bigger boots and a smile you could open beer bottles
with was laying it down fast and hard already. I wanted to bounce. I
bounced. Ange bounced with me. We were sweating freely in the evening,
which reeked of perspiration and pot smoke. Warm bodies crushed in on
all sides of us. They bounced too.

``Don't trust anyone over 25!'' she shouted.

We roared. We were one big animal throat, roaring.

``Don't trust anyone over 25!''

``\emph{Don't trust anyone over 25!}''

``Don't trust anyone over 25!''

``\emph{Don't trust anyone over 25!}''

``Don't trust anyone over 25!''

``\emph{Don't trust anyone over 25!}''

She banged some hard chords on her guitar and the other guitarist, a
little pixie of a girl whose face bristled with piercings, jammed in,
going wheedle-dee-wheedle-dee-dee up high, past the twelfth fret.

``It's our goddamned city! It's our goddamned country. No terrorist can
take it from us for so long as we're free. Once we're not free, the
terrorists win! Take it back! Take it back! You're young enough and
stupid enough not to know that you can't possibly win, so you're the
only ones who can lead us to victory! \emph{Take it back!}''

``TAKE IT BACK!'' we roared. She jammed down hard on her guitar. We
roared the note back and then it got really really LOUD.

\fancybreak{\#}

I danced until I was so tired I couldn't dance another step. Ange
danced alongside of me. Technically, we were rubbing our sweaty bodies
against each other for several hours, but believe it or not, I totally
wasn't being a horn-dog about it. We were dancing, lost in the godbeat
and the thrash and the screaming -- TAKE IT BACK! TAKE IT BACK!

When I couldn't dance anymore, I grabbed her hand and she squeezed
mine like I was keeping her from falling off a building. She dragged
me toward the edge of the crowd, where it got thinner and cooler. Out
there, on the edge of Dolores Park, we were in the cool air and the
sweat on our bodies went instantly icy. We shivered and she threw her
arms around my waist. ``Warm me,'' she commanded. I didn't need a
hint. I hugged her back. Her heart was an echo of the fast beats from
the stage -- breakbeats now, fast and furious and wordless.

She smelled of sweat, a sharp tang that smelled great. I knew I
smelled of sweat too. My nose was pointed into the top of her head,
and her face was right at my collarbone. She moved her hands to my
neck and tugged.

``Get down here, I didn't bring a stepladder,'' is what she said and I
tried to smile, but it's hard to smile when you're kissing.

Like I said, I'd kissed three girls in my life. Two of them had never
kissed anyone before. One had been dating since she was 12. She had
issues.

None of them kissed like Ange. She made her whole mouth soft, like the
inside of a ripe piece of fruit, and she didn't jam her tongue in my
mouth, but slid it in there, and sucked my lips into her mouth at the
same time, so it was like my mouth and hers were merging. I heard
myself moan and I grabbed her and squeezed her harder.

Slowly, gently, we lowered ourselves to the grass. We lay on our sides
and clutched each other, kissing and kissing. The world disappeared so
there was only the kiss.

My hands found her butt, her waist. The edge of her t-shirt. Her warm
tummy, her soft navel. They inched higher. She moaned too.

``Not here,'' she said. ``Let's move over there.'' She pointed across the
street at the big white church that gives Mission Dolores Park and the
Mission its name. Holding hands, moving quickly, we crossed to the
church. It had big pillars in front of it. She put my back up against
one of them and pulled my face down her hers again. My hands went
quickly and boldly back to her shirt. I slipped them up her front.

``It undoes in the back,'' she whispered into my mouth. I had a boner
that could cut glass. I moved my hands around to her back, which was
strong and broad, and found the hook with my fingers, which were
trembling. I fumbled for a while, thinking of all those jokes about
how bad guys are at undoing bras. I was bad at it. Then the hook
sprang free. She gasped into my mouth. I slipped my hands around,
feeling the wetness of her armpits -- which was sexy and not at all
gross for some reason -- and then brushed the sides of her breasts.

That's when the sirens started.

They were louder than anything I'd ever heard. A sound like a physical
sensation, like something blowing you off your feet. A sound as loud
as your ears could process, and then louder.

``DISPERSE IMMEDIATELY,'' a voice said, like God rattling in my skull.

``THIS IS AN ILLEGAL GATHERING. DISPERSE IMMEDIATELY.''

The band had stopped playing. The noise of the crowd across the street
changed. It got scared. Angry.

I heard a click as the PA system of car-speakers and car-batteries in
the tennis courts powered up.

``TAKE IT BACK!''

It was a defiant yell, like a sound shouted into the surf or screamed
off a cliff.

``TAKE IT BACK!''

The crowd \emph{growled}, a sound that made the hairs on the back of my
neck stand up.

``\emph{TAKE IT BACK}!'' they chanted. ``TAKE IT BACK TAKE IT BACK TAKE IT
BACK!''

The police moved in in lines, carrying plastic shields, wearing Darth
Vader helmets that covered their faces. Each one had a black truncheon
and infra-red goggles. They looked like soldiers out of some
futuristic war movie. They took a step forward in unison and every one
of them banged his truncheon on his shield, a cracking noise like the
earth splitting. Another step, another crack. They were all around the
park and closing in now.

``DISPERSE IMMEDIATELY,'' the voice of God said again. There were
helicopters overhead now. No floodlights, though. The infrared
goggles, right. Of course. They'd have infrared scopes in the sky,
too. I pulled Ange back against the doorway of the church, tucking us
back from the cops and the choppers.

``TAKE IT BACK!'' the PA roared. It was Trudy Doo's rebel yell and I
heard her guitar thrash out some chords, then her drummer playing,
then that big deep bass.

``TAKE IT BACK!'' the crowd answered, and they boiled out of the park at
the police lines.

I've never been in a war, but now I think I know what it must be
like. What it must be like when scared kids charge across a field at
an opposing force, knowing what's coming, running anyway, screaming,
hollering.

``DISPERSE IMMEDIATELY,'' the voice of God said. It was coming from
trucks parked all around the park, trucks that had swung into place in
the last few seconds.

That's when the mist fell. It came out of the choppers, and we just
caught the edge of it. It made the top of my head feel like it was
going to come off. It made my sinuses feel like they were being
punctured with ice-picks. It made my eyes swell and water, and my
throat close.

Pepper spray. Not 100 thousand Scovilles. A million and a half. They'd
gassed the crowd.

I didn't see what happened next, but I heard it, over the sound of
both me and Ange choking and holding each other. First the choking,
retching sounds. The guitar and drums and bass crashed to a halt. Then
coughing.

Then screaming.

The screaming went on for a long time. When I could see again, the
cops had their scopes up on their foreheads and the choppers were
flooding Dolores Park with so much light it looked like
daylight. Everyone was looking at the Park, which was good news,
because when the lights went up like that, we were totally visible.

``What do we do?'' Ange said. Her voice was tight, scared. I didn't
trust myself to speak for a moment. I swallowed a few times.

``We walk away,'' I said. ``That's all we can do. Walk away. Like we were
just passing by. Down to Dolores and turn left and up towards 16th
Street. Like we're just passing by. Like this is none of our
business.''

``That'll never work,'' she said.

``It's all I've got.''

``You don't think we should try to run for it?''

``No,'' I said. ``If we run, they'll chase us. Maybe if we walk, they'll
figure we haven't done anything and let us alone. They have a lot of
arrests to make. They'll be busy for a long time.''

The park was rolling with bodies, people and adults clawing at their
faces and gasping. The cops dragged them by the armpits, then lashed
their wrists with plastic cuffs and tossed them into the trucks like
rag-dolls.

``OK?'' I said.

``OK,'' she said.

And that's just what we did. Walked, holding hands, quickly and
business-like, like two people wanting to avoid whatever trouble
someone else was making. The kind of walk you adopt when you want to
pretend you can't see a panhandler, or don't want to get involved in a
street-fight.

It worked.

We reached the corner and turned and kept going. Neither of us dared
to speak for two blocks. Then I let out a gasp of air I hadn't know
I'd been holding in.

We came to 16th Street and turned down toward Mission Street. Normally
that's a pretty scary neighborhood at 2AM on a Saturday night. That
night it was a relief -- same old druggies and hookers and dealers and
drunks. No cops with truncheons, no gas.

``Um,'' I said as we breathed in the night air. ``Coffee?''

``Home,'' she said. ``I think home for now. Coffee later.''

``Yeah,'' I agreed. She lived up in Hayes Valley. I spotted a taxi
rolling by and I hailed it. That was a small miracle -- there are
hardly any cabs when you need them in San Francisco.

``Have you got cabfare home?''

``Yeah,'' she said. The cab-driver looked at us through his window. I
opened the back door so he wouldn't take off.

``Good night,'' I said.

She put her hands behind my head and pulled my face toward her. She
kissed me hard on the mouth, nothing sexual in it, but somehow more
intimate for that.

``Good night,'' she whispered in my ear, and slipped into the taxi.

Head swimming, eyes running, a burning shame for having left all those
Xnetters to the tender mercies of the DHS and the SFPD, I set off for
home.

\fancybreak{\#}

Monday morning, Fred Benson was standing behind Ms Galvez's desk.

``Ms Galvez will no longer be teaching this class,'' he said, once we'd
taken our seats. He had a self-satisfied note that I recognized
immediately. On a hunch, I checked out Charles. He was smiling like it
was his birthday and he'd been given the best present in the world.

I put my hand up.

``Why not?''

``It's Board policy not to discuss employee matters with anyone except
the employee and the disciplinary committee,'' he said, without even
bothering to hide how much he enjoyed saying it.

``We'll be beginning a new unit today, on national security. Your
SchoolBooks have the new texts. Please open them and turn to the first
screen.''

The opening screen was emblazoned with a DHS logo and the title: WHAT
EVERY AMERICAN SHOULD KNOW ABOUT HOMELAND SECURITY.

I wanted to throw my SchoolBook on the floor.

\fancybreak{\#}

I'd made arrangements to meet Ange at a cafe in her neighborhood after
school. I jumped on the BART and found myself sitting behind two guys
in suits. They were looking at the San Francisco Chronicle, which
featured a full-page post-mortem on the ``youth riot'' in Mission
Dolores Park. They were tutting and clucking over it. Then one said to
the other, ``It's like they're brainwashed or something. Christ, were
we ever that stupid?''

I got up and moved to another seat.

\chapter{Chapter 13}

\epigraph{This chapter is dedicated to Books-A-Million, a chain of
  gigantic bookstores spread across the USA. I first encountered
  Books-A-Million while staying at a hotel in Terre Haute, Indiana (I
  was giving a speech at the Rose Hulman Institute of Technology later
  that day). The store was next to my hotel and I really needed some
  reading material -- I'd been on the road for a solid month and I'd
  read everything in my suitcase, and I had another five cities to go
  before I headed home. As I stared intently at the shelves, a clerk
  asked me if I needed any help. Now, I've worked at bookstores
  before, and a knowledgeable clerk is worth her weight in gold, so I
  said sure, and started to describe my tastes, naming authors I'd
  enjoyed. The clerk smiled and said, ``I've got just the book for
  you,'' and proceeded to take down a copy of my first novel, Down and
  Out in the Magic Kingdom. I busted out laughing, introduced myself,
  and had an absolutely lovely chat about science fiction that almost
  made me late to give my speech!}
{Books-A-Million
\url{http://www.booksamillion.com/ncom/books?&isbn=0765319853}}

``They're total whores,'' Ange said, spitting the word out. ``In fact,
that's an insult to hardworking whores everywhere.
\discretionary{They}{are}{They're}, they're
\emph{profiteers.}''

We were looking at a stack of newspapers we'd picked up and brought to
the cafe. They all contained ``reporting'' on the party in Dolores Park
and to a one, they made it sound like a drunken, druggy orgy of kids
who'd attacked the cops. \emph{USA Today} described the cost of the ``riot''
and included the cost of washing away the pepper-spray residue from
the gas-bombing, the rash of asthma attacks that clogged the city's
emergency rooms, and the cost of processing the eight hundred arrested
``rioters.''

No one was telling our side.

``Well, the Xnet got it right, anyway,'' I said. I'd saved a bunch of
the blogs and videos and photostreams to my phone and I showed them to
her. They were first-hand accounts from people who'd been gassed, and
beaten up. The video showed us all dancing, having fun, showed the
peaceful political speeches and the chant of ``Take It Back'' and Trudy
Doo talking about us being the only generation that could believe in
fighting for our freedoms.

``We need to make people know about this,'' she said.

``Yeah,'' I said, glumly. ``That's a nice theory.''

``Well, why do you think the press doesn't ever publish our side?''

``You said it, they're whores.''

``Yeah, but whores do it for the money. They could sell more papers and
commercials if they had a controversy. All they have now is a crime --
controversy is much bigger.''

``OK, point taken. So why don't they do it? Well, reporters can barely
search regular blogs, let alone keep track of the Xnet. It's not as if
that's a real adult-friendly place to be.''

``Yeah,'' she said. ``Well, we can fix that, right?''

``Huh?''

``Write it all up. Put it in one place, with all the links. A single
place where you can go that's intended for the press to find it and
get the whole picture. Link it to the HOWTOs for Xnet. Internet users
can get to the Xnet, provided they don't care about the DHS finding
out what they've been surfing.''

``You think it'll work?''

``Well, even if it doesn't, it's something positive to do.''

``Why would they listen to us, anyway?''

``Who wouldn't listen to M1k3y?''

I put down my coffee. I picked up my phone and slipped it into my
pocket. I stood up, turned on my heel, and walked out of the cafe. I
picked a direction at random and kept going. My face felt tight, the
blood gone into my stomach, which churned.

\emph{They know who you are,} I thought. \emph{They know who M1k3y
  is.} That was it. If Ange had figured it out, the DHS had too. I was
doomed. I had known that since they let me go from the DHS truck, that
someday they'd come and arrest me and put me away forever, send me to
wherever Darryl had gone.

It was all over.

She nearly tackled me as I reached Market Street. She was out of
breath and looked furious.

``What the \emph{hell} is your problem, mister?''

I shook her off and kept walking. It was all over.

She grabbed me again. ``Stop it, Marcus, you're scaring me. Come on,
talk to me.''

I stopped and looked at her. She blurred before my eyes. I couldn't
focus on anything. I had a mad desire to jump into the path of a Muni
trolley as it tore past us, down the middle of the road. Better to die
than to go back.

``Marcus!'' She did something I'd only seen people do in the movies. She
slapped me, a hard crack across the face. ``Talk to me, dammit!''

I looked at her and put my hand to my face, which was stinging hard.

``No one is supposed to know who I am,'' I said. ``I can't put it any
more simply. If you know, it's all over. Once other people know, it's
all over.''

``Oh god, I'm sorry. Look, I only know because, well, because I
blackmailed Jolu. After the party I stalked you a little, trying to
figure out if you were the nice guy you seemed to be or a secret
axe-murderer. I've known Jolu for a long time and when I asked him
about you, he gushed like you were the Second Coming or something, but
I could hear that there was something he wasn't telling me. I've known
Jolu for a long time. He dated my older sister at computer camp when
he was a kid. I have some really good dirt on him. I told him I'd go
public with it if he didn't tell me.''

``So he told you.''

``No,'' she said. ``He told me to go to hell. Then I told him something
about me. Something I'd never told anyone else.''

``What?''

She looked at me. Looked around. Looked back at me. ``OK. I won't swear
you to secrecy because what's the point? Either I can trust you or I
can't.

``Last year, I --'' she broke off. ``Last year, I stole the standardized
tests and published them on the net. It was just a lark. I happened to
be walking past the principal's office and I saw them in his safe, and
the door was hanging open. I ducked into his office -- there were six
sets of copies and I just put one into my bag and took off again. When
I got home, I scanned them all and put them up on a Pirate Party
server in Denmark.''

``That was \emph{you}?'' I said.

She blushed. ``Um. Yeah.''

``Holy crap!'' I said. It had been huge news. The Board of Education
said that its No Child Left Behind tests had cost tens of millions of
dollars to produce and that they'd have to spend it all over again now
that they'd had the leak. They called it ``edu-terrorism.'' The news had
speculated endlessly about the political motivations of the leaker,
wondering if it was a teacher's protest, or a student, or a thief, or
a disgruntled government contractor.

``That was YOU?''

``It was me,'' she said.

``And you told Jolu this --''

``Because I wanted him to be sure that I would keep the secret. If he
knew \emph{my} secret, then he'd have something he could use to put me in
jail if I opened my trap. Give a little, get a little. Quid pro quo,
like in Silence of the Lambs.''

``And he told you.''

``No,'' she said. ``He didn't.''

``But --''

``Then I told him how into you I was. How I was planning to totally
make an idiot of myself and throw myself at you. \emph{Then} he told me.''

I couldn't think of anything to say then. I looked down at my
toes. She grabbed my hands and squeezed them.

``I'm sorry I squeezed it out of him. It was your decision to tell me,
if you were going to tell me at all. I had no business --''

``No,'' I said. Now that I knew how she'd found out, I was starting to
calm down. ``No, it's good you know. \emph{You}.''

``Me,'' she said. ``Li'l ol' me.''

``OK, I can live with this. But there's one other thing.''

``What?''

``There's no way to say this without sounding like a jerk, so I'll just
say it. People who date each other -- or whatever it is we're doing
now -- they split up. When they split up, they get angry at each
other. Sometimes even hate each other. It's really cold to think about
that happening between us, but you know, we've got to think about it.''

``I solemnly promise that there is nothing you could ever do to me that
would cause me to betray your secret. Nothing. Screw a dozen
cheerleaders in my bed while my mother watches. Make me listen to
Britney Spears. Rip off my laptop, smash it with hammers and soak it
in sea-water. I promise. Nothing. Ever.''

I whooshed out some air.

``Um,'' I said.

``Now would be a good time to kiss me,'' she said, and turned her face
up.

\fancybreak{\#}

M1k3y's next big project on the Xnet was putting together the ultimate
roundup of reports of the DON'T TRUST party at Dolores Park. I put
together the biggest, most bad-ass site I could, with sections showing
the action by location, by time, by category -- police violence,
dancing, aftermath, singing. I uploaded the whole concert.

It was pretty much all I worked on for the rest of the night. And the
next night. And the next.

My mailbox overflowed with suggestions from people. They sent me dumps
off their phones and their pocket-cameras. Then I got an email from a
name I recognized -- Dr Eeevil (three ``e''s), one of the prime
maintainers of ParanoidLinux.

\edialog{M1k3y}

\edialog{I have been watching your Xnet experiment with great
  interest. Here in Germany, we have much experience with what happens
  with a government that gets out of control.}

\edialog{One thing you should know is that every camera has a unique
  ``noise signature'' that can be used to later connect a picture with a
  camera. That means that the photos you're republishing on your site
  could potentially be used to identify the photographers, should they
  later be picked up for something else.}

\edialog{Luckily, it's not hard to strip out the signatures, if you
  care to. There's a utility on the ParanoidLinux distro you're using
  that does this -- it's called photonomous, and you'll find it in
  /usr/bin. Just read the man pages for documentation. It's simple
  though.}

\edialog{Good luck with what you're doing. Don't get caught. Stay
  free. Stay paranoid.}

\edialog{Dr Eeevil}

I de-fingerprintized all the photos I'd posted and put them back up,
along with a note explaining what Dr Eeevil had told me, warning
everyone else to do the same. We all had the same basic ParanoidXbox
install, so we could all anonymize our pictures. There wasn't anything
I could do about the photos that had already been downloaded and
cached, but from now on we'd be smarter.

That was all the thought I gave the matter that night, until I got
down to breakfast the next morning and Mom had the radio on, playing
the NPR morning news.

``Arabic news agency Al-Jazeera is running pictures, video and
first-hand accounts of last weekend's youth riot in Mission Dolores
park,'' the announcer said as I was drinking a glass of orange juice. I
managed not to spray it across the room, but I \emph{did} choke a little.

``Al-Jazeera reporters claim that these accounts were published on the
so-called 'Xnet,' a clandestine network used by students and Al-Quaeda
sympathizers in the Bay Area. This network's existence has long been
rumored, but today marks its first mainstream mention.''

Mom shook her head. ``Just what we need,'' she said. ``As if the police
weren't bad enough. Kids running around, pretending to be guerillas
and giving them the excuse to really crack down.''

``The Xnet weblogs have carried hundreds of reports and multimedia
files from young people who attended the riot and allege that they
were gathered peacefully until the police attacked \emph{them}. Here is one
of those accounts.

``'All we were doing was dancing. I brought my little brother. Bands
played and we talked about freedom, about how we were losing it to
these jerks who say they hate terrorists but who attack us though
we're not terrorists we're Americans. I think they hate freedom, not
us.

``We danced and the bands played and it was all fun and good and then
the cops started shouting at us to disperse. We all shouted take it
back! Meaning take America back. The cops gassed us with pepper
spray. My little brother is twelve. He missed three days of school. My
stupid parents say it was my fault. How about the police? We pay them
and they're supposed to protect us but they gassed us for no good
reason, gassed us like they gas enemy soldiers.'

``Similar accounts, including audio and video, can be found on
Al-Jazeera's website and on the Xnet. You can find directions for
accessing this Xnet on NPR's homepage.''

Dad came down.

``Do you use the Xnet?'' he said. He looked intensely at my face. I felt
myself squirm.

``It's for video-games,'' I said. ``That's what most people use it
for. It's just a wireless network. It's what everyone did with those
free Xboxes they gave away last year.''

He glowered at me. ``Games? Marcus, you don't realize it, but you're
providing cover for people who plan on attacking and destroying this
country. I don't want to see you using this Xnet. Not anymore. Do I
make myself clear?''

I wanted to argue. Hell, I wanted to shake him by the shoulders. But I
didn't. I looked away. I said, ``Sure, Dad.'' I went to school.

\fancybreak{\#}

At first I was relieved when I discovered that they weren't going to
leave Mr Benson in charge of my social studies class. But the woman
they found to replace him was my worst nightmare.

She was young, just about 28 or 29, and pretty, in a wholesome kind of
way. She was blonde and spoke with a soft southern accent when she
introduced herself to us as Mrs Andersen. That set off alarm bells
right away. I didn't know \emph{any} women under the age of sixty that
called themselves ``Mrs.''

But I was prepared to overlook it. She was young, pretty, she sounded
nice. She would be OK.

She wasn't OK.

``Under what circumstances should the federal government be prepared to
suspend the Bill of Rights?'' she said, turning to the blackboard and
writing down a row of numbers, one through ten.

``Never,'' I said, not waiting to be called on. This was
easy. ``Constitutional rights are absolute.''

``That's not a very sophisticated view.'' She looked at her
seating-plan. ``Marcus. For example, say a policeman conducts an
improper search -- he goes beyond the stuff specified in his
warrant. He discovers compelling evidence that a bad guy killed your
father. It's the only evidence that exists. Should the bad guy go
free?''

I knew the answer to this, but I couldn't really explain it. ``Yes,'' I
said, finally. ``But the police shouldn't conduct improper searches --''

``Wrong,'' she said. ``The proper response to police misconduct is
disciplinary action against the police, not punishing all of society
for one cop's mistake.'' She wrote ``Criminal guilt'' under point one on
the board.

``Other ways in which the Bill of Rights can be superseded?''

Charles put his hand up. ``Shouting fire in a crowded theater?''

``Very good --'' she consulted the seating plan -- ``Charles. There are
many instances in which the First Amendment is not absolute. Let's
list some more of those.''

Charles put his hand up again. ``Endangering a law enforcement
officer.''

``Yes, disclosing the identity of an undercover policeman or
intelligence officer. Very good.'' She wrote it down. ``Others?''

``National security,'' Charles said, not waiting for her to call on him
again. ``Libel. Obscenity. Corruption of minors. Child
porn. Bomb-making recipes.'' Mrs Andersen wrote these down fast, but
stopped at child porn. ``Child porn is just a form of obscenity.''

I was feeling sick. This was not what I'd learned or believed about my
country. I put my hand up.

``Yes, Marcus?''

``I don't get it. You're making it sound like the Bill of Rights is
optional. It's the Constitution. We're supposed to follow it
absolutely.''

``That's a common oversimplification,'' she said, giving me a fake
smile. ``But the fact of the matter is that the framers of the
Constitution intended it to be a living document that was revised over
time. They understood that the Republic wouldn't be able to last
forever if the government of the day couldn't govern according to the
needs of the day. They never intended the Constitution to be looked on
like religious doctrine. After all, they came here fleeing religious
doctrine.''

I shook my head. ``What? No. They were merchants and artisans who were
loyal to the King until he instituted policies that were against their
interests and enforced them brutally. The religious refugees were way
earlier.''

``Some of the Framers were descended from religious refugees,'' she
said.

``And the Bill of Rights isn't supposed to be something you pick and
choose from. What the Framers hated was tyranny. That's what the Bill
of Rights is supposed to prevent. They were a revolutionary army and
they wanted a set of principles that everyone could agree to. Life,
liberty and the pursuit of happiness. The right of people to throw off
their oppressors.''

``Yes, yes,'' she said, waving at me. ``They believed in the right of
people to get rid of their Kings, but --'' Charles was grinning and
when she said that, he smiled even wider.

``They set out the Bill of Rights because they thought that having
absolute rights was better than the risk that someone would take them
away. Like the First Amendment: it's supposed to protect us by
preventing the government from creating two kinds of speech, allowed
speech and criminal speech. They didn't want to face the risk that
some jerk would decide that the things that he found unpleasant were
illegal.''

She turned and wrote, ``Life, liberty and the pursuit of happiness'' on
it.

``We're getting a little ahead of the lesson, but you seem like an
advanced group.'' The others laughed at this, nervously.

``The role of government is to secure for citizens the rights of life,
liberty and the pursuit of happiness. In that order. It's like a
filter. If the government wants to do something that makes us a little
unhappy, or takes away some of our liberty, it's OK, providing they're
doing it to save our lives. That's why the cops can lock you up if
they think you're a danger to yourself or others. You lose your
liberty and happiness to protect life. If you've got life, you might
get liberty and happiness later.''

Some of the others had their hands up. ``Doesn't that mean that they
can do anything they want, if they say it's to stop someone from
hurting us in the future?''

``Yeah,'' another kid said. ``This sounds like you're saying that
national security is more important than the Constitution.''

I was so proud of my fellow students then. I said, ``How can you
protect freedom by suspending the Bill of Rights?''

She shook her head at us like we were being very stupid. ``The
'revolutionary' founding fathers \emph{shot traitors} and spies. They
didn't believe in absolute freedom, not when it threatened the
Republic. Now you take these Xnet people --''

I tried hard not to stiffen.

``-- these so-called jammers who were on the news this morning. After
this city was attacked by people who've declared war on this country,
they set about sabotaging the security measures set up to catch the
bad guys and prevent them from doing it again. They did this by
endangering and inconveniencing their fellow citizens --''

``They did it to show that our rights were being taken away in the name
of protecting them!'' I said. OK, I shouted. God, she had me so
steamed. ``They did it because the government was treating \emph{everyone}
like a suspected terrorist.''

``So they wanted to prove that they shouldn't be treated like
terrorists,'' Charles shouted back, ``so they acted like terrorists? So
they committed terrorism?''

I boiled.

``Oh for Christ's sake. Committed terrorism? They showed that universal
surveillance was more dangerous than terrorism. Look at what happened
in the park last weekend. Those people were dancing and listening to
music. How is \emph{that} terrorism?''

The teacher crossed the room and stood before me, looming over me
until I shut up. ``Marcus, you seem to think that nothing has changed
in this country. You need to understand that the bombing of the Bay
Bridge changed everything. Thousands of our friends and relatives lie
dead at the bottom of the Bay. This is a time for national unity in
the face of the violent insult our country has suffered --''

I stood up. I'd had enough of this ``everything has changed''
crapola. ``National unity? The whole point of America is that we're the
country where dissent is welcome. We're a country of dissidents and
fighters and university dropouts and free speech people.''

I thought of Ms Galvez's last lesson and the thousands of Berkeley
students who'd surrounded the police-van when they tried to arrest a
guy for distributing civil rights literature. No one tried to stop
those trucks when they drove away with all the people who'd been
dancing in the park. I didn't try. I was running away.

Maybe everything \emph{had} changed.

``I believe you know where Mr Benson's office is,'' she said to me. ``You
are to present yourself to him immediately. I will \emph{not} have my
classes disrupted by disrespectful behavior. For someone who claims to
love freedom of speech, you're certainly willing to shout down anyone
who disagrees with you.''

I picked up my SchoolBook and my bag and stormed out. The door had a
gas-lift, so it was impossible to slam, or I would have slammed it.

I went fast to Mr Benson's office. Cameras filmed me as I went. My
gait was recorded. The arphids in my student ID broadcast my identity
to sensors in the hallway. It was like being in jail.

``Close the door, Marcus,'' Mr Benson said. He turned his screen around
so that I could see the video feed from the social studies
classroom. He'd been watching.

``What do you have to say for yourself?''

``That wasn't teaching, it was \emph{propaganda}. She told us that the
Constitution didn't matter!''

``No, she said it wasn't religious doctrine. And you attacked her like
some kind of fundamentalist, proving her point. Marcus, you of all
people should understand that everything changed when the bridge was
bombed. Your friend Darryl --''

``Don't you say a goddamned word about him,'' I said, the anger bubbling
over. ``You're not fit to talk about him. Yeah, I understand that
everything's different now. We used to be a free country. Now we're
not.''

``Marcus, do you know what 'zero-tolerance' means?''

I backed down. He could expel me for ``threatening behavior.'' It was
supposed to be used against gang kids who tried to intimidate their
teachers. But of course he wouldn't have any compunctions about using
it on me.

``Yes,'' I said. ``I know what it means.''

``I think you owe me an apology,'' he said.

I looked at him. He was barely suppressing his sadistic smile. A part
of me wanted to grovel. It wanted to beg for his forgiveness for all
my shame. I tamped that part down and decided that I would rather get
kicked out than apologize.

``Governments are instituted among men, deriving their just powers from
the consent of the governed, that whenever any form of government
becomes destructive of these ends, it is the right of the people to
alter or abolish it, and to institute new government, laying its
foundation on such principles, and organizing its powers in such form,
as to them shall seem most likely to effect their safety and
happiness.'' I remembered it word for word.

He shook his head. ``Remembering things isn't the same as understanding
them, sonny.'' He bent over his computer and made some clicks. His
printer purred. He handed me a sheet of warm Board letterhead that
said I'd been suspended for two weeks.

``I'll email your parents now. If you are still on school property in
thirty minutes, you'll be arrested for trespassing.''

I looked at him.

``You don't want to declare war on me in my own school,'' he said. ``You
can't win that war. GO!''

I left.

\chapter{Chapter 14}

\epigraph{This chapter is dedicated to the incomparable Mysterious
  Galaxy in San Diego, California. The Mysterious Galaxy folks have
  had me in to sign books every time I've been in San Diego for a
  conference or to teach (the Clarion Writers' Workshop is based at
  San Diego State University in nearby La Jolla, CA), and every time I
  show up, they pack the house. This is a store with a loyal following
  of die-hard fans who know that they'll always be able to get great
  recommendations and great ideas at the store. In summer 2007, I took
  my writing class from Clarion down to the store for the midnight
  launch of the final Harry Potter book and I've never seen such a
  rollicking, awesomely fun party at a store.}  {Mysterious Galaxy
  \footnote{\url{http://mysteriousgalaxy.booksense.com/NASApp/store/Product?s=showproduct&isbn=9780765319852}}:
  7051 Clairemont Mesa Blvd., Suite \#302 San Diego, CA
  USA 92111 +1 858 268 4747}

The Xnet wasn't much fun in the middle of the school-day, when all the
people who used it were in school. I had the piece of paper folded in
the back pocket of my jeans, and I threw it on the kitchen table when
I got home. I sat down in the living room and switched on the TV. I
never watched it, but I knew that my parents did. The TV and the radio
and the newspapers were where they got all their ideas about the
world.

The news was terrible. There were so many reasons to be
scared. American soldiers were dying all over the world. Not just
soldiers, either. National guardsmen, who thought they were signing up
to help rescue people from hurricanes, stationed overseas for years
and years of a long and endless war.

I flipped around the 24-hour news networks, one after another, a
parade of officials telling us why we should be scared. A parade of
photos of bombs going off around the world.

I kept flipping and found myself looking at a familiar face. It was
the guy who had come into the truck and spoken to Severe-Haircut woman
when I was chained up in the back. Wearing a military uniform. The
caption identified him as Major General Graeme Sutherland, Regional
Commander, DHS.

``I hold in my hands actual literature on offer at the so-called
concert in Dolores Park last weekend.'' He held up a stack of
pamphlets. There'd been lots of pamphleteers there, I
remembered. Wherever you got a group of people in San Francisco, you
got pamphlets.

``I want you to look at these for a moment. Let me read you their
titles. WITHOUT THE CONSENT OF THE GOVERNED: A CITIZEN'S GUIDE TO
OVERTHROWING THE STATE. 
\discretionary{Here}{is}{Here's} one, DID THE SEPTEMBER 11TH BOMBINGS
REALLY HAPPEN? And another, HOW TO USE THEIR SECURITY AGAINST
THEM. This literature shows us the true purpose of the illegal
gathering on Saturday night. This wasn't merely an unsafe gathering of
thousands of people without proper precaution, or even toilets. It was
a recruiting rally for the enemy. It was an attempt to corrupt
children into embracing the idea that America shouldn't protect
herself.

``Take this slogan, DON'T TRUST ANYONE OVER 25. What better way to
ensure that no considered, balanced, adult discussion is ever injected
into your pro-terrorist message than to exclude adults, limiting your
group to impressionable young people?

``When police came on the scene, they found a recruitment rally for
America's enemies in progress. The gathering had already disrupted the
nights of hundreds of residents in the area, none of whom had been
consulted in the planning of this all night rave party.

``They ordered these people to disperse -- that much is visible on all
the video -- and when the revelers turned to attack them, egged on by
the musicians on stage, the police subdued them using non-lethal crowd
control techniques.

``The arrestees were ring-leaders and provocateurs who had led the
thousands of impressionistic young people there to charge the police
lines. 827 of them were taken into custody. Many of these people had
prior offenses. More than 100 of them had outstanding warrants. They
are still in custody.

``Ladies and gentlemen, America is fighting a war on many fronts, but
nowhere is she in more grave danger than she is here, at home. Whether
we are being attacked by terrorists or those who sympathize with
them.''

A reporter held up a hand and said, ``General Sutherland, surely you're
not saying that these children were terrorist sympathizers for
attending a party in a park?''

``Of course not. But when young people are brought under the influence
of our country's enemies, it's easy for them to end up over their
heads. Terrorists would love to recruit a fifth column to fight the
war on the home front for them. If these were my children, I'd be
gravely concerned.''

Another reporter chimed in. ``Surely this is just an open air concert,
General? They were hardly drilling with rifles.''

The General produced a stack of photos and began to hold them
up. ``These are pictures that officers took with infra-red cameras
before moving in.'' He held them next to his face and paged through
them one at a time. They showed people dancing really rough, some
people getting crushed or stepped on. Then they moved into sex stuff
by the trees, a girl with three guys, two guys necking
together. ``There were children as young as ten years old at this
event. A deadly cocktail of drugs, propaganda and music resulted in
dozens of injuries. It's a wonder there weren't any deaths.''

I switched the TV off. They made it look like it had been a riot. If
my parents thought I'd been there, they'd have strapped me to my bed
for a month and only let me out afterward wearing a tracking collar.

Speaking of which, they were going to be \emph{pissed} when they found out
I'd been suspended.

\fancybreak{\#}

They didn't take it well. Dad wanted to ground me, but Mom and I
talked him out of it.

``You know that vice-principal has had it in for Marcus for years,'' Mom
said. ``The last time we met him you cursed him for an hour
afterward. I think the word 'asshole' was mentioned repeatedly.''

Dad shook his head. ``Disrupting a class to argue against the
Department of Homeland Security --''

``It's a social studies class, Dad,'' I said. I was beyond caring
anymore, but I felt like if Mom was going to stick up for me, I should
help her out. ``We were talking about the DHS. Isn't debate supposed to
be healthy?''

``Look, son,'' he said. He'd taking to calling me ``son'' a lot. It made
me feel like he'd stopped thinking of me as a person and switched to
thinking of me as a kind of half-formed larva that needed to be guided
out of adolescence. I hated it. ``You're going to have to learn to live
with the fact that we live in a different world today. You have every
right to speak your mind of course, but you have to be prepared for
the consequences of doing so. You have to face the fact that there are
people who are hurting, who aren't going to want to argue the finer
points of Constitutional law when their lives are at stakes. We're in
a lifeboat now, and once you're in the lifeboat, no one wants to hear
about how mean the captain is being.''

I barely restrained myself from rolling my eyes.

``I've been assigned two weeks of independent study, writing one paper
for each of my subjects, using the city for my background -- a history
paper, a social studies paper, an English paper, a physics paper. It
beats sitting around at home watching television.''

Dad looked hard at me, like he suspected I was up to something, then
nodded. I said goodnight to them and went up to my room. I fired up my
Xbox and opened a word-processor and started to brainstorm ideas for
my papers. Why not? It really was better than sitting around at home.

\fancybreak{\#}

I ended up IMing with Ange for quite a while that night. She was
sympathetic about everything and told me she'd help me with my papers
if I wanted to meet her after school the next night. I knew where her
school was -- she went to the same school as Van -- and it was all the
way over in the East Bay, where I hadn't visited since the bombs went.

I was really excited at the prospect of seeing her again. Every night
since the party, I'd gone to bed thinking of two things: the sight of
the crowd charging the police lines and the feeling of the side of her
breast under her shirt as we leaned against the pillar. She was
amazing. I'd never been with a girl as \ldots aggressive as her before. It
had always been me putting the moves on and them pushing me away. I
got the feeling that Ange was as much of a horn-dog as I was. It was a
tantalizing notion.

I slept soundly that night, with exciting dreams of me and Ange and
what we might do if we found ourselves in a secluded spot somewhere.

The next day, I set out to work on my papers. San Francisco is a good
place to write about. History? Sure, it's there, from the Gold Rush to
the WWII shipyards, the Japanese internment camps, the invention of
the PC. Physics? The Exploratorium has the coolest exhibits of any
museum I've ever been to. I took a perverse satisfaction in the
exhibits on soil liquefaction during big quakes. English? Jack London,
Beat Poets, science fiction writers like Pat Murphy and Rudy
Rucker. Social studies? The Free Speech Movement, Cesar Chavez, gay
rights, feminism, anti-war movement\ldots

I've always loved just learning stuff for its own sake. Just to be
smarter about the world around me. I could do that just by walking
around the city. I decided I'd do an English paper about the Beats
first. City Lights books had a great library in an upstairs room where
Alan Ginsberg and his buddies had created their radical druggy
poetry. The one we'd read in English class was \emph{Howl} and I would
never forget the opening lines, they gave me shivers down my back:

\begin{flushleft}
\itshape
\setlength{\parskip}{.5\baselineskip}
I saw the best minds of my generation destroyed by madness, starving
hysterical naked,

dragging themselves through the negro streets at dawn looking for an
angry fix,

angelheaded hipsters burning for the ancient heavenly connection to
the starry dynamo in the machinery of night\ldots

\end{flushleft}

I liked the way he ran those words all together, ``starving hysterical
naked.'' I knew how that felt. And ``best minds of my generation'' made
me think hard too. It made me remember the park and the police and the
gas falling. They busted Ginsberg for obscenity over Howl -- all about
a line about gay sex that would hardly have caused us to blink an eye
today. It made me happy somehow, knowing that we'd made some
progress. That things had been even more restrictive than this before.

I lost myself in the library, reading these beautiful old editions of
the books. I got lost in Jack Kerouac's \emph{On the Road}, a novel I'd
been meaning to read for a long time, and a clerk who came up to check
on me nodded approvingly and found me a cheap edition that he sold me
for six bucks.

I walked into Chinatown and had dim sum buns and noodles with
hot-sauce that I had previously considered to be pretty hot, but which
would never seem anything like hot ever again, not now that I'd had an
Ange special.

As the day wore on toward the afternoon, I got on the BART and
switched to a San Mateo bridge shuttle bus to bring me around to the
East Bay. I read my copy of \emph{On the Road} and dug the scenery whizzing
past. \emph{On the Road} is a semi-auto\-bio\-gra\-phi\-cal novel about Jack
Kerouac, a druggy, hard-drinking writer who goes hitchhiking around
America, working crummy jobs, howling through the streets at night,
meeting people and parting ways. Hipsters, sad-faced hobos, con-men,
muggers, scumbags and angels. There's not really a plot -- Kerouac
supposedly wrote it in three weeks on a long roll of paper, stoned out
of his mind -- only a bunch of amazing things, one thing happening
after another. He makes friends with self-destructing people like Dean
Moriarty, who get him involved in weird schemes that never really work
out, but still it works out, if you know what I mean.

There was a rhythm to the words, it was luscious, I could hear it
being read aloud in my head. It made me want to lie down in the bed of
a pickup truck and wake up in a dusty little town somewhere in the
central valley on the way to LA, one of those places with a gas
station and a diner, and just walk out into the fields and meet people
and see stuff and do stuff.

It was a long bus ride and I must have dozed off a little -- staying
up late IMing with Ange was hard on my sleep-schedule, since Mom still
expected me down for breakfast. I woke up and changed buses and before
long, I was at Ange's school.

She came bounding out of the gates in her uniform -- I'd never seen
her in it before, it was kind of cute in a weird way, and reminded me
of Van in her uniform. She gave me a long hug and a hard kiss on the
cheek.

``Hello you!'' she said.

``Hiya!''

``Whatcha reading?''

I'd been waiting for this. I'd marked the passage with a
finger. ``Listen: 'They danced down the streets like dingledodies, and
I shambled after as I've been doing all my life after people who
interest me, because the only people for me are the mad ones, the ones
who are mad to live, mad to talk, mad to be saved, desirous of
everything at the same time, the ones that never yawn or say a
commonplace thing, but burn, burn, burn like fabulous yellow roman
candles exploding like spiders across the stars and in the middle you
see the blue centerlight pop and everybody goes ``Awww!'''''

She took the book and read the passage again for herself. ``Wow,
dingledodies! I love it! Is it all like this?''

I told her about the parts I'd read, walking slowly down the sidewalk
back toward the bus-stop. Once we turned the corner, she put her arm
around my waist and I slung mine around her shoulder. Walking down the
street with a girl -- my girlfriend? Sure, why not? -- talking about
this cool book. It was heaven. Made me forget my troubles for a little
while.

``Marcus?''

I turned around. It was Van. In my subconscious I'd expected this. I
knew because my conscious mind wasn't remotely surprised. It wasn't a
big school, and they all got out at the same time. I hadn't spoken to
Van in weeks, and those weeks felt like months. We used to talk every
day.

``Hey, Van,'' I said. I suppressed the urge to take my arm off of Ange's
shoulders. Van seemed surprised, but not angry, more ashen,
shaken. She looked closely at the two of us.

``Angela?''

``Hey, Vanessa,'' Ange said.

``What are you doing here?''

``I came out to get Ange,'' I said, trying to keep my tone neutral. I
was suddenly embarrassed to be seen with another girl.

``Oh,'' Van said. ``Well, it was nice to see you.''

``Nice to see you too, Vanessa,'' Ange said, swinging me around,
marching me back toward the bus-stop.

``You know her?'' Ange said.

``Yeah, since forever.''

``Was she your girlfriend?''

``What? No! No way! We were just friends.''

``You \emph{were} friends?''

I felt like Van was walking right behind us, listening in, though at
the pace we were walking, she would have to be jogging to keep up. I
resisted the temptation to look over my shoulder for as long as
possible, then I did. There were lots of girls from the school behind
us, but no Van.

``She was with me and Jose-Luis and Darryl when we were arrested. We
used to ARG together. The four of us, we were kind of best friends.''

``And what happened?''

I dropped my voice. ``She didn't like the Xnet,'' I said. ``She thought
we would get into trouble. That I'd get other people into trouble.''

``And that's why you stopped being friends?''

``We just drifted apart.''

We walked a few steps. ``You weren't, you know, 
boy\-friend/\discretionary{}{}{}girl\-friend
friends?''

``No!'' I said. My face was hot. I felt like I sounded like I was lying,
even though I was telling the truth.

Ange jerked us to a halt and studied my face.

``Were you?''

``No! Seriously! Just friends. Darryl and her -- well, not quite, but
Darryl was so into her. There was no way --''

``But if Darryl hadn't been into her, you would have, huh?''

``No, Ange, no. Please, just believe me and let it go. Vanessa was a
good friend and we're not anymore, and that upsets me, but I was never
into her that way, all right?

She slumped a little. ``OK, OK. I'm sorry. I don't really get along
with her is all. We've never gotten along in all the years we've known
each other.''

Oh ho, I thought. This would be how it came to be that Jolu knew her
for so long and I never met her; she had some kind of thing with Van
and he didn't want to bring her around.

She gave me a long hug and we kissed, and a bunch of girls passed us
going \emph{woooo} and we straightened up and headed for the
bus-stop. Ahead of us walked Van, who must have gone past while we
were kissing. I felt like a complete jerk.

Of course, she was at the stop and on the bus and we didn't say a word
to each other, and I tried to make conversation with Ange all the way,
but it was awkward.

The plan was to stop for a coffee and head to Ange's place to hang out
and ``study,'' i.e. take turns on her Xbox looking at the Xnet. Ange's
mom got home late on Tuesdays, which was her night for yoga class and
dinner with her girls, and Ange's sister was going out with her
boyfriend, so we'd have the place to ourselves. I'd been having pervy
thoughts about it ever since we'd made the plan.

We got to her place and went straight to her room and shut the
door. Her room was kind of a disaster, covered with layers of clothes
and notebooks and parts of PCs that would dig into your stocking feet
like caltrops. Her desk was worse than the floor, piled high with
books and comics, so we ended up sitting on her bed, which was OK by
me.

The awkwardness from seeing Van had gone away somewhat and we got her
Xbox up and running. It was in the center of a nest of wires, some
going to a wireless antenna she'd hacked into it and stuck to the
window so she could tune in the neighbors' WiFi. Some went to a couple
of old laptop screens she'd turned into standalone monitors, balanced
on stands and bristling with exposed electronics. The screens were on
both bedside tables, which was an excellent setup for watching movies
or IMing from bed -- she could turn the monitors sidewise and lie on
her side and they'd be right-side-up, no matter which side she lay on.

We both knew what we were really there for, sitting side by side
propped against the bedside table. I was trembling a little and
super-conscious of the warmth of her leg and shoulder against mine,
but I needed to go through the motions of logging into Xnet and seeing
what email I'd gotten and so on.

There was an email from a kid who liked to send in funny phone-cam
videos of the DHS being really crazy -- the last one had been of them
disassembling a baby's stroller after a bomb-sniffing dog had shown an
interest in it, taking it apart with screwdrivers right on the street
in the Marina while all these rich people walked past, staring at them
and marveling at how weird it was.

I'd linked to the video and it had been downloaded like crazy. He'd
hosted it on the Internet Archive's Alexandria mirror in Egypt, where
they'd host anything for free so long as you'd put it under the
Creative Commons license, which let anyone remix it and share it. The
US archive -- which was down in the Presidio, only a few minutes away
-- had been forced to take down all those videos in the name of
national security, but the Alexandria archive had split away into its
own organization and was hosting anything that embarrassed the USA.

This kid -- his handle was Kameraspie -- had sent me an even better
video this time around. It was at the doorway to City Hall in Civic
Center, a huge wedding cake of a building covered with statues in
little archways and gilt leaves and trim. The DHS had a secure
perimeter around the building, and Kameraspie's video showed a great
shot of their checkpoint as a guy in an officer's uniform approached
and showed his ID and put his briefcase on the X-ray belt.

It was all OK until one of the DHS people saw something he didn't like
on the X-ray. He questioned the General, who rolled his eyes and said
something inaudible (the video had been shot from across the street,
apparently with a homemade concealed zoom lens, so the audio was
mostly of people walking past and traffic noises).

The General and the DHS guys got into an argument, and the longer they
argued, the more DHS guys gathered around them. Finally, the General
shook his head angrily and waved his finger at the DHS guy's chest and
picked up his briefcase and started to walk away. The DHS guys shouted
at him, but he didn't slow. His body language really said, ``I am
totally, utterly pissed.''

Then it happened. The DHS guys ran after the general. Kameraspie
slowed the video down here, so we could see, in frame-by-frame slo-mo,
the general half-turning, his face all like, ``No freaking way are you
about to tackle me,'' then changing to horror as three of the giant DHS
guards slammed into him, knocking him sideways, then catching him at
the middle, like a career-ending football tackle. The general --
middle aged, steely grey hair, lined and dignified face -- went down
like a sack of potatoes and bounced twice, his face slamming off the
sidewalk and blood starting out of his nose.

The DHS hog-tied the general, strapping him at ankles and wrists. The
general was shouting now, really shouting, his face purpling under the
blood streaming from his nose. Legs swished by in the tight
zoom. Passing pedestrians looked at this guy in his uniform, getting
tied up, and you could see from his face that this was the worst part,
this was the ritual humiliation, the removal of dignity. The clip
ended.

``Oh my dear sweet Buddha,'' I said looking at the screen as it faded to
black, starting the video again. I nudged Ange and showed her the
clip. She watched wordless, jaw hanging down to her chest.

``Post that,'' she said. ``Post that post that post that post that!''

I posted it. I could barely type as I wrote it up, describing what I'd
seen, adding a note to see if anyone could identify the military man
in the video, if anyone knew anything about this.

I hit publish.

We watched the video. We watched it again.

My email pinged.

\edialog{I totally recognize that dude -- you can find his bio on
  Wikipedia. He's General Claude Geist. He commanded the joint UN
  peacekeeping mission in Haiti.}

I checked the bio. There was a picture of the general at a press
conference, and notes about his role in the difficult Haiti
mission. It was clearly the same guy.

I updated the post.

Theoretically, this was Ange's and my chance to make out, but that
wasn't what we ended up doing. We crawled the Xnet blogs, looking for
more accounts of the DHS searching people, tackling people, invading
them. This was a familiar task, the same thing I'd done with all the
footage and accounts from the riots in the park. I started a new
category on my blog for this, AbusesOfAuthority, and filed them
away. Ange kept coming up with new search terms for me to try and by
the time her mom got home, my new category had seventy posts,
headlined by General Geist's City Hall takedown.

\fancybreak{\#}

I worked on my Beat paper all the next day at home, reading the
Kerouac and surfing the Xnet. I was planning on meeting Ange at
school, but I totally wimped out at the thought of seeing Van again,
so I texted her an excuse about working on the paper.

There were all kinds of great suggestions for AbusesOfAuthority coming
in; hundreds of little and big ones, pictures and audio. The meme was
spreading.

It spread. The next morning there were even more. Someone started a
new blog called AbusesOfAuthority that collected hundreds more. The
pile grew. We competed to find the juiciest stories, the craziest
pictures.

The deal with my parents was that I'd eat breakfast with them every
morning and talk about the projects I was doing. They liked that I was
reading Kerouac. It had been a favorite book of both of theirs and it
turned out there was already a copy on the bookcase in my parents'
room. My dad brought it down and I flipped through it. There were
passages marked up with pen, dog-eared pages, notes in the margin. My
dad had really loved this book.

It made me remember a better time, when my Dad and I had been able to
talk for five minutes without shouting at each other about terrorism,
and we had a great breakfast talking about the way that the novel was
plotted, all the crazy adventures.

But the next morning at breakfast they were both glued to the radio.

``Abuses of Authority -- it's the latest craze on San Francisco's
notorious Xnet, and it's captured the world's attention. Called
A-oh-A, the movement is composed of 'Little Brothers' who watch back
against the Department of Homeland Security's anti-ter\-ror\-ism measures,
documenting the failures and excesses. The rallying cry is a popular
viral video clip of a General Claude Geist, a retired three-star
general, being tackled by DHS officers on the sidewalk in front of
City Hall. Geist hasn't made a statement on the incident, but
commentary from young people who are upset with their own treatment
has been fast and furious.

``Most notable has been the global attention the movement has
received. Stills from the Geist video have appeared on the front pages
of newspapers in Korea, Great Britain, Germany, Egypt and Japan, and
broadcasters around the world have aired the clip on prime-time
news. The issue came to a head last night, when the British
Broadcasting Corporation's National News Evening program ran a special
report on the fact that no American broadcaster or news agency has
covered this story. Commenters on the BBC's website noted that BBC
America's version of the news did not carry the report.''

They brought on a couple of interviews: British media watchdogs, a
Swedish Pirate Party kid who made jeering remarks about America's
corrupt press, a retired American newscaster living in Tokyo, then
they aired a short clip from Al-Jazeera, comparing the American press
record and the record of the national news-media in Syria.

I felt like my parents were staring at me, that they knew what I was
doing. But when I cleared away my dishes, I saw that they were looking
at each other.

Dad was holding his coffee cup so hard his hands were shaking. Mom was
looking at him.

``They're trying to discredit us,'' Dad said finally. ``They're trying to
sabotage the efforts to keep us safe.''

I opened my mouth, but my mom caught my eye and shook her
head. Instead I went up to my room and worked on my Kerouac
paper. Once I'd heard the door slam twice, I fired up my Xbox and got
online.

\edialog{Hello M1k3y. This is Colin Brown. I'm a producer with the
  Canadian Broadcasting Corporation's news programme The
  National. We're doing a story on Xnet and have sent a reporter to
  San Francisco to cover it from there. Would you be interested in
  doing an interview to discuss your group and its actions?}

I stared at the screen. Jesus. They wanted to \emph{interview} me about ``my
group''?

\edialog{Um thanks no. I'm all about privacy. And it's not ``my
  group.'' But thanks for doing the story!}

A minute later, another email.

\edialog{We can mask you and ensure your anonymity. You know that the
  Department of Homeland Security will be happy to provide their own
  spokesperson. I'm interested in getting your side.}

I filed the email. He was right, but I'd be crazy to do this. For all
I knew, he \emph{was} the DHS.

I picked up more Kerouac. Another email came in. Same request,
different news-agency: KQED wanted to meet me and record a radio
interview. A station in Brazil. The Australian Broadcasting
Corporation. Deutsche Welle. All day, the press requests came in. All
day, I politely turned them down.

I didn't get much Kerouac read that day.

\fancybreak{\#}

``Hold a \erratum{press-conference}{press conference},'' is what Ange said, as we sat in the cafe
near her place that evening. I wasn't keen on going out to her school
anymore, getting stuck on a bus with Van again.

``What? Are you crazy?''

``Do it in Clockwork Plunder. Just pick a trading post where there's no
PvP allowed and name a time. You can login from here.''

PvP is player-versus-player combat. Parts of Clockwork Plunder were
neutral ground, which meant that we could theoretically bring in a ton
of noob reporters without worrying about gamers killing them in the
middle of the \erratum{press-conference}{press conference}.

``I don't know anything about press conferences.''

``Oh, just google it. I'm sure someone's written an article on holding
a successful one. I mean, if the President can manage it, I'm sure you
can. He looks like he can barely tie his shoes without help.''

We ordered more coffee.

``You are a very smart woman,'' I said.

``And I'm beautiful,'' she said.

``That too,'' I said.

\chapter{Chapter 15}

\epigraph{This chapter is dedicated to Chapters/Indigo, the national
  Canadian megachain. I was working at Bakka, the independent science
  fiction bookstore, when Chapters opened its first store in Toronto
  and I knew that something big was going on right away, because two
  of our smartest, best-informed customers stopped in to tell me that
  they'd been hired to run the science fiction section. From the
  start, Chapters raised the bar on what a big corporate bookstore
  could be, extending its hours, adding a friendly cafe and lots of
  seating, installing in-store self-service terminals and stocking the
  most amazing variety of titles.}
{Chapters/Indigo:
\url{http://www.chapters.indigo.ca/books/Little-Brother-Cory-Doctorow/9780765319852-item.html}}

I blogged the \erratum{press-conference}{press conference} even before I'd sent out the
invitations to the press. I could tell that all these writers wanted
to make me into a leader or a general or a supreme guerrilla
commandant, and I figured one way of solving that would be to have a
bunch of Xnetters running around answering questions too.

Then I emailed the press. The responses ranged from puzzled to
enthusiastic -- only the Fox reporter was ``outraged'' that I had the
gall to ask her to play a game in order to appear on her TV show. The
rest of them seemed to think that it would make a pretty cool story,
though plenty of them wanted lots of tech support for signing onto the
game

I picked 8PM, after dinner. Mom had been bugging me about all the
evenings I'd been spending out of the house until I finally spilled
the beans about Ange, whereupon she came over all misty and kept
looking at me like, my-little-boy's-growing-up. She wanted to meet
Ange, and I used that as leverage, promising to bring her over the
next night if I could ``go to the movies'' with Ange tonight.

Ange's mom and sister were out again -- they weren't real
stay-at-homes -- which left me and Ange alone in her room with her
Xbox and mine. I unplugged one of her bedside screens and attached my
Xbox to it so that we could both login at once.

Both Xboxes were idle, logged into Clockwork Plunder. I was pacing.

``It's going to be fine,'' she said. She glanced at her
screen. ``Patcheye Pete's Market has 600 players in it now!'' We'd
picked Patcheye Pete's because it was the market closest to the
village square where new players spawned. If the reporters weren't
already Clockwork Plunder players -- ha! -- then that's where they'd
show up. In my blog post I'd asked people generally to hang out on the
route between Patcheye Pete's and the spawn-gate and direct anyone who
looked like a disoriented reporter over to Pete's.

``What the hell am I going to tell them?''

``You just answer their questions -- and if you don't like a question,
ignore it. Someone else can answer it. It'll be fine.''

``This is insane.''

``This is perfect, Marcus. If you want to really screw the DHS, you
have to embarrass them. It's not like you're going to be able to
out-shoot them. Your only weapon is your ability to make them look
like morons.''

I flopped on the bed and she pulled my head into her lap and stroked
my hair. I'd been playing around with different haircuts before the
bombing, dying it all kinds of funny colors, but since I'd gotten out
of jail I couldn't be bothered. It had gotten long and stupid and
shaggy and I'd gone into the bathroom and grabbed my clippers and
buzzed it down to half an inch all around, which took zero effort to
take care of and helped me to be invisible when I was out jamming and
cloning arphids.

I opened my eyes and stared into her big brown eyes behind her
glasses. They were round and liquid and expressive. She could make
them bug out when she wanted to make me laugh, or make them soft and
sad, or lazy and sleepy in a way that made me melt into a puddle of
horniness.

That's what she was doing right now.

I sat up slowly and hugged her. She hugged me back. We kissed. She was
an amazing kisser. I know I've already said that, but it bears
repeating. We kissed a lot, but for one reason or another we always
stopped before it got too heavy.

Now I wanted to go farther. I found the hem of her t-shirt and
tugged. She put her hands over her head and pulled back a few
inches. I knew that she'd do that. I'd known since the night in the
park. Maybe that's why we hadn't gone farther -- I knew I couldn't
rely on her to back off, which scared me a little.

But I wasn't scared then. The impending \erratum{press-conference}{press conference}, the fights
with my parents, the international attention, the sense that there was
a movement that was careening around the city like a wild pinball --
it made my skin tingle and my blood sing.

And she was beautiful, and smart, and clever and funny, and I was
falling in love with her.

Her shirt slid off, her arching her back to help me get it over her
shoulders. She reached behind her and did something and her bra fell
away. I stared goggle-eyed, motionless and breathless, and then she
grabbed \emph{my} shirt and pulled it over my head, grabbing me and pulling
my bare chest to hers.

We rolled on the bed and touched each other and ground our bodies
together and groaned. She kissed all over my chest and I did the same
to her. I couldn't breathe, I couldn't think, I could only move and
kiss and lick and touch.

We dared each other to go forward. I undid her jeans. She undid
mine. I lowered her zipper, she did mine, and tugged my jeans off. I
tugged off hers. A moment later we were both naked, except for my
socks, which I peeled off with my toes.

It was then that I caught sight of the bedside clock, which had long
ago rolled onto the floor and lay there, glowing up at us.

``Crap!'' I yelped. ``It starts in two minutes!'' I couldn't freaking
believe that I was about to stop what I was about to stop doing, when
I was about to stop doing it. I mean, if you'd asked me, ``Marcus, you
are about to get laid for the firstest time EVAR, will you stop if I
let off this nuclear bomb in the same room as you?'' the answer would
have been a resounding and unequivical \emph{NO}.

And yet we stopped for this.

She grabbed me and pulled my face to hers and kissed me until I
thought I would pass out, then we both grabbed our clothes and more or
less dressed, grabbing our keyboards and mice and heading for Patcheye
Pete's.

\fancybreak{\#}

You could easily tell who the press were: they were the noobs who
played their characters like staggering drunks, weaving back and forth
and up and down, trying to get the hang of it all, occasionally
hitting the wrong key and offering strangers all or part of their
inventory, or giving them accidental hugs and kicks.

The Xnetters were easy to spot, too: we all played Clockwork Plunder
whenever we had some spare time (or didn't feel like doing our
homework), and we had pretty tricked-out characters with cool weapons
and booby-traps on the keys sticking out of our backs that would cream
anyone who tried to snatch them and leave us to wind down.

When I appeared, a system status message displayed M1K3Y HAS ENTERED
PATCHEYE PETE'S -- WELCOME SWABBIE WE OFFER FAIR TRADE FOR FINE
BOOTY. All the players on the screen froze, then they crowded around
me. The chat exploded. I thought about turning on my voice-paging and
grabbing a headset, but seeing how many people were trying to talk at
once, I realized how confusing that would be. Text was much easier to
follow and they couldn't misquote me (heh heh).

I'd scouted the location before with Ange -- it was great campaigning
with her, since we could both keep each other wound up. There was a
high-spot on a pile of boxes of salt-rations that I could stand on and
be seen from anywhere in the market.

\edialog{Good evening and thank you all for coming. My name is M1k3y
  and I'm not the leader of anything. All around you are Xnetters who
  have as much to say about why we're here as I do. I use the Xnet
  because I believe in freedom and the Constitution of the United
  States of America. I use Xnet because the DHS has turned my city
  into a police-state where we're all suspected terrorists. I use Xnet
  because I think you can't defend freedom by tearing up the Bill of
  Rights. I learned about the Constitution in a California school and
  I was raised to love my country for its freedom. If I have a
  philosophy, it is this:}

\edialog{Governments are instituted among men, deriving their just
  powers from the consent of the governed, that whenever any form of
  government becomes destructive of these ends, it is the right of the
  people to alter or abolish it, and to institute new government,
  laying its foundation on such principles, and organizing its powers
  in such form, as to them shall seem most likely to effect their
  safety and happiness.}

\edialog{I didn't write that, but I believe it. The DHS does not
  govern with my consent.}

\edialog{Thank you}

I'd written this the day before, bouncing drafts back and forth with
Ange. Pasting it in only took a second, though it took everyone in the
game a moment to read it. A lot of the Xnetters cheered, big showy
pirate ``Hurrah''s with raised sabers and pet parrots squawking and
flying overhead.

Gradually, the journalists digested it too. The chat was running past
fast, so fast you could barely read it, lots of Xnetters saying things
like ``Right on'' and ``America, love it or leave it'' and ``DHS go home''
and ``America out of San Francisco,'' all slogans that had been big on
the Xnet blogosphere.

\edialog{M1k3y, this is Priya Rajneesh from the BBC. You say you're
  not the leader of any movement, but do you believe there is a
  movement? Is it called the Xnet?}

Lots of answers. Some people said there wasn't a movement, some said
there was and lots of people had ideas about what it was called: Xnet,
Little Brothers, Little Sisters, and my personal favorite, the United
States of America.

They were really cooking. I let them go, thinking of what I could
say. Once I had it, I typed,

\edialog{I think that kind of answers your question, doesn't it?
  There may be one or more movements and they may be called Xnet or
  not.}

\edialog{M1k3y, I'm Doug Christensen from the Washington Internet
  Daily. What do you think the DHS should be doing to prevent another
  attack on San Francisco, if what they're doing isn't successful.}

More chatter. Lots of people said that the terrorists and the
government were the same -- either literally, or just meaning that
they were equally bad. Some said the government knew how to catch
terrorists but preferred not to because ``war presidents'' got
re-elected.

\edialog{I don't know}

I typed finally.

\edialog{I really don't. I ask myself this question a lot because I
  don't want to get blown up and I don't want my city to get blown
  up. Here's what I've figured out, though: if it's the DHS's job to
  keep us safe, they're failing. All the crap they've done, none of it
  would stop the bridge from being blown up again. Tracing us around
  the city? Taking away our freedom? Making us suspicious of each
  other, turning us against each other? Calling dissenters traitors?
  The point of terrorism is to terrify us. The DHS terrifies me.}

\edialog{I don't have any say in what the terrorists do to me, but if
  this is a free country then I should be able to at least say what my
  own cops do to me. I should be able to keep them from terrorizing
  me.}

\edialog{I know that's not a good answer. Sorry.}

\edialog{What do you mean when you say that the DHS wouldn't stop
  terrorists?  How do you know?}

\edialog{Who are you?}

\edialog{I'm with the Sydney Morning Herald.}

\edialog{I'm 17 years old. I'm not a straight-A student or
  anything. Even so, I figured out how to make an Internet that they
  can't wiretap. I figured out how to jam their person-tracking
  technology. I can turn innocent people into suspects and turn guilty
  people into innocents in their eyes. I could get metal onto an
  airplane or beat a no-fly list. I figured this stuff out by looking
  at the web and by thinking about it. If I can do it, terrorists can
  do it. They told us they took away our freedom to make us safe. Do
  you feel safe?}

\edialog{In Australia? Why yes I do}

The pirates all laughed.

More journalists asked questions. Some were sympathetic, some were
hostile. When I got tired, I handed my keyboard to Ange and let her be
M1k3y for a while. It didn't really feel like M1k3y and me were the
same person anymore anyway. M1k3y was the kind of kid who talked to
international journalists and inspired a movement. Marcus got
suspended from school and fought with his dad and wondered if he was
good enough for his kick-ass girlfriend.

By 11PM I'd had enough. Besides, my parents would be expecting me home
soon. I logged out of the game and so did Ange and we lay there for a
moment. I took her hand and she squeezed hard. We hugged.

She kissed my neck and murmured something.

``What?''

``I said I love you,'' she said. ``What, you want me to send you a
telegram?''

``Wow,'' I said.

``You're that surprised, huh?''

``No. Um. It's just -- I was going to say that to you.''

``Sure you were,'' she said, and bit the tip of my nose.

``It's just that I've never said it before,'' I said. ``So I was working
up to it.''

``You still haven't said it, you know. Don't think I haven't
noticed. We girls pick upon these things.''

``I love you, Ange Carvelli,'' I said.

``I love you too, Marcus Yallow.''

We kissed and nuzzled and I started to breathe hard and so did
she. That's when her mom knocked on the door.

``Angela,'' she said, ``I think it's time your friend went home, don't
you?''

``Yes, mother,'' she said, and mimed swinging an axe. As I put my socks
and shoes on, she muttered, ``They'll say, that Angela, she was such a
good girl, who would have thought it, all the time she was in the back
yard, helping her mother out by sharpening that hatchet.''

I laughed. ``You don't know how easy you have it. There is \emph{no way} my
folks would leave us alone in my bedroom until 11 o'clock.''

``11:45,'' she said, checking her clock.

``Crap!'' I yelped and tied my shoes.

``Go,'' she said, ``run and be free! Look both ways before crossing the
road! Write if you get work! Don't even stop for a hug! If you're not
out of here by the count of ten, there's going to be \emph{trouble},
mister. One. Two. Three.''

I shut her up by leaping onto the bed, landing on her and kissing her
until she stopped trying to count. Satisfied with my victory, I
pounded down the stairs, my Xbox under my arm.

Her mom was at the foot of the stairs. We'd only met a couple
times. She looked like an older, taller version of Ange -- Ange said
her father was the short one -- with contacts instead of glasses. She
seemed to have tentatively classed me as a good guy, I and appreciated
it.

``Good night, Mrs Carvelli,'' I said.

``Good night, Mr Yallow,'' she said. It was one of our little rituals,
ever since I'd called her Mrs Carvelli when we first met.

I found myself standing awkwardly by the door.

``Yes?'' she said.

``Um,'' I said. ``Thanks for having me over.''

``You're always welcome in our home, young man,'' she said.

``And thanks for Ange,'' I said finally, hating how lame it sounded. But
she smiled broadly and gave me a brief hug.

``You're very welcome,'' she said.

The whole bus ride home, I thought over the \erratum{press-conference}{press conference}, thought
about Ange naked and writhing with me on her bed, thought about her
mother smiling and showing me the door.

My mom was waiting up for me. She asked me about the movie and I gave
her the response I'd worked out in advance, cribbing from the review
it had gotten in the \emph{Bay Guardian}.

As I fell asleep, the \erratum{press-conference}{press conference} came back. I was really proud
of it. It had been so cool, to have all these big-shot journos show up
in the game, to have them listen to me and to have them listen to all
the people who believed in the same things as me. I dropped off with a
smile on my lips.

\fancybreak{\#}

I should have known better.

\begin{news}
XNET LEADER: I COULD GET METAL ONTO AN AIRPLANE

%\medskip
DHS DOESN'T HAVE MY CONSENT TO GOVERN

%\medskip
XNET KIDS: USA OUT OF SAN FRANCISCO
\end{news}

Those were the \emph{good} headlines. Everyone sent me the articles to
blog, but it was the last thing I wanted to do.

I'd blown it, somehow. The press had come to my press-con\-fer\-ence and
concluded that we were terrorists or terrorist dupes. The worst was
the reporter on Fox News, who had apparently shown up anyway, and who
devoted a ten-minute commentary to us, talking about our ``criminal
treason.'' Her killer line, repeated on every news-outlet I found, was:

``They say they don't have a name. I've got one for them. Let's call
these spoiled children Cal-Quaeda. They do the terrorists' work on the
home front. When -- not if, but when -- California gets attacked
again, these brats will be as much to blame as the House of Saud.''

Leaders of the anti-war movement denounced us as fringe elements. One
guy went on TV to say that he believed we had been fabricated by the
DHS to discredit them.

The DHS had their own \erratum{press-conference}{press conference} announcing that they would
double the security in San Francisco. They held up an arphid cloner
they'd found somewhere and demonstrated it in action, using it to
stage a car-theft, and warned everyone to be on their alert for young
people behaving suspiciously, especially those whose hands were out of
sight.

They weren't kidding. I finished my Kerouac paper and started in on a
paper about the Summer of Love, the summer of 1967 when the anti-war
movement and the hippies converged on San Francisco. The guys who
founded Ben and Jerry's -- old hippies themselves -- had founded a
hippie museum in the Haight, and there were other archives and
exhibits to see around town.

But it wasn't easy getting around. By the end of the week, I was
getting frisked an average of four times a day. Cops checked my ID and
questioned me about why I was out in the street, carefully eyeballing
the letter from Chavez saying that I was suspended.

I got lucky. No one arrested me. But the rest of the Xnet weren't so
lucky. Every night the DHS announced more arrests, ``ringleaders'' and
``operatives'' of Xnet, people I didn't know and had never heard of,
paraded on TV along with the arphid sniffers and other devices that
had been in their pockets. They announced that the people were ``naming
names,'' compromising the ``Xnet network'' and that more arrests were
expected soon. The name ``M1k3y'' was often heard.

Dad loved this. He and I watched the news together, him gloating, me
shrinking away, quietly freaking out. ``You should see the stuff
they're going to use on these kids,'' Dad said. ``I've seen it in
action. They'll get a couple of these kids and check out their friends
lists on IM and the speed-dials on their phones, look for names that
come up over and over, look for patterns, bringing in more
kids. They're going to unravel them like an old sweater.''

I canceled Ange's dinner at our place and started spending even more
time there. Ange's little sister Tina started to call me ``the
house-guest,'' as in ``is the house-guest eating dinner with me
tonight?'' I liked Tina. All she cared about was going out and partying
and meeting guys, but she was funny and utterly devoted to Ange. One
night as we were doing the dishes, she dried her hands and said,
conversationally, ``You know, you seem like a nice guy, Marcus. My
sister's just crazy about you and I like you too. But I have to tell
you something: if you break her heart, I will track you down and pull
your scrotum over your head. It's not a pretty sight.''

I assured her that I would sooner pull my own scrotum over my head
than break Ange's heart and she nodded. ``So long as we're clear on
that.''

``Your sister is a nut,'' I said as we lay on Ange's bed again, looking
at Xnet blogs. That is pretty much all we did: fool around and read
Xnet.

``Did she use the scrotum line on you? I hate it when she does
that. She just loves the word 'scrotum,' you know. It's nothing
personal.''

I kissed her. We read some more.

``Listen to this,'' she said. ``Police project four to six \emph{hundred}
arrests this weekend in what they say will be the largest coordinated
raid on Xnet dissidents to date.''

I felt like throwing up.

``We've got to stop this,'' I said. ``You know there are people who are
doing \emph{more} jamming to show that they're not intimidated? Isn't that
just \emph{crazy}?''

``I think it's brave,'' she said. ``We can't let them scare us into
submission.''

``What? No, Ange, no. We can't let hundreds of people go to \emph{jail}. You
haven't been there. I have. It's worse than you think. It's worse than
you can imagine.''

``I have a pretty fertile imagination,'' she said.

``Stop it, OK? Be serious for a second. I won't do this. I won't send
those people to jail. If I do, I'm the guy that Van thinks I am.''

``Marcus, I'm being serious. You think that these people don't know
they could go to jail? They believe in the cause. You believe in it
too. Give them the credit to know what they're getting into. It's not
up to you to decide what risks they can or can't take.''

``It's my responsibility because if I tell them to stop, they'll stop.''

``I thought you weren't the leader?''

``I'm not, of course I'm not. But I can't help it if they look to me
for guidance. And so long as they do, I have a responsibility to help
them stay safe. You see that, right?''

``All I see is you getting ready to cut and run at the first sign of
trouble. I think you're afraid they're going to figure out who \emph{you}
are. I think you're afraid for \emph{you}.''

``That's not fair,'' I said, sitting up, pulling away from her.

``Really? Who's the guy who nearly had a heart attack when he thought
that his secret identity was out?''

``That was different,'' I said. ``This isn't about me. You know it
isn't. Why are you being like this?''

``Why are \emph{you} like this?'' she said. ``Why aren't \emph{you} willing to be
the guy who was brave enough to get all this started?''

``This isn't brave, it's suicide.''

``Cheap teenage melodrama, M1k3y.''

``Don't call me that!''

``What, 'M1k3y'? Why not, \emph{M1k3y}?''

I put my shoes on. I picked up my bag. I walked home.

\fancybreak{\#}

\edialog{Why I'm not jamming}

\edialog{I won't tell anyone else what to do, because I'm not
  anyone's leader, no matter what Fox News thinks.}

\edialog{But I am going to tell you what \emph{I} plan on doing. If you
  think that's the right thing to do, maybe you'll do it too.}

\edialog{I'm not jamming. Not this week. Maybe not next. It's not
  because I'm scared. It's because I'm smart enough to know that I'm
  better free than in prison. They figured out how to stop our tactic,
  so we need to come up with a new tactic. I don't care what the
  tactic is, but I want it to work. It's \emph{stupid} to get
  arrested. It's only jamming if you get away with it.}

\edialog{There's another reason not to jam. If you get caught, they
  might use you to catch your friends, and their friends, and their
  friends. They might bust your friends even if they're not on Xnet,
  because the DHS is like a maddened bull and they don't exactly worry
  if they've got the right guy.}

\edialog{I'm not telling you what to do.}

\edialog{But the DHS is dumb and we're smart. Jamming proves that
  they can't fight terrorism because it proves that they can't even
  stop a bunch of kids. If you get caught, it makes them look like
  they're smarter than us.}

\edialog{THEY AREN'T SMARTER THAN US! We are smarter than them. Let's
  be smart. Let's figure out how to jam them, no matter how many goons
  they put on the streets of our city.}

I posted it. I went to bed.

I missed Ange.

\fancybreak{\#}

Ange and I didn't speak for the next four days, including the weekend,
and then it was time to go back to school. I'd almost called her a
million times, written a thousand unsent emails and IMs.

Now I was back in Social Studies class, and Mrs Andersen greeted me
with voluble, sarcastic courtesy, asking me sweetly how my ``holiday''
had been. I sat down and mumbled nothing. I could hear Charles
snicker.

She taught us a class on Manifest Destiny, the idea that the Americans
were destined to take over the whole world (or at least that's how she
made it seem) and seemed to be trying to provoke me into saying
something so she could throw me out.

I felt the eyes of the class on me, and it reminded me of M1k3y and
the people who looked up to him. I was sick of being looked up to. I
missed Ange.

I got through the rest of the day without anything making any kind of
mark on me. I don't think I said eight words.

Finally it was over and I hit the doors, heading for the gates and the
stupid Mission and my pointless house.

I was barely out the gate when someone crashed into me. He was a young
homeless guy, maybe my age, maybe a little older. He wore a long,
greasy overcoat, a pair of baggy jeans, and rotting sneakers that
looked like they'd been through a wood-chipper. His long hair hung
over his face, and he had a pubic beard that straggled down his throat
into the collar of a no-color knit sweater.

I took this all in as we lay next to each other on the sidewalk,
people passing us and giving us weird looks. It seemed that he'd
crashed into me while hurrying down Valencia, bent over with the
burden of a split backpack that lay beside him on the pavement,
covered in tight geometric doodles in magic-marker.

He got to his knees and rocked back and forth, like he was drunk or
had hit his head.

``Sorry buddy,'' he said. ``Didn't see you. You hurt?''

I sat up too. Nothing felt hurt.

``Um. No, it's OK.''

He stood up and smiled. His teeth were shockingly white and straight,
like an ad for an orthodontic clinic. He held his hand out to me and
his grip was strong and firm.

``I'm really sorry.'' His voice was also clear and intelligent. I'd
expected him to sound like the drunks who talked to themselves as they
roamed the Mission late at night, but he sounded like a knowledgeable
bookstore clerk.

``It's no problem,'' I said.

He stuck out his hand again.

``Zeb,'' he said.

``Marcus,'' I said.

``A pleasure, Marcus,'' he said. ``Hope to run into you again sometime!''

Laughing, he picked up his backpack, turned on his heel and hurried
away.

\fancybreak{\#}

I walked the rest of the way home in a bemused fug. Mom was at the
kitchen table and we had a little chat about nothing at all, the way
we used to do, before everything changed.

I took the stairs up to my room and flopped down in my chair. For
once, I didn't want to login to the Xnet. I'd checked in that morning
before school to discover that my note had created a gigantic
controversy among people who agreed with me and people who were
righteously pissed that I was telling them to back off from their
beloved sport.

I had three thousand projects I'd been in the middle of when it had
all started. I was building a pinhole camera out of legos, I'd been
playing with aerial kite photography using an old digital camera with
a trigger hacked out of silly putty that was stretched out at launch
and slowly snapped back to its original shape, triggering the shutter
at regular intervals. I had a vacuum tube amp I'd been building into
an ancient, rusted, dented olive-oil tin that looked like an
archaeological find -- once it was done, I'd planned to build in a
dock for my phone and a set of 5.1 surround-sound speakers out of
tuna-fish cans.

I looked over my workbench and finally picked up the pinhole
camera. Methodically snapping legos together was just about my speed.

I took off my watch and the chunky silver two-finger ring that showed
a monkey and a ninja squaring off to fight and dropped them into the
little box I used for all the crap I load into my pockets and around
my neck before stepping out for the day: phone, wallet, keys,
wifinder, change, batteries, retractable cables\ldots I dumped it all out
into the box, and found myself holding something I didn't remember
putting in there in the first place.

It was a piece of paper, grey and soft as flannel, furry at the edges
where it had been torn away from some larger piece of paper. It was
covered in the tiniest, most careful handwriting I'd ever seen. I
unfolded it and held it up. The writing covered both sides, running
down from the top left corner of one side to a crabbed signature at
the bottom right corner of the other side.

The signature read, simply: ZEB.

I picked it up and started to read.

\edialog{Dear Marcus}

\edialog{You don't know me but I know you. For the past three months,
  since the Bay Bridge was blown up, I have been imprisoned on
  Treasure Island. I was in the yard on the day you talked to that
  Asian girl and got tackled. You were brave. Good on you.}

\edialog{I had a burst appendix the day afterward and ended up in the
  infirmary. In the next bed was a guy named Darryl. We were both in
  recovery for a long time and by the time we got well, we were too
  much of an embarrassment to them to let go.}

\edialog{So they decided we must really be guilty. They questioned us
  every day. You've been through their questioning, I know. Imagine it
  for months. Darryl and I ended up cell-mates. We knew we were
  bugged, so we only talked about inconsequentialities. But at night,
  when we were in our cots, we would softly tap out messages to each
  other in Morse code (I knew my HAM radio days would come in useful
  sometime).}

\edialog{At first, their questions to us were just the same crap as
  ever, who did it, how'd they do it. But after a little while, they
  switched to asking us about the Xnet. Of course, we'd never heard of
  it. That didn't stop them asking.}

\edialog{Darryl told me that they brought him arphid cloners, Xboxes,
  all kinds of technology and demanded that he tell them who used
  them, where they learned to mod them. Darryl told me about your
  games and the things you learned.}

\edialog{Especially: The DHS asked us about our friends. Who did we
  know?  What were they like? Did they have political feelings? Had
  they been in trouble at school? With the law?}

\edialog{We call the prison Gitmo-by-the-Bay. It's been a week since
  I got out and I don't think that anyone knows that their sons and
  daughters are imprisoned in the middle of the Bay. At night we could
  hear people laughing and partying on the mainland.}

\edialog{I got out last week. I won't tell you how, in case this
  falls into the wrong hands. Maybe others will take my route.}

\edialog{Darryl told me how to find you and made me promise to tell
  you what I knew when I got back. Now that I've done that I'm out of
  here like last year. One way or another, I'm leaving this
  country. Screw America.}

\edialog{Stay strong. They're scared of you. Kick them for me. Don't
  get caught.}

\edialog{Zeb}

There were tears in my eyes as I finished the note. I had a disposable
lighter somewhere on my desk that I sometimes used to melt the
insulation off of wires, and I dug it out and held it to the note. I
knew I owed it to Zeb to destroy it and make sure no one else ever saw
it, in case it might lead them back to him, wherever he was going.

I held the flame and the note, but I couldn't do it.

Darryl.

With all the crap with the Xnet and Ange and the DHS, I'd almost
forgotten he existed. He'd become a ghost, like an old friend who'd
moved away or gone on an exchange program. All that time, they'd been
questioning him, demanding that he rat me out, explain the Xnet, the
jammers. He'd been on Treasure Island, the abandoned military base
that was halfway along the demolished span of the Bay Bridge. He'd
been so close I could have swam to him.

I put the lighter down and re-read the note. By the time it was done,
I was weeping, sobbing. It all came back to me, the lady with the
severe haircut and the questions she'd asked and the reek of piss and
the stiffness of my pants as the urine dried them into coarse canvas.

``Marcus?''

My door was ajar and my mother was standing in it, watching me with a
worried look. How long had she been there?

I armed the tears away from my face and snorted up the snot. ``Mom,'' I
said. ``Hi.''

She came into my room and hugged me. ``What is it? Do you need to
talk?''

The note lay on the table.

``Is that from your girlfriend? Is everything all right?''

She'd given me an out. I could just blame it all on problems with Ange
and she'd leave my room and leave me alone. I opened my mouth to do
just that, and then this came out:

``I was in jail. After the bridge blew. I was in jail for that whole
time.''

The sobs that came then didn't sound like my voice. They sounded like
an animal noise, maybe a donkey or some kind of big cat noise in the
night. I sobbed so my throat burned and ached with it, so my chest
heaved.

Mom took me in her arms, the way she used to when I was a little boy,
and she stroked my hair, and she murmured in my ear, and rocked me,
and gradually, slowly, the sobs dissipated.

I took a deep breath and Mom got me a glass of water. I sat on the
edge of my bed and she sat in my desk chair and I told her everything.

Everything.

Well, most of it.

\chapter{Chapter 16}

\epigraph{This chapter is dedicated to San Francisco's Booksmith,
  ensconced in the storied Haight-Ashbury neighborhood, just a few
  doors down from the Ben and Jerry's at the exact corner of Haight
  and Ashbury. The Booksmith folks really know how to run an author
  event -- when I lived in San Francisco, I used to go down all the
  time to hear incredible writers speak (William Gibson was
  unforgettable). They also produce little baseball-card-style trading
  cards for each author -- I have two from my own appearances there.}
{Booksmith\footnote{\url{http://thebooksmith.booksense.com}} 1644 Haight St. San
Francisco CA 94117 USA +1 415 863 8688}

At first Mom looked shocked, then outraged, and finally she gave up
altogether and just let her jaw hang open as I took her through the
interrogation, pissing myself, the bag over my head, Darryl. I showed
her the note.

``Why --?''

In that single syllable, every recrimination I'd dealt myself in the
night, every moment that I'd lacked the bravery to tell the world what
it was really about, why I was really fighting, what had really
inspired the Xnet.

I sucked in a breath.

``They told me I'd go to jail if I talked about it. Not just for a few
days. Forever. I was -- I was scared.''

Mom sat with me for a long time, not saying anything. Then, ``What
about Darryl's father?''

She might as well have stuck a knitting needle in my chest. Darryl's
father. He must have assumed that Darryl was dead, long dead.

And wasn't he? After the DHS has held you illegally for three months,
would they ever let you go?

But Zeb got out. Maybe Darryl would get out. Maybe me and the Xnet
could help get Darryl out.

``I haven't told him,'' I said.

Now Mom was crying. She didn't cry easily. It was a British thing. It
made her little hiccoughing sobs much worse to hear.

``You will tell him,'' she managed. ``You will.''

``I will.''

``But first we have to tell your father.''

\fancybreak{\#}

Dad no longer had any regular time when he came home. Between his
consulting clients -- who had lots of work now that the DHS was
shopping for data-mining startups on the peninsula -- and the long
commute to Berkeley, he might get home any time between 6PM and
midnight.

Tonight Mom called him and told him he was coming home \emph{right now}. He
said something and she just repeated it: \emph{right now}.

When he got there, we had arranged ourselves in the living room with
the note between us on the coffee table.

It was easier to tell, the second time. The secret was getting
lighter. I didn't embellish, I didn't hide anything. I came clean.

I'd heard of coming clean before but I'd never understood what it
meant until I did it. Holding in the secret had dirtied me, soiled my
spirit. It had made me afraid and ashamed. It had made me into all the
things that Ange said I was.

Dad sat stiff as a ramrod the whole time, his face carved of
stone. When I handed him the note, he read it twice and then set it
down carefully.

He shook his head and stood up and headed for the front door.

``Where are you going?'' Mom asked, alarmed.

``I need a walk,'' was all he managed to gasp, his voice breaking.

We stared awkwardly at each other, Mom and me, and waited for him to
come home. I tried to imagine what was going on in his head. He'd been
such a different man after the bombings and I knew from Mom that what
had changed him were the days of thinking I was dead. He'd come to
believe that the terrorists had nearly killed his son and it had made
him crazy.

Crazy enough to do whatever the DHS asked, to line up like a good
little sheep and let them control him, drive him.

Now he knew that it was the DHS that had imprisoned me, the DHS that
had taken San Francisco's children hostage in Gitmo-by-the-Bay. It
made perfect sense, now that I thought of it. Of course it had been
Treasure Island where I'd been kept. Where else was a ten-minute
boat-ride from San Francisco?

When Dad came back, he looked angrier than he ever had in his life.

``You should have told me!'' he roared.

Mom interposed herself between him and me. ``You're blaming the wrong
person,'' she said. ``It wasn't Marcus who did the kidnapping and the
intimidation.''

He shook his head and stamped. ``I'm not blaming Marcus. I know
\emph{exactly} who's to blame. Me. Me and the stupid DHS. Get your shoes
on, grab your coats.''

``Where are we going?''

``To see Darryl's father. Then we're going to Barbara Stratford's
place.''

\fancybreak{\#}

I knew the name Barbara Stratford from somewhere, but I couldn't
remember where. I thought that maybe she was an old friend of my
parents, but I couldn't exactly place her.

Meantime, I was headed for Darryl's father's place. I'd never really
felt comfortable around the old man, who'd been a Navy radio operator
and ran his household like a tight ship. He'd taught Darryl Morse code
when he was a kid, which I'd always thought was cool. It was one of
the ways I knew that I could trust Zeb's letter. But for every cool
thing like Morse code, Darryl's father had some crazy military
discipline that seemed to be for its own sake, like insisting on
hospital corners on the beds and shaving twice a day. It drove Darryl
up the wall.

Darryl's mother hadn't liked it much either, and had taken off back to
her family in Minnesota when Darryl was ten -- Darryl spent his
summers and Christmases there.

I was sitting in the back of the car, and I could see the back of
Dad's head as he drove. The muscles in his neck were tense and kept
jumping around as he ground his jaws.

Mom kept her hand on his arm, but no one was around to comfort me. If
only I could call Ange. Or Jolu. Or Van. Maybe I would when the day
was done.

``He must have buried his son in his mind,'' Dad said, as we whipped up
through the hairpin curves leading up Twin Peaks to the little cottage
that Darryl and his father shared. The fog was on Twin Peaks, the way
it often was at night in San Francisco, making the headlamps reflect
back on us. Each time we swung around a corner, I saw the valleys of
the city laid out below us, bowls of twinkling lights that shifted in
the mist.

``Is this the one?''

``Yes,'' I said. ``This is it.'' I hadn't been to Darryl's in months, but
I'd spent enough time here over the years to recognize it right off.

The three of us stood around the car for a long moment, waiting to see
who would go and ring the doorbell. To my surprise, it was me.

I rang it and we all waited in held-breath silence for a minute. I
rang it again. Darryl's father's car was in the driveway, and we'd
seen a light burning in the living room. I was about to ring a third
time when the door opened.

``Marcus?'' Darryl's father wasn't anything like I remembered
him. Unshaven, in a housecoat and bare feet, with long toenails and
red eyes. He'd gained weight, and a soft extra chin wobbled beneath
the firm military jaw. His thin hair was wispy and disordered.

``Mr Glover,'' I said. My parents crowded into the door behind me.

``Hello, Ron,'' my mother said.

``Ron,'' my father said.

``You too? What's going on?''

``Can we come in?''

\fancybreak{\#}

His living room looked like one of those news-segments they show about
abandoned kids who spend a month locked in before they're rescued by
the neighbors: frozen meal boxes, empty beer cans and juice bottles,
moldy cereal bowls and piles of newspapers. There was a reek of cat
piss and litter crunched underneath our feet. Even without the cat
piss, the smell was incredible, like a bus-station toilet.

The couch was made up with a grimy sheet and a couple of greasy
pillows and the cushions had a dented, much-slept-upon look.

We all stood there for a long silent moment, embarrassment
overwhelming every other emotion. Darryl's father looked like he
wanted to die.

Slowly, he moved aside the sheets from the sofa and cleared the
stacked, greasy food-trays off of a couple of the chairs, carrying
them into the kitchen, and, from the sound of it, tossing them on the
floor.

We sat gingerly in the places he'd cleared, and then he came back and
sat down too.

``I'm sorry,'' he said vaguely. ``I don't really have any coffee to offer
you. I'm having more groceries delivered tomorrow so I'm running low
--''

``Ron,'' my father said. ``Listen to us. We have something to tell you,
and it's not going to be easy to hear.''

He sat like a statue as I talked. He glanced down at the note, read it
without seeming to understand it, then read it again. He handed it
back to me.

He was trembling.

``He's --''

``Darryl is alive,'' I said. ``Darryl is alive and being held prisoner on
Treasure Island.''

He stuffed his fist in his mouth and made a horrible groaning sound.

``We have a friend,'' my father said. ``She writes for the \emph{Bay
Guardian}. An investigative reporter.''

That's where I knew the name from. The free weekly \emph{Guard\-ian} often
lost its reporters to bigger daily papers and the Internet, but
Barbara Stratford had been there forever. I had a dim memory of having
dinner with her when I was a kid.

``We're going there now,'' my mother said. ``Will you come with us, Ron?
Will you tell her Darryl's story?''

He put his face in his hands and breathed deeply. Dad tried to put his
hand on his shoulders, but Mr Glover shook it off violently.

``I need to clean myself up,'' he said. ``Give me a minute.''

Mr Glover came back downstairs a changed man. He'd shaved and gelled
his hair back, and had put on a crisp military dress uniform with a
row of campaign ribbons on the breast. He stopped at the foot of the
stairs and kind of gestured at it.

``I don't have much clean stuff that's presentable at the moment. And
this seemed appropriate. You know, if she wanted to take pictures.''

He and Dad rode up front and I got in the back, behind him. Up close,
he smelled a little of beer, like it was coming through his pores.

\fancybreak{\#}

It was midnight by the time we rolled into Barbara Stratford's
driveway. She lived out of town, down in Mountain View, and as we sped
down the 101, none of us said a word. The high-tech buildings
alongside the highway streamed past us.

This was a different Bay Area to the one I lived in, more like the
suburban America I sometimes saw on TV. Lots of freeways and
subdivisions of identical houses, towns where there weren't any
homeless people pushing shopping carts down the sidewalk -- there
weren't even sidewalks!

Mom had phoned Barbara Stratford while we were waiting for Mr Glover
to come downstairs. The journalist had been sleeping, but Mom had been
so wound up she forgot to be all British and embarrassed about waking
her up. Instead, she just told her, tensely, that she had something to
talk about and that it had to be in person.

When we rolled up to Barbara Stratford's house, my first thought was
of the Brady Bunch place -- a low ranch house with a brick baffle in
front of it and a neat, perfectly square lawn. There was a kind of
abstract tile pattern on the baffle, and an old-fashioned UHF TV
antenna rising from behind it. We wandered around to the entrance and
saw that there were lights on inside already.

The writer opened the door before we had a chance to ring the
bell. She was about my parents' age, a tall thin woman with a
hawk-like nose and shrewd eyes with a lot of laugh-lines. She was
wearing a pair of jeans that were hip enough to be seen at one of the
boutiques on Valencia Street, and a loose Indian cotton blouse that
hung down to her thighs. She had small round glasses that flashed in
her hallway light.

She smiled a tight little smile at us.

``You brought the whole clan, I see,'' she said.

Mom nodded. ``You'll understand why in a minute,'' she said. Mr Glover
stepped from behind Dad.

``And you called in the Navy?''

``All in good time.''

We were introduced one at a time to her. She had a firm handshake and
long fingers.

Her place was furnished in Japanese minimalist style, just a few
precisely proportioned, low pieces of furniture, large clay pots of
bamboo that brushed the ceiling, and what looked like a large, rusted
piece of a diesel engine perched on top of a polished marble plinth. I
decided I liked it. The floors were old wood, sanded and stained, but
not filled, so you could see cracks and pits underneath the varnish. I
\emph{really} liked that, especially as I walked over it in my stocking
feet.

``I have coffee on,'' she said. ``Who wants some?''

We all put up our hands. I glared defiantly at my parents.

``Right,'' she said.

She disappeared into another room and came back a moment later bearing
a rough bamboo tray with a half-gallon thermos jug and six cups of
precise design but with rough, sloppy decorations. I liked those too.

``Now,'' she said, once she'd poured and served. ``It's very good to see
you all again. Marcus, I think the last time I saw you, you were maybe
seven years old. As I recall, you were very excited about your new
video games, which you showed me.''

I didn't remember it at all, but that sounded like what I'd been into
at seven. I guessed it was my Sega Dreamcast.

She produced a tape-recorder and a yellow pad and a pen, and twirled
the pen. ``I'm here to listen to whatever you tell me, and I can
promise you that I'll take it all in confidence. But I can't promise
that I'll do anything with it, or that it's going to get published.''
The way she said it made me realize that my Mom had called in a pretty
big favor getting this lady out of bed, friend or no friend. It must
be kind of a pain in the ass to be a big-shot investigative
reporter. There were probably a million people who would have liked
her to take up her cause.

Mom nodded at me. Even though I'd told the story three times that
night, I found myself tongue-tied. This was different from telling my
parents. Different from telling Darryl's father. This -- this would
start a new move in the game.

I started slowly, and watched Barbara take notes. I drank a whole cup
of coffee just explaining what ARGing was and how I got out of school
to play. Mom and Dad and Mr Glover all listened intently to this
part. I poured myself another cup and drank it on the way to
explaining how we were taken in. By the time I'd run through the whole
story, I'd drained the pot and I needed a piss like a race-horse.

Her bathroom was just as stark as the living-room, with a brown,
organic soap that smelled like clean mud. I came back in and found the
adults quietly watching me.

Mr Glover told his story next. He didn't have anything to say about
what had happened, but he explained that he was a veteran and that his
son was a good kid. He talked about what it felt like to believe that
his son had died, about how his ex-wife had had a collapse when she
found out and ended up in a hospital. He cried a little, unashamed,
the tears streaming down his lined face and darkening the collar of
his dress-uniform.

When it was all done, Barbara went into a different room and came back
with a bottle of Irish whiskey. ``It's a Bushmills 15 year old rum-cask
aged blend,'' she said, setting down four small cups. None for me. ``It
hasn't been sold in ten years. I think this is probably an appropriate
time to break it out.''

She poured them each a small glass of the liquor, then raised hers and
sipped at it, draining half the glass. The rest of the adults followed
suit. They drank again, and finished the glasses. She poured them new
shots.

``All right,'' she said. ``Here's what I can tell you right now. I
believe you. Not just because I know you, Lillian. The story sounds
right, and it ties in with other rumors I've heard. But I'm not going
to be able to just take your word for it. I'm going to have to
investigate every aspect of this, and every element of your lives and
stories. I need to know if there's anything you're not telling me,
anything that could be used to discredit you after this comes to
light. I need everything. It could take weeks before I'm ready to
publish.

``You also need to think about your safety and this Darryl's safety. If
he's really an 'un-person' then bringing pressure to bear on the DHS
could cause them to move him somewhere much further away. Think
Syria. They could also do something much worse.'' She let that hang in
the air. I knew she meant that they might kill him.

``I'm going to take this letter and scan it now. I want pictures of the
two of you, now and later -- we can send out a photographer, but I
want to document this as thoroughly as I can tonight, too.''

I went with her into her office to do the scan. I'd expected a
stylish, low-powered computer that fit in with her decor, but instead,
her spare-bedroom/office was crammed with top-of-the-line PCs, big
flat-panel monitors, and a scanner big enough to lay a whole sheet of
newsprint on. She was fast with it all, too. I noted with some
approval that she was running ParanoidLinux. This lady took her job
seriously.

The computers' fans set up an effective white-noise shield, but even
so, I closed the door and moved in close to her.

``Um, Barbara?''

``Yes?''

``About what you said, about what might be used to discredit me?''

``Yes?''

``What I tell you, you can't be forced to tell anyone else, right?''

``In theory. Let me put it this way. I've gone to jail twice rather
than rat out a source.''

``OK, OK. Good. Wow. Jail. Wow. OK.'' I took a deep breath. ``You've
heard of Xnet? Of M1k3y?''

``Yes?''

``I'm M1k3y.''

``Oh,'' she said. She worked the scanner and flipped the note over to
get the reverse. She was scanning at some unbelievable resolution,
10,000 dots per inch or higher, and on-screen it was like the output
of an electron-tunneling microscope.

``Well, that does put a different complexion on this.''

``Yeah,'' I said. ``I guess it does.''

``Your parents don't know.''

``Nope. And I don't know if I want them to.''

``That's something you're going to have to work out. I need to think
about this. Can you come by my office? I'd like to talk to you about
what this means, exactly.''

``Do you have an Xbox Universal? I could bring over an installer.''
 
``Yes, I'm sure that can be arranged. When you come by, tell the
receptionist that you're Mr Brown, to see me. They know what that
means. No note will be taken of you coming, and all the security
camera footage for the day will be automatically scrubbed and the
cameras deactivated until you leave.''

``Wow,'' I said. ``You think like I do.''

She smiled and socked me in the shoulder. ``Kiddo, I've been at this
game for a hell of a long time. So far, I've managed to spend more
time free than behind bars. Paranoia is my friend.''

\fancybreak{\#}

I was like a zombie the next day in school. I'd totaled about three
hours of sleep, and even three cups of the Turk's caffeine mud failed
to jump-start my brain. The problem with caffeine is that it's too
easy to get acclimated to it, so you have to take higher and higher
doses just to get above normal.

I'd spent the night thinking over what I had to do. It was like runnin
though a maze of twisty little passages, all alike, every one leading
to the same dead end. When I went to Barbara, it would be over for
me. That was the outcome, no matter how I thought about it.

By the time the school day was over, all I wanted was to go home and
crawl into bed. But I had an appointment at the \emph{Bay Guardian}, down
on the waterfront. I kept my eyes on my feet as I wobbled out the
gate, and as I turned into 24th Street, another pair of feet fell into
step with me. I recognized the shoes and stopped.

``Ange?''

She looked like I felt. Sleep-deprived and raccoon-eyed, with sad
brackets in the corners of her mouth.

``Hi there,'' she said. ``Surprise. I gave myself French Leave from
school. I couldn't concentrate anyway.''

``Um,'' I said.

``Shut up and give me a hug, you idiot.''

I did. It felt good. Better than good. It felt like I'd amputated part
of myself and it had been reattached.

``I love you, Marcus Yallow.''

``I love you, Angela Carvelli.''

``OK,'' she said breaking it off. ``I liked your post about why you're
not jamming. I can respect it. What have you done about finding a way
to jam them without getting caught?''

``I'm on my way to meet an investigative journalist who's going to
publish a story about how I got sent to jail, how I started Xnet, and
how Darryl is being illegally held by the DHS at a secret prison on
Treasure Island.''

``Oh.'' She looked around for a moment. ``Couldn't you think of anything,
you know, ambitious?''

``Want to come?''

``I am coming, yes. And I would like you to explain this in detail if
you don't mind.''

After all the re-tellings, this one, told as we walked to Potrero
Avenue and down to 15th Street, was the easiest. She held my hand and
squeezed it often.

We took the stairs up to the \emph{Bay Guardian}'s offices two at a
time. My heart was pounding. I got to the reception desk and told the
bored girl behind it, ``I'm here to see Barbara Stratford. My name is
Mr Green.''

``I think you mean Mr Brown?''

``Yeah,'' I said, and blushed. ``Mr Brown.''

She did something at her computer, then said, ``Have a seat. Barbara
will be out in a minute. Can I get you anything?''

``Coffee,'' we both said in unison. Another reason to love Ange: we were
addicted to the same drug.

The receptionist -- a pretty latina woman only a few years older than
us, dressed in Gap styles so old they were actually kind of
hipster-retro -- nodded and stepped out and came back with a couple of
cups bearing the newspaper's masthead.

We sipped in silence, watching visitors and reporters come and
go. Finally, Barbara came to get us. She was wearing practically the
same thing as the night before. It suited her. She quirked an eyebrow
at me when she saw that I'd brought a date.

``Hello,'' I said. ``Um, this is --''

``Ms Brown,'' Ange said, extending a hand. Oh, yeah, right, our
identities were supposed to be a secret. ``I work with Mr Green.'' She
elbowed me lightly.

``Let's go then,'' Barbara said, and led us back to a board-room with
long glass walls with their blinds drawn shut. She set down a tray of
Whole Foods organic Oreo clones, a digital recorder, and another
yellow pad.

``Do you want to record this too?'' she asked.

Hadn't actually thought of that. I could see why it would be useful if
I wanted to dispute what Barbara printed, though. Still, if I couldn't
trust her to do right by me, I was doomed anyway.

``No, that's OK,'' I said.

``Right, let's go. Young lady, my name is Barbara Stratford and I'm an
investigative reporter. I gather you know why I'm here, and I'm
curious to know why you're here.''

``I work with Marcus on the Xnet,'' she said. ``Do you need to know my
name?''

``Not right now, I don't,'' Barbara said. ``You can be anonymous if you'd
like. Marcus, I asked you to tell me this story because I need to know
how it plays with the story you told me about your friend Darryl and
the note you showed me. I can see how it would be a good adjunct; I
could pitch this as the origin of the Xnet. 'They made an enemy
they'll never forget,' that sort of thing. But to be honest, I'd
rather not have to tell that story if I don't have to.

``I'd rather have a nice clean tale about the secret prison on our
doorstep, without having to argue about whether the prisoners there
are the sort of people likely to walk out the doors and establish an
underground movement bent on destabilizing the federal government. I'm
sure you can understand that.''

I did. If the Xnet was part of the story, some people would say, see,
they need to put guys like that in jail or they'll start a riot.

``This is your show,'' I said. ``I think you need to tell the world about
Darryl. When you do that, it's going to tell the DHS that I've gone
public and they're going to go after me. Maybe they'll figure out then
that I'm involved with the Xnet. Maybe they'll connect me to M1k3y. I
guess what I'm saying is, once you publish about Darryl, it's all over
for me no matter what. I've made my peace with that.''

``As good be hanged for a sheep as a lamb,'' she said. ``Right. Well,
that's settled. I want the two of you to tell me everything you can
about the founding and operation of the Xnet, and then I want a
demonstration. What do you use it for? Who else uses it? How did it
spread? Who wrote the software? Everything.''

``This'll take a while,'' Ange said.

``I've got a while,'' Barbara said. She drank some coffee and ate a fake
Oreo. ``This could be the most important story of the War on
Terror. This could be the story that topples the government. When you
have a story like this, you take it very carefully.''

\chapter{Chapter 17}

\epigraph{This chapter is dedicated to Waterstone's, the national UK
  bookselling chain. Waterstone's is a chain of stores, but each one
  has the feel of a great independent store, with tons of personality,
  great stock (especially audiobooks!), and knowledgeable staff.}
{Waterstones \url{http://www.waterstones.com}}

So we told her. I found it really fun, actually. Teaching people how
to use technology is always exciting. It's so cool to watch people
figure out how the technology around them can be used to make their
lives better. Ange was great too -- we made an excellent team. We'd
trade off explaining how it all worked. Barbara was pretty good at
this stuff to begin with, of course.

It turned out that she'd covered the crypto wars, the period in the
early nineties when civil liberties groups like the Electronic
Frontier Foundation fought for the right of Americans to use strong
crypto. I dimly knew about that period, but Barabara explained it in a
way that made me get goose-pimples.

It's unbelievable today, but there was a time when the government
classed crypto as a munition and made it illegal for anyone to export
or use it on national security grounds. Get that? We used to have
illegal \emph{math} in this country.

The National Security Agency were the real movers behind the ban. They
had a crypto standard that they said was strong enough for bankers and
their customers to use, but not so strong that the mafia would be able
to keep its books secret from them. The standard, DES-56, was said to
be practically unbreakable. Then one of EFF's millionaire co-founders
built a \$250,000 DES-56 cracker that could break the cipher in two
hours.

Still the NSA argued that it should be able to keep American citizens
from possessing secrets it couldn't pry into. Then EFF dealt its
death-blow. In 1995, they represented a Berkeley mathematics grad
student called Dan Bernstein in court. Bernstein had written a crypto
tutorial that contained computer code that could be used to make a
cipher stronger than DES-56. Millions of times stronger. As far as the
NSA was concerned, that made his article into a weapon, and therefore
unpublishable.

Well, it may be hard to get a judge to understand crypto and what it
means, but it turned out that the average Appeals Court judge isn't
real enthusiastic about telling grad students what kind of articles
they're allowed to write. The crypto wars ended with a victory for the
good guys when the 9th Circuit Appellate Division Court ruled that
code was a form of expression protected under the First Amendment --
``Congress shall make no law abridging the freedom of speech.'' If
you've ever bought something on the Internet, or sent a secret
message, or checked your bank-balance, you used crypto that EFF
legalized. Good thing, too: the NSA just isn't that smart. Anything
they know how to crack, you can be sure that terrorists and mobsters
can get around too.

Barbara had been one of the reporters who'd made her reputation from
covering the issue. She'd cut her teeth covering the tail end of the
civil rights movement in San Francisco, and she recognized the
similarity between the fight for the Constitution in the real world
and the fight in cyberspace.

So she got it. I don't think I could have explained this stuff to my
parents, but with Barbara it was easy. She asked smart questions about
our cryptographic protocols and security procedures, sometimes asking
stuff I didn't know the answer to -- sometimes pointing out potential
breaks in our procedure.

We plugged in the Xbox and got it online. There were four open WiFi
nodes visible from the board room and I told it to change between them
at random intervals. She got this too -- once you were actually
plugged into the Xnet, it was just like being on the Internet, only
some stuff was a little slower, and it was all anonymous and
unsniffable.

``So now what?'' I said as we wound down. I'd talked myself dry and I
had a terrible acid feeling from the coffee. Besides, Ange kept
squeezing my hand under the table in a way that made me want to break
away and find somewhere private to finish making up for our first
fight.

``Now I do journalism. You go away and I research all the things you've
told me and try to confirm them to the extent that I can. I'll let you
see what I'm going to publish and I'll let you know when it's going to
go live. I'd prefer that you \emph{not} talk about this with anyone else
now, because I want the scoop and because I want to make sure that I
get the story before it goes all muddy from press speculation and DHS
spin.

``I \emph{will} have to call the DHS for comment before I go to press, but
I'll do that in a way that protects you to whatever extent
possible. I'll also be sure to let you know before that happens.

``One thing I need to be clear on: this isn't your story anymore. It's
mine. You were very generous to give it to me and I'll try to repay
the gift, but you don't get the right to edit anything out, to change
it, or to stop me. This is now in motion and it won't stop. Do you
understand that?''

I hadn't thought about it in those terms but once she said it, it was
obvious. It meant that I had launched and I wouldn't be able to recall
the rocket. It was going to fall where it was aimed, or it would go
off course, but it was in the air and couldn't be changed
now. Sometime in the near future, I would stop being Marcus -- I would
be a public figure. I'd be the guy who blew the whistle on the DHS.

I'd be a dead man walking.

I guess Ange was thinking along the same lines, because she'd gone a
color between white and green.

``Let's get out of here,'' she said.

\fancybreak{\#}

Ange's mom and sister were out again, which made it easy to decide
where we were going for the evening. It was past supper time, but my
parents had known that I was meeting with Barbara and wouldn't give me
any grief if I came home late.

When we got to Ange's, I had no urge to plug in my Xbox. I had had all
the Xnet I could handle for one day. All I could think about was Ange,
Ange, Ange. Living without Ange. Knowing Ange was angry with me. Ange
never going to talk to me again. Ange never going to kiss me again.

She'd been thinking the same. I could see it in her eyes as we shut
the door to her bedroom and looked at each other. I was hungry for
her, like you'd hunger for dinner after not eating for days. Like
you'd thirst for a glass of water after playing soccer for three hours
straight.

Like none of that. It was more. It was something I'd never felt
before. I wanted to eat her whole, devour her.

Up until now, she'd been the sexual one in our relationship. I'd let
her set and control the pace. It was amazingly erotic to have \emph{her}
grab \emph{me} and take off my shirt, drag my face to hers.

But tonight I couldn't hold back. I wouldn't hold back.

The door clicked shut and I reached for the hem of her t-shirt and
yanked, barely giving her time to lift her arms as I pulled it over
her head. I tore my own shirt over my head, listening to the cotton
crackle as the stitches came loose.

Her eyes were shining, her mouth open, her breathing fast and
shallow. Mine was too, my breath and my heart and my blood all roaring
in my ears.

I took off the rest of our clothes with equal zest, throwing them into
the piles of dirty and clean laundry on the floor. There were books
and papers all over the bed and I swept them aside. We landed on the
unmade bedclothes a second later, arms around one another, squeezing
like we would pull ourselves right through one another. She moaned
into my mouth and I made the sound back, and I felt her voice buzz in
my vocal chords, a feeling more intimate than anything I'd ever felt
before.

She broke away and reached for the bedstand. She yanked open the
drawer and threw a white pharmacy bag on the bed before me. I looked
inside. Condoms. Trojans. One dozen spermicidal. Still sealed. I
smiled at her and she smiled back and I opened the box.

\fancybreak{\#}

I'd thought about what it would be like for years. A hundred times a
day I'd imagined it. Some days, I'd thought of practically nothing
else.

It was nothing like I expected. Parts of it were better. Parts of it
were lots worse. While it was going on, it felt like an
eternity. Afterwards, it seemed to be over in the blink of an eye.

Afterwards, I felt the same. But I also felt different. Something had
changed between us.

It was weird. We were both shy as we put our clothes on and puttered
around the room, looking away, not meeting each other's eyes. I
wrapped the condom in a kleenex from a box beside the bed and took it
into the bathroom and wound it with toilet paper and stuck it deep
into the trash-can.

When I came back in, Ange was sitting up in bed and playing with her
Xbox. I sat down carefully beside her and took her hand. She turned to
face me and smiled. We were both worn out, trembly.

``Thanks,'' I said.

She didn't say anything. She turned her face to me. She was grinning
hugely, but fat tears were rolling down her cheeks.

I hugged her and she grabbed tightly onto me. ``You're a good man,
Marcus Yallow,'' she whispered. ``Thank you.''

I didn't know what to say, but I squeezed her back. Finally, we
parted. She wasn't crying any more, but she was still smiling.

She pointed at my Xbox, on the floor beside the bed. I took the
hint. I picked it up and plugged it in and logged in.

Same old same old. Lots of email. The new posts on the blogs I read
streamed in. Spam. God did I get a lot of spam. My Swedish mailbox was
repeatedly ``joe-jobbed'' -- used as the return address for spams sent
to hundreds of millions of Internet accounts, so that all the bounces
and angry messages came back to me. I didn't know who was behind
it. Maybe the DHS trying to overwhelm my mailbox. Maybe it was just
people pranking. The Pirate Party had pretty good filters, though, and
they gave anyone who wanted it 500 gigabytes of email storage, so I
wasn't likely to be drowned any time soon.

I filtered it all out, hammering on the delete key. I had a separate
mailbox for stuff that came in encrypted to my public key, since that
was likely to be Xnet-related and possibly sensitive. Spammers hadn't
figured out that using public keys would make their junk mail more
plausible yet, so for now this worked well.

There were a couple dozen encrypted messages from people in the web of
trust. I skimmed them -- links to videos and pics of new abuses from
the DHS, horror stories about near-escapes, rants about stuff I'd
blogged. The usual.

Then I came to one that was only encrypted to my public key. That
meant that no one else could read it, but I had no idea who had
written it. It said it came from Masha, which could either be a handle
or a name -- I couldn't tell which.

\edialog{M1k3y}

\edialog{You don't know me, but I know you.}

\edialog{I was arrested the day that the bridge blew. They questioned
  me. They decided I was innocent. They offered me a job: help them
  hunt down the terrorists who'd killed my neighbors.}

\edialog{It sounded like a good deal at the time. Little did I
  realize that my actual job would turn out to be spying on kids who
  resented their city being turned into a police state.}

\edialog{I infiltrated Xnet on the day it launched. I am in your web
  of trust. If I wanted to spill my identity, I could send you email
  from an address you'd trust. Three addresses, actually. I'm totally
  inside your network as only another 17-year-old can be. Some of the
  email you've gotten has been carefully chosen misinformation from me
  and my handlers.}

\edialog{They don't know who you are, but they're coming close. They
  continue to turn people, to compromise them. They mine the social
  network sites and use threats to turn kids into informants. There
  are hundreds of people working for the DHS on Xnet right now. I have
  their names, handles and keys. Private and public.}

\edialog{Within days of the Xnet launch, we went to work on
  exploiting ParanoidLinux. The exploits so far have been small and
  insubstantial, but a break is inevitable. Once we have a zero-day
  break, you're dead.}

\edialog{I think it's safe to say that if my handlers knew that I was
  typing this, my ass would be stuck in Gitmo-by-the-Bay until I was
  an old woman.}

\edialog{Even if they don't break ParanoidLinux, there are poisoned
  ParanoidXbox distros floating around. They don't match the
  checksums, but how many people look at the checksums? Besides me and
  you? Plenty of kids are already dead, though they don't know it.}

\edialog{All that remains is for my handlers to figure out the best
  time to bust you to make the biggest impact in the media. That time
  will be sooner, not later. Believe.}

\edialog{You're probably wondering why I'm telling you this.}

\edialog{I am too.}

\edialog{Here's where I come from. I signed up to fight
  terrorists. Instead, I'm spying on Americans who believe things that
  the DHS doesn't like. Not people who plan on blowing up bridges, but
  protestors. I can't do it anymore.}

\edialog{But neither can you, whether or not you know it. Like I say,
  it's only a matter of time until you're in chains on Treasure
  Island. That's not if, that's when.}

\edialog{So I'm through here. Down in Los Angeles, there are some
  people. They say they can keep me safe if I want to get out.}

\edialog{I want to get out.}

\edialog{I will take you with me, if you want to come. Better to be a
  fighter than a martyr. If you come with me, we can figure out how to
  win together. I'm as smart as you. Believe.}

\edialog{What do you say?}

\edialog{Here's my public key.}

\edialog{Masha}

\fancybreak{\#}

When in trouble or in doubt, run in circles, scream and shout.

Ever hear that rhyme? It's not good advice, but at least it's easy to
follow. I leapt off the bed and paced back and forth. My heart thudded
and my blood sang in a cruel parody of the way I'd felt when we got
home. This wasn't sexual excitement, it was raw terror.

``What?'' Ange said. ``What?''

I pointed at the screen on my side of the bed. She rolled over and
grabbed my keyboard and scribed on the touchpad with her
fingertip. She read in silence.

I paced.

``This has to be lies,'' she said. ``The DHS is playing games with your
head.''

I looked at her. She was biting her lip. She didn't look like she
believed it.

``You think?''

``Sure. They can't beat you, so they're coming after you using Xnet.''

``Yeah.''

I sat back down on the bed. I was breathing fast again.

``Chill out,'' she said. ``It's just head-games. Here.''

She never took my keyboard from me before, but now there was a new
intimacy between us. She hit reply and typed,

\edialog{Nice try.}

She was writing as M1k3y now, too. We were together in a way that was
different from before.

``Go ahead and sign it. We'll see what she says.''

I didn't know if that was the best idea, but I didn't have any better
ones. I signed it and encrypted it with my private key and the public
key Masha had provided.

The reply was instant.

\edialog{I thought you'd say something like that.}

\edialog{Here's a hack you haven't thought of. I can anonymously
  tunnel video over DNS. Here are some links to clips you might want
  to look at before you decide I'm full of it. These people are all
  recording each other, all the time, as insurance against a
  back-stab. It's pretty easy to snoop off them as they snoop on each
  other.}

\edialog{Masha}

Attached was source-code for a little program that appeared to do
exactly what Masha claimed: pull video over the Domain Name Service
protocol.

Let me back up a moment here and explain something. At the end of the
day, every Internet protocol is just a sequence of text sent back and
forth in a proscribed order. It's kind of like getting a truck and
putting a car in it, then putting a motorcycle in the car's trunk,
then attaching a bicycle to the back of the motorcycle, then hanging a
pair of Rollerblades on the back of the bike. Except that then, if you
want, you can attach the truck to the Rollerblades.

For example, take Simple Mail Transport Protocol, or SMTP, which is
used for sending email.

Here's a sample conversation between me and my mail server, sending a
message to myself:

\begin{serverlog}
\query{HELO littlebrother.com.se}

\response{250 mail.pirateparty.org.se Hello mail.pirateparty.org.se,
  pleased to meet you}

\query{MAIL FROM:m1k3y@littlebrother.com.se}

\response{250 2.1.0 m1k3y@littlebrother.com.se... Sender ok}

\query{RCPT TO:m1k3y@littlebrother.com.se}

\response{250 2.1.5 m1k3y@littlebrother.com.se... Recipient ok}

\query{DATA}

\response{354 Enter mail, end with ``.'' on a line by itself}

\query{When in trouble or in doubt, run}
\query{in circles, scream and shout}
\query{.}

\response{250 2.0.0 k5SMW0xQ006174 Message accepted for delivery}

\query{QUIT}

\response{221 2.0.0 mail.pirateparty.org.se closing connection}

\response{Connection closed by foreign host.}

\end{serverlog}

This conversation's grammar was defined in 1982 by Jon Postel, one of
the Internet's heroic forefathers, who used to literally run the most
important servers on the net under his desk at the University of
Southern California, back in the paleolithic era.

Now, imagine that you hooked up a mail-server to an IM session. You
could send an IM to the server that said 

\edialog{HELO littlebrother.com.se}

and it would reply with 

\edialog{250 mail.pirateparty.org.se Hello
mail.pirateparty.org.se, pleased to meet you.}

In other words, you
could have the same conversation over IM as you do over SMTP. With the
right tweaks, the whole mail-server business could take place inside
of a chat. Or a web-session. Or anything else.

This is called ``tunneling.'' You put the SMTP inside a chat ``tunnel.''
You could then put the chat back into an SMTP tunnel if you wanted to
be really weird, tunneling the tunnel in another tunnel.

In fact, every Internet protocol is susceptible to this process. It's
cool, because it means that if you're on a network with only Web
access, you can tunnel your mail over it. You can tunnel your favorite
P2P over it. You can even tunnel Xnet -- which itself is a tunnel for
dozens of protocols -- over it.

Domain Name Service is an interesting and ancient Internet protocol,
dating back to 1983. It's the way that your computer converts a
computer's name -- like \url{pirateparty.org.se} -- to the IP number that
computers actually use to talk to each other over the net, like
\url{204.11.50.136.} It generally works like magic, even though it's got
millions of moving parts -- every ISP runs a DNS server, as do most
governments and lots of private operators. These DNS boxes all talk to
each other all the time, making and filling requests to each other so
no matter how obscure the name is you feed to your computer, it will
be able to turn it into a number.

Before DNS, there was the HOSTS file. Believe it or not, this was a
single document that listed the name and address of \emph{every single
computer} connected to the Internet. Every computer had a copy of
it. This file was eventually too big to move around, so DNS was
invented, and ran on a server that used to live under Jon Postel's
desk. If the cleaners knocked out the plug, the entire Internet lost
its ability to find itself. Seriously.

The thing about DNS today is that it's everywhere. Every network has a
DNS server living on it, and all of those servers are configured to
talk to each other and to random people all over the Internet.

What Masha had done was figure out a way to tunnel a video-streaming
system over DNS. She was breaking up the video into billions of pieces
and hiding each of them in a normal message to a DNS server. By
running her code, I was able to pull the video from all those DNS
servers, all over the Internet, at incredible speed. It must have
looked bizarre on the network histograms, like I was looking up the
address of every computer in the world.

But it had two advantages I appreciated at once: I was able to get the
video with blinding speed -- as soon as I clicked the first link, I
started to receive full-screen pictures, without any jitter or
stuttering -- and I had no idea where it was hosted. It was totally
anonymous.

At first I didn't even clock the content of the video. I was totally
floored by the cleverness of this hack. Streaming video from DNS? That
was so smart and weird, it was practically \emph{perverted}.

Gradually, what I was seeing began to sink in.

It was a board-room table in a small room with a mirror down one
wall. I knew that room. I'd sat in that room, while Severe-Haircut
woman had made me speak my password aloud. There were five comfortable
chairs around the table, each with a comfortable person, all in DHS
uniform. I recognized Major General Graeme Sutherland, the DHS Bay
Area commander, along with Severe Haircut. The others were new to
me. They all watched a video screen at the end of the table, on which
there was an infinitely more familiar face.

Kurt Rooney was known nationally as the President's chief strategist,
the man who returned the party for its third term, and who was
steaming towards a fourth. They called him ``Ruthless'' and I'd seen a
news report once about how tight a rein he kept his staffers on,
calling them, IMing them, watching their every motion, controlling
every step. He was old, with a lined face and pale gray eyes and a
flat nose with broad, flared nostrils and thin lips, a man who looked
like he was smelling something bad all the time.

He was the man on the screen. He was talking, and everyone else was
focused on his screen, everyone taking notes as fast as they could
type, trying to look smart.

``-- say that they're angry with authority, but we need to show the
country that it's terrorists, not the government, that they need to
blame. Do you understand me? The nation does not love that city. As
far as they're concerned, it is a Sodom and Gomorrah of fags and
atheists who deserve to rot in hell. The only reason the country cares
what they think in San Francisco is that they had the good fortune to
have been blown to hell by some Islamic terrorists.

``These Xnet children are getting to the point where they might start
to be useful to us. The more radical they get, the more the rest of
the nation understands that there are threats everywhere.''

His audience finished typing.

``We can control that, I think,'' Severe Haircut Lady said. ``Our people
in the Xnet have built up a lot of influence. The Manchurian Bloggers
are running as many as fifty blogs each, flooding the chat channels,
linking to each other, mostly just taking the party line set by this
M1k3y. But they've already shown that they can provoke radical action,
even when M1k3y is putting the brakes on.''

Major General Sutherland nodded. ``We have been planning to leave them
underground until about a month before the midterms.'' I guessed that
meant the mid-term elections, not my exams. ``That's per the original
plan. But it sounds like --''

``We've got another plan for the midterms,'' Rooney said. ``Need-to-know,
of course, but you should all probably not plan on traveling for the
month before. Cut the Xnet loose now, as soon as you can. So long as
they're moderates, they're a liability. Keep them radical.''

The video cut off.

Ange and I sat on the edge of the bed, looking at the screen. Ange
reached out and started the video again. We watched it. It was worse
the second time.

I tossed the keyboard aside and got up.

``I am \emph{so sick} of being scared,'' I said. ``Let's take this to
Barbara and have her publish it all. Put it all on the net. Let them
take me away. At least I'll know what's going to happen then. At least
then I'll have a little \emph{certainty} in my life.''

Ange grabbed me and hugged me, soothed me. ``I know baby, I know. It's
all terrible. But you're focusing on the bad stuff and ignoring the
good stuff. You've created a movement. You've outflanked the jerks in
the White House, the crooks in DHS uniforms. You've put yourself in a
position where you could be responsible for blowing the lid off of the
entire rotten DHS thing.

``Sure they're out to get you. Course they are. Have you ever doubted
it for a moment? I always figured they were. But Marcus, \emph{they
  don't know who you are}. Think about that. All those people, money,
guns and spies, and you, a seventeen year old high school kid --
you're still beating them. They don't know about Barbara. They don't
know about Zeb. You've jammed them in the streets of San Francisco and
humiliated them before the world. So stop moping, all right? You're
winning.''

``They're coming for me, though. You see that. They're going to put me
in jail forever. Not even jail. I'll just disappear, like
Darryl. Maybe worse. Maybe Syria. Why leave me in San Francisco? I'm a
liability as long as I'm in the USA.''

She sat down on the bed with me.

``Yeah,'' she said. ``That.''

``That.''

``Well, you know what you have to do, right?''

``What?'' She looked pointedly at my keyboard. I could see the tears
rolling down her cheeks. ``No! You're out of your mind. You think I'm
going to run off with some nut off the Internet? Some spy?''

``You got a better idea?''

I kicked a pile of her laundry into the air. ``Whatever. Fine. I'll
talk to her some more.''

``You talk to her,'' Ange said. ``You tell her you and your girlfriend
are getting out.''

``What?''

``Shut up, dickhead. You think you're in danger? I'm in just as much
danger, Marcus. It's called guilt by association. When you go, I go.''
She had her jaw thrust out at a mutinous angle. ``You and I -- we're
together now. You have to understand that.''

We sat down on the bed together.

``Unless you don't want me,'' she said, finally, in a small voice.

``You're kidding me, right?''

``Do I look like I'm kidding?''

``There's no way I would voluntarily go without you, Ange. I could
never have asked you to come, but I'm ecstatic that you offered.''

She smiled and tossed me my keyboard.

``Email this Masha creature. Let's see what this chick can do for us.''

I emailed her, encrypting the message, waiting for a reply. Ange
nuzzled me a little and I kissed her and we necked. Something about
the danger and the pact to go together -- it made me forget the
awkwardness of having sex, made me freaking horny as hell.

We were half naked again when Masha's email arrived.

\edialog{Two of you? Jesus, like it won't be hard enough already.}

\edialog{I don't get to leave except to do field intelligence after a
  big Xnet hit. You get me? The handlers watch my every move, but I go
  off the leash when something big happens with Xnetters. I get sent
  into the field then.}

\edialog{You do something big. I get sent to it. I get us both
  out. All three of us, if you insist.}

\edialog{Make it fast, though. I can't send you a lot of email,
  understand?  They watch me. They're closing in on you. You don't
  have a lot of time. Weeks? Maybe just days.}

\edialog{I need you to get me out. That's why I'm doing this, in case
  you're wondering. I can't escape on my own. I need a big Xnet
  distraction. That's your department. Don't fail me, M1k3y, or we're
  both dead. Your girlie too.}

\edialog{Masha}

My phone rang, making us both jump. It was my mom wanting to know when
I was coming home. I told her I was on my way. She didn't mention
Barbara. We'd agreed that we wouldn't talk about any of this stuff on
the phone. That was my dad's idea. He could be as paranoid as me.

``I have to go,'' I said.

``Our parents will be --''

``I know,'' I said. ``I saw what happened to my parents when they thought
I was dead. Knowing that I'm a fugitive isn't going to be much
better. But they'd rather I be a fugitive than a prisoner. That's what
I think. Anyway, once we disappear, Barbara can publish without
worrying about getting us into trouble.''

We kissed at the door of her room. Not one of the hot, sloppy numbers
we usually did when parting ways. A sweet kiss this time. A slow
kiss. A goodbye kind of kiss.

\fancybreak{\#}

BART rides are introspective. When the train rocks back and forth and
you try not to make eye contact with the other riders and you try not
to read the ads for plastic surgery, bail bondsmen and AIDS testing,
when you try to ignore the graffiti and not look too closely at the
stuff in the carpeting. That's when your mind starts to really churn
and churn.

You rock back and forth and your mind goes over all the things you've
overlooked, plays back all the movies of your life where you're no
hero, where you're a chump or a sucker.

Your brain comes up with theories like this one:

\emph{If the DHS wanted to catch M1k3y, what better way than to lure him
into the open, panic him into leading some kind of big, public Xnet
event? Wouldn't that be worth the chance of a compromising video
leaking?}

Your brain comes up with stuff like that even when the train ride only
lasts two or three stops. When you get off, and you start moving, the
blood gets running and sometimes your brain helps you out again.

Sometimes your brain gives you solutions in addition to problems.

\chapter{Chapter 18}

\epigraph{This chapter is dedicated to Vancouver's multilingual Sophia
  Books, a diverse and exciting store filled with the best of the
  strange and exciting pop culture worlds of many lands. Sophia was
  around the corner from my hotel when I went to Van to give a talk at
  Simon Fraser University, and the Sophia folks emailed me in advance
  to ask me to drop in and sign their stock while I was in the
  neighborhood. When I got there, I discovered a treasure-trove of
  never-before-seen works in a dizzying array of languages, from
  graphic novels to thick academic treatises, presided over by
  good-natured (even slapstick) staff who so palpably enjoyed their
  jobs that it spread to every customer who stepped through the
  door.}
{Sophia Books \url{http://www.sophiabooks.com/} 450 West Hastings St.,
Vancouver, BC Canada V6B1L1 +1 604 684 0484}

There was a time when my favorite thing in the world was putting on a
cape and hanging out in hotels, pretending to be an invisible vampire
whom everyone stared at.

It's complicated, and not nearly as weird as it sounds. The Live
Action Role Playing scene combines the best aspects of D\&D with drama
club with going to sci-fi cons.

I understand that this might not make it sound as appealing to you as
it was to me when I was 14.

The best games were the ones at the Scout Camps out of town: a hundred
teenagers, boys and girls, fighting the Friday night traffic, swapping
stories, playing handheld games, showing off for hours. Then debarking
to stand in the grass before a group of older men and women in
bad-ass, home-made armor, dented and scarred, like armor must have
been in the old days, not like it's portrayed in the movies, but like
a soldier's uniform after a month in the bush.

These people were nominally paid to run the games, but you didn't get
the job unless you were the kind of person who'd do it for
free. They'd have already divided us into teams based on the
questionnaires we'd filled in beforehand, and we'd get our team
assignments then, like being called up for baseball sides.

Then you'd get your briefing packages. These were like the briefings
the spies get in the movies: here's your identity, here's your
mission, here's the secrets you know about the group.

From there, it was time for dinner: roaring fires, meat popping on
spits, tofu sizzling on skillets (it's northern California, a
vegetarian option is not optional), and a style of eating and drinking
that can only be described as quaffing.

Already, the keen kids would be getting into character. My first game,
I was a wizard. I had a bag of beanbags that represented spells --
when I threw one, I would shout the name of the spell I was casting --
fireball, magic missile, cone of light -- and the player or ``monster''
I threw it at would keel over if I connected. Or not -- sometimes we
had to call in a ref to mediate, but for the most part, we were all
pretty good about playing fair. No one liked a dice lawyer.

By bedtime, we were all in character. At 14, I wasn't super-sure what
a wizard was supposed to sound like, but I could take my cues from the
movies and novels. I spoke in slow, measured tones, keeping my face
composed in a suitably mystical expression, and thinking mystical
thoughts.

The mission was complicated, retrieving a sacred relic that had been
stolen by an ogre who was bent on subjugating the people of the land
to his will. It didn't really matter a whole lot. What mattered was
that I had a private mission, to capture a certain kind of imp to
serve as my familiar, and that I had a secret nemesis, another player
on the team who had taken part in a raid that killed my family when I
was a boy, a player who didn't know that I'd come back, bent on
revenge. Somewhere, of course, there was another player with a similar
grudge against me, so that even as I was enjoying the camaraderie of
the team, I'd always have to keep an eye open for a knife in the back,
poison in the food.

For the next two days, we played it out. There were parts of the
weekend that were like hide-and-seek, some that were like wilderness
survival exercises, some that were like solving crossword puzzles. The
game-masters had done a great job. And you really got to be friends
with the other people on the mission. Darryl was the target of my
first murder, and I put my back into it, even though he was my
pal. Nice guy. Shame I'd have to kill him.

I fireballed him as he was seeking out treasure after we wiped out a
band of orcs, playing rock-papers-scissors with each orc to determine
who would prevail in combat. This is a lot more exciting than it
sounds.

It was like summer camp for drama geeks. We talked until late at night
in tents, looked at the stars, jumped in the river when we got hot,
slapped away mosquitos. Became best friends, or lifelong enemies.

I don't know why Charles's parents sent him LARPing. He wasn't the
kind of kid who really enjoyed that kind of thing. He was more the
pulling-wings-off-flies type. Oh, maybe not. But he just was not into
being in costume in the woods. He spent the whole time mooching
around, sneering at everyone and everything, trying to convince us all
that we weren't having the good time we all felt like we were
having. You've no doubt found that kind of person before, the kind of
person who is compelled to ensure that everyone else has a rotten
time.

The other thing about Charles was that he couldn't get the hang of
simulated combat. Once you start running around the woods and playing
these elaborate, semi-military games, it's easy to get totally
adrenalized to the point where you're ready to tear out someone's
throat. This is not a good state to be in when you're carrying a prop
sword, club, pike or other utensil. This is why no one is ever allowed
to hit anyone, under any circumstances, in these games. Instead, when
you get close enough to someone to fight, you play a quick couple
rounds of rock-paper-scissors, with modifiers based on your
experience, armaments, and condition. The referees mediate
disputes. It's quite civilized, and a little weird. You go running
after someone through the woods, catch up with him, bare your teeth,
and sit down to play a little roshambo. But it works -- and it keeps
everything safe and fun.

Charles couldn't really get the hang of this. I think he was perfectly
capable of understanding that the rule was no contact, but he was
simultaneously capable of deciding that the rule didn't matter, and
that he wasn't going to abide by it. The refs called him on it a bunch
of times over the weekend, and he kept on promising to stick by it,
and kept on going back. He was one of the bigger kids there already,
and he was fond of ``accidentally'' tackling you at the end of a
chase. Not fun when you get tackled into the rocky forest floor.

I had just mightily smote Darryl in a little clearing where he'd been
treasure-hunting, and we were having a little laugh over my extreme
sneakiness. He was going to go monstering -- killed players could
switch to playing monsters, which meant that the longer the game wore
on, the more monsters there were coming after you, meaning that
everyone got to keep on playing and the game's battles just got more
and more epic.

That was when Charles came out of the woods behind me and tackled me,
throwing me to the ground so hard that I couldn't breathe for a
moment. ``Gotcha!'' he yelled. I only knew him slightly before this, and
I'd never thought much of him, but now I was ready for murder. I
climbed slowly to my feet and looked at him, his chest heaving,
grinning. ``You're so dead,'' he said. ``I totally got you.''

I smiled and something felt wrong and sore in my face. I touched my
upper lip. It was bloody. My nose was bleeding and my lip was split,
cut on a root I'd face-planted into when he tackled me.

I wiped the blood on my pants-leg and smiled. I made like I thought
that it was all in fun. I laughed a little. I moved towards him.

Charles wasn't fooled. He was already backing away, trying to fade
into the woods. Darryl moved to flank him. I took the other
flank. Abruptly, he turned and ran. Darryl's foot hooked his ankle and
sent him sprawling. We rushed him, just in time to hear a ref's
whistle.

The ref hadn't seen Charles foul me, but he'd seen Charles's play that
weekend. He sent Charles back to the camp entrance and told him he was
out of the game. Charles complained mightily, but to our satisfaction,
the ref wasn't having any of it. Once Charles had gone, he gave \emph{us}
both a lecture, too, telling us that our retaliation was no more
justified than Charles's attack.

It was OK. That night, once the games had ended, we all got hot
showers in the scout dorms. Darryl and I stole Charles's clothes and
towel. We tied them in knots and dropped them in the urinal. A lot of
the boys were happy to contribute to the effort of soaking
them. Charles had been very enthusiastic about his tackles.

I wish I could have watched him when he got out of his shower and
discovered his clothes. It's a hard decision: do you run naked across
the camp, or pick apart the tight, piss-soaked knots in your clothes
and then put them on?

He chose nudity. I probably would have chosen the same. We lined up
along the route from the showers to the shed where the packs were
stored and applauded him. I was at the front of the line, leading the
applause.

\fancybreak{\#}

The Scout Camp weekends only came three or four times a year, which
left Darryl and me -- and lots of other LARPers -- with a serious LARP
deficiency in our lives.

Luckily, there were the Wretched Daylight games in the city
hotels. Wretched Daylight is another LARP, rival vampire clans and
vampire hunters, and it's got its own quirky rules. Players get cards
to help them resolve combat skirmishes, so each skirmish involves
playing a little hand of a strategic card game. Vampires can become
invisible by cloaking themselves, crossing their arms over their
chests, and all the other players have to pretend they don't see them,
continuing on with their conversations about their plans and so
on. The true test of a good player is whether you're honest enough to
go on spilling your secrets in front of an ``invisible'' rival without
acting as though he was in the room.

There were a couple of big Wretched Daylight games every month. The
organizers of the games had a good relationship with the city's hotels
and they let it be known that they'd take ten unbooked rooms on Friday
night and fill them with players who'd run around the hotel, playing
low-key Wretched Daylight in the corridors, around the pool, and so
on, eating at the hotel restaurant and paying for the hotel
WiFi. They'd close the booking on Friday afternoon, email us, and we'd
go straight from school to whichever hotel it was, bringing our
knapsacks, sleeping six or eight to a room for the weekend, living on
junk-food, playing until three AM. It was good, safe fun that our
parents could get behind.

The organizers were a well-known literacy charity that ran kids'
writing workshops, drama workshops and so on. They had been running
the games for ten years without incident. Everything was strictly
booze- and drug-free, to keep the organizers from getting busted on
some kind of corruption of minors rap. We'd draw between ten and a
hundred players, depending on the weekend, and for the cost of a
couple movies, you could have two and a half days' worth of solid fun.

One day, though, they lucked into a block of rooms at the Monaco, a
hotel in the Tenderloin that catered to arty older tourists, the kind
of place where every room came with a goldfish bowl, where the lobby
was full of beautiful old people in fine clothes, showing off their
plastic surgery results.

Normally, the mundanes -- our word for non-players -- just ignored us,
figuring that we were skylarking kids. But that weekend there happened
to be an editor for an Italian travel magazine staying, and he took an
interest in things. He cornered me as I skulked in the lobby, hoping
to spot the clan-master of my rivals and swoop in on him and draw his
blood. I was standing against the wall with my arms folded over my
chest, being invisible, when he came up to me and asked me, in
accented English, what me and my friends were doing in the hotel that
weekend?

I tried to brush him off, but he wouldn't be put off. So I figured I'd
just make something up and he'd go away.

I didn't imagine that he'd print it. I really didn't imagine that it
would get picked up by the American press.

``We're here because our prince has died, and so we've had to come in
search of a new ruler.''

``A prince?''

``Yes,'' I said, getting into it. ``We're the Old People. We came to
America in the 16th Century and have had our own royal family in the
wilds of Pennsylvania ever since. We live simply in the woods. We
don't use modern technology. But the prince was the last of the line
and he died last week. Some terrible wasting disease took him. The
young men of my clan have left to find the descendants of his
great-uncle, who went away to join the modern people in the time of my
grandfather. He is said to have multiplied, and we will find the last
of his bloodline and bring them back to their rightful home.''

I read a lot of fantasy novels. This kind of thing came easily to me.

``We found a woman who knew of these descendants. She told us one was
staying in this hotel, and we've come to find him. But we've been
tracked here by a rival clan who would keep us from bringing home our
prince, to keep us weak and easy to dominate. Thus it is vital we keep
to ourselves. We do not talk to the New People when we can help
it. Talking to you now causes me great discomfort.''

He was watching me shrewdly. I had uncrossed my arms, which meant that
I was now ``visible'' to rival vampires, one of whom had been slowly
sneaking up on us. At the last moment, I turned and saw her, arms
spread, hissing at us, vamping it up in high style.

I threw my arms wide and hissed back at her, then pelted through the
lobby, hopping over a leather sofa and deking around a potted plant,
making her chase me. I'd scouted an escape route down through the
stairwell to the basement health-club and I took it, shaking her off.

I didn't see him again that weekend, but I \emph{did} relate the story to
some of my fellow LARPers, who embroidered the tale and found lots of
opportunities to tell it over the weekend.

The Italian magazine had a staffer who'd done her master's degree on
Amish anti-technology communities in rural Pennsylvania, and she
thought we sounded awfully interesting. Based on the notes and taped
interviews of her boss from his trip to San Francisco, she wrote a
fascinating, heart-wrenching article about these weird, juvenile
cultists who were crisscrossing America in search of their ``prince.''
Hell, people will print anything these days.

But the thing was, stories like that get picked up and
republished. First it was Italian bloggers, then a few American
bloggers. People across the country reported ``sightings'' of the Old
People, though whether they were making it up, or whether others were
playing the same game, I didn't know.

It worked its way up the media food-chain all the way to the \emph{New York
Times}, who, unfortunately, have an unhealthy appetite for
fact-checking. The reporter they put on the story eventually tracked
it down to the Monaco Hotel, who put them in touch with the LARP
organizers, who laughingly spilled the whole story.

Well, at that point, LARPing got a lot less cool. We became known as
the nation's foremost hoaxers, as weird, pathological liars. The press
who we'd inadvertently tricked into covering the story of the Old
People were now interested in redeeming themselves by reporting on how
unbelievably weird we LARPers were, and that was when Charles let
everyone in school know that Darryl and I were the biggest LARPing
weenies in the city.

That was not a good season. Some of the gang didn't mind, but we
did. The teasing was merciless. Charles led it. I'd find plastic fangs
in my bag, and kids I passed in the hall would go ``bleh, bleh'' like a
cartoon vampire, or they'd talk with fake Transylvanian accents when I
was around.

We switched to ARGing pretty soon afterwards. It was more fun in some
ways, and it was a lot less weird. Every now and again, though, I
missed my cape and those weekends in the hotel.

\fancybreak{\#}

The opposite of esprit d'escalier is the way that life's
embarrassments come back to haunt us even after they're long past. I
could remember every stupid thing I'd ever said or done, recall them
with picture-perfect clarity. Any time I was feeling low, I'd
naturally start to remember other times I felt that way, a hit-parade
of humiliations coming one after another to my mind.

As I tried to concentrate on Masha and my impending doom, the Old
People incident kept coming back to haunt me. There'd been a similar,
sick, sinking doomed feeling then, as more and more press outlets
picked up the story, as the likelihood of someone figuring out that it
had been me who'd sprung the story on the stupid Italian editor in the
designer jeans with crooked seams, the starched collarless shirt, and
the oversized metal-rimmed glasses.

There's an alternative to dwelling on your mistakes. You can learn
from them.

It's a good theory, anyway. Maybe the reason your subconscious dredges
up all these miserable ghosts is that they need to get closure before
they can rest peacefully in humiliation afterlife. My subconscious
kept visiting me with ghosts in the hopes that I would do something to
let them rest in peace.

All the way home, I turned over this memory and the thought of what I
would do about ``Masha,'' in case she was playing me. I needed some
insurance.

And by the time I reached my house -- to be swept up into melancholy
hugs from Mom and Dad -- I had it.

\fancybreak{\#}

The trick was to time this so that it happened fast enough that the
DHS couldn't prepare for it, but with a long enough lead time that the
Xnet would have time to turn out in force.

The trick was to stage this so that there were too many present to
arrest us all, but to put it somewhere that the press could see it and
the grownups, so the DHS wouldn't just gas us again.

The trick was to come up with something with the media friendliness of
the levitation of the Pentagon. The trick was to to stage something
that we could rally around, like 3,000 Berkeley students refusing to
let one of their number be taken away in a police van.

The trick was to put the press there, ready to say what the police
did, the way they had in 1968 in Chicago.

It was going to be some trick.

I cut out of school an hour early the next day, using my customary
techniques for getting out, not caring if it would trigger some kind
of new DHS checker that would result in my parents getting a note.

One way or another, my parents' last problem after tomorrow would be
whether I was in trouble at school.

I met Ange at her place. She'd had to cut out of school even earlier,
but she'd just made a big deal out of her cramps and pretended she was
going to keel over and they sent her home.

We started to spread the word on Xnet. We sent it in email to trusted
friends, and IMmed it to our buddy lists. We roamed the decks and
towns of Clockwork Plunder and told our team-mates. Giving everyone
enough information to get them to show up but not so much as to tip
our hand to the DHS was tricky, but I thought I had just the right
balance:

\edialog{VAMPMOB TOMORROW}

\edialog{If you're a goth, dress to impress. If you're not a goth,
  find a goth and borrow some clothes. Think vampire.}

\edialog{The game starts at 8:00AM sharp. SHARP. Be there and ready
  to be divided into teams. The game lasts 30 minutes, so you'll have
  plenty of time to get to school afterward.}

\edialog{Location will be revealed tomorrow. Email your public key to
  \url{m1k3y@littlebrother.pirateparty.org.se} and check your messages at
  7AM for the update. If that's too early for you, stay up all
  night. That's what we're going to do.}

\edialog{This is the most fun you will have all year, guaranteed.}

\edialog{Believe.}

\edialog{M1k3y}

Then I sent a short message to Masha.

\edialog{Tomorrow}

\edialog{M1k3y}

A minute later, she emailed back:

\edialog{I thought so. VampMob, huh? You work fast. Wear a red
  hat. Travel light.}

\fancybreak{\#}

What do you bring along when you go fugitive? I'd carried enough heavy
packs around enough scout camps to know that every ounce you add cuts
into your shoulders with all the crushing force of gravity with every
step you take -- it's not just one ounce, it's one ounce that you
carry for a million steps. It's a ton.

``Right,'' Ange said. ``Smart. And you never take more than three days'
worth of clothes, either. You can rinse stuff out in the sink. Better
to have a spot on your t-shirt than a suitcase that's too big and
heavy to stash under a plane-seat.''

She'd pulled out a ballistic nylon courier bag that went across her
chest, between her breasts -- something that made me get a little
sweaty -- and slung diagonally across her back. It was roomy inside,
and she'd set it down on the bed. Now she was piling clothes next to
it.

``I figure that three t-shirts, a pair of pants, a pair of shorts,
three changes of underwear, three pairs of socks and a sweater will do
it.''

She dumped out her gym bag and picked out her toiletries. ``I'll have
to remember to stick my toothbrush in tomorrow morning before I head
down to Civic Center.''

Watching her pack was impressive. She was ruthless about it all. It
was also freaky -- it made me realize that the next day, I was going
to go away. Maybe for a long time. Maybe forever.

``Do I bring my Xbox?'' she asked. ``I've got a ton of stuff on the
hard-drive, notes and sketches and email. I wouldn't want it to fall
into the wrong hands.''

``It's all encrypted,'' I said. ``That's standard with ParanoidXbox. But
leave the Xbox behind, there'll be plenty of them in LA. Just create a
Pirate Party account and email an image of your hard-drive to
yourself. I'm going to do the same when I get home.''

She did so, and queued up the email. It was going to take a couple
hours for all the data to squeeze through her neighbor's WiFi network
and wing its way to Sweden.

Then she closed the flap on the bag and tightened the compression
straps. She had something the size of a soccer-ball slung over her
back now, and I stared admiringly at it. She could walk down the
street with that under her shoulder and no one would look twice -- she
looked like she was on her way to school.

``One more thing,'' she said, and went to her bedside table and took out
the condoms. She took the strips of rubbers out of the box and opened
the bag and stuck them inside, then gave me a slap on the ass.

``Now what?'' I said.

``Now we go to your place and do your stuff. It's time I met your
parents, no?''

She left the bag amid the piles of clothes and junk all over the
floor. She was ready to turn her back on all of it, walk away, just to
be with me. Just to support the cause. It made me feel brave, too.

\fancybreak{\#}

Mom was already home when I got there. She had her laptop open on the
kitchen table and was answering email while talking into a headset
connected to it, helping some poor Yorkshireman and his family
acclimate to living in Louisiana.

I came through the door and Ange followed, grinning like mad, but
holding my hand so tight I could feel the bones grinding together. I
didn't know what she was so worried about. It wasn't like she was
going to end up spending a lot of time hanging around with my parents
after this, even if it went badly.

Mom hung up on the Yorkshireman when we got in.

``Hello, Marcus,'' she said, giving me a kiss on the cheek. ``And who is
this?''

``Mom, meet Ange. Ange, this is my Mom, Lillian.'' Mom stood up and gave
Ange a hug.

``It's very good to meet you, darling,'' she said, looking her over from
top to bottom. Ange looked pretty acceptable, I think. She dressed
well, and low-key, and you could tell how smart she was just by
looking at her.

``A pleasure to meet you, Mrs Yallow,'' she said. She sounded very
confident and self-assured. Much better than I had when I'd met her
mom.

``It's Lillian, love,'' she said. She was taking in every detail. ``Are
you staying for dinner?''

``I'd love that,'' she said.

``Do you eat meat?'' Mom's pretty acclimated to living in California.

``I eat anything that doesn't eat me first,'' she said.

``She's a hot-sauce junkie,'' I said. ``You could serve her old tires and
she'd eat 'em if she could smother them in salsa.''

Ange socked me gently in the shoulder.

``I was going to order Thai,'' Mom said. ``I'll add a couple of their
five-chili dishes to the order.''

Ange thanked her politely and Mom bustled around the kitchen, getting
us glasses of juice and a plate of biscuits and asking three times if
we wanted any tea. I squirmed a little.

``Thanks, Mom,'' I said. ``We're going to go upstairs for a while.''

Mom's eyes narrowed for a second, then she smiled again. ``Of course,''
she said. ``Your father will be home in an hour, we'll eat then.''

I had my vampire stuff all stashed in the back of my closet. I let
Ange sort through it while I went through my clothes. I was only going
as far as LA. They had stores there, all the clothing I could need. I
just needed to get together three or four favorite tees and a favorite
pair of jeans, a tube of deodorant, a roll of dental floss.

``Money!'' I said.

``Yeah,'' she said. ``I was going to clean out my bank account on the way
home at an ATM. I've got maybe five hundred saved up.''

``Really?''

``What am I going to spend it on?'' she said. ``Ever since the Xnet, I
haven't had to even pay any service charges.''

``I think I've got three hundred or so.''

``Well, there you go. Grab it on the way to Civic Center in the
morning.''

I had a big book-bag I used when I was hauling lots of gear around
town. It was less conspicuous than my camping pack. Ange went through
my piles mercilessly and culled them down to her favorites.

Once it was packed and under my bed, we both sat down.

``We're going to have to get up really early tomorrow,'' she said.

``Yeah, big day.''

The plan was to get messages out with a bunch of fake VampMob
locations tomorrow, sending people out to secluded spots within a few
minutes' walk of Civic Center. We'd cut out a spray-paint stencil that
just said VAMPMOB CIVIC CENTER -\textgreater -\textgreater{} that
I we would spray-paint at
those spots around 5AM. That would keep the DHS from locking down the
Civic Center before we got there. I had the mailbot ready to send out
the messages at 7AM -- I'd just leave my Xbox running when I went out.

``How long. . .'' She trailed off.

``That's what I've been wondering, too,'' I said. ``It could be a long
time, I suppose. But who knows? With Barbara's article coming out --''
I'd queued an email to her for the next morning, too -- ``and all,
maybe we'll be heroes in two weeks.''

``Maybe,'' she said and sighed.

I put my arm around her. Her shoulders were shaking.

``I'm terrified,'' I said. ``I think that it would be crazy not to be
terrified.''

``Yeah,'' she said. ``Yeah.''

Mom called us to dinner. Dad shook Ange's hand. He looked unshaved and
worried, the way he had since we'd gone to see Barbara, but on meeting
Ange, a little of the old Dad came back. She kissed him on the cheek
and he insisted that she call him Drew.

Dinner was actually really good. The ice broke when Ange took out her
hot-sauce mister and treated her plate, and explained about Scoville
units. Dad tried a forkful of her food and went reeling into the
kitchen to drink a gallon of milk. Believe it or not, Mom still tried
it after that and gave every impression of loving it. Mom, it turned
out, was an undiscovered spicy food prodigy, a natural.

Before she left, Ange pressed the hot-sauce mister on Mom. ``I have a
spare at home,'' she said. I'd watched her pack it in her
backpack. ``You seem like the kind of woman who should have one of
these.''

\chapter{Chapter 19}

\epigraph{This chapter is dedicated to the MIT Press Bookshop, a store
  I've visited on every single trip to Boston over the past ten
  years. MIT, of course, is one of the legendary origin nodes for
  global nerd culture, and the campus bookstore lives up to the
  incredible expectations I had when I first set foot in it. In
  addition to the wonderful titles published by the MIT press, the
  bookshop is a tour through the most exciting high-tech publications
  in the world, from hacker zines like 2600 to fat academic
  anthologies on video-game design. This is one of those stores where
  I have to ask them to ship my purchases home because they don't fit
  in my suitcase.}  {MIT Press Bookstore\footnote{
  \url{http://web.mit.edu/bookstore/www/}} Building E38, 77
  Massachusetts Ave., Cambridge, MA USA 02139-4307 +1 617 253 5249}

Here's the email that went out at 7AM the next day, while Ange and I
were spray-painting VAMP-MOB CIVIC CENTER $\to \to$ at strategic locations
around town.

\edialog{RULES FOR VAMPMOB}

\edialog{You are part of a clan of daylight vampires. You've
  discovered the secret of surviving the terrible light of the
  sun. The secret was cannibalism: the blood of another vampire can
  give you the strength to walk among the living.}

\edialog{You need to bite as many other vampires as you can in order
  to stay in the game. If one minute goes by without a bite, you're
  out. Once you're out, turn your shirt around backwards and go
  referee -- watch two or three vamps to see if they're getting their
  bites in.}

\edialog{To bite another vamp, you have to say ``Bite!'' five times
  before they do. So you run up to a vamp, make eye-contact, and shout
  ``bite bite bite bite bite!'' and if you get it out before she does,
  you live and she crumbles to dust.}

\edialog{You and the other vamps you meet at your rendezvous are a
  team. They are your clan. You derive no nourishment from their
  blood.}

\edialog{You can ``go invisible'' by standing still and folding your
  arms over your chest. You can't bite invisible vamps, and they can't
  bite you.}

\edialog{This game is played on the honor system. The point is to
  have fun and get your vamp on, not to win.}

\edialog{There is an end-game that will be passed by word of mouth as
  winners begin to emerge. The game-masters will start a whisper
  campaign among the players when the time comes. Spread the whisper
  as quickly as you can and watch for the sign.}

\edialog{M1k3y}

\edialog{bite bite bite bite bite!}

We'd hoped that a hundred people would be willing to play
VampMob. We'd sent out about two hundred invites each. But when I sat
bolt upright at 4AM and grabbed my Xbox, there were \emph{400} replies
there. Four \emph{hundred}.

I fed the addresses to the bot and stole out of the house. I descended
the stairs, listening to my father snore and my mom rolling over in
their bed. I locked the door behind me.

At 4:15 AM, Potrero Hill was as quiet as the countryside. There were
some distant traffic rumbles, and once, a car crawled past me. I
stopped at an ATM and drew out \$320 in twenties, rolled them up and
put a rubber-band around them, and stuck the roll in a zip-up pocket
low on the thigh of my vampire pants.

I was wearing my cape again, and a ruffled shirt, and tuxedo pants
that had been modded to have enough pockets to carry all my little
bits and pieces. I had on pointed boots with silver-skull buckles, and
I'd teased my hair into a black dandelion clock around my head. Ange
was bringing the white makeup and had promised to do my eyeliner and
black nail-polish. Why the hell not? When was the next time I was
going to get to play dressup like this?

Ange met me in front of her house. She had her backpack on too, and
fishnet tights, a ruffled gothic lolita maid's dress, white
face-paint, elaborate kabuki eye-makeup, and her fingers and throat
dripped with silver jewelry.

``You look \emph{great}!'' we said to each other in unison, then laughed
quietly and stole off through the streets, spray-paint cans in our
pockets.

\fancybreak{\#}

As I surveyed Civic Center, I thought about what it would look like
once 400 VampMobbers converged on it. I expected them in ten minutes,
out front of City Hall. Already the big plaza teemed with commuters
who neatly sidestepped the homeless people begging there.

I've always hated Civic Center. It's a collection of huge wed\-ding-cake
buildings: court houses, museums, and civic buildings like City
Hall. The sidewalks are wide, the buildings are white. In the tourist
guides to San Francisco, they manage to photograph it so that it looks
like Epcot Center, futuristic and austere.

But on the ground, it's grimy and gross. Homeless people sleep on all
the benches. The district is empty by 6PM except for drunks and
druggies, because with only one kind of building there, there's no
legit reason for people to hang around after the sun goes down. It's
more like a mall than a neighborhood, and the only businesses there
are bail-bonds\-men and liquor stores, places that cater to the families
of crooks on trial and the bums who make it their nighttime home.

I really came to understand all of this when I read an interview with
an amazing old urban planner, a woman called Jane Jacobs who was the
first person to really nail why it was wrong to slice cities up with
freeways, stick all the poor people in housing projects, and use
zoning laws to tightly control who got to do what where.

Jacobs explained that real cities are organic and they have a lot of
variety -- rich and poor, white and brown, Anglo and Mex, retail and
residential and even industrial. A neighborhood like that has all
kinds of people passing through it at all hours of the day or night,
so you get businesses that cater to every need, you get people around
all the time, acting like eyes on the street.

You've encountered this before. You go walking around some older part
of some city and you find that it's full of the coolest looking
stores, guys in suits and people in fashion-rags, upscale restaurants
and funky cafes, a little movie theater maybe, houses with elaborate
paint-jobs. Sure, there might be a Starbucks too, but there's also a
neat-looking fruit market and a florist who appears to be three
hundred years old as she snips carefully at the flowers in her
windows. It's the opposite of a planned space, like a mall. It feels
like a wild garden or even a woods: like it \emph{grew}.

You couldn't get any further from that than Civic Center. I read an
interview with Jacobs where she talked about the great old
neighborhood they knocked down to build it. It had been just that kind
of neighborhood, the kind of place that happened without permission or
rhyme or reason.

Jacobs said that she predicted that within a few years, Civic Center
would be one of the worst neighborhoods in the city, a ghost-town at
night, a place that sustained a thin crop of weedy booze shops and
flea-pit motels. In the interview, she didn't seem very glad to have
been vindicated; she sounded like she was talking about a dead friend
when she described what Civic Center had become.

Now it was rush hour and Civic Center was as busy at it could be. The
Civic Center BART also serves as the major station for Muni trolley
lines, and if you need to switch from one to another, that's where you
do it. At 8AM, there were thousands of people coming up the stairs,
going down the stairs, getting into and out of taxis and on and off
buses. They got squeezed by DHS checkpoints by the different civic
buildings, and routed around aggressive panhandlers. They all smelled
like their shampoos and colognes, fresh out of the shower and armored
in their work suits, swinging laptop bags and briefcases. At 8AM,
Civic Center was business central.

And here came the vamps. A couple dozen coming down Van Ness, a couple
dozen coming up Market. More coming from the other side of
Market. More coming up from Van Ness. They slipped around the side of
the buildings, wearing the white face-paint and the black eyeliner,
black clothes, leather jackets, huge stompy boots. Fishnet fingerless
gloves.

They began to fill up the plaza. A few of the business people gave
them passing glances and then looked away, not wanting to let these
weirdos into their personal realities as they thought about whatever
crap they were about to wade through for another eight hours. The
vamps milled around, not sure when the game was on. They pooled
together in large groups, like an oil spill in reverse, all this black
gathering in one place. A lot of them sported old-timey hats, bowlers
and toppers. Many of the girls were in full-on elegant gothic lolita
maid costumes with huge platforms.

I tried to estimate the numbers. 200. Then, five minutes later, it was
300. 400. They were still streaming in. The vamps had brought friends.

Someone grabbed my ass. I spun around and saw Ange, laughing so hard
she had to hold her thighs, bent double.

``Look at them all, man, look at them all!'' she gasped. The square was
twice as crowded as it had been a few minutes ago. I had no idea how
many Xnetters there were, but easily 1000 of them had just showed up
to my little party. Christ.

The DHS and SFPD cops were starting to mill around, talking into their
radios and clustering together. I heard a far-away siren.

``All right,'' I said, shaking Ange by the arm. ``All right, let's \emph{go}.''

We both slipped off into the crowd and as soon as we encountered our
first vamp, we both said, loudly, ``Bite bite bite bite bite!'' My
victim was a stunned -- but cute -- girl with spider-webs drawn on her
hands and smudged mascara running down her cheeks. She said, ``Crap,''
and moved away, acknowledging that I'd gotten her.

The call of ``bite bite bite bite bite'' had scrambled the other nearby
vamps. Some of them were attacking each other, others were moving for
cover, hiding out. I had my victim for the minute, so I skulked away,
using mundanes for cover. All around me, the cry of ``bite bite bite
bite bite!'' and shouts and laughs and curses.

The sound spread like a virus through the crowd. All the vamps knew
the game was on now, and the ones who were clustered together were
dropping like flies. They laughed and cussed and moved away, clueing
the still-in vamps that the game was on. And more vamps were arriving
by the second.

8:16. It was time to bag another vamp. I crouched low and moved
through the legs of the straights as they headed for the BART
stairs. They jerked back with surprise and swerved to avoid me. I had
my eyes laser-locked on a set of black platform boots with steel
dragons over the toes, and so I wasn't expecting it when I came face
to face with another vamp, a guy of about 15 or 16, hair gelled
straight back and wearing a PVC Marilyn Manson jacket draped with
necklaces of fake tusks carved with intricate symbols.

``Bite bite bite --'' he began, when one of the mundanes tripped over
him and they both went sprawling. I leapt over to him and shouted
``bite bite bite bite bite!'' before he could untangle himself again.

More vamps were arriving. The suits were really freaking out. The game
overflowed the sidewalk and moved into Van Ness, spreading up toward
Market Street. Drivers honked, the trolleys made angry \emph{ding}s. I
heard more sirens, but now traffic was snarled in every direction.

It was freaking \emph{glorious}.

BITE BITE BITE BITE BITE!

The sound came from all around me. There were so many vamps there,
playing so furiously, it was like a roar. I risked standing up and
looking around and found that I was right in the middle of a giant
crowd of vamps that went as far as I could see in every direction.

BITE BITE BITE BITE BITE!

This was even better than the concert in Dolores Park. That had been
angry and rockin', but this was -- well, it was just \emph{fun}. It was
like going back to the playground, to the epic games of tag we'd play
on lunch breaks when the sun was out, hundreds of people chasing each
other around. The adults and the cars just made it more fun, more
funny.

That's what it was: it was \emph{funny}. We were all laughing now.

But the cops were really mobilizing now. I heard helicopters. Any
second now, it would be over. Time for the endgame.

I grabbed a vamp.

``Endgame: when the cops order us to disperse, pretend you've been
gassed. Pass it on. What did I just say?''

The vamp was a girl, tiny, so short I thought she was really young,
but she must have been 17 or 18 from her face and the smile. ``Oh,
that's wicked,'' she said.

``What did I say?''

``Endgame: when the cops order us to disperse, pretend you've been
gassed. Pass it on. What did I just say?''

``Right,'' I said. ``Pass it on.''

She melted into the crowd. I grabbed another vamp. I passed it on. He
went off to pass it on.

Somewhere in the crowd, I knew Ange was doing this too. Somewhere in
the crowd, there might be infiltrators, fake Xnetters, but what could
they do with this knowledge? It's not like the cops had a choice. They
were going to order us to disperse. That was guaranteed.

I had to get to Ange. The plan was to meet at the Founder's Statue in
the Plaza, but reaching it was going to be hard. The crowd wasn't
moving anymore, it was \emph{surging}, like the mob had in the way down to
the BART station on the day the bombs went off. I struggled to make my
way through it just as the PA underneath the helicopter switched on.

``THIS IS THE DEPARTMENT OF HOMELAND SECURITY. YOU ARE ORDERED TO
DISPERSE IMMEDIATELY.''

Around me, hundreds of vamps fell to the ground, clutching their
throats, clawing at their eyes, gasping for breath. It was easy to
fake being gassed, we'd all had plenty of time to study the footage of
the partiers in Mission Dolores Park going down under the pepper-spray
clouds.

``DISPERSE IMMEDIATELY.''

I fell to the ground, protecting my pack, reaching around to the red
baseball hat folded into the waistband of my pants. I jammed it on my
head and then grabbed my throat and made horrendous retching noises.

The only ones still standing were the mundanes, the salarymen who'd
been just trying to get to their jobs. I looked around as best as I
could at them as I choked and gasped.

``THIS IS THE DEPARTMENT OF HOMELAND SECURITY. YOU ARE ORDERED TO
DISPERSE IMMEDIATELY. DISPERSE IMMEDIATELY.'' The voice of god made my
bowels ache. I felt it in my molars and in my femurs and my spine.

The salarymen were scared. They were moving as fast as they could, but
in no particular direction. The helicopters seemed to be directly
overhead no matter where you stood. The cops were wading into the
crowd now, and they'd put on their helmets. Some had shields. Some had
gas masks. I gasped harder.

Then the salarymen were running. I probably would have run too. I
watched a guy whip a \$500 jacket off and wrap it around his face
before heading south toward Mission, only to trip up and go
sprawling. His curses joined the choking sounds.

This wasn't supposed to happen -- the choking was just supposed to
freak people out and get them confused, not panic them into a
stampede.

There were screams now, screams I recognized all too well from the
night in the park. That was the sound of people who were scared
spitless, running into each other as they tried like hell to get away.

And then the air-raid sirens began.

I hadn't heard that sound since the bombs went off, but I would never
forget it. It sliced through me and went straight into my balls,
turning my legs into jelly on the way. It made me want to run away in
a panic. I got to my feet, red cap on my head, thinking of only one
thing: Ange. Ange and the Founders' Statue.

Everyone was on their feet now, running in all directions,
screaming. I pushed people out of my way, holding onto my pack and my
hat, heading for Founders' Statue. Masha was looking for me, I was
looking for Ange. Ange was out there.

I pushed and cursed. Elbowed someone. Someone came down on my foot so
hard I felt something go \emph{crunch} and I shoved him so he went down. He
tried to get up and someone stepped on him. I shoved and pushed.

Then I reached out my arm to shove someone else and strong hands
grabbed my wrist and my elbow in one fluid motion and brought my arm
back around behind my back. It felt like my shoulder was about to
wrench out of its socket, and I instantly doubled over, hollering, a
sound that was barely audible over the din of the crowd, the thrum of
the choppers, the wail of the sirens.

I was brought back upright by the strong hands behind me, which
steered me like a marionette. The hold was so perfect I couldn't even
think of squirming. I couldn't think of the noise or the helicopter or
Ange. All I could think of was moving the way that the person who had
me wanted me to move. I was brought around so that I was face-to-face
with the person.

It was a girl whose face was sharp and rodent-like, half-hidden by a
giant pair of sunglasses. Over the sunglasses, a mop of bright pink
hair, spiked out in all directions.

``You!'' I said. I knew her. She'd taken a picture of me and threatened
to rat me out to truant watch. That had been five minutes before the
alarms started. She'd been the one, ruthless and cunning. We'd both
run from that spot in the Tenderloin as the klaxon sounded behind us,
and we'd both been picked up by the cops. I'd been hostile and they'd
decided that I was an enemy.

She -- Masha -- became their ally.

``Hello, M1k3y,'' she hissed in my ear, close as a lover. A shiver went
up my back. She let go of my arm and I shook it out.

``Christ,'' I said. ``You!''

``Yes, me,'' she said. ``The gas is gonna come down in about two
minutes. Let's haul ass.''

``Ange -- my girlfriend -- is by the Founders' Statue.''

Masha looked over the crowd. ``No chance,'' she said. ``We try to make it
there, we're doomed. The gas is coming down in two minutes, in case
you missed it the first time.''

I stopped moving. ``I don't go without Ange,'' I said.

She shrugged. ``Suit yourself,'' she shouted in my ear. ``Your funeral.''

She began to push through the crowd, moving away, north, toward
downtown. I continued to push for the Founders' Statue. A second
later, my arm was back in the terrible lock and I was being swung
around and propelled forward.

``You know too much, jerk-off,'' she said. ``You've seen my face. You're
coming with me.''

I screamed at her, struggled till it felt like my arm would break, but
she was pushing me forward. My sore foot was agony with every step, my
shoulder felt like it would break.

With her using me as a battering ram, we made good progress through
the crowd. The whine of the helicopters changed and she gave me a
harder push. ``RUN!'' she yelled. ``Here comes the gas!''

The crowd noise changed, too. The choking sounds and scream sounds got
much, much louder. I'd heard that pitch of sound before. We were back
in the park. The gas was raining down. I held my breath and \emph{ran}.

We cleared the crowd and she let go of my arm. I shook it out. I
limped as fast as I could up the sidewalk as the crowd thinned and
thinned. We were heading towards a group of DHS cops with riot shields
and helmets and masks. As we drew near them, they moved to block us,
but Masha held up a badge and they melted away like she was Obi Wan
Kenobi, saying ``These aren't the droids you're looking for.''

``You goddamned \emph{bitch},'' I said as we sped up Market Street. ``We have
to go back for Ange.''

She pursed her lips and shook her head. ``I feel for you, buddy. I
haven't seen my boyfriend in months. He probably thinks I'm
dead. Fortunes of war. We go back for your Ange, we're dead. If we
push on, we have a chance. So long as we have a chance, she has a
chance. Those kids aren't all going to Gitmo. They'll probably take a
few hundred in for questioning and send the rest home.''

We were moving up Market Street now, past the strip joints where the
little encampments of bums and junkies sat, stinking like open
toilets. Masha guided me to a little alcove in the shut door of one of
the strip places. She stripped off her jacket and turned it inside out
-- the lining was a muted stripe pattern, and with the jacket's seams
reversed, it hung differently. She produced a wool hat from her pocket
and pulled it over her hair, letting it form a jaunty, off-center
peak. Then she took out some make-up remover wipes and went to work on
her face and fingernails. In a minute, she was a different woman.

``Wardrobe change,'' she said. ``Now you. Lose the shoes, lose the
jacket, lose the hat.'' I could see her point. The cops would be
looking very carefully at anyone who looked like they'd been a part of
the VampMob. I ditched the hat entirely -- I'd never liked ball
caps. Then I jammed the jacket into my pack and got out a long-sleeved
tee with a picture of Rosa Luxembourg on it and pulled it over my
black tee. I let Masha wipe my makeup off and clean my nails and a
minute later, I was clean.

``Switch off your phone,'' she said. ``You carrying any arphids?''

I had my student card, my ATM card, my Fast Pass. They all went into a
silvered bag she held out, which I recognized as a radio-proof Faraday
pouch. But as she put them in her pocket, I realized I'd just turned
my ID over to her. If she was on the other side\ldots

The magnitude of what had just happened began to sink in. In my mind,
I'd pictured having Ange with me at this point. Ange would make it two
against one. Ange would help me see if there was something amiss. If
Masha wasn't all she said she was.

``Put these pebbles in your shoes before you put them on --''

``It's OK. I sprained my foot. No gait recognition program will spot me
now.''

She nodded once, one pro to another, and slung her pack. I picked up
mine and we moved. The total time for the changeover was less than a
minute. We looked and walked like two different people.

She looked at her watch and shook her head. ``Come on,'' she said. ``We
have to make our rendezvous. Don't think of running, either. You've
got two choices now. Me, or jail. They'll be analyzing the footage
from that mob for days, but once they're done, every face in it will
go in a database. Our departure will be noted. We are both wanted
criminals now.''

\fancybreak{\#}

\textperiodcentered She got us off Market Street on the next block,
swinging back into the Tenderloin. I knew this neighborhood. This was
where we'd gone hunting for an open WiFi access-point back on the day,
playing Harajuku Fun Madness.

``Where are we going?'' I said.

``We're about to catch a ride,'' she said. ``Shut up and let me
concentrate.''

We moved fast, and sweat streamed down my face from under my hair,
coursed down my back and slid down the crack of my ass and my
thighs. My foot was \emph{really} hurting and I was seeing the streets of
San Francisco race by, maybe for the last time, ever.

It didn't help that we were ploughing straight uphill, moving for the
zone where the seedy Tenderloin gives way to the nosebleed real-estate
values of Nob Hill. My breath came in ragged gasps. She moved us
mostly up narrow alleys, using the big streets just to get from one
alley to the next.

We were just stepping into one such alley, Sabin Place, when someone
fell in behind us and said, ``Freeze right there.'' It was full of evil
mirth. We stopped and turned around.

At the mouth of the alley stood Charles, wearing a halfhearted VampMob
outfit of black t-shirt and jeans and white face-paint. ``Hello,
Marcus,'' he said. ``You going somewhere?'' He smiled a huge, wet
grin. ``Who's your girlfriend?''

``What do you want, Charles?''

``Well, I've been hanging out on that traitorous Xnet ever since I
spotted you giving out DVDs at school. When I heard about your
VampMob, I thought I'd go along and hang around the edges, just to see
if you showed up and what you did. You know what I saw?''

I said nothing. He had his phone in his hand, pointed at
us. Recording. Maybe ready to dial 911. Beside me, Masha had gone
still as a board.

``I saw you \emph{leading} the damned thing. And I \emph{recorded} it, Marcus. So
now I'm going to call the cops and we're going to wait right here for
them. And then you're going to go to pound-you-in-the-ass prison, for
a long, long time.''

Masha stepped forward.

``Stop right there, chickie,'' he said. ``I saw you get him away. I saw
it all --''

She took another step forward and snatched the phone out of his hand,
reaching behind her with her other hand and bringing it out holding a
wallet open.

``DHS, dick-head,'' she said. ``I'm DHS. I've been running this twerp
back to his masters to see where he went. I \emph{was} doing that. Now
you've blown it. We have a name for that. We call it 'Obstruction of
National Security.' You're about to hear that phrase a lot more
often.''

Charles took a step backward, his hands held up in front of him. He'd
gone even paler under his makeup. ``What? No! I mean -- I didn't know!
I was trying to \emph{help}!''

``The last thing we need is a bunch of high school Junior G-men
'helping' buddy. You can tell it to the judge.''

He moved back again, but Masha was fast. She grabbed his wrist and
twisted him into the same judo hold she'd had me in back at Civic
Center. Her hand dipped back to her pockets and came out holding a
strip of plastic, a handcuff strip, which she quickly wound around his
wrists.

That was the last thing I saw as I took off running.

\fancybreak{\#}

I made it as far as the other end of the alley before she caught up
with me, tackling me from behind and sending me sprawling. I couldn't
move very fast, not with my hurt foot and the weight of my pack. I
went down in a hard face-plant and skidded, grinding my cheek into the
grimy asphalt.

``Jesus,'' she said. ``You're a goddamned idiot. You didn't \emph{believe}
that, did you?''

My heart thudded in my chest. She was on top of me and slowly she let
me up.

``Do I need to cuff you, Marcus?''

I got to my feet. I hurt all over. I wanted to die.

``Come on,'' she said. ``It's not far now.''

\fancybreak{\#}

'It' turned out to be a moving van on a Nob Hill side-street, a
sixteen-wheeler the size of one of the ubiquitous DHS trucks that
still turned up on San Francisco's street corners, bristling with
antennas.

This one, though, said ``Three Guys and a Truck Moving'' on the side,
and the three guys were very much in evidence, trekking in and out of
a tall apartment building with a green awning. They were carrying
crated furniture, neatly labeled boxes, loading them one at a time
onto the truck and carefully packing them there.

She walked us around the block once, apparently unsatisfied with
something, then, on the next pass, she made eye-contact with the man
who was watching the van, an older black guy with a kidney-belt and
heavy gloves. He had a kind face and he smiled at us as she led us
quickly, casually up the truck's three stairs and into its
depth. ``Under the big table,'' he said. ``We left you some space there.''

The truck was more than half full, but there was a narrow corridor
around a huge table with a quilted blanket thrown over it and
bubble-wrap wound around its legs.

Masha pulled me under the table. It was stuffy and still and dusty
under there, and I suppressed a sneeze as we scrunched in among the
boxes. The space was so tight that we were on top of each other. I
didn't think that Ange would have fit in there.

``Bitch,'' I said, looking at Masha.

``Shut up. You should be licking my boots thanking me. You would have
ended up in jail in a week, two tops. Not Gitmo-by-the-Bay. Syria,
maybe. I think that's where they sent the ones they really wanted to
disappear.''

I put my head on my knees and tried to breathe deeply.

``Why would you do something so stupid as declaring war on the DHS
anyway?''

I told her. I told her about being busted and I told her about Darryl.

She patted her pockets and came up with a phone. It was
Charles's. ``Wrong phone.'' She came up with another phone. She turned
it on and the glow from its screen filled our little fort. After
fiddling for a second, she showed it to me.

It was the picture she'd snapped of us, just before the bombs blew. It
was the picture of Jolu and Van and me and --

Darryl.

I was holding in my hand proof that Darryl had been with us minutes
before we'd all gone into DHS custody. Proof that he'd been alive and
well and in our company.

``You need to give me a copy of this,'' I said. ``I need it.''

``When we get to LA,'' she said, snatching the phone back. ``Once you've
been briefed on how to be a fugitive without getting both our asses
caught and shipped to Syria. I don't want you getting rescue ideas
about this guy. He's safe enough where he is -- for now.''

I thought about trying to take it from her by force, but she'd already
demonstrated her physical skill. She must have been a black-belt or
something.

We sat there in the dark, listening to the three guys load the truck
with box after box, tying things down, grunting with the effort of
it. I tried to sleep, but couldn't. Masha had no such problem. She
snored.

There was still light shining through the narrow, obstructed corridor
that led to the fresh air outside. I stared at it, through the gloom,
and thought of Ange.

My Ange. Her hair brushing her shoulders as she turned her head from
side to side, laughing at something I'd done. Her face when I'd seen
her last, falling down in the crowd at VampMob. All those people at
VampMob, like the people in the park, down and writhing, the DHS
moving in with truncheons. The ones who disappeared.

Darryl. Stuck on Treasure Island, his side stitched up, taken out of
his cell for endless rounds of questioning about the terrorists.

Darry's father, ruined and boozy, unshaven. Washed up and in his
uniform, ``for the photos.'' Weeping like a little boy.

My own father, and the way that he had been changed by my
disappearance to Treasure Island. He'd been just as broken as Darryl's
father, but in his own way. And his face, when I told him where I'd
been.

That was when I knew that I couldn't run.

That was when I knew that I had to stay and fight.

\fancybreak{\#}

Masha's breathing was deep and regular, but when I reached with
glacial slowness into her pocket for her phone, she snuffled a little
and shifted. I froze and didn't even breathe for a full two minutes,
counting one hippopotami, two hippopotami.

Slowly, her breath deepened again. I tugged the phone free of her
jacket-pocket one millimeter at a time, my fingers and arm trembling
with the effort of moving so slowly.

Then I had it, a little candy-bar shaped thing.

I turned to head for the light, when I had a flash of memory: Charles,
holding out his phone, waggling it at us, taunting us. It had been a
candy-bar-shaped phone, silver, plastered in the logos of a dozen
companies that had subsidized the cost of the handset through the
phone company. It was the kind of phone where you had to listen to a
commercial every time you made a call.

It was too dim to see the phone clearly in the truck, but I could feel
it. Were those company decals on its sides? Yes? Yes. I had just
stolen \emph{Charles's} phone from Masha.

I turned back around slowly, slowly, and slowly, slowly, \emph{slowly}, I
reached back into her pocket. \emph{Her} phone was bigger and bulkier, with
a better camera and who knew what else?

I'd been through this once before -- that made it a little
easier. Millimeter by millimeter again, I teased it free of her
pocket, stopping twice when she snuffled and twitched.

I had the phone free of her pocket and I was beginning to back away
when her hand shot out, fast as a snake, and grabbed my wrist, hard,
fingertips grinding away at the small, tender bones below my hand.

I gasped and stared into Masha's wide-open, staring eyes.

""You are such an idiot,'' she said, conversationally, taking the phone
from me, punching at its keypad with her other hand. ``How did you plan
on unlocking this again?''

I swallowed. I felt bones grind against each other in my wrist. I bit
my lip to keep from crying out.

She continued to punch away with her other hand. ``Is this what you
thought you'd get away with?'' She showed me the picture of all of us,
Darryl and Jolu, Van and me. ``This picture?''

I didn't say anything. My wrist felt like it would shatter.

``Maybe I should just delete it, take temptation out of your way.'' Her
free hand moved some more. Her phone asked her if she was sure and she
had to look at it to find the right button.

That's when I moved. I had Charles's phone in my other hand still, and
I brought it down on her crushing hand as hard as I could, banging my
knuckles on the table overhead. I hit her hand so hard the phone
shattered and she yelped and her hand went slack. I was still moving,
reaching for her other hand, for her now-unlocked phone with her thumb
still poised over the OK key. Her fingers spasmed on the empty air as
I snatched the phone out of her hand.

I moved down the narrow corridor on hands and knees, heading for the
light. I felt her hands slap at my feet and ankles twice, and I had to
shove aside some of the boxes that had walled us in like a Pharaoh in
a tomb. A few of them fell down behind me, and I heard Masha grunt
again.

The rolling truck door was open a crack and I dove for it, slithering
out under it. The steps had been removed and I found myself hanging
over the road, sliding headfirst into it, clanging my head off the
blacktop with a thump that rang my ears like a gong. I scrambled to my
feet, holding the bumper, and desperately dragged down on the
door-handle, slamming it shut. Masha screamed inside -- I must have
caught her fingertips. I felt like throwing up, but I didn't.

I padlocked the truck instead.

\chapter{Chapter 20}

\epigraph{This chapter is dedicated to The Tattered Cover, Denver's
  legendary independent bookstore. I happened upon The Tattered Cover
  quite by accident: Alice and I had just landed in Denver, coming in
  from London, and it was early and cold and we needed coffee. We
  drove in aimless rental-car circles, and that's when I spotted it,
  the Tattered Cover's sign. Something about it tingled in my
  hindbrain -- I knew I'd heard of this place. We pulled in (got a
  coffee) and stepped into the store -- a wonderland of dark wood,
  homey reading nooks, and miles and miles of bookshelves.}
{The Tattered Cover\footnote{
\url{http://www.tatteredcover.com/NASApp/store/Product?s=showproduct&isbn=9780765319852}}
1628 16th St., Denver, CO USA 80202 +1 303 436 1070}

None of the three guys were around at the moment, so I took off. My
head hurt so much I thought I must be bleeding, but my hands came away
dry. My twisted ankle had frozen up in the truck so that I ran like a
broken marionette, and I stopped only once, to cancel the
photo-deletion on Masha's phone. I turned off its radio -- both to
save battery and to keep it from being used to track me -- and set the
sleep timer to two hours, the longest setting available. I tried to
set it to not require a password to wake from sleep, but that required
a password itself. I was just going to have to tap the keypad at least
once every two hours until I could figure out how to get the photo off
of the phone. I would need a charger, then.

I didn't have a plan. I needed one. I needed to sit down, to get
online -- to figure out what I was going to do next. I was sick of
letting other people do my planning for me. I didn't want to be acting
because of what Masha did, or because of the DHS, or because of my
dad. Or because of Ange? Well, maybe I'd act because of Ange. That
would be just fine, in fact.

I'd just been slipping downhill, taking alleys when I could, merging
with the Tenderloin crowds. I didn't have any destination in
mind. Every few minutes, I put my hand in my pocket and nudged one of
the keys on Masha's phone to keep it from going asleep. It made an
awkward bulge, unfolded there in my jacket.

I stopped and leaned against a building. My ankle was killing
me. Where was I, anyway?

O'Farrell, at Hyde Street. In front of a dodgy ``Asian Massage Parlor.''
My traitorous feet had taken me right back to the beginning -- taken
me back to where the photo on Masha's phone had been taken, seconds
before the Bay Bridge blew, before my life changed forever.

I wanted to sit down on the sidewalk and bawl, but that 
\discretionary{would}{not}{wouldn't} solve
my problems. I had to call Barbara Stratford, tell her what had
happened. Show her the photo of Darryl.

What was I thinking? I had to show her the video, the one that Masha
had sent me -- the one where the President's Chief of Staff gloated at
the attacks on San Francisco and admitted that he knew when and where
the next attacks would happen and that he wouldn't stop them because
they'd help his man get re-elected.

That was a plan, then: get in touch with Barbara, give her the
documents, and get them into print. The VampMob had to have really
freaked people out, made them think that we really were a bunch of
terrorists. Of course, when I'd been planning it, I had been thinking
of how good a distraction it would be, not how it would look to some
NASCAR Dad in Nebraska.

I'd call Barbara, and I'd do it smart, from a payphone, putting my
hood up so that the inevitable CCTV wouldn't get a photo of me. I dug
a quarter out of my pocket and polished it on my shirt-tail, getting
the fingerprints off it.

I headed downhill, down and down to the BART station and the payphones
there. I made it to the trolley-car stop when I spotted the cover of
the week's \emph{Bay Guardian}, stacked in a high pile next to a homeless
black guy who smiled at me. ``Go ahead and read the cover, it's free --
it'll cost you fifty cents to look inside, though.''

The headline was set in the biggest type I'd seen since 9/11:

INSIDE GITMO-BY-THE-BAY

Beneath it, in slightly smaller type:

``How the DHS has kept our children and friends in secret prisons on
our doorstep.

``By Barbara Stratford, Special to the Bay Guardian''

The newspaper seller shook his head. ``Can you believe that?'' he
said. ``Right here in San Francisco. Man, the government \emph{sucks}.''

Theoretically, the \emph{Guardian} was free, but this guy appeared to have
cornered the local market for copies of it. I had a quarter in my
hand. I dropped it into his cup and fished for another one. I didn't
bother polishing the fingerprints off of it this time.

``We're told that the world changed forever when the Bay Bridge was
blown up by parties unknown. Thousands of our friends and neighbors
died on that day. Almost none of them have been recovered; their
remains are presumed to be resting in the city's harbor.

``But an extraordinary story told to this reporter by a young man who
was arrested by the DHS minutes after the explosion suggests that our
own government has illegally held many of those thought dead on
Treasure Island, which had been evacuated and declared off-limits to
civilians shortly after the bombing\ldots''

I sat down on a bench -- the same bench, I noted with a prickly
hair-up-the-neck feeling, where we'd rested Darryl after escaping from
the BART station -- and read the article all the way through. It took
a huge effort not to burst into tears right there. Barbara had found
some photos of me and Darryl goofing around together and they ran
alongside the text. The photos were maybe a year old, but I looked so
much \emph{younger} in them, like I was 10 or 11. I'd done a lot of growing
up in the past couple months.

The piece was beautifully written. I kept feeling outraged on behalf
of the poor kids she was writing about, then remembering that she was
writing about \emph{me}. Zeb's note was there, his crabbed handwriting
reproduced in large, a half-sheet of the newspaper. Barbara had dug up
more info on other kids who were missing and presumed dead, a long
list, and asked how many had been stuck there on the island, just a
few miles from their parents' doorsteps.

I dug another quarter out of my pocket, then changed my mind. What was
the chance that Barbara's phone wasn't tapped? There was no way I was
going to be able to call her now, not directly. I needed some
intermediary to get in touch with her and get her to meet me somewhere
south. So much for plans.

What I really, really needed was the Xnet.

How the hell was I going to get online? My phone's wifinder was
blinking like crazy -- there was wireless all around me, but I didn't
have an Xbox and a TV and a ParanoidXbox DVD to boot from. WiFi, WiFi
everywhere\ldots

That's when I spotted them. Two kids, about my age, moving among the
crowd at the top of the stairs down into the BART.

What caught my eye was the way they were moving, kind of clumsy,
nudging up against the commuters and the tourists. Each had a hand in
his pocket, and whenever they met one another's eye, they
snickered. They couldn't have been more obvious jammers, but the crowd
was oblivious to them. Being down in that neighborhood, you expect to
be dodging homeless people and crazies, so you don't make eye contact,
don't look around at all if you can help it.

I sidled up to one. He seemed really young, but he couldn't have been
any younger than me.

``Hey,'' I said. ``Hey, can you guys come over here for a second?''

He pretended not to hear me. He looked right through me, the way you
would a homeless person.

``Come on,'' I said. ``I don't have a lot of time.'' I grabbed his
shoulder and hissed in his ear. ``The cops are after me. I'm from
Xnet.''

He looked scared now, like he wanted to run away, and his friend was
moving toward us. ``I'm serious,'' I said. ``Just hear me out.''

His friend came over. He was taller, and beefy -- like Darryl. ``Hey,''
he said. ``Something wrong?''

His friend whispered in his ear. The two of them looked like they were
going to bolt.

I grabbed my copy of the \emph{Bay Guardian} from under my arm and rattled
it in front of them. ``Just turn to page 5, OK?''

They did. They looked at the headline. The photo. Me.

``Oh, dude,'' the first one said. ``We are \emph{so} not worthy.'' He grinned
at me like crazy, and the beefier one slapped me on the back.

``No \emph{way} --'' he said. ``You're M --''

I put a hand over his mouth. ``Come over here, OK?''

I brought them back to my bench. I noticed that there was something
old and brown staining the sidewalk underneath it. Darryl's blood? It
made my skin pucker up. We sat down.

``I'm Marcus,'' I said, swallowing hard as I gave my real name to these
two who already knew me as M1k3y. I was blowing my cover, but the \emph{Bay
Guardian} had already made the connection for me.

``Nate,'' the small one said. ``Liam,'' the bigger one said. ``Dude, it is
\emph{such} an honor to meet you. You're like our all-time hero --''

``Don't say that,'' I said. ``Don't say that. You two are like a flashing
advertisement that says, 'I am jamming, please put my ass in
Gitmo-by-the-Bay. You couldn't be more obvious.''

Liam looked like he might cry.

``Don't worry, you didn't get busted. I'll give you some tips, later.''
He brightened up again. What was becoming weirdly clear was that these
two really \emph{did} idolize M1k3y, and that they'd do anything I
said. They were grinning like idiots. It made me uncomfortable, sick
to my stomach.

``Listen, I need to get on Xnet, now, without going home or anywhere
near home. Do you two live near here?''

``I do,'' Nate said. ``Up at the top of California Street. It's a bit of
a walk -- steep hills.'' I'd just walked all the way down them. Masha
was somewhere up there. But still, it was better than I had any right
to expect.

``Let's go,'' I said.

\fancybreak{\#}

Nate loaned me his baseball hat and traded jackets with me. I didn't
have to worry about gait-recognition, not with my ankle throbbing the
way it was -- I limped like an extra in a cowboy movie.
	
Nate lived in a huge four-bedroom apartment at the top of Nob
Hill. The building had a doorman, in a red overcoat with gold brocade,
and he touched his cap and called Nate, ``Mr Nate'' and welcomed us all
there. The place was spotless and smelled of furniture polish. I tried
not to gawp at what must have been a couple million bucks' worth of
condo.

``My dad,'' he explained. ``He was an investment banker. Lots of life
insurance. He died when I was 14 and we got it all. They'd been
divorced for years, but he left my mom as beneficiary.''

From the floor-to-ceiling window, you could see a stunning view of the
other side of Nob Hill, all the way down to Fisherman's Wharf, to the
ugly stub of the Bay Bridge, the crowd of cranes and trucks. Through
the mist, I could just make out Treasure Island. Looking down all that
way, it gave me a crazy urge to jump.

I got online with his Xbox and a huge plasma screen in the living
room. He showed me how many open WiFi networks were visible from his
high vantage point -- twenty, thirty of them. This was a good spot to
be an Xnetter.

There was a \emph{lot} of email in my M1k3y account. 20,000 new messages
since Ange and I had left her place that morning. Lots of it was from
the press, asking for followup interviews, but most of it was from the
Xnetters, people who'd seen the \emph{Guardian} story and wanted to tell me
that they'd do anything to help me, anything I needed.

That did it. Tears started to roll down my cheeks.

Nate and Liam exchanged glances. I tried to stop, but it was no
good. I was sobbing now. Nate went to an oak book-case on one wall and
swung a bar out of one of its shelves, revealing gleaming rows of
bottles. He poured me a shot of something golden brown and brought it
to me.

``Rare Irish whiskey,'' he said. ``Mom's favorite.''

It tasted like fire, like gold. I sipped at it, trying not to choke. I
didn't really like hard liquor, but this was different. I took several
deep breaths.

``Thanks, Nate,'' I said. He looked like I'd just pinned a medal on
him. He was a good kid.

``All right,'' I said, and picked up the keyboard. The two boys watched
in fascination as I paged through my mail on the gigantic screen.

What I was looking for, first and foremost, was email from Ange. There
was a chance that she'd just gotten away. There was always that
chance.

I was an idiot to even hope. There was nothing from her. I started
going through the mail as fast as I could, picking apart the press
requests, the fan mail, the hate mail, the spam\ldots

And that's when I found it: a letter from Zeb.

``It wasn't nice to wake up this morning and find the letter that I
thought you would destroy in the pages of the newspaper. Not nice at
all. Made me feel -- hunted.

``But I've come to understand why you did it. I don't know if I can
approve of your tactics, but it's easy to see that your motives were
sound.

``If you're reading this, that means that there's a good chance you've
gone underground. It's not easy. I've been learning that. I've been
learning a lot more.

``I can help you. I should do that for you. You're doing what you can
for me. (Even if you're not doing it with my permission.)

``Reply if you get this, if you're on the run and alone. Or reply if
you're in custody, being run by our friends on Gitmo, looking for a
way to make the pain stop. If they've got you, you'll do what they
tell you. I know that. I'll take that risk.

``For you, M1k3y.''

``Wooooah,'' Liam breathed. ``Duuuuude.'' I wanted to smack him. I turned
to say something awful and cutting to him, but he was staring at me
with eyes as big as saucers, looking like he wanted to drop to his
knees and worship me.

``Can I just say,'' Nate said, ``can I just say that it is the biggest
honor of my entire life to help you? Can I just say that?''

I was blushing now. There was nothing for it. These two were totally
star-struck, even though I wasn't any kind of star, not in my own mind
at least.

``Can you guys --'' I swallowed. ``Can I have some privacy here?''

They slunk out of the room like bad puppies and I felt like a tool. I
typed fast.

``I got away, Zeb. And I'm on the run. I need all the help I can get. I
want to end this now.'' I remembered to take Masha's phone out of my
pocket and tickle it to keep it from going to sleep.

They let me use the shower, gave me a change of clothes, a new
backpack with half their earthquake kit in it -- energy bars,
medicine, hot and cold packs, and an old sleeping-bag. They even
slipped a spare Xbox Universal already loaded with ParanoidXbox on it
into there. That was a nice touch. I had to draw the line at a
flaregun.

I kept on checking my email to see if Zeb had replied. I answered the
fan mail. I answered the mail from the press. I deleted the hate
mail. I was half-expecting to see something from Masha, but chances
were she was halfway to LA by now, her fingers hurt, and in no
position to type. I tickled her phone again.

They encouraged me to take a nap and for a brief, shameful moment, I
got all paranoid like maybe these guys were thinking of turning me in
once I was asleep. Which was idiotic -- they could have turned me in
just as easily when I was awake. I just couldn't compute the fact that
they thought \emph{so much} of me. I had known, intellectually, that there
were people who would follow M1k3y. I'd met some of those people that
morning, shouting BITE BITE BITE and vamping it up at Civic
Center. But these two were more personal. They were just nice, goofy
guys, they coulda been any of my friends back in the days before the
Xnet, just two pals who palled around having teenage
adventures. They'd volunteered to join an army, my army. I had a
responsibility to them. Left to themselves, they'd get caught, it was
only a matter of time. They were too trusting.

``Guys, listen to me for a second. I have something serious I need to
talk to you about.''

They almost stood at attention. It would have been funny if it wasn't
so scary.

``Here's the thing. Now that you've helped me, it's really
dangerous. If you get caught, I'll get caught. They'll get anything
you know out of you --'' I held up my hand to forestall their
protests. ``No, stop. You haven't been through it. Everyone
talks. Everyone breaks. If you're ever caught, you tell them
everything, right away, as fast as you can, as much as you
can. They'll get it all eventually anyway. That's how they work.

``But you won't get caught, and here's why: you're not jammers
anymore. You are retired from active duty. You're a --'' I fished in my
memory for vocabulary words culled from spy thrillers -- ``you're a
sleeper cell. Stand down. Go back to being normal kids. One way or
another, I'm going to break this thing, break it wide open, end it. Or
it will get me, finally, do me in. If you don't hear from me within 72
hours, assume that they got me. Do whatever you want then. But for the
next three days -- and forever, if I do what I'm trying to do -- stand
down. Will you promise me that?''

They promised with all solemnity. I let them talk me into napping, but
made them swear to rouse me once an hour. I'd have to tickle Masha's
phone and I wanted to know as soon as Zeb got back in touch with me.

\fancybreak{\#}

The rendezvous was on a BART car, which made me nervous. They're full
of cameras. But Zeb knew what he was doing. He had me meet him in the
last car of a certain train departing from Powell Street Station, at a
time when that car was filled with the press of bodies. He sidled up
to me in the crowd, and the good commuters of San Francisco cleared a
space for him, the hollow that always surrounds homeless people.

``Nice to see you again,'' he muttered, facing into the doorway. Looking
into the dark glass, I could see that there was no one close enough to
eavesdrop -- not without some kind of high-efficiency mic rig, and if
they knew enough to show up here with one of those, we were dead
anyway.

``You too, brother,'' I said. ``I'm -- I'm sorry, you know?''

``Shut up. Don't be sorry. You were braver than I am. Are you ready to
go underground now? Ready to disappear?''

``About that.''

``Yes?''

``That's not the plan.''

``Oh,'' he said.

``Listen, OK? I have -- I have pictures, video. Stuff that really
\emph{proves} something.'' I reached into my pocket and tickled Masha's
phone. I'd bought a charger for it in Union Square on the way down,
and had stopped and plugged it in at a cafe for long enough to get the
battery up to four out of five bars. ``I need to get it to Barbara
Stratford, the woman from the \emph{Guardian}. But they're going to be
watching her -- watching to see if I show up.''

``You don't think that they'll be watching for me, too? If your plan
involves me going within a mile of that woman's home or office --''

``I want you to get Van to come and meet me. Did Darryl ever tell you
about Van? The girl --''

``He told me. Yes, he told me. You don't think they'll be watching her?
All of you who were arrested?''

``I think they will. I don't think they'll be watching her as hard. And
Van has totally clean hands. She never cooperated with any of my --'' I
swallowed. ``With my projects. So they might be a little more relaxed
about her. If she calls the Bay Guardian to make an appointment to
explain why I'm just full of crap, maybe they'll let her keep it.''

He stared at the door for a long time.

``You know what happens when they catch us again.'' It wasn't a
question.

I nodded.

``Are you sure? Some of the people that were on Treasure Island with us
got taken away in helicopters. They got taken \emph{offshore}. There are
countries where America can outsource its torture. Countries where you
will rot forever. Countries where you wish they would just get it over
with, have you dig a trench and then shoot you in the back of the head
as you stand over it.''

I swallowed and nodded.

``Is it worth the risk? We can go underground for a long, long time
here. Someday we might get our country back. We can wait it out.''

I shook my head. ``You can't get anything done by doing nothing. It's
our \emph{country}. They've taken it from us. The terrorists who attack us
are still free -- but \emph{we're not}. I can't go underground for a year,
ten years, my whole life, waiting for freedom to be handed to
me. Freedom is something you have to take for yourself.''

\fancybreak{\#}

That afternoon, Van left school as usual, sitting in the back of the
bus with a tight knot of her friends, laughing and joking the way she
always did. The other riders on the bus took special note of her, she
was so loud, and besides, she was wearing that stupid, giant floppy
hat, something that looked like a piece out of a school play about
Renaissance sword fighters. At one point they all huddled together,
then turned away to look out the back of the bus, pointing and
giggling. The girl who wore the hat now was the same height as Van,
and from behind, it could be her.

No one paid any attention to the mousy little Asian girl who got off a
few stops before the BART. She was dressed in a plain old school
uniform, and looking down shyly as she stepped off. Besides, at that
moment, the loud Korean girl let out a whoop and her friends followed
along, laughing so loudly that even the bus driver slowed down,
twisted in his seat and gave them a dirty look.

Van hurried away down the street with her head down, her hair tied
back and dropped down the collar of her out-of-style bubble
jacket. She had slipped lifts into her shoes that made her two wobbly,
awkward inches taller, and had taken her contacts out and put on her
least-favored glasses, with huge lenses that took up half her
face. Although I'd been waiting in the bus-shelter for her and knew
when to expect her, I hardly recognized her. I got up and walked along
behind her, across the street, trailing by half a block.

The people who passed me looked away as quickly as possible. I looked
like a homeless kid, with a grubby cardboard sign, street-grimy
overcoat, huge, overstuffed knapsack with duct-tape over its rips. No
one wants to look at a street-kid, because if you meet his eye, he
might ask you for some spare change. I'd walked around Oakland all
afternoon and the only person who'd spoken to me was a Jehovah's
Witness and a Scientologist, both trying to convert me. It felt gross,
like being hit on by a pervert.

Van followed the directions I'd written down carefully. Zeb had passed
them to her the same way he'd given me the note outside school --
bumping into her as she waited for the bus, apologizing profusely. I'd
written the note plainly and simply, just laying it out for her: I
know you don't approve. I understand. But this is it, this is the most
important favor I've ever asked of you. Please. Please.

She'd come. I knew she would. We had a lot of history, Van and I. She
didn't like what had happened to the world, either. Besides, an evil,
chuckling voice in my head had pointed out, she was under suspicion
now that Barbara's article was out.

We walked like that for six or seven blocks, looking at who was near
us, what cars went past. Zeb told me about five-person trails, where
five different undercovers traded off duties following you, making it
nearly impossible to spot them. You had to go somewhere totally
desolate, where anyone at all would stand out like a sore thumb.

The overpass for the 880 was just a few blocks from the Coliseum BART
station, and even with all the circling Van did, it didn't take long
to reach it. The noise from overhead was nearly deafening. No one else
was around, not that I could tell. I'd visited the site before I
suggested it to Van in the note, taking care to check for places where
someone could hide. There weren't any.

Once she stopped at the appointed place, I moved quickly to catch up
to her. She blinked owlishly at me from behind her glasses.

``Marcus,'' she breathed, and tears swam in her eyes. I found that I was
crying too. I'd make a really rotten fugitive. Too sentimental.

She hugged me so hard I couldn't breathe. I hugged her back even
harder.

Then she kissed me.

Not on the cheek, not like a sister. Full on the lips, a hot, wet,
steamy kiss that seemed to go on forever. I was so overcome with
emotion --

No, that's bull. I knew exactly what I was doing. I kissed her back.

Then I stopped and pulled away, nearly shoved her away. ``Van,'' I
gasped.

``Oops,'' she said.

``Van,'' I said again.

``Sorry,'' she said. ``I --''

Something occurred to me just then, something I guess I should have
seen a long, long time before.

``You \emph{like} me, don't you?''

She nodded miserably. ``For years,'' she said.

Oh, God. Darryl, all these years, so in love with her, and the whole
time she was looking at me, secretly wanting me. And then I ended up
with Ange. Ange said that she'd always fought with Van. And I was
running around, getting into so much trouble.

``Van,'' I said. ``Van, I'm so sorry.''

``Forget it,'' she said, looking away. ``I know it can't be. I just
wanted to do that once, just in case I never --'' She bit down on the
words.

``Van, I need you to do something for me. Something important. I need
you to meet with the journalist from the Bay Guardian, Barbara
Stratford, the one who wrote the article. I need you to give her
something.'' I explained about Masha's phone, told her about the video
that Masha had sent me.

``What good will this do, Marcus? What's the point?''

``Van, you were right, at least partly. We can't fix the world by
putting other people at risk. I need to solve the problem by telling
what I know. I should have done that from the start. Should have
walked straight out of their custody and to Darryl's father's house
and told him what I knew. Now, though, I have evidence. This stuff --
it could change the world. This is my last hope. The only hope for
getting Darryl out, for getting a life that I don't spend underground,
hiding from the cops. And you're the only person I can trust to do
this.''

``Why me?''

``You're kidding, right? Look at how well you handled getting
here. You're a pro. You're the best at this of any of us. You're the
only one I can trust. That's why you.''

``Why not your friend Angie?'' She said the name without any inflection
at all, like it was a block of cement.

I looked down. ``I thought you knew. They arrested her. She's in Gitmo
-- on Treasure Island. She's been there for days now.'' I had been
trying not to think about this, not to think about what might be
happening to her. Now I couldn't stop myself and I started to sob. I
felt a pain in my stomach, like I'd been kicked, and I pushed my hands
into my middle to hold myself in. I folded there, and the next thing I
knew, I was on my side in the rubble under the freeway, holding myself
and crying.

Van knelt down by my side. ``Give me the phone,'' she said, her voice an
angry hiss. I fished it out of my pocket and passed it to her.

Embarrassed, I stopped crying and sat up. I knew that snot was running
down my face. Van was giving me a look of pure revulsion. ``You need to
keep it from going to sleep,'' I said. ``I have a charger here.'' I
rummaged in my pack. I hadn't slept all the way through the night
since I acquired it. I set the phone's alarm to go off every 90
minutes and wake me up so that I could keep it from going to
sleep. ``Don't fold it shut, either.''

``And the video?''

``That's harder,'' I said. ``I emailed a copy to myself, but I can't get
onto the Xnet anymore.'' In a pinch, I could have gone back to Nate and
Liam and used their Xbox again, but I didn't want to risk it. ``Look,
I'm going to give you my login and password for the Pirate Party's
mail-server. You'll have to use Tor to access it -- Homeland Security
is bound to be scanning for people logging into p-party mail.''

``Your login and password,'' she said, looking a little surprised.

``I trust you, Van. I know I can trust you.''

She shook her head. ``You \emph{never} give out your passwords, Marcus.''

``I don't think it matters anymore. Either you succeed or I -- or it's
the end of Marcus Yallow. Maybe I'll get a new identity, but I don't
think so. I think they'll catch me. I guess I've known all along that
they'd catch me, some day.''

She looked at me, furious now. ``What a waste. What was it all for,
anyway?''

Of all the things she could have said, nothing could have hurt me
more. It was like another kick in the stomach. What a waste, all of
it, futile. Darryl and Ange, gone. I might never see my family
again. And still, Homeland Security had my city and my country caught
in a massive, irrational shrieking freak-out where anything could be
done in the name of stopping terrorism.

Van looked like she was waiting for me to say something, but I had
nothing to say to that. She left me there.

\fancybreak{\#}

Zeb had a pizza for me when I got back ``home'' -- to the tent under a
freeway overpass in the Mission that he'd staked out for the night. He
had a pup tent, military surplus, stenciled with SAN FRANCISCO LOCAL
HOMELESS COORDINATING BOARD.

The pizza was a Dominos, cold and clabbered, but delicious for all
that. ``You like pineapple on your pizza?''

Zeb smiled condescendingly at me. ``Freegans can't be choosy,'' he said.

``Freegans?''

``Like vegans, but we only eat free food.''

``Free food?''

He grinned again. ``You know -- \emph{free} food. From the free food store?''

``You stole this?''

``No, dummy. It's from the other store. The little one out behind the
store? Made of blue steel? Kind of funky smelling?''

``You got this out of the garbage?''

He flung his head back and cackled. ``Yes indeedy. You should \emph{see}
your face. Dude, it's OK. It's not like it was rotten. It was fresh --
just a screwed up order. They threw it out in the box. They sprinkle
rat poison over everything at closing-time, but if you get there
quick, you're OK. You should see what grocery stores throw out! Wait
until breakfast. I'm going to make you a fruit salad you won't
believe. As soon as one strawberry in the box goes a little green and
fuzzy, the whole thing is out --''

I tuned him out. The pizza was fine. It wasn't as if sitting in the
dumpster would infect it or something.  If it was gross, that was only
because it came from Domino's -- the worst pizza in town. I'd never
liked their food, and I'd given it up altogether when I found out that
they bankrolled a bunch of ultra-crazy politicians who thought that
global warming and evolution were satanic plots.

It was hard to shake the feeling of grossness, though.

But there \emph{was} another way to look at it. Zeb had showed me a secret,
something I hadn't anticipated: there was a whole hidden world out
there, a way of getting by without participating in the system.

``Freegans, huh?''

``Yogurt, too,'' he said, nodding vigorously. ``For the fruit salad. They
throw it out the day after the best-before date, but it's not as if it
goes green at midnight. It's yogurt, I mean, it's basically just
rotten milk to begin with.''

I swallowed. The pizza tasted funny. Rat poison. Spoiled yogurt. Furry
strawberries. This would take some getting used to.

I ate another bite. Actually, Domino's pizza sucked a little less when
you got it for free.

Liam's sleeping bag was warm and welcoming after a long, emotionally
exhausting day. Van would have made contact with Barbara by now. She'd
have the video and the picture. I'd call her in the morning and find
out what she thought I should do next. I'd have to come in once she
published, to back it all up.

I thought about that as I closed my eyes, thought about what it would
be like to turn myself in, the cameras all rolling, following the
infamous M1k3y into one of those big, columnated buildings in Civic
Center.

The sound of the cars screaming by overhead turned into a kind of
ocean sound as I drifted away. There were other tents nearby, homeless
people. I'd met a few of them that afternoon, before it got dark and
we all retreated to huddle near our own tents. They were all all older
than me, rough looking and gruff. None of them looked crazy or
violent, though. Just like people who'd had bad luck, or made bad
decisions, or both.

I must have fallen asleep, because I don't remember anything else
until a bright light was shined into my face, so bright it was
blinding.

``That's him,'' said a voice behind the light.

``Bag him,'' said another voice, one I'd heard before, one I'd heard
over and over again in my dreams, lecturing to me, demanding my
passwords. Severe-haircut-woman.

The bag went over my head quickly and was cinched so tight at the
throat that I choked and threw up my freegan pizza. As I spasmed and
choked, hard hands bound my wrists, then my ankles. I was rolled onto
a stretcher and hoisted, then carried into a vehicle, up a couple of
clanging metal steps. They dropped me into a padded floor. There was
no sound at all in the back of the vehicle once they closed the
doors. The padding deadened everything except my own choking.

``Well, hello again,'' she said. I felt the van rock as she crawled in
with me. I was still choking, trying to gasp in a breath. Vomit filled
my mouth and trickled down my windpipe.

``We won't let you die,'' she said. ``If you stop breathing, we'll make
sure you start again. So don't worry about it.''

I choked harder. I sipped at air. Some was getting through. Deep,
wracking coughs shook my chest and back, dislodging some more of the
puke. More breath.

``See?'' she said. ``Not so bad. Welcome home, M1k3y. We've got somewhere
very special to take you.''

I relaxed onto my back, feeling the van rock. The smell of used pizza
was overwhelming at first, but as with all strong stimuli, my brain
gradually grew accustomed to it, filtered it out until it was just a
faint aroma. The rocking of the van was almost comforting.

That's when it happened. An incredible, deep calm that swept over me
like I was lying on the beach and the ocean had swept in and lifted me
as gently as a parent, held me aloft and swept me out onto a warm sea
under a warm sun. After everything that had happened, I was caught,
but it didn't matter. I had gotten the information to Barbara. I had
organized the Xnet. I had won. And if I hadn't won, I had done
everything I could have done. More than I ever thought I could do. I
took a mental inventory as I rode, thinking of everything that I had
accomplished, that \emph{we} had accomplished. The city, the country, the
world was full of people who wouldn't live the way DHS wanted us to
live. We'd fight forever. They couldn't jail us all.

I sighed and smiled.

She'd been talking all along, I realized. I'd been so far into my
happy place that she'd just gone away.

``-- smart kid like you. You'd think that you'd know better than to
mess with us. We've had an eye on you since the day you walked out. We
would have caught you even if you hadn't gone crying to your lesbo
journalist traitor. I just don't get it -- we had an understanding,
you and me\ldots''

We rumbled over a metal plate, the van's shocks rocking, and then the
rocking changed. We were on water. Heading to Treasure Island. Hey,
Ange was there. Darryl, too. Maybe.

\fancybreak{\#}

The hood didn't come off until I was in my cell. They didn't bother
with the cuffs at my wrists and ankles, just rolled me off the
stretcher and onto the floor. It was dark, but by the moonlight from
the single, tiny, high window, I could see that the mattress had been
taken off the cot. The room contained me, a toilet, a bed-frame, and a
sink, and nothing else.

I closed my eyes and let the ocean lift me. I floated away. Somewhere,
far below me, was my body. I could tell what would happen next. I was
being left to piss myself. Again. I knew what that was like. I'd
pissed myself before. It smelled bad. It itched. It was humiliating,
like being a baby.

But I'd survived it.

I laughed. The sound was weird, and it drew me back into my body, back
to the present. I laughed and laughed. I'd had the worst that they
could throw at me, and I'd survived it, and I'd \emph{beaten them}, beaten
them for months, showed them up as chumps and despots. I'd \emph{won}.

I let my bladder cut loose. It was sore and full anyway, and no time
like the present.

The ocean swept me away.

\fancybreak{\#}

When morning came, two efficient, impersonal guards cut the bindings
off of my wrists and ankles. I still couldn't walk -- when I stood, my
legs gave way like a stringless marionette's. Too much time in one
position. The guards pulled my arms over their shoulders and
half-dragged/half-carried me down the familiar corridor. The bar codes
on the doors were curling up and dangling now, attacked by the salt
air.

I got an idea. ``Ange!'' I yelled. ``Darryl!'' I yelled. My guards yanked
me along faster, clearly disturbed but not sure what to do about
it. ``Guys, it's me, Marcus! Stay free!''

Behind one of the doors, someone sobbed. Someone else cried out in
what sounded like Arabic. Then it was cacophony, a thousand different
shouting voices.

They brought me to a new room. It was an old shower-room, with the
shower-heads still present in the mould tiles.

``Hello, M1k3y,'' Severe Haircut said. ``You seem to have had an eventful
morning.'' She wrinkled her nose pointedly.

``I pissed myself,'' I said, cheerfully. ``You should try it.''

``Maybe we should give you a bath, then,'' she said. She nodded, and my
guards carried me to another stretcher. This one had restraining
straps running its length. They dropped me onto it and it was ice-cold
and soaked through. Before I knew it, they had the straps across my
shoulders, hips and ankles. A minute later, three more straps were
tied down. A man's hands grabbed the railings by my head and released
some catches, and a moment later I was tilted down, my head below my
feet.

``Let's start with something simple,'' she said. I craned my head to see
her. She had turned to a desk with an Xbox on it, connected to an
expensive-looking flat-panel TV. ``I'd like you to tell me your login
and password for your Pirate Party email, please?''

I closed my eyes and let the ocean carry me off the beach.

``Do you know what waterboarding is, M1k3y?'' Her voice reeled me
in. ``You get strapped down like this, and we pour water over your
head, up your nose and down your mouth. You can't suppress the gag
reflex. They call it a simulated execution, and from what I can tell
from this side of the room, that's a fair assessment. You won't be
able to fight the feeling that you're dying.''

I tried to go away. I'd heard of waterboarding. This was it, real
torture. And this was just the beginning.

I couldn't go away. The ocean didn't sweep in and lift me. There was a
tightness in my chest, my eyelids fluttered. I could feel clammy piss
on my legs and clammy sweat in my hair. My skin itched from the dried
puke.

She swam into view above me. ``Let's start with the login,'' she said.

I closed my eyes, squeezed them shut.

``Give him a drink,'' she said.

I heard people moving. I took a deep breath and held it.

The water started as a trickle, a ladleful of water gently poured over
my chin, my lips. Up my upturned nostrils. It went back into my
throat, starting to choke me, but I wouldn't cough, wouldn't gasp and
suck it into my lungs. I held onto my breath and squeezed my eyes
harder.

There was a commotion from outside the room, a sound of chaotic boots
stamping, angry, outraged shouts. The dipper was emptied into my face.

I heard her mutter something to someone in the room, then to me she
said, ``Just the login, Marcus. It's a simple request. What could I do
with your login, anyway?''

This time, it was a bucket of water, all at once, a flood that didn't
stop, it must have been gigantic. I couldn't help it. I gasped and
aspirated the water into my lungs, coughed and took more water in. I
knew they wouldn't kill me, but I couldn't convince my body of
that. In every fiber of my being, I knew I was going to die. I
couldn't even cry -- the water was still pouring over me.

Then it stopped. I coughed and coughed and coughed, but at the angle I
was at, the water I coughed up dribbled back into my nose and burned
down my sinuses.

The coughs were so deep they hurt, hurt my ribs and my hips as I
twisted against them. I hated how my body was betraying me, how my
mind couldn't control my body, but there was nothing for it.

Finally, the coughing subsided enough for me to take in what was going
on around me. People were shouting and it sounded like someone was
scuffling, wrestling. I opened my eyes and blinked into the bright
light, then craned my neck, still coughing a little.

The room had a lot more people in it than it had had when we
started. Most of them seemed to be wearing body armor, helmets, and
smoked-plastic visors. They were shouting at the Treasure Island
guards, who were shouting back, necks corded with veins.

``Stand down!'' one of the body-armors said. ``Stand down and put your
hands in the air. You are under arrest!''

Severe haircut woman was talking on her phone. One of the body armors
noticed her and he moved swiftly to her and batted her phone away with
a gloved hand. Everyone fell silent as it sailed through the air in an
arc that spanned the small room, clattering to the ground in a shower
of parts.

The silence broke and the body-armors moved into the room. Two grabbed
each of my torturers. I almost managed a smile at the look on Severe
Haircut's face when two men grabbed her by the shoulders, turned her
around, and yanked a set of plastic handcuffs around her wrists.

One of the body-armors moved forward from the doorway. He had a video
camera on his shoulder, a serious rig with blinding white light. He
got the whole room, circling me twice while he got me. I found myself
staying perfectly still, as though I was sitting for a portrait.

It was ridiculous.

``Do you think you could get me off of this thing?'' I managed to get it
all out with only a little choking.

Two more body armors moved up to me, one a woman, and began to unstrap
me. They flipped their visors up and smiled at me. They had red
crosses on their shoulders and helmets.

Beneath the red crosses was another insignia: CHP. California Highway
Patrol. They were State Troopers.

I started to ask what they were doing there, and that's when I saw
Barbara Stratford. She'd evidently been held back in the corridor, but
now she came in pushing and shoving. ``There you are,'' she said,
kneeling beside me and grabbing me in the longest, hardest hug of my
life.

That's when I knew it -- Guantanamo by the Bay was in the hands of its
enemies. I was saved.

\chapter{Chapter 21}

\epigraph{This chapter is dedicated to Pages Books in Toronto,
  Canada. Long a fixture on the bleedingly trendy Queen Street West
  strip, Pages is located over the road from CityTV and just a few
  doors down from the old Bakka store where I worked. We at Bakka
  loved having Pages down the street from us: what we were to science
  fiction, they were to everything else: hand-picked material
  representing the stuff you'd never find elsewhere, the stuff you
  didn't know you were looking for until you saw it there. Pages also
  has one of the best news-stands I've ever seen, row on row of
  incredible magazines and zines from all over the world.}
{Pages Books \url{http://pagesbooks.ca/} 256 Queen St W, Toronto, ON M5V
1Z8 Canada +1 416 598 1447}

They left me and Barbara alone in the room then, and I used the
working shower head to rinse off -- I was suddenly embarrassed to be
covered in piss and barf. When I finished, Barbara was in tears.

``Your parents --'' she began.

I felt like I might throw up again. God, my poor folks. What they must
have gone through.

``Are they here?''

``No,'' she said. ``It's complicated,'' she said.

``What?''

``You're still under arrest, Marcus. Everyone here is. They can't just
sweep in and throw open the doors. Everyone here is going to have to
be processed through the criminal justice system. It could take, well,
it could take months.''

``I'm going to have to stay here for \emph{months}?''

She grabbed my hands. ``No, I think we're going to be able to get you
arraigned and released on bail pretty fast. But pretty fast is a
relative term. I wouldn't expect anything to happen today. And it's
not going to be like those people had it. It will be humane. There
will be real food. No interrogations. Visits from your family.

``Just because the DHS is out, it doesn't mean that you get to just
walk out of here. What's happened here is that we're getting rid of
the bizarro-world version of the justice system they'd instituted and
replacing it with the old system. The system with judges, open trials
and lawyers.

``So we can try to get you transferred to a juvie facility on the
mainland, but Marcus, those places can be really rough. Really, really
rough. This might be the best place for you until we get you bailed
out.''

Bailed out. Of course. I was a criminal -- I hadn't been charged yet,
but there were bound to be plenty of charges they could think of. It
was practically illegal just to think impure thoughts about the
government.

She gave my hands another squeeze. ``It sucks, but this is how it has
to be. The point is, it's \emph{over}. The Governor has thrown the DHS out
of the State, dismantled every checkpoint. The Attorney General has
issued warrants for any law-enforcement officers involved in 'stress
interrogations' and secret imprisonments. They'll go to jail, Marcus,
and it's because of what you did.''

I was numb. I heard the words, but they hardly made sense. Somehow, it
was over, but it wasn't over.

``Look,'' she said. ``We probably have an hour or two before this all
settles down, before they come back and put you away again. What do
you want to do? Walk on the beach? Get a meal? These people had an
incredible staff room -- we raided it on the way in. Gourmet all the
way.''

At last a question I could answer. ``I want to find Ange. I want to
find Darryl.''

\fancybreak{\#}

I tried to use a computer I found to look up their cell-numbers, but
it wanted a password, so we were reduced to walking the corridors,
calling out their names. Behind the cell-doors, prisoners screamed
back at us, or cried, or begged us to let them go. They didn't
understand what had just happened, couldn't see their former guards
being herded onto the docks in plastic handcuffs, taken away by
California state SWAT teams.

``Ange!'' I called over the din, ``Ange Carvelli! Darryl Glover! It's
Marcus!''

We'd walked the whole length of the cell-block and they hadn't
answered. I felt like crying. They'd been shipped overseas -- they
were in Syria or worse. I'd never see them again.

I sat down and leaned against the corridor wall and put my face in my
hands. I saw Severe Haircut Woman's face, saw her smirk as she asked
me for my login. She had done this. She would go to jail for it, but
that wasn't enough. I thought that when I saw her again, I might kill
her. She deserved it.

``Come on,'' Barbara said, ``Come on, Marcus. Don't give up. There's more
around here, come on.''

She was right. All the doors we'd passed in the cellblock were old,
rusting things that dated back to when the base was first built. But
at the very end of the corridor, sagging open, was a new high-security
door as thick as a dictionary. We pulled it open and ventured into the
dark corridor within.

There were four more cell-doors here, doors without bar codes. Each
had a small electronic keypad mounted on it.

``Darryl?'' I said. ``Ange?''

``Marcus?''

It was Ange, calling out from behind the furthest door. Ange, my Ange,
my angel.

``Ange!'' I cried. ``It's me, it's me!''

``Oh God, Marcus,'' she choked out, and then it was all sobs.

I pounded on the other doors. ``Darryl! Darryl, are you here?''

``I'm here.'' The voice was very small, and very hoarse. ``I'm here. I'm
very, very sorry. Please. I'm very sorry.''

He sounded\ldots broken. Shattered.

``It's me, D,'' I said, leaning on his door. ``It's Marcus. It's over --
they arrested the guards. They kicked the Department of Homeland
Security out. We're getting trials, open trials. And we get to testify
against \emph{them}.''

``I'm sorry,'' he said. ``Please, I'm so sorry.''

The California patrolmen came to the door then. They still had their
camera rolling. ``Ms Stratford?'' one said. He had his faceplate up and
he looked like any other cop, not like my savior. Like someone come to
lock me up.

``Captain Sanchez,'' she said. ``We've located two of the prisoners of
interest here. I'd like to see them released and inspect them for
myself.''

``Ma'am, we don't have access codes for those doors yet,'' he said.

She held up her hand. ``That wasn't the arrangement. I was to have
complete access to this facility. That came direct from the Governor,
sir. We aren't budging until you open these cells.'' Her face was
perfectly smooth, without a single hint of give or flex. She meant it.

The Captain looked like he needed sleep. He grimaced. ``I'll see what I
can do,'' he said.

\fancybreak{\#}

They did manage to open the cells, finally, about half an hour
later. It took three tries, but they eventually got the right codes
entered, matching them to the arphids on the ID badges they'd taken
off the guards they'd arrested.

They got into Ange's cell first. She was dressed in a hospital gown,
open at the back, and her cell was even more bare than mine had been
-- just padding all over, no sink or bed, no light. She emerged
blinking into the corridor and the police camera was on her, its
bright lights in her face. Barbara stepped protectively between us and
it. Ange stepped tentatively out of her cell, shuffling a
little. There was something wrong with her eyes, with her face. She
was crying, but that wasn't it.

``They drugged me,'' she said. ``When I wouldn't stop screaming for a
lawyer.''

That's when I hugged her. She sagged against me, but she squeezed
back, too. She smelled stale and sweaty, and I smelled no better. I
never wanted to let go.

That's when they opened Darryl's cell.

He had shredded his paper hospital gown. He was curled up, naked, in
the back of the cell, shielding himself from the camera and our
stares. I ran to him.

``D,'' I whispered in his ear. ``D, it's me. It's Marcus. It's over. The
guards have been arrested. We're going to get bail, we're going home.''

He trembled and squeezed his eyes shut. ``I'm sorry,'' he whispered, and
turned his face away.

They took me away then, a cop in body-armor and Barbara, took me back
to my cell and locked the door, and that's where I spent the night.

\fancybreak{\#}

I don't remember much about the trip to the courthouse. They had me
chained to five other prisoners, all of whom had been in for a lot
longer than me. One only spoke Arabic -- he was an old man, and he
trembled. The others were all young. I was the only white one. Once we
had been gathered on the deck of the ferry, I saw that nearly everyone
on Treasure Island had been one shade of brown or another.

I had only been inside for one night, but it was too long. There was a
light drizzle coming down, normally the sort of thing that would make
me hunch my shoulders and look down, but today I joined everyone else
in craning my head back at the infinite gray sky, reveling in the
stinging wet as we raced across the bay to the ferry-docks.

They took us away in buses. The shackles made climbing into the buses
awkward, and it took a long time for everyone to load. No one
cared. When we weren't struggling to solve the geometry problem of six
people, one chain, narrow bus-aisle, we were just looking around at
the city around us, up the hill at the buildings.

All I could think of was finding Darryl and Ange, but neither were in
evidence. It was a big crowd and we weren't allowed to move freely
through it. The state troopers who handled us were gentle enough, but
they were still big, armored and armed. I kept thinking I saw Darryl
in the crowd, but it was always someone else with that same beaten,
hunched look that he'd had in his cell. He wasn't the only broken one.

At the courthouse, they marched us into interview rooms in our shackle
group. An ACLU lawyer took our information and asked us a few
questions -- when she got to me, she smiled and greeted me by name --
and then led us into the courtroom before the judge. He wore an actual
robe, and seemed to be a in a good mood.

The deal seemed to be that anyone who had a family member to post bail
could go free, and everyone else got sent to prison. The ACLU lawyer
did a lot of talking to the judge, asking for a few more hours while
the prisoners' families were rounded up and brought to the
court-house. The judge was pretty good about it, but when I realized
that some of these people had been locked up since the bridge blew,
taken for dead by their families, without trial, subjected to
interrogation, isolation, torture -- I wanted to just break the chains
myself and set everyone free.

When I was brought before the judge, he looked down at me and took off
his glasses. He looked tired. The ACLU lawyer looked tired. The
bailiffs looked tired. Behind me, I could hear a sudden buzz of
conversation as my name was called by the bailiff. The judge rapped
his gavel once, without looking away from me. He scrubbed at his eyes.

``Mr Yallow,'' he said, ``the prosecution has identified you as a flight
risk. I think they have a point. You certainly have more, shall we
say, \emph{history}, than the other people here. I am tempted to hold you
over for trial, no matter how much bail your parents are prepared to
post.''

My lawyer started to say something, but the judge silenced her with a
look. He scrubbed at his eyes.

``Do you have anything to say?''

``I had the chance to run,'' I said. ``Last week. Someone offered to take
me away, get me out of town, help me build a new identity. Instead I
stole her phone, escaped from our truck, and ran away. I turned over
her phone -- which had evidence about my friend, Darryl Glover, on it
-- to a journalist and hid out here, in town.''

``You stole a phone?''

``I decided that I couldn't run. That I had to face justice -- that my
freedom wasn't worth anything if I was a wanted man, or if the city
was still under the DHS. If my friends were still locked up. That
freedom for me wasn't as important as a free country.''

``But you did steal a phone.''

I nodded. ``I did. I plan on giving it back, if I ever find the young
woman in question.''

``Well, thank you for that speech, Mr Yallow. You are a very well
spoken young man.'' He glared at the prosecutor. ``Some would say a very
brave man, too. There was a certain video on the news this morning. It
suggested that you had some legitimate reason to evade the
authorities. In light of that, and of your little speech here, I will
grant bail, but I will also ask the prosecutor to add a charge of
Misdemeanor Petty Theft to the count, as regards the matter of the
phone. For this, I expect another \$50,000 in bail.''

He banged his gavel again, and my lawyer gave my hand a squeeze.

He looked down at me again and re-seated his glasses. He had dandruff,
there on the shoulders of his robe. A little more rained down as his
glasses touched his wiry, curly hair.

``You can go now, young man. Stay out of trouble.''

\fancybreak{\#}

I turned to go and someone tackled me. It was Dad. He literally lifted
me off my feet, hugging me so hard my ribs creaked. He hugged me the
way I remembered him hugging me when I was a little boy, when he'd
spin me around and around in hilarious, vomitous games of airplane
that ended with him tossing me in the air and catching me and
squeezing me like that, so hard it almost hurt.

A set of softer hands pried me gently out of his arms. Mom. She held
me at arm's length for a moment, searching my face for something, not
saying anything, tears streaming down her face. She smiled and it
turned into a sob and then she was holding me too, and Dad's arm
encircled us both.

When they let go, I managed to finally say something. ``Darryl?''

``His father met me somewhere else. He's in the hospital.''

``When can I see him?''

``It's our next stop,'' Dad said. He was grim. ``He doesn't --'' He
stopped. ``They say he'll be OK,'' he said. His voice was choked.

``How about Ange?''

``Her mother took her home. She wanted to wait here for you, but\ldots''

I understood. I felt full of understanding now, for how all the
families of all the people who'd been locked away must feel. The
courtroom was full of tears and hugs, and even the bailiffs couldn't
stop it.

``Let's go see Darryl,'' I said. ``And let me borrow your phone?''

I called Ange on the way to the hospital where they were keeping
Darryl -- San Francisco General, just down the street from us -- and
arranged to see her after dinner. She talked in a hurried whisper. Her
mom wasn't sure whether to punish her or not, but Ange didn't want to
tempt fate.

There were two state troopers in the corridor where Darryl was being
held. They were holding off a legion of reporters who stood on tiptoe
to see around them and get pictures. The flashes popped in our eyes
like strobes, and I shook my head to clear it. My parents had brought
me clean clothes and I'd changed in the back seat, but I still felt
gross, even after scrubbing myself in the court-house bathrooms.

Some of the reporters called my name. Oh yeah, that's right, I was
famous now. The state troopers gave me a look, too -- either they'd
recognized my face or my name when the reporters called it out.

Darryl's father met us at the door of his hospital room, speaking in a
whisper too low for the reporters to hear. He was in civvies, the
jeans and sweater I normally thought of him wearing, but he had his
service ribbons pinned to his breast.

``He's sleeping,'' he said. ``He woke up a little while ago and he
started crying. He couldn't stop. They gave him something to help him
sleep.''

He led us in, and there was Darryl, his hair clean and combed,
sleeping with his mouth open. There was white stuff at the corners of
his mouth. He had a semi-private room, and in the other bed there was
an older Arab-looking guy, in his 40s. I realized it was the guy I'd
been chained to on the way off of Treasure Island. We exchanged
embarrassed waves.

Then I turned back to Darryl. I took his hand. His nails had been
chewed to the quick. He'd been a nail-biter when he was a kid, but
he'd kicked the habit when we got to high school. I think Van talked
him out of it, telling him how gross it was for him to have his
fingers in his mouth all the time.

I heard my parents and Darryl's dad take a step away, drawing the
curtains around us. I put my face down next to his on the pillow. He
had a straggly, patchy beard that reminded me of Zeb.

``Hey, D,'' I said. ``You made it. You're going to be OK.''

He snored a little. I almost said, ``I love you,'' a phrase I'd only
said to one non-family-member ever, a phrase that was weird to say to
another guy. In the end, I just gave his hand another squeeze. Poor
Darryl.

\chapter{Epilogue}

\epigraph{This chapter is dedicated to Hudson Booksellers, the
  booksellers that are in practically every airport in the USA. Most
  of the Hudson stands have just a few titles (though those are often
  surprisingly diverse), but the big ones, like the one in the AA
  terminal at Chicago's O'Hare, are as good as any neighborhood
  store. It takes something special to bring a personal touch to an
  airport, and Hudson's has saved my mind on more than one long
  Chicago layover.}
{Hudson Booksellers\\
\url{http://www.hudsongroup.com/HudsonBooksellers_s.html}}

Barbara called me at the office on July 4th weekend. I wasn't the only
one who'd come into work on the holiday weekend, but I was the only
one whose excuse was that my day-release program wouldn't let me leave
town.

In the end, they convicted me of stealing Masha's phone. Can you
believe that? The prosecution had done a deal with my lawyer to drop
all charges related to ``Electronic terrorism'' and ``inciting riots'' in
exchange for my pleading guilty to the misdemeanor petty theft
charge. I got three months in a day-release program with a half-way
house for juvenile defenders in the Mission. I slept at the halfway
house, sharing a dorm with a bunch of actual criminals, gang kids and
druggie kids, a couple of real nuts. During the day, I was ``free'' to
go out and work at my ``job.''

``Marcus, they're letting her go,'' she said.

``Who?''

``Johnstone, Carrie Johnstone,'' she said. ``The closed military tribunal
cleared her of any wrongdoing. The file is sealed. She's being
returned to active duty. They're sending her to Iraq.''

Carrie Johnstone was Severe Haircut Woman's name. It came out in the
preliminary hearings at the California Superior Court, but that was
just about all that came out. She wouldn't say a word about who she
took orders from, what she'd done, who had been imprisoned and
why. She just sat, perfectly silent, day after day, in the courthouse.

The Feds, meanwhile, had blustered and shouted about the Governor's
``unilateral, illegal'' shut-down of the Treasure Island facility, and
the Mayor's eviction of fed cops from San Francisco. A lot of those
cops had ended up in state prisons, along with the guards from
Gitmo-by-the-Bay.

Then, one day, there was no statement from the White House, nothing
from the state capitol. And the next day, there was a dry, tense
\erratum{press-conference}{press conference} held jointly on the steps of the Governor's mansion,
where the head of the DHS and the governor announced their
``understanding.''

The DHS would hold a closed, military tribunal to investigate
``possible errors in judgment'' committed after the attack on the Bay
Bridge. The tribunal would use every tool at its disposal to ensure
that criminal acts were properly punished. In return, control over DHS
operations in California would go through the State Senate, which
would have the power to shut down, inspect, or re-prioritize all
homeland security in the state.

The roar of the reporters had been deafening and Barbara had gotten
the first question in. ``Mr Governor, with all due respect: we have
incontrovertible video evidence that Marcus Yallow, a citizen of this
state, native born, was subjected to a simulated execution by DHS
officers, apparently acting on orders from the White House. Is the
State really willing to abandon any pretense of justice for its
citizens in the face of illegal, barbaric \emph{torture}?'' Her voice
trembled, but didn't crack.

The Governor spread his hands. ``The military tribunals will accomplish
justice. If Mr Yallow -- or any other person who has cause to fault
the Department of Homeland Security -- wants further justice, he is,
of course, entitled to sue for such damages as may be owing to him
from the federal government.''

That's what I was doing. Over twenty thousand civil lawsuits were
filed against the DHS in the week after the Governor's
announcement. Mine was being handled by the ACLU, and they'd filed
motions to get at the results of the closed military tribunals. So
far, the courts were pretty sympathetic to this.

But I hadn't expected this.

``She got off totally Scot-free?''

``The press release doesn't say much. 'After a thorough examination of
the events in San Francisco and in the special anti-terror detention
center on Treasure Island, it is the finding of this tribunal that Ms
Johnstone's actions do not warrant further discipline.' There's that
word, 'further' -- like they've already punished her.''

I snorted. I'd dreamed of Carrie Johnstone nearly every night since I
was released from Gitmo-by-the-Bay. I'd seen her face looming over
mine, that little snarly smile as she told the man to give me a
``drink.''

``Marcus --'' Barbara said, but I cut her off.

``It's fine. It's fine. I'm going to do a video about this. Get it out
over the weekend. Mondays are big days for viral video. Everyone'll be
coming back from the holiday weekend, looking for something funny to
forward around school or the office.''

I saw a shrink twice a week as part of my deal at the halfway
house. Once I'd gotten over seeing that as some kind of punishment, it
had been good. He'd helped me focus on doing constructive things when
I was upset, instead of letting it eat me up. The videos helped.

``I have to go,'' I said, swallowing hard to keep the emotion out of my
voice.

``Take care of yourself, Marcus,'' Barbara said.

Ange hugged me from behind as I hung up the phone. ``I just read about
it online,'' she said. She read a million newsfeeds, pulling them with
a headline reader that sucked up stories as fast as they ended up on
the wire. She was our official blogger, and she was good at it,
snipping out the interesting stories and throwing them online like a
short order cook turning around breakfast orders.

I turned around in her arms so that I was hugging her from in
front. Truth be told, we hadn't gotten a lot of work done that day. I
wasn't allowed to be out of the halfway house after dinner time, and
she couldn't visit me there. We saw each other around the office, but
there were usually a lot of other people around, which kind of put a
crimp in our cuddling. Being alone in the office for a day was too
much temptation. It was hot and sultry, too, which meant we were both
in tank-tops and shorts, a lot of skin-to-skin contact as we worked
next to each other.

``I'm going to make a video,'' I said. ``I want to release it today.''

``Good,'' she said. ``Let's do it.''

Ange read the press-release. I did a little monologue, synched over
that famous footage of me on the water-board, eyes wild in the harsh
light of the camera, tears streaming down my face, hair matted and
flecked with barf.

``This is me. I am on a waterboard. I am being tortured in a simulated
execution. The torture is supervised by a woman called Carrie
Johnstone. She works for the government. You might remember her from
this video.''

I cut in the video of Johnstone and Kurt Rooney. ``That's Johnstone and
Secretary of State Kurt Rooney, the president's chief strategist.''

\emph{''The nation does not love that city. As far as they're concerned, it
is a Sodom and Gomorrah of fags and atheists who deserve to rot in
hell. The only reason the country cares what they think in San
Francisco is that they had the good fortune to have been blown to hell
by some Islamic terrorists.''}

``He's talking about the city where I live. At last count, 4,215 of my
neighbors were killed on the day he's talking about. But some of them
may not have been killed. Some of them disappeared into the same
prison where I was tortured. Some mothers and fathers, children and
lovers, brothers and sisters will never see their loved ones again --
because they were secretly imprisoned in an illegal jail right here in
the San Francisco Bay. They were shipped overseas. The records were
meticulous, but Carrie Johnstone has the encryption keys.'' I cut back
to Carrie Johnstone, the footage of her sitting at the board table
with Rooney, laughing.

I cut in the footage of Johnstone being arrested. ``When they arrested
her, I thought we'd get justice. All the people she broke and
disappeared. But the president --'' I cut to a still of him laughing
and playing golf on one of his many holidays ``-- and his Chief
Strategist --'' now a still of Rooney shaking hands with an infamous
terrorist leader who used to be on ``our side'' ``-- intervened. They
sent her to a secret military tribunal and now that tribunal has
cleared her. Somehow, they saw nothing wrong with all of this.''

I cut in a photomontage of the hundreds of shots of prisoners in their
cells that Barbara had published on the Bay Guardian's site the day we
were released. ``We elected these people. We pay their
salaries. They're supposed to be on our side. They're supposed to
defend our freedoms. But these people --'' a series of shots of
Johnstone and the others who'd been sent to the tribunal ``-- betrayed
our trust. The election is four months away. That's a lot of
time. Enough for you to go out and find five of your neighbors -- five
people who've given up on voting because their choice is 'none of the
above.'

``Talk to your neighbors. Make them promise to vote. Make them promise
to take the country back from the torturers and thugs. The people who
laughed at my friends as they lay fresh in their graves at the bottom
of the harbor. Make them promise to talk to their neighbors.

``Most of us choose none of the above. It's not working. You have to
choose -- choose freedom.

``My name is Marcus Yallow. I was tortured by my country, but I still
love it here. I'm seventeen years old. I want to grow up in a free
country. I want to live in a free country.''

I faded out to the logo of the website. Ange had built it, with help
from Jolu, who got us all the free hosting we could ever need on
Pigspleen.

The office was an interesting place. Technically we were called
Coalition of Voters for a Free America, but everyone called us the
Xnetters. The organization -- a charitable nonprofit -- had been
co-founded by Barbara and some of her lawyer friends right after the
liberation of Treasure Island. The funding was kicked off by some tech
millionaires who couldn't believe that a bunch of hacker kids had
kicked the DHS's ass. Sometimes, they'd ask us to go down the
peninsula to Sand Hill Road, where all the venture capitalists were,
and give a little presentation on Xnet technology. There were about a
zillion startups who were trying to make a buck on the Xnet.

Whatever -- I didn't have to have anything to do with it, and I got a
desk and an office with a storefront, right there on Valencia Street,
where we gave away ParanoidXbox CDs and held workshops on building
better WiFi antennas. A surprising number of average people dropped in
to make personal donations, both of hardware (you can run
ParanoidLinux on just about anything, not just Xbox Universals) and
cash money. They loved us.

The big plan was to launch our own ARG in September, just in time for
the election, and to really tie it in with signing up voters and
getting them to the polls. Only 42 percent of Americans showed up at
the polls for the last election -- nonvoters had a huge majority. I
kept trying to get Darryl and Van to one of our planning sessions, but
they kept on declining. They were spending a lot of time together, and
Van insisted that it was totally nonromantic. Darryl wouldn't talk to
me much at all, though he sent me long emails about just about
everything that wasn't about Van or terrorism or prison.

Ange squeezed my hand. ``God, I hate that woman,'' she said.

I nodded. ``Just one more rotten thing this country's done to Iraq,'' I
said. ``If they sent her to my town, I'd probably become a terrorist.''

``You did become a terrorist when they sent her to your town.''

``So I did,'' I said.

``Are you going to Ms Galvez's hearing on Monday?''

``Totally.'' I'd introduced Ange to Ms Galvez a couple weeks before,
when my old teacher invited me over for dinner. The teacher's union
had gotten a hearing for her before the board of the Unified School
District to argue for getting her old job back. They said that Fred
Benson was coming out of (early) retirement to testify against her. I
was looking forward to seeing her again.

``Do you want to go get a burrito?''

``Totally.''

``Let me get my hot-sauce,'' she said.

I checked my email one more time -- my PirateParty email, which still
got a dribble of messages from old Xnetters who hadn't found my
Coalition of Voters address yet.

The latest message was from a throwaway email address from one of the
new Brazilian anonymizers.

\edialog{Found her, thanks. You didn't tell me she was so h4wt.}

``Who's \emph{that} from?''

I laughed. ``Zeb,'' I said. ``Remember Zeb? I gave him Masha's email
address. I figured, if they're both underground, might as well
introduce them to one another.''

``He thinks Masha is \emph{cute}?''

``Give the guy a break, he's clearly had his mind warped by
circumstances.''

``And you?''

``Me?''

``Yeah -- was your mind warped by circumstances?''

I held Ange out at arm's length and looked her up and down and up and
down. I held her cheeks and stared through her thick-framed glasses
into her big, mischievous tilted eyes. I ran my fingers through her
hair.

``Ange, I've never thought more clearly in my whole life.''

She kissed me then, and I kissed her back, and it was some time before
we went out for that burrito.

\backmatter
\section{Afterword
by Bruce Schneier}

I'm a security technologist.  My job is making people secure.

I think about security systems and how to break them.  Then, how to
make them more secure.  Computer security systems.  Surveillance
systems.  Airplane security systems and voting machines and RFID chips
and everything else.

Cory invited me into the last few pages of his book because he wanted
me to tell you that security is fun.  It's incredibly fun.  It's cat
and mouse, who can outsmart whom, hunter versus hunted fun.  I think
it's the most fun job you can possibly have.  If you thought it was
fun to read about Marcus outsmarting the gait-recognition cameras with
rocks in his shoes, think of how much more fun it would be if you were
the first person in the world to think of that.

Working in security means knowing a lot about technology.  It might
mean knowing about computers and networks, or cameras and how they
work, or the chemistry of bomb detection.  But really, security is a
mindset.  It's a way of thinking.  Marcus is a great example of that
way of thinking.  He's always looking for ways a security system
fails.  I'll bet he couldn't walk into a store without figuring out a
way to shoplift.  Not that he'd do it -- there's a difference between
knowing how to defeat a security system and actually defeating it --
but he'd know he could.

It's how security people think. We're constantly looking at security
systems and how to get around them; we can't help it.

This kind of thinking is important no matter what side of security
you're on.  If you've been hired to build a shoplift-proof store,
you'd better know how to shoplift.  If you're designing a camera
system that detects individual gaits, you'd better plan for people
putting rocks in their shoes.  Because if you don't, you're not going
to design anything good.

So when you're wandering through your day, take a moment to look at
the security systems around you.  Look at the cameras in the stores
you shop at.  (Do they prevent crime, or just move it next door?)  See
how a restaurant operates.  (If you pay after you eat, why don't more
people just leave without paying?)  Pay attention at airport security.
(How could you get a weapon onto an airplane?)  Watch what the teller
does at a bank.  (Bank security is designed to prevent tellers from
stealing just as much as it is to prevent you from stealing.)  Stare
at an anthill.  (Insects are all about security.)  Read the
Constitution, and notice all the ways it provides people with security
against government. Look at traffic lights and door locks and all the
security systems on television and in the movies.  Figure out how they
work, what threats they protect against and what threats they don't,
how they fail, and how they can be exploited.

Spend enough time doing this, and you'll find yourself thinking
differently about the world.  You'll start noticing that many of the
security systems out there don't actually do what they claim to, and
that much of our national security is a waste of money.  You'll
understand privacy as essential to security, not in opposition.
You'll stop worrying about things other people worry about, and start
worrying about things other people don't even think about.

Sometimes you'll notice something about security that no one has ever
thought about before.  And maybe you'll figure out a new way to break
a security system.

It was only a few years ago that someone invented phishing.

I'm frequently amazed how easy it is to break some pretty big-name
security systems.  There are a lot of reasons for this, but the big
one is that it's impossible to prove that something is secure.  All
you can do is try to break it. -- if you fail, you know that it's
secure enough to keep \emph{you} out, but what about someone who's smarter
than you? Anyone can design a security system so strong he himself
can't break it.

Think about that for a second, because it's not obvious.  No one is
qualified to analyze their own security designs, because the designer
and the analyzer will be the same person, with the same limits.
Someone else has to analyze the security, because it has to be secure
against things the designers didn't think of.

This means that all of us have to analyze the security that other
people design.  And surprisingly often, one of us breaks it.  Marcus's
exploits aren't far-fetched; that kind of thing happens all the time.
Go onto the net and look up ``bump key'' or ``Bic pen Kryptonite lock'';
you'll find a couple of really interesting stories about seemingly
strong security defeated by pretty basic technology.

And when that happens, be sure to publish it on the Internet
somewhere. Secrecy and security aren't the same, even though it may
seem that way.  Only bad security relies on secrecy; good security
works even if all the details of it are public.

And publishing vulnerabilities forces security designers to design
better security, and makes us all better consumers of security.  If
you buy a Kryptonite bike lock and it can be defeated with a Bic pen,
you're not getting very good security for your money.  And, likewise,
if a bunch of smart kids can defeat the DHS's antiterrorist
technologies, then it's not going to do a very good job against real
terrorists.

Trading privacy for security is stupid enough; not getting any actual
security in the bargain is even stupider.

So close the book and go.  The world is full of security systems.
Hack one of them.

Bruce Schneier \url{http://www.schneier.com}

\section{Afterword
by Andrew ``bunnie'' Huang, Xbox Hacker}
 
Hackers are explorers, digital pioneers. It's in a hacker's nature to
question conventions and be tempted by intricate problems. Any complex
system is sport for a hacker; a side effect of this is the hacker's
natural affinity for problems involving security. Society is a large
and complex system, and is certainly not off limits to a little
hacking. As a result, hackers are often stereotyped as iconoclasts and
social misfits, people who defy social norms for the sake of
defiance. When I hacked the Xbox in 2002 while at MIT, I wasn't doing
it to rebel or to cause harm; I was just following a natural impulse,
the same impulse that leads to fixing a broken iPod or exploring the
roofs and tunnels at MIT.
 
Unfortunately, the combination of not complying with social norms and
knowing ``threatening'' things like how to read the arphid on your
credit card or how to pick locks causes some people to fear
hackers. However, the motivations of a hacker are typically as simple
as ``I'm an engineer because I like to design things.'' People often ask
me, ``Why did you hack the Xbox security system?'' And my answer is
simple: First, I own the things that I buy. If someone can tell me
what I can and can't run on my hardware, then I don't own it. Second,
because it's there. It's a system of sufficient complexity to make
good sport. It was a great diversion from the late nights working on
my PhD.
 
I was lucky. The fact that I was a graduate student at MIT when I
hacked the Xbox legitimized the activity in the eyes of the right
people. However, the right to hack shouldn't only be extended to
academics. I got my start on hacking when I was just a boy in
elementary school, taking apart every electronic appliance I could get
my hands on, much to my parents' chagrin. My reading collection
included books on model rocketry, artillery, nuclear weaponry and
explosives manufacture -- books that I borrowed from my school library
(I think the Cold War influenced the reading selection in public
schools). I also played with my fair share of ad-hoc fireworks and
roamed the open construction sites of houses being raised in my
Midwestern neighborhood. While not the wisest of things to do, these
were important experiences in my coming of age and I grew up to be a
free thinker because of the social tolerance and trust of my
community.
 
Current events have not been so kind to aspiring hackers. Little
Brother shows how we can get from where we are today to a world where
social tolerance for new and different thoughts dies altogether. A
recent event highlights exactly how close we are to crossing the line
into the world of Little Brother. I had the fortune of reading an
early draft of Little Brother back in November 2006. Fast forward two
months to the end of January 2007, when Boston police found suspected
explosive devices and shut down the city for a day. These devices
turned out to be nothing more than circuit boards with flashing LEDs,
promoting a show for the Cartoon Network. The artists who placed this
urban graffiti were taken in as suspected terrorists and ultimately
charged with felony; the network producers had to shell out a \$2
million settlement, and the head of the Cartoon Network resigned over
the fallout.

Have the terrorists already won? Have we given in to fear, such that
artists, hobbyists, hackers, iconoclasts, or perhaps an unassuming
group of kids playing Harajuku Fun Madness, could be so trivially
implicated as terrorists?

There is a term for this dysfunction--it is called an autoimmune
disease, where an organism's defense system goes into overdrive so
much that it fails to recognize itself and attacks its own
cells. Ultimately, the organism self-destructs. Right now, America is
on the verge of going into anaphylactic shock over its own freedoms,
and we need to inoculate ourselves against this. Technology is no cure
for this paranoia; in fact, it may enhance the paranoia: it turns us
into prisoners of our own device. Coercing millions of people to strip
off their outer garments and walk barefoot through metal detectors
every day is no solution either. It only serves to remind the
population every day that they have a reason to be afraid, while in
practice providing only a flimsy barrier to a determined adversary.

The truth is that we can't count on someone else to make us feel free,
and M1k3y won't come and save us the day our freedoms are lost to
paranoia. That's because M1k3y is in you and in me--Little Brother is
a reminder that no matter how unpredictable the future may be, we
don't win freedom through security systems, cryptography,
interrogations and spot searches. We win freedom by having the courage
and the conviction to live every day freely and to act as a free
society, no matter how great the threats are on the horizon.

Be like M1k3y: step out the door and dare to be free.

\section{Bibliography}
No writer creates from scratch -- we all engage in what Isaac Newton
called ``standing on the shoulders of giants.'' We borrow, plunder and
remix the art and culture created by those around us and by our
literary forebears.

If you liked this book and want to learn more, there are plenty of
sources to turn to, online and at your local library or bookstore.

Hacking is a great subject. All science relies on telling other people
what you've done so that they can verify it, learn from it, and
improve on it, and hacking is all about that process, so there's
plenty published on the subject.

{
\setlength{\parskip}{0.5\baselineskip plus 0.5\baselineskip}

Start with Andrew ``Bunnie'' Huang's ``Hacking the Xbox,'' (No Starch
Press, 2003) a wonderful book that tells the story of how Bunnie, then
a student at MIT, reverse-engineered the Xbox's anti-tampering
mechanisms and opened the way for all the subsequent cool hacks for
the platform. In telling the story, Bunnie has also created a kind of
Bible for reverse engineering and hardware hacking.

Bruce Schneier's ``Secrets and Lies'' (Wiley, 2000) and ``Beyond Fear''
(Copernicus, 2003) are the definitive lay-person's texts on
understanding security and thinking critically about it, while his
``Applied Cryptography'' (Wiley, 1995) remains the authoritative source
for understanding crypto. Bruce maintains an excellent blog and
mailing list at \url{schneier.com/blog.} Crypto and security are the realm
of the talented amateur, and the ``cypherpunk'' movement is full of
kids, home-makers, parents, lawyers, and every other stripe of person,
hammering away on security protocols and ciphers.

There are several great magazines devoted to this subject, but the two
best ones are 2600: The Hacker Quarterly, which is full of
pseudonymous, boasting accounts of hacks accomplished, and O'Reilly's
MAKE magazine, which features solid HOWTOs for making your own
hardware projects at home.

The online world overflows with material on this subject, of
course. Ed Felten and Alex J Halderman's Freedom to Tinker
(\url{http://www.freedom-to-tinker.com}) is a blog maintained by two
fantastic Princeton engineering profs who write lucidly about
security, wiretapping, anti-copying technology and crypto.

Don't miss Natalie Jeremijenko's ``Feral Robotics'' at UC San Diego
(\url{http://xdesign.ucsd.edu/feralrobots/}). Natalie and her students
rewire toy robot dogs from Toys R Us and turn them into bad-ass
toxic-waste detectors. They unleash them on public parks where big
corporations have dumped their waste and demonstrate in media-friendly
fashion how toxic the ground is.

Like many of the hacks in this book, the tunneling-over-DNS stuff is
real. Dan Kaminsky, a tunneling expert of the first water, published
details in 2004 (\url{http://www.doxpara.com/bo2004.ppt}).

The guru of ``citizen journalism'' is Dan Gillmor, who is pres\-ently
running Center for Citizen Media at Harvard and UC Berkeley -- he also
wrote a hell of a book on the subject, ``We, the Media'' (O'Reilly,
2004).

If you want to learn more about hacking arphids, start with Annalee
Newitz's Wired Magazine article ``The RFID Hacking Underground''
(\url{http://www.wirednews.com/wired/archive/14.05/rfid.html}). Adam
Greenfield's ``Everyware'' (New Riders Press, 2006) is a chilling look
at the dangers of a world of RFIDs.

Neal Gershenfeld's Fab Lab at MIT (\url{http://fab.cba.mit.edu}) is
hacking out the world's first real, cheap ``3D printers'' that can pump
out any object you can dream of. This is documented in Gershenfeld's
excellent book on the subject, ``Fab'' (Basic Books, 2005).

Bruce Sterling's ``Shaping Things'' (MIT Press, 2005) shows how arphids
and fabs could be used to force companies to build products that don't
poison the world.

Speaking of Bruce Sterling, he wrote the first great book on hackers
and the law, ``The Hacker Crackdown'' (Bantam, 1993), which is also the
first book published by a major publisher that was released on the
Internet at the same time (copies abound; see
\url{http://stuff.mit.edu/hacker/hacker.html} for one). It was reading
this book that turned me on to the Electronic Frontier Foundation,
where I was privileged to work for four years.

The Electronic Frontier Foundation (\url{http://www.eff.org}) is a
charitable membership organization with a student rate. They spend the
money that private individuals give them to keep the Internet safe for
personal liberty, free speech, due process, and the rest of the Bill
of Rights. They're the Internet's most effective freedom fighters, and
you can join the struggle just by signing up for their mailing list
and writing to your elected officials when they're considering selling
you out in the name of fighting terrorism, piracy, the mafia, or
whatever bogeyman has caught their attention today. EFF also helps
maintain , The Onion Router, which is a real technology you can use
\emph{right now} to get out of your government, school or library's
censoring firewall (\url{http://tor.eff.org}).

EFF has a huge, deep website with amazing information aimed at a
general audience, as do the American Civil Liberties Union
(\url{http://aclu.org}), Public Knowledge
(\url{http://publicknowledge.org}), FreeCulture
(\url{http://freeculture.org}), Creative Commons
(\url{http://creativecommons.org}) -- all of which also are worthy of
your support. FreeCulture is an international student movement that
actively recruits kids to found their own local chapters at their high
schools and universities. It's a great way to get involved and make a
difference.

A lot of websites chronicle the fight for cyberliberties, but few go
at it with the verve of Slashdot, ``News for Nerds, Stuff That Matters''
(\url{http://slashdot.org}).

And of course, you \emph{have to} visit Wikipedia, the collaborative,
net-authored encyclopedia that anyone can edit, with more than
1,000,000 entries in English alone. Wikipedia covers hacking and
counterculture in astonishing depth and with amazing,
up-to-the-nanosecond currency. One caution: you can't just look at the
entries in Wikipedia. It's really important to look at the ``History''
and ``Discussion'' links at the top of every Wikipedia page to see how
the current version of the truth was arrived it, get an appreciation
for the competing points-of-view there, and decide for yourself whom
you trust.

If you want to get at some \emph{real} forbidden knowledge, have a skim
around Cryptome (\url{http://cryptome.org}), the world's most amazing
archive of secret, suppressed and liberated information. Cryptome's
brave publishers collect material that's been pried out of the state
by Freedom of Information Act requests or leaked by whistle-blowers
and publishes it.

The best fictional account of the history of crypto is, hands-down,
Neal Stephenson's Cryptonomicon (Avon, 2002). Stephenson tells the
story of Alan Turing and the Nazi Enigma Machine, turning it into a
gripping war-novel that you won't be able to put down.

The Pirate Party mentioned in Little Brother is real and thriving in
Sweden (\url{http://www.piratpartiet.se}), Denmark, the USA and France
at the time of this writing (July, 2006). They're a little out-there,
but a movement takes all kinds.

Speaking of out-there, Abbie Hoffman and the Yippies did indeed try to
levitate the Pentagon, throw money into the stock exchange, and work
with a group called the Up Against the Wall Motherf\_\_\_\_\_ers. Abbie
Hoffman's classic book on ripping off the system, ``Steal This Book,''
is back in print (Four Walls Eight Windows, 2002) and it's also online
as a collaborative wiki for people who want to try to update it
(\url{http://stealthiswiki.nine9pages.com}).

Hoffman's autobiography, ``Soon to Be a Major Motion Picture'' (also in
print from Four Walls Eight Windows) is one of my favorite memoirs
ever, even if it is highly fictionalized. Hoffman was an incredible
storyteller and had great activist instincts. If you want to know how
he really lived his life, though, try Larry Sloman's ``Steal This
Dream'' (Doubleday, 1998).

More counterculture fun: Jack Kerouac's ``On the Road'' can be had in
practically any used bookstore for a buck or two. Allan Ginsberg's
``HOWL'' is online in many places, and you can hear him read it if you
search for the MP3 at archive.org. For bonus points, track down the
album ``Tenderness Junction'' by the Fugs, which includes the audio of
Allan Ginsberg and Abbie Hoffman's levitation ceremony at the
Pentagon.

This book couldn't have been written if not for George Orwell's
magnificent, world-changing ``1984,'' the best novel ever published on
how societies go wrong. I read this book when I was 12 and have read
it 30 or 40 times since, and every time, I get something new out of
it. Orwell was a master of storytelling and was clearly sick over the
totalitarian state that emerged in the Soviet Union. 1984 holds up
today as a genuinely frightening work of science fiction, and it is
one of the novels that literally changed the world. Today, ``Orwellian''
is synonymous with a state of ubiquitous surveillance, doublethink,
and torture.

Many novelists have tackled parts of the story in Little
Brother. Daniel Pinkwater's towering comic masterpiece, ``Alan
Men\-del\-sohn: The Boy From Mars'' (presently in print as part of the
omnibus ``5 Novels,'' Farrar, Straus and Giroux, 1997) is a book that
every geek needs to read. If you've ever felt like an outcast for
being too smart or weird, READ THIS BOOK. It changed my life.

On a more contemporary front, there's Scott Westerfeld's ``So
Yesterday'' (Razorbill, 2004), which follows the adventures of cool
hunters and counterculture jammers. Scott and his wife Justine
Larbalestier were my partial inspiration to write a book for young
adults -- as was Kathe Koja. Thanks, guys.
}

\section{Acknowledgments}

This book owes a tremendous debt to many writers, friends, mentors,
and heroes who made it possible.

For the hackers and cypherpunks: Bunnie Huang, Seth Schoen, Ed Felten,
Alex Halderman, Gweeds, Natalie Jeremijenko, Emmanuel Goldstein, Aaron
Swartz

For the heroes: Mitch Kapor, John Gilmore, John Perry Barlow, Larry
Lessig, Shari Steele, Cindy Cohn, Fred von Lohmann, Jamie Boyle,
George Orwell, Abbie Hoffman, Joe Trippi, Bruce Schneier, Ross Dowson,
Harry Kopyto, Tim O'Reilly

For the writers: Bruce Sterling, Kathe Koja, Scott Westerfeld, Justine
Larbalestier, Pat York, Annalee Newitz, Dan Gillmor, Daniel Pinkwater,
Kevin Pouslen, Wendy Grossman, Jay Lake, Ben Ros\-en\-baum

For the friends: Fiona Romeo, Quinn Norton, Danny O'Brien, Jon
Gilbert, danah boyd, Zak Hanna, Emily Hurson, Grad Conn, John Henson,
Amanda Foubister, Xeni Jardin, Mark Frauenfelder, David Pescovitz,
John Battelle, Karl Levesque, Kate Miles, Neil and Tara-Lee Doctorow,
Rael Dornfest, Ken Snider

For the mentors: Judy Merril, Roz and Gord Doctorow, Harriet Wolff,
Jim Kelly, Damon Knight, Scott Edelman

Thank you all for giving me the tools to think and write about these
ideas.

\section{Creative Commons}

Creative Commons Legal Code

Attribution-NonCommercial-ShareAlike 3.0 Unported

    CREATIVE COMMONS CORPORATION IS NOT A LAW FIRM AND DOES NOT
PROVIDE LEGAL SERVICES. DISTRIBUTION OF THIS LICENSE DOES NOT CREATE
AN ATTORNEY-CLIENT RELATIONSHIP. CREATIVE COMMONS PROVIDES THIS
INFORMATION ON AN ``AS-IS'' BASIS. CREATIVE COMMONS MAKES NO WARRANTIES
REGARDING THE INFORMATION PROVIDED, AND DISCLAIMS LIABILITY FOR
DAMAGES RESULTING FROM ITS USE.

License

THE WORK (AS DEFINED BELOW) IS PROVIDED UNDER THE TERMS OF THIS
CREATIVE COMMONS PUBLIC LICENSE (``CCPL'' OR ``LICENSE''). THE WORK IS
PROTECTED BY COPYRIGHT AND/OR OTHER APPLICABLE LAW. ANY USE OF THE
WORK OTHER THAN AS AUTHORIZED UNDER THIS LICENSE OR COPYRIGHT LAW IS
PROHIBITED.

BY EXERCISING ANY RIGHTS TO THE WORK PROVIDED HERE, YOU ACCEPT AND
AGREE TO BE BOUND BY THE TERMS OF THIS LICENSE. TO THE EXTENT THIS
LICENSE MAY BE CONSIDERED TO BE A CONTRACT, THE LICENSOR GRANTS YOU
THE RIGHTS CONTAINED HERE IN CONSIDERATION OF YOUR ACCEPTANCE OF SUCH
TERMS AND CONDITIONS.

1. Definitions

   1. ``Adaptation'' means a work based upon the Work, or upon the Work
and other pre-existing works, such as a translation, adaptation,
derivative work, arrangement of music or other alterations of a
literary or artistic work, or phonogram or performance and includes
cinematographic adaptations or any other form in which the Work may be
recast, transformed, or adapted including in any form recognizably
derived from the original, except that a work that constitutes a
Collection will not be considered an Adaptation for the purpose of
this License. For the avoidance of doubt, where the Work is a musical
work, performance or phonogram, the synchronization of the Work in
timed-relation with a moving image (``synching'') will be considered an
Adaptation for the purpose of this License.

   2. ``Collection'' means a collection of literary or artistic works,
such as encyclopedias and anthologies, or performances, phonograms or
broadcasts, or other works or subject matter other than works listed
in Section 1(g) below, which, by reason of the selection and
arrangement of their contents, constitute intellectual creations, in
which the Work is included in its entirety in unmodified form along
with one or more other contributions, each constituting separate and
independent works in themselves, which together are assembled into a
collective whole. A work that constitutes a Collection will not be
considered an Adaptation (as defined above) for the purposes of this
License.

   3. ``Distribute'' means to make available to the public the original
and copies of the Work or Adaptation, as appropriate, through sale or
other transfer of ownership.

   4. ``License Elements'' means the following high-level license
attributes as selected by Licensor and indicated in the title of this
License: Attribution, Noncommercial, ShareAlike.
 
   5. ``Licensor'' means the individual, individuals, entity or entities
that offer(s) the Work under the terms of this License.

   6. ``Original Author'' means, in the case of a literary or artistic
work, the individual, individuals, entity or entities who created the
Work or if no individual or entity can be identified, the publisher;
and in addition (i) in the case of a performance the actors, singers,
musicians, dancers, and other persons who act, sing, deliver, declaim,
play in, interpret or otherwise perform literary or artistic works or
expressions of folklore; (ii) in the case of a phonogram the producer
being the person or legal entity who first fixes the sounds of a
performance or other sounds; and, (iii) in the case of broadcasts, the
organization that transmits the broadcast.

   7. ``Work'' means the literary and/or artistic work offered under the
terms of this License including without limitation any production in
the literary, scientific and artistic domain, whatever may be the mode
or form of its expression including digital form, such as a book,
pamphlet and other writing; a lecture, address, sermon or other work
of the same nature; a dramatic or dramatico-musical work; a
choreographic work or entertainment in dumb show; a musical
composition with or without words; a cinematographic work to which are
assimilated works expressed by a process analogous to cinematography;
a work of drawing, painting, architecture, sculpture, engraving or
lithography; a photographic work to which are assimilated works
expressed by a process analogous to photography; a work of applied
art; an illustration, map, plan, sketch or three-dimensional work
relative to geography, topography, architecture or science; a
performance; a broadcast; a phonogram; a compilation of data to the
extent it is protected as a copyrightable work; or a work performed by
a variety or circus performer to the extent it is not otherwise
considered a literary or artistic work.

   8. ``You'' means an individual or entity exercising rights under this
License who has not previously violated the terms of this License with
respect to the Work, or who has received express permission from the
Licensor to exercise rights under this License despite a previous
violation.
 
   9. ``Publicly Perform'' means to perform public recitations of the
Work and to communicate to the public those public recitations, by any
means or process, including by wire or wireless means or public
digital performances; to make available to the public Works in such a
way that members of the public may access these Works from a place and
at a place individually chosen by them; to perform the Work to the
public by any means or process and the communication to the public of
the performances of the Work, including by public digital performance;
to broadcast and rebroadcast the Work by any means including signs,
sounds or images.

  10. ``Reproduce'' means to make copies of the Work by any means
including without limitation by sound or visual recordings and the
right of fixation and reproducing fixations of the Work, including
storage of a protected performance or phonogram in digital form or
other electronic medium.

2. Fair Dealing Rights. Nothing in this License is intended to reduce,
limit, or restrict any uses free from copyright or rights arising from
limitations or exceptions that are provided for in connection with the
copyright protection under copyright law or other applicable laws.

3. License Grant. Subject to the terms and conditions of this License,
Licensor hereby grants You a worldwide, royalty-free, non-exclusive,
perpetual (for the duration of the applicable copyright) license to
exercise the rights in the Work as stated below:

   1. to Reproduce the Work, to incorporate the Work into one or more
Collections, and to Reproduce the Work as incorporated in the
Collections;

   2. to create and Reproduce Adaptations provided that any such
Adaptation, including any translation in any medium, takes reasonable
steps to clearly label, demarcate or otherwise identify that changes
were made to the original Work. For example, a translation could be
marked ``The original work was translated from English to Spanish,'' or
a modification could indicate ``The original work has been modified.'';

   3. to Distribute and Publicly Perform the Work including as
incorporated in Collections; and,
 
   4. to Distribute and Publicly Perform Adaptations.

The above rights may be exercised in all media and formats whether now
known or hereafter devised. The above rights include the right to make
such modifications as are technically necessary to exercise the rights
in other media and formats. Subject to Section 8(f), all rights not
expressly granted by Licensor are hereby reserved, including but not
limited to the rights described in Section 4(e).

4. Restrictions. The license granted in Section 3 above is expressly
made subject to and limited by the following restrictions:

   1. You may Distribute or Publicly Perform the Work only under the
terms of this License. You must include a copy of, or the Uniform
Resource Identifier (URI) for, this License with every copy of the
Work You Distribute or Publicly Perform. You may not offer or impose
any terms on the Work that restrict the terms of this License or the
ability of the recipient of the Work to exercise the rights granted to
that recipient under the terms of the License. You may not sublicense
the Work. You must keep intact all notices that refer to this License
and to the disclaimer of warranties with every copy of the Work You
Distribute or Publicly Perform. When You Distribute or Publicly
Perform the Work, You may not impose any effective technological
measures on the Work that restrict the ability of a recipient of the
Work from You to exercise the rights granted to that recipient under
the terms of the License. This Section 4(a) applies to the Work as
incorporated in a Collection, but this does not require the Collection
apart from the Work itself to be made subject to the terms of this
License. If You create a Collection, upon notice from any Licensor You
must, to the extent practicable, remove from the Collection any credit
as required by Section 4(d), as requested. If You create an
Adaptation, upon notice from any Licensor You must, to the extent
practicable, remove from the Adaptation any credit as required by
Section 4(d), as requested.

   2. You may Distribute or Publicly Perform an Adaptation only under:
(i) the terms of this License; (ii) a later version of this License
with the same License Elements as this License; (iii) a Creative
Commons jurisdiction license (either this or a later license version)
that contains the same License Elements as this License (e.g.,
Attribution-NonCommercial-ShareAlike 3.0 US) (``Applicable
License''). You must include a copy of, or the URI, for Applicable
License with every copy of each Adaptation You Distribute or Publicly
Perform. You may not offer or impose any terms on the Adaptation that
restrict the terms of the Applicable License or the ability of the
recipient of the Adaptation to exercise the rights granted to that
recipient under the terms of the Applicable License. You must keep
intact all notices that refer to the Applicable License and to the
disclaimer of warranties with every copy of the Work as included in
the Adaptation You Distribute or Publicly Perform. When You Distribute
or Publicly Perform the Adaptation, You may not impose any effective
technological measures on the Adaptation that restrict the ability of
a recipient of the Adaptation from You to exercise the rights granted
to that recipient under the terms of the Applicable License. This
Section 4(b) applies to the Adaptation as incorporated in a
Collection, but this does not require the Collection apart from the
Adaptation itself to be made subject to the terms of the Applicable
License.

   3. You may not exercise any of the rights granted to You in Section
3 above in any manner that is primarily intended for or directed
toward commercial advantage or private monetary compensation. The
exchange of the Work for other copyrighted works by means of digital
file-sharing or otherwise shall not be considered to be intended for
or directed toward commercial advantage or private monetary
compensation, provided there is no payment of any monetary
compensation in con-nection with the exchange of copyrighted works.

   4. If You Distribute, or Publicly Perform the Work or any
Adaptations or Collections, You must, unless a request has been made
pursuant to Section 4(a), keep intact all copyright notices for the
Work and provide, reasonable to the medium or means You are utilizing:
(i) the name of the Original Author (or pseudonym, if applicable) if
supplied, and/or if the Original Author and/or Licensor designate
another party or parties (e.g., a sponsor institute, publishing
entity, journal) for attribution (``Attribution Parties'') in Licensor's
copyright notice, terms of service or by other reasonable means, the
name of such party or parties; (ii) the title of the Work if supplied;
(iii) to the extent reasonably practicable, the URI, if any, that
Licensor specifies to be associated with the Work, unless such URI
does not refer to the copyright notice or licensing information for
the Work; and, (iv) consistent with Section 3(b), in the case of an
Adaptation, a credit identifying the use of the Work in the Adaptation
(e.g., ``French translation of the Work by Original Author,'' or
``Screenplay based on original Work by Original Author''). The credit
required by this Section 4(d) may be implemented in any reasonable
manner; provided, however, that in the case of a Adaptation or
Collection, at a minimum such credit will appear, if a credit for all
contributing authors of the Adaptation or Collection appears, then as
part of these credits and in a manner at least as prominent as the
credits for the other contributing authors. For the avoidance of
doubt, You may only use the credit required by this Section for the
purpose of attribution in the manner set out above and, by exercising
Your rights under this License, You may not implicitly or explicitly
assert or imply any connection with, sponsorship or endorsement by the
Original Author, Licensor and/or Attribution Parties, as appropriate,
of You or Your use of the Work, without the separate, express prior
written permission of the Original Author, Licensor and/or Attribution
Parties.

   5. For the avoidance of doubt:
 
         1. Non-waivable Compulsory License Schemes. In those
jurisdictions in which the right to collect royalties through any
statutory or compulsory licensing scheme cannot be waived, the
Licensor reserves the exclusive right to collect such royalties for
any exercise by You of the rights granted under this License;
 
         2. Waivable Compulsory License Schemes. In those
jurisdictions in which the right to collect royalties through any
statutory or compulsory licensing scheme can be waived, the Licensor
reserves the exclusive right to collect such royalties for any
exercise by You of the rights granted under this License if Your
exercise of such rights is for a purpose or use which is otherwise
than noncommercial as permitted under Section 4(c) and otherwise
waives the right to collect royalties through any statutory or
compulsory licensing scheme; and,

         3. Voluntary License Schemes. The Licensor reserves the right
to collect royalties, whether individually or, in the event that the
Licensor is a member of a collecting society that administers
voluntary licensing schemes, via that society, from any exercise by
You of the rights granted under this License that is for a purpose or
use which is otherwise than noncommercial as permitted under Section
4(c).
 
   6. Except as otherwise agreed in writing by the Licensor or as may
be otherwise permitted by applicable law, if You Reproduce, Distribute
or Publicly Perform the Work either by itself or as part of any
Adaptations or Collections, You must not distort, mutilate, modify or
take other derogatory action in relation to the Work which would be
prejudicial to the Original Author's honor or reputation. Licensor
agrees that in those jurisdictions (e.g. Japan), in which any exercise
of the right granted in Section 3(b) of this License (the right to
make Adaptations) would be deemed to be a distortion, mutilation,
modification or other derogatory action prejudicial to the Original
Author's honor and reputation, the Licensor will waive or not assert,
as appropriate, this Section, to the fullest extent permitted by the
applicable national law, to enable You to reasonably exercise Your
right under Section 3(b) of this License (right to make Adaptations)
but not otherwise.

5. Representations, Warranties and Disclaimer

UNLESS OTHERWISE MUTUALLY AGREED TO BY THE PARTIES IN WRITING AND TO
THE FULLEST EXTENT PERMITTED BY APPLICABLE LAW, LICENSOR OFFERS THE
WORK AS-IS AND MAKES NO REPRESENTATIONS OR WARRANTIES OF ANY KIND
CONCERNING THE WORK, EXPRESS, IMPLIED, STATUTORY OR OTHERWISE,
INCLUDING, WITHOUT LIMITATION, WARRANTIES OF TITLE, MERCHANTABILITY,
FITNESS FOR A PARTICULAR PURPOSE, NONINFRINGEMENT, OR THE ABSENCE OF
LATENT OR OTHER DEFECTS, ACCURACY, OR THE PRESENCE OF ABSENCE OF
ERRORS, WHETHER OR NOT DISCOVERABLE. SOME JURISDICTIONS DO NOT ALLOW
THE EXCLUSION OF IMPLIED WARRANTIES, SO THIS EXCLUSION MAY NOT APPLY
TO YOU.

6. Limitation on Liability. EXCEPT TO THE EXTENT REQUIRED BY
APPLICABLE LAW, IN NO EVENT WILL LICENSOR BE LIABLE TO YOU ON ANY
LEGAL THEORY FOR ANY SPECIAL, INCIDENTAL, CONSEQUENTIAL, PUNITIVE OR
EXEMPLARY DAMAGES ARISING OUT OF THIS LICENSE OR THE USE OF THE WORK,
EVEN IF LICENSOR HAS BEEN ADVISED OF THE POSSIBILITY OF SUCH DAMAGES.

7. Termination

   1. This License and the rights granted hereunder will terminate
automatically upon any breach by You of the terms of this
License. Individuals or entities who have received Adaptations or
Collections from You under this License, however, will not have their
licenses terminated provided such individuals or entities remain in
full compliance with those licenses. Sections 1, 2, 5, 6, 7, and 8
will survive any termination of this License.

   2. Subject to the above terms and conditions, the license granted
here is perpetual (for the duration of the applicable copyright in the
Work). Notwithstanding the above, Licensor reserves the right to
release the Work under different license terms or to stop distributing
the Work at any time; provided, however that any such election will
not serve to withdraw this License (or any other license that has
been, or is required to be, granted under the terms of this License),
and this License will continue in full force and effect unless
terminated as stated above.

8. Miscellaneous

   1. Each time You Distribute or Publicly Perform the Work or a
Collection, the Licensor offers to the recipient a license to the Work
on the same terms and conditions as the license granted to You under
this License.
 
   2. Each time You Distribute or Publicly Perform an Adaptation,
Licensor offers to the recipient a license to the original Work on the
same terms and conditions as the license granted to You under this
License.

   3. If any provision of this License is invalid or unenforceable
under applicable law, it shall not affect the validity or
enforceability of the remainder of the terms of this License, and
without further action by the parties to this agreement, such
provision shall be reformed to the minimum extent necessary to make
such provision valid and enforceable.

   4. No term or provision of this License shall be deemed waived and
no breach consented to unless such waiver or consent shall be in
writing and signed by the party to be charged with such waiver or
consent.
  
   5. This License constitutes the entire agreement between the
parties with respect to the Work licensed here. There are no
understandings, agreements or representations with respect to the Work
not specified here. Licensor shall not be bound by any additional
provisions that may appear in any communication from You. This License
may not be modified without the mutual written agreement of the
Licensor and You.

   6. The rights granted under, and the subject matter referenced, in
this License were drafted utilizing the terminology of the Berne
Convention for the Protection of Literary and Artistic Works (as
amended on September 28, 1979), the Rome Convention of 1961, the WIPO
Copyright Treaty of 1996, the WIPO Performances and Phonograms Treaty
of 1996 and the Universal Copyright Convention (as revised on July 24,
1971). These rights and subject matter take effect in the relevant
jurisdiction in which the License terms are sought to be enforced
according to the corresponding provisions of the implementation of
those treaty provisions in the applicable national law. If the
standard suite of rights granted under applicable copyright law
includes additional rights not granted under this License, such
additional rights are deemed to be included in the License; this
License is not intended to restrict the license of any rights under
applicable law.

    Creative Commons Notice

    Creative Commons is not a party to this License, and makes no
warranty whatsoever in connection with the Work. Creative Commons will
not be liable to You or any party on any legal theory for any damages
whatsoever, including without limitation any general, special,
incidental or consequential damages arising in connection to this
license. Notwithstanding the foregoing two (2) sentences, if Creative
Commons has expressly identified itself as the Licensor hereunder, it
shall have all rights and obligations of Licensor.

    Except for the limited purpose of indicating to the public that
the Work is licensed under the CCPL, Creative Commons does not
authorize the use by either party of the trademark ``Creative Commons''
or any related trademark or logo of Creative Commons without the prior
written consent of Creative Commons. Any permitted use will be in
compliance with Creative Commons' then-current trademark usage
guidelines, as may be published on its website or otherwise made
available upon request from time to time. For the avoidance of doubt,
this trademark restriction does not form part of this License.

    Creative Commons may be contacted at \url{http://creativecommons.org/}.

%<a rel="license"
%href="http://creativecommons.org/licenses/by-nc-sa/3.0/us/"> 
%<img
%alt="Creative Commons License" style="border-width:0"
%src="http://i.creativecommons.org/l/by-nc-sa/3.0/us/88x31.png"/> </a>
%<br/> <span xmlns:dc="http://purl.org/dc/elements/1.1/"
%href="http://purl.org/dc/dcmitype/Text" property="dc:title"
%rel="dc:type"> Little Brother </span>
%
%by
%
%<a xmlns:cc="http://creativecommons.org/ns#"
%href="http://craphound.com/littlebrother"
%property="cc:attributionName" rel="cc:attributionURL"> Cory Doctorow
%</a>
%
%is licensed under a
%
%<a rel="license"
%href="http://creativecommons.org/licenses/by-nc-sa/3.0/us/"> Creative
%Commons Attribution-Noncommercial-Share Alike 3.0 United States
%License </a>.


Little Brother
by
Cory Doctorow
is licensed under a 
Creative Commons Attribution-Noncommercial-Share Alike 3.0 United
States License.

\LaTeX{} by Mikael Vejdemo Johansson. Reworked and adapted by Harald Geyer.


\end{document}
