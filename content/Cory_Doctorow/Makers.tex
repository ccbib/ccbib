\hyphenation{mo-no-poly car-ne-gie pro-ject pro-gress mo-dem rou-lette
  browse-wrap Use-net mon-as-tery mo-dems}
\hyphenation{co-me-dic polt-roon stove-pipe Ma-dame scru-ta-ble star-tling}
\hyphenation{heal-thily lim-ou-sines wrest-lers tan-trum push-over un-asked
  bras-siere bro-th-er}
\hyphenation{Can-a-da Fred-rick teen-agers wrest-ler Cha-vez Tho-mas 
  a-nom-a-lies sur-veil-lance ar-mies ref-u-gee ref-u-gees bris-tling
  eve-ning man-chu-ria man-chu-ri-an mid-terms me-di-um jap-a-nese}
\hyphenation{spend-ers googl-ing tour-ist tour-ists leg-end-ary}
\hyphenation{Dan-iel Van-essa Doc-to-row Ste-phen-son}
\hyphenation{de-cade sur-veilled rout-ers Wol-fen-stein teen-ager to-night}
\hyphenation{his-to-gram an-o-nym-ize Ga-la-xy sym-pa-the-tic}
\hyphenation{ar-phid ar-phids Found-ers}
\hyphenation{stran-ger stran-gers shoul-der-blades dump-ling dump-lings}
\hyphenation{ice-pack guard-rail Sep-tem-ber boot-able e-co-nom-ist}
\hyphenation{grown-ups roos-ter shoe-laces li-quid-i-ty}
\hyphenation{side-arm}
\hyphenation{wo-man wo-men tan-trum tan-trums Le-nin-grad zom-bie bunk-house}
\hyphenation{up-tick bio-mass}
\hyphenation{of-fi-cial of-fi-cial-ly gov-ern-ment}
\hyphenation{heal-thy Or-ville spark-ling}
\hyphenation{ves-ti-bule Law-rence au-to-no-mous}
\hyphenation{sau-sage door-step staf-fer}
\hyphenation{tree-trunk}
\hyphenation{to-ron-to}
\hyphenation{qua-dril-lion-aire qua-dril-lion-aires}
\hyphenation{sports-jack-et sports-jack-ets}
\hyphenation{work-space skunk-works}
\hyphenation{kings-ton}


%% Traditional Layout: lines start with :: and have hard lineends
%\newenvironment{blog}{\renewcommand{\\}{\newline{}:\,}\newcommand\blogpar{\\}:\,}{}

%% Recommended layout, use distinct font but no hard lineends for
%% extra fexibility
\newenvironment{blog}{\renewcommand{\\}{ }\newcommand\blogpar\par\sffamily}{}

\newenvironment{email}{%
  \begin{flushleft}%
  \setlength{\parindent}{0pt}%
  \setlength{\parskip}{0.5\baselineskip}}{\end{flushleft}\medskip}

\newenvironment{projection}{%
  \begin{flushleft}%
  \spaceskip=1.5\fontdimen2\font plus 1.5\fontdimen3\font minus \fontdimen4\font%
  }{\end{flushleft}}

\newenvironment{speaker}{%
  \spaceskip=1.5\fontdimen2\font plus 1.5\fontdimen3\font minus \fontdimen4\font}{}

\newcommand{\headline}[1]{\smallskip\textuppercase{#1}\smallskip}

\newcommand\megaphone\relax

\hyphenation{Su-zan-ne mo-da-fi-nil Les-ter}

\begin{document}
\raggedbottom


\title{Makers}
\author{Cory Doctorow
  \thanks{doctorow@craphound.com}}
\lowertitleback{Tor Books: 978-0765312792\\HarperCollins UK/Voyager:
978-0007325221}
\date{Based on Version from 24 Jan 2012}
\maketitle

\subsection{About this download}

There’s a dangerous group of anti-copyright activists out there who
pose a clear and present danger to the future of authors and
publishing. They have no respect for property or laws. What’s more,
they’re powerful and organized, and have the ears of lawmakers and
the press.

I’m speaking, of course, of the legal departments at ebook
publishers.

These people don’t believe in copyright law. Copyright law says
that when you buy a book, you own it. You can give it away, you can
lend it, you can pass it on to your descendants or donate it to the
local homeless shelter. Owning books has been around for longer
than publishing books has. Copyright law has \emph{always}
recognized your right to own your books. When copyright laws are
made\dash{}by elected officials, acting for the public good\dash{}they always
safeguard this right.

But ebook publishers don’t respect copyright law, and they don’t
believe in your right to own property. Instead, they say that when
you “buy” an ebook, you’re really only \emph{licensing} that book,
and that copyright law is superseded by the thousands of farcical,
abusive words in the license agreement you click through on the way
to sealing the deal. (Of course, the button on their website says,
“Buy this book” and they talk about “Ebook sales” at conferences\dash{}no
one says, “License this book for your Kindle” or “Total licenses of
ebooks are up from 0.00001\% of all publishing to 0.0001\% of all
publishing, a 100-fold increase!”)

I say to hell with them. You bought it, you own it. I believe in
copyright law’s guarantee of ownership in your books.

So you own this ebook. The license agreement (see below), is from
Creative Commons and it gives you even \emph{more} rights than you
get to a regular book. Every word of it is a gift, not a
confiscation. Enjoy.

What do I want from you in return? Read the book. Tell your
friends. Review it on Amazon or at your local bookseller. Bring it
to your bookclub. Assign it to your students (older students,
please\dash{}that sex scene is a scorcher) (\emph{now} I’ve got your
attention, don’t I?). As Woody Guthrie wrote:

“This song is Copyrighted in U.S., under Seal of Copyright
\#154085, for a period of 28 years, and anybody caught singin’ it
without our permission, will be mighty good friends of ourn, cause
we don’t give a dern. Publish it. Write it. Sing it. Swing to it.
Yodel it. We wrote it, that’s all we wanted to do.”

Oh yeah. Also: if you like it,
\href{http://craphound.com/makers/buy}{buy it} or
\href{http://craphound.com/makers/donate}{donate a copy} to a
worthy, cash-strapped institution.

Why am I doing this? Because my problem isn’t piracy, it’s
obscurity (thanks, @timoreilly for this awesome aphorism). Because
free ebooks sell print books. Because I copied my ass off when I
was 17 and grew up to spend practically every discretionary cent I
have on books when I became an adult. Because I can’t stop you from
sharing it (zeroes and ones aren’t ever going to get harder to
copy); and because readers have shared the books they loved
forever; so I might as well enlist you to the cause.

I have always dreamt of writing sf novels, since I was six years
old. Now I do it. It is a goddamned dream come true, like growing
up to be a cowboy or an astronaut, except that you don’t get
oppressed by ranchers or stuck on the launchpad in an adult diaper
for 28 hours at a stretch. The idea that I’d get dyspeptic over
people\dash{}\emph{readers}\dash{}celebrating what I write is goddamned
\emph{bizarre}.

So, download this book.

Some rules of the road:

It’s kind of a tradition around here that my readers convert my
ebooks to their favorite formats and send them to me here, and it’s
one that I love! If you’ve converted these files to another format,
send them to me (doctorow@craphound.com, subject Makers Conversion)
and I’ll host them, but before you do, make sure you read the
following:

\begin{itemize}
\item
  Only one conversion per format, first come, first serve. That means
  that if someone’s already converted the file to a Femellhebber 3000
  document, that’s the one you’re going to find here. I just don’t
  know enough about esoteric readers to adjudicate disputes about
  what the ideal format is for your favorite device.
\item
  Make sure include a link to the reader as well. When you send me an
  ebook file, make sure that you include a link to the website for
  the reader technology as well so that I can include it below.
\item
  No cover art. The text of this book is freely copyable, the cover,
  not so much. The rights to it are controlled by my publisher, so
  don’t include it with your file.
\item
  No DRM. The Creative Commons license prohibits sharing the file
  with “DRM” (sometimes called “copy-protection”) on it, and that’s
  fine by me. Don’t send me the book with DRM on it. If you’re
  converting to a format that has a DRM option, make sure it’s
  switched off.
\end{itemize}
\subsection{A word to professors, librarians, and people who want to donate money to me}

Every time I put a book online for free, I get emails from readers
who want to send me donations for the book. I appreciate their
generous spirit, but I’m not interested in cash donations, because
my publishers are really important to me. They contribute
immeasurably to the book, improving it, introducing it to audience
I could never reach, helping me do more with my work. I have no
desire to cut them out of the loop.

But there has to be some good way to turn that generosity to good
use, and I think I’ve found it.

Here’s the deal: there are lots of professors and librarians who’d
love to get hard-copies of this book into their students’ and
patrons’ hands, but don’t have the budget for it.

There are generous people who want to send some cash my way to
thank me for the free ebooks.

I’m proposing that we put them together.

If you’re a prof or librarian and you want a free copy of Makers,
email freemakers@gmail.com with your name and the name and address
of your school. It’ll be posted below by my fantastic helper, Olga
Nunes, so that potential donors can see it.

If you enjoyed the electronic edition of Makers and you want to
donate something to say thanks, check below to find a teacher or
librarian you want to support. Then go to Amazon, BN.com, or your
favorite electronic bookseller and order a copy to the classroom,
then email a copy of the receipt (feel free to delete your address
and other personal info first!) to freemakers@gmail.com so that
Olga can mark that copy as sent. If you don’t want to be publicly
acknowledged for your generosity, let us know and we’ll keep you
anonymous, otherwise we’ll thank you on the donate page.

Check http://craphound.com/makers/donate for profs, librarians and
similar people seeking donations.

\subsection{This file is licensed under a Creative Commons US Attribution-NonCommercial-ShareAlike license:}

\href{http://creativecommons.org/licenses/by-nc-sa/3.0/}{http://creativecommons.org/licenses/by-nc-sa/3.0/}

You are free:

to Share \dash{} to copy, distribute and transmit the work

to Remix \dash{} to adapt the work

Under the following conditions:

Attribution \dash{} You must attribute the work in the manner specified
by the author or licensor (but not in any way that suggests that
they endorse you or your use of the work).

Noncommercial \dash{} You may not use this work for commercial purposes.

Share Alike \dash{} If you alter, transform, or build upon this work, you
may distribute the resulting work only under the same or similar
license to this one.

With the understanding that:

Waiver \dash{} Any of the above conditions can be waived if you get
permission from the copyright holder. Other Rights \dash{} In no way are
any of the following rights affected by the license: Your fair
dealing or fair use rights; The author’s moral rights; Rights other
persons may have either in the work itself or in how the work is
used, such as publicity or privacy rights. Notice \dash{} For any reuse
or distribution, you must make clear to others the license terms of
this work.

\subsection{Dedication:}

For “the risk-takers, the doers, the makers of things.”

\chapter{PART I}

Suzanne Church almost never had to bother with the blue blazer
these days. Back at the height of the dot-boom, she’d put on her
business journalist drag\dash{}blazer, blue sailcloth shirt, khaki
trousers, loafers\dash{}just about every day, putting in her obligatory
appearances at splashy press-conferences for high-flying IPOs and
mergers. These days, it was mostly work at home or one day a week
at the San Jose Mercury News’s office, in comfortable light
sweaters with loose necks and loose cotton pants that she could
wear straight to yoga after shutting her computer’s lid.

Blue blazer today, and she wasn’t the only one. There was Reedy
from the NYT’s Silicon Valley office, and Tribbey from the WSJ, and
that despicable rat-toothed jumped-up gossip columnist from one of
the UK tech-rags, and many others besides. Old home week, blue
blazers fresh from the dry-cleaning bags that had guarded them
since the last time the NASDAQ broke 5,000.

The man of the hour was Landon Kettlewell\dash{}the kind of outlandish
prep-school name that always seemed a little made up to her\dash{}the new
CEO and front for the majority owners of Kodak/Duracell. The
despicable Brit had already started calling them Kodacell. Buying
the company was pure Kettlewell: shrewd, weird, and ethical in a
twisted way.

“Why the hell have you done this, Landon?” Kettlewell asked himself
into his tie-mic. Ties and suits for the new Kodacell execs in the
room, like surfers playing dress-up. “Why buy two dinosaurs and
stick ’em together? Will they mate and give birth to a new
generation of less-endangered dinosaurs?”

He shook his head and walked to a different part of the stage,
thumbing a PowerPoint remote that advanced his slide on the
jumbotron to a picture of a couple of unhappy cartoon brontos
staring desolately at an empty nest. “Probably not. But there is a
good case for what we’ve just done, and with your indulgence, I’m
going to lay it out for you now.”

“Let’s hope he sticks to the cartoons,” Rat-Toothed hissed beside
her. His breath smelled like he’d been gargling turds. He had a
not-so-secret crush on her and liked to demonstrate his
alpha-maleness by making half-witticisms into her ear. “They’re
about his speed.”

She twisted in her seat and pointedly hunched over her computer’s
screen, to which she’d taped a thin sheet of polarized plastic that
made it opaque to anyone shoulder-surfing her. Being a halfway
attractive woman in Silicon Valley was more of a pain in the ass
than she’d expected, back when she’d been covering rustbelt
shenanigans in Detroit, back when there was an auto industry in
Detroit.

The worst part was that the Brit’s reportage was just spleen-filled
editorializing on the lack of ethics in the valley’s board-rooms (a
favorite subject of hers, which no doubt accounted for his
fellow-feeling), and it was also the crux of Kettlewell’s schtick.
The spectacle of an exec who talked ethics enraged Rat-Toothed more
than the vilest baby-killers. He was the kind of revolutionary who
liked his firing squads arranged in a circle.

“I’m not that dumb, folks,” Kettlewell said, provoking a stagey
laugh from Mr Rat-Tooth. “Here’s the thing: the market had valued
these companies at less than their cash on hand. They have twenty
billion in the bank and a 16 billion dollar market-cap. We just
made four billion dollars, just by buying up the stock and taking
control of the company. We could shut the doors, stick the money in
our pockets, and retire.”

Suzanne took notes. She knew all this, but Kettlewell gave good
sound-bite, and talked slow in deference to the kind of reporter
who preferred a notebook to a recorder. “But we’re not gonna do
that.” He hunkered down on his haunches at the edge of the stage,
letting his tie dangle, staring spacily at the journalists and
analysts. “Kodacell is bigger than that.” He’d read his email that
morning then, and seen Rat-Toothed’s new moniker. “Kodacell has
goodwill. It has infrastructure. Administrators. Physical plant.
Supplier relationships. Distribution and logistics. These companies
have a lot of useful plumbing and a lot of priceless reputation.

“What we don’t have is a product. There aren’t enough buyers for
batteries or film\dash{}or any of the other stuff we make\dash{}to occupy or
support all that infrastructure. These companies slept through the
dot-boom and the dot-bust, trundling along as though none of it
mattered. There are parts of these businesses that haven’t changed
since the fifties.

“We’re not the only ones. Technology has challenged and killed
businesses from every sector. Hell, IBM
\emph{doesn’t make computers anymore}! The very idea of a travel
agent is inconceivably weird today! And the record labels, oy, the
poor, crazy, suicidal, stupid record labels. Don’t get me started.

“Capitalism is eating itself. The market works, and when it works,
it commodifies or obsoletes everything. That’s not to say that
there’s no money out there to be had, but the money won’t come from
a single, monolithic product line. The days of companies with names
like ’General Electric’ and ’General Mills’ and ’General Motors’
are over. The money on the table is like krill: a billion little
entrepreneurial opportunities that can be discovered and exploited
by smart, creative people.

“We will brute-force the problem-space of capitalism in the twenty
first century. Our business plan is simple: we will hire the
smartest people we can find and put them in small teams. They will
go into the field with funding and communications
infrastructure\dash{}all that stuff we have left over from the era of
batteries and film\dash{}behind them, capitalized to find a place to live
and work, and a job to do. A business to start. Our company isn’t a
project that we pull together on, it’s a \emph{network} of
like-minded, cooperating autonomous teams, all of which are
empowered to do whatever they want, provided that it returns
something to our coffers. We will explore and exhaust the realm of
commercial opportunities, and seek constantly to refine our tactics
to mine those opportunities, and fill our hungry belly. This
company isn’t a company anymore: this company is a network, an
approach, a sensibility.”

Suzanne’s fingers clattered over her keyboard. The Brit chuckled
nastily. “Nice talk, considering he just made a hundred thousand
people redundant,” he said. Suzanne tried to shut him out: yes,
Kettlewell was firing a company’s worth of people, but he was also
saving the company itself. The prospectus had a decent severance
for all those departing workers, and the ones who’d taken advantage
of the company stock-buying plan would find their pensions
augmented by whatever this new scheme could rake in. If it worked.

“Mr Kettlewell?” Rat-Toothed had clambered to his hind legs.

“Yes, Freddy?” Freddy was Rat-Toothed’s given name, though Suzanne
was hard pressed to ever retain it for more than a few minutes at a
time. Kettlewell knew every business-journalist in the Valley by
name, though. It was a CEO thing.

“Where will you recruit this new workforce from? And what kind of
entrepreneurial things will they be doing to ’exhaust the realm of
commercial activities’?”

“Freddy, we don’t have to recruit anyone. They’re beating a path to
our door. \emph{This} is a nation of manic entrepreneurs, the kind
of people who’ve been inventing businesses from video arcades to
photomats for centuries.” Freddy scowled skeptically, his jumble of
grey tombstone teeth protruding. “Come on, Freddy, you ever hear of
the Grameen Bank?”

Freddy nodded slowly. “In India, right?”

“Bangladesh. Bankers travel from village to village on foot and by
bus, finding small co-ops who need tiny amounts of credit to buy a
cellphone or a goat or a loom in order to grow. The bankers make
the loans and advise the entrepreneurs, and the payback rate is
fifty times higher than the rate at a regular lending institution.
They don’t even have a written lending agreement:
entrepreneurs\dash{}real, hard-working entrepreneurs\dash{}you can trust on a
handshake.”

“You’re going to help Americans who lost their jobs in your
factories buy goats and cellphones?”

“We’re going to give them loans and coordination to start
businesses that use information, materials science, commodified
software and hardware designs, and creativity to wring a profit
from the air around us. Here, catch!” He dug into his suit-jacket
and flung a small object toward Freddy, who fumbled it. It fell
onto Suzanne’s keyboard.

She picked it up. It looked like a keychain laser-pointer, or maybe
a novelty light-saber.

“Switch it on, Suzanne, please, and shine it, oh, on that wall
there.” Kettlewell pointed at the upholstered retractable wall that
divided the hotel ballroom into two functional spaces.

Suzanne twisted the end and pointed it. A crisp rectangle of green
laser-light lit up the wall.

“Now, watch this,” Kettlewell said.

\begin{projection}
NOW WATCH THIS
\end{projection}

The words materialized in the middle of the rectangle on the
distant wall.

“Testing one two three,” Kettlewell said.

\begin{projection}
TESTING ONE TWO THREE
\end{projection}

“Donde esta el bano?”

\begin{projection}
WHERE IS THE BATHROOM
\end{projection}

“What is it?” said Suzanne. Her hand wobbled a little and the
distant letters danced.

\begin{projection}
WHAT IS IT
\end{projection}

“This is a new artifact designed and executed by five previously
out-of-work engineers in Athens, Georgia. They’ve mated a tiny
Linux box with some speaker-independent continuous speech
recognition software, a free software translation engine that can
translate between any of twelve languages, and an extremely
high-resolution LCD that blocks out words in the path of the
laser-pointer.

“Turn this on, point it at a wall, and start talking. Everything
said shows up on the wall, in the language of your choosing,
regardless of what language the speaker was speaking.”

All the while, Kettlewell’s words were scrolling by in black block
caps on that distant wall: crisp, laser-edged letters.

“This thing wasn’t invented. All the parts necessary to make this
go were just lying around. It was \emph{assembled}. A gal in a
garage, her brother the marketing guy, her husband overseeing
manufacturing in Belgrade. They needed a couple grand to get it all
going, and they’ll need some life-support while they find their
natural market.

“They got twenty grand from Kodacell this week. Half of it a loan,
half of it equity. And we put them on the payroll, with benefits.
They’re part freelancer, part employee, in a team with backing and
advice from across the whole business.

“It was easy to do once. We’re going to do it ten thousand times
this year. We’re sending out talent scouts, like the artists and
representation people the record labels used to use, and they’re
going to sign up a lot of these bands for us, and help them to cut
records, to start businesses that push out to the edges of
business.

“So, Freddy, to answer your question, no, we’re not giving them
loans to buy cellphones and goats.”

Kettlewell beamed. Suzanne twisted the laser-pointer off and made
ready to toss it back to the stage, but Kettlewell waved her off.

“Keep it,” he said. It was suddenly odd to hear him speak without
the text crawl on that distant wall. She put the laser pointer in
her pocket and reflected that it had the authentic feel of cool,
disposable technology: the kind of thing on its way from a
startup’s distant supplier to the schwag bags at high-end
technology conferences to blister-packs of six hanging in the
impulse aisle at Fry’s.

She tried to imagine the technology conferences she’d been to with
the addition of the subtitling and translation and couldn’t do it.
Not conferences. Something else. A kids’ toy? A tool for
Starbucks-smashing anti-globalists, planning strategy before a WTO
riot? She patted her pocket.

Freddy hissed and bubbled like a teakettle beside her, fuming.
“What a cock,” he muttered. “Thinks he’s going to hire ten thousand
teams to replace his workforce, doesn’t say a word about what
\emph{that} lot is meant to be doing now he’s shitcanned them all.
Utter bullshit. Irrational exuberance gone berserk.”

Suzanne had a perverse impulse to turn the wand back on and splash
Freddy’s bilious words across the ceiling, and the thought made her
giggle. She suppressed it and kept on piling up notes, thinking
about the structure of the story she’d file that day.

Kettlewell pulled out some charts and another surfer in a suit came
forward to talk money, walking them through the financials. She’d
read them already and decided that they were a pretty credible bit
of fiction, so she let her mind wander.

She was a hundred miles away when the ballroom doors burst open and
the unionized laborers of the former Kodak and the former Duracell
poured in on them, tossing literature into the air so that it
snowed angry leaflets. They had a big drum and a bugle, and they
shook tambourines. The hotel rent-a-cops occasionally darted
forward and grabbed a protestor by the arm, but her colleagues
would immediately swarm them and pry her loose and drag her back
into the body of the demonstration. Freddy grinned and shouted
something at Kettlewell, but it was lost in the din. The
journalists took a lot of pictures.

Suzanne closed her computer’s lid and snatched a leaflet out of the
air. \textuppercase{WHAT ABOUT US?} it began, and talked about the workers who’d
been at Kodak and Duracell for twenty, thirty, even forty years,
who had been conspicuously absent from Kettlewell’s stated plans to
date.

She twisted the laser-pointer to life and pointed it back at the
wall. Leaning in very close, she said, “What are your plans for
your existing workforce, Mr Kettlewell?”

\begin{projection}
WHAT ARE YOUR PLANS FOR YOUR EXISTING WORKFORCE MR KETTLEWELL
\end{projection}

She repeated the question several times, refreshing the text so
that it scrolled like a stock ticker across that upholstered wall,
an illuminated focus that gradually drew all the attention in the
room. The protestors saw it and began to laugh, then they read it
aloud in ragged unison, until it became a chant: \textuppercase{WHAT ARE YOUR
PLANS}\dash{}\emph{thump} of the big drum\dash{}\textuppercase{FOR YOUR EXISTING WORKFORCE}
\emph{thump} MR \emph{thump} KETTLEWELL?

Suzanne felt her cheeks warm. Kettlewell was looking at her with
something like a smile. She liked him, but that was a personal
thing and this was a truth thing. She was a little embarrassed that
she had let him finish his spiel without calling him on that
obvious question. She felt tricked, somehow. Well, she was making
up for it now.

On the stage, the surfer-boys in suits were confabbing, holding
their thumbs over their tie-mics. Finally, Kettlewell stepped up
and held up his own laser-pointer, painting another rectangle of
light beside Suzanne’s.

“I’m glad you asked that, Suzanne,” he said, his voice barely
audible.

\begin{projection}
I’M GLAD YOU ASKED THAT SUZANNE
\end{projection}

The journalists chuckled. Even the chanters laughed a little. They
quieted down.

“I’ll tell you, there’s a downside to living in this age of
wonders: we are moving too fast and outstripping the ability of our
institutions to keep pace with the changes in the world.”

Freddy leaned over her shoulder, blowing shit-breath in her ear.
“Translation: you’re ass-fucked, the lot of you.”

\begin{projection}
TRANSLATION YOUR ASS FUCKED THE LOT OF YOU
\end{projection}

Suzanne yelped as the words appeared on the wall and reflexively
swung the pointer around, painting them on the ceiling, the
opposite wall, and then, finally, in miniature, on her computer’s
lid. She twisted the pointer off.

Freddy had the decency to look slightly embarrassed and he slunk
away to the very end of the row of seats, scooting from chair to
chair on his narrow butt. On stage, Kettlewell was pretending very
hard that he hadn’t seen the profanity, and that he couldn’t hear
the jeering from the protestors now, even though it had grown so
loud that he could no longer be heard over it. He kept on talking,
and the words scrolled over the far wall.

\begin{projection}
THERE IS NO WORLD IN WHICH KODAK AND DURACELL GO ON MAKING FILM AND
BATTERIES

THE COMPANIES HAVE MONEY IN THE BANK BUT IT HEMORRHAGES OUT THE
DOOR EVERY DAY

WE ARE MAKING THINGS THAT NO ONE WANTS TO BUY

THIS PLAN INCLUDES A GENEROUS SEVERANCE FOR THOSE STAFFERS WORKING
IN THE PARTS OF THE BUSINESS THAT WILL CLOSE DOWN
\end{projection}

\dash{}Suzanne admired the twisted, long-way-around way of saying, “the
people we’re firing.” Pure CEO passive voice. She couldn’t type
notes and read off the wall at the same time. She whipped out her
little snapshot and monkeyed with it until it was in video mode and
then started shooting the ticker.

\begin{projection}
BUT IF WE ARE TO MAKE GOOD ON THAT SEVERANCE WE NEED TO BE IN
BUSINESS

WE NEED TO BE BRINGING IN A PROFIT SO THAT WE CAN MEET OUR
OBLIGATIONS TO ALL OUR STAKEHOLDERS SHAREHOLDERS AND WORKFORCE
ALIKE

WE CAN’T PAY A PENNY IN SEVERANCE IF WE’RE BANKRUPT

WE ARE HIRING 50000 NEW EMPLOYEES THIS YEAR AND THERE’S NOTHING
THAT SAYS THAT THOSE NEW PEOPLE CAN’T COME FROM WITHIN

CURRENT EMPLOYEES WILL BE GIVEN CONSIDERATION BY OUR SCOUTS

ENTREPRENEURSHIP IS A DEEPLY AMERICAN PRACTICE AND OUR WORKERS ARE
AS CAPABLE OF ENTREPRENEURIAL ACTION AS ANYONE

I AM CONFIDENT WE WILL FIND MANY OF OUR NEW HIRES FROM WITHIN OUR
EXISTING WORKFORCE

I SAY THIS TO OUR EMPLOYEES IF YOU HAVE EVER DREAMED OF STRIKING
OUT ON YOUR OWN EXECUTING ON SOME AMAZING IDEA AND NEVER FOUND THE
MEANS TO DO IT NOW IS THE TIME AND WE ARE THE PEOPLE TO HELP
\end{projection}

Suzanne couldn’t help but admire the pluck it took to keep speaking
into the pointer, despite the howls and bangs.

“C’mon, I’m gonna grab some bagels before the protestors get to
them,” Freddy said, plucking at her arm\dash{}apparently, this was his
version of a charming pickup line. She shook him off
authoritatively, with a whip-crack of her elbow.

Freddy stood there for a minute and then moved off. She waited to
see if Kettlewell would say anything more, but he twisted the
pointer off, shrugged, and waved at the hooting protestors and the
analysts and the journalists and walked off-stage with the rest of
the surfers in suits.

She got some comments from a few of the protestors, some details.
Worked for Kodak or Duracell all their lives. Gave everything to
the company. Took voluntary pay-cuts under the old management five
times in ten years to keep the business afloat, now facing layoffs
as a big fat thank-you-suckers. So many kids. Such and such a
mortgage.

She knew these stories from Detroit: she’d filed enough copy with
varying renditions of it to last a lifetime. Silicon Valley was
supposed to be different. Growth and entrepreneurship\dash{}a failed
company was just a stepping-stone to a successful one, can’t win
them all, dust yourself off and get back to the garage and start
inventing. There’s a whole world waiting out there!

Mother of three. Dad whose bright daughter’s university fund was
raided to make ends meet during the “temporary” austerity measures.
This one has a Down’s Syndrome kid and that one worked through
three back surgeries to help meet production deadlines.

Half an hour before she’d been full of that old Silicon Valley
optimism, the sense that there was a better world a-borning around
her. Now she was back in that old rustbelt funk, with the feeling
that she was witness not to a beginning, but to a perpetual ending,
a cycle of destruction that would tear down everything solid and
reliable in the world.

She packed up her laptop and stepped out into the parking lot.
Across the freeway, she could make out the bones of the Great
America fun-park roller-coasters whipping around and around in the
warm California sun.

These little tech-hamlets down the 101 were deceptively utopian.
All the homeless people were miles north on the streets of San
Francisco, where pedestrian marks for panhandling could be had,
where the crack was sold on corners instead of out of the trunks of
fresh-faced, friendly coke-dealers’ cars. Down here it was giant
malls, purpose-built dot-com buildings, and the occasional
fun-park. Palo Alto was a university-town theme-park, provided you
steered clear of the wrong side of the tracks, the East Palo Alto
slums that were practically shanties.

Christ, she was getting melancholy. She didn’t want to go into the
office\dash{}not today. Not when she was in this kind of mood. She would
go home and put her blazer back in the closet and change into yoga
togs and write her column and have some good coffee.

She nailed up the copy in an hour and emailed it to her editor and
poured herself a glass of Napa red (the local vintages in Michigan
likewise left something to be desired) and settled onto her porch,
overlooking the big reservoir off 280 near San Mateo.

The house had been worth a small fortune at the start of the
dot-boom, but now, in the resurgent property boom, it was worth a
large fortune and then some. She could conceivably sell this badly
built little shack with its leaky hot-tub for enough money to
retire on, if she wanted to live out the rest of her days in Sri
Lanka or Nebraska.

“You’ve got no business feeling poorly, young lady,” she said to
herself. “You are as well set-up as you could have dreamed, and you
are right in the thick of the weirdest and best time the world has
yet seen. And Landon Kettlewell knows your name.”

She finished the wine and opened her computer. It was dark enough
now with the sun set behind the hills that she could read the
screen. The Web was full of interesting things, her email full of
challenging notes from her readers, and her editor had already
signed off on her column.

She was getting ready to shut the lid and head for bed, so she
pulled her mail once more.

\begin{email}
From: kettlewell-l@skunkworks.kodacell.com\\
To: schurch@sjmercury.com\\
Subject: Embedded journalist?

Thanks for keeping me honest today, Suzanne. It’s the hardest
question we’re facing today: what happens when all the things
you’re good at are no good to anyone anymore? I hope we’re going to
answer that with the new model.

You do good work, madam. I’d be honored if you’d consider joining
one of our little teams for a couple months and chronicling what
they do. I feel like we’re making history here and we need someone
to chronicle it.

I don’t know if you can square this with the Merc, and I suppose
that we should be doing this through my PR people and your editor,
but there comes a time about this time every night when I’m just
too goddamned hyper to bother with all that stuff and I want to
just DO SOMETHING instead of ask someone else to start a process to
investigate the possibility of someday possibly maybe doing
something.

Will you do something with us, if we can make it work? 100 percent
access, no oversight? Say you will. Please.

Your pal,

Kettlebelly
\end{email}

She stared at her screen. It was like a work of art; just look at
that return address, “kettlewell-l@skunkworks.kodacell.com”\dash{}for
kodacell.com to be live and accepting mail, it had to have been
registered the day before. She had a vision of Kettlewell checking
his email at midnight before his big press-conference, catching
Freddy’s column, and registering kodacell.com on the spot, then
waking up some sysadmin to get a mail server answering at
skunkworks.kodacell.com. Last she’d heard, Lockheed-Martin was
threatening to sue anyone who used their trademarked term “Skunk
Works” to describe a generic R\&D department. That meant that
Kettlewell had moved so fast that he hadn’t even run this project
by legal. She was willing to bet that he’d already ordered new
business-cards with the address on them.

There was a guy she knew, an editor at a mag who’d assigned himself
a plum article that he’d run on his own cover. He’d gotten a
book-deal out of it. A half-million dollar book-deal. If Kettlewell
was right, then the exclusive book on the inside of the first year
at Kodacell could easily make that advance. And the props would be
mad, as the kids said.

Kettlebelly! It was such a stupid frat-boy nickname, but it made
her smile. He wasn’t taking himself seriously, or maybe he was, but
he wasn’t being a pompous ass about it. He was serious about
changing the world and frivolous about everything else. She’d have
a hard time being an objective reporter if she said yes to this.

She couldn’t possibly decide at this hour. She needed a night’s
sleep and she had to talk this over with the Merc. If she had a
boyfriend, she’d have to talk it over with him, but that wasn’t a
problem in her life these days.

She spread on some expensive duty-free French wrinkle-cream and
brushed her teeth and put on her nightie and double-checked the
door locks and did all the normal things she did of an evening.
Then she folded back her sheets, plumped her pillows and stared at
them.

She turned on her heel and stalked back to her computer and thumped
the spacebar until the thing woke from sleep.

\begin{email}
From: schurch@sjmercury.com\\
To: kettlewell-l@skunkworks.kodacell.com\\
Subject: Re: Embedded journalist?

Kettlebelly: that is one dumb nickname. I couldn’t possibly
associate myself with a grown man who calls himself Kettlebelly.

So stop calling yourself Kettlebelly, immediately. If you can do
that, we’ve got a deal.

Suzanne
\end{email}

There had come a day when her readers acquired email and the paper
ran her address with her byline, and her readers had begun to write
her and write her and write her. Some were amazing, informative,
thoughtful notes. Some were the vilest, most bilious trolling. In
order to deal with these notes, she had taught herself to pause,
breathe, and re-read any email message before clicking send.

The reflex kicked in now and she re-read her note to
Kettlebelly\dash{}Kettlewell!\dash{}and felt a crimp in her guts. Then she hit
send.

She needed to pee, and apparently had done for some time, without
realizing it. She was on the toilet when she heard the ping of new
incoming mail.

\begin{email}
From: kettlewell-l@skunkworks.kodacell.com\\
To: schurch@sjmercury.com\\
Subject: Re: Embedded journalist?

I will never call myself Kettlebelly again.

Your pal,

Kettledrum.
\end{email}

Oh-shit-oh-shit-oh-shit. She did a little two-step at her bed’s
edge. Tomorrow she’d go see her editor about this, but it just felt
\emph{right}, and exciting, like she was on the brink of an event
that would change her life forever.

It took her three hours of mindless Web-surfing, including a truly
dreary Hot-Or-Not clicktrance and an hour’s worth of fiddling with
tweets from the press-conference, before she was able to lull
herself to sleep. As she nodded off, she thought that Kettlewell’s
insomnia was as contagious as his excitement.

\begin{center}\rule{3in}{0.4pt}\end{center}

Hollywood, Florida’s biggest junkyard was situated in the rubble of
a half-built ghost-mall off Taft Street. Suzanne’s Miami airport
rental car came with a GPS, but the little box hadn’t ever heard of
the mall; it was off the map. So she took a moment in the
sweltering parking-lot of her coffin hotel to call her interview
subject again and get better coordinates.

“Yeah, it’s ’cause they never finished building the mall, so the
address hasn’t been included in the USGS maps. The open GPSes all
have these better maps made by geohackers, but the rental car
companies have got a real hard-on for official map-data. Morons.
Hang on, lemme get my GPS out and I’ll get you some decent
lat-long.”

His voice had a pleasant, youthful, midwestern sound, like a
Canadian newscaster: friendly and enthusiastic as a puppy. His name
was Perry Gibbons, and if Kettlewell was to be believed, he was the
most promising prospect identified by Kodacell’s talent-scouts.

The ghost-mall was just one of many along Taft Street, ranging in
size from little corner plazas to gigantic palaces with broken-in
atria and cracked parking lots. A lot of the malls in California
had crashed, but they’d been turned into flea-markets or day-cares,
or, if they’d been abandoned, they hadn’t been abandoned like this,
left to go to ruin. This reminded her of Detroit before she’d left,
whole swaths of the inner city emptied of people, neighborhoods
condemned and bulldozed and, in a couple of weird cases, actually
\emph{farmed} by enterprising city-dwellers who planted crops, kept
livestock, and rode their mini tractors beneath the beam of the
defunct white-elephant monorail.

The other commonality this stretch of road shared with Detroit was
the obesity of the people she passed. She’d felt a little
self-conscious that morning, dressing in a light short-sleeved
blouse and a pair of shorts\dash{}nothing else would do, the weather was
so hot and drippy that even closed-toe shoes would have been
intolerable. At 45, her legs had slight cellulite saddlebags and
her tummy wasn’t the washboard it had been when she was 25. But
here, on this stretch of road populated by people so fat they could
barely walk, so fat that they were de-sexed marshmallows with faces
like inflatable toys, she felt like a toothpick.

The GPS queeped when she came up on the junkyard, a sprawling,
half-built discount mall whose waist-high walls had been used to
parcel out different kinds of sorted waste. The mall had been
planned with wide indoor boulevards between the shops wide enough
for two lanes of traffic, and she cruised those lanes now in the
hertzmobile, looking for a human. Once she reached the center of
the mall\dash{}a dry fountain filled with dusty Christmas-tree
ornaments\dash{}she stopped and leaned on the horn.

She got out of the car and called, “Hello? Perry?” She could have
phoned him but it always seemed so wasteful spending money on
airtime when you were trying to talk to someone within shouting
range.

“Suzanne!” The voice came from her left. She shielded her eyes from
the sun’s glare and peered down a spoke of mall-lane and caught her
first glimpse of Perry Gibbons. He was standing in the basket of a
tall cherry-picker, barechested and brown. He wore a sun-visor and
big work gloves, and big, baggy shorts whose pockets jangled as he
shinnied down the crane’s neck.

She started toward him tentatively. Not a lot of business-reporting
assignments involved spending time with half-naked, sun-baked dudes
in remote southern junkyards. Still, he sounded nice.

“Hello!” she called. He was young, 22 or 23, and already had
squint-creases at the corners of his eyes. He had a brace on one
wrist and his steel-toed boots were the mottled grey of a
grease-puddle on the floor of a muffler and brake shop.

He grinned and tugged off a glove, stuck out his hand. “A pleasure.
Sorry for the trouble finding this place. It’s not easy to get to,
but it’s cheap as hell.”

“I believe it.” She looked around again\dash{}the heaps of interesting
trash, the fountain-dish filled with thousands of shining
ornaments. The smell was a mixture of machine-oil and salt, jungle
air, Florida swamp and Detroit steel. “So, this place is pretty
cool. Looks like you’ve got pretty much everything you could
imagine.”

“And then some.” This was spoken by another man, one who puffed
heavily up from behind her. He was enormous, not just tall but fat,
%TODO: Decide on markup for uppercase
as big around as a barrel. His green tee-shirt read IT’S FUN TO USE
LEARNING FOR EVIL! in blocky, pixelated letters. He took her hand
and shook it. “I love your blog,” he said. “I read it all the
time.” He had three chins, and eyes that were nearly lost in his
apple cheeks.

“Meet Lester,” Perry said. “My partner.”

“Sidekick,” Lester said with a huge wink. “Sysadmin slash hardware
hacker slash dogsbody slashdot org.”

She chuckled. Nerd humor. Ar ar ar.

“Right, let’s get started. You wanna see what I do, right?” Perry
said.

“That’s right,” Suzanne said.

“Lead the way, Lester,” Perry said, and gestured with an arm, deep
into the center of the junkpile. “All right, check this stuff out
as we go.” He stuck his hand through the unglazed window of a
never-built shop and plucked out a toy in a battered box. “I love
these things,” he said, handing it to her.

%TODO: Markup
She took it. It was a Sesame Street Elmo doll, labeled BOOGIE
WOOGIE ELMO.

“That’s from the great Elmo Crash,” Perry said, taking back the box
and expertly extracting the Elmo like he was shelling a nut. “The
last and greatest generation of Elmoid technology, cast into an
uncaring world that bought millions of Li’l Tagger washable
graffiti kits instead after Rosie gave them two thumbs up on her
Christmas shopping guide.

“Poor Elmo was an orphan, and every junkyard in the world has
mountains of mint-in-package BWEs, getting rained on, waiting to
start their long, half-million-year decomposition.

“But check this out.” He flicked a multitool off his belt and
extracted a short, sharp scalpel-blade. He slit the grinning,
disco-suited Elmo open from chin to groin and shucked its furry
exterior and the foam tissue that overlaid its skeleton. He slid
the blade under the plastic cover on its ass and revealed a little
printed circuit board.

“That’s an entire Atom processor on a chip, there,” he said. “Each
limb and the head have their own subcontrollers. There’s a
high-powered digital-to-analog rig for letting him sing and dance
to new songs, and an analog-to-digital converter array for
converting spoken and danced commands to motions. Basically, you
dance and sing for Elmo and he’ll dance and sing back for you.”

Suzanne nodded. She’d missed that toy, which was a pity. She had a
five year old goddaughter in Minneapolis who would have loved a
Boogie Woogie Elmo.

They had come to a giant barn, set at the edge of a
story-and-a-half’s worth of anchor store. “This used to be where
the contractors kept their heavy equipment,” Lester rumbled, aiming
a car-door remote at the door, which queeped and opened.

Inside, it was cool and bright, the chugging air-conditioners
efficiently blasting purified air over the many work-surfaces. The
barn was a good 25 feet tall, with a loft and a catwalk circling it
halfway up. It was lined with metallic shelves stacked neatly with
labeled boxes of parts scrounged from the junkyard.

Perry set Elmo down on a workbench and worked a miniature USB cable
into his chest-cavity. The other end terminated with a PDA with a
small rubberized photovoltaic cell on the front.

“This thing is running InstallParty\dash{}it can recognize any hardware
and build and install a Linux distro on it without human
intervention. They used a ton of different suppliers for the BWE,
so every one is a little different, depending on who was offering
the cheapest parts the day it was built. InstallParty doesn’t care,
though: one-click and away it goes.” The PDA was doing all kinds of
funny dances on its screen, montages of playful photoshopping of
public figures matted into historical fine art.

“All done. Now, have a look\dash{}this is a Linux computer with some of
the most advanced robotics ever engineered. No sweatshop stuff,
either, see this? The solder is too precise to be done by
hand\dash{}that’s because it’s from India. If it was from Cambodia, you’d
see all kinds of wobble in the solder: that means that tiny, clever
hands were used to create it, which means that somewhere in the
device’s karmic history, there’s a sweatshop full of crippled
children inhaling solder fumes until they keel over and are dumped
in a ditch. This is the good stuff.

“So we have this karmically clean robot with infinitely malleable
computation and a bunch of robotic capabilities. I’ve turned these
things into wall-climbing monkeys; I’ve modded them for a woman
from the University of Miami at the Jackson Memorial who used their
capability to ape human motions in physiotherapy programs with
nerve-damage cases. But the best thing I’ve done with them so far
is the Distributed Boogie Woogie Elmo Motor Vehicle Operation
Cluster. Come on,” he said, and took off deeper into the barn’s
depths.

They came to a dusty, stripped-down Smart car, one of those tiny
two-seat electric cars you could literally buy out of a vending
machine in Europe. It was barely recognizable, having been reduced
to its roll-cage, drive-train and control-panel. A gang of naked
robot Elmos were piled into it.

“Wake up boys, time for a demo!” Perry shouted, and they sat up and
made canned, tinny Elmo “oh boy” noises, climbing into position on
the pedals, around the wheel, and on the gear-tree.

“I got the idea when I was teaching an Elmo to play Mario Brothers.
I thought it’d get a decent diggdotting. I could get it to speedrun
all of the first level using an old paddle I’d found and
rehabilitated, and I was trying to figure out what to do next. The
dead mall across the way is a drive-in theater, and I was out front
watching the silent movies, and one of them showed all these cute
little furry animated whatevers collectively driving a car. It’s a
really old sight-gag, I mean, like racial memory old. I’d seen the
Little Rascals do the same bit, with Alfalfa on the wheel and
Buckwheat and Spanky on the brake and clutch and the doggy working
the gearshift.

“And I thought, Shit, I could do that with Elmos. They don’t have
any networking capability, but they can talk and they can parse
spoken commands, so all I need is to designate one for left and one
for right and one for fast and one for slow and one to be the eyes,
barking orders and they should be able to do this. And it works!
They even adjust their balance and centers of gravity when the car
swerves to stay upright at their posts. Check it out.” He turned to
the car. “Driving Elmos, ten-HUT!” They snapped upright and ticked
salutes off their naked plastic noggins. “In circles, DRIVE,” he
called. The Elmos scrambled into position and fired up the car and
in short order they were doing donuts in the car’s little indoor
pasture.

“Elmos, HALT” Perry shouted and the car stopped silently, rocking
gently. “Stand DOWN.” The Elmos sat down with a series of tiny
thumps.

Suzanne found herself applauding. “That was amazing,” she said.
“Really impressive. So that’s what you’re going to do for Kodacell,
make these things out of recycled toys?”

Lester chuckled. “Nope, not quite. That’s just for starters. The
Elmos are all about the universal availability of cycles and
apparatus. Everywhere you look, there’s devices for free that have
everything you need to make anything do anything.

“But have a look at part two, c’mere.” He lumbered off in another
direction, and Suzanne and Perry trailed along behind him.

“This is Lester’s workshop,” Perry said, as they passed through a
set of swinging double doors and into a cluttered wonderland. Where
Perry’s domain had been clean and neatly organized, Lester’s area
was a happy shambles. His shelves weren’t orderly, but rather,
crammed with looming piles of amazing junk: thrift-store wedding
dresses, plaster statues of bowling monkeys, box kites, knee-high
tin knights-in-armor, seashells painted with American flags,
presidential action-figures, paste jewelry and antique cough-drop
tins.

“You know how they say a sculptor starts with a block of marble and
chips away everything that doesn’t look like a statue? Like he can
\emph{see} the statue in the block? I get like that with garbage: I
see the pieces on the heaps and in roadside trash and I can just
\emph{see} how it can go together, like this.”

He reached down below a work-table and hoisted up a huge triptych
made out of three hinged car-doors stood on end. Carefully, he
unfolded it and stood it like a screen on the cracked concrete
floor.

The inside of the car-doors had been stripped clean and polished to
a high metal gleam that glowed like sterling silver. Spot-welded to
it were all manner of soda tins, pounded flat and cut into gears,
chutes, springs and other mechanical apparatus.

“It’s a mechanical calculator,” he said proudly. “About half as
powerful as Univac. I milled all the parts using a laser-cutter.
What you do is, fill this hopper with GI Joe heads, and this hopper
with Barbie heads. Crank this wheel and it will drop a number of
M\&Ms equal to the product of the two values into this hopper,
here.” He put three scuffed GI Joe heads in one hopper and four
scrofulous Barbies in another and began to crank, slowly. A
music-box beside the crank played a slow, irregular rendition of
“Pop Goes the Weasel” while the hundreds of little coin-sized gears
turned, flipping switches and adding and removing tension to
springs. After the weasel popped a few times, twelve brown M\&Ms
fell into an outstretched rubber hand. He picked them out carefully
and offered them to her. “It’s OK. They’re not from the trash,” he
said. “I buy them in bulk.” He turned his broad back to her and
heaved a huge galvanized tin washtub full of brown M\&Ms in her
direction. “See, it’s a bit-bucket!” he said.

Suzanne giggled in spite of herself. “You guys are hilarious,” she
said. “This is really good, exciting nerdy stuff.” The gears on the
mechanical computer were really sharp and precise; they looked like
you could cut yourself on them. When they ground over the polished
surfaces of the car-doors, they made a sound like a box of
toothpicks falling to the floor: click-click, clickclickclick,
click. She turned the crank until twelve more brown M\&Ms fell
out.

“Who’s the Van Halen fan?”

Lester beamed. “Might as well jump\dash{}JUMP!” He mimed heavy-metal
air-guitar and thrashed his shorn head up and down as though he
were headbanging with a mighty mane of hair-band locks. “You’re the
first one to get the joke!” he said. “Even Perry didn’t get it!”

“Get what?” Perry said, also grinning.

“Van Halen had this thing where if there were any brown M\&Ms in
their dressing room they’d trash it and refuse to play. When I was
a kid, I used to \emph{dream} about being so famous that I could
act like that much of a prick. Ever since, I’ve afforded a great
personal significance to brown M\&Ms.”

She laughed again. Then she frowned a little. “Look, I hate to
break this party up, but I came here because Kettlebelly\dash{}crap,
Kettle\emph{well}\dash{}said that you guys exemplified everything that he
wanted to do with Kodacell. This stuff you’ve done is all very
interesting, it’s killer art, but I don’t see the business-angle.
So, can you help me out here?”

“That’s step three,” Perry said. “C’mere.” He led her back to his
workspace, to a platform surrounded by articulated arms terminated
in webcams, like a grocery scale in the embrace of a metal spider.
“three-d scanner,” he said, producing a Barbie head from Lester’s
machine and dropping it on the scales. He prodded a button and a
nearby screen filled with a three-dimensional model of the head,
flattened on the side where it touched the surface. He turned the
head over and scanned again and now there were two digital versions
of the head on the screen. He moused one over the other until they
lined up, right-clicked a drop-down menu, selected an option and
then they were merged, rotating.

“Once we’ve got the three-d scan, it’s basically Plasticine.” He
distorted the Barbie head, stretching it and squeezing it with the
mouse. “So we can take a real object and make this kind of protean
hyper-object out of it, or drop it down to a wireframe and skin it
with any bitmap, like this.” More fast mousing\dash{}Barbie’s head turned
into a gridded mesh, fine filaments stretching off along each
mussed strand of plastic hair. Then a Campbell’s Cream of Mushroom
Soup label wrapped around her like a stocking being pulled over her
head. There was something stupendously weird and simultaneously
very comic about the sight, the kind of inherent comedy in a
cartoon stretched out on a blob of Silly Putty.

“So we can build anything out of interesting junk, with any shape,
and then we can digitize the shape. Then we can do anything we like
with the shape. Then we can output the shape.” He typed quickly and
another machine, sealed and mammoth like an outsized photocopier,
started to grunt and churn. The air filled with a smell like Saran
Wrap in a microwave.

“The goop we use in this thing is epoxy-based. You wouldn’t want to
build a car out of it, but it makes a mean doll-house. The last
stage of the output switches to inks, so you get whatever bitmap
you’ve skinned your object with baked right in. It does about one
cubic inch per minute, so this job should be almost done now.”

He drummed his fingers on top of the machine for a moment and then
it stopped chunking and something inside it went \emph{clunk}. He
lifted a lid and reached inside and plucked out the barbie head,
stretched and distorted, skinned with a Campbell’s Soup label. He
handed it to Suzanne. She expected it to be warm, like a squashed
penny from a machine on Fisherman’s Wharf, but it was cool and had
the seamless texture of a plastic margarine tub and the heft of a
paperweight.

“So, that’s the business,” Lester said. “Or so we’re told. We’ve
been making cool stuff and selling it to collectors on the web for
you know, gigantic bucks. We move one or two pieces a month at
about ten grand per. But Kettlebelly says he’s going to
industrialize us, alienate us from the product of our labor, and
turn us into an assembly line.”

“He didn’t say any such thing,” Perry said. Suzanne was aware that
her ears had grown points. Perry gave Lester an affectionate slug
in the shoulder. “Lester’s only kidding. What we need is a couple
of dogsbodies and some bigger printers and we’ll be able to turn
out more modest devices by the hundred or possibly the thousand. We
can tweak the designs really easily because nothing is coming off a
mold, so there’s no setup charge, so we can do limited runs of a
hundred, redesign, do another hundred. We can make ’em to order.”

“And we need an MBA,” Lester said. “Kodacell’s sending us a
business manager to help us turn junk into pesos.”

“Yeah,” Perry said, with a worried flick of his eyes. “Yeah, a
business manager.”

“So, I’ve known some business geeks who aren’t total assholes,”
Lester said. “Who care about what they’re doing and the people
they’re doing it with. Respectful and mindful. It’s like
lawyers\dash{}they’re not all scumbags. Some of them are totally awesome
and save your ass.”

Suzanne took all this in, jotting notes on an old-fashioned
spiral-bound shirt-pocket notebook. “When’s he arriving?”

“Next week,” Lester said. “We’ve cleared him a space to work and
everything. He’s someone that Kettlewell’s people recruited up in
Ithaca and he’s going to move here to work with us, sight unseen.
Crazy, huh?”

“Crazy,” Suzanne agreed.

“Right,” Perry said. “That’s next week, and this aft we’ve got some
work to do, but now I’m ready for lunch. You guys ready for
lunch?”

Something about food and really fat guys, it seemed like an awkward
question to Suzanne, like asking someone who’d been horribly
disfigured by burns if he wanted to toast a marshmallow. But Lester
didn’t react to the question\dash{}of course not, he had to eat, everyone
had to eat.

“Yeah, let’s do the IHOP.” Lester trundled back to his half of the
workspace, then came back with a cane in one hand. “There’s like
three places to eat within walking distance of here if you don’t
count the mobile Mexican burrito wagon, which I don’t, since it’s a
rolling advertisement for dysentery. The IHOP is the least
objectionable of those.”

“We could drive somewhere,” Suzanne said. It was coming up on noon
and the heat once they got outside into the mall’s ruins was like
the steam off a dishwasher. She plucked at her blouse a couple of
times.

“It’s the only chance to exercise we get,” Perry said. “It’s pretty
much impossible to live or work within walking distance of anything
down here. You end up living in your car.”

And so they hiked along the side of the road. The sidewalk was a
curious mix of old and new, the concrete unworn but still overgrown
by tall sawgrass thriving in the Florida heat. It brushed up
against her ankles, hard and sharp, unlike the grass back home.

They were walking parallel to a ditch filled with sluggish,
brackish water and populated by singing frogs, ducks, ibises, and
mosquitoes in great number. Across the way were empty lots,
ghost-plazas, dead filling stations. Behind one of the filling
stations, a cluster of tents and shacks.

“Squatters?” she asked, pointing to the shantytown.

“Yeah,” Perry said. “Lots of that down here. Some of them are the
paramilitary wing of the AARP, old trailer-home retirees who’ve run
out of money and just set up camp here. Some are bums and junkies,
some are runaways. It’s not as bad as it looks\dash{}they’re pretty comfy
in there. We bring ’em furniture and other good pickings that show
up at the junkyard. The homeless with the wherewithal to build
shantytowns, they haven’t gone all animal like the shopping cart
people and the scary beachcombers.” He waved across the malarial
ditch to an old man in a pair of pressed khaki shorts and a crisp
Bermuda shirt. “Hey Francis!” he called. The old man waved back.
“We’ll have some IHOP for you ’bout an hour!” The old man ticked a
salute off his creased forehead.

“Francis is a good guy. Used to be an aerospace engineer if you can
believe it. Wife had medical problems and he went bust taking care
of her. When she died, he ended up here in his double-wide and
never left. Kind of the unofficial mayor of this little patch.”

Suzanne stared after Francis. He had a bit of a gimpy leg, a limp
she could spot even from here. Beside her, Lester was puffing. No
one was comfortable walking in Florida, it seemed.

It took another half hour to reach the IHOP, the International
House of Pancakes, which sat opposite a mini-mall with only one
still-breathing store, a place that advertised 99-cent t-shirts,
which struck Suzanne as profoundly depressing. There was a junkie
out front of 99-Cent Tees, a woman with a leathery tan and a tiny
tank-top and shorts that made her look a little like a Tenderloin
hooker, but not with that rat’s-nest hair, not even in the ’Loin.
She wobbled uncertainly across the parking lot to them.

“Excuse me,” she said, with an improbable Valley Girl accent.
“Excuse me? I’m hoping to get something to eat, it’s for my kid,
she’s nursing, gotta keep my strength up.” Her naked arms and legs
were badly tracked out, and Suzanne had a horrified realization
that among the stains on her tank-top were a pair of spreading
pools of breast milk, dampening old white, crusted patches over her
sagging breasts. “For my baby. A dollar would help, a dollar.”

There were homeless like this in San Francisco, too. In San Jose as
well, she supposed, but she didn’t know where they hid. But
something about this woman, cracked out and tracked out, it freaked
her out. She dug into her purse and got out a five dollar bill and
handed it to the homeless woman. The woman smiled a snaggletoothed
stumpy grin and reached for it, then, abruptly, grabbed hold of
Suzanne’s wrist. Her grip was damp and weak.

“Don’t you fucking look at me like that. You’re not better than me,
bitch!” Suzanne tugged free and stepped back quickly. “That’s
right, run away! Bitch! Fuck you! Enjoy your lunch!”

She was shaking. Perry and Lester closed ranks around her. Lester
moved to confront the homeless woman.

“The fuck you want lard ass? You wanna fuck with me? I got a knife,
you know, cut your ears off and feed ’em to ya.”

Lester cocked his head like the RCA Victor dog. He towered over the
skinny junkie, and was five or six times wider than her.

“You all right?” he said gently.

“Oh yeah, I’m just fine,” she said. “Why, you looking for a
party?”

He laughed. “You’re joking\dash{}I’d crush you!”

She laughed too, a less crazy, more relaxed sound. Lester’s voice
was a low, soothing rumble. “I don’t think my friend thinks she’s
any better than you. I think she just wanted to help you out.”

The junkie flicked her eyes back and forth. “Listen can you spare a
dollar for my baby?”

“I think she just wanted to help you. Can I get you some lunch?”

“Fuckers won’t let me in\dash{}won’t let me use the toilet even. It’s not
humane. Don’t want to go in the bushes. Not dignified to go in the
bushes.”

“That’s true,” he said. “What if I get you some take out, you got a
shady place you could eat it? Nursing’s hungry work.”

The junkie cocked her head. Then she laughed. “Yeah, OK, yeah.
Sure\dash{}thanks, thanks a lot!”

Lester motioned her over to the menu in the IHOP window and waited
with her while she picked out a helping of caramel-apple waffles,
sausage links, fried eggs, hash browns, coffee, orange juice and a
chocolate malted. “Is that all?” he said, laughing, laughing, both
of them laughing, all of them laughing at the incredible,
outrageous meal.

They went in and waited by the podium. The greeter, a black guy
with corn-rows, nodded at Lester and Perry like an old friend. “Hey
Tony,” Lester said. “Can you get us a go-bag with some take-out for
the lady outside before we sit down?” He recited the astounding
order.

Tony shook his head and ducked it. “OK, be right up,” he said. “You
want to sit while you’re waiting?”

“We’ll wait here, thanks,” Lester said. “Don’t want her to think
we’re bailing on her.” He turned and waved at her.

“She’s mean, you know\dash{}be careful.”

“Thanks, Tony,” Lester said.

Suzanne marveled at Lester’s equanimity. Nothing got his goat. The
doggie bag arrived. “I put some extra napkins and a couple of
wet-naps in there,” Tony said, handing it to him.

“Great!” Lester said. “You guys sit down, I’ll be back in a
second.”

Perry motioned for Suzanne to follow him to a booth. He laughed.
“Lester’s a good guy,” he said. “The best guy I know, you know?”

“How do you know him?” she asked, taking out her notepad.

“He was the sysadmin at a company that was making three-d printers,
and I was a tech at a company that was buying them, and the
products didn’t work, and I spent a lot of time on the phone with
him troubleshooting them. We’d get together in our off-hours and
hack around with neat little workbench projects, stuff we’d come up
with at work. When both companies went under, we got a bunch of
their equipment at bankruptcy auctions. Lester’s uncle owned the
junkyard and he offered us space to set up our workshops and the
rest is history.”

Lester joined them again. He was laughing. “She is \emph{funny},”
he said. “Kept hefting the sack and saying, ’Christ what those
bastards put on a plate, no wonder this country’s so goddamned
fat!’” Perry laughed, too. Suzanne chuckled nervously and looked
away.

He slid into the booth next to her and put a hand on her shoulder.
“It’s OK. I’m a guy who weighs nearly 400 pounds. I know I’m a big,
fat guy. If I was sensitive about it, I couldn’t last ten minutes.
I’m not proud of being as big as I am, but I’m not ashamed either.
I’m OK with it.”

“You wouldn’t lose weight if you could?”

“Sure, why not? But I’ve concluded it’s not an option anymore. I
was always a fat kid, and so I never got good at sports, never got
that habit. Now I’ve got this huge deficit when I sit down to
exercise, because I’m lugging around all this lard. Can’t run more
than a few steps. Walking’s about it. Couldn’t join a pick-up game
of baseball or get out on the tennis court. I never learned to
cook, either, though I suppose I could. But mostly I eat out, and I
try to order sensibly, but just look at the crap they feed us at
the places we can get to\dash{}there aren’t any health food restaurants
in the strip malls. Look at this menu,” he said, tapping a
pornographic glossy picture of a stack of glistening waffles oozing
with some kind of high-fructose lube. “Caramel pancakes with
whipped cream, maple syrup and canned strawberries. When I was a
kid, we called that \emph{candy}. These people will sell you an
eight dollar, 18 ounce plate of candy with a side of sausage, eggs,
biscuits, bacon and a pint of orange juice. Even if you order this
stuff and eat a third of it, a quarter of it, that’s probably too
much, and when you’ve got a lot of food in front of you, it’s
pretty hard to know when to stop.”

Suzanne couldn’t help it; she blurted out: “But willpower\dash{}“

“Sure, will-power. Will-power \emph{nothing}. The thing is, when
three quarters of America are obese, when half are dangerously
obese, like me, years off our lives from all the fat\dash{}that tells you
that this isn’t a will-power problem. We didn’t get less willful in
the last fifty years. Might as well say that all those people who
died of the plague lacked the will-power to keep their houses free
of rats. Fat isn’t moral, it’s \emph{epidemiological}. There are a
small number of people, a tiny minority, whose genes are
short-circuited in a way that makes them less prone to retaining
nutrients. That’s a maladaptive trait through most of human
history\dash{}burning unnecessary calories when you’ve got to chase down
an antelope to get more, that’s no way to live long enough to pass
on your genes! So you and Perry over here with your little skinny
selves, able to pack away transfats and high-fructose corn-syrup
and a pound of candy for breakfast at the IHOP, you’re not doing
this on will-power\dash{}you’re doing it by expressing the somatotype of
a recessive, counter-survival gene.

“Would I like to be thinner? Sure. But I’m not gonna let the fact
that I’m genetically better suited to famine than feast get to me.
Speaking of, let’s eat. Tony, c’mere, buddy. I want a plate of
candy!” He was smiling, and brave, and at that moment, Suzanne
thought that she could get a crush on this guy, this big, smart,
talented, funny, lovable guy. Then reality snapped back and she saw
him as he was, sexless, lumpy, almost grotesque. The overlay of
his, what, his \emph{inner beauty} on that exterior, it disoriented
her. She looked back over her notes.

“So, you say that there’s a third coming out to work with you?”

“To \emph{live} with us,” Perry said. “That’s part of the deal.
Geek houses, like in the old college days. We’re going to be a
power-trio: two geeks and a suit, lean and mean. The suit’s name is
Tjan, and he’s Singaporean by way of London by way of Ithaca, where
Kettlebelly found him. We’ve talked on the phone a couple times and
he’s moving down next week.”

“He’s moving down without ever having met you?”

“Yeah, that’s the way it goes. It’s like the army or something for
us: once you’re in you get dispatched here or there. It was in the
contract. We already had a place down here with room for Tjan, so
we put some fresh linen on the guest-bed and laid in an extra
toothbrush.”

“It’s a little nervous-making,” Lester said. “Perry and I get along
great, but I haven’t had such good luck with business-types. It’s
not that I’m some kind of idealist who doesn’t get the need to make
money, but they can be so condescending, you know?”

Suzanne nodded. “That’s a two-way street, you know. ’Suits’ don’t
like being talked down to by engineers.”

Lester raised a hand. “Guilty as charged.”

“So what’re you planning to do for the rest of the week?” It was
Wednesday, and she’d counted on getting this part of the story by
Saturday, but here she was going to have to wait, clearly, until
this Tjan arrived.

“Same stuff as we always do. We build crazy stuff out of junk, sell
it to collectors, and have fun. We could go to the Thunderbird
Drive In tonight if you want, it’s a real classic, flea-market by
day and drive in by night, practically the last one standing.”

Perry cut in. “Or we could go to South Beach and get a good meal,
if that’s more your speed.”

“Naw,” Suzanne said. “Drive in sounds great, especially if it’s
such a dying breed. Better get a visit in while there’s still
time.”

They tried to treat her but she wouldn’t let them. She never let
anyone buy her so much as a cup of coffee. It was an old
journalism-school drill, and she was practically the only scribbler
she knew who hewed to it: some of the whores on the Silicon Valley
papers took in free computers, trips, even spa days!\dash{}but she had
never wavered.

\begin{center}\rule{3in}{0.4pt}\end{center}

The afternoon passed quickly and enchantingly. Perry was working on
a knee-high, articulated Frankenstein monster built out of
hand-painted seashells from a beach-side kitsch market. They said
%TODO: Markup
GOD BLESS AMERICA and SOUVENIR OF FLORIDA and CONCH REPUBLIC and
each had to be fitted out for a motor custom built to conform to
its contours.

“When it’s done, it will make toast.”

“Make toast?”

“Yeah, separate a single slice off a loaf, load it into a
top-loading slice-toaster, depress the lever, time the toast-cycle,
retrieve the toast and butter it. I got the idea from old-time
backup-tape loaders. This plus a toaster will function as a loosely
coupled single system.”

“OK, that’s really cool, but I have to ask the boring question,
Perry. Why? Why build a toast-robot?”

Perry stopped working and dusted his hands off. He was really
built, and his shaggy hair made him look younger than his
crows-feet suggested. He turned a seashell with a half-built motor
%TODO: Markup
in it over and spun it like a top on the hand-painted WEATHER IS
HERE/WISH YOU WERE BEAUTIFUL legend.

“Well, that’s the question, isn’t it? The simple answer: people buy
them. Collectors. So it’s a good hobby business, but that’s not
really it.

“It’s like this: engineering is all about constraint. Given a span
of foo feet and materials of tensile strength of bar, build a
bridge that doesn’t go all fubared. Write a fun video-game for an
eight-bit console that’ll fit in 32K. Build the fastest airplane,
or the one with the largest carrying capacity\ldots{} But these days,
there’s not much traditional constraint. I’ve got the engineer’s
most dangerous luxury: plenty. All the computational cycles I’ll
ever need. Easy and rapid prototyping. Precision tools.

“Now, it may be that there is a suite of tasks lurking
\emph{in potentia} that demand all this resource and more\dash{}maybe I’m
like some locomotive engineer declaring that 60 miles per hour is
the pinnacle of machine velocity, that speed is cracked. But I
don’t see many of those problems\dash{}none that interest me.

“What I’ve got here are my own constraints. I’m challenging myself,
using found objects and making stuff that throws all this
computational capacity at, you know, these \emph{trivial} problems,
like car-driving Elmo clusters and seashell toaster-robots. We have
so much capacity that the trivia expands to fill it. And all that
capacity is junk-capacity, it’s leftovers. There’s enough
computational capacity in a junkyard to launch a space-program, and
that’s by design. Remember the iPod? Why do you think it was so
prone to scratching and going all gunky after a year in your
pocket? Why would Apple build a handheld technology out of
materials that turned to shit if you looked at them cross-eyed?
It’s because the iPod was only meant to last a year!

“It’s like tailfins\dash{}they were cool in the Tailfin Cretaceous, but
wouldn’t it have been better if they could have disappeared from
view when they became aesthetically obsolete, when the space age
withered up and blew away? Oh, not really, obviously, because it’s
nice to see a well-maintained land-yacht on the highway every now
and again, if only for variety’s sake, but if you’re going to
design something that is meant to be \emph{au fait} then presumably
you should have some planned obsolescence in there, some
end-of-lifing strategy for the aesthetic crash that follows any
couture movement. Here, check this out.”

He handed her a white brick, the size of a deck of cards. It took
her a moment to recognize it as an iPod. “Christ, it’s
\emph{huge},” she said.

“Yeah, isn’t it just. Remember how small and shiny this thing was
when it shipped? ’A thousand songs in your pocket!'”

That made her actually laugh out loud. She fished in her pocket for
her earbuds and dropped them on the table where they clattered like
M\&Ms. “I \emph{think} I’ve got about 40,000 songs on those.
Haven’t run out of space yet, either.”

He rolled the buds around in his palm like a pair of dice. “You
won’t\dash{}I stopped keeping track of mine after I added my
hundred-thousandth audiobook. I’ve got a bunch of the Library of
Congress in mine as high-rez scans, too. A copy of the Internet
Archive, every post ever made on Usenet\ldots{} Basically, these things
are infinitely capacious, given the size of the media we work with
today.” He rolled the buds out on the workbench and laughed. “And
that’s just the point! Tomorrow, we’ll have some new extra fat kind
of media and some new task to perform with it and some new storage
medium that will make these things look like an old iPod. Before
that happens, you want this to wear out and scuff up or get lost\dash{}”

“I lose those things all the time, like a set a month.”

“There you go then! The iPods were too big to lose like that, but
just \emph{look at them}.” The iPod’s chrome was scratched to the
point of being fogged, like the mirror in a gas-station toilet. The
screen was almost unreadable for all the scratches. “They had
scratch-proof materials and hard plastics back then. They
\emph{chose} to build these things out of Saran Wrap and tin-foil
so that by the time they doubled in capacity next year, you’d have
already worn yours out and wouldn’t feel bad about junking them.

“So I’m building a tape-loading seashell robot toaster out of
discarded obsolete technology because the world is full of
capacious, capable, disposable junk and it cries out to be used
again. It’s a potlatch: I have so much material and computational
wealth that I can afford to waste it on frivolous junk. I think
that’s why the collectors buy it, anyway.”

“That brings us back to the question of your relationship with
Kodacell. They want to do what, exactly, with you?”

“Well, we’ve been playing with some mass-production techniques, the
three-d printer and so on. When Kettlebelly called me, he said that
he wanted to see about using the scanner and so on to make a lot of
these things, at a low price-point. It’s pretty perverse when you
think about it: using modern technology to build replicas of
obsolete technology rescued from the dump, when these replicas are
bound to end up back here at the dump!” He laughed. He had nice
laugh-lines around his eyes. “Anyway, it’s something that Lester
and I had talked about for a long time, but never really got around
to. Too much like retail. It’s bad enough dealing with a couple
dozen collectors who’ll pay ten grand for a sculpture: who wants to
deal with ten thousand customers who’ll go a dollar each for the
same thing?”

“But you figure that this Tjan character will handle all the
customer stuff?”

“That’s the idea: he’ll run the business side, we’ll get more time
to hack; everyone gets paid. Kodacell’s got some micro-sized
marketing agencies, specialized PR firms, creative shippers, all
kinds of little three-person outfits that they’ve promised to hook
us up with. Tjan interfaces with them, we do our thing, enrich the
shareholders, get stock ourselves. It’s supposed to be all upside.
Hell, if it doesn’t work we can just walk away and find another
dump and go back into the collectors’ market.”

He picked up his half-finished shell and swung a lamp with a
magnifying lens built into it over his workspace. “Hey, just a sec,
OK? I’ve just figured out what I was doing wrong before.” He took
up a little tweezers and a plastic rod and probed for a moment,
then daubed some solder down inside the shell’s guts. He tweezed a
wire to a contact and the shell made a motorized sound, a peg
sticking out of it began to move rhythmically.

“Got it,” he said. He set it down. “I don’t expect I’m going to be
doing many more of these projects after next week. This kind of
design, we could never mass-produce it.” He looked a little
wistful, and Suzanne suppressed a smile. What a tortured artiste
this Florida junkyard engineer was!

As the long day drew to a close, they went out for a walk in the
twilight’s cool in the yard. The sopping humidity of the day
settled around them as the sun set in a long summer blaze that
turned the dry fountain full of Christmas ornaments into a
luminescent bowl of jewels.

“I got some real progress today,” Lester said. He had a cane with
him and he was limping heavily. “Got the printer to output complete
mechanical logical gates, all in one piece, Almost no assembly,
just daisy-chain them on a board. And I’ve been working on a
standard snap-on system for lego-bricking each gate to the next.
It’s going to make it a lot easier to ramp up production.”

“Yeah?” Perry said. He asked a technical question about the
printer, something about the goop’s tensile strength that Suzanne
couldn’t follow. They went at it, hammer and tongs, talking through
the abstruse details faster than she could follow, walking more and
more quickly past the vast heaps of dead technology and half-built
mall stores.

She let them get ahead of her and stopped to gather her thoughts.
She turned around to take it all in and that’s when she caught
sight of the kids sneaking into Perry and Lester’s lab.

“Hey!” she shouted, in her loudest Detroit voice. “What are you
doing there?” There were three of them, in Miami Dolphins jerseys
and shiny bald-shaved heads and little shorts, the latest
inexplicable rapper style which made them look more like drag
queens in mufti than tough-guys.

They rounded on her. They were heavyset and their eyebrows were
bleached blond. They had been sneaking into the lab’s side-door,
looking about as inconspicuous as a trio of nuns.

“Get lost!” she shouted. “Get out of here! Perry, Lester!”

They were coming closer now. They didn’t move so well, puffing in
the heat, but they clearly had mayhem on their minds. She reached
into her purse for her pepper spray and held it before her
dramatically, but they didn’t stop coming.

Suddenly, the air was rent by the loudest sound she’d ever heard,
like she’d put her head inside a foghorn. She flinched and misted a
cloud of aerosol capsicum ahead of her. She had the presence of
mind to step back quickly, before catching a blowback, but she
wasn’t quick enough, for her eyes and nose started to burn and
water. The sound wouldn’t stop, it just kept going on, a sound like
her head was too small to contain her brain, a sound that made her
teeth ache. The three kids had stopped and staggered off.

“You OK?” The voice sounded like it was coming from far, far away,
though Lester was right in front of her. She found that she’d
dropped to her knees in the teeth of that astonishing noise.

She let him help her to her feet. “Jesus,” she said, putting a hand
to her ears. They rang like she’d been at a rave all night. “What
the hell?”

“Anti-personnel sonic device,” Lester said. She realized that he
was shouting, but she could barely hear it. “It doesn’t do any
permanent damage, but it’ll scare off most anyone. Those kids
probably live in the shantytown we passed this morning. More and
more of them are joining gangs. They’re our neighbors, so we don’t
want to shoot them or anything.”

She nodded. The ringing in her ears was subsiding a little. Lester
steadied her. She leaned on him. He was big and solid. He wore the
same cologne as her father had, she realized.

She moved away from him and smoothed out her shorts, dusting off
her knees. “Did you invent that?”

“Made it using a HOWTO I found online,” he said. “Lot of kids
around here up to no good. It’s pretty much a homebrew civil
defense siren\dash{}rugged and cheap.”

She put a finger in each ear and scratched at the itchy buzzing.
When she removed them, her hearing was almost back to normal. “I
once had an upstairs neighbor in Cambridge who had a stereo system
that loud\dash{}never thought I’d hear it again.”

Perry came and joined them. “I followed them a bit, they’re way
gone now. I think I recognized one of them from the campsite. I’ll
talk to Francis about it and see if he can set them right.”

“Have you been broken into before?”

“A few times. Mostly what we worry about is someone trashing the
printers. Everything else is easy to replace, but when Lester’s old
employer went bust we bought up about fifty of these things at the
auction and I don’t know where we’d lay hands on them again.
Computers are cheap and it’s not like anyone could really
\emph{steal} all this junk.” He flashed her his good-looking,
confident smile again.

“What time do the movies start?”

Lester checked his watch. “About an hour after sunset. If we leave
now we can get a real dinner at a Haitian place I know and then
head over to the Thunderbird. I’ll hide under a blanket in the back
seat so that we can save on admission!”

She’d done that many times as a kid, her father shushing her and
her brother as they giggled beneath the blankets. The thought of
giant Lester doing it made her chuckle. “I think we can afford to
pay for you,” she said.

The dinner was good\dash{}fiery spicy fish and good music in an old tiki
bar with peeling grass wallpaper that managed to look vaguely
Haitian. The waiters spoke Spanish, not French, though. She let
herself be talked into two bottles of beer\dash{}about one and a half
more than she would normally take\dash{}but she didn’t get light-headed.
The heat and humidity seemed to rinse the alcohol right out of her
bloodstream.

They got to the movies just at dusk. It was just like she
remembered from being a little girl and coming with her parents.
Children in pajamas climbed over a jungle-gym to one side of the
lot. Ranked rows of cars faced the huge, grubby white projection
walls. They even showed one of those scratchy old “Let’s all go to
the lobby and get ourselves a treat” cartoon shorts with the
dancing hot-dogs before the movie.

The nostalgia filled her up like a balloon expanding in her chest.
She hadn’t ever seen a computer until she was ten years old, and
that had been the size of a chest-freezer, with less capability
than one of the active printed-computer cards that came in glossy
fashion magazines with come-ons for perfume and weight-loss.

The world had been stood on its head so many times in the
intervening thirty-plus years that it was literally dizzying\dash{}or was
that the beer having a delayed effect? Suddenly all the certainties
she rested on\dash{}her 401k, her house, her ability to navigate the
professional world in a competent manner\dash{}seemed to be built on
shifting sands.

They’d come in Lester’s car, a homemade auto built around two
electric Smart cars joined together to form a kind of mini-sedan
with room enough for Lester to slide into the driver’s perch with
room to spare. Once they arrived, they unpacked clever folding
chairs and sat them beside the car, rolled down the windows, and
turned up the speakers. It was a warm night, but not sticky the way
it had been that day, and the kiss of the wind that rustled the
leaves of the tall palms ringing the theater was like balm.

The movie was something forgettable about bumbling detectives on
the moon, one of those trendy new things acted entirely by animated
dead actors who combined the virtues of box-office draw and cheap
labor. There might have been a couple of fictional actors in there
too, it was hard to say, she’d never really followed the movies
except as a place to escape to. There was real magic and escape in
a drive-in, though, with the palpable evidence of all those other
breathing humans in the darkened night watching the magic story
flicker past on the screen, something that went right into her
hindbrain. Before she knew it, her eyelids were drooping and then
she found herself jerking awake. This happened a couple times
before Lester slipped a pillow under her head and she sank into it
and fell into sleep.

She woke at the closing credits and realized that she’d managed to
prop the pillow on Lester’s barrel-chest. She snapped her head up
and then smiled embarrassedly at him. “Hey, sleepyhead,” he said.
“You snore like a bandsaw, you know it?”

She blushed. “I don’t!”

“You do,” he said.

“I do?”

Perry, on her other side nodded. “You do.”

“God,” she said.

“Don’t worry, you haven’t got anything on Lester,” Perry said.
“I’ve gone into his room some mornings and found all the pictures
lying on the floor, vibrated off their hooks.”

It seemed to her that Lester was blushing now.

“I’m sorry if I spoiled the movie,” she said.

“Don’t sweat it,” Lester said, clearly grateful for the change of
subject. “It was a lousy movie anyway. You drowned out some truly
foul dialogue.”

“Well, there’s that.”

“C’mon, let’s go back to the office and get you your car. It’s an
hour to Miami from here.”

She was wide awake by the time she parked the rent-a-car in the
coffin-hotel’s parking lot and crawled into her room, slapping the
air-con buttons up to full to clear out the stifling air that had
baked into the interior during the day.

She lay on her back in the dark coffin for a long time, eyes open
and slowly adjusting to the idiot lights on the control panel,
until it seemed that she was lying in a space capsule hurtling
through the universe at relativistic speeds, leaving behind
history, the world, everything she knew. She sat up, wide awake, on
West Coast time suddenly, and there was no way she would fall
asleep now, but she lay back down and then she did, finally.

The alarm woke her seemingly five minutes later. She did a couple
laps around the parking lot, padding around, stretching her legs,
trying to clear her head\dash{}her internal clock thought that it was
4AM, but at 7AM on the east coast, the sun was up and the heat had
begun to sizzle all the available moisture into the air. She left
the hotel and drove around Miami for a while. She needed to find
some toiletries and then a cafe where she could sit down and file
some copy. She’d tweeted a bunch of working notes and posted a few
things to her blog the day before, but her editor expected
something more coherent for those who preferred their news a little
more digested.

By the time she arrived at Perry’s junkyard, the day had tipped for
afternoon, the sun no longer straight overhead, the heat a little
softer than it had been the day before. She settled in for another
day of watching the guys work, asking the occasional question. The
column she’d ended up filing had been a kind of wait-and-see piece,
describing the cool culture these two had going between them, and
asking if it could survive scaling up to mass production. Now she
experimented with their works-in-progress, sculptures and machines
that almost worked, or didn’t work at all, but that showed the
scope of their creativity. Kettlewell thought that there were a
thousand, ten thousand people as creative as these two out there,
waiting to be discovered. Could it be true?

“Sure,” Perry said, “why not? We’re just here because someone
dropped the barrier to entry, made it possible for a couple of
tinkerers to get a lot of materials and to assemble them without
knowing a whole lot about advanced materials science. Wasn’t it
like this when the Internet was starting out?”

“Woah,” Suzanne said. “I just realized that you wouldn’t really
remember those days, back in the early nineties.”

“Sure I remember them. I was a kid, but I remember them fine!”

She felt very old. “The thing was that no one really suspected that
there were so many liberal arts majors lurking in the nation’s
universities, dying to drop out and learn perl and HTML.”

Perry cocked his head. “Yeah, I guess that’s analogous. The legacy
of the dotcom years for me is all this free infrastructure, very
cheap network connections and hosting companies and so on. That, I
guess, combined with people willing to use it. I never really
thought of it, but there must have been a lot of people hanging
around in the old days who thought email and the net were pretty
sketchy, right?”

She waved her hands at him. “Perry, lad, you don’t know the half of
it. There are \emph{still} executives in the rustbelt who spend
bailout money on secretaries to print out their email and then
dictate replies into tape recorders to be typed and sent.”

He furrowed his thick eyebrows. “You’re joking,” he said.

She put her hand on her heart. “I kid you not. I knew people in the
newsroom at the Detroit Free Press. There are whole industries in
this country that are living in the last century.

“Well, for me, all that dotcommie stuff was like putting down a
good base, making it easy for people like me to get parts and
build-logs and to find hardware hackers to jam with.”

Perry got engrossed in a tricky bit of engine-in-seashell then and
she wandered over to Lester, who was printing out more Barbie heads
for a much larger version of his mechanical computer. “It’ll be
able to add, subtract, and multiply any two numbers up to 99,” he
said. “It took decades to build a vacuum-tube machine that could do
that much\dash{}I’m doing it with \emph{switches} in just three revs. In
your face, UNIVAC!”

She laughed. He had a huge bag of laser-cut soda-can switches that
he was soldering onto a variety of substrates from polished
car-doors to a bamboo tiki-bar. She looked closely at the solder.
“Is this what sweatshop solder looks like?”

He looked confused, then said, “Oh! Right, Perry’s thing. Yeah,
anything not done by a robot has this artisanal quality of
blobbiness, which I quite like, it’s aesthetic, like a painting
with visible brushstrokes. But Perry’s right: if you see solder
like this on anything that there are a million of, then you know
that it was laid down by kids and women working for slave wages.
There’s no way it’s cheaper to make a million solders by hand than
by robot unless your labor force is locked in, force-fed
amphetamine, and destroyed for anything except prostitution inside
of five years. But here, in something like this, so handmade and
one of a kind, I think it gives it a nice cargo-cult neoprimitive
feel. Like a field of hand-tilled furrows.”

She nodded. Today she was keeping her computer out, writing down
quotes and tweeting thoughts as they came. They worked side by side
in companionable silence for a while as she killed a couple
thousand spams and he laid down a couple dozen blobs of solder.

“How do you like Florida?” he said, after straightening up and
cracking his back.

She barely stopped typing, deep into some email: “It’s all right, I
suppose.”

“There’s great stuff here if you know where to look. Want me to
show you around a little tonight? It’s Friday, after all.”

“Sounds good. Is Perry free?”

It took her a second to register that he hadn’t answered. She
looked up and saw he was blushing to the tips of his ears. “I
thought we could go out just the two of us. Dinner and a walk
around the deco stuff on Miami Beach?”

“Oh,” she said. And the weird thing was, she took it seriously for
a second. She hadn’t been on a date in something like a year, and
he was a really nice guy and so forth. But professional ethics made
that impossible, and besides.

And besides. He was huge. He’d told her he weighed nearly 400
pounds. So fat, he was, essentially, sexless. Round and unshaped,
doughy.

All of these thoughts in an instant and then she said, “Oh, well.
Listen, Lester, it’s about professional ethics. I’m here on a story
and you guys are really swell, but I’m here to be objective. That
means no dating. Sorry.” She said it in the same firm tones as
she’d used to turn down their offer to treat her at the IHOP: a
fact of life, something she just didn’t do. Like turning down a
glass of beer by saying, “No thanks, I don’t drink.” No value
judgment.

But she could see that she had let her thinking show on her face,
if only for the briefest moment. Lester stiffened and his nostrils
flared. He wiped his hands on his thighs, then said, in a light
tone, “Sure, no problem. I understand completely. Should have
thought of that. Sorry!”

“No problem,” she said. She pretended to work on her email a while
longer, then said, “Well, I think I’ll call it a day. See you
Monday for Tjan’s arrival, right?”

“Right!” he said, too brightly, and she slunk away to her car.

She spent the weekend blogging and seeing the beach. The people on
the beach seemed to be of another species from the ones she saw
walking the streets of Hollywood and Miami and Lauderdale. They had
freakishly perfect bodies, the kind of thing you saw in an
anatomical drawing or a comic-book\dash{}so much muscular definition that
they were practically cross-hatched. She even tried out the nude
beach, intrigued to see these perfect specimens in the
all-together, but she chickened out when she realized that she’d
need a substantial wax-job before her body hair was brought down to
norms for that strip of sand.

She did get an eyeful of several anatomically correct drawings
before taking off again. It made her uncomfortably horny and aware
of how long it had been since her last date. That got her thinking
of poor Lester, buried underneath all that flesh, and that got her
thinking about the life she’d chosen for herself, covering the
weird world of tech where the ground never stood still long enough
for her to get her balance.

So she retreated to blog in a cafe, posting snippets and
impressions from her days with the boys, along with photos. Her
readers were all over it, commenting like mad. Half of them thought
it was disgusting\dash{}so much suffering and waste in the world and
these guys were inventing \$10,000 toys out of garbage. The other
half wanted to know where to go to buy one for themselves. Halfway
through Sunday, her laptop battery finally died, needing a fresh
weekly charge, so she retreated again, to the coffin, to wait for
Monday and the new day that would dawn for Perry and Lester and
Kodacell\dash{}and her.

Tjan turned out to be a lot older than she’d expected. She’d
pictured him as about 28, smart and preppie like they all were when
they were fresh out of B-school and full of Management Wisdom.
Instead, he was about forty, balding, with a little pot-belly and
thinning hair. He dressed like an English professor, blue-jeans and
a checked shirt and a tweedy sports-coat that he’d shucked within
seconds of leaving the terminal at Miami airport and stepping into
the blast-furnace heat.

They’d all come in Lester’s big, crazy car, and squishing back in
with Tjan’s suitcases was like a geometry trick. She found herself
half on Perry’s lap, hugging half a big duffel-bag that seemed to
be full of bricks.

“Books,” Tjan said. “Just a little personal library. It’s a bad
habit, moving the physical objects around, but I’m addicted.” He
had a calm voice that might in fact be a little dull, a prof’s
monotone.

They brought him to Perry and Lester’s place, which was three
condos with the dividing walls knocked out in a complex that had
long rust-streaks down its sides and rickety balconies that had
been eaten away by salt air. There was a guardhouse at the front of
the complex, but it was shuttered, abandoned, and graffiti tagged.

Tjan stepped out of the car and put his hands on his hips and
considered the building. “It could use a coat of paint,” he said.
Suzanne looked closely at him\dash{}he was so deadpan, it was hard to
tell what was on his mind. But he slipped her a wink.

“Yeah,” Perry said. “It could at that. On the bright side:
spacious, cheap and there’s a pool. There’s a lot of this down here
since the housing market crashed. The condo association here
dissolved about four years ago, so there’s not really anyone who’s
in charge of all the common spaces and stuff, just a few condo
owners and speculators who own the apartments. Suckers, I’m
thinking. Our rent has gone down twice this year, just for asking.
I’m thinking we could probably get them to pay us to live here and
just keep out the bums and stuff.”

The living quarters were nearly indistinguishable from the workshop
at the junkyard: strewn with cool devices in various stages of
disassembly, detritus and art. The plates and dishes and glasses
all had IHOP and Cracker Barrel logos on them. “From thrift shops,”
Lester explained. “Old people steal them when they get their
earlybird specials, and then when they die their kids give them to
Goodwill. Cheapest way to get a matched set around here.”

Tjan circled the three adjoined cracker-box condos like a dog
circling his basket. Finally, he picked an unoccupied master
bedroom with moldy lace curtains and a motel-art painting of an
abstract landscape over the headboard. He set his suitcase down on
the faux-Chinoise chest of drawers and said, “Right, I’m done.
Let’s get to work.”

They took him to the workshop next and his expression hardly
changed as they showed him around, showed him their cabinets of
wonders. When they were done, he let them walk him to the IHOP and
he ordered the most austere thing on the menu, a peanut-butter and
jelly sandwich that was technically on the kids’ menu\dash{}a kids’ menu
at a place where the grownups could order a plate of candy!

“So,” Perry said. “So, Tjan, come on buddy, give it to me
straight\dash{}you hate it? Love it? Can’t understand it?”

Tjan set down his sandwich. “You boys are very talented,” he said.
“They’re very good inventions. There are lots of opportunities for
synergy within Kodacell: marketing, logistics, even packing
materials. There’s a little aerogel startup in Oregon that Kodacell
is underwriting that you could use for padding when you ship.”

Perry and Lester looked at him expectantly. Suzanne broke the
silence. “Tjan, did you have any artistic or design ideas about the
things that these guys are making?”

Tjan took another bite of sandwich and sipped at his milk. “Well,
you’ll have to come up with a name for them, something that
identifies them. Also, I think you should be careful with
trademarked objects. Any time you need to bring in an IP lawyer,
you’re going to run into huge costs and time delays.”

They waited again. “That’s it?” Perry said. “Nothing about the
designs themselves?”

“I’m the business-manager. That’s editorial. I’m artistically
autistic. Not my job to help you design things. It’s my job to sell
the things you design.”

“Would it matter what it was we were making? Would you feel the
same if it was toothbrushes or staplers?”

Tjan smiled. “If you were making staplers I wouldn’t be here,
because there’s no profit in staplers. Too many competitors.
Toothbrushes are a possibility, if you were making something really
revolutionary. People buy about 1.6 toothbrushes a year, so there’s
lots of opportunity to come up with an innovative design that sells
at a good profit over marginal cost for a couple seasons before it
gets cloned or out-innovated. What you people are making has an
edge because it’s you making it, very bespoke and distinctive. I
think it will take some time for the world to emerge an effective
competitor to these goods, provided that you can build an initial
marketplace mass-interest in them. There aren’t enough people out
there who know how to combine all the things you’ve combined here.
The system makes it hard to sell anything above the marginal cost
of goods, unless you have a really innovative idea, which can’t
stay innovative for long, so you need continuous invention and
re-invention too. You two fellows appear to be doing that. I don’t
know anything definitive about the aesthetic qualities of your
gadgets, nor how useful they’ll be, but I \emph{do} understand
their distinctiveness, so that’s why I’m here.”

It was longer than all the speeches he’d delivered since arriving,
put together. Suzanne nodded and made some notes. Perry looked him
up and down.

“You’re, what, an ex-B-school prof from Cornell, right?”

“Yes, for a few years. And I ran a company for a while, doing
import-export from emerging economy states in the former Soviet
bloc.”

“I see,” Perry said. “So you’re into what, a new company every 18
months or something?”

“Oh no,” Tjan said, and he had a little twinkle in his eye and the
tiniest hint of a smile. “Oh no. Every six months. A year at the
outside. That’s my deal. I’m the business guy with the short
attention span.”

“I see,” Perry said. “Kettlewell didn’t mention this.”

At the junkyard, Tjan wandered around the Elmo-propelled Smart car
and peered at its innards, watched the Elmos negotiate their
balance and position with minute movements and acoustic signals. “I
wouldn’t worry about it if I were you,” he said. “You guys aren’t
temperamentally suited to doing just one thing.”

Lester laughed. “He’s got you there, dude,” he said, slapping Perry
on the shoulder.

Suzanne got Tjan out for dinner that night. “My dad was in
import-export and we travelled a lot, all over Asia and then the
former Soviets. He sent me away when I was 16 to finish school in
the States, and there was no question but that I would go to
Stanford for business school.”

“Nice to meet a fellow Californian,” she said, and sipped her wine.
They’d gone to one of the famed Miami deco restaurants and the fish
in front of her was practically a sculpture, so thoroughly plated
it was.

“Well, I’m as Californian as\ldots{}”

“\ldots{} as possible, under the circumstances,” she said and laughed.
“It’s a Canadian joke, but it applies equally well to Californians.
So you were in B-school when?”

“Ninety eight to 2001. Interesting times to be in the Valley. I
read your column, you know.”

She looked down at her plate. A lot of people had read the column
back then. Women columnists were rare in tech, and she supposed she
was good at it, too. “I hope I get remembered as more than the
chronicler of the dot-com boom, though,” she said.

“Oh, you will,” he said. “You’ll be remembered as the chronicler of
this\dash{}what Kettlewell and Perry and Lester are doing.”

“What you’re doing, too, right?”

“Oh, yes, what I’m doing too.”

A robot rollerbladed past on the boardwalk, turning the occasional
somersault. “I should have them build some of those,” Tjan said,
watching the crowd turn to regard it. It hopped onto and off of the
curb, expertly steered around the wandering couples and the
occasional homeless person. It had a banner it streamed out behind
it: \textuppercase{CAP’N JACKS PAINTBALL AND FANBOAT TOURS GET SHOT AND GET WET
MIAMI KEY WEST LAUDERDALE.}

“You think they can?”

“Sure,” Tjan said. “Those two can build anything. That’s the point:
any moderately skilled practitioner can build anything these days,
for practically nothing. Back in the old days, the blacksmith just
made every bit of ironmongery everyone needed, one piece at a time,
at his forge. That’s where we’re at. Every industry that required a
factory yesterday only needs a garage today. It’s a real return to
fundamentals. What no one ever could do was join up all the
smithies and all the smiths and make them into a single logical
network with a single set of objectives. That’s new and it’s what I
plan on making hay out of. This will be much bigger than dot-com.
It will be much harder, too\dash{}bigger crests, deeper troughs. This is
something to chronicle all right: it will make dot-com look like a
warm up for the main show.

“We’re going to create a new class of artisans who can change
careers every 10 months, inventing new jobs that hadn’t been
imagined a year before.”

“That’s a pretty unstable market,” Suzanne said, and ate some
fish.

“That’s a \emph{functional} market. Here’s what I think the point
of a good market is. In a good market, you invent something and you
charge all the market will bear for it. Someone else figures out
how to do it cheaper, or decides they can do it for a slimmer
margin\dash{}not the same thing, you know, in the first case someone is
more efficient and in the second they’re just less greedy or less
ambitious. They do it and so you have to drop your prices to
compete. Then someone comes along who’s less greedy or more
efficient than both of you and undercuts you again, and again, and
again, until eventually you get down to a kind of firmament, a
baseline that you can’t go lower than, the cheapest you can produce
a good and stay in business. That’s why straightpins, machine
screws and reams of paper all cost basically nothing, and make
damned little profit for their manufacturers.

“So if you want to make a big profit, you’ve got to start over
again, invent something new, and milk it for all you can before the
first imitator shows up. The more this happens, the cheaper and
better everything gets. It’s how we got here, you see. It’s what
the system is \emph{for}. We’re approaching a kind of pure and
perfect state now, with competition and invention getting easier
and easier\dash{}it’s producing a kind of superabundance that’s amazing
to watch. My kids just surf it, make themselves over every six
months, learn a new interface, a new entertainment, you name it.
Change-surfers\ldots{}” He trailed off.

“You have kids?”

“In St Petersburg, with their mother.”

She could tell by his tone that it had been the wrong question to
ask. He was looking hangdog. “Well, it must be nice to be so much
closer to them than you were in Ithaca.”

“What? No, no. The St Petersburg in \emph{Russia}.”

“Oh,” she said.

They concentrated on their food for a while.

“You know,” he said, after they’d ordered coffee and desert, “it’s
all about abundance. I want my kids to grow up with abundance, and
whatever is going on right now, it’s providing abundance in
abundance. The self-storage industry is bigger than the recording
industry, did you know that? All they do is provide a place to put
stuff that we own that we can’t find room for\dash{}that’s
superabundance.”

“I have a locker in Milpitas,” she said.

“There you go. It’s a growth industry.” He drank his coffee. On the
way back to their cars, he said, “My daughter, Lyenitchka, is four,
and my son, Sasha, is one. I haven’t lived with their mother in
three years.” He made a face. “Sasha’s circumstances were
complicated. They’re good kids, though. It just couldn’t work with
their mother. She’s Russian, and connected\dash{}that’s how we met, I was
hustling for my import-export business and she had some good
connections\dash{}so after the divorce there was no question of my taking
the kids with me. But they’re good kids.”

“Do you see them?”

“We videoconference. Who knew that long-distance divorce was the
killer app for videoconferencing?”

“Yeah.”

That week, Suzanne tweeted constantly, filed two columns, and
blogged ten or more items a day: photos, bits of discussion between
Lester, Perry and Tjan, a couple videos of the Boogie Woogie Elmos
doing improbable things. Turned out that there was quite a cult
following for the BWE, and the news that there was a trove of some
thousands of them in a Hollywood dump sent a half-dozen pilgrims
winging their way across the nation to score some for the
collectors’ market. Perry wouldn’t even take their money: “Fella,”
he told one persistent dealer, “I got forty \emph{thousand} of
these things. I won’t miss a couple dozen. Just call it good
karma.”

When Tjan found out about it he pursed his lips for a moment, then
said, “Let me know if someone wants to pay us money, please. I
think you were right, but I’d like to have a say, all right?”

Perry looked at Suzanne, who was videoing this exchange with her
keychain. Then he looked back at Tjan, “Yeah, of course.
Sorry\dash{}force of habit. No harm done, though, right?”

That footage got downloaded a couple hundred times that night, but
once it got slashdotted by a couple of high-profile headline
aggregators, she found her server hammered with a hundred thousand
requests. The Merc had the horsepower to serve them all, but you
never knew: every once in a while, the web hit another tipping
point and grew by an order of magnitude or so, and then all the
server-provisioning\dash{}calculated to survive the old
slashdottings\dash{}shredded like wet kleenex.

\begin{email}
From: kettlewell-l@skunkworks.kodacell.com\\
To: schurch@sjmercury.com\\
Subject: Re: Embedded journalist?

This stuff is amazing. Amazing! Christ, I should put you on the
payroll. Forget I wrote that. But i should. You’ve got a fantastic
eye. I have never felt as in touch with my own business as I do at
this moment. Not to mention proud! Proud\dash{}you’ve made me so proud of
the work these guys are doing, proud to have some role in it.

Kettlebelly
\end{email}

She read it sitting up in her coffin, just one of several hundred
emails from that day’s blog-posts and column. She laughed and
dropped it in her folder of correspondence to answer. It was nearly
midnight, too late to get into it with Kettlewell.

Then her computer rang\dash{}the net-phone she forwarded her cellphone to
when her computer was live and connected. She’d started doing that
a couple years back, when soft-phones really stabilized, and her
phone bills had dropped to less than twenty bucks a month, down
from several hundred. It wasn’t that she spent a lot of time within
arm’s reach of a live computer, but given that calls routed through
the laptop were free, she was perfectly willing to defer her calls
until she was.

“Hi Jimmy,” she said\dash{}her editor, back in San Jose. 9PM Pacific time
on a weeknight was still working hours for him.

“Suzanne,” he said.

She waited. She’d half expected him to call with a little shower of
praise, an echo of Kettlewell’s note. Jimmy wasn’t the most
effusive editor she’d had, but it made his little moments of praise
more valuable for their rarity.

“Suzanne,” he said again.

“Jimmy,” she said. “It’s late here. What’s up?”

“So, it’s like this. I love your reports but it’s not Silicon
Valley news. It’s Miami news. McClatchy handed me a thirty percent
cut this morning and I’m going to the bone. I am firing a third of
the newsroom today. Now, you are a stupendous writer and so I said
to myself, ‘I can fire her or I can bring her home and have her
write about Silicon Valley again,’ and I knew what the answer had
to be. So I need you to come home, just wrap it up and come home.”

He finished speaking and she found herself staring at her
computer’s screen. Her hands were gripping the laptop’s edges so
tightly it hurt, and the machine made a plasticky squeak as it
began to bend.

“I can’t do that, Jimmy. This is stuff that Silicon Valley needs to
know about. This may not be what’s happening \emph{in} Silicon
Valley, but it sure as shit is what’s happening \emph{to} Silicon
Valley.” She hated that she’d cussed\dash{}she hadn’t meant to. “I know
you’re in a hard spot, but this is the story I need to cover right
now.”

“Suzanne, I’m cutting a third of the newsroom. We’re going to be
covering stories within driving distance of this office for the
foreseeable future, and that’s it. I don’t disagree with a single
thing you just said, but it doesn’t matter: if I leave you where
you are, I’ll have to cut the guy who covers the school boards and
the city councils. I can’t do that, not if I want to remain a daily
newspaper editor.”

“I see,” she said. “Can I think about it?”

“Think about what, Suzanne? This has not been the best day for me,
I have to tell you, but I don’t see what there is to think about.
This newspaper no longer has correspondents who work in Miami and
London and Paris and New York. As of today, that stuff comes from
bloggers, or off the wire, or whatever\dash{}but not from our payroll.
You work for this newspaper, so you need to come back here, because
the job you’re doing does not exist any longer. The job you have
with us is here. You’ve missed the night-flight, but there’s a
direct flight tomorrow morning that’ll have you back by lunchtime
tomorrow, and we can sit down together then and talk about it, all
right?”

“I think\dash{}” She felt that oh-shit-oh-shit feeling again, that
needing-to-pee feeling, that tension from her toes to her nose.
“Jimmy,” she said. “I need a leave of absence, OK?”

“What? Suzanne, I’m sure we owe you some vacation but now isn’t the
time\dash{}”

“Not a vacation, Jimmy. Six months leave of absence, without pay.”
Her savings could cover it. She could put some banner ads on her
blog. Florida was cheap. She could rent out her place in
California. She was six steps into the plan and it had only taken
ten seconds and she had no doubts whatsoever. She could talk to
that book-agent who’d pinged her last year, see about getting an
advance on a book about Kodacell.

“Are you quitting?”

“No, Jimmy\dash{}well, not unless you make me. But I need to stay here.”

“The work you’re doing there is fine, Suzanne, but I worked really
hard to protect your job here and this isn’t going to help make
that happen.”

“What are you saying?”

“If you want to work for the Merc, you need to fly back to San
Jose, where the Merc is published. I can’t make it any clearer than
that.”

No, he couldn’t. She sympathized with him. She was really well paid
by the Merc. Keeping her on would mean firing two junior writers.
He’d cut her a lot of breaks along the way, too\dash{}let her feel out
the Valley in her own way. It had paid off for both of them, but
he’d taken the risk when a lot of people wouldn’t have. She’d be a
fool to walk away from all that.

She opened her mouth to tell him that she’d be on the plane in the
morning, and what came out was, “Jimmy, I really appreciate all the
work you’ve done for me, but this is the story I need to write. I’m
sorry about that.”

“Suzanne,” he said.

“Thank you, Jimmy,” she said. “I’ll get back to California when I
get a lull and sort out the details\dash{}my employee card and stuff.”

“You know what you’re doing, right?”

“Yeah,” she said. “I do.”

When she unscrewed her earpiece, she discovered that her neck was
killing her. That made her realize that she was a
forty-five-year-old woman in America without health insurance. Or
regular income. She was a journalist without a journalistic organ.

She’d have to tell Kettlewell, who would no doubt offer to put her
on the payroll. She couldn’t do that, of course. Neutrality was
hard enough to maintain, never mind being financially compromised.

She stepped out of the coffin and sniffed the salty air. Living in
the coffin was expensive. She’d need to get a condo or something. A
place with a kitchen where she could prep meals. She figured that
Perry’s building would probably have a vacancy or two.

\begin{center}\rule{3in}{0.4pt}\end{center}

The second business that Tjan took Perry into was even more
successful than the first, and that was saying something. It only
took a week for Tjan to get Perry and Lester cranking on a Kitchen
Gnome design that mashed together some Homeland Security
gait-recognition software with a big solid-state hard-disk and a
microphone and a little camera, all packaged together in one of a
couple hundred designs of a garden-gnome figurine that stood six
inches tall. It could recognize every member of a household by the
way they walked and play back voice-memos for each. It turned out
to be a killer tool for context-sensitive reminders to kids to do
the dishes, and for husbands, wives and roommates to nag each other
without getting on each others’ nerves. Tjan was really jazzed
about it, as it tied in with some theories he had about the
changing US demographic, trending towards blended households in
urban centers, with three or more adults co-habitating.

“This is a rich vein,” he said, rubbing his hands together. “Living
communally is hard, and technology can make it easier. Roommate
ware. It’s the wave of the future.”

There was another Kodacell group in San Francisco, a design outfit
with a bunch of stringers who could design the gnomes for them and
they did great work. The gnomes were slightly lewd-looking, and
they were the product of a generative algorithm that varied each
one. Some of the designs that fell out of the algorithm were
jaw-droppingly weird\dash{}Perry kept a three-eyed, six-armed version on
his desk. They tooled up to make them by the hundred, then the
thousand,then the tens of thousand. The fact that each one was
different kept their margins up, but as the Gnomes gained
popularity their sales were steadily eroded by knock-offs, mostly
from Eastern Europe.

The knockoffs weren’t as cool-looking\dash{}though they were certainly
weirder looking, like the offspring of a Norwegian troll and an
anime robot\dash{}but they were more feature-rich. Some smart hacker in
Russia was packing all kinds of functionality onto a single chip,
so that their trolls cost less and did more: burglar alarms,
baby-monitors, streaming Internet radio source, and low-reliability
medical diagnostic that relied on quack analysis of eye pigment,
tongue coating and other newage (rhymes with sewage) indicators.

Lester came back from the Dollar Store with a big bag of trolls, a
dozen different models, and dumped them out on Tjan’s desk, up in
old foreman’s offices on the catwalk above the workspaces. “Christ,
would you look at these? They’re selling them for less than our
cost to manufacture. How do we compete with this?”

“We don’t,” Tjan said, and rubbed his belly. “Now we do the next
thing.”

“What’s the next thing?” Perry said.

“Well, the first one delivered a return-on-investment at about
twenty times the rate of any Kodak or Duracell business unit in the
history of either company. But I’d like to shoot for thirty to
forty times next, if that’s all right with you. So let’s go see
what you’ve invented this week and how we can commercialize it.”

Perry and Lester just looked at each other. Finally, Lester said,
“Can you repeat that?”

“The typical ROI for a Kodacell unit in the old days was about four
percent. If you put a hundred dollars in, you’d get a hundred and
four dollars out, and it would take about a year to realize. Of
course, in the old days, they wouldn’t have touched a new business
unless they could put a hundred million in and get a hundred and
four million out. Four million bucks is four million bucks.

“But here, the company put fifty thousand into these dolls and
three months later, they took seventy thousand out, after paying
our salaries and bonuses. That’s a forty percent ROI. Seventy
thousand bucks isn’t four million bucks, but forty percent is forty
percent. Not to mention that our business drove similar margins in
three other business units.”

“I thought we’d screwed up by letting these guys eat our lunch,”
Lester said, indicating the dollar-store trolls.

“Nope, we got in while the margins were high, made a good return,
and now we’ll get out as the margins drop. That’s not screwing up,
that’s doing the right thing. The next time around, we’ll do
something more capital intensive and we’ll take out an even higher
margin: so show me something that’ll cost two hundred grand to get
going and that we can pull a hundred and sixty thou’s worth of
profit out of for Kodacell in three months. Let’s do something
ambitious this time around.”

Suzanne took copious notes. There’d been a couple weeks’
awkwardness early on about her scribbling as they talked, or
videoing with her keychain. But once she’d moved into the building
with the guys, taking a condo on the next floor up, she’d become
just a member of the team, albeit a member who tweeted nearly every
word they uttered to a feed that was adding new subscribers by the
tens of thousands.

“So, Perry, what have you got for Tjan?” she asked.

“I came up with the last one,” he said, grinning\dash{}they always ended
up grinning when Tjan ran down economics for them. “Let Lester take
this one.”

Lester looked shy\dash{}he’d never fully recovered from Suzanne turning
him down and when she was in the room, he always looked like he’d
rather be somewhere else. He participated in the message boards on
her blog though, the most prolific poster in a field with thousands
of very prolific posters. When he posted, others listened: he was
witty, charming and always right.

“Well, I’ve been thinking a lot about roommate-ware, ’cause I know
that Tjan’s just crazy for that stuff. I’ve been handicapped by the
fact that you guys are such excellent roomies, so I have to think
back to my college days to remember what a bad roommate is like,
where the friction is. Mostly, it comes down to resource
contention, though: I wanna cook, but your dishes are in the sink;
I wanna do laundry but your boxers are in the dryer; I wanna watch
TV, but your crap is all over the living room sofa.”

Living upstairs from the guys gave her fresh insight into how the
Kodacell philosophy would work out. Kettlewell was really big on
communal living, putting these people into each other’s pockets
like the old-time geek houses of pizza-eating hackers, getting that
in-the-trenches camaraderie. It had taken a weekend to put the most
precious stuff in her California house into storage and then turn
over the keys to a realtor who’d sort out leasing it for her. The
monthly check from the realtor left more than enough for her to pay
the rent in Florida and then some, and once the UPS man dropped off
the five boxes of personal effects she’d chosen, she was
practically at home.

She sat alone over the guys’ apartments in the evenings, windows
open so that their muffled conversations could drift in and form
the soundtrack as she wrote her columns. It made her feel curiously
with, but not of, their movement\dash{}a reasonable proxy for
journalistic objectivity in this age of relativism.

“Resource contention readily decomposes into a bunch of smaller
problems, with distinctive solutions. Take dishes: every dishwasher
should be designed with a ’clean’ and a ’dirty’
compartment\dash{}basically, two logical dishwashers. You take clean
dishes out of the clean side, use them, and put them into the dirty
side. When the dirty side is full, the clean side is empty, so you
cycle the dishwasher and the clean side becomes dirty and
vice-versa. I had some sketches for designs that would make this
happen, but it didn’t feel right: making dishwashers is too
industrial for us. I either like making big chunks of art or little
silver things you can carry in your pocket.”

She smiled despite herself. She was drawing a half-million readers
a day by doing near-to-nothing besides repeating the mind-blowing
conversations around her. It had taken her a month to consider
putting ads on the site\dash{}lots of feelers from blog “micro-labels”
who wanted to get her under management and into their banner
networks, and she broke down when one of them showed her a little
spreadsheet detailing the kind of long green she could expect to
bring in from a couple of little banners, with her getting the
right to personally approve every advertiser in the network. The
first month, she’d made more money than all but the most senior
writers on the Merc. The next month, she’d outstripped her own old
salary. She’d covered commercial blogs, the flamboyant
attention-whores who’d bought stupid cars and ridiculous bimbos
with the money, but she’d always assumed they were in a different
league from a newspaper scribbler. Now she supposed all the money
meant that she should make it official and phone in a resignation
to Jimmy, but they’d left it pretty ambiguous as to whether she was
retiring or taking a leave of absence and she was reluctant to
collapse that waveform into the certainty of saying goodbye to her
old life.

“So I got to thinking about snitch-tags, radio frequency ID gizmos.
Remember those? When we started talking about them a decade ago,
all the privacy people went crazy, totally sure that these things
would be bad news. The geeks dismissed them as not understanding
the technology. Supposedly, an RFID can only be read from a couple
inches away\dash{}if someone wanted to find out what RFIDs you had on
your person, they’d have to wand you, and you’d know about it.”

“Yeah, that was bull,” Perry said. “I mean, sure you can’t read an
RFID unless it’s been excited with electromagnetic radiation, and
\emph{sure} you can’t do that from a hundred yards without frying
everything between you and the target. But if you had a subway
turnstile with an exciter built into it, you could snipe all the
tag numbers from a distant roof with a directional antenna. If
those things had caught on, there’d be exciters everywhere and
you’d be able to track anyone you wanted\dash{}Christ, they even put
RFIDs in the hundred-dollar bill for a while! Pickpockets could
have figured out whose purse was worth snatching from half a mile a
way!”

“All true,” Lester said. “But that didn’t stop these guys. There
are still a couple of them around, limping along without many
customers. They print the tags with inkjets, sized down to about a
third the size of a grain of rice. Mostly used in supply-chain
management and such. They can supply them on the cheap.

“Which brings me to my idea: why not tag everything in a group
household, and use the tags to figure out who left the dishes in
the sink, who took the hammer out and didn’t put it back, who put
the empty milk-carton back in the fridge, and who’s got the TV
remote? It won’t solve resource contention, but it will limit the
social factors that contribute to it.” He looked around at them.
“We can make it fun, you know, make cool RFID sticker designs, mod
the little gnome dolls to act as terminals for getting reports.”

Suzanne found herself nodding along. She could use this kind of
thing, even though she lived alone, just to help her find out where
she left her glasses and the TV remote.

Perry shook his head, though. “When I was a kid, I had a really bad
relationship with my mom. She was really smart, but she didn’t have
a lot of time to reason things out with me, so often as not she’d
get out of arguing with me by just changing her story. So I’d say,
’Ma, can I go to the mall this aft?’ and she’d say, ‘Sure, no
problem.’ Then when I was getting ready to leave the house, she’d
ask me where I thought I was going. I’d say, ‘To the mall, you
said!’ and she’d just deny it. Just deny it, point blank.

“I don’t think she even knew she was doing it. I think when I asked
her if I could go, she’d just absentmindedly say yes, but when it
actually came time to go out, she’d suddenly remember all my
unfinished chores, my homework, all the reasons I should stay home.
I think every kid gets this from their folks, but it made me
fucking crazy. So I got a mini tape recorder and I started to
\emph{tape} her when she gave me permission. I thought I’d really
nail her the next time she changed her tune, play her own words
back in her ear.

“So I tried it, and you know what happened? She gave me nine kinds
of holy hell for wearing a wire and then she said it didn’t matter
what she’d said that morning, she was my mother and I had chores to
do and no \emph{how} was I going \emph{anywhere} now that I’d
started sneaking around the house with a hidden recorder. She took
it away and threw it in the trash. And to top it off, she called me
’J. Edgar’ for a month.

“So here’s my question: how would you feel if the next time you
left the dishes in the sink, I showed up with the audit trail for
the dishes and waved it in your face? How would we get from that
point to a happy, harmonious household? I think you’ve mistaken the
cause for the effect. The problem with dishes in the sink isn’t
just that it’s a pain when I want to cook a meal: it’s that when
you leave them in the sink, you’re being inconsiderate. And the
\emph{reason} you’ve left them in the sink, as you’ve pointed out,
is that putting dishes in the dishwasher is a pain in the ass: you
have to bend over, you have to empty it out, and so on. If we moved
the dishwasher into the kitchen cupboards and turned half of them
into a dirty side and half into a clean side, then disposing of
dishes would be as easy as getting them out.”

Lester laughed, and so did Tjan. “Yeah, yeah\dash{}OK. Point taken. But
these RFID things, they’re so frigging cheap and potentially
useful. I just can’t believe that they’ve never found a single
really compelling use in all this time. It just seems like an
opportunity that’s going to waste.”

“Maybe it’s a dead end. Maybe it’s an ornithopter. Inventors spent
hundreds of years trying to build an airplane that flew by flapping
its wings, and it was all a rat-hole.”

“I guess,” Lester said. “But I don’t like the idea.”

“Like it or don’t, “ Perry said, “doesn’t affect whether it’s true
or not.”

But Lester had a sparkle in his eye, and he disappeared into his
workshop for a week, and wouldn’t let them in, which was unheard of
for the big, gregarious giant. He liked to drag the others in
whenever he accomplished anything of note, show it off to them like
a big kid.

That was Sunday. Monday, Suzanne got a call from her realtor. “Your
tenants have vanished,” she said.

“Vanished?” The couple who’d rented her place had been as reliable
as anyone she’d ever met in the Valley. He worked at a PR agency,
she worked in marketing at Google. Or maybe he worked in marketing
and she was in PR at Google\dash{}whatever, they were affluent,
well-spoken, and had paid the extortionate rent she’d charged
without batting an eye.

“They normally paypal the rent to me on the first, but not this
month. I called and left voicemail the next day, then followed up
with an email. Yesterday I went by the house and it was empty. All
their stuff was gone. No food in the fridge. I think they might
have taken your home theater stuff, too.”

“You’re fucking kidding me,” Suzanne said. It was 11AM in Florida
and she was into her second glass of lemonade as the sun began to
superheat the air. Back in California, it was 8AM. Her realtor was
pulling long hours, and it wasn’t her fault. “Sorry. Right. OK,
what about the deposit?”

“You waived it.”

She had. It hadn’t seemed like a big deal at the time. The distant
owner of the condo she was renting in Florida hadn’t asked for one.
“So I did. Now what?”

“You want to swear out a complaint against them?”

“With the police?”

“Yeah. Breach of contract. Theft, if they took the home theater. We
can take them to collections, too.”

Goddamned marketing people had the collective morals of a snake.
All of them useless, conniving, shallow\dash{}she never should have\ldots

“Yeah, OK. And what about the house?”

“We can find you another tenant by the end of the month, I’m sure.
Maybe a little earlier. Have you thought any more about selling
it?”

She hadn’t, though the realtor brought it up every time they spoke.
“Is now a good time?”

“Lot of new millionaires in the Valley shopping for houses,
Suzanne. More than I’ve seen in years.” She named a sum that was a
third higher than the last time they’d talked it over.

“Is it peaking?”

“Who knows? It might go up, it might collapse again. But now is the
best time to sell in the past ten years. You’d be smart to do it.”

She took a deep breath. The Valley was dead, full of venal
marketing people and buck-chasers. Here in Florida, she was on the
cusp of the next thing, and it wasn’t happening in the Valley: it
was happening everywhere \emph{except} the Valley, in the cheap
places where innovation could happen at low rents. Leaky hot tub,
incredible property taxes, and the crazy roller-coaster ride\dash{}up 20
percent this month, down forty next. The bubble was going to burst
some day and she should sell out now.

“Sell it,” she said.

“You’re going to be a wealthy lady,” the realtor said.

“Right,” Suzanne said.

“I have a buyer, Suzanne. I didn’t want to pressure you. But I can
sell it by Friday. Close escrow next week. Cash in hand by the
fifteenth.”

“Jesus,” she said. “You’re joking.”

“No joke,” the realtor said. “I’ve got a waiting list for houses on
your block.”

And so Suzanne got on an airplane that night and flew back to San
Jose and took a pricey taxi back to her place. The marketdroids had
left it in pretty good shape, clean and tidy, clean sheets in the
linen cupboard. She made up her bed and reflected that this would
be the last time she made this bed\dash{}the next time she stripped the
sheets, they’d go into a long-term storage box. She’d done this
before, on her way out of Detroit, packing up a life into boxes and
shoving it into storage. What had Tjan said? “The self-storage
industry is bigger than the recording industry, did you know that?
All they do is provide a place to put stuff that we own that we
can’t find room for\dash{}that’s superabundance.”

Before bed she posted a classified on Craigslist for a couple
helpers to work on boxing stuff, emailed Jimmy to see if he wanted
lunch, and looked up the address for the central police station to
swear out her complaint. The amp, speakers, and A/V switcher were
all missing from her home theater.

She had a dozen helpers to choose from the next morning. She picked
two who came with decent references, marveling that it was suddenly
possible in Silicon Valley to get anyone to show up anywhere for
ten bucks an hour. The police sergeant who took the complaint was
sympathetic and agreed with her choice to get out of town. “I’ve
had it with this place, too. Soon as my kids are out of high-school
I’m moving back to Montana. I miss the weather.”

She didn’t think of the marketdroids again until the next day, when
she and her helpers were boxing up the last of her things and
loading them into her U-Haul. Then a BMW convertible screeched
around the corner and burned rubber up to her door.

The woman marketdroid was driving, looking crazy and disheveled,
eyes red-rimmed, one heel broken off of her shoes.

“What the FUCK is your problem, lady?” she said, as she leapt out
of her car and stalked toward Suzanne.

Instinctively, Suzanne shrank back and dropped the box of books she
was holding. It spilled out over her lawn.

“Fiona?” she said. “What’s happened?”

“I was \emph{arrested.} They came to my workplace and led me out in
handcuffs. I had to make \emph{bail}.”

Suzanne’s stomach shrank to a little pebble, impossibly heavy.
“What was I supposed to do? You two took off with my home
theater!”

“What home theater? Everything was right where you left it when I
went. I haven’t lived here in \emph{weeks}. Tom left me last month
and I moved out.”

“You moved out?”

“Yeah, bitch, I \emph{moved out}. Tom was your tenant, not me. If
he ripped something off, that’s between you and him.”

“Look, Fiona, wait, hold up a second. I tried to call you, I sent
you email. No one was paying the rent, no one told me that you’d
moved out, and no one answered when I tried to find out what had
happened.”

“That sounds like an \emph{explanation,} she said, hissing. “I’m
waiting for a fucking \emph{apology.} They took me to
\emph{prison}.”

Suzanne knew that the local lockup was a long way from prison. “I
apologize,” she said. “Can I get you a cup of coffee? Would you
like to use the shower or anything?”

The woman glared at her a moment longer, then slowly folded in on
herself, collapsing, coughing and sobbing on the lawn.

Suzanne stood with her arms at her sides for a moment. Her
Craigslist helpers had gone home, so she was all alone, and this
woman, whom she’d met only once before, in passing, was clearly
having some real problems. Not the kind of thing she dealt with a
lot\dash{}her life didn’t include much person-to-person hand-holding.

But what can you do? She knelt beside Fiona in the grass and took
her hand. “Let’s get you inside, OK?”

At first it was as though she hadn’t heard, but slowly she
straightened up and let Suzanne lead her into the house. She was
twenty-two, twenty-three, young enough to be Suzanne’s daughter if
Suzanne had gone in for that sort of thing. Suzanne helped her to
the sofa and sat her down amid the boxes still waiting to go into
the U-Haul. The kitchen was packed up, but she had a couple bottles
of Diet Coke in the cooler and she handed one to the girl.

“I’m really sorry, Fiona. Why didn’t you answer my calls or
email?”

She looked at Suzanne, her eyes lost in streaks of mascara. “I
don’t know. I didn’t want to talk about it. He lost his job last
month and kind of went crazy, told me he didn’t want the
responsibility anymore. What responsibility? But he told me to go,
told me it would be best for both of us if we were apart. I thought
it was another girl, but I don’t know. Maybe it was just craziness.
Everyone I know out here is crazy. They all work a hundred hours a
week, they get fired or quit their jobs every five months.
Everything is so expensive. My rent is three quarters of my
salary.”

“It’s really hard,” Suzanne said, thinking of the easy, lazy days
in Florida, the hackers’ idyll that Perry and Lester enjoyed in
their workshops.

“Tom was on antidepressants, but he didn’t like taking them. When
he was on them, he was pretty good, but when he went off, he turned
into\ldots{} I don’t know. He’d cry a lot, and shout. It wasn’t a good
relationship, but we moved out here from Oregon together, and I’d
known him all my life. He was a little moody before, but not like
he was here.”

“When did you speak to him last?” Suzanne had found a couple of
blister-packs of anti-depressants in the medicine chest. She hoped
that wasn’t Tom’s only supply.

“We haven’t spoken since I moved out.”

An hour later, the mystery was solved. The police went to Tom’s
workplace and discovered that he’d been fired the week before. They
tried the GPS in his car and it finked him out as being in a ghost
mall’s parking lot near his old office. He was dead behind the
wheel, a gun in his hand, shot through the heart.

Suzanne took the call and though she tried to keep her end of the
conversation quiet and neutral, Fiona\dash{}still on the sofa, drinking
the warm, flat Coke\dash{}knew. She let out a moan like a dog that’s been
kicked, and then a scream. For Suzanne, it was all unreal,
senseless. The cops told her that her home theater components were
found in the trunk of the car. No note.

“God, oh God, Jesus, you selfish shit fucking bastard,” Fiona
sobbed. Awkwardly, Suzanne sat down beside her and took her into a
one-armed hug. Her helpers were meeting her at the self-storage the
next day to help her unload the U-Haul.

“Do you have someone who can stay with you tonight?” Suzanne asked,
praying the answer was yes. She had a house to move out of. Christ,
she felt so cold-blooded, but she was on a goddamned schedule.

“Yes, I guess.” Fiona scrubbed at her eyes with her fists. “Sure.”

Suzanne sighed. The lie was plain. “Who?”

Fiona stood up and smoothed out her skirt. “I’m sorry,” she said,
and started for the door.

Groaning inwardly, Suzanne blocked her. “You’ll stay on the sofa,”
she said. “You’re not driving in this state. I’ll order in pizza.
Pepperoni mushroom OK?”

Looking defeated, Fiona turned on her heel and went back to the
sofa.

Over pizza, Suzanne pulled a few details out of her. Tom had fallen
into a funk when the layoffs had started in his office\dash{}they were
endemic across the Valley, another bust was upon them. His behavior
had grown worse and worse, and she’d finally left, or been thrown
out, it wasn’t clear. She was on thin ice at Google, and they were
laying people off too, and she was convinced that being led out in
handcuffs would be the straw that broke the camel’s back.

“I should move back to Oregon,” she said, dropping her slice back
on the box-top.

Suzanne had heard a lot of people talk about giving up on the
Valley since she’d moved there. It was a common thing, being beaten
down by life in the Bay Area. You were supposed to insert a pep
talk here, something about hanging in, about the opportunities
here.

“Yes,” she said, “that’s a good idea. You’re young, and there’s a
life for you there. You can start something up, or go to work for
someone else’s startup.” It felt weird coming out of her mouth,
like a betrayal of the Valley, of some tribal loyalty to this
tech-Mecca. But after all, wasn’t she selling up and moving east?

“There’s nothing in Oregon,” Fiona said, snuffling.

“There’s something everywhere. Let me tell you about some friends
of mine in Florida,” and she told her, and as she told her, she
told herself. Hearing it spoken aloud, even after having written
about it and written about it, and been there and DONE it, it was
different. She came to understand how fucking \emph{cool} it all
was, this new, entrepreneurial, inventive, amazing thing she was
engaged in. She’d loved the contrast of nimble software companies
when compared with gigantic, brutal auto companies, but what her
boys were doing, it made the software companies look like lumbering
lummoxes, crashing around with their fifty employees and their big
purpose-built offices.

Fiona was disbelieving, then interested, then excited. “They just
make this stuff, do it, then make something else?”

“Exactly\dash{}no permanence except for the team, and they support each
other, live and work together. You’d think that because they live
and work together that they don’t have any balance, but it’s the
opposite: they book off work at four or sometimes earlier, go to
movies, go out and have fun, read books, play catch. It’s amazing.
I’m never coming back here.”

And she never would.

She told her editor about this. She told her friends who came to a
send-off party at a bar she used to go to when she went into the
office a lot. She told her cab driver who picked her up to take her
to the airport and she told the bemused engineer who sat next to
her all the way back to Miami. She had the presence of mind not to
tell the couple who bought her house for a sum of money that seemed
to have at least one extra zero at the end\dash{}maybe two.

And so when she got back to Miami, she hardly noticed the
incredible obesity of the man who took the money for the gas in her
leased car\dash{}now that she was here for the long haul she’d have to
look into getting Lester to help her buy a used Smart-car from a
junker lot\dash{}and the tin roofs of the shantytowns she passed looked
tropical and quaint. The smell of swamp and salt, the pea-soup
humidity, the bass thunder of the boom-cars in the traffic around
her\dash{}it was like some kind of sweet homecoming for her.

Tjan was in the condo when she got home and he spotted her from the
balcony, where he’d been sunning himself and helped her bring up
her suitcases of things she couldn’t bear to put in storage.

“Come down to our place for a cup of coffee once you’re settled
in,” he said, leaving her. She sluiced off the airplane grease that
had filled her pores on the long flight from San Jose to Miami and
changed into a cheap sun-dress and a pair of flip-flops that she’d
bought at the Thunderbird Flea Market and headed down to their
place.

Tjan opened the door with a flourish and she stepped in and stopped
short. When she’d left, the place had been a reflection of their
jumbled lives: gizmos, dishes, parts, tools and clothes strewn
everywhere in a kind of joyful, eye-watering hyper-mess, like an
enormous kitchen junk-drawer.

Now the place was \emph{spotless}\dash{}and what’s more, it was
\emph{minimalist}. The floor was not only clean, it was visible.
Lining the walls were translucent white plastic tubs stacked to the
ceiling.

“You like it?”

“It’s amazing,” she said. “Like Ikea meets \emph{Barbarella}. What
happened here?”

Tjan did a little two-step. “It was Lester’s idea. Have a look in
the boxes.”

She pulled a couple of the tubs out. They were jam-packed with
books, tools, cruft and crud\dash{}all the crap that had previously
cluttered the shelves and the floor and the sofa and the coffee
table.

“Watch this,” he said. He unvelcroed a wireless keyboard from the
side of the TV and began to type: T-H-E C-O. . . The field
autocompleted itself: \textuppercase{THE COUNT OF MONTE CRISTO}, and brought up a
picture of a beaten-up paperback along with links to web-stores,
reviews, and the full text. Tjan gestured with his chin and she saw
that the front of one of the tubs was pulsing with a soft blue
glow. Tjan went and pulled open the tub and fished for a second
before producing the book.

“Try it,” he said, handing her the keyboard. She began to type
experimentally: U-N and up came UNDERWEAR (14). “No way,” she
said.

“Way,” Tjan said, and hit return, bringing up a thumbnail gallery
of fourteen pairs of underwear. He tabbed over each, picked out a
pair of Simpsons boxers, and hit return. A different tub started
glowing.

“Lester finally found a socially beneficial use for RFIDs. We’re
going to get rich!”

“I don’t think I understand,” she said.

“Come on,” he said. “Let’s get to the junkyard. Lester explains
this really well.”

He did, too, losing all of the shyness she remembered, his eyes
glowing, his sausage-thick fingers dancing.

“Have you ever alphabetized your hard drive? I mean, have you ever
spent any time concerning yourself with where on your hard drive
your files are stored, which sectors contain which files? Computers
abstract away the tedious, physical properties of files and leave
us with handles that we use to persistently refer to them,
regardless of which part of the hard drive currently holds those
particular bits. So I thought, with RFIDs, you could do this with
the real world, just tag everything and have your furniture keep
track of where it is.

“One of the big barriers to roommate harmony is the correct
disposition of stuff. When you leave your book on the sofa, I have
to move it before I can sit down and watch TV. Then you come after
me and ask me where I put your book. Then we have a fight. There’s
stuff that you don’t know where it goes, and stuff that you don’t
know where it’s been put, and stuff that has nowhere to put it. But
with tags and a smart chest of drawers, you can just put your stuff
wherever there’s room and ask the physical space to keep track of
what’s where from moment to moment.

“There’s still the problem of getting everything tagged and
described, but that’s a service business opportunity, and where
you’ve got other shared identifiers like ISBNs you could use a
cameraphone to snap the bar-codes and look them up against public
databases. The whole thing could be coordinated around ’spring
cleaning’ events where you go through your stuff and photograph it,
tag it, describe it\dash{}good for your insurance and for forensics if
you get robbed, too.”

He stopped and beamed, folding his fingers over his belly. “So,
that’s it, basically.”

Perry slapped him on the shoulder and Tjan drummed his forefingers
like a heavy-metal drummer on the side of the workbench they were
gathered around.

They were all waiting for her. “Well, it’s very cool,” she said, at
last. “But, the whole white-plastic-tub thing. It makes your
apartment look like an Ikea showroom. Kind of inhumanly minimalist.
We’re Americans, we like celebrating our stuff.”

“Well, OK, fair enough,” Lester said, nodding. “You don’t have to
put everything away, of course. And you can still have all the
decor you want. This is about clutter control.”

“Exactly,” Perry said. “Come check out Lester’s lab.”

“OK, this is pretty perfect,” Suzanne said. The clutter was gone,
disappeared into the white tubs that were stacked high on every
shelf, leaving the work-surfaces clear. But Lester’s
works-in-progress, his keepsakes, his sculptures and triptychs were
still out, looking like venerated museum pieces in the stark
tidiness that prevailed otherwise.

Tjan took her through the spreadsheets. “There are ten teams that
do closet-organizing in the network, and a bunch of shippers,
packers, movers, and storage experts. A few furniture companies. We
adopted the interface from some free software inventory-management
apps that were built for illiterate service employees. Lots of big
pictures and autocompletion. And we’ve bought a hundred RFID
printers from a company that was so grateful for a new customer
that they’re shipping us 150 of them, so we can print these things
at about a million per hour. The plan is to start our sales through
the consultants at the same time as we start showing at trade-shows
for furniture companies. We’ve already got a huge order from a
couple of local old-folks’ homes.”

They walked to the IHOP to have a celebratory lunch. Being back in
Florida felt just right to her. Francis, the leader of the
paramilitary wing of the AARP, threw them a salute and blew her a
kiss, and even Lester’s nursing junkie friend seemed to be in a
good mood.

When they were done, they brought take-out bags for the junkie and
Francis in the shantytown.

“I want to make some technology for those guys,” Perry said as they
sat in front of Francis’s RV drinking cowboy coffee cooked over a
banked wood-stove off to one side. “Room-mate-ware for homeless
people.”

Francis uncrossed his bony ankles and scratched at his mosquito
bites. “A lot of people think that we don’t buy stuff, but it’s not
true,” he said. “I shop hard for bargains, but there’s lots of
stuff I spend more on because of my lifestyle than I would if I had
a real house and steady electricity. When I had a chest-freezer, I
could bulk buy ground round for about a tenth of what I pay now
when I go to the grocery store and get enough for one night’s
dinner. The alternative is using propane to keep the fridge going
overnight, and that’s not cheap, either. So I’m a kind of premium
customer. Back at Boeing, we loved the people who made small
orders, because we could charge them such a premium for custom
work, while the big airlines wanted stuff done so cheap that half
the time we lost money on the deal.”

Perry nodded. “There you have it\dash{}roommate-ware for homeless people,
a great and untapped market.”

Suzanne cocked her head and looked at him. “You’re sounding awfully
commerce-oriented for a pure and unsullied engineer, you know?”

He ducked his head and grinned and looked about twelve years old.
“It’s infectious. Those little kitchen gnomes, we sold nearly a
half-million of those things, not to mention all the spin-offs.
That’s a half-million \emph{lives}\dash{}a half-million
\emph{households}\dash{}that we changed just by thinking up something
cool and making it real. These RFID things of Lester’s\dash{}we’ll sign a
couple million customers with those. People will change everything
about how they live from moment to moment because of something
Lester thought up in my junkyard over there.”

“Well, there’s thirty million of us living in what the social
workers call ‘marginal housing,’” Francis said, grinning wryly. He
had a funny smile that Suzanne had found adorable until he
explained that he had an untreated dental abscess that he couldn’t
afford to get fixed. “So that’s a lot of difference you could
make.”

“Yeah,” Perry said. “Yeah, it sure is.”

That night, she found herself still blogging and answering
emails\dash{}they always piled up when she travelled and took a couple of
late nights to clear out\dash{}after nine PM, sitting alone in a pool of
light in the back corner of Lester’s workshop that she had staked
out as her office. She yawned and stretched and listened to her old
back crackle. She hated feeling old, and late nights made her feel
old\dash{}feel every extra ounce of fat on her tummy, feel the lines
bracketing her mouth and the little bag of skin under her chin.

She stood up and pulled on a light jacket and began to switch off
lights and get ready to head home. As she poked her head in Tjan’s
office, she saw that she wasn’t the only one working late.

“Hey, you,” she said. “Isn’t it time you got going?”

He jumped like he’d been stuck with a pin and gave a little yelp.
“Sorry,” he said, “didn’t hear you.”

He had a cardboard box on his desk and had been filling it with his
personal effects\dash{}little one-off inventions the guys had made for
him, personal fetishes and tchotchkes, a framed picture of his
kids.

“What’s up?”

He sighed and cracked his knuckles. “Might as well tell you now as
tomorrow morning. I’m resigning.”

She felt a flash of anger and then forced it down and forcibly
replaced it with professional distance and curiosity. Mentally she
licked her pencil-tip and flipped to a blank page in her reporter’s
notebook.

“Oh yes?”

“I’ve had another offer, in Westchester County. Westinghouse has
spun out its own version of Kodacell and they’re looking for a new
vice-president to run the division. That’s me.”

“Good job,” she said. “Congratulations, Mr Vice-President.”

He shook his head. “I emailed Kettlewell half an hour ago. I’m
leaving in the morning. I’m going to say goodbye to the guys over
breakfast.”

“Not much notice,” she said.

“Nope,” he said, a note of anger creeping into his voice. “My
contract lets Kodacell fire me on one day’s notice, so I insisted
on the right to quit on the same terms. Maybe Kettlewell will get
his lawyers to write better boilerplate from here on in.”

When she had an angry interview, she habitually changed the subject
to something sensitive: angry people often say more than they
intend to. She did it instinctively, not really meaning to psy-ops
Tjan, whom she thought of as a friend, but not letting that get in
the way of the story. “Westinghouse is doing what, exactly?”

“It’ll be as big as Kodacell’s operation in a year,” he said.
“George Westinghouse personally funded Tesla’s research, you know.
The company understands funding individual entrepreneurs. I’m going
to be training the talent scouts and mentoring the financial
people, then turning them loose to sign up entrepreneurs for the
Westinghouse network. There’s a competitive market for garage
inventors now.” He laughed. “Go ahead and print that,” he said.
“Blog it tonight. There’s competition now. We’re giving two points
more equity and charging half a point less on equity than the
Kodacell network.”

“That’s amazing, Tjan. I hope you’ll keep in touch with me\dash{}I’d love
to follow your story.”

“Count on it,” he said. He laughed. “I’m getting a week off every
eight weeks to scout Russia. They’ve got an incredible culture of
entrepreneurship.”

“Plus you’ll get to see your kids,” Suzanne said. “That’s really
good.”

“Plus, I’ll get to see my kids,” he admitted.

“How much money is Westinghouse putting into the project?” she
asked, replacing her notional notebook with a real one, pulled from
her purse.

“I don’t have numbers, but they’ve shut down the whole appliances
division to clear the budget for it.” She nodded\dash{}she’d seen news of
the layoffs on the wires. Mass demonstrations, people out of work
after twenty years’ service. “So it’s a big budget.”

“They must have been impressed with the quarterlies from
Kodacell.”

Tjan folded down the flaps on his box and drummed his fingers on
it, squinting at her. “You’re joking, right?”

“What do you mean?”

“Suzanne, they were impressed by \emph{you}. Everyone knows that
quarterly numbers are easy to cook\dash{}anything less than two annual
reports is as likely to be enronning as real fortune-making. But
\emph{your} dispatches from here\dash{}they’re what sold them. It’s
what’s convincing \emph{everyone}. Kettlewell said that three
quarters of his new recruits come on board after reading your
descriptions of this place. That’s how \emph{I} ended up here.”

She shook her head. “That’s very flattering, Tjan, but\dash{}”

He waved her off and then, surprisingly, came around the desk and
hugged her. “But nothing, Suzanne. Kettlewell, Lester,
Perry\dash{}they’re all basically big kids. Full of enthusiasm and
invention, but they’ve got the emotional maturity and sense of
scale of hyperactive five year olds. You and me, we’re grownups.
People take us seriously. It’s easy to get a kid excited, but when
a grownup chimes in you know there’s some there there.”

Suzanne recovered herself after a second and put away her notepad.
“I’m just the person who writes it all down. You people are making
it happen.”

“In ten years’ time, they’ll remember you and not us,” Tjan said.
“You should get Kettlewell to put you on the payroll.”

Kettlewell himself turned up the next day. Suzanne had developed an
intuitive sense of the flight-times from the west coast and so for
a second she couldn’t figure out how he could possibly be standing
there\dash{}nothing in the sky could get him from San Jose to Miami for a
seven AM arrival.

“Private jet,” he said, and had the grace to look slightly
embarrassed. “Kodak had eight of them and Duracell had five. We’ve
been trying to sell them all off but no one wants a used jet these
days, not even Saudi princes or Columbian drug-lords.”

“So, basically, it was going to waste.”

He smiled and looked eighteen\dash{}she really did feel like the only
grownup sometimes\dash{}and said, “Zackly\dash{}it’s practically environmental.
Where’s Tjan?”

“Downstairs saying goodbye to the guys, I think.”

“OK,” he said. “Are you coming?”

She grabbed her notebook and a pen and beat him out the door of her
rented condo.

“What’s this all about,” Tjan said, looking wary. The guys were
hang-dog and curious looking, slightly in awe of Kettlewell, who
did little to put them at their ease\dash{}he was staring intensely at
Tjan.

“Exit interview,” he said. “Company policy.”

Tjan rolled his eyes. “Come on,” he said. “I’ve got a flight to
catch in an hour.”

“I could give you a lift,” Kettlewell said.

“You want to do the exit interview between here and the airport?”

“I could give you a lift to JFK. I’ve got the jet warmed up and
waiting.”

Sometimes, Suzanne managed to forget that Kodacell was a
multi-billion dollar operation and that Kettlewell was at its helm,
but other times the point was very clear.

“Come on,” he said, “we’ll make a day of it. We can stop on the way
and pick up some barbecue to eat on the plane. I’ll even let you
keep your seat in the reclining position during take-off and
landing. Hell, you can turn your cell-phone on\dash{}just don’t tell the
Transport Security Administration!”

Tjan looked cornered, then resigned. “Sounds good to me,” he said
and Kettlewell shouldered one of the two huge duffel-bags that were
sitting by the door.

“Hi, Kettlewell,” Perry said.

Kettlewell set down the duffel. “Sorry, sorry. Lester, Perry, it’s
really good to see you. I’ll bring Suzanne back tonight and we’ll
all go out for dinner, OK?”

Suzanne blinked. “I’m coming along?”

“I sure hope so,” Kettlewell said.

Perry and Lester accompanied them down in the elevator.

“Private jet, huh?” Perry said. “Never been in one of those.”

Kettlewell told them about his adventures trying to sell off
Kodacell’s private air force.

“Send one of them our way, then,” Lester said.

“Do you fly?” Kettlewell said.

“No,” Perry said. “Lester wants to take it apart. Right, Les?”

Lester nodded. “Lots of cool junk in a private jet.”

“These things are worth millions, guys,” Kettlewell said.

“No, someone \emph{paid} millions for them,” Perry said. “They’re
\emph{worth} whatever you can sell them for.”

Kettlewell laughed. “You’ve had an influence around here, Tjan,” he
said. Tjan managed a small, tight smile.

Kettlewell had a driver waiting outside of the building who loaded
the duffels into the spacious trunk of a spotless dark town-car
whose doors chunked shut with an expensive sound.

“I want you to know that I’m really not angry at all, OK?”
Kettlewell said.

Tjan nodded. He had the look of a man who was steeling himself for
a turn in an interrogation chamber. He’d barely said a word since
Kettlewell arrived. For his part, Kettlewell appeared oblivious to
all of this, though Suzanne was pretty sure that he understood
exactly how uncomfortable this was making Tjan.

“The thing is, six months ago, nearly everyone was convinced that I
was a fucking moron, that I was about to piss away ten billion
dollars of other people’s money on a stupid doomed idea. Now
they’re copying me and poaching my best people. So this is good
news for me, though I’m going to have to find a new business
manager for those two before they get picked up for turning planes
into component pieces.”

Suzanne’s PDA vibrated whenever the number of online news stories
mentioning her or Kodacell or Kettlewell increased or decreased
sharply. She used to try to read everything, but it was impossible
to keep up\dash{}now all she wanted was to keep track of whether the
interestingness-index was on the uptick or downtick.

It had started to buzz that morning and the pitch had increased
steadily until it was actually uncomfortable in her pocket.
Irritated, she yanked it out and was about to switch it off when
the lead article caught her eye.

\headline{KODACELL LOSES TJAN TO WESTINGHOUSE}

The by-line was Freddy. Feeling like a character in a horror movie
who can’t resist the compulsion to look under the bed, Suzanne
thumbed the PDA’s wheel and brought up the whole article.

\begin{blog}
Kodacell business-manager Tjan Lee Tang, whose adventures
we’ve\\
followed through Suzanne Church’s gushing, besotted blog posts
\end{blog}

She looked away and reflexively reached toward the delete button.
The innuendo that she was romantically involved with one or more of
the guys had circulated on her blog’s message boards and around the
diggdots ever since she’d started writing about them. No woman
could possibly be writing about this stuff because it was
important\dash{}she had to be “with the band,” a groupie or a whore.

Combine that with Rat-Toothed Freddy’s sneering tone and she was
instantly sent into heart-thundering rage. She deleted the post and
looked out the window. Her pager buzzed some more and she looked
down. The same article, being picked up on blogs, on some of the
bigger diggdots, and an AP wire.

She forced herself to re-open it.

\begin{blog}
has been hired to head up a new business unit on behalf of
the\\ multinational giant Westinghouse. The appointment stands as
more\\ proof of Church’s power to cloud men’s minds with pretty
empty\\ words about the half-baked dot-com schemes that have
oozed out of\\ Silicon Valley and into every empty and dead
American suburb.
\end{blog}

It was hypnotic, like staring into the eyes of a serpent. Her pulse
actually thudded in her ears for a second before she took a few
deep breaths and calmed down enough to finish the article, which
was just more of the same: nasty personal attacks, sniping, and
innuendo. Freddy even managed to imply that she was screwing all of
them\dash{}and Kettlewell besides.

Kettlewell leaned over her shoulder and read.

“You should send him an email,” he said. “That’s disgusting. That’s
not reportage.”

“Never get into a pissing match with a skunk,” she said. “What
Freddy wants is for me to send him mail that he can publish along
with more snarky commentary. When the guy you’re arguing with
controls the venue you’re arguing in, you can’t possibly win.”

“So blog him,” Kettlewell said. “Correct the record.”

“The record is correct,” she said. “It’s never been incorrect. I’ve
written an exhaustive record that is there for everyone to see. If
people believe this, no amount of correction will help.”

Kettlewell made a face like a little boy who’d been told he
couldn’t have a toy. “That guy is poison,” he said. “Those
quote-marks around blog.”

“Let him add his quote-marks,” she said. “My daily readership is
higher than the Merc’s paid circulation this week.” It was true.
After a short uphill climb from her new URL, she’d accumulated
enough readers that the advertising revenue dwarfed her old salary
at the Merc, an astonishing happenstance that nevertheless kept her
bank-account full. She clicked a little. “Besides, look at this,
there are three dozen links pointing at this story so far and all
of them are critical of him. We don’t need to stick up for
ourselves\dash{}the world will.”

Saying it calmed her and now they were at the airport. They cruised
into a private gate, away from the militarized gulag that fronted
Miami International. A courteous security guard waved them through
and the driver confidently piloted the car up to a wheeled jetway
beside a cute, stubby little toy jet. On the side, in cursive
script, was the plane’s name: Suzanne.

She looked accusatorially at Kettlewell.

“It was called that when I bought the company,” he said,
expressionless but somehow mirthful behind his curved surfer
shades. “But I kept it because I liked the private joke.”

“Just no one tell Freddy that you’ve got an airplane with my name
on it or we’ll never hear the fucking end of it.”

She covered her mouth, regretting her language, and Kettlewell
laughed, and so did Tjan, and somehow the ice was broken between
them.

“No \emph{way} flying this thing is cost-effective,” Tjan said.
“Your CFO should be kicking your ass.”

“It’s a little indulgence,” Kettlewell said, bounding up the steps
and shaking hands with a small, neat woman pilot, an
African-American with corn-rows peeking out under her smart peaked
cap. “Once you’ve flown in your own bird, you never go back.”

“This is a \emph{monstrosity},” Tjan said as he boarded. “What this
thing eats up in hangar fees alone would be enough to bankroll
three or four teams.” He settled into an oversized Barcalounger of
a seat and accepted a glass of orange juice that the pilot poured
for him. “Thank you, and no offense.”

“None taken,” she said. “I agree one hundred percent.”

“See,” Tjan said.

Suzanne took her own seat and her own glass and buckled in and
watched the two of them, warming up for the main event, realizing
that she’d been brought along as a kind of opening act.

“They paying you more?”

“Yup,” Tjan said. “All on the back-end. Half a point on every
dollar brought in by a team I coach or whose members I mentor.”

Kettlewell whistled. “That’s a big share,” he said.

“If I can make my numbers, I’ll take home a million this year.”

“You’ll make those numbers. Good negotiations. Why didn’t you ask
us for the same deal?”

“Would you have given it to me?”

“You’re a star,” Kettlewell said, nodding at Suzanne, whose
invisibility to the conversation popped like a bubble. “Thanks to
her.”

“Thanks, Suzanne,” Tjan said.

Suzanne blushed. “Come on, guys.”

Tjan shook his head. “She doesn’t really understand. It’s actually
kind of charming.”

“We might have matched the offer.”

“You guys are first to market. You’ve got a lot of procedures in
place. I wanted to reinvent some wheels.”

“We’re too \emph{conservative} for you?”

Tjan grinned wickedly. “Oh yes,” he said. “I’m going to do business
in \emph{Russia}.”

Kettlewell grunted and pounded his orange juice. Around them, the
jet’s windows flashed white as they broke through the clouds and
the ten thousand foot bell sounded.

“How the hell are you going to make anything that doesn’t collapse
under its own weight in Russia?”

“The corruption’s a problem, sure,” Tjan said. “But it’s offset by
the entrepreneurship. Some of those cats make the Chinese look lazy
and unimaginative. It’s a shame that so much of their efforts have
been centered on graft, but there’s no reason they couldn’t be
focused on making an honest ruble.”

They fell into a discussion of the minutiae of Perry and Lester’s
businesses, franker than any business discussion she’d ever heard.
Tjan talked about the places where they’d screwed up, and places
where they’d scored big, and about all the plans he’d made for
Westinghouse, the connections he had in Russia. He even talked
about his kids and his ex in St Petersburg, and Kettlewell admitted
that he’d known about them already.

For Kettlewell’s part, he opened the proverbial kimono wide,
telling Tjan about conflicts within the board of directors,
poisonous holdovers from the pre-Kodacell days who sabotaged the
company from within with petty bureaucracy, even the problems he
was having with his family over the long hours they were working.
He opened the minibar and cracked a bottle of champagne to toast
Tjan’s new job, and they mixed it with more orange juice, and then
there were bagels and schmear, fresh fruit, power bars, and canned
Starbucks coffees with deadly amounts of sugar and caffeine.

When Kettlewell disappeared into the tiny\dash{}but
marble-appointed\dash{}bathroom, Suzanne found herself sitting alone with
Tjan, almost knee to knee, lightheaded from lack of sleep and
champagne and altitude.

“Some trip,” she said.

“You’re the best,” he said, wobbling a little. “You know that? Just
the best. The stuff you write about these guys, it makes me want to
stand up and salute. You make us all seem so fucking
\emph{glorious}. We’re going to end up taking over the world
because you inspire us so. Maybe I shouldn’t tell you this, because
you’re not very self-conscious about it right now, but Suzanne, you
won’t believe it because you’re so goddamned modest, too. It’s what
makes your writing so right, so believable\dash{}”

Kettlewell stepped out of the bathroom. “Touching down soon,” he
said, and patted them each on the shoulder as he took his seat. “So
that’s about it, then,” he said, and leaned back and closed his
eyes. Suzanne was accustomed to thinking of him as
twenty-something, the boyish age of the magazine cover portraits
from the start of his career. Now, eyes closed on his private jet,
harsh upper atmosphere sun painting his face, his crowsfeet and the
deep vertical brackets around his mouth revealed him for someone
pushing a youthful forty, kept young by exercise and fun and the
animation of his ideas.

“Guess so,” Tjan said, slumping. “This has been one of the more
memorable experiences of my life, Kettlewell, Suzanne. Not entirely
pleasant, but pleasant on the whole. A magical time in the
clouds.”

“Once you’ve flown private, you’ll never go back to coach,”
Kettlewell said, smiling, eyes still closed. “You still think my
CFO should spank me for not selling this thing?”

“No,” Tjan said. “In ten years, if we do our jobs, there won’t be
five companies on earth that can afford this kind of thing\dash{}it’ll be
like building a cathedral after the Protestant Reformation. While
we have the chance, we should keep these things in the sky. But you
should give one to Lester and Perry to take apart.”

“I was planning to,” Kettlewell said. “Thanks.”

Suzanne and Kettlewell got off the plane and Tjan didn’t look back
when they’d landed at JFK. “Should we go into town and get some
bialy to bring back to Miami?” Kettlewell said, squinting at the
bright day on the tarmac.

“Bring deli to Miami?”

“Right, right,” he said. “Forget I asked. Besides, we’d have to
charter a chopper to get into Manhattan and back without dying in
traffic.”

Something about the light through the open hatch or the sound or
the smell\dash{}something indefinably New York\dash{}made her yearn for Miami.
The great cities of commerce like New York and San Francisco seemed
too real for her, while the suburbs of Florida were a kind of
endless summer camp, a dreamtime where anything was possible.

“Let’s go,” she said. The champagne buzz had crashed and she had a
touch of headache. “I’m bushed.”

“Me too,” Kettlewell said. “I left San Jose last night to get into
Miami before Tjan left. Not much sleep. Gonna put my seat back and
catch some winks, if that’s OK?”

“Good plan,” Suzanne said.

Embarrassingly, when they were fully reclined, their seats nearly
touched, forming something like a double bed. Suzanne lay awake in
the hum of the jets for a while, conscious of the breathing human
beside her, the first man she’d done anything like share a bed with
in at least a year. The last thing she remembered was the ten
thousand foot bell going off and then she slipped away into sleep.

\begin{center}\rule{3in}{0.4pt}\end{center}

\begin{blog}
Perry thought that they’d sell a million Home Awares in six\\
months. Lester thought he was nuts, that number was too
high.\blogpar{} “Please,” he said, “I \emph{invented} these things
but there aren’t a\\ million roommate households in all of
America. We’ll sell half\\ a million tops, total.
\end{blog}

Lester always complained when she quoted him directly in her blog
posts, but she thought he secretly enjoyed it.

\begin{blog}
Today the boys shipped their millionth unit. It took six weeks.
\end{blog}

They’d uncorked a bottle of champagne when unit one million
shipped. They hadn’t actually shipped it, per se. The manufacturing
was spread out across forty different teams all across the country,
even a couple of Canadian teams. The RFID printer company had
re-hired half the workers they’d laid off the year before, and had
them all working overtime to meet demand.

\begin{blog}
What’s exciting about this isn’t just the money that these
guys\\ have made off of it, or the money that Kodacell will
return to\\ its shareholders, it’s the ecosystem that these
things have\\ enabled. There’re at least ten competing commercial
systems for\\ organizing, tagging, sharing, and describing Home
Aware objects.\\ Parents love them for their kids. School
teachers love them.\\ Seniors’ homes.
\end{blog}

The seniors’ homes had been Francis’s idea. They’d brought him in
to oversee some of the production engineering, along with some of
the young braves who ran around the squatter camps. Francis knew
which ones were biddable and he kept them to heel. In the evenings,
he’d join the guys and Suzanne up on the roof of the workshop on
folding chairs, with beers, watching the sweaty sunset.

\begin{blog}
They’re not the sole supplier. That’s what an ecosystem is
all\\ about, creating value for a lot of players. All this
competition\\ is great news for you and me, because it’s already
driven the\\ price of Home Aware goods down by forty percent.
That means that\\ Lester and Perry are going to have to invent
something new, soon,\\ before the margin disappears
altogether\dash{}and that’s also good\\ news for you and me.
\end{blog}

“Are you coming?” Lester had dated a girl for a while, someone he
met on Craigslist, but she’d dumped him and Perry had confided that
she’d left him because he didn’t live up to the press he’d gotten
in Suzanne’s column. When he got dumped, he became even touchier
about Suzanne, caught at a distance from her that was defined by
equal parts of desire and resentment.

“Up in a minute,” she said, trying to keep her smile light and
noncommittal. Lester was very nice, but there were times when she
caught him staring at her like a kicked puppy and it made her
uncomfortable. Naturally, this increased his discomfort as well.

On the roof they already had a cooler of beers going and beside it
a huge plastic tub of brightly colored machine-parts.

“Jet engine,” Perry said. The months had put a couple pounds on him
and new wrinkles, and given him some grey at the temples, and laugh
lines inside his laugh lines. Perry was always laughing at
everything around them (“They fucking \emph{pay me} to do this,”
he’d told her once, before literally collapsing to the floor,
rolling with uncontrollable hysteria). He laughed again.

“Good old Kettlebelly,” she said. “Must have broken his heart.”

Francis held up a curved piece of cowling. “This thing wasn’t going
to last anyway. See the distortion here and here? This thing was
designed in a virtual wind-tunnel and machine-lathed. We tried that
a couple times, but the wind-tunnel sims were never detailed enough
and the forms that flew well in the machine always died a premature
death in the sky. Another two years and he’d have had to have it
rebuilt anyway, and the Koreans who built this charge shitloads for
parts.”

“Too bad,” Lester said. “It’s pretty. Gorgeous, even.” He mimed its
curve in the air with a pudgy hand, that elegant swoop.

“Aerospace loves the virtual wind-tunnel,” Francis said, and glared
at the cowling. “You can use evolutionary algorithms in the sim and
come up with really efficient designs, in theory. And computers are
cheaper than engineers.”

“Is that why you were laid off?” Suzanne said.

“I wasn’t laid off, girl,” he said. He jiggled his lame foot. “I
retired at 65 and was all set up but the pension plan went bust. So
I missed a month of medical and they cut me off and I ended up
uninsured. When the wife took sick, bam, that was it, wiped right
out. But I’m not bitter\dash{}why should the poor be allowed to live,
huh?”

His acolytes, three teenagers in do-rags from the shantytown,
laughed and went on to pitching bottle-caps off the edge of the
roof.

“Stop that, now,” he said, “you’re getting the junkyard all dirty.
Christ, you’d think that they grew up in some kind of zoo.” When
Francis drank, he got a little mean, a little dark.

“So, kids,” Perry said, wandering over to them, hands in pockets.
Silhouetted against the setting sun, biceps bulging, muscular chest
tapering to his narrow hips, he looked like a Greek statue. “What
do you think of the stuff we’re building?”

They looked at their toes. “’S OK,” one of them grunted.

“Answer the man,” Francis snapped. “Complete sentences, looking up
and at him, like you’ve got a shred of self-respect. Christ, what
are you, five years old?”

They shifted uncomfortably. “It’s fine,” one of them said.

“Would you use it at home?”

One of them snorted. “No, man. My dad steals anything nice we get
and sells it.”

“Oh,” Perry said.

“Fucker broke in the other night and I caught him with my ipod.
Nearly took his fucking head off with my cannon before I saw who it
was. Fucking juice-head.”

“You should have fucked him up,” one of the other kids said. “My ma
pushed my pops in front of a bus one day to get rid of him, guy
broke both his legs and never came back.”

Suzanne knew it was meant to shock them, but that didn’t take away
from its shockingness. In the warm fog of writing and living in
Florida, it was easy to forget that these people lived in a
squatter camp and were technically criminals, and received no
protection from the law.

Perry, though, just squinted into the sun and nodded. “Have you
ever tried burglar alarms?”

The kids laughed derisively and Suzanne winced, but Perry was
undaunted. “You could be sure that you woke up whenever anyone
entered, set up a light and siren to scare them off.”

“I want one that fires spears,” the one with the juice-head father
said.

“Blowtorches,” said the one whose mother pushed his father under a
bus.

“I want a force-field,” the third one said, speaking for the first
time. “I want something that will keep anyone from coming in,
period, so I don’t have to sleep one eye up, ’cause I’ll be safe.”

The other two nodded, slowly.

“Damn straight,” Francis said.

That was the last time Francis’s acolytes joined them on the
rooftop. Instead, when they finished work they went home, walking
slowly and talking in low murmurs. With just the grownups on the
roof, it was a lot more subdued.

“What’s that smoke?” Lester said, pointing at the black billowing
column off to the west, in the sunset’s glare.

“House-fire,” Francis said. “Has to be. Or a big fucking car-wreck,
maybe.”

Perry ran down the stairs and came back up with a pair of
high-power binox. “Francis, that’s your place,” he said after a
second’s fiddling. He handed the binox to Francis. “Just hit the
button and they’ll self-stabilize.”

“That’s my place,” Francis said. “Oh, Christ.” He’d gone gray and
seemed to have sobered up instantly. His lips were wet, his eyes
bright.

They drove over at speed, Suzanne wedged into Lester’s
frankensmartcar, practically under his armpit, and Perry traveling
with Francis. Lester still wore the same cologne as her father, and
when she opened the window, its smell was replaced by the
burning-tires smell of the fire.

They arrived to discover a fire-truck parked on the side of the
freeway nearest the shantytown. The fire-fighters were standing
soberly beside it, watching the fire rage across the canal.

They rushed for the footbridge and a firefighter blocked their
way.

“Sorry, it’s not safe,” he said. He was Latino, good looking, like
a movie star, bronze skin flickering with copper highlights from
the fire.

“I live there,” Francis said. “That’s my home.”

The firefighter looked away. “It’s not safe,” he said.

“Why aren’t you fighting the fire?” Suzanne said.

Francis’s head snapped around. “You’re not fighting the fire!
You’re going to let our houses burn!”

A couple more fire-fighters trickled over. Across the river, the
fire had consumed half of the little settlement. Some of the
residents were operating a slow and ponderous bucket-brigade from
the canal, while others ran into the unburned buildings and emerged
clutching armloads of belongings, bits of furniture, boxes of
photos.

“Sir,” the movie-star said, “the owner of this property has asked
us not to intervene. Since there’s no imminent risk to life and no
risk of the burn spreading off his property, we can’t trespass to
put out the fire. Our hands are tied.”

“The owner?” Francis spat. “This land is in title dispute. The
court case has been underway for twenty years now. What owner?”

The movie-star shrugged. “That’s all I know, sir.”

Across the canal, the fire was spreading, and the bucket brigade
was falling back. Suzanne could feel the heat now, like putting
your face in the steam from a boiling kettle.

Francis seethed, looking from the firemen and their truck back to
the fire. He looked like he was going to pop something, or start
shouting, or charge into the flames.

Suzanne grabbed his hand and walked him over to the truck and
grabbed the first firefighter she encountered.

“I’m Suzanne Church, from the San Jose Mercury News, a McClatchy
paper. I’d like to speak to the commanding officer on the scene,
please.” She hadn’t been with the Merc for months, but she hadn’t
been able to bring herself to say,
\emph{I’m Suzanne Church with SuzanneChurch.org.} She was pretty
sure that no matter how high her readership was and how profitable
her ad sales were, the fire-fighter wouldn’t have been galvanized
into the action that was invoked when she mentioned the name of a
real newspaper.

He hopped to, quickly moving to an older man, tapping him on the
shoulder, whispering in his ear. Suzanne squeezed Francis’s hand as
the fire-chief approached them. She extended her hand and talked
fast. “Suzanne Church,” she said, and took out her notebook, the
key prop in any set piece involving a reporter. “I’m told that you
are going to let those homes burn because someone representing
himself as the title-holder to that property has denied you entry.
However, I’m also told that the title to that land is in dispute
and has been in the courts for decades. Can you resolve this for
me, Chief\ldots{}?”

“Chief Brian Wannamaker,” he said. He was her age, with the
leathery skin of a Florida native who spent a lot of time out of
doors. “I’m afraid I have no comment for you at this time.”

Suzanne kept her face deadpan, and gave Francis’s hand a warning
squeeze to keep him quiet. He was trembling now. “I see. You can’t
comment, you can’t fight the fire. Is that what you’d like me to
write in tomorrow’s paper?”

The Chief looked at the fire for a moment. Across the canal, the
bucket-brigaders were losing worse than ever. He frowned and
Suzanne saw that his hands were clenched into fists. “Let me make a
call, OK?” Without waiting for an answer, he turned on his heel and
stepped behind the fire-engine, reaching for his cellphone.

Suzanne strained to hear his conversation, but it was inaudible
over the crackle of the fire. When she turned around again, Francis
was gone. She caught sight of him again in just a moment, running
for the canal, then jumping in and landing badly in the shallow,
swampy water. He hobbled across to the opposite bank and began to
laboriously climb it.

A second later, Perry followed. Then Lester.

“Chief!” she said, going around the engine and pointing. The Chief
had the phone clamped to his head still, but when he saw what was
going on, he snapped it shut, dropped it in his pocket and started
barking orders.

Now the fire-fighters \emph{moved}, boiling across the bridge,
uncoiling hoses, strapping on tanks and masks. They worked in easy,
fluid concert, and it was only seconds before the water and foam
hit the flames and the smoke changed to white steam.

The shantytown residents cheered. The fire slowly receded. Perry
and Lester had Francis, holding him back from charging into the
fray as the fire-fighters executed their clockwork dance.

The steam was hot enough to scald, and Suzanne pulled the collar of
her blouse up over her face. Around her were the shantytowners,
mothers with small children, old men, and a seemingly endless
parade of thug-life teenagers, the boys in miniature cycling shorts
and do-rags, the girls in bandeau tops, glitter makeup, and skirts
made from overlapping strips of rag, like post-apocalyptic hula
outfits. Their faces were tight, angry, smudged with smoke and
pinkened by the heat.

She saw the one whose father had reportedly been pushed under a bus
by his mother, and he grimaced at her. “What we gonna do now?”

“I don’t know,” she said. “Are you all right? Is your family all
right?”

“Don’t got nowhere to sleep, nowhere to go,” he said. “Don’t even
have a change of clothes. My moms won’t stop crying.”

There were tears in his eyes. He was all of fifteen, she realized.
He’d seemed much older on the roof. She gathered him into her arms
and gave him a hug. He was stiff and awkward at first and then he
kind of melted into her, weeping on her shoulder. She stroked his
back and murmured reassuringly. Some of the other shantytowners
looked at the spectacle, then looked away. Even a couple of his
homeboys\dash{}whom she’d have bet would have laughed and pointed at this
show of weakness\dash{}only looked and then passed on. One had tears
streaking the smoke smudges on his face.

\emph{For someone who isn’t good at comforting people, I seem to be doing a lot of it},
she thought.

Francis and Lester and Perry found her and Francis gave the boy a
gruff hug and told him everything would be fine.

The fire was out now, the firefighter hosing down the last embers,
going through the crowd and checking for injuries. A TV news crew
had set up and a pretty black reporter in her twenties was doing a
stand up.

“The illegal squatter community has long been identified as a
problem area for gang and drug activity by the Broward County
Sheriff’s office. The destruction here seems total, but it’s
impossible to say whether this spells the end of this encampment,
or whether the denizens will rebuild and stay on.”

Suzanne burned with shame. That could have been her. When she’d
first seen this place, it had been like something out of a
documentary on Ethiopia. As she’d come to know it, it had grown
homier. The residents built piecemeal, one wall at a time, one
window, one poured concrete floor, as they could afford it. None of
them had mortgages, but they had neat vegetable gardens and
walkways spelled out in white stones with garden gnomes standing
guard.

The reporter was staring at her\dash{}and naturally so; she’d been
staring at the reporter. Glaring at her.

“My RV,” Francis said, pointing, distracting her. It was a charred
wreck. He went to the melted doors and opened them, stepping back
as a puff of smoke rose from the inside. A fire-fighter spotted it
and diverted a stream of water into the interior, soaking Francis
and whatever hadn’t burned. He turned and shouted something at the
fire-fighter, but he was already hosing down something else.

Inside Francis’s trailer, they salvaged a drenched photo-album, a
few tools, and a lock-box with some of his papers in it. He had
backed up his laptop to his watch that morning, so his data was
safe. “I kept meaning to scan these in,” he said, paging through
the photos in the soaked album. “Should have done it.”

Night was falling, the mosquitoes singing and buzzing. The neat
little laneways and homey, patchwork buildings lay in ruins around
them.

The shantytowners clustered in little groups or picked through the
ruins. Drivers of passing cars slowed down to rubberneck, and a few
shouted filthy, vengeful things at them. Suzanne took pictures of
their license plates. She’d publish them when she got home.

A light drizzle fell. Children cried. The swampy sounds of cicadas
and frogs and mosquitoes filled the growing dark and then the
streetlights flicked on all down the river of highway, painting
everything in blue-white mercury glow.

“We’ve got to get tents up,” Francis said. He grabbed a couple of
young men and gave them orders, things to look for\dash{}fresh water,
plastic sheeting, anything with which to erect shelters.

Lester started to help them, and Perry stood with his hands on his
hips, next to Suzanne.

“Jesus Christ,” he said. “This is a fucking disaster. I mean, these
people are used to living rough, but this\dash{}” he broke off, waving
his hands helplessly. He wiped his palms off on his butt, then
grabbed Francis.

“Get them going,” he said. “Get them to gather up their stuff and
walk them down to our place. We’ve got space for everyone for now
at least.”

Francis looked like he was going to say something, then he stopped.
He climbed precariously up on the hood of Lester’s car and shouted
for people to gather round. The boys he bossed around took up the
call and it wasn’t long before nearly everyone was gathered around
them.

“Can everyone hear? This is as loud as I go.”

There were murmurs of assent. Suzanne had seen him meet with his
people before in the daylight and the good times, seen the respect
they afforded to him. He wasn’t the leader, per se, but when he
spoke, people listened. It was a characteristic she’d encountered
in the auto-trade and in technology, in the ones the others all
gravitated to. Charismatics.

“We’ve got a place to stay a bit up the road for tonight. It’s
about a half hour walk. It’s indoors and there’s toilets, but maybe
not much to make beds out of. Take what you can carry for about a
mile, you can come back tomorrow for the rest. You don’t have to
come, but this isn’t going to be any fun tonight.”

A woman came forward. She was young, but not young enough to be a
homegirl. She had long dark hair and she twisted her hands as she
spoke in a soft voice to Francis. “What about our stuff? We can’t
leave it here tonight. It’s all we’ve got.”

Francis nodded. “We need ten people to stand guard in two shifts of
five tonight. Young people. You’ll get flashlights and phones,
coffee and whatever else we can give you. Just keep the
rubberneckers out.” The rubberneckers were out of earshot. The
account they’d get of this would come from the news-anchor who’d
tell them how dangerous and dirty this place was. They’d never see
what Suzanne saw, ten men and women forming up to one side of the
crowd. Young braves and homegirls, people her age, their faces
solemn.

Francis oversaw the gathering up of belongings. Suzanne had never
had a sense of how many people lived in the shantytown but now she
could count them as they massed up by the roadside and began to
walk: a hundred, a little more than a hundred. More if you counted
the surprising number of babies.

Lester conferred briefly with Francis and then Francis tapped three
of the old timers and two of the mothers with babes in arms and
they crammed into Lester’s car and he took off. Suzanne walked by
the roadside with the long line of refugees, listening to their
murmuring conversation, and in a few minutes, Lester was back to
pick up more people, at Francis’s discretion.

Perry was beside her now, his eyes a million miles away.

“What now?” she said.

“We put them in the workshop tonight, tomorrow we help them build
houses.”

“At your place? You’re going to let them stay?”

“Why not? We don’t use half of that land. The landlord gets his
check every month. Hasn’t been by in five years. He won’t care.”

She took a couple more steps. “Perry, I’m going to write about
this,” she said.

“Oh,” he said. They walked further. A small child was crying. “Of
course you are. Well, fuck the landlord. I’ll sic Kettlewell on him
if he squawks.”

“What do you think Kettlewell will think about all this?”

“This? Look, this is what I’ve been saying all along. We need to
make products for these people. They’re a huge untapped market.”

What she wanted to ask was \emph{What would Tjan say about this?}
but they didn’t talk about Tjan these days. Kettlewell had promised
them a new business manager for weeks, but none had appeared. Perry
had taken over more and more of the managerial roles, and was
getting less and less workshop time in. She could tell it
frustrated him. In her discussions with Kettlewell, he’d confided
that it had turned out to be harder to find suits than it was
finding wildly inventive nerds. Lots of people \emph{wanted} to run
businesses, but the number who actually seemed likely to be capable
of doing so was only a small fraction.

They could see the junkyard now. Perry pulled out his phone and
called his server and touch-toned the codes to turn on all the
lights and unlock all the doors.

They lost a couple of kids in the aisles of miraculous junk, and
Francis had to send out bigger kids to find them and bring them
back, holding the treasures they’d found to their chests. Lester
kept going back for more old-timers, more mothers, more stragglers,
operating his ferry service until they were all indoors in the
workshop.

“This is the place,” Francis said. “We’ll stay indoors here
tonight. Toilets are there and there\dash{}orderly lines, no shoving.”

“What about food?” asked a man with a small boy sleeping over his
shoulder.

“This isn’t the Red Cross, Al,” Francis snapped. “We’ll organize
food for ourselves in the morning.”

Perry whispered in his ear. Francis shook his head, and Perry
whispered some more.

“There will be food in the morning. This is Perry. It’s his place.
He’s going to go to Costco for us when they open.”

The crowd cheered and a few of the women hugged him. Some of the
men shook his hand. Perry blushed. Suzanne smiled. These people
were good people. They’d been through more than Suzanne could
imagine. It felt right that she could help them\dash{}like making up for
every panhandler she’d ignored and every passed-out drunk she’d
stepped over.

There were no blankets, there were no beds. The squatters slept on
the concrete floor. Young couples spooned under tables. Children
snuggled between their parents, or held onto their mothers. As the
squatters dossed down and as Suzanne walked past them to get to her
car her heart broke a hundred times. She felt like one of those
Depression-era photographers walking through an Okie camp, a
rending visual at each corner.

Back at her rented condo, she found herself at the foot of her
comfortable bed with its thick duvet\dash{}she liked keeping the AC
turned up enough to snuggle under a blanket\dash{}and the four pillows.
She was in her jammies, but she couldn’t climb in between those
sheets.

She couldn’t.

And then she was back in her car with all her blankets, sheets,
pillows, big towels\dash{}even the sofa cushions, which the landlord was
not going to be happy about\dash{}and speeding back to the workshop.

She let herself in and set about distributing the blankets and
pillows and towels, picking out the families, the old people. A
woman\dash{}apparently able-bodied and young, but skinny\dash{}sat up and said,
“Hey, where’s one for me?” Suzanne recognized the voice. The junkie
from the IHOP. Lester’s friend. The one who’d grabbed her and
cursed her.

She didn’t want to give the woman a blanket. She only had two left
and there were old people lying on the bare floor.

“Where’s one for me?” the woman said more loudly. Some of the
sleepers stirred. Some of them sat up.

Suzanne was shaking. Who the hell was she to decide who got a
blanket? Did being rude to her at the IHOP disqualify you from
getting bedding when your house burned down?

Suzanne gave her a blanket, and she snatched one of the sofa
cushions besides.

\emph{It’s why she’s still alive,} Suzanne thought.
\emph{How she’s survived.}

She gave away the last blanket and went home to sleep on her naked
bed underneath an old coat, a rolled-up sweater for a pillow. After
her shower, she dried herself on tee-shirts, having given away all
her towels to use as bedding.

The new shantytown went up fast\dash{}faster than she’d dreamed possible.
The boys helped. Lester downloaded all the information he could
find on temporary shelters\dash{}building out of mud, out of sandbags,
out of corrugated cardboard and sheets of plastic\dash{}and they tried
them all. Some of the houses had two or more rickety-seeming
stories, but they all felt solid enough as she toured them,
snapping photos of proud homesteaders standing next to their
handiwork.

Little things went missing from the workshops\dash{}tools, easily pawned
books and keepsakes, Perry’s wallet\dash{}and they all started locking
their desk-drawers. There were junkies in among the squatters, and
desperate people, and immoral people, them too. One day she found
that her cute little gold earrings weren’t beside her desk-lamp,
where she’d left them the night before and she practically burst
into tears, feeling set-upon on all sides.

She found the earrings later that day, at the bottom of her purse,
and that only made things worse. Even though she hadn’t voiced a
single accusation, she’d accused every one of the squatters in her
mind that day. She found herself unable to meet their eyes for the
rest of the week.

“I have to write about this,” she said to Perry. “This is part of
the story.” She’d stayed clear of it for a month, but she couldn’t
go on writing about the successes of the Home Aware without writing
about the workforce that was turning out the devices and add-ons by
the thousands, all around her, in impromptu factories with
impromptu workers.

“Why?” Perry said. He’d been a dervish, filling orders, training
people, fighting fires. By nightfall, he was hollow-eyed and
snappish. Lester didn’t join them on the roof anymore. He liked to
hang out with Francis and some of the young men and pitch
horseshoes down in the shantytown, or tinker with the composting
toilets he’d been installing at strategic crossroads through the
town. “Can’t you just concentrate on the business?”

“Perry, this \emph{is} the business. Kettlewell hasn’t sent a
replacement for Tjan and you’ve filled in and you’ve turned this
place into something like a worker-owned co-op. That’s important
news\dash{}the point of this exercise is to try all the different
businesses that are possible and see what works. If you’ve found
something that works, I should write about it. Especially since
it’s not just solving Kodacell’s problem, it’s solving the problem
for all of those people, too.”

Perry drank his beer in sullen silence. “I don’t want Kettlewell to
get more involved in this. It’s going good. Scrutiny could kill
it.”

“You’ve got nothing to be embarrassed about here,” she said.
“There’s nothing here that isn’t as it should be.”

Perry looked at her for a long moment. He was at the end of his
fuse, trying to do too much, and she regretted having brought it
up. “You do what you have to do,” he said.

\begin{blog}
The original shantytown was astonishing. Built around a nexus
of\\ trailers and RVs that didn’t look in the least roadworthy,
the\\ settlers had added dwelling on dwelling to their little
patch of\\ land. They started with plastic sheeting and poles,
and when they\\ could afford it, they replaced the sheets, one at
a time, with\\ bricks or poured concrete and re-bar. They
thatched their roofs\\ with palm-leaves, shingles, linoleum,
corrugated tin\dash{}even\\ plywood with flattened beer-cans. Some
walls were wood. Some had\\ windows. Some were made from old
car-doors, with hand-cranked\\ handles to lower them in the day,
then roll them up again at\\ night when the mosquitoes came out.
Most of the settlers slept on\\ nets.\blogpar{} A second wave had
moved into the settlement, just as I arrived,\\ and rather than
building out\dash{}and farther away from their\\ neighbors’ latrines,
water-pump and mysterious sources of\\ electrical power\dash{}they
built \emph{up}, on top of the existing\\ structures, shoring up
the walls where necessary. It wasn’t\\ hurricane proof, but
neither are the cracker-box condos that\\ “property owners”
occupy. They made contractual arrangements with\\ the dwellers of
the first stories, paid them rent. A couple with\\ second-story
rooms opposite one another in one of the narrow\\ “streets”
consummated their relationship by building a sky-bridge\\ between
their rooms, paying joint rent to two landlords.\blogpar{} The thing
these motley houses had in common, all of them, was\\ ingenuity
and pride of work. They had neat vegetable gardens,\\
flower-boxes, and fresh paint. They had kids’ bikes leaned up\\
against their walls, and the smell of good cooking in the air.\\
They were homely homes.\blogpar{} Many of the people who lived in
these houses worked regular\\ service jobs, walking three miles
to the nearest city bus stop\\ every morning and three miles back
every evening. They sent\\ their kids to school, faking local
addresses with PO boxes. Some\\ were retired. Some were just down
on their luck.\blogpar{} They helped each other. When something
precious was stolen, the\\ community pitched in to find the
thieves. When one of them\\ started a little business selling
sodas or sandwiches out of her\\ shanty, the others patronized
her. When someone needed medical\\ care, they chipped in for a
taxi to the free clinic, or someone\\ with a working car drove
them. They were like the neighbors of\\ the long-lamented
American town, an ideal of civic virtue that is\\ so remote in
our ancestry as to have become mythical. There were\\ eyes on the
street here, proud residents who knew what everyone\\ was about
and saw to it that bad behavior was curbed before it\\ could get
started.\blogpar{} Somehow, it burned down. The fire department won’t
investigate,\\ because this was an illegal homestead, so they
don’t much care\\ about how the fire started. It took most of the
homes, and most\\ of their meager possessions. The water got the
rest. The fire\\ department wouldn’t fight the fire at first,
because someone at\\ city hall said that the land’s owner
wouldn’t let them on the\\ property. As it turns out, the owner
of that sad strip of land\\ between an orange grove and the side
of a four-lane highway is\\ unknown\dash{}a decades-old dispute over
title has left it in legal\\ limbo that let the squatters settle
there. It’s suspicious all\\ right\dash{}various entities had tried to
evict the squatters\\ before, but the legal hassles left them in
happy limbo. What the\\ law couldn’t accomplish, the fire
did.\blogpar{} The story has a happy ending. The boys have moved the
squatters\\ into their factory, and now they have “live-work”
condos that\\ look like something Dr Seuss designed [photo
gallery]. Like the\\ Central Park shantytown of the last century,
these look like they\\ were “constructed by crazy poets and
distributed by a whirlwind\\ that had been drinking,” as a press
account of the day had it.\blogpar{} Last year, the city completed a
new housing project nearby to\\ here, and social workers
descended on the shantytowners to get\\ them to pick up and move
to these low-rent high-rises. The\\ shantytowners wouldn’t go:
“It was too expensive,” said Mrs X,\\ who doesn’t want her family
back in Oklahoma to know she’s\\ squatting with her husband and
their young daughter. “We can’t\\ afford \emph{any} rent, not if
we want to put food on the table on\\ what we earn.”\blogpar{} She
made the right decision: the housing project is an urban\\
renewal nightmare, filled with crime and junkies, the kind of\\
place where little old ladies triple-chain their doors and
order\\ in groceries that they pay for with direct debit,
unwilling to\\ keep any cash around.\blogpar{} The squatter village
was a shantytown, but it was no slum. It was\\ a neighborhood
that could be improved. And the boys are doing\\ that: having
relocated the village to their grounds, they’re\\ inventing and
remixing new techniques for building cheap and\\ homey shelter
fast. [profile: ten shanties and the technology\\ inside them]
\end{blog}

The response was enormous and passionate. Dozens of readers wrote
to tell her that she’d been taken in by these crooks who had stolen
the land they squatted. She’d expected that\dash{}she’d felt that way
herself, when she’d first walked past the shantytown.

But what surprised her more were the message-board posts and emails
from homeless people who’d been living in their cars, on the
streets, in squatted houses or in shanties. To read these, you’d
think that half her readership was sleeping rough and getting
online at libraries, Starbuckses, and stumbled wireless networks
that they accessed with antique laptops on street-corners.

“Kettlewell’s coming down to see this,” Perry said.

Her stomach lurched. She’d gotten the boys in trouble. “Is he
mad?”

“I couldn’t tell\dash{}I got voicemail at three AM.” Midnight in San
Jose, the hour at which Kettlewell got his mad impulses. “He’ll be
here this afternoon.”

“That jet makes it too easy for him to get around,” she said, and
stretched out her back. Sitting at her desk all morning answering
emails and cleaning up some draft posts before blogging them had
her in knots. It was practically lunch-time.

“Perry,” she began, then trailed off.

“It’s all right,” he said. “I know why you did it. Christ, we
wouldn’t be where we are if you hadn’t written about us. I’m in no
position to tell you to stop now.” He swallowed. The month since
the shantytowners had moved in had put five years on him. His tan
was fading, the wrinkles around his eyes deeper, grey salting his
stubbly beard and short hair. “But you’ll help me with Kettlewell,
right?”

“I’ll come along and write down what he says,” she said. “That
usually helps.”

\begin{center}\rule{3in}{0.4pt}\end{center}

\begin{blog}
Kodacell is supposed to be a new way of doing business.\\
Decentralized, net-savvy, really twenty-first century. The\\
suck-up tech press and tech-addled bloggers have been
trumpeting\\ its triumph over all other modes of
commerce.\blogpar{} But what does decentralization really mean? On
her “blog” this\\ week, former journalist Suzanne Church reports
that the inmates\\ running the flagship Kodacell asylum in
suburban Florida have\\ invited an entire village of homeless
squatters to take up\\ residence at their factory
premises.\blogpar{} Describing their illegal homesteading as
“live-work” condos that\\ Dr Seuss might have designed, Kodacell
shill Church goes on to\\ describe how this captive, live-in
audience has been converted to\\ a workforce for Kodacell’s most
profitable unit (“most\\ profitable” is a relative term: to date,
this unit has turned a\\ profit of about 1.5 million, per the
last quarterly report; by\\ contrast the old Kodak’s most
profitable unit made twenty times\\ that in its last quarter of
operation).\blogpar{} America has a grand tradition of this kind of
indentured living:\\ the coal-barons’ company towns of the 19th
century are the\\ original model for this kind of industrial
practice in the USA.\\ Substandard housing and only one employer
in town\dash{}that’s the\\ kind of brave new world that Church’s
boyfriend Kettlewell has\\ created.\blogpar{} A reader writes: “I
live near the shantytown that was relocated\\ to the Kodacell
factory in Florida. It was a dangerous slum full\\ of drug
dealers. None of the parents in my neighborhood let their\\ kids
ride their bikes along the road that passed it by\dash{}it was\\ a
haven for all kinds of down-and-out trash.”\blogpar{} There you have
it, the future of the American workforce:\\ down-and-out junkie
squatters working for starvation wages.
\end{blog}

“Kettlewell, you can’t let jerks like Freddy run this company. He’s
just looking to sell banner-space. This is how the Brit rags
write\dash{}it’s all meanspirited sniping.” Suzanne had never seen
Kettlewell so frustrated. His surfer good looks were fading fast\dash{}he
was getting a little paunch on him and his cheeks were sagging off
his bones into the beginnings of jowls. His car had pulled up to
the end of the driveway and he’d gotten out and walked through the
shantytown with the air of a man in a dream. The truckers who
pulled in and out all week picking up orders had occasionally had a
curious word at the odd little settlement, but for Suzanne it had
all but disappeared into her normal experience. Kettlewell made it
strange and even a little outrageous, just by his stiff, outraged
walk through its streets.

“You think I’m letting \emph{Freddy} drive this decision?” He had
spittle flecks on the corners of his mouth. “Christ, Suzanne,
you’re supposed to be the adult around here.”

Perry looked up from the floor in front of him, which he had been
staring at intently. Suzanne caught his involuntary glare at
Kettlewell before he dropped his eyes again. Lester put a big meaty
paw on Perry’s shoulder. Kettlewell was oblivious.

“Those people can’t stay, all right? The shareholders are baying
for blood. The fucking liability\dash{}Christ, what if one of those
places burns down? What if one of them knifes another one? We’re on
the hook for everything they do. We could end up being on the hook
for a fucking \emph{cholera epidemic}.”

Irrationally, Suzanne burned with anger at Freddy. He had written
every venal, bilious word with the hope that it would result in a
scene just like this one. And not because he had any substantive
objection to what was going on: simply because he had a need to
deride that which others hailed. He wasn’t afflicting the mighty,
though: he was taking on the very meekest, people who had
\emph{nothing}, including a means of speaking up for themselves.

Perry looked up. “You’ve asked me to come up with something new and
incredible every three to six months. Well, this is new and
incredible. We’ve built a living lab on our doorstep for exploring
an enormous market opportunity to provide low-cost, sustainable
technology for use by a substantial segment of the population who
have no fixed address. There are millions of American squatters and
billions of squatters worldwide. They have money to spend and no
one else is trying to get it from them.”

Kettlewell thrust his chin forward. “How many millions? How much
money do they have to spend? How do you know that any of this will
make us a single cent? Where’s the market research? Was there any?
Or did you just invite a hundred hobos to pitch their tent out
front of my factory on the strength of your half-assed guesses?”

Lester held up a hand. “We don’t have any market research,
Kettlewell, because we don’t have a business-manager on the team
anymore. Perry’s been taking that over as well as his regular work,
and he’s been working himself sick for you. We’re flying by the
seat of our pants here because you haven’t sent us a pilot.”

“You need an MBA to tell you not to turn your workplace into a
slum?” Kettlewell said. He was boiling. Suzanne very carefully
pulled out her pad and wrote this down. It was all she had, but
sometimes it was enough.

Kettlewell noticed. “Get out,” he said. “I want to talk with these
two alone.”

“No,” Suzanne said. “That’s not our deal. I get to document
everything. \emph{That’s} the deal.”

Kettlewell glared at her, and then he deflated. He sagged and took
two steps to the chair behind Perry’s desk and collapsed into it.

“Put the notebook away, Suzanne, please?”

She silently shook her head at him. He locked eyes with her for a
moment, then nodded curtly. She resumed writing.

“Guys, the major shareholders are going to start dumping their
stock this week. A couple of pension funds, a merchant bank. It’s
about ten, fifteen percent of the company. When that happens, our
ticker price is going to fall by sixty percent or more.”

“They’re going to short us because they don’t like what we’ve done
here?” Perry said. “Christ, that’s ridiculous!”

Kettlewell sighed and put his face in his hands, scrubbed at his
eyes. “No, Perry, no. They’re doing it because they can’t figure
out how to value us. Our business units have an industry-high
return on investment, but there’s not enough of them. We’ve only
signed a thousand teams and we wanted ten thousand, so ninety
percent of the money we had to spend is sitting in the bank at
garbage interest rates. We need to soak up that money with big
projects\dash{}the Hoover Dam, Hong Kong Disneyland, the Big Dig. All
we’ve got are little projects.”

“So it’s not our fault then, is it?” Lester said. Perry was staring
out the window.

“No, it’s not your fault, but this doesn’t help. This is a disaster
waiting to turn into a catastrophe.”

“Calm down, Landon,” Perry said. “Calm down for a sec and listen to
me, OK?”

Kettlewell looked at him and sighed. “Go ahead.”

“There are more than a billion squatters worldwide. San Francisco
has been giving out tents and shopping carts ever since they ran
out of shelter beds in the nineties. From Copenhagen to Capetown,
there are more and more people who are going off the grid, often in
the middle of cities.”

Suzanne nodded. “They farm Detroit, in the ruins of old buildings.
Raise crops and sell them. Chickens, too. Even pigs.”

“There’s something there. These people have money, like I said.
They buy and sell in the stream of commerce. They often have to buy
at a premium because the services and goods available to them are
limited\dash{}think of how a homeless person can’t take advantage of
bulk-packaged perishables because she doesn’t have a fridge. They
are the spirit of ingenuity, too\dash{}they mod their cars, caves,
anything they can find to be living quarters. They turn RVs into
permanent homes. They know more about tents, sleeping bags and
cardboard than any UN SHELTER specialist. These people need
housing, goods, appliances, you name it. It’s what Tjan used to
call a green-field market: no one else knows it’s there. You want
something you can spend ungodly amounts of money on? This is it.
Get every team in the company to come up with products for these
people. Soak up every cent they spend. Better us providing them
with quality goods at reasonable prices than letting them get
ripped off by the profiteers who have a captive market. This plant
is a living lab: this is the kind of market intelligence you can’t
buy, right here. We should set up more of these. Invite squatters
all over the country to move onto our grounds, test out our
products, help us design, build and market them. We can recruit
traveling salespeople to go door to door in the shanties and take
orders. Shit, man, you talk about the Grameen Bank all the time\dash{}why
not go into business providing these people with easy microcredit
without preying on them the way the banks do? Then we could loan
them money to buy things that we sell them that they use to better
their lives and earn more money so they can pay us back and buy
more things and borrow more money\dash{}”

Kettlewell held up a hand. “I like the theory. It’s a nice story.
But I have to sell this to my Board, and they want more than
stories: where can I get the research to back this up?”

“We’re it,” Perry said. “This place, right here. There’s no numbers
to prove what I’m saying is right because everyone who knows it’s
right is too busy chasing after it and no one else believes it. But
right here, if we’re allowed to do this\dash{}right here we can prove it.
We’ve got the capital in our account, we’re profitable, and we can
roll those profits back into more R\&D for the future of the
company.”

Suzanne was writing so fast she was getting a hand cramp. Perry had
never given speeches like this, even a month before. Tjan’s leaving
had hurt them all, but the growth it had precipitated in Perry was
stunning.

Kettlewell argued more, but Perry was a steamroller and Suzanne was
writing down what everyone said and that kept it all civil, like a
silent camera rolling in the corner of the room. No one looked at
her, but she was the thing they were conspicuously not looking at.

Francis took the news calmly. “Sound business strategy. Basically,
it’s what I’ve been telling you to do all along, so I’m bound to
like it.”

It took a couple weeks to hive off the Home Aware stuff to some of
the other Kodacell business-units. Perry flew a bunch, spending
days in Minnesota, Oregon, Ohio, and Michigan overseeing the
retooling efforts that would let him focus on his new project.

By the time he got back, Lester had retooled their own workspace,
converting it to four functional areas: communications, shelter,
food and entertainment. “They were Francis’s idea,” he said.
Francis’s gimpy leg was bothering him more and more, but he’d
overseen the work from a rolling ergonomic office-chair. “It’s his
version of the hierarchy of needs\dash{}stuff he knows for sure we can
sell.”

It was the first time the boys had launched something new without
knowing what it was, where they’d started with a niche and decided
to fill it instead of starting with an idea and looking for a niche
for it.

“You’re going to underestimate the research time,” Francis said
during one of their flip-chart brainstorms, where they had been
covering sheet after sheet with ideas for products they could
build. “Everyone underestimates research time. Deciding what to
make is always harder than making it.” He’d been drinking less
since he’d gotten involved in the retooling effort, waking earlier,
bossing around his young-blood posse to get him paper, bricks,
Tinkertoys.

He was right. Suzanne steadily recorded the weeks ticking by as the
four competing labs focus-grouped, designed, tested and scrapped
all manner of “tchotchkes for tramps,” as Freddy had dubbed it in a
spiraling series of ever-more-bilious columns. But the press was
mostly positive: camera crews liked to come by and shoot the
compound. One time, the pretty black reporter from the night of the
fire came by and said very nice things during her standup. Her name
was Maria and she was happy to talk shop with Suzanne, endlessly
fascinated by a “real” journalist who’d gone permanently slumming
on the Internet.

“The problem is that all this stuff is too specialized, it has too
many prerequisites,” Perry said, staring at a waterproof,
cement-impregnated bag that could be filled with a hose, allowed to
dry, and used as a self-contained room. “This thing is great for
refugees, but it’s too one-size-fits all for squatters. They have
to be able to heavily customize everything they use to fit into
really specialized niches.”

More squatters had arrived to take up residence with them\dash{}families,
friends, a couple of dodgy drifters\dash{}and a third story was going
onto the buildings in the camp. They were even more Dr Seussian
than the first round, idiosyncratic structures that had to be built
light to avoid crushing the floors below them, hanging out over the
narrow streets, corkscrewing like vines seeking sun.

He kept staring, and would have been staring still had he not heard
the sirens. Three blue-and-white Broward County sheriff’s cars were
racing down the access road into their dead mall, sirens howling,
lights blazing.

They screeched to a halt at the shantytown’s edge and their doors
flew open. Four cops moved quickly into the shantytown, while two
more worked the radios, sheltering by the cars.

“Jesus Christ,” Perry said. He ran for the door, but Suzanne
grabbed him.

“Don’t run toward armed cops,” she said. “Don’t do anything that
looks threatening. Slow down, Perry.”

He took a couple deep breaths. Then he looked around his lab for a
while, frantically muttering, “Where the fuck did I put it?”

“Use Home Aware,” she said. He shook his head, grimaced, went to a
keyboard and typed MEGAPHONE. One of the lab-drawers started to
throb with a white glow.

He pulled out the megaphone and went to his window.

\megaphone{“ATTENTION POLICE,”} he said. \megaphone{“THIS IS THE LEASEHOLDER FOR THIS
PROPERTY. WHY ARE YOU RUNNING AROUND WITH YOUR GUNS DRAWN? WHAT IS
GOING ON?”}

The police at the cars looked toward the workshop, then back to the
shantytown, then back to the workshop.

\megaphone{“SERIOUSLY. THIS IS NOT COOL. WHAT ARE YOU DOING HERE?”}

One of the cops grabbed the mic for his own loudhailer. \megaphone{“THIS IS
THE BROWARD COUNTY SHERIFF’S DEPARTMENT. WE HAVE RECEIVED
INTELLIGENCE THAT AN ARMED FUGITIVE IS ON THESE PREMISES. WE HAVE
COME TO RETRIEVE HIM.”}

\megaphone{“WELL, THAT’S WEIRD. NONE OF THE CHILDREN, CIVILIANS AND
HARDWORKING PEOPLE HERE ARE FUGITIVES AS FAR AS I KNOW. CERTAINLY
THERE’S NO ONE ARMED AROUND HERE. WHY DON’T YOU GET BACK IN YOUR
CARS AND I’LL COME OUT AND WE’LL RESOLVE THIS LIKE CIVILIZED
PEOPLE, OK?”}

The cop shook his head and reached for his mic again, and then
there were two gunshots, a scream, and a third.

Perry ran for the door and Suzanne chased after him, trying to stop
him. The cops at the cars were talking intently into their radios,
though it was impossible to know if they were talking to their
comrades in the shantytown or to their headquarters. Perry burst
out of the factory door and there was another shot and he spun
around, staggered back a step, and fell down like a sack of grain.
There was blood around his head. Suzanne stuck her hand in her
mouth to stifle a scream and stood helplessly in the doorway of the
workshop, just a few paces from Perry.

Lester came up behind her and firmly moved her aside. He lumbered
deliberately and slowly and fearlessly to Perry’s side, knelt
beside him, touched him gently. His face was grey. Perry thrashed
softly and Suzanne let out a sound like a cry, then remembered
herself and took out her camera and began to shoot and shoot and
shoot: the cops, Lester with Perry like a tragic Pieta, the
shantytowners running back and forth screaming. Snap of the cops
getting out of their cars, guns in hands, snap of them fanning out
around the shantytown, snap of them coming closer and closer, snap
of a cop pointing his gun at Lester, ordering him away from Perry,
snap of a cop approaching her.

“It’s live,” she said, not looking up from the viewfinder. “Going
out live to my blog. Daily readership half a million. They’re
watching you now, every move. Do you understand?”

The officer said, “Put the camera down, ma’am.”

She held the camera. “I can’t quote the First Amendment from
memory, not exactly, but I know it well enough that I’m not moving
this camera. It’s live, you understand\dash{}every move is going out
live, right now.”

The officer stepped back, turned his head, muttered in his mic.

“There’s an ambulance coming,” he said. “Your friend was shot with
a nonlethal rubber bullet.”

“He’s bleeding from the head,” Lester said. “From the eye.”

Suzanne shuddered.

Ambulance sirens in the distance. Lester stroked Perry’s hair.
Suzanne took a step back and panned it over Perry’s ruined face,
bloody and swollen. The rubber bullet must have taken him either
right in the eye or just over it.

“Perry Mason Gibbons was unarmed and posed no threat to Sheriff’s
Deputy Badge Number 5724\dash{}” she zoomed in on it\dash{}“when he was shot
with a rubber bullet in the eye. He is unconscious and bloody on
the ground in front of the workshop where he has worked quietly and
unassumingly to invent and manufacture new technologies.”

The cop knew when to cut his losses. He turned aside and walked
back into the shantytown, leaving Suzanne to turn her camera on
Perry, on the EMTs who evacced him to the ambulance, on the three
injured shantytowners who were on the ambulance with him, on the
corpse they wheeled out on his own gurney, one of the newcomers to
the shantytown, a man she didn’t recognize.

They operated on Perry all that night, gingerly tweezing fragments
of bone from his shattered left orbit out of his eye and face. Some
had floated to the back of the socket and posed a special risk of
brain damage, the doctor explained into her camera.

Lester was a rock, sitting silently in the waiting room, talking
calmly and firmly with the cops and over the phone to Kettlewell
and the specially impaneled board-room full of Kodacell lawyers who
wanted to micromanage this. Rat-Toothed Freddy filed a column in
which he called her a “grandstanding bint,” and accused Kodacell of
harboring dangerous fugitives. He’d dug up the fact that one of the
newcomers to the shantytown\dash{}not the one they’d killed, that was a
bystander\dash{}was wanted for holding up a liquor-store with a corkscrew
the year before.

Lester unscrewed his earphone and scrubbed at his eyes.
Impulsively, she leaned over and gave him a hug. He stiffened up at
first but then relaxed and enfolded her in his huge, warm arms. She
could barely make her arms meet around his broad, soft back\dash{}it was
like hugging a giant loaf of bread. She squeezed tighter and he did
too. He was a good hugger.

“You holding in there, kiddo?” she said.

“Yeah,” he murmured into her neck. “No.” He squeezed tighter. “As
well as I need to, anyway.”

The doctor pried them apart to tell them that the EEG and fMRI were
both negative for any brain-damage, and that they’d managed to
salvage the eye, probably. Kodacell was springing for all the care
he needed, cash money, no dorking around with the fucking HMO, so
the doctors had put him through every machine on the premises in a
series of farcically expensive tests.

“I hope they sue the cops for the costs,” the doctor said. She was
Pakistani or Bangladeshi, with a faint accent, and very pretty even
with the dark circles under her eyes. “I read your columns,” she
said, shaking Suzanne’s hand. “I admire the work you do,” she said,
shaking Lester’s hand. “I was born in Delhi. We were squatters who
were given a deed to our home and then evicted because we couldn’t
pay the taxes. We had to build again, in the rains, outside of the
city, and then again when we were evicted again.”

She had two brothers who were working for startups like Kodacell’s,
but run by other firms: one was backed by McDonald’s, the other by
the AFL-CIO’s investment arm. Suzanne did a little interview with
her about her brothers’ projects\dash{}a bike-helmet that had been
algorithmically evolved for minimum weight and maximum protection;
a smart skylight that deformed itself to follow light based on
simple phototropic controllers. The brother working on bike-helmets
was riding a tiger and could barely keep up with orders; he was
consuming about half of the operational capacity of the McDonald’s
network and climbing fast.

Lester joined in, digging on the details. He’d been following the
skylights in blogs and on a list or two, and he’d heard of the
doctor’s brother, which really tweaked her, she was visibly proud
of her family.

“But your work is most important. Things for the homeless. We get
them in here sometimes, hurt, off the ambulances. We usually turn
them away again. The ones who sell off the highway medians and at
the traffic lights.” Suzanne had seen them, selling homemade
cookies, oranges, flowers, newspapers, plasticky toys, sad or
beautiful handicrafts. She had a carved coconut covered in
intricate scrimshaw that she’d bought from a little girl who was
all skin and bones except for her malnourished pot-belly.

“They get hit by cars?”

“Yes,” the doctor said. “Deliberately, too. Or beaten up.”

Perry was moved out of the operating theater to a recovery room and
then to a private room and by then they were ready to collapse,
though there was so much email in response to her posts that she
ended up pounding on her computer’s keyboard all the way home as
Lester drove them, squeezing the bridge of his nose to stay awake.
She didn’t even take her clothes off before collapsing into bed.

\begin{center}\rule{3in}{0.4pt}\end{center}

“They need the tools to make any other tools,” is what Perry said
when he returned from the hospital, the side of his head still
swaddled in bandages that draped over his injured eye. They’d
shaved his head at his insistence, saying that he wasn’t going to
try to keep his hair clean with all the bandages. It made him look
younger, and his fine skull-bones stood out through his thin scalp
when he finally came home. Before he’d looked like a outdoorsman
engineer: now he looked like a radical, a pirate.

“They need the tools that will let them build anything else, for
free, and use it or sell it.” He gestured at the rapid prototyping
machines they had, the three-d printer and scanner setups. “I mean
something like that, but I want it to be capable of printing out
the parts necessary to assemble another one. Machines that can
reproduce themselves.”

Francis shifted in his seat. “What are they supposed to do with
those?”

“Everything,” Perry said, his eye glinting. “Make your kitchen
fixtures. Make your shoes and hat. Make your kids’ toys\dash{}if it’s in
the stores, it should be a downloadable too. Make toolchests and
tools. Make it and build it and sell it. Make other printers and
sell them. Make machines that make the goop we feed into the
printers. Teach a man to fish, Francis, teach a man to fucking
\emph{fish}. No top-down ’solutions’ driven by ’market
research’”\dash{}his finger-quotes oozed sarcasm\dash{}“the thing that we need
to do is make these people the authors of their own destiny.”

They put up the sign that night: \textuppercase{AUTHOR OF YOUR OWN DESTINY}, hung
over the workshop door. Suzanne trailed after Perry transcribing
the rants that spilled out of his mouth as he explained it to
Lester and Francis, and then to Kettlewell when he called, and then
to the pretty young black lady from the TV who by now had figured
out that there was a real story in her backyard, then to an NPR man
on the phone, and then to a CNN crew who drove in from Miami and
filmed the shantytown and the workshop like Japanese tourists at
Disney World, never having ventured into the skanky, failed
strip-mall suburbs just outside of town.

Francis had a protege who had a real dab touch with the 3-D
printers. The manufacturer, Lester’s former employer, had been out
of business for two years by then, so all the service on the
machines had to be done on the premises. Francis’s protege\dash{}the one
who claimed his mother had pushed his father under a bus, his name
was Jason\dash{}watched Lester work on recalcitrant machines silently for
a couple days, then started to hand him the tool he needed next
without having to be asked. Then he diagnosed a problem that had
stumped Lester all morning. Then he suggested an improvement to the
feedstock pump that increased the mean time between failures by a
couple hours.

“No, man, no, not like that,” Jason said to one of the small gang
of boys he was bossing. “Gently, or you’ll snap it off.” The boy
snapped it off and Jason pulled another replacement part out of a
tub and said, “See, like \emph{this},” and snapped it on. The small
gang of boys regarded him with something like awe.

“How come no girls?” Suzanne said as she interviewed him while he
took a smoke-break. Perry had banned cigarettes from all indoor
workshops, nominally to keep flames away from the various
industrial chemicals and such, but really just to encourage the
shantytowners to give up the habit that they couldn’t afford
anyway. He’d also leaned on the shantytowners who’d opened up small
shops in their houses to keep cigs out of the town, without a lot
of success.

“Girls aren’t interested in this stuff, lady.”

“You think?” There was a time when she would have objected, but it
was better to let these guys say it out loud, hear themselves say
it.

“No. Maybe where you come from, OK? Don’t know. But here girls are
different. They do good in school but when they have babies they’re
done. I mean, hey, it’s not like I don’t \emph{want} girls in the
team, they’d be great. I love girls. They fuckin’ \emph{work}, you
know. No bullshit, no screwing around. But I know every girl in
this place and none of ’em are even interested, OK?”

Suzanne cocked one eyebrow just a little and Jason shifted
uncomfortably. He scratched his bare midriff and shuffled. “I do,
all of them. Why would they? One girl, a roomful of boys, it’d be
gross. They’d act like jerks. There’s no way we’d get anything
done.”

Suzanne lifted her eyebrow one hair higher. He squirmed harder.

“So all right, that’s not their fault. But I got enough work, all
right? Too much to do without spending time on that. It’s not like
any girls have \emph{asked} to join up. I’m not keeping them out.”

Suzanne jotted a couple of notes, keeping perfectly mum.

“Well, I’d like to have them in the workshop, OK? Maybe I should
ask some of them if they’d come. Shit, if I can teach these apes, I
can teach a girl. They’re smart. Girls’d make this place a little
better to work in. Lots of them trying to support their families,
so they need the money, too.”

There was a girl there by the afternoon. The next day, there were
two more. They seemed like quick studies, despite their youth and
their lip-gloss. Suzanne approved.

\begin{center}\rule{3in}{0.4pt}\end{center}

Lester stayed long enough to see the first prototype
printer-printers running, then he lit out with a duffel bag jammed
into the back of his modded Smart car. “Where are you going?”
Suzanne said as Perry looked on gloomily. “I’ll come and visit you.
I want to follow your story.” Truth be told, she was sorry to see
him go, very sorry. He was such a rock, such an anchor for Perry’s
new crazy pirate energy and for the madness around them. He hadn’t
given much notice (not to her\dash{}Perry didn’t seem that surprised).

“I can’t really talk about it,” he said. “Nondisclosure.”

“So it’s a new job,” she said. “You’re going to work for Tjan?”
Tjan’s Westinghouse operation was fully rocking. He had fifty teams
up the eastern seaboard, ten in the midwest and was rumored to have
twice as many in Eastern Europe.

He grinned. “Oh, Suzanne, don’t try to journalist me.” He reached
out and hugged her in a cloud of her father’s cologne. “You’re
fantastic, you know that? No, I’m not going to a job. It’s a thing
that’s an amazing opportunity, you know?”

She didn’t, but then he was gone and boy did she miss him.

Perry and she went out for dinner in Miami the next night with a
PhD candidate from Pepperdine’s B-school, eating at the same deco
patio that she’d dined at with Tjan. Perry wore a white shirt open
to reveal his tangle of wiry chest hair and the waitress couldn’t
keep her eyes off of him. He had a permanent squint now, and a scar
that made his eyebrow into a series of small hills.

“I was just in Greensboro, Miss,” the PhD candidate said. He was in
his mid-twenties, young and slick, his only nod to academe a small
goatee. “I used to spend summers there with my grandpa.” He talked
fast, flecks of spittle in the corners of his mouth, eyes wide,
fork stabbing blindly at the bits of crab-cake on his plate. “There
wasn’t \emph{anything} left there, just a couple gas-stations and a
7-Eleven, shit, they’d even closed the Wal-Mart. But now, but now,
it’s \emph{alive} again, it’s buzzing and hopping. Every empty
storefront is full of people playing and tinkering, just a little
bit of money in their pockets from a bank or a company or a fund.
They’re doing the dumbest things, mind you: tooled-leather laptop
cases, switchblade knives with thumb drives in the handles, singing
and dancing lawn-Santas that yodel like hillbillies.”

“I’d buy a tooled-leather laptop case,” Perry said, swilling a
sweaty bottle of beer. He waggled his funny eyebrow and rubbed his
fuzzy scalp.

“The rate of employment is something like ninety-five percent,
which it hasn’t been in like a hundred years. If you’re not
inventing stuff, you’re keeping the books for someone who is, or
making sandwiches for them, or driving delivery vehicles around.
It’s like a tiny, distributed gold rush.”

“Or like the New Deal,” Suzanne said. That was how she’d come to
invite him down, after she’d read his paper coining the term New
Work to describe what Perry was up to, comparing it to Roosevelt’s
public-investment plan that spent America free of the Depression.

“Yeah, exactly, exactly! I’ve got research that shows that one in
five Americans is employed in the New Work industry. Twenty
percent!”

Perry’s lazy eye opened a little wider. “No way,” he said.

“Way,” the PhD candidate said. He finished his caipirinha and shook
the crushed ice at a passing waiter, who nodded and ambled to the
bar to get him a fresh one. “You should get on the road and write
about some of these guys,” he said to Suzanne. “They need some ink,
some phosphors. They’re pulling up stakes and moving to the small
towns their parents came from, or to abandoned suburbs, and just
\emph{doing it}. Bravest fucking thing you’ve seen in your life.”

The PhD candidate stayed out the week, and went home with a
suitcase full of the parts necessary to build a three-d printer
that could print out all of the parts necessary to build a three-d
printer.

Lester emailed her from wherever it was he’d gone, and told her
about the lovely time he was having. It made her miss him sharply.
Perry was hardly ever around for her now, buried in his work,
buried with the kids from the shantytown and with Francis. She
looked over her last month’s blogs and realized that she’d been
turning in variations on the same theme for all that time. She knew
it was time to pack a duffel bag of her own and go see the bravest
fucking thing she’d seen in her life.

“Bye, Perry,” she said, stopping by his workbench. He looked up at
her and saw the bag and his funny eyebrow wobbled.

“Leaving for good?” he said. He sounded unexpectedly bitter.

“No!” she said. “No! Just a couple weeks. Going to get the rest of
the story. But I’ll be back, count on it.”

He grunted and slumped. He was looking a lot older now, and beaten
down. His hair, growing out, was half grey, and he’d gotten gaunt,
his cheekbones and forehead springing out of his face. On impulse,
she gave him a hug like the ones she’d shared with Lester. He
returned it woodenly at first, then with genuine warmth. “I will be
back, you know,” she said. “You’ve got plenty to do here, anyway.”

“Yeah,” he said. “Course I do.”

She kissed him firmly on the cheek and stepped out the door and
into her car and drove to Miami International.

\begin{center}\rule{3in}{0.4pt}\end{center}

Tjan met her at Logan and took her bag. “I’m surprised you had the
time to meet me,” she said. The months had been good to him,
slimming down his pot-belly and putting a twinkle in his eye.

“I’ve got a good organization,” he said, as they motored away
toward Rhode Island, through strip-mall suburbs and past boarded-up
chain restaurants. Everywhere there were signs of industry:
workshops in old storefronts, roadside stands selling disposable
music players, digital whoopee cushions, and so forth. “I barely
have to put in an appearance.”

Tjan yawned hugely and constantly. “Jet-lag,” he apologized. “Got
back from Russia a couple days ago.”

“Did you see your kids?” she said. “How’s business there?”

“I saw my kids,” he said, and grinned. “They’re amazing, you know
that? Good kids, unbelievably smart. Real little operators. The
older one, Lyenitchka, is running a baby-sitting service\dash{}not
baby-sitting herself, you see, but recruiting other kids to do the
sitting for her while she skims a management fee and runs the
quality control.”

“She’s your daughter all right,” she said. “So tell me everything
about the Westinghouse projects.”

She’d been following them, of course, lots of different little
startups, each with its own blogs and such. But Tjan was quite
fearless about taking her through their profits and losses and
taking notes on it all kept her busy until she reached her hotel.
Tjan dropped her off and promised to pick her up the next morning
for a VIP tour of the best of his teams, and she went to check in.

She was in the middle of receiving her key when someone grabbed her
shoulder and squeezed it. “Suzanne bloody Church! What are you
doing here, love?”

The smell of his breath was like a dead thing, left to fester. She
turned around slowly, not wanting to believe that of all the hotels
in rural Rhode Island, she ended up checking into the same one as
Rat-Toothed Freddy.

“Hey, Freddy,” she said. Seeing him gave her an atavistic urge to
stab him repeatedly in the throat with the hotel stick-pen. He was
unshaven, his gawky Adam’s apple bobbing up and down, and he
swallowed and smiled wetly. “Nice to see you.”

“Fantastic to see you, too! I’m here covering a shareholder meeting
for Westinghouse, is that what you’re here for, too?”

“No,” she said. She knew the meeting was on that week, but hadn’t
planned on attending it. She was done with press conferences,
preferring on-the-ground reporting. “Well, nice to see you.”

“Oh, do stay for a drink,” he said, grinning more widely, exposing
those grey teeth in a shark’s smile. “Come on\dash{}they have a free
cocktail hour in this place. I’ll have to report you to the
journalist’s union if you turn down a free drink.”

“I don’t think ’bloggers’ have to worry about the journalist’s
union,” she said, making sarcastic finger-quotes in case he didn’t
get the message. He still didn’t. He laughed instead.

“Oh, love, I’m sure they’ll still have you even if you have lapsed
away from the one true faith.”

“Good night, Freddy,” was all she could manage to get out without
actually hissing through her teeth.

“OK, good night,” he said, moving in to give her a hug. As he
loomed toward her, she snapped.

“Freeze, mister. You are not my friend. I do not want to touch you.
You have poor personal hygiene and your breath smells like an
overflowing camp-toilet. You write vicious personal attacks on me
and on the people I care about. You are unfair, meanspirited, and
you write badly. The only day I wouldn’t piss on you, Freddy, is
the day you were on fire. Now get the fuck out of my way before I
kick your tiny little testicles up through the roof of your reeking
mouth.”

She said it quietly, but the desk-clerks behind her overheard it
anyway and giggled. Freddy’s smile only wobbled, but then returned,
broader than ever.

“Well said,” he said and gave her a single golf-clap. “Sleep well,
Suzanne.”

She boiled all the way to her room and when she came over hungry,
she ordered in room service, not wanting to take the chance that
Rat-Toothed Freddy would still be in the lobby.

\begin{center}\rule{3in}{0.4pt}\end{center}

Tjan met her as she was finishing her coffee in the breakfast room.
She hadn’t seen Freddy yet.

“I’ve got five projects slated for you to visit today,” Tjan said,
sliding into the booth beside her. Funnily now that he was in the
cold northeast, he was dressing like a Floridian in blue jeans and
a Hawai’ian barkcloth shirt with a bright spatter of pineapples and
Oscar Mayer Wienermobiles. Back in Florida, he’d favored
unflattering nylon slacks and white shirts with ironed collars.

The projects were fascinating and familiar. The cultural
differences that distinguished New England New Work from Florida
New Work were small but telling: a lot more woodcraft, in a part of
the country where many people had grown up in their grandfathers’
woodworking shops. A little more unreflexive kitsch, like the
homely kittens and puppies that marched around the reactive,
waterproof, smash-proof screens integrated into a bio-monitoring
crib.

At the fourth site, she was ambushed by a flying hug. Tjan laughed
as she nearly went down under the weight of a strong, young woman
who flung her arms around Suzanne’s neck. “Holy \emph{crap} it’s
good to see you!”

Suzanne untangled herself and got a look at her hugger. She had
short mousy hair, twinkling blue eyes, and was dressed in overalls
and a pretty flowered blouse, scuffed work boots and stained and
torn work-gloves. “Uh\ldots{}” she said, then it clicked. “Fiona?”

“Yeah! Didn’t Tjan tell you I was here?” The last time she’d seen
this woman, she was weeping over pizza and getting ready to give up
on life. Now she was practically vibrating.

“Uh, no,” she said, shooting a look at Tjan, who was smiling like
the Buddha and pretending to inspect a pair of shoes with
gyroscopically stabilized retractable wheels in the heels.

“I’ve been here for months! I went back to Oregon, like you told me
to, and then I saw a recruiting ad for Westinghouse and I sent them
my CV and then I got a videoconference interview and then, bam, I
was on an airplane to Rhode Island!”

Suzanne blinked. \emph{I} told you to go back to Oregon? Well,
maybe she had. That was a lifetime ago.

The workshop was another dead mall, this one a horseshoe of
storefronts separated by flimsy gyprock. The Westinghousers had cut
through the walls with drywall knives to join all the stores
together. The air was permeated with the familiar
Saran-Wrap-in-a-microwave tang of three-d printers. The parking lot
was given over to some larger apparatus and a fantastical
children’s jungle-gym in the shape of a baroque, spired pirate
fortress, with elegantly curved turrets, corkscrew sky-bridges, and
flying buttresses crusted over with ornate, grotesque gargoyles.
Children swarmed over it like ants, screeching with pleasure.

“Well, you’re looking really good, Fiona,” Suzanne said.
\emph{Still not great with people}, she thought. Fiona, though, was
indeed looking good, and beaming. She wasn’t wearing the crust of
cosmetics and hair-care products she’d affected in the corporate
Silicon Valley world. She glowed pink.

“Suzanne,” Fiona said, getting serious now, taking her by the
shoulders and looking into her eyes. “I can’t thank you enough for
this. This has saved my life. It gave me something to live for. For
the first time in my life, I am doing something I’m proud of. I go
to bed every night thankful and happy that I ended up here. Thank
you, Suzanne. Thank you.”

Suzanne tried not to squirm. Fiona gave her another long hug. “It’s
all your doing,” Suzanne said at last. “I just told you about it.
You’ve made this happen for you, OK?”

“OK,” Fiona said, “but I still wouldn’t be here if it wasn’t for
you. I love you, Suzanne.”

Ick. Suzanne gave her another perfunctory hug and got the hell out
of Dodge.

\begin{center}\rule{3in}{0.4pt}\end{center}

“What’s with the jungle-gym?” It really had been something, fun and
Martian-looking.

“That’s the big one,” Tjan said with a big grin. “Most people don’t
even notice it, they think it’s daycare or something. Well, that’s
how it started out, but then some of the sensor people started
noodling with jungle-gym components that could tell how often they
were played with. They started modding the gym every night, adding
variations on the elements that saw the most action, removing the
duds. Then the CAD people added an algorithm that would take the
sensor data and generate random variations on the same basis.
Finally, some of the robotics people got in on the act so that the
best of the computer-evolved designs could be instantiated
automatically: now it’s a self-modifying jungle-gym. The kids love
it. It is the crack cocaine of jungle-gyms, though we won’t be
using that in the marketing copy, of course.”

“Of course,” Suzanne said drily. She’d automatically reached for
her notepad and started writing when Tjan started talking. Now,
reviewing her notes, she knew that she was going to have to go back
and get some photos of this. She asked Tjan about it.

“The robots go all night, you know. Not much sleep if you do
that.”

No going back to the hotel to see Freddy, what a pity. “I’ll grab a
couple blankets from the hotel to keep warm,” she said.

“Oh, you needn’t,” he said. “That crew has a set of bleachers with
gas-heaters for the night crew and their family to watch from. It’s
pretty gorgeous, if you ask me.”

They had a hasty supper of burgers at a drive-through and then went
back to the jungle-gym project. Suzanne ensconced herself at
someone’s vacated desk for a couple hours and caught up on email
before finally emerging as the sun was dipping swollen and red
behind the mall. She set herself up on the bleachers, and Fiona
found her with a thermos of coffee and a flask of whisky. They
snuggled under a blanket amid a small crowd of geeks, an outdoor
slumber party under the gas-heaters’ roar.

Gradually, the robots made an appearance. Most of them humped along
like inchworms, carrying chunks of new playground apparatus in
coils of their long bodies. Some deployed manipulator arms, though
they didn’t have much by way of hands at their ends. “We just use
rare-earth magnets,” Fiona said. “Less fiddly than trying to get
artificial vision that can accurately grasp the bars.”

Tjan nudged her and pointed to a new tower that was going up. The
robots were twisting around themselves to form a scaffold, while
various of their number crawled higher and higher, snapping modular
pieces of high-impact plastic together with \emph{snick} sounds
that were audible over the whine of their motors.

Suzanne switched on her camera’s night-vision mode and got
shooting. “Where did you get all these robots?”

Tjan grinned. “It’s an open design\dash{}the EPA hired Westinghouse to
build these to work on sensing and removing volatile organic
compounds on Superfund sites. Because we did the work for the
government, we had to agree not to claim any design copyright or
patents in the outcome. There’s a freaking warehouse full of this
stuff at Westinghouse, all kinds of crazy things that Westinghouse
abandoned because they weren’t proprietary enough and they were
worried that they’d have to compete on the open market if they
tried to productize them. Suits us just fine, though.”

The field was aswarm with glinting metal inchworm robots now,
shifting back and forth, boiling and roiling and picking up
enormous chunks of climber like cartoon ants carrying away a picnic
basket. The playground was being transformed before her eyes, in
ways gross and subtle, and it was enchanting to watch.

“Can I go out and have a look?” she said. “I mean, is it safe?”

“Sure,” Fiona said. “Of course! Our robots won’t harm you; they
just nuzzle you and then change direction.”

“Still, try to stay out of their way,” Tjan said. “Some of that
stuff they’re moving around is heavy.”

So she waded out onto the playground and carefully picked her way
through the robot swarm. Some crawled over her toes. A couple
twined between her feet and nearly tripped her up and once she
stepped on one and it went still and waited politely for her to
step off.

Once in the thick of it all, she switched on her video and began to
record through the night filter. Standing there amid the whirl and
racket and undulating motion of the jungle gym as it reconfigured
itself, she felt like she’d arrived at some posthuman future where
the world no longer needed her or her kind. Like humanity’s
creations had evolved past their inventors.

She was going to have to do a \emph{lot} of writing before bed.

Freddy was checking out in the lobby when Tjan dropped her off at
5AM. It was impossible to sneak past him, and he gave her a nasty,
bucktoothed smile as she passed by him. It distracted her and made
the writing come more slowly, but she was a pro and her readers had
sent in a lot of kind mail, and there was one from Lester, still
away on his mysterious errand but sounding happier than he had in
months, positively giddy.

She set the alarm-clock so that she could be awake for her next
stop, outside of North Carolina’s Research Triangle, where some
local millionaires had backed a dozen New Work teams.

Another three weeks of this stuff and she’d get to go home\dash{}Florida.
The condo was home now, and the junkyard. Hot and sticky and
inventive and ever-changing. She fell asleep thinking of it and
smiling.

It was two weeks more before Lester caught up with her, in Detroit
of all places. Going back to the old place hadn’t been her idea,
she’d been dragged back by impassioned pleas from the local Ford
and GM New Work teams, who were second-generation-unemployed, old
rust-belt families who’d rebooted with money from the companies
that had wrung their profit from their ancestors and abandoned
them.

The big focus in the rustbelt was eradicating the car. Some were
building robots that could decommission leaky gas-stations and
crater out the toxic soil. Some were building car-disassembly
plants that reclaimed materials from the old beasts’ interiors.
Between the Ford and GM teams with their latest bail-out and those
funded by the UAW out of the settlements they’d won from the
auto-makers, Detroit was springing up anew.

Lester emailed her and said that he’d seen on her blog that she was
headed to Detroit, and did she want to meet him for dinner, being
as he’d be in town too?

They ate at Devil’s Night, a restaurant in one of the reclaimed
mansions in Brush Park, a neighborhood of wood-frame buildings that
teenagers had all but burned to the ground over several decades’
worth of Halloweens. In Detroit, Devil’s Night was the
pre-Halloween tradition of torching abandoned buildings, and all of
Brush Park had been abandoned for years, its handsome houses
attractive targets for midnight firebugs.

Reclaiming these buildings was an artisanal practice of urethaning
the charred wood and adding clever putty, cement, and glass to
preserve the look of a burned out hulk while restoring structural
integrity. One entire floor of the restaurant was missing, having
been replaced by polished tempered one-way glass that let upstairs
diners look down on the bald spots and cleavage of those eating
below.

Suzanne showed up a few minutes late, having gotten lost wandering
the streets of a Detroit that had rewritten its map in the decades
since she’d left. She was flustered, and not just because she was
running late. There was a lingering awkwardness between her and
Lester and her elation at seeing him again had an inescapable
undercurrent of dread.

When the waiter pointed out her table, she told him he was
mistaken. Lester wasn’t there, some stranger was: short-haired,
burly, with a few days’ stubble. He wore a smart blazer and a loose
striped cotton shirt underneath. He was beaming at her.

“Suzanne,” he said.

Her jaw literally dropped. She realized she was standing with her
mouth open and shut it with a snap. “Lester?” she said,
wonderingly.

He got up, still smiling, even laughing a little, and gave her a
hug. It was Lester all right. That smell was unmistakable, and
those big, warm paws he called hands.

When he let go of her, he laughed again. “Oh, Suzanne, I could
\emph{not} have asked for any better reaction than this.
\emph{Thank you}.” They were drawing stares. Dazedly, she sat down.
So did he.

“Lester?” she said again.

“Yes, it’s me,” he said. “I’ll tell you about it over dinner. The
waiter wants to take our drink orders.”

Theatrically, she ordered a double Scotch. The waiter rattled off
the specials and Suzanne picked one at random. So did Lester.

“So,” he said, patting his washboard tummy. “You want to know how I
got to this in ten weeks, huh?”

“Can I take notes?” Suzanne said, pulling out her pad.

“Oh by all means,” he said. “I got a discount on my treatments on
the basis that you would end up taking notes.”

The clinic was in St Petersburg, Russia, in a neighborhood filled
with Russian dentists who catered to American health tourists who
didn’t want to pay US prices for crowns. The treatment hadn’t
originated there: The electromuscular stimulation and chemical
therapy for skin-tightening was standard for rich new mothers in
Hollywood who wanted to get rid of pregnancy bellies. The
appetite-suppressing hormones had been used in the Mexican pharma
industry for years. Stem-cells had been an effective substitute for
steroids when it came to building muscle in professional athletic
circles the world round. Genomic therapy using genes cribbed from
hummingbirds boosted metabolism so that the body burned 10,000
calories a day sitting still.

But the St Petersburg clinic had ripped, mixed and burned these
different procedures to make a single, holistic treatment that had
dropped Lester from 400 to 175 pounds in ten weeks.

“Is that safe?” she said.

“Everyone asks that,” he said, laughing. “Yeah, it’s safe if
they’re monitoring you and standing by with lots of diagnostic
equipment. But if you’re willing to take slower losses, you can go
on a way less intensive regime that won’t require supervision. This
stuff is the next big grey-market pharma gold. They’re violating
all kinds of pharma patents, of course, but that’s what Cuba and
Canada are for, right? Inside of a year, every fat person in
America is going to have a bottle of pills in his pocket, and
inside of two years, there won’t be any fat people.”

She shook her head. “You look\ldots{} Lester, you look
\emph{incredible}. I’m so proud of you.”

He ducked his head. He really did look amazing. Dropping the weight
had taken off ten years, and between that and the haircut and the
new clothes, he was practically unrecognizable.

“Does Perry know?”

“Yeah,” Lester said. “I talked it over with him before I opted for
it. Tjan had mentioned it in passing, it was a business his ex-wife
was tangled up with through her mafiyeh connections, and once I had
researched it online and talked to some people who’d had the
treatment, including a couple MDs, I decided to just do it.”

It had cost nearly everything he’d made from Kodacell, but it was a
small price to pay. He insisted on getting dinner.

Afterward, they strolled through the fragrant evening down Woodward
Avenue, past the deco skyscrapers and the plowed fields and
community gardens, their livestock pens making soft animal noises.

“It’s wonderful to see you again, Lester,” she said truthfully.
She’d really missed him, even though his participation on her
message boards had hardly let up (though it had started coming in
at weird hours, something explained by the fact that he’d been in
Russia). Walking alongside of him, smelling his smell, seeing him
only out of the corner of her eye, it was like nothing had
changed.

“It’s great to see you again too.” Tentatively, he took her hand in
his big paw. His hand was warm but not sweaty, and she realized it
had been a long time since anyone had held her hand. Heart
pounding, she gave his hand a squeeze.

Their conversation and their walk rambled on, with no outward
acknowledgment of the contact of hand on hand, but her hand
squeezed his softly now and again, or he squeezed hers, and then
they were at her hotel. \emph{How did that happen?} she asked
herself.

But then they were having a nightcap, and then he was in the
elevator with her and then he was at the door of her room, and the
blood was roaring in her ears as she stuck her credit-card in the
reader to open it.

\emph{Wait}, she tried to say. \emph{Lester, hang on a second,} is
what she tried to say, but her tongue was thick in her mouth. He
stepped through the door with her, then said, “Uh, I need to use
the bathroom.”

With relief, she directed him to the small water closet. The room
was basic\dash{}now that she was her own boss, she wasn’t springing for
Crowne Plazas and Hiltons, this was practically a coffin\dash{}and there
was nowhere to sit except the bed. Her laptop was open and there
was a lot of email in her inbox, but for once, she didn’t care. She
was keenly attuned to the water noises coming from behind the door,
each new sound making her jump a little. What was he doing in
there, inserting a fucking diaphragm?

She heard him work the latch on the door and she put on her best
smile. Her stomach was full of butterflies. He smiled back and sat
down on the bed next to her, taking her hand again. His hand was
moist from being washed, and a little slippery. She didn’t mind.
Wordlessly, she put her head on his barrel chest. His heart was
racing, and so was hers.

Gradually, they leaned back, until they were side by side on the
bed, her head still on his chest. Moving like she was in a dream,
she lifted her head from his chest and stared into his eyes. They
were wide and scared. She kissed him, softly. His lips were
trembling and unyielding. She kissed him more insistently, running
her hands over his chest and shoulders, putting one leg over him.
He closed his eyes and kissed her back. He wasn’t bad, but he was
scared or nervous and all jittery.

She kissed his throat, breathing in the smell, savoring the rough
texture of his three-day beard. Tentatively, he put his hands on
her back, stroked her, worked gradually towards her bottom. Then he
stopped.

“What’s wrong?” she said, propping herself up on her forearms,
still straddling him.

She saw that there were tears in his eyes.

“Lester? What’s wrong?”

He opened his mouth and then shut it. Tears slid off his face into
his ears. She blotted them with a corner of hotel-pillow.

She stroked his hair. “Lester?”

He gave out a choked sob and pushed her away. He sat up and put his
face in his hands. His back heaved. She stroked his shoulders
tentatively.

Finally, he seemed to get himself under control. He sniffled.

“I have to go,” he said.

“Lester, what’s \emph{wrong}?”

“I can’t do this,” he said. “I\ldots{}”

“Just tell me,” she said. “Whatever it is, tell me.”

“You didn’t want me before.” He said it simply without accusation,
but it stung like he’d slapped her in the face.

“Oh, Lester,” she said, moving to hug him, but he pushed her away.

“I have to go,” he said, drawing himself up to his full height. He
was tall, though he’d never seemed it before, but oh, he was tall,
six foot four or taller. He filled the room. His eyes were red and
swollen, but he put on a smile for her. “Thanks, Suzanne. It was
really good to see you again. I’ll see you in Florida.”

She stood up and moved quickly to him, stood on tiptoe to put her
arms around his neck and hug him fiercely. He hugged her back and
she kissed him on the cheek.

“I’ll see you in Florida,” she said.

And then he was gone. She sat on the edge of her bed and waited for
tears, but they didn’t come. So she picked up her laptop and
started to work through her mountain of email.

\begin{center}\rule{3in}{0.4pt}\end{center}

When she saw him again, he was coming down the drive leading to the
shantytown and the factory. She was having tea in the tea-room that
had opened in a corkscrew spire high above the rest of the
shantytown. The lady who operated it called herself Mrs Torrence,
and she was exquisitely antique but by no means frail, and when she
worked the ropes on her dumbwaiter to bring up supplies from the
loading area on the ground, her biceps stood at attention like
Popeye’s. There was a rumor that Mrs Torrence used to be a man, or
still was, under her skirts, but Suzanne didn’t pay attention to
it.

Lester came down the drive grinning and bouncing on the balls of
his feet. Perry had evidently been expecting him, for he came
racing through the shantytown and pelted down the roadway and threw
himself at Lester, grabbing him in a crazy, exuberant, whooping
hug. Francis gimped out a moment later and gave him a solemn
handshake. She hadn’t blogged their meeting in Detroit, so if
Francis and Perry knew about Lester’s transformation, they’d found
out without hearing it from her.

She finished recording the homecoming from Mrs Torrence’s crow’s
nest, then paid the grinning old bag and took the stairs two at a
time, hurrying to catch up with Lester and his crowd.

Lester accepted her hug warmly but distantly, letting go a fraction
of a second before she did. She didn’t let it get to her. He had
drawn a crowd now, with Francis’s protege printer-techs in the
innermost circle, and he was recounting the story of his
transformation. He had them as spellbound as a roomful of Ewoks
listening to C3PO.

“Shit, why don’t we sell that stuff?” Jason said. He’d taken a real
interest in the business end of their three-d printer project.

“Too much competition,” Lester said. “There are already a dozen
shops tooling up to make bathtub versions of the therapy here in
America. Hundreds more in Eastern Europe. There just won’t be any
profit in it by the time we get to market. Getting thin on the
cheap’s going to be \emph{easy}. Hell, all it takes to do it is the
stuff you’d use for a meth lab. You can buy all that in a kit from
a catalog.”

Jason nodded, but looked unconvinced.

Suzanne took Lester’s return as her cue to write about his
transformation. She snapped more pics of him, added some video. He
gave her ten minutes’ description of the therapies he’d undergone,
and named a price for the therapy that was substantially lower than
a couple weeks at a Hollywood fat-farm, and far more effective.

The response was amazing. Every TV news-crew in the greater Miami
area made a pilgrimage to their factory to film Lester working in a
tight t-shirt over a three-d printer, wrangling huge vats of
epoxy-mix goop in the sun with sweat beading over his big,
straining biceps.

Her message boards exploded. It seemed that a heretofore
unsuspected contingent of her growing readership was substantially
obese. And they had friends. Lester eventually gave up on posting,
just so he could get some work done. They had the printers to the
point where they could turn out new printers, but the whole system
was temperamental and needed careful nursing. Lester was more
interested in what people had to say on the engineering
message-boards than chatting with the fatties.

The fatties were skeptical and hopeful in equal measures. The big
fight was over whether there was anything to this, whether Lester
would keep the weight off, whether the new skinny Lester was really
Lester, whether he’d undergone surgery or had his stomach stapled.
America’s wallets had been cleaned out by so many snake-oil
peddlers with a “cure” for obesity that no one could believe what
they saw, no matter how much they wanted to.

Lord, but it was bringing in the readers, not to mention the
advertising dollars. The clearing price for a thousand weight-loss
ads targeted to affluent, obese English-speakers was over fifty
bucks, as compared with her customary CPM of three bucks a thou.
Inside of a week, she’d made enough to buy a car. It was weird
being her own circulation and ad-sales department, but it wasn’t as
hard as she’d worried it might be\dash{}and it was intensely satisfying
to have such a nose-to-tail understanding of the economics of her
production.

“You should go,” Lester told her as she clicked him through her
earnings spreadsheet. “Jesus, this is insane. You know that these
fatties actually follow me around on the net now, asking me
questions in message boards about engineering? The board moderators
are asking me to post under an assumed name. Madame, your public
has spoken. There is a dire need for your skills in St Petersburg.
Go. They have chandeliers in the subways and caviar on tap. All the
blini you can eat. Bear steaks.”

She shook her head and slurped at the tea he’d brought her. “You’re
joking. It’s all mafiyeh there. Scary stuff. Besides, I’m covering
this beat right now, New Work.”

“New Work isn’t going anywhere, Suzanne. We’ll be here when you get
back. And this story is one that needs your touch. They’re
micro-entrepreneurs solving post-industrial problems. It’s the same
story you’ve been covering here, but with a different angle. Take
that money and buy yourself a business-class ticket to St
Petersburg and spend a couple weeks on the job. You’ll clean up.
They could use the publicity, too\dash{}someone to go and drill down on
which clinics are legit and which ones are clip-joints. You’re
perfect for the gig.”

“I don’t know,” she said. She closed her eyes. Taking big chances
had gotten her this far and it would take her farther, she knew.
The world was your oyster if you could stomach a little risk.

“Yeah,” she said. “Yeah, hell yeah. You’re totally right, Lester.”

“Zasterovyeh!”

“What you said!”

“It’s cheers,” he said. “You’ll need to know that if you’re going
to make time in Petrograd. Let me go send some email and get you
set up. You book a ticket.”

\begin{center}\rule{3in}{0.4pt}\end{center}

And just like that she was off to Russia. Lester insisted that she
buy a business-class ticket, and she discovered to her bemusement
that British Airways had about three classes above business,
presumably with even more exclusive classes reserved to royalty and
peers of the realm. She luxuriated in fourteen hours of reclining
seats and warm peanuts and in-flight connectivity, running a brief
videoconference with Lester just because she could. Tjan had sent
her a guide to the hotels and she’d opted for the Pribaltiyskaya, a
crumbling Stalin-era four-star of spectacular, Vegasesque
dimensions. The facade revealed the tragedy of the USSR’s
unrequited love-affair with concrete, as did the cracks running up
the walls of the lobby.

They checked her into the hotel with the nosiest questionnaire
ever, a two-pager on government stationary that demanded to know
her profession, employer, city of birth, details of family, and so
forth. An American businessman next to her at the check-in counter
saw her puzzling over it. “Just make stuff up,” he said. “I always
write that I come from 123 Fake Street, Anytown, California, and
that I work as a professional paper-hanger. They don’t check on it,
except maybe the mob when they’re figuring out who to mug. First
time in Russia?”

“It shows, huh?”

“You get used to it,” he said. “I come here every month on
business. You just need to understand that if it seems ridiculous
and too bad to be true, it is. They have lots of rules here, but no
one follows ’em. Just ignore any unreasonable request and you’ll
fit right in.”

“That’s good advice,” she said. He was middle-aged, but so was she,
and he had nice eyes and no wedding ring.

“Get a whole night’s sleep, don’t drink the so-called ’champagne’
and don’t change money on the streets. Did you bring melatonin and
modafinil?”

She stared blankly at him. “Drugs?”

“Sure. One tonight to sleep, one in the morning to wake up, and do
it again tomorrow and you’ll be un-lagged. No booze or caffeine,
either, not for the first couple days. Melatonin’s over the
counter, even in the States, and modafinil’s practically legal. I
have extra, here.” He dug in his travel bag and came up with some
generic Walgreens bottles.

“That’s OK,” she said, handing her credit card to a pretty young
clerk. “Thanks, though.”

He shook his head. “It’s your funeral,” he said. “Jet-lag is way
worse for you than this stuff. It’s over the counter stateside. I
don’t leave home without it. Anyway, I’m in room 1422. If it’s two
in the morning and you’re staring at the ceiling and regretting it,
call me and I’ll send some down.”

Was he hitting on her? Christ, she was so tired, she could barely
see straight. There was no way she was going to need any help
getting to sleep. She thanked him again and rolled her suitcase
across the cavernous lobby with its gigantic chandeliers and to the
elevators.

But sleep didn’t come. The network connection cost a
fortune\dash{}something she hadn’t seen in years\dash{}and the number of worms
and probes bouncing off her firewall was astronomical. The
connection was slow and frustrating. Come 2AM, she was, indeed,
staring at the ceiling.

Would you take drugs offered by a stranger in a hotel lobby? They
were in a \emph{Walgreens bottle} for chrissakes. How bad could
they be? She picked up the house-phone on the chipped bedstand and
punched his hotel room.

“Lo?”

“Oh Christ, I woke you up,” she said. “I’m sorry.”

“’Sok. Lady from check-in, right? Gimme your room number, I’ll send
up a melatonin now and a modafinil for the morning. No sweatski.”

“Uh,” she hadn’t thought about giving a strange man her room
number. In for a penny, in for a pound. “2813,” she said.
“Thanks.”

“Geoff,” he said. “It’s Geoff. New York\dash{}upper West Side. Work in
health products.”

“Suzanne,” she said. “Florida, lately. I’m a writer.”

“Good night, Suzanne. Pills are en route.”

“Good night, Geoff. Thanks.”

“Tip the porter a euro, or a couple bucks. Don’t bother with
rubles.”

“Oh,” she said. It had been a long time since her last visit
overseas. She’d forgotten how much minutiae was involved.

He hung up. She put on a robe and waited. The porter took about
fifteen minutes, and handed her a little envelope with two pills in
it. He was about fifteen, with a bad mustache and bad skin, and bad
teeth that he displayed when she handed him a couple of dollar
bills.

A minute later, she was back on the phone.

“Which one is which?”

“Little white one is melatonin. That’s for now. My bad.”

She saw him again in the breakfast room, loading a plate with
hard-boiled eggs, potato pancakes, the ubiquitous caviar, salami,
and cheeses. In his other hand he balanced a vat of porridge with
strawberry jam and enough dried fruit to keep a parrot zoo happy
for a month.

“How do you keep your girlish figure if you eat like that?” she
said, settling down at his table.

“Ah, that’s a professional matter,” he said. “And I make it a point
never to discuss bizniz before I’ve had two cups of coffee.” He
poured himself a cup of decaf. “This is number two.”

She picked her way through her cornflakes and fruit salad. “I
always feel like I don’t get my money’s worth out of buffet
breakfasts,” she said.

“Don’t worry,” he said. “I’ll make up for you.” He pounded his
coffee and poured another cup. “Humanity returns,” he said, rubbing
his thighs. “Marthter, the creature waketh!” he said in high Igor.

She laughed.

“You are really into, uh, \emph{substances}, aren’t you?” she
said.

“I am a firm believer in better living through chemistry,” he said.
He pounded another coffee. “Ahhh. Coffee and modafinil are an
amazing combo.”

She’d taken hers that morning when the alarm got her up. She’d been
so tired that it actually made her feel nauseated to climb out of
bed, but the modafinil was getting her going. She knew a little
about the drug, and figured that if the TSA approved it for use by
commercial pilots, it couldn’t be that bad for you.

“So, my girlish figure. I work for a firm that has partners here in
Petersburg who work on cutting-edge pharma products, including some
stuff the FDA is dragging its heels on, despite widespread
acceptance in many nations, this one included. One of these is a
pill that overclocks your metabolism. I’ve been on it for a year
now, and even though I am a stone calorie freak and pack away five
or six thousand calories a day, I don’t gain an ounce. I actually
have to remember to eat enough so that my ribs don’t start
showing.”

Suzanne watched him gobble another thousand calories. “Is it
healthy?”

“Compared to what? Being fat? Yes. Running ten miles a day and
eating a balanced diet of organic fruit and nuts? No. But when the
average American gets the majority of her calories from soda-pop,
’healthy’ is a pretty loaded term.”

It reminded her of that talk with Lester, a lifetime ago in the
IHOP. Slowly, she found herself telling him about Lester’s story.

“Wait a second, you’re Suzanne \emph{Church}? New Work Church? San
Jose Mercury News Church?”

She blushed. “You can’t \emph{possibly} have heard of me,” she
said.

He rolled his eyes. “Sure. I shoulder-surfed your name off the
check-in form and did a background check on you last night just so
I could chat you up over breakfast.”

It was a joke, but it gave her a funny, creeped-out feeling.
“You’re kidding?”

“I’m kidding. I’ve been reading you for freaking \emph{years}. I
followed Lester’s story in detail. Professional interest. You’re
the voice of our generation, woman. I’d be a philistine if I didn’t
read your column.”

“You’re not making me any less embarrassed, you know.” It took an
effort of will to keep from squirming.

He laughed hard enough to attract stares. “All right, I \emph{did}
spend the night googling you. Better?”

“If that’s the alternative, I’ll take famous, I suppose,” she
said.

“You’re here writing about the weight loss clinics, then?”

“Yes,” she said. It wasn’t a secret, but she hadn’t actually gone
out of her way to mention it. After all, there might not be any
kind of story after all. And somewhere in the back of her mind was
the idea that she didn’t want to tip off some well-funded newsroom
to send out its own investigative team and get her scoop.

“That is fantastic,” he said. “That’s just, wow, that’s the best
news I’ve had all year. You taking an interest in our stuff, it’s
going to really push it over the edge. You’d think that selling
weight-loss to Americans would be easy, but not if it involves any
kind of travel: 80 percent of those lazy insular fucks don’t even
have passports. Ha. Don’t quote that. Ha.”

“Ha,” she said. “Don’t worry, I won’t. Look, how about this, we’ll
meet in the lobby around nine, after dinner, for a cup of coffee
and an interview?” She had gone from intrigued to flattered to
creeped-out with this guy, and besides, she had her first clinic
visit scheduled for ten and it was coming up on nine and who knew
what a Russian rush-hour looked like?

“Oh. OK. But you’ve got to let me schedule you for a visit to some
of our clinics and plants\dash{}just to see what a professional shop we
run here. No gold-teeth-shiny-suit places like you’d get if you
just picked the top Google AdWord. Really American-standard places,
better even, Scandinavian-standard, a lot of our doctors come over
from Sweden and Denmark to get out from under the socialist
medicine systems there. They run a tight ship, ya shore, you
betcha,” he delivered this last in a broad Swedish bork-bork-bork.

“Um,” she said. “It all depends on scheduling. Let’s sort it out
tonight, OK?”

“OK,” he said. “Can’t \emph{wait}.” He stood up with her and gave
her a long, two-handed handshake. “It’s a real honor to meet you,
Suzanne. You’re one of my real heros, you know that?”

“Um,” she said again. “Thanks, Geoff.”

He seemed to sense that he’d come on too strong. He looked like he
was about to apologize.

“That’s really kind of you to say,” she said. “It’ll be good to
catch up tonight.”

He brightened. It was easy enough to be kind, after all.

She had the front desk call her a taxi\dash{}she’d been repeatedly warned
off of gypsy cabs and any vehicle that one procured by means of a
wandering tout. She got into the back, had the doorman repeat the
directions to Lester’s clinic twice to the cabbie, watched him
switch on the meter and checked the tariff, then settled in to
watch St Petersburg go flying by.

She switched on her phone and watched it struggle to associate with
a Russian network. They were on the road for all of five
minutes\dash{}long enough to note the looming bulk of the Hermitage and
the ripples left by official cars slicing through the traffic with
their blue blinking lights\dash{}when her phone went nutso. She looked at
it\dash{}she had ten texts, half a dozen voicemails, a dozen new clipped
articles, and it was ringing with a number in New York.

She bumped the New York call to voicemail. She didn’t recognize the
number. Besides, if the world had come to an end while she was
asleep, she wanted to know some details before she talked to anyone
about it. She paged back through the texts in reverse
chronological\dash{}the last five were increasingly panicked messages
from Lester and Perry. Then one from Tjan. Then one from
Kettlebelly. They all wanted to discuss “the news” whatever that
was. One from her old editor at the Merc asking if she was
available for comment about “the news.” Tjan, too. The first one
was from Rat-Toothed Freddy, that snake.

“Kodacell’s creditors calling in debts. Share price below one cent.
Imminent NASDAQ de-listing. Comments?”

Her stomach went cold, her breakfast congealed into a hard lump.
The clipped articles had quotes from Kettlewell (“We will see to it
that all our employees are paid, our creditors are reimbursed, and
our shareholders are well-done-by through an orderly wind-down”),
Perry (“Fuck it\dash{}I was doing this shit before Kodacell, don’t expect
to stop now”) and Lester (“It was too beautiful and cool to be
real, I guess.”) Where she was mentioned, it was usually in a snide
context that made her out to be a disgraced pitchwoman for a failed
movement.

Which she was. Basically.

Her phone rang. Kettlewell.

“Hi, Kettlewell,” she said.

“Where have you been?” he said. He sounded really edgy. It was the
middle of the night in California.

“I’m in St Petersburg,” she said. “In Russia. I only found out
about ten seconds ago. What happened?”

“Oh Christ. Who knows? Cascading failure. Fell short of last
quarter’s estimates, which started a slide. Then a couple lawsuits
filed. Then some unfavorable press. The share price kept falling,
and things got worse. Your basic clusterfuck.”

“But you guys had great numbers overall\dash{}”

“Sure, if you looked at them our way, they were great. If you
looked at them the way the Street looks at them, we were in deep
shit. Analysts couldn’t figure out how to value us. Add a little
market chaos and some old score-settling assholes, like that fucker
Freddy, and it’s a wonder we lasted as long as we did. They’re
already calling us the twenty first century Enron.”

“Kettlewell,” she said, “I lived through a couple of these, and
something’s not right. When the dotcoms were going under, their
CEOs kept telling everyone everything was all right, right up to
the last minute. They didn’t throw in the towel. They stood like
captains on the bridge of sinking ships.”

“So?”

“So what’s going on here. It sounds like you’re whipped. Why aren’t
you fighting? There were lots of dotcoms that tanked, but a few of
those deep-in-denial CEOs pulled it off, restructured and came out
of it alive. Why are you giving up?”

“Suzanne, oh, Suzanne.” He laughed, but it wasn’t a happy laugh.
“You think that this happened overnight? You think that this
problem just cropped up yesterday and I tossed in the towel?”

Oh. “Oh.”

“Yeah. We’ve been tanking for months. I’ve been standing on the
bridge of this sinking ship with my biggest smile pasted on for two
consecutive quarters now. I’ve thrown out the most impressive
reality distortion field the business world has ever seen. Just
because I’m giving up doesn’t mean I gave up without a fight.”

Suzanne had never been good at condolences. She hated funerals.
“Landon, I’m sorry. It must have been very hard\dash{}”

“Yeah,” he said. “Well, sure. I wanted you to have the scoop on
this, but I had to talk to the press once the story broke, you
understand.”

“I understand,” she said. “Scoops aren’t that important anyway.
I’ll tell you what. I’ll post a short piece on this right away,
just saying, ’Yes, it’s true, and I’m getting details. Then I’ll do
interviews with you and Lester and Perry and put up something
longer in a couple of hours. Does that work?”

He laughed again, no humor in it. “Yeah, that’ll be \emph{fine}.”

“Sorry, Kettlewell.”

“No, no,” he said. “No, it’s OK.”

“Look, I just want to write about this in a way that honors what
you’ve done over the past two years. I’ve never been present at the
birth of anything remotely this important. It deserves to be
described well.”

It sounded like he might be crying. There was a snuffling sound.
“You’ve been amazing, Suzanne. We couldn’t have done it without
you. No one could have described it better. Great deeds are
irrelevant if no one knows about them or remembers them.”

Her phone was beeping. She snuck a peek. It was her old editor.
“Listen,” she said. “I have to go. There’s a call coming in I
\emph{have} to take. I can call you right back.”

“Don’t,” he said. “It’s OK. I’m busy here anyway. This is a big
day.” His laugh was like a dog’s bark.

“Take care of yourself, Kettlewell,” she said. “Don’t let the
bastards grind you down.”

“Nil carborundum illegitimis to you, too.”

She clicked over to her editor. “Jimmy,” she said. “Long time no
speak. Sorry I missed your calls before\dash{}I’m in Russia on a story.”

“Hello, Suzanne,” he said. His voice had an odd, strained quality,
or maybe that was just her mood, projecting. “I’m sorry, Suzanne.
You’ve been doing good work. The best work of your career, if you
ask me. I follow it closely.”

It made her feel a little better. She’d been uncomfortable about
the way she and Jimmy had parted ways, but this was vindicating. It
emboldened her. “Jimmy, what the hell do I do now?”

“Christ, Suzanne, I don’t know. I’ll tell you what not to do,
though. Off the record.”

“Off the record.”

“Don’t do what I’ve done. Don’t hang grimly onto the last planks
from the sinking ship, chronicling the last few struggling, sinking
schmucks’ demise. It’s no fun being the stenographer for the fall
of a great empire. Find something else to cover.”

The words made her heart sink. Poor Jimmy, stuck there in the
Merc’s once-great newsroom, while the world crumbled around him. It
must have been heartbreaking.

“Thanks,” she said. “You want an interview?”

“What? No, woman. I’m not a ghoul. I wanted to call and make sure
you were all right.”

“Jimmy, you’re a prince. But I’ll be OK. I land on my feet. You’ve
got someone covering this story, so give her my number and have her
call me and I’ll give her a quote.”

“Really, Suzanne\dash{}”

“It’s \emph{fine}, Jimmy.”

“Suzanne,” he said. “We don’t cover that kind of thing from our
newsroom anymore. Just local stuff. National coverage comes from
the wires or from the McClatchy national newsroom.”

She sucked in air. Could it be possible? Her first thought when
Jimmy called was that she’d made a terrible mistake by leaving the
Merc, but if this was what the paper had come to, she had left just
in time, even if her own life-raft was sinking, it had kept her
afloat for a while.

“The offer still stands, Jimmy. I’ll talk to anyone you want to
assign.”

“You’re a sweetheart, Suzanne. What are you in Russia for?”

She told him. Screw scoops, anyway. Not like Jimmy was going to
send anyone to \emph{Russia}, he couldn’t even afford to dispatch a
reporter to Marin County by the sounds of things.

“What a story!” he said. “Man!”

“Yeah,” she said. “Yeah I guess it is.”

“You \emph{guess}? Suzanne, this is the single most important issue
in practically every American’s life\dash{}there isn’t one in a thousand
who doesn’t worry endlessly about his weight.”

“Well, I have been getting really good numbers on this.” She named
the figure. He sucked air between his teeth. “That’s what the whole
freaking \emph{chain} does on a top story, Suzanne. You’re
outperforming fifty local papers \emph{combined.}”

“Yeah?”

“Hell yeah,” he said. “Maybe I should ask you for a job.”

When he got off the phone, she spoke to Perry, and then to Lester.
Lester said that he wanted to go traveling and see his old friends
in Russia and that if she was still around in a couple weeks, maybe
he’d see her there. Perry was morose and grimly determined. He was
on the verge of shipping his three-d printers and he was sure he
could do it, even if he didn’t have the Kodacell network for
marketing and logistics. He didn’t even seem to register it when
she told him that she was going to be spending some time in
Russia.

Then she had to go into the clinic and ask intelligent questions
and take pictures and record audio and jot notes and pay attention
to the small details so that she would be able to write the best
account possible.

They dressed well in Russia, in the clinics. Business casual, but
well tailored and made from good material. The Europeans knew from
textiles, and expert tailoring seemed to be in cheap supply here.

She’d have to get someone to run her up a blue blazer and a white
shirt and a decent skirt. It would be nice to get back into
grown-up clothes after a couple years’ worth of Florida casual.

She’d see Geoff after dinner that night, get more detail for the
story. There was something big here in the medical tourism
angle\dash{}not just weight loss but gene therapy, too, and voodoo
stem-cell stuff and advanced prostheses and even some crazy
performance enhancement stuff that had kept Russia out of the past
Olympics.

She typed her story notes and answered the phone calls. One special
call she returned once she was sitting in her room, relaxed, with a
cup of coffee from the in-room coffee-maker.

“Hello, Freddy,” she said.

“Suzanne, darling!” He sounded like he was breathing hard.

“What can I do for you?”

“Just wanted a quote, love, something for color.”

“Oh, I’ve got a quote for you.” She’d given the quote a lot of
thought. Living with the squatters had broadened her vocabulary
magnificently.

“And those are your good points,” she said, taking a sip of coffee.
“Goodbye, Freddy.”

\chapter{PART II}

The drive from Orlando down to Hollywood got worse every time Sammy
took it. The turnpike tolls went up every year and the road surface
quality declined, and the gas prices at the clip-joints were
heart-attack-inducing. When Sammy started at Disney Imagineering a
decade before, the company had covered your actual expenses\dash{}just
collect the receipts and turn them in for cash back. But since
Parks had been spun off into a separate company with its own
shareholders, the new austerity measures meant that the
bean-counters in Burbank set a maximum per-mile reimbursement and
never mind the actual expense.

Enough of this competitive intelligence work and Sammy would go
broke.

Off the turnpike, it was even worse. The shantytowns multiplied and
multiplied. Laundry lines stretched out in the parking-lots of
former strip-malls. Every traffic-light clogged with aggressive
techno-tchotchke vendors, the squeegee bums of the twenty-first
century, with their pornographic animatronic dollies and infinitely
varied robot dogs. Disney World still sucked in a fair number of
tourists (though not nearly so many as in its golden day), but they
were staying away from Miami in droves. The snowbirds had died off
in a great demographic spasm over the past decade, and their
children lacked the financial wherewithal to even think of
over-wintering in their parents’ now-derelict condos.

The area around the dead Wal-Mart was particularly awful. The
shanties here rose three, even four stories into the air, clustered
together to make medieval street-mazes. Broward County had long
since stopped enforcing the property claims of the bankruptcy
courts that managed the real-estate interests of the former owners
of the fields and malls that had been turned into the new towns.

By the time he pulled into the Wal-Mart’s enormous parking lot, the
day had heated up, his air-con had conked, and he’d accumulated a
comet-tail of urchins who wanted to sell him a computer-generated
bust of himself in the style of a Roman emperor\dash{}they worked on
affiliate commission for some three-d printer jerk in the shanties,
and they had a real aggressive pitch, practically flinging their
samples at him.

He pushed past them and wandered through the open-air market
stalls, a kind of cruel parody of the long-gone Florida
flea-markets. These gypsies sold fabricated parts that could be
modded to make single-shot zip guns and/or bongs and/or
illegal-gain wireless antennae. They sold fruit smoothies and
suspicious “beef” jerky. They sold bootleg hardcopies of Mexican
fotonovelas and bound printouts of Japanese fan-produced
tentacle-porn comics. It was all damnably eye-catching and
intriguing, even though Sammy knew that it was all junk.

Finally, he reached the ticket-window in front of the Wal-Mart and
slapped down five bucks on the counter. The guy behind the counter
was the kind of character that kept the tourists away from Florida:
shaven-headed, with one cockeyed eyebrow that looked like a set of
hills, a three-day beard and skin tanned like wrinkled leather.

“Hi again!” Sammy said, brightly. Working at Disney taught you to
talk happy even when your stomach was crawling\dash{}the castmember’s
grin.

“Back again?” the guy behind the counter laughed. He was missing a
canine tooth and it made him look even more sketchy. “Christ, dude,
we’ll have to invent a season’s pass for you.”

“Just can’t stay away,” Sammy said.

“You’re not the only one. You’re a hell of a customer for the ride,
but you haven’t got anything on some of the people I get
here\dash{}people who come practically every day. It’s flattering, I tell
you.”

“You made this, then?”

“Yeah,” he said, swelling up with a little pigeon-chested puff of
pride. “Me and Lester, over there.” He gestured at a fit, greying
man sitting on a stool before a small cocktail bar built into a
scavenged Orange Julius stand\dash{}God knew where these people got all
their crap from. He had the look of one of the fatkins, unnaturally
thin and muscled and yet somehow lazy, the combination of a ten
kilocalorie diet, zero body-fat and non-steroidal muscle enhancers.
Ten years ago, he would have been a model, but today he was just
another ex-tubbalard with a serious food habit. Time was that
Disney World was nigh-unnavigable from all the powered wheelchairs
carting around morbidly obese Americans who couldn’t walk from ride
to ride, but these days it looked more like an ad for a gymnasium,
full of generically buff fatkins in tight-fitting clothes.

“Good work!” he said again in castmemberese. “You should be very
proud!”

The proprietor smiled and took a long pull off a straw hooked into
the distiller beside him. “Go on, get in there\dash{}flatterer!”

Sammy stepped through the glass doors and found himself in an
air-conditioned cave of seemingly infinite dimension. The old
Wal-Mart had been the size of five football fields, and a cunning
arrangement of curtains and baffles managed to convey all that
space without revealing its contents. Before him was the ride
vehicle, in a single shaft of spotlight.

Gingerly, he stepped into it. The design was familiar\dash{}there had
been a glut of these things before the fatkins movement took hold,
stair-climbing wheelchairs that used gyro-stabilizers to pitch,
yaw, stand and sit in a perpetual controlled fall. The Disney World
veterans of their heyday remembered them as failure-prone behemoths
that you needed a forklift to budge when they died, but the ride
people had done something to improve on the design. These things
performed as well as the originals, though they were certainly
knock-offs\dash{}no\emph{how} were these cats shelling out fifty grand a
pop for the real deal.

The upholstered seat puffed clouds of dust into the spotlight’s
shaft as he settled into the chair and did up his lap-belt. The
little LCD set into the control panel lit up and started to play
the standard video spiel, narrated in grizzled voice-over.

\begin{speaker}
WELCOME TO THE CABINET OF WONDERS

THERE WAS A TIME WHEN AMERICA HELD OUT THE PROMISE OF A NEW WAY OF
LIVING AND WORKING. THE NEW WORK BOOM OF THE TEENS WAS A PERIOD OF
UNPARALLELED INVENTION, A CAMBRIAN EXPLOSION OF CREATIVITY NOT SEEN
SINCE THE TIME OF EDISON\dash{}AND UNLIKE EDISON, THE PEOPLE WHO INVENTED
THE NEW WORK REVOLUTION WEREN’T RIP-OFF ARTISTS AND FRAUDS.

THEIR MARVELOUS INVENTIONS EMERGED AT THE RATE OF FIVE OR SIX PER
WEEK. SOME DANCED, SOME SANG, SOME WERE HELPMEETS AND SOME WERE
MERE JESTERS.

TODAY, NEARLY ALL OF THESE WONDERFUL THINGS HAVE VANISHED WITH THE
COLLAPSE OF NEW WORK. THEY’VE ENDED UP BACK IN THE TRASH HEAPS THAT
INSPIRED THEM.

HERE IN THE CABINET OF WONDERS, WE ARE PRESERVING THESE LAST
REMNANTS OF THE GOLDEN AGE, A SINGLE BEACON OF LIGHT IN A TIME OF
DARKNESS.

AS YOU MOVE THROUGH THE RIDESPACE, PLEASE REMAIN SEATED. HOWEVER,
YOU MAY PAUSE YOUR VEHICLE TO GET A CLOSER LOOK BY MOVING THE
JOYSTICK TOWARD YOURSELF. PULL THE JOYSTICK UP TO CUE NARRATION
ABOUT ANY OBJECT.

MOVE THE JOYSTICK TO THE LEFT, TOWARDS THE MINUS-ONE, IF YOU THINK
AN ITEM IS UGLY, UNWORTHY OR MISPLACED. MOVE THE JOYSTICK TO THE
RIGHT, TOWARD THE PLUS-ONE, IF YOU THINK AN ITEM IS PARTICULARLY
PLEASING. YOUR FEEDBACK WILL BE FACTORED INTO THE CONTINUOUS
REARRANGEMENT OF THE CABINET, WHICH TAKES PLACE ON A
MINUTE-BY-MINUTE BASIS, DRIVEN BY THE ROBOTS YOU MAY SEE CRAWLING
AROUND THE FLOOR OF THE CABINET.

THE RIDE LASTS BETWEEN TEN MINUTES AND AN HOUR, DEPENDING ON HOW
OFTEN YOU PAUSE.

PLEASE ENJOY YOURSELF, AND REMEMBER WHEN WE WERE GOLDEN.
\end{speaker}

This plus-one/minus-one business was new to him. It had been a mere
four days since he’d been up here, but like so many other of his
visits, they’d made major rehabs to their ride in the amount of
time it would have taken Imagineering to write a memo about the
possibility of holding a design-review meeting.

He velcroed his camera’s wireless eye to his lapel, tapped the
preset to correct for low light and motion, and hit the joystick.
The wheelchair stood up with wobbly grace, and began to roll
forward on two wheels, heeling over precipitously as it cornered
into the main space of the ride. The gyros could take it, he knew,
but it still thrilled him the way that a fast, out-of-control
go-kart did, miles away from the safe rides back in Disney.

The chair screeched around a corner and pulled into the first
scene, a diorama littered with cross-sectioned cars. Each one was
kitted out with different crazy technologies\dash{}dashboard gods that
monitored and transmitted traffic heuristics, parallel-parking
autopilots, peer-to-peer music-sharing boxes, even an amphibious
retrofit on a little hybrid that apparently worked, converting the
little Bug into a water-Bug.

The chair swooped around each one, pausing while the narration
played back reminisces by the inventors, or sometimes by the owners
of the old gizmos. The stories were pithy and sweet and always
funny. These were artifacts scavenged from the first days of a
better nation that had died a-borning.

Then on to the kitchen, and the bathrooms\dash{}bathroom after bathroom,
with better toilets, better showers, better tubs, better floors and
better lights\dash{}bedrooms, kids’ rooms. One after another, a hyper
museum.

The decor was miles ahead of where it had been the last time he’d
been through. There were lots of weird grace-notes, like
taxidermied alligators, vintage tourist pennants, chintz lamps, and
tiny dioramae of action figures.

He paused in front of a fabric printer surrounded by custom tees
and knit caps and three-d video-game figurines machine-crocheted
from bright yarns, and was passed by another chair. In it was a
cute woman in her thirties, white-blond shaggy hair luminous in the
spotlight over the soft-goods. She paused her chair and lovingly
reached out to set down a pair of appliqued shorts with organic
LEDs pulsing and swirling around the waistband. “Give it a
plus-one, OK? These were my best sellers,” she said, smiling a
dazzling beach-bunny smile at him. She wheeled away and paused at
the next diorama to set down a doll-house in a child’s room
diorama.

Wow\dash{}they were getting user-generated content in the \emph{ride}.
Holy crap.

He finished out the ride with a keen hand on the plus-one/minus-one
lever, carefully voting for the best stuff and against the stuff
that looked out of place\dash{}like a pornographic ceramic bong that
someone had left in the midst of a clockwork animatronic jug-band
made from stitched-together stuffed animals.

Then it was over, and he was debarking in what had been the
Wal-Mart’s garden center. The new bright sun made him tear up, and
he fished out his shades.

“Hey, mister, c’mere, I’ve got something better than sunglasses for
you!” The guy who beckoned him over to a market-stall had the look
of an aging bangbanger: shaved head, tattoos, ridiculous cycling
shorts with some gut hanging over them.

“See these? Polarizing contact-lenses\dash{}prescription or optically
neutral. Everyone in India is into these things, but we make ’em
right here in Florida.” He lifted a half-sphere of filmy plastic
from his case and peeled back his eyelid and popped it in. His
whole iris was tinted black, along with most of the whites of his
eyes. Geometric shapes like Maori tattoos were rendered in charcoal
grey across the lenses. “I can print you up a set in five minutes,
ten bucks for plain, twenty if you want them bit-mapped.”

“I think I’ll stick with my shades, thanks,” Sammy said.

“C’mon, the ladies love these things. Real conversation starter.
Make you look all anime and shit, guy like you can try this kind of
thing out for twenty bucks, you know, won’t hurt.”

“That’s all right,” Sammy said.

“Just try a pair on, then, how about that. I printed an extra set
last Wednesday and they’ve only got a shelf-life of a week, so
these’ll only be good for another day. Fresh in a sealed package.
You like ’em. you buy a pair at full price, c’mon that’s as good as
you’re going to get.”

Before Sammy knew it, he was taking receipt of a sealed plastic
packet in hot pink with a perforated strip down one side. “Uh,
thanks\ldots{}” he said, as he began to tuck it into a pocket. He hated
hard-sells, he was no good at them. It was why he bought all his
cars online now.

“Naw, that’s not the deal, you got to try them on, otherwise how
can you buy them once you fall in love with ’em? They’re safe man,
go on, it’s easy, just like putting in a big contact lens.”

Sammy thought about just walking away, but the other vendors were
watching him now, and the scrutiny sapped his will. “My hands are
too dirty for this,” he said. The vendor silently passed him a
sealed sterile wipe, grinning.

Knowing he was had, he wiped his hands, tore open the package, took
out the lenses and popped them one at a time into his eyes. He
blinked a couple times. The world was solarized and grey, like he
was seeing it through a tinted windscreen.

“Oh man, you look bad-\emph{ass},” the vendor said. He held up a
hand mirror.

Sammy looked. His eyes were shiny black beads, like a mouse’s eyes,
solid save for a subtle tracery of Mickey Mouse heads at the
corners. The trademark infringement made him grin, hard and
spitless. He looked ten years younger, like those late-teen
hipsters whose parents dragged them to Walt Disney World, who
showed up in bangbanger threads and sneered and scratched their
groins and made loud remarks about how suckballs it all was. His
conservative buzz-cut looked more like a retro-skinhead thing, and
his smooth-shaved, round cheeks made him boyish.

“Those are good for two days tops\dash{}your eyes start getting itchy,
you just toss ’em. You want a pair that’s good for a week, twenty
dollah with the Mickeys. I got Donalds and Astro Boys and all kinds
of shit, just have a look through my flash book. Some stuff I drew
myself, even.”

Playing along now, Sammy let himself be led on a tour of the
flash-book, which featured the kind of art he was accustomed to
seeing in tattoo parlor windows: skulls and snakes and scorpions
and naked ladies. Mickey Mouse giving the finger, Daisy Duck with a
strap-on, Minnie Mouse as a dominatrix. The company offered a
bounty for turning in trademark infringers, but somehow he doubted
that the company lawyers would be able to send this squatter a
cease-and-desist letter.

In the end, he bought one of each of the Disney sets.

“You like the mouse, huh?”

“Sure,” he said.

“I never been. Too expensive. This is all the ride I want, right
here.” He gestured at the dead Wal-Mart.

“You like that huh?”

“Man, it’s cool! I go on that sometimes, just to see what it’s
turned into. I like that it’s always different. And I like that
people add their own stuff. It makes me feel, you know\ldots{}”

“What?”

Suddenly, the vendor dropped his hard-case bangbanger facade.
“Those were the best days of my life. I was building three-d
printers, making them run. My older brother liked to fix cars, and
so did my old man, but who needs a car, where you going to go? The
stuff I built, man, it could make \emph{anything}. I don’t know why
or how it ended, but while it was going, I felt like the king of
the goddamned world.”

It felt less fun and ironic now. There were tears bright on the
vendor’s black-bead eyes. He was in his mid-twenties, younger than
he’d seemed at first. If he’d been dressed like a suburban
home-owner, he would have looked like someone smart and
accomplished, with lively features and clever hands. Sammy felt
obscurely ashamed.

“Oh,” he said. “Well, I spent those years working a straight job,
so it didn’t really touch me.”

“That’s your loss, man,” the vendor said. The printer behind him
was spitting out the last of Sammy’s contact-lenses, in sealed
plastic wrap. The vendor wrapped them up and put them in a brown
liquor-store bag.

Sammy plodded through the rest of the market with his paper bag. It
was all so depressing. The numbers at Disney World were down, way
down, and it was his job to figure out how to bring them up again,
without spending too much money. He’d done it before a couple of
times, with the live-action role-playing stuff, and with the
rebuild of Fantasyland as an ironic goth hangout (being a wholly
separate entity from the old Walt Disney Company had its
advantages). But to do it a third time\dash{}Christ, he had no idea how
he’d get there. These weird-ass Wal-Mart squatters had seemed
promising, but could you possibly transplant something like this to
a high-throughput, professional location-based entertainment
product?

The urchins were still in the parking lot with their Roman emperor
busts. He held his hands out to ward them off and found himself
holding onto a bust of his own head. One of the little rats had
gotten a three-d scan of his head while he was walking by and had
made the bust on spec. He looked older in Roman emperor guise than
he did in his mind’s eye, old and tired, like an emperor in
decline.

“Twenty dollah man, twenty, twenty,” the kid said. He was about 12,
and still chubby, with long hair that frizzed away from his head in
a dandelion halo.

“Ten,” Sammy said, clutching his tired head. It was smooth as epoxy
resin, and surprisingly light. There was a lot of different goop
you could run through those three-d printers, but whatever they’d
used for this, it was featherweight.

The kid looked shrewd. “Twenty dollah and I get rid of these other
kids, OK?”

Sammy laughed. He passed the kid a twenty, taking care to tuck his
wallet deep into the inside pocket of his jacket. The kid whistled
shrilly and the rest of the kids melted away. The entrepreneur made
the twenty disappear, tapped the side of his nose, and took off
running back into the market stalls.

It was hot and muggy and Sammy was tired, and the drive back to
Orlando was another five hours if the traffic was against him\dash{}and
these days, everything was against him.

\begin{center}\rule{3in}{0.4pt}\end{center}

Perry’s funny eyebrow twitched as he counted out the day’s take.
This gig was all cream, all profit. His overheads amounted to a
couple hundred a month to Jason and his crew to help with the robot
and machinery maintenance in the Wal-Mart, half that to some of the
shantytown girls to dust and sweep after closing, and a retainer to
a bangbanger pack that ran security at the ride and in the market.
Plus he got the market-stall rents, and so when the day was over,
only the first hundred bucks out of the till went into overheads
and the rest split even-steven with Lester.

Lester waited impatiently, watching him count twice before
splitting the stack. Perry rolled up his take and dropped it into a
hidden pocket sewn into his cargo shorts.

“Someday you’re going to get lucky and some chick is going to reach
down and freak out, buddy,” Lester said.

“Better she finds my bank-roll than my prostate,” Perry said.
Lester spent a lot of time thinking about getting lucky, making up
for a lifetime of bad luck with girls.

“OK, let’s get changed,” Lester said. As usual, he was wearing
tight-fitting jeans that owed a little debt to the bangbanger
cycling shorts, something you would have had to go to a gay bar to
see when Perry was in college. His shirt clung to his pecs and was
tailored down to his narrow waist. It was a fatkins style, the kind
of thing you couldn’t wear unless you had a uniquely adversarial
relationship with your body and metabolism.

“No, Lester, no.” Perry said. “I said I’d go on this double date
with you, but I didn’t say anything about letting you dress me up
for it.” The two girls were a pair that Lester had met at a fatkins
club in South Beach the week before, and he’d camera-phoned their
pic to Perry with a scrawled drunken note about which one was his.
They were attractive enough, but the monotonic fatkins devotion to
sybartism was so tiresome. Perry didn’t see much point in hooking
up with a girl he couldn’t have a good technical discussion with.

“Come \emph{on}, it’s good stuff, you’ll love it.”

“If I have to change clothes, I’m not interested.” Perry folded his
arms. In truth, he wasn’t interested, period. He liked his little
kingdom there, and he could get everything he needed from burritos
to RAM at the market. He had a chest freezer full of bankruptcy
sale organic MREs, for variety.

“Just the shirt then\dash{}I had it printed just for you.”

Perry raised his funny eyebrow. “Let’s see it.”

Lester turned to his latest car, a trike with huge, electric blue
back tires, and popped the trunk, rummaged, and proudly emerged
holding a bright blue Hawai’ian print shirt.

“Lester, are those . . . turds?”

“It’s transgressivist moderne,” Lester said, hopping from foot to
foot. “Saw it in the New York Times, brought the pic to Gabriela in
the market, she cloned it, printed it, and sent it out for
stitching\dash{}an extra ten buck for same-day service.”

“I am \emph{not} wearing a shirt covered in steaming piles of shit,
Lester. No, no, no. A googol times no.”

Lester laughed. “Christ, I had you going, didn’t I? Don’t worry, I
wouldn’t actually have let you go out in public wearing this. But
how about \emph{this}?” he said with a flourish, and brought out
another shirt. Something stretchy and iridescent, like an
oil-slick. It was sleeveless. “It’ll really work with your biceps
and pecs. Also: looks pretty good compared to the turd shirt,
doesn’t it? Go on, try it on.”

“Lester Banks, you are the gayest straight man I know,” Perry said.
He shucked his sweaty tee and slipped into the shirt. Lester gave
him a big thumbs-up. He examined his reflection in the blacked-out
glass doors of the Wal-Mart.

“Yeah, OK,” he said. “Let’s get this over with.”

“Your enthusiasm, your best feature,” Lester said.

Their dates were two brunettes with deep tans and whole-eye
cosmetic contacts that hid their pupils in favor of featureless
expanses of white, so they looked like their eyes had rolled back
into their heads, or maybe like they were wearing cue-balls for
glass eyes. Like most of the fatkins girls Perry had met, they
dressed to the nines, ate like pigs, drank like fishes, and talked
about nothing but biotech.

“So I’m thinking, sure, mitochrondrial lengthening \emph{sounds}
like it should work, but if that’s so, why have we been screwing
around with it for thirty years without accomplishing anything?”
His date, Moira, worked at a law office, and she came up to his
chest, and it was hard to tell with those eyes, but it seemed like
she was totally oblivious to his complete indifference to
mitochondria.

He nodded and tried not to look bored. South Beach wasn’t what it
had once been, or maybe Perry had changed. He used to love to come
here to people-watch, but the weirdos of South Beach seemed too
precious when compared with the denizens of his own little
settlement out on the Hollywood freeway.

“Let’s go for a walk on the beach,” Lester said, digging out his
wallet and rubbing his card over the pay-patch on the table.

“Good idea,” Perry said. Anything to get off this patio and away
from the insufferable club music thundering out of the speakers
pole-mounted directly over their table.

The beach was gorgeous, so there was that. The sunset behind them
stained the ocean bloody and the sand was fine and clean. Around
their feet, Dade County beachcombers wormed endlessly through the
sand, filtering out all the gunk, cig butts, condoms, needles,
wrappers, loose change, wedding rings, and forgotten sunglasses.
Perry nudged one with his toe and it roombaed away, following its
instinct to avoid human contact.

“How do you figure they keep the vags from busting those open for
whatever they’ve got in their bellies?” Perry said, looking over
his date’s head at Lester, who was holding hands with his girl,
carrying her shoes in his free hand.

“Huh? Oh, those things are built like tanks. Have to be to keep the
sand out. You need about four hours with an air-hammer to bust one
open.”

“You tried it?”

Lester laughed. “Who, me?”

Now it was Perry’s date’s turn to be bored. She wandered away
toward the boardwalk, with its strip of novelty sellers. Perry
followed, because he had a professional interest in the kind of
wares they carried. Most of them originated on one of his printers,
after all. Plus, it was the gentlemanly thing to do.

“What have we here?” he said as he pulled up alongside her. She was
trying on a bracelet of odd, bony beads.

“Ectopic fetuses,” she said. “You know, like the Christian fundies
use for stem-cell research? You quicken an unfertilized egg in
vitro and you get a little ball of fur and bone and skin and
stem-cells. It can never be a human, so it has no soul, so it’s not
murder to harvest them.”

The vendor, a Turkish teenager with a luxurious mustache, nodded.
“Every bead made from naturally occurring foetus-bones.” He handed
one to Perry.

It was dry and fragile in his hand. The bones were warm and porous,
and in tortured Elephant Man shapes that he recoiled from
atavistically.

“Good price,” the Turkish kid said. He had practically no accent at
all, and was wearing a Japanese baseball-team uniform and spray-on
foot-coverings. Thoroughly Americanized. “Look here,” he said, and
gestured at a little corner of his table.

It was covered in roses made from fabric\dash{}small and crude, with
pin-backs. Perry picked one up. It had a certain naive charm. The
fabric was some kind of very delicate leather\dash{}

“It’s skin,” his date said. “Foetal skin.”

He dropped it. His fingers tingled with the echo of the feeling of
the leather. \emph{Jesus I hate biotech}. The rose fluttered past
the table to the sandy boardwalk, and the Turkish kid picked it up
and blew it clean.

“Sorry,” Perry said, sticking his hands in his pockets. His date
bought a bracelet and a matching choker made of tiny bones and
teeth, and the Turkish kid, leering, helped her fasten the
necklace. When they returned to Lester and his date, Perry knew the
evening was at a close. The girls played a couple rounds of
eye-hockey, unreadable behind their lenses, and Perry shrugged
apologetically at Lester.

“Well then,” Lester said, “it sure has been a nice night.” Lester
got smooched when they saw the girls off in a pedicab. In the buzz
and hum of its flywheel, Perry got a damp and unenthusiastic
handshake.

“Win some, lose some,” Lester said as the girls rolled away in a
flash of muscular calves from the pair of beach-perfect cabbies
pedaling the thing.

“You’re not angry?” Perry said.

“Nah,” Lester said. “I get laid too much as it is. Saps me of my
precious bodily fluids. Gotta keep some chi inside, you know?”

Perry raised up his funny eyebrow and made it dance.

“Oh, OK,” Lester said. “You got me. I’m meeting mine later, after
she drops her friend off.”

“I’ll get a cab home then, shall I?”

“Take my car,” Lester said. “I’ll get a ride back in the morning.
No way you’ll get a taxi to take you to our neighborhood at this
hour.”

Perry’s car had been up on blocks for a month, awaiting his
attention to its failing brakes and mushy steering. So it was nice
to get behind the wheel of Lester’s Big Daddy Roth trike and give
it a little gas out on the interstate, the smell of the swamp and
biodiesel from the big rigs streaming past the windscreen. The road
was dark and treacherous with potholes, but Perry got into the
rhythm of it and found he didn’t want to go home, quite, so he kept
driving, into the night. He told himself that he was scouting dead
malls for future expansion, but he had kids who’d video-documented
the status of all the likely candidates in the hood, and he kept
tabs on his choicest morsels via daily sat photos that he
subscribed to in his morning feed.

What the hell was he doing with his life? The Wal-Mart ride was a
lark\dash{}it had been Lester’s idea, but Lester had lost interest and
Perry had done most of the work. They weren’t quite squatting the
Wal-Mart: Perry paid rent to a state commission that collected in
escrow for the absentee landlord. It was a fine life, but the days
blurred one into the next, directionless. Building the ride had
been fun, setting up the market had been fun, but running
them\dash{}well, he might as well be running a laundromat for all the
mental acuity his current job required.

“You miss it,” he said to himself over the whistle of the wind and
the hiss of the fat contact-patches on the rear tires. “You want to
be back in the shit, inventing stuff, making it all happen.”

For the hundredth time, he thought about calling Suzanne Church. He
missed her, too, and not just because she made him famous (and now
he was no longer famous). She put it all in perspective for him,
and egged him on to greater things. She’d been their audience, and
they’d all performed for her, back in the golden days.

It was, what, 5AM in Russia? Or was it two in the afternoon? He had
her number on his speed-dial, but he never rang it. He didn’t know
what he’d tell her.

He could call Tjan, or even Kettlebelly, just ring them out of the
blue, veterans together shooting the shit. Maybe they could have a
Kodacell reunion, and get together to sing the company song,
wearing the company t-shirt.

He pulled the car off at a truck stop and bought an ice-cream
novelty from a vending machine with a robotic claw that scooped the
ice-cream, mushed it into the cone, then gave it a haircut so that
it looked like Astro Boy’s head, then extended the cone on a
robotic claw. It made him smile. Someone had invented this thing.
It could have been him. He knew where you could download
vision-system libraries, and force-feedback libraries. He knew
where you could get plans for the robotics, and off-the-shelf
motors and sensors. Christ, these days he had a good idea where you
could get the ice-cream wholesale, and which crooked
vending-machine interests he’d have to grease to get his stuff into
truck-stops.

He was thirty four years old, he was single and childless, and he
was eating an ice-cream in a deserted truck-stop at two in the
morning by the side of a freeway in south Florida. He bossed a
low-budget tourist attraction and he ran a pirate flea-market.

What the hell was he doing with his life?

Getting mugged, that’s what.

They came out of the woods near the picnic tables, four
bangbangers, but young ones, in their early teens. Two had
guns\dash{}nothing fancy, just AK-47s run off a computer-controlled mill
somewhere in an industrial park. You saw them all over the place,
easy as pie to make, but the ammo was a lot harder to come by. So
maybe they were unloaded.

Speaking of unloaded. He was about to piss his pants.

“Wallet,” one of them said. He had a bad mustache that reminded him
of the Turkish kid on the beach. Probably the same hormones that
gave kids mustaches gave them bad ideas like selling fetus jewelry
or sticking up people by the ice-cream machines at late night
truck-stops. “Keys,” he said. “Phone,” he added.

Perry slowly set down the ice-cream cone on the lid of the
trash-can beside him. He’d only eaten one spike off Astro-Boy’s
head.

His vision telescoped down so that he was looking at that kid, at
his mustache, at the gun in his hands. He was reaching for his
wallet, slowly. He’d need to hitch a ride back to town. Canceling
the credit-cards would be tough, since he’d stored all the
identity-theft passwords and numbers in his phone, which they were
about to take off him. And he’d have to cancel the phone, for that
matter.

“Do you have an older brother named Jason?” his mouth said, while
his hands were still being mugged.

“What?”

“Works a stall by the Wal-Mart ride, selling contact lenses?”

The kid’s eyes narrowed. “You don’t know me, man. You don’t want to
know me. Better for your health if you don’t know me.”

His hands were passing over his phone, his wallet, his
keys\dash{}Lester’s keys. Lester would be glad to have an excuse to build
a new car.

“Only I own the Wal-Mart ride, and I’ve known Jason a long time. I
gave him his first job, fixing the printers. You look like him.”

The kid’s three buddies were beginning their slow fade into the
background. The kid was visibly on the horns of a dilemma. The gun
wavered. Perry’s knees turned to water.

“You’re that guy?” the kid said. He peered closer. “Shit, you
are.”

“Keep it all,” Perry said. His mouth wasn’t so smart. Knowing who
mugged you wasn’t good for your health.

“Shit,” the kid said. The gun wavered. Wavered.

“Come \emph{on},” one of his buddies said. “Come on, man!”

“I’ll be there in a minute,” the kid said, his voice flat.

Perry knew he was a dead man.

“I’m really sorry,” the kid said, once his friends were out of
range.

“Me too,” said Perry.

“You won’t tell my brother?”

Perry froze. Time dilated. He realized that his fists were clenched
so tight that his knuckles hurt. He realized that he had a zit on
the back of his neck that was rubbing against his collar. He
realized that the kid had a paperback book stuck in the waistband
of his bangbanger shorts, which was unusual. It was a fantasy
novel. A Conan novel. Wow.

Time snapped back.

“I won’t tell your brother,” he said. Then he surprised himself,
“But you’ve got to give me back the credit-cards and leave the car
at the market in the morning.”

The kid nodded. Then he seemed to realize he was holding a gun on
Perry. He lowered it. “Yeah, that’s fair,” he said. “Can’t use the
fucking cards these days anyway.”

“Yeah,” Perry said. “Well, there’s some cash there anyway.” He
realized he had five hundred bucks in a roll in a hidden pocket in
his shorts.

“You get home OK?”

“I’ll thumb a ride,” Perry said.

“I can call you a taxi,” the kid said. “It’s not safe to hang
around here.”

“That’s really nice of you,” Perry said. “Thanks.”

The kid took out a little phone and prodded it for a minute. “On
the way,” he said. “The guns aren’t loaded.”

“Oh, well,” Perry said. “Good to know.”

An awkward silence spread between them. “Look, I’m really sorry,”
the kid said. “We don’t really do this. It’s our first night. My
brother would really kill me.”

“I won’t tell him,” Perry said. His heart was beating again, not
thundering or keeping ominously still. “But you know, this isn’t
smart. You’re going to stick someone up who has bullets and he’s
gonna shoot you.”

“We’ll get ammo,” the kid said.

“And shoot him? That’s only a little better, you know.”

“What do you want me to say?” the kid said, looking young and
petulant. “I apologized.”

“Come by tomorrow with the car and let’s talk, all right?”

Lester didn’t even notice that his car was missing until the kid
drove up with it, and when he asked about it, Perry just raised his
funny eyebrow at him. That funny eyebrow, it had the power to cloud
men’s minds.

“What’s your name?” Perry asked the kid, giving him the spare stool
by the ticket-window. It was after lunch time, when the punishing
heat slowed everyone to a sticky crawl, and the crowd was thin\dash{}one
or two customers every half hour.

“Glenn,” the kid said. In full daylight, he looked older. Perry had
noticed that the shantytowners never stopped dressing like
teenagers, wearing the fashions of their youths forever, so that a
walk through the market was like a tour through the teen fashions
of the last thirty years.

“Glenn, you did me a real solid last night.”

Glenn squirmed on his stool. “I’m sorry about that\dash{}”

“Me too,” Perry said. “But not as sorry as I might have been. You
said it was your first night. Is that true?”

“Car-jacking, sure,” the kid said.

“But you get into other shit, don’t you? Mugging? Selling a little
dope? Something like that?”

“Everyone does that,” Glenn said. He looked sullen.

“Maybe,” Perry said. “And then a lot of them end up doing a stretch
in a work-camp. Sometimes they get bit by water-moccasins and don’t
come out. Sometimes, one of the other prisoners hits them over the
head with a shovel. Sometimes you just lose three to five years of
your life to digging ditches.”

Glenn said nothing.

“I’m not trying to tell you how to run your life,” Perry said. “But
you seem like a decent kid, so I figure there’s more in store for
you than getting killed or locked up. I know that’s pretty normal
around here, but you don’t have to go that way. Your brother
didn’t.”

“What the fuck do you know about it, anyway?” The kid was up now,
body language saying he wanted to get far away, fast.

“I could ask around the market,” Perry said, as though the kid
hadn’t spoken. “Someone here has got to be looking for someone to
help out. You could open your own stall.”

The kid said, “It’s all just selling junk to idiots. What kind of
job is that for a man?”

“Selling people stuff they can’t be bothered to make for themselves
is a time-honored way of making a living. There used to be
professional portrait photographers who’d take a pic of your family
for money. They were even considered artists. Besides, you don’t
have to sell stuff you download. You can invent stuff and print
that.”

“Get over it. Those days are over. No one cares about inventions
anymore.”

It nailed Perry between the eyes, like a slaughterhouse bolt.
“Yeah, yeah,” he said. He didn’t want to talk to this kid any more
than this kid wanted to talk to him. “Well, if I can’t talk you out
of it, it’s your own business. . .” He started to rearrange his
ticket-desk.

The kid saw his opportunity for freedom and bolted. He was probably
headed for his brother’s stall and then the long walk to wherever
he planned on spending his day. Everything was a long walk from
here, or you could wait for the busses that ran on the hour during
business-hours.

Perry checked out the car, cleaned out the empties and the roaches
and twists from the back seat, then parked it. A couple more people
came by to ride his ride, and he took their money.

Lester had just finished his largest-ever flattened-soda-can
mechanical computer, it snaked back and forth across the whole of
the old Wal-Mart solarium, sheets of pressboard with precision-cut
gears mounted on aviation bearings\dash{}Francis had helped him with
those. All day, he’d been listening to the racket of it grinding
through its mighty 0.001KHz calculations, dumping carloads of M\&Ms
into its output hopper. You programmed it with regulation
baseballs, footballs, soccer-balls, and wiffleballs: dump them in
the input hopper and they would be sorted into the correct chutes
to trigger the operations. With a whopping one kilobit of memory,
the thing could best any of the early vacuum tube computers without
a single electrical component, and Lester was ready to finally
declare victory over the cursed Univac.

Perry let himself be coaxed into the work-room, deputizing Francis
to man the ticket-desk, and watched admiringly as Lester put the
machine through its paces.

“You’ve done it,” Perry said.

“Well, I gotta blog it,” Lester said. “Run some benchmarks, really
test it out against the old monsters. I’m thinking of using it to
brute-force the old Nazi Enigma code. That’ll show those dirty Nazi
bastards! We’ll win the war yet!”

Perry found himself giggling. “You’re the best, man,” he said to
Lester. “It’s good that there’s at least one sane person around
here.”

“Don’t flatter yourself, Perry.”

“I was talking about \emph{you}, Lester.”

“Uh-oh,” Lester said. He scooped a double handful of brown M\&Ms up
from the output hopper and munched them. “It’s not a good sign when
you start accusing me of being the grownup in our partnership. Have
some M\&Ms and tell me about it.”

Perry did, unburdening himself to his old pal, his roommate of ten
years, the guy he’d gone to war with and started businesses with
and collaborated with.

“You’re restless, Perry,” Lester said. He put nine golf-balls, a
ping-pong ball, and another nine golf balls in the machine’s input
hopper. Two and a third seconds later, eighty one M\&Ms dropped
into the output hopper. “You’re just \emph{bored}. You’re a maker,
and you’re running things instead of making things.”

“No one cares about made things anymore, Les.”

“That’s sort of true,” Lester said. “I’ll allow you that. But it’s
only sort of true. What you’re missing is how much people care
about organizations still. That was the really important thing
about the New Work: the way we could all come together to execute,
without a lot of top-down management. The bangbanger arms dealers,
the bio-terrorists and fatkins suppliers\dash{}they all run on social
institutions that we perfected back then. You’ve got something like
that here with your market, a fluid social institution that you
couldn’t have had ten or fifteen years ago.”

“If you say so,” Perry said. The M\&Ms were giving him heartburn.
Cheap chocolate didn’t really agree with his stomach.

“I do. And so the answer is staring you right in the face: go
invent some social institutions. You’ve got one creeping up here in
the ride. There are little blogospheres of fans who coordinate what
they’re going to bring down and where they’re going to put it.
Build on that.”

“No one’s going to haul ass across the country to ride this ride,
Les. Get real.”

“Course not.” Lester beamed at him. “I’ve got one word for you,
man: franchise!”

“Franchise?”

“Build dupes of this thing. Print out anything that’s a one of a
kind, run them as franchises.”

“Won’t work,” Perry said. “Like you said, this thing works because
of the hardcore of volunteer curators who add their own stuff to
it\dash{}it’s always different. Those franchises would all be static, or
would diverge\ldots{} It’d just be boring compared to this.”

“Why should they diverge? Why should they be static? You could
network them, dude! What happens in one, happens in all. The
curators wouldn’t just be updating one exhibit, but all of them.
Thousands of them. Millions of them. A gigantic physical wiki. Oh,
it’d be so very very very cool, Perry. A cool
\emph{social institution.}”

“Why don’t you do it?”

“I’m gonna. But I need someone to run the project. Someone who’s
good at getting people all pointed in the same direction. You, pal.
You’re my hero on this stuff.”

“You’re such a flatterer.”

“You love it, baby,” Lester said, and fluttered his long eyelashes.
“Like the lady said to the stamp collector, philately will get you
everywhere.”

“Oy,” Perry said. “You’re fired.”

“You can’t fire me, I’m a volunteer!”

Lester dropped six golf-balls and a heavy medicine ball down the
hopper. The machine ground and chattered, then started dropping
hundred-loads of M\&Ms\dash{}100, 200, 300, 400, 500, 600, 700\dash{}then some
change.

“What operation was that?” Perry said. He’d never seen Lester pull
out the medicine ball.

“Figure it out,” Lester said.

Perry thought for a moment. Six squared? Six cubed? Log six? “Six
\emph{factorial}? My God you’re weird, Les.”

“Genius is never appreciated.” He scooped up a double-handful of
brown M\&Ms. “In your face, Von Neumann! Let’s see your precious
ENIAC top \emph{this}!”

\begin{center}\rule{3in}{0.4pt}\end{center}

A month later, Perry was clearing security at Miami International,
looking awkward in long trousers, closed-sole shoes, and a denim
jacket. It was autumn in Boston, and he couldn’t show up in
flip-flops and a pair of cutoffs. The security guards gave his
leathery, lopsided face a hard look. He grinned like a pirate and
made his funny eyebrow twitch, a stunt that earned him half an hour
behind the screen and a date with Doctor Jellyfinger.

“What, exactly, do you think I’ve got hidden up there?” he asked as
he gripped the railing and tried not to let the illegitimati
carborundum.

“It’s procedure, sir.”

“Well, the doc said my prostate was the size of a guava about a
month ago\dash{}in your professional opinion, has it shrunk or grown? I
mean, while you’re up there.”

The TSA man didn’t like that at all. A minute later, Perry was
buckling up and leaving the little room with an exaggerated
bowlegged gait. He tipped an imaginary hat at the guard’s
retreating back and said, “Call me!” in a stagey voice.

It was the last bit of fun he had for the next four hours, crammed
in the tin can full of recycled discount air-traveller flatulence
and the clatter of fingers on keyboards and the gabble of a hundred
phone conversations as the salarymen on the flight stole a few
minutes of cramped productivity from the dead travel time.

Touching down in Boston and getting his luggage, he felt like he’d
landed on an alien planet. The feeling of disorientation and
foreignness was new to Perry. He was used to being supremely
comfortable, in control\dash{}confident. But he was nervous now, maybe
even scared, a little.

He dialed Tjan. “I’ve got my bags,” he said.

“I’ll be right around,” Tjan said. “Really looking forward to
seeing you.”

There were more cops than passengers in the arrivals area at Logan,
and they watched Tjan warily as he pulled up and swung open a door
of his little sports-car.

“What the fuck is this, a Porsche?” Perry said as he folded himself
awkwardly into the front seat, stepping in through the sun-roof,
pulling his bag down into his lap after him.

“It’s a Lada. I had it imported\dash{}they’re all over Russia.
Evolutionary algorithm used to produce a
minimum-materials/maximum-strength chassis. It’s nice to see you,
Perry.”

“It’s nice to see you, Tjan,” he said. The car was so low to the
ground that it felt like he was riding luge. Tjan hammered
mercilessly on the gearbox, rocketing them to Cambridge at such
speed that Perry barely had time to admire the foliage, except at
stop-lights.

They were around the campus now, taking a screeching right off Mass
Ave onto a tree-lined street of homely two-storey brick houses.
Tjan pulled up in front of one and popped the sun-roof. The cold
air that rushed in was as crisp as an apple, unlike any breath of
air to be had in Florida, where there was always a mushiness, a
feeling of air that had been filtered through the moist lungs of
Florida’s teeming fauna.

Perry climbed out of the little Russian sports-car and twisted his
back and raised his arms over his head until his spine gave and
popped and crackled.

Tjan followed, and then he shut down the car with a remote that
made it go through an impressive and stylish series of clicks,
clunks and chirps before settling down over its wheels, dropping
the chassis to a muffler-scraping centimeter off the ground.

“Come on,” he said. “I’ll show you your room.”

Tjan’s porch sagged, with a couple kids’ bikes triple-locked to it
and an all-covering chalk mosaic over every inch of it. The wood
creaked and gave beneath their feet.

The door sprang open and revealed a pretty little girl, nine or ten
years old, in blue-jeans and a hoodie sweater that went nearly to
her ankles, the long sleeves bunched up like beach-balls on her
forearms. The hood hung down to her butt\dash{}it was East Coast
bangbanger, as reinterpreted through the malls.

“Daddy!” she said, and put her arms around Tjan’s waist, squeezing
hard.

He pried her loose and then hoisted her by the armpits up to
eye-height. “What have you done to your brother?”

“Nothing he didn’t deserve,” she said, with a smile that showed
dimples and made her little nose wrinkle.

Tjan looked over at Perry. “This is my daughter, Lyenitchka, who is
about to be locked in the coal cellar until she learns to stop
torturing her younger brother. Lyenitchka, this is Perry Gibbons,
upon whom you have already made an irreparably bad first
impression.” He shook her gently Perrywards.

“Hello, Perry,” she said, giggling, holding out one hand. She had a
faint accent, which made her sound like a tiny, skinny Bond
villainess.

He shook gravely. “Nice to meet you,” he said.

“You got your kids,” Perry said, once she was gone.

“For the school year. Me and the ex, we had a heart-to-heart about
the Russian education system and ended up here: I get the kids from
September to June, but not Christmases or Easter holidays. She gets
them the rest of the time, and takes them to a family dacha in
Ukraine, where she assures me there are hardly any mafiyeh kids to
influence my darling daughter.”

“You must be loving this,” Perry said.

Tjan’s face went serious. “This is the best thing that’s ever
happened to me.”

“I’m really happy for you, buddy.”

They had burgers in the back-yard, cooking on an electric grill
that was caked with the smoking grease of a summer’s worth of
outdoor meals. The plastic table-cloth was weighed down with
painted rocks and the corners blew up in the freshening autumn
winds. Lyenitchka’s little brother appeared when the burgers began
to spit and smoke on the grill, a seven-year-old in metallic mesh
trousers and shirt wrought with the logo of a cartoon Cossack
holding a laser-sword aloft.

“Sasha, meet Perry.” Sasha looked away, then went off to swing on a
tire-swing hanging from the big tree.

“You’ve got good kids,” Perry said, handing Tjan a beer from the
cooler under the picnic table.

“Yup,” Tjan said. He flipped the burgers and then looked at both of
them. Lyenitchka was pushing her brother on the swing, a little too
hard. Tjan smiled and looked back down at his burgers.

Tjan cut the burgers in half and dressed them to his kids’ exacting
standards. They picked at them, pushed them onto each other’s
plates and got some into their mouths.

“I’ve read your briefing on the ride,” Tjan said, once his kids had
finished and eaten half a package of Chutney Oreos for dessert.
“It’s pretty weird stuff.”

Perry nodded and cracked another beer. The cool air was weirding
him out, awakening some atavistic instinct to seek a cave. “Yup,
weird as hell. But they love it. Not just the geeks, either, though
they eat it up, you should see it. Obsessive doesn’t begin to cover
it. But the civilians come by the hundreds, too. You should hear
them when they come out: ’Jee-zus, I’d forgotten about those
dishwasher-stackers, they were wicked! Where can I get one of those
these days you figger?’ The nostalgia’s thick enough to cut with a
knife.”

Tjan nodded. “I’ve been going over your books, but I can’t figure
out if you’re profitable.”

“Sorry, that’s me. I’m pretty good at keeping track of numbers, but
getting them massaged into a coherent picture\dash{}”

“Yeah, I know.” Tjan got a far-away look. “How’d you make out on
Kodacell, Perry? Finance-wise?”

“Enough to open the ride, buy a car. Didn’t lose anything.”

“Ah.” Tjan fiddled with his beer. “Listen, I got rich off of
Westinghouse. Not fuck-the-service-here-I’m-buying-this-restaurant
rich, but rich enough that I never have to work again. I can spend
the rest of my life in this yard, flipping burgers, taking care of
my kids, and looking at porn.”

“Well, you were the suit. Getting rich is what suits do. I’m just a
grunt.”

Tjan had the good grace to look slightly embarrassed. “Now here’s
the thing. I don’t \emph{have} to work, but, Perry, I have
\emph{no idea} what I’m going to do if I don’t work. The kids are
at school all day. Do you have any idea how much daytime TV sucks?
Playing the stock market is completely nuts, it’s all gone sideways
and upside down. I got an education so I wouldn’t \emph{have} to
flip burgers for the rest of my life.”

“What are you saying, Tjan?”

“I’m saying yes,” Tjan said, grinning piratically. “I’m saying that
I’ll join your little weird-ass hobby business and I’ll open
another ride here for the Massholes. I’ll help you run the
franchising op, collect fees, make it profitable.”

Perry felt his face tighten.

“What? I thought you’d be happy about this.”

“I am,” Perry said. “But you’re misunderstanding something. These
aren’t meant to be profitable businesses. I’m done with that. These
are art, or community, or something. They’re museums. Lester calls
them \emph{wunderkammers}\dash{}cabinets of wonders. There’s no
franchising op the way you’re talking about it. It’s ad hoc. It’s a
protocol we all agree on, not a business arrangement.”

Tjan grunted. “I don’t think I understand the difference between a
agreed-upon protocol and a business arrangement.” He held up his
hand to fend off Perry’s next remark. “But it doesn’t matter. You
can let people have the franchise for free. You can claim that
you’re not letting anyone have anything, that they’re letting
themselves in for their franchise. It doesn’t matter to me.

“But Perry, here’s something you’re going to have to understand:
it’s going to be nearly impossible \emph{not to} make a business
out of this. Businesses are great structures for managing big
projects. It’s like trying to develop the ability to walk without
developing a skeleton. Once in a blue moon, you get an octopus, but
for the most part, you get skeletons. Skeletons are good shit.”

“Tjan, I want you to come on board to help me create an octopus,”
Perry said.

“I can try,” Tjan said, “but it won’t be easy. When you do cool
stuff, you end up making money.”

“Fine,” Perry said. “Make money. But keep it to a minimum, OK?”

\begin{center}\rule{3in}{0.4pt}\end{center}

The next time Perry turned up at Logan, it was colder than the
inside of an icebox and shitting down grey snow with the
consistency of frozen custard.

“Great weather for an opening,” he said, once he’d climbed through
the roof of Tjan’s car and gotten snow all over the leather
upholstery. “Sorry about the car.”

“Don’t sweat it, the kids are murder on leather. I should trade
this thing in on something that’s less of a deathtrap anyway.”

Tjan was balder than he’d been in September, and skinnier. He had a
three-day beard that further hollowed out his normally round
cheeks. The Lada sports-car fishtailed a little as they navigated
the tunnels back toward Cambridge, the roads slick and icy.

“We scored an excellent location,” Tjan said. “I told you that, but
check this out.” They were right in the middle of a built-up area
of Boston, something that felt like a banking district, with
impressive towers. It took Perry a minute to figure out what Tjan
was pointing at.

“That’s the site?” There was a mall on the corner, with a boarded
up derelict Hyatt overtopping it, rising high into the sky. “But
it’s right in the middle of town!”

“Boston’s not Florida,” Tjan said. “Lots of people here don’t have
cars. There were some dead malls out in Worcester and the like, but
I got this place for nothing. The owners haven’t paid taxes in the
ten years since the hotel folded, and the only shops that were left
open were a couple of Azerbaijani import-export guys, selling junky
stuff from India.

“We gutted the whole second floor and turned the ground-floor
food-court into a flea-market. There’s an old tunnel connecting
this to the T and I managed to get it re-opened, so I expect we’ll
get some walk-in.”

Perry marveled. Tjan had a suit’s knack for pulling off the
ambitious. Perry had never tried to even rent an apartment in a big
city, figuring that any place where land was at a premium was a
place where people willing to spend more than him could be found.
Give him a ghost-mall that was off the GPS grid anytime.

“Have you managed to fill the flea market?” It had taken Perry a
long time to fill his, and still he had a couple of dogs\dash{}a tarot
reader and a bong stall, a guy selling high-pressure spray-paint
cans and a discount porn stall that sold naked shovelware by the
petabyte.

“Yeah, I got proteges up and down New England. A lot of them
settled here after the crash. One place is as good as another, and
the housing was wicked-cheap once the economy disappeared. They
upped stakes and came to Boston as soon as I put the word out. I
think everyone’s waiting for the next big thing.”

“You think?”

“Perry, New Work is the most important thing that ever happened to
some of those people. It was the high-point of their lives. It was
the only time they ever felt useful.”

Perry shook his head. “Don’t you think that’s sad?”

Tjan negotiated a tricky tunnel interchange and got the car pointed
to Cambridge. “No, Perry, I don’t think it’s sad. Jesus Christ, you
can’t believe that. Why do you think I’m helping you? You and me
and all the rest of them, we did something \emph{important}. The
world changed. It’s continuing to change. Have you stopped to think
that one in five American workers picked up and moved somewhere
else to do New Work projects? That’s one of the largest American
resettlements since the dustbowl. The average New Work collective
shipped more inventions per year than Edison Labs at its peak. In a
hundred years, when they remember the centuries that were
America’s, they’ll count this one among them, because of what we
made.

“So no, Perry, I don’t think it’s sad.”

“I’m sorry. Sorry, OK? I didn’t mean it that way. But it’s tragic,
isn’t it, that the dream ended? That they’re all living out there
in the boonies, thinking of their glory days?”

“Yes, that \emph{is} sad. But that’s why I agreed to do the
ride\dash{}not to freeze the old projects in amber, but to create a new
project that we can all participate in again. These people uprooted
their lives to follow us, it’s the least we can do to give them
something back for that.”

Perry stewed on that the rest of the way to Tjan’s, staring at the
sleet, hand resting against the icy window-glass.

\begin{center}\rule{3in}{0.4pt}\end{center}

Sammy checked in to a Comfort Inn tucked into the thirty-seventh
storey of the Bank of America building in downtown Boston. The
lobby was empty, the security-guard’s desk unmanned. B of A was in
receivership, and not doing so hot at that, as the fact that they
had let out their executive floors to a discount business-hotel
testified.

The room was fine, though\dash{}small and windowless, but fine: power,
shower, toilet and bed, all he demanded in a hotel room. He ate the
packet of nuts he’d bought at the airport before jumping on the T
and then checked his email. He had more of it than he could
possibly answer\dash{}he didn’t think he’d ever had an empty in-box.

But he picked off anything that looked important, including a note
from his ex-, who was now living in the Keys on a squatter beach
and wanted to know if he could loan her a hundred bucks. No sense
of how she’d pay him back without work. But Michelle was
resourceful and probably good for it. He paypalled it to her,
feeling like a sucker for hoping that she might repay it in person.
He’d been single since she’d left him the year before and he was
lonely and hard-up.

He’d landed at two and by the time he was done with all the
bullshit, it was after dinner time and he was hungry as hell.
Boston was full of taco-wagons and kebab stands that he’d passed on
the walk in, and he hustled out onto the street to see if any were
still open. He got a huge garlicky kebab and ate it in the lee of a
frozen ATM shelter, wolfing it without tasting it.

He went and scouted the location of the new ride. He’d gotten wind
of it online\dash{}none of his idiot colleagues could be bothered to read
the public email lists of the competitors they were supposedly in
charge of oppo researching. Shaking loose the budget to get a
discount flight to Boston had been a major coup, requiring
horse-trading, blackmail, and passive-aggressive gaming of the
system. With the ridiculously low per-diem and hotel allowance he’d
still go home a couple hundred bucks out-of-pocket. Why did he even
do his job? He should just play by the rules and get nothing done.

And get fired. Or passed up for promotion, which was practically
the same thing.

The new ride was in an impressive urban mall. He’d spent his
college years in Philly and had passed many a happy day in malls
like this one, cruising for girls or camping out on a bench with
his books and a smoothie. Unlike the crappy roadside malls of
Florida, there had been nothing but the best stores in them, the
property values too high to make anything but high-margin,
high-turnover, high-ticket shops viable.

So it was especially sad to see this mall turned over to the junky
stalls and junkier ride\dash{}like a fat, washed-up supermodel sentenced
to a talk-show appearance for her shoplifting arrests. He
approached the doors with trepidation. He was resolved not to buy
anything from the market\dash{}no busts or contact lenses\dash{}and had stuck
his wallet in his front pocket on the way over.

The mall was like a sauna. He shucked his jacket and sweater and
hung them over one arm. The whole ground floor had been given over
to flimsy market-stalls. He skulked among them, trying to
simultaneously take note of their contents and avoid their owners’
notice.

He came to realize that he needn’t skulk. It seemed like half of
Boston had turned out\dash{}not just young people, either. There were
plenty of tweedy academics, big working-class Southie boys with
thick accents, recent immigrants with Scandie-chic clothes. They
chattered and laughed and mixed freely and ate hot food out of huge
cauldrons or off of clever electric grills. The smells made his
stomach growl, even though he’d just polished off a kebab the size
of his head.

The buzz of the crowd reminded him of something, what was it? A
premiere, that was it. When they opened a new ride or area at the
Park, there was the same sense of thrilling anticipation, of
excitement and eagerness. That made it worse\dash{}these people had no
business being this excited about something so. . . lowbrow? Cheap?
Whatever it was, it wasn’t worthy.

They were shopping like fiends. A mother with a baby on her hip
pushed past him, her stroller piled high with shopping bags
screened with giant, pixellated Belgian pastries. She was laughing
and the baby on her hip was laughing too.

He headed for the escalator, whose treads had been anodized in
bright colors, something he’d never seen before. He let it carry
him upstairs, but looked down, and so he was nearly at the top
before he realized that the guy from the Florida ride was standing
there, handing out fliers and staring at Sammy like he knew him
from somewhere.

It was too late to avoid him. Sammy put on his best castmember
smile. “Hello there!”

The guy grinned and wiggled his eyebrow. “I know you from
somewhere,” he said slowly.

“From Florida,” Sammy said, with an apologetic shrug. “I came up to
see the opening.”

“No \emph{way}!” The guy had a huge smile now, looked like was
going to hug him. “You’re shitting me!”

“What can I say? I’m a fan.”

“That’s \emph{incredible.} Hey, Tjan, come here and meet this guy.
What’s your name?”

Sammy tried to think of another name, but drew a blank. “Mickey,”
he said at last, kicking himself.

“Tjan, this is Mickey. He’s a regular on the ride in Florida and
he’s come up here just to see the opening.”

Tjan had short hair and sallow skin, and dressed like an
accountant, but his eyes were bright and sharp as they took Sammy
in, looking him up and down quickly. “Well that’s certainly
flattering.” He reached into his creased blazer and pulled out a
slip of paper. “Have a couple comp tickets then\dash{}the least we can do
for your loyalty.” The paper was festooned with holograms and
smart-cards and raised bumps containing RFIDs, but Sammy knew that
you could buy standard anti-counterfeiting stock like it from a
mail-order catalog.

“That’s mighty generous of you,” he said, shaking Tjan’s dry, firm
hand.

“Our pleasure,” the other guy said. “Better get in line, though, or
you’re gonna be waiting a long, long time.” He had a satisfied
expression. Sammy saw that what he’d mistaken for a crowd of people
was in fact a long, jostling queue stretching all the way around
the escalator mezzanine and off one of the mall’s side corridors.

Feeling like he’d averted a disaster, Sammy followed the length of
the queue until he came to its end. He popped in a headphone and
set up his headline reader to text-to-speech his day’s news. He’d
fallen behind, what with the air travel and all. Most of the stuff
in his cache came in from his co-workers, and it was the most
insipid crap anyway, but he had to listen to it or he’d be odd man
out at the watercooler when he got back.

He listened with half an ear and considered the gigantic crowd
stretching away as far as the eye could see. Compared with the
re-opening of Fantasyland, it was nothing\dash{}goths from all over the
world had flocked to central Florida for that, Germans and Greeks
and Japanese and even some from Mumbai and Russia. They’d filled
the park to capacity, thrilled with the delightful perversity of
chirpy old Disney World remade as a goth theme park.

But a line this long in Boston, in the dead of winter, for
something whose sole attraction was that there was another one like
it by a shitty forgotten b-road outside of Miami? Christ on an
Omnimover.

The line moved, just a little surge, and there was a cheer all down
the mall’s length. People poured past him headed for the line’s
tail, vibrating with excitement. But the line didn’t move again for
five minutes, then ten. Then another surge, but maybe that was just
people crowding together more. Some of the people in line were
drinking beers out of paper bags and getting raucous.

“What’s going on?” someone hollered from behind him. The cry was
taken up, and then the line shuddered and moved forward some. Then
nothing.

Thinking, \emph{screw this}, Sammy got out of line and walked to
the front. Tjan was there, working the velvet rope, letting people
through in dribs and drabs. He caught sight of Sammy and gave him a
solemn nod. “They’re all taking too long to ride,” he said. “I tell
them fifteen minutes max, get back in line if you want to see more,
but what can you do?”

Sammy nodded sympathetically. The guy with the funny eyebrow put in
an appearance from behind the heavy black curtains. “Send through
two more,” he said, and grabbed Sammy, tugging him in.

Behind the curtain, it was dim and spotlit, almost identical to
Florida, and half a dozen vehicles waited. Sammy slid into one and
let the spiel wash over him.

\begin{speaker}
THERE WAS A TIME WHEN AMERICA HELD OUT THE PROMISE OF A NEW WAY OF
LIVING AND WORKING. THE NEW WORK BOOM OF THE TEENS WAS A PERIOD OF
UNPARALLELED INVENTION, A CAMBRIAN EXPLOSION OF CREATIVITY NOT SEEN
SINCE THE TIME OF EDISON\dash{}AND UNLIKE EDISON, THE PEOPLE WHO INVENTED
THE NEW WORK REVOLUTION WEREN’T RIP-OFF ARTISTS AND FRAUDS.
\end{speaker}

The layout was slightly different due to the support pillars, but
as similar to the Florida version as geography allowed. Robots
humped underfoot moving objects, keeping them in sync with the
changes in Florida. He’d read on the message boards that Florida
would stay open late so that the riders could collaborate with the
attendees at the Boston premiere, tweeting back and forth to one
another.

The other chairs in the ride crawled around each exhibit, reversing
and turning slowly. Riders brought their chairs up alongside one
another and conferred in low voices, over the narration from the
scenery. He thought he saw a couple making out\dash{}a common enough
occurrence in dark rides that he’d even exploited a few times when
planning out rides that would be likely to attract amorous
teenagers. They had a key demographic: too young to leave home, old
enough to pay practically anything for a private spot to score some
nookie.

The air smelled of three-d printer, the cheap smell of truck-stops
where vending machines outputted cheap kids’ toys. Here it wasn’t
cheap, though: here it smelled futuristic, like the first time
someone had handed him a printed prop for one of his rides\dash{}it had
been a head for an updated Small World ride. Then it had smelled
like something foreign and new and exciting and frightening, like
the first days of a different world.

Smelling that again, remembering the crowds outside waiting to get
in, Sammy started to get a sick feeling, the kebab rebounding on
him. Moving as if in a dream, he reached down into his lap and drew
out a small utility knife. There would be infrared cameras, but he
knew from experience that they couldn’t see through ride vehicles.

Slowly, he fingered the access panel’s underside until he found a
loose corner. He snicked out the knife’s little blade\dash{}he’d brought
an entire suitcase just so he could have a checked bag to store
this in\dash{}and tugged at the cables inside. He sawed at them with
small movements, feeling the copper wires inside the insulation
give way one strand at a time. The chair moved jerkily, then not at
all. He snipped a few more wires just to be sure, then tucked them
all away.

“Hey!” he called. “My chair’s dead!” He had fetched up in a central
pathway where the chairs tried to run cloverleafs around four
displays. A couple chairs swerved around him. He thumped the panel
dramatically, then stepped out and shook his head. He contrived to
step on three robots on the way to another chair.

“Is yours working?” he asked the kid riding in it, all of ten years
old and of indefinite gender.

“Yeah,” the kid said. It scooted over. “There’s room for both of
us, get in.”

\emph{Christ, don’t they have stranger-danger in the north?} He
climbed in beside the kid and contrived to slide one sly hand under
the panel. Teasing out the wires the second time was easier, even
one-handed. He sliced through five large bundles this time before
the chair ground to a halt, its gyros whining and rocking it from
side-to-side.

The kid looked at him and frowned. “These things are shit,” it said
with real vehemence, climbing down and kicking one of its tires,
and then kicking a couple of the floor-level robots for good
measure. They’d landed another great breakdown spot: directly in
front of a ranked display of raygun-shaped appliances and objects.
He remembered seeing that one in its nascent stage, back in
Florida\dash{}just a couple of toy guns, which were presently joined by
three more, then there were ten, then fifty, then a high wall of
them, striking and charming. The chair’s breakdown position neatly
blocked the way.

“Guess we’d better walk out,” he said. He stepped on a couple more
robots, making oops noises. The kid enthusiastically kicked robots
out of its way. Chairs swerved around them as other riders tried to
navigate. They were approaching the exit when Sammy spotted a
charge-plate for the robots. They were standard issue for robotic
vacuum cleaners and other semi-autonomous appliances, and he’d had
one in his old apartment. They were supposed to be safe as
anything, but a friend’s toddler had crawled over to his and shoved
a stack of dimes into its recessed jack and one of them had shorted
it out in a smoking, fizzing fireworks display.

“You go on ahead, I’m going to tie my shoes.”

Sammy bent down beside the charge plate, his back to the kid and
the imagined cameras that were capturing his every move, and
slipped the stack of coins he’d taken from his pocket into the
little slot where the robots inserted their charging stamen.

The ensuing shower of sparks was more dramatic than he’d
remembered\dash{}maybe it was the darkened room. The kid shrieked and ran
for the EXIT sign, and he took off too, at a good clip. They’d get
the ride up and running soon enough, but maybe not tonight, not if
they couldn’t get the two chairs he’d toasted out of the room.

There was the beginnings of chaos at the exit. There was that Tjan
character, giving him an intense look. He tried to head for the
down escalator, but Tjan cut him off.

“What’s going on in there?”

“Damnedest thing,” he said, trying to keep his face composed. “My
chair died. Then another one\dash{}a little kid was riding in it. Then
there was a lot of electrical sparks, and I walked out. Crazy.”

Tjan cocked his head. “I hope you’re not hurt. We could have a
doctor look at you; there are a couple around tonight.”

It had never occurred to Sammy that professional types might turn
out for a ride like this, but of course it was obvious. There were
probably off-duty cops, local politicians, lawyers, and the like.

“I’m fine,” he said. “Don’t worry about me. Maybe you should send
someone in for the people still in there, though?”

“That’s being taken care of. I’m just sorry you came all the way
from Florida for this kind of disappointment. That’s just brutal.”
Tjan’s measuring stare was even more intense.

“Uh, it’s OK. I had meetings here this week. This was just a cool
bonus.”

“Who do you work for, Mickey?”

Shit.

“Insurance company,” he said.

“That’d be Norwich Union, then, right? They’ve got a headquarters
here.”

Sammy knew how this went. Norwich Union didn’t have headquarters
here. Or they did. He’d have to outguess Tjan with his answer.

“Are you going to stay open tonight?”

Tjan nodded, though it wasn’t clear whether he was nodding because
he was answering in the affirmative or because his suspicions had
been confirmed.

“Well then, I should be going.”

Tjan put out a hand. “Oh, please stay. I’m sure we’ll be running
soon; you should get a whole ride through.”

“No, really, I have to go.” He shook off the hand and pelted down
the escalator and out into the freezing night. His blood sang in
his ears. They probably wouldn’t get the ride running that night at
all. They probably would send that whole carnival crowd home,
disappointed. He’d won some kind of little victory over something.

He’d felt more confident of his victory when he was concerned with
the guy with the funny eyebrow\dash{}with Perry. He’d seemed little more
than a bum, a vag. But this Tjan reminded him of the climbers he’d
met through his career at Walt Disney World: keenly observant and
fast formulators of strategies. Someone who could add two and two
before you’d know that there was such a thing as four.

Sammy walked back to his hotel as quickly as he could, given the
icy sidewalks underfoot, and by the time he got to the lobby of the
old office tower his face hurt\dash{}forehead, cheeks and nose. He’d
booked his return flight for a day later, thinking he’d do more
reccies of the new site before writing his report and heading home,
but there was no way he was facing down that Tjan guy again.

What had prompted him to sabotage the ride? It was something
primal, something he hadn’t been in any real control of. He’d been
in some kind of fugue-state. But he’d packed the little knife in
his suitcase and he’d slipped it into his pocket before leaving the
room. So how instinctive could it possibly have been?

He had a vision of the carnival atmosphere in the market stalls
outside and knew that even after the ride had broken down, the
crowd had lingered, laughing and browsing and enjoying a night’s
respite from the world and the cold city. The Whos down in
Who-ville had gone on singing even after he’d Grinched their ride.

That was it. The ride didn’t just make use of user-created
content\dash{}it \emph{was} user-created content. He could never convince
his bosses in Orlando to let him build anything remotely like this,
and given enough time, it would surely overtake them. That
Tjan\dash{}someone like him wouldn’t be involved if there wasn’t some
serious money opportunity on the line.

He’d seen the future that night and he had no place in it.

\begin{center}\rule{3in}{0.4pt}\end{center}

It only took a week on the Boston ride before they had their third
and fourth nodes. The third was outside of San Francisco, in a
gigantic ghost-mall that was already being used as a flea-market.
They had two former anchor-stores, one of which was being squatted
by artists who needed studio space. The other one made a perfect
location for a new ride, and the geeks who planned on building it
had cut their teeth building elaborate Burning Man confectioneries
together, so Perry gave them his blessing.

The fourth was to open in Raleigh, in the Research Triangle, where
the strip malls ran one into the next. The soft-spoken, bitingly
ironic southerners who proposed it were the daughters of old IBM
blue-tie stalwarts who’d been running a women’s tech collective
since they realized they couldn’t afford college and dropped out
together. They wanted to see how much admission they could charge
if they let it be known that they would plow their profits into
scholarship funds for local women.

Perry couldn’t believe that these people wanted to open their own
rides, nor that they thought they needed his permission to do so.
He was reminded of the glory days of New Work, when every day there
were fifty New Work sites with a hundred new gizmos, popping up on
the mailing lists, looking for distributors, recruiting, competing,
swarming, arguing, forming and reforming. Watching Tjan cut the
deals whereby these people were granted permission to open their
own editions of the ride felt like that, and weirder still.

“Why do they need our permission? The API’s wide open. They can
just implement. Are they sheep or something?”

Tjan gave him an old-fashioned look. “They’re being polite,
Perry\dash{}they’re giving you face for being the progenitor of the
ride.”

“I don’t like it,” Perry said. “I didn’t get anyone’s permission to
include their junk in the ride. When we get a printer to clone
something that someone brings here, we don’t get their permission.
Why the hell is seeking permission considered so polite? Shit, why
not send me a letter asking me if I mind receiving an email? Where
does it end?”

“They’re trying to be nice to you Perry, that’s all.”

“Well I don’t like it,” Perry said. “How about this: from now on
when someone asks for permission we tell them no, we don’t give out
\emph{or} withhold permission for joining the network, but we hope
that they’ll join it anyway. Maybe put up a FAQ on the site.”

“You’ll just confuse people.”

“I won’t be confusing them, man! I’ll be educating them!”

“How about if you add a Creative Commons license to it? Some of
them are very liberal.”

“I don’t \emph{want} to license this. You have to \emph{own}
something to license it. A license is a way of saying, ‘Without
this license, you’re forbidden to do this.’ You don’t need a
license to click a link and load a webpage\dash{}no one has to give you
permission to do this and no one could take it away from you.
Licensing just gives people even worse ideas about ownership and
permission and property!”

“It’s your show,” Tjan said.

“No it \emph{isn’t}! That’s the \emph{point}!”

“OK, OK, it’s not your show. But we’ll do it your way. You are a
lovable, cranky weirdo, you know it?”

They did it Perry’s way. He was scheduled to go back to Florida a
few days later, but he changed his ticket to go out to San
Francisco and meet with the crew who were implementing the ride
there. One of them taught interaction design at SFSU and brought
him in to talk to the students. He wasn’t sure what he was going to
talk to them about, but when he got there, he found himself telling
the story of how he and Lester and Tjan and Suzanne and Kettlebelly
had built and lost the New Work movement, without even trying. It
was a fun story to tell from start to finish, and they talked
through the lunch break and then a group of students took him to a
bar in the Mission with a big outdoor patio where he went on
telling war stories until the sun had set and he’d drunk so much
beer he couldn’t tell stories any longer.

They were all ten or fifteen years younger than him, and the girls
were pretty and androgynous and the boys were also pretty and
androgynous, not that he really swung that way. Still, it was fine
being surrounded by the catcalling, joking, bullshitting crowd of
young, pretty, flirty people. They hugged him a lot, and two of the
prettier girls (who, he later realized, were a lot more interested
in each other than him) took him back to a capsule hotel built
across three parking-spots and poured him into bed and tucked him
in.

He had a burrito the size of a football for breakfast, stuffed with
shredded pig-parts and two kinds of sloppy beans. He washed it down
with a quart of a cinnamon/rice drink called horchata that was
served ice-cold and did wonders for his hangover.

A couple hours’ noodling on his laptop and a couple bags of Tecate
later and he was feeling almost human. Early mariachis strolled the
street with electric guitars that controlled little tribes of
dancing, singing knee-high animatronics, belting out old Jose
Alfredo Jimenez tunes.

It was shaping up to be a good day. His laptop rang and he screwed
in his headset and started talking to Tjan.

“Man, this place is excellent,” he said. “I had the best night I’ve
had in years last night.”

“Well then you’ll love this: there’s a crew in Madison that want to
do the same thing and could use a little guidance. They spoke to me
this morning and said they’d be happy to spring for the airfare.
Can you make a six o’clock flight at SFO?”

They gave him cheese in Madison and introduced him to the
biohackers who were the spiritual progeny of the quirky moment when
Madison was one of six places where stem cells could be legally
researched. The biohackers gave him the willies. One had gills. One
glowed in the dark. One was orange and claimed to photosynthesize.

He got his hosts to bring him to the ratskeller where they sat down
to comedy-sized beers and huge, suspicious steaming wursts.

“Where’s your site?”

“We were thinking of building one\dash{}there’s a lot of farmland around
here.” Either the speaker was sixteen years old or Perry was
getting to be such a drunken old fart that everyone seemed sixteen.
He wasn’t old enough to shave, anyway. Perry tried to remember his
name and couldn’t. Jet-lag or sleepdep or whatever.

“That’s pretty weird. Everywhere else, they’re just moving into
spaces that have been left vacant.”

“We haven’t got many of those. All the offices and stuff are being
occupied by heavily funded startups.”

“Heavily funded startups? In this day and age?”

“Superbabies,” the kid said with a shrug. “It’s all anyone here
thinks about anymore. That and cancer cures. I think superbabies
are crazy\dash{}imagine being a twenty-year-old superbaby, with
two-decade-old technology in your genes. In your germline! Breeding
other obsolete superbabies. Crazy. But the Chinese are investing
heavily.”

“So no dead malls? Christ, that’s like running out of sand or
hydrogen or something. Are we still in America?”

The kid laughed. “The campus is building more student housing
because none of us can afford the rents around here anymore. But
there’s lots of farmland, like I said. Won’t be a problem to throw
up a prefab and put the ride inside it. It’ll be like putting up a
haunted cornfield at Halloween. Used to do that every year to raise
money for the ACLU, back in Nebraska.”

“Wow.” He wanted to say, \emph{They have the ACLU in Nebraska?} but
he knew that wasn’t fair. The midwesterners he’d met had generally
been kick-ass geeks and hackers, so he had no call to turn his nose
up at this kid. “So why do you want to do this?”

The kid grinned. “Because there’s got to be a way to do something
cool without moving to New York. I like it around here. Don’t want
to live in some run-down defaulted shit-built condo where the mice
are hunchbacked. Like the wide-open spaces. But I don’t want to be
a farmer or an academic or run a student bar. All that stuff is a
dead-end, I can see it from here. I mean, who drinks beer anymore?
There’s much sweeter highs out there in the real world.”

Perry looked at his beer. It was in a themed stein with
Germano-Gothic gingerbread worked into the finish. It felt like it
had been printed from some kind of ceramic/epoxy hybrid. You could
get them at traveling carny midways, too.

“I like beer,” he said.

“But you’re\dash{}” The kid broke off.

“Old,” Perry said. “’Sok. You’re what, 16?”

“21,” the kid said. “I’m a late bloomer. Devoting resources to more
important things than puberty.”

Two more kids slid into their booth, a boy and a girl who actually
did look 21. “Hey Luke,” the girl said, kissing him on the cheek.

Luke, that was his name. Perry came up with a mnemonic so he
wouldn’t forget it again\dash{}Nebraska baby-faced farm boy, that was
like Luke Skywalker. He pictured the kid swinging a lightsaber and
knew he’d keep the name for good now.

“This is Perry Gibbons,” Luke said. “Perry, this is Hilda and
Ernie. Guys, Perry’s the guy who built the ride I was telling you
about.”

Ernie shook his hand. “Man, that’s the coolest shit I’ve ever seen,
wow. What the hell are you doing here? I love that stuff. Wow.”

Hilda flicked his ear. “Stop drooling, fanboy,” she said.

Ernie rubbed his ear. Perry nodded uncertainly.

“Sorry. It’s just\dash{}well, I’m a big fan is all.”

“That’s really nice of you,” Perry said. He’d met a couple people
in Boston and San Francisco who called themselves his fans, and he
hadn’t known what to say to them, either. Back in the New Work days
he’d meet reporters who called themselves fans, but that was just
blowing smoke. Now he was meeting people who seemed to really mean
it. Not many, thank God.

“He’s just like a puppy,” Hilda said, pinching Ernie’s cheek. “All
enthusiasm.”

Ernie rubbed his cheek. Luke reached out abruptly and tousled both
of their hair. “These two are going to help me build the ride,” he
said. “Hilda’s an amazing fundraiser. Last year she ran the
fundraising for a whole walk-in clinic.”

“Women’s health clinic or something?” Perry asked. He was starting
to sober up a little. Hilda was one of those incredible, pneumatic
midwestern girls that he’d seen at five minute intervals since
getting off his flight in Madison. He didn’t think he’d ever met
one like her.

“No,” Hilda said. “Metabolic health. Lots of people get the fatkins
treatment at puberty, either because their fatkins parents talk
them into it or because they hate their baby fat.”

Perry shook his head. “Come again?”

“You think eating ten thousand calories a day is easy? It’s hell on
your digestive system. Not to mention you spend a fortune on food.
A lot of people get to college and just switch to high-calorie
powdered supplements because they can’t afford enough real food to
stay healthy, so you’ve got all these kids sucking down vanilla
slurry all day just to keep from starving. We provide counseling
and mitigation therapies to kids who want it.”

“And when they get out of college\dash{}do they get the treatment
again?”

“You can’t. The mitigation’s permanent. People who take it have to
go through the rest of their lives taking supplements and eating
sensibly and exercising.”

“Do they get fat?”

She looked away, then down, then back up at him. “Yes, most of them
do. How could they not? Everything around them is geared at people
who need to eat five times as much as they do. Even the salads all
have protein powder mixed in with them. But it is \emph{possible}
to eat right. You’ve never had the treatment, have you?”

Perry shook his head. “Trick metabolism though. I can eat like a
hog and not put on an ounce.”

Hilda reached out and squeezed his bicep. “Really\dash{}and I suppose
that all that lean muscle there is part of your trick metabolism,
too?”

She left her hand where it was.

“OK, I do a fair bit of physical labor too. But I’m just saying\dash{}if
they get fat again after they reverse the treatment\dash{}”

“There are worse things than being fat.”

Her hand still hadn’t moved. He looked at Ernie, whom he’d assumed
was her boyfriend, to see how he was taking it. Ernie was looking
somewhere else, though, across the ratshkeller, at the huge TV that
was showing competitive multiplayer gaming, apparently some kind of
championships. It was as confusing as a hundred air-hockey games
being played on the same board, with thousands of zipping, jumping,
firing entities and jump-cuts so fast that Perry couldn’t imagine
how you’d make sense of it.

The girl’s hand was still on his arm, and it was warm. His mouth
was dry but more beer would be a bad idea. “How about some water?”
he said, in a bit of a croak.

Luke jumped up to get some, and a silence fell over the table. “So
this clinic, how’d you fundraise for it?”

“Papercraft,” she said. “I have a lot of friends who are into
paper-folding and we modded a bunch of patterns. We did really big
pieces, too\dash{}bed-frames, sofas, kitchen-tables, chairs\dash{}”

“Like actual furniture?”

“Like actual furniture,” she said with a solemn nod. “We used huge
sheets of paper and treated them with stiffening, waterproofing and
fireproofing agents. We did a frat house’s outdoor bar and sauna,
with a wind-dynamo\dash{}I even made a steam engine.”

“You made a steam engine out of paper?” He was agog.

“You mean to say that \emph{you’re} surprised by building stuff out
of unusual materials?”

Perry laughed. “Point taken.”

“We just got a couple hundred students to do some folding in their
spare time and then sold it on. Everyone on campus needs
bookshelves, so we started with those\dash{}using accordion-folded arched
supports under each shelf. We could paint or print designs on them,
too, but a lot of people liked them all-white. Then we did chairs,
desks, kitchenette sets, placemats\dash{}you name it. I called the
designs ‘Multiple Origami.’”

Perry sprayed beer out his nose. “That’s awesome!” he said, wiping
up the mess with a kleenex that she extracted from a folded paper
purse. Looking closely, he realized that the white baseball cap she
was wearing was also folded out of paper.

She laughed and rummaged some more in her handbag, coming up with a
piece of stiff card. Working quickly and nimbly, she gave it a few
deft folds along pre-scored lines, and a moment later she was
holding a baseball hat that was the twin of the one she was
wearing. She leaned over the table and popped it on his head.

Luke came back with the water and set it down between them, pouring
out glasses for everyone.

“Smooth lid,” he said, touching the bill of Perry’s cap.

“Thanks,” Perry said, draining his water and pouring another glass.
“Well, you people certainly have some pretty cool stuff going on
here.”

“This is a great town,” Luke said expansively, as though he had
travelled extensively and settled on Madison, Wisconsin as a truly
international hotspot. “We’re going to build a kick-ass ride.”

“You going to make it all out of paper?”

“Some of it, anyway,” Luke said. “Hilda wouldn’t have it any other
way, right?”

“This one’s your show, Luke,” she said. “I’m just a fundraiser.”

“Anyone hungry?” Hilda said. “I want to go eat something that
doesn’t have unidentified organ-meat mixed in.”

“Go on without me,” Ernie said. “I got money on this game.”

“Homework,” Luke said.

Perry had just eaten, and had planned on spending this night in his
room catching up on email. “Yeah, I’m starving,” he said. He felt
like a high-school kid, but in a good way.

They went out for Ukrainian food, which Perry had never had before,
but the crepes and the blood sausage were tasty enough. Mostly,
though, he was paying attention to Hilda, who was running down her
war stories from the Multiple Origami fundraiser. There were funny
ones, sad ones, scary ones, triumphant ones.

Every one of her stories reminded him of one of his own. She was an
organizer and so was he and they’d been through practically the
same shit. They drank gallons of coffee afterward, getting chucked
out when the restaurant closed and migrating to a cafe on the main
drag where they had low tables and sofas, and they never stopped
talking.

“You know,” Hilda said, stretching and yawning, “it’s coming up on
four AM.”

“No way,” he said, but his watch confirmed it. “Christ.” He tried
to think of a casual way of asking her to sleep with him. For all
their talking, they’d hardly touched on romance\dash{}or maybe there’d
been romance in every word.

“I’ll walk you to your hotel,” she said.

“Hey, that’s really nice of you,” he said. His voice sounded fakey
and forced in his ears. All of a sudden, he wasn’t tired at all,
instead his heart was hammering in his chest and his blood sang in
his ears.

There was hardly any talk on the way back to the hotel, just the
awareness of her steps and his in time with one another over the
cold late-winter streets. No traffic at that hour, and hardly a
sound from any of the windows they passed. The town was theirs.

At the door to his hotel\dash{}another stack of the ubiquitous capsules,
these geared to visiting parents\dash{}they stopped. They were looking at
one another like a couple of googly-eyed kids at the end of a date
in a sitcom.

“Um, what’s your major?” he said.

“Pure math,” she said.

“I think I know what that is,” he said. It was freezing out on the
street. “Theory, right?”

“Pure math as opposed to applied math,” she said. “Do you really
care about this?”

“Um,” he said. “Well, yes. But not very much.”

“I’ll come into your hotel room, but we’re not having sex, OK?”

“OK,” he said.

There was room enough for the two of them in the capsule, but only
just. These were prefabbed in bulk and they came in different
sizes\dash{}in the Midwest they were large, the ones stacked up in San
Francisco parking spots were small. Still, he and Hilda were almost
in each other’s laps, and he could smell her, feel wisps of her
hair tickling his ear.

“You’re really nice,” he said. Late at night, his ability to be
flippant evaporated. He was left with simple truths, simply
declared. “I like you a lot.”

“Well then you’ll have to come back to Madison and check in on the
ride, won’t you?”

“Um,” he said. He had a planning meeting with Luke and the rest of
his gang the next day, then he was supposed to be headed for Omaha,
where Tjan had set up another crew for him to speak to. At this
rate, he would get back to Florida some time in June.

“Perry, you’re not a career activist, are you?”

“Nope,” he said. “I hadn’t really imagined that there was such a
thing.”

“My parents. Both of them. Here’s what being a career activist
means: you are on the road most of the time. When you get on the
road, you meet people, have intense experiences with them\dash{}like
going to war or touring with a band. You fall in love a thousand
times. And then you leave all those people behind. You get off a
plane, turn some strangers into best friends, get on a plane and
forget them until you come back into town, and then you take it all
back up again.

“If you want to survive this, you’ve got to love that. You’ve got
to get off a plane, meet people, fall in love with them, treasure
every moment, and know that moments are all you have. Then you get
on a plane again and you love them forever. Otherwise, every new
meeting is sour because you know how soon it will end. It’s like
starting to say your summer-camp goodbyes before you’ve even
unpacked your duffel-bag. You’ve got to embrace\dash{}or at least
forget\dash{}that every gig will end in a day or two.”

Perry took a moment to understand this, swallowed a couple times,
then nodded. Lots of people had come in and out of his factory and
his ride over the years. Lester came and went. Suzanne was gone.
Tjan was gone but was back again. Kettlebelly was no longer in his
life at all, a ghost of a memory with a great smile and good
cologne. Already he was forgetting the faces in Boston, the faces
in San Francisco. Hilda would be a memory in a month.

Hilda patted his hand. “I have friends in practically every city in
America. My folks campaigned for stem cells up and down every red
state in the country. I even met superman before he died. He knew
my name. I spent ten years on the road with them, back and forth.
The Bush years, a couple years afterward. You can live this way and
you can be happy, but you’ve got to have right mind.

“What it means is you’ve got to be able to say things to people you
meet, like, ‘You’re really nice,’ and mean it, really mean it. But
you’ve also got to be cool with the fact that really nice people
will fall out of your life every week, twice a week, and fall back
into it or not. I think you’re very nice, too, but we’re not gonna
be a couple, ever. Even if we slept together tonight, you’d be gone
tomorrow night. What you need to ask yourself is whether you want
to have friends in every city who are glad to see you when you get
off the plane, or ex-girlfriends in every city who might show up
with their new boyfriends, or not at all.”

“Are you telling me this to explain why we’re not going to sleep
together? I just figured you were dating that guy, Ernie.”

“Ernie’s my brother,” she said. “And yeah, that’s kind of why I’m
telling you this. I’ve never gone on what you might call a date.
With my friends, it tends to be more like, you work together, you
hang out together, you catch yourself looking into one another’s
eyes a couple times, then you do a little circling around and then
you end up in your bed or their bed having hard, energetic sex and
then you sort out some details and then it lasts as long as it
lasts. We’ve done a compressed version of that tonight, and we’re
up to the sex, and so I thought we should lay some things on the
table, you should forgive the expression.”

Perry thought back to his double-date with Lester. The girl had
been pretty and intelligent and would have taken him home if he’d
made the least effort. He hadn’t, though. This girl was
inappropriate in so many ways: young, rooted to a city thousands of
miles from home\dash{}why had he brought her back to the hotel?

A thought struck him. “Why do you think I’m going to be getting on
and off planes for the rest of my life? I’ve got a home to get
to.”

“You haven’t been reading the message boards, have you?”

“Which message boards?”

“For ride-builders. There are projects starting up everywhere.
People like what they’ve heard and what they’ve seen, and they
remember you from the old days and want to get in on the magic
you’re going to bring. A lot of us know each other anyway, from
other joint projects. Everyone’s passing the hat to raise your
airfare and arguing about who’s sofa you’re going to stay on.”

He’d known that they were there. There were always message-boards.
But they were just talk\dash{}he never bothered to read them. That was
Lester’s job. He wanted to make stuff, not chatter. “Jesus, when
the hell was someone going to tell me?”

“Your guy in Boston, we’ve been talking to him. He said not to bug
you, that you were busy enough as it is.”

He did, did he? In the old days, Tjan had been in charge of
planning and he’d been in charge of the ideas: in charge of what to
plan. Had they come full circle without him noticing? If they had,
was that so bad?

“Man, I was really looking forward to spending a couple nights in
my own bed.”

“Is it much more comfortable than this one?” She thumped the narrow
coffin-bed, which was surprisingly comfortable, adjustable, heated,
and massaging.

He snorted. “OK, I sleep on a futon on the floor back home, but
it’s the principle of the thing. I just miss home, I guess.”

“So go home for a couple days after this stop, or the next one.
Charge up your batteries and do your laundry. But I have a feeling
that home is going to be your suitcase pretty soon, Perry my dear.”
Her voice was thick with sleep, her eyes heavy-lidded and bleary.

“You’re probably right.” He yawned as he spoke. “Hell, I know
you’re right. You’re a real smarty.”

“And I’m too tired to go home,” she said, “so I’m a smarty who’s
staying with you.”

He was suddenly wide awake, his heart thumping. “Um, OK,” he said,
trying to sound casual.

He turned back the sheets, then, standing facing into the cramped
corner, took off his jeans and shoes and socks, climbing in between
the sheets in his underwear and tee. There were undressing
noises\dash{}exquisite ones\dash{}and then she slithered in behind him,
snuggled up against him. With a jolt, he realized that her bare
breasts were pressed to his back. Her arm came around him and
rested on his stomach, which jumped like a spring uncoiling. He
felt certain his erection was emitting a faint cherry-red glow. Her
breath was on his neck.

He thought about casually rolling onto his back so that he could
kiss her, but remembered her admonition that they would not be
having sex. Her fingertips traced small circles on his stomach.
Each time they grazed his navel, his stomach did a flip.

He was totally awake now, and when her lips very softly\dash{}so softly
he barely felt it\dash{}brushed against the base of his skull, he let out
a soft moan. Her lips returned, and then her teeth, worrying at the
tendons at the back of his neck with increasing roughness, an
exquisite pain-pleasure that was electric. He was panting, her hand
was flat on his stomach now, gripping him. His erection strained
toward it.

Her hips ground against him and she moved her mouth toward his ear,
nipping at it, the tip of her tongue touching the whorls there. Her
hand was on the move now, sliding over his ribs, her fingertips at
his nipple, softly and then harder, giving it an abrupt hard pinch
that had some fingernail in it, like a bite from little teeth. He
yelped and she giggled in his ear, sending shivers up his spine.

He reached back behind him awkwardly and put his hand on her ass,
discovering that she was bare there, too. It was wide and hard,
foam rubber over steel, and he kneaded it, digging his fingers in.
She groaned in his ear and tugged him onto his back.

As soon as his shoulders hit the narrow bed, she was on him, her
elbows on his biceps, pinning him down, her breasts in his face,
fragrant and soft. Her hot, bare crotch ground against his
underwear. He bit at her tits, hard little bites that made her
gasp. He found a stiff nipple and sucked it into his mouth, beating
at it with his tongue. She pressed her crotch harder against his,
hissed something that might have been \emph{yesssss}.

She straightened up so that she was straddling him and looking
imperiously down on him. Her braids swung before her. Her eyes were
exultant. Her face was set in an expression of fierce concentration
as she rocked on him.

He dug his fingers into her ass again, all the way around, so that
they brushed against her labia, her asshole. He pulled at her,
dragging her up her body, tugging her vagina toward his mouth. Once
she saw what he was after, she knee-walked up the bed in three or
four quick steps and then she was on his face. Her smell and her
taste and her texture and temperature filled his senses, blotting
out the room, blotting out introspection, blotting out everything
except for the sweet urgency.

He sucked at her labia before slipping his tongue up her length,
letting it tickle her ass, her opening, her clit. In response, she
ground against him, planting her opening over his mouth and he
tongue-fucked her in hard, fast strokes. She reached back and took
hold of his cock, slipping her small, strong hand under the
waistband of his boxers and curling it around his rigid shaft,
pumping vigorously.

He moaned into her pussy and that set her shuddering. Now he had
her clit sucked into his mouth and he was lapping at its engorged
length with short strokes. Her thighs were clamped over his ears,
but he could still make out her cries, timed with the shuddering of
her thighs, the spasmodic grip on his cock.

Abruptly she rolled off of him and the world came back. They hadn’t
kissed yet. They hadn’t said a word. She lay beside him, half on
top of him, shuddering and making kittenish sounds. He kissed her
softly, then more forcefully. She bit at his lips and his tongue,
sucking it into her mouth and chewing at it while her fingernails
raked his back.

Her breathing became more regular and she tugged at the waistband
of his boxers. He got the message and yanked them off, his cock
springing free and rocking slightly, twitching in time with his
pulse. She smiled a cat-ate-the-canary grin and went to work
kissing his neck, his chest\dash{}hard bites on his nipples that made him
yelp and arch his back\dash{}his stomach, his hips, his pubes, his
thighs. The teasing was excruciating and exquisite. Her juices
dried on his face, the smell caught in his nose, refreshing his
eros with every breath.

Her tongue lapped eagerly at his balls like a cat with a saucer of
milk. Long, slow strokes, over his sack, over the skin between his
balls and his thighs, over his perineum, tickling his ass as he’d
tickled hers. She pulled back and spat out a pube and laughed and
dove back in, sucking softly at his sack, then, in one swift
motion, taking his cock to the hilt.

He shouted and then moaned and her head bobbed furiously along the
length of his shaft, her hand squeezing his balls. It took only
moments before he dug his hands hard into the mattress and groaned
through clenched teeth and fired spasm after spasm down her throat,
her nose in his pubes, his cock down her throat to the base. She
refused to let him go, swirling her tongue over the head while he
was still super-sensitive, making him grunt and twitch and buck
involuntarily, all the while her hand caressing his balls, rubbing
at his prostate over the spot between his balls and his ass.

Finally she worked her way back up his body licking her lips and
kissing as she went.

“Hello,” she said as she buried her face in his throat.

“Wow,” he said.

“So if you’re going to be able to live in the moment and have no
regrets, this is a pretty good place to start. It’d be a hell of a
shock if we saw each other twice in the next year\dash{}are we going to
be able to be friends when we do? Will the fact that I fucked your
brains out make things awkward?”

“That’s why you jumped me?”

“No, not really. I was horny and you’re hot. But that’s a good
post-facto reason.”

“I see. You know, you haven’t actually fucked my brains out,” he
said.

“Yet,” she said. She retrieved her backpack from beside the bed,
dug around it in, and produced a strip of condoms. “Yet.”

He licked his lips in anticipation, and a moment later she was
unrolling the condom down his shaft with her talented mouth. He
laughed and then took her by the waist and flipped her onto her
back. She grabbed her ankles and pulled her legs wide and he dove
between her, dragging the still-sensitive tip of his cock up and
down the length of her vulva a couple times before sawing it in and
out of her opening, sinking to the hilt.

He wanted to fuck her gently but she groaned urgent demands in his
ear to pound her harder, making satisfied sounds each time his
balls clapped against her ass.

She pushed him off her and turned over, raising her ass in the air,
pulling her labia apart and looking over her shoulder at him. They
fucked doggy-style then, until his legs trembled and his knees
ached, and then she climbed on him and rocked back and forth,
grinding her clit against his pubis, pushing him so deep inside
her. He mauled her tits and felt the pressure build in his balls.
He pulled her to him, thrust wildly, and she hissed dirty
encouragement in his ear, begging him to fill her, ordering him to
pound her harder. The stimulation in his brain and between his legs
was too much to bear and he came, lifting them both off the bed
with his spasms.

“Wow,” he said.

“Yum,” she said.

“Jesus, it’s 8AM,” he said. “I’ve got to meet with Luke in three
hours.”

“So let’s take a shower now, and set an alarm for half an hour
before he’s due,” she said. “Got anything to eat.”

“That’s what I like about you Hilda,” he said. “Businesslike.
Vigorous. Living life to the hilt.”

Her dimples were pretty and luminous in the hints of light emerging
from under the blinds. “Feed me,” she said, and nipped at his
earlobe.

In the shoebox-sized fridge, he had a cow-shaped brick of Wisconsin
cheddar that he’d been given when he stepped off the plane. They
broke chunks off it and ate it in bed, then started in on the bag
of soy crispies his hosts in San Francisco had given him. They
showered slowly together, scrubbing one-another’s backs, set an
alarm, and sacked out for just a few hours before the alarm roused
them.

They dressed like strangers, not embarrassed, just too groggy to
take much notice of one another. Perry’s muscles ached pleasantly,
and there was another ache, dull and faint, even more pleasant, in
his balls.

Once they were fully clothed, she grabbed him and gave him a long
hug, and a warm kiss that started on his throat and moved to his
mouth, with just a hint of tongue at the end.

“You’re a good man, Perry Gibbons,” she said. “Thanks for a lovely
night. Remember what I told you, though: no regrets, no looking
back. Be happy about this\dash{}don’t mope, don’t miss me. Go on to your
next city and make new friends and have new conversations, and when
we see each other again, be my friend without any awkwardness. All
right?”

“I get it,” he said. He felt slightly irritated. “Only one thing.
We weren’t going to sleep together.”

“You regret it?”

“Of course not,” he said. “But it’s going to make this injunction
of yours hard to understand. I’m not good at anonymous one night
stands.”

She raised one eyebrow at him. “Earth to Perry: this wasn’t
anonymous, and it wasn’t a one-night stand. It was an intimate,
loving relationship that happened to be compressed into a single
day.”

“Loving?”

“Sure. If I’d been with you for a month or two, I would have fallen
in love. You’re just my type. So I think of you as someone I love.
That’s why I want to make sure you understand what this all
means.”

“You’re a very interesting person,” he said.

“I’m smart,” she said, and cuddled him again. “You’re smart. So be
smart about this and it’ll be forever sweet.”

She left him off at the spot where he was supposed to meet Luke and
the rest of his planning team to go over schematics and theory and
practice. All of these discussions could happen online\dash{}they did, in
fact\dash{}but there was something about the face-to-face connection. The
meeting ran six hours before he was finally saved by his impending
flight to Nebraska.

Sleepdep came down on him like a hammer as he checked in for his
flight and began the ritual security-clearing buck-and-wing. He
missed a cue or two and ended up getting a “detailed hand search”
but even that didn’t wake him up. He fell asleep in the waiting
room and in the plane, in the taxi to his hotel.

But when he dropped down onto his hotel bed, he couldn’t sleep. The
hotel was the spitting image of the one he’d left in Wisconsin,
minus Hilda and the musky smell the two of them had left behind
after their roll in the hay.

It had been years since he’d had a regular girlfriend and he’d
never missed it. There had been women, high-libido fatkins girls
and random strangers, some who came back for a date or two. But no
one who’d meant anything or whom he’d wanted to mean anything. The
closest he’d come had been\dash{}he sat up with a start and realized that
the last woman he’d had any strong feelings for had been Suzanne
Church.

\begin{center}\rule{3in}{0.4pt}\end{center}

Kettlewell emerged from New Work rich. He’d taken home large
bonuses every year that Kodacell had experienced growth\dash{}a better
metric than turning an actual ahem profit\dash{}and he’d invested in a
diverse portfolio that had everything from soybeans to software in
it, along with real estate (oops) and fine art. He believed in the
New Work, believed in it with every fiber of his being, but an
undiverse portfolio was flat-out irresponsible.

The New Work crash had killed the net worth of a lot of
irresponsible people.

Living in the Caymans got boring after a year. The kids hated the
international school, scuba diving amazed him by going from
endlessly, meditatively fascinating to deadly dull in less than a
year. He didn’t want to sail. He didn’t want to get drunk. He
didn’t want to join the creepy zillionaires on their sex tours of
the Caribbean and wouldn’t have even if his wife would have stood
for it.

A year after the New Work crash, he filed a 1040 with the IRS and
paid them forty million dollars in back taxes and penalties, and
repatriated his wealth to an American bank.

Now he lived in a renovated housing project on Potrero Hill in San
Francisco, all upscale now with restored, kitschy window-bars and
vintage linoleum and stucco ceilings. He had four units over two
floors, with cleverly knocked-through walls and a spiral staircase.
The kids freaking loved the staircase.

Suzanne Church called him from SFO to let him know that she was on
her way in, having cleared security and customs after a scant hour.
He found himself unaccountably nervous about her now, and realized
with a little giggle that he had something like a crush on her.
Nothing serious\dash{}nothing his wife needed to worry about\dash{}but she was
smart and funny and attractive and incisive and fearless, and it
was a hell of a combination.

The kids were away at school and his wife was having a couple of
days camping with the girls in Yosemite, which facts lent a little
charge to Suzanne’s impending visit. He looked up the AirBART
schedule and calculated how long he had until she arrived at the
24th Street station, a brisk 20 minute walk from his place.

Minutes, just minutes. He checked the guest-room and then did a
quick mirror check. His months in the Caymans had given him a deep
tan that he’d kept up despite San Francisco’s grey skies. He still
looked like a surfer, albeit with just a little daddy-paunch\dash{}he’d
gained more weight through his wife’s pregnancies than she had and
only hard, aneurysm-inducing cycling over and around Potrero Hill
had knocked it off again. His jeans’ neat rows of pockets and
Mobius seams were a little outdated, but they looked good on him,
as did his Hawai’ian print shirt with its machine-screw motif.

Finally he plopped down to read a book and waited for Suzanne, and
managed to get through a whole page in the intervening ten
minutes.

“Kettlebelly!” she hollered as she came through the door. She took
him in a hug that smelled of stale airplane and restless sleep and
gave him a thorough squeezing.

She held him at arm’s length and they sized each other up. She’d
been a well-preserved mid-forties when he’d seen her last,
buttoned-down in a California-yoga-addict way. Now she was years
older, and her time in Russia had given her a forest of smile-lines
at the corners of her mouth and eyes. She had a sad, wise turn to
her face that he’d never seen there before, like a painted Pieta.
Her hands had gone a little wrinkly, her knuckles more prominent,
but her fingernails were beautifully manicured and her clothes were
stylish, foreign, exotic and European.

She laughed huskily and said, “You haven’t changed a bit.”

“Ouch,” he said. “I’m older and wiser, I’ll have you know.”

“It doesn’t show,” she said. “I’m older, but no wiser.”

He took her hand and looked at the simple platinum band on her
finger. “But you’re married now\dash{}nothing wises you up faster in my
experience.”

She looked at her hand. “Oh, that. No. That’s just to keep the
wolves at bay. Married women aren’t the same kinds of targets that
single ones are. Give me water, and then a beer, please.”

Glad to have something to do, he busied himself in the kitchen
while she prowled the place. “I remember when these places were
bombed-out, real ghettos.”

“What did you mean about being a target?”

“St Pete’s, you know. Lawless state. Everyone’s on the make. I had
a bodyguard most of the time, but if I wanted to go to a
restaurant, I didn’t want to have to fend off the dating-service
mafiyeh who wanted to offer me the deal of a lifetime on a
green-card marriage.”

“Jeez.”

“It’s another world, Landon. You know what the big panic there is
this week? A cult of ecstatic evangelical Christians who
’hypnotize’ women in the shopping malls and steal their babies to
raise as soldiers to the Lord. God knows how much of it is true.
These guys don’t bathe, and dress in heavy coats with big beards
all year round. I mean, freaky, really freaky.”

“They hypnotize women?”

“Weird, yeah? And the \emph{driving}! Anyone over the age of fifty
who knows how to drive got there by being an apparat in the Soviet
days, which means that they learned to drive when the roads were
empty. They don’t signal, they straddle lanes, they can’t park\dash{}I
mean, they \emph{really} can’t park. And drunk! Everyone, all the
time! You’ve never seen the like. Imagine a frat party the next
day, with a lot of innocent bystanders, hookers, muggers and
pickpockets.”

Landon looked at her. She was animated and vivid, thin\dash{}age had
brought out her cheekbones and her eyes. Had she had a chin-tuck?
It was common enough\dash{}all the medical tourists loved Russia. Maybe
she was just well-preserved.

She made a show of sniffing herself. “Phew! I need a shower! Can I
borrow your facilities?”

“Sure,” he said. “I put clean towels out in the kids’
bathroom\dash{}upstairs and second on the right.”

She came down with her fine hair slicked back over her ears, her
face scrubbed and shining. “I’m a new woman,” she said. “Let’s go
somewhere and eat something, OK?”

He took her for pupusas at a Salvadoran place on Goat Hill. They
slogged up and down the hills and valleys, taking the steps cut
into the steep sides, walking past the Painted Ladies\dash{}grand, gaudy
Victorian wood-frames\dash{}and the wobbly, heavy canvas bubble-houses
that had sprung up where the big quake and landslides had washed
away parts of the hills.

“I’d forgotten that they had hills like that,” she said, greedily
guzzling an horchata. Her face was streaked with sweat and
flushed\dash{}it made her look prettier, younger.

“My son and I walk them every day.”

“You drag a little kid up and down that every \emph{day}? Christ,
that’s child abuse!”

“Well, he poops out after a couple of peaks and I end up carrying
him.”

“You \emph{carry} him? You must be some kind of superman.” She gave
his bicep a squeeze, then his thigh, then slapped his butt. “A fine
specimen. Your wife’s a lucky woman.”

He grinned. Having his wife in the conversation made him feel less
at risk.
\emph{That’s right, I’m married and we both know it. This is just fun flirting. Nothing more.}

They bit into their pupusas\dash{}stuffed cornmeal dumplings filled with
grilled pork and topped with shredded cabbage and hot sauce\dash{}and
grunted and ate and ordered more.

“What are these called again?”

“Pupusas, from El Salvador.”

“Humph. In my day, we ate Mexican burritos the size of a football,
and we were grateful.”

“No one eats burritos anymore,” he said, then covered his mouth,
aware of how pretentious that sounded.

“Dahling,” she said, “burritos are \emph{so} 2005. You \emph{must}
try a pupusa\dash{}it’s what all the most charming Central American
peasants are eating now.”

They both laughed and stuffed their faces more. “Well, it was
either here or one of the fatkins places with the triple-decker
stuffed pizzas, and I figured\dash{}”

“They really do that?”

“The fatkins? Yeah\dash{}anything to get that magical 10,000 calories any
day. It must be the same in Russia, right? I mean, they invented
it.”

“Maybe for fifteen minutes. But most of them don’t bother\dash{}they get
a little metabolic tweak, not a wide-open throttle like that.
Christ, what it must do to your digestive system to process 10,000
calories a day!”

“Chacun a son gout,” he said, essaying a Gallic shrug.

She laughed again and they ate some more. “I’m starting to feel
human at last.”

“Me too.”

“It’s still mid-afternoon, but my circadian thinks it’s 2AM. I need
to do something to stay awake or I’ll be up at four tomorrow
morning.”

“I have some modafinil,” he said.

“Swore ’em off. Let’s go for a walk.”

They did a little more hill-climbing and then headed into the
Mission and window-shopped the North African tchotchke emporia that
were crowding out the Mexican rodeo shops and hairdressers. The
skin drums and rattles were laser-etched with intricate
designs\dash{}Coca Cola logos, the UN Access to Essential Medicines
Charter, Disney characters. It put them both in mind of the old
days of the New Work, and the subject came up again, hesitant at
first and then a full-bore reminisce.

Suzanne told him stories of the things that Perry and Lester had
done that she’d never dared report on, the ways they’d skirted the
law and his orders. He told her a few stories of his own, and they
rocked with laughter in the street, staggering like drunks,
pounding each other on the backs, gripping their knees and stomachs
and doubling over to the curious glances of the passers-by.

It was fine, that day, Landon thought. Some kind of great sorrow
that he’d forgotten he’d carried lifted from him and his chest and
shoulders expanded and he breathed easy. What was the sorrow? The
death of the New Work. The death of the dot-coms. The death of
everything he’d considered important and worthy, its fading into
tawdry, cheap nostalgia.

They were sitting in the grass in Dolores Park now, watching the
dogs and their people romp among the robot pooper-scoopers. He had
his arm around her shoulders, like war-buddies on a bender (he told
himself) and not like a middle-aged man flirting with a woman he
hadn’t seen in years.

And then they were lying down, the ache of laughter in their
bellies, the sun on their faces, the barks and happy shouts around
them. Their hands twined together (but that was friendly too, Arab
men held hands walking down the street as a way of showing
friendship).

Now their talk had banked down to coals, throwing off an occasional
spark when one or the other would remember some funny anecdote and
grunt out a word or two that would set them both to gingerly
chuckling. But their hands were tied and their breathing was in
sync, and their flanks were touching and it wasn’t just friendly.

Abruptly, she shook her hand free and rolled on her side. “Listen,
married man, I think that’s enough of that.”

He felt his face go red. His ears rang. “Suzanne\dash{}what\dash{}” He was
sputtering.

“No harm no foul, but let’s keep it friendly, all right.”

The spell was broken, and the sorrow came back. He looked for the
right thing to say. “God I miss it,” he said. “Oh, Suzanne, God, I
miss it so much, every day.”

Her face fell, too. “Yeah.” She looked away. “I really thought we
were changing the world.”

“We were,” he said. “We did.”

“Yeah,” she said again. “But it didn’t matter in the end, did it?
Now we’re older and our work is forgotten and it’s all come to
nothing. Petersburg is nice, but who gives a shit? Is that what I’m
going to do with the rest of my life, hang around Petersburg
blogging about the mafiyeh and medical tourism? Just shoot me
now.”

“I miss the people. I’d meet ten amazing creative geniuses every
day\dash{}at least! Then I’d give them money and they’d make amazing
stuff happen with it. The closest I come to that now is my kids,
watching them learn and build stuff, which is really great, don’t
get me wrong, but it’s nothing like the old days.”

“I miss Lester. And Perry. Tjan. The whole gang of them, really.”
She propped herself up on one elbow and then shocked him by kissing
him hard on the cheek. “Thanks, Kettlebelly. Thank you so much for
putting me in the middle of all that. You changed my life, that’s
for sure.”

He felt the imprint of her lips glowing on his cheek and grinned.
“OK, here’s an idea: let’s go buy a couple bottles of wine, sit on
my patio, get a glow on, and then call Perry and see what he’s up
to.”

“Oh, that’s a good one,” she said. “That’s a \emph{very} good
one.”

A few hours later, they sat on the horsehair club-sofa in
Kettlewell’s living room and hit a number he’d never taken out of
his speed-dial. “Hi, this is Perry. Leave a message.”

“Perry!” they chorused. They looked at each other, at a loss for
what to say next, then dissolved in peals of laughter.

“Perry, it’s Suzanne and Kettlebelly. What the hell are you up to?
Call us!”

They looked at the phone with renewed hilarity and laughed some
more. But by the time the sun was setting over Potrero Hill and
Suzanne’s jet-lag was beating her up again, they’d both descended
into their own personal funks. Suzanne went up to the guest room
and put herself to bed, not bothering to brush her teeth or even
change into her nightie.

\begin{center}\rule{3in}{0.4pt}\end{center}

Perry touched down in Miami in a near-coma, his eyes gummed shut by
several days’ worth of hangovers chased by drink. Sleep deprivation
made him uncoordinated, so he tripped twice deplaning, and his
voice was a barely audible rasp, his throat sore with a cold he’d
picked up in Texas or maybe it was Oklahoma.

Lester was waiting beyond the luggage carousels, grinning like a
holy fool, tall and broad-shouldered and tanned, dressed in fatkins
pimped-out finery, all tight stretch-fabrics and glitter.

“Oh man, you look like shit,” he said, breaking off from the
fatkins girl he’d been chatting up. Perry noticed that he was
holding his phone, a sure sign that he’d gotten her number.

“Ten,” Perry said, grinning through the snotty rheum of his cold.
“Ten rides.”

“Ten rides?” Lester said.

“Ten. San Francisco, Austin, Minneapolis, Omaha, Oklahoma City,
Madison, Bellingham, Chapel Hill and\dash{}” He faltered. “And\dash{}Shit. I
forget. It’s all written down.”

Lester took his bag from him and set it down, then crushed him in
an enormous, muscular hug that whiffed slightly of the ketosis
fumes that all the fatkins exuded.

“You did good, cowboy,” he said. “Let’s mosey back to the ranch,
feed you and put you to bed, s’awright?”

“Can I sleep in?”

“Of course.”

“Until April?”

Lester laughed and slipped one of Perry’s arms over his shoulders
and picked up his suitcase and walked them back through the parking
lot to his latest hotrod.

Perry breathed in the hot, wet air as they went, feeling it open
his chest and nasal passages. His eyes were at half mast, but the
sight of the sickly roadside palms, the wandering vendors on the
traffic islands with their net bags full of ipods and vpods\dash{}he was
home, and his body knew it.

Lester cooked him a huge plate of scrambled eggs with corned beef,
pastrami, salami and cheese, with a mountain of sauerkraut on top.
“There you go, fatten you up. You’re all skinny and haggard,
buddy.” Lester was an expert at throwing together high-calorie
meals on short order.

Perry stuffed away as much as he could, then collapsed on his old
bed with his old sheets and his old pillows, and in seconds he was
sleeping the best sleep he’d had in months.

When he woke the next day, his cold had turned into a horrible,
wet, crusty thing that practically had his face glued to his
pillow. Lester came in, took a good look at him, and came back with
a quart of fresh orange juice, a pot of tea, and a stack of dry
toast, along with a pack of cold pills.

“Take \emph{all} of this and then come down to the ride when you’re
ready. I’ll hold down the fort for another couple days if that’s
what it takes.”

Perry spent the day in his bathrobe, shuttling between the living
room and the sun-chairs on the patio, letting the heat bake some of
the snot out of his head. Lester’s kindness and his cold made him
nostalgic for his youth, when his father doted on his illnesses.

Perry’s father was a little man. Perry\dash{}no giant himself\dash{}was taller
than the old man by the time he turned 13. His father had always
reminded him of some clever furry animal, a raccoon or badger. He
had tiny hands and his movements were small and precise and
careful.

They were mostly cordial and friendly, but distant. His father
worked as a CAD/CAM manager in a machine shop, though he’d started
out his career as a plain old machinist. Of all the machinists he’d
started with at the shop, only he had weathered the transition to
the new computerized devices. The others had all lost their jobs or
taken early retirement or just quit, but his father had taken to
CAD/CAM with total abandon, losing himself in the screens and
staggering home bleary after ten or fifteen hours in front of the
screen.

But that all changed when Perry took ill. Perry’s father loved to
play nurse. He’d book off from work and stay home, ferrying up
gallons of tea and beef broth, flat ginger-ale and dry toast, cold
tablets and cough syrup. He’d open the windows when it was warm and
then run around the house shutting them at the first sign of a cool
breeze.

Best of all was what his father would do when Perry got restless:
he and Perry would go down to the living-room, where the upright
piano stood. It had been Perry’s grandfather’s, and the old
man\dash{}who’d died before Perry was born\dash{}had been a jazz pianist who’d
played sessions with everyone from Cab Calloway to Duke Ellington.

“You ready, P?” his father would ask.

Perry always nodded, watching his father sit down at the bench and
try a few notes.

Then his father would play, tinkling and then pounding, running up
and down the keyboard in an improvised jazz recital that could go
for hours, sometimes only ending once Perry’s mom came home from
work at the framing shop.

Nothing in Perry’s life since had the power to capture him the way
his father’s music did. His fingers danced, literally \emph{danced}
on the keys, walking up and down them like a pair of high-kicking
legs, making little comedy movements. The little stubby fingers
with their tufts of hair on the knuckles, like goat’s legs, nimbly
prancing and turning.

And then there was the \emph{music}. Perry sometimes played with
the piano and he’d figured out that if you hit every other key with
three fingers, you got a chord. But Perry’s dad almost never made
chords: he made anti-chords, sounds that involved those mysterious
black keys and clashed in a way that was \emph{precisely} not a
chord, that jangled and jarred.

The anti-chords made up anti-tunes. Somewhere in the music there’d
be one or more melodies, often the stuff that Perry listened to in
his room, but sometimes old jazz and blues standards.

The music would settle into long runs of improvisational noise that
wasn’t \emph{quite} noise. That was the best stuff, because Perry
could never tell if there was a melody in there. Sometimes he’d be
sure that he had the know of it, could tell what was coming next, a
segue into “Here Comes the Sun” or “Let the Good Times Roll” or
“Merrily We Roll Along,” but then his father would get to that spot
and he’d move into something else, some other latent pattern that
was unmistakable in hindsight.

There was a joke his dad liked, “Time flies like an arrow, fruit
flies like a banana.” This was funny in just that way: you expected
one thing, you got something else, and when your expectations fell
apart like that, it was pure hilarity. You wanted to clutch your
sides and roll on the floor sometimes, it was so funny.

His dad usually closed his eyes while he played, squeezing them
shut, letting his mouth hang open slightly. Sometimes he grunted or
scatted along with his playing but more often he grunted out
something that was kind of the \emph{opposite} of what he was
playing, just like sometimes the melody and rhythms he played on
the piano were sometimes the opposite of the song he was playing,
something that was exactly and perfectly opposite, so you couldn’t
hear it without hearing the thing it was the opposite of.

The game would end when his dad began to improvise on parts of the
piano besides the keys, knocking on it, reaching in to pluck its
strings like a harp, rattling Perry’s teacup on its saucer just
so.

Nothing made him feel better faster. It was a tonic, a fine one,
better than pills and tea and toast, daytime TV and flat
ginger-ale.

As Perry got older, he and the old man had their share of fights
over the normal things: girls, partying, school\ldots{} But every time
Perry took ill, he was transported back to his boyhood and those
amazing piano recitals, his father’s stubby fingers doing their
comic high-kicks and pratfalls on the keys, the grunting anti-song
in the back of his throat, those crazy finales with teacups and
piano strings.

Now he stared morosely at the empty swimming pool six stories below
his balcony, filled with blowing garbage, leaves, and a huge wasps’
nest. His father’s music was in his ears, distantly now and fading
with his cold. He should call the old man, back home in Westchester
County, retired now. They talked only rarely these days, three or
four times a year on birthdays and anniversaries. No fight had
started their silence, only busy lives grown apart.

He should call the old man, but instead he got dressed and went for
a jog around the block, trying to get the wet sick wheeze out of
his whistling breath, stopping a couple times to blow his nose. The
sun was like a blowtorch on his hair, which had grown out of his
normal duckling fuzz into something much shaggier. His head baked,
the cold baked with it, and by the time he got home and chugged a
quart of orange juice, he was feeling fully human again and ready
for a shower, street clothes and a turn at the old ticket-window at
work.

The queue snaked all the way through the market and out to the
street, where the line had a casual, party kind of atmosphere. The
market kids were doing a brisk business in popsicles, homemade
colas, and clever origami stools and sun-beds made from recycled
cardboard. Some of the kids recognized him and waved, then returned
to their hustle.

He followed the queue through the stalls. The vendors were happier
than the kids, if that was possible, selling stuff as fast as they
could set it out. The queue had every conceivable kind of person in
it: old and young, hipsters and conservative rawboned southerners,
Latina moms with their babies, stone-faced urban homeboys,
crackers, and Miami Beach queers in pastel shorts. There were old
Jewish couples and smartly turned out European tourists with their
funny two-tone shag cuts and the filter masks that they smoked
around. There was a no-fooling Korean tour group, of the sort he’d
seen now and again in Disney World, led by a smart lady in a
sweltering little suit, holding an umbrella over her head.

“Lester, what the \emph{fuck}?” he said, grinning and laughing as
he clapped Lester on the shoulder, taking a young mall-goth’s five
bucks out of a hand whose fingernails were painted with chipped
black polish. “What the hell is going on here?”

Lester laughed. “I was saving this for a surprise, buddy. Record
crowds\dash{}growing every day. There’s a line up in the morning no
matter how early I open and no matter what time I close, I turn
people away.”

“How’d they all find out about it?”

Lester shrugged. “Word of mouth,” he said. “Best advertising you
can have. Shit, Perry, you just got back from ten cities where they
want to clone this thing\dash{}how did \emph{they} find out about it?”

Perry shook his head and marveled at the queue some more. The
Korean tour group was coming up on them, and Perry nudged Lester
aside and got out his ticket-roll, the familiar movements lovely
after all that time on the road.

The tour guide put a stack of twenties down on the counter. “I got
fifty of ’em,” she said. “That’s two hundred and fifty bucks.” She
had an American accent, somewhere south of the Mason-Dixon line.
Perry had been expecting a Korean accent, broken English.

Perry riffled the bills. “I’ll take your word for it.”

She winked at him. “They got off the plane and they were all like,
’Screw Disney, we have one of those in Seoul, what’s \emph{new},
what’s \emph{American}?’ So I took them here. You guys totally
rock.”

He could have kissed her. His heart took wing. “In you go,” he
said. “Lester will get you the extra ride vehicles.”

“They’re all in there already,” he said. “I’ve been running the
whole fleet for two weeks and I’ve got ten more on order.”

Perry whistled. “You shoulda said,” he said, then turned back to
the tour guide. “It might be a little bit of a wait.”

“Ten, fifteen minutes,” Lester said.

“No \emph{problem},” she said. “They’ll wait till kingdom come,
provided there’s good shopping to be had.” Indeed the tour group
was at the center of a pack of vendor-kids, hawking busts and
tattoos, contacts and action-figures, kitchenware and cigarette
lighters.

Once she was gone, Lester gave his shoulder another squeeze. “I
hired two more kids to bring the ride cars back around to the
entrance.” When Perry had left, that had been a once-daily chore,
something you did before shutting down for the night.

“Holy crap,” Perry said, watching the tour group edge toward the
entrance, slip inside in ones and twos.

“It’s amazing, isn’t it?” Lester said. “And wait till you see the
ride!”

Perry didn’t get a chance to ride until much later that day, once
the sun had set and the last market-stall had been shut and the
last rider had been chased home, when he and Lester slugged back
bottles of flat distilled water from their humidity-still and sat
on the ticket counter to get the weight off their tired feet.

“Now we ride,” Lester said. “You’re going to \emph{love} this.”

The first thing he noticed was that the ride had become a lot less
open. When he’d left, there’d been the sense that you were in a
giant room\dash{}all that dead Wal-Mart\dash{}with little exhibits spread
around it, like the trade-floor at a monster-car show. But now the
exhibits had been arranged out of one another’s sight-lines, and
some of the taller pieces had been upended to form baffles. It was
much more like a carny haunted house trade-show floor now.

The car circled slowly in the first “room,” which had accumulated a
lot of junk that wasn’t mad inventions from the heyday of New Work.
There was a chipped doll-cradle, and a small collection of girls’
dolls, a purse spilled on the floor with photos of young girls
clowning at a birthday party. He reached for the joystick with
irritation and slammed it toward minus one\dash{}what the hell was this
crap?

Next was a room full of boys’ tanks and cars and trading cards,
some in careful packages and frames, some lovingly scuffed and
beaten up. They were from all eras, and he recognized some of his
beloved toys from his own boyhood among the mix. The items were
arranged in concentric rings\dash{}one of the robots’ default patterns
for displaying materials\dash{}around a writhing tower of juddering,
shuddering domestic robots that had piled one atop the other. The
vogue for these had been mercifully brief, but it had been intense,
and for Perry, the juxtaposition of the cars and the cards, the
tanks and the robots made something catch in his throat. There was
a statement here about the drive to automate household chores and
the simple pleasure of rolling an imaginary tank over the imaginary
armies of your imaginary enemies. So, too, something about the
collecting urge, the need to get every card in a set, and then to
get each in perfect condition, and then to arrange them in perfect
order, and then to forget them altogether.

His hand had been jerking the joystick to plus one all this time
and now he became consciously aware of this.

The next room had many of the old inventions he remembered, but
they were arranged not on gleaming silver tables, but were mixed in
with heaps of clothing, mountains of the brightly colored
ubiquitous t-shirts that had gone hand in hand with every New Work
invention and crew. Mixed in among them were some vintage tees from
the dotcom era, and perched on top of the mountain, staring
glassily at him, was a little girl-doll that looked familiar; he
was almost certain that he’d seen her in the first doll room.

The next room was built out of pieces of the old “kitchen” display,
but there was disarray now, dishes in the sink and a plate on the
counter with a cigarette butted out in the middle of it. Another
plate lay in three pieces on the linoleum before it.

The next room was carpeted with flattened soda tins that crunched
under the chair’s wheels. In the center of them, a neat workbench
with ranked tools.

The ride went on and on, each room utterly different from how he’d
left it, but somehow familiar too. The ride he’d left had
celebrated the New Work and the people who’d made it happen, and so
did this ride, but this ride was less linear, less about display
more\dash{}

“It’s a story,” he said when he got off.

“I think so too,” Lester said. “It’s been getting more and more
story-like. The way that doll keeps reappearing. I think that
someone had like ten of them and just tossed them out at regular
intervals and then the plus-oneing snuck one into every scene.”

“It’s got scenes! That’s what they are, scenes. It’s like a Disney
ride, one of those dark rides in Fantasyland.”

“Except those suck and our ride rocks. It’s more like Pirates of
the Caribbean.”

“Have it your way. Whatever, how freaking weird is that?”

“Not so weird. People see stories like they see faces in clouds.
Once we gave them the ability to subtract the stuff that felt wrong
and reinforce the stuff that felt right, it was only natural that
they’d anthropomorphize the world into a story.”

Perry shook his head. “You think?”

“We have this guy, a cultural studies prof, who comes practically
every day. He’s been telling me all about it. Stories are how we
understand the world, and technology is how we choose our stories.

“Check out the Greeks. All those Greek plays, they end with the
\emph{deus ex machina}\dash{}the playwright gets tired of writing, so he
trots a god out on stage to simply point a finger at the players
and make it all better. You can’t do that in a story today, but
back then, they didn’t have the tools to help them observe and
record the world, so as far as they could tell, that’s how stuff
worked!

“Today we understand a little more about the world, so our stories
are about people figuring out what’s causing their troubles and
changing stuff so that those causes go away. Causal stories for a
causal universe. Thinking about the world in terms of causes and
effects makes you seek out causes and effects\dash{}even where there are
none. Watch how gamblers play, that weird cargo-cult feeling that
the roulette wheel came up black a third time in a row so the next
spin will make it red. It’s not superstition, it’s kind of the
\emph{opposite}\dash{}it’s causality run amok.”

“So this is the story that has emerged from our collective
unconscious?”

Lester laughed. “That’s a little pretentious, I think. It’s more
like those Japanese crabs.”

“Which Japanese crabs?”

“Weren’t you there when Tjan was talking about this? Or was that in
Russia? Anyway. There are these crabs in Japan, and if they have
anything that looks like a face on the backs of their shells, the
fishermen throw them back because it’s bad luck to eat a crab with
a face on its shell. So the crabs with face-like shells have more
babies. Which means that gradually, the crabs’ shells get more
face-like, since all non-face-like shells are eliminated from the
gene-pool. This leads the fishermen to raise the bar on their
selection criteria, so they will eat crabs with shells that are a
little face-like, but not \emph{very} face-like. So all the
slightly face-like crab-shells are eliminated, leaving behind
moderately face-like shells. This gets repeated over several
generations, and now you’ve got these crabs that have vivid faces
on their shells.

“We let our riders eliminate all the non-story-like elements from
the ride, and so what’s left behind is more and more story-like.”

“But the plus-one/minus-one lever is too crude for this, right? We
should give them a pointer or something so they can specify
individual elements they don’t like.”

“You want to encourage this?”

“Don’t you?”

Lester nodded vigorously. “Of course I do. I just thought that
you’d be a little less enthusiastic about it, you know, because so
much of the New Work stuff is being de-emphasized.”

“You kidding? This is what the New Work was all about: group
creation! I couldn’t be happier about it. Seriously\dash{}this is so much
cooler than anything that I could have built. And now with the
network coming online soon\dash{}wow. Imagine it. It’s going to be so
fucking weird, bro.”

“Amen,” Lester said. He looked at his watch and yelped. “Shit, late
for a date! Can you get yourself home?”

“Sure,” Perry said. “Brought my wheels. See you later\dash{}have a good
one.”

“She’s amazing,” Lester said. “Used to weigh 900 pounds and was
shut in for ten years. Man has she got an imagination on her. She
can do this thing\dash{}”

Perry put his hands over his ears. “La la la I’m not listening to
you. TMI, Lester. Seriously. Way way TMI.”

Lester shook his head. “You are such a prude, dude.”

Perry thought about Hilda for a fleeting moment, and then grinned.
“That’s me, a total puritan. Go. Be safe.”

“Safe, sound, and slippery,” Lester said, and got in his car.

Perry looked around at the shuttered market, rooftops glinting in
the rosy tropical sunset. Man he’d missed those sunsets. He snorted
up damp lungsful of the tropical air and smelled dinners cooking at
the shantytown across the street. It was different and bigger and
more elaborate every time he visited it, which was always less
often than he wished.

There was a good barbecue place there, Dirty Max’s, just a hole in
the wall with a pit out back and the friendliest people. There was
always a mob scene around there, locals greasy from the ribs in
their hands, a big bucket overflowing with discarded bones.

Wandering towards it, he was amazed by how much bigger it had grown
since his last visit. Most buildings had had two stories, though a
few had three. Now almost all had four, leaning drunkenly toward
each other across the streets. Power cables, network cables and
clotheslines gave the overhead spaces the look of a carelessly spun
spider’s web. The new stories were most remarkable because of what
Francis had explained to him about the way that additional stories
got added: most people rented out or sold the right to build on top
of their buildings, and then the new upstairs neighbors in turn
sold \emph{their} rights on. Sometimes you’d get a third-storey
dweller who’d want to build atop two adjacent buildings to make an
extra-wide apartment for a big family, and that required
negotiating with all of the “owners” of each floor of both
buildings.

Just looking at it made his head hurt with all the tangled property
and ownership relationships embodied in the high spaces. He heard
the easy chatter out the open windows and music and crying babies.
Kids ran through the streets, laughing and chasing each other or
bouncing balls or playing some kind of networked RPG with their
phones that had them peeking around corners, seeing another player
and shrieking and running off.

The grill-woman at the barbecue joint greeted him by name and the
men and women around it made space for him. It was friendly and
companionable, and after a moment Francis wandered up with a couple
of his proteges. They carried boxes of beer.

“Hey hey,” Francis said. “Home again, huh?”

“Home again,” Perry said. He wiped rib-sauce off his fingers and
shook Francis’s hand warmly. “God, I’ve missed this place.”

“We missed having you,” Francis said. “Big crowds across the way,
too. Seems like you hit on something.”

Perry shook his head and smiled and ate his ribs. “What’s the story
around here?”

“Lots and lots,” Francis said. “There’s a whole net-community thing
happening. Lots of traffic on the AARP message-boards from other
people setting these up around the country.”

“So you’ve hit on something, too.”

“Naw. When it’s railroading time, you get railroads. When it’s
squatter time, you get squats. You know they want to open a
7-Eleven here?”

“No!” Perry laughed and choked on ribs and then guzzled some beer
to wash it all down.

Francis put a wrinkled hand over his heart. He still wore his
wedding band, Perry saw, despite his wife’s being gone for decades.
“I swear it. Just there.” He pointed to one of the busier corners.

“And?”

“We told them to fuck off,” Francis said. “We’ve got lots of
community-owned businesses around here that do everything a
7-Eleven could do for us, without taking the wealth out of our
community and sending it to some corporate jack-off. Some soreheads
wanted to see how much money we could get out of them, but I just
kept telling them\dash{}whatever 7-Eleven gives us, it’ll only be because
they think they can get more \emph{out of us}. They saw reason.
Besides, I’m in charge\dash{}I always win my arguments.”

“You are the most benevolent of dictators,” Perry said. He began to
work on another beer. Beer tasted better outside in the heat and
the barbecue smoke.

“I’m glad someone thinks so,” Francis said.

“Oh?”

“The 7-Eleven thing left a lot of people pissed at me. There’s
plenty around here that don’t remember the way it started off. To
them, I’m just some alter kocker who’s keeping them down.”

“Is it serious?” Perry knew that there was the potential for
serious, major lawlessness from his little settlement. It wasn’t a
failing condo complex rented out to Filipina domestics and weird
entrepreneurs like him. It was a place where the cops would love an
excuse to come in with riot batons (his funny eyebrow twitched) and
gas, the kind of place where there almost certainly were a few very
bad people living their lives. Miami had bad people, too, but the
bad people in Miami weren’t his problem.

And the bad people and the potential chaos were what he loved about
the place, too. He’d grown up in the kind of place where everything
was predictable and safe and he’d hated every minute of it. The
glorious chaos around him was just as he liked it. The wood-smoke
curled up his nose, fragrant and all-consuming.

“I don’t know anymore. I thought I’d retire and settle down and
take up painting. Now I’m basically a mob boss. Not the bad kind,
but still. It’s a lot of work.”

“Pimpin’ ain’t easy.” Perry saw the shocked look on Francis’s face
and added hastily, “Sorry\dash{}not calling you a pimp. It’s a song lyric
is all.”

“We got pimps here now. Whores, too. You name it, we got it. It’s
still a good place to live\dash{}better than Miami, if you ask me\dash{}but it
could go real animal. Bad, bad animal.”

Hard to believe, standing there in the wood-smoke, licking his
fingers, drinking his beer. His cold seemed to have been baked out
by the steamy swampy heat.

“Well, Francis, if anyone knows how to keep peace, it’s you.”

“Social workers come around, say the same thing. But there’s people
around here with little kids, they worry that the social workers
could force them out, take away their children.”

It wasn’t like Francis to complain like this, it wasn’t in his
nature, but here it was. The strain of running things was showing
on him. Perry wondered if his own strain was showing that way. Did
he complain more these days? Maybe he did.

An uncomfortable silence descended upon them. Perry drank his beer,
morosely. He thought of how ridiculous it was to be morose about
the possibility that he was being morose, but there you had it.

Finally his phone rang and saved him from further conversation. He
looked at the display and shook his head. It was Kettlewell again.
That first voicemail had made him laugh aloud, but when they hadn’t
called back for a couple days, he’d figured that they had just had
a little too much wine and placed the call.

Now they were calling back, and it was still pretty early on the
West Coast. Too early for them to have had too much wine, unless
they’d really changed.

“Perry Perry Perry!” It was Kettlebelly. He sounded like he might
be drunk, or merely punch-drunk with excitement. Perry remembered
that he got that way sometimes.

“Kettlewell, how are you doing?”

“I’m here too, Perry. I cashed in my return ticket.”

“Suzanne?”

“Yeah,” she said. She too sounded punchy, like they’d been having a
fit of the giggles just before calling. “Kettlewell’s family have
taken me in, wayward wanderer that I am.”

“You two sound pretty, um, happy.”

“We’ve been having an amazing time,” Kettlewell said. His
speakerphone made him sound like he was at the bottom of a well.
“Mostly reminiscing about you guys. What the hell are you up to? We
tried to follow it on the net, but it’s all jumbled. What’s this
about a \emph{story}?”

“Story?”

“I keep reading about this ride of yours and its story. I couldn’t
make any sense of it.”

“I haven’t read any of this, but Lester and I were talking about
some stuff to do with stories tonight. I didn’t know anyone else
was talking about this, though. Where’d you see it?”

“I’ll email it to you,” Suzanne said. “I was going to blog it
tonight anyway.”

“So you two are just hanging around San Francisco giggling and
walking down memory lane?”

“Well, yeah! It’s about time, too. We’ve all been separated for too
long. We want a reunion, Perry.”

“A reunion?”

“We want to come down for a visit and see what you’re doing and
hang out. You wouldn’t believe how much fun we’ve been having,
Perry, seriously.” Kettlewell sounded like he’d been huffing
nitrous or something. “Have \emph{you} been having fun?”

He thought about the question. “Um, kind of?” He told them about
his travels, a quick thumbnail sketch, struggling to remember which
city he’d been to when, leaving out the crazy sex\dash{}which came back
to him in a rush, that night with Hilda in the coffin, like a warm
hallucination. “On balance, yes. It’s been fun.”

“Right, so we want to come down and have fun with you and Lester.
He’s still hanging around, right?”

Lester had told him about the history he had with Suzanne, and
there was something in the way she asked after Lester that
suggested to Perry that there was still something there.

“You kidding? You’d have to pry us apart with a crowbar.”

“See, I told you so,” Suzanne said. “This guy thought that Lester
might have gotten bored and wandered off.”

“Never! Plus anyone who follows his message board traffic and blogs
would know that he was right here, minding the shop.” And you’re
reading his blog, aren’t you, Suzanne? He didn’t need to say it. He
could almost hear her blush over the line.

“So how about tomorrow?”

“For what?”

“For us coming to town. I’ll bring the wife and kids. We’ll rent
out a couple hotel rooms and spend a week there. It’ll be a
blast.”

“Tomorrow?”

“We could get the morning flight and be there for breakfast. You
got a good hotel? Not a coffin hotel, not with the kids.”

Perry’s heart beat faster. He did miss these two, and they were so
punchy, so gleeful. He’d love to see them. He muted his phone.

“Hey, Francis? That guesthouse down the road, is it still
running?”

“Lulu’s? Sure. They just built another storey and took over the top
floor of the place next door.”

“Perfect.” He unmuted. “How’d you like to stay in a squatter
guesthouse in the shantytown?”

“Um,” Kettlewell said, but Suzanne laughed.

“Oh hell \emph{yes},” she said. “Get that look off your face,
Kettlewell, this is an \emph{adventure}.”

“We’d love it,” Kettlewell said.

“Great, I’ll make you a reservation. How long are you staying?”

“Until we leave,” Suzanne said.

“Right,” Perry said and laughed himself. They were different
people, these two, from the people he remembered, but they were
also old friends. And they were coming to see him tomorrow. “OK,
lemme go make your reservations.”

Francis walked him over and the landlord fussed over the two of
them like they were visiting dignitaries. Perry looked the place
over and it was completely charming. He spotted what he thought was
probably a hooker and a trick taking a room for the night, but you
got that at the Hilton, too.

By the time he got home he was sure that he’d sleep like a log. He
could barely keep his eyes open on the drive. But after he climbed
into bed and closed his eyes, he found that he couldn’t sleep at
all. Something about being back in his own room in his own bed felt
alien and exciting. He got up and paced the apartment and then
Lester came home from his date with the fatkins nympho, full of
improbable stories and covered in little hickeys.

“You won’t believe who’s coming for a visit,” Perry said.

“Steve Jobs. He’s come down from the lamasery and renounced
Buddhism. He wants to give a free computer to every visitor.”

“Close,” Perry said. “Kettlebelly and Suzanne Church. Coming
\emph{tomorrow} for a stay of unspecified duration. It’s a reunion.
It’s a \emph{reunion} you big sonofabitch! Woot! Woot!” Perry did a
little two-step. “A reunion!”

Lester looked confused for a second, and then for another second he
looked, what, upset? and then he was grinning and jumping up and
down with Perry. “Reunion!”

He felt like he’d barely gotten to sleep when his phone rang. The
clock showed six AM, and it was Kettlebelly and Suzanne, bleary,
jet-lagged and grouchy from their one-hour post-flight security
processing.

“We want breakfast,” Suzanne said.

“We’ve gotta open the ride, Suzanne.”

“At six in the morning? Come on, you’ve got hours yet before you
have to be at work. How about you and Lester meet us at the IHOP?”

“Jesus,” he said.

“Come \emph{on}! Kettlebelly’s kids are dying for something to eat
and his wife looks like she’s ready to eat \emph{him}. It’s been
years, dude! Get your ass in the shower and down to the
International House of Pancakes!”

Lester didn’t rouse easy, but Perry knew all the tricks for getting
his old pal out of bed, they were practically married after all.

They arrived just in time for the morning rush but Tony greeted
them with a smile and sent them straight to the front of the line.
Lester ordered his usual (“Bring me three pounds of candy with a
side of ground animal parts and potatoes”) and they waited
nervously for Suzanne and the clan Kettlewell to turn up.

They arrived in a huge bustle of taxis and luggage and two
wide-eyed, jet-lagged children hanging off of Kettlewell and Mrs
Kettlewell, whom neither of them had ever met. She was a small,
youthful woman in her mid-forties with artfully styled hair and
big, abstract chunky silver jewelry. Suzanne had gone all Eurochic,
rail-thin and smoking, with quiet, understated dark clothes.
Kettlewell had a real daddy belly on him now, a little pot that his
daughter thumped rhythmically from her perch on his hip.

“Sit, sit,” Perry said to them, getting up to help them stack their
luggage at either end of the long table down the middle of the
IHOP. Big family groups with tons of luggage were par for the
course in Florida, so they didn’t really draw much attention beyond
mild irritation from the patrons they jostled as they got everyone
seated.

Perry was mildly amused to see that Lester and Suzanne ended up
sitting next to one another and were already chatting avidly and
close up, in soft voices that they had to lean in very tight to
hear.

He was next to Mrs Kettlewell, whose name, it transpired, was
Eva\dash{}“As in Extra-Vehicular Activity,” she said, geeking out with
him. Kettlewell was in the bathroom with his daughter and son, and
Mrs Kettlewell\dash{}Eva\dash{}seemed relieved at the chance for a little adult
conversation.

“You must be a very patient woman,” Perry said, laughing at all the
ticklish noise and motion of their group.

“Oh, that’s me all right,” Eva said. “Patience is my virtue. And
you?”

“Oh, patience is something I value very much in other people.”
Perry said. It made Eva laugh, which showed off her pretty
laugh-lines and dimples. He could see how this woman and Kettlewell
must complement each other.

She rocked her head from side to side and took a long swig of the
coffee that their waiter had distributed around the table, topping
up from the carafe he’d left behind. “Thank God for legal
stimulants.”

“Long flight?”

“Traveling with larvae is always a challenge,” she said. “But they
dug it hard. You should have seen them at the windows.”

“They’d never been on a plane before?”

“I like to go camping,” she said with a shrug. “Landon’s always on
me to take the kids to Hawaii or whatever, but I’m always like,
‘Man, you spend half your fucking life in a tin can\dash{}why do you want
to start your holidays in one? Let’s go to Yosemite and get muddy.’
I haven’t even taken them to Disneyland!”

Perry put the back of his hand to his forehead. “That’s heresy
around here,” he said. “You going to take them to Disney World
while you’re in Florida? It’s a lot bigger, you know\dash{}and it’s a
different division. Really different feel, or so I’m told.”

“You kidding? Perry, we came here for \emph{your} ride. It’s
famous, you know.”

“Net.famous, maybe. A little.” He felt his cheeks burning. “Well,
there will be one in your neck of the woods soon enough.” He told
her about the Burning Man collective and the plan to build one down
the 101, south of San Francisco International.

Kettlebelly returned then with the kids, and he managed to get them
into their seats while sucking back a coffee and eating a biscuit
from the basket in the center of the table, breaking off bits to
shove in the kids’ mouths whenever they protested.

“These are some way tired kids,” he said, leaning over to give his
wife a kiss. Perry thought he saw Suzanne flick a look at them
then, but it might have been his imagination. Suzanne and Lester
were off in their own world, after all.

“The plane almost crashed,” said the little girl next to Perry. She
had a halo of curly hair like a dandelion clock and big solemn dark
eyes and a big wet mouth set between apple-round cheeks.

“Did it really?” Perry said. She was seven or eight he thought, the
bossy big sister who’d been giving orders to her little brother
from the moment they came through the door.

She nodded solemnly. He looked at Eva, who shrugged.

“Really?” he said.

“Really,” she said, nodding vigorously now. “There were terrists on
the plane who wanted to blow it up, but the sky marshas stopped
them.”

“How could you tell they were ’terrists’?”

She clicked her tongue and rolled her eyes. “They were
\emph{whispering},” she said. “Just like on
\emph{Captain President and the Freedom Fighters}.” He knew
something of this cartoon, mostly because of all the knock-off
merch for sale in the market stalls in front of the ride.

“I see,” he said. “Well, I’m glad the Sky Marshas stopped them. Do
you want pancakes?”

“I want caramel apple chocolate pancakes with blueberry banana
sauce,” she said, rolling one pudgy finger along the description in
the glossy menu, beneath an oozing food-porn photo. “And my brother
wants a chocolate milkshake and a short stack of happy face clown
waffles with strawberry sauce, but not too many because he’s still
a baby and can’t eat much.”

“You’ll become as fat as your daddy if you eat like that,” Perry
said. Eva snorted beside him.

“No,” she said. “I’m gonna be a fatkins.”

“I see,” he said. Eva shook her head.

“It’s the goddamned fatkins agitprop games,” Eva said. “They come
free with everything now\dash{}digital cameras, phones, even in cereal
boxes. You have to eat a minimum number of calories per level or
you starve to death. This one is a champeen.”

“I’m nationally ranked,” the little girl said, not looking up from
the menu.

Perry looked across the table and discovered that Suzanne had
covered Lester’s hand with hers and that Lester was laughing along
with her at something funny. Something about that made him a little
freaked out, like Lester was making time with his sister or their
mom.

“Suzanne,” he said. “What’s happening with you these days,
anyway?”

“Petersburg is what’s happening with me,” she said, with a hoarse
little chuckle. “Petersburg is like Detroit crossed with Paris.
Completely decrepit and decadent. There’s a serial killer who’s
been working the streets for five years there and the biggest
obstacle to catching him is that the first cops on the scene let
rubberneckers bribe them to take home evidence as souvenirs.”

“No way!” Lester said.

“Oh, da, big vay,” she said, dropping into a comical Boris and
Natasha accent. “Bolshoi vay.”

“So why are you there?”

“It’s like home for me. It’s got enough of Detroit’s old brutal,
earthy feel, plus enough of Silicon Valley’s manic hustle, it just
feels right.”

“You going to settle in there?”

“Well, put that way, no. I couldn’t hack it for the long term. But
at this time in my life, it’s been just right. But it’s good to get
back to the States, too. I’m thinking of hanging out here for a
couple months. Russia’s so cheap, I’ve got a ton saved up. Might as
well blow it before inflation kills it.”

“You keep your money in rubles?”

“Hell no\dash{}no one uses rubles except tourists. I’m worried about
another run of \emph{US} inflation. I mean, have you looked around
lately? You’re living in a third world country, buddy.”

A waiter came between them, handing out heaping, steaming plates of
food. Lester, who’d finished his first breakfast while they waited,
had ordered a second breakfast, which arrived along with the rest
of them. Mountains of food stacked up on the table, side-plates
crowding jugs of apple juice and carafes of coffee.

Incredibly, the food kept coming\dash{}multiple syrup-jugs, plates of
hash-browns, baskets of biscuits and bowls of white sausage gravy.
Perry hadn’t paid much attention when orders were being taken, but
from the looks of things, he was eating with a bunch of IHOP
virgins, unaccustomed to the astonishing portions to be had there.

He cocked his funny eyebrow at Suzanne, who laughed. “OK, not quite
a third-world country. But not a real industrial nation anymore,
either. Maybe more like the end-days of Rome or something. Drowning
in wealth and wallowing in poverty.” She forked up a mouthful of
hash browns and chased them with coffee. Perry attacked his own
plate.

Kettlewell fed the kids, sneaking bites in-between, while Eva
looked on approvingly. “You’re a good man, Landon Kettlewell,” she
said, slicing up her steak and eggs into small, precise cubes,
wielding the knife like an artist.

“You just enjoy your breakfast, my queen,” he said, spooning
oatmeal with raisins, bananas, granola and boysenberry jam into the
little boy’s mouth.

“We got you presents,” the little girl said, taking a break from
shoveling banana-chocolate caramel apples into her mouth.

“Really?” Perry raised his funny eyebrow and she giggled. He did it
again, making it writhe like a snake. She snarfed choco-banana
across the table, then scooped it up and put it back in her mouth.

She nodded vigorously. “Dad, give them their presents!”

Kettlewell said, “Someone has to feed your brother, you know.”

“I’ll do it,” she said. She forked up some of his oatmeal and
attempted to get it into the little boy’s face. “Presents!”

Kettlewell dug through the luggage-cluster under the table and came
up with an overstuffed diaper bag, then pawed through it for a long
time, urged on by his daughter who kept chanting “Presents!
Presents! Presents!” while attempting to feed her little brother.
Eva and Lester and Suzanne took up the chant. They were drawing
stares from nearby tables, but Perry didn’t mind. He was laughing
so hard his sides hurt.

Finally Kettlewell held a paper bag aloft triumphantly, then
clapped a hand over his daughter’s mouth and shushed the rest.

“You guys are really hard to shop for,” he said. “What the hell do
you get for two guys who not only have everything, but \emph{make}
everything?”

Suzanne nodded. “Damned right. We spent a whole day looking for
something.”

“What is it?”

“Well,” Kettlewell said. “We figured that it should be something
useful, not decorative. You guys have decorative coming out of your
asses. So that left us with tools. We wanted to find you a tool
that you didn’t have, and that you would appreciate.”

Suzanne picked up the story. “I thought we should get you an
antique tool, something so well-made that it was still usable. But
to be useful, it had to be something no one had improved on, and
that had in fact been degraded by modern manufacturing techniques.

“At first we looked at old tape-measures, but I remembered that you
guys were mostly using keychain laser range-finders these days.
Screwdrivers, pliers, and hammers were all out\dash{}I couldn’t find a
damned thing that looked any better than what you had around here.
The state of the art is genuinely progressing.

“There were a lot of nice old brass spirit-levels and hand-lathed
plumb-bobs but they were more decorative than useful by a damned
sight. Great old steel work-helmets looked cool, but they weighed
about a hundred times what the safety helmets around here weigh.

“We were going to give in and try to bring you guys a big goddamned
tube-amp, or maybe some Inuit glass knives, but I didn’t see you
having much of a use for either.

“Which is how we came to give up on tools per se and switched over
to leisure\dash{}sports tools. There was a much richer vein. Wooden bats,
oh yes, and real pigskin footballs that had nice idiosyncratic spin
that you’d have to learn to compensate for. But when we found
these, we knew we’d hit pay-dirt.”

She picked up Kettlewell’s paper sack with a flourish and unzipped
it. A moment later she presented them with two identical packages
wrapped in coarse linen paper hand-stamped with Victorian woodcuts
of sporting men swinging bats and charging the line with pigskins
under their arms.

“Ta-dah!”

The kids echoed it. “These are the best presents,” the little girl
confided in Perry as he picked delicately at the exquisite paper.

The paper gave way in folds and curls, and then he and Lester both
held their treasures aloft.

“Baseball gloves!” Perry said.

“A catcher’s mitt and a fielder’s glove,” Kettlewell said. “You
look at that catcher’s mitt. 1910!” It was black and bulbous, the
leather soft and yielding, with a patina of fine cracks like an old
painting. It smelled like oil and leather, an old rich smell like a
gentleman’s club or an expensive briefcase. Perry tried it on and
it molded itself to his hand, snug and comfortable. It practically
cried out to have a ball thrown at it.

“And this fielder’s glove,” Kettlewell went on, pointing at the
glove Lester held. It was the more traditional tan color, comically
large like the glove of a cartoon character. It too had the look of
ancient, well-loved leather, the same mysterious smell of hide and
oil. Perry touched it with a finger and it felt like a woman’s
cheek, smooth and soft. “Rawlings XPG6. The Mickey Mantle. Early
1960s\dash{}the ultimate glove.”

“You got the whole sales pitch, huh, darling?” Eva said, not
unkindly, but Kettlewell flushed and glared at her for a moment.

Perry broke in. “Guys, these are\dash{}wow. Incredible.”

“They’re better than the modern product,” Suzanne said. “That’s the
point. You can’t print these or fab these. They’re wonderful
because they’re so well made \emph{and so well-used}! The only way
to make a glove this good would be to fab it and then give it to
several generations of baseball players to love and use for fifty
to a hundred years.”

Perry turned over the catcher’s mitt. Over a hundred years old.
This wasn’t something to go in a glass case. Suzanne was right:
this was a great glove because people had played with it, all the
time. It needed to be played with or it would get out of practice.

“I guess we’re going to have to buy a baseball,” Perry said.

The little girl beside him started bouncing up and down.

“Show him,” Suzanne said, and the girl dove under the table and
came up with two white, fresh hard balls. Once he fitted one to the
pocket of his glove, it felt so perfectly right\dash{}like a key in a
lock. This pocket had held a lot of balls over the years.

Lester had put a ball in the pocket of his glove, too. He tossed it
lightly in the air and caught it, then repeated the trick. The look
of visceral satisfaction on his face was unmistakable.

“These are \emph{great} presents, guys,” Perry said. “Seriously.
Well done.”

They all beamed and murmured and then the ball Lester was tossing
crashed to the table and broke a pitcher of blueberry syrup, upset
a carafe of orange juice, and rolled to a stop in the chocolate
mess in front of the little girl, who laughed and laughed and
laughed.

“And \emph{that} is why we don’t play with balls indoors,” Suzanne
said, looking as stern as she could while obviously trying very
hard not to bust out laughing.

The waiters were accustomed to wiping up spills and Lester was
awkwardly helpful. While they were getting everything set to rights
again, Perry looked at Eva and saw her lips tightly pursed as she
considered her husband. He followed Kettlebelly’s gaze and saw that
he was watching Suzanne (who was laughingly restraining Lester from
doing any more “cleaning”) intently. In a flash, Perry thought he
had come to understanding. Oh dear, he thought.

The kids loved the shantytown. The little girl\dash{}Ada, “like the
programming language,” Eva said\dash{}insisted on being set down so she
could tread the cracked cement walkways herself, head whipping back
and forth to take the crazy-leaning buildings in, eyes following
the zipping motor-bikes and bicycles as they wove in and out of the
busy streets. The shantytowners were used to tourists in their
midst. A few yardies gave them the hairy eyeball, but then they saw
Perry was along and they found something else to pay attention to.
That made Perry feel obscurely proud. He’d been absent for months,
but even the corner boys knew who he was and didn’t want to screw
with him.

The guesthouse’s landlady greeted them at the door, alerted to
their coming by the jungle telegraph. She shook Perry’s hand
warmly, gave Ada a lollipop, and chucked the little boy (Pascal,
“like the programming language,” said Eva, with an eye-roll) under
the chin. Check-in was a lot simpler than at a coffin-hotel or a
Hilton: just a brief discussion of the available rooms and a quick
tour. The Kettlewells opted for the lofty attic, which could fit
two three-quarter width beds and a crib, and overlooked the curving
streets from a high vantage; Suzanne took a more quotidian room
just below, with lovely tile mosaics made from snipped-out sections
of plastic fruit and smashed novelty soda bottles. (The landlady
privately assured Perry that her euphemistic “hourly trade” was in
a different part of the guesthouse altogether, with its own
staircase).

A few hours later, Perry was alone again, working his ticket
counter. The Kettlewells were having naps, Lester and Suzanne had
gone off to see some sights, and the crowd for the ride was already
large, snaking through the market, thick with vendors and hustling
kids trying to pry the visitors loose of their bankrolls.

He felt like doing a carny barker spiel,
\emph{Step right up, step right up, this way to the great egress!}
But the morning’s visitors didn’t seem all that frivolous\dash{}they were
serious-faced and sober.

“Everything OK?” he asked a girl who was riding for at least the
second time. She was a midwestern-looking giantess in her early
twenties with big white front teeth and broad shoulders, wearing a
faded Hoosiers ball-cap and a lot of coral jewelry. “I mean, you
don’t look like you’re having a fun time.”

“It’s the story,” she said. “I read about it online and I didn’t
really believe it, but now I totally see it. But you made it,
right? It didn’t just\ldots{} \emph{happen}, did it?”

“No, it just happened,” Perry said. This girl was a little
spooky-looking. He put his hand over his heart. “On my honor.”

“It can’t be,” she said. “I mean, the story is like
\emph{right there}. Someone must have made it.”

“Maybe they did,” Perry said. “Maybe a bunch of people thought it
would be fun to make a story out of the ride and came by to do
it.”

“That’s probably it,” the girl said. “The other thing, that’s just
ridiculous.”

She was gone and on the ride before he could ask her what this
meant, and the three bangbangers behind her just wanted tickets,
not conversation.

An hour later, she was back.

“I mean the message boards,” she said. “Don’t you follow your
referers? There’s a guy in Osceola who says that this is, I don’t
know, like the story that’s inside our collective unconsciousness.”
Perry restrained a smile at the malapropism. “Anyway a lot of
people agree. I don’t think so, though. No offense, mister, but I
think that this is just a prank or something.”

“Something,” Perry said. But she rode twice more that day, and she
wasn’t the only one. It was a day of many repeat riders, and the
market-stall people came by to complain that the visitors weren’t
buying much besides the occasional ice-cream or pork cracklin.

Perry shrugged and told them to find something that these people
wanted to buy, then. One or two of the miniatures guys got gleams
in their eyes and bought tickets for the ride (Perry charged them
half price) and Perry knew that by the time the day was out,
there’d be souvenir ride-replicas to be had.

Lester and Suzanne came by after lunchtime and Lester relieved him,
leaving him to escort Suzanne back to the shantytown and the
Kettlewells.

“You two seem to be getting on well,” Perry said, jerking his head
back at Lester as they walked through the market.

Suzanne looked away. “This is amazing, Perry,” she said, waving her
hand at the market stalls, a gesture that took in the spires of the
shantytown and the ride, too. “You have done
something \ldots{} stupendous, you know it? I mean, if you had a slightly
different temperament, I’d call this a cult. But it seems like
you’re not in charge of anything\dash{}”

“That’s for sure!”

“\dash{}even though you’re still definitely \emph{leading} things.”

“No way\dash{}I just go where I’m told. Tjan’s leading.”

“I spoke to Tjan before we came out, and he points the finger at
you. ‘I’m just keeping the books and closing the contracts.’ That’s
a direct quote.”

“Well maybe no one’s leading. Not everything needs a leader,
right?”

Suzanne shook her head at him. “There’s a leader, sweetie, and it’s
you. Have a look around. Last I checked, there were three more
rides going operational this week, and five more in the next month.
Just looking at your speaking calendar gave me a headache\dash{}”

“I have a speaking calendar?”

“You do indeed, and it’s a busy one. You knew that though, right?”

Tjan sent him email all the time telling him about this group or
that, where he was supposed to go and give a talk, but he’d never
seen a calendar. But who had time to look at the website anymore?

“I suppose. I knew I was supposed to get on a plane again in a
couple weeks.”

“So that’s what a leader is\dash{}someone who gets people mobilized and
moving.”

“I met a girl in Madison, Wisconsin, you’d probably get along
with.” Thinking of Hilda made him smile and feel a little horny, a
little wistful. He hadn’t gotten fucked in mind and body like that
since his twenties.

“Maybe I’ll meet her. Is she working on a local ride?”

“You’re going to go to the other rides?”

“I got to write about something, Perry. Otherwise my pageviews fall
off and I can’t pay my rent. This is a story\dash{}a big one, and no one
else has noticed it yet. That kind of story can turn into the kind
of money you buy a house with. I’m speaking from experience here.”

“You think?”

She put her hand over her heart. “I’m good at spotting these. Man,
you’ve got a cult on your hands here.”

“What?”

“The story people. I’ve been reading the message boards and blogs.
It’s where I get all my best tips.”

Perry shook his head. Everyone else was more on top of this stuff
than him. He was going to have to spend less time hacking the ride
and more time reading the interweb, clearly.

“It was all Lester’s idea, anyway,” he said.

She looked down with an unreadable expression. He hazarded a guess
as to what that was about.

“Things are getting tight between you two, huh?”

“Christ it doesn’t show that much does it?”

“No,” he lied. “I just know Lester is all.”

“He’s something else,” she said.

Suzanne needed some sundries, so he directed her to a little bodega
in the back room of one of the houses. He told her he’d meet her at
the guesthouse and took a seat in the lobby. He was still beat from
the cold and the jet-lag, the work and the sheer exhaustion.

On the road he’d had momentum dragging him from one thing to the
next, flights to catch, speeches to make. Back at home, confronted
with routine, it was like his inertia was disappearing.

Eva Kettlewell thundered down the stairs three at a time with a
sound like a barely controlled fall, burst into the lobby and
headed for the door, her back rigid, her arms swinging, her face a
picture of rage.

She went out the door like a flash and then stood in the street for
a moment before striking out, seemingly at random.

Uh-oh, Perry thought.

\begin{center}\rule{3in}{0.4pt}\end{center}

Sammy didn’t dare go back to the ride for weeks after the debacle
in Boston. He’d been spotted by the Chinese guy and the
bummy-looking guy who said he’d designed the ride, that much was
sure. They probably suspected him of having sabotaged the Boston
ride.

But he couldn’t stay away. Work was dismal. The other execs at
Disney World were all amazingly petty, and always worse before the
quarterly numbers came out. Management liked to chase any kind of
bad numbers with a few ritual beheadings.

The new Fantasyland had been a feather in Sammy’s cap that had kept
him safe from politics for a long time, but not anymore. Now it was
getting run down: cigarette burns, graffiti, and every now and
again someone would find a couple having pervy eyeliner sex in the
bushes.

He’d loved to work openings in Fantasyland’s heyday. He’d stand
just past the castle-gate and watch the flocking crowds of
black-clad, lightly sweating, white-faced goth kids pour through
it, blinking in the unnatural light of the morning. A lot of them
took drugs and partied all night and then capped it off with an
early morning at Fantasyland\dash{}Disney had done focus groups, and
they’d started selling the chewy things that soothed the clenched
jaws brought on by dance-drugs.

But now he hated the raven-garbed customers who sallied into his
park like they owned the joint. A girl\dash{}maybe 16\dash{}walked past on
vinyl platform heels with two gigantic men in their thirties behind
her, led on thin black leather leashes. A group of whippet-thin
boys in grey dusters with impossibly high sprays of teased electric
blue hair followed. Then a group of heavily pierced older women,
their faces rattling.

Then it was a river of black, kids in chains and leather, leathery
grownups who dressed like surly kids. They formed neat queues by
their favorite rides\dash{}the haunted houses, the graveyard
walk-through, the coffin coaster, the river of blood\dash{}and puffed
cloves through smokeless hookahs. At least he hoped it was cloves.

The castmembers in Sammy’s Fantasyland were no better than the
guests. They were pierced, dyed, teased, and branded to within an
inch of their lives, even gothier than the goths who made the long
pilgrimages to ride his unwholesome rides.

The worst of it was that there weren’t \emph{enough} of them
anymore. The goth scene, which had shown every sign of surging and
re-surging every five years, seemed finally to be dying. Numbers
were down. A couple of goth-themed parks in the area had shuttered,
as had the marshy one in New Orleans (admittedly that might have
been more to do with the cholera outbreak).

Last month, he’d shut down the goth toddler-clothing shop and put
its wares on deep online discount. All his little nieces and
nephews were getting bat-wing onesies, skull platform-booties and
temporary hair-dye and tattoos for Christmas. Now he just had to
get rid of the other ten million bucks’ worth of merch.

“Morning, Death,” he said. The kid’s real name was Darren
Weinberger, but he insisted on being called Death Waits, which
given his pudgy round cheeks and generally eager-to-please
demeanor, was funny enough that it had taken Sammy a full year to
learn to control his grin when he said it.

“Sammy! Good morning\dash{}how’re you doing?”

“The numbers stink,” Sammy said. “You must have noticed.”

Death’s grin vanished. “I noticed. Time for a new ride, maybe.” No
one called them “attractions” anymore\dash{}all that old Orwellian
Disneyspeak had been abolished. “They love the coaster and the
free-fall. Thrill rides are always crowd-pleasers.”

Death Waits had worked at Disney for three years now, since the age
of 16, and he had grown up coming to the park, one of the rare
Orlando locals. Sammy had come to rely on him for what he thought
of as insight into the “goth street.” He never said that aloud,
because he knew how much it sounded like “whatever you crazy kids
are into these days.”

But this wasn’t helpful. “I \emph{know} that everyone likes thrill
rides, but how the hell can you compete with the gypsy coasters?”
They set up their coasters by the road and ran them until there was
an injury serious enough to draw the law\dash{}a week or two at best. You
could order the DIY coaster kits from a number of suppliers across
the US and Mexico, put them up with cranes and semi-skilled labor
and wishful thinking, start taking tickets, and when the inevitable
catastrophe ensued, you could be packed and on the lam in a couple
hours.

“Gypsy coasters? They suck. We’ve got theming. Our rides are
\emph{art}. That stuff is just \emph{engineering}.” Death Waits was
a good kid, but he was a serious imbiber of the kool-aid. “Maybe
try dance parties again?” They’d tried a string of all-night raves,
but the fights, drugs, and sex were just too much for the upper
management, no matter how much money they brought in.

Sammy shook his head morosely. “I’ve told you that a company this
size can’t afford the risks from that sort of thing.” A few more
goths straggled in. They headed for the walk-through, which
probably meant they planned to get high or make out, something he’d
given up on trying to prevent. Anything to get the numbers up. He
and the security staff had come to an understanding on this and no
one was telling his boss or his colleagues.

“I should just bulldoze the whole fucking thing and start over.
What comes after goth, anyway? Are ravers back? Hippies? Punks?
Chavs?”

Death Waits was staring at him with round eyes. “You wouldn’t
really\dash{}”

He waved at the kid. This was his whole life. “No, Death, no. We’re
not going to bulldoze this place. You’ve got a job for life here.”
It was a lie of such amazing callousness that Sammy felt a twinge
of remorse while saying it. Those twinges didn’t come often. But
Death Waits looked a lot happier once the words were out of his
mouth\dash{}goths with big candy-apple cheeks were pretty unconvincing
gloom-meisters.

Sammy stalked back to the nearest utilidor entrance, over by what
had been the Pinocchio Village Haus. He’d turned the redesign over
to a designer who’d started out as a lit major and whose admiration
for the dark and twisted elements of the original Pinocchio tale by
Carlo Collodi shone through. Now it featured murals of donkeys
being flensed by fish, hectic Pleasure Island. Hanged Pinocchio on
his gibbet dangled over the condiment bar, twitching and thrashing.
The smell of stale grease rose from it like a miasma, clashing with
the patchouli they pumped out from the underground misters.

Down into the tunnels and then into a golf cart and out to his
office. He had time to paw desultorily at the mountain of
merchandise samples that had come in over the week since he’d last
tackled it\dash{}every plaster-skull vendor and silver cross-maker in the
world saw him as a ticket to easy street. None had twigged to the
fact that they were \emph{reducing} their goth-themed merch these
days. Still, going through merch had been his task for three years
now and it was a hard habit to break. He liked the lick-and-stick
wounds with dancing maggots that were activated by body-heat. The
skeletal bikers with flocking algorithms that led them into noisy
demolition derbies were a great idea, too, since you’d have to buy
another set after a couple hours’ play.

His desk was throbbing pink, which meant that he was late for
something. He slapped at it, read the message that came up,
remembered that there was a weekly status meeting for theme-leaders
that he’d been specifically instructed to attend. He didn’t go to
these things if he could help it. The time-markers who ran
Adventureland and Tomorrowland and so on were all boring curatorial
types who thought that change was what you gave a sucker back from
a ten at a frozen-banana wagon.

The theme-leaders met in a sumptuous board-room that had been
themed in the glory years of the unified Walt Disney Company. It
had renewable tropical hardwood panelling, a beautiful garden and a
koi pond, and an aviary that teemed with chirruping bright birds
borrowed from the Animal Kingdom menagerie. The table was a slab of
slate with a brushed finish over its pits and shelves, the chairs
were so ergonomic that they had zero adjustment controls, because
they knew much better than you ever could how to arrange themselves
for your maximum comfort.

He was the last one through the door, and they all turned to stare
at him. They all dressed for shit, in old fashioned slacks and
high-tech walking shoes, company pocket-tees or baseball jerseys.
None of them had a haircut that was worth a damn, not even the two
women execs who co-ran Main Street. They dressed like the Middle
Americans they catered to, or maybe a little better.

Sammy had always been a sharp dresser. He liked shirts that looked
like good cotton but had a little stretch built into them so they
rested tight at his chest, which was big, and tight at his waist,
which was small. He liked jeans in whatever style jeans were being
worn in Barcelona that year, which meant black jeans cut very
square and wide-legged, ironed stiff without a crease. He had
shades that had been designed to make his face look a little
vulpine, a trait that he’d always known he had. It put people on
edge if you looked a little wolfy.

He stopped outside the door of the board-room and squared up his
shoulders. He was the youngest person on the board, and he’d always
been the biggest, cockiest bastard in the room. He had to remember
that if he was going to survive this next hour.

He came through the door and stopped and looked at the people
around the table and waited for everyone to notice him. They looked
so midwestern and goofy, and he gave them his wolfy smile\dash{}hello,
little piggies, here to blow your house down.

“Hey, kids,” he said, and grabbed the coffee carafe and a mug off
the sideboard. He filled his cup, then passed the carafe off, as
though every meeting began with the passing-around of the low-grade
stimulants. He settled into his seat and looked around
expectantly.

“Glad you could make it, Sammy.” That was Wiener, who generally
chaired the meetings. Theoretically, it was a rotating chairship,
but there’s a certain kind of person who naturally ends up running
every meeting, and Ron Wiener was that kind of person. He co-ran
Tomorrowland with three faceless nonentities who had been promoted
above their competence due to his inexplicable loyalty to them, and
between the four of them, they’d managed to keep Tomorrowland the
most embarrassingly badly themed part of the park. “We were just
talking about you.”

“I love being the subject of conversation,” Sammy said. He slurped
loudly at his coffee.

“What we were talking about was the utilization numbers from
Fantasyland.”

Which sucked, Sammy knew. They’d been in free-fall for months now,
and looking around at those cow-like midwestern faces, Sammy
understood that it was time for the knives to come out.

“They suck,” Sammy said brightly. “That’s why we’re about to change
things up.”

That preempted them. “Can you explain that some?” Wiener said,
clicking his pen and squaring up his notepad. These jerks and their
paper-fetish.

Sammy did his best thinking on his feet and on the move. Confident.
Wolfy. You’re better than these jerks with their pads and their
corn-fed notions. He sucked in a breath and began to pace and use
his hands.

“We’re going to take out every under-utilized ride in the land,
effective immediately. Lay off the dead-wood employees. We’re going
to get a couple off-the-shelf thrill rides and give them a solid
working-over for theming\dash{}build our own ride vehicles, queue areas
and enclosures, big ones, weenies that will draw your eye from
outside the main gate. But that’s just a stopgap.

“Next I’m going to start focus-grouping the fatkins. They’re
ready-made for this stuff. All about having fun. Most of those
ex-fatties used to pack this place when they were stuck in electric
wheelchairs, but now they’re too busy\dash{}” he stopped himself from
saying “fucking”\dash{}“Having more adult fun to come back, but anyone
who can afford fatkins has discretionary income and we should have
a piece of it.

“It’s hard to say without research, but I’m willing to bet that
these guys will respond strongly to nostalgia. I’m thinking of
reinstating the old Fantasyland dark-rides, digging parts out of
storage, whatever we haven’t auctioned off on the collectibles
market, anyway, and cloning the rest, but remaking them with a
little, you know, darkness. Like the Pinocchio thing, but more so.
Captain Hook’s grisly death. Tinker Bell’s inherent porniness. What
kind of friendship did Snow White have with the dwarfs? You see
where I’m going. Ironic\dash{}we haven’t done ironic in a long time. It’s
probably due for a comeback.”

They stared at him in shocked silence.

“You say you’re going to do this when?” Wiener said. He’d want to
know so he could get someone senior to intervene.

“You know, research first. We’ll shut down the crap rides next week
and can the dead-wood. Want to commission the research today if I
can. Start work on the filler thrill-rides next week too.”

He sat down. They continued to boggle.

“You’re serious about this?”

“About what? Getting rid of unprofitable stuff? Researching
profitable directions? Yes and yes.”

There were other routine agenda items, which reminded Sammy of why
he didn’t come to these meetings. He spent the time surfing
readymade coasters and checking the intranet for engineer
availability. He was just getting into the HR records to see who
he’d have to lay off when they finally wound down and he sauntered
out, giving his wolfy grin to all, with a special flash of it for
Wiener.

\begin{center}\rule{3in}{0.4pt}\end{center}

“Death, I’d like a word, please?”

“I’d be delighted.” Death talked like someone who’d learned to talk
by being a precocious reader. He over-pronounced his words, spoke
in complete sentences, and paused at the commas. Sammy knew that
speech pattern well, since he’d worked hard to train himself out of
it. It was a geek accent, and it made you sound like a smart-ass
instead of a sharp operator. You got that way if you grew up trying
to talk with a grown-up vocabulary and a child’s control of your
speech-muscles; you learned to hold your chin and cheeks still
while you spoke to give you a little precision-boost. That was the
geek accent.

“Remember what we talked about this morning?”

“Building a thrill ride?”

“Yes,” Sammy said. He’d forgotten that Death Waits had suggested
that in the first place. Good\dash{}that was a good spin. “I’ve decided
to take your suggestion. Of course, we need to make room for it, so
I’m going to shut down some of the crap\dash{}you know which ones I
mean.”

Death Waits was green under his white makeup. “You mean\dash{}”

“All the walk-throughs. The coffin coaster, of course. The flying
bats. Maybe one or two others. And I’m going to need to make some
layoffs, of course. Gotta make room.”

“You’re going to lay people off? How many people? We’re already
barely staffed.” Death was the official arbiter of shift-changing,
schedule-swapping and cross-scheduling. If you wanted to take an
afternoon off to get your mom out of the hospital or your dad out
of jail, he was the one to talk to.

“That’s why I’m coming to you. If I shut down six of the rides\dash{}”
Death gasped. Fantasyland had 10 rides in total. “Six of the rides.
How many of the senior staffers can I get rid of and still have the
warm bodies to keep everything running?” Senior people cost a
\emph{lot} more than the teenagers who came through. He could hire
six juniors for what Death cost him. Frigging Florida labor laws
meant that you had to give cost-of-living raises every year, and it
added up.

Death looked like he was going to cry.

“I’ve got my own estimates,” Sammy said. “But I wanted to get a
reality check from you, since you’re right there, on the ground.
I’d hate to leave too much fat on the bone.”

He knew what effect this would have on the kid. Death blinked back
his tears, put his fist under his chin and pulled out his phone and
started scribbling on it. He had a list of every employee in there
and he began to transfer names from it to another place.

“They’ll be back, right? To operate the new rides?”

“The ones we don’t bring back, we’ll get them unemployment
counseling. Enroll them in a networking club for the jobless, one
of the really good ones. We can get a group rate. A job reference
from this place goes a long way, too. They’ll be OK.”

Death looked at him, a long look. The kid wasn’t stupid, Sammy
knew. None of these people were stupid, not Wiener, not the kid,
not the goths who led each other around Fantasyland on leashes. Not
the fatkins who’d soon pack the place. They were none of them
stupid. They were just\dash{}soft. Unwilling to make the hard choices.
Sammy was good at hard choices.

\begin{center}\rule{3in}{0.4pt}\end{center}

Perry got home that night and walked in on Lester and Suzanne. They
were tangled on the living-room carpet, mostly naked, and Lester
blushed right to his ass-cheeks when Perry came through the door.

“Sorry, sorry!” Lester called as he grabbed a sofa cushion and
passed it to Suzanne, then got one for himself. Perry averted his
eyes and tried not to laugh.

“Jesus, guys, what’s wrong with the bedroom?”

“We would’ve gotten there eventually,” Lester said as he helped
Suzanne to her feet. Perry pointedly turned to face the wall. “You
were supposed to be at dinner with the gang,” Lester said.

“Close-up on the ride was crazy. Everything was changing and the
printers were out of goop. Lots of action on the network\dash{}Boston and
San Francisco are introducing a lot of new items to the ride. By
the time I got to the guest-house, the Kettlewells were already
putting the kids to bed.” He decided not to mention Eva’s angry
storm-out to Suzanne. No doubt she had already figured out that all
was not well in the House of Kettlewell.

Suzanne ahem’d.

“Sorry, sorry,” Lester said. “Let’s talk about this later, OK?
Sorry.”

They scurried off to Lester’s room and Perry whipped out a
computer, put on some short humor videos in shuffle-mode, and
grabbed a big tub of spare parts he kept around to fiddle with. It
could be soothing to take apart and reassemble a complex mechanism,
and sometimes you got ideas from it.

Five minutes later, he heard the shower running and then Suzanne
came into the living room.

“I’m going to order some food. What do you feel like?”

“Whatever you get, you’ll have to order it from one of the fatkins
places. It’s not practical to feed Lester any other way. Get me a
small chicken tikka pizza.”

She pored over the stack of menus in the kitchen. “Does Food in
Twenty Minutes really deliver in 20 minutes?”

“Usually 15. They do most of the prep in the vans and use a lot of
predictive math in their routing. There’s usually a van within
about ten minutes of here, no matter what the traffic. They deliver
to traffic-jams, too, on scooters.”

Suzanne made a face. “I thought \emph{Russia} was weird.” She
showed the number on the brochure to her phone and then started to
order.

Lester came out a minute later, dressed to the nines as always. He
was barely capable of entering his bedroom without effecting a
wardrobe change.

He gave Perry a slightly pissed off look and Perry shrugged
apologetically, though he didn’t feel all that bad. Lester’s
fault.

Christ on a bike, it was weird to think of the two of them
together, especially going at it on the living room rug like a
couple of horny teens. Suzanne had always been the grownup in their
little family. But that had been back when there was a big company
involved. Something about being a piece of a big company made you
want to act like you’d always figured grownups should act. Once you
were a free agent, there wasn’t any reason not to embrace your
urges.

When the food came, the two of them attacked it like hungry dogs.
It was clear that they’d forgotten their embarrassment and were
planning another retreat to the bedroom once they’d refueled. Perry
left.

\begin{center}\rule{3in}{0.4pt}\end{center}

“Hey, Francis.” Francis was sitting on the second-storey balcony of
his mayoral house, surveying the electric glow of the shantytown.
As usual now, he was alone, without any of his old gang of boys
hanging around him. He waved an arm toward Perry and beckoned him
inside, buzzing him in with his phone.

Perry tracked up the narrow stairs, wondering how Francis
negotiated them with his bad knee and his propensity to have one
beer too many.

“What’s the good word?”

“Oh, not much,” Perry said. He helped himself to a beer. They made
it in the shantytown and fortified it with fruits, like a Belgian
beer. The resulting suds were strong and sweet. This one was
raspberry and it tasted a little pink, like red soda.

“Your friends aren’t getting along too good, is what I hear.”

“Really.” Nothing was much of a secret in this place.

“The little woman’s taken a room of her own down the road. My wife
did that to me once. Crazy broad. That’s their way sometimes. Get
so mad they just need to walk away.”

“I get that mad, too,” Perry said.

“Oh, hell, me too, all the time. But men usually don’t have the
guts to pack a suitcase and light out. Women have the guts. They’re
nothing but guts.”

Perry cursed. Why hadn’t Kettlebelly called him? What was going
on?

He called Kettlebelly.

“Hi, Perry!”

“Hi, Landon. What’s up?”

“Up?”

“Yeah, how are things?”

“Things?”

“Well, I hear Eva took off. That sort of thing. Anything we can
talk about?”

Kettlewell didn’t say anything.

“Should I come over?”

“No,” he said. “I’ll meet you somewhere. Where?”

Francis wordlessly passed Kettlewell a beer as he stepped out onto
the terrace.

“So?”

“They’re in a motel not far from here. The kids love coffins.”

Francis opened another beer for himself. “Hard to imagine a kid
loved a coffin more than your kids loved this place this
afternoon.”

“Eva’s pretty steamed at me. It just hasn’t been very good since I
retired. I guess I’m pretty hard to live with all the time.”

Perry nodded. “I can see that.”

“Thanks,” Kettlewell said. “Also.” He took a pull off his beer.
“Also I had an affair.”

Both men sucked air between their teeth.

“With her best friend.”

Perry coughed a little.

“While Eva was pregnant.”

“You’re still breathing? Patient woman,” Francis said.

“She’s a good woman,” Kettlewell said. “The best. Mother of my
children. But it made her a little crazy-jealous.”

“So what’s the plan, Kettlewell? You’re a good man with a plan,”
Perry said.

“I have to give her a night off to cool down and then we’ll see.
Never any point in doing this while she’s hot. Tomorrow morning,
it’ll come together.”

The next morning, Perry found himself desperately embroiled in
ordering more goop for the three-d printers. \emph{Lots more}. The
other rides had finally come online in the night, after
interminable network screw-ups and malfing robots and printers and
scanners that wouldn’t cooperate, but now there were seven rides in
the network, seven rides whose riders were rearranging, adding and
subtracting, and there was reconciling to do. The printers hummed
and hummed.

“The natives are restless,” Lester said, pointing a thumb over his
shoulder at the growing queue of would-be riders. “We going to be
ready to open soon?”

Perry had fallen into a classic nerd trap of having almost solved a
problem and not realizing that the last three percent of the
solution would take as long as the rest of it put together.
Meanwhile, the ride was in a shambles as robots attempted to print
and arrange objects to mirror those around the nation.

“Soon soon,” Perry said. He stood up and looked around at the
shambles. “I lie. This crap won’t be ready for hours yet. Sorry.
Fuck it. Open up.”

Lester did.

“I know, I know, but that’s the deal with the ride. It’s got to get
in sync. You know we’ve been working on this for months now. It’s
just growing pains. Here, I’ll give you back your money you come
back tomorrow, it’ll all be set to rights.”

The angry rider was a regular, one of the people who came by every
morning to ride before work. She was gaunt and tall and geeky and
talked like an engineer, with the nerd accent.

“What kind of printer?” Lester broke in. Perry hid his snicker with
a cough. Lester would get her talking about the ins and outs of her
printer, talking shop, and before you knew it she’d be mollified.

Perry sold another ticket, and another.

“Hi again!” It was the creepy guy, the suit who’d shown up in
Boston. Tjan had a crazy theory about why he’d left the Boston
launch in such a hurry, but who knew?

“Hi there,” Perry said. “Long time no see. Back from Boston, huh?”

“For months.” The guy was grinning and sweating and didn’t look
good. He had a fresh bruise on his cheek with a couple of knuckle
prints clearly visible. “Can’t wait to get back on the ride. It’s
been too long.”

\begin{center}\rule{3in}{0.4pt}\end{center}

Sammy had been through a rehab and knew how they went. You laid off
a bunch of people in one fast, hard big bang. Hired some
unemployment coaches for the senior unionized employees, scheduled
a couple of “networking events” where they could mingle with other
unemployed slobs and pass around home-made business cards.

You needed a Judas goat, someone who’d talk up the rehab to the
other employees, whom you could rely on. Death Waits had been his
Judas goat for the Fantasyland goth makeover. He’d tirelessly
evangelized the idea to his co-workers, had found goth tru-fans
who’d blog the hell out of every inch of the rehab, had run every
errand no matter how menial.

But his passion didn’t carry over to dismantling the goth rehab.
Sammy should have anticipated that, but he had totally failed to do
so. He was just so used to thinking of Death Waits as someone who
was a never-questioning slave to the park.

“Come on, cheer up! Look at how cool these thrill rides are going
to be. Those were your idea, you know. Check out the coffin-cars
and the little photo-op at the end that photoshops all the riders
into zombies. That’s got to be right up your alley, right? Your
friends are going to love this.”

Death moped as only a goth could. He performed his duties slowly
and unenthusiastically. When Sammy pinned him down with a direct
question, he let his bangs fall over his eyes, looked down at his
feet, and went silent.

“Come on, what the hell is going on? The fences were supposed to be
up this morning!” The plan had been to get the maintenance crews in
before rope-drop to fence off the doomed rides so that the
dismantling could begin. But when he’d shown up at eight, there was
no sign of the fences, no sign of the maintenance crews and the
rides were all fully staffed.

Death looked at his feet. Sammy bubbled with rage. If you couldn’t
trust your own people, you were lost. There were already enough
people around the park looking for a way to wrong-foot him.

“Death, I’m talking to you. For Christ’s sake, don’t be such a
goddamned baby. You shut down the goddamned rides and send those
glue-sniffers home. I want a wrecking crew here by lunchtime.”

Death Waits looked at his feet some more. His floppy black wings of
hair covered his face, but from the snuffling noises, Sammy knew
there was some crying going on underneath all that hair.

“Suck it up,” he said. “Or go home.”

Sammy turned on his heel and started for the door, and that was
when Death Waits leapt on his back, dragged him to the ground and
started punching him. He wasn’t much of a puncher, but he did have
a lot of chunky silver skull-rings that really stung. He pasted a
couple good ones on Sammy before Sammy came to his senses and threw
the skinny kid off of him. Strangely, Sammy’s anger was dissipated
by the actual, physical violence. He had never thrown a punch in
his life and he was willing to bet the same was true of Death
Waits. There was something almost funny about an actual punch-up.

Death Waits picked himself up and looked at Sammy. The kid’s
eyeliner was in smears down his cheeks and his hair was standing up
on end. Sammy shook his head slowly.

“Don’t bother cleaning out your locker. I’ll have your things sent
to you. And don’t stop on your way out of the park, either.”

He could have called security, but that would have meant sitting
there with Death Waits until they arrived. The kid would go and he
would never come back. He was disgraced.

And leave he did. Sammy had Death Wait’s employee pass deactivated
and the contents of his locker\dash{}patchouli-reeking black tee-shirts
and blunt eyeliner pencils\dash{}sent by last-class mail to his house. He
cut off Death Waits’s benefits. He had the deadwood rides shuttered
and commenced their destruction, handing over any piece
recognizable as coming from a ride to the company’s auction
department to list online. Anything to add black to his bottom
line.

But his cheek throbbed where Death had laid into him, and he’d lost
his fire for the new project. Were fatkins a decent-sized market
segment? He should have commissioned research on it. But he’d
needed to get a plan in the can in time to mollify the executive
committee. Plus he knew what his eyes told him every day: the park
was full of fatkins, and always had been.

The ghost of Death Waits was everywhere. Sammy had to figure out
for himself whom to fire, and how to do it. He didn’t really know
any of the goth kids that worked the rides these days. Death Waits
had hired and led them. There were lots of crying fits and threats,
and the kids he didn’t fire acted like they were next, and if it
hadn’t been for the need to keep revenue flowing, Sammy would have
canned all of them.

Then he caught wind of what they were all doing with their
severance pay: traveling south to Hollywood and riding that
goddamned frankenride in the dead Wal-Mart, trying to turn it into
goth paradise. Judging from the message-boards he surfed, the whole
thing had been Death Waits’s idea. Goddamn it.

It was Boston all over again. He’d pulled the plug and the machine
kept on moving. The hoardings went up and the rides came down, but
all his former employees and their weird eyeliner pervert pals all
went somewhere else and partied on just the same. His attendance
numbers were way down, and the photobloggers posting shots of black
clouds of goths at the frankenride made it clear where they’d all
gone.

\emph{Fine}, he thought, \emph{fine. Let’s go have a look.}

The guy with the funny eyebrow made him immediately, but didn’t
seem to be suspicious. Maybe they never figured out what he’d done
in Boston. The goth kids were busy in the market stalls or hanging
around smoking clove and patchouli hookahs and they ignored him as
a square and beneath their notice.

The ride had changed a great deal since his last fated visit. He’d
heard about The Story, of course\dash{}the dark-ride press had reported
on it in an editorial that week. But now The Story\dash{}which, as he
could perceive it, was an orderly progression of what seemed to be
someone’s life unfolding from childhood naivete to adolescent
exuberance to adult cynicism to a nostalgic, elderly delight\dash{}was
augmented by familiar accoutrements.

There was a robot zombie-head from one of the rides he’d torn down
yesterday. And here was half the sign from the coffin coaster. A
bat-wing bush from the hedge-maze. The little bastards had stolen
the deconstructed ride-debris and brought it here.

By the time he got off the ride, he was grinning ferociously. By
tomorrow there’d be copies of all that trademarked ride-stuff
rolling off the printers in ten cities around the United States.
That was a major bit of illegal activity, and he knew where he
could find some hungry attack-lawyers who’d love to argue about it.
He jumped on the ride again and got his camera configured for
low-light shooting.

\begin{center}\rule{3in}{0.4pt}\end{center}

Eva showed up on Perry’s doorstep that night after dinner. Lester
and Suzanne had gone off to the beach and Perry was alone, updating
his inventory of tchotchkes with a camera and an old computer,
getting everything stickered with RFIDs.

She had the kids in tow. Ada spotted the two old, lovely baseball
mitts on the crowded coffee table and made a bee-line for them,
putting one over each hand and walking around smacking them
together to hear the leathery sound, snooping in drawers and
peering at the business-end of an arc-welder that Perry hastily
snapped up and put on a high shelf, which winked once to let him
know that it had tracked the movement and noted the location of the
tool.

The little boy, Pascal, rode on his mother’s hip. Eva had clearly
had a bit of a cry, but had gotten over it. Now she was determined,
with her jaw thrust out and her chin up-tilted.

“I don’t know what to do about him. He’s been driving me crazy
since he retired. You know he had an affair?”

“He told me.”

She laughed. “He tells \emph{everyone}. He’s boasting, you know?
Whatever. I know why he did it. Mid-life crisis. But before that,
it was early-adulthood crisis. And adolescent crisis. That guy
doesn’t know what to do with himself. He’s a good man, but he’s out
of his fucking mind if he’s not juggling a hundred balls.”

Perry tried out a noncommittal shrug.

“You’re his buddy, I know. But you have to see that it’s true,
right? I love him, I really do, but he’s got a self-destructive
streak a mile wide. It doesn’t matter how much he loves me or the
kids, if he’s not torturing himself with work, he’s got to come up
with something else to screw up his life. I thought that we were
going to spend the next twenty years raising the kids, doing
volunteer work, and traveling. Not much chance of that though. You
saw how he was looking at Suzanne.”

“You think he and Suzanne\dash{}”

“No, I asked him and he said no. Then I talked to \emph{her} and
she told me that she wouldn’t ever let something like that happen.
Her I believe.” She sat down and dandled the little boy until he
gurgled contentedly. Perry heard Ada going crazy in the kitchen
with a mechanical sphincter he’d been building. “Rides are a lot of
fun, Perry. Your ride, it’s amazing. But I don’t want to ride a
ride for the rest of my life, and Landon is a ride that doesn’t
stop. You can’t get off.”

Perry was at a loss. “I’ve never had a relationship that lasted
more than six months, Eva. I’ve got no business giving you advice
on this stuff. Kettlewell is pretty amazing, though. It sounds like
you’ve got him pretty wired, right? You know that if he’s busy,
he’s happy, and when he’s slack, he’s miserable. Sounds like if you
keep him busy, he’ll be the kind of guy you want him to be, even if
you won’t have much time to play with him.”

She unholstered a tit and stuck it in the boy’s mouth and Perry
looked at the carpet. She laughed. “You are such a geek,” she said.
“OK, fine. I hear what you’re saying. So how do I get him busy
again? Can you use him around here?”

“Here?” Perry thought about it. “I don’t think we need much empire
building around here.”

“I thought you’d say that. Perry, what the hell am I going to do?”

There was a tremendous crash from the kitchen, a shriek of
surprise, then a small “oops.”

“Ada!” Eva called. “What now?”

“I was playing ball in the house,” Ada said in the same small
voice. “Even though you have told me not to. And I broke something.
I should have listened to you.”

Eva shook her head. “Plays me like a goddamned cello,” she said.
“I’m sorry, Perry. We’ll pay for whatever it was.”

He patted her arm. “You forget who you’re talking to. I love fixing
stuff. Don’t sweat it.”

“Whatever\dash{}I’ll buy you one and you can use it for parts. Ada! What
did you break, anyway?”

“Made of seashells, by the toaster. It’s twitching.”

“Toast-making seashell robot,” Perry said. “No sweat\dash{}it was due for
an overhaul, anyway.”

“Christ,” she said. “Toast-making \emph{seashell} robot?”

“Kettlewell is why we gave up making that kind of thing,” he said.

“Have you seen him?”

“I’ve seen him.”

“How penitent was he?”

He thought back to Kettlewell’s long puss on Francis’s terrace.
“Yeah, pretty penitent. He’s pretty worried, I’d say.”

She nodded. “All right then. Maybe he’s learned a lesson. Ada! Stop
breaking things and get your shoes back on!”

“We going back to Daddy?”

“Yes,” she said.

“Good,” Ada said.

They were barely out the door when Suzanne and Lester came in. They
nodded at Perry and disappeared into the bedroom. Ten minutes
later, Suzanne stomped out again. She barely looked at Perry as she
disappeared into the corridor, slamming the door behind her.

Perry waited five minutes to see if Lester would come out on his
own. This happened sometimes with the fatkins girls; love among the
fatkins was stormy and unpredictable and Lester seemed to like
bragging about the melt-downs they experienced, each one an oddity
of sybaritic fatkins culture to boast about.

But Lester didn’t come out this time. Perry thought about calling
him or sending him an email. Finally, Perry went and knocked at his
door.

“Oh, go back to the living room, I’ll come out, I’ll come out.”

Perry went back and moused desultorily at some ride-fan blogs for a
while, listening for Lester’s door opening. Finally, out he came,
long-faced and puffy-eyed.

Perry shook his head. Was everyone miserable tonight?

“Hello, Lester,” he said. “Something on your mind?”

He barked a humorless laugh. “With her, I’m still fat.”

Perry nodded as though he understood, though he didn’t.

“Since fatkins, I’ve felt like, I don’t know, a real person. When I
was big, I was invisible and totally asexual. I didn’t think about
having sex with anyone and no one ever thought about having sex
with me. When I felt something for a woman, it was more like a big,
romantic love, like I was a beast and she was a beauty and we could
enjoy some kind of chaste, spiritual love.

“Fatkins made me \ldots{} whole. A whole person, with a life below my belt
as well as above my neck. I know it looks gross and desperate to
you, but to me it’s a celebration. Every time I get together with a
fatkins girl and we’re, you know, partying\dash{}for both of us it
becomes something really intimate. A denial of pain. A fuck you to
the universe that made us so gross and untouchable.”

“And with her, you’re still fat, huh?”

Lester winced. “Yeah, it’s my problem. I guess I really resent her
for not wanting me when I was big, though I totally get why she
wouldn’t have.”

“Maybe you’re angry that she wants you now.”

“Huh.” Lester looked at his hands, which he was dry-washing in his
lap. “OK, maybe. Why should she want me now? I’m the same person,
after all.”

“Except that you’re whole now.”

“Urk.” Lester started pacing. “Who broke the toast-robot?”

“Kettlewell’s daughter, Ada. Eva was over with the kids. She moved
out on Kettlebelly.” He thought about whether he should tell
Lester. What the hell. “She thinks he’s in love with Suzanne.”

“Jesus,” Lester said. “Maybe we should swap. I’ll take Eva and he
can take Suzanne.”

“You’re such a pig,” Perry said.

“You know us fatkins\dash{}fuck, food and folly.”

“So what’s going on with you and Suzanne now?”

“She’s gone away until I can get naked around her without either
bursting into tears or making sarcastic remarks.”

Jesus. Crying. Perry couldn’t remember when he’d ever seen Lester
cry. It was waterworks city these days around here.

“Ah.” Perry just wanted this day to be over. He missed Hilda,
though he barely knew her. It would have been nice to have someone
here at home with him, someone he could cuddle up to in bed and
talk this all out with. Maybe he should call Tjan. He hit the
button on his computer that made the TV blink the time in Morse
code. It was 1AM. He’d have to be up in six hours to get the ride
up and running. Screw all this, he was going to bed. He hadn’t even
gotten a single email from Hilda since he’d left Madison. Not that
he’d sent one to her, of course.

Lester was still snoring when Perry slipped out of the condo, a
bulb of juice and a microwavable venison and quail-egg breakfast
burrito under his arm. He had a little glove-box microwave and by
the time he hit his first red light, the burrito was nuclear-hot
and ready to eat. He gobbled it one-handed while he made his way to
the ride.

There were two cop cars at the end of the driveway leading to the
parking lot. Broward County sheriff’s deputy black-and-whites,
parked horizontally to blockade the drive.

Perry pulled over and got out of his car slowly, keeping his hands
in plain sight. The doors of the cruisers opened, too. The deputies
already had their mirrorshades on, though the sun was still rising,
and they set down their coffees on the hood of the cars.

“This yours?” A deputy said, jerking his thumb over his shoulder at
the flea market and the ride.

Perry knew better than to answer any questions. “Can I help you?”

“We’re shutting you down, buddy, sorry.” The cop was young, Latina
and female, her partner was older, white and male, with the ruddy
complexion that Perry associated with old time Florida cops.

“What’s the charge?”

“There’s no charge,” the male cop said. He sounded like he was
angry already and anything Perry said would just make him angrier.
“We charge you if we’re going to arrest you. We’re enforcing an
injunction. Now, if you try to get past us, we’ll come up with a
charge and \emph{then} we’ll arrest you.”

“Can I see the injunction?”

“Sure, you can go to the courthouse and see the injunction.”

“Aren’t you supposed to have a copy of it to show to me?”

“Am I?” The cop’s grin was mean and impatient.

“Can I go and get some stuff from my office?”

“If you want to get arrested you can.” He pulled a dyspeptic face
and drank some coffee, then got back into his cruiser.

The other cop had the grace to look faintly embarrassed at her
asshole partner, but then she, too, got back in her car.

Perry thought furiously about this. The cop was clearly itching to
bust his ass. Maybe he hated the ride, or this duty, or maybe he
hated Perry\dash{}maybe he was one of the cops who had raided the
shantytown all those years before. Perry had taken a pretty big
settlement off the county over the shot in his head, and it was a
sure bet that a lot of cops had suffered for it and now harbored
some enmity for him.

As bad as this was, it was about to get worse. The goth kids who’d
been hanging around in droves lately\dash{}they didn’t seem like the sort
with a lot of good instincts when it came to dealing with authority
figures. Then there were the flea-market stall owners, who’d be
coming over the road to open their shops in an hour or so. This
could get really goddamned ugly.

He needed a lawyer, and someone to front for him with the lawyer.
He could call Tjan\dash{}he would call him, in fact, but not just yet.
There were limits to what Tjan could do from Boston, after all.

He got back in his car and peeled across the road to the shantytown
and the guesthouse.

“Kettlewell!” He thumped the door. “Come on, Landon, it’s me,
Perry. It’s an emergency.”

He heard Eva curse, then heard movement. “Whazzit?”

“Sorry, man, I wouldn’t have woken you but it’s a real emergency.”

“Fire?”

“No. Cops. They’ve shut down the ride.”

Kettlewell opened the door a crack and stared at him with a
red-rimmed, hung-over eye. “Cops shut down the ride?”

“Yeah, they say there’s an injunction.”

“Gimme a sec, gotta put some pants on.” He closed the door. As
Perry listened to the sounds of him getting dressed, he reflected
that he’d done Eva the favor she’d been seeking: he’d found
something to keep Kettlewell busy.

Kettlewell quizzed him intensely as they drove back across the road
to the police-cars. He called Tjan and got voicemail, left a brief
message, then got out of the car and stood still outside it, waving
at the cop-cars.

“What?”

The male cop looked even more dyspeptic.

“Hi there! I wondered if I could get you to explain what’s going on
here so we can open up shop again?”

“We’ve shut you down to enforce an injunction.”

“What injunction is that?”

“A court injunction.”

“Which court?”

The cop looked really angry for a second, then he got back in his
car and fished around. “Broward County.” He sounded aggrieved.

“Is that the injunction there?” Kettlewell said.

“No,” the cop said, too quickly. They both knew he was lying,
jerking them around.

“Can I see it? Does it have information about who to talk to to get
the injunction lifted?” Kettlewell’s tone was even, pleasant and
very adult. The voice of someone used to being obeyed.

“You’ll have to go to the courthouse. They open in a couple
hours.”

“I’d really like to see it.”

“Oh for chrissakes,” the female cop said. “Just show it to them,
Tom. God.” She spat on the ground. Her partner gave her a look,
then handed the paper over to Kettlewell, who pored over it
intently. Perry shoulder surfed him and gathered that they were
being shut down for infringing Disney Parks Company trademarks.
That was weird. You could hardly go ten feet in Florida without
tripping over a bootleg Mickey, so why should the market-stalls’
Mickey designs trigger legal action?

“All right, then,” Kettlewell said. “Let’s make some phone calls.”

They got in the car and drove across the road to the shantytown.
There was a tea-house that opened early and they commandeered its
window table and spread out their things. Perry called Lester and
woke him up. It took two or three tries to get his head around
it\dash{}Lester couldn’t figure out why they’d shut down the
market-stalls, but once he got that the ride was down too, he woke
up fast and promised to meet them.

Kettlewell’s conversation with Tjan was a lot more heated. Perry
tried to eavesdrop but couldn’t make any sense of it.

“All the rides are down,” he said once he’d dropped the phone to
bounce a couple times on the tabletop, making the coffees shiver.
“Every one of them was shut down by the cops this morning.”

“You’re shitting me. But they don’t all sell the same stuff.”

“They were shut down because of Disney trademarks in the ride
itself, or so it seems. Now, what are we going to do? Tjan’s hired
a lawyer for the Boston group and we can hire one for here, but I
don’t think we’re going to be able to hire fixers everywhere that
there’s a ride. That’s going to be really expensive. Disney’s filed
all the injunctions at the state level\dash{}they have an industry
association they work through that has cooperating attorneys in
every city in the country, so it was easy for them.”

“Holy crap.”

“Yeah. Who did you piss off, Perry?”

Damned if he knew. He literally couldn’t think of a single person
who’d want to do this\dash{}someone had convinced the Disney company to
clobber him like Godzilla going after Tokyo. It just didn’t make
any sense.

“So what do we do?”

Kettlewell looked at him. “I have no clue, Perry. You aren’t a
company. You aren’t a network of companies. You aren’t an industry
association. No one can speak for you. You can’t lobby or even
field a spokesman. I mean, none of that stuff works for you\dash{}and
that’s the only way I know to fight back in court.”

“I thought we were immune to this stuff. If there’s no one to sue,
how can they sue us?”

“If there’s no one to sue, there’s no one to show up in court and
object, either.”

“Yeah.”

“I don’t think we can incorporate you in time to make a
difference,” Kettlewell said. “So we need to think of something
else.”

Suzanne slid into the booth beside them. Her hair was tied back and
her makeup was spare and severe. She had on European-cut trousers,
high like a bolero-dancer’s, and a loose, flowing white cotton
over-shirt on top of a luminescent pink tank. Perry couldn’t tell
whether it was formal or informal, but it looked good and a little
intimidatingly foreign. She didn’t meet Perry’s eye.

“Brief me,” she said. She held out her phone and put it in record
mode.

Kettlewell ran it down quickly and she nodded, jotting notes.

“So what happens next?”

“Not much we can do,” Kettlewell said.

“The riders will be along shortly. Oh, and the merchants.” Perry
still couldn’t catch her eye.

“I’ll go take some pictures,” she said.

“Be careful,” Perry said.

She mugged for him. “Sweetie, I take pictures of the mafiyeh.” Then
it was all right between them again, somehow.

“Right,” Kettlewell said. “How’s our time looking?”

“Got thirty minutes until the first of the merchants show up. An
hour until the riders start turning up.”

“You don’t have a lawyer, do you?”

Perry quirked his funny eyebrow.

“Stupid question. OK. Right, I’ll make some more calls. Let’s get
some people out of bed.”

“What can I do?”

Kettlewell looked at him. “Huh. Um. This is really my beat now. I
suppose you could go keep Suzanne company.”

“Gee, thanks.”

“Something wrong with Suzanne?”

“Nothing’s wrong with Suzanne,” he said. “OK, off I go.”

He set off on foot. The shantytown had woken up now, people getting
ready for the hike to the early busses into places where the few
remaining jobs were.

He took his phone out and tossed it from hand to hand. Then he
called the number that he’d programmed in all those days ago in
Madison but had never bothered to call. He forgot until the ringing
started that it was another time-zone there\dash{}an hour or two earlier.
But when Hilda answered, she sounded wide awake.

“Nice of you to call,” she said.

“Nice of you to answer.” Her voice sent a thrill up his spine.

“We’ve got cops outside of the ride here,” she said. “We’ve only
been live for a week, too.”

“They’re at every ride,” he said. “They shut us down too.”

“Well, what are you going to do about it?”

“What am \emph{I} going to do about it?”

“Sure, this is your thing, Perry. We woke up and discovered the
cops this morning and the first thing everyone did was wonder when
you’d call with the plan.”

“You’re kidding. What do I know about cops?”

“What do any of us know about cops? All we know is we built this
thing after you came and talked to us about it and now it’s been
shut down, so we’re waiting for you to tell us what to do next.”

He groaned and sat down on a curb. “Oh, crap.”

Then she sighed heavily at the other end. “OK, Perry, you need to
pull it together. We need you now. We need something that explains
what’s going on, what to do next, and how to do it. There’s a lot
of energy out here, a lot of people ready to fight. Just point us
in the right direction.”

“I have a guy who’s trying to figure that out right now.”

“Perfect. Now you need to set up a conference call with every ride
operator so we can talk this over. Get online and post a time and
an address. I’ll chat it up and make some calls. You make some
calls too. Everyone likes to hear from you. They like to know
you’re on their side.”

“Right,” he said, getting back to his feet, turning around to get
his computer out of his trunk. “Right. That’s totally the right
thing to do. I’m on it.”

“Good man,” she said.

A little pause stretched between them. “So,” he said. “How you
doing, apart from all this?”

Her laugh was merry. “I thought you’d never ask. I’m looking
forward to your next visit, is how I’m doing.”

“Really?”

“Of course really.”

“You sounded a little pissed at me there is all.” He sounded like a
lovesick teenager. “I mean\dash{}” He broke off.

“Your ass needed kicking, was all.” Pause. “I’m not pissed at you,
though. When are you coming for a visit?”

“Got me,” he said. “I guess I should, right?” He really sounded
like a teenager.

“You need to visit all the sites, check in on how we’re doing.”
Pause. “Plus you should come hang out with me some.”

He almost pointed out all her warnings about only having a
one-night stand and not missing the people he was away from and so
forth, but stayed his tongue. The fact that she wanted him to come
for a visit was overshadowing everything, even the looming crisis
with the cops.

“It’s a deal.”

“Deal.”

“Well, bye.”

“Bye.”

He almost said, “You hang up first,” but that would have been too
much. Instead he just kept the phone at his ear until he heard her
click.

Suzanne was pointing and shooting like mad. Perry sat down on the
cracked pavement beside her and unfolded his computer and started
sending out emails, setting up a conference-channel. He gave
Suzanne a short version of his talk with Hilda, being careful not
to give a hint of his feelings for her.

“She sounds like a sensible girl,” Suzanne said. “You should go and
pay her another visit.”

He blushed and she socked him in the shoulder.

“Take your call,” she said. The cops were giving them the hairy
eyeball, and Perry screwed in his headset.

The conference channel was filling up. Perry checked off names as
reps from all the rides came online. There was a lot of tight,
tense chatter, jokes about the fuzz.

“OK,” Perry said. “Let’s get it started. There’s cops blockading
every ride, right? Use the poll please.” He posted a poll to the
conference page and it quickly got to 100 percent green. “So I just
found the cops outside of mine, too, and I’m not sure what to do
about it. I’ve got some dough for a lawyer, but I can’t afford
lawyers for everyone. To make that work, we’d have to fly attorneys
to every city with a ride in it, and that’s not practical as I’m
sure you can tell.”

A half-dozen flags went up in the conference page. “I need someone
to play moderator, ’cause I can’t talk and mod at the same time.
How about you, Hilda?”

“OK,” she said. “I’m Hilda Hammersen, from the Madison group. Post
one-line summaries of your points and I’ll set a speaker order.”

The conference page filled up. There was the official back channel
at the bottom where the text was spilling by too fast for Perry to
parse, and he knew that there were lots of unofficial back-channels
in use, too. He covered the mic and sighed. He had nothing to say
to these people. He didn’t have any answers.

“Right. So who knows what we should do?” The back-channel went
crazy. Hilda started green-lighting speakers with their flags up.

“Why are you asking us, Perry? You’ve got to run this.” The voice
was petulant and Perry saw that it was one of the Boston crew,
which made him wonder what Tjan was going to do when he discovered
that Perry was doing this.

The page pinkened and then sank into red. The other people on the
call clearly thought this was BS, which was a relief to Perry.
Hilda cued up the next speaker.

“We could set up information pickets at the gates to each ride
hitting people up for donations for our legal defense\dash{}get the press
to cover it and maybe we could bring in enough to fight all the
injunctions.”

The pink lightened a little, went back to neutral white, turned a
little green. Perry slowed down the back-channel a little and
skimmed it:

\begin{blog}%TODO
No way could we bring in enough, that’s like 30 grand each I get
a couple hundred people here in the morning and that would mean a
hundred and fifty bucks each

No no it’s totally do-able we can raise that easy just set up
some paypals and publicize the shit out of it
\end{blog}

The next speaker was talking. “What if we got the maintenance bots
to break open the doors and carry the ride outside where everyone
can see it?”

Bright red. Dumb idea.

Perry broke in. “I’m worried that when people show up it’ll provoke
some kind of confrontation with the law. It could get ugly here.
How can we keep that cooled out?”

Green.

“That’s totally got to be our top priority,” Hilda said.

Next speaker. “OK, so the best way to keep people calm is to tell
them that there’s an alternative to going nuts, which maybe could
be raising money for a legal defense.”

Green-ish. “What about finding pro-bono lawyers? What about the
ACLU or EFF?”

Greener.

The back-channel filled up with URLs and phone numbers and email
addresses.

“OK, time’s running out here,” Perry said. “You guys need to
organize a call-around to those orgs and see if they’ll help us
out. Pass the hat at your rides, try to find lawyers. Everyone keep
reporting in all day\dash{}especially if you get a win anywhere. I’m
going to go take care of things here.”

Hilda IMed him\dash{}“Good luck, Perry. You’ll kick ass.”

Perry started to IM back, but a shadow fell across his screen. It
was Jason, who ran the contact-lens stall. He was staring at the
two cop-cars quizzically, looking groggy but growing alarmed.

Perry closed his lid and got to his feet. “Morning, Jason.” Behind
Jason were five or six other vendors. The sellers who lived in the
shantytown and could therefore walk to work were always first in.
Soon the commuters would start arriving in their beater cars.

“Hey, Perry,” Jason said. He was chewing on an unlit cigarette, a
disgusting habit that was only marginally less gross than smoking
them. He’d tried toothpicks, but nothing would satisfy his oral
cravings like a filter-tip. At least he didn’t light them. “What’s
up?”

Perry told him what he knew, which wasn’t much. Jason listened
carefully, as did the other vendors who arrived. “They’re fucking
with you, man. The cops, Disney, all of them. Just fucking with
you. You go ahead and hire a lawyer to go to the court for you and
see how far it gets you. They’re not playing by any rules, they’re
not interested in the law you broke or whatever. They just want to
fuck with you.”

Suzanne appeared over Perry’s shoulder.

“I’m Suzanne Church, Jason. I’m a reporter.”

“Sure, I know you. You were there when they burned down the old
place.”

“That was me. I think you’re right. They’re fucking with you guys.
I want to report on that because it might be that exposing it makes
it harder to continue. Can I record what you guys say and do?”

Jason grinned and slid the soggy cig from one corner of his mouth
to the other and back again. “Sure, that’s cool with me.” He turned
to the other sellers: “You guys don’t mind, do you?” They joked and
laughed and said no. Perry let out a breath slowly. These guys
didn’t want a confrontation with the cops\dash{}they knew better than him
that they couldn’t win that one.

Suzanne started interviewing them. The cops got out of their cars
and stared at them. The woman cop had her mirrorshades on now, and
so the both of them looked hard and eyeless. Perry looked away
quickly.

The vendors with cars were pulling them around to the roadside
leading up to the ride, unpacking merchandise and setting it out on
their hoods. Vendors from the shantytown headed home and came back
with folding tables and blankets. These guys were business-people.
They weren’t going to let the law stand in the way of putting food
on the table for their families.

The cops got back into their cars. Kettlewell worked his way
cautiously across the freeway, climbing laboriously over the
median. He had changed into a smart blazer and slacks, with a crisp
white shirt that hid his incipient belly. He looked like the
Kettlewell of old, the kind of man used to giving orders and
getting respect.

“Hey, man,” Perry said. Kettlewell’s easy smile was reassuring.

“Perry,” he said, throwing an arm around his shoulders and leading
him away. “Come here and talk with me.”

They stood in the lee of one of the sickly palms that stood by the
roadside. The day was coming up hot and Perry’s t-shirt stuck to
his chest, though Kettlewell seemed dry and in control.

“What’s going on, Perry?”

“Well, we did a phoner this morning with all the ride operators.
They’re going to work on raising money for the defense and getting
pro-bono lawyers from the EFF or the ACLU or something.”

Kettlewell did a double-take. “Wait, what? They’re going to ask the
ACLU? They can’t be trusted, Perry. They’re
\emph{impact litigators}\dash{}they’ll take cases to make a point, even
when it’s not in their clients’ best interests.”

“What could be more in our interests than getting lawyers to fight
these bogus injunctions?”

Kettlewell blew out a long breath. “OK, table it. Table it. Here’s
what I’ve been pulling together: we’ve got a shitkicking corporate
firm that used to handle the Kodacell business that’s sending out a
partner to go to the Broward County court this morning to get the
injunction lifted. They’re doing this as a freebie, but I told them
that they could handle the business if we put together all the
rides into one entity.”

Now it was Perry’s turn to boggle. “What kind of entity?”

“We have to incorporate them all, get them all under one umbrella
so that we can defend them all in one go. Otherwise there’s no way
we’re going to be able to save them. Without a corporate entity,
it’s like trying to herd cats. Besides, you need some kind of
structure, a formal constitution or something for this thing.
You’ve got a network protocol, and that’s it. There’s money at
stake here\dash{}potentially some big money\dash{}and you can’t run something
like that on a handshake. It’s too vulnerable. You’ll get embezzled
or sued into oblivion before you even have a chance to grow. So
I’ve started the paperwork to get everything under one banner.”

Perry counted to ten, backwards. “Landon, I’m really thankful that
you’re helping us out here. You’re probably going to save our
asses. But you can’t put everything under one banner\dash{}you can’t just
declare to these people that their projects are ours\dash{}”

“Of course they’re yours. They’re using your IP, your protocols,
your designs.\ldots{} If they don’t come on board, you can just threaten
to sue them\dash{}”

“Landon! Please listen to me. We are not going to effect a hostile
takeover of my friends. They are equal owners of everything we do
here. And no offense, but if you ever mention suing other projects
over our ’IP’”\dash{}he made sarcastic finger quotes\dash{}“then we’re through
having any discussions about this. OK?”

Kettlewell snorted air through his nostrils. “My apologies, I
didn’t realize that this was such a sensitive area for you.” Perry
boggled at this\dash{}lawsuits against ride operators! “But I can get
that. Here’s the thing, Perry. Without some kind of fast-moving
structure you’re going to be dead. Even if we repel the boarders
this morning, they’ll be back tomorrow and the day after. You need
something stronger than a bunch of friends who have loose
agreements. You need a legal entity that can speak for everyone.
Maybe that’s a co-op or a charity or something else, but it’s got
to exist. You may not think you have any say over these other
rides, but does everyone else agree? What if you get sued for
someone’s bad deeds in Minneapolis? What if some ride operator sues
\emph{you} to put you out of business?”

Perry’s head swam. He hated conversations like this. He didn’t have
any good answer for Kettlewell’s objections, but it was ridiculous.
No one from a ride was going to sue him. Or maybe they would, if he
got all grabby and went MINE MINE MINE and incorporated everything
with him at the top. Hilda said he was the one they all looked to,
but that was because he would never try to hijack their projects.

“No.”

“No what?”

“No to all of it. We have to defend this thing, but we’re not going
to do it by trying to tie everyone down to contracts and agreements
where I get to control everything. Maybe a co-op is the right way
to go, but we can’t just declare a co-op and force everyone to be
members. We have to get everyone to agree, everyone who’s involved,
and then they can elect a council or something and work out some
kind of uniform agreement. I mean, that’s how all the good free
software projects work. There’s authority, but it’s not all
unilateral and imperious. I’m not interested in that. I’d rather
shut this down than declare myself pope-emperor of ride-land.”

Kettlewell scrubbed his eyes with his fists. Up close, the lines in
his face were deep-sunk, his eyeballs bloodshot and hung over.
“You’re killing me, you know that? What good is principle going to
do when they knock this fucking thing down and slap you with a
gigantic lawsuit?”

Perry shrugged. “I really appreciate what you’ve done, but I’d
rather lose it than fuck it up.”

They stared at each other for a long time. Cars whizzed past. Perry
felt like a big jerk. Kettlewell had done amazing work for him this
morning, just out of the goodness of his own heart, and Perry had
repaid him by being a stiff-necked dickwad. He felt an overwhelming
desire to take it back, just put Kettlewell in charge and let him
run the whole show. Just shrug his shoulders and abdicate.

He looked down at the ground and up into the straggly palms, then
heaved a sigh.

“Landon, I’m sorry, OK, but that’s just how it is. I totally dig
that you’re saying that we’re risking everything by not doing it
your way, but from my seat, doing it your way will kill it anyway.
So we need a better answer.”

Kettlewell scrubbed his eyes some more. “You and my wife sound like
you’d get along.”

Perry waited for him to go on, but it became clear he had nothing
more to say.

Perry went back to the cop cars just as the first gang of goths
showed up to take a ride.

\chapter{PART III}

Sammy had filled a cooler and stuck it in the back-seat of his car
the night before, programmed his coffee-maker, and when his alarm
roused him at 3AM, he hit the road. First he guzzled his thermos of
lethal coffee, then reached around in back for bottles of icy
distilled water. He kept the windows rolled down and breathed in
the swampy, cool morning air, the most promising air of the Florida
day, before it all turned to steam and sizzle.

He didn’t bother looking for truck-stops when he needed to piss,
just pulled over on the turnpike’s side and let fly. Why not? At
that hour, it was just him and the truckers and the tourists with
morning flights.

He reached Miami ahead of schedule and had a diner-breakfast big
enough to kill a lesser man, a real fatkins affair. He got back on
the road groaning from the chow and made it to the old Wal-Mart
just as the merchants were setting up their market on the
roadside.

When he’d done the Boston ride, he’d been discouraged that they’d
kept on with their Who-ville Xmas even though he’d grinched away
all their fun, but this time he was expecting something like this.
Watching these guys sell souvenirs at the funeral for the ride made
him feel pretty good this time around: their disloyalty had to be a
real morale-killer for those ride-operators.

The cops were getting twitchy, which made him grin. Twitchy cops
were a key ingredient for bad trouble. He reached behind him and
pulled an iced coffee from the cooler and cracked it, listening to
the hiss as the embedded CO2 cartridge forced bubbles through it.

Now here came a suit. He looked like a genuine mighty morphin’
power broker, which made Sammy worry, because a guy like that
hadn’t figured into his plans, but look at that; he was having a
huge fight with the eyebrow guy and now the eyebrow guy was running
away from him.

Getting the lawyers to agree to spring the budget to file in every
location where there was a ride had been tricky. Sammy had had to
fudge a little on his research, claim that they were bringing in
real money, tie it to the drop in numbers in Florida, and generally
do a song and dance, but it was all worth it. These guys clearly
didn’t know whether to shit or go blind.

Now eyebrow man was headed for the cop-cars and the entrance, and
there, oh yes, there it was. Five cars’ worth of goths, lugging
bags full of some kind of home-made or scavenged
horror-memorabilia, pulling up short at the entrance.

They piled out of their cars and started milling around, asking
questions. Some approached the cops, who seemed in no mood to chat.
The body-language could be read at 150 feet:
{\setlength{\parindent}{0pt}

Goth: But officer, I wanna get on this riiiiiide.

Cop: You sicken me.

Goth: All around me is gloom, gloom. Why can’t I go on my
riiiiiide?

Cop: I would like to arrest you and lock you up for being a weird,
sexually ambiguous melodramatic who’s dumb enough to hang around
out of doors, all in black, in \emph{Florida}.

Goth: Can I take your picture? I’m gonna put it on my blog and then
everyone will know what a meanie you are.

Cop: Yap yap yap, little bitch. You go on photographing me and
mouthing off, see how long it is before you’re in cuffs in the back
of this car.

Scumbag street-vendors: Ha ha ha, look at these goth kids mouthing
off to the law, that cop must have minuscule testicles!

Cop: Don’t make me angry, you wouldn’t like me when I’m angry.

Eyebrow guy: Um, can everyone just be nice? I’d prefer that this
all not go up in flames.

Scumbags, goths: Hurr hurr hurr, shuttup, look at those dumb cops,
ahahaha.

Cops: Grrrr.

Eyebrow: Oh, shit.}

Four more cars pulled up. Now the shoulder was getting really
crowded and freeway traffic was slowing to a crawl.

More goths piled out. Family cars approached the snarl, slowed,
then sped up again, not wanting to risk the craziness. Maybe some
of them would get on the fucking turnpike and drive up to Orlando,
where the real fun was.

The four-lane road was down to about a lane and a half, and milling
crowds from the shantytown and the arriving cars were clogging what
remained of the thoroughfare. Now goths were parking their cars way
back at the intersection and walking over, carrying the objects
they’d planned to sacrifice to the ride and smoking clove
cigarettes.

Sammy saw Death Waits before Death Waits turned his head, and so
Sammy had time to duck down before he was spotted. He giggled to
himself and chugged his coffee, crouched down below the window.

The situation was heating up now. Lots of people were asking
questions of the cops. People trying to drive through got shouted
at by the people in the road. Sometimes a goth would slam a fist
down on a hood and there’d be a little bit of back and forth. It
was a powder-keg, and Sammy decided to touch it off.

He swung his car out into the road and hit the horn and revved his
engine, driving through the crowd just a hair faster than was safe.
People slapped his car as it went by and he just leaned on the
horn, ploughing through, scattering people who knocked over
vendors’ tables and stepped on their wares.

In his rear-view, he saw the chaos begin. Someone threw a punch,
someone slipped, someone knocked over a table of infringing merch.
Wa-hoo! Party time!

He hit the next left, then pointed his car at the freeway. He
reached back and snagged another can of coffee and went to work on
it. As the can hissed open, he couldn’t help himself: he chuckled.
Then he laughed\dash{}a full, loud belly-laugh.

\begin{center}\rule{3in}{0.4pt}\end{center}

Perry watched it happen as though it were all a dream: The crowds
thickening. The cops getting out of their cars and putting their
hands on their belts. A distant siren. More people milling around,
hanging out in the middle of the road, like idiots, \emph{idiots}.
Then that jerk in the car\dash{}what the hell was he thinking, he was
going to kill someone!

And then it all exploded. There was a knot of fighting bodies over
by the tables, and the knot was getting bigger. The cops were
running for them, batons out, pepper-spray out. Perry shouted
something, but he couldn’t hear himself. In a second the crowd
noises had gone from friendly to an angry roar.

Perry spotted Suzanne watching it all through the viewfinder on her
phone, presumably streaming it live, then shouted again, an unheard
warning, as a combatant behind her swung wide and clocked her in
the head. She went down and he charged for her.

He’d just reached her when a noise went off that dropped him to his
knees. It was their antipersonnel sound-cannon, which meant that
Lester was around here somewhere. The sound was a physical thing,
it made his bowels loose and made his head ring like a gong.
Thought was impossible. Everything was impossible except curling up
and wrapping your hands around your head.

Painfully, he raised his head and opened his eyes. All around him,
people were on their knees. The cops, though, had put giant
industrial earmuffs on, the kind of thing you saw jackhammer
operators wearing. They were moving rapidly toward\ldots{} Lester who
was in a pickup truck with the AP horn stuck in the cargo bed,
wired into the cigarette lighter. They had guns drawn and Lester
was looking at them wide-eyed, hands in the air.

Their mouths were moving, but whatever they were saying was
inaudible. Perry took his phone out of his pocket and aimed it at
them. He couldn’t move without spooking them and possibly knocking
himself out from the sound, but he could rodneyking them as they
advanced on Lester. He could practically read Lester’s thoughts:
\emph{If I move to switch this off, they’ll shoot me dead.}

The cops closed on Lester and then the sour old male cop was up in
the bed and he had Lester by the collar, throwing him to the
ground, pointing his gun. His partner moved quickly and efficiently
around the bed, eventually figuring out how to unplug the horn. The
silence rang in his head. He couldn’t hear anything except a
dog-whistle whine from his abused eardrums. Around him, people
moved sluggishly, painfully.

He got to his feet as quick as he could and drunk-walked to the
truck. Lester was already in plastic cuffs and leg-restraints, and
the big, dead-eyed cop was watching an armored police bus roll
toward them in the eerie silence of their collective deafness.

Perry managed to switch his phone over to streaming, so that it was
uploading everything instead of recording it locally. He faded back
behind some of the cars for cover and kept rolling as the riot bus
disgorged a flying squadron of helmeted cops who began to
methodically and savagely grab, cuff, and toss the groaning crowd
lying flat on the ground. He wanted to add narration, but he didn’t
trust himself to whisper, since he couldn’t hear his own voice.

A hand came down on his shoulder and he jumped, squeaked, and fell
into a defensive pose, waiting for the truncheon to hit him, but it
was Suzanne, grim faced, pointing her own phone. She had a
laminated press-pass out in her free hand and was holding it up
beside her head like a talisman. She pointed off down the road,
where some of the goth kids who’d just been arriving when things
went down were more ambulatory, having been somewhat shielded from
the noise. They were running and being chased by cops. She made a
little scooting gesture and Perry understood that she meant he
should be following them, getting the video. He sucked in a big
breath and nodded once and set off. She gave his hand a firm
squeeze and he felt that her palms were slick with sweat.

He kept low and moved slow, keeping the viewfinder up so that he
could keep the melee in shot. He hoped like hell that someone
watching this online would spring for his bail.

Miraculously, he reached the outlier skirmish without being
spotted. He recorded the cops taking the goths down, cuffing them,
and hooding one kid who was thrashing like a fish on a hook. It
seemed that he would never be spotted. He crept forward, slowly,
slowly, trying to feel invisible and unnoticed, trying to project
it.

It worked. He was getting incredible footage. He was practically on
top of the cops before anyone noticed him. Then there was a shout
and a hand grabbed for his phone and the spell was broken. Suddenly
his heart was thundering, his pulse pounding in his ears.

He turned on his heel and ran. A mad giggle welled up in his chest.
His phone was still streaming, presumably showing wild, nauseous
shots of the landscape swinging past as he pumped his arm. He was
headed for the ride, for the rear entrance, where he knew he could
take cover. He felt the footsteps thud behind him, dimly heard the
shouts\dash{}but his temporary deafness drowned out the words.

He had his fob out before he reached the doors and he badged in,
banging the fob over the touch-plate an instant before slamming
into the crash-bar and the doors swung open. He waited in agitation
for the doors to hiss shut slowly after him and then it was the
gloom of the inside of the ride, dark in his sun-adjusted
eyesight.

It was only when the doors shivered behind him that he realized
what he’d just done. They’d break in and come and get him, and in
the process, they’d destroy the ride, for spite. His eyes were
adjusting to the gloom now and he made out the familiar/unfamiliar
shapes of the dioramas, now black and lacy with goth memorabilia.
This place gave him calm and joy. He would keep them from
destroying it.

He set his phone down on the floor, propped against a plaster skull
so that the doorway was in the shot. He walked to the door and
shouted as loud as he could, his voice inaudible in his own ears.
“I’m coming out now!” he shouted. “I’m opening the doors!”

He waited for a two-count, then reached for the lock. He turned it
and let the door crash open as two cops in riot-visors came
through, pepper-spray at the fore. He was down on the ground,
writhing and clawing at his face in an instant, and the phone
caught it all.

\begin{center}\rule{3in}{0.4pt}\end{center}

All Perry wanted was for someone to cut the plastic cuffs off so he
could scrub at his eyes, though he knew that would only make it
worse. The riot-bus sounded like an orgy, moaning and groaning with
dozens of voices every time the bus jounced over a pothole.

Perry was on the floor of the bus, next to a kid\dash{}judging from the
voice\dash{}who cursed steadily the whole way along. One hard jounce made
their heads connect and they both cussed, then apologized to one
another, then laughed a little.

“My name’s Perry.” His voice sounded like he was underwater, but he
could hear. The pepper spray seemed to have cleared out his sinuses
and given him back some of his hearing.

“I’m Death Waits.” He said it without any drama. Perry wasn’t sure
if he’d heard right. He supposed he had. Goth kids.

“Nice to meet you.”

“Likewise.” Their heads were banged together again. They laughed
and cursed.

“Christ my face hurts,” Perry said.

“I’m not surprised. You look like a tomato.”

“You can see?”

“Lucky me, yup. I got a pretty good couple of whacks on the back
and shoulders once I was down, but no gas.”

“Lucky you all right.”

“I’m more pissed that I lost the tombstone I brought down. It was a
real rarity, and it was hard to get, too. I bet it got tromped.”

“Tombstone, huh?”

“From the Graveyard Walk at Disney. They tore it down last week.”

“And you were bringing it to add it to the ride?”

“Sure\dash{}that’s where it belongs.”

Perry’s face still burned, but the pain was lessening. Before it
had been like his face was on fire. Now it was like a million fire
ants biting him. He tried to put it out of his mind by
concentrating on the pain in his wrists where the plastic straps
were cutting into him.

“Why?”

There was a long silence. “Has to go somewhere. Better there than
in a vault or in the trash.”

“How about selling it to a collector?”

“You know, it never occurred to me. It means too much to go to a
collector.”

“The tombstone means too much?”

“I know it sounds stupid, but it’s true. You heard that Disney’s
tearing out all the goth stuff? Fantasyland meant a lot to some of
us.”

“You didn’t feel like it was, what, co-opting you?”

“Dude, you can buy goth clothes at a chain of mall-stores. We’re
all over the mainstream/non-mainstream fight. If Disney wants to
put together a goth homeland, that’s all right with me. And that
ride, it was the best place to remember it. You know that it got
copied over every night to other rides around the country? So all
the people who loved the old Disney could be part of the memorial,
even if they couldn’t come to Florida. We had the idea last week
and everyone loved it.”

“So you were putting stuff from Disney rides into my ride?”

“Your ride?”

“Well, I built it.”

“No fucking way.”

“Way.” He smiled and that made his face hurt.

“Dude, that is the coolest thing ever. You built that? How did\dash{}How
do you become the kind of person who can build one of those things?
I’m out of work and trying to figure out what to do next.”

“Well, you could join one of the co-ops that’s building the other
rides.”

“Sure, I guess. But I want to be the kind of person who invents the
idea of making something like that. Did you get an electrical
engineering degree or something?”

“Just picked it up as I went along. You could do the same, I’m
sure. But hang on a sec\dash{}you were putting stuff from Disney rides
into my ride?”

“Well, yeah. But it was stuff they’d torn down.”

Perry’s eyes streamed. This couldn’t be a coincidence, stuff from
Disney rides showing up in his ride and the cops turning up to
enforce a court order Disney got. But he couldn’t blame this kid,
who sounded like a real puppy-dog.

“Wait, you don’t think the cops were there because\dash{}”

“Probably. No hard feelings though. I might have done the same in
your shoes.”

“Oh shit, I am \emph{so sorry}. I didn’t think it through at
\emph{all}, I can see that now. Of course they’d come after you.
They must totally hate you. I used to work there, they just hate
anything that takes a Florida tourist dollar. It’s why they built
the monorail extension to Orlando airport\dash{}to make sure that from
the moment you get off the plane, you don’t spend a nickel on
anything that they don’t sell you. I used to think it was cool,
because they built such great stuff, but then they went after the
new Fantasyland\dash{}”

“You can’t be a citizen of a themepark,” Perry said.

The kid barked a laugh. “Man, how true is \emph{that}? You’ve
nailed it, pal.”

Perry managed to crack an eye, painfully, and catch a blurry look
at the kid: a black Edward Scissorhands dandelion clock of hair,
eyeliner, frock-coat\dash{}but a baby-face with cheeks you could probably
see from the back of his head. About as threatening as a Smurf.
Perry felt a sudden, delayed rush of anger. How \emph{dare} they
beat up kids like this “Death Waits”\dash{}all he wanted to do was ride a
goddamned ride! He wasn’t a criminal, wasn’t out rolling old ladies
or releasing malicious bioorganisms on the beach!

The bus turned a sharp corner and their heads banged together
again. They groaned and then the doors were being opened and Perry
squeezed his eyes shut again.

Rough hands seized him and marched him into the station house. The
crowd susurrations were liquid in his screwed-up ears. He couldn’t
smell or see, either. He felt like he was in some kind of terrible
sensory deprivation nightmare, and it made him jerky, so whenever a
hand took him and guided him to another station in the check-in
process (his wallet lifted from his pocket, his cheek swabbed, his
fingers pressed against a fingerprint scanner) he flinched
involuntarily. The hands grew rougher and more insistent. At one
point, someone peeled open his swollen eyelid, a feeling like being
stabbed in the eye, and his retina was scanned. He screamed and
heard laughter, distant through his throbbing eardrums.

It galvanized him. He forced his eyes open, glaring at the cops
around him. Mostly they were Florida crackers, middle-aged guys
with dead-eyed expressions of impersonal malevolence. There was a
tiny smattering of brown faces and women’s faces, but they were but
a sprinkling when compared to the dominant somatype of Florida
law.

The next time someone grabbed him to shove him towards the next
station on this quest, he jerked his arm away and sat down. He’d
seen protestors do this before, and knew that it was hard to move a
sitting man expeditiously or with dignity. Hands seized him by the
arms, and he flailed until he was free, remaining firmly seated.
The laughter was turning to anger now. Beside him, someone else
sat. Death Waits, looking white-faced and round-eyed. More people
hit the floor. A billy-club was shoved under his arm, which was
then twisted into an agonizing position. He was suddenly ready to
give up the fight and go along, but he couldn’t get to his feet
fast enough. With a sickening \emph{crack}, his arm broke. He had a
moment’s lucid awareness that a bone had broken in his body, and
then the pain was on him and he choked out a shout, then a louder
one, and then everything went dark.

\begin{center}\rule{3in}{0.4pt}\end{center}

As it turned out, his prison infirmary time didn’t last long at
all. Kettlewell had faded fast from the riot, headed back to the
guesthouse and got the lawyers on the phone. He’d shown them the
stream off of Perry’s phone and they were in front of a judge
before Perry reached the jail.

Perry was led out of the infirmary with his arm in a sling. His
face was still painfully swollen, and he’d managed to turn an ankle
as well. At least his hearing was coming back.

Kettlewell took Perry’s good arm and gave him a soulful hug that
embarrassed him. Kettlewell led him outside, to where a big cab was
waiting. In it were the family Kettlewell, Lester, and Suzanne.
Lester had a couple bandages taped to his face and when Suzanne
smiled, he saw her lips were stained red and one of her front teeth
had been knocked out.

He managed a brave smile. “Looks like you guys got the full
treatment, huh?”

Suzanne squeezed his hand. “Nothing that can’t be fixed.” Ada and
Pascal looked goggle-eyed at them. Ada was popping Korean
lotus-bean walnut cakes into her mouth from a greasy paper bag, and
she offered them silently to Perry, who took one just to be polite,
but found after the first bite that he wasn’t really hungry after
all.

Kettlewell and Perry fought about what to do next, but Kettlewell
prevailed. He took them to a private doctor who photographed them
and examined them and x-rayed them, documenting everything while
Ada Kettlewell played camera-woman with her phone, videoing it
all.

“I don’t think suing the police is going to help, Landon,” Perry
said. Suzanne nodded vigorously. The three victims were in paper
examining gowns, and the Kettlewells were still in street clothes,
which gave them a real advantage in the self-confidence
department.

“It’ll help if we cash out a big settlement\dash{}it’ll bankroll our
defense against the Disney trademark claims. IP lawyers charge more
than God per hour. I got the injunction lifted, but we’re still
going to have to go to court, and that’s not going to be cheap.”

It needled Perry\dash{}he didn’t like the idea of being embroiled in the
legal system in the first place, and while he could grudgingly
admit a certain elegance in using cash settlements from the law to
fund their defense in court, the whole business made him squirm.

Eva sat down beside him. “I can tell this sucks for you, Perry.”
Ada whispered the word \emph{sucks} and giggled, and Eva rolled her
eyes. “But there’s fifty people we \emph{didn’t} bail out in there,
who are all of them going to have to figure out their own way
through the legal system. You can’t run a business if your
customers risk a solid beating and jail time just for showing up.”

\emph{I don’t want to run a business,} he thought, but he knew that
was petulant. He was the man with the roll of bills down his pants.
“There are fifty people still in the slam?”

Kettlewell nodded. Suzanne had her camera out and she was
recording. It had been a long time since Perry had really felt the
camera’s eye on him. It was one thing to be recorded by some
friends for remembrance, but now Suzanne’s camera seemed like the
gaze of posterity. He needed to rise to it, he knew.

“Let’s get them out. All of them.”

Kettlewell raised his eyebrows. “And how do you plan on doing
that?”

“We’ll charge it to the business,” Perry said. Lester chuckled and
gave him a thump on the back. “It’s a legit expense\dash{}these are our
\emph{customers} after all.”

Kettlewell shook his head at all of them, then he left the doctor’s
office. He already had his phone stuck to his head and was talking
with the lawyer before he got out of earshot.

Perry and Lester and Suzanne and Eva exchanged mischievous glances,
grinning with unexpected delight. Pascal, riding on Eva’s hip, woke
up and started crying and Eva handed him to Lester while she went
for the diaper bag.

“Here we go again,” Lester said, wrinkling his nose and holding the
wailing Pascal at arm’s length.

Suzanne got it all with her phone, then she flipped it shut and
gave Lester a hard kiss on the cheek.

“Fatherhood would suit you,” she said.

He went bright red. “Don’t you get any ideas,” he said. Suzanne
laughed and skipped away, looking all of ten.

Perry felt huge. Larger than life. The adventure was beginning
anew, with these good people whom he loved like family. He had the
work and the people, and who needed anything more.

It was a feeling that lasted all the way back to the ride.

But then he surveyed the ride itself and found it in utter ruins,
far worse than it had been left when he’d been dragged out of it.
Every single exhibit was smashed, strewn here and there.

He couldn’t believe it. He brought up the clean-up lights, flooding
the place, and then he saw what he’d missed at first: the smashed
exhibits were not smashed exhibits\dash{}they were \emph{replicas} of
smashed exhibits. At every ride in the country, police had gone in
smashing, and every other ride in the country had faithfully
reproduced the damage, dutiful printers churning out replica
detritus and dutiful robots placing it with micrometer precision.

He began to laugh and couldn’t stop. Lester came in and immediately
got the joke and laughed along with him. They managed to stop
laughing just long enough to explain it to Suzanne and Kettlewell,
who didn’t find it nearly as funny as they did. Suzanne took
pictures.

Finally he got down to business, opening the change-log and rolling
the ride back through the “revisions” to its unsmashed state. It
would take the robots a long time to set everything right again,
but at least he didn’t have to oversee it.

Instead, he tracked down as many of the market-stall vendors as he
could locate in the shantytown and made sure they were all
right\dash{}they were, though they’d lost some inventory. He comped them
all a month’s rent and made sure they knew that steps were being
taken to keep it from happening again. He knew that they could make
nearly as much money selling from a roadside or online, and he
wanted to keep them happy. Besides, it wasn’t their fault.

He was exhausted and his arm was really starting to gripe him. He
found himself stopping in the street every few steps to rub his
eyes and force himself on. Francis came on him when he was like
that, leaning against the prefab concrete wall of one of the tall,
twisty shanties, and he took Perry’s car-keys away and drove him
home. Perry was in too much of a state by the time he got there to
think about how Francis would get back\dash{}he was already lying in bed
before it occurred to him that the old man with the gimpy leg
probably walked the ten miles home.

He woke up later that night to sex noises from Lester’s room and he
recognized Suzanne’s voice. Later, he woke again to hear the tail
end of another argument between Lester and Suzanne, and then
Suzanne storming out of the apartment. \emph{Oh, goody}, he
thought. He lay on his back, trying to find sleep again\dash{}the clock
said 3AM\dash{}and found thoughts of Hilda drifting unbidden into his
mind.

It was silly\dash{}they’d only spent one night together, and he had to
admit that as great as the sex had been, he’d had better with the
fatkins gymnasts you could pick up down on South Beach. She was too
young for him. She lived in \emph{Wisconsin}. But there were
touches in the ride that had originated with her instantiation\dash{}he
looked over the logs every now and then\dash{}and he found himself
contemplating them with sentimental smiles.

He fell asleep again and only woke when he rolled over on his bad
arm and yelped himself awake. The smell of waffles, bacon and eggs
was strong in the apartment. He couldn’t be bothered to figure out
how to shower with his cast on, so he pulled on a pair of shorts
and let himself into the living room.

Lester was at the stove, cooking up half a pig and pouring maple
batter into the waffle-iron. He waved a spatula at him and pointed
out at the terrace. Perry stepped out and saw Suzanne and Tjan and
Tjan’s little kids\dash{}what were their names? Lyenitchka and the little
boy? Man, the whole family was here.

“Your arm is broken,” Lyenitchka said, pointing at him.

Perry nodded gravely. “That’s true. Want to sign my cast?” He was
pretty sure that he had a grease-pencil that would mark the
surface, though the hospital had sworn that it would shed dirt, ink
and anything else he threw at it.

She nodded vigorously. Tjan looked him over and gave a little wave,
then Perry went back into the living room and asked his computer to
find the grease-pencil.

“Thought you’d be busy in Boston,” he said, while Lyenitchka
painstakingly spelled out her name, going over the letters to get
them to show up dark\dash{}the cast surface really didn’t want to suck up
any tint.

“Boston came out OK. We had lawyers on tap at the start and the
vibe was cool. I incorporated there, so it was easier than you guys
had it. But some of the others were hit bad, like San Francisco and
Madison.”

“\emph{Madison}?” Perry was alarmed by how alarmed he sounded.

“Mass arrests. The cops there are real hard-cases, with all this
antipersonnel gear left over from the stem-cell riots.”

Perry jerked and spoiled Lyenitchka’s writing. He patted her head
and set his arm back down where she could get at it. He groaned.

“They’re mostly still in. We’re trying to get them bailed out, but
the judge at the arraignment set bail pretty high.”

“I’ll post it,” Perry said. “I can put up my savings or
something\ldots{}”

Tjan looked uncomfortable. “Perry, there are 250 people in the
lockup in Wisconsin. Some of them are going to skip out, it’s
nearly a certainty. If you bail them all out, you’ll go broke. I
mean, it’s good to see you and I’m sorry you got hurt and all
respect, but don’t be an idiot.”

Perry felt himself go belligerent. His hands went into fists and
his broken wing protested. That brought him back to reality. He
forced himself to smile.

“There’s a girl in Madison, I want to make sure she’s OK.”

Tjan and Suzanne stared at him for a second. Then Lester clapped
him across the back from behind him, startling him and making him
squeak. “Big fella!” he crowed. “I should have known.”

Perry gave him a mock glare. “\emph{You} have no right to say
\emph{anything} on this score.” He darted a glance at Suzanne and
saw that she was blushing. Tjan took this in and nodded, as though
his suspicions had just been confirmed.

“Fair enough,” Tjan said. “Let’s make some inquiries about the
young lady. What’s her name?”

“Hilda Hammersen.”

Tjan’s eyebrows shot up. “Hilda \emph{Hammersen}? From the mailing
lists? \emph{That} Hilda?”

Hilda was the queen of the mailing lists\dash{}brash, quick, and
argumentative, but never the kind of person who started flamewars.
Hilda’s arguments were hot and fast, and she always won. Perry had
watched her admiringly from the sidelines, only weighing in
occasionally, but he seemed to remember now that she’d taken Tjan
to the cleaners once on an issue of protocol resolution.

“That’s the one,” Perry said.

“I always pictured her as being about fifty, with a machete between
her teeth,” Lester said. “No offense.”

“Lyenitchka, go get my phone from my bed-stand,” Perry said,
patting the girl on the shoulder. When she got back he went through
his photos of Hilda with them.

Lester made a wolf-whistle and Suzanne punched him in the shoulder
and took the phone away.

“She’s very pretty,” Suzanne said, disapprovingly. “And very
young.”

“Oh yes, dating younger people is \emph{so} sleazy,” Lester said
with a chuckle. Suzanne squirmed and even Perry had to laugh.

“Guys, here it is. I need to spring Hilda, and we need to do
something about all those customers and supporters and so on who
went to jail today. We need to fight all the injunctions\dash{}all of
them\dash{}and prevent them from recurring.”

“And we need to eat breakfast, which is ready,” Lester said,
gesturing at the table behind him, which was stacked high with
waffles, sausages, eggs, toast, and pitchers of juice and carafes
of coffee.

Lyenitchka and Sasha looked at each other and ran to the table,
taking seats next to one another. The adults followed and soon they
were eating. Perry managed a waffle and a sausage, but then he went
off to his room. Hilda was in the slam in Madison, and who the hell
knew what the antipersonnel stuff the Madison cops used had done to
her. He just wanted to get on a fucking plane and \emph{go there}.

Halfway through his shower, he knew that that was what he was going
to do. He packed a shoulder-bag, took a couple more painkillers,
and walked out into the living room.

“Guys, I’m going to Madison. I’ll be back in a day or two. We’ll
work everything out over the phone, OK?”

Lester and Suzanne came over to him. “You going to be OK, buddy?”
Lester said.

“I’ll be fine,” he said.

“We can spring her from here,” Tjan said. “We have the Internet,
you know.”

“I know,” Perry said. “You do that, OK? And tell her I’ll be there
as soon as I can.”

The security at the airport went bonkers over him. The perfect
storm: a fresh arrest, a suspicious cast, and a ticket bought with
cash. He missed the first two flights to Chicago, but by
mid-afternoon he was landing at O’Hare and submitting to an interim
screening procedure before boarding for Madison. His phone rang in
the middle of the screening, and the wrinkly old TSA goon-lady
primly informed him that he might as well get that since once the
phone rings, they have to start the procedure over again.

“Tjan,” he said.

“They can’t spring her today. Tomorrow, though.”

He closed his eyes and shut out the TSA goon. She had a huge
bouffant of copper hair, and a midwesterner’s sense of
proportionality when it came to eye-shadow and rouge. She was the
kind of woman who could call you “honey” and make it sound like
“Islamofascist faggot.”

“Why not, Tjan?”

There was a pause. “She’s in the infirmary and they won’t release
her until tomorrow.”

“Infirmary.”

“Nothing serious\dash{}she took a knock on the head and they want to hold
her for observation.”

He pictured a copper’s electrified billy-club coming down on
shining blond hair and felt like throwing up.

“Perry? Buddy. She’s OK, really. I had our lawyer visit her in the
prison infirmary and she swears she looks great. The lawyer’s name
is Candice\dash{}take a cab to her office from the airport. OK?”

“Why is she in the prison infirmary, Tjan? Why can’t she be moved
to a real hospital?”

“It’s just a liability thing. The police don’t want to risk the
suit if she goes complicated on them between hospitals.”

“Jesus.”

“Seriously, she’s fine. We’ve got a good lawyer on the scene.”

But Perry had a bad feeling. The TSA goon picked up on it and gave
him a little bit of extra attention. Acting nervous or agitated in
an airport was a one-way ticket to a cavity search.

But then he was lifting off and headed for Madison, and though the
time crawled on the one-hour flight, it was, after all, only an
hour. He even napped briefly, though a sky marshall woke him
shortly after for a random bag-search. His fellow passengers\dash{}badly
dressed midwesterners and a couple of hipster students\dash{}all turned
their bags out in the cramped cabin and then got back in their
seats for the landing.

Perry had meant to phone in a car reservation at O’Hare, but the
extra search had eaten up the time he’d allocated for it, and now
all the rental counters were sold out. Reluctantly, he got into a
taxi and asked the driver to take him to the office of the lawyers
that Tjan had hired.

The cabbie was a young African kid with a shaved head. He had a
dent in one temple and more dents in one of his wrists, visible as
he let his long hands drape over the steering wheel.

“I know where it is,” he said when Perry gave him the address.
“That lawyer, she is very good. She helped me with the Homeland
Security.”

The kid was young, 21 or 22, with a studious air, despite his old
injuries. He reminded Perry of the shantytowners, people who didn’t
always get medical attention for their ailments, people who were
often missing a tooth or two, who had mysterious lumps from
badly-set bones or scars or funny eyebrows like his. The
midwesterners on the plane had been flawless as action-figures, but
Perry’s friends and this African kid looked like something carved
out of coal and chalk.

Perry was one big jitter from the trip and the coffee and the pills
for his arm, but he found himself drawn into conversation as they
whizzed past the fields and malls, the factories and office-parks.

“I’m from Gulu, in Uganda. There has been civil war there for
thirty five years. I studied chemical engineering through the
African Virtual University wiki-program, and qualified for a Chavez
scholarship here in Madison.” His accent was light but exotic, the
African rolling of the Rs, the British-sounding vowel-shifts. “But
the Homeland Security didn’t want to renew my visa last year. They
said I had financial irregularities. I was paypalling to a friend
in Kampala who withdrew it in shillings and sent it to my family in
giros. Homeland Security said that I was \emph{money laundering}. I
thought I’d be sent away or put in prison, but Ms Candice wrote
them a letter and they vanished.” He snapped his long, knuckly
fingers for emphasis.

“Jesus. Well, that’s good. She’s going to help me get my girlfriend
out of jail.” Perry realized he’d just called Hilda his girlfriend,
which would be news to her, but there it was.

“You don’t need to worry. She’ll get your friend free.”

Perry nodded and tried to close his eyes and relax. He couldn’t.
What the hell had happened to the world. It had seemed so exciting
when his father was bringing home new shapes he’d spun off his
CAD/CAM rig. When Perry had started to trade designs with people,
to effortlessly find people on the net who wanted to collaborate
with him and vice-versa. When Perry had started a business making
cool art out of free junk and selling it off an Internet connection
that was likewise free.

Free, free, free. No need to talk to a government, or grovel for a
curator, or put up with an agent or a boss. He’d just assumed all
along that he’d end up living in a world where all those parasites
and bullies and middlemen would just blow away in the wind.

But they’d all found jobs in the new world. They weren’t needed
anymore, but that didn’t mean that they went away. Now they were
wanding him in airports and suing him for trademark infringement
and busting his girlfriend and breaking his arm and giving hassle
to this poor African kid who’d taught himself to be an engineer
with a ferchrissakes \emph{wiki}.

He dry-swallowed another pain-killer and then remembered that
taking the pills meant he wouldn’t be able to get a drink, which he
could sure as shit use.

“My name’s Perry,” he said.

“Richard,” the driver said. “We’re almost there, Perry. I wish you
the very best of luck.”

“You too,” he said. The driver shook his hand warmly after getting
his luggage out of the trunk, a limp handshake by North American
standards, but gentle and friendly nonetheless. His dented wrist
flexed oddly as the half-knit bones there moved.

The lawyer’s office was not what Perry was expecting. It looked
like someone’s living room, with a couple of overstuffed sofas, a
dozing cat, and the lawyer, Candice, who was a young-looking woman
in her mid-twenties. She dressed in jeans and an oversized UW
sweatshirt, with a laptop perched on one knee. She had a friendly,
open face, framed with lots of curly brown hair.

“You must be Perry,” she said, setting the laptop down and giving
him an unexpected hug. “That was from Hilda. I saw her a couple
hours ago. She was very adamant that I pass it on to you.”

“Nice to meet you," he said, accepting a cup of tea from an
insulated jug on a cardboard side-board. “Hilda is all right?”

“Sit down,” the lawyer said.

Perry’s stomach turned a somersault. “Hilda’s all right?”

“Sit.”

Perry sat.

“She was gassed with a neurotoxin that has given her a temporary
but severe form of Parkinson’s disease. Normally it just renders
people immobile, but one in a million has a reaction like this.
It’s just bad luck that Hilda was one of them.”

“She was \emph{gassed}?”

“They all were. There was a hell of a fight, as I understand it. It
really looks like it was the cops’ fault. Someone told them that
there were printed guns in the ride-location and they used extreme
and disproportionate force.”

“I see,” Perry said. His blood whooshed in his ears. Printed guns?
No frigging way. Sure, ray-guns in some of the exhibits. But
nothing that fired anything. He felt tears begin to stream down his
face. The lawyer moved to his sofa and put her arm around his
shoulders.

“She’s going to be fine,” Candice said. “The Parkinson’s is rare,
but it goes away in 100 percent of the the cases where it occurs.
What this means is that we’ve got an amazing chance of taking a
huge bite out of the local law that we can use to fund future
defense. Tjan told me that that’s the strategy and I think it’s
sound. Plus the harder we hit the law today, the more reluctant
they’ll be to rush off half-cocked the next time someone trumps up
a BS trademark claim. It could be much worse, Perry. There’s a kid
who lost an eye to a rubber bullet.”

Perry fisted the tears away. “Let’s go get her,” he said.

“They say she shouldn’t be moved,” Candice said.

“What does our doctor say?”

“I phoned a couple MDs this afternoon and got conflicting stories.
Everyone agrees that not moving her is safer than moving her,
though. The only disagreement is about how dangerous it would be to
move her.”

“Let’s go see her, then.”

“That we can do.”

Perry had trouble with the search at the prison hospital. His cast
and their scanners didn’t get along and they couldn’t be satisfied
with a hand search. For a couple minutes it looked like he was
going to be kept out, but Candice\dash{}who had changed into a power-suit
before they left the office\dash{}put on a stern voice and demanded to
speak to the duty sergeant, and then to his commanding officer, and
in ten minutes, they were on the hospital ward, where the
metal-railed beds had prisoners handcuffed to them.

“Hilda?” She looked sunken and sick, her face slack and her jaw
askew. Her eyes opened and rolled crazily, they focused on him. Her
body shook through two waves of tremors before she was able to
raise a shaking hand toward him, trailing IV tubes. She was trying
to say his name, but it wouldn’t come out, just a series of plosive
Ps.

But then he took her hand and felt its fine warmth, the calluses he
remembered from all those months ago, and he felt better. Actually
better. Felt some peace for the first time in a long time.

“Hello, Hilda,” he said, and he was smiling so broadly his face
hurt, and tears were running down his cheeks and dripping off his
nose and running into his mouth. She was weeping, too, her head
vibrating like a bobble-doll. He bent over her and took her head in
his hands, burying them in her thick blond hair, and kissed her on
the lips. She shook under him, but she kissed him back, he could
feel her lips move on his.

They kissed for a long time. He subconsciously took note of the
fact that Candice had moved back, giving them some privacy. When
the kiss broke, he had an overwhelming desire to tell her he loved
her, but they hadn’t taken that step yet, and maybe a prison
hospital bed wasn’t the right place to make pronouncements of
love.

“I love you,” he said softly, in her ear, kissing the lobe. “I love
you, Hilda.”

She cried harder, and made choking sobs. He hugged her as hard as
he dared. Candice came back and stood by them.

“They think that she’ll be better in the morning. She’s already
much better off than she was just a couple hours ago. Sleep’s the
only thing for it. They’ve got her mildly sedated, too.”

Hilda smelled like he remembered, the undersmell beneath her
shampoo and the chemicals clinging to her hair. It took him back to
their night together, and he stroked her cheek.

“I’ll stay here,” he said.

“I don’t think that they’re going to let you do that, Perry. This
is a prison, not a hospital.”

“I’ll stay here,” he said again. “Just make it happen, OK? We’re
going to sue them into a smoking hole, right? That’s got to give us
some leverage. I’ll stay here.”

She sighed and looked at him for a long time, but he wouldn’t take
his eyes off of Hilda. His broken arm throbbed and he was out of
painkillers. They’d have painkillers here.

Candice went away, and then, a while later, she came back. “Stay
here,” she said. “I’ll come and get you in the morning.”

“Thanks,” he said. Then he thought that he should say something
more, and he turned around, but the lawyer had gone.

He fell asleep holding Hilda’s hand with his good hand, and woke up
with an unbelievable pain in his broken arm and couldn’t find a
nurse. He bit down on the pain and spent a long watch that night
staring at Hilda, thinking of all she meant to him and how weird it
was that she meant so much when they’d had so brief a moment
together. They hadn’t let him bring his phone in, or he’d have
taken a thousand pictures of her face in repose. He nodded off
again.

He woke when she did, stirring in her bed. Her movements were still
weak and feeble, but they lacked the uncontrolled tremors of the
night before. He leaned in for a kiss, not caring about his sour
breath or hers.

“Good morning,” he said.

“Morning, gorgeous,” she said, and took him in a soft, sleepy hug.

Candice sprung them and took them across town to her doctor, a
young man who took great care in examining Hilda, explaining
patiently which fluids he was drawing and which tests he planned on
running on them. Perry had noticed that midwesterners came in two
flavors: big Scandinavian Aryans with giant shoulders and easy
smiles, and exchange students and immigrants in varying shades of
brown, who looked hurt and bent alongside of the natives\dash{}looked
like the people he knew from back home, people who didn’t have
ready access to medical care or good nutrition in their formative
years.

The doctor was Vietnamese, but he was at least a couple generations
in, judging by his accent, and he had the same midwestern smile and
seemed big and bulky compared with the Vietnamese people Perry knew
in Florida. He watched the man peer intently at a screen after
taping some electrodes to Hilda’s head, and felt like he’d come to
some land of Norse giants.

The doctor eventually told Hilda to go home and rest, and she
promised she would. Perry and she got into the back of Candice’s
car and cuddled up to one another, dozing. It wasn’t until Perry
got back with her to her apartment\dash{}every stick of furniture made
from clever cardboard\dash{}and emptied out his pockets that he
remembered to switch his phone on again.

He was down to his boxers and she was in cotton PJs with sexy
cowgirls printed on them, and when he powered the phone up, it went
bonkers, lighting up like a Christmas tree, vibrating, and making
urgent bleats.

“Shit,” he said, and began to sort through the alerts while his
back and neck muscles tightened. He sat on the edge of the bed and
prodded at the phone with his right hand, holding it awkwardly in
his left hand, trying to work around the cast. Hilda took the phone
and held it for him so he could work more freely and they both read
what was going on.

A second round of lawsuits had been filed that night, and the
injunctions had been reinstated. The story about the rides being a
source of printed arms and munitions had spread, and in San
Francisco the ride had been taken apart by Homeland Security bomb
robots that had detonated several key pieces of equipment. Three of
the San Francisco ride-crew ended up in the hospital after clashes
with overreacting cops.

Hilda nodded and took the phone from him and set it down.

“Right, what’s the game-plan?”

“How should I know?” Perry said. He could hear the whine in his
voice. “I just build stuff. Tjan and Candice say that they think we
can sue the cops over the brutality and use the money to fund legal
defenses, but Disney’s denial-of-service attacking us in the
courtroom. They’re also getting all this destruction dealt to us by
the cops.”

“You know how you eat an elephant? One bite at a time. Let’s break
this down into small component pieces and work on solutions to
them, then call up the troops and let them know what’s going on.
I’ll get a conference call set up while we chat.”

She was still moving slowly and weakly, and he tried to get her to
put down her laptop and rest, but she wasn’t having any of it.

And so they worked, dividing the problem up into manageable pieces:
incorporating a nonprofit co-op, writing the by-laws, getting the
word out through the press, re-opening the rides, putting together
scrapbooks of the carnage wrought.

It all seemed do-able once it was reduced to its component parts.
Perry put it all online and then conferenced Tjan and Kettlewell
in.

“Perry, do you think it’s a good idea to tell our enemies how we
plan to respond to them?”

Hilda shook her head and put a hand on Perry’s good arm to calm him
down before he answered Kettlewell. “That’s how we do it over on
our side. Their side is all about secrecy. Our side trades the
advantage of surprise for the advantage of openness. You watch\dash{}by
tonight we’ll have by-laws drafted, press-releases, exhaustive
documentation. You watch.”

On the screen, Lester’s face suddenly hove into view, fish-eye
distorted by his proximity to the lens. Hilda gave an amused squeak
and pulled back.

“So that’s Yoko, huh?” Lester said, grinning. “Cute! Listen guys,
don’t let these suits talk you out of what you’re doing. This is
the right thing. I’m on all the message boards and stuff and
they’re all champing to do something for real.”

“Yoko?” Hilda said. She raised an adorable eyebrow.

“Just a figure of speech,” Lester said. “I’m Lester. You must be
Hilda. Perry’s told us practically nothing about you, which is
probably a sign of something or other.”

Hilda regarded Perry with mock coolness. “Oh really?”

“Lester,” Perry said. “I love you like a brother. Shut the fuck up
already.”

Lester made a little whipping motion. Suddenly he was gone from the
picture, and they saw Suzanne pulling him away by one ear. Hilda
snorted. “I like her,” she said. Suzanne gave them a wave and Tjan
and Kettlewell came back into frame.

They made their goodbyes and hung up. Now Hilda and Perry were
alone, together, in her bedroom, laptops shut, day done\dash{}though it
was hardly gone noon\dash{}and the silence stretched.

“Thanks for coming, Perry,” she said.

“I\dash{}” He broke off. He didn’t know what to say. They had only known
each other for a day, only had a one-night stand. She probably
thought that he was a giant creep. “I was worried.” he said. “Um.
You should probably rest up some more, right?”

He got up and headed for the door.

“Where do you think you’re going?” she said.

“Figured I’d let you rest,” he said with a half-shrug.

“Get in this bed this instant, young man,” she said, slapping the
bed beside her. “And get those stinky clothes off before you do\dash{}I
won’t have you getting my sheets all covered in your
travel-grime.”

He felt the foolish grin spread across his face and he skinned out
of his clothes as fast as he could with his cast on.

\begin{center}\rule{3in}{0.4pt}\end{center}

They didn’t leave the house until suppertime, freshly showered
(she’d been a delightful help in scrubbing those spots where the
cast impeded access) and changed. Perry took a painkiller after the
shower, which kicked in as they went out the door, and the autumn
evening was crisp and sharp.

They got as far as the corner before the man approached them.
“Perry Gibbons, isn’t it?” He had an English accent, and a little
pot-belly, and a big white bubble-jacket and a scarf wound round
his throat.

“That’s right,” Perry said. He looked at the guy. “Do I know you?”

“No, I don’t think so. But I’ve followed you in the press. Quite
remarkable.”

“Thanks,” Perry said. Being recognized\dash{}how weird was that. Cool
that it happened in front of Hilda. “This is Hilda,” he said. She
took the man’s hand, and he grinned, showing two long rat-like
front teeth.

“Fred,” he said. “What an absolute delight running into you out
here of all places. What are you doing in town?”

“Just visiting with friends,” Perry said.

“Wasn’t there some kind of dust-up at your place in Florida? I saw
what they did to the ride here, what a bloody mess.”

“Yeah,” Perry said. He pointed at his casted arm. “Seemed like a
good time to get out of Dodge.”

Hilda said, “We’re getting some dinner, if you’d like to come
along.”

“I wouldn’t want to intrude.”

“No, it’s no sweat, we’ve got a whole bunch of people associated
with the ride meeting us. You’d be more than welcome.”

“Goodness, that \emph{is} hospitable of you. How can I refuse?”

Luke and Ernie were there with their girlfriends, and there were
more kids, midwestern and healthy even if they weren’t necessarily
all Scandic, some Vietnamese kids, some Hmong, some desis descended
from the H1B diaspora. They had a gigantic meal in a student place
that was heavy on the potatoes and beers the size of your head,
which Perry resisted for a couple hours until he figured that he’d
metabolized most of the painkiller and then started in, getting
just short of roaring drunk. He told them war stories, told them
about Death Waits, told them about the co-op and the plan to fight
back.

“That just doesn’t sound right to me,” said a friend of Luke’s, a
law-school grad student who had been bending Perry’s ear all night
with stories from his law-clinic work defending university students
from music-industry lawsuits. “I mean, sure, go after the cops
because they roughed you guys up, but how much money do the cops
have? You gotta target some fat cash, and for that you want to go
after Disney. Abuse of trademark, abuse of process, something like
that. The standard’s pretty high, but if you can get a judgement,
the money is incredible. You could take them to the cleaners.”

Perry looked blearily at him. He was young, like all of them, but
he had a good rhetorical style that Perry recognized as something
born of real confidence. He knew his stuff, or thought he did. He
had a strawberry mark on his high forehead that looked like a map
of a distant island, and Perry thought that the mark probably threw
off the kid’s opponents. “So we sue Disney and five years from now
we cash in\dash{}how does that help us now?”

The kid nodded. “I hoped you’d ask me that. I’ve been thinking
about this a lot lately. Here’s what you need to do, dude, here’s
the fucking thing.” The room had grown silent. Everyone leaned
closer. Fred poured Perry another beer from the pitcher in the
middle of the table. “Here’s how you do it. You raise investment
capital for it. There’s a ton of money in this, a ton. Disney’s got
deep pockets and you’ve got a great case.

“But like you say, it’ll take ten, fifteen years to get the money
out of them. And it’ll cost a mil in legal fees on the way. So what
you do is, you create an investment syndicate. You can maybe get
thirty million out of Disney, plus whatever the jury awards in
punitives, and if you keep half of it, you can deliver a fifteen-x
return on investment. So go find a millionaire and borrow sixteen
million, and turn the defense over to him.”

Perry was dumbstruck. “You’re joking. How can that possibly work?”

“It’s how patent lawsuits work! Some dickhead engineer gets a bogus
patent for his doomed startup, and as they’re sinking into the mud,
some venture capitalist comes and buys the company up just so it
can go around and threaten other companies with real businesses for
violating the patent. They ask for sums just below what it would
cost to get the US Patent and Trademark Office to invalidate the
patent, and everyone ponies up. Venture capitalism is the major
source of funding for commercial lawsuits these days.”

Fred laughed and clapped. “Brilliant! Perry, that’s just brilliant.
Are you going to do it?”

Perry looked at the table, doodling in the puddles of beer with a
fingertip. “I just want to get back to making stuff, you know. This
is nuts. Devoting ten years of my life to suing someone?”

“You don’t have to do the suing. That’s the point. You outsource
that. You get the money; someone else does the business stuff.”
Hilda put her arm around his shoulders. “Give the suits something
to occupy themselves with\dash{}otherwise they get antsy and stir up
trouble.”

Perry and Hilda laughed like it was the funniest thing they’d ever
heard. Fred and the others joined in, and Perry scrawled a drunken
note to Tjan and Kettlewell with the info. The party broke up not
long after, amid much chortling and snorting, and they staggered
home. Fred gave Perry a warm handshake and treated Hilda to a
lingering, sloppy hug until she pushed him off, laughing even
harder.

“All right then,” Perry said, “home again home again.”

Hilda gave his groin a friendly honk and then made a dash for it,
and he gave chase.

\begin{center}\rule{3in}{0.4pt}\end{center}

PHOTO: A Drunken Perry Gibbons Gets a How’s Your Father From
Ride-Bride Hilda Hammersen

MADISON, WI: Say you managed to inspire some kind of “movement” of
techno-utopians who built a network of amusement park rides that
guide their visitors through an illustrated history of the last
dotcom bubble.

Say that your merry band of unwashed polyamorous info-hippies was
overtaken by jackbooted thugs from one of the dinosauric media
empires of yesteryear, whose legal machinations resulted in
nationwide raids, beatings, gassings, and the total shutdown of
your “movement.”

What would you do? Sue? Call a press-conference? Bail your loyal
followers out of the slam?

Get laid, get shitfaced, and let a bunch of students spitball
bullshit ideas for fighting back?

If you picked the latter, you’re in good company. Last night, Perry
Gibbons, soi-disant “founder” of the rideafarian religious cult,
was spotted out for drinks and cuddles with a group of
twentysomething students in the backwater town of Madison, WI, a
place better known for its cheddar than its activism.

While Gibbons regaled the impressionable post-adolescents with
tales of his derring-do, he avidly noted their strategic
suggestions for solving his legal, paramilitary, and technical
problems.

One suggestion that drew Gibbons’s attention and admiration was to
approach venture capitalists and beg them for the capital to sue
Disney and then use the settlements from the suits to pay back the
VCs.

This mind-croggling Ponzi scheme is the closest thing to a business
model we’ve yet heard of from the chip-addled techno-hippies of the
New Work and its post-boom incarnation.

One can only imagine how our Ms Church will cover this in her
fan-blog: breathless admiration for Mr Gibbons’s cunning in
soliciting yet more “way out of the box” thinking from the Junior
Guevaras of the Great Midwest, no doubt.

Perhaps Gibbons can be afforded a little sympathy, though. His
latest encounter with Florida law left him with a broken arm and it
may be that the pain medication is primarily responsible for
Gibbons’s fancy thinking. If that’s the case, we can only hope that
his young, blond Scandie nursie will carefully minister him back to
health (while his comrades rot in gaol around the country).

This organization needs to die before it gets someone killed.

Comments? Write to Freddy at honestfred@techstink.co.uk

\begin{center}\rule{3in}{0.4pt}\end{center}

Lester interrupted Suzanne’s phone-call to break in and announce
that he’d run Rat-Toothed Freddy to ground: the reporter had caught
the first flight from Madison to Chicago and then gone west to San
Jose. The TSA had flagged him as a person-of-interest and were
watching his movements, and a little digging on its website could
cause it to disclose Freddy’s every airborne movement.

Suzanne relayed this to Perry.

“Don’t you go there,” she said. “He’s gunning for the San Francisco
crew, and he’s hoping for a confrontation or a denunciation so that
he can print it. He gets idees fixes that he worries at like a
terrier, going for more bile.”

“Is he a psycho? What the hell is his beef with me?”

“I think that he thinks that technology hasn’t lived up to its
promise and that we should all be demanding better of our tech. So
for him, that means that anyone who actually \emph{likes}
technology is the enemy, the worst villain, undermining the case
for bringing tech up to its true potential.”

“Fuck, that is so twisted.”

“And given the kind of vile crap he writes, the only readers he has
are nut-cases who get off on seeing people who are actually
creating stuff flayed alive for their failures. They egg him
on\dash{}ever see one of his letters columns? If he changed to actual
reportage, telling the balanced stories of what was going on in the
world, they’d jump ship for some other hate-monger. He’s a
lightning-rod for assholes\dash{}he’s the king of the trolls.”

Perry looked away. “What do I do?”

“You could try to starve him. If you don’t show your head, he can’t
report on you, except by making stuff up\dash{}and made-up stuff gets
boring, even for the kinds of losers who read his stuff.”

“But I’ve got work to do.”

“Yeah, yeah you do. Maybe you’ve just got to take your lumps. Every
complex ecosystem has parasites after all. Maybe you just call up
San Francisco and brief them on what to expect from this guy and
take it from there.”

Once they were off the line, Lester came up behind her and hugged
her at the waist, squeezing the little love-handles there,
reminding her of how long it had been since she’d made it to yoga.

“You think that’ll work?”

“Maybe. I’ve been talking to the \emph{New Journalism Review} about
writing a piece on moral responsibility and paid journalism, and if
I can bang it out this aft, I bet they’ll publish it tomorrow.”

“What’s that going to do?”

“Well, it’ll distract him from Perry, maybe. It might get his
employer to take a hard look at what he’s writing\dash{}I mean that piece
is just lies, mischaracterizations, and editorial masquerading as
reportage.” She put her lid down and paced around the condo,
looking at the leaves floating in the pool. “It’ll give me some
satisfaction.”

Lester gave her a hug, and it smelled of the old days and the old
Lester, the giant, barrel-chested pre-fatkins Lester. It took her
back to a simpler time, when they’d had to worry about commercial
competition, not police raids.

She hugged him back. He was all hard muscle and zero body-fat
underneath his tight shirt. She’d never dated anyone that fit, not
even back in high-school. It was a little disorienting, and it made
her feel especially old and saggy sometimes, though he never seemed
to notice.

Speaking of which, she felt his erection pressing against her
midriff, and tried to hide her grin. “Gimme a couple hours, all
right?”

She dialed the NJR editor’s number as she slid into her chair and
pulled up a text-editor. She knew what she planned on writing, but
it would help to be able to share an outline with the NJR if she
was going to get this out in good time. Working with editors was a
pain after years of writing for the blog, but sometimes you wanted
someone else’s imprimatur on your work.

Five hours later, the copy was filed. She rocked back in her chair
and stretched her arms high over her head, listening to the crackle
of her spine. She’d been half-frozen by the air conditioning, so
she’d turned it off and opened a window, and now the condo was hot
and muggy. She stripped down to her underwear and headed for the
shower, but before she could make it, she was intercepted by
Lester.

He fell on her like a dog on dinner, and hours slipped by as they
made the apartment even muggier. Lester’s athleticism in the sack
was flattering, but sometimes boundless to the point of irritation.
She was rescued from it this time by the doorbell.

Lester put on a bathrobe and answered the door, and she heard the
sounds of the family Kettlewell spilling in, the kids’ little
footfalls pounding up and down the corridors. Hurriedly, Suzanne
threw on a robe and ducked across the corridor into the bathroom,
but not before catching sight of Eva and Landon. Eva’s expression
was grimly satisfied; Landon looked stricken. Fuck it, anyway.
She’d never given him any reason to hope, and he had no business
hoping.

Halfway through her shower, she heard someone moving around in the
bathroom, and thinking it was Lester, she stuck her head around the
curtain, only to find Ada on the pot, little jeans around her
ankles. “I hadda make,” Ada said, with a shrug.

Christ. What was she doing back here, anyway? She’d missed it all
so much from Petersburg. But she hadn’t really bargained for this.
It was only a matter of time until Tjan showed up too, surely
they’d be wanting a council of war after Freddy’s opening salvo.

She waited for the little girl to flush (ouch! hot water!) and got
dressed as discreetly as possible.

By the time she got to the balcony where the council of war was
under way, the two little girls, Lyenitchka and Ada, had gotten
Pascal up on the sofa and were playing dress up with him,
hot-gluing Barbie heads to his cheeks and arms and chubby knees,
like vacantly staring warts.

“Do you like him?”

“I think he looks wonderful, girls. Is that glue OK for him,
though?”

Ada nodded vigorously. “I’ve been gluing things to my brother with
that stuff forever. Dad says it’s OK so long as I don’t put it in
his eyes.”

“Your dad’s a smart man.”

“He’s in love with you,” Lyenitchka said, and giggled. Ada slugged
her in the arm.

“That’s supposed to be a secret, stupid,” Ada said.

Flustered, Suzanne ducked out onto the patio and shut the door
behind her. Eva and Tjan and Kettlewell all turned to look at her.

“Suzanne!” Tjan said. “Nice article.”

“Is it up already?”

“Yeah, just a couple minutes ago.” Tjan held up his phone. “I’ve
got a watch-list for anything to do with Freddy that gets a lot of
link-love in a short period. Your piece rang the cherries.”

She took the phone from him and looked at the list of links that
had been found to the \emph{NJR} piece. Three of the diggdots had
picked up the story, since they loved to report on anything that
made fun of Freddy\dash{}he was a frequent savager of their readers’
cherished beliefs, after all\dash{}and thence it had wormed its way all
around the net. In the time she’d needed to take a shower, her
story had been read by about three million people. She felt a
twinge of regret for not publishing it on her blog\dash{}that would have
been some serious advertising coin.

“Well, there you have it.”

“What do you suppose he’ll come back with?” Kettlewell said, then
looked uncomfortably at Eva. She pretended not to notice, and
continued to stare at the grimy Hollywood palms, swimming pools and
freeways.

“Something nasty and full of lies, no doubt.”

\begin{center}\rule{3in}{0.4pt}\end{center}

Nerd Groupie Church Finds Fatkins Love with Ride Sidekick

Sources close to the Hollywood, Florida ride-cult have revealed
that Suzanne Church, the celebrity blogger who helped inflate the
New Work stock bubble, is in the midst of a romantic entanglement
with one of the cult’s co-founders.

Church recently came out of retirement in St Petersburg, where she
has been producing PR\^{}H\^{}H journalistic accounts of the new
generation of Russian experimental plastic surgery butchers.

Church was lured back by the promise of a story about the
ride-network that was founded by her old pals from the New Work
pump-and-dump, Lester Banks and Perry Gibbons. Now on the scene are
more familiar faces: Landon Kettlewell, the disgraced former CEO of
Kodacell, and Tjan Tang, the former business manager of the
Banks/Gibbons scam.

But not long after arriving on the scene, Church fell in with
Banks, an early fatkins and stalwart of the New Work movement, a
technologist who entranced his fellow engineers with his accounts
of the New Work’s many “inventions”\dash{}prompting one message-board
commenter to characterize him as “a cross between Steve Wozniak and
the Reverend Sun Myung Moon.”

Now, eyewitness accounts have them going at it like shagging
marmots, as the bio-enhanced Banks falls on Church’s wrinkly
carcass half a dozen times a day, apparently consummating a romance
that blossomed while Banks was, to put it bluntly, a giant fat
bastard. It seems that radical weight-loss has put Banks into the
category of “blokes that Suzanne Church is willing to play hide the
sausage with.”

All this would be mere sordid gossip but for the fact that Church
is once again glowingly chronicling the adventures of the Florida
cultists, playing journalist, without a shred of impartiality or
disclosure.

One can only imagine when the other, financial shoe will drop. For
wherever Church goes, money isn’t far behind: surely there’s a
financial aspect to this business with the ride.

UPDATE:

Indeed there is: further anonymous tipsterism reveals that papers
have been filed to create a “co-operative” structured like a
classic Ponzi scheme, in which franchise operators of the ride are
expected to pay membership dues further up the ladder. All the
romance of Church’s accounts will certainly find a fresh batch of
suckers\dash{}if there’s one thing we know about Suzanne Church, it’s
that she knows how to separate a mark from his money.

\begin{center}\rule{3in}{0.4pt}\end{center}

Lester ran the ride basically on his own that week, missing his
workshop and his tinkering, thinking of Suzanne, wishing that Perry
was back already. He wasn’t exactly a people person, and there were
a \emph{lot} of people.

“I brought some stuff,” the goth kid said as he paid for his
ticket, hefting two huge duffel bags. “That’s still OK, right?”

Was it? Damned if Lester knew. The kid had a huge bruise covering
half of his face, and Lester thought he recognized him from the
showdown\dash{}Death Waits, that’s what Perry had said.

“Sure, it’s fine.”

“You’re Lester, right?”

Christ, another one.

“Yes, that’s me.”

“Honest Fred is full of shit. I’ve been reading your posts since
forever. That guy is just jealous because your girlfriend outed him
for being such a lying asshole.”

“Yeah.” Death Waits wasn’t the first one to say words to this
effect\dash{}Suzanne had had that honor\dash{}and he wouldn’t be the last. But
Lester wanted to forget it. He’d liked the moments of fame he’d
gained from Suzanne’s writing, from his work on the message boards.
He’d even had a couple of fanboys show up to do a little interview
for their podcast about his mechanical computer. That had been
nice. But “blokes that Suzanne Church is willing to play hide the
sausage with”\dash{}ugh.

Suzanne was holding it together as far as he could tell. But she
didn’t seem as willing to stick her neck out to broker little
peaces between Tjan and Kettlewell anymore, and those two were
going at it hammer and tongs now, each convinced that he was in
charge. Tjan reasoned that since he actually ran one of the
most-developed rides in the network that he should be the
executive, with Kettlewell as a trusted adviser. Kettlewell clearly
felt that he deserved the crown because he’d actually run global
businesses, as opposed to Tjan, who was little more than a middle
manager.

Neither had said exactly that, but that was only because whenever
they headed down that path, Suzanne interposed herself and
distracted them.

No one asked Lester or Perry, even though they were the ones who’d
invented it all. It was all so fucked up. Why couldn’t he just make
stuff and do stuff? Why did it always have to turn into a plan for
world domination? In Lester’s experience, most world-domination
plans went sour, while a hefty proportion of modest plans to Make
Something Cool actually worked out pretty well, paid the bills, and
put food on the table.

The goth kid looked expectantly at him. “I’m a huge fan, you know.
I used to work for Disney, and I was always watching what you did
to get ideas for new stuff we should do. That’s why it’s so totally
suckballs that they’re accusing you of ripping them off\dash{}we rip you
off all the time.”

Lester felt like he was expected to do something with that
information\dash{}maybe deliver it to some lawyer or whatever. But would
it make a difference? He couldn’t get any spit in his mouth over
legal fights. Christ\dash{}legal fights!

“Thanks. You’re Death Waits, right? Perry told me about you.”

The kid visibly swelled. “Yeah. I could help around here if you
wanted, you know. I know a lot about ride-operating. I used to
train the ride-runners at Disney, and I could work any position. If
you wanted.”

“We’re not really hiring\dash{}” Lester began.

“I’m not looking for a job. I could just, you know, help. I don’t
have a job or anything right now.”

Lester needed to pee. And he was sick of sitting here taking
people’s money. And he wanted to go play with his mechanical
computer, anyway.

“Lester? Who’s the kid taking ticket money?” Suzanne’s hug was
sweaty and smelled good.

“Look at this,” Lester said. He flipped up his magnifying goggles
and handed her the soda can. He’d cut away a panel covering the
whole front of the can, and inside he’d painstakingly assembled
sixty-four flip-flops. He turned the crank on the back of the can
slowly, and the correct combination of rods extended from the back
of the can, indicating the values represented on the flip-flops
within. “It’s a sixty-four bit register. We could build a
shitkicking Pentium out of a couple million of these.”

He turned the crank again. The can smelled of solder and it had a
pleasant weight in his hand. The mill beside him hummed, and on his
screen, the parts he’d CADded up rotated in wireframe. Suzanne was
at his side and he’d just built something completely teh awesome.
He’d taken his shirt off somewhere along the afternoon’s lazy, warm
way and his skin prickled with a breeze.

He turned to take Suzanne in his arms. God he loved her. He’d been
in love with her for years now and she was his.

“Look at how cool this thing is, just look.” He used a tweezer to
change the registers again and gave it a little crank. “I got the
idea from the old Princeton Institute Electronic Computer Project.
All these geniuses, von Neumann and Dyson, they brought in their
kids for the summer to wind all the cores they’d need for their
RAM. Millions of these things, wound by the kids of the smartest
people in the universe. What a cool way to spend your summer.

“So I thought I’d prototype the next generation of these, a 64-bit
version that you could build out of garbage. Get a couple hundred
of the local kids in for the summer and get them working. Get them
to understand just how these things work\dash{}that’s the problem with
integrated circuits, you can’t take them apart and see how they
work. How are we going to get another generation of tinkerers
unless we get kids interested in how stuff works?”

“Who’s the kid taking ticket money?”

“He’s a fan, that kid that Perry met in jail. Death Waits. The one
who brought in the Disney stuff.”

He gradually became aware that Suzanne was rigid and shaking in his
arms.

“What’s wrong?”

Her face was purple now, her hands clenched into fists. “What’s
wrong? Lester, what’s wrong? You’ve left a total stranger, who, by
his own admission, is a recently terminated employee of a company
that is trying to bankrupt you and put you in jail. You’ve left him
in charge of an expensive, important capital investment, and given
him the authority to collect money on your behalf. Do you really
need to ask me what’s wrong?”

He tried to smile. “It’s OK, it’s OK, he’s only\dash{}”

“Only what? Only your possible doom? Christ, Perry, you don’t even
have fucking \emph{insurance} on that business.”

Did she just call him Perry? He carefully set down the Coke can and
looked at her.

“I’m down here busting my \emph{ass} for you two, fighting cops,
letting that shit Freddy smear my name all over the net, and what
the hell are you doing to save yourself? You’re in here playing
with Coke cans!” She picked it up and shook it. He heard the works
inside rattling and flinched towards it. She jerked it out of his
reach and threw it, \emph{threw it} hard at the wall. Hundreds of
little gears and ratchets and rods spilled out of it.

“Fine, Lester, fine. You go on being an emotional ten-year-old. But
stop roping other people into this. You’ve got people all over the
country depending on you and you are just \emph{abdicating} your
responsibility to them. I won’t be a part of it.” She was crying
now. Lester had no idea what to say now.

“It’s not enough that Perry’s off chasing pussy, you’ve got to pick
this moment to take French leave to play with your toys. Christ,
the whole bunch of you deserve each other.”

Lester knew that he was on the verge of shouting at her, really
tearing into her, saying unforgivable things. He’d been there
before with other friends, and no good ever came of it. He wanted
to tell her that he’d never asked for the responsibility, that he’d
lived up to it anyway, that no one had asked her to put her neck on
the line and it wasn’t fair to blame him for the shit that Freddy
was putting her through. He wanted to tell her that if she was in
love with Perry, she should be sleeping with Perry, and not him. He
wanted to tell her that she had no business reaming him out for
doing what he’d always done: sit in his workshop.

He wanted to tell her that she had never once seen him as a sexual
being when he was big and fat, but that he had no trouble seeing
her as one now that she was getting old and a little saggy, and so
where did she get off criticizing his emotional maturity?

He wanted to say all of this, and he wanted to take back his 64-bit
register and nurse it back to health. He’d been in a luminous
creative fog when he’d built that can, and who knew if he’d be able
to reconstruct it?

He wanted to cry, to blubber at her for the monumental unfairness
of it all. He stood stiffly up from his workbench and turned on his
heel and walked out. He expected Suzanne to call out to him, but
she didn’t. He didn’t care, or at least he didn’t want to.

\begin{center}\rule{3in}{0.4pt}\end{center}

Sammy skipped three consecutive Theme-Leaders’ meetings, despite
increasingly desperate requests for his presence. The legal team
was eating every spare moment he had, and he hadn’t been able to
get audience research to get busy on his fatkins project. Now he
was behind schedule\dash{}not surprising, given that he’d pulled his
schedule out of his ass to shut up Wiener and co\dash{}and dealing with
lawyers was making him crazy.

And to top it all off, the goddamned rides were back up and
running.

So the last thing he wanted was a visit from Wiener.

“They’re suing us, you know. They raised \emph{venture capital} to
sue us, because we have such deep pockets. You know that, Sammy?”

“I know it, Wiener. People sue us all the time. Venture capitalists
have deep pockets, too, you know\dash{}when we win, we’ll take them to
the cleaners. Christ, why am I having this conversation with you?
Don’t you have something productive to do? Is Tomorrowland so
fucking \emph{perfect} that you’ve come around to help me with my
little projects?”

“Someone’s a little touchy today,” Wiener said, wagging a finger.
“I just wanted to see if you wanted some help coming up with a
strategy for getting out of this catastrophe, but since you mention
it, I \emph{do} have work I could be doing. I’ll see you at the
next Theme-Leaders’ meeting, Sam. Missing three is grounds for
disciplinary action, you know.”

Sammy sat back in his chair and looked coolly at Wiener. Threats
now. Disciplinary action. He kept on his best poker face, looking
past Wiener’s shoulder (a favorite trick for staring down
adversaries\dash{}just don’t meet their eyes). In his peripheral vision,
he saw Wiener wilt, look away and then turn and leave the room.

He waited until the door had shut, then slumped in his seat and put
his face in his hands. God, and shit, and damn. How did it all go
so crapola? How did he end up with a theme-area that was half-shut,
record absenteeism, and even a goddamned \emph{union organizer}
just the day before, whom he’d had to have security remove. Florida
laws being what they were, it was a rare organizer brave enough to
try to come on an employer’s actual premises to do his dirty work,
no one wanted a two-year rap without parole for criminal trespass
and interference with trade. The kid had been young, about the same
age as Death Waits and the castmembers, and had clearly been
desperate to collect his bounty from SEIU. He’d gone hard,
struggling and kicking, shouting slogans at the wide-eyed
castmembers and few guests who watched him go away.

Having him taken away had given Sammy a sick feeling. They hadn’t
had one of those vultures on the premises in three years, and never
on Sammy’s turf.

What next, what next? How much worse could it get?

“Hi, Sammy.” Hackelberg wasn’t the head of the legal department,
but he was as high up in the shadowy organization as Sammy ever
hoped to meet. He was old and leathery, the way that natives to the
Sunbelt could be. He loved to affect ice-cream suits and had even
been known to carry a cane. When he was in casual conversation, he
talked “normal”\dash{}like a Yankee newscaster. But the more serious he
got, the deeper and thicker his drawl got. Sammy never once
believed that this was accidental. Hackelberg was as premeditated
as they came.

“I was just about to come over and see you,” Sammy lied. Whatever
problem had brought Hackelberg down to his office, it would be
better to seem as though he was already on top of it.

“I expect you were.” \emph{Were} came out \emph{Wuh}\dash{}when the drawl
got that far into the swamps that quickly, disaster was on the
horizon. Hackelberg let the phrase hang there.

Sammy sweated. He was good at this game, but Hackelberg was better.
Entertainment lawyers were like fucking vampires, evil embodied. He
looked down at his desk.

“Sammy. They’re coming back after us\dash{}”
\emph{They-ah comin’ back aft-ah us}. “Those ride people. They did
what we thought they’d do, incorporating into a single entity that
we can sue once and kill for good, but then they did something
else. Do you know what they did, Sammy?”

Sammy nodded. “They’re countersuing. We knew they’d do that,
right?”

“We didn’t expect they’d raise a war-chest like the one they’ve
pulled together. They have a \emph{business-plan} built around
suing us for the next fifteen years, Sammy. They’re practically
ready to float an IPO. Have you seen this?” He handed Sammy a
hardcopy of a chic little investment newsletter that was so
expensive to subscribe to that he’d suspected until now that it
might just be a rumor.

\headline{HOW DO YOU GET RID(E) OF A BILLION?}

The Kodacell experiment recognized one fundamental truth: it’s easy
to turn ten thousand into two hundred thousand, but much harder to
turn ten million into two hundred million. Scaling an investment up
to gigascale is so hard, it’s nearly impossible.

But a new paradigm in investment that’s unfolding around us that
might actually solve the problem: venture-financed litigation.
Twenty or thirty million sunk into litigation can bankrupt a twenty
billion-dollar firm, transferring to the investors whatever assets
remain after legal fees.

It sounds crazy, and only time will tell whether it proves to be
sustainable. But the founder of the strategy, Landon Kettlewell,
has struck gold for his investors more than once\dash{}witness the
legendary rise and fall of Kodacell, the entity that emerged from
the merger of Kodak and Duracell. Investors in the first two rounds
and the IPO on Kodacell brought home 30X returns in three years (of
course, investors who stayed in too long came away with nothing).

Meanwhile, Kettlewell’s bid to take down Disney Parks looks
good\dash{}the legal analysis of the vexatious litigation and unfair
competition charges have legal scholars arguing and adding up the
zeros. Most damning is the number of former Disney Parks employees
(or “castmembers” in the treacly dialect of the Magic Kingdom)
who’ve posted information about the company’s long-term plan to
sabotage Kettlewell’s clients.

Likewise fascinating is the question of whether the jury will be
able to distinguish between Disney Parks, whose corporate
citizenship is actually pretty good, from Disney Products, whose
record has been tainted by a string of disastrous child-labor,
safety, and design flaws (astute readers will be thinking of the
“flammable pajamas” flap of last year, and CEO Robert Montague’s
memorable words, “Parents who can’t keep their kids away from
matches have no business complaining about \emph{our}
irresponsibility”). Punitive jury awards are a wild-card in this
kind of litigation, but given the trends in recent years, things
look bad for Disney Parks.

Bottom line: should your portfolio include a litigation-investment
component? Yes, unequivocally. While risky and slow to mature,
litigation-investments promise a staggering return on investment
not seen in decades. A million or two carefully placed with the
right litigation fund could pay off enough to make it all
worthwhile. This is creative destruction at its finest: the old
dinosaurs like Disney Parks are like rich seams of locked-away
capital begging to be liquidated and put to work at nimbler firms.

How can you tell if you’ve got the right fund? Come back next week,
when we’ll have a Q\&A with a litigation specialist at Credit
Suisse/First Boston.

\begin{center}\rule{3in}{0.4pt}\end{center}

“There’s litigation specialists at Credit Suisse?”

He was big, Hackelberg, though he often gave the impression of
being smaller through his habitual slouch. But when he pulled
himself up, it was like a string in the center of the top of his
head was holding him erect, like he was hovering off the ground,
like he was about to leap across the desk and go for your throat.
His lower jaw rocked from side to side.

“They do now, Sammy. Every investment bank has one, including the
one that the chairman of our board is a majority shareholder in.”

Sammy swallowed. “But they’ve got just as deep pockets as we
do\dash{}can’t we just fight these battles out and take the money off of
them when we win?”

“If we win.”

Sammy saw his opportunity to shift the blame. “If we’ve been acting
on good legal advice, why wouldn’t we win?”

Hackelberg inhaled slowly, his chest filling and filling until his
ice-cream suit looked like it might pop. His jaw clicked from side
to side. But he didn’t say anything. Sammy tried to meet that cool
gaze, but he couldn’t out-stare the man. The silence stretched.
Sammy got the message: this was not a problem that originated in
the legal department. This was a problem that originated with him.

He looked away. “How do we solve this?”

“We need to raise the cost of litigation, Samuel. The only reason
this is viable is that it’s cost-effective to sue us. When we raise
the cost of litigation, we reduce its profitability.”

“How do we raise the cost of litigation?”

“You have a fertile imagination, Sammy. I have no doubt that you
will be able to conceive of innumerable means of accomplishing this
goal.”

“I see.”

“I hope you do. I really hope you do. Because we have an
alternative to raising the cost of litigation.”

“Yes?”

“We could sacrifice an employee or two.”

Sammy picked up his water-glass and discovered that it was empty.
He turned away from his desk to refill it from his filter and when
he turned back, the lawyer had gone. His mouth was dry as cotton
and his hands were shaking.

Raise the cost of litigation, huh?

He grabbed his laptop. There were ways to establish anonymous email
accounts, but he didn’t know them. Figuring that out would take up
the rest of the afternoon, he realized, as he called up a couple of
FAQs.

In the course of a career as varied and ambitious as Sammy’s, it
was often the case that you ran across an email address for someone
you never planned on contacting, but you never knew, and a wise
planner makes space for lots of outlier contingencies.

Sammy hadn’t written down these email addresses. He’d committed
them to memory.

\begin{center}\rule{3in}{0.4pt}\end{center}

Death Waits was living the dream. He took people’s money and
directed them to the ride’s entrance, making them feel welcome,
talking ride trivia. Some of his pals spotted him at the desk and
enviously demanded to know how he came to be sitting on the other
side of the wicket, and he told them the incredible story of the
fatkins who’d simply handed over the reins.

This, this was how you ran a ride. None of that artificial gloopy
sweetness that defined the Disney experience: instead, you got a
personal, informal, human-scale experience. Chat people up, find
out their hopes and dreams, make admiring noises at the artifacts
they’d brought to add to the ride, kibbitz about where they might
place them.\ldots

Around him, the bark of the vendors. One of them, an old lady in a
blinding white sun-dress, came by to ask him if he wanted anything
from the coffee-cart.

There had been a time, those first days when they’d rebuilt
Fantasyland, when he’d really felt like he was part of the magic.
No, The Magic, with capital letters. Something about the shared
experience of going to a place with people and having an experience
with them, that was special. It must be why people went to church.
Not that Disney had been a religion for him, exactly. But when he
watched the park he’d grown up attending take on the trappings that
adorned his favorite clubs, his favorite movies and games\dash{}man, it
had been a piece of magic.

And to be a part of it. To be an altar boy, if not a priest, in
that magical cathedral they’d all built together in Orlando!

But it hadn’t been real. He could see that now.

At Disney, Death Waits had been a customer, and then an employee
(“castmember”\dash{}he corrected himself reflexively). What he wanted,
though, was to be a \emph{citizen}. A citizen of The Magic\dash{}which
wasn’t a Magic Kingdom, since kingdoms didn’t have citizens, they
had subjects.

He started to worry about whether he was going to get a lunch break
by about two, and by three he was starving. Luckily that’s when
Lester came back. He thanked Death profusely, which was nice, but
he didn’t ask Death to come back the next day.

“Um, when can I come back and do this some more?”

“You \emph{want} to do this?”

“I told you that this morning\dash{}I love it. I’m good at it, too.”

Lester appeared to think it over. “I don’t know, man. I kind of put
you in the hot-seat today, but I don’t really have the authority to
do it. I could get into trouble\dash{}”

Death waved him off. “Don’t sweat it, then,” he said with as much
chirp as he could muster, which was precious fucking little. He
felt like his heart was breaking. It was worse than when he’d
finally asked out a co-worker who’d worked the Pinocchio Village
Haus and she had her looked so horrified that he’d made a joke out
of it, worried about a sexual harassment complaint.

Lester clearly caught some of that, for he thought some more and
then waved his hands. “Screw her anyway. Meet me here at ten
tomorrow. You’re in.”

Death wasn’t sure he’d heard him right. “You’re kidding.”

“No man, you want it, you got it. You’re good at it, like you
said.”

“Holy\dash{}thanks. Thank you so much. I mean it. Thank you!” He made
himself stop blithering. “Nice to meet you,” he said finally. “Have
a great evening!” Yowch. He was speaking castmemberese.
\emph{Nice one, Darren}.

He’d saved enough out of his wages from his first year at Disney to
buy a little Shell electric two-seater, and then he’d gone way into
debt buying kits to mod it to look like a Big Daddy Roth
coffin-dragster. The car sat alone at the edge of the lot. Around
him, a slow procession of stall-operators, with their arms full,
headed for the freeway and across to the shantytown.

Meanwhile, he nursed his embarrassment and tried to take comfort in
the attention that his gleaming, modded car evinced. He loved the
decorative spoilers, the huge rear tires, the shining muffler-pipes
running alongside the bulging running-boards. He stepped in and
gripped the bat-shaped gearshift, adjusted the headstone-shaped
headrest, and got rolling. It was a long drive back home to
Melbourne, and he was reeling from the day’s events. He wished he’d
gotten someone to snap a pic of him at the counter. Shit.

He pulled off at a filling station after a couple hours. He needed
a piss and something with guarana if he was going to make it the
rest of the way home. It was all shut down, but the automat was
still open. He stood before the giant, wall-sized glassed-in
refrigerator and dithered over the energy-drinks. There were
chocolate ones, salty ones, colas and cream sodas, but a friend had
texted him a picture of a semi-legal yogurt smoothie with taurine
and modafinil that sounded really good.

He spotted it and reached to tap on the glass and order it just as
the fat guy came up beside him. Fat guys were rare in the era of
fatkins, it was practically a fashion-statement to be chunky, but
this guy wasn’t fashionable. He had onion-breath that Death could
smell even before he opened his mouth, and he was wearing a greasy
windbreaker and baggy jeans. He had a comb-over and needed a
shave.

“What the hell are you supposed to be?”

“I’m not anything,” Death Waits said. He was used to shit-kickers
and tourists gawping at his shock of black hair with its viridian
green highlights, his white face-paint and eyeliner, his contact
lenses that made his whole eyes into zombie-white cue-balls. You
just had to ignore them.

“You don’t look like nothing to me. You look like something.
Something you’d dress up a six year old as for Halloween. I mean,
what the fuck?” He was talking quietly and without rancor, but he
had a vibe like a basher. He must have arrived at the deserted
rest-stop while Death Waits was having a piss.

Death Waits looked around for a security cam. These rest-stops
always had a license-plate cam at the entrance and a couple of
anti-stickup cams around the cashier. He spotted the camera.
Someone had hung a baseball hat over its lens.

He felt his balls draw up toward his abdomen and his breathing
quicken. This guy was going to fucking mug him. Shit shit shit.
Maybe take his \emph{car}.

“OK,” Death said, “nice talking to you.” He tried to step around
the guy, but he side-stepped to block Death’s path, then put a hand
on Death’s shoulder\dash{}it was strong. Death had been mugged once
before, but the guy hadn’t touched him; he’d just told him, fast
and mean, to hand over his wallet and phone and then had split.

“I’m not done,” the guy said.

“Look, take my wallet, I don’t want any trouble.” Apart from two
glorious sucker-punches at Sammy, Death had never thrown a punch,
not since he’d flunked out of karate lessons at the local
strip-mall when he was twelve. He liked to dance and he could run a
couple miles without getting winded, but he’d seen enough real
fights to know that it was better to get away than to try to strike
out if you didn’t know what you were doing.

“You don’t want any trouble, huh?”

Death held out his wallet. He could cancel the cards. Losing the
cash would hurt now that he didn’t have a day-job, but it was
better than losing his teeth.

The guy smiled. His onion breath was terrible.

“\emph{I} want trouble.” Without any pre-amble or wind-up, the guy
took hold of the earring that Death wore in his tragus, the little
knob of cartilage on the inside of his ear, and briskly tore it out
of Death’s head.

It was so sudden, the pain didn’t come at once. What came first was
a numb feeling, the blood draining out of his cheeks and the color
draining out of the world, and his brain double- and
triple-checking what had just happened.
\emph{Did someone just tear a piece out of my ear? Tear? Ear?}

Then the pain roared in, all of his senses leaping to keen
awareness before maxing out completely. He heard a crashing sound
like the surf, smelled something burning, a light appeared before
his eyes, an acrid taste flooded his mouth and his ear felt like
there was a hot coal nestled in it, charring the flesh.

With pain came the plan: \emph{get the fuck out of there}. He took
a step back and turned to run, but there was something tangled in
his feet\dash{}the guy had bridged the distance between them quickly,
very quickly, and had hooked a foot around his ankle. He was going
to fall over. He landed in a runner’s crouch and tried to start
running, but a boot caught him in the butt, like an old-timey
comedy moment, and he went sprawling, his chin smacking into the
pavement, his teeth clacking together with a sound that echoed in
his head.

“Get the fuck up,” the guy said. He was panting a little, sounding
excited. That sound was the scariest thing so far. This guy wanted
to kill him. He could hear that. He was some kind of truck-stop
murderer.

Death’s fingers were encrusted in heavy silver rings\dash{}stylized
skulls, a staring eyeball, a coffin-shaped poisoner’s ring that he
sometimes kept artificial sweetener in, an ankh, an alien head with
insectile eyes\dash{}and he balled his hands into fists, thinking of
everything he’d ever read about throwing a punch without breaking
your knuckles.
\emph{Get close. Keep your fist tight, thumb outside. Don’t wind up or he’ll see it coming.}

He slowly turned over. The guy’s eyes were in shadow. His belly
heaved with each excited pant. From this angle, Death could see the
guy had a gigantic boner. The thought of what that might bode sent
him into overdrive. He couldn’t afford to let this guy beat him
up.

He backed up to the rail that lined the walkway and pulled himself
upright. He cowered in on himself as much as he could, hoping that
the guy would close with him, so he could get in one good punch. He
muttered indistinctly, softly, hoping to make the man lean in. His
ring-encrusted hands gripped the railings.

The guy took a step toward him. His lips were wet, his eyes shone.
He had a hand in his pocket and Death realized that getting his
attacker close in wouldn’t be smart if he had a knife.

The hand came out. It was pudgy and stub-fingered, and the
fingernails were all gnawed down to the quick. Death looked at it.
Spray-can. Pepper-spray? Mace? He didn’t wait to find out. He
launched himself off the railing at the fat man, going for his wet,
whistling cave of a mouth.

The man nodded as he came for him and let him paste one on him.
Death's rings drew blood on the fat cheek and rocked the guy's head
back a bit. The man stepped back and armed away the blood with his
sleeve. Death was running for his car, hand digging into his pocket
for his phone. He managed to get the phone out and his hand on the
door handle before the fat man caught up, breathing heavily, air
whistling through his nose.

He punched Death in the mouth in a vastly superior rendition of
Death’s sole brave blow, a punch so hard Death’s neck made a
crackling sound as his head rocked away, slamming off the car’s
frame, ringing like a gong. Death began to slide down the car’s
door, and only managed to turn his face slightly when the man
sprayed him with his little aerosol can.

Mace. Death’s breath stopped in his lungs and his face felt as if
he’d plunged it into boiling oil. His eyes felt worse, like dirty
fingers were sandpapering over his eyeballs. He choked and fell
over and heard the man laugh.

Then a boot caught him in the stomach and while he was doubled
over, it came down again on his skinny shin. The sound of the bone
breaking was loud enough to be heard over the roaring of the blood
in his ears. He managed to suck in a lungful of air and scream it
out, and the boot connected with his mouth, kicking him hard and
making him bite his tongue. Blood filled his mouth.

A rough hand seized him by the hair and the rasping breath was in
his ears.

“You should just shut the fuck up about Disney on the fucking
Internet, you know that, kid?”

The man slammed his head against the pavement.

“Just. Shut. The. Fuck. Up.” Bang, bang, bang. Death thought he’d
lose consciousness soon\dash{}he’d had no idea that pain could be this
intense. But he didn’t lose consciousness for a long, long time.
And the pain could be a lot more intense, as it turned out.

\begin{center}\rule{3in}{0.4pt}\end{center}

Sammy didn’t want the writer meeting him at his office. His
organization had lots of people who’d been loyal to the old gothy
park and even to Death Waits. They plotted against him. They wrote
about him on the fucking Internet, reporting on what he’d eaten for
lunch and who’d shouted at him in his office and how the numbers
were declining and how none of the design crews wanted to work on
his new rides.

The writer couldn’t come to the office\dash{}couldn’t come within miles
of the park. In fact, if Sammy had had his way, they would have
done this all by phone, but when he’d emailed the writer, he’d said
that he was in Florida already and would be happy to come and meet
up.

Of course he was in Florida\dash{}he was covering the ride.

The trick was to find a place where no one, but no one, from work
would go. That meant going as touristy as possible\dash{}something
overpriced and kitschy.

Camelot was just the place. It had once been a demolition derby
stadium, and then had done turns as a skate-park, a dance-club and
a discount wicker furniture outlet. Now it was Orlando’s number two
Arthurian-themed dining experience, catering to package-holiday
consolidators who needed somewhere to fill the gullets of their
busloads of tourists. Watching men in armor joust at low speed on
glue-factory nags took care of an evening’s worth of entertainment,
too.

Sammy parked between two giant air-conditioned tour coaches, then
made his way to the entrance. He’d told the guy what he looked
like, and the guy had responded with an obvious publicity shot that
made him look like Puck from a boys’-school performance of
\emph{A Midsummer Night’s Dream}\dash{}unruly hair, mischievous grin.

When he turned up, though, he was ten years older, a cigarette
jammed in the yellowing crooked stumps of his teeth. He needed a
shower and there was egg on the front of his denim jacket.

“I’m Sammy,” Sammy said. “You must be Freddy.”

Freddy spat the cigarette to one side and shook with him. The
writer’s palms were clammy and wet.

“Pleasure to meet you,” Freddy said. “Camelot, huh?”

“Taste of home for you, I expect,” Sammy said. “Tally ho. Pip
pip.”

Freddy scrunched his face up in an elaborate sneer. “You are
joking, right?”

“I’m joking. If I wanted to give you a taste of home, I’d have
invited you to the Rose and Crown Pub in Epcot: ’Have a jolly ol’
good time at the Rose and Crown!’”

“Still joking, I trust?”

“Still joking,” Sammy said. “This place does a decent roast beef,
and it’s private enough.”

“Private in the sense of full of screaming stupid tourists stuffing
their faces?”

“Exactly.” Sammy took a step toward the automatic doors.

“Before we go in, though,” Freddy said. “Before we go in. Why are
you talking to me at all, Mr Disney Parks Executive?”

He was ready for this one. “I figured that sooner or later you’d
want to know more about this end of the story that you’ve been
covering. I figured it was in my employer’s best interest to see to
it that you got my version.”

The reporter’s grin was wet and mean. “I thought it was something
like that. You understand that I’m going to write this the way I
see it, not the way you spin it, right?”

Sammy put a hand on his heart. “Of course. I never would have asked
anything less of you.”

The reporter nodded and stepped inside the air-conditioned,
horsey-smelling depths of Camelot. The greeter had acne and a pair
of tights that showed off his skinny knock-knees. He took off his
great peaked cap with its long plume and made a stiff little bow.
“Greetings, milords, to Camelot. Yon feast awaits, and our brave
knights stand ready to do battle for their honor and your
amusement.”

Freddy rolled his eyes at Sammy, but Sammy made a little scooting
gesture and handed the greeter their tickets, which were ringside.
If he was going to go to a place like Camelot, he could at least
get the best seats in the house.

They settled in and let the serving wench\dash{}whose fancy contact
lenses, piercings, and electric blue pony-tails were seriously
off-theme\dash{}take their roast beef orders and serve them gigantic
pewter tankards of “ale”; Bud Light, and the logo was stamped into
the sides of the tankards.

“Tell me your story, then,” Freddy said. The tourists around them
were noisy and already a little drunk, their conversation loud to
be heard over the looping soundtrack of ren faire polka music.

“Well, I don’t know how much you know about the new Disney Parks
organization. A lot of people think of us as being just another
subsidiary of the Mouse, like back in the old days. But since the
IPO, we’re our own company. We license some trademarks from Disney
and operate rides based on them, but we also aggressively license
from other parties\dash{}Warners, Universal, Nintendo. Even the French
comic-book publisher responsible for Asterix. That means that we
get a lot of people coming in and out of the organization,
contractors or consultants working on designing a single ride or
show.

“That creates a lot of opportunities for corporate espionage.
Knowing what properties we’re considering licensing gives the
competition a chance to get there ahead of us, to land an exclusive
deal that sets us back on square one. It’s ugly stuff\dash{}they call it
’competitive intelligence’ but it’s just spying, plain old spying.

“All of our employees have been contacted, one time or another, by
someone with an offer\dash{}get me a uniform, or a pic of the design
roughs, or a recording of the soundtrack, or a copy of the
contracts, and I’ll make it worth your while. From street-sweepers
to senior execs, the money is just sitting there, waiting for us to
pick it up.”

The wench brought them their gigantic pewter plates of roast-beef,
Yorkshire pudding, parsnips, and a mountain of french fries,
presumably to appease the middle-American appetites of the more
unadventurous diners.

Freddy sliced off a throat-plugging lump of beef and skewered it on
his fork.

“You’re going to tell me that the temptation overwhelmed one of
your employees, yes?” He shoved the entire lump into his mouth and
began to masticate it, cheeks pouched out, looking like a kid with
a mouthful of bubble-gum.

“Precisely. Our competitors don’t want to compete with us on a
level playing field. They are, more than anything, imitators. They
take the stuff that we carefully build, based on extensive
research, design and testing, and they clone it for parking-lot
amusement rides. There’s no attention to detail. There’s no
attention to safety! It’s all cowboys and gypsies.”

Freddy kept chewing, but he dug in the pockets of his sports-coat
and came up with a small stubby notebook and a ball-point. He
jotted some notes, shielding the pad with his body.

“And these crass imitators enter into our story how?” Freddy asked
around his beef.

“You know about these New Work people\dash{}they call themselves
’re-mixers’ but that’s just a smokescreen. They like to cloak
themselves in some post-modern, ’Creative Commons’ legitimacy, but
when it comes down to it, they made their fortune off the
intellectual property of others, uncompensated use of designs and
technologies that others had invested in and created.

“So when they made a ride, it wasn’t much of much. Like some kind
of dusty Commie museum, old trophies from their last campaign. But
somewhere along the way, they hooked up with one of these brokers
who specializes in sneaking our secrets out of the park and into
the hands of our competitors and quick as that, they were
profitable\dash{}nationally franchised, even.” He stopped to quaff his
Bud Light and surreptitiously checked out the journalist to see how
much of this he was buying. Impossible to say. He was still
masticating a cheekful of rare roast, juice overflowing the corners
of his mouth. But his hand moved over his pad and he made an
impatient go-on gesture with his head, swallowing some of his
payload.

“We fired some of the people responsible for the breaches, but
there will be more. With 50,000 castmembers\dash{}” The writer snorted a
laugh at the Disney-speak and choked a little, washing down the
last of his mouthful with a chug of beer. “\dash{}50,000 \emph{employees}
it’s inevitable that they’ll find more. These ex-employees,
meanwhile, have moved to the last refuge of the scoundrel: Internet
message boards, petulant tweets, and whiny blogs, where they’re
busily running us down. We can’t win, but at least we can stanch
the bleeding. That’s why we’ve brought our lawsuits, and why we’ll
bring the next round.”

The journalist’s hand moved some more, then he turned a fresh page.
“I see, I see. Yes, all fascinating, really. But what about these
countersuits?”

“More posturing. Pirates love to put on aggrieved airs. These guys
ripped us off and got caught at it, and now they want to sue us for
their trouble. You know how counter-suits work: they’re just a bid
to get a fast settlement: ’Well, I did something bad but so did
you, why don’t we shake hands and call it a day?’”

“Uh huh. So you’re telling me that these intellectual property
pirates made a fortune knocking off your rides and that they’re
only counter-suing you to get a settlement out of you, huh?”

“That’s it in a nutshell. I wanted to sit down with you, on
background, and just give you our side of things, the story you
won’t get from the press-releases. I know you’re the only one
trying to really get at the story behind the story with these
people.”

Freddy had finished his entire roast and was working his way
through the fries and limp Yorkshire pudding. He waved vigorously
at their serving wench and hollered, “More here, love!” and quaffed
his beer.

Sammy dug into his cold dinner and speared up a forkful, waiting
for Freddy to finish swallowing.

“Well, that’s a very neat little story, Mr Disney Executive off the
record on background.” Sammy felt a vivid twinge of anxiety.
Freddy’s eyes glittered in the torchlight. “Very neat indeed.

“Let me tell you one of my own. When I was a young man, before I
took up the pen, I worked a series of completely rubbish jobs. I
cleaned toilets, I drove a taxi, I stocked grocery shelves. You may
ask how this qualified me to write about the technology industry.
Lots of people have, in fact, asked that.

“I’ll tell you why it qualifies me. It qualifies me because unlike
all the ivory-tower bloggers, rich and comfortable geeks whose
masturbatory rants about Apple not honoring their warranties are
what passes for corporate criticism online, I’ve been there. I’m
not from a rich family, I didn’t get to go to the best schools, no
one put a PC in my bedroom when I was six. I worked for an honest
living before I gave up honest work to write.

“As much as the Internet circle-jerk disgusts me, it’s not a patch
on the businesses themselves. You Disney people with your minimum
wage and all the sexual harassment you can eat labor policies in
your nice right-to-work state, you get away with murder. Anyone who
criticizes you does so on your own terms: Is Disney exploiting its
workers too much? Is it being too aggressive in policing its
intellectual property? Should it be nicer about it?

“I’m the writer who doesn’t watch your corporations on your own
terms. I don’t care if another business is unfairly competing with
your business. I care that your business is unfair to the world.
That it aggressively exploits children to get their parents to
spend money they don’t have on junk they don’t need. I care that
your workers can’t unionize, make shit wages, and get fired when
they complain or when you need to flex your power a little.

“I grew up without any power at all. When I was working for a
living, I had no say at all in my destiny. It didn’t matter how
much shit a boss wanted to shovel on me, all I could do was stand
and take it. Now I’ve got some power, and I plan on using it to
setting things to rights.”

Sammy chewed his roast long past the point that it was ready to
swallow. The fact that he’d made an error was readily apparent from
the start of Freddy’s little speech, but with each passing minute,
the depth of his error grew. He’d really fucked up. He felt like
throwing up. This guy was going to fuck him, he could tell.

Freddy smiled and quaffed and wiped at his beard with the
embroidered napkin. “Oh, look\dash{}the jousting’s about to start,” he
said. Knights in armor on horseback circled the arena, lances held
high. The crowd applauded and an announcer came on the PA to tell
them each knight’s name, referring them to a program printed on
their placemats. Sammy pretended to be interested while Freddy
cheered them on, that same look of unholy glee plain on his face.

The knights formed up around the ring and their pimply squires came
out of the gate and tended to them. There was a squire and knight
right in front of them, and the squire tipped his hat to them.
Freddy handed the kid a ten-dollar bill. Sammy never tipped live
performers; he hated buskers and panhandlers. It all reminded him
of stuffing a stripper’s G-string. He liked his media a little more
impersonal than that. But Freddy was looking at him, so with a weak
little smile, he handed the squire the smallest thing in his
wallet\dash{}a twenty.

The jousting began. It was terrible. The “knights” couldn’t ride
worth a damn, their “lances” missed one another by farcical
margins, and their “falls” were so obviously staged that even the
chubby ten year old beside him was clearly unimpressed.

“Got to go to the bathroom,” he said into Freddy’s ear. In leaning
over, he contrived to get a look at the reporter’s notebook. It was
covered in obscene doodles of Mickey Mouse with a huge erection,
Minnie dangling from a noose. There wasn’t a single word written on
it. What little blood was left in Sammy’s head drained into his
feet, which were leaden and uncoordinated on the long trip to the
filthy toilets.

He splashed cold water on his face in the sink, and then headed
back toward his seat. He never made it. From the top of the stairs
leading down to ringside, he saw Freddy quaffing more ale and
flirting with the wench. The thunder of horse-hooves and the
soundtrack of cinematic music drowned out all sounds, but nothing
masked the stink of the manure falling from the horses, half of
which were panicking (the other half appeared to be drugged).

This was a mistake. He thought Freddy was a gossip reporter who
liked juicy stories. Turned out he was also one of those tedious
anti-corporate types who would happily hang Sammy out to dry. Time
to cut his losses.

He turned on his heel and headed for the door. The doorman was
having a cigarette with a guy in a sports-coat who was wearing a
manager badge on his lapel.

“Leaving so soon? The show’s only just getting started!” The
manager was sweating under his sports-coat. He had a thin mustache
and badly died chestnut hair cut like a Lego character’s.

“Not interested,” Sammy said. “All the off-theme stuff distracted
me. Nose-rings. Blue hair. Cigarettes.” The doorman guiltily
flicked his cigarette into the parking lot. Sammy felt a little
better.

“I’m sorry to hear that, sir,” the manager said. He was prematurely
grey under the dye-job, for he couldn’t have been more than
thirty-five. Thirty-five years old and working a dead-end job like
this\dash{}Sammy was thirty-five. This is where he might end up if his
screw-ups came back to haunt him. “Would you like a comment-card?”

“No,” Sammy said. “Any outfit that can’t figure out clean toilets
and decent theming on its own can’t benefit from my advice.” The
doorman flushed and looked away, but the manager’s smile stayed
fixed and calm. Maybe he was drugged, like the horses. It bothered
Sammy. “Christ, how long until this place gets turned into a
roller-derby again?”

“Would you like a refund, sir?” the manager asked. He looked out at
the parking lot. Sammy followed his gaze, looking above the cars,
and realized, suddenly, that he was standing in a cool tropical
evening. The sky had gone the color of a ripe plum, with proud
palms silhouetted against it. The wind made them sway. A few clouds
scudded across the moon’s luminous face, and the smell of citrus
and the hum of insects and the calls of night birds were vivid on
the evening air.

He’d been about to say something cutting to the manager, one last
attempt to make the man miserable, but he couldn’t be bothered. He
had a nice screened-in porch behind his house, with a hammock. He’d
sat in it on nights like this, years ago. Now all he wanted to do
was sit in it again.

“Good night,” he said, and headed for his car.

\begin{center}\rule{3in}{0.4pt}\end{center}

Perry’s cast \emph{stank}. It had started to go a little skunky on
the second day, but after a week it was like he had a dead animal
stuck to his shoulder. A rotting dead animal. A rotting, itchy dead
animal.

“I don’t think you’re supposed to be doing this on your own,” Hilda
said, as he sawed awkwardly at it with the utility knife. It was
made of something a lot tougher than the fiberglass one he’d had
when he broke his leg falling off the roof as a kid (he’d been up
there scouting out glider possibilities).

“So you do it,” he said, handing her the knife. He couldn’t stand
the smell for one second longer.

“Uh-uh, not me, pal. No way that thing is supposed to come off
anytime soon. If you’re going to cripple yourself, you’re going to
have to do it on your own.”

He made a rude sound. “Fuck hospitals, fuck doctors, and fuck this
fucking cast. My arm barely hurts these days. We can splint it once
I get this off, that’ll immobilize it. They told me I’d need this
for \emph{six weeks}. I can’t wear this for six weeks. I’ll go
nuts.”

“You’ll go lame if you take it off. Your poor mother, you must have
driven her nuts.”

He slipped and cut himself and winced, but tried not to let her
know, because that’s exactly what she’d predicted would happen.
After a couple days together, she’d become an expert at predicting
exactly which of his escapades would end in disaster. It was a
little spooky.

Blood oozed out from under the cast and slicked his hand.

“Right, off to the hospital. I told you you’d get this thing wet if
you got in the shower. I told you that it would stink and rot and
itch if you did. I told you to let me give you a sponge bath.”

“I’m not insured.”

“We’ll go to the free clinic.”

Defeated, he let her lead him to her car.

She helped him buckle in, wrinkling her nose. “What’s wrong, baby?”
she said, looking at his face. “What are you moping about?”

“It’s just the cast,” he said, looking away.

She grabbed him by the chin and turned him to face her. “Look,
don’t do that. Do \emph{not} do that. If something’s bothering you,
we’re going to talk about it. I didn’t sign up to fall in love with
the strong silent type. You’ve been sulking all day, now what’s it
about?”

He smiled in spite of himself. “All right, I give in. I miss home.
They’re all in the middle of it, running the ride and stuff, and
I’m here.” He felt a moment’s worry that she’d be offended. “Not
that I don’t love being here with you, but I’m feeling guilty\dash{}”

“OK, I get it. Of course you feel guilty. It’s your project, it’s
in trouble, and you’re not taking care of it. Christ, Perry, is
that all? I would have been disappointed if this wasn’t worrying
you. Let’s go to Florida then.”

“What?”

She kissed the tip of his nose. “Take me to Florida, let’s meet
your friends.”

“But\ldots{}” Were they moving in together or something? He was totally
smitten with this girl, but that was \emph{fast}. Even for Perry.
“Don’t you need to be here?”

“They can live without me. It’s not like I’m proposing to move in
with you. I’ll come back here after a while. But I’m only doing two
classes this term and they’re both offered by distance-ed. Let’s
just go.”

“When?”

“After the hospital. You need a new cast, stinkmeister. Roll down
your window a little, OK? Whew!”

The doctors warned him to let the new cast set overnight before
subjecting it to the rigors of a TSA examination, so they spent one
more night at Hilda’s place. Perry spent it going over the mailing
list traffic and blog posts, confirming the plane tickets, ordering
a car to meet them at the Miami airport. He finally managed to
collapse into bed at 3AM, and Hilda grabbed him, dragged him to
her, and spooned him tightly.

“Don’t worry, baby. Your friends and I will get along great.”

He hadn’t realized that he’d been worrying about this, but once she
pointed it out, it was obvious. “You’re not worried?”

She ran her hands over his furry chest and tummy. “No, of course
not. Your friends will love me or I’ll have them killed. More to
the point, they’ll love me because you love me and I love you and
they love you, too.”

“What does Ernie think of me?” he said, thinking of her brother for
the first time since they’d hooked up all those months ago.

“Oh, hum,” she said. He stiffened. “No, it’s OK,” she said, rubbing
his tummy some more. It tickled. “He’s glad I’m with someone I care
about, and he loves the ride. He’s just, you know. Protective of
his big sister.”

“What’s he worried about?”

“Just what you’d expect. We live thousands of miles apart. You’re
ten years older than me. You’ve been getting into the kind of
trouble that attracts armed cops. Wouldn’t you be protective if you
were my bro?”

“I was an only child, but sure, OK, I see that.”

“It’s nothing,” she said. “Really. Bring him a nice souvenir from
Florida when we come back to Madison, take him out for a couple
beers and it’ll all be great.”

“So we’re cool? All the families are in agreement? All the stars
are in alignment? Everything is hunky and/or dory?”

“Perry Gibbons, I love you dearly. You love me. We’ve got a cause
to fight for, and it’s a just one with many brave comrades fighting
alongside of us. What could possibly go wrong?”

“What could possibly go wrong?” Perry said. He drew in a breath to
start talking.

“It was rhetorical, goofball. It’s also three in the morning.
Sleep, for tomorrow we fly.”

\begin{center}\rule{3in}{0.4pt}\end{center}

Lester didn’t want to open the ride, but someone had to. Someone
had to, and it wasn’t Perry, who was off with his midwestern honey.
Lester would have loved to sleep in and spend the day in his
workshop rebuilding his 64-bit registers\dash{}he’d had some good ideas
for improving on the initial design, and he still had the CAD
files, which were the hard part anyway.

He walked slowly across the parking lot, the sunrise in his eyes, a
cup of coffee steaming in his hand. He’d almost gone to the fatkins
bars the night before\dash{}he’d almost gone ten, fifteen times, every
time he thought of Suzanne storming out of his lab, but he’d stayed
home with the TV and waited for her to turn up or call or post
something to her blog or turn up on IM, and when none of those
things had happened by 4AM, he tumbled into bed and slept for three
hours until his alarm went off again.

Blearily, he sat himself down behind the counter, greeted some of
the hawkers coming across the road, and readied his ticket-roll.

The first customers arrived just before nine\dash{}an East Indian family
driving a car with Texas plates. Dad wore khaki board-shorts and a
tank-top and leather sandals, Mom was in a beautiful silk sari, and
the kids looked like mall-bangbangers in designer versions of the
stuff the wild kids in the shantytown went around in.

They came out of the ride ten minutes later and asked for their
money back.

“There’s nothing in there,” the dad said, almost apologetically.
“It’s empty. I don’t think it’s supposed to be empty, is it?”

Lester put the roll of tickets into his pocket and stepped into the
Wal-Mart. His eyes took a second to adjust to the dark after the
brightness of the rising Florida sun. When they were fully
adjusted, though, he could see that the tourist was right. Busy
robots had torn down all the exhibits and scenes, leaving nothing
behind but swarming crowds of bots on the floor, dragging things
offstage. The smell of the printers was hot and thick.

Lester gave the man his money back.

“Sorry, man, I don’t know what’s going on. This kind of thing
should be impossible. It was all there last night.”

The man patted him on the shoulder. “It’s all right. I’m an
engineer\dash{}I know all about crashes. It just needs some debugging,
I’m sure.”

Lester got out a computer and started picking through the logs.
This kind of failure really should be impossible. Without manual
oversight, the bots weren’t supposed to change more than five
percent of the ride in response to another ride’s changes. If all
the other rides had torn themselves down, it might have happened,
but they hadn’t, had they?

No, they hadn’t. A quick check of the logs showed that none of the
changes had come from Madison, or San Francisco, or Boston, or
Westchester, or any of the other ride-sites.

Either his robots had crashed or someone had hacked the system. He
rebooted the system and rolled it back to the state from the night
before and watched the robots begin to bring the props back from
offstage.

How the hell could it have happened? He dumped the logs and began
to sift through them. He kept getting interrupted by riders who
wanted to know when the ride would come back up, but he didn’t
know, the robots’ estimates were oscillating wildly between ten
minutes and ten hours. He finally broke off to write up a little
quarter-page flier about it and printed out a couple hundred of
them on some neon yellow paper stock he had lying around, along
with a jumbo version that he taped over the price-list.

It wasn’t enough. Belligerent riders who’d traveled for hours to
see the ride wanted a human explanation, and they pestered him
ceaselessly. All the hawkers felt like they deserved more
information than the rubes, and they pestered him even more. All he
wanted to do was write some regexps that would help him figure out
what was wrong so he could fix it.

He wished that Death kid would show up already. He was supposed to
be helping out from now on and he seemed like the kind of person
who would happily jaw with the marks until the end of time.

Eventually he gave up. He set the sign explaining what had happened
(or rather, not explaining, since he didn’t fucking know yet) down
in the middle of the counter, bolted it down with a couple of
lock-bolts, and retreated to the ride’s interior and locked the
smoked-glass doors behind him.

Once he had some peace and quiet, it took only him a few minutes to
see where the changes had originated. He verified the info three
times, not because he wasn’t sure, but because he couldn’t tell if
this was good news or bad news. He read some blogs and discovered
lots of other ride-operators were chasing this down but none of
them had figured it out yet.

Grinning hugely, he composed a hasty post and CCed it to a bunch of
mailing lists, then went out to find Kettlebelly and Tjan.

He found them in the guesthouse, sitting down to a working
breakfast, with Eva and the kids at the end of the table. Tjan’s
little girl was trying to feed Pascal, but not doing a great job of
it; Tjan’s son sat on his lap, picking at his clown-face pancakes.

“Morning guys!”

Suzanne narrowed her eyes and looked away. The table fell
quiet\dash{}even the kids sensed that something was up. “Who’s watching
the ride, Lester?” Tjan asked, quietly.

“It’s shut,” he said cheerfully.

“\emph{Shut}?” Tjan spoke loudly enough that everyone jumped a
little. Lyenitchka accidentally stabbed Pascal with the spoon and
he started to wail. Suzanne stood up from the table and walked
quickly out of the guesthouse, holding on to her phone as a kind of
thin pretense of having to take a call. Lester chose to ignore
her.

Lester held his hands out placatingly. “It’s OK\dash{}it’s just down for
a couple hours. I had to reset it after what happened last night.”

Lester waited.

“All right,” Eva said, “I’ll bite. What happened last night?”

“Brazil came online!” Lester said. “Like twenty rides opened there.
But they got their protocol implementation a little wrong so when I
showed up, the whole ride had been zeroed out. I’m sure I can help
them get it right; in the meantime I’ve got the ride resetting
itself and I’ve blackholed their changes temporarily.” He grinned
sunnily. “How fucking cool is that? Brazil!”

They smiled weakly back. “I don’t think I understand, Lester,”
Kettlewell said. “Brazil? We don’t have any agreements with anyone
in Brazil.”

“We have agreements with everyone in Brazil!” Lester said. “We’ve
got an open protocol and a server that anyone can connect to.
That’s an agreement, that’s all a protocol is.”

Kettlewell shook his head. “You’re saying that all anyone needed to
do to reprogram our ride\dash{}”

“\dash{}was to connect to it and send some changes. Trust is assumed in
the system.”

“Trust is \emph{assumed}? You haven’t changed this?”

Lester took a step back. “No, I haven’t changed it. The whole
system is open\dash{}that’s the point. We can’t just start requiring
logins to get on the network. The whole thing would collapse\dash{}it’d
be like putting locks on the bathroom and then taking the only key
for yourself. We just can’t do it.”

Kettlewell looked like he was going to explode. Tjan put a hand on
his arm. Slowly, Kettlewell sat back down. Tjan took a sip of his
coffee.

“Lester, can you walk me through this one more time?”

Lester rocked back and forth a little. They were all watching him
now, except for Suzanne, who was fuming somewhere or getting ready
to go home to Russia, or something.

“We have a published protocol for describing changes to the
ride\dash{}it’s built on Git3D’s system for marking up and syncing
three-d models of objects; it’s what we used all through the
Kodacell days for collaboration. The way you get a ride online is
to sync up with our version-server and then instantiate a copy.
Then any changes you make get synced back and we instantiate them.
Everyone stays in sync, give or take a couple hours.”

“But you had passwords on the Subversion server for objects,
right?”

“Yeah, but we didn’t design this one to take passwords. It’s a lot
more ad-hoc\dash{}we wanted to be sure that people we didn’t know could
get in and play.”

Kettlewell put his face in his hands and groaned.

Tjan rolled his eyes. “I think what Kettlewell’s trying to say is
that things have changed since those carefree days\dash{}we’re in a spot
now where if Disney or someone else who hated us wanted to attack
us, this would be a prime way of doing it.”

Lester nodded. “Yeah, I figured that. Openness always costs
something. But we get a lot of benefits out of openness too. The
way it works now is that no one ride can change more than five
percent of the status quo within 24 hours without a manual
approval. The problem was that the Brazilians opened, like,
\emph{fifty} rides at the same time, and each of them zeroed out
and tried to sync that and between them they did way more than 100
percent. It’d be pretty easy to set things up so that no more than
five percent can be changed, period, within a 24-hour period,
without manual approval.”

“If you can do that, why not set every change to require approval?”
Kettlewell said.

“Well, for starters because we’d end up spending all our time
clicking OK for five-centimeter adjustments to prop-positioning.
But more importantly, it’s because the system is all about
community\dash{}we’re not in charge, we’re just part of the network.”

Kettlewell made a sour face and muttered something. Tjan patted his
arm again. “You guys \emph{are} in charge, as much as you’d like
not to be. You’re the ones facing the legal hassles, you’re the
ones who invented it.”

“We didn’t, really,” Lester said. “This was a real standing on the
shoulders of giants project. We made use of a bunch of stuff that
was on the shelf already, put it together, and then other people
helped us refine it and get it working well. We’re just part of the
group, like I keep saying.” He had a thought. “Besides, if we were
in charge, Brazil wouldn’t have been able to zero us out.

“You guys are being really weird and suit-y about this, you know?
I’ve fixed the problem: no one can take us down like this again. It
just won’t happen. I’ve put the fix on the version-server for the
codebase, so everyone else can deploy it if they want to. The
problem’s solved. We’ll be shut for an hour or two, but who cares?
You’re missing the big picture: Brazil opened fifty rides
\emph{yesterday}! I mean, it sucks that we didn’t notice until it
screwed us up, but Brazil’s got it all online. Who’s next? China?
India?”

“Russia?” Kettlewell said, looking at the door that Suzanne had
left by. He was clearly trying to needle Lester.

Lester ignored him. “I’d love to go to Brazil and check out how
they’ve done it. I speak a little Portuguese even\dash{}enough to say,
’Are you 18 yet?’ anyway.”

“You’re \emph{weird},” Lyenitchka said. Ada giggled and said,
“Weird!”

Eva shook her head. “The kids have got a point,” she said. “You
people are all a little weird. Why are you fighting? Tjan, Landon,
you came here to manage the business side of things, and that’s
what you’re doing. Lester, you’re in charge of the creative and
technical stuff and that’s what you’re doing. Without Lester, you
two wouldn’t have any business to run. Without these guys, you’d be
in jail or something by now. Make peace, because you’re on the same
side. I’ve got enough children to look after here.”

Kettlewell snapped a nod at her. “Right as ever, darling. OK, I
apologize, all right?”

“Me too,” Lester said. “I was kidding about going to Brazil\dash{}at
least while Perry’s still away.”

“He’s coming home,” Tjan said. “He called me this morning. He’s
bringing the girl, too.”

“Yoko!” Lester said, and grinned. “OK, someone should get online
and find out how all the other rides are coping with this. I’m sure
they’re going nutso out there.”

“You do that,” Kettlewell said. “We’ve got another call with the
lawyers in ten minutes.”

“How’s all that going?”

“Let me put it this way,” Kettlewell said, and for a second he was
back in his glory days, slick and formidable, a shark. “I
liquidated my shares in Disney this morning. They’re down fifty
points since the NYSE opened. You wait until Tokyo wakes up,
they’re going to bail and bail and bail.”

Lester smiled back. “OK, well that’s good, then.”

He hunkered down with a laptop and got his homebrew wireless rig up
and running\dash{}a card would have been cheaper, but his rig gave him
lots of robustness against malicious interference, multi-path and
plain old attenuation\dash{}and got his headline reader running.

He set to reading the posts and dispelling the popups that tried to
call his attention to this or that. His filters had lots to tell
him about, and the areas of his screen designated for different
interests were starting to pinken as they accumulated greater
urgency.

He waved them away and concentrated on getting through to all the
ride-maintainers who had questions about his patches. But there was
one pink area that wouldn’t go. It was his serendipity zone, where
things that didn’t match his filters but had lots of
interestingness\dash{}comments and reposts from people he paid attention
to\dash{}and some confluence with his keywords turned up.

Impatiently, he waved it up, and a page made of bits of
LiveJournals and news reports and photo-streams assembled itself.

His eye fell first on the photos. But for the shock of black and
neon green hair, he wouldn’t have recognized the kid in the
pictures as Death Waits. His face was a ruin. His nose was a bloody
rose, his eyes were both swollen shut. One ear was
ruined\dash{}apparently he’d been dragged some distance with that side of
his head on the ground. His cheeks were pulpy and bruised. Then he
clicked through to the photos from where they’d found Death, before
they’d cleaned him up in the ambulance, and he had to turn his head
away and breathe deeply. Both legs and both arms were clearly
broken, with at least one compound fracture. His crotch\dash{}Jesus.
Lester looked away again, then quickly closed the window.

He switched to text accounts from Death’s friends who’d been to see
him in the hospital. He would live, but he might not walk again. He
was lucid, and he was telling stories about the man who’d beaten
him\dash{}

\emph{You should just shut the fuck up about Disney on the fucking Internet, you know that, kid?}

Lester got up and went to find Kettlewell and Tjan and Suzanne\dash{}oh,
especially Suzanne\dash{}again. He didn’t think for one second that Death
would have invented that. In fact, it was just the sort of brave
thing that the gutsy little kid might have had the balls to report
on.

Every step he took, he saw that ruin of a face, the compound
fracture, the luminous blood around his groin. He made it halfway
to the guesthouse before he found himself leaning against a shanty,
throwing up. Tears and bile streaming down his face, chest heaving,
Lester decided that this wasn’t about fun anymore. Lester came to
understand what it meant to be responsible for people’s lives. When
he stood up and wiped his face on the tail of his tight, glittering
shirt, he was a different person.

\begin{center}\rule{3in}{0.4pt}\end{center}

Sweating in the suffocating afternoon heat, his re-casted arm on
fire, Hilda had shown him the article about Death Waits while they
were being screened for their connection at O’Hare. The TSA guy was
swabbing his cast with a black-powder residue detector, and as
Perry read it, he let out an involuntary yelp and a jump that sent
him back for a full round of tertiary screening. No date with Dr.
Jellyfinger, though it was a close thing.

Hilda was deep in her own phone, probing ferociously at it,
occasionally picking it up and talking into it, then poking at it
some more. Neither of them looked out the windows much, though in
his mind, Perry had rehearsed this homecoming as a kind of tour of
his territory, picking out which absurd landmarks he’d point out,
which funny stories he’d tell, pausing to nuzzle Hilda’s throat.

But by the time he’d absorbed the mailing-list traffic and done a
couple phoners with the people back in Madison\dash{}particularly Ernie,
who was freaking about Death Waits and calling for tight physical
security for all their people\dash{}they were pulling in at the ride. The
cabbie, a Turk, wasn’t very cool about the neighborhood, and he
kept slowing down on the side of the road and offering to let them
out there, and Perry kept insisting that he take them all the way.

“No, you can’t just drop me here, man. For the tenth time, I’ve got
a fucking \emph{cast} on my \emph{broken arm}. I’m not carrying my
suitcase a mile from here. I live there. It’s safe. God, it’s not
like I’m asking you to take me to a war-zone.”

He didn’t want to tip the guy, but he did. The cabbie was just
trying to play it safe. Lots of people tried to play it safe. It
didn’t make them assholes, even if it did make them ineffectual and
useless.

While Perry tipped him, Hilda pulled the suitcase out of the cab’s
trunk and she’d barely had time to shut the lid when the driver
roared off like he was trying to outrun a sniper.

Perry grimaced. This was supposed to be a triumphant homecoming. He
was supposed to be showing off his toys, all he’d wrought, to this
girl. The town was all around them and they were about to charge in
without even pausing to consider its Dr Seuss wonderment.

“Wait a sec,” Perry said. He took her hand. “See that? That was the
first shanty they built. Five stories now.” The building was made
of prefab concrete for the first couple stories, then successively
lighter materials, with the roof-shack made of bamboo. “The designs
are experimental, from the Army Corps of Engineers mostly, but they
say they’ll stand a force-five hurricane.” He grimaced again.
“Probably not the bamboo one, of course.”

“Of course,” Hilda said. “What’s that one?” She’d picked up on his
mood, she knew he wanted to show her around before they ended up
embroiled in ride-politics and work again.

“You’ve got a good eye, my dear. That’s the finest BBQ on the
continent. See how the walls are a little sooty looking? That’s
carbonized ambrosia, a mix of fat and spice and hickory that you
could scrape off and bottle as perfume.”

“Eww.”

“You haven’t tried Lemarr’s ribs yet,” he said, and goosed her. She
squeaked and punched him in the shoulder. He showed her the
tuck-shops, the kids playing, the tutor’s place, the day-care
center, the workshops, taking her on a grand-circle tour of this
place he’d help conjure into existence.

“Now there’s someone I haven’t seen in far too long,” Francis said.
He’d aged something fierce in the last year, booze making his face
subside into a mess of wrinkles and pouches and broken
blood-vessels. He gave Perry a hard hug that smelled of booze, and
it wasn’t even lunchtime.

“Francis, meet Hilda Hammersen; Hilda, meet Francis Clammer:
aerospace engineer and gentleman of leisure.”

He took her hand and feinted a kiss at it, and Hilda good-naturedly
rolled her eyes at this.

“What do you think of our lovely little settlement, then, Ms
Hammersen?”

“It’s like something out of a fairy-tale,” she said. “You hear
stories about Christiania and how good and peaceful it all was, but
whenever you see squatters on TV, it’s always crack houses and
drive-bys. You’ve really got something here.”

Francis nodded. “We get a bad rap, but we’re no different really
from any other place where people take pride in what they own. I
built my place, with my two hands. If Jimmy Carter had been there
with Habitat for Humanity, we would have gotten no end of good
press. Because we did it without a dead ex-president on the scene,
we’re crooks. Perry tell you about what the law does around here?’

Perry nodded. “Yeah. She knows.”

Francis patted his cast. “Nice hardware, buddy. So when some
Bible-thumping do-gooder gives you a leg up, you’re a folk-hero.
Help yourself, you’re a CHUD. It’s the same with you people and
your ride. If you had the backing of a giant corporation with claws
sunk deep into kids’ brains, you’d be every package-tour operator’s
wet dream. Build it yourself in the guts of a dead shopping center,
and you’re some kind of slimy underclass.”

“Maybe that’s true,” said Hilda. “But it’s not necessarily true.
Back in Madison, the locals love us, they think we do great stuff.
After the law came after us, they came by with food and money and
helped us rebuild. Scrappy activists get a lot of love in this
country, too. Not everyone wants a big corporation to spoon-feed
them.”

“Off in hippie college-towns you’ll always find people with enough
brains to realize that their neighbors aren’t the boogieman. But
there ain’t so many hippie college towns these days. I wish you two
luck, but I think you’d be nuts to walk out the door in the morning
expecting anything better than a kick in the teeth.”

That made Perry think of Death Waits, and the sense of urgency came
back to him. “OK, we have to go now,” he said. “Thanks, Francis.”

“Nice to meet you, young woman,” he said, and when he smiled, it
was a painful thing, all pouches and wrinkles and sags, and he
gimped away with his limp more pronounced than ever.

They tracked down the crew at the tea-house’s big table. Everyone
roared greetings at them when they came through the door, a proper
homecoming, but when Perry counted heads, he realized that there
was no one watching the ride.

“Guys, who’s running the ride?”

They told him about Brazil then, and Hilda listened with her head
cocked, her face animated with surprise, dismay, then delight. “You
say there are \emph{fifty} rides open?”

“All at once,” Lester said. “All in one go.”

“Holy mother of poo,” Hilda breathed. Perry couldn’t even bring
himself to say \emph{anything}. He couldn’t even imagine Brazil in
his head\dash{}jungles? beaches? He knew nothing about the country.
They’d built \emph{fifty} rides, without even making contact with
him. He and Lester had designed the protocol to be open because
they thought it would make it easier for others to copy what they’d
done, but he’d never thought\dash{}

It was like vertigo, that feeling.

“So you’re Yoko, huh?” Lester said finally. It made everyone smile,
but the tension was still there. Something big had just happened,
bigger than any of them, bigger than the beating that had been laid
on Death Waits, bigger than anything Perry had ever done. From his
mind to a nation on another continent\dash{}

“You’re the sidekick, huh?” Hilda said.

Lester laughed. “Touche. It’s very nice to meet you and thank you
for bringing him back home. We were starting to miss him, though
God alone knows why.”

“I plan on keeping him,” she said, giving his bicep a squeeze. It
brought Perry back to them. The little girls were staring at Hilda
with saucer eyes. It made him realize that except for Suzanne and
Eva, their whole little band was boys, all boys.

“Well, I’m home now,” he said. He knelt down and showed the girls
his cast. “I got a new one,” he said. “They had to throw the old
one out. So I need your help decorating this. Do you think you
could do the job?”

Lyenitchka looked critically at the surface. “I think we could do
the gig,” she said. “What do you think, partner?”

Tjan snorted out his nose, but she was so solemn that the rest kept
quiet. Ada matched Lyenitchka’s critical posture and then nodded
authoritatively. “Sure thing, partner.”

“It’s a date,” Perry said. “We’re gonna head home and put down our
suitcases and come back and open the ride if it’s ready. It’s time
Lester got some time off. I’m sure Suzanne will appreciate having
him back again.”

Another silence fell over the group, tense as a piano wire. Perry
looked from Lester to Suzanne and saw in a second what was up. He
had time to notice that his first emotional response was to be
intrigued, not sorry or scared. Only after a moment did he have the
reaction he thought he should have\dash{}a mixture of sadness for his
friend and irritation that they had yet another thing to deal with
in the middle of a hundred other crises.

Hilda broke the tension\dash{}“It was great to meet you all. Dinner
tonight, right?”

“Absolutely,” Kettlewell said, seizing on this. “Leave it to
us\dash{}we’ll book someplace just great and have a great dinner to
welcome you guys back.”

Eva took his arm. “That’s right,” she said. “I’ll get the girls to
pick it out.” The little girls jumped up and down with excitement
at this, and the baby brothers caught their excitement and made
happy kid-screeches that got everyone smiling again.

Perry gave Lester a solemn, supportive hug, kissed Suzanne and Eva
on the cheeks (Suzanne smelled good, something like sandalwood),
shook hands with Tjan and Kettlewell and tousled all four kids
before lighting out for the ride, gasping out a breath as they
stepped into the open air.

\begin{center}\rule{3in}{0.4pt}\end{center}

Death Waits regained consciousness several times over the next
week, aware each time that he was waking up in a hospital bed on a
crowded ward, that he’d woken here before, and that he hurt and
couldn’t remember much after the beating had started.

But after a week or so, he found himself awake and aware\dash{}he still
hurt all over, a dull and distant stoned ache that he could tell
was being kept at bay by powerful painkillers. There was someone
waiting for him.

“Hello, Darren,” the man said. “I’m an attorney working for your
friends at the ride. My name is Tom Levine. We’re suing Disney and
we wanted to gather some evidence from you.”

Death didn’t like being called Darren, and he didn’t want to talk
to this dork. He’d woken up with a profound sense of anger,
remembering the dead-eyed guy shouting about Disney while bouncing
his head off the ground, knowing that Sammy had done this, wanting
nothing more than to get ahold of Sammy and, and\ldots{} That’s where he
ran out of imagination. He was perfectly happy drawing
medieval-style torture chambers and vampires in his sketch book,
but he didn’t actually have much stomach for, you know,
\emph{violence}.

Per se.

“Can we do this some other time?” His mouth hurt. He’d lost four
teeth and had bitten his tongue hard enough to need stitches. He
could barely understand his own words.

“I wish we could, but time is of the essence here. You’ve heard
that we’re bringing a suit against Disney, right?”

“No,” Death said.

“Must have come up while you were out. Anyway, we are, for unfair
competition. We’ve got a shot at cleaning them out, taking them for
every cent. We’re going through the pre-trial motions now and
there’s been a motion to summarily exclude any evidence related to
your beating from the proceedings. We think that’s BS. It’s clear
from what you’ve told your friends that they wanted to shut you up
because you were making them look bad. So what we need is more
information from you about what this guy said to you, and what
you’d posted before, and anything anyone at Disney said to you
while you were working there.”

“You know that that guy said he was beating me up because I talked
about this stuff in the first place?”

The lawyer waved a hand. “There’s no way they’ll come after you
now. They look like total assholes for doing this. They’re scared
stupid. Now, I’m going to want to formally depose you later, but
this is a pre-deposition interview just to get clear on
everything.”

The guy leaned forward and suddenly Death Waits had a bone-deep
conviction that the guy was about to punch him. He gave a little
squeak and shrank away, then cried out again as every inch of his
body awoke in hot agony, a feeling like grating bones beneath his
skin.

“Woah, take it easy there, champ,” the lawyer said.

Death Waits held back tears. The guy wasn’t going to hit him, but
just the movement in his direction had scared him like he’d leapt
out holding an axe. The magnitude of his own brokenness began to
sink in and now he could barely hold back the tears.

“Look, the guys who run the ride have told me that I have to get
this from you as soon as I can. If we’re going to keep the ride
safe and nail the bastards who did this to you, I need to do this.
If I had my way, I wouldn’t bug you, but I’ve got my orders, OK?”

Death snuffled back the tears. The back of his throat felt like it
had been sanded with a rusty file. “Water,” he croaked.

The lawyer shook his head. “Sorry buddy, just the IV, I’m afraid.
The nurses were very specific. Let’s start, OK, and then we’ll be
done before you know it.”

Defeated, Death closed his eyes. “Start,” he said, his voice like
something made from soft tar left too long in the sun.

\begin{center}\rule{3in}{0.4pt}\end{center}

Sammy knew he was a dead man. The only thing keeping him alive was
legal’s reluctance to read the net. Hackelberg had a couple of
juniors who kept watch-lists running on hot subjects, but they
liked to print them out and mark them up, and that meant that they
lagged a day or two behind the blogosphere.

The Death Waits thing was a freaking disaster. The guy was just
supposed to put a scare into him, not cripple him for life. Every
time Sammy thought about what would happen when the Death Waits
thing percolated up to him, he got gooseflesh.

Damn that idiot thug anyway. Sammy had been very clear. The guy who
knew the guy who knew the guy had been reassuring on the phone when
Sammy put in the order\dash{}sure, sure, nothing too rough, just a little
shoving around.

And what’s worse is the idiot kid hadn’t gotten the hint. Sammy
didn’t get it. If a stranger beat him half to death and told him to
stop hanging out in message-boards, well, the message-boards would
go. Damned right they would.

And with Freddy, there was a shoe waiting to drop. Freddy wouldn’t
report on their interview, he was pretty sure of that. “Off the
record” means something, even to “journalists” like Honest Freddy.
But Freddy wasn’t going to be nice to him in follow-ups, that much
was sure. And if\dash{}when!\dash{}Freddy got wind of the Death Waits
situation\ldots

He began to hyperventilate.

“I’m going to go check on the construction,” he said to his
personal assistant, a new girl they’d sent up when his last one had
defected to work for Wiener (Wiener!) after Sammy’d shouted at her
for putting through a press-call from some blogger who wanted to
know when Fantasyland would be re-opening.

It had been a mistake to shut down Fantasyland just to get the
other managers off his back. Sure the rides were sick dogs, but
there had been life in them still. Construction sites don’t bring
in visitors, and the numbers for the park were down and everyone
was looking at him. Never mind that the only reason the numbers had
been as high as they were was that Sammy had saved everyone’s ass
when he’d done the goth rehab. Never mind that the real reason that
numbers were down was that no one else in management had the guts
to keep the park moving and improving.

He slowed his step on Main Street, USA, and forced himself to pay
attention to his surroundings. The stores on Main Street had been
co-opted into helping him dump all the superfluous goth
merchandise, and it was in their windows and visible through their
doors. The fatkins pizza-stands and ice-cream wagons were doing a
brisk trade around the castle roundabout. The crowd was
predominantly veering to the left, toward Adventureland and
Frontierland and Liberty Square, while the right side of the plaza,
which held the gateways to Fantasyland and Tomorrowland, was
conspicuously sparse. He’d known that his numbers were down, but
standing in the crowd’s flow, he could feel it.

He cleared the castle and stood for a moment at the brink of
Fantasyland. It should be impossible to stand here at one in the
afternoon\dash{}there should be busy rushes of people pushing past to get
on the rides and to eat and to buy stuff, but now there were just a
few kids in eyeliner puffing cloves in smokeless hookahs and a
wasteland of hoardings painted a shade Imagineering called “go-away
green” for its ability to make the eye slide right past it.

He’d left the two big coasters open, and they had decent queues,
but that was it. No one was in the stores, and no one was bothering
with the zombie maze. Clouds of dust and loud destruction noises
rose over the hoardings, and he slipped into a staff door and
threaded his way onto one of the sites, pausing to pick up a safety
helmet with mouse-ears.

At least these crews were efficient. He’d long ago impressed on the
department that hired construction contractors the necessity of
decommissioning old rides with extreme care so as to preserve as
much of the collectible value of the finishings and trim as
possible. It was a little weird\dash{}Disney customers howled like stuck
pigs when you shut down their rides, then fought for the chance to
spend fortunes buying up the dismembered corpses of their favored
amusements.

He watched some Cuban kids carefully melting the hot glue that had
held the skull trim-elements to the pillar of the Dia de los
Muertos facade, setting them atop a large pile of other
trim\dash{}scythes, hooded figures, tombstones\dash{}with a layer of aerogel
beneath to keep the garriture from scratching. The whole area
behind the hoardings was like this\dash{}rides in pieces, towers of
fiberglass detritus sandwiched between layers of aerogel.

They’d done this before, when he’d taken Fantasyland down, and he’d
fretted every moment about how long the tear-down was taking. There
were exciting new plans lurking in the wings then, waiting to leap
onstage and take shape. He’d had some of the ride components
fabricated by a contractor in Kissimmee, but large chunks of the
construction had to take place onsite. The advantage had been his:
cheap fabricators, new materials, easy collaboration between remote
contractors and his people on-site. No one had ever executed new
rides as fast and as well as he had. The things had basically built
themselves.

Now the competition was using the same tech and it was a fucking
disaster for him. Worse and worse: he had no plans for what was to
come afterward. He’d thought that he’d just grab some of the
audience research people, throw together a fatkins focus group or
two, and give Imagineering two weeks to come up with some designs
they could put up fast. He knew from past experience that design
expanded to fill the time available to it, and that the best stuff
usually emerged in the first ten days anyway, and after that it was
all committee group-think.

But no one from audience research wanted to return his calls, no
one from Imagineering was willing to work for him, and no one
wanted to visit a section of the park that was dominated by
construction hoardings and demolition dust.

What the hell was happening at the Miami ride, anyway? He could
follow it online, run the three-d flythroughs of the ride as it
stood, even download and print his own versions of the ride
objects, but none of that told him what it \emph{felt like} to get
on the ride, to be in its clanking bowels, surrounded by other
riders, pointing and marveling and laughing at the scenes and
motion.

Rides were things that you had to ride to understand. Describing a
ride was like talking about a movie\dash{}so abstract and remote. Like
talking about sex versus having sex.

Sammy loved rides. Or he used to, anyway. So much more than films,
so much more than books\dash{}so immersive and human, and the whole crowd
thing, all the other people waiting to ride it or just getting off
it. It had started with coasters\dash{}doesn’t every kid love
coasters?\dash{}but he’d ended up a connoisseur, a gourmand who loved
every species of ride, from thrill-rides to monorails, carousels to
dark-rides.

There’d been a time when he’d ridden every ride in the park once a
week, and every ride in every nearby park once a month. That had
been years before. Now he sat in an office and made important
decisions and he was lucky if he made it onto a ride once a week.

Not that it mattered anymore. He’d screwed up so bad that it was
only a matter of time until he ended up on the bread-line. Or in
jail.

He realized he was staring glumly at the demolition, and pulled
himself upright, sucked in a few breaths, mentally kicked himself
in the ass and told himself to stop feeling sorry for himself.

A young woman pried loose another resin skull finial and added it
to the pile, placed another sheet of aerogel on top of it.

People loved these little tchotchkes. They had a relationship with
Disney Parks that made them want to come again and again, to own a
piece of the place. They came for visits and then they visited in
their hearts and they came back to bring their hearts home. It was
an extremely profitable dynamic.

That’s what those ride people up in the Wal-Mart were making their
hay on\dash{}anyone could replicate the ride in their back-yard. You
didn’t have to fly from Madison to Orlando to have a little
refresher experience. It was right there, at the end of the road.

If only there was some way to put his rides, his park, right there
in the riders’ homes, in their literal back-yards. Being able to
look at the webcams and take a three-d fly-through was one thing,
but it wasn’t the physical, visceral experience of being there.

The maintenance crew had finished all the trim and now they were
going after the props and animatronics. They never used to sell
these off, because manufacturing the guts of a robot was too
finicky to do any more than you had to\dash{}it was far better to
repurpose them, like the America Sings geese that had all their
skin removed and found a new home as smart-talking robots in the
pre-show for the old Star Tours.

But now it all could be printed to order, fabbed and shipped in.
They weren’t even doing their own machining at Imagineering
anymore\dash{}that was all mail-order fulfillment. Just email a three-d
drawing to a shop and you’d have as many as you wanted the next
day, FedEx guaranteed. Sammy’s lips drew back from his teeth as he
considered the possibility that the Wal-Mart ride people had
ordered their parts from the same suppliers. Christ on a bike, what
a mess.

And there, in the pit of despair, at the bottom of his downward
arc, Sammy was hit by a bolt of inspiration:

Put Disney into people’s living rooms! Put printers into their
homes that decorated a corner of their rooms with a replica of a
different ride every day. You could put it on a coffee table, or
scale it up to fill your basement rumpus-room. You could have a
magic room that was a piece of the park, a souvenir that never let
go of Disney, there in your home. The people who were willing to
spend a fortune on printed skull finials would cream for this! It
would be like actually living there, in the park. It would be
Imagineering Eye for the Fan Guy.

He could think of a hundred ways to turn this into money. Give away
the printers and sell subscriptions to the refresh. Sell the
printers and give away the refreshes. Charge sponsors to modify the
plans and target different product placements to different users.
The possibilities were endless. Best of all, it would extend the
reach of Disney Parks further than the stupid ride could ever go\dash{}it
would be there, on the coffee table, in the rumpus room, in your
school gym or at your summer place.

He loved it. Loved it! He actually laughed aloud. What a
\emph{great} idea! Sure he was in trouble\dash{}big trouble. But if he
could get this thing going\dash{}and it would go, \emph{fast}\dash{}then
Hackelberg would get his back. The lawyer didn’t give a shit if
Sammy lived or died, but he would do anything to protect the
company’s interests.

Sure, no one from Imagineering had been willing to help him design
new rides. They all had all the new ride design projects they could
use. Audience research too. But this was new, \emph{new new}, not
old new, and new was always appealing to a certain kind of novelty
junkie in Imagineering. He’d find help for this, and then he’d pull
together a business-plan, and a timeline, and a critical path, and
he’d start executing. He wanted a prototype out the door in a week.
Christ, it couldn’t be that hard\dash{}those Wal-Mart ride assholes had
published the full schematics for their toys already. He could just
rip them off. Turnabout is fair play, after all.

\begin{center}\rule{3in}{0.4pt}\end{center}

Hilda left Perry after a couple hours working the ticket-booth
together. She wanted to go for a shower and a bit of an explore,
and it was a secret relief to both of them to get some time apart
after all that time living in each others’ pockets. They were
intimate strangers still, not yet attuned to each others’ moods and
needs for privacy, and a little separation was welcome.

Welcome, too, was Perry’s old post there at the ticket counter,
like Lucy’s lemonade stand in Peanuts. The riders came on thick, a
surprising number of them knew his name and wanted to know how his
arm was. They were all watching the drama unfold online. They knew
about the Brazilian rides coming online and the patch Lester had
run. They all felt a proprietary interest in this thing. It made
him feel good, but a little weird. He could deal with having
friends, and customers, but fans?

When he got off work, he wandered over to the shantytown with a
bunch of the vendors, to have a customary after-work beer and plate
of ribs. He was about to get his phone out and find Hilda when he
spotted her, gnawing on a greasy bone with Suzanne and Eva.

“Well, \emph{hello}!” he said, delighted, skipping around the
barbecue pit to collect a greasy kiss from Hilda, and more chaste
but equally greasy pecks on the cheek from Suzanne and Eva. “Looks
like you’ve found the best place in town!”

“We thought we’d show her around,” Suzanne said. She and Eva had
positioned each other on either side of Hilda, using her as a
buffer, but it was great to see that they were on something like
speaking terms. Perry had no doubt that Suzanne hadn’t led
Kettlewell on (they all had crushes on her, he knew it), but that
didn’t mean that Eva wouldn’t resent her anyway. If their positions
were reversed, he would have had a hard time controlling his
jealousy.

“They’ve been wonderful,” Hilda said, offering him a rib. He
introduced her to the market-stall sellers who’d come over with him
and there was more greasy handshaking and hugging, and the
proprietor of the joint started handing around more ribs, more
beers, and someone brought out a set of speakers and suction-cupped
their induction-surfaces to a nearby wall, and Perry dropped one of
his earbuds into them and set it to shuffle and they had music.

Kids ran past them in shrieking hordes, playing some kind of big
game that they’d all been obsessed with. Perry saw that Ada and
Lyenitchka were with them, clutching brightly colored mobiles and
trying to read their screens while running away from another gang
of kids who were clearly “it,” taking exaggerated care not to run
into invisible obstacles indicated on the screens.

“It was great to get back into the saddle,” Perry said, digging
into some ribs, getting sauce on his fingers. “I had no idea how
much I’d been missing it.”

Hilda nodded. “I could tell, anyway. You’re a junkie for it. You’re
like the ones who show up all googly-eyed about the ’story’ that’s
supposedly in there. You act like that’s a holy box.”

Suzanne nodded solemnly. “She’s right. The two of you, you and
Lester, you’re so into that thing, you’re the biggest fanboys in
the world. You know what they call it, the fans, when they get
together to chat about the stuff they love? Drooling. As in, ’Did
you see the drool I posted this morning about the new girl’s
bedroom scene?’ You drool like no one’s business when you talk
about that thing. It’s a holy thing for you.”

“You guys sound like you’ve been comparing notes,” Perry said,
making his funny eyebrow dance.

Eva arched one of her fine, high eyebrows in response. In some
ways, she was the most beautiful of all of them, the most
self-assured and poised. “Of course we were, sonny. Your young lady
here needed to know that you aren’t an axe-murderer.” The women’s
camaraderie was almost palpable. Suzanne and Eva had clearly
patched up whatever differences they’d had, which was probably bad
news for Kettlewell.

“Where is Lester, anyway?” He hadn’t planned on asking, but
Suzanne’s mention of his name led him to believe he could probably
get away with it.

“He’s talking to Brazil,” Suzanne said. “It’s all he’s done, all
day long.”

Talking to Brazil. Wow. Perry’d thought of Brazil as a kind of
abstract thing, fifty rogue nodes on the network that had
necessitated a hurried software patch. Not as a bunch of people.
But of course, there they were, in Brazil, real people by the
dozens, maybe even hundreds, building rides.

“He doesn’t speak Spanish, though,” Perry said.

“Neither do they, dork,” Hilda said, giving him an elbow in the
ribs. “Portuguese.”

“They all speak some English and he’s using automated translation
stuff for the hard concepts.”

“Does that work? I mean, any time I’ve tried to translate a
web-page in Japanese or Hebrew, it’s kind of read like noun noun
noun noun verb noun random.”

Suzanne shook her head. “That’s how most of the world experiences
most of the net, Perry. Anglos are just about the only people on
earth who don’t read the net in languages other than their own.”

“Well, good for Lester then,” he said.

Suzanne made a sour face that let him know that whatever peace
prevailed between her and Lester, it was fragile. “Good for him,”
she said.

“Where are the boys?”

“Landon and Tjan have them,” Eva said. “They’ve been holed up with
your lawyers going over strategy with them. When I walked out, they
were trying to get the firm’s partners to take shares in the
corporation that owns the settlement in lieu of cash up front.”

“Man that’s all too weird for me,” Perry said. “I wish we could
just run this thing like a business: make stuff people want to give
us money for, collect the money, and spend it.”

“You are such a nerd fatalist,” Suzanne said. “Getting involved in
the more abstract elements of commerce doesn’t make you into a
suit. If you don’t participate and take an interest, you’ll always
be out-competed by those who do.”

“Bull,” Perry said. “They can get a court to order us to make pi
equal to three, or to ensure that other people don’t make Mickey
heads in their rides, or that our riders don’t think of Disney when
they get into one of our chairs, but they’ll never be able to
enforce it.”

Suzanne suddenly whirled on him. “Perry Gibbons, you aren’t that
stupid, so stop acting like you are.” She touched his cast. “Look
at this thing on your arm. Your superior technology can \emph{not}
make inferior laws irrelevant. You’re assuming that the machinery
of state is unwilling to completely shut you down in order to make
you comply with some minor law. You’re totally wrong. They’ll come
after you and break your head.”

Perry rocked back on his heels. He was suddenly furious, even if
somewhere in his heart of hearts he knew that she was right and he
was mostly angry at being shown up in front of Hilda. “I’ve been
hearing that all my life, Suzanne. I don’t buy it. Look, it just
keeps getting cheaper and easier to make something like what we’ve
built. To get a printer, to get goop, to make stuff, to download
stuff, to message and IM with people who’ll help you make stuff. To
learn how to make it. Look, the world is getting better because
we’re getting better at routing around the bullies. We can play
their game, or we can invent a new game.

“I refuse to be sucked into playing their game. If we play their
game, we end up just like them.”

Suzanne shook her head sadly. “It’s a good thing you’ve got Tjan
and Kettlewell around then, to do the dirty work. I just hope you
can spare them a little pity from atop your moral high-ground.”

She took Eva by the arm and led her away, leaving Perry, shaking,
with Hilda.

“Bitch,” he said, kicking the ground. He balled his hands into
fists and then quickly relaxed them as his broken arm ground and
twinged from the sudden tensing.

Hilda took him by the arm. “You two clearly have a \emph{lot} of
history.”

He took a couple deep breaths. “She was so out of line there. What
the hell, anyway? Why should I have to\dash{}” He stopped. He could tell
when he was repeating himself.

“I don’t think that she would be telling you that stuff if she
didn’t think you needed to hear it.”

“You sound like you’re on her side. I thought you were a fiery
young revolutionary. You think we should all put on suits and
incorporate?”

“I think that if you’ve got skilled people willing to help you, you
owe it to them to value their contribution. I’ve heard you complain
about ’suits’ twenty times in the past week. Two of those suits are
on your side. They’re putting themselves on the line, just like
you. Hell, they’re doing the shit-work while you get to do all the
inventing and fly around the country and get laid by hot
groupies.”

She kissed his cheek, trying to make a joke of it, but she’d really
hurt his feelings. He felt like weeping. It was all out of his
control. His destiny was not his to master.

“OK, let’s go apologize to Kettlewell and Tjan.”

She laughed, but he’d only been halfway kidding. What he really
wanted to do was have a big old dinner at home with Lester, just
the two of them in front of the TV, eating Lester’s fatkins
cuisine, planning a new invention. He was tired of all these
people. Even Suzanne was an outsider. It had just been him and
Lester in the old days, and those had been the best days.

Hilda put her arm around his shoulders and nuzzled his neck. “Poor
Perry,” she said. “Everyone picks on him.”

He smiled in spite of himself.

“Come on, sulkypants, let’s go find Lester and he can call me
’Yoko’ some more. That always cheers you up.”

\begin{center}\rule{3in}{0.4pt}\end{center}

It was two weeks before Death Waits could sit up and prod at a
keyboard with his broken hands. Some of his pals brought a laptop
around and they commandeered a spare dining tray to keep it
on\dash{}Death’s lap was in no shape to support anything heavy with sharp
corners.

The first day, he was reduced to tears of frustration within
minutes of starting. He couldn’t use the shift key, couldn’t really
use the mouse\dash{}and the meds made it hard to concentrate and remember
what he’d done.

But there were people on the other end of that computer, human
friends whom he could communicate with if only he could re-learn to
use this tool that he’d lived with since he was old enough to sit
up on his own.

So laboriously, peck by peck, key by key, he learned to use it
again. The machine had a mode for disabled people, for
\emph{cripples}, and once he hit on this, it went faster. The mode
tried to learn from him, learn his tremors and mis-keys, his errors
and cursing, and so emerge something that was uniquely his
interface. It was a kind of a game to watch the computer try to
guess what was meant by his mashed keystrokes and spastic
pointer-movements\dash{}he turned on the webcam and aimed it at his eye,
and switched it to retinal scanner mode, giving it control of the
pointer, then watched in amusement as the wild leaping of the
cursor every time a needle or a broken bone shifted inside his body
was becalmed into a graceful, normalized curve.

It was humiliating to be a high-tech cripple and the better the
technology worked, the more prone it was to reducing him to tears.
He might be like this for the rest of his life. He might never walk
without a limp again. Might never dance. Might never be able to
reach for and lift objects again. He’d never find a woman, never
have a family, never have grandkids.

But this was offset by the real people with their real chatter. He
obsessively flew through the Brazilian mode, strange and wonderful
but nowhere near what he loved from “his” variation on the ride. He
could roll through all the different changes he’d made with his
friends to the ride in Florida, and he became subtly attuned to
which elements were wrong and which were right.

It was on one of these flythroughs that he encountered The Story,
leaping out of the ride so vividly that he yelped like he’d flexed
his IV into a nerve again.

There it was\dash{}irrefutable and indefinable. When you rode through
there was an escalating tension, a sense of people who belonged to
these exhibits going through hard changes, growing up and out.

Once he’d seen it, he couldn’t un-see it. When he and his pals had
started to add their own stuff to the ride, the story people had
been giant pains in the ass, accusing them of something they called
“narricide”\dash{}destroying the fragile story that humanity had laid
bare there.

Now that he’d seen it too, he wanted to protect it. But he could
see by skimming forward and back through the change-log and trying
different flythroughs that the story wasn’t being undermined by the
goth stuff they were bringing in; it was being enhanced. It was
telling the story he knew, of growing up with an indefinable need
to be \emph{different}, to reject the mainstream and to embrace
this subculture and aesthetic.

It was the story of his tribe and sub-species and it got realer the
more he played it. God, how could he have \emph{missed it}? It made
him want to cry, though that might have been the meds. Some of it
made him want to laugh, too.

He tried, laboriously, to compose a message-board post that
expressed what he was feeling, but every attempt came out sounding
like those story mystics he’d battled. He understood now why they’d
sounded so hippy-trippy.

So he rode the ride, virtually, again and again, spotting the
grace-notes and the sly wit and the wrenching emotion that the
collective intelligence of all those riders had created.
Discovered? It was like the story was there all along, lurking like
the statue inside a block of marble.

Oh, it was wonderful. He was ruined, maybe forever, but it was
wonderful. And he’d been a part of it.

He went back to writing that message-board post. He’d be laid in
that bed for a long time yet. He had time to rewrite.

\begin{center}\rule{3in}{0.4pt}\end{center}

\headline{IF YOU CAN’T BEAT THEM, RIP THEM OFF}

A new initiative from the troubled Disney Parks corporation shows
how a little imagination can catapult an ambitious exec to the top
of the corporate ladder.

Word has it that Samuel R.D. Page, the Vice President for
Fantasyland (I assure you, I am NOT making that up) has been kicked
upstairs to Senior Vice President for Remote Delivery of Park
Experience (I’m not making that up, either). Insiders in the
company tell us that “Remote Delivery of Park Experience” is a plan
to convince us to give The Mouse a piece of our homes which will be
constantly refreshed via a robot three-dimensional printer with
miniatures of the Disney park.

If this sounds familiar, it should. It’s a pale imitation of the
no-less-ridiculous (if slightly less evil) “rides” movement
pioneered by Perry Gibbons and Lester Banks, previously the
anti-heroes of the New Work pump-and-dump scandal.

Imitation is meant to be the sincerest form of flattery, and if so,
Gibbons and his cultists must be blushing fire-engine red.

This is cheap irony, Disney-style. After all, it’s only been a
month since the company launched ten separate lawsuits against
various incarnations of the ride for trademark violation, and it’s
now trying to duck the punishing countersuits that have risen up in
their wake.

Most ironic of all, word has it that Page was responsible for both
ends of this: the lawsuits against the ride and the decision to
turn his company into purveyors of cheap knockoffs of the ride.

Page is best known among Park aficionados for having had the
“foresight” to gut the children’s “Fantasyland” district in Walt
Disney World and replace it with a jumped up version of Hot Topic,
a goth-themed area that drew down the nation’s eyeliner supply to
dangerously low levels.

It was apparently that sort of “way-out-of-the-box” “genius” that
led Page to his latest round of disasters: the lawsuits, an
abortive rebuilding of Fantasyland, and now this “Remote Delivery”
scam.

What’s next? The Mouse has already shipped Disney Dollars, an
abortive home-wares line, a disastrous fine-art chain, and oversaw
the collapse of the collectible cel-art market. With “visionaries”
like Page at the helm, the company can’t help but notch up more
“successes.”

\begin{center}\rule{3in}{0.4pt}\end{center}

Death was deep into the story now. The Brazilians had forked off
their own ride\dash{}they’d had their own New Work culture, too, centered
in the favelas, so they had different stories to tell. Some of the
ride operators imported a few of their scenes, tentatively, and
some of the ride fans were recreating the Brazil scenes on their
own passes through the ride.

It was all in there, if you knew where to look for it, and the best
part was, no one had written it. It had written itself. The
collective judgement of people who rode through had turned chaos
into coherence.

Or had it? The message-boards were rife with speculation that The
Story had been planted by someone\dash{}maybe the ride’s creators, maybe
some clan of riders\dash{}who’d inserted it deliberately. These
discussions bordered on the metaphysical: what was an “organic”
ride decision? It made Death Waits’s head swim.

The thing that was really doing his head in, though, was the Disney
stuff. Sammy\dash{}he couldn’t even think of Sammy without a sick feeling
in his stomach, crashing waves of nausea that transcended even his
narcotic haze\dash{}Sammy was making these grotesque parodies of the
ride. He was pushing them out to the world’s living rooms. Even the
deleted rides from the glory days of the goth Fantasyland, in
time-limited miniature. If he’d still been at Disney Parks, he
would have loved this idea. It was just what he loved, the
knowledge that he was sharing experience with his people around the
world, part of a tribe even if he couldn’t see them.

Now, in the era of the ride, he could see how dumb this was. How
thin and shallow and commercial. Why should they have to pay some
giant evil corporation to convene their community?

He kept trying to write about The Story, kept failing. It wouldn’t
come. But Sammy\dash{}he knew what he wanted to say about Sammy. He typed
until they sedated him, and then typed some more when he woke up.
He had old emails to refer to. He pasted them in.

After three days of doing this, the lawyer came back. Tom Levine
was dressed in a stern suit with narrow lapels and a tie pierced
with some kind of frat pin. He wasn’t much older than Death, but he
made Death feel like a little kid.

“I need to talk to you about your Internet activity,” he said,
sitting down beside him. He’d brought along a salt-water taffy
assortment bought from the roadside, cut into double-helix
molecules and other odd biological forms\dash{}an amoeba, a skeleton.

“OK?” Death said. They’d switched him to something new for the pain
that day, and given him a rocker-switch he could use to drizzle it
into his IV when it got bad. He’d hit it just before the lawyer
came to see him and now he couldn’t concentrate much. Plus he
wasn’t used to talking. Writing online was better. He could write
something, save it, go back and re-read it later and clean it up if
it turned out he’d gone off on a stoned ramble.

“You know we’re engaged in some very high-stakes litigation here,
right, Darren?”

He hated it when people called him Darren.

“Death,” he said. His toothless lisp was pathetic, like an old
wino’s.

“Death, OK. This high-stakes litigation needs a maximum of caution
and control. This is a fifteen-year journey that ends when we’ve
broken the back of the company that did this to you. It ends when
we take them for every cent, bankrupt their executives, take their
summer homes, freeze their accounts. You understand that?”

Death hadn’t really understood that. It sounded pretty tiring.
Exhausting. Fifteen years. He was only nineteen now. He’d be
thirty-four, and that was only if the lawyer was estimating
correctly.

“Oh,” he said.

“Well, not that you’re going to have to take part in fifteen years’
worth of this. It’s likely we’ll be done with your part in a year,
tops. But the point is that when you go online and post material
that’s potentially harmful to this case\dash{}”

Death closed his eyes. He’d posted the wrong thing. This had been a
major deal when he was at Disney, what he was and wasn’t allowed to
post about\dash{}though in practice, he’d posted about everything,
sticking the private stuff in private discussions.

“Look, you can’t write about the case, or anything involved with
it, that’s what it comes down to. If you write about that stuff and
you say the wrong thing, you could blow this whole suit. They’d get
away clean.”

Death shook his head. Not write about it at \emph{all}?

“No,” he said. “No.”

“I’m not asking you, Death. I can get a court order if I have to.
This is serious\dash{}it’s not some funny little game. There are billions
on the line here. One wrong word, one wrong post and \emph{pfft},
it’s all over. And nothing in email, either\dash{}it’s likely everything
you write is going to go through discovery. Don’t write anything
personal in any of your mail\dash{}nothing you wouldn’t want in a court
record.”

“I can’t do that,” Death said. He sounded like a fucking retard,
between talking through his mashed mouth and talking through the
tears. “I can’t. I live in email.”

“Well, now you’ll have a reason to go outside. This isn’t up for
negotiation. When I was here last, I thought I made the seriousness
of this case clear to you. I’m frankly amazed that you were
immature and irresponsible enough to write what I’ve read.”

“I can’t\dash{}” Death said.

The lawyer purpled. He didn’t look like a happy-go-lucky tanned
preppie anymore. He looked Dad-scary, like one of those fathers in
Disney who was about to seriously lose his shit and haul off and
smack a whiny kid. Death’s own Pawpaw, who’d stood in for his
father, had gone red like that whenever he “mouthed off,” a sin
that could be committed even without opening his mouth. He had an
instinctive curl-up-and-hide reaction to it, and the lawyer seemed
to sense this, looming over him. He felt like he was about to be
eaten.

“You listen to me, \emph{Darren}\dash{}this is not the kind of thing you
fuck up. This isn’t something \emph{I’m} going to fuck up. I win my
cases and you’re not going to change that. There’s too much at
stake here for you to blow it all with your childish, selfish\dash{}”

He seemed to catch himself then, and he snorted a hot breath
through his nose that blew over Death’s face. “Listen, there’s a
lot on the line here. More money than you or I are worth. I’m
trying to help you out here. Whatever you write, whatever you say,
it’s going to be very closely scrutinized. From now on, you should
treat every piece of information that emanates from your fingertips
as likely to be covered on the evening news and repeated to
everyone you’ve ever met. No matter how private you think you’re
being, it’ll come out. It’s not pretty, and I know you didn’t ask
for it, but you’re here, and there’s nothing you can do to change
that.

He left then, embarrassed at losing his temper, embarrassed at
Death’s meek silence. Death poked at his laptop some. He thought
about writing down more notes, but that was probably in the same
category.

He closed his eyes and now, \emph{now} he felt the extent of his
injuries, felt them truly for the first time since he’d woken up in
this hospital. There were deep, grinding pains in his legs\dash{}both
knees broken, fracture in the left thigh. His ribs hurt every time
he breathed. His face was a ruin, his mouth felt like he had
twisted lumps of hamburger glued to his torn lips. His dick\dash{}well,
they’d catheterized him, but that didn’t account for the feelings
down there. He’d been kicked repeatedly and viciously, and they
told him that the reconstructive surgeries\dash{}surgeries, plural\dash{}would
take some time, and nothing was certain until they were done.

He’d managed to pretend that his body wasn’t there for so long as
he was able to poke at the computer. Now it came back to him. He
had the painkiller rocker-switch and the pain wasn’t any worse than
what passed for normal, but he had an idea that if he hit it enough
times, he’d be able to get away from his body for a while again.

He tried it.

\begin{center}\rule{3in}{0.4pt}\end{center}

Hilda and Lester sat uncomfortably on the sofa next to each other.
Perry had hoped they’d hit it off, but it was clear after Lester
tried his Yoko joke again that the chemistry wasn’t there. Now they
were having a rare moment of all-look-same-screen, the TV switched
on like in an old comedy, no one looking at their own laptop.

The tension was thick, and Perry was sick of it.

He reached for his computer and asked it to find him the baseball
gloves. Two of the drawers on the living-room walls glowed pink. He
fetched the gloves down, tossed one to Lester, and picked up his
ball.

“Come on,” he said. “TV is historically accurate, but it’s not very
social.”

Lester got up from the sofa, a slow smile spreading on his face,
and Hilda followed a minute later. Outside, by the cracked pool, it
was coming on slow twilight and that magic, tropical blood-orange
sky like a swirl of sorbet.

Lester and Perry each put on their gloves. Perry’d worn his now and
again, but had never had a real game of catch with it. Lester
lobbed an easy toss to him and when it smacked his glove, it felt
so \emph{right}, the sound and the vibration and the fine cloud of
dust that rose up from the mitt’s pocket, Christ, it was like a
sacrament.

He couldn’t lob the ball back, because of his busted wing, so he
handed the ball to Hilda. “You’re my designated right arm,” he
said. She smiled and chucked the ball back to Lester.

They played until the twilight deepened to velvety warm dark and
humming bugs and starlight. Each time he caught a ball, something
left Perry, some pain long held in his chest, evanesced into the
night air. His catching arm, stiff from being twisted by the weight
of the cast on his other hand, unlimbered and became fluid. His
mind was becalmed.

None of them talked, though they sometimes laughed when a ball went
wild, and both Perry and Lester went “ooh,” when Lester made a
jump-catch that nearly tumbled him into the dry pool.

Perry hadn’t played a game of catch since he was a kid. Catch
wasn’t his dad’s strong suit, and he and his friends had liked
video-games better than tossing a ball, which was pretty dull by
comparison.

But that night it was magic, and when it got to full dark and they
could barely see the ball except as a second moon hurtling white
through the air, they kept tossing it a few more times before Perry
dropped it into the pocket of his baggy shorts. “Let’s get a
drink,” he said.

Lester came over and gave him a big, bearish hug. Then Hilda joined
them. “You stink,” Lester said, “Seriously, dude. Like the ass of a
dead bear.”

That broke them up and set them to laughing together, a giggling
fit that left them gasping, Lester on all fours. Perry’s arm forgot
to hurt and he moved to kiss Hilda on the cheek and instead she
turned her head to kiss him full on the lips, a real juicy, steamy
one that made his ear-wax melt.

“Drinks,” Hilda said, breaking the kiss.

They went upstairs, holding the mitts, and had a beer together on
the patio, talking softly about nothing in particular, and then
Lester hugged them good night and then they all went to bed, and
Perry put his face into the hair at the back of Hilda’s neck and
told her he loved her, and Hilda snuggled up to him and they fell
asleep.

\begin{center}\rule{3in}{0.4pt}\end{center}

\headline{A GAME OF CATCH}

Pop-quiz: Your empire is crumbling around your ears. Your
supporters are hospitalized by jackboot thugs for sticking up for
you.

The lawsuits are mounting and fly-by-night MBAs have determined to
use your non-profit, info-hippie ride project to get right by
embarking on 20 years of litigation.

What do you do?

Well, if you’re like Perry Gibbons, Lester Banks and Hilda
Hammersen, you go out into the backyard and throw a ball around for
a while, then you have a big cuddle and head inside.

The pictures shown here were captured by a neighbor of the cult
leaders last night, at their palatial condos in Hollywood,
Florida.

The three are ring-leaders of the loose-knit organization that
manages the “rides” that dot ten cities in America and are present
in fifty cities in Brazil. Their project came to national attention
when Disney brought suit against them, securing injunctions against
the rides that resulted in riots and bloodshed.

One supporter of the group, the outspoken “Death Waits,” a former
Disney employee, has been hospitalized for over a week following a
savage beating that he claims resulted from his Internet posting
about the unhealthy obsession Disney executive Samuel R.D. Page
(see previous coverage) bore for the ride.

Everyone needs to unwind now and then, but sources at the hospital
where Death Waits lies abed say that he has had no visits from the
cult leaders since he took his beating in their service.

No doubt these three have more important things to do\dash{}like play
catch.

\begin{center}\rule{3in}{0.4pt}\end{center}

Suzanne said, “Look, you can’t let crazy people set your agenda. If
you want to visit this Death kid, you should. If you don’t, you
shouldn’t. But don’t let Freddy psy-ops you into doing something
you don’t want to do. Maybe he does have a rat in your building.
Maybe he’s got a rat at the hospital. Maybe, though, he just scored
some stills off a flickr stream, maybe he’s watching new photos
with some face-recognition stuff.”

Perry looked up from his screen, still scowling. “People do that?”

“Sure\dash{}stalkerware! I use it myself, just to see what photos of me
are showing up online. I scour every photo-feed published for
anything that appears to be a photo of me. Most of it’s from
blogjects, CCTV cameras and crap like that. You should see what
it’s like on days I go to London\dash{}you can get photographed 800 times
a day there without trying. So yeah, if I was Freddy and I wanted
to screw with you, I’d be watching every image feed for your pic,
and mine, and Lester’s. We just need to assume that that’s going
on. But look at what he actually reported on: you went out and
played catch and then hugged after your game. It’s not like he
caught you cornholing gators while smoking spliffs rolled in
C-notes.”

“What does that guy have against us, anyway?”

Suzanne sighed. “Well, at first I think it was that \emph{I} liked
you, and that you were trying to do something consistent with what
he thought everyone should be doing. After all, if anyone were to
follow his exhortations, they’d have to be dumb enough to be taking
him seriously, and for that they deserve all possible
disapprobation.

“These days, though, he hates you for two reasons. The first is
that you failed, which means that you’ve got to have some kind of
moral deficiency. The second is that we keep pulling his pants down
in public, which makes him even angrier, since pulling down
people’s pants is \emph{his} job.

“I know it’s armchair psychology, but I think that Freddy just
doesn’t like himself very much. At the end of the day, people who
are secure and happy don’t act like this.”

Perry’s scowl deepened. “I’d like to kick him in the fucking
balls,” he said. “Why can’t he just let us be? We’ve got enough
frigging problems.”

“I just want to go and visit this kid,” Lester said, and they were
back where they started.

“But we know that this Freddy guy has an informant in the hospital,
he about says as much in this article. If we go there, he wins,”
Perry said.

Hilda and Lester just looked at him. Finally he smiled and
relented. “OK, Freddy isn’t going to run my life. If it’s the right
thing to visit this kid, it’s the right thing. Let’s do it.”

“We’ll go after the ride shuts tonight,” Lester said. “All of us.
I’ll buy him a fruit basket and bring him a mini.” The minis were
Lester’s latest mechanical computers, built inside of sardine cans,
made of miniaturized, printed, high-impact alloys. They could add
and subtract numbers up to ten, using a hand crank on the side,
registering their output on a binary display of little windows that
were covered and uncovered by tiny shutters. He’d built his first
the day before, using designs supplied by some of his people in
Brazil and tweaking them to his liking.

The day was as close to a normal day on the ride as Perry could
imagine. The crowd was heavy from the moment he opened, and he had
to go back into the depths and kick things back into shape a couple
times, and one of the chairs shut down, and two of the merchants
had a dispute that degenerated into a brawl. Just another day
running a roadside attraction in Florida.

Lester spelled him off for the end of the day, then they counted
the take and said good night to the merchants and all piled into
one of Lester’s cars and headed for the hospital.

“You liking Florida?” Lester called over the seat as they inched
forward in the commuter traffic on the way into Melbourne.

“It’s hot; I like that,” Hilda said.

“You didn’t mention the awesome aesthetics,” Lester said.

Suzanne rolled her eyes. “Ticky-tacky chic,” she said.

“I love it here,” Lester said. “That contrast between crass,
overdeveloped, cheap, nasty strip-malls and unspoiled tropical
beauty. It’s gorgeous \emph{and} it tickles my funny bone.”

Hilda squinted out the window as though she were trying to see what
Lester saw, like someone staring at a random-dot stereogram in a
mall-store, trying to make the three-d image pop out.

“If you say so,” she said. “I don’t find much attractive about
human settlement, though. If it needs to be there, it should just
be invisible as possible. We fundamentally live in ugly boxes, and
efforts to make them pretty never do anything for me except call
attention to how ugly they are. I kinda wish that everything was
built to disappear as much as possible so we could concentrate on
the loveliness of the world.”

“You get that in Madison?” Lester said.

“Nope,” she said. “I’ve never seen any place designed the way I’d
design one. Maybe I’ll do that someday.”

Perry loved her just then, for that. The casual “oh, yeah, the
world isn’t arranged to my satisfaction, maybe I’ll rearrange it
someday.”

The duty-nurse was a bored Eastern European who gave them a
half-hearted hard time about having too many people visit Death
Waits all at once, but who melted when Suzanne gave her a little
talk in Russian.

“What was that all about?” Perry whispered to her as they made
their way along the sour-smelling ward.

“Told her we would keep it down\dash{}and complimented her on her
manicure.”

Lester shook his head. “I haven’t been in a place like this in so
long. The fatkins places are nothing like it.”

Hilda snorted. “More upscale, I take it?” Lester and Hilda hadn’t
really talked about the fatkins thing, but Perry suddenly
remembered the vehemence with which Hilda had denounced the kids
who were talked into fatkins treatments in their teens and wondered
if she and Lester should be clearing the air.

“Not really\dash{}but more functional. More about, I don’t know, pursuing
your hobby. Less about showing up in an emergency.”

Hilda snorted again and they were at Death’s room. They walked past
his roommates, an old lady with her teeth out, sleeping with her
jaw sagging down, and a man in a body-cast hammering on a
video-game controller and staring fixedly at the screen at the foot
of his bed.

Then they came upon Death Waits. Perry had only seen him briefly,
and in bad shape even then, but now he was a wreck, something from
a horror movie or an atrocity photo. Perry swallowed hard as he
took in the boy’s wracked, skinny body, the casts, the sunken eyes,
the shaved head, the caved-in face and torn ears.

He was fixedly watching TV, which seemed to be showing a golf show.
His thumb was poised over a rocker-switch connected to the IV in
his arm.

Death looked at them with dull eyes at first, not recognizing them
for a moment. Then he did, and his eyes welled up with tears. They
streamed down his face and his chin and lip quivered, and then he
opened his mouth and started to bawl like a baby.

Perry was paralyzed\dash{}transfixed by this crying wreck. Lester, too,
and Suzanne. They all took a minute step backward, but Hilda pushed
past them and took his hand and stroked his hair and went
\emph{shhh}, \emph{shhh}. His bawling become more uncontrolled,
louder, and his two roommates complained, calling to him to shut
up, and Suzanne moved back and drew the curtains around each of
their beds. Strangely, this silenced them.

Gradually, Death’s cries became softer, and then he snuffled and
snorted and Hilda gave him a kleenex from her purse. He wiped his
face and blew his nose and squeezed the kleenex tight in his hand.
He opened his mouth, shut it, opened and shut it.

Then, in a whisper, he told them his story. The man in the
parking-lot and his erection. The hospital. Posting on the message
boards.

The lawyer.

“\emph{What}?” Perry said, loud enough that they all jumped and
Death Waits flinched pathetically in his hospital bed. Hilda
squeezed his arm hard. “Sorry, sorry,” Perry muttered. “But this
lawyer, what did he say to you?”

Perry listened for a time. Death Waits spoke in a low monotone,
pausing frequently to draw in shuddering breaths that were almost
sobs.

“Fucking \emph{bastards},” Perry said. “Evil, corporate, immoral,
sleazy\dash{}”

Hilda squeezed his arm again. “Shh,” she said. “Take it easy.
You’re upsetting him.”

Perry was so angry he could barely see, barely think. He was
trembling, and they were all staring at him, but he couldn’t stop.
Death had shrunk back into himself, squeezed his eyes shut.

“I’ll be back in a minute,” Perry said. He felt like he was
suffocating. He walked out of the room so fast it was practically a
jog, then pounded on the elevator buttons, waited ten seconds and
gave up and ran down ten flights of stairs. He got outside into the
coolness of the hazy night and sucked in huge lungsful of wet air,
his heart hammering in his chest.

He had his phone in his hand and he had scrolled to Kettlewell’s
number, but he kept himself from dialing it. He was in no shape to
discuss this with Kettlewell. He wanted witnesses there when he did
it, to keep him from doing something stupid.

He went back inside. The security guards watched him closely, but
he forced himself to smile and act calm and they didn’t stop him
from boarding the elevator.

“I’m sorry,” he said to all of them. “I’m sorry,” he said to Death
Waits. “Let me make something very, very clear: you are free to use
the Internet as much as you want. You are free to tell your story
to anyone you want to tell it to. Even if it screws up my case,
you’re free to do that. You’ve given up enough for me already.”

Death looked at him with watery eyes. “Really?” he said. It came
out in a hoarse whisper.

Perry moved the breakfast tray that covered Death’s laptop, then
opened the laptop and positioned it where Death could reach it.
“It’s all yours, buddy. Whatever you want to say, say it. Let your
freak flag fly.”

Death cried again then, silent tears slipping down his hollow
cheeks. Perry got him some kleenex from the bathroom and he blew
his nose and wiped his face and grinned at them all, a toothless,
wet, ruined smile that made Perry’s heart lurch. Jesus, Jesus,
Jesus. What the hell was he doing? This kid\dash{}he would never get the
life he’d had back.

“Thank you, thank you, thank you,” Death said.

“Please don’t be grateful to me,” Perry said. “We owe you the
thanks around here. Remember that. We haven’t done you any favors.
All the favors around here have come from you.

“Any lawyer shows up here again representing me, I want you to
email me.”

In the car back, no one said anything until they were within sight
of the shantytown. “Kettlewell isn’t going to like this,” Suzanne
said.

“Yeah, I expect not,” Perry said. “He can go fuck himself.”

\begin{center}\rule{3in}{0.4pt}\end{center}

Imagineering sent the prototype up to Sammy as soon as it was
ready, the actual engineers who’d been working on it shlepping it
into his office.

He’d been careful to cultivate their friendship through the weeks
of production, taking them out for beers and delicately letting
them know that they were just the sort of people who really
understood what Disney Parks was about, not like those philistines
who comprised the rest of the management layer at Disney. He
learned their kids’ names and forwarded jokes to them by email. He
dropped by their break-room and let them beat him at pinball on
their gigantic, bizarre, multi-board homebrew machine, letting them
know just how cool said machine was.

Now it was paying off. Judging from the device he was looking at, a
breadbox-sized, go-away-green round-shouldered smooth box that it
took two of them to carry in.

“Watch this,” one of them said. He knocked a complicated pattern on
the box’s top and a hidden hatch opened out of the side, yawning
out and forming a miniature staircase from halfway down the box’s
surface to the ground. There was soft music playing inside the box:
a jazzy, uptempo futuristic version of
\emph{When You Wish upon a Star}.

A little man appeared in the doorway. He looked like he was made of
pipe-cleaners and he took the stairs in three wobbling strides. He
ignored them as he lurched around the box’s perimeter until he came
to a far corner, then another hatch slid away and the little man
reached inside and tugged out the plug and the end of the
power-cord. He hugged the plug to his chest and began to wander
around Sammy’s desk, clearly looking for an electrical outlet.

“It’s a random-walk search algorithm,” one of the Imagineers said.
“Watch this.” After a couple of circuits of Sammy’s desk the little
robot went to the edge and jumped, hanging on to the power-cable,
which unspooled slowly from the box like a belay-line, gently
lowering the man to the ground. A few minutes later, he had found
the electrical outlet and plugged in the box.

The music inside stilled and a fanfare began. The trumpeting
reached a joyous peak\dash{}“It’s found a network connection”\dash{}and then
subsided into marching-band music. There was a smell like
Saran-Wrap in the microwave. A moment later, another pipe-cleaner
man emerged from the box, lugging a chunk of plastic that looked
like the base of a rocket in an old-timey science fiction movie.

The first pipe-cleaner man was shinnying up the power cable. He
crested the desktop and joined his brother in ferrying out more
parts. Each one snapped into the previous one with a Lego-like
\emph{click}. Taking shape on the desktop in slow stages, the
original, 1955 Tomorrowland, complete with the rocket to the moon,
the Clock of the World and\dash{}

“Dairy Farmers of America Present the Cow of Tomorrow?” Sammy said,
peering at the little brass plaque on the matchbox-sized diorama,
which showed a cow with an IV in her hock, watching a video of a
pasture. “You’re kidding me.”

“No!” one Imagineer said. “It’s all for real\dash{}the archives have all
these tight, high-rez three-d models of all the rides the Park’s
ever seen. This is totally historically accurate.”

The Kaiser Aluminum Hall of Fame. The Monsanto Hall of Chemistry.
Thimble Drome Flight Circle, with tiny flying miniature airplanes.

“Holy crap,” Sammy said. “People \emph{paid} to see these things?”

“Go on,” the other Imagineer said. “Take the roof off the Hall of
Chemistry.”

Sammy did, and was treated to a tiny, incredibly detailed three-d
model of the Hall’s interior exhibits, complete with tiny people in
1950s garb marveling at the truly crappy exhibits.

“We print to 1200 dpi with these. We can put pupils on the eyeballs
at that rez.”

The pieces were still trundling out. Sammy picked up the Monsanto
Hall of Chemistry and turned it over and over in his hands, looking
at the minute detail, admiring the way all the pieces snapped
together.

“It’s kind of brittle,” the first Imagineer said. He took it from
Sammy and gave it a squeeze and it cracked with a noise like an
office chair rolling over a sheet of bubble-wrap. The pieces fell
to the desk.

A pipe-cleaner man happened upon a shard after a moment and hugged
it to his chest, then toddled back into the box with it.

“There’s a little optical scanner in there\dash{}it’ll figure out which
bit this piece came from and print another one. Total construction
of this model takes about two hours.”

“You built this entire thing from scratch in three weeks?”

The Imagineers laughed. “No, no\dash{}no way! No, almost all the code and
designs came off the net. Most of this stuff was developed by New
Work startups back in the day, or by those ride weirdos down in
Hollywood. We just shoved it all into this box and added the models
for some of our old rides from the archives. This was easy,
man\dash{}easy!”

Sammy’s head swam. Easy! This thing was undeniably super-cool. He
wanted one. Everyone was going to want one!

“You can print these as big as you want, too\dash{}if we gave it enough
time, space and feedstock, it’d run these buildings at full size.”

The miniature Tomorrowland was nearly done. It was all brave, sad
white curves, like the set of a remake of Rollerball, and featured
tiny people in 1950s clothes, sun-dresses and salaryman hats,
black-rimmed glasses and scout uniforms for the boys.

Sammy goggled at it. He moved the little people around, lifted off
the lids.

“Man, I’d seen the three-d models and flythroughs, but they’re
nothing compared to actually seeing it, owning it. People will want
libraries of these things. Whole rooms devoted to them.”

“Umm,” one of the Imagineers said. Sammy knew his name, but he’d
forgotten it. He had a whole complicated scheme for remembering
people’s names by making up stories about them, but it was a lot of
work. “Well, about that. This feedstock is very fast-setting, but
it doesn’t really weather well. Even if you stored it in a dark,
humidity-controlled room, it’d start to delaminate and fall to
pieces within a month or two. Leave it in the living room in direct
sunlight and it’ll crumble within a couple days.”

Sammy pursed his lips and thought for a while. “Please, please tell
me that there’s something proprietary we can require in the
feedstock that can make us into the sole supplier of consumables
for this thing.”

“Maybe? We could certainly tag the goop with something proprietary
and hunt for it when we do the build, refuse to run on anyone
else’s goop. Of course, that won’t be hard to defeat\dash{}”

“We’ll sue anyone who tries it,” Sammy said. “Oh, boys, you’ve
outdone yourselves. Seriously. If I could give you a raise, I
would. As it is, take something home from the architectural salvage
lot and sell it on eBay. It’s as close to a bonus as this fucking
company’s going to pay any of us.”

They looked at him quizzically, with some alarm and he smiled and
spread his hands. “Ha ha, only serious boys. Really\dash{}take some stuff
home. You’ve earned it. Try and grab something from the ride-system
itself, that’s got the highest book-value.”

They left behind a slim folder with production notes and estimates,
suppliers who would be likely to bid on a job like this. He’d need
a marketing plan, too\dash{}but this was farther than he ever thought
he’d get. He could show this to legal and to the board, and yes, to
Wiener and the rest of the useless committee. He could get everyone
lined up behind this and working on it. Hell, if he spun it right
they’d all be fighting to have their pet projects instantiated with
it.

He fiddled with a couple of overnight shippers’ sites for a while,
trying to figure out what it would cost to sell these in the Park
and have them waiting on the marks’ doorsteps when they got back
home. There were lots of little details like that, but ultimately,
this was good and clean\dash{}it would extend the Parks’ reach right into
the living rooms of their customers, giving them a new reason to
think of the Park every day.

\begin{center}\rule{3in}{0.4pt}\end{center}

Kettlewell and Tjan looked up when Perry banged through the door of
the tea-house they’d turned into their de facto headquarters.

Perry had gone through mad and back to calm on the ride home, but
as he drew closer to the tea-house, passing the people in the
streets, the people living their lives without lawyers or bullshit,
his anger came back. He’d even stopped outside the tea-house and
breathed deeply, but his heart was pounding and his hands kept
balling into fists and sometimes, man, sometimes you’ve just got to
go for it.

He got to the table and grabbed the papers there and tossed them
over his shoulder.

“You’re fired,” he said. “Pack up and go, I want you out by
morning. You’re done here. You don’t represent the ride and you
never will. Get lost.” He didn’t know he was going to say it until
he said it, but it felt right. This was what he was
feeling\dash{}\emph{his} project had been stolen and bad things were
being done in \emph{his} name and it was going to stop, right now.

Tjan and Kettlewell got to their feet and looked at him, faces
blank with shock. Kettlewell recovered first. “Perry, let’s sit
down and do an exit interview, all right? That’s traditional.”

Perry was shaking with anger now. These two friends of his, they’d
fucking screwed him\dash{}committed their dirty work in his name. But
Kettlewell was holding a chair out to him and the others in the
tea-house were staring and he thought about Eva and the kids and
the baseball gloves, and he sat down.

He squeezed his thighs hard with his clenching hands, drew in a
deep breath, and recited what Death Waits had told him in an even,
wooden voice.

“So that’s it. I don’t know if you instructed the lawyers to do
this or only just distanced yourself enough from them to let them
do this on their own. The point is that the way you’re running this
campaign is victimizing people who believe in us, making life worse
for people who already got a shitty, shitty deal on our account. I
won’t have it.”

Kettlewell and Tjan looked at each other. They’d both stayed
poker-faced through Perry’s accusation, and now Kettlewell made a
little go-ahead gesture at Tjan.

“There’s no excuse for what that lawyer did. We didn’t authorize
it, we didn’t know it had happened, and we wouldn’t have permitted
it if we had. In a suit like this, there are a lot of moving parts
and there’s no way to keep track of all of them all of the time.
You don’t know what every ride operator in the world is up to, you
don’t even know where all the rides in the world \emph{are}. That’s
in the nature of a decentralized business.

“But here’s the thing: the lawyer was at least partly right.
Everything that kid blogs, emails, and says will potentially end up
in the public record. Like it or not, that kid can no longer
consider himself to have a private life, not until the court case
is up. Neither can you or I, for that matter. That’s in the nature
of a lawsuit\dash{}and it’s not something any of us can change at this
point.”

Perry heard him as from a great distance, through the whooshing of
the blood in his ears. He couldn’t think of anything to say to
that.

Tjan and Kettlewell looked at each other.

“So even if we’re ’fired’\dash{}” Tjan said at last, making sarcastic
finger-quotes, “this problem won’t go away. We’ve floated the
syndicate and given control of the legal case to them. If you try
to ditch it, you’re going to have to contend with \emph{their}
lawsuits, too.”

“I didn’t\dash{}” Perry started. But he had, he’d signed all kinds of
papers: first, papers that incorporated the ride-runners’ co-op;
and, second, papers that gave legal representation over to the
syndicate.

“Perry, I’m the chairman of the Boston ride collective. I’m their
rep on the co-op’s board. You can’t fire me. You didn’t hire me.
They did. So stop breathing through your nose like a locomotive and
calm down. None of us wanted that lawyer to go after that kid.”

He knew they were making sense but he didn’t want to care. He’d
ended up in this place because these supposed pals of his had
screwed up.

He knew that he was going to end up making up with them, going to
end up getting deeper into this. He knew that this was how good
people did shitty things: one tiny rotten compromise at a time.
Well, he wasn’t going to go there.

“Tomorrow morning,” he said. “Gone. We can figure out by email how
to have a smooth transition, but no more of this. Not on my head.
Not on my account.”

He stalked away, which is what he should have done in the first
place. Fuck being reasonable. Reasonable sucked.

\begin{center}\rule{3in}{0.4pt}\end{center}

Death found out about the Disney-in-a-Box printers seconds after
they were announced. He’d been tuning his feed-watchers to give him
news about the Disney Parks for nearly a decade, and this little PR
item on the Disney Parks newswire rang all the cherries on his
filters, flagging the item red and rocketing it to the top of his
news playlist, making all the icons in the sides of his screen
bounce with delight.

The announcement made him want to throw up. They were totally
ripping off the rides, and he knew for a fact that most of the
three-d meshes of the old yesterland rides and even the
contemporary ones were fan-made, so those’d be ripped off, too.

And the worst part was, he could feel himself getting excited. This
was just the kind of thing that would have given him major fanboy
drool as recently as a month ago.

He just stared angrily at his screen. Being angry made the
painkillers wear off, so the madder he got the more he hurt. He
could nail the rocker-switch and dose himself with more of whatever
the painkiller plugged into his IV was today, but since Perry and
Lester and their girlfriends (had that other one been Suzanne
Church? It sure looked like her) had told him he could use his
laptop again, he’d stayed off the juice as much as possible. The
computer could make him forget he hurt.

He looked at the clock. It was 4AM. The blinds on the ward were
shut most of the time, and he kept to his own schedule, napping and
then surfing, then nodding off and then surfing some more. The
hospital staff just left his food on the table beside him if he was
asleep when it arrived, though they woke him for his sponge baths
and to stick fresh needles in his arms, which were filled with
bruisey collapsed veins.

There was no one he could tell about this. Sure, there were
chat-rooms with 24/7 chatter from Disney freaks, but he didn’t much
want to chat with them. Some of his friends would still be up and
tweaking, but Christ, who wanted to IM with a speed freak at four
in the morning? His typing was down to less than 30 wpm, and he
couldn’t keep it up for long. What he really wanted was to talk to
someone about this.

He really wanted to talk to Perry about this. He should send him an
email, but he had the inkling of an idea and he didn’t want to put
it in writing, because it was a deliciously naughty idea.

It was dumb to even think about phoning him, he barely knew him,
and no one liked to get calls at four am. Besides\dash{}he’d
checked\dash{}Perry’s number was unlisted.

\begin{email}
From: deathw@deathwait.er\\
To: pgibbons@hollywood.ride\\
Subject: What’s your phone number?

Perry, I know that it’s presumptuous, but I’d really like to talk
to you v2v about something important that I’d prefer not to put in
writing. I don’t have any right to impose on you, especially not
after you’ve already done me the kindness of coming to see me in
the hospital, but I hope you’ll send me your number anyway.
Alternatively, please call me on my enum\dash{}1800DEATHWAITS-GGFSAH.

Your admirer,

Death Waits
\end{email}

It was five minutes later when his laptop rang. It was unnaturally
loud on the ward, and he heard his roommates stir when the tone
played. He didn’t have a headset\dash{}Christ, he was an idiot. Wait,
there was one, dangling from the TV. No mic, but at least he could
pair it with his laptop for sound. He stabbed at the mute button
and reached for the headset and slipped it on. Then he held the
computer close to his face and whispered “Hello?” into its little
mic. His voice was a croak, his ruined mouth distorting the word.
Why had he decided to call this guy? He was such an idiot.

“This is Perry Gibbons. Is that Death Waits?”

“Yes, sorry, I don’t have a mic. Can you hear me OK?”

“If I turn the volume all the way up I can.”

There was an awkward silence. Death tried to think of how to
begin.

“What’s on your mind, Death?”

“I didn’t expect you to be awake at this hour.”

“I had a rough night,” Perry said. It occurred to Death that he was
talking to one of his heros, a man who had come to visit him in the
hospital that day. He grew even more tongue-tied.

“What happened?”

“Nothing important,” Perry said and swallowed, and Death suddenly
understood that Perry had had a rough night because of \emph{him},
because of what \emph{he’d} told Perry. It made him want to cry.

“I’m sorry,” Death said.

“What’s on your mind, Death?” Perry said again.

Death told him what he’d found, about the Disney printers. He read
Perry the URLs so he could look them up.

“OK, that’s interesting,” Perry said. Death could tell he didn’t
really think it was that interesting.

“I haven’t told you my idea yet.” He groped for the words. His
mouth had gone dry. “OK, so Disney’s going to ship these things to
tons of people’s houses, they’ll sell them cheap at the parks and
mail them as freebies to Magic Kingdom Club gold-card holders. So
in a week or two, there’s going to be just, you know, tons of these
across the country.”

“Right.”

“So here’s my idea: what if you could get them to build non-Disney
stuff? What if you could send them plans for stuff from the rides?
What if you could just download your friends’ designs? What if this
was opened wide.”

Perry chuckled on the other end of the line, then laughed,
full-throated and full of merriment. “I like the way you think,
kid,” he said, once he’d caught his breath.

And then this amazing thing happened. Perry Gibbons
\emph{brainstormed} with him about the kinds of designs they could
push out to these things. It was like some kind of awesome dream
come true. Perry was treating him like a peer, loving his ideas,
keying off of them.

Then a dismal thought struck him. “Wait though, wait. They’re using
their own goop for the printers. Every design we print makes them
richer.”

Perry laughed again, really merry. “Oh, that kind of thing never
works. They’ve been trying to tie feedstock to printers since the
inkjet days. We go through that like wet kleenex.”

“Isn’t that illegal?”

“Who the fuck knows? It shouldn’t be. I don’t care about illegal
anymore. Legal gets you lawyers. Come on, dude\dash{}what’s the point of
being all into some anti-authoritarian subculture if you spend all
your time sucking up to the authorities?”

Death laughed, which actually hurt quite a bit. It was the first
laugh he’d had since he’d ended up in the hospital, maybe the first
one since he’d been fired from Disney World, and as much as it
hurt, it felt good, too, like a band being loosened from around his
broken ribs.

His roommates stirred and one of them must have pushed the nurse
call button, because shortly thereafter, the formidable Ukrainian
nurse came in and savagely told him off for disturbing the ward at
five in the morning. Perry heard and said his goodbyes, like they
were old pals who’d chatted too long, and Death Waits rang off and
fell into a light doze, grinning like a maniac.

\begin{center}\rule{3in}{0.4pt}\end{center}

Hilda eyed Perry curiously. “That sounded like an interesting
conversation,” she said. She was wearing a long t-shirt of his that
didn’t really cover much, and she looked delicious in it. It was
all he could do to keep from grabbing her and tossing her on the
bed\dash{}of course, the cast meant that he couldn’t really do that. And
Hilda wasn’t exactly smiling, either.

“Sorry, I didn’t mean to wake you up,” he said.

“It wasn’t the talking that did it, it was you not being there in
the first place. Gave me the toss-and-turns.”

She came over to him then, the lean muscles in her legs flexing as
she crossed the living room. She took his laptop away and set it
down on the coffee-table, then took off his headset. He was wearing
nothing but boxers, and she reached down and gave his dick a
companionable honk before sitting down next to him and giving him a
kiss on the cheek, the throat and the lips.

“So, Perry,” she said, looking into his eyes. “What the fuck are
you doing sitting in the living room at 5 am talking to your
computer? And why didn’t you come to bed last night? I’m not going
to be hanging out in Florida for the rest of my life. I woulda
thought you’d want to maximize your Hilda-time while you’ve got the
chance.”

She smiled to let him know she was kidding around, but she was
right, of course.

“I’m an idiot, Hilda. I fired Tjan and Kettlewell, told them to get
lost.”

“I don’t know why you think that’s such a bad idea. You need
business-people, probably, but it doesn’t need to be those guys.
Sometimes you can have too much history with someone to work with
him. Besides, anything can be un-said. You can change your mind in
a week or a month. Those guys aren’t doing anything special. They’d
come back to you if you asked ’em. You’re Perry motherfuckin’
Gibbons. You rule, dude.”

“You’re a very nice person, Hilda Hammersen. But those guys are
running our legal defense, which we’re going to need, because I’m
about to do something semi-illegal that’s bound to get us sued
again by the same pack of assholes as last time.”

“Disney?” She snorted. “Have you ever read up on the history of the
Disney Company? The old one, the one Walt founded? Walt Disney
wasn’t just a racist creep, he was also a mad inventor. He kept
coming up with these cool high-tech ways of making
cartoons\dash{}sticking real people in them, putting them in color,
adding sync-sound. People loved it all, but it drove him out of
business. It was all too expensive.

“So he recruited his brother, Roy Disney, who was just a banker, to
run the business. Roy turned the business around, watching the
income and the outgo. But all this came at a price: Roy wanted to
tell Walt how to run the business. More to the point, he wanted to
tell Walt that he couldn’t just spend millions from the company
coffers on weird-ass R\&D projects, especially not when the company
was still figuring out how to exploit the \emph{last} R\&D project
Walt had chased. But it was Walt’s company, and he’d overrule Roy,
and Roy would promise that it was going to put them in the
poorhouse and then he’d figure out how to make another million off
of Walt’s vision, because that’s what the money guy is supposed to
do.

“Then after the war, Walt went to Roy and said, ’Give me \$17
million, I’m going to build a theme-park. And Roy said, ’You can’t
have it and what’s a theme-park?’ Walt threatened to fire Roy, the
way he always had, and Roy pointed out that Disney was now a
\emph{public} company with shareholders who weren’t going to let
Walt cowboy around and piss away their money on his toys.”

“So how’d he get Disneyland built?”

“He quit. He started his own company, WED, for Walter Elias Disney.
He poached all the geniuses away from the studios and turned them
into his ’Imagineers’ and cashed in his life-insurance policy and
raised his own dough and built the park, and then made Roy buy the
company back from him. I’m guessing that that felt pretty good.”

“It sounds like it must’ve,” Perry said. He was feeling thoughtful,
and buzzed from the sleepless night, and jazzed from his
conversation with Death Waits. He had an idea that they could push
designs out to the printers that were like the Disney designs, but
weird and kinky and subversive and a little disturbing.

“I can understand why you’d be nervous about ditching your suits,
but they’re just that, suits. At some level, they’re all
interchangeable, mercenary parts. You want someone to watch the
bottom line, but not someone who’ll run the show. If that’s not
these guys, hey, that’s cool. Find a couple more suits and run
them.”

“Jesus, you really \emph{are} Yoko, aren’t you?” Lester was wearing
his boxers and a bleary grin, standing in the living room’s doorway
where Hilda had stood a minute before. It was past 6AM now, and
there were waking up sounds through the whole condo, toilets
flushing, a car starting down in the parking lot.

“Good morning, Lester,” Hilda said. She smiled when she said it, no
offense taken, all good, all good.

“You fired who now, Perry?” Lester dug a pint of chocolate
ice-cream out of the freezer and attacked it with a self-heating
ceramic spoon that he’d designed specifically for this purpose.

“I got rid of Kettlewell and Tjan,” Perry said. He was blushing. “I
would have talked to you about it, but you were with Suzanne. I had
to do it, though. I had to.”

“I hate what happened to Death Waits. I hate that we’ve got some of
the blame for it. But, Perry, Tjan and Kettlewell are part of our
outfit. It’s their show, too. You can’t just go shit-canning them.
Not just morally, either. Legally. Those guys own a piece of this
thing and they’re keeping the lawyers at bay too. They’re managing
all the evil shit so we don’t have to. I don’t want to be in charge
of the evil, and neither do you, and hiring a new suit isn’t going
to be easy. They’re all predatory, they all have delusions of
grandeur.”

“You two have the acumen to hire better representation than those
two,” Hilda said. “You’re experienced now, and you’ve founded a
movement that plenty of people would kill to be a part of. You just
need better management structure: an executive you can overrule
whenever you need to. A lackey, not a boss.”

Lester acted as though he hadn’t heard her. “I’m being pretty
mellow about this, buddy. I’m not making a big deal out of the fact
that you did this without consulting me, because I know how rough
it must have been to discover that this wickedness had gone down in
our name, and I might have done the same. But it’s the cold light
of day now and it’s time to go over there together and have a chat
with Tjan and Kettlewell and talk this over and sort it out. We
can’t afford to burn all this to the ground and start over now.”

Perry knew it was reasonable, but screw reasonable. Reasonable was
how good people ended up doing wrong. Sometimes you had to be
unreasonable.

“Lester, they violated our trust. It was their responsibility to do
this thing and do it right. They didn’t do that. They didn’t look
closely at this thing so that they wouldn’t have to put the brakes
on if it turned out to be dirty. Which do you think those two would
rather have happen: we run a cool project that everyone loves, or
we run a lawsuit that makes ten billion dollars for their
investors? They’re playing a different game from us and their
victory condition isn’t ours. I don’t want to be reasonable. I want
to do the right thing. You and me could have sold out a thousand
times over the years and made money instead of doing good, but we
didn’t. We didn’t because it’s better to be right than to be
reasonable and rich. You say we can’t afford to get rid of those
two. I say we can’t afford not to.”

“You need to get a good night’s sleep, buddy,” Lester said. He was
blowing through his nose, a sure sign that he was angry. It made
Perry’s hackles go up\dash{}he and Lester didn’t fight much but when they
did, hoo-boy. “You need to mellow out and see that what you’re
talking about is abandoning our friends, Kettlewell and Tjan, to
make our own egos feel a little better. You need to see that we’re
risking everything, risking spending our lives in court and losing
everything we’ve ever built.”

A Zen-like calm descended on Perry. Hilda was right. Suits were
everywhere, and you could choose your own. You didn’t need to let
the Roy Disneys of the world call the shots.

“I’m sorry you feel that way, Lester. I hear everything you’re
saying, but you know what, it’s going to be my way. I understand
that what I want to do is risky, but there’s no way I can go on
doing what I’m doing and letting things get worse and worse. Making
a little compromise here and there is how you end up selling out
everything that’s important. We’re going to find other
business-managers and we’re going to work with them to make a
smooth transition. Maybe we’ll all come out of this friends later
on. They want to do something different from what I want to do is
all.”

This wasn’t calming Lester down at all. “Perry, this isn’t your
project to do what you want with. This belongs to a lot of us. I
did most of the work in there.”

“You did, buddy. I get that. If you want to stick with them, that’s
how it’ll go. No hard feelings. I’ll go off and do my own thing,
run my own ride. People who want to connect to my network, no
sweat, they can do it. That’s cool. We’ll still be friends. You can
work with Kettlewell and Tjan.” Perry could hardly believe these
words were coming out of his mouth. They’d been buddies forever,
inseparable.

Hilda took his hand silently.

Lester looked at him with increasing incredulity. “You don’t mean
that.”

“Lester, if we split, it would break my heart. There wouldn’t be a
day that went by from now to the end of time that I didn’t regret
it. But if we keep going down this path, it’s going to cost me my
soul. I’d rather be broke than evil.” Oh, it felt so \emph{good} to
be saying this. To finally affirm through deed and word that he was
a good person who would put ethics before greed, before comfort
even.

Lester looked at Hilda for a moment. “Hilda, this is probably
something that Perry and I should talk about alone, if you don’t
mind.”

“\emph{I} mind, Lester. There’s nothing you can’t say in front of
her.”

Lester apparently had nothing to say to that, and the silence made
Perry uncomfortable. Lester had tears in his eyes, and that hit
Perry in the chest like a spear. His friend didn’t cry often.

He crossed the room and hugged Lester. Lester was wooden and
unyielding.

“Please, Lester. Please. I hate to make you choose, but you have to
choose. We’re on the same side. We’ve always been on the same side.
Neither of us are the kind of people who send lawyers after kids in
hospital. Never. I want to make it good again. We can have the kind
of gig where we do the right thing and the cool thing. Come on,
Lester. Please.”

He let go of Lester. Lester turned on his heel and walked back into
his bedroom. Perry knew that that meant he’d won. He smiled at
Hilda and hugged her. She was a lot more fun to hug than Lester.

\begin{center}\rule{3in}{0.4pt}\end{center}

Sammy was at his desk looking over the production prototype for the
Disney-in-a-Box (R) units that Imagineering had dropped off that
morning when his phone rang. Not his desk phone\dash{}his cellular phone,
with the call-return number blocked.

“Hello?” he said. Not many people had this number\dash{}he didn’t like
getting interrupted by the phone. People who needed to talk to him
could talk to his secretary first.

“Hi, Sammy. Have I caught you at a bad time?” He could hear the
sneer in the voice and then he could see the face that went with
the sneer: Freddy. Shit. He’d given the reporter his number back
when they were arranging their disastrous face-to-face.

“It’s not a good time, Freddy,” he said. “If you call my
secretary\dash{}”

“I just need a moment of your time, sir. For a quote. For a story
about the ride response to your printers\dash{}your Disney-in-a-Box
Circle-R, Tee-Em, Circle-C.”

Sammy felt his guts tense up. Of course those ride assholes would
have known about the printers. That’s what press-releases were for.
Somewhere on their message-boards he was sure that there was some
discussion of them. He hadn’t had time to look for it, though, and
he didn’t want to use the Disney Parks competitive intel people on
this stuff, because after the Death Waits debacle (debacle on
debacle, ack, he could be such a fuck-up) he didn’t want to have
any train of intel-gathering on the group pointing back to him.

“I’m not familiar with any response,” Sammy said. “I’m afraid I
can’t comment\dash{}”

“Oh, it’ll only take a moment to explain it,” Freddy said and then
launched into a high-speed explanation before Sammy could object.
They were delivering their own three-d models for the printers, and
had even gotten hold of one of the test units Disney had passed out
last week. They claimed to have reverse-engineered the goop that it
ran on, so that anyone’s goop could print to it.

“So, what I’m looking for is a quote from Disney on this. Do you
condone this? Did you anticipate it? What if someone prints an
AK-47 with it?”

“No one’s going to print a working AK-47 with this,” Sammy said.
“It’s too brittle. AK-47 manufacturing is already sadly in great
profusion across our inner cities, anyway. As to the rest of it\dash{}”
He closed his eyes and took a couple of deep breaths. “As to the
rest of it, that would be something you’d have to speak to one of
my legal colleagues about. Would you like me to put you through to
them?”

Freddy laughed. “Oh come on, Sammy. A little something on
background, no attribution? You going to sue them? Have them beaten
up?”

Sammy felt his face go white. “I’m sure I don’t know what you’re
talking about\dash{}”

“Word has it that the Death Waits kid came up with this. He used to
be your protege, no? And I hear that Kettlewell and Tjan have been
kicked out of the organization\dash{}no one around to call the lawyers
out on their behalf. Seems like a golden opportunity to strike.”

Sammy seethed. He’d been concentrating on making new stuff, great
stuff. Competitive stuff, to be sure, but in the end, the reason
for making the Disney-in-a-Box devices had been to make them, make
them as cool as he could imagine. To plus them and re-plus them, in
the old slang of Walt Disney, making the thing because the thing
could be made and the world would be a more fun place once it was.

Now here was this troll egging him on to go to war again with those
ride shit-heads, to spend his energies destroying instead of
creating. The worst part? It was all his fault. He’d brought his
own destruction: the reporter, Death Waits, even the lawsuit. All
the result of his bad planning and dumb decisions. God, he was a
total fuck-up.

Disney-in-a-Box sat on his desk, humming faintly\dash{}not humming like a
fridge hums, but actually humming in a baritone hum, humming a
medley of magic-users’ songs from Disney movies, like a living
thing. Every once in a while it would clear its throat and mutter
and even snore a little. There would be happy rustles and whispered
conversations from within the guts of the thing. It was plussed all
the way to hell and back. It had been easy, as more and more
Imagineers had come up with cool features to add to the firmware,
contributing them to the versioning system, and he’d been able to
choose from among them and pick the best of the lot, making a
device that rivaled Walt’s 1955 Disneyland itself for originality,
excitement, and cool.

“I’ll just say you declined to comment, then?”

Asshole.

“You write whatever you need to write, Freddy,” he said. A hatch
opened a tiny bit on the top of the cube and a pair of eyes peered
out, then it slammed shut and there was a round of convincing
giggles and scurrying from within the box. This could be huge, if
Sammy didn’t fuck it up by worrying too much about what someone
else was up to.

“Oh, and one other thing: it looks like the Death Waits kid is
going to be discharged from the hospital this week.”

\begin{center}\rule{3in}{0.4pt}\end{center}

He wasn’t ready to leave the hospital. For starters, he couldn’t
walk yet, and there were still times when he could barely remember
where he was, and there was the problem of the catheter. But the
insurance company and the hospital had concurred that he’d had all
the treatment he needed\dash{}even if his doctor hadn’t been able to look
him in the eye when this was explained\dash{}and it was time for him to
go home. Go away. Go anywhere.

He’d put it all in his LJ, the conversation as best as he could
remember it, the way it made him feel. The conversation he’d had
with Perry and the idea he’d had for pwning Disney-in-a-Box. He
didn’t even know if his apartment was still there\dash{}he hadn’t been
back in weeks and the rent was overdue.

And the comments came flooding in. First a couple dozen from his
friends, then hundreds, then thousands. Raging fights\dash{}some people
accused him of being a fakester sock-puppet aimed at gathering
sympathy or donations (!)\dash{}side-conversations, philosophical
arguments.

Buried in there, offers from real world and online friends to meet
him at the hospital, to get him home, to take care of him. It was
unbelievable. There was a small fortune\dash{}half-a-year’s wages at his
old job\dash{}waiting in his paypal, and if this was all to be believed,
there was a cadre of people waiting just outside that door to meet
him.

The nurse who came to get him looked rattled. “Your friends are
here,” she said in her Boris-and-Natasha accent, and gave him a
disapproving look as she disconnected his hoses and pipes so
swiftly he didn’t have time to register the pain he felt. She
pulled on a pair of Salvation Army underpants\dash{}the first pair he’d
worn in weeks\dash{}and a pair of new, dark blue-jeans and a Rotary
picnic t-shirt dated three years before. The shirt was a small and
it still hung from him like a tent.

“You will use canes?” she asked. He’d had some physiotherapy that
week and he could take one or two doddering steps on crutches, but
canes? No way.

“I can’t,” he said, picturing himself sprawled on the polished
concrete floor, with what was left of his face bashed in from the
fall.

“Wheelchair,” she said to someone in the hall, and an orderly came
in pushing a chair with a squeaky wheel\dash{}though the chair itself was
a pretty good one, at least as good as the ones they rented at
Disney, which were nearly indestructible. He let the nurse transfer
him to it with her strong hands in his armpits and under his knees.
A bag containing his laptop and a few cards and things that had
shown up at the hospital was dumped into his lap and he clutched it
to himself as he was wheeled to the end of the corridor and around
the corner, where the nurse’s station, the elevators, the common
area and his \emph{fans} were.

They weren’t just his pals, though there were a few of them there,
but also a big crowd of people he’d never met, didn’t recognize.
There were goths, skinny and pale and draped in black, but they
were outnumbered by the subculture civilians, normal-looking,
slightly hippieish, old and young. When he hove into sight, they
burst into a wild cheer. The orderly stopped pushing his chair and
the nurse rushed forward to shush them sternly, but it barely
dampened the calls. There were wolf whistles, cheers, calls,
disorganized chants, and then two very pretty girls\dash{}he hadn’t
thought about “pretty” anything in a long, long time\dash{}unfurled a
banner that said DEATH WAITS in glittery hand-drawn letters, with a
little skull dotting the I in WAITS.

The nurse read the banner and reached to tear it out of their
hands, but they folded it back. She came to him and hissed in his
ear, something about getting security to get rid of these people if
they were bothering him, and he realized that she thought DEATH
WAITS was a \emph{threat} and that made him laugh so hard he
choked, and she flounced off in a deeply Slavic huff.

And then he was among his welcoming party, and it \emph{was} a
party\dash{}there were cake and clove cigarettes in smoke-savers and cans
of licorice coffee, and everyone wanted to talk with him and take
their pictures with him, and the two pretty girls took turns making
up his face, highlighting his scars to make him fit for a Bela
Lugosi role. The were called Lacey and Tracey, and they were
sisters who went to the ride every day, they said breathlessly, and
they’d seen the story he’d described, seen it with their own eyes,
and it was something that was as personal as the twin language
they’d developed to communicate with one another when they were
little girls.

His old friends surrounded him: guys who marveled at his recovery,
girls who kissed his cheek and messed up Tracey and Lacey’s makeup.
Some of them had new tattoos to show him\dash{}one girl had gotten a
full-leg piece showing scenes from the ride, and she slyly pulled
her skirt all the way up, all the way up, to show him where it all
started.

Security showed up and threw them all out into the street, where
the heat was oppressive and wet, but the air was fresh and full of
smells that weren’t sickness or medicine, which made Death Waits
feel like he could get up and dance. Effervescent citrus and
biodiesel fumes, moist vegetation and the hum of lazy high noon
bugs.

“Now, it’s all arranged,” one of the straight-looking ones told
him. He’d figured out that these were the pure story people, who’d
read his descriptions and concluded that he’d seen something more
than anyone else. They all wanted a chance to talk to him, but
didn’t seem too put out that he was spending most of his time with
his old mates. “Don’t worry about a thing.” Car after car appeared,
taking away more of the party. “Here you go.”

Another car pulled up, an all-electric kneeling number with a huge
cargo space. They wheeled the chair right into it, and then two of
the story-hippies helped him transfer into the seat. “My mom was in
a wheelchair for ten years before she passed,” a hippie told him.
He was older and looked like an English teacher Death Waits had
quite liked in grade ten. He strapped Death Waits in like a pro and
off they went.

They were ten minutes into Melbourne traffic\dash{}Death marveling at
buildings, signs, people, in every color, without the oppressive
white-and-gore colors of everything in the hospital\dash{}when the
English teacher dude looked shyly at Death.

“You think it’s real\dash{}the Story, I mean\dash{}don’t you?”

Death thought about this for a second. He’d been very focused on
the Park-in-a-Box printers for the past week, which felt like an
eternity to him, but he remembered his obsession with the story
fondly. It required a kind of floaty non-concentration to really
see it, a meditative state he’d found easy to attain with all the
painkillers.

“It’s real,” he said.

The English teacher and two of his friends seemed to relax a
little. “We think so too.”

They pulled up to his condo\dash{}how’d they know where he lived?\dash{}and
parked right next to his car! He could see where the tow had kind
of fucked-up the rear bumper, but other than that, it was just as
he remembered it, and it looked like someone had given it a wash,
too. The English teacher put his car in park and came around to
open his door just as the rest of the welcoming party came out of
his building, pushing\dash{}

A stair-climbing wheelchair, the same kind that they used in the
ride. Death laughed aloud with delight when he saw it rolling
toward him, handling the curb easily, hardly a bump, and the two
pretty girls, Tracey and Lacey, transferred him into it, and both
contrived to brush their breasts and jasmine-scented hair across
his cheeks as they did so, and he felt the first stirrings in his
ruined groin that he’d felt since before his beating.

He laughed like a wild-man, and they all laughed with him and
someone put a clove cigarette between his lips and he drew on it,
coughed a little, and then had another drag before he rolled into
the elevator.

The girls put him to bed hours later. His apartment had been
spotless and he had every confidence that it would be spotless
again come night-time. The party had spent the rest of the day and
most of the night talking about the story that they’d seen in the
ride, where they’d seen it, what it meant. There was a lot of
debate about whether they had any business rating things now that
the story had shown itself to them. The story was the product of
unconscious effort, and it should be left to unconscious effort.

But the counter-argument was that they had a duty to garden the
story, or possibly to sharpen its telling, or to protect it from
people who couldn’t see it or wouldn’t see it.

At first Death didn’t know what to make of all this talk. At first
he found it funny and more than a little weird to be taking the
story this seriously. It was beautiful, but it was an accidental
beauty. The ride was the important thing, the story was its
effect.

But these people convinced him that they were right, that the story
\emph{had} to be important. After all, it had inspired all of them,
hadn’t it? The ride was just technology\dash{}the story was what the ride
was \emph{for}.

His head swam with it.

“We’ve got to protect it,” he said finally, after listening to the
argument, after eating the food with which they’d filled his
fridge, after talking intensely with Tracey (or possibly Lacey)
about their parents’ unthinking blandness, after letting the
English teacher guy (whose name was Jim) take him to the toilet,
after letting his old goth pals play some music some mutual friends
had just mixed.

“We’ve got to protect it and sharpen it. The story wants to get out
and there will be those who can’t see it.” He didn’t care that his
speech was mangled by his fucked-up face. He’d seen his face in the
mirror and Tracey and Lacey had done a nice job in making it up\dash{}he
looked like a latter-day Marilyn Manson, his twisted mouth a
ghoulish smear. The doctors had talked about giving him another
series of surgeries to fix his lip, a set of implanted dentures to
replace the missing teeth, had even mentioned that there were
specialist clinics where he could get a new set budded and grown
right out of his own gums. That had been back when the mysterious
forces of the lawsuit and the ride were paying his bills.

Now he contemplated his face in the mirror and told himself he’d
get used to this, he’d come to like it, it would be a trademark. It
would make him gothier than goth, for life, always an outsider,
always one of the weird ones, like the old-timers who’d come to
Disney with their teenaged, eye-rolling kids. Goths’ kids were
never goths, it seemed\dash{}more like bang-bangers or jocky-looking
peak-performance types, or hippies or gippies or dippies or tippies
or whatever. But their parents were still proudly flying their
freak-flags, weird to the grave.

“We’ll let everyone know about it,” he said, thinking not of
\emph{everyone} but of all the cool subculture kids he’d grown up
with and worshipped and been rejected by and dated and loved and
hated\dash{}“and we’ll make it part of \emph{everyone}’s story. We’ll
protect it, guys. Of course we’ll protect it.”

That settled the argument. Death hadn’t expected that. Since when
did he get the last word on any subject? Since now. They were
following his lead.

And then the girls put him to bed, shyly helping him undress, each
of them leaning over him to kiss him good night. Tracey’s kiss was
sisterly, on the cheek, her spicy perfume and her jet-black hair
caressing him. Lacey’s kiss was anything but sisterly. She mashed
her breasts to his chest and thrust her tongue into his mouth,
keeping her silver eyes open and staring deep into his, her fingers
working busily in his hair.

She broke the kiss off with a gasp and a giggle. She traced the
ruin of his mouth with a fingertip, breathing heavily, and let it
slide lower, down his chest. He found himself actually \emph{hard},
the first pleasurable sensation he’d had in his dick since that
fateful night. From the corridor came an impatient cough\dash{}Tracey,
waiting for Lacey to get going.

Lacey rolled her eyes and giggled again and then slid her hand the
rest of the way down, briefly holding his dick and then encircling
his balls with her fingers before kissing him again on the twist of
his lips and backing out of the room, whispering, “Sleep well, see
you in the morning.”

Death lay awake and staring at the ceiling for a long time after
they had gone. The English teacher dude had left him with a bedpan
for the night and many of them had promised to return in rotations
indefinitely during the days, helping him out with dressing and
shopping and getting him in and out of his marvelous chair.

He stared and stared at that ceiling, and then he reached for his
laptop, there beside the bed, the same place it had lived when he
was in the hospital. He fired it up and went straight to today’s
fly-throughs of the ride and ran through them from different
angles\dash{}facing backward and sideways, looking down and looking up,
noting all the elements that felt like \emph{story} and all the
ones that didn’t, wishing he had his plus-one/minus-one joystick
with him to carve out the story he was seeing.

\begin{center}\rule{3in}{0.4pt}\end{center}

Lester wouldn’t work the ride anymore, so Perry took it on his own.
Hilda was in town buying groceries\dash{}his chest-freezer of gourmet
surplus food had blown its compressor and the contents had spoiled
in a mess of venison and sour blueberry sauce and duck pancakes\dash{}and
he stood alone. Normally he loved this, being the carnival barker
at the middle of the three-ring circus of fans, tourists and
hawkers, but today his cast itched, he hadn’t slept enough, and
there were lawyers chasing him. Lots of lawyers.

A caravan of cars pulled into the lot like a Tim Burton version of
a funeral, a long train of funnycar hearses with jacked-up rear
wheels and leaning chimney-pots, gargoyles and black bunting with
super-bright black-light LEDs giving them a commercially eldritch
glow. Mixed in were some straight cars, and they came and came and
came, car on car. The hawkers got out more stuff, spread it out
further, and waited while the caravan maneuvered itself into
parking spots, spilling out into the street.

Riders got out of the cars, mostly super-skinny goths\dash{}a line of
special low-calorie vegan versions of Victorian organ-meat
delicacies had turned a mom-and-pop cafe in Portland, Oregon, into
a Fortune 500 company a few years before\dash{}in elaborate DIY
costumery. It shimmered darkly, petticoats and toppers, bodices and
big stompy boots and trousers cut off in ribbons at the knees.

The riders converged on one of the straight cars, a beige mini-van,
and crowded around it. A moment later, they were moving toward
Perry’s ticket-taking stand. The crowd parted as they approached
and in Perry saw whom they’d been clustered around. It was a skinny
goth kid in a wheelchair like the ones they kept in the ride\dash{}they’d
get that every now and again, a guest in his own chair, just
needing a little wireless +1/-1 box. His hair was shaggy and black
with green highlights, stuck out like an anime cosplayer’s. He was
white as Wonder Bread, with something funny about his mouth. His
legs were in casts that had been wrapped with black gauze, and a
pair of black pointy shoes had been slid over his toes, tipped with
elaborate silver curlicues.

The chair zipped forward and Perry recognized him in a flash: Death
Waits! He felt his mouth drop open and he shut it and came around
the stand.

“No way!” he said, and grabbed Death’s hand, encrusted in chunky
silver jewelry, a different stylized animal skull on each finger.
Death’s ruined mouth pulled up in a kind of smile.

“Nice to see you,” he said, limply squeezing Perry’s hand. “It was
very kind of you to visit me in the hospital.”

Perry thought of all the things that had happened since then and
wondered how much of it, if any, Death had a right to know about.
He leaned in close, conscious of all the observers. “I’m out of the
lawsuit. We are. Me and Lester. Fired those guys.” Behind his
reflective contacts, Death’s eyes widened a touch.

He slumped a little. “Because of me?”

Perry thought some. “Not exactly. But in a way. It wasn’t us.”

Death smiled. “Thank you.”

Perry straightened up. “Looks like you brought down a good crowd,”
he said. “Lots of friends!”

Death nodded. “Lots of friends these days,” he said. An attractive
young woman came over and squeezed his shoulder.

They were such a funny bunch in their DIY goth-frocks,
micro-manufactured customized boots, their elaborate tattoos and
implants and piercings, but for all that, cuddly and earnest with
the shadows visible of the geeks they’d been. Perry felt he was
smiling so broadly it almost hurt.

“Rides are on me, gang,” he said. “In you go. Your money’s no good
here. Any friend of Death Waits rides for free today.”

They cheered and patted him on the back as they went through, and
Death Waits looked like he’d grown three inches in his wheelchair,
and the pretty girl kissed Perry’s cheek as she went by, and Death
Waits had a smile so big you could hardly tell there was anything
wrong with his mouth.

They rode it through six times in a row, and as they came back
around for another go and another, they talked intently about the
story, the story, the story. Perry knew about the story, he’d seen
it, and he and Lester had talked it over now and again, but he was
still constantly amazed by its ability to inspire riders.

Paying customers slipped in and out, too, and seemed to catch some
of the infectious intensity of the story group. They went away in
pairs, talking about the story, and shopped the market stalls for a
while before coming back to ride again, to look for more story.

They’d never named the ride. It had always been “the ride.” Not
even a capital “R.” For a second, Perry wondered if they’d end up
calling it “The Story” in the end.

\begin{center}\rule{3in}{0.4pt}\end{center}

Perry got his Disney-in-a-Box through a circuitous route, getting
one of the hawkers’ brothers to order it to a PO box in Miami, to
which Perry would drive down to pick it up and take it back.

Lester roused himself from the apartment when Perry told him it had
arrived. Lester and Suzanne had been AWOL for days, sleeping in
until Perry left, coming back after Perry came back, until it felt
like they were just travelers staying in the same hotel.

He hadn’t heard a peep from Kettlewell or Tjan, either. He guessed
that they were off figuring things out with their money people. The
network of ride operators had taken the news with equanimity\dash{}Hilda
had helped him write the message so that it kind of implied that
everything was under control and moving along nicely.

But when Perry emailed Lester to say he was going to drive down to
the PO box the next morning before opening the ride, Lester emailed
back in minutes volunteering to come with him.

He had coffee ready by the time Perry got out of the shower. It was
still o-dark-hundred outside, the sun not yet risen, and they
hardly spoke as they got into the car, but soon they were on the
open road.

“Kettlewell and Tjan aren’t going to sue you,” Lester said. There
it was, all in a short sentence:
\emph{I’ve been talking to them. I’ve been figuring out if I’m with you or with them. I’ve been saving your ass. I’ve been deciding to be on your side.}

“Good news,” Perry said. “That would have really sucked.”

Perry waited for the rest of the drive for Lester to say something,
but he didn’t. It was a long drive.

The whole way back, Lester talked about the Disney-in-a-Box.
There’d been some alien autopsy videos of them posted online
already, engineers taking them to bits, making guesses about and
what they did and how. Lester had watched the videos avidly and he
held his own opinions, and he was eager to get at the box and find
answers for himself. It was the size of an ice-chest, too big to
fit on his lap, but he kept looking over his shoulder at it.

The box-art, a glossy pic of two children staring goggle-eyed at a
box from which Disneoid marvels were erupting, looked a little like
the Make Your Own Monster toy Perry’d had as a boy. It actually
made his heart skip a beat the way that that old toy had. Really,
wasn’t that every kid’s dream? A machine that created wonders from
dull feedstock?

They got back to the ride long before it was due to open and Perry
asked Lester if he wanted to get a second breakfast in the tea-room
in the shantytown, but Lester begged off, heading for his workshop
to get to grips with the Box.

So Perry alone waited for the ride to open, standing at his
familiar spot behind the counter. The hawkers came and nodded hello
to him. A customer showed up. Another. Perry took their money.

The ticket-counter smelled of sticky beverages spilled and left to
bake in the heat, a sour-sweet smell like bile. His chair was an
uncomfortable bar-stool he’d gotten from a kitchen-surplus place,
happy for the bargain. He’d logged a lot of hours in that chair. It
had wreaked havoc on his lower spine and tenderized his ass.

He and Lester had started this as a lark, but now it was a
movement, and not one that was good for his mental health. He
didn’t want to be sitting on that stool. He might as well be
working in a liquor store\dash{}the skill-set was the same.

Hilda broke his reverie by calling his phone. “Hey, gorgeous,” she
said. She bounded out of bed fully formed, without any intervening
stages of pre-coffee, invertebrate, pre-shower, and Homo erectus.
He could hear that she was ready to catch the world by the ankle
and chew her way up its leg.

“Hey,” he said.

“Uh oh. Mr Badvibes is back. You and Lester fight in the car?”

“Naw,” he said. “That was fine. Just\ldots{}” He told her about the
smell and the stool and working at a liquor store.

“Get one of those home-slices running the market stalls to take
over the counter, and take me to the beach, then. It’s been weeks
and I still haven’t seen the ocean. I’m beginning to think it’s an
urban legend.”

So that’s what he did. Hilda drove up in a bikini that made his jaw
drop, and bought a pair of polarizing contacts from Jason, and
Perry turned the till over to one of the more trustworthy vendors,
and they hit the road.

Hilda nuzzled him and prodded him all the way to the beach, kissing
him at the red lights. The sky was blue and clear as far as the eye
could see in all directions, and they bought a bag of oranges, a
newspaper, beach-blankets, sun-block, a picnic lunch, and a book of
replica vintage luggage stickers from hawkers at various
stop-points.

They unpacked the trunk in the parking garage and stepped out into
the bright day, and that’s when they noticed the wind. It was
blowing so hard it took Hilda’s sarong off as soon as she stepped
out onto the street. Perry barely had time to snatch the cloth out
of the air. The wind howled.

They looked up and saw the palm-trees bending like drawn bows, the
hot-dog vendors and shave-ice carts and the jewelry hawkers
hurriedly piling everything into their cars.

“Guess the beach is cancelled,” Hilda said, pointing out over the
ocean. There, on the horizon, was a wall of black cloud, scudding
rapidly toward them in the raging wind. “Shoulda checked the
weather.”

The wind whipped up stinging clouds of sand and debris. It gusted
hard and actually blew Hilda into Perry. He caught her and they
both laughed nervously.

“Is this a hurricane?” she asked, joking, not joking, tension in
her voice.

“Probably not.” He was thinking of Hurricane Wilma, though, the
year he’d moved to Florida. No one had predicted Wilma, which had
been a tropical storm miles off the coast until it wasn’t, until it
was smashing a 50km-wide path of destruction from Key West to
Kissimmee. He’d been working a straight job as a structural
engineer for a condo developer, and he’d seen what a good blow
could do to the condos of Florida, which were built mostly from
dreams, promises, spit, and kleenex.

Wilma had left cars stuck in trees, trees stuck in houses, and it
had blown just like this when it hit. There was a crackle in the
air, and the sighing of the wind turned to groans, seeming to come
from everywhere at once\dash{}the buildings were moaning in their bones
as the winds buffeted them.

“We have to get out of here,” Perry said. “Now.”

They got up to the second storey of the parking garage when the
whole building moaned and shuddered beneath them, like a tremor.
They froze on the stairwell. Somewhere in the garage, something
crashed into something else with a sound like thunder, and then it
was echoed with an actual thunder-crack, a sound like a hundred
rifles fired in unison.

Hilda looked at him. “No way. Not further up. Not in this
building.”

He agreed. They pelted down the street and into the first sleeting
showers coming out of a sky that was now dirty grey and low. A
sandwich board advertising energy beverages spun through the air
like a razor-edged frisbee, trailing a length of clothesline that
had tethered it to the front of some beach-side cafe. On the beach
across the road, beachcomber robots burrowed into the sand, trying
to get safe from the wind, but were foiled again and again, rolled
around like potato bugs into the street, into the sea, into the
buildings. They seizured like dying things. Perry felt an
irrational urge to rescue them.

“High ground,” Hilda said, pointing away from the beach. “High
ground and find a basement. Just like a twister.”

A sheet of water lifted off the surface of the sea and swept across
the road at them, soaking them to the skin, followed by a sheet of
sand that coated them from head to toe. It was all the
encouragement they needed. They ran.

They ran, but the streets were running with rain now and more
debris was rolling past them. They got up one block and sloshed
across the road. They made it halfway up the next block, past a
coffee shop and a surf-shop in low-slung buildings, and the wind
literally lifted them off their feet and slammed them to the
ground. Perry grabbed Hilda and dragged her into an alley behind
the surf-shop. There were dumpsters there, and a recessed doorway,
and they squeezed past the dumpster and into the doorway.

Now in the lee, they realized how loud the storm had been. Their
ears rang with it, and rang again with another thunderclap. Their
chests heaved and they shivered, grabbing each other. The doorway
stank of piss and the crackling ozone around them.

“This place, holy fuck, it’s about to lift off and fly away,” Hilda
said, panting. Perry’s unbroken arm throbbed and he looked down to
see a ragged cut running the length of his forearm. From the
Dumpster?

“It’s a big storm,” Perry said. “They come through now and again.
Sometimes they blow away.”

“What do they blow away? Trailers? Apartment buildings?” They were
both spitting sand and Perry’s arm oozed blood.

“Sometimes!” Perry said. They huddled together and listened to the
wind lashing at the buildings around them. The Dumpster blocking
their doorway groaned, and then it actually slid a few inches.
Water coursed down the alley before them, with debris caught in it:
branches, trash, then an electric motorcycle, scratching against
the road as it rattled through the river.

They watched it pass without speaking, then both of them screamed
and scrambled back as a hissing, soaked house-cat scrambled over
the dumpster, landing practically in their laps, clawing at them
with hysterical viciousness.

“Fuck!” Hilda said as it caught hold of her thumb with its teeth.
She pushed at its face ineffectually, hissing with pain, and Perry
finally worked a thumb into the hinge of its jaw and forced it
open. The cat sprang away, clawing up his face, leaping back onto
the Dumpster.

Hilda’s thumb was punctured many times, already running free with
blood. “I’m going to need rabies shots,” she said. “But I’ll
live.”

They cuddled, in the blood and the mud, and watched the river swell
and run with more odd debris: clothes and coolers, beer bottles and
a laptop, cartons of milk and someone’s purse. A small palm-tree. A
mailbox. Finally, the river began to wane, the rain to falter.

“Was that it?” Hilda said.

“Maybe,” Perry said. He breathed in the moist air. His arms
throbbed\dash{}one broken, the other torn open. The rain was petering out
fast now, and looking up, he could see blue sky peeking through the
dirty, heavy clouds, which were scudding away as fast as they’d
rolled in.

“Next time, we check the weather before we go to the beach,” he
said.

She laughed and leaned against him and he yelped as she came into
contact with his hurt arm. “We got to get you to a hospital,” she
said. “Get that looked at.”

“You too,” he said, pointing at her thumb. It was all so weird and
remote now, as they walked through the Miami streets, back toward
the garage. Other shocked people wandered the streets, weirdly
friendly, smiling at them like they all shared a secret.

The beach-front was in shambles, covered in blown trash and mud,
uprooted trees and fallen leaves, broken glass and rolled cars.
Perry hit the car radio before they pulled out of the garage. An
announcer reported that Tropical Storm Henry had gone about three
miles inland before petering out to a mere sun-shower, along with
news about the freeways and hospitals being equally jammed.

“Huh,” Perry said. “Well, what do we do now?”

“Let’s find a hotel room,” Hilda said. “Have showers, get something
to eat.”

It was a weird and funny idea, and Perry liked it. He’d never
played tourist in Florida, but what better place to do so? They
gathered their snacks from the back of the car and used the first
aid kit in the trunk to tape themselves up.

They tried to reach Lester but no one answered. “He’s probably at
the ride,” Perry said. “Or balls-deep in reverse-engineering the
Disney Box thing. OK, let’s find a hotel room.”

Everything on the beach was fully booked, but as they continued
inland for a couple blocks, they came upon coffin hotels stacked
four or five capsules high, painted gay Miami deco pastels,
installed in rows in old storefronts or stuck in street-parking
spots, their silvered windows looking out over the deserted
boulevards.

“Should we?” Perry said, gesturing at them.

“If we can get an empty one? Damn right\dash{}these things are going to
be in serious demand in pretty short order.”

Stepping into the coffin hotel transported Perry back to his days
on the road, his days staying at coffin hotel after coffin hotel,
to his first night with Hilda, in Madison. One look at Hilda told
him she felt the same. They washed each other slowly, as though
they were underwater, cleaning out one-another’s wounds, sluicing
away the caked on mud and grime blown deep into their ears and the
creases of their skin, nestled against their scalps.

They lay down in bed, naked, together, spooned against one another.
“You’re a good man, Perry Gibbons,” Hilda said, snuggling against
him, hand moving in slow circles on his tummy.

They slept that way and got back on the road long past dark,
driving the blasted freeway slowly, moving around the broken glass
and blown out tires that remained.

The path of the hurricane followed the coast straight to Hollywood,
a line of smashed trees and car wrecks and blown-off roofs that
made the nighttime drive even more disorienting.

They went straight back to the condo, but Lester wasn’t there.
Worry nagged at Perry. “Take me to the ride?” he said, after he’d
paced the apartment a few times.

Hilda looked up from the sofa, where she had collapsed the instant
they came through the door, arm flung over her face. “You’re
shitting me,” she said. “It’s nearly midnight, and we’ve been in a
hurricane.”

Perry squirmed. “I’ve got a bad feeling, OK? And I can’t drive
myself.” He flapped his busted arm at her.

Hilda looked at him, her eyes narrowed. “Look, don’t be a jerk, OK?
Lester’s a big boy. He’s probably just out with Suzanne. He’d have
called you if there’d been a problem.”

He looked at her, bewildered by the ferocity of her response. “OK,
I’ll call a cab,” he said, trying for a middle ground.

She jumped up from the couch. “Whatever. Fine. Let me get my keys.
Jesus.”

He had no idea how he’d angered her, but it was clear that he had,
and the last thing he wanted was to get into a car with her, but he
couldn’t think of a way of saying that without escalating things.

So they drove in white-lipped silence to the ride, Hilda tense with
anger, Perry tense with worry, both of them touchy as cats, neither
saying a word.

But when they pulled up to the ride, they both let out a gasp. It
was lit with rigged floodlights and car headlights, and it was
swarming with people. As they drew closer, they saw that the market
stalls were strewn across the parking lot, in smashed pieces. As
they drew closer still, they saw that the ride itself was staring
eyeless at them, window-glass smashed.

Perry was out of the car even before it stopped rolling, Hilda
shouting something after him. Lester was just on the other side of
the ride-entrance, wearing a paper mask and rubber boots, wading in
three-inch deep, scummy water.

Perry splashed to a halt. “Holy shit,” he breathed. The ride was
lit with glow-sticks, waterproof lamps, and LED torches, and the
lights reflected crazily from the still water that filled it as far
as the eye could see, way out into the gloom.

Lester looked up at him. His face was lined and exhausted, and it
gleamed with sweat. “Storm broke out all the windows and trashed
the roof, then flooded us out. It did a real number on the market,
too.” His voice was dead.

Perry was wordless. Bits of the ride-exhibits floated in the water,
along with the corpses of the robots.

“No drainage,” Lester said. “The code says drainage, but there’s
none here. I never noticed it before. I’m going to rig a pump, but
my workshop’s pretty much toast.” Lester’s workshop had been in the
old garden-center at the side of the ride. It was all glass. “We
had some pretty amazing winds.”

Perry felt like he should be showing off his wound to prove that he
hadn’t been fucking off while the disaster was underway, but he
couldn’t bring himself to do so. “We got caught in it in Miami,” he
said.

“Wondered where you were. The kid who was minding the shop just cut
and run when the storm rolled in.”

“He did? Christ, what an irresponsible asshole. I’ll break his
neck.”

A slimy raft of kitchen gnomes\dash{}their second business
venture\dash{}floated past silently in the harsh watery light. The smell
was almost unbearable.

“It wasn’t his job\dash{}” Lester’s voice cracked on \emph{job}, and he
breathed deeply. “It wasn’t his job, Perry. It was your job. You’re
running around, having a good time with your girlfriend, firing
lawyers\dash{}” He stopped and breathed again. “You know that they’re
going to sue us, right? They’re going to turn us into a smoking
ruin because you fired them, and what the fuck are you going to do
about that? Whose job is that?”

“I thought you said they weren’t going to sue,” Perry said. It came
out in an embarrassed mumble. Lester had never talked to him like
this. Never.

“Kettlewell and Tjan aren’t going to sue,” Lester said. “The
lawyers you fired, the venture capitalists who backed them? They’re
going to turn us into paste.”

“What would you have preferred?” Hilda said. She was standing in
the doorway, away from the flood, watching them intently. Her eyes
were raccoon-bagged, but she was rigid with anger. Perry could
hardly look at her. “Would you have preferred to have those fuckers
go around destroying the lives of your supporters in order to
enrich a few pig assholes?”

Lester just looked at her.

“Well?”

“Shut up, Yoko,” he said. “We’re having a private conversation
here.”

Perry’s jaw dropped, and Hilda was already in motion, sloshing into
the water in her sandals. She smacked Lester across the cheek, a
crack that echoed back over the water and walls.

Lester brought his hand up to his reddening face. “Are you done?”
he said, his voice hard.

Hilda looked at Perry. Lester looked at Perry. Perry looked at the
water.

“I’ll meet you by the car,” Perry said. It came out in a mumble.
They held for a moment, the three of them, then Hilda walked out
again, leaving Lester and Perry looking at one another.

“I’m sorry,” Perry said.

“About Hilda? About the lawsuits? About skipping out?”

“About everything,” he said. “Let’s fix this up, OK?”

“The ride? I don’t even know if I want to. Why bother? It’ll cost a
fortune to get it online, and they’ll only shut it down again with
the lawsuit. Why bother.”

“So we won’t fix the ride. Let’s fix us.”

“Why bother,” Lester said, and it came out in the same mumble.

The watery sounds of the room and the smell and the harsh reflected
rippling light made Perry want to leave. “Lester\dash{}” he began.

Lester shook his head. “There’s nothing more we can do tonight,
anyway. I’ll rent a pump in the morning.”

“I’ll do it,” Perry said. “You work on the Disney-in-a-Box thing.”

Lester laughed, a bitter sound. “Yeah, OK, buddy. Sure.”

Out in the parking-lot, the hawkers were putting their stalls back
together as best they could. The shantytown was lit up and Perry
wondered how it had held together. Pretty good, is what he
guessed\dash{}they met and exceeded county code on all of those plans.

Hilda honked the horn at him. She was fuming behind the wheel and
they drove in silence. He felt numb and wrung out and he didn’t
know what to say to her. He lay awake in bed that night waiting to
hear Lester come home, but he didn’t.

\begin{center}\rule{3in}{0.4pt}\end{center}

Sammy loved his morning meetings. They all came to his office, all
the different park execs, creatives, and emissaries from the old
partner companies that had spun off to make movies and merch and
educational materials. They all came each day to talk to him about
the next day’s Disney-in-a-Box build. They all came to beg him to
think about adding in something from their franchises and cantons
to the next installment.

There were over a million DiaBs in the field now, and they weren’t
even trying to keep up with orders anymore. Sammy loved looking at
the online auction sites to see what the boxes were going for\dash{}he
knew that some of his people had siphoned off a carload or two of
the things to e-tail out the back door. He loved that. Nothing was
a better barometer of your success than having made something other
people cared enough about to steal.

He loved his morning meetings, and he conducted them with the flair
of a benevolent emperor. He’d gotten a bigger office\dash{}technically it
was a board-room for DiaB strategy, but Sammy \emph{was} the DiaB
strategy. He’d outfitted it with fan-photos of their DiaB shrines
in their homes, with kids watching enthralled as the day’s model
was assembled before their eyes. The hypnotic fascination in their
eyes was unmistakable. Disney was the focus of their daily lives,
and all they wanted was more, more, more. He could push out five
models a day, ten, and they’d go nuts for them.

But he wouldn’t. He was too cunning. One model a day was all. Leave
them wanting more. Never breathe a hint of what the next day’s
model would be\dash{}oh, how he loved to watch the blogs and the chatter
as the models self-assembled, the heated, time-bound fights over
what the day’s model was going to be.

“Good morning, Ron,” he said. Wiener had been lobbying to get a
Main Street build into the models for weeks now, and Sammy was
taking great pleasure in denying it to him without shutting down
all hope. Getting Ron Wiener to grovel before him every morning was
better than a cup of coffee.

“I’ve been thinking about what you said, and you’re right,” Wiener
said. He always started the meeting by telling Sammy how right he
was to reject his last idea. “The flag-pole and marching-band scene
would have too many pieces. House cats would knock it over. We need
something more unitary, more visually striking. So here’s what I’ve
been thinking: what about the fire-engine?”

Sammy raised an indulgent eyebrow.

“Kids \emph{love} fire trucks. All the colors are in the printer’s
gamut\dash{}I checked. We could create a Mickey-and-Friends fire-crew to
position around it, a little barn for it.”

“The only thing I liked about firetrucks when I was a kid was that
the word started with ’f’ and ended with ’uck’\dash{}” Sammy smiled when
he said it, and waited for Wiener to fake hilarity, too. The others
in the room\dash{}other park execs, some of their licensing partners, a
few advertisers\dash{}laughed too. Officially, this was a “brainstorming
session,” but everyone knew that it was all about getting the nod
from Sammy.

Wiener laughed dutifully and slunk away. More supplicants came
forward.

“How about this?” She was very cute\dash{}dressed in smart, dark clothes
that were more Lower East Side than Orlando. She smelled good,
too\dash{}one of the new colognes that hinted at free monomers, like hot
plastic or a new-bought tire. Cat-slanted green eyes completed the
package.

“What you got there?” She was from an ad agency, someone Disney
Parks had done business with at some point. Agencies had been
sending their people to these meetings too, trying to get a
co-branding coup for one of their clients.

“It’s a series of three, telling a little story. Beginning, middle
and end. The first one is a family sitting down to breakfast, and
you can see, it’s the same old crap, boring microwave omelets and
breakfast puddings. Mom’s bored, dad’s more bored, and sis and
brother here are secretly dumping theirs onto mom’s and dad’s
plates. All this stuff is run using the same printers, so it looks
very realistic.”

It did indeed. Sammy hadn’t thought about it, but he supposed it
was only natural that the omelets were printed\dash{}how else could
General Mills get that uniformity? He should talk to some of the
people in food services about getting some of that tech to work at
the parks.

“So in part two, they’re setting up the kitchen around this mystery
box\dash{}one part Easy-Bake lightbulb oven, one part Tardis. You know
what that is?”

Sammy grinned. “Why yes, I believe I do.” Their eyes met in a
fierce look of mutual recognition. “It’s a breakfast printer, isn’t
it?” The other supplicants in the room sucked in a collective
breath. Some chuckled nervously.

“It’s about moving the apparatus to the edge. Bridging the last
mile. Why not? This one will do waffles, breakfast cereals, bagels
and baked goods, small cakes. New designs every day\dash{}something for
mom and dad, something for the kids, something for the sullen
teens. We’re already doing this at the regional plants and
distributorships, on much larger scales. But getting our stuff into
consumers’ homes, getting them \emph{subscribed} to our food\dash{}”

Sammy held up a hand. “I see,” he said. “And our people are already
primed for home-printing experiences. They’re right in your sweet
spot.”

“Part three, Junior and little sis are going cuckoo for Cocoa
Puffs, but these things are shaped \emph{like them}, with their
portraits on each sugar-lump. Mom and dad are eating tres
sophistique croissants and delicate cakes. Look at Rover here, with
his own cat-shaped dog-biscuit. See how happy they all are?”

Sammy nodded. “Shouldn’t this all be under nondisclosure?” he
said.

“Probably, but what are you gonna do? You guys are pretty good at
keeping secrets, and if you decide to shaft us by selling out to
one of our competitors, we’re probably dead, anyway. I’ll be able
to ship out half a million units in the first week, then we can
ramp production if need be\dash{}lots of little parts-and-assembly
subcontractors will take the work if we offer.”

Sammy liked the way she talked. Like someone who didn’t need to
spend a lot of time screwing around, planning, like someone who
could just make it happen.

“You’re launching when?”

“Three days after you start running this campaign,” she said,
without batting an eyelash.

“My name’s Sammy,” he said. “How’s Thursday?”

“Launch on Sunday?” She shook her head. “It’s tricky, Sunday
launches. Gotta pay everyone scale-and-a-half.” She gave him a
wink. “What the hell, it’s not my money.” She stuck out her hand.
She was wearing a couple of nice chunky obsidian rings in abstract
curvy shapes, looking a little porny in their suggestion of breasts
and thighs. He shook her hand and it was warm and dry and strong.

“Well, that’s this week taken care of,” Sammy said, and pointedly
cleared the white-board surface running the length of the table.
The others groaned and got up and filed out. The woman stayed
behind.

“Dinah,” she said. She handed him a card and he noted the agency.
Dallas-based, not New York, but he could tell she was a
transplant.

“You got any breakfast plans?” It was hardly gone 9AM\dash{}Sammy liked
to get these meetings started early. “I normally get something sent
in, but your little prototypes there\ldots{}”

She laughed. It was a pretty laugh. She was a couple years older
than him, and she wore it well. “Do I have breakfast plans? Sammy
my boy, I’m nothing \emph{but} breakfast plans! I have a launch on
Sunday, remember?”

“Heh. Oh yeah.”

“I’m on the next flight to DFW,” she said. “I’ve got a cab waiting
to take me to the airport.”

“I wonder if you and I need to talk over some details,” Sammy
said.

“Only if you want to do it in the taxi.”

“I was thinking we could do it on the plane,” he said.

“You’re going to buy a ticket?”

“On my plane,” he said. They’d given him use of one of the company
jets when he started really ramping production on the DiaBs.

“Oh yes, I think that can be arranged,” she said. “It’s Sammy,
right?”

“Right,” he said. They left the building and had an altogether
lovely flight to Dallas. Very productive.

\begin{center}\rule{3in}{0.4pt}\end{center}

Lester hadn’t left Suzanne’s apartment in days. She’d rented a
place in the shantytown\dash{}bemused at the idea of paying rent to a
squatter, but pleased to have a place of her own now that Lester
and Perry’s apartment had become so tense.

Technically, he was working on the Disney printers, which she found
interesting in an abstract way. They had a working one and a couple
of disassembled ones, and watching the working one do its thing was
fascinating for a day or two, but then it was just a three-d TV
with one channel, broadcasting one frame per day.

She dutifully wrote about it, though, and about Perry’s ongoing
efforts to re-open the ride. She got the sense from him that he was
heading for flat-ass broke. Lester and he had always been casual
about money, but buying all new robots, more printers, replacement
windows, fixing the roof\dash{}none of it was cheap. And with the market
in pieces, he wasn’t getting any rent.

She looked over Lester’s shoulder for the fiftieth time. “How’s it
going?”

“Don’t write about this, OK?”

He’d never said that to her.

“I’ll embargo it until you ship.”

He grunted. “Fine, I guess. OK, well, I’ve got it running on
generic goop, that part was easy. I can also load my own designs,
but that requires physical access to the thing, in order to load
new firmware. They don’t make it easy, which is weird. It’s like
they don’t plan on updating it once it’s in the field\dash{}maybe they
just plan on replacing them at regular intervals.”

“Why’s the firmware matter to you?”

“Well, that’s where it stores information about where to get the
day’s designs. If we’re going to push our own designs to it, we
need to give people an easy way to tell it to tune in to our feed,
and the best way to do that is to change the firmware. The
alternative would be, oh, I don’t know, putting another machine
upstream of it to trick it into thinking that it’s accessing their
site when it’s really going to ours. That means getting people to
configure another machine\dash{}no one but a few hardcore geeks will want
to do that.”

Suzanne nodded. She wondered if “a few hardcore geeks” summed up
the total audience for this project in any event. She didn’t
mention it, though. Lester’s brow was so furrowed you could lose a
dime in the crease above his nose.

“Well, I’m sure you’ll get it,” she said.

“Yeah. It’s just a matter of getting at the boot-loader. I could
totally do this if I could get at the boot-loader.”

Suzanne knew what a boot-loader was, just barely. The thing that
chose which OS to load when you turned it on. She wondered if every
daring, sexy technology project started like this, a cranky hacker
muttering angrily about boot-loaders.

Suzanne missed Russia. She’d had a good life there, covering the
biotech scene. Those hackers were a lot scarier than Lester and
Perry, but they were still lovable and fascinating in their own
way. Better than the Ford and GM execs she used to have to cozy up
to.

She’d liked the manic hustle of Russia, the glamour and the
squalor. She’d bought a time-share dacha that she could spend
weekends at, and the ex-pats in Petersburg had rollicking parties
and dinners where they took apart the day’s experiences on Planet
Petrograd.

“I’m going out, Lester,” she said. Lester looked up from the DiaB
and blinked a few times, then seemed to rewind the conversation.

“Hey,” he said. “Oh, hey. Sorry, Suzanne. I’m just\dash{}I’m trying to
work instead of think these days. Thinking just makes me angry. I
don’t know what to do\dash{}” He broke off and thumped the side of the
printer.

“How’s Perry getting on with rebuilding?”

“He’s getting on,” Lester said. “As far as I know. I read that the
Death Waits kid and his people had come by to help. Whatever that
means.”

“He freaks me out,” Suzanne said. “I mean, I feel terrible for him,
and he seemed nice enough in the hospital. But all those people\dash{}the
way they follow him around. It’s just weird. Like the charismatic
cults back home.” She realized she’d just called Russia “home” and
it made her frown. Just how long was she going to stay here with
these people, anyway?

Lester hadn’t noticed. “I guess they all feel sorry for him. And
they like what he has to say about stories. I just can’t get a lot
of spit in my mouth over the ride these days, though. It feels like
something we did and completed and should move on from.”

Suzanne didn’t have anything to say, and Lester wasn’t particularly
expecting anything, he was giving off a palpable let-me-work vibe,
so she let herself out of the apartment\dash{}her apartment!\dash{}and headed
out into the shantytown. On the way to the ride, she passed the
little tea-house where Kettlewell and Tjan had done their scheming
and she suddenly felt very, very old. The only grownup on-site.

She was about to cross the freeway to the ride when her phone rang.
She looked at the face and then nearly dropped it. Freddy was
calling her.

“Hello, Suzanne,” he said. The gloat in his voice was unmistakable.
He had something really slimy up his sleeve.

“How can I help you?”

“I’m calling for comment on a story,” he said. “It’s my
understanding that your lad, Perry, pitched a tantie and fired the
business-managers of the ride, and has told the lawyers
representing him against Disney that he intends to drop the suit.”

“Is there a question in there?”

“Oh, there are many questions in there, my darling. For starters, I
wondered how it could possibly be true if you haven’t written about
it on your little ’blog’\dash{}” even over the phone, she could hear the
sarcastic quotes. “\dash{}You seem to be quite comprehensive in
documenting the undertakings of your friends down there in
Florida.”

“Are you asking me to comment on why I haven’t commented?”

“For starters.”

“Have you approached Perry for a comment?”

“I’m afraid he was rather abrupt. And I couldn’t reach his Valkyrie
of the Midwest, either. So I’m left calling on you, Suzanne. Any
comment?”

Suzanne stared across the road at the ride. She’d been gassed
there, chased by armed men, watched a war there.

“The ride doesn’t have much formal decision-making process,” she
said finally. “That means that words like ’fired’ don’t really
apply here. The boys might have a disagreement about the best way
to proceed, but if that’s the case, you’ll have to talk to them
about it.”

“Are you saying that you don’t know if your boyfriend’s best friend
is fighting with his business partners? Don’t you all live
together?”

“I’m saying that if you want to find out what Lester and Perry are
doing, you’ll have to ask Lester and Perry.”

“And the living together thing?”

“We don’t live together,” she said. It was technically true.

“Really?” Freddy said.

“Do we have a bad connection?”

“You don’t live together?”

“No.”

“Where do you live then?”

“My place,” she said. “Have your informants been misinforming you?
I hope you haven’t been paying for your information, Freddy. I
suppose you don’t, though. I suppose there’s no end of cranks who
really enjoy spiteful gossip and are more than happy to email you
whatever fantasies they concoct.”

Freddy tsked. “And you don’t know what’s happened to Kettlewell and
Tjan?”

“Have you asked them?”

“I will,” he said. “But since you’re the ranking reporter on the
scene.”

“I’m just a blogger, Freddy. A busy blogger. Good afternoon.”

The call left her shaking, though she was proud of how calm she’d
kept her voice. What a goddamned troll. And she was going to have
to write about this now.

There were ladders leaned up against the edge of the ride, and a
motley crew of roofers and glaziers on them and on the roof,
working to replace the gaping holes the storm had left. The workers
mostly wore black and had dyed hair and lots of metal flashing from
their ears and faces as they worked. A couple had stripped to the
waist, revealing full-back tattoos or even more piercings and
subcutaneous implants, like armor running over their spines and
shoulder-blades. A couple of boom-boxes blasted out grinding,
incoherent music with a lot of electronic screams.

Around the ride, the market-stalls were coming back, rebuilt from a
tower of fresh-sawed lumber stacked in the parking-lot. This was a
lot more efficient, with gangs of vendors quickly sawing the lumber
to standard sizes, slapping each one with a positional sensor, then
watching the sensor’s lights to tell them when it was properly
lined up with its mates, and then slipping on corner-clips that
held it all together. Suzanne watched as a whole market stall came
together this way, in the space of five minutes, before the vendors
moved on to their next stall. It was like a high-tech version of an
Amish barn-raising, performed by bandanna-clad sketchy hawkers
instead of bearded technophobes.

She found Perry inside, leaning over a printer, tinkering with its
guts, LED torches clipped to the temples of his glasses. He was
hampered by having only one good arm, and he pressed her into
service passing him tools for a good fifteen minutes before he
straightened up and really looked at her.

“You come down to help out?”

“To write about it, actually.”

The room was a hive of activity. A lot of goth kids of various ages
and degrees of freakiness, a few of the squatter kids, some people
she recognized from the second coming of Death Waits. She couldn’t
see Death Waits, though.

“Well, that’s good.” He powered up the printer and the air filled
with the familiar smell of Saran-Wrap-in-a-microwave. She had an
eerie flashback to her first visit to this place, when they’d
showed her how they could print mutated, Warhol-ized Barbie heads.
“How’s Lester getting on with cracking that printer?”

\emph{Why don’t you ask him yourself?} She didn’t say it. She
didn’t know why Lester had come to her place after the flood
instead of going home, why he stiffened up and sniffed when she
mentioned Perry’s name, why he looked away when she mentioned
Hilda.

“Something about firmware.”

He straightened his back more, making it pop and gave her his
devilish grin, the one where his wonky eyebrow went up and down.
“It’s always firmware,” he said, and laughed a little. Maybe they
were both remembering those old days, the Boogie Woogie Elmos.

“Looks like you’ve got a lot of help,” Suzanne said, getting out a
little steno pad and a pen.

Perry nodded at it, and she was struck by how many times they’d
stood like this, a few feet apart, her pen poised over her pad.
She’d chronicled so much of this man’s life.

“They’re good people, these folks. Some of them have some carpentry
or electronics experience, the rest are willing to learn. It’s
going faster than I thought it would. Lots of support from out in
the world, too\dash{}people sending in cash to help with replacement
parts.”

“Have you heard from Kettlewell or Tjan?”

The light went out of his face. “No,” he said.

“How about from the lawyers?”

“No comment,” he said. It didn’t sound like a joke.

“Come on, Perry. People are starting to ask questions. Someone’s
going to write about this. Do you want your side told or not?”

“Not,” he said, and disappeared back into the guts of the printer.

She stared at his back for a long while before turning on her heel,
muttering, “Fuck,” and walking back out into the sunshine. There’d
been a musty smell in the ride, but out here it was the Florida
smell of citrus and car-fumes, and sweat from the people around
her, working hard, trying to wrest a living from the world.

She walked back across the freeway to the shantytown and ran into
Hilda coming the other way. The younger woman gave her a cool look
and then looked away, and crossed.

That was just about enough, Suzanne thought. Enough playtime with
the kids. Time to go find some grownups. She wasn’t here for her
health. If Lester didn’t want to hang out with her, if Perry had
had enough of her, it was time to go do something else.

She went back to her room, where Lester was still working on his
DiaB project. She took out her suitcase and packed with the
efficiency of long experience. Lester didn’t notice, not even when
she took the blouse she’d hand-washed and hung to dry on the back
of his chair, folded it and put it in her suitcase and zipped it
shut.

She looked at his back working over the bench for a long time. He
had a six-pack of chocolate pudding beside him, and a wastebasket
overflowing with food wrappers and boxes. He shifted in his seat
and let out a soft fart.

She left. She paid the landlady through the end of the week. She
could send Lester an email later.

The cab took her to Miami. It wasn’t until she got to the airport
that she realized she had no idea where she was going. Boston? San
Francisco? Petersburg? She opened her laptop and began to price out
last minute tickets. The rush of travelers moved around her and she
was jostled many times.

The standby sites gave her a thousand options\dash{}Miami to JFK to
Heathrow to Petersburg, Miami to Frankfurt to Moscow to Petersburg,
Miami to Dallas to San Francisco.\ldots{} The permutations were
overwhelming, especially since she wasn’t sure where she wanted to
be.

Then she heard something homey and familiar: a large group of
Russian tourists walking past, talking loudly in Russian,
complaining about the long flight, the bad food, and the
incompetence of their tour operator. She smiled to see the old men
with their high-waisted pants and the old women with their bouffant
hair.

She couldn’t help but eavesdrop\dash{}at their volume, she would have
been hard-pressed \emph{not} to listen in. A little boy and girl
tore ass around the airport, under the disapproving glares from DHS
goons, and they screamed as they ran, “Disney World! Disney World!
Disney World!”

She’d never been\dash{}she’d been to a couple of the kitschy Gulag parks
in Russia, and she’d grown up with Six Flags coaster parks and
Ontario Place and the CNE in Toronto, not far from Detroit. But
she’d never been to The Big One, the place that even now managed to
dominate the world’s consciousness of theme-parks.

She asked her standby sites to find her a room in a Disney hotel
instead, looking for an inclusive rate that would get her onto the
rides and pay for her meals. These were advertised at roadside
kiosks at 100-yard intervals on every freeway in Florida, so she
suspected they were the best deal going.

A moment of browsing showed her that she’d guessed wrong. A week in
Disney cost a heart-stopping sum of money\dash{}the equivalent of six
months’ rent in Petersburg. How did all these Russians afford this
trip? What the hell compelled people to part with these sums?

She was going to have to find out. It was research. Plus she needed
a vacation.

She booked in, bought a bullet-train ticket, and grabbed the handle
of her suitcase. She examined her welcome package as she waited for
the train. She was staying at something called the Polynesian
Resort hotel, and the brochure showed a ticky-tacky tiki-themed set
of longhouses set on an ersatz white-sand beach, with a crew of
Mexican and Cuban domestic workers in leis, Hawai’ian shirts, and
lava-lavas waving and smiling. Her package included a complimentary
luau\dash{}the pictures made it clear this was nothing like the tourist
luaus she’d attended in Maui. On top of that, she was entitled to a
“character breakfast” with a wage-slave in an overheated plush
costume, and an hour with a “resort counsellor” who’d help her plan
her trip for maximal fun.

The bullet-train came and took on the passengers, families bouncing
with anticipation, joking and laughing in every language spoken.
These people had just come through a US Customs checkpoint and they
were acting like the world was a fine place. She decided there must
be something to this Disney business.

\begin{center}\rule{3in}{0.4pt}\end{center}

Death Waits waited, and waited and waited for the ride to come back
online. He split his days between hanging out at home, writing
about the story, running the fly-throughs from the other rides,
watching what was happening in Brazil, answering his fan-mail; the
rest of the time he spent with his new friends down at the site of
the ride, encouraging them to pitch in and help Perry and Lester to
get the thing back up and running. Fast, please. It was driving him
bonkers not to be able to ride any longer. After everything he’d
been through, he deserved a ride.

His friends were wonderful. Wonderful! Lacey especially. She was a
nurse and a goddess of mercy. The money that flooded into his
paypals whenever his friends let it be known that he needed more
covered all his expenses. He never wanted for companionship,
conversation, helpmeets, or respect. It was a wonderful life.

If only the ride would come online.

He woke next to Lacey, she asleep still, her hair spread out across
the pillow in a fall of shiny black with blue highlights\dash{}she’d
given him a matching dye-job a few days before and they looked like
a matched set now. He let his hands lazily trace her soft skin, the
outlines of her tattoos, her implants and piercings. He felt a
stirring between his legs.

Lacey yawned and woke and kissed him. “Good morning, my handsome
man,” she said.

“Good morning, my beautiful woman. What’s the plan for today?”

“Whatever you want,” she said.

“Breakfast, then down to the ride,” he said. “I’ll do my email and
writing there today.”

“Something before breakfast?” she asked, with a lopsided smile that
was adorable.

“Oh yes, please,” he said, his voice breathy.

\begin{center}\rule{3in}{0.4pt}\end{center}

The smell at the Wal-Mart was overpowering. It was one part sharp
mold, one part industrial disinfectant, a citrus smell that made
your eyes water and your sinuses burn.

“I’ve rented some big blowers,” Perry said. “They’ll help air the
place out. If that doesn’t work, I might have to resurface the
floor, which would be rough\dash{}it could take a week to get that done
properly.”

“A week?” Death said. Jesus. No way. Not another week. He didn’t
know it for sure, but he had a feeling that a lot of these people
would stop showing up eventually if there was no ride for them to
geek out over. He sure would.

“You smell that? We can’t close the doors and the windows and leave
it like this.”

Death’s people, standing around them, listening in, nodded. It was
true. You’d melt people’s lungs if you shut them up with these
fumes.

“How can I help?” Death said. It was his constant mantra with
Perry. Sometimes he didn’t think Perry liked him very much, and it
was good to keep on reminding him that Death and his buddies were
here to be part of the solution. That Perry needed them.

“The roof is just about done, the robots are back online. The
dividers should be done today. I’ve got the chairs stripped down
for routine maintenance, I could use a couple people for that.”

“What’s Lester working on?” Death said.

“You’d have to ask him.”

Death hadn’t seen Lester in days, which was weird. He hoped Lester
didn’t dislike him. He worried a lot about whether people liked him
these days. He’d thought that Sammy liked him, after all.

“Where is he?”

“Don’t know.”

Perry put dark glasses on.

Death Waits took the hint. “Come on,” he said to Lacey, who patted
him on the hand as he lifted up in his chair and rolled out to the
van. “Let’s just call him.”

“Lo?”

“It’s Death Waits. We’re down at the ride, but there’s not much to
do around here. I thought maybe we could help you with whatever you
were working on?”

“What do you know about what I’m working on?” Lester said.

“Um. Nothing.”

“So how do you know you want to help?”

Death Waits closed his eyes. He wanted to help these two. They’d
made something important, didn’t they know that?

“What are you working on?”

“Nothing,” Lester said.

“Come on,” Death said. “Come on. We just want to pitch in. I love
you guys. You changed my life. Let me contribute.”

Lester snorted. “Cross the road, go straight for two hundred yards,
turn left at the house with the Cesar Chavez mural, and I’ll meet
you there.”

“You mean go into the\dash{}” Death didn’t know what it was called. He
always tried not to look at it when he came to the ride. That slum
across the road. He knew it was somehow connected with the ride,
but in the same way that the administrative buildings at Disney
were connected with the parks. The big difference was that Disney’s
extraneous buildings were shielded from view by berms and painted
go-away green. The weird town across the road was
\emph{right there}.

“Yeah, across the road into the shantytown.”

“OK,” Death said. “See you soon.” He hung up and patted Lacey’s
hand. “We’re going over there,” he said, pointing into the
shantytown.

“Is it safe?”

He shrugged. “I guess so.” He loved his chair, loved how tall it
made him, loved how it turned him into a half-ton cyborg who could
raise up on his rear wheels and rock back and forth like a triffid.
Now he felt very vulnerable\dash{}a crippled cyborg whose apparatus cost
a small fortune, about to go into a neighborhood full of people who
were technically homeless.

“Should we drive?”

“I think we can make it across,” he said. Traffic was light, though
the cars that bombed past were doing 90 or more. He started to
gather up a few more of his people, but reconsidered. It was a
little scary to be going into the town, but he couldn’t afford to
freak out Lester by showing up with an entourage.

The guardrail shielding the town had been bent down and flattened
and the chair wheeled over it easily, with hardly a bump. As they
crossed this border, they crossed over to another world. There were
cooking smells\dash{}barbecue and Cuban spices\dash{}and a little hint of
septic tank or compost heap. The buildings didn’t make any sense to
Death’s eye, they curved or sloped or twisted or leaned and seemed
to be made of equal parts pre-fab cement and aluminum and scrap
lumber, laundry lines, power lines, and graffiti.

Death was used to drawing stares, even before he became a cyborg
with a beautiful woman beside him, but this was different. There
were eyes everywhere. Little kids playing in the street\dash{}hadn’t
these people heard of stranger danger\dash{}stopped to stare at him with
big shoe-button eyes. Faces peered out of windows from the ground
on up to the third storey. Voices whispered and called.

Lacey gave them her sunniest smile and even waved at the little
kids, and Death tried nodding at some of the homeys staring at him
from the window of what looked like a little diner.

Death hadn’t known what to expect from this little town, but he
certainly hadn’t pictured so many little shops. He realized that he
thought of shops as being somehow civilized\dash{}tax paying,
license-bearing entities with commercial relationships with
suppliers, with cash-registers and employees. Not lawless and
wild.

But every ground-floor seemed to have at least a small shop,
advertised with bright OLED pixel-boards that showed rotating
enticements\dash{}\emph{Productos de Dominica, Beautiful for Ladies, OFERTA!!!, Fantasy Nails}.
He passed twenty different shops in as many steps, some of them
seemingly nothing more than a counter recessed into the wall with a
young man sitting behind it, grinning at them.

Lacey stopped at one and bought them cans of coffee and small
Mexican pastries dusted with cinnamon. He watched a hundred pairs
of eyes watch Lacey as she drew out her purse and paid. At first he
thought of the danger, but then he realized that if anyone was to
mug them, it would be in full sight of all these people.

It was a funny thought. He’d grown up in sparse suburbs where you’d
never see anyone walking or standing on the sidewalks or their
porches. Even though it was a “nice” neighborhood, there were
muggings and even killings at regular, horrific intervals. Walking
there felt like taking your life into your hands.

Here, in this crowded place with a human density like a Disney
park, it felt somehow safer. Weird.

They came to what had to be the Cesar Chavez mural\dash{}a Mexican in a
cowboy hat standing like a preacher on the tailgate of a truck,
surrounded by more Mexicans, farmer-types in cotton shirts and
blue-jeans and cowboy hats. They turned left and rounded a corner
into a little cul-de-sac with a confusion of hopscotches chalked
onto the ground, ringed by parked bicycles and scooters. Lester
stood among them, eating a churro in a piece of wax-paper.

“You seem to be recovering quickly,” he said, sizing up Death in
his chair. “Good to see it.” He seemed a little distant, which
Death chalked up to being interrupted.

“It’s great to see you again,” Death said. “My friends and I have
been coming by the ride every day, helping out however we can, but
we never see you there, so I thought I’d call you.”

“You’d call me.”

“To see if we could help,” Death said. “With whatever you’re
doing.”

“Come in,” Lester said. He gestured behind him and Death noticed
for the first time the small sign that said
\emph{HOTEL ROTHSCHILD,} with a stately peacock behind it.

The door was a little narrow for his rolling chair, but he managed
to get it in with a little back-and-forth, but once inside, he was
stymied by the narrow staircase leading up to the upper floors. The
lobby\dash{}such as it was\dash{}was completely filled by him, Lacey and
Lester, and even if the chair could have squeezed up the stairs, it
couldn’t have cornered to get there.

Lester looked embarrassed. “Sorry, I didn’t think of that. Um. OK,
I could rig a winch and hoist the chair up if you want. We’d have
to belt you in, but it’s do-able. There are masts for pulleys on
the top floor\dash{}it’s how they get the beds into the upper stories.”

“I can get up on canes,” Death Waits said. “Is it safe to leave my
chair outside, though?”

Lester’s eyebrows went up. “Well of course\dash{}sure it is.” Death felt
weird for having asked. He backed the chair out and locked the
transmission, feeling silly. Who was going to hot-wire a
wheelchair? He was such a dork. Lacey handed him his canes and he
stood gingerly. He’d been making his way to the bathroom and back
on canes all week, but he hadn’t tried stairs yet. He hoped Lester
wasn’t too many floors up.

Lester turned out to be on the third floor, and by the time they
reached it, Death Waits was dripping sweat and his eyeliner had run
into his eyes. Lacey dabbed at him with her gauzy scarf and fussed
over him. Death caught Lester looking at the two of them with a
little smirk, so he pushed Lacey away and steadied his breathing
with an effort.

“OK,” he said. “All done.”

“Great,” Lester said. “This is what I’m working on. You talked to
Perry about it before, right? The Disney-in-a-Box printers. Well,
I’ve cracked it. We can load our own firmware onto it\dash{}just stick it
on a network with a PC, and the PC will find it and update it. Then
it becomes an open box\dash{}it’ll accept anyone’s goop. You can send it
your own plans.”

Death hadn’t seen a DiaB in person yet. Beholding it and knowing
that he was the reason that Lester and Perry were experimenting
with it in the first place made him feel a sense of excitement he
hadn’t felt since the goth rehab of Fantasyland began.

“So how does this tie in to the ride?” Death asked. “I was thinking
of building rides in miniature, but at that scale, will it really
impress people? No, I don’t think so.

“So instead I was thinking that we could just push out details from
the ride, little tabletop-sized miniatures showing a piece every
day. Maybe whatever was newest. And you could have multiple feeds,
you know, like an experimental trunk for objects that people in one
region liked\dash{}”

Lester was shaking his head and holding up his hands. “Woah, wait a
second. No, no, no\dash{}” Death was used to having his friends hang on
his every word when he was talking about ideas for the ride and the
story, so this brought him up short. He reminded himself who he was
talking to.

“Sorry,” he said. “Got ahead of myself.”

“Look,” Lester said, prodding at the printer. “This thing is its
own thing. We’re about more than the ride here. I know you really
like it, and that’s very cool, but there’s no way that everything I
do from now on is going to be about that fucking thing. It was a
lark, it’s cool, it’s got its own momentum. But these boxes are
going to be their own thing. I want to show people how to take
control of the stuff in their living rooms, not advertise my little
commercial project to them.”

Death couldn’t make sense out of this. It sounded like Lester
didn’t \emph{like} the ride. How was that possible? “I don’t get
it,” he said at last. Lester was making him look like an idiot in
front of Lacey, too. He didn’t like how this was going at all.

Lester picked up a screwdriver. “You see this? It’s a tool. You can
pick it up and you can unscrew stuff or screw stuff in. You can use
the handle for a hammer. You can use the blade to open paint cans.
You can throw it away, loan it out, or paint it purple and frame
it.” He thumped the printer. “This thing is a tool, too, but it’s
not \emph{your} tool. It belongs to someone else\dash{}Disney. It isn’t
interested in listening to you or obeying you. It doesn’t want to
give you more control over your life.

“This thing reminds me of life before fatkins. It was my very own
personal body, but it wasn’t under my control. What’s the word the
academics use? ‘Agency.’ I didn’t have any agency. It didn’t matter
what I did, I was just this fat thing that my brain had to lug
around behind it, listening to its never-ending complaints and
aches and pains.

“If you don’t control your life, you’re miserable. Think of the
people who don’t get to run their own lives: prisoners,
reform-school kids, mental patients. There’s something inherently
awful about living like that. Autonomy makes us happy.”

He thumped the top of the printer again. “So here’s this stupid
thing, which Disney gives you for free. It looks like a tool, like
a thing that you use to better your life, but in reality, it’s a
tool that Disney uses to control your life. You can’t program it.
You can’t change the channel. It doesn’t even have an
\emph{off switch}. That’s what gets me exercised. I want to
redesign this thing so it gets converted from something that
controls to something that gives you control.”

Lester’s eyes shone. Death hurt from head to toe, from the climb
and the aftermath of the beating, and the life he’d lived. Lester
was telling him that the ride wasn’t important to him anymore, that
he’d be doing this other thing with the printer next, and then
something else, and then something else. He felt a great,
unexpected upwelling of bitterness at the thought.

“So what about the ride?”

“The ride? I told you. I’m done with it. It’s time to do the next
thing. You said you wanted to help out, right?”

“With the ride,” Death said patiently, with the manner of someone
talking to a child.

Lester turned his back on Death.

“I’m done with the ride,” Lester said. “I don’t want to waste your
time.” It was clear he meant, \emph{You’re wasting my time.} He
bent over the printer.

Lacey looked daggers at his shoulders, then turned to help Death
down the stairs. His canes clattered on the narrow staircase, and
it was all he could do to keep from crying.

\begin{center}\rule{3in}{0.4pt}\end{center}

Suzanne rode the bullet-train from Miami airport in air-conditioned
amusement, watching the Mickey-shaped hang-straps rock back and
forth. She’d bought herself a Mickey waffle and a bucket-sized Diet
Coke in the dining car and fended off the offers of plush
animatronic toys that were clearly descended from Boogie-Woogie
Elmo.

Now she watched the kids tear ass up and down the train, or sit
mesmerized by the videos and interactives set up at the ends of the
cars. The train was really slick, and judging from the brochure she
found in the seat-pocket, there was another one from the Orlando
airport. These things were like chutes leading from the luggage
carousel straight into the parks. Disney had figured out how to
make sure that every penny spent by its tourists went straight into
its coffers.

The voice-over announcements as they pulled into the station were
in English, Chinese, Spanish, Farsi and Russian\dash{}in that order\dash{}and
displayed on the porters’ red coats with brass buttons were
name-badges with the flags of many nations, denoting the languages
they spoke. They wore mouse-ears, and Suzanne\dash{}a veteran of
innumerable hotels\dash{}could not dissuade one from taking her
suitcase.

He brought her to a coach-station and saw her aboard a bus marked
for the Polynesian, decorated with tiki-lamps, bamboo, and
palm-fronds (she touched one and discovered that it was vinyl). He
refused her tip as they saw her aboard, and then stood and waved
her off with his white gloves and giant white smile. She had to
chuckle as she pulled away, amazed at how effective these little
touches were. She felt her muscles loosening, little involuntary
chuckles rising in her throat. The coach was full of parents and
children from all over the world, grinning and laughing and hugging
and talking excitedly about the day ahead of them.

The coach let them off to a group of Hawai’ian-shirt-clad staff who
shouted “Aloha!” at them as they debarked, and picked up their
luggage with swift, cheerful, relentless efficiency. Her check-in
was so painless she wasn’t sure it was over until a nice young lady
who looked Chechen picked up her bag for her and urged her out to
the grounds, which were green and lush, like nothing she’d seen
since landing in Florida. She was surrounded by the hotel
structures, long-houses decorated with Polynesian masks and stalked
by leggy ibises and chirping tropical birds. Before her was a
white-sand beach fronting onto an artificial lake ringed with other
luxury hotels: a gigantic 1970s Soviet A-frame building and a
gingerbread-choked Victorian hotel. The lake was ringed with a
monorail track and plied by handsome paddle-wheeler ferry-boats.

She stared gape-jawed at this until the bellhop gently tugged at
her elbow, giving her a dazzling smile.

Her room was the kind of thing you’d see Lucy and Ricky checking
into on honeymoon in an old \emph{I Love Lucy} episode\dash{}wicker
ceiling fans, bamboo furniture, a huge hot-tub shaped like a
seashell. Outside, a little terrace looking over the lake, with a
pair of cockatoos looking quizzically at her. The bellhop waved at
them and they cawed at her and flew off. Suzanne must have made a
disappointed noise, because the bellhop patted her on the arm and
said, “Don’t worry, we feed them here, they come back all the time.
Greedy birdies!”

She tipped the bellhop five bucks once she’d been given the grand
tour of the room\dash{}a tame Internet connection that was “kid-friendly”
and a likewise censored video-on-demand service, delivery pizza or
sushi, information on park hours, including the dazzling array of
extras she could purchase. It turned out that resort guests were
eligible to purchase priority passes for boarding rides ahead of
the plebes, and for entering parks early and staying late. This
made Suzanne feel right at home\dash{}it was very Russian in its
approach: the more you spent, the better your time was.

She bought it all: all the fast-passes and priority cards, all of
it loaded into a grinning Mickey on a lanyard, a wireless pendant
that would take care of her everywhere she went in the park,
letting her spend money like water.

Thus girded, she consulted with her bellhop some more and laid out
an itinerary. Once she’d showered she found she didn’t want to wear
any of her European tailored shorts and blouses. She wanted to
disappear into the Great American Mass. The hotel gift shop
provided her with a barkcloth Hawai’ian shirt decorated with
tessellated Disney trademarks and a big pair of loose shorts, and
once she donned them, she saw that she could be anyone now, any
tourist in the park. A pair of cheap sunglasses completed the look
and she paid for it all by waving her Mickey necklace at the
register, spending money like water.

She passed the rest of the day at the Magic Kingdom, taking a ferry
from the hotel’s pier to the Victorian wrought-iron docks on the
other side of the little artificial lake. As she cleared the
turnstiles into Main Street, USA, her heart quickened. Kids rushed
past her, chased by their parents’ laughing calls to slow down.
Balloon sellers and old-fashioned popcorn machines jostled for
space in the crowd, and a brass band was marching down the street
in straw boaters and red striped jackets, playing a Sousa march.

She ambled up the road, peering in the adorable little shop
windows, like the shops in a fancy casino, all themed artificial
facades that were, in back, all one shop, linked through the length
of the street.

She reached the castle before she realized it, and saw that it was
shorter than it had appeared. Turning around and looking back down
Main Street, she saw that the trees lining the sides of the street
had been trimmed so they got progressively smaller from the gates
to the castle, creating a kind of false perspective line. She
laughed now, amused by the accomplishment of the little trompe
l’oeil.

She squeezed past the hordes of Asian tourists taking precisely the
same picture of the castle, one after another, a phenomenon she’d
observed at other famous landmarks. For some Japanese shutterbugs,
the holiday photo experience was as formal as the Stations of the
Cross, with each picture of each landmark rigidly prescribed by
custom and unwritten law.

Now she was under the castle and headed for what her map assured
her was Fantasyland. Just as she cleared the archway, she
remembered her conversations with that Death Waits kid about
Fantasyland: this was the part that had been made over as a goth
area, and then remade as the Happiest Construction Site on Earth.

And so it was. The contrast was stark. From fairy castle to
green-painted construction sidings. From smiling, well-turned out
“castmembers” to construction workers with butt-crack-itis and
grouchy expressions. Fantasyland was like an ugly scar on the
blemish-free face of a Barbie doll.

She liked it.

Something about all that artifice, all that cunning work to cover
up all the bodies a company like Disney would have buried under its
manicured Main Street\dash{}it had given her a low-level, tooth-grinding
headache, a kind of anger at the falseness of it all. Here, she
could see the bodies as they buried them.

Out came her camera and she went on the prowl, photographing and
photographing, seeking high ground from which to catch snaps over
the siding. She’d look at the satellite pics of this spot later.

Now she knew what her next project would be: she would document
this scar. She’d dig up the bodies.

Just for completeness’ sake, she went on some of the rides. Her
super-fancy pass let her sail past the long lines of bored kids,
angry dads, exhausted moms. She captured their expressions with her
camera.

The rides were all right. She was sick of rides, truth be told. As
an art-form, they were wildly overrated. Some of them made her sick
and some of them were like mildly interesting trips through
someone’s collection of action-figures in a dark room. The Disney
rides didn’t even let you drive, like Lester’s ride did, and you
didn’t get to vote on them.

By the time the sun had gone down, she was ready to go back to the
room and start writing. She wanted to get all this down, the beauty
and the terror, the commerce lurking underneath the friendly
facade. As the day lengthened into night, there were more and more
screaming children, more angry parents. She caught parents smacking
kids, once, twice, got her camera out, caught three more.

They sent a big pupu platter up to her room with a dish of poi and
a hollow pineapple filled with rum. She took her computer out onto
her lanai and looked out over the lake. An ibis came by and
demanded some of her dinner scraps. She obliged it and it gave her
a cold look, as if determining whether she’d be good for dessert,
then flew off.

She began to write.

\begin{center}\rule{3in}{0.4pt}\end{center}

Something had changed between Kettlewell and Eva since they’d left
Florida with the kids. It wasn’t just the legal hassles, though
there were plenty of those. They’d gone to Florida with a second
chance\dash{}a chance for him to be a mover again, a chance for her to
have a husband who was happy with his life again.

Now he found himself sneaking past her when she was in the living
room and they slept back to back in bed with as much room between
them as possible.

Ada missed Lyenitchka and spent all her time in her bedroom IMing
her friend or going questing with her in their favorite game, which
involved Barbies, balrogs, and buying outfits. Pascal missed all
the attention he had received as the designated mascot of the two
little girls.

It was not a high point in the history of the Kettlewell clan.

“Hello?”

“Landon Kettlewell?”

“Hello, Freddy,” he said.

“My fame precedes me,” the journalist said. Kettlewell could hear
the grin in his voice. That voice was unmistakable\dash{}Kettlewell had
heard it in the occassional harassing voicemail that Suzanne
forwarded on.

“How are you?”

“Oh, I’m very well sir, and kind of you to ask, yes indeed. I hear
you’re not doing so well, though?”

“I can’t complain.”

“I wish you would, though.” You could tell, Freddy thought he was a
funny son of a bitch. “Seriously, Mr Kettlewell. I’m calling to
follow up on the story of the litigation that Perry Gibbons and
Lester Banks are facing for unilaterally canceling the arrangement
you’d made to finance their litigation. I’m hoping that you’ll give
me a quote that might put this into perspective. Is the defense
off? Will Gibbons and Banks be sued? Are you a party to the suit?”

“Freddy?”

“Yes, Mr Kettlewell.”

“I am not a child, nor am I a fool, nor am I a sucker. I’m also not
a hothead. You can’t goad me into saying something. You can’t trick
me into saying something. I haven’t hung up on you yet, but I will
unless you can give me a single good reason to believe that any
good could possibly come out of talking to you.”

“I’m going to write this story and publish it today. I can either
write that you declined to comment or I can write down whatever
comment you might have on the matter. You tell me which is
fairer?”

“Goodbye, Freddy.”

“Wait, wait! Just wait.”

Kettlewell liked the pleading note in Freddy’s voice.

“What is it, Freddy?”

“Can I get you to comment on the general idea of litigation
investment? A lot of people followed your lead in seeking out
litigation investment opportunities. There’s lots of money tied up
in it these days. Do incidents like the one in Florida mean that
litigation investment is a dead strategy?”

“Of course not,” Kettlewell snapped. He shouldn’t be talking to
this man, but the question drove him bonkers. He’d invented
litigation investment. “Those big old companies have two common
characteristics: they’ve accumulated more assets than they know
what to do with, and they’ve got poisonous, monopolistic cultures
that reward executives who break the law to help the company turn a
buck. None of that’s changed, and so long as that’s all true, there
will be little companies with legit gripes against big companies
that can be used as investment vehicles for unlocking all that dead
Fortune 100 capital and putting it to work.”

“But aren’t Fortune 100 companies investing in litigation funds?”

Kettlewell suppressed a nasty laugh. “Yeah, so what?”

“Well, if this is about destroying Fortune 100 companies\dash{}”

“It’s about wringing positive social value out of the courts and
out of investment. The way it used to work, there were only two
possible outcomes when a big company did something rotten: either
they’d get away scot-free or they’d make some lawyers very, very
rich. Litigation funds fix that. They socialize the cost of
bringing big companies to heel, and they free up the capital that
these big companies have accumulated.”

“But when a big company invests in destroying another big
company\dash{}”

“Sometimes you get a forest where a few trees end up winning, they
form a canopy that keeps all the sunlight from reaching the floor.
Now, this is stable for forests, but stability is the \emph{last}
thing you want in a market. Just look at what happens when one of
those big trees falls over: whoosh! A million kinds of life are
spawned on the floor, fighting for the light that tree had hogged
for itself. In a market, when you topple a company that’s come to
complacently control some part of the ecosystem, you free up that
niche for new innovators.”

“And why is that better than stability? Don’t the workers at these
companies deserve the security that comes from their employers’
survival?”

“Oh come on, Freddy. Stop beating that drum. If you’re an employee
and you want to get a good deal out of an employer, you’re better
off if you’ve got fifty companies you could work for than just
one.”

“So you’re saying that if you destroy Disney with your lawsuit, the
fifty thousand people who work at Walt Disney World will be able
to, what, work for those little rides like your friends have
built?”

“They’ll find lots of work, Freddy. If we make it possible for
anyone to open an innovative little ride without worrying about
getting clobbered by a big old monopolist. You like big
corporations so much?”

“Yes, but it’s not little innovative startups that invest in these
funds, is it?”

“It’s they who benefit once the fund takes up their cause.”

“And how’s that working out for the ride people you’re meant to be
helping out? They rejected you, didn’t they?”

Kettlewell really hated Freddy, he realized. Not just a little\dash{}he
had a deep and genuine loathing. “Oh, for fuck’s sake. You don’t
like little companies. You don’t like big companies. You don’t like
workers’ co-ops. What do you want us to do, Freddy? You want us to
just curl up under a rock and die? You sit there and make up your
funny names for things; you make your snarky little commentaries,
but how much good have \emph{you} done for the world, you
complaining, sniping little troll?”

The line got very quiet. “Can I quote you?”

“You certainly can,” Kettlewell huffed. In for a penny, in for a
pound. “You can print that, and you can kiss my ass.”

“Thank you, Mr Kettlewell,” Freddy said. “I’ll certainly take the
suggestion under advisement.”

Kettlewell stood in his home office and stared at the four walls.
Upstairs, Pascal was crying. He did that a lot lately. Kettlewell
breathed deeply and tried to chill out.

Someone was knocking at his door, though. He answered it
tentatively. The kid he found there was well-scrubbed, black, in
his twenties, and smiling amiably.

“Landon Kettlewell?”

“Who’s suing me?” Kettlewell could spot a process server a mile
away.

The guy shrugged and made a little you-got-me smile. “Couldn’t say,
sir,” he said, and handed Kettlewell the envelope, holding it so
that the header was clearly visible to the camera set into the
lapel of his shirt.

“You want me to sign something?” Kettlewell said.

“It’s all right, sir,” the kid said and pointed at the camera.
“It’s all caught on video.”

“Oh, right,” Kettlewell said. “Want a cup of water? Coffee?”

“I expect you’re going to be too busy to entertain, sir,” the kid
said, and ticked a little salute off his forehead. “But you seem
like a nice guy. Good luck with it all.”

Kettlewell watched him go, then closed the door and walked back to
his office, opening the envelope and scanning it. No surprises
there\dash{}the shareholders in the investment syndicate that had backed
Lester and Perry were suing him for making false representations
about his ability to speak for them.

Tjan called him a minute later.

“They got you too, huh?” Kettlewell said.

“Just left. Wish I could say it was unexpected.”

“Wish I could say I blamed them,” Kettlewell said.

“Hey, you should see what the ride’s been doing this week since
Florida went down,” Tjan said. “It’s totally mutated. I think it’s
mostly coming from the Midwest, though those Brazilians seem to
keep on logging in somehow too.”

“How many rides are there in South America, anyways?”

“Brazilians of them!” Tjan said with a mirthless chuckle.
“Impossible to say. They’ve got some kind of variant on the
protocol that lets a bunch of them share one network address. I
think some of them aren’t even physical rides, just virtual
flythroughs. Some are directly linked, some do a kind of mash-up
between their current norms and other rides’ current norms. It’s
pretty weird.”

Kettlewell paced. “Well, at least someone’s having a good time.”

“They’re going to nail us to the wall,” Tjan said. “Both of us.
Probably the individual ride-operators, too. They’re out for
blood.”

“It’s not like they even lost much money.”

“They didn’t need to\dash{}they feel like they lost the money they might
have won from Disney.”

“But that was twenty years away, and highly speculative.”

Tjan sighed heavily on the other end of the phone. “Landon, you’re
a very, very good finance person. The best I’ve ever met, but you
really need to understand that even the most speculative investor
is mostly speculating about how he’s going to spend all the money
you’re about to make him. If investors didn’t count their chickens
before they hatched, you’d never raise a cent.”

“Yeah,” Kettlewell said. He knew it, but he couldn’t soak it in.
He’d won and lost so many fortunes\dash{}his own and others’\dash{}that he’d
learned to take it all in stride. Not everyone else was so
sanguine.

“So what do we do about it? I don’t much want to lose everything.”

“You could always go back to Russia,” Kettlewell said, suddenly
feeling short-tempered. Why did he always have to come up with the
plan? “Sorry. You know what the lawyers are going to tell us.”

“Yeah. Sue Perry and Lester.”

“And we told Lester we wouldn’t do that. It was probably a mistake
to do this at all, you know.”

“No, don’t say that. The idea was a really good one. You might have
saved their asses if they’d played along.”

“And if I’d kept the lawyers on a shorter leash.”

They both sat in glum silence.

“How about if we defend ourselves by producing evidence that they
reneged on a deal we’d made in good faith. Then the bastards can
sue Perry and Lester and we’ll still be keeping our promise.”

Kettlewell tried to picture Perry in a courtroom. He’d never been
the most even-keeled dude and since he’d been shot and had his arm
broken and been gassed, he was almost pathological.

“I’ve got a better idea,” he said, growing excited as it unfolded
in his mind. He had that burning sensation he got sometimes when he
knew he was having a real doozy. “How about if we approach each of
the individual ride co-ops and see if they’ll join the lawsuit
separately from the umbrella org? Play it right and we’ll have the
lawsuit back on, without having to get our asses handed to us and
without having to destroy Perry and Lester!”

Tjan laughed. “That’s\dash{}that’s\ldots{} Wow! Genius. Yeah, OK, right! The
Boston group is in, I’ll tell you that much. I’m sure we can get
half a dozen more in, too. Especially if we can get Perry to agree
not to block it, which I’m sure he’ll do after I have a little talk
with him. This’ll work!”

“Sometimes the threat of total legal destruction can have a
wonderful, clarifying effect on one’s mind,” Kettlewell said drily.
“How’re the kids?”

“Lyenitchka is in a sulk. She wants to go back to Florida and she
wants to see Ada some more. Plus she’s upset that we never made it
to Disney World.”

Kettlewell flopped down on his couch. “Have you seen Suzanne’s blog
lately?”

Tjan laughed. “Yeah. Man, she’s giving it to them with both
barrels. Makes me feel sorry for ’em.”

“Um, you \emph{do} know that we’re suing them for everything
they’ve got, right?”

“Well, yes. But that’s just money. Suzanne’s going to take their
balls.”

They exchanged some more niceties and promised that they’d get
together face-to-face real soon and Kettlewell hung up. From behind
him, he heard someone fidgeting.

“Kids, you know you aren’t supposed to come into my office.”

“Sounds like things have gotten started up again.” It wasn’t the
kids, it was Eva. He sat up. She was standing with her arms folded
in the doorway of his office, staring at him.

“Yeah,” he said, mumbling a little. She was really beautiful, his
wife, and she put up with a hell of a lot. He felt obscurely
ashamed of the way that he’d treated her. He wished he could stand
up and give her a warm hug. He couldn’t.

Instead, she sat beside him. “Sounds like you’ll be busy.”

“Oh, I just need to get all the individual co-ops on board, talk to
the lawyers, get the investors off my back. Have a shareholders’
meeting. It’ll be fine.”

Her smile was little and sad. “I’m going, Landon,” she said.

The blood drained from his face. She’d left him plenty, over the
years. He’d deserved it. But it had always been white-hot, in the
middle of a fight, and it had always ended with some kind of
reconciliation. This time, it had the feeling of something planned
and executed in cold blood.

He sat up and folded his hands in his lap. He didn’t know what else
to do.

Her smile wilted. “It’s not going to work, you and me. I can’t live
like this, lurching from crisis to crisis. I love you too much to
watch that happen. I hate what it turns me into. You’re only happy
when you’re miserable, you know that? I can’t do that forever.
We’ll be part of each others’ lives forever, but I can’t be Mrs
Stressbunny forever.”

None of this was new. She’d shouted variations on this at him at
many times in their relationship. The difference was that now she
wasn’t shouting. She was calm, assured, sad but not crying. Behind
her in the hallway, he saw that she’d packed her suitcase, and the
little suitcases the kids used when they travelled together.

“Where will you go?”

“I’m going to stay with Lucy, from college. She’s living down the
peninsula in Mountain View. She’s got room for the kids.”

He felt like raging at her, promising her a bitter divorce and
custody suit, but he couldn’t do it. She was completely right,
after all. Even though his first impulse was to argue, he couldn’t
do it just then.

So she left, and Kettlewell was alone in his nice apartment with
his phone and his computer and his lawsuits and his mind fizzing
with ideas.

\begin{center}\rule{3in}{0.4pt}\end{center}

The last thing Sammy wanted was a fight. Dinah’s promo was making
major bank for the company\dash{}and he was taking more and more meetings
in Texas with Dinah, which was a hell of a perk. They’d shipped two
million of the DiaBs, and were projecting ten million in the first
quarter. Park admission was soaring and the revenue from the
advertising was going to cover the entire cost of the next rev of
the DiaBs, which would be better, faster, smaller and cheaper.

That business with Death Waits and the new Fantasyland and the
ride\dash{}what did it matter now? He’d been so focused on the details
that he’d lost track of the big picture. Walt Disney had made his
empire by figuring out how to do the next thing, not wasting his
energy on how to protect the last thing. It had all been a mistake,
a dumb mistake, and now he was back on track. From all appearances,
the lawsuits were on the verge of blowing away, anyway.
Fantasyland\dash{}he’d turned that over to Wiener, of all people, and he
was actually doing some good stuff there. Really running with the
idea of restoring it as a nostalgia site aimed squarely at fatkins,
with lots of food and romantic kiddie rides that no kid would want
to ride in the age of the break-neck coaster.

The last thing he wanted was a fight. What he wanted was to make
assloads of money for the company, remake himself as a power in the
organization.

But he was about to have a fight.

Hackelberg came into his office unannounced. Sammy had some of the
Imagineers in, showing him prototypes of the next model, which was
being designed for more reliable shipping and easier packing.
Hackelberg was carrying his cane today, wearing his ice-cream suit,
and was flushed a deep, angry red that seemed to boil up from his
collar.

One look from his blazing eyes was enough to send the Imagineers
scurrying. They didn’t even take their prototype with them.
Hackelberg closed the door behind them.

“Hello, Samuel,” he said.

“Nice to see you. Can I offer you a glass of water? Iced tea?”

Hackelberg waved the offers away. “They’re using your boxes to
print their own designs,” he said.

“What?”

“Those freaks with their home-made ride. They’ve just published a
system for printing their own objects on your boxes.”

Sammy rewound the conversations he’d had with the infosec people in
Imagineering about what countermeasures they’d come up with, what
they were proof against. He was pissed that he was finding out
about this from Hackelberg. If Lester and Perry were hacking the
DiaBs, they would be talking about it nonstop, running their mouths
on the Internet. Back when he was his own competitive intelligence
specialist, he would have known about this project the second it
began. Now he was trying to find a competitive intelligence person
who knew his ass from his elbow, so far without success.

“Well, that’s regrettable, obviously, but so long as we’re still
selling the consumables\ldots{}” The goop was a huge profit-maker for
the company. They bought it in bulk, added a proprietary, precisely
mixed chemical that the printer could check for in its hoppers, and
sold it to the DiaB users for a two thousand percent markup. If you
tried to substitute a competitor’s goop, the machine would reject
it. They shipped out new DiaBs with only half a load of goop, so
that the first purchase would come fast. It was making more money,
week-on-week, than popcorn.

“The crack they’re distributing also disables the checking for the
watermark. You can use any generic goop in them.”

Sammy shook his head and restrained himself from thumping his hand
down on the desk. He wanted to scream.

“We’re not suing them, are we?”

“Do you think that’s wise, Samuel?”

“I’m no legal expert. You tell me. Maybe we can take stronger
countermeasures with the next generation\dash{}” He gestured at the
prototype on his desk.

“And abandon the two million units we’ve shipped to date?”

Sammy thought about it. Those families might hang on to their
original two million forever, or until they wore out. Maybe he
should be building them to fall apart after six months of use, to
force updates.

“It’s just so unfair. They’re ripping us off. We spent the money on
those units so that we could send our message out. What the hell is
wrong with those people? Are they compulsive? Do they \emph{have}
to destroy every money-making business?”

Hackelberg sat back. “Samuel, I think it’s time we dealt with
them.”

Sammy’s mind was still off on the strategies for keeping Lester and
Perry at bay, though. Sure, a six-month obsolescence curve would do
it. Or they could just charge money for the DiaBs now that people
were starting to understand what they were for. Hell, they could
just make the most compelling stuff for a DiaB to print and maybe
that would be enough.

Hackelberg tapped the tip of his cane once, sharply. Sammy came
back to the conversation. “So that’s settled. Filing suit today.
We’re going to do a discovery on them that’ll split them open from
asshole to throat. No more of this chickenshit police stuff\dash{}we’re
going to figure out every source of income these bastards have,
we’re going to take away their computers, we’re going down to their
ISPs and getting their emails and instant messages.

“And as we’ve seen, they’re going to retaliate. That’s fine. We’re
not treating these people as a couple of punk pirates who go down
at the first sign of trouble. Not anymore. We know that these
people are the competition. We’re going to make an example of them.
They’re the first ones to attack on this front, but they won’t be
the last. We’re vulnerable, Samuel, but we can contain that
vulnerability with enough deterrent.”

Hackelberg seemed to be expecting something of Sammy, but Sammy was
damned if he knew what it was. “OK,” he said lamely.

Hackelberg’s smile was like a jack o’lantern’s. “That means that
we’ve got to be prepared for their discovery on \emph{us}. I need
to know every single detail of this DiaB project, including the
things I’d find if I went through your phone records and your
email. Because they \emph{will} be going through them. They’ll be
putting you and your operation under the microscope.”

Sammy restrained his groan. “I’ll have it for you,” he said. “Give
me a day or two.”

He saw Hackelberg out of his office as quickly as he could, then
shut the door. Hackelberg wanted everything, and that meant
\emph{everything}, including his playmates from the advertising
industry\dash{}everything. He was becoming the kind of executive who
emitted strategic intelligence, rather than the kind who gathered
it. That wouldn’t do. That wasn’t the natural order of things.

He sat down at his computer. Someone had to do the competitive
intelligence work around here and it looked like it would have to
be him.

\begin{center}\rule{3in}{0.4pt}\end{center}

What the World Can Learn from Disney

Suzanne Church

It’s easy to dismiss Disney. They make more lawsuits than rides
these days. They have a reputation for Polyannaish chirpiness.
Their corporate communications veer from Corporate Passive Voice
Third Person to a syrupy, condescending kiddee-speak that’s
calculated to drive children into a frenzy of parent-nagging
screeches.

But if you haven’t been to a Disney Park in a while, you don’t know
what you’re missing. I’ve been in Walt Disney World for a week now,
and I’m here to tell you, it’s pretty good. No, it’s better than
that\dash{}it’s \emph{amazing}.

You’ve probably heard about the attention to detail: the roofline
over Fantasyland features sagging, Georgian tiles, crazy chimneys,
and subtly animated gargoyles (left over from a previous, goth-ier
incarnation of this part of the park). You don’t see this unless
you raise your eyes above the busy, intriguing facades that front
the rides, above the masterfully painted signage, and higher still.
In other words, unless you’re someone like me, looking for details,
you won’t spot them. They’re there as pure gold-plating, they’re
there because someone who took pride in his work
\emph{put them there}.

It tells you something about the people behind the scenes here.
People who care about their jobs work here. It’s easy to forget
that when you’re thinking about Disney, a company whose reputation
these days has more to do with whom they sue than with what they
make.

But oh, what they make. There’s a safari park here, something like
a zoo but without that stuff that makes you feel like you’re
participating in some terrible exercise that strips noble animals
of their dignity for our amusement. Instead, the animals here roam
free, near their hairless monkey cousins, separated from them by
water features, camouflaged ditches, simulated ancient ruins [more
details].

That’s just one of six parks, each subdivided into six or seven
“lands,” each land with its own unique charm, culture, and customs.
That’s not counting the outlying areas: two new towns, golf
courses, a velodrome, a preserved marshland that you can tour in a
skiff with a local naturist. In these days of cheap fabrication,
it’s easy to forget what you can do with several billion dollars
and the kind of hubris that leads you to dredge lakes, erect papier
mache mountains, and create your own toy mass-transit system.

Of course, Disney Parks are no strangers to small scale
fabrication. See their tiny, clever Disney-in-a-Box devices, which
I have chronicled here from the other side. On the one hand, these
things are networked volumetric printers, but on the other, they
are superb category-busters that have achieved an entirely
justifiable\dash{}yet still staggering\dash{}market penetration in just a few
months.

I came here ready to be bored and disgusted and fleeced of every
nickel. I am disappointed. The parks are tremendous at separating
people from money, it’s true. They’ve structured each promenade and
stroll so that even a walk to the bathroom can create a
Mommy-Daddy-Want-It-NOW situation. For such a happy place, there
certainly are a lot of weepy children and frustrated parents.

But it’s hard to fault Disney for being a business that makes a lot
of money. That’s the point, after all. And it can’t be cheap to
keep the tens of thousands of “castmembers” (yes, they really do
call them that, even when they’re earning minimum wage and work
jobs with all the glamour of a bathroom attendant) hanging around,
picking up litter and confronting every new “guest” with eerily
convincing cheer.

As for “bored” and “disgusted”\dash{}not yet. Bored\dash{}it’s impossible to
imagine such a thing. For starters, the world’s middle classes have
converged here in a sort of bourgeois UN, and you can get a lot of
pleasure out of watching a Chinese “little emperor” with doting
parents in tow making friends with a tiny perfect Russian mafiyeh
princess whose parents flick nervously at their nicotine inhalers
and scout the building facades for hidden cameras.

Of course, if people-watching isn’t your thing, there are the rides
themselves, which make art out of the shoebox diorama. There are
luaus, indoor scuba diving with live sharks, and an island of
genuinely sleazy nightclubs where you can get propositioned for
some improbable acts that are hardly family friendly. These last
appear to be largely populated by the “castmembers” seeking a
little after-work action.

Disgusted? I think if I were a parent, there’d be parts of the
experience that drove me nuts. But once you get to know the rhythm
of the place, you start to see that there are navigable pathways
that don’t lead through any commercial areas\dash{}fantastic adventure
playgrounds, nature hikes, petting zoos, horseback rides, sports
training. And for every kid who’s having a blood-sugar meltdown
after consuming half a quart of high-fructose lube slathered on a
cinnamon bun, there’s another who is standing open-mouthed with
complete bodily wonder, at some stupendous spectacle, clearly
forming neuronal connections of a sort that will create the
permanent predisposition to an appreciation of spectacle, wonder,
and beauty.

This is the kind of place where you have to love the sin and hate
the sinner. The company may sue and resort to dirty tricks, but
it’s also chock full of real artists making real art.

If you haven’t been for a visit, you should. Honestly. Oh, by all
means, also go somewhere unspoiled (if you can find it). Go
camping. Go to one of the rides I’ve written so much about. But if
you want to see the bright side of what billions can do\dash{}the stuff
you never get from outside the walls of this fortress of fun\dash{}buy a
ticket.

\begin{center}\rule{3in}{0.4pt}\end{center}

The barman at Suzanne’s hotel started building her a Lapu-Lapu as
she came up the stairs. The drink involved a hollow pineapple,
overproof rum, and an umbrella, and she’d concluded that it
contained the perfect dosage of liquid CNS depressant to unwind her
after a day of battle at the parks. That day she’d spent following
around the troupes of role-playing actors at Disney’s Hollwood
Studios: a cast of a hundred costumed players who acted out a
series of interlocking comedies set in the black-and-white days of
Hollywood. They were fearlessly cheeky, grabbing audience members
and conscripting them in their plays.

Now she was footsore and there was still a nighttime at Epcot in
her future. The barman passed her the pineapple and she thumped her
lanyard against the bar twice\dash{}once to pay for the drink and once to
give him a generous tip. He was gay as a goose, but fun to look at,
and he flirted with her for kicks.

“Gentleman caller for you, Suzanne,” he said, tilting his head.
“You temptress.”

She looked in the direction indicated and took in the man sitting
on the bar-stool. He didn’t have the look of a harried dad and he
was too old to be a love-flushed honeymooner. In sensible
tropical-weight slacks and a western shirt, he was impossible to
place. He smiled and gave her a little wave.

“What?”

“He came in an hour ago and asked for you.”

She looked back at the man. “What’s your take on him?”

“I think he works here. He didn’t pay with an employee card, but he
acted like it.”

“OK,” she said, “send out a search party if I’m not back in an
hour.”

“Go get him, tiger,” the barman said, giving her hand a squeeze.

She carried her pineapple with her and drifted down the bar.

“Hello there,” she said.

“Ms Church,” the man said. He had a disarming, confident smile. “My
name is Sammy Page.”

She knew the name, of course. The face, too, now that she thought
about it. He offered her his hand. She didn’t take it. He put it
down, then wiped it on his trouser-leg.

“Are you having a good time?”

“A lovely time, thank you.” She sipped her drink and wished it was
a little more serious and intimidating. It’s hard to do frosty when
you’re holding a rum-filled pineapple with a paper parasol.

His smile faltered. “I read your article. I can’t believe I missed
it. I mean, you’ve been here for six days and I just figured it out
today? I’m a pretty incompetent villain.”

She let a little smile slip out at that. “Well, it’s a big
Internet.”

“But I \emph{love} your stuff. I’ve been reading it since, well,
back when I lived in the Valley. I used to get the Merc actually
delivered on paper.”

“You are a walking fossil, aren’t you?”

He bobbed his head. “So it comes down to this. I’ve been very
distracted with making things besides lawsuits lately, as you know.
I’ve been putting my energy into doing stuff, not preventing stuff.
It’s been refreshing.”

She grubbed in her pocket and came up with a little steno book and
a pencil. “Do you mind if I take notes?”

He gulped. “Can this all be on background?”

She hefted her notebook. “No,” she said finally. “If there’s
anything that needs publishing, I’m going to have to publish it. I
can respect the fact that you’re speaking to me with candor, but
frankly, Mr Page, you haven’t earned the privilege of speaking on
background.”

He sipped at his drink\dash{}a more grown-up highball, with a lone
ice-cube in it, maybe a Scotch and soda. “OK, right. Well, then, on
the record, but candorously. I loved your article. I love your work
in general. I’m really glad to have you here, because I think we
make great stuff and we’re making more of it than ever. Your latest
post was right on the money\dash{}we care about our work here. That’s how
we got to where we are.”

“But you devote a lot of your resources to other projects here,
don’t you? I’ve heard about you, Mr Page. I’ve interviewed Death
Waits.” He winced and she scribbled a note, leaving him on
tenterhooks while she wrote. Something cold and angry had hold of
her writing arm. “I’ve interviewed him and heard what he has to say
about this place, what you have done.”

“My hands aren’t the cleanest,” he said. “But I’m trying to atone.”
He swallowed. The barman was looking at them. “Look, can I take you
for a walk, maybe? Someplace more private?”

She thought about it. “Let me get changed,” she said. “Meet you in
the lobby in ten.”

She swapped her tennis shoes for walking sandals and put on a clean
shirt and long slacks, then draped a scarf over her shoulders like
a shawl. Outside, the sunset was painting the lagoon bloody. She
was just about to rush back down to the lobby when she stopped and
called Lester, her fingers moving of their own volition.

“Hey, you,” he said. “Still having fun in Mauschwitz?”

“It keeps getting weirder here, let me tell you,” she said. She
told him about Sammy showing up, wanting to talk with her.

“Ooh, I’m jealous,” Lester said. “He’s my arch-rival, after all.”

“I hadn’t thought of it that way. He \emph{is} kind of cute\dash{}”

“Hey!”

“In a slimy, sharky way. Don’t worry, Lester. I miss you, you
know?”

“Really?”

“Really. I think I’m about done here. I’m going to come home
soon.”

There was a long pause, then a snuffling sound. She realized he was
crying. He slurped. “Sorry. That’s great, babe. I missed you.”

“I\dash{}I missed you too. Listen, I’ve got to go meet this guy.”

“Go, go. Call me after dinner and tell me how it goes. Meanwhile,
I’m going to go violate the DiaB some more.”

“Channel it, that’s right.”

“Right on.”

Sammy met her in the lobby. “I thought we could go for a walk
around the lake,” he said. “There’s a trail that goes all the way
around. It’s pretty private.”

She looked at the lake. At twelve o’clock, the main gates of the
Magic Kingdom; at three, the retro A-frame Contemporary hotel, at
nine, the wedding-cake Grand Floridian Resort.

“Lead on,” she said. He led her out onto the artificial white-sand
beach and around, and a moment later they were on a pathway paved
with octagonal tiles, each engraved with the name of a family and a
year.

“I really liked your article.”

“You said that.”

They walked a while longer. “It reminded me of why I came here. I
worked for startups, and they were fun, but they were ephemeral. No
one expected something on the Web to last for half a century. Maybe
the brand survives, but who knows? I mean, who remembers Yahoo!
anymore? But for sure, anything you built then would be gone in a
year or two, a decade tops.

“But here\ldots{}” He waved his hands. They were coming around the bend
for the Contemporary now, and she could see it in all its absurd
glory. It had been kept up so that it looked like it might have
been erected yesterday, but the towering white A-frame structure
with the monorail running through its midriff was clearly of
another era. It was like a museum piece, or a bit of artillery on
the field at a civil war reenactment.

“I see.”

“It’s about the grandiosity, the permanence. The belief in doing
something\dash{}anything\dash{}that will endure.”

“You didn’t need to bring me someplace private to tell me that.”

“No, I didn’t.” He swallowed. “It’s hard because I want to tell you
something that will compromise me if I say it.”

“And I won’t let you off the hook by promising to keep it
confidential.”

“Exactly.”

“Well, you’re on the horns of a dilemma then, aren’t you?” The sun
was nearly set now, and stones at their feet glittered from
beneath, sprinkled with twinkling lights. It made the evening,
scented with tropical flowers and the clean smell of the lake, even
more lovely. A cool breeze fluffed her hair.

He groaned. She had to admit it, she was enjoying this. Was it any
less than this man deserved?

“Let me try this again. I have some information that, if I pass it
on to you, could save your friends down in Hollywood from terrible
harm. I can only give you this information on the condition that
you take great pains to keep me from being identified as the
source.”

They’d come to the Magic Kingdom now. Behind them, the main gates
loomed, and a pufferbelly choo-choo train blew its whistle as it
pulled out of the station. Happy, exhausted children ran across the
plaza, heading for the ferry docks and the monorail ramps. The
stones beneath her feet glittered with rainbow light, and tropical
birds called to each other from the Pirates of the Caribbean
Adventure Island in the middle of the lake.

“Hum,” she said. The families laughed and jostled each other. “Hum.
OK, one time only. This one is off the record.”

Sammy looked around nervously. “Keep walking,” he said. “Let’s get
past here and back into the private spots.”

\emph{But it’s the crowds that put me in a generous mood.} She
didn’t say it. She’d give him this one. What harm could it do? If
it was something she had to publish, she could get it from another
source.

“They’re going to sue your friends.”

“So what else is new?”

“No, personally. They’re going to the mattresses. Every trumped up
charge they can think of. But the point here isn’t to get the cops
to raid them, it’s to serve discovery on every single
communication, every document, every file. Open up everything. Root
through every email until they find something to hang them with.”

“You say ’they’\dash{}aren’t \emph{you} ’they’?”

It was too dark to see his face now, but she could tell the
question made him uncomfortable.

“No. Not anymore.” He swallowed and looked out at the lake. “Look,
I’m doing something now\dash{}something\ldots{} \emph{amazing}. The DiaB, it’s
breaking new ground. We’re putting three-d printers into every
house in America. What your friend Lester is doing, it’s actually
\emph{helping} us. We’re inventing a whole new\dash{}”

“Business?”

“No, not just a business. A world. It’s what the New Work was
missing\dash{}a three-d printer in every living room. A killer app. There
were personal computers and geeks for years before the spreadsheet
came along. Then there was a reason to put one in every house. Then
we got the Internet, the whole software industry. A new world.
That’s where we’re headed. It’s all I want to do. I don’t want to
spend the rest of my life suing people. I want to
\emph{do stuff}.”

He kicked at the rushes that grew beside the trail. “I want to be
remembered for that. I want \emph{that} to be my place in the
history books\dash{}not a bunch of lawsuits.”

Suzanne walked along beside him in silence for a time. “OK, so what
do you want me to do about it?”

“I thought that if\dash{}” He shut up. “Look, I tried this once before. I
told that Freddy bastard everything in the hopes that he’d come
onto my side and help me out. He screwed me. I’m not saying you’re
Freddy, but\dash{}”

Suzanne stopped walking. “What do you want from me, sir? You have
hardly been a friend to me and mine. It’s true that you’ve made
something very fine, but it’s also true that you helped sabotage
something every bit as fine. You’re painting yourself as the victim
of some mysterious ‘them.’ But as near as I can work out, the only
difference between you and ’them’ is that you’re having a little
disagreement with them. I don’t like to be used as part of your
corporate head-games and power-struggles.”

“Fine,” he said. “Fine. I deserve that. I deserve no better. Fine.
Well, I tried.”

Suzanne refused to soften. Grown men sulking did not inspire any
sympathy in her. Whatever he wanted to tell her, it wasn’t worth
going into his debt.

He gave a shuddering sigh. “Well, I’ve taken you away from your
evening of fun. Can I make it up to you? Would you like to come
with me on some of my favorite rides?”

This surprised her a little, but when she thought about it, she
couldn’t see why not. “Sure,” she said.

\begin{center}\rule{3in}{0.4pt}\end{center}

Taking a guest around Disney World was like programming a playlist
for a date or a car-trip. Sammy had done it three or four times for
people he was trying to win over (mostly women he was trying to
screw) and he refined his technique every time.

So he took her to the Carousel of Progress. It was the oldest
untouched ride in the park, a replica of the one that Walt himself
had built for GE at the 1964 World’s Fair. There had been attempts
to update it over the years, but they’d all been ripped out and the
show restored to its mid-sixties glory.

It was a revolving theater where robots danced and sang and talked
through the American Century, from the last days of the coal stove
up to the dawn of the space age. It had a goofy, catchy song,
cornball jokes, and he relished playing guide and telling his
charges about the time that the revolving theater had trapped a
careless castmember in its carousel and crushed her to death. That
juxtaposition of sunny, goofy American corporate optimism and the
macabre realities of operating a park where a gang of half-literate
minimum-wage workers spent their days shovelling the world’s rich
children into modified threshing machines\dash{}it was delicious.

Suzanne’s body language told him the whole story from the second
she sat down, arms folded, a barely contained smirk on her lips.
The lights played over the GE logo, which had acquired an even more
anachronistic luster since the last time he’d been. Now that GE had
been de-listed from the NYSE, it was only a matter of time before
they yanked the sponsorship, but for now, it made the ride seem
like it was part time-machine. Transported back to the corporate
Pleistocene, when giant dinocorps thundered over the plains.

The theater rotated to the first batch of singing, wise-cracking
robots. Her eyebrows shot up and she shook her head bemusedly. Out
came the second batch, the third\dash{}now they were in the fabulous
forties and the Andrews Sisters played while grandma and grandpa
robot watched a bulging fish-eye TV and sister got vibrated by an
electric slimming belt. The jokes got worse, the catchy
jingle\dash{}“There’s a great big beautiful tomorrow, shining at the end
of every daaaaay!”\dash{}got repeated with more vigor.

“It’s like an American robot performance of
\emph{Triumph of the Will},” she whispered to him, and he cracked
up. They were the only two in the theater. It was never full, and
he himself had taken part in spitball exercises brainstorming
replacements, but institutionally, Disney Parks couldn’t bring
itself to shut it down. There was always some excuse\dash{}rabid fans,
historical interest, competing priorities\dash{}but it came down to the
fact that no one wanted to bring the axe down on the robot family.

The final segment now, the whole family enjoying a futuristic
Christmas with a high-tech kitchen whose voice-activated stove went
haywire. All the robots were on stage for the segment, and they
exhorted the audience to sing and clap along. Sammy gave in and
clapped, and a second later, Suzanne did, too, laughing at the
silliness of it all. When the house lights came up and the
bored\dash{}but unsquashed\dash{}castmember spieled them out of the ride, Sammy
had a bounce in his step and the song in his head.

“That was \emph{terrible}!” Suzanne said.

“Isn’t it great?”

“God, I’ll never get that song out of my head.” They moved through
the flashing lights of Tomorrowland.

“Look at that\dash{}no line on Space Mountain,” Sammy said, pointing.

So they rode Space Mountain\dash{}twice. Then they caught the fireworks.
Then Sammy took her over to Tom Sawyer Island on a maintenance boat
and they sat up in the tree house and watched as the park heaved
and thronged, danced and ran, laughed and chattered.

“Hear the rustling?”

“Yeah, what is that, rabbits or something?”

“Giant rats.” Sammy grinned in the dark. “Giant, feral rats.”

“Come on, you’re joking.”

“Cross my heart. We drain the lake every now and then and they
migrate to the island. No predators. Lots of dropped french
fries\dash{}it’s ratopia here. They get as big as cats. Bold little
fuckers too. No one likes to be here alone at night.”

“What about us?”

“We’re together.”

The rustling grew louder and they held their breath. A bold rat
like a raccoon picked its way across the path below them. Then two
more. Suzanne shivered and Sammy did, too. They were huge, feral,
menacing.

“Want to go?”

“Hell \emph{yes},” she said. She fumbled in her purse and came out
with a bright little torch that shone like a beacon. You weren’t
supposed to use bright lights on the island after hours while the
rest of the park was open, but Sammy was glad of it.

Back on the mainland, they rode Big Thunder Mountain and moseyed
over to the new, half-rebuilt Fantasyland. The zombie maze was
still open, and they got lost in it amid the groans, animatronic
shamblers, and giggling kids running through the hedges.

Something happened in the maze. Between entering it and leaving it,
they lost their cares. Instead of talking about the park and
Hackelberg, they talked about ways of getting out of the maze,
talked about which zombie was coming next, about the best zombie
movies they’d ever seen, about memorable Halloweens. As they neared
the exit, they started to strategize about the best ride to go on
next. Suzanne had done the Haunted Mansion twice when she first
arrived and now\dash{}

“Come on, it’s such a cliche,” Sammy said. “Anyone can be a Haunted
Mansion fan. It’s like being a Mickey fan. It takes real character
to be a Goofy fan.”

“You’re a Goofy fan, I take it?”

“Indeed. And I’m also a Jungle Cruise man.”

“More corny jokes?”

“’We’ve been \emph{dying} to have you’\dash{}talk about cornball humor.”

They rode both. The park was closing, and all around them, people
were streaming away from the rides. No lines at all, not even in
front of the rollercoasters, not even in front of Dumbo, not even
in front of the ultra-violent fly-over of the world of the zombies
(nee Peter Pan’s Flight, and a perennial favorite).

“You know, I haven’t just \emph{enjoyed} the park like this in
years.” He was wearing a huge foam Goofy hat that danced and bobbed
on his head, trying to do little pas-de-deux with the other Goofy
hats in the vicinity. It also let out the occassional chuckle and
snatch of song.

“Shut up,” Suzanne said. “Don’t talk about magic. Live magic.”

They closed the park, letting themselves get herded off of Main
Street along with the last stragglers. He looked over his shoulder
as they moved through the arches under the train-station. The night
crew was moving through the empty Main Street, hosing down the
streets, sweeping, scrubbing. As he watched, the work lights came
on, throwing the whole thing into near-daylight illumination,
making it seem less like an enchanted wonderland and more like a
movie set, an artifice. A sham.

It was one in the morning and he was exhausted. And Hackelberg was
going to sue.

“Sammy, what do you want me to do, blackmail him?”

“I don’t know\dash{}sure. Why not? You could call him and say, ’I hear
you’re working on this lawsuit, but don’t you think it’s
hypocritical when you’ve been doing all this bad stuff\dash{}’”

“I don’t blackmail people.”

“Fine. Tell your friends, then. Tell some lawyers. That could
work.”

“Sammy, I think we’re going to have to fight this suit on its
merits, not on the basis of some sneaky intel. I appreciate the
risk you’re putting yourself to\dash{}”

“We ripped off some of Lester’s code for the DiaB.” He blurted it
out, not believing he was hearing himself say it. “I didn’t know it
at the time. The libraries were on the net and my guys were in a
hurry, and they just imported it into the build and left it
there\dash{}they rewrote it with the second shipment, but we put out a
million units running a library Lester wrote for volumetric
imaging. It was under some crazy viral open source license and we
were supposed to publish all our modifications, and we never did.”

Suzanne threw her head back and laughed, long and hard. Sammy found
himself laughing along with her.

“OK,” she said. “OK. That’s a good one. I’ll tell Lester about it.
Maybe he’ll want to use it. Maybe he’ll want to sue.”

Sammy wanted to ask her if she’d keep his name out of it, but he
couldn’t ask. He’d gone to Hackelberg with the info as soon as he’d
found out and they’d agreed to keep it quiet. The Imagineers
responsible had had a very firm talking to, and had privately
admitted to a curious and aghast Sammy over beers that everyone
everywhere did this all the time, that it was so normal as to be
completely unremarkable. He was pretty sure that a judge wouldn’t
see it that way.

Suzanne surprised him by giving him a strong, warm hug. “You’re not
the worst guy in the world, Sammy Page,” she said. “Thanks for
showing me around your park.”

\begin{center}\rule{3in}{0.4pt}\end{center}

Kettlewell had been almost pathetic in his interest in helping
Lester out. Lester got the impression that he’d been sitting around
his apartment, moping, ever since Eva had taken the kids and gone.
As Lester unspooled the story for him\dash{}Suzanne wouldn’t tell him how
she’d found this out, and he knew better than to ask\dash{}Kettlewell
grew more and more excited. By the time Lester was through, he was
practically slobbering into the phone.

“Oh, oh, oh, this is going to be a \emph{fun} phoner,” he said.

“You’ll do it, then? Even after everything?”

“Does Perry know you’ve called me?”

Lester swallowed. “No,” he said. “I don’t talk to Perry much these
days.”

Kettlewell sighed. “What the hell am I going to do with you two?”

“I’m sorry,” Lester said.

“Don’t be sorry. Be happy. Someone should be happy around here.”

\begin{center}\rule{3in}{0.4pt}\end{center}

Herve Guignol chaired the executive committee. Sammy had known him
for years. They’d come east together from San Jose, where Guignol
had run the entertainment side of eBay. They’d been recruited by
Disney Parks at the same time, during the hostile takeover and
breakup, and they’d had their share of nights out, golf games, and
stupid movies together.

But when Guignol was wearing his chairman’s hat, it was like he was
a different person. The boardroom was filled with huge, ergonomic
chairs, the center of the table lined with bottles of imported
water and trays of fanciful canapes in the shapes of Disney
characters. Sammy sat to Guignol’s left and Hackelberg sat to his
right.

Guignol brought the meeting to order and the rest of the committee
stopped chatting and checking email and looked expectant. At the
touch of a button, the door swung shut with an authoritative clunk
and shutters slid down over the window.

“Welcome, and thank you for attending on such short notice. You
know Augustus Hackelberg; he has something to present to you.”

Hackelberg climbed to his feet and looked out at them. He didn’t
look good.

“An issue has arisen\dash{}” Sammy loved the third person passive voice
that dominated corporate meetings. Like the issue had arisen all on
its own, spontaneously. “A decision that was taken has come back to
bite us.” He explained about the DiaBs and the code, laying it out
more or less as it happened, though of course he downplayed his
involvement in advising Sammy to go ahead and ship.

The committee asked a few intense questions, none directed at
Sammy, who kept quiet, though he instinctively wanted to defend his
record. They took a break after an hour, and Sammy found himself in
a corner with Guignol.

“What do you think?” Sammy asked him.

Guignol grimaced. “I think we’re pretty screwed. Someone is going
to have to take a fall for this, you know. It’s going to cost us a
fortune.”

Sammy nodded. “Well, unless we just settle with them,” he said.
“You know\dash{}we drop the suit we just filed and they drop theirs.\ldots{}”
He had hoped that this would come out on its own, but it was clear
that Hackelberg wasn’t going to offer it up himself. He was too in
love with the idea of getting his hands on Perry and Lester.

Guignol rocked his head from side to side. “You think they’d go for
it?”

Sammy dropped his voice to a whisper and turned away from the rest
of the room to confound any lip-readers. “I think they’ve
\emph{offered} to do that.”

Guignol cut his eyes over to Hackelberg and Sammy nodded,
imperceptibly.

Guignol moved away, leaving Sammy to eat a Mickey head built from
chunks of salmon and hamachi. Guignol moved among the committee,
talking to a few members. Sammy recognized the
behavior\dash{}consolidating power. Hard to remember that this was the
guy he’d played savage, high-stakes games of putt-putt golf with.

The meeting reconvened. No one looked at Sammy. They all looked at
Hackelberg.

“What about trying to settle the suit?” Guignol said.

Hackelberg flushed. “I don’t know if that’s possible\dash{}”

“What about if we offer to settle in exchange for dropping the suit
we’ve just filed?”

Hackelberg’s hands squeezed the side of the table. “I don’t think
that that would be a wise course of action. This is the opportunity
we’ve been waiting for\dash{}the chance to crack them wide open and see
what’s going on inside. Discover just what they’ve taken from us
and how. Out them for all their bad acts.”

Guignol nodded. “OK, that’s true. Now, as I understand it, every
DiaB we shipped with this Banks person’s code on it is a separate
act of infringement. We shipped a million of them. What’s the
potential liability per unit?”

“Courts usually award\dash{}”

Guignol knocked quietly on the table. “What’s the
\emph{potential liability}\dash{}what’s the size of the bill a court
\emph{could} hand down, if a jury was involved? If, say, this
became part of someone’s litigation portfolio.”

Hackelberg looked away. “It’s up to five hundred thousand per
separate act of infringement.”

Guignol nodded. “So, we’re looking at a ceiling on the liability at
\$500 billion, then?”

“Technically, yes. But\dash{}”

“I propose that we offer a settlement, quid-pro-quo with this Banks
person. We drop our suit if he indemnifies us from damages for
his.”

“Seconded,” said someone at the table. Things were picking up
steam. Sammy bit the inside of his cheek to keep his smile in
check.

“Wait,” Hackelberg said. “Gentlemen and lady, please. While it’s
true that damages can technically run to \$500,000 per
infringement, that simply isn’t done. Not to entities like this
firm. Listen, we \emph{wrote} that law so we could sue people who
took from \emph{us}. It won’t be used against us. We will face, at
worst, a few hundred dollars per act of infringement. Still a
sizable sum of money, but in the final analysis\dash{}”

“Thank you,” Guignol said. “All in favor of offering a
settlement?”

It was unanimous\dash{}except for Hackelberg.

\begin{center}\rule{3in}{0.4pt}\end{center}

Sammy got his rematch with Hackelberg when the quarterly financials
came out. It was all that black ink, making him giddy.

“I don’t want to be disrespectful,” he said, knowing that in
Hackelberg’s books, there could be nothing more disrespectful than
challenging him. “But we need to confront some business realities
here.”

Hackelberg’s office was nothing like Sammy had expected\dash{}not a
southern gentleman’s study lined with hunting trophies and framed
ancestral photos. It was as spare as the office of a temp, almost
empty save for a highly functional desk, built-in bookcases lined
with law-books, and a straight-backed chair. It was ascetic,
severe, and it was more intimidating than any dark-wood den could
hope to be.

Hackelberg’s heavy eyelids drooped a little, the corners of his
eyes going down with them. It was like staring down a gator. Sammy
resisted the urge to look away.

“The numbers don’t lie. DiaB is making us a fortune, and most of
it’s coming from the platform, not the goop and not the increased
visitor numbers. We’re making money because other people are
figuring out ways to use our stuff. It’s our fastest-growing
revenue source and if it continues, we’re going to end up being a
DiaB company with a side-business in theme-parks.

“That’s the good news. The bad news is that these characters in the
ghost mall have us in their crosshairs. They’re prying us open
faster than we can lock ourselves down. But here’s another way of
looking at it: every time they add another feature to the DiaB,
they make owning a DiaB more attractive, which makes it easier for
us to sell access to the platform to advertisers.”

Hackelberg held up his hands. “Samuel, I think I’ve heard enough.
Your job is to figure out new businesses for us to diversify into.
My job is to contain our liability and protect our brand and
investors. It sounds a lot to me like you’re saying that you want
me to leave off doing my job so that you can do yours.”

Sammy squirmed. “No, that’s not it at all. We both want to protect
the business. I’m not saying that you need to give these guys a
free ride. What I’m saying is, suing these guys is \emph{not} good
for our business. It costs us money, goodwill\dash{}it distracts us from
doing our jobs.”

Hackelberg leaned back and looked coolly into Sammy’s eyes. “What
are you proposing as an alternative, then?”

The idea had come to Sammy in the shower one morning, as he
mentally calculated the size of his coming quarterly bonus. A great
idea. Out of the box thinking. The right answer to the question
that no one had thought to ask. It had seemed so \emph{perfect}
then. Now, though\dash{}

“I think we should buy them out.”

Hackelberg’s thin, mirthless grin made his balls shrivel up.

Sammy held up his hands. “Here, look at this. I drew up some
figures. What they’re earning. What we earn from them. Growth
estimates over the next five quarters. It’s not just some random
idea I had in the shower. This makes \emph{sense}.” He passed over
a sheaf of papers, replete with pie-charts.

Hackelberg set it down in the center of his desk, perfectly square
to the corners. He flipped through the first five pages, then
squared the stack up again.

“You’ve done a lot of work here, Samuel. I can really see that.”

He got up from his straight-backed chair, lifted Sammy’s papers
between his thumb and forefinger, and crossed to the wall. There
was a shredder there, its maw a wide rectangle, the kind of thing
that you can stick entire hardcover books (or hard drives) into.
Calmly, Hackelberg fed Sammy’s paper into the shredder,
fastidiously holding the paper-clipped corner between thumb and
forefinger, then dropping the corner in once the rest had been
digested.

“I won’t ask you for your computer,” he said, settling back into
his chair. “But I expect that you will back up your other data and
then send the hard-drive to IT to be permanently erased. I don’t
want any record of this, period. I want this done by the end of
business today.”

Sammy’s mouth hung open. He shut it. Then he opened it again.

Abruptly, Hackelberg stood, knocking his chair to the ground behind
him.

“Not one word, do you understand me? Not one solitary word, you
goddamned idiot! We’re in the middle of being sued by these people.
I \emph{know} you know this, since it’s your fault that it’s
happening. I know that you know that the stakes are the
\emph{entire} company. Now, say a jury were to discover that we
were considering buying these assholes out? Say a jury were to
decide that our litigation was a base stratagem to lower the asking
price for their, their \emph{company}\dash{}” The word dripped with
sarcasm\dash{}“what do you suppose would happen? If you had the sense of
a five year old, you’d have known better than to do this. Good
Christ, Page, I should have security escort you to the gate.

“Turn on your heel and go weep in the corridor. Don’t stand in my
office for one more second. Get your computer to IT by 2PM. I will
check. That goes for anyone you worked with on this, anyone who has
a copy of this information. Now, leave.” Sammy stood rooted in
place. “LEAVE, you ridiculous little dog’s-pizzle, get out of my
sight!”

Sammy drew in a deep breath. He thought about saying something
like, \emph{You can’t talk to me like that}, but it was very likely
that Hackelberg could talk to him just like that. He felt
light-headed and a little sick, and he backed slowly out of the
office.

Standing in the corridor, he began to shake. He pounded the
elevator button, and felt the eyes of Hackelberg’s severe secretary
burning into his back. Abruptly, he turned away and yanked open the
staircase door so hard it smashed into the wall with a loud bang.
He took the stairs in a rush of desperate claustrophobia, wanting
more than anything to get outside, to breathe in the fresh air.

He stumbled on the way down, falling a couple of steps and smashing
into the wall on the landing. He stood, pressed against the wall,
the cold cinder block on his cheek, which felt like it might be
bruised. The pain was enough to bring him back to his senses.

This is ridiculous. He had the right answer. Hackelberg was wrong.
Hackelberg didn’t run the company. Yes, it was hard to get anything
done without his sign-off, but it wasn’t impossible. Going behind
Hackelberg’s back to the executive committee could cost him his
job, of course.

Of course.

Sammy realized that he didn’t actually \emph{care} if he lost his
job. Oh, the thought made his chest constrict and thoughts of
living in a refrigerator box materialize in his mind’s eye, but
beyond that, he really didn’t care. It was such a goddamned
roller-coaster ride\dash{}Sammy smiled grimly at the metaphor. You guess
right, you end up on top. You guess wrong, you bottom out. He spent
half his career lording it over the poor guessers and the other
half panicking about a bad guess he’d made. He thought of Perry and
Lester, thought of that night in Boston. He’d killed their ride and
the party had gone on all the same. They had something, in that
crazy shantytown, something pure and happy, some camaraderie that
he’d always assumed he’d get someday, but that had never
materialized.

If this was his dream job, how much worse would unemployment really
be?

He would go to the executive committee. He would not erase his
numbers. He set off for his office, moving quickly, purposefully,
head up. A last stand, how exciting, why not?

He piloted the little golf-cart down the back road and was nearly
at his building’s door when he spotted the security detail. Three
of them, in lightweight Disney cop uniforms, wearing ranger hats
and looking around alertly. Hackelberg must have sent them there to
make sure that he followed through with deleting his data.

He stopped the golf cart abruptly and reversed out of the driveway
before the guards spotted him. He needed to get his files somewhere
that Hackelberg wouldn’t be able to retrieve them. He zipped down
the service roads, thinking furiously.

The answer occurred to him in the form of a road-sign for the
Polynesian hotel. He turned up its drive and parked the golf-cart.
As he stepped out, he removed his employee badge and untucked his
shirt. Now he was just another sweaty fresh-arrived tourist, Dad
coming in to rendezvous with Mom and the kids, back from some banal
meeting that delayed his arrival, hasn’t even had time to change
into a t-shirt.

He headed straight for the sundries store and bought a postage-paid
Walt Disney World postcard with a little magnetic patch mounted on
one corner. You filled up the memory with a couple hours’ worth of
video and as many photos as you wanted and mailed it off. The
pixelated display on the front played a slide show of the images\dash{}at
least once a year, some honeymoon couple would miss this fact and
throw a couple racy bedroom shots in the mix, to the perennial
delight of the mail room.

He hastily wrote some banalities about the great time he and the
kids were having in Disney World, then he opened his computer and
looked up the address that the Church woman had checked in under.
He addressed it, simply, to “Suzanne,” to further throw off the
scent, then he slipped it into a mail-slot with a prayer to the
gods of journalist shield laws.

He walked as calmly as he could back to his golf-cart, clipping on
his employee badge and tucking his shirt back in. Then he motored
calmly to his office building. The Disney cops were sweating under
the mid-day sun.

“Mr Page?”

“Yes,” he said.

“I’m to take your computer to IT, sir.”

“I don’t think so,” Sammy said, with perfect calm. “I think we’ll
GO up to my office and call a meeting of the executive committee
instead.”

The security guard was young, Latino, and skinny. His short
back-and-sides left his scalp exposed to the sun. He took his hat
off and mopped his forehead with a handkerchief, exposing a line of
acne where his hat-band irritated the skin. It made Sammy feel
sorry for the kid\dash{}especially considering that Sammy earned more
than 20 times the kid’s salary.

“This really isn’t your job, I know,” Sammy said, wondering where
all this sympathy for the laboring classes had come from, anyway?
“I don’t want to make it hard for you. We’ll go inside. You can
hang on to the computer. We’ll talk to some people. If they tell
you to go ahead, you go ahead. Otherwise, we go see them, all
right?”

He held his computer out to the kid, who took it.

“Let’s go up to my office now,” he said.

The kid shook his head. “I’m supposed to take this\dash{}”

“I know, I know. But we have a deal.” The kid looked like he would
head out anyway. “And there are backups in my office, so you need
to come and get those, too.”

That did it. The kid looked a little grateful as they went inside,
where the air conditioning was blowing icy cold.

“You should have waited in the lobby, Luis,” Sammy said, reading
the kid’s name off his badge. “You must be boiled.”

“I had instructions,” Luis said.

Sammy made a face. “They don’t sound like very reasonable
instructions. All the more reason to sort this out, right?”

Sammy had his secretary get Luis a bottle of cold water and a
little plate of grapes and berries out of the stash he kept for his
visitors, then he called Guignol from his desk phone.

“It’s Sammy. I need to call an emergency meeting of the exec
committee,” he said without preamble.

“This is about Hackelberg, isn’t it?”

“He’s already called you?”

“He was very persuasive.”

“I can be persuasive, too. Give me a chance.”

“You know what will happen if you push this?”

“I might save the company.”

“You might,” Guignol said. “And you might\dash{}”

“I know,” Sammy said. “What the hell, it’s only a career.”

“You can’t keep your data\dash{}Hackelberg is right about that.”

“I can send all the backups and my computer to your office right
now.”

“I was under the impression that they were all on their way to IT
for disposal.”

“Not yet. There’s a security castmember in my office with me named
Luis. If you want to call dispatch and have them direct him to
bring this stuff to you instead\dash{}”

“Sammy, do you understand what you’re doing here?”

Sammy suppressed a mad giggle. “I do,” he said. “I understand
exactly what I’m doing. I want to help you all understand that,
too.”

“I’m calling security dispatch now.”

A moment later, Luis’s phone rang and the kid listened intently,
nodding unconsciously. Once he’d hung up, Sammy passed him his
backups, hardcopy and computer. “Let’s go,” he said.

“Right,” Luis said, and led the way.

It was a short ride to the casting office building, where Guignol
had his office. The wind felt terrific on his face, drying his
sweat. It had been a long day.

When they pulled up, Sammy let Luis lead the way again, badging in
behind him, following him up to the seventh-floor board-room. at
the end of the Gold Coast where the most senior offices were.

Guignol met them at the door and took the materials from Luis, then
ushered Sammy in. Sammy caught Luis’s eye, and Luis surprised him
by winking and slipping him a surreptitious thumbs-up, making Sammy
feel like they shared a secret.

There were eight on the executive committee, but they travelled a
lot. Sammy had expected to see no more than four. There were two.
And Hackelberg, of course. The lawyer was the picture of saurian
calm.

Sammy sat down at the table and helped himself to a glass of water,
watching a ring pool on the table’s polished and waxed wooden
surface.

“Samuel,” Hackelberg said, shaking his head. “I hoped it wouldn’t
come to this.”

Sammy took a deep breath, looking for that don’t-give-a-shit calm
that had suffused him before. It was there still, not as potent,
but there. He drew upon it.

“Let’s put this to the committee, shall we? I mean, we already know
how we feel.”

“That won’t be necessary,” Hackelberg said. “The committee has
already voted on this.”

Sammy closed his eyes and rubbed the bridge of his nose. He looked
at Hackelberg, who was smiling grimly, a mean grin that went all
the way to the corners of his eyes.

Sammy looked around at Guignol and the committee members. They
wouldn’t meet his eye. Guignol gestured Luis into the room and
handed him Sammy’s computer, papers, and backups. He leaned in and
spoke quietly to him. Luis turned and left.

Guignol cleared his throat. “There’s nothing else to discuss,
then,” he said. “Thank you all for coming.”

In his heart, Sammy had known this was coming. Hackelberg would
beat him to the committee\dash{}never let him present his side. Watching
the lawyer get up stiffly and leave with slow, dignified steps,
Sammy had a moment’s intuition about what it must be like to be
that man\dash{}possessed of a kind of cold, furious power that came from
telling everyone that not obeying you to the letter would put them
in terrible danger. He knew that line of reasoning: It was the same
one he got from the TSA at the airport before they bent him over
and greased him up.
\emph{You can’t understand the grave danger we all face. You must obey me, for only I can keep it at bay.}

He waited for the rest of the committee to file out. None of them
would meet his eye. Then it was just him and Guignol. Sammy raised
his eyebrows and spread out his hands, miming
\emph{What happens now?}

“You won’t be able to get anything productive done until IT gets
through with your computer. Take some time off. Call up Dinah and
see if she wants to grab some holiday time.”

“We split,” Sammy said. He drank his water and stood up. “I’ve just
got one question before I go.”

Guignol winced but stood his ground. “Go ahead,” he said.

“Don’t you want to know what the numbers looked like?”

“It’s not my job to overrule legal\dash{}”

“We’ll get to that in a second. It’s not the question. The question
is, don’t you \emph{want to know}?”

Guignol sighed. “You know I want to know. Of course I want to know.
This isn’t about me and what I want, though. It’s about making sure
we don’t endanger the shareholders\dash{}”

“So ignoring this path, sticking our heads in the sand, that’s
\emph{good} for the shareholders?”

“No, of course it’s not good for the shareholders. But it’s better
than endangering the whole company\dash{}”

Sammy nodded. “Well, how about if we both take some time off and
drive down to Hollywood. It’d do us some good.”

“Sammy, I’ve got a job to do\dash{}”

“Yeah, but without your computer\ldots{}”

Guignol looked at him. “What did you do?”

“It’s not what I did. It’s what I might have done. I’m going to be
a good boy and give Hackelberg a list of everyone I might have
emailed about this. All those people are losing their computers to
the big magnet at IT.”

“But you never emailed me about this\dash{}”

“You sure? I might have. It’s the kind of thing I might have done.
Maybe your spam-filter ate it. You never know. That’s what IT’s
for.”

Guignol looked angry for a moment, then laughed. “You are such a
shithead. Fuck that lawyer asshole anyway. What are you driving
these days?”

“Just bought a new Dell Luminux,” Sammy said, grinning back.
“Rag-top.”

“When do we leave?”

“I’ll pick you up at 6AM tomorrow. Beat the morning traffic.”

\begin{center}\rule{3in}{0.4pt}\end{center}

Suzanne was getting sick of breakfast in bed. It was hard to
imagine that such a thing was possible, but there it was. Lester
stole out from between the covers before 7AM every day, and then,
half an hour later, he was back with a laden tray, something new
every day. She’d had steaks, burritos, waffles, home-made granola,
fruit-salad with Greek yogurt, and today there were eggs Benedict
with fresh-squeezed grapefruit juice. The tray always came with a
French press of fresh-ground Kona coffee, a cloth napkin, and her
computer, so she could read the news.

In theory, this was a warm ritual that ensured that they had
quality time together every day, no matter what. In practice,
Lester was so anxious about the food and whether she was enjoying
it that she couldn’t really enjoy it. Plus, she wasn’t a fatkins,
so three thousand calorie breakfasts weren’t good for her.

Most of all, it was the pressure to be a happy couple, to have
cemented over the old hurts and started anew. She felt it every
moment, when Lester climbed into the shower with her and soaped her
back, when he brought home flowers, and when he climbed into bed
with her in the morning to eat breakfast with her.

She picked at her caviar and blini glumly and poked at her
computer. Beside her, Lester hoovered up three thousand calories’
worth of fried dough and clattered one-handed on his machine.

“This is delicious, babe, thanks,” she said, with as much sincerity
as she could muster. It was really generous and nice of him to do
this. She was just a bitter old woman who couldn’t be happy no
matter what was going on in her life.

There was voicemail on her computer, which was unusual. Most people
sent her email. This originated from a pay phone on the Florida
Turnpike.

“Ms Church, this is\dash{}ah, this is a person whom you recently had the
acquaintance of, while on your holidays. I have a confidential
matter to discuss with you. I’m travelling to your location with a
colleague today and should arrive mid-morning. I hope you can make
some time to meet with me.”

She listened to it twice. Lester leaned over.

“What’s that all about?”

“You’re not going to believe it. I think it’s that Disney guy, the
guy I told you about. The one Death used to work for.”

“He’s coming \emph{here}?”

“Apparently.”

“Woah. Don’t tell Perry.”

“You think?”

“He’d tear that guy’s throat out with his teeth.” Lester took a
bite of blini. “I might help.”

Suzanne thought about Sammy. He hadn’t been the sort of person she
could be friends with, but she’d known plenty of his kind in her
day, and he was hardly the worst of the lot. He barely rated above
average on the corporate psychopath meter. Somewhere in there,
there was a real personality. She’d seen it.

“Well, then I guess I’d better meet with him alone.”

“It sounds like he wants a doctor-patient meeting anyway.”

“Or confessor-penitent.”

“You think he’ll leak you something.”

“That’s a pretty good working theory when it comes to this kind of
call.”

Lester ate thoughtfully, then reached over and hit a key on her
computer, replaying the call.

“He sounds, what, giddy?”

“That’s right, he does, doesn’t he. Maybe it’s good news.”

Lester laughed and took away her dishes, and when he came back in,
he was naked, stripped and ready for the shower. He was a very
handsome man, and he had a devilish grin as he whisked the blanket
off of her.

He stopped at the foot of the bed and stared at her, his grin
quirking in a way she recognized instantly. She didn’t have to look
down to know that he was getting hard. In the mirror of his eyes,
she was beautiful. She could see it plainly. When she looked into
the real mirror at the foot of the bed, draped with gauzy
sun-scarves and crusted around the edges with kitschy tourist
magnets Lester brought home, she saw a saggy, middle-aged woman
with cottage-cheese cellulite and saddle-bags.

Lester had slept with more fatkins girls than she could count,
women made into doll-like mannequins by surgery and chemical
enhancements, women who read sex manuals in public places and
boasted about their Kegel weight-lifting scores.

But when he looked at her like that, she knew that she was the most
beautiful woman he’d ever loved, that he would do anything for her.
That he loved her as much as he could ever love anyone.

\emph{What the hell was I complaining about?} she thought as he
fell on her like a starving man.

\begin{center}\rule{3in}{0.4pt}\end{center}

She met Sammy in their favorite tea-room, the one perched up on a
crow’s nest four storeys up a corkscrew building whose supplies
came up on a series of dumbwaiters and winches that shrouded its
balconies like vines.

She staked out the best table, the one with the panoramic view of
the whole shantytown, and ordered a plate of the tiny shortbread
cakes that were the house specialty, along with a gigantic mug of
nonfat decaf cappuccino.

Sammy came up the steps red-faced and sweaty, wearing a Hawai’ian
shirt and Bermuda shorts, like some kind of tourist. Or like he was
on holidays? Behind him came a younger man, with severe little
designer glasses, dressed in the conventional polo-shirt and slacks
uniform of the corporate exec on a non-suit day.

Suzanne sprinkled an ironic wave at them and gestured to the
mismatched school-room chairs at her table. The
waitress\dash{}Shayna\dash{}came over with two glasses of water and a paper
napkin dispenser. The men thanked her and mopped their faces and
drank their water.

“Good drive?”

Sammy nodded. His friend looked nervous, like he was wondering what
might have been swimming in his water glass. “This is some place.”

“We like it here.”

“Is there, you know, a bathroom?” the companion asked.

“Through there.” Suzanne pointed.

“How do you deal with the sewage around here?”

“Sewage? Mr Page, sewage is \emph{solved}. We feed it into our
generators and the waste heat runs our condenser purifiers. There
was talk of building one big one for the whole town, but that
required way too much coordination and anyway, Perry was convinced
that having central points of failure would be begging for a
disaster. I wrote a series on it. If you’d like I can send you the
links.”

The Disney exec made some noises and ate some shortbread, peered at
the chalk-board menu and ordered some Thai iced tea.

“Look, Ms Church\dash{}Suzanne\dash{}thank you for seeing me. I would have
understood completely if you’d told me to go fuck myself.”

Suzanne smiled and made a go-on gesture.

“Before my friend comes back from the bathroom, before we meet up
with anyone from your side, I just want you to know this. What
you’ve done, it’s changed the world. I wouldn’t be here today if it
wasn’t for you.”

He had every appearance of being completely sincere. He was a
little road-crazed and windblown today, not like she remembered him
from Orlando. What the hell had happened to him? What was he here
for?

His friend came back and Sammy said, “I ordered you a Thai iced
tea. This is Suzanne Church, the writer. Ms Church, this is Herve
Guignol, co-director of the Florida regional division of Disney
Parks.”

Guignol was more put-together and stand-offish than Sammy. He shook
her hand and made executive sounding grunts at her. He was young,
and clearly into playing the role of exec. He reminded Suzanne of
fresh Silicon Valley millionaires who could go from pizza-slinging
hackers to suit-wearing biz-droids who bullshitted knowledgeably
about EBITDA overnight.

\emph{What the hell are you two here for?}

“Mr Page\dash{}”

“Sammy, call me Sammy, please. Did you get my postcard?”

“That was from you?” She’d not been able to make heads or tails of
it when it arrived in the mail the day before and she’d chucked it
out as part of some viral marketing campaign she didn’t want to get
infected by.

“You got it?”

“I threw it out.”

Sammy went slightly green.

“But it’ll still be in the trash,” she said. “Lester never takes it
out, and I haven’t.”

“Um, can we go and get it now, all the same?”

“What’s on it?”

Sammy and Guignol exchanged a long look. “Let’s pretend that I gave
you a long run-up to this. Let’s pretend that we spent a lot of
time with me impressing on you that this is confidential, and not
for publication. Let’s pretend that I charmed you and made sure you
understood how much respect I have for you and your friends here\dash{}”

“I get it,” Suzanne said, trying not to laugh.
\emph{Not for publication}\dash{}really!

“OK, let’s pretend all that. Now I’ll tell you: what’s on that
postcard is the financials for a Disney Parks buyout of your
friends’ entire operation here. DiaBolical, the ride, all of it.”

Suzanne had been expecting a lot of things, but this wasn’t one of
them. It was loopy. Daffy. Not just weird, but inconceivable. As
though he’d said, “I sent you our plans to carve your portrait on
the moon’s surface with a green laser.” But she was a pro. She kept
her face still and neutral, and calmly swallowed her cappuccino.

“I see.”

“And there are\dash{}there are people at Disney who feel like this idea
is so dangerous that it doesn’t even warrant discussion. That it
should be suppressed.”

Guignol cleared his throat. “That’s the consensus,” he said.

“And normally, I’d say, hey, sure, the consensus. That’s great. But
I’ll tell you, I drew up these numbers because I was curious, I’m a
curious guy. I like to think laterally, try stuff that might seem
silly at first. See where it goes. I’ve had pretty good
instincts.”

Guignol and Suzanne snorted at the same time.

“And an imperfect record,” Sammy said. Suzanne didn’t want to like
him, but there was something forthright about him that she couldn’t
help warming to. There was no subtlety or scheming in this guy.
Whatever he wanted, you could see it right on his face. Maybe he
was a psycho, but he wasn’t a sneak.

“So I ran these numbers for my own amusement, to see what they
would look like. Assume that your boys want, say, 30 times gross
annual revenue for a buyout. Say that this settles our lawsuit\dash{}not
theirs, just ours, so we don’t have to pay for the trademark suit
to go forward. Assume that they generate one DiaBolical-scale idea
every six months\dash{}” Suzanne found herself nodding along, especially
at this last one. “Well, you make those assumptions and you know
what comes out of it?”

Suzanne let the numbers dance behind her own eyelids. She’d
followed all the relevant financials closely for years, so closely
that they were as familiar as her monthly take-home and mortgage
payments had been, back when she had a straight job and a straight
life.

“Well, you’d make Lester and Perry \emph{very} wealthy,” she said.
“After they vested out, they’d be able to live off the interest
alone.”

Sammy nodded judiciously. His sidekick looked alarmed. “Yup. And
for us?”

“Well, assuming your last quarterly statement was accurate\dash{}”

“We were a little conservative,” Sammy said. The other man nodded
reflexively.

\emph{You were very conservative,} she thought.
\emph{DiaB’s making you a fortune and you didn’t want to advertise that to the competition.}

“Assuming that, well, you guys earn back your investment in, what,
18 months?”

“I figure a year. But 18 months would be good.”

“If you vest the guys out over three years, that means\dash{}”

“100 percent ROI, plus or minus 200 percent,” Sammy said. “For less
money than we’ll end up spending on our end of the lawsuit.”

Guignol was goggling at them both. Sammy drank his Thai iced-tea,
slurping noisily. He signalled for another one.

“And you sent me these financials on a postcard?”

“There was some question about whether they’d be erased before I
could show them to anyone, and I knew there was no way I’d be given
the chance to re-create them independently. It seemed prudent to
have a backup copy.”

“A backup copy in my hands?”

“Well, at least I knew you wouldn’t give it up without a fight.”
Sammy shrugged and offered her a sunny smile.

“We’d better go rescue that postcard from the basket before Lester
develops a domestic instinct and takes out the trash, then,”
Suzanne said, pushing away from the table. Shayna brought the bill
and Sammy paid it, overtipping by a factor of ten, which endeared
him further to Suzanne. She couldn’t abide rich people who stiffed
on the tip.

Suzanne walked them through the shantytown, watching their
reactions closely. She liked to take new people here. She’d
witnessed its birth and growth, then gone away during its
adolescence, and now she got to enjoy its maturity. Crowds of kids
ran screeching and playing through the streets, adults nodded at
them from their windows, wires and plumbing and antennas crowded
the skies above them. The walls shimmered with murals and graffiti
and mosaics.

Sammy treated it like he had his theme park, seeming to take in
every detail with a connoisseur’s eye; Guignol was more nervous,
clearly feeling unsafe amid the cheerful lawlessness. They came
upon Francis and a gang of his kids, building bicycles out of
stiffened fabric and strong monofilament recycled from packing
crates.

“Ms Church,” Francis said gravely. He’d given up drinking, maybe
for good, and he was clear-eyed and charming in his engineer’s
coveralls. The kids\dash{}boys \emph{and} girls, Suzanne noted
approvingly\dash{}continued to work over the bikes, but they were clearly
watching what Francis was up to.

“Francis, please meet Sammy and his colleague, Herve. They’re here
for a story I’m working on. Gentlemen, Francis is the closest thing
we have to a mayor around here.”

Francis shook hands all around, but Sammy’s attention was riveted
on the bicycles.

Francis picked one up with two fingers and handed it to him. “Like
it? We got the design from a shop in Liberia, but we made our own
local improvements. The trick is getting the stiffener to stay
liquid long enough to get the fabric stretched out in the right
proportion.”

Sammy took the frame from him and spun it in one hand like a baton.
“And the wheels?”

“Mostly we do solids, which stay in true longer. We use the carbon
stiffener on a pre-cut round of canvas or denim, then fit a
standard tire. They go out of true after a while. You just apply
some solvent to them and they go soft again and you re-true them
with a compass and a pair of tailor’s shears, then re-stiffen them.
You get maybe five years of hard riding out of a wheel that way.”

Sammy’s eyes were round as saucers. He took one of the proffered
wheels and spun it between opposing fingertips. Then, grinning, he
picked up another wheel and the bike-frame and began to
\emph{juggle} them, one-two-three, hoop-la! Francis looked amused,
rather than pissed\dash{}giving up drink had softened his temper. His
kids stopped working and laughed. Sammy laughed too. He transferred
the wheels to his left hand, then tossed the frame high the air,
spun around and caught it and then handed it all back to Francis.
The kids clapped and he took a bow.

“I didn’t know you had it in you,” Guignol said, patting him on the
shoulder.

Sammy, sweating and grinning like a fool, said, “Yeah, it’s not
something I get a lot of chances to do around the office. But did
you see that? It was light enough to juggle! I mean, how exciting
is all this?” He swept his arm around his head. “Between the sewage
and the manufacturing and all these kids\dash{}” He broke off. “What do
you do about education, Suzanne?”

“Lots of kids bus into the local schools, or ride. But lots more
home-school these days. We don’t get a very high caliber of public
school around here.”

“Might that have something to do with all the residents who don’t
pay property tax?” Guignol said pointedly.

Suzanne nodded. “I’m sure it does,” she said. “But it has more to
do with the overall quality of public education in this state. 47th
in the nation for funding.”

They were at her and Lester’s place now. She led them through the
front door and picked up the trash-can next to the little table
where she sorted the mail after picking it up from her PO box at a
little strip mall down the road.

There was the postcard. She handed it silently to Sammy, who held
it for a moment, then reluctantly passed it to Guignol. “You’d
better hang on to it,” he said, and she sensed that there was
something bigger going on there.

“Now we go see Lester,” Suzanne said.

He was behind the building in his little workshop, hacking
DiaBolical. There were five different DiaBs running around him,
chugging and humming. The smell of goop and fuser and heat filled
the room, and an air-conditioner like a jet-engine labored to keep
things cool. Still, it was a few degrees warmer inside than out.

“Lester,” Suzanne shouted over the air-conditioner din, “we have
visitors.”

Lester straightened up from his keyboard and wiped his palms and
turned to face them. He knew who they were based on his earlier
conversation with Suzanne, but he also clearly recognized Sammy.

“You!” he said. “You work for Disney?”

Sammy blushed and looked away.

Lester turned to Suzanne. “This guy used to come up, what, twice,
three times a week.”

Sammy nodded and mumbled something. Lester reached out and snapped
off the AC, filling the room with eerie silence and stifling heat.
“What was that?”

“I’m a great believer in competitive intelligence.”

“You work for Disney?”

“They both work for Disney, Lester,” Suzanne said. “This is Sammy
and Herve.” \emph{Herve doesn’t do much talking,} she mentally
added, \emph{but he seems to be in charge}.

“That’s right,” Sammy said, seeming to come to himself at last.
“And it’s an honor to formally meet you at last. I run the DiaB
program. I see you’re a fan. I’ve read quite a bit about you, of
course, thanks to Ms Church here.”

Lester’s hands closed and opened, closed and opened. “You were,
what, you were sneaking around here?”

“Have I mentioned that I’m a great fan of \emph{your work}? Not
just the ride, either. This DiaBolical, well, it’s\dash{}”

“What are you doing here?”

Suzanne had expected something like this. Lester wasn’t like Perry,
he wouldn’t go off the deep-end with this guy, but he wasn’t going
to be his best buddy, either. Still, someone needed to intervene
before this melted down altogether.

“Lester,” she said, putting her hand on his warm shoulder. “Do you
want to show these guys what you’re working on?”

He blew air through his nose a couple times, then settled down. He
even smiled.

“This one,” he said, pointing to a DiaBolical, “I’ve got it running
an experimental firmware that lets it print out hollow components.
They’re a lot lighter and they don’t last as long. But they’re also
way less consumptive on goop. You get about ten times as much
printing out of them.”

Suzanne noted that this bit of news turned both of the Disney execs
a little green. They made a lot of money selling goop, she knew.

“This one,” Lester continued, patting a DiaB that was open to the
elements, its imps lounging in its guts, “we mix some serious epoxy
in with it, some carbon fibers. The printouts are practically
indestructible. There are some kids around here who’ve been using
it to print parts for bicycles\dash{}”

“Those were printed on \emph{this}?” Sammy said.

“We ran into Francis and his gang,” Suzanne explained.

Lester nodded. “Yeah, it’s not perfect, though. The epoxy clogs up
the works and the imps really don’t like it. I only get two or
three days out of a printer after I convert it. I’m working on
changing the mix to fix that, though.”

“After all,” Guignol noted sourly, “it’s not as if you have to pay
for new DiaBs when you break one.”

Lester smiled nastily at him. “Exactly,” he said. “We’ve got a
great research subsidy around here.”

Guignol looked away, lips pursed.

“This one,” Lester said, choosing not to notice, “this one is the
realization of an age-old project.” He pointed to the table next to
it, where its imps were carefully fitting together some very fine
parts.

Sammy leaned in close, inspecting their work. After a second, he
hissed like a teakettle, then slapped his knee.

Now Lester’s smile was more genuine. He loved it when people
appreciated his work. “You figured it out?”

“You’re printing DiaBs!”

“Not the whole thing,” Lester said. “A lot of the logic needs an
FPGA burner. And we can’t do some of the conductive elements,
either. But yeah, about 90 percent of the DiaB can be printed in a
DiaB.”

Suzanne hadn’t heard about this one, though she remembered earlier
attempts, back in the golden New Work days, the dream of
self-replicating machines. Now she looked close, leaning in next to
Sammy, so close she could feel his warm breath. There was
something, well, \emph{spooky} about the imps building a machine
using another one of the machines.

“It’s, what, it’s like it’s alive, and reproducing itself,” Sammy
said.

“Don’t tell me this never occurred to you,” Lester said.

“Honestly? No. It never did. Mr Banks, you have a uniquely twisted,
fucked up imagination, and I say that with the warmest
admiration.”

Guignol leaned in, too, staring at it.

“It’s so obvious now that I see it,” he said.

“Yeah, all the really great ideas are like that,” Lester said.

Sammy straightened up and shook Lester’s hand. “Thank you for the
tour, Lester. You have managed to simultaneously impress and
depress me. You are one sharp motherfucker.”

Lester preened and Suzanne suppressed a giggle.

Sammy held his hand up like he was being sworn in. “I’m dead
serious, man. This is amazing. I mean, we manage some pretty
out-of-the-box thinking at Disney, right? We may not be as nimble
as some little whacked out co-op, but for who we are\dash{}I think we do
a good job.

“But you, man, you blow us out of the water. This stuff is just
\emph{crazy}, like it came down from Mars. Like it’s from the
future.” He shook his head. “It’s humbling, you know.”

Guignol looked more thoughtful than he had to this point. He and
Lester stared at Sammy, wearing similar expressions of bemusement.

“Let’s go into the apartment,” Suzanne said. “We can sit down and
have a chat.”

They trooped up the stairs together. Guignol expressed admiration
for the weird junk-sculptures that adorned each landing, made by a
local craftswoman and installed by the landlord. They sat around
the living room and Lester poured iced coffee out of a pitcher in
the fridge, dropping in ice-cubes molded to look like legos.

They rattled their drinks and looked uncomfortably at one another.
Suzanne longed to whip out her computer and take notes, or at least
a pad, or a camera, but she restrained herself. Guignol looked
significantly at Sammy.

“Lester, I’m just going to say it. Would you sell your business to
us? The ride, DiaBolical, all of it? We could make you a very, very
rich man. You and Perry. You would have the freedom to go on doing
what you’re doing, but we’d put it in our production chain,
mass-market the hell out of it, get it into places you’ve never
seen. At its peak, New Work\dash{}which you were only a small part of,
remember\dash{}touched 20 percent of Americans. \emph{90 percent} of
Americans have been to a Disney park. We’re a bigger tourist draw
than \emph{all of Great Britain.} We can give your ideas legs.”

Lester began to chuckle, then laugh, then he was doubled over,
thumping his thighs. Suzanne shook her head. In just a few short
moments, she’d gotten used to the idea, and it was growing on her.

Guignol looked grim. “It’s not a firm offer\dash{}it’s a chance to open a
dialogue, a negotiation. Talk the possibility over. A good
negotiation is one where we both start by saying what we want and
work it over until we get to the point where we’re left with what
we both need.”

Lester wiped tears from his eyes. “I don’t think that you grasp the
absurdity of this situation, fellas. For starters, Perry will never
go for it. I mean \emph{never.}” Suzanne wondered about that. And
wondered whether it mattered. The two had hardly said a word to
each other in months.

“What’s more, the rest of the rides will never, never, \emph{never}
go in for it. That’s also for sure.

“Finally, what the fuck are you talking about? Me go to work for
you? Us go to work for you? What will you do, stick Mickey in the
ride? He’s already in the ride, every now and again, as you well
know. You going to move me up to Orlando?”

Sammy waggled his head from side to side. “I have a deep
appreciation for how weird this is, Lester. To tell you the truth,
I haven’t thought much about your ride or this little town. As far
as I’m concerned, we could just buy it and then turn around and
sell it back to the residents for one dollar\dash{}we wouldn’t want to
own or operate any of this stuff, the liability is too huge.
Likewise the other rides. We don’t care about what you did
\emph{yesterday}\dash{}we care about what you’re going to do tomorrow.

“Listen, you’re a smart guy. You make stuff that we can’t dream of,
that we lack the institutional imagination to dream of. We need
\emph{that}. What the hell is the point of fighting you, suing you,
when we can put you on the payroll? And you know what? Even if we
throw an idiotic sum of money at you, even if you never make
anything for us, we’re still ahead of the game if you stop making
stuff \emph{against} us.

“I’m putting my cards on the table here. I know your partner is
going to be even harder to convince, too. None of this is going to
be easy. I don’t care about easy. I care about what’s right. I’m
sick of being in charge of sabotaging people who make awesome
stuff. Aren’t you sick of being sabotaged? Wouldn’t you like to
come work some place where we’ll shovel money and resources at your
projects and keep the wolves at bay?”

Suzanne was impressed. This wasn’t the same guy whom Rat-Toothed
Freddy had savaged. It wasn’t the same guy that Death Waits had
described. He had come a long way. Even Guignol\dash{}whom, she
suspected, needed to be sold on the idea almost as much as
Lester\dash{}was nodding along by the end of it.

Lester wasn’t though: “You’re wasting your time, mister. That’s all
there is to it. I am not going to go and work for\dash{}” a giggle
escaped his lips “\dash{}Disney. It’s just\dash{}”

Sammy held his hands up in partial surrender. “OK, OK. I won’t push
you today. Think about it. Talk it over with your buddy. I’m a
patient guy.” Guignol snorted. “I don’t want to lean on you here.”

They took their leave, though Suzanne found out later that they’d
taken a spin around the ride before leaving. Everyone went on the
ride.

Lester shook his head at the door behind them.

“Can you believe that?”

Suzanne smiled and squeezed his hand. “You’re funny about this, you
know that? Normally, when you encounter a new idea, you like to
play with it, think it through, see what you can make of it. With
this, you’re not even willing to noodle with it.”

“You can’t seriously think that this is a good idea\dash{}”

“I don’t know. It’s not the dumbest idea I’ve ever heard. Become a
millionaire, get to do whatever you want? It’ll sure make an
interesting story.”

He goggled at her.

“Kidding,” she said, thinking,
\emph{It would indeed make an interesting story, though.} “But
where are you going from here? Are you going to stay here
forever?”

“Perry would never go for it\dash{}” Lester said, then stopped.

“You and Perry, Lester, how long do you think that’s going to
last.”

“Don’t you go all Yoko on me, Suzanne. We’ve got one of those
around here already\dash{}”

“I don’t like this Yoko joke, Lester. I never did. Hilda doesn’t
want to drive Perry away from you. She wants to make the rides
work. And it sounds like that’s what Perry wants, too. What’s wrong
with them doing that? Especially if you can get them a ton of money
to support it?”

Lester stared at her, open-mouthed. “Honey\dash{}”

“Think about it, Lester. Your most important virtue is your
expansive imagination. Use it.”

She watched this sink in. It did sink in. Lester listened to her,
which surprised her every now and again. Most relationships seemed
to be negotiations or possibly competitions. With Lester it was a
conversation.

She gave him a hug that seemed to go on forever.

\begin{center}\rule{3in}{0.4pt}\end{center}

Sammy was glad he was driving. The mood Guignol was in, he’d have
wrecked the car. “That was \emph{not} the plan, Sammy,” he said.
“The plan was to get the data, talk it over\dash{}”

“The first casualty of any battle is the battle-plan,” Sammy said,
threading them through the press of tourist busses and commuter
cars.

“I thought the first casualty was the truth.”

They’d spent too long at the ride, then gotten stuck in the
afternoon rush hour out of Miami. “That too. Look, I’m proposing to
spend a tenth of the profits from the DiaB on this venture. In any
other circumstance, I would do it with a \emph{purchase order}. The
only reason it’s a big deal is\dash{}”

“That it carries enough legal liability to destroy the company.
Sammy, didn’t you listen to Hackelberg?”

“The reason I still work at Disney is that it’s the kind of company
where the lawyers don’t \emph{always} set the agenda.”

Guignol drummed his hands on the dashboard. Sammy pulled over and
gassed up. At the next pump was a minivan with Kansas plates. Dad
was a dumpy Korean guy, Mom was a dumpy white midwesterner with a
country-and-western denim jacket, and the back seat was filled with
vibrating children, two girls and a boy. The kids were screaming
and fighting, the girls trying to draw on the boy’s face with
candy-flavored lipstick and kiddie mascara, the boy squirming
mightily and lashing out at them with his gameboy.

Dad and Mom were having their own heated discussion as Dad gassed
up, Sammy eavesdropped enough to hear that they were fighting over
Dad’s choice of taking the toll roads instead of the cheaper,
slower alternative route. The kids were shouting so loud, though\dash{}

“You keep that up and we’re not going to Disney World!”

It was the magic sentence, the litmus test for Disney’s currency.
As it rose and fell, so did the efficacy of the threat. If Sammy
could, he’d take a video of the result every time this was
uttered.

The kids looked at Dad and shrugged. “Who cares?” the eldest sister
said, and grabbed the boy again.

Sammy turned to Guignol and waggled his eyebrows. Once he was back
in the car, he said, “You know, it’s risky doing anything. But
riskiest of all is doing nothing.”

Guignol shook his head and pulled out his computer.

He spent a lot of time looking at the numbers while Sammy fought
traffic. Finally he closed his computer, put his head back and shut
his eyes. Sammy drove on.

“You think this’ll work?” Guignol said.

“Which part?

“You think if you buy these guys out\dash{}”

“Oh, that part. Sure, yeah, slam dunk. They’re cheap. Like I say,
we could make back the whole nut just by settling the lawsuit. The
hard part is going to be convincing them to sell.”

“And Hackelberg.”

“That’s your job, not mine.”

Guignol slid the seat back so it was flat as a bed. “Wake me when
we hit Orlando.”

\begin{center}\rule{3in}{0.4pt}\end{center}

It took IT three days to get Sammy his computer back. His secretary
managed as best as she could, but he wasn’t able to do much without
it.

When he got it back at last, he eagerly downloaded his backlog of
mail. It beggared the imagination. Even after auto-filtering it,
there were hundreds of new messages, things he had to pay real
attention to. When he was dealing with this stuff in little spurts
every few minutes all day long, it didn’t seem like much, but it
sure piled up.

He enlisted his secretary to help him with sorting and responding.
After an hour she forwarded one back to him with a bold red flag.

It was from Freddy. He got an instant headache, the feeling halfway
between a migraine and the feeling after you bang your head against
the corner of a table.

\begin{blog}
Sammy, I’m disappointed in you. I thought we were friends. Why
do I have to learn about your bizarre plan to buy out Gibbons and
Banks from strangers. I do hope you’ll give me a comment on the
story?
\end{blog}

He’d left the financials with Guignol, who had been discreetly
showing them around to the rest of the executive committee in
closed door, off-site meetings. One of them must have blabbed,
though\dash{}or maybe it was a leak at Lester’s end.

He tasted his lunch and bile as his stomach twisted. It wasn’t
fair. He had a real chance of making this happen\dash{}and it would be a
source of genuine good for all concerned.

He got halfway through calling Guignol’s number, then put the phone
down. He didn’t know who to call. He’d put himself in an unwinnable
position. As he contemplated the article that Freddy would probably
write, he realized that he would almost certainly lose his job over
this, too. Maybe end up on the wrong end of a lawsuit. Man, that
seemed to be his natural state at Disney. Maybe he was in the wrong
job.

He groaned and thumped himself on the forehead. All he wanted to do
was have good ideas and make them happen.

Basically, he wanted to be Lester.

Then he knew who he had to call.

“Ms Church?”

“We’re back to that, huh? That’s probably not a good sign.”

“Suzanne then.”

“Sammy, you sound like you’re about to pop a testicle. Spit it
out.”

“Do you think I could get a job with Lester?”

“You’re not joking, are you?”

“Freddy found out about the buyout offer.”

“Oh.”

“Yeah.”

“So I’m gonna be in search of employment. All I ever wanted to do
was come up with cool ideas and execute them\dash{}”

“Shush now. Freddy found out about this, huh? Not surprising. He’s
got a knack for it. It’s just about his only virtue.”

“Urgh.”

“However, it’s also his greatest failing. I’ve given this a lot of
thought, since my last run in with Rat-Toothed Freddy.”

“You call him that to his face?”

“Not yet. But I look forward to it. Tell you what, give me an hour
to talk to some people here, and I’ll get back to you.”

An hour? “An hour?”

“He’ll keep you squirming for at least that long. He loves to make
people squirm. It’s good journalism\dash{}shakes loose some new
developments.”

“An hour?”

“Have you got a choice?”

“An hour, then.”

\begin{center}\rule{3in}{0.4pt}\end{center}

Suzanne didn’t knock on Lester’s door. Lester would fall into
place, once Perry was in.

She found him working the ride, Hilda back in the maintenance bay,
tweaking some of the robots. His arm was out of the cast, but it
was noticeably thinner than his good left arm, weak and pale and
flabby.

“Hello, Suzanne.” He was formal, like he always was these days, and
it saddened her, but she pressed on.

“Perry, we need to shut down for a while, it’s urgent.”

“Suzanne, this is a busy time, we just can’t shut down\dash{}”

She thumped her hand on his lemonade-stand counter. “Cut it out,
Perry. I have never been an alarmist, you know that. I understand
intimately what it means to shut this place down. Look, I know that
things haven’t been so good between us, between \emph{any} of us,
for a long time. But I am your dear friend, and you are mine, no
matter what’s going on at this second, and I’m telling you that you
need to shut this down and we need to talk. Do it, Perry.”

He gave her a long, considering look.

“Please?”

He looked at the little queue of four or five people, pretending
not to eavesdrop, waiting their turn.

“Sorry, folks, you heard the lady. Family emergency. Um, here\dash{}” He
rummaged under the counter, came up with scraps of paper. “Mrs
Torrence’s tearoom across the street\dash{}they make the best cappuccino
in the hood, and the pastries are all baked fresh. On me, OK?”

“Come on,” Suzanne said. “Time’s short.”

She accompanied him to the maintenance bay and they pulled the
doors shut behind them. Hilda looked up from her robot, wiping her
hands on her shorts. She was really lovely, and the look on her
face when she saw Perry was pure adoration. Suzanne’s heart welled
up for the two of them, such a perfect picture of young love.

Then Hilda saw Suzanne, and her expression grew guarded, tense.
Perry took Hilda’s hand.

“What’s this about, Suzanne?” he said.

“Let me give this to you in one shot, OK?” They nodded. She ran it
down for them. Sammy and Guignol, the postcard and the funny
circumstances of their visit\dash{}the phone call.

“So here’s the thing. He wants to buy you guys out. He doesn’t want
the ride or the town. He just wants\dash{}I don’t know\dash{}the
\emph{creativity}. The PR win. He wants peace. And the real news
is, he’s over a barrel. Freddy’s forcing his hand. If we can make
that problem go away, we can ask for \emph{anything}.”

Hilda’s jaw hung slack. “You have to be kidding\dash{}”

Perry shushed her. “Suzanne, why are you here? Why aren’t you
talking to Lester about this? Why hasn’t Lester talked to me about
this. I mean, just what the fuck is going on?”

She winced. “I didn’t talk to Lester because I thought he’d be
easier to sell on this than you are. This is a golden opportunity
and I thought that you would be conflicted as hell about it and I
thought if I talked to you first, we could get past that. I don’t
really have a dog in this fight, except that I want all parties to
end up not hating each other. That’s where you’re headed now\dash{}you’re
melting down in slow motion. How long since you and Lester had a
conversation together, let alone a real meal? How long since we all
sat around and laughed? Every good thing comes to some kind of end,
and then the really good things come to a beginning again.

“You two \emph{were} the New Work. Lots of people got blisteringly
rich off of New Work, but not you. Here’s a chance for you to get
what you deserve for a change. You solve this\dash{}and you \emph{can}
solve it, and not just for you, but for that Death kid, you can get
him justice that the courts will take fifteen years to deliver.”

Perry scowled. “I don’t care about money\dash{}”

“Yes, that’s admirable. I have one other thing; I’ve been saving it
for last, waiting to see if you’d come up with it on your own.”

“What?”

“Why is time of the essence?”

“Because Freddy’s going to out this dirtball\dash{}”

“And how do we solve that?”

Hilda grinned. “Oh, this part I like.”

Suzanne laughed. “Yeah.”

“What?” Perry said.

“Freddy’s good at intelligence gathering, but he’s not so good at
distinguishing truth from fiction. In my view, this presents a
fascinating opportunity. Depending on what we leak to him and how,
we can turn him into\dash{}”

“A laughing stock?”

“A puddle of deliquesced organ meat.”

Perry began to laugh. “You’re saying that you think that we should
do this deal for \emph{spite}?”

“Yeah, that’s the size of it,” Suzanne said.

“I love it,” he said.

Hilda laughed too. Suzanne extended her hand to Perry and he shook
it. Then she shook with Hilda.

“Let’s go find Lester.”

\begin{center}\rule{3in}{0.4pt}\end{center}

By the time the call came, Sammy was ready to explode. He got in a
golf cart and headed to the Animal Kingdom Lodge, which backed onto
the safari park portion of the Animal Kingdom. He snuck himself
onto the roof of the grand hotel, which had a commanding view of
the artificial savanna. He watched a family of giraffes graze,
using the zoom on his phone to resolve the hypnotic patterns of the
little calf. It calmed him. But the sound of his phone ringing
startled him so much he nearly did a half-gainer off the roof.
Heart hammering, he answered it.

“Is this Sammy?”

“Yes,” he said.

“Landon Kettlewell,” the voice on the other side said. Sammy knew
the name, of course. But he hadn’t been expecting a call from him.

“Hello, Mr Kettlewell.”

“The boys have asked me to negotiate this deal for them. It makes
sense\dash{}it’ll be hard to make this happen without my contributions. I
hope you agree.”

“It does make sense,” Sammy said noncommittally. This wasn’t the
best day of his life. The giraffes were moving off, but a flock of
cranes was wheeling overhead in quiet splendor.

“I’ll tell you where we’re at. We’re going to do a deal with you, a
fair one. But a condition of the deal is that we are going to
destroy Freddy.”

“What?”

“We’re going to leak him bad intel on the deal. Lots of it. Give
him a whole story. Wait until he publishes it, and then\dash{}”

Sammy sat down on the roof. This was going to be a long
conversation.

\begin{center}\rule{3in}{0.4pt}\end{center}

Perry ground his teeth and squeezed his beer. The idea of doing
this in a big group had seemed like a good idea. Dirty Max’s was
certainly full of camaraderie, the smell of roasting meat and the
chatter of nearly a hundred voices. He heard Hilda laughing at
something Lester said to her, and there were Kettlewell and his
kids, fingers and faces sticky with sauce.

Lester had set up the projector and they’d hung sheets over one of
the murals for a screen, and brought out a bunch of wireless
speakers that they’d scattered around the courtyard. It looked,
smelled, sounded, and tasted like a carnival.

But Perry couldn’t meet anyone’s eye. He just wanted to go home and
get under the covers. They were about to destroy Freddy, which had
also seemed like a hell of a lark at the time, but now\dash{}

“Perry.” It was Sammy, up from Orlando, wearing the classic
Mickey-gives-the-finger bootleg tee.

“Can you get fired for that?” Perry pointed.

Sammy shook his head. “Actually, it’s official. I had them produced
last year\dash{}they’re a big seller. If you can’t beat ’em\ldots{} Here\dash{}” He
dug in the backpack he carried and pulled out another. “You look
like a large, right?”

Perry took it from him, held it up. Shrugging, he put down his beer
and skinned his tee, then pulled on the Mickey-flips-the-bird. He
looked down at his chest. “It’s a statement.”

“Have you and Lester given any thought to where you’re going to
relocate, after?”

Perry drew in a deep breath. “I think Lester wants to come to
Orlando. But I’m going to go to Wisconsin. Madison.”

“You’re what now?”

Perry hadn’t said anything about this to anyone except Hilda.
Something about this Disney exec, it made him want to spill the
beans. “I can’t go along with this. I’m going to bow out. Do
something new. I’ve been in this shithole for what feels like my
whole life now.”

Sammy looked poleaxed. “Perry, that wasn’t the deal\dash{}”

“Yeah, I know. But think about this: do you want me there if I hate
it, resent it? Besides, it’s a little late in the day to back
out.”

Sammy reeled. “Christ almighty. Well, at least you’re not going to
end up my employee.”

Francis\dash{}who had an uncanny knack for figuring out the right moment
to step into a conversation\dash{}sidled over. “Nice shirt, Perry.”

“Francis, this is Sammy.” Francis had a bottle of water and a plate
of ribs, so he extended a friendly elbow.

“We’ve met\dash{}showed him the bicycle factory.”

Sammy visibly calmed himself. “That’s right, you did. Amazing, just
amazing.”

“All this is on Sammy,” Perry said, pointing at the huge barbecue
smoker, the crowds of sticky-fingered gorgers. “He’s the Disney
guy.”

“Hence the shirts, huh?”

“Exactly.”

“So what’s the rumpus, exactly?” Francis asked. “It’s all been
hush-hush around here for a solid week.”

“I think we’re about to find out,” Perry said, nodding at the
gigantic screen, which rippled in the sultry Florida night-breeze,
obscured by blowing clouds of fragrant smoke. It was lit up now,
showing CNNfn, two pan-racial anchors talking silently into the
night.

The speakers popped to life and gradually the crowd noises dimmed.
People moved toward the screen, all except Francis and Perry and
Sammy, who hung back, silently watching the screen.

“\dash{}guest on the show is Freddy Niedbalski, a technology reporter for
the notorious British technology publication \emph{Tech Stink}.
Freddy has agreed to come on \emph{Countdown} to break a story that
will go live on \emph{Tech Stink}’s website in about ten minutes.”
The camera zoomed out to show Freddy, sitting beside the anchor
desk in an armchair. His paunch was more pronounced than it had
been when Perry had seen him in Madison, and there was something
wrong with his makeup, a color mismatch that made him look like
he’d slathered himself with Man-Tan. Still, he was grinning evilly
and looking like he could barely contain himself.

“Thank you, Tania-Luz, it’s a pleasure.”

“Now, take us through the story. You’ve been covering it for a long
time, haven’t you?”

“Oh yes. This is about the so-called ’New Work’ cult, and its
aftermath. I’ve broken a series of scandals involving these
characters over the years\dash{}weird sex, funny money, sweatshop labor.
These are the people who spent all that money in the New Work
bubble, and then went on to found an honest-to-God slum that they
characterized as a ‘living laboratory.’”\dash{}out came the sarcastic
finger-quotes\dash{}“but, as near as anyone can work out was more of a
human subject experiment gone mad. They pulled off these bizarre
stunts with the help of some of the largest investment funds on the
planet.”

Perry looked around at the revellers. They were chortling, pointing
at each other, mugging for the camera. Freddy’s words made Perry
uncomfortable\dash{}maybe there was something to what he said. But there
was Francis, unofficial mayor of the shantytown, smiling along with
the rest. They hadn’t been perfect, but they’d left the world a
better place than they’d found it.

“There are many personalities in this story, but tonight’s
installment has two main players: a venture capitalist named Landon
Kettlewell and a Disney Parks senior vice president called Sammy
Page. Technically, these two hate each others’ guts\dash{}” Sammy and
Kettlewell toasted each other through the barbecue smoke. “But
they’ve been chumming up to one another lately as they brokered an
improbable deal to shaft everyone else in the sordid mess.”

“A deal that you’ve got details on for us tonight?”

“Exactly. My sources have turned up reliable memos and other
intelligence indicating that the investors behind the shantytown
are about to \emph{take over Disney Parks}. It all stems from a
lawsuit that was brought on behalf of a syndicate of operators of
bizarre, trademark infringing rides that were raided off the backs
of complaints from Disney Parks. These raids, and a subsequent and
very suspicious beating of an ex-Disney Park employee, led to the
creation of an investment syndicate to fund a monster lawsuit
against Disney Parks, one that could take the company down.

“The investment syndicate found an unlikely ally in the person of
Sammy Page, the senior VP from Disney Parks, who worked with them
to push through a plan where they would settle the lawsuit in
exchange for a controlling interest in Disney Parks.”

The anchors looked suitably impressed. Around the screen, the
partiers had gone quiet, even the kids, mesmerized by Freddy’s
giant head, eyes rolling with irony and mean humor.

“And that’s just for starters. The deal required securing the
cooperation of the beaten-up ex-Disney employee, who goes by the
name of ’Death Waits’\dash{}no, really!\dash{}and \emph{he} required that he be
made a vice president of the new company as well, running the
’Fantasyland’ section of the Florida park. In the new structure,
the two founders of the New Work scam, Perry Gibbons and Lester
Banks are to oversee the Disneyfication of the activist rides
around the country, selling out their comrades, who signed over
control of their volunteer-built enterprises as part of the earlier
lawsuit.”

The male anchor shook his head. “If this is true, it’s the
strangest turn in American corporate history.”

“Oh yes,” Freddy said. “These people are like some kind of poison,
a disease that affects the judgement of all those around them\dash{}”

“If it’s true,” the male anchor continued, as if Freddy hadn’t
spoken. “But is it? Our next guest denies all of this, and claims
that Mr Niedbalski has his facts all wrong. Tjan Lee Tang is the
chairman of Massachusetts Ride Theorists, a nonprofit that operates
three of the spin-off rides in New England. He is in our Boston
studios. Welcome, Mr Tang.”

Freddy’s expression was priceless: a mixture of raw terror and
contempt. He tried to cover it, but only succeeded in looking
constipated. On the other half of the split-screen, Tjan beamed
sunnily at them.

“Hi there!” he said. “Greetings from the blustery Northeast.”

“Mr Tang, you’ve heard what our guest has to say about the latest
developments in the extraordinary story of the rides you helped
create. Do you have any comment?”

“I certainly do. Freddy, old buddy, you’ve been had. Whomever your
leak was in Disney, he was putting you on. There is not one single
word of truth to anything you had to say.” He grinned wickedly. “So
what else is new?”

Freddy opened his mouth and Tjan held up one hand. “No, wait, let
me finish. I know it’s your schtick to come after us this way,
you’ve been at it for years. I think it’s because you have an
unrequited crush on Suzanne Church.

“Here’s what’s really happening. Lester Banks and Perry Gibbons
have taken jobs with Disney Parks as part of a straightforward
deal. They’re going to do research and development there, and
Disney is settling its ongoing lawsuit with us with a seventy
million dollar cash settlement. Half goes to the investors. Some of
the remainder will go to buy the underlying titles to the
shantytown and put them in a trust to be managed by a co-operative
of residents. The rest is going into another trust that will be
disbursed in grants to people operating rides around the country.
There’s a non-monetary part of the deal, too: all rides get a
perpetual, worldwide license on all Disney trademarks for use in
the rides.”

The announcers smiled and nodded.

“We think this is a pretty good win. The rides go on. The
shantytown goes on. Lester and Perry get to do great work in a
heavily resourced lab environment.”

Tania-Luz turned to Freddy. “It seems that your story is in
dispute. Do you have further comment?”

Freddy squirmed. A streak of sweat cut through his pancake makeup
as the camera came in for a closeup. “Well, if this is true, I’d
want to know why Disney would make such a generous offer\dash{}”

“Generous?” Tjan said. He snorted. “We were asking for
\emph{eight billion} in punitive damages. They got off easy!”

Freddy acted like he hadn’t heard. “Unless the terms of this
so-called deal are published and subject to scrutiny\dash{}”

“We posted them about five minutes ago. You could have just asked
us, you know.”

Freddy’s eyes bugged out. “We have no way of knowing whether what
this man is saying is true\dash{}”

“Actually, you do. Like I say, it’s all online. The deals are
signed. Securities filings and everything.”

Freddy got up out of his seat.
“\emph{Would you shut up and let me finish}?” he screamed.

“Sorry, sorry,” Tjan said with a chuckle. He was enjoying this way
too much. “Go on.”

“And what about Death Waits? He’s been a pawn all along in this
game you’ve played with other people’s lives. What happens to him
as you all get rich?”

Tjan shrugged. “He got a large cash settlement too. He seemed
pretty happy about it\dash{}”

Freddy was shaking. “You can’t just sell off your lawsuit\dash{}”

“We were looking to get compensated for bad acts. We got
compensated for them, and we did it without tying up the public
courts. Everybody wins.” He cocked his head. “Except you, of
course.”

“This was a fucking ambush,” Freddy said, pointing his fingers at
the two coiffed and groomed anchors, who shied away dramatically,
making him look even crazier. He stormed off the stage, cursing,
every word transmitted by his still-running wireless mic. He
shouted at an invisible security guard to get out of his way. Then
they heard him make a phone-call, presumably to his editor,
shouting at him to kill the article, nearly weeping in frustration.
The anchors and Tjan pasted on unconvincing poker-faces, but around
the BBQ pit, it was all howls of laughter, which turned to shrieks
when Freddy finally figured out that he was still on a live mic.

Perry and Sammy locked eyes and grinned. Perry ticked a little
salute off his forehead at Sammy and hefted his tee. Then he turned
on his heel and walked off into the night, the fragrant smell of
the barbecue smoke and the sound of the party behind him.

He parked his car at home and trudged up the stairs. Hilda had
packed her suitcase that morning. He had a lot more than a
suitcase’s worth of stuff around the apartment, but as he threw a
few t-shirts\dash{}including his new fake bootleg Mickey tee\dash{}and some
underwear in a bag, he suddenly realized that he didn’t care about
any of it.

Then he happened upon the baseball glove. The cloud of old leather
smell it emitted when he picked it up made tears spring into his
eyes. He hadn’t cried through any of this process, though, and he
wasn’t about to start now. He wiped his eyes with his forearm and
reverently set the glove into his bag and shut it. He carried both
bags downstairs and put them in the trunk, then he drove to just a
little ways north of the ride and called Hilda to let her know he
was ready to go.

She didn’t say a word when she got in the car, and neither did he,
all the way to Miami airport. He took his frisking and secondary
screening in stoic silence, and once they were seated on the
Chicago flight, he put his head down on Hilda’s shoulder and she
stroked his hair until he fell asleep.

\begin{center}\rule{3in}{0.4pt}\end{center}

Epilogue

Lester was in his workshop when Perry came to see him. He had the
yoga mat out and he was going through the slow exercises that his
physiotherapist had assigned to him, stretching his crumbling bones
and shrinking muscles, trying to keep it all together. He’d fired
three physios, but Suzanne kept finding him new ones, and (because
she loved him) prettier ones.

He was down on all fours, his ass stuck way up in the air, when
Perry came through the door. He looked back through his ankles and
squinted at the upside-down world. Perry’s expression was carefully
neutral, the same upside-down as it would be right-side-up. He
grunted and went down to his knees, which crackled like popcorn.

“That doesn’t sound good,” Perry remarked mildly.

“Funny man,” Lester said. “Get over here and help me up, will
you?”

Perry went down in a crouch before him. There was something funny
about his eye, the whole side of his head. He smelled a little
sweaty and a little gamy, but the face was the one Lester knew so
well. Perry held out his strong, leathery hands, and after a
moment, Lester grasped them and let Perry drag him to his feet.

They stood facing one another for an uncomfortable moment, hands
clasped together. Then Perry flung his arms wide and shouted, “Here
I am!”

Lester laughed and embraced his old friend, not seen or heard from
these last 15 years.

\begin{center}\rule{3in}{0.4pt}\end{center}

Lester’s workshop had a sofa where he entertained visitors and took
his afternoon nap. Normally, he’d use his cane to cross from his
workbench to the sofa, but seeing Perry threw him for such a loop
that he completely forgot until he was a pace or two away from it
and then he found himself flailing for support as his hips started
to give way. Perry caught him under the shoulders and propped him
up. Lester felt a rush of shame color his cheeks.

“Steady there, cowboy,” Perry said.

“Sorry, sorry,” Lester muttered.

Perry lowered him to the sofa, then looked around. “You got
anything to drink? Water? I didn’t really expect the bus would take
as long as it did.”

“You’re taking the bus around Burbank?” Lester said. “Christ,
Perry, this is Los Angeles. Even homeless people drive cars.”

Perry looked away and shook his head. “The bus is cheaper.” Lester
pursed his lips. “You got anything to drink?”

“In the fridge,” Lester said, pointing to a set of nested clay pot
evaporative coolers. Perry grinned at the jury-rigged cooler and
rummaged around in its mouth for a while. “Anything, you know,
buzzy? Guarana? Caffeine, even?”

Lester gave an apologetic shrug. “Not me, not anymore. Nothing goes
into my body without oversight by a team of very expensive
nutritionists.”

“You don’t look so bad,” Perry said. “Maybe a little skinny\dash{}”

Lester cut him off. “Not bad like the people you see on TV, huh?
Not bad like the dying ones.” The fatkins had overwhelmed the
nation’s hospitals in successive waves of sickened disintegrating
skeletons whose brittle bones and ruined joints had outstripped
anyone’s ability to cope with them. The only thing that kept the
crisis from boiling over entirely was the fast mortality that
followed on the first symptoms\dash{}difficulty digesting, persistent
stiffness. Once you couldn’t keep down high-calorie slurry, you
just starved to death.

“Not like them,” Perry agreed. He had a bit of limp, Lester saw,
and his old broken arm hung slightly stiff at his side.

“I’m doing OK,” Lester said. “You wouldn’t believe the medical
bills, of course.”

“Don’t let Freddy know you’ve got the sickness,” Perry said. “He’d
love that story\dash{}’fatkins pioneer pays the price\dash{}’”

“Freddy! Man, I haven’t thought of that shitheel in\dash{}Christ, a
decade, at least. Is he still alive?”

Perry shrugged. “Might be. I’d think that if he’d keeled over
someone would have asked me to pitch in to charter a bus to go piss
on his grave.”

Lester laughed hard, so hard he hurt his chest and had to sag back
into the sofa, doing deep yoga breathing until his ribs felt
better.

Perry sat down opposite him on the sofa with a bottle of Lester’s
special thrice-distilled flat water in a torpedo-shaped bottle.
“Suzanne?” he asked.

“Good,” Lester said. “Spends about half her time here and half on
the road. Writing, still.”

“What’s she on to now?”

“Cooking, if you can believe it. Molecular gastronomy\dash{}food hackers
who use centrifuges to clarify their consomme. She says she’s never
eaten better. Last week it was some kid who’d written a genetic
algorithm to evolve custom printable molecules that can bridge two
unharmonius flavors to make them taste good together\dash{}like, what do
you need to add to chocolate and sardines to make them freakin’
delicious?”

“Is there such a molecule?”

“Suzanne says there is. She said that they misted it into her face
with a vaporizer while she ate a sardine on a slab of dark
chocolate and it tasted better than anything she’d ever had
before.”

“OK, that’s just wrong,” Perry said. The two of them were grinning
at each other like fools.

Lester couldn’t believe how good it felt to be in the same room as
Perry again after all these years. His old friend was much older
than the last time they’d seen each other. There was a lot of grey
in his short hair, and his hairline was a lot higher up his
forehead. His knuckles were swollen and wrinkled, and his face had
deep lines, making him look carved. He had the leathery skin of a
roadside homeless person, and there were little scars all over his
arms and a few on his throat.

“How’s Hilda?” Lester asked.

Perry looked away. “That’s a name I haven’t heard in a while,” he
said.

“Yowch. Sorry.”

“No, that’s OK. I get email blasts from her every now and again.
She’s chipper and scrappy as always. Fighting the good fight.
Fatkins stuff again\dash{}same as when I met her. Funny how that fight
never gets old.”

“Hardy har har,” Lester said.

“OK, we’re even,” Perry said. “One-one on the faux-pas master’s
tournament.”

They chatted about inconsequentialities for a while, stories about
Lester’s life as the closeted genius at Disney Labs, Perry’s life
on the road, getting itinerant and seasonal work at little
micro-factories.

“Don’t they recognize you?”

“Me? Naw, it’s been a long time since I got recognized. I’m just
the guy, you know, he’s handy, keeps to himself. Probably going to
be moving on soon. Good with money, always has a quiet suggestion
for tweaking an idea to make it return a little higher on the
investment.”

“That’s you, all right. All except the ’keeps to himself’ part.”

“A little older, a little wiser. Better to keep your mouth shut and
be thought a fool than to open it and remove all doubt.”

“Thank you, Mister Twain. You and Huck been on the river a while
then?”

“No Huck,” he said. His smile got sad, heartbreakingly sad. This
wasn’t the Perry Lester knew. Lester wasn’t the same person,
either. They were both broken. Perry was alone, though\dash{}gregarious
Perry, always making friends. Alone.

“So, how long are you staying?”

“I’m just passing through, buddy. I woke up in Burbank this morning
and I thought, ‘Shit, Lester’s in Burbank, I should say hello.’ But
I got places to go.”

“Come on, man, stay a while. We’ve got a guest-cottage out back, a
little mother-in-law apartment. There are fruit trees, too.”

“Living the dream, huh?” He sounded unexpectedly bitter.

Lester was embarrassed for his wealth. Disney had thrown so much
stock at him in the beginning and Suzanne had sold most of it and
wisely invested it in a bunch of micro-funds; add to that the money
she was raking in from the affiliate sites her Junior
Woodchucks\dash{}kid-reporters she’d trained and set up in business\dash{}ran,
and they never had to worry about a thing.

“Well, apart from dying. And working here.” As soon as the words
were out of his mouth, he wished he could take them back. He never
let on that he wasn’t happy at the Mouse, and the dying thing\dash{}well,
Suzanne and he liked to pretend that medical science would cure
what it had brought.

Perry, though, he just nodded as if his suspicions were confirmed.
“Must be hard on Suzanne.”

Now that was hitting the nail on the head. “You always were a
perceptive son of a bitch.”

“She never said fatkins was good for you. She just reported the
story. The people who blame her\dash{}”

This was the elephant in the room whenever Lester and Suzanne
talked about his health. Between the two of them, they’d
popularized fatkins, sent millions winging to Russia for the
clinics, fuelled the creation of the clinics in the US and Mexico.

But they never spoke of it. Never. Now Perry was talking about it,
still talking:

“\dash{}the FDA, the doctors. That’s what we pay them for. The way I see
it, you’re a victim, their victim.”

Lester couldn’t say anything. Words stoppered themselves up in his
mouth like a cork. Finally, he managed to choke out, “Change the
subject, OK?”

Perry looked down. “Sorry. I’m out of practice with people.”

“I hope you’ll stay with us,” he said, thinking
\emph{I hope you leave soon and never come back}.

“You miss it, huh?”

“Sometimes.”

“You said working here\dash{}”

“Working here. They said that they wanted me to come in and help
them turn the place around, help them reinvent themselves. Be
nimble. Shake things up. But it’s like wrestling a tar-baby. You
push, you get stuck. You argue for something better and they tell
you to write a report, then no one reads the report. You try to get
an experimental service running and no one will reconfigure the
firewall. Turn the place around?” He snorted. “It’s like turning
around a battleship by tapping it on the nose with a toothpick.”

“I hate working with assholes.”

“They’re not assholes, that’s the thing, Perry. They’re some really
smart people. They’re nice. We have them over for dinner. They’re
fun to eat lunch with. The thing is,
\emph{every single one of them feels the same way I do}. They
\emph{all} have cool shit they want to do, but they can’t do it.”

“Why?”

“It’s like an emergent property. Once you get a lot of people under
one roof, the emergent property seems to be crap. No matter how
great the people are, no matter how wonderful their individual
ideas are, the net effect is shit.”

“Reminds me of reliability calculation. Like if you take two
components that are 90 percent reliable and use them in a design,
the outcome is 90 percent of 90 percent\dash{}81 percent. Keep adding 90
percent reliable components and you’ll have something that explodes
before you get it out of the factory.

“Maybe people are like that. If you’re 90 percent non-bogus and ten
percent bogus, and you work with someone else who’s 90 percent
non-bogus, you end up with a team that’s 81 percent non-bogus.”

“I like that model. It makes intuitive sense. But fuck me, it’s
depressing. It says that all we do is magnify each others’ flaws.”

“Well, maybe that’s the case. Maybe flaws are multiplicative.”

“So what are virtues?”

“Additive, maybe. A shallower curve.”

“That’d be an interesting research project, if you could come up
with some quantitative measurements.”

“So what do you do around here all day?”

Lester blushed.

“What?”

“I’m building bigger mechanical computers, mostly. I print them out
using the new volumetrics and have research assistants assemble
them. There’s something soothing about them. I have an Apple ][+
clone running entirely on physical gates made out of extruded
plastic skulls. It takes up an entire building out on one of the
lots and when you play Pong on it, the sound of the jaws clacking
is like listening to corpse beetles skeletonizing an elephant.”

“I think I’d like to see that,” Perry said, laughing a little.

“That can be arranged,” Lester said.

They were like gears that had once emerged from a mill with
perfectly precise teeth, gears that could mesh and spin against
each other, transferring energy.

They were like gears that had been ill-used in machines, apart from
each other, until their precise teeth had been chipped and bent, so
that they no longer meshed.

They were like gears, connected to one another and mismatched,
clunking and skipping, but running still, running still.

\begin{center}\rule{3in}{0.4pt}\end{center}

Perry and Lester rode in the back of the company car, the driver an
old Armenian who’d fled Azerbaijan, whom Lester introduced as
Kapriel. It seemed that Lester and Kapriel were old friends, which
made sense, since Lester couldn’t drive himself, and in Los
Angeles, you didn’t go anywhere except by car. The relationship
between a man and his driver would be necessarily intimate.

Perry couldn’t bring himself to feel envious of Lester having a
chauffeured car, though it was clear that Lester was embarrassed by
the luxury. It was too much like an invalid’s subsidy to feel
excessive.

“Kap,” Lester said, stirring in the nest of paper and parts and
empty health-food packages that he’d made of the back-seat.

Kapriel looked over his shoulder at them. “Home now?” He barely had
an accent, but when he turned his head, Perry saw that one ear had
been badly mangled, leaving behind a misshapen fist of scar.

“No,” Lester said. “Let’s eat out tonight. How about Musso and
Frank?”

“Ms Suzanne says\dash{}”

“We don’t need to tell her,” Lester said.

Perry spoke in a low voice, “Lester, I don’t need anything special.
Don’t make yourself sick\dash{}”

“Perry, buddy, shut the fuck up, OK? I can have a steak and a beer
and a big-ass dessert every now and again. Purified medicated
fatkins-chow gets old. My colon isn’t going to fall out of my
asshole in terror if I send a cheeseburger down there.”

They parked behind Musso and Frank and let the valet park the town
car. Kapriel went over to the Walk of Fame to take pictures of the
robotic movie stars doing acrobatic busking acts, and they went
into the dark cave of the restaurant, all dark wood, dark carpets,
pictures of movie stars on the walls. The maitre d’ gave them a
look, tilted his head, looked again. Calmly, Lester produced a
hundred-dollar bill and slid it across the podium.

“We’d like Orson Welles’s table, please,” he said.

The maitre d’\dash{}an elderly, elegant Mexican with a precise spade
beard\dash{}nodded affably. “Give me five minutes, gentlemen. Would you
care to have a drink in the bar?”

They sat at the long counter and Perry ordered a Scotch and soda.
Lester ordered water, then switched his order to beer, then
non-alcoholic beer, then beer again. “Sorry,” he said to the
waitress. “Just having an indecisive kind of night, I guess.”

Perry tried to figure out if Lester had been showing off with the
c-note, and decided that he hadn’t been. He’d just gone native in
LA, and a hundred for the maitre d’ when you’re in a hurry can’t be
much for a senior exec.

Lester sipped gingerly at his beer. “I like this place,” he said,
waving the bottle at the celebrity caricatures lining the walls.
“It’s perfect Hollyweird kitsch. Celebrities who usually eat out in
some ultra-modern place come here. They come because they’ve always
come\dash{}to sit in Orson Welles’s booth.”

“How’s the food?”

“Depends on what you order. The good stuff is great. You down for
steaks?”

“I’m down for whatever,” Perry said. Lester was in his medium here,
letting the waiter unfold his napkin and lay it over his lap
without taking any special notice of the old man.

The food was delicious, and they even got to glimpse a celebrity,
though neither Perry nor Lester knew who the young woman was, nor
what she was famous for. She was surrounded by children who came
over from other tables seeking autographs, and more than one patron
snapped a semi-subtle photo of her.

“Poor girl,” Perry said with feeling.

“It’s a career decision here. You decide to become famous because
you want that kind of life. Sometimes you even kid yourself that
it’ll last forever\dash{}that in thirty years, they’ll come into Musso
and Frank and ask for Miss Whatshername’s table. Anyone who wants
to know what stardom looks like can find out\dash{}and no one becomes a
star by accident.”

“You think?” Perry said. “I mean, we were celebs, kind of, for a
while there\dash{}”

“Are you saying that that happened by accident?”

“I never set out to get famous\dash{}”

“You took part in a national movement, Perry. You practically
\emph{founded} it. What did you think was going to happen\dash{}”

“You’re saying that we were just attention whores\dash{}”

“No, Perry, no. We weren’t \emph{just} attention whores. We were
attention whores \emph{and} we built and ran cool shit. There’s
nothing wrong with being an attention whore. It’s an attention
economy. If you’re going to be a working stiff, you should pick a
decent currency to get paid in. But you can’t sit there and tell me
that it didn’t feel good, didn’t feel \emph{great} to have all
those people looking up to us, following us into battle, throwing
themselves at us\dash{}”

Perry held up his hands. His friend was looking more alive than he
had at any time since Perry had been ushered into his workshop. He
sat up straight, and the old glint of mischief and good humor was
in his eye.

“I surrender, buddy, you’re right.” They ordered desserts, heavy
“diplomat puddings”\dash{}bread pudding made with cake and cherries, and
Lester dug in, after making Perry swear not to breathe a word of it
to Suzanne. He ate with such visible pleasure that Perry felt like
a voyeur.

“How long did you say you were in town for?”

“I’m just passing through,” Perry said. He had only planned on
maybe seeing Lester long enough for lunch or something. Now it
seemed a foregone conclusion that he’d be put up in the “guest
cottage.” He thought about getting back on the road. There was a
little gang in Oregon that made novelty school supplies, they were
always ramping up for their busy season at this time of year. They
were good people to work for.

“Come on, where you got to be? Stay a week. I’ll put you on the
payroll as a consultant. You can give lunch-hour talks to the R\&D
team, whatever you want.”

“Lester, you just got through telling me how much you hate your
job\dash{}”

“That’s the beauty of contracting\dash{}you don’t stick around long
enough to hate it, and you never have to worry about the org chart.
Come on, pal\dash{}”

“I’ll think about it.”

Lester fell asleep on the car ride home, and Kapriel didn’t mind if
Perry didn’t want to chat, so he just rolled his windows down and
watched the LA lights scream past as they hit the premium lanes on
the crosstown freeways, heading to Lester’s place in Topanga
Canyon. When they arrived, Lester roused himself heavily, clutched
his stomach, then raced for the house. Kapriel shook his head and
rolled his eyes, then showed Perry to the front door and shook his
hand.

\begin{center}\rule{3in}{0.4pt}\end{center}

In the morning, he prowled Lester and Suzanne’s place like a
burglar. The guesthouse had once served as Lester’s workshop and it
had the telltale leavings of a busy inventor\dash{}drawers and tubs of
parts, a moldy coffee-cup in a desk-drawer, pens and toys and
unread postal spam in piles. What it didn’t have was a kitchen, so
Perry helped himself to the key that Lester had left him with the
night before and wandered around the big house, looking for the
kitchen.

It turned out to be on the second floor, a bit of weird
architectural design that was characteristic of the place, which
had started as a shack in the hills on several acres of land and
then grown and grown as successive generations of owners had added
extensions, seismic retrofitting, and new floors.

Perry found the pantries filled with high-tech MREs, each
nutritionally balanced and fortified in ways calculated to make
Lester as healthy as possible. Finally, he found a small cupboard
clearly devoted to Suzanne’s eating, with boxes of breakfast cereal
and, way in the back, a little bag of Oreos. He munched
thoughtfully on the cookies while drinking more of the flat,
thrice-distilled water.

He heard Lester totter into a bathroom on the floor above, and
called “Good morning,” up a narrow, winding staircase.

Lester groaned back at him, a sound that Perry hadn’t heard in
years, that theatrical oh-my-shit-it’s-another-day sound.

He clomped down the stairs with his cane, wearing a pair of
boxer-shorts and rubber slippers. He was gaunt, the hair on his
sunken chest gone wiry grey, and the skin around his torso sagged.
From the neck down, he looked a hundred years old. Perry looked
away.

“Morning, bro,” Lester said, and took a vacuum-sealed pouch out of
a medical white box over the sink, tore it open, added purified
water, and put it in the microwave. The smell was like wet
cardboard in a dumpster. Perry wrinkled his nose.

“Tastes better than it smells. Or looks,” Lester said. “Very easy
on the digestion. Which I need. Never let me pig out like that
again, OK?”

He collapsed heavily into a stool and closed his sunken eyes.
Without opening them, he said, “So, are you in?”

“Am I in?”

“You going to come on board as my consultant?”

“You were serious about that, huh?”

“Perry, they can’t fire me. If I quit, I lose my health bennies,
which means I’ll be broke in a month. Which puts us at an impasse.
I’m past feeling guilty about doing nothing much all day long, but
that doesn’t mean I’m not bored.”

“You make it sound so attractive.”

“You got something better to do?”

“I’m in.”

\begin{center}\rule{3in}{0.4pt}\end{center}

Suzanne came home a week later and found them sitting up in the
living room. They’d pushed all the furniture up against the walls
and covered the floor with board-game boards, laid edge-to-edge or
overlapping. They had tokens, cards and money from several of the
games laid out around the rims of the games.

“What the blistering fuck?” she said good naturedly. Lester had
told her that Perry was around, so she’d been prepared for
something odd, but this was pretty amazing, even so. Lester held up
a hand for silence and rolled two dice. They skittered across the
floor, one of them slipping through the heating-grating.

“Three points,” Perry said. “One for not going into the grating,
two for going into the grating.”

“I thought we said it was two points for not going into the
grating, and one for dropping it?”

“Let’s call it 1.5 points for each.”

“Gentlemen,” Suzanne said, “I believe I asked a question? To wit,
’What the blistering fuck\dash{}’”

“Calvinball,” Lester said. “Like in the old Calvin and Hobbes
strips. The rules are, the rules can never be the same twice.”

“And you’re supposed to wear a mask,” Perry said. “But we kept
stepping on the pieces.”

“No peripheral vision,” Lester said.

“Caucus race!” Perry yelled, and took a lap around the world.
Lester struggled to his feet, then flopped back down.

“I disbelieve,” he said, taking up two ten-sided dice and rolling
them. “87,” he said.

“Fine,” Perry said. He picked up a Battleship board and said, “B7,”
and then he said, “What’s the score, anyway?”

“Orange to seven,” Lester said.

“Who’s orange?”

“You are.”

“Shit. OK, let’s take a break.”

Suzanne tried to hold in her laughter, but she couldn’t. She ended
up doubled over, tears streaming down her face. When she
straightened up, Lester hobbled to her and gave her a surprisingly
strong welcome-home hug. He smelled like Lester, like the man she’d
shared her bed with all these years.

Perry held out his hand to her and she yanked him into a long, hard
hug.

“It’s good to have you back, Perry,” she said, once she’d kissed
both his cheeks.

“It’s fantastic to see you, Suzanne,” he said. He was thinner than
she remembered, with snow on the roof, but he was still handsome as
a pirate.

“We missed you. Tell me everything you’ve been up to.”

“It’s not interesting,” he said. “Really.”

“I find that difficult to believe.”

So he told them stories from the road, and they were interesting in
a kind of microcosm sort of way. Stories about interesting
characters he’d met, improbable meals he’d eaten, bad working
conditions, memorable rides hitched.

“So that’s it?” Suzanne said. “That’s what you’ve done?”

“It’s what I do,” he said.

“And you’re happy?”

“I’m not sad,” he said.

She shook her head involuntarily. Perry stiffened.

“What’s wrong with not sad?”

“There’s nothing wrong with it, Perry. I’m\dash{}” she faltered, searched
for the words. “Remember when I first met you, met both of you, in
that ghost mall? You weren’t just happy, you were hysterical.
Remember the Boogie-Woogie Elmos? The car they drove?”

Perry looked away. “Yeah,” he said softly. There was a hitch in his
voice.

“All I’m saying is, it doesn’t have to be this way. You could\dash{}”

“Could what?” he said. He sounded angry, but she thought that he
was just upset. “I could go work for Disney, sit in a workshop all
day making crap no one cares about? Be the wage-slave for the end
of my days, a caged monkey for some corporate sultan’s zoo?” The
phrase was Lester’s, and Suzanne knew then that Perry and Lester
had been talking about it.

Lester, leaning heavily against her on the sofa (they’d pushed it
back into the room, moving aside pieces of the Calvinball game),
made a warning sound and gave her knee a squeeze. Aha, definitely
territory they’d covered before then.

“You two have some of the finest entrepreneurial instincts I’ve
ever encountered,” she said. Perry snorted.

“What’s more, I’ve never seen you happier than you were back when I
first met you, making stuff for the sheer joy of it and selling it
to collectors. Do you know how many collectors would pony up for an
original Gibbons/Banks today? You two could just do that forever\dash{}”

“Lester’s medical\dash{}”

“Lester’s medical nothing. You two get together on this, you could
make so much money, we could buy Lester his own hospital.”
\emph{Besides, Lester won’t last long no matter what happens}. She
didn’t say it, but there it was. She’d come to grips with the
reality years ago, when his symptoms first appeared\dash{}when \emph{all}
the fatkins’ symptoms began to appear. Now she could think of it
without getting that hitch in her chest that she’d gotten at first.
Now she could go away for a week to work on a story without weeping
every night, then drying her eyes and calling Lester to make sure
he was still alive.

“I’m not saying you need to do this to the exclusion of everything
else, or forever\dash{}” \emph{there is no forever for Lester} “\dash{}but you
two would have to be insane not to try it. Look at this board-game
thing you’ve done\dash{}”

“Calvinball,” Perry said.

“Calvinball. Right. You were made for this. You two make each other
better. Perry, let’s be honest here. You don’t have anything better
to do.”

She held her breath. It had been years since she’d spoken to Perry,
years since she’d had the right to say things like that to him.
Once upon a time, she wouldn’t have thought twice, but now\dash{}

“Let me sleep on it,” Perry said.

Which meant no, of course. Perry didn’t sleep on things. He decided
to do things. Sometimes he decided wrong, but he’d never had
trouble deciding.

That night, Lester rubbed her back, the way he always did when she
came back from the road, using the hand-cream she kept on her
end-table. His hands had once been so \emph{strong}, mechanic’s
hands, stubby-fingered pistons he could drive tirelessly into the
knots in her back. Now they smoothed and petted, a rub, not a
massage. Every time she came home, it was gentler, somehow more
loving. But she missed her massages. Sometimes she thought she
should tell him not to bother anymore, but she was afraid of what
it would mean to end this ritual\dash{}and how many more rituals would
end in its wake.

It was the briefest backrub yet and then he slid under the covers
with her. She held him for a long time, spooning him from behind,
her face in the nape of his neck, kissing his collar bone the way
he liked, and he moaned softly.

“I love you, Suzanne,” he said.

“What brought that on?”

“It’s just good to have you home,” he said.

“You seem to have been taking pretty good care of yourself while I
was away, getting in some Perry time.”

“I took him to Musso and Frank,” he said. “I ate like a pig.”

“And you paid the price, didn’t you?”

“Yeah. For days.”

“Serves you right. That Perry is \emph{such} a bad influence on my
boy.”

“I’ll miss him.”

“You think he’ll go, then?”

“You know he will.”

“Oh, honey.”

“Some wounds don’t heal,” he said. “I guess.”

“I’m sure it’s not that,” Suzanne said. “He loves you. I bet this
is the best week he’s had in years.”

“So why wouldn’t he want to stay?” Lester’s voice came out in the
petulant near-sob she had only ever heard when he was in extreme
physical pain. It was a voice she heard more and more often
lately.

“Maybe he’s just afraid of himself. He’s been on the run for a long
time. You have to ask yourself, what’s he running from? It seems to
me that he’s spent his whole life trying to avoid having to look
himself in the eye.”

Lester sighed and she squeezed him tight. “How’d we get so screwed
up?”

“Oh, baby,” she said, “we’re not screwed up. We’re just people who
want to do things, big things. Any time you want to make a
difference, you face the possibility that you’ll, you know, make a
difference. It’s a consequence of doing things with consequences.”

“Gak,” he said. “You always get so Zen-koan when you’re on the
road.”

“Gives me time to reflect. Were you reading?”

“Was I reading? Suzanne, I read your posts whenever I feel lonely.
It’s kind of like having you home with me.”

“You’re sweet.”

“Did you really eat sardines on sorbet toast?”

“Don’t knock it. It’s better than it sounds. Lots better.”

“You can keep it.”

“Listen to Mr Musso and Frank\dash{}boy, you’ve got no business
criticizing anyone else’s food choices.”

He heaved a happy sigh. “I love you, Suzanne Church.”

“You’re a good man, Lester Banks.”

\begin{center}\rule{3in}{0.4pt}\end{center}

Perry met them at the breakfast table the next morning as Suzanne
was fiddling with the espresso machine, steaming soy milk for her
latte. He wore a pair of Lester’s sloppy drawstring pants and a
t-shirt for a motorcycle shop in Kansas City that was spotted with
old motor-oil stains.

“Bom dia,” he said, and chucked Lester on the shoulder. He was
carrying himself with a certain stiffness, and Suzanne thought,
\emph{Here it comes; he’s going to say goodbye. Perry Gibbons, you bastard.}

“Morning,” Lester said, brittle and chipper.

Perry dug around on Suzanne’s non-medicated food-shelf for a while
and came up with a bagel for the toaster and a jar of peanut
butter. No one said anything while he dug around for the big bread
knife, found the cutting board, toasted the bagel, spread peanut
butter, and took a bite. Suzanne and Lester just continued to eat,
in uncomfortable silence. \emph{Tell him,} Suzanne urged silently.
\emph{Get it over with, damn you.}

“I’m in,” Perry said, around a mouthful of bagel, looking away.

Suzanne saw that he had purple bags under his eyes, like he hadn’t
slept a wink all night.

“I’m staying. If you’ll have me. Let’s make some stuff.”

He put the bagel down and swallowed. He looked back at Lester and
the two old comrades locked eyes for a long moment.

Lester smiled. “All right!” He danced a shuffling step, mindful of
his sore hips. “All right, buddy, \emph{fuckin’ A}! Yeah!”

Suzanne tried to fade then, to back out of the room and let them do
their thing, but Lester caught her arm and drew her into an
embrace, tugging on her arm with a strength she’d forgotten he
had.

He gave her a hard kiss. “I love you, Suzanne Church,” he said.
“You’re my savior.”

Perry made a happy sound behind her.

“I love you, too, Lester,” she said, squeezing his skinny, brittle
back.

Lester let go of her and she turned to face Perry. Tears pricked
his eyes, and she found that she was crying too. She gave him a
hug, and felt the ways that his body had changed since she’d held
him back in Florida, back in some forgotten time. He was thicker,
but still solid, and he smelled the same. She put her lips close to
his ear and whispered, “You’re a good man, Perry Gibbons.”

\begin{center}\rule{3in}{0.4pt}\end{center}

Lester gave his notice that morning. Though it was 8PM in Tehran
when Lester called, Sammy was at his desk.

“Why are you telling me this, Lester?”

“It says in my contract that I have to give my notice to you,
specifically.”

“Why the hell did I put that there?” Sammy’s voice sounded far
away\dash{}not just in Iran. It sounded like he had travelled through
time, too.

“Politics, I think,” he said.

“Hard to remember. Probably wanted to be sure that someone like
Wiener wouldn’t convince you to quit, switch companies, and hire
you again.”

“Not much risk of that now,” Lester said. “Let’s face it, Sammy, I
don’t actually do anything for the company.”

“Nope. That’s right. We’re not very good at making use of people
like you.”

“Nope.”

“Well, email me your paperwork and I’ll shove it around. How much
notice are you supposed to give?”

“Three months’.”

“Yowch. Whatever. Just pack up and go home. Gardening leave.”

It had been two years since Lester’d had any contact with Sammy,
but it was clear that running Iranian ops had mellowed him out.
Harder to get into trouble with women there, anyway.

“How’s Iran treating you?”

“The Middle East operation is something else, boy. You’d like it
here. The post-war towns all look like your squatter city\dash{}the
craziest buildings you ever saw. They love the DiaBs though\dash{}we get
the most fantastic designs through the fan channels.\ldots{}” He trailed
off. Then, with a note of suspicion: “What are you going to do
now?”

Ah. No sense in faking it. “Perry and I are going to go into
business together. Making kinetic sculptures. Like the old days.”

“No \emph{way}! Perry \emph{Gibbons}? You two are back together?
Christ, we’re all doomed.” He was laughing. “Sculptures\dash{}like that
toast robot? And he wants to go into \emph{business}? I thought he
was some kind of Commie.”

Lester had a rush of remembrance, the emotional memory of how much
he’d hated this man and everything he stood for. What had happened
to him over the years that he counted this sneak, this thug, as his
colleague? What had he sold when he sold out?

“Perry Gibbons,” Lester said, and drew in a breath. “Perry Gibbons
is the sharpest entrepreneur I’ve ever met. He can’t \emph{help}
but make businesses. He’s an artist who anticipates the market a
year ahead of the curve. He could be a rich man a hundred times
over if he chose. Commie? Page, you’re not fit to keep his books.”

The line went quiet, the eerie silence of a net-connection with no
packets routing on it. “Goodbye, Lester,” Sammy said at length.

Lester wanted to apologize. He wanted \emph{not} to want to
apologize. He swallowed the apology and disconnected the line.

\begin{center}\rule{3in}{0.4pt}\end{center}

When it was time for bed, Suzanne shut her lid and put the computer
down beside the sofa. She stepped carefully around the pieces of
the Calvinball game that still covered the living room floor and
stepped into a pair of slippers. She slid open the back door and
hit the switch for the yard’s flood-light. The last thing she
wanted to do was trip into the pool.

She picked her way carefully down the flagstones that led to the
workshop, where the lights burned merrily in the night. There was
no moon tonight, and the stars were laid out like a bag of
synthetic diamonds arrayed on a piece of black velour in a street
market stall.

She peered through the window before she went around to the door,
the journalist in her wanting to fix an image of the moment in her
mind before she moved in and disturbed it. That was the problem
with being a reporter\dash{}everything changed the instant you started
reporting on it. By now, there wasn’t a person alive who didn’t
know what it means to be in the presence of a reporter. She was a
roving Panopticon.

The scene inside the workshop was eerie. Perry and Lester stood
next to each other, cheek by jowl, hunched over something on the
workbench. Perry had a computer open in front of him, and he was
typing, Lester holding something out of sight.

How many times had she seen this tableau? How many afternoons had
she spent in the workshop in Florida, watching them hack a robot,
build a sculpture, turn out the latest toy for Tjan’s amusement,
Kettlewell’s enrichment? The postures were identical\dash{}though their
bodies had changed, the hair thinner and grayer. Like someone had
frozen one of those innocent moments in time for a decade, then
retouched it with wizening makeup and hair-dye.

She must have made a noise, because Lester looked up\dash{}or maybe it
was just the uncanny, semi-psychic bond between an old married
couple. He grinned at her like he was ten years old and she grinned
back and went around to the door.

“Hello, boys,” she said. They straightened up, both of them
unconsciously cradling their low backs, and she suppressed a grin.
\emph{My little boys, all grown up}.

“Darling!” Lester said. “Come here, have a look!”

He put his arm over her shoulders and walked her to the bench,
leaning on her a little.

It was in pieces, but she could see where it was going: a pair of
familiar boxy shapes, two of Lester’s mechanical computers, their
cola-can registers spilling away in a long daisy-chain of
worm-gears and rotating shafts. One figure was big and
round-shouldered like a vintage refrigerator. The other was
cockeyed, half its gears set higher than the other half. Each had a
single, stark mechanical arm extended before it, and at the end of
each arm was a familiar cracked and fragrant baseball glove.

Lester put a ball into one of the gloves and Perry hammered away at
the keyboard. Very, very slowly, the slope-shouldered robot drew
its mechanical arm back\dash{}“We used one of the open-source prosthestic
plans,” Lester whispered in the tense moment. Then it lobbed a soft
underhand toss to the lopsided one.

The ball arced through the air and the other bot repositioned its
arm in a series of clattering jerks. It seemed to Suzanne that the
ball would miss the glove and bounce off of the robot’s carapace,
and she winced. Then, at the very last second, the robot
repositioned its arm with one more fast jerk, and the ball fell
into the pocket.

A moment later, the lopsided bot\dash{}Perry, it was Perry, that was easy
to see\dash{}tossed the ball to the round-shouldered one, who was clearly
her Lester, as she’d first known him. Lester-bot caught the ball
with a similar series of jerks and returned the volley.

It was magic to watch the robots play their game of catch. Suzanne
was mesmerized, mouth open. Lester squeezed her shoulder with
uncontained excitement.

The Lester-bot lobbed one to Perry-bot, but Perry-bot flubbed the
toss. The ball made a resounding gong sound as it bounced off of
Perry-bot’s carapace, and Perry-bot wobbled.

Suzanne winced, but Lester and Perry both dissolved in gales of
laughter. She watched the Perry-bot try to get itself re-oriented,
aligning its torso to face Lester-bot and she saw that it
\emph{was} funny, very funny, like a particularly great cartoon.

“They do that on purpose?”

“Not exactly\dash{}but there’s no way they’re going to be perfect, so we
built in a bunch of stuff that would make it funnier when it
happened. It is now officially a feature, not a bug.” Perry glowed
with pride.

“Isn’t it bad for them to get beaned with a baseball?” she asked as
Lester carefully handed the ball to Perry-bot, who lobbed it to
Lester-bot again.

“Well, yeah. But it’s kind of an artistic statement,” Perry said,
looking away from them both. “About the way that friendships always
wear you down, like upper and lower molars grinding away at each
other.”

Lester squeezed her again. “Over time, they’ll knock each other
apart.”

Tears pricked at Suzanne’s eyes. She blinked them away. “Guys, this
is great.” Her voice cracked, but she didn’t care. Lester squeezed
her tighter.

“Come to bed soon, hon,” she said to Lester. “I’m going away again
tomorrow afternoon\dash{}New York, a restaurant opening.”

“I’ll be right up,” Lester said, and kissed the top of her head.
She’d forgotten that he was that tall. He didn’t stand all the way
up.

She went to bed, but she couldn’t sleep. She crossed to the window
and drew back the curtain and looked out at the backyard\dash{}the scummy
swimming pool she kept forgetting to do something about, the heavy
grapefruit and lemon trees, the shed. Perry stood on the shed’s
stoop, looking up at the night sky. She pulled the curtains around
herself an instant before he looked up at her.

Their eyes met and he nodded slowly.

“Thank you,” she mouthed silently.

He blew her a kiss, stuck out a foot, and then bowed slightly over
his outstretched leg.

She let the curtain fall back into place and went back to bed.
Lester climbed into bed with her a few minutes later and spooned up
against her back, his face buried in her neck.

She fell asleep almost instantly.

\chapter{Acknowledgements}

Thanks to Andrew Leonard and Salon for publishing this when it was
\emph{Themepunks}.

Thanks to Patrick Nielsen Hayden, Irene Gallo, Pablo Defendini,
Justin Golenbock, Liz Gorinksy, Tom Doherty and the many wonderful
people at Tor for their good work putting this book into the
world.

Likewise thanks to Sarah Hodgson, Alice Moss and Victoria Barnsley
at HarperCollins for making this book happen in the UK.

Thanks to my agents, Russell Galen, Danny Baror and Justin Manask.

Thanks to my mother, Dr Roslyn Doctorow, who remains the sharpest
proofer in the business.

Thanks to my business partners at Boing Boing, the staff of MAKE:
Magazine, and to all the makers who let me hold their skateboards
while they welded the killer robots.

And thanks, of course, to Alice and Poesy, who are the reason for
all of it.
\end{document}
