\hyphenation{mo-no-poly car-ne-gie pro-ject pro-gress mo-dem rou-lette
  browse-wrap Use-net mon-as-tery mo-dems}
\hyphenation{co-me-dic polt-roon stove-pipe Ma-dame scru-ta-ble star-tling}
\hyphenation{heal-thily lim-ou-sines wrest-lers tan-trum push-over un-asked
  bras-siere bro-th-er}
\hyphenation{Can-a-da Fred-rick teen-agers wrest-ler Cha-vez Tho-mas 
  a-nom-a-lies sur-veil-lance ar-mies ref-u-gee ref-u-gees bris-tling
  eve-ning man-chu-ria man-chu-ri-an mid-terms me-di-um jap-a-nese}
\hyphenation{spend-ers googl-ing tour-ist tour-ists leg-end-ary}
\hyphenation{Dan-iel Van-essa Doc-to-row Ste-phen-son}
\hyphenation{de-cade sur-veilled rout-ers Wol-fen-stein teen-ager to-night}
\hyphenation{his-to-gram an-o-nym-ize Ga-la-xy sym-pa-the-tic}
\hyphenation{ar-phid ar-phids}


\begin{document}
%\setlength{\emergencystretch}{1ex}
\raggedbottom

\begin{center}
\textbf{\huge\textsf{Other People's Money}}

\medskip
Cory Doctorow

\end{center}

\bigskip

\begin{flushleft}
This story is part of Cory Doctorow’s short story collection
“With a Little Help” published by himself. It is licensed under a
\href{http://creativecommons.org/licenses/by-nc-sa/}
{Creative Commons Attribution-NonCommercial-ShareAlike 3.0} license.

\bigskip

The whole volume is available at:
\texttt{http://craphound.com/walh/}

\medskip

The volume has been split into individual stories for the purpose of the
\href{http://ccbib.org}{Creative Commons Bibliothek.}
The introduction and similar accompanying texts are available under the 
title:
\end{flushleft}
\begin{center}
With a Little Help -- Extra Stuff
\end{center}

\newpage

\section{Other People's Money}

Gretl's stall in the dead WalMart off the I-5 in Pico Rivera was not 
the busiest spot in the place, but that was how she liked it. Time to 
think was critical to her brand of functional sculpture, and reflection 
was the scarcest commodity of all in 2027.

Which is why she was hoping that the venture capitalist would just 
leave her alone. He wasn't a paying customer, he wasn't a fellow artist 
-- he wanted to \emph{buy} her, and he was thirty years too late.

“You know, I pitched you guys in 1999. On Sand Hill Road. One of the 
founding partners. Kleiner, I think. The guy ate a salad all through my 
slide-deck. When I was done, he wiped his mouth, looked over my 
shoulder, and told me he didn't think I'd scale. That was it. He didn't 
even pick up my business card. When I looked back as I was going out 
the door, I saw him sweep it into the trash with the wrapper from his 
sandwich.”

The VC -- young, with the waxy, sweaty look of someone who ate a lot of 
GM yogurt to try to patch his biochemistry -- shook his head. “That 
wasn't us. We're a franchise -- based here in LA. I just opened up the 
Inglewood branch. But I can see how that would have soured you on us. 
Did you ever get your VC?”

Gretl tossed her tablet with a crash on top of an overflowing barrel of 
primo plastics and wiped her hands on the cunningly stitched dress 
quilted from back pockets of vintage bootleg Levis, their frayed, 
misspelled red tags on proud display. “Son, that was 1999. Within a 
year, VCs weren't writing term-sheets. They were doing cram-downs on 
anything halfway decent in their portfolios, forcing out the founders, 
trying to flip them before the market cratered. But it wasn't that 
pitch that soured me on Sand Hill Road --”

“We're in Inglewood.”

“Yes, you said.” What the hell, it was Wednesday and she had all 
her week's commissions done already. The VC was at least pretty, if you 
liked them young. He had good teeth -- they all had good teeth now -- 
and a cute bump in the bridge of his nose that spoke of a little bit of 
brawling before his B-school days. “OK, here's the thing. I had 
running code, a half-million users. That was big numbers then. We did 
moderation matching -- a heuristic that figured out whether a message 
on a message board was flamebait, flagging up the worst offenders to 
volunteers who blindly checked each other. The BBC was hand-moderating 
a million message-board posts a \emph{day} back then. We could do 
better. But no one thought we'd scale up -- our customers were little 
guys, hotrodder boards, cooking boards. Most of them were getting 
everything for free in exchange for serving as our `reference 
customers,' which was how all those biz-dev weasels did things back 
then.

“By 2007, we were `Web 2.0.' I mean, we'd been Web 2.0 since Web 0.9, 
but now it seemed like the world was ready for us. All we needed was 
some capital to pay for the features our freeloading reference 
customers wanted. I met every single shitweasel -- excuse me, junior 
analyst -- on Sand Hill and brain dumped. They wrote great reports. We 
got nothing. No one was doing investments then, either: it was all 
acquisition driven. Stupid Sarbanes-Oxley killed IPOs and the VC went 
with it.”

The stall across the way was half the size of hers. The old Shenzen 
couple that ran it were real gnarly, covered in old burn scars from 
working in the plastic tag factory where they'd met. Now they sold 
nostalgic hardware, old working specialty appliances and devices from 
the WTO's heyday. They were highly complementary to Gretl's own 
business, which is why they had such a friendly relationship. The old 
woman, she called herself Chloe, was giving her a little hand-gesture 
that meant, “Do you need help getting rid of this jerk?”

“It's OK,” Gretl said to her, waving. “Want to get lunch in 
twenty minutes?”

The old lady rocked back and forth. “Not nutritionist food,” she 
said. Gretl nodded enthusiastically. Nutritionist food wasn't even food 
-- just nutrients and flavoring. It was 80 percent of the stalls in the 
food-court, since the capital costs of a food printer and feedstock 
were practically nil, and any food hacker could differentiate himself 
by thinking up exotic new texture/\-flavor/\-temperature combos.

“Twenty minutes, Mr VC.”

“Udhay,” he said. “Udhay Gonzales.” He passed her a card, 
laser-etched on a jumbo lima bean. She pocketed it.

“You'd have thought I'd learned my lesson by then, but no, sir. I am 
the original glutton for punishment. After Bubble 2.0, I took my best 
coders, our CFO, and a dozen of our users and did a little health-care 
startup, brokering carbon-neutral medical travel plans to Fortune 500s. 
Today that sounds like old hat, but back then, it was sexy. No one 
seriously believed that we could get out from under the HMOs, but 
between Virgin's cheap bulk-ticket sales and the stellar medical deals 
in Venezuela, Argentina and Cuba, it was the only cost-effective way. 
And once the IWWWW signed up 80 percent of the US workforce through 
World of Starcraft guilds, no employer could afford to skimp on health 
insurance. The word would go out during that night's raids and by the 
morning, you'd have picket lines in front of every branch office.

“We had all the right connections, but by then I was a 40-year old 
woman, and that's as close as you can come to invisible in this society 
without having brown skin or a janitor's uniform. I didn't even get a 
chance to get ignored in the offices. We couldn't even get meetings -- 
not once they found my YASNS profiles and saw what I looked like and 
the codgers in my social network.

“So that's when I threw in the towel. I bought a Dremel tool. Then a 
hot glue gun. Then a CNC lathe. Then a mill. Then I got serious.”

“Well, it seems to have worked out for you.” The VC leaned over the 
display cabinet. She saw his reflection in the clear top. His eyes were 
wide with genuine admiration. OK, OK, she thought. OK, you get another 
five minutes, Udhay Gonzales.

She opened the lid and made fortune-teller passes over her pieces with 
her hands. “Pick them up, that's what they're for.”

He went for the fish first. Its scales were individual slices from the 
skins of old Nokia phones -- back when it was just Nokia, not Marvel 
Comics Mobile -- each articulated on its own little sprig of memory 
wire. The gills were scuffed iPod backings, the logos just recognizable 
under the fog of scratches. The eyes bore HP and Playstation logos, 
respectively, and the lips were made from inner-tube strips that bore 
the smallest recognizable logomarks. As he lifted it, it settled into 
his hand, arching back to find his thumb and palm, nestling in there.

“It'll work like an old-time phone,” she said. “It'll even do a 
little lookup from old-style exchange numbers to different identity 
registers and try to get you a voice-call with someone.”

“Do people really do that?”

“Some do. Most just want it for the object-ness of it. It's got a lot 
of emotion.” The scuffs, that's what did it. They were like stories, 
those scratches, each one a memento mori for some long-dead instant in 
some stranger's life.

He picked up another piece. This one was purely sculptural, made from 
several generations of iPhones, their screens carved into abstract 
shapes and then painted with networked OLEDs that stitched them 
together into a single display. The abstract shapes and colors combined 
with the device's aggressive incursions on your PAN to give the sense 
of holding a vampire, something transgressive and savage. Dangerous. 
“When was the last time you owned a device that felt that 
dangerous?”

“Never!” The VC seemed to surprise himself with his vehemence. He 
fumbled the device, caught it, set it down reverently.

Gretl laughed. “Oh, you can be rougher than that. My little critters 
love adapting to hard circumstances.” She tossed the vampire high in 
the sky, let it come down on the floor, having righted itself in the 
air to take the drop on its armored back. “You can't break it, it's 
made of garbage.”

The VC fondled each of her pieces, making genuine appreciative noises. 
She could tell the difference between the genuine article and the fakes.

“I remember all these things from when I was little,” he said at 
last. “I wanted them all so badly. Each one seemed impossibly 
wonderful and out of reach.”

“Yeah,” she said. “That's what does it, all right. That feeling 
right there. You watched these go from fetish item to six-for-a-buck in 
the blister packs at the pharmacy check-out. This gives them back their 
dignity.”

“Can I ask how many of these you sell?”

“Enough,” she said. “As many as I can make. I mostly do 
commissions, but only with people who come down in person. I won't sell 
online. Getting off email was the best gift I ever gave myself.”

“You are hard to reach,” he said.

“Nope. I'm easy to reach -- you just have to haul ass here to Pico 
Rivera. There's even parking, if you're that kind of pervert.”

“I think I see why you aren't interested in capital,” he said. 
“You can't scale this up -- not with all the money in the world.”

Gretl laughed. “You VCs -- scale, scale, scale! It's all you think 
of. You're wrong, as it turns out. This business decomposes into four 
elements: materials acquisition, design, fabrication and retail. They 
all scale like crazy.

“Take materials. After the WTO, the Chinese spent 25 years 
brute-forcing the problem-space of all possible 3D plastic objects that 
an American might pay money for. There is no shortage of that stuff -- 
most of it is sitting in international waters somewhere on a container 
ship, waiting for someone to pay the carbon taxes to land it somewhere. 
I can bring in all the junk electronics and chassis and parts that I 
want, and I print the actuators, controllers, wires and the rest of it 
here.

“Design? Design's easy. Roll the parts through the tumbler and let 
each one get scanned up good. Then run the evolutionary algorithm to 
see how they can fit together. I just watch it, tweaking it, culling 
the ugly mutants, cultivating the pretty ones. I can do fifty original 
designs in a day, and by the time I'm done with any random container, 
I'll have used up more than 80 percent of its payload. The rest goes to 
some feedstockers to be eaten by bacteria.

“Manufacturing -- that's just monkey labor. Easy. Every kid takes 
shop class nowadays, especially the girls.”

“I made cars for my parents' anniversary,” he said.

“Fuel-cell?”

He snorted. “No one wants to drive a truck anymore. Sub-micro solar. 
Fast little things.” He picked up the fish again. “And retail, 
that's just you, here. So if you could scale up, why don't you?”

“Why should I? I'm making incredible money now. I could stand to 
double my operation, but for that I'd need, what, 60 grand? What's the 
smallest angel round you do at your franchise?”

“We're very nimble.”

“How nimble?”

He mumbled something.

“Speak up!”

“Three hundred kay,” he said, blushing. “But it doesn't have to 
be all to you. We could roll your round up with five or six similar 
firms--”

“And increase my communications and bureaucracy overhead by 3,000 
percent. Yeah, that sounds \emph{swell.} I net enough after expenses 
that I could double every quarter if I wanted to. But I'm growing 
organically, cherry-picking my best contractors and getting them on the 
payroll, expanding poco a poco. I'm sixty years old, Mr Gonzales, and I 
don't need to grow like a tumor anymore.”

He put the fish back down. It flopped.

“You say you're nimble. But from where I sit, you're not nimble 
enough. You're starting off in the 300 grand range, and you're probably 
averaging a million in your angel round, ten or twenty for Series A, 
seventy for Series B. I can turn 60 grand into 600 in six months. 
That's pretty good for me, as an individual. But I can't turn your 
million into ten million -- not in six years. What does your franchise 
have under management?”

“We're a gigafund,” he said. He managed to make it sound like a 
boast.

She shook her head. “You poor, poor boy. How are you going to spend a 
billion dollars in \$300,000 increments? You'll be sitting on three 
quarters of that by the time you cash out the fund.”

“It's the smallest amount that a franchisee can take,” he said.

“Well, sure. The parent company's got what, half a trillion under 
management? Don't look so surprised. Yes, I keep up to date on the 
shenanigans you Mighty Morphin' Power Brokers get up to in Silly 
Valley. No \emph{wonder} they're franchising! But the secret is, big 
money is dumb money. I can spend a hundred bucks so smart that I turn 
it into fifteen hundred. You look like a smart kid, you could probably 
make a thousand. But you'll never do the same trick with your billion 
in other people's money. Whoever sold you that franchise conned you, 
sonny.”

He looked glum.

“Oh, cheer up,” she said. “You're a young man. Getting shafted by 
VCs builds character. Look at me!”

He picked up the fish again. She knew what he was going to ask without 
having to wait. She named the price. “But for you, a ten percent 
discount.”

He shook his head and put it back. “I can't afford that,” he said.

“What are you doing tonight?”

He cocked an eyebrow at her. “Don't worry, I'm not interested in your 
youthful limbs. I just have a spot on my third shift. One of my girls 
is pregnant and she's taking some maternity. You pull six hours 
starting at 11PM and you can take that home.”

“I'm not supposed to moonlight.” He caressed the fish's scales. 
They rippled under his finger.

“It's due diligence,” she said.

He smiled. He was very pretty. And he'd built two cars -- not bad. He'd 
do OK. Maybe he'd even work out and end up one of her regulars.

“Think about it. I close down at 6PM. You come by then, if you're 
interested, and I'll give you the details for the fabrica.”

She locked her cabinets and set out her “Gone to lunch” sign, then 
hopped over the display case, vaulting it the way she'd learned to do 
in yogacrobatics class in Silver Lake.

“Lunch time?”

Mrs. Huang called to one of her daughters to come out and staff the 
booth, then came around on her cane.

“No nutritionist food,” she said.

“Certainly not,” Gretl said, sprinkling a wave at the VC as he 
moved off among the stalls in the dead WalMart.

\section{Afterword}

I have an odd and productive relationship with \emph{Forbes} magazine. 
I'm far from a typical \emph{Forbes} reader, but they've commissioned 
several articles and this short story from me, and the commissions are 
always challenging and just weird enough to inspire. Here, the brief 
was to write about the future of entrepreneurship. I'd been thinking a 
lot about how \emph{little} it costs to start a business, and how 
predatory and awful many of the investors I'd met were, and I came up 
with this -- a Socratic dialog between a startupist and a VC who can't 
find anyone to take his money.

\end{document}
