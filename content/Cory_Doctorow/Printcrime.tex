\hyphenation{mo-no-poly car-ne-gie pro-ject pro-gress mo-dem rou-lette
  browse-wrap Use-net mon-as-tery mo-dems}
\hyphenation{co-me-dic polt-roon stove-pipe Ma-dame scru-ta-ble star-tling}
\hyphenation{heal-thily lim-ou-sines wrest-lers tan-trum push-over un-asked
  bras-siere bro-th-er}
\hyphenation{Can-a-da Fred-rick teen-agers wrest-ler Cha-vez Tho-mas 
  a-nom-a-lies sur-veil-lance ar-mies ref-u-gee ref-u-gees bris-tling
  eve-ning man-chu-ria man-chu-ri-an mid-terms me-di-um jap-a-nese}
\hyphenation{spend-ers googl-ing tour-ist tour-ists leg-end-ary}
\hyphenation{Dan-iel Van-essa Doc-to-row Ste-phen-son}
\hyphenation{de-cade sur-veilled rout-ers Wol-fen-stein teen-ager to-night}
\hyphenation{his-to-gram an-o-nym-ize Ga-la-xy sym-pa-the-tic}
\hyphenation{ar-phid ar-phids}


\begin{document}

\raggedbottom

\begin{center}
\textbf{\huge\textsf{{Printcrime}}}
\end{center}

%\setlength{\emergencystretch}{1ex}

\emph{Forematter:}

This story is part of Cory Doctorow’s 2007 short story collection
“Overclocked: Stories of the Future Present,” published by
Thunder’s Mouth, a division of Avalon Books. It is licensed under a
Creative Commons Attribution-NonCommercial-ShareAlike 2.5 license,
about which you’ll find more at the end of this file.

This story and the other stories in the volume are available at:

\texttt{http://craphound.com/overclocked}

You can buy Overclocked at finer bookstores everywhere, including
\href{http://www.amazon.com/exec/obidos/ASIN/1560259817/downandoutint-20}{Amazon.}

In the words of Woody Guthrie:

“This song is Copyrighted in U.S., under Seal of Copyright
\#154085, for a period of 28 years, and anybody caught singin it
without our permission, will be mighty good friends of ourn, cause
we don’t give a dern. Publish it. Write it. Sing it. Swing to it.
Yodel it. We wrote it, that’s all we wanted to do.”

Overclocked is dedicated to Pat York, who made my stories better.

\section{Introduction to Printcrime:}

Printcrime came out of a discussion I had with a friend who’d been
to hear a spokesman for the British recording industry talk about
the future of “intellectual property.” The record exec opined the
recording industry’s great and hysterical spasm would form the
template for a never-ending series of spasms as 3D printers,
fabricators and rapid prototypers laid waste to every industry that
relied on trademarks or patents.

My friend thought that, as kinky as this was, it did show a fair
amount of foresight, coming as it did from the notoriously
technosqueamish record industry.

I was less impressed.

It’s almost certainly true that control over the production of
trademarked and patented objects will diminish over the coming
years of object-on-demand printing, but to focus on 3D printers’
impact on \emph{trademarks} is a stupendously weird idea.

It’s as if the railroad were looming on the horizon, and the most
visionary thing the futurists of the day can think of to say about
it is that these iron horses will have a disastrous effect on the
hardworking manufacturers of oat-bags for horses. It’s true, as far
as it goes, but it’s so tunnel-visioned as to be practically
blind.

When Nature magazine asked me if I’d write a short-short story for
their back-page, I told them I’d do it, then went home, sat down on
the bed and banged this one out. They bought it the next morning,
and we were in business.

\section{Printcrime}

\textsf{(Originally published in Nature Magazine, January 2006)}

The coppers smashed my father’s printer when I was eight. I
remember the hot, cling-film-in-a-microwave smell of it, and Da’s
look of ferocious concentration as he filled it with fresh goop,
and the warm, fresh-baked feel of the objects that came out of it.

The coppers came through the door with truncheons swinging, one of
them reciting the terms of the warrant through a bullhorn. One of
Da’s customers had shopped him. The ipolice paid in high-grade
pharmaceuticals\dash{}performance enhancers, memory supplements,
metabolic boosters. The kind of thing that cost a fortune over the
counter; the kind of thing you could print at home, if you didn’t
mind the risk of having your kitchen filled with a sudden crush of
big, beefy bodies, hard truncheons whistling through the air,
smashing anyone and anything that got in the way.

They destroyed grandma’s trunk, the one she’d brought from the old
country. They smashed our little refrigerator and the purifier unit
over the window. My tweetybird escaped death by hiding in a corner
of his cage as a big, booted foot crushed most of it into a sad
tangle of printer-wire.

Da. What they did to him. When he was done, he looked like he’d
been brawling with an entire rugby side. They brought him out the
door and let the newsies get a good look at him as they tossed him
in the car, while a spokesman told the world that my Da’s
organized-crime bootlegging operation had been responsible for at
least twenty million in contraband, and that my Da, the desperate
villain, had resisted arrest.

I saw it all from my phone, in the remains of the sitting room,
watching it on the screen and wondering how, just \emph{how} anyone
could look at our little flat and our terrible, manky estate and
mistake it for the home of an organized crime kingpin. They took
the printer away, of course, and displayed it like a trophy for the
newsies. Its little shrine in the kitchenette seemed horribly
empty. When I roused myself and picked up the flat and rescued my
peeping poor tweetybird, I put a blender there. It was made out of
printed parts, so it would only last a month before I’d need to
print new bearings and other moving parts. Back then, I could take
apart and reassemble anything that could be printed.

By the time I turned eighteen, they were ready to let Da out of
prison. I’d visited him three times\dash{}on my tenth birthday, on his
fiftieth, and when Ma died. It had been two years since I’d last
seen him and he was in bad shape. A prison fight had left him with
a limp, and he looked over his shoulder so often it was like he had
a tic. I was embarrassed when the minicab dropped us off in front
of the estate, and tried to keep my distance from this ruined,
limping skeleton as we went inside and up the stairs.

“Lanie,” he said, as he sat me down. “You’re a smart girl, I know
that. Trig. You wouldn’t know where your old Da could get a printer
and some goop?”

I squeezed my hands into fists so tight my fingernails cut into my
palms. I closed my eyes. “You’ve been in prison for ten years, Da.
Ten. Years. You’re going to risk another ten years to print out
more blenders and pharma, more laptops and designer hats?”

He grinned. “I’m not stupid, Lanie. I’ve learned my lesson. There’s
no hat or laptop that’s worth going to jail for. I’m not going to
print none of that rubbish, never again.” He had a cup of tea, and
he drank it now like it was whisky, a sip and then a long,
satisfied exhalation. He closed his eyes and leaned back in his
chair.

“Come here, Lanie, let me whisper in your ear. Let me tell you the
thing that I decided while I spent ten years in lockup. Come here
and listen to your stupid Da.”

I felt a guilty pang about ticking him off. He was off his rocker,
that much was clear. God knew what he went through in prison.
“What, Da?” I said, leaning in close.

“Lanie, I’m going to print more printers. Lots more printers. One
for everyone. That’s worth going to jail for. That’s worth
anything.”

\section{Creative Commons License Deed}

Attribution-NonCommercial-ShareAlike 2.5

You are free:

* to Share\dash{}to copy, distribute, display, and perform the work

* to Remix\dash{}to make derivative works

Under the following conditions:

* Attribution. You must attribute the work in the manner specified
by the author or licensor.

* Noncommercial. You may not use this work for commercial
purposes.

* Share Alike. If you alter, transform, or build upon this work,
you may distribute the resulting work only under a license
identical to this one.



* For any reuse or distribution, you must make clear to others the
license terms of this work.

* Any of these conditions can be waived if you get permission from
the copyright holder.

Disclaimer: Your fair use and other rights are in no way affected
by the above.

This is a human-readable summary of the Legal Code (the full
license):

http://creativecommons.org/licenses/by-nc-sa/2.5/legalcode

\end{document}
