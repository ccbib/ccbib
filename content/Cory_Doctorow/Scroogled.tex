\hyphenation{mo-no-poly car-ne-gie pro-ject pro-gress mo-dem rou-lette
  browse-wrap Use-net mon-as-tery mo-dems}
\hyphenation{co-me-dic polt-roon stove-pipe Ma-dame scru-ta-ble star-tling}
\hyphenation{heal-thily lim-ou-sines wrest-lers tan-trum push-over un-asked
  bras-siere bro-th-er}
\hyphenation{Can-a-da Fred-rick teen-agers wrest-ler Cha-vez Tho-mas 
  a-nom-a-lies sur-veil-lance ar-mies ref-u-gee ref-u-gees bris-tling
  eve-ning man-chu-ria man-chu-ri-an mid-terms me-di-um jap-a-nese}
\hyphenation{spend-ers googl-ing tour-ist tour-ists leg-end-ary}
\hyphenation{Dan-iel Van-essa Doc-to-row Ste-phen-son}
\hyphenation{de-cade sur-veilled rout-ers Wol-fen-stein teen-ager to-night}
\hyphenation{his-to-gram an-o-nym-ize Ga-la-xy sym-pa-the-tic}
\hyphenation{ar-phid ar-phids Found-ers}
\hyphenation{stran-ger stran-gers shoul-der-blades dump-ling dump-lings}
\hyphenation{ice-pack guard-rail Sep-tem-ber boot-able e-co-nom-ist}
\hyphenation{grown-ups roos-ter shoe-laces li-quid-i-ty}
\hyphenation{side-arm}
\hyphenation{wo-man wo-men tan-trum tan-trums Le-nin-grad zom-bie bunk-house}
\hyphenation{up-tick bio-mass}
\hyphenation{of-fi-cial of-fi-cial-ly gov-ern-ment}
\hyphenation{heal-thy Or-ville spark-ling}
\hyphenation{ves-ti-bule Law-rence au-to-no-mous}
\hyphenation{sau-sage door-step staf-fer}
\hyphenation{tree-trunk}
\hyphenation{to-ron-to}
\hyphenation{qua-dril-lion-aire qua-dril-lion-aires}
\hyphenation{sports-jack-et sports-jack-ets}
\hyphenation{work-space skunk-works}
\hyphenation{kings-ton}


\begin{document}
%\setlength{\emergencystretch}{1ex}
\raggedbottom

\begin{center}
\textbf{\huge\textsf{Scroogled}}

\medskip
Cory Doctorow

\end{center}

\bigskip

\begin{flushleft}
This story is part of Cory Doctorow’s short story collection
“With a Little Help” published by himself. It is licensed under a
\href{http://creativecommons.org/licenses/by-nc-sa/}
{Creative Commons Attribution-NonCommercial-ShareAlike 3.0} license.

\bigskip

The whole volume is available at:
\texttt{http://craphound.com/walh/}

\medskip

The volume has been split into individual stories for the purpose of the
\href{http://ccbib.org}{Creative Commons Bibliothek.}
The introduction and similar accompanying texts are available under the 
title:
\end{flushleft}
\begin{center}
With a Little Help -- Extra Stuff
\end{center}

\newpage

\section{Scroogled}

“Give me six lines written by the most honorable of men, and I will 
find an excuse in them to hang him.” - Cardinal Richelieu

Greg landed at SFO at 8PM, but by the time he made it to the front of 
the customs line it was after midnight. He had it good -- he'd been in 
first class, first off the plane, brown as a nut and loose-limbed after 
a month on the beach at Cabo, SCUBA diving three days a week, bumming 
around and flirting with French college girls the rest of the time. 
When he'd left San Francisco a month before, he'd been a 
stoop-shouldered, pot-bellied wreck -- now he was a bronze god, drawing 
appreciative looks from the stews at the front of the plane.

In the four hours he spent in the customs line, he fell from god back 
to man. His warm buzz wore off, the sweat ran down the crack of his 
ass, and his shoulders and neck grew so tense that his upper back felt 
like a tennis racket. The batteries on his iPod died after the third 
hour, leaving him with nothing to do except eavesdrop on the 
middle-aged couple ahead of him.

“They've starting googling us at the border,” she said. “I told 
you they'd do it.”

“I thought that didn't start until next month?” The man had brought 
a huge sombrero on board, carefully stowing it in its own overhead 
locker, and now he was stuck alternately wearing it and holding it.

Googling at the border. Christ. Greg vested out from Google six months 
before, cashing in his options and “taking some me time,” which 
turned out to be harder than he expected. Five months later, what he'd 
mostly done is fix his friends' PCs and websites, and watch daytime TV, 
and gain ten pounds, which he blamed on being at home, instead of in 
the Googleplex, with its excellent 24-hour gym.

The writing had been on the wall. Google had a whole pod of lawyers in 
charge of dealing with the world's governments, and scumbag lobbyists 
on the Hill to try to keep the law from turning them into the world's 
best snitch. It was a losing battle. The US Government had spent \$15 
\emph{billion} on a program to fingerprint and photograph visitors at 
the border, and hadn't caught \emph{a single} terrorist. Clearly, the 
public sector was not equipped to Do Search Right.

The DHS officers had bags under their eyes as they squinted at their 
screens, prodding mistrustfully at their keyboards with sausage 
fingers. No wonder it was taking four hours to get out of the goddamned 
airport.

“Evening,” he said, as he handed the man his sweaty passport. The 
man grunted and swiped it, then stared at his screen, clicking. A lot. 
He had a little bit of dried food in the corner of his mouth and his 
tongue crept out and licked at it as he concentrated.

“Want to tell me about June, 1998?”

Greg turned rotated his head this way and that. “I'm sorry?”

“You posted a message to alt.burningman on June 17, 1998 about your 
plan to attend Burning Man. You posted, `Would taking shrooms be a 
really bad idea?'”

\tb

It was 3AM before they let him out of the “secondary screening” 
room. The interrogator was an older man, so skinny he looked like he'd 
been carved out of wood. His questions went a lot further than the 
Burning Man shrooms. They were just the start of Greg's problems.

“I'd like to know more about your hobbies. Are you interested in 
model rocketry?”

“What?”

“Model rocketry.”

“No,” Greg said. “No, I'm not.” Thinking of all the explosives 
that model rocketry people surrounded themselves with.

The man made a note, clicked some more. “You see, I ask because I see 
a heavy spike of ads for model rocketry supplies showing up alongside 
your search results and Google mail.”

Greg felt his guts spasm. “You're looking at my searches and 
email?” He hadn't touched a keyboard in a month, but he knew that 
what you put into the searchbar was more intimate than what you told 
your father-confessor. He'd seen enough queries to know that.

“Calm down, please. No, I'm not looking at your searches.” The man 
made a bitter lemon face and went on in a squeaky voice. “That would 
be \emph{unconstitutional}. You weren't listening to me. We see the 
\emph{ads} that show up when you read your mail and do your searching. 
I have a brochure explaining it, I'll give it to you when we're through 
here.”

“But the ads don't mean \emph{anything} -- I get ads for Ann Coulter 
ringtones whenever I get email from my friend who lives in Coulter, 
Iowa!”

The man nodded. “I understand, sir. And that's just why I'm here 
talking to you, instead of just looking at this screen. Why do you 
suppose model rocket ads show up so frequently for you?”

He thought for a moment. “OK, just do this. Go to Google and search 
for `coffee fanciers', all right?” He'd been very active in the 
group, helping them build out the site for their coffee-of-the-month 
subscription service. The blend they were going to launch with was 
called “Jet Fuel.” “Jet Fuel” and “Launch” -- that'd 
probably make Google barf up model rocket ads. Not that he would know 
-- he blocked all the ads in his browser.

\tb

They were in the home stretch when the carved man found the Hallowe'en 
photos. They were buried three screens deep in the search results for 
“Greg Lupinski,” and Greg hadn't noticed them.

“It was a Gulf War themed party,” he said. “In the Castro.”

“And you're dressed as --?”

“A suicide bomber.” Just saying the words in an airport made him 
nervous, as though uttering them would cause the handcuffs to come out.

“Come with me, Mr Lupinski.”

\tb

The search lasted a long time. They swabbed him in places he didn't 
know he had. He asked about a lawyer. They told him that he could call 
all the lawyers he wanted once he was out of the Customs sterile area.

“Good night, Mr Lupinski.” This was a new interrogator, a man who'd 
wanted to know about the reason that he'd sought both night diving and 
deep diving specialist certification from the PADI instructor in Cabo. 
The guy implied that Greg had been training to be an al-Qaeda frogman, 
and didn't seem to believe that Greg had just wanted to do all the 
certifications he could, pursuing diving the way he pursued everything: 
thoroughly.

But now the man with the frogman fantasy was bidding him a good night 
and releasing him from the secondary screening area. His suitcases 
stood alone by the baggage carousel. When he picked them up, he saw 
that they had been opened and then inexpertly closed. Some of his 
clothes stuck out from around the edges.

At home, he saw that all the fake “pre-Colombian” statues had been 
broken, and that his white cotton Mexican shirt -- folded and fresh 
from his laundry-lady -- had a boot-print in the middle of it. His 
clothes no longer smelled of Mexico. Now they smelled of airports and 
machine oil.

The mailman had dropped an entire milk-crate of mail off at his place 
that day, but he couldn't even begin to confront it. All he could think 
of, as the sun rose over the Mission, turning the Victorian houses they 
called “painted ladies” vivid colors, was what it meant to be 
googled.

He wasn't going to sleep. No \emph{way}. He needed to talk about this. 
And there was only one person who he could talk to, and luckily, she 
was usually awake around now.

\tb

Maya had started at Google two years after him, but had gotten a much 
bigger grant of stock than he had. She knew exactly what she was going 
to do with it, too, once she vested: take her dogs and her girlfriend 
and head to Florence, for good. Learn Italian, take in the museums, sit 
in the cafes. It was she who'd convinced him to go to Mexico: anywhere, 
she said, anywhere that he could reboot his existence.

Maya had two giant chocolate Labs and a very, very patient girlfriend 
who'd put up with anything except being dragged around Dolores Park at 
6AM by 350 pounds of drooling brown canine.

She went for her Mace as he jogged towards her, then did a double-take 
and threw her arms open, dropping the leashes and stamping on them with 
one sneaker, a practiced gesture. “Where's the rest of you? Dude, you 
look \emph{hawt}!”

He took the hug, suddenly self-conscious of the way he smelled after a 
night of invasive googling. “Maya,” he said. “Maya, what do you 
know about the DHS?”

She stiffened and the dogs whined. She looked around, then nodded up at 
the tennis courts. “Top of the light standard there, don't look, 
there. That's one of our muni WiFi access points. Wide-angle webcam. 
Face away from it when you talk. Lip-readers.”

He parsed this out slowly. Google's free municipal WiFi program was a 
hit in every city where it played, and in the grand scheme of things, 
it hadn't cost much to put WiFi access points up on light standards and 
other power-ready poles around town. Especially not when measured 
against the ability to serve ads to people based on where they were 
sitting. He hadn't paid much attention when they'd made the webcams on 
all those access points public -- there'd been a day's worth of 
blogstorm while people looked out over their childhood streets or 
patrolled prostitution strolls, fingering johns, but it had blown over.

Now he felt -- \emph{watched}.

Feeling silly, he kept his lips together and mumbled, “You're 
joking.”

“Come with me,” she said, facing squarely away from the pole.

\tb

The dogs weren't happy about having their walks cut short, and they let 
it be known in the kitchen as Maya fixed coffee for them -- barking, 
banging into the table and rocking it. Maya's girlfriend Laurie called 
out from the bedroom and Maya went back to talk to her, then emerged, 
looking flustered.

“It started with China,” she said. “Once we moved our servers 
onto the mainland, they went under Chinese jurisdiction. They could 
google everyone going through our servers.” Greg knew what that 
meant: if you visited a page with Google ads on it, if you used Google 
maps, if you used Google mail -- even if you \emph{sent} mail to a 
gmail account -- Google was collecting your info, forever.

“They were using us to build profiles of people. Not arresting them, 
you understand. But when they had someone they wanted to arrest, they'd 
come to us for a profile and find a reason to bust them. There's hardly 
anything you can do on the net that isn't illegal in China.”

Greg shook his head. “Why did they put the servers in China?”

“The government said they'd block them if they didn't. And Yahoo was 
there.” They both made a face. Somewhere along the way, Google had 
become obsessed with Yahoo, more worried about what the competition was 
doing than how they were performing. “So we did it. But a lot of us 
didn't like the idea.”

She sipped her coffee and lowered her voice. One of the dogs whined. 
“I made it my 20 percent project.” Googlers were supposed to devote 
20 percent of their time to blue-sky projects. “Me and my pod. We 
call it the googlecleaner. It goes deep into the database and 
statistically normalizes you. Your searches, your gmail histograms, 
your browsing patterns. All of it.”

“The search ads?”

“Ah,” she grimaced. “Yes, the DHS. So we brokered a compromise 
with the DHS. They'd stop asking to go fishing in our search records 
and we'd let them see what ads got displayed for you.”

Greg felt sick. “Why? Don't tell me Yahoo was doing it already --”

“No, no. Well, yes. Sure. Yahoo was already doing it. But that wasn't 
it. You know, Republicans \emph{hate} Google. We are overwhelmingly 
registered Democrat. So we're doing what we can to make peace with them 
before they clobber us. This isn't PII --” Personally Identifying 
Information, the toxic smog of the information age “-- it's just 
\emph{metadata}. So it's only slightly evil.”

“If it's all so innocuous, why all this cloak-and-dagger stuff?”

She sighed and hugged the dog that was butting her with his huge, 
anvil-shaped head. “The spooks are like public lice. They get 
everywhere. Once we let them in, everything suddenly got a lot more -- 
secret. Some of our meetings have to have spooks present, it's like 
being in some Soviet ministry, with a political officer always there, 
watching everything. And the security clearance. Now we're divided into 
these two camps: the cleared and the suspect. We all know who isn't 
cleared, but no one knows why. I'm cleared. Lucky me -- being a homo no 
longer disqualifies you for access to seekrit crap. No cleared person 
wants to even eat lunch with an un-clearable. And every now and again, 
one of your teammates will get pulled off your project `for security 
reasons', whatever that means.”

Greg felt very tired. “So now I'm feeling lucky I got out of the 
airport alive. I suppose I might have ended up in Gitmo if it had gone 
badly, huh?”

She was staring at him intently, her eyes flicking from side to side. 
He waited, but she didn't say anything.

“What?”

“What I'm about to tell you, you can't ever repeat it, OK?”

“Um, OK? You're not going to tell me you're a deep-cover Al-Quaeda 
suicide bomber?”

“Nothing so simple. Here's the thing: the airport DHS scrutiny is a 
gating function. It lets the spooks narrow down their search criteria. 
Once you get pulled aside for secondary at the border, you become a 
`person of interest,' and they never, ever let up. They'll check the 
webcams for your face and gait. Read your mail. Log your searches.”

“I thought you said the courts wouldn't let them --”

“The courts won't let them \emph{indiscriminately} google you. But 
once you get into the system, it becomes a \emph{selective} search. All 
legal. And once they start googling you, they \emph{always} find 
something.”

“You mean to say they've got a boiler-room of midwestern housewives 
reading the email of everyone who ever got a second look at the border? 
Sounds like the world's shittiest job.”

“If only. No, this is all untouched by human hands. All your data is 
fed into a big hopper that checks for `suspicious patterns' and 
gradually builds the case against you, using deviation from statistical 
norms to prove that you're guilty of \emph{something}. It's just a 
variation of the way we spot search-spammers” -- the “optimizers” 
who tried to get their Viagra scams and Ponzi schemes to come to the 
top of the search results “-- but instead of lowering your search 
rank, we increase your probability of being sent to Syria. And of 
course, they google all of \emph{us}, everyone who works on anything 
`sensitive.'”

“Naturally,” Greg said. He felt like he was going to throw up. He 
felt like never using a search engine again. “How the hell did this 
\emph{happen}? It's such a \emph{good} place. `Don't be evil,' 
right?” That was the corporate motto, and for Greg, it had been a 
huge part of his reason for taking his fresh-minted computer science 
PhD from Stanford directly to Google.

Maya's laugh was bitter and cynical. “Don't be evil? Come on, Greg. 
Don't you remember what it was like when we started censoring the 
Chinese search results, and we all asked how that could be anything but 
evil? The company line was hilarious: `We're not doing evil -- we're 
giving them access to a better search tool! If we showed them search 
results they couldn't get to, that would just frustrate them. It would 
be a \emph{bad user experience}. If we hadn't lost our don't-be-evil 
cherry by then, we surely did the day we took that one.”

“Now what?” Greg pushed a dog away from him and Maya looked hurt.

“Now you're a person of interest, Greg. Googlestalked. Now, you live 
your life with someone watching over your shoulder, all the time. You 
know the mission statement, right? `Organize all human knowledge.' 
That's \emph{everything}. Give it five years, we'll know how many turds 
were in the bowl before you flushed. Combine that with automated 
suspicion of anyone who matches a statistical picture of a bad guy and 
you're --”

“I'm scroogled.”

“Totally.”

“Thanks, Maya,” he said. “Thanks anyway.”

“Sit down,” she said. The dog that had been bumping at his legs was 
at it again. Maya took both dogs down the hall to the bedroom and he 
heard her muffled argument with her girlfriend. She came back without 
the dogs.

“I can fix this,” she said in a whisper so low it was practically a 
hiss. “I can googleclean you.”

“But you're under constant scrutiny --”

“By DHS agents. Once they fired all non-native-born Americans from 
the DHS, it got a lot fatter and stupider. I can googleclean you, 
Greg.”

“I don't want you to get into trouble.”

She shook her head. “I'm already doomed. I built the googlecleaner. 
Every day since then has been borrowed time -- now it's just a matter 
of waiting for someone to point out my expertise and history to the DHS 
and, oh, I don't know. Whatever it is they do to people like me in the 
War on Abstract Nouns.”

Greg remembered the questioning at the airport. The search. His shirt, 
the bootprint in the middle of it.

“Do it,” he said.

\tb

The ads were weird. He hadn't really paid attention to them in years. 
The blocker got rid of most of them, but Google changed its code often 
enough that their little text ads showed up on a lot of his pages. They 
stayed subliminal mostly -- only clunkers like that Ann Coulter 
ringtone ad made it past his eyes into his brain.

Now the clunkers were everywhere: Intelligent Design Facts, Online 
Seminary Degree, Terror Free Tomorrow, Porn Blocker Software, 
Homosexuality and Satan. He clicked through a couple of these and found 
himself in some kind of alternate universe Internet, full of weird 
opinions about the evils of being gay, the certainty of the young 
Earth, the need for eternal national vigilance.

Then he started to notice something weird about the search results 
themselves. After unpacking his suitcase and opening his mail, he spent 
two weeks sitting at home on his ass, surfing. His pre-Mexico belly was 
reemerging, so he decided to do something about it. No burritos for 
lunch today -- he'd go to that holistic place Maya had told him about. 
Vegan low-fat cuisine couldn't possibly be as gross as it sounded.

“Did you mean `Hungarian Restaurants'?”

He snorted. No, he'd meant “holistic restaurants,” you dumbass 
search-engine. It nagged at him. He pulled up his search history and 
went back through the results, printing out the pages. Then he logged 
out of his Google account and went back through the same searches, 
comparing the results to the logged-in pages. The differences were 
striking. A search for “democratic primary” pointed to anti-Hillary 
rants on angry blogs when he was logged in, and to information on 
volunteering for the DNC when he was logged out. Searching for 
“abortion clinic” while logged out listed the nearest Planned 
Parenthood office; searching while logged in gave him information about 
Campaign Life, ProLife.com, and the ProLife alliance. Good thing he 
wasn't pregnant.

This was Maya's googlecleaner at work. It was like the stories of 
people who asked their TiVos to record an episode of “Queer Eye” 
and then got inundated with suggestions for other “gay shows” -- 
“My TiVo thinks I'm gay,” was the title of one article he 
remembered. Google had been experimenting with “personalized” 
search results before he left the country -- here it was, in all its 
glory.

Google thought he was a conservative Christian Republican who supported 
the War on Terror and many other abstract nouns.

He logged out of Google -- that was simple. Five minutes later, he 
logged in again. His entire address book was in there. He logged out 
again. Logged back in. His calendar -- when was his parents' 
anniversary again? Logged out. Logged back in. Needed his bookmarked 
locations in Maps. Logged out.

He stopped trying. Google was where his friendships lived -- all those 
people he stayed connected to on Orkut. It was where his relationships 
lived: all that archived email, all those addresses in his 
address-book. It was his family photos, his bookmarks. Hell, his search 
history -- his real search history -- was like an outboard brain, 
remembering which parts of the unplumbable Internet he cared about, so 
that he didn't have to remember it the hard way, with the meat in his 
skull.

Google had a copy of him -- all the parts of him that navigated the 
world and the people in it. Google owned that copy, and without it, he 
couldn't be himself anymore. He'd just have to stay logged in.

\tb

Greg mashed the keys on the laptop next to his bed, bringing the screen 
to life. He squinted at the toolbar clock: 4:13AM! Christ, who was 
pounding on his door at this hour?

He shouted “Coming!” in a muzzy voice and pulled on a robe and 
slippers. He shuffled down the hallway, turning on lights as he went, 
squinting. At the door, he squinted through the peephole, peering at -- 
Maya.

He undid the chains and the deadbolt and yanked the door open and Maya 
rushed in past him, followed by the dogs, followed by her girlfriend, 
Laurie, whom he'd last seen at a Christmas party at Google, in a 
fabulous cocktail dress and an elaborate up-do. Now she was wearing a 
freebie Google Summer of Code sweatshirt, jeans, and a frown that 
started between her eyebrows and intensified all the way down her face.

Maya was sheened with sweat, her hair sticking to her forehead. She 
scrubbed at her eyes, which were red and lined.

“Pack a bag,” she said, in a hoarse croak.

“What?”

“Whatever you can't live without. A couple changes of clothes. 
Anything you're sentimental about -- shoebox of pictures, your 
grandfather's razor, whatever. But keep it small, something you can 
carry. We're traveling light.”

“Maya, what are you --”

She took him by the shoulders. “Do. It,” she said. “Don't ask 
questions right now. There's no time.”

“Where do you want to --”

“Mexico, probably. Don't know yet. \emph{Pack}, dammit.” She pushed 
past him into his bedroom and started yanking open drawers.

“Maya,” he said, sharply, “I'm not going anywhere until you tell 
me what's going on.”

She glared at him and pushed her hair away from her face. “The 
googlecleaner lives. I shut it down, walked away from it, after I did 
you. It was too dangerous to use anymore. But I still get buginizer 
notifications when new bugs get filed against it, I'm still in B as the 
project's owner. Someone filed eight bugs against it this week. 
Someone's used it six times to smear six very specific accounts.”

“Who's using it?”

“Well, I'll give you a hint. Let me tell you who's been cleaned this 
week --” She listed six candidates, four Republican and two Democrat, 
who were all in the running for the primaries.

“Googlers are blackwashing political candidates?”

“Not Googlers. This is all coming from offsite. The IP block is 
registered in DC. And the IPs are all also used by Gmail users. And 
those Gmail users --”

“You spied on gmail accounts?”

“I'm leaving in two minutes, with or without you. You can interrupt 
me to ask me questions, or you can listen.” She gave him another 
look. Laurie stood in the door of the bedroom, holding the dogs by the 
collars and looking down at the floor.

“Good. OK. Yes. I did spy on their email. Of course I did. Everyone 
does it, now and again, and for a lot worse reasons that this.

“It's our lobbying firm. The ones who invented the Swift Boat 
Veterans for Truth. Remember them? It was a stink when we hired them, 
but Google couldn't afford to be `that company full of registered 
Democrats' forever. We needed friends in Congress. These guys could do 
it for us.”

“But they're ruining politicians' careers!”

“Yeah. They certainly are. And who benefits when they do that?”

Laurie spoke, at last. “Other politicians.”

He felt his pulse beating in his temples. “We should tell someone.”

“Yeah,” Maya said. “How? They know everything about us. They can 
see every search. Every email. Every time we've been caught on the 
webcams. Who is in our social network -- you know that if you've got 
more than fifteen Orkut buddies, it's statistically certain that you're 
no more than three steps to someone who's contributed money to a 
`terrorist' cause? Remember the airport? Imagine a lot more of that.”

“Maya,” he said, carefully. “I think you're over-reacting. You 
don't need to go to Mexico. You can just quit. We can do a startup 
together or something. Or you can move to the country and raise dogs. 
Whatever. This is crazy --”

“They came to see me today,” she said. “At work. Two of the 
political officers -- the minders who monitor our sensitive projects. 
And they asked me a lot of very heavy questions.”

“About the googlecleaner?”

“About my friends and family. About my search history. About my 
political beliefs.”

“Jesus.”

“They were sending me a message. They were letting me know that they 
were onto me. They're watching every click and every search. It's time 
to go -- time to get out of range.”

“There's a Google office in Mexico, you know.”

“Are you coming, Greg? We're going now.”

“Laurie, what do you think of this?”

Laurie thumped the dogs between the shoulders. “Maya showed me what 
Google knows about me. It's like there's a little me in there, a copy 
of me. Like I'm pinned down under a jar with a ball of ether. My 
parents left East Germany in `65 -- they used to tell me about the 
Stasi. They'd put everything about you in your file -- even unpatriotic 
jokes. Lately I've been feeling...watched. All the time. Like I can't 
live without leaving a trail. Like I'm throwing off a smog of data and 
it can't be gotten rid of.”

“We're going now, Greg. Now. Are you coming?”

Greg looked at the dogs. “I've got some pesos left over,” he said. 
“You take them. Be careful, OK?”

She looked like she was going to slug him. Then she softened and gave 
him a ferocious hug. “Be careful yourself,” she whispered in his 
ear.

\tb

They came for him a week later. At home, in the middle of the night, 
just as he'd imagined it. Their knock was nothing like Maya's 
tentative, nervous thump. They went bang-bang-bang, confident, knowing 
that they had every right to be there and not caring who else came 
after them.

Two men. One stayed by the door and didn't say anything. The other was 
a smiler, short and rumpled, in a sports coat with a small stain on one 
lapel and a cloisonné American flag on the other. “Computer Fraud 
and Abuse Act,” he said, by way of introduction. “'Exceeding 
authorized access, and by means of such conduct having obtained 
information.' Ten years for a first offense, ever since the PATRIOT Act 
extended it. I have it on the best of authority that what you and your 
friend did to your Google records qualifies. And oh, what will come out 
in the trial. All the stuff you whitewashed out of your profile.”

Greg had been playing this scene out in his head for a week. He'd had 
all kinds of brave things to say, planned out in advance. He'd even 
written some down, to see how they looked. It had given him something 
to do while the knots in his stomach tightened, while he waited to hear 
from Maya.

“I'd like to call a lawyer,” is all he managed. It came out in a 
whisper.

“You can do that,” the man said. “But hear me out first.”

Greg found his voice. “I'd like to see your badge.”

The man's basset-hound face lit up as he hissed a laugh. “Oh, Greg, 
buddy. I'm not a cop. I work for --” He named the DC firm in Google's 
employ. The inventors of swiftboating. “You're a Googler. You're part 
of the family. We couldn't send the police after you without talking 
with you first. There's an offer I'd like to make.”

Greg made coffee. It gave him something to do with his hands while he 
tried to find that bravery he'd been honing all week. “I'll go to the 
press,” he said. “I've written this all up. I'll go straight to 
them.”

The guy nodded as if thinking it over. “Well, sure. You could walk 
into the Chronicle's office in the morning and spill everything you 
need. They'd try to find a confirming source. They won't find it. Maybe 
you'll try to show them what your profile looks like today? Well, tell 
you what, it looks just like it looked the day you landed at SFO. Greg, 
buddy, why don't you hear me out before you start trying to figure out 
how to fight me? I'm in the win-win business. I'm in the business of 
figuring out how to get all parties what they need. I'm very good at 
it. You don't even want to know what I'm billing Google for this little 
tete-a-tete. By the way, those are excellent beans, but you want to 
give them a little rinse first, takes some of the bitterness out and 
brings up the oils. Here, pass me a colander?”

Greg watched in numb bemusement as the man took off his jacket and hung 
it over a kitchen chair, then undid his cuffs and rolled them up, 
slipping a cheap digital watch into his pocket. Then he poured the 
beans back out of the grinder and into Greg's colander and did things 
at the sink.

He was a little pudgy, and very pale. He needed a haircut -- had unruly 
curls at his neck. It made Greg relax, somehow. This guy had the social 
gracelessness of a nerd, felt like a real Googler, obsessed with the 
minutiae. He knew his way around a coffee-grinder, too.

“We're drafting a team for Building 49 --”

“There is no building 49,” Greg said, automatically.

“Yeah,” the guy said, with a private little smile. “There's no 
Building 49. And we're putting together a team, with its own buginizer, 
to own googlecleaner. Maya's code wasn't very efficient. Every time 
someone runs it, it clobbers the whole farm. And it's got plenty of 
bugs. We've asked around and there's consensus on this. You'd be the 
right guy, and it wouldn't matter what you knew if you were back inside 
--”

“No, I wouldn't,” Greg said. “You're on crack.”

“Hear me out. There's money involved. Good work, too. Smart 
colleagues. A direction for your life. A chance to participate in the 
political life of your country --”

Greg gave a bitter laugh. “Unbelievable,” he said. “If you think 
I'm going to help you smear political candidates in exchange for 
favors, you're even crazier than I thought.”

“Greg,” he said, “Greg, you're right. That was dumb. No one is 
going to do that anymore. We're just going to -- clean things up a 
little. For some select people. You know what I mean, right? Every 
Google profile is a little scary under close inspection. Close 
inspection is the order of the day in politics. You stand for office 
and they'll look at your kids, your brothers, your ex-girlfriends. Now 
that your search history is available to so many people, it won't be 
that hard to look into that too. Your Orkut network, your old Usenet 
messages, your searches, all of it.” He loaded the cafetiere and 
depressed the plunger, his face screwed up in solemn concentration. He 
held out his hand and Greg got down two coffee mugs -- Google mugs, of 
course -- and passed them to him.

“We're going to do for our friends just what Maya did for you. Just 
give them a little cleanup. Preserve their privacy. That's all -- I 
promise you, that's all.”

Greg sipped the coffee, but didn't taste it. “And whichever 
candidates you \emph{don't} clean --”

“Yeah,” the guy said. “Yeah, you're right. It'll be tough for 
them.”

“You can go now,” Greg said.

“Oh, Greg,” the guy said. He plucked his jacket off his chair-back 
and shrugged it on, felt in the inside pocket and produced a small 
stack of paper, folded into quarters. He smoothed it out and put it on 
the table.

Greg looked quickly and saw the rows of results he'd seen on the DHS 
man's screen, back at the airport, when this all started. “I don't 
care,” he said. “Tell the world about my search history. Go ahead. 
In five years, everyone will have had their search history ruptured. 
We'll all be guilty.”

“It's not your history,” the man said. He divided the stack into 
two piles, and pointed to names on the top sheet of each. One was 
Maya's. The other was a candidate whose campaign Greg had contributed 
to for the last three elections.

“You get five weeks' vacation a year. You can go to Cabo for the 
SCUBA. The options package is very generous, too.”

The man sat down and drank some coffee. Greg tried some more of his 
own. It didn't taste so bad. It was, in fact, more delicious than 
anything that had ever come out of his kitchen. The man knew what he 
was doing.

The best years of Greg's life had been spent at Google. Smart people. 
Amazing work environment. Wonderful technology. Nothing in the world 
like it. When you worked at G, you had the best model train set in the 
universe to play with. Organizing all of human knowledge.

“You can pick your team, of course,” the man said.

Greg poured himself another cup of delicious coffee.

\tb

The new Congress took eleven working days to pass the Securing and 
Enumerating America's Communications and Hypertext Act, which 
authorized the DHS and the NSA to outsource up to 80 percent of its 
intelligence and analysis work to private contractors.

Theoretically, the contracts were open to a competitive bidding 
process, but within the secure group at Google, in building 49, there 
was no question of who would win those contracts. If Google had spent 
\$15 billion on a program to catch bad guys at the border, you can bet 
that they would have caught them -- governments just aren't equipped to 
Do Search Right.

Greg looked himself in the eye that morning as he shaved -- the 
security minders didn't like hacker-stubble, and they weren't shy about 
telling you so -- and realized that today was his first day as a de 
facto intelligence agent in the US government.

How bad would it be? Wasn't it better to have Google doing this stuff 
than some ham-fisted spook?

He had himself convinced by the time he parked at the Googleplex, among 
the hybrid cars and bulging bike-racks. He stopped for an organic 
smoothie on the way to his desk, then sat down and sipped.

The rumpled man hadn't been to the G since Greg went back to work, but 
it often felt like his influence was all around them in building 49. He 
wasn't any less rumpled today -- he could have been wrapped in 
saran-wrap on the day he brought Greg back to work and refrigerated for 
all that he hadn't changed a hair.

“Hi, Greg,” he said, sliding into the chair next to his. His 
podmates stood up in unison and left the room.

“Just tell me what it is,” Greg said. “Just spit it out. You want 
me to pwn NORAD and start World War III, right?”

“Nothing so obvious,” the man said, patting his shoulder. “Just a 
little search-job.”

“Yeah?”

“There's a person we want to find. A person who's left the country, 
apparently headed for Mexico. She knows certain things that are, as of 
today, classified. She needs to be briefed on her new 
responsibilities.”

Greg stood up. “I'm not going to find Maya for you.” He pulled on 
his jacket.

“There are plenty of people here who will. It's up to you, though. 
You can work here with her, being productive, or you can find out just 
how rotten life can get -- while she works here, being productive with 
your co-workers.”

Greg stared at him, his hands balled into fists.

“Come on,” the rumpled man said. “Greg, we both know how this 
goes. When you said yes to me in your kitchen, you lost the option of 
saying no. It's not so bad, is it? Who would you rather have doing the 
nation's intelligence: you and your pals here in the Valley, or a bunch 
of straight-edge code-grinders in Virginia?”

Greg turned on his heel and left. He made it all the way to the parking 
lot before he stopped and kicked a wall so hard he felt something give 
way in his foot.

Then he limped back to his desk, hung his jacket on his chair, and 
logged back in.

\tb

It was a week later when his key-card failed to open the door to 
Building 49. The idiot red LED shone at him every time he swiped it. He 
swiped it and swiped it. Any other building and there'd be someone to 
tailgate on, people trickling in and out all day. But the Googlers in 
49 only emerged for meals, and sometimes not even that.

Swipe, swipe, swipe.

“Greg, can I see you, please?”

The rumpled man hadn't shaved in a couple of days. He put an arm around 
Greg's shoulders and Greg smelled his citrusy aftershave. It was the 
same cologne that his divemaster in Baja had worn when they went out to 
the bars in the evening. Greg couldn't remember his name. Juan-Carlos? 
Juan-Luis?

The man's arm around his shoulders was firm, steering him away from the 
door, out onto the immaculate lawn, past the kitchen's herb garden. 
“We're giving you a couple of days off,” he said.

Greg felt a cold premonition that sank all the way to his balls. 
“Why?” Had he done something wrong? Was he going to jail?

“It's Maya.” The man turned him around, met his eyes with his 
bottomless basset-hound gaze. “It's Maya. Killed herself. In 
Guatemala. I'm sorry, Greg.”

Greg seemed to hurtle away from himself, to a place miles above, a 
Google Earth view of the Googleplex, looking down on himself and the 
rumpled man as a pair of dots, two pixels, tiny and insignificant. He 
willed himself to tear at his hair, to drop to his knees and weep.

From a long way away, he heard himself say, “I don't need any time 
off. I'm OK.”

From a long way away, he heard the rumpled man insist.

But one-pixel Greg wouldn't be turned aside. The argument persisted for 
a long time, and then the two pixels moved into Building 49 and the 
door swung shut behind them.

\section{Afterword}

This one came as a commission from \emph{Radar} magazine -- now 
defunct, a casualty of the 2008 crash, but in 2007, this was the most 
widely circulated “lifestyle” magazine in the US. They asked me to 
write about “the day Google became evil.” I didn't want to cheap 
out and just write about the company selling out to some evil 
millionaire. If Google ever turned evil, it would be because a) evil 
had a compelling business-model and b) evil lay at the end of a 
compelling technical challenge.

I spent a lot of time talking off-the-record to Googlers, who are, to a 
one, the nicest people I know (OK, one exception springs to mind, but 
let's not air our dirty laundry in public, right?). I also had an 
incredibly productive conversation with the Electronic Frontier 
Foundation's Kevin Bankston, a profound and sharp-witted privacy lawyer.

I wanted to capture a company that was full of good people who do bad. 
There are lots of these. For example, \emph{all} the Microsoft 
employees I know are fantastic and smart and caring and principled. But 
ethically and technically, most of what comes out of Redmond is a 
train-wreck. It's anti-synergy: a firm that is far less than the sum of 
its parts. I could easily see Google turning into that. I wish I 
understood how groups of good people trying to do good can do bad.
\end{document}
