\hyphenation{mo-no-poly car-ne-gie pro-ject pro-gress mo-dem rou-lette
  browse-wrap Use-net mon-as-tery mo-dems}
\hyphenation{co-me-dic polt-roon stove-pipe Ma-dame scru-ta-ble star-tling}
\hyphenation{heal-thily lim-ou-sines wrest-lers tan-trum push-over un-asked
  bras-siere bro-th-er}
\hyphenation{Can-a-da Fred-rick teen-agers wrest-ler Cha-vez Tho-mas 
  a-nom-a-lies sur-veil-lance ar-mies ref-u-gee ref-u-gees bris-tling
  eve-ning man-chu-ria man-chu-ri-an mid-terms me-di-um jap-a-nese}
\hyphenation{spend-ers googl-ing tour-ist tour-ists leg-end-ary}
\hyphenation{Dan-iel Van-essa Doc-to-row Ste-phen-son}
\hyphenation{de-cade sur-veilled rout-ers Wol-fen-stein teen-ager to-night}
\hyphenation{his-to-gram an-o-nym-ize Ga-la-xy sym-pa-the-tic}
\hyphenation{ar-phid ar-phids}


\begin{document}
%\setlength{\emergencystretch}{1ex}
\raggedbottom

\begin{center}
\textbf{\huge\textsf{The Right Book}}

\medskip
Cory Doctorow

\end{center}

\bigskip

\begin{flushleft}
This story is part of Cory Doctorow’s short story collection
“With a Little Help” published by himself. It is licensed under a
\href{http://creativecommons.org/licenses/by-nc-sa/}
{Creative Commons Attribution-NonCommercial-ShareAlike 3.0} license.

\bigskip

The whole volume is available at:
\texttt{http://craphound.com/walh/}

\medskip

The volume has been split into individual stories for the purpose of the
\href{http://ccbib.org}{Creative Commons Bibliothek.}
The introduction and similar accompanying texts are available under the 
title:
\end{flushleft}
\begin{center}
With a Little Help -- Extra Stuff
\end{center}

\newpage

\section{The Right Book}

Now (-ish)

The thing that Arthur liked best about owning his own shop was that he 
could stock whatever he pleased, and if you didn't like it, you could 
just shop somewhere else. So there in the window were four ancient 
Cluedo sets rescued from a car-boot sale in Sussex; a pair of trousers 
sewn from a salvaged WWII bivouac tent; a small card advertising the 
availability of artisanal truffles hand-made by an autistically gifted 
chocolatier in Islington; a brick of Pu'er tea that had been made in 
Guyana by a Chinese family who'd emigrated a full century previous; 
and, just as of now, six small, handsomely made books.

The books were a first for Arthur. He'd always loved reading the 
things, but he'd worked at bookshops before opening his own little 
place in Bow, and he knew the book-trade well enough to stay well away. 
They were bulky, these books, and low-margin (Low margin? Two-for-three 
titles actually \emph{lost} money!), and honestly, practically no one 
read books anymore and what they did read was mostly rubbish. Selling 
books depressed Arthur.

These little buggers were different, though. He reached into the window 
-- the shop was so small he could reach it without leaving his stool 
behind the till -- and plucked one out and handed it to the kid who'd 
just asked for it. She was about 15, with awkward hair and skin and 
posture and so on, but the gleam in her eye that said, “Where have 
you been all my life?” as he handed her the book.

“They're all carrying them in school,” she said. “Never thought 
I'd find one in a shop, though. How much?”

Arthur compared the book to his cheat-sheet behind the counter. This 
one had a cover made from old Hacks tins, resurfaced with a spectral 
spiderweb of rotting Irish lace. The chapters within had a whopping 
aggregate score of 98 percent, meaning that 98 percent of the writing 
community had rated them aces or above. Even before he looked to the 
price column on his sheet, he knew he was going to have to disappoint 
her.

“That one's seventy quid, love,” he said. He armored himself for 
the inevitable shock, disbelief and protestation, but she just hung her 
head, resigned.

“Figures,” she said.

He ran his fingers down the spines until he found a cheaper one -- 
bound with floppy felt screened with a remixed Victorian woodcut of a 
woman with tentacles for arms. “This one's got mostly the same text, 
but I can let you have it for, erm,” he looked at the sheet again, 
thinking about the wholesale price, about his margin. “Call it 
twenty-five pounds.”

She shook her head again, gave him a wry smile. “Still too much. I 
should have known. It's mostly the posh kids who've got `em, the kind 
who turn up at school with a tenner just for lunch money.”

“You could just read it online, you know.”

“Oh, I do,” she said. “Been following it since it started.” Her 
eyes flicked down. “Wrote a little, too -- didn't make it into the 
top 100, though.”

The Story So Far was part game, part competition, part creative writing 
exercise, a massive shared universe drama with dozens of sub-plots, 
mysteries, betrayals, crosses, and double-crosses. Everyone kept saying 
it was only a matter of time until the big publishers started to 
cherry-pick the best writers from the message-boards, but in the 
meantime, there were these little hand-made editions, each one paying a 
small, honor-system royalty to the authors they anthologized.

“Have you tried asking your teachers for help?” He knew as soon as 
he asked it that it was the wrong sort of question. She rolled her eyes 
with adolescent eloquence, then looked down again. “Only you might be 
able to get credit for it -- independent study type of thing?”

She rolled her eyes again.

“Right,” he said. “Right. Well, sorry I couldn't be more help.” 
The little bell over the door jingled merrily as she left.

\tb

“Back again?”

She had her school bag in her hands, zip opened, bag gaping. He was 
reminded of all those terrible little signs that said “No more than 
two school kids in the shop at any one time.” Fancy that -- imagine 
if it said “No more than two women in the shop” or “No more than 
two Asians in the shop” -- kids were the last group you could treat 
like second class citizens without being called a bigot.

“Where do you get your copies of The Book?” she said. The Book, 
with the capital letters -- the one book with a thousand covers, a 
million tables of contents, each one not so much published as made, as 
curated.

“There's a man,” he said. “Art student at UCL. He's got a little 
stall at the weekend in the parking structure where Borough Market used 
to be.”

“So you just buy them from some bloke? Does he make them?”

“I suppose so -- he gives me that impression, anyway.” He liked her 
shrewd, unembarrassed, direct questioning. Not a single scruple or a 
hint that she was embarrassed to be interrogating him about the 
intimate details of his trade.

“Do you have, like, an exclusive arrangement with him?”

“No, no, nothing like that.” Her hands were digging through the 
bag, looking for something.

“Would you think about carrying these?”

She'd clearly bound them herself. Someone had taught her to really sew, 
her Gran, maybe. You could see it in the neat stitching that ran the 
binding and the spine, holding together the nylon and the denim, taken 
from a pair of jeans, a backpack. The end-papers were yellowed page 
three girls from the Star, strategically cropped just below the 
nipples. He'd been reading The Story So Far ever since those first six 
copies had sold out in forty eight hours, and he had an eye for the 
table of contents now, and he flipped to each volume's list, giving 
them a long look.

“Who's Chloe Autumn?”

She didn't look down, looked at him with a look that was totally 
unapologetic. “I am,” she said. “It's one way to get my stuff 
into print, innit?” She grinned. It was a very grown up grin.

“What do you think you want for them?”

“Those four I figure you can have cheaply -- say fifteen pounds each. 
You can sell them for thirty, then. That's fair, I think.”

It was more than fair. His UCL student wasn't carrying anything for 
less than forty now, and was only offering him a 40 percent discount.

“What about returns?”

“What's a return?”

He reached under the counter and brought out the shooting stick he used 
as a spare stool. “Have a seat,” he said. “Let me explain some 
things. Want a cup of tea? It's Pu'er. Chinese. Mostly.”

\tb

+75 years (or so)

The kids in the shop were like kids everywhere. That weird, hyperaware 
thing that came from the games they played all the time, even in their 
sleep; the flawless skin and teeth (because no parent would dare choose 
otherwise at conception), the loud, hooting calls that rippled through 
the little social groups whenever a particularly bon mot vibrated its 
way through their tight little networks, radiating at the speed of 
light.

Chloe watched them keenly from her perch behind the counter. After 
seventy-some years perching on a stool, she'd finally done away with 
it. The exoskeleton she'd been fitted for on her 90th birthday would 
lock very handily into a seated position that took all the pressure off 
her bum and knees and hips. It was all rather glorious.

Kids came into the store every day now, and in ever-increasing numbers. 
She flicked her eyes sideways and menued over to her graph of young 
people in the shop over time, warming herself on the upward trend.

It was Arthur's 110th birthday today, the mad old sod, and he was meant 
to be coming into the shop for one of his rare tours of inspection. 
That had the staff all a-twitter. He was something of a legend, the man 
who'd started the distributorship that put small, carefully curated 
handsful of books into the few retailers across the land who'd let 
young people in. No one could have predicted how well books and Halal 
fried chicken went together.

“How long have you known him, then?” Marcel, her store manager, was 
only a few years older than the kids who ghosted past her counter, 
playing some weird round of their game, listening to cues only they 
could hear, heads all cocked identically.

“Let me put it this way -- the first time we met, I was riding a 
brontosaurus.”

He did her the favor of a smile, radiant and handsome as a movie 
marquee. They were all like that these days. Thankfully she was old 
enough not to feel self-conscious about it.

“Seriously, Chloe, when did you meet him?”

“I was fourteen -- no fifteen. That was before he was Sir Arthur 
Levitt, Savior of English Literacy, you understand.”

“And before you were Chloe Autumn, superstar author?” He was 
kidding her. They'd stopped caring about what she wrote decades before 
he was born, but he knew about her history and liked to tease. He had 
an easy way about him, and it showed in the staff.

“Before then, yes.”

“I still don't quite understand what it was he did -- what was so 
different about his bookshop?”

“It wasn't a bookshop,” she said. “You didn't know that part?” 
He shook his head. “Well, that's the most important part. It wasn't a 
bookshop. Back then, bookshops were practically the only place you 
could get a book. Oh, sure, the newsagents might carry a few titles, 
but they were the same titles, all around the country. Bookshops are 
fine if you already love books, but how do you fall in love with books? 
Where does it start? There have to be books everywhere, in places where 
you go before you know you're a reader. That was the secret.”

“So how'd he do it?”

“I'll tell you how,” Arthur said. He'd padded up to the counter on 
the oiled, carefully balanced carapace of his exoskeleton, moving as 
spryly as a jaguar. His eyes glittered with mad, birdy glee. “Hello, 
Chloe,” he said.

“Happy birthday, love,” she said, uncurling herself and levering 
herself up on tiptoe -- the gyros whining -- to give him a kiss on the 
cheek. “Arthur, this is Marcel.”

They shook hands.

“I'll tell you how,” Arthur said again, clearly enjoying the chance 
to unfurl one of his old, well-oiled stories. “It was all about 
connecting kids up with their local neighborhoods and the tastes there. 
Kids know what their friends want to read. We had them curate their own 
anthologies of the best, most suitable material from The Story So Far, 
put all that local knowledge to work. The right book for the right 
person in the right place. You've got to give them a religious 
experience before you can lure them into coming to church regular.”

“Arthur thinks reading is a religion,” Chloe said, noting Marcel's 
puzzled expression.

“Obsolete, you mean?” Marcel said.

Arthur opened his mouth, shut it, prepared to have an argument. Chloe 
short-circuited it by reaching under the counter and producing a 
carefully wrapped package.

“Happy birthday, you old sod,” she said, and handed it to Arthur.

He was clearly delighted. Slowly, he picked at the wrapping paper, 
making something of a production of it, so much so that the kids 
started to drift over to watch. He peeled back a corner, revealing the 
spine of the book, the neat stitching, the nylon from an old, old 
backpack, the worn denim, the embroidered title on the spine.

“You didn't,” he said.

“I certainly did,” she said, “now finish unwrapping it so that we 
can have some cake.”

\tb

150 years from now(ish)

The young man blinked his eyes at the coruscating lights and struggled 
into a seated position, brushing off the powdery residue of his 
creation. “The Story So Far?” he said.

“The Story So Far,” a voice agreed with him from a very long way 
off and so close in, it was practically up his nose.

“Better than Great Expectations again,” he said, getting to his 
feet, digging through the costumes on the racks around him. Knowledge 
slotted itself in his head, asserting itself. Plots, other characters, 
what had come before, the consensus about where things might go next. 
He didn't like the consensus. He began to dress himself.

“Tell me about the reader,” he said. The voice was back in an 
instant, describing the child (four), the circumstances of his birth 
and life, his interests. “So I'm a picture book?”

“No,” the voice said. “He's reading in chapters now. It's the 
cognitive fashion, here.” At \emph{here}, more knowledge asserted 
itself, the shape of the comet on which they all resided, their 
hurtling trajectory, a seed-pod of humanity on its way \emph{elsewhere}.

“Right,” he said, putting on gloves, picking out a moustache and a 
sword and a laser-blaster. “Let's go sell some books.”

\section{Afterword}

This is another story that was inspired by Patrick Nielsen Hayden; 
specifically by his very nice rant about how the collapse of small, 
local book distributors that served grocers and pharmacies -- and the 
rise of national distributors who serve big-box stores -- has destroyed 
the primary means by which new readers enter the field. It's all well 
and good to have terrific giant bookstores (or fabulous neighborhood 
stores, for that matter), but people don't go into those stores unless 
they already love books. In the past, the love affair with books often 
began outside of bookstores, in grocers and pharmacies, where you might 
happen upon any number of quirky, hand-picked paperbacks stocked by the 
local distributor. With the choice of books available outside of 
bookstores narrowed to the handful of titles with national 
distribution, it's far less likely that any given reader will discover 
“the right book” -- the one that turns her into a book-junkie for 
the rest of her life.

Thus, this story. \emph{The Bookseller}, Britain's oldest publishing 
trade magazine, commissioned a story from me for its 150th anniversary 
issue -- three parts, depicting the future of bookselling in 50, 100 
and 150 years.
\end{document}

