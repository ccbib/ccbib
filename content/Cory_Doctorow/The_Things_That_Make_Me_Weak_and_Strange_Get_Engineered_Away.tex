\hyphenation{mo-no-poly car-ne-gie pro-ject pro-gress mo-dem rou-lette
  browse-wrap Use-net mon-as-tery mo-dems}
\hyphenation{co-me-dic polt-roon stove-pipe Ma-dame scru-ta-ble star-tling}
\hyphenation{heal-thily lim-ou-sines wrest-lers tan-trum push-over un-asked
  bras-siere bro-th-er}
\hyphenation{Can-a-da Fred-rick teen-agers wrest-ler Cha-vez Tho-mas 
  a-nom-a-lies sur-veil-lance ar-mies ref-u-gee ref-u-gees bris-tling
  eve-ning man-chu-ria man-chu-ri-an mid-terms me-di-um jap-a-nese}
\hyphenation{spend-ers googl-ing tour-ist tour-ists leg-end-ary}
\hyphenation{Dan-iel Van-essa Doc-to-row Ste-phen-son}
\hyphenation{de-cade sur-veilled rout-ers Wol-fen-stein teen-ager to-night}
\hyphenation{his-to-gram an-o-nym-ize Ga-la-xy sym-pa-the-tic}
\hyphenation{ar-phid ar-phids Found-ers}
\hyphenation{stran-ger stran-gers shoul-der-blades dump-ling dump-lings}
\hyphenation{ice-pack guard-rail Sep-tem-ber boot-able e-co-nom-ist}
\hyphenation{grown-ups roos-ter shoe-laces li-quid-i-ty}
\hyphenation{side-arm}
\hyphenation{wo-man wo-men tan-trum tan-trums Le-nin-grad zom-bie bunk-house}
\hyphenation{up-tick bio-mass}
\hyphenation{of-fi-cial of-fi-cial-ly gov-ern-ment}
\hyphenation{heal-thy Or-ville spark-ling}
\hyphenation{ves-ti-bule Law-rence au-to-no-mous}
\hyphenation{sau-sage door-step staf-fer}
\hyphenation{tree-trunk}
\hyphenation{to-ron-to}
\hyphenation{qua-dril-lion-aire qua-dril-lion-aires}
\hyphenation{sports-jack-et sports-jack-ets}
\hyphenation{work-space skunk-works}
\hyphenation{kings-ton}


\begin{document}
%\setlength{\emergencystretch}{1ex}
\raggedbottom

\begin{center}
\textbf{\huge\textsf{The Things That Make Me Weak and Strange Get 
Engineered Away}}

\medskip
Cory Doctorow

\end{center}

\bigskip

\begin{flushleft}
This story is part of Cory Doctorow’s short story collection
“With a Little Help” published by himself. It is licensed under a
\href{http://creativecommons.org/licenses/by-nc-sa/}
{Creative Commons Attribution-NonCommercial-ShareAlike 3.0} license.

\bigskip

The whole volume is available at:
\texttt{http://craphound.com/walh/}

\medskip

The volume has been split into individual stories for the purpose of the
\href{http://ccbib.org}{Creative Commons Bibliothek.}
The introduction and similar accompanying texts are available under the 
title:
\end{flushleft}
\begin{center}
With a Little Help -- Extra Stuff
\end{center}

\newpage
\section{The Things That Make Me Weak and Strange Get Engineered Away}
\begin{flushleft}
\small
\textsf{"Cause it's gonna be the future soon,\\
"And I won't always be this way,\\
"When the things that make me weak and strange get engineered away"\\
}
\hfill \textit{--Jonathan Coulton, The Future Soon}
\end{flushleft}

Lawrence's cubicle was just the right place to chew on a thorny logfile 
problem: decorated with the votive fetishes of his monastic order, a 
thousand calming, clarifying mandalas and saints devoted to helping him 
think clearly.

From the nearby cubicles, Lawrence heard the ritualized muttering of a 
thousand brothers and sisters in the Order of Reflective Analytics, a 
susurration of harmonized, concentrated thought. On his display, he 
watched an instrument widget track the decibel level over time, the 
graph overlayed on a 3D curve of normal activity over time and space. 
He noted that the level was a little high, the room a little more 
anxious than usual.

He clicked and tapped and thought some more, massaging the logfile to 
see if he could make it snap into focus and make sense, but it 
stubbornly refused to be sensible. The data tracked the custody chain 
of the bitstream the Order munged for the Securitat, and somewhere in 
there, a file had grown by 68 bytes, blowing its checksum and becoming 
An Anomaly.

Order lore was filled with Anomalies, loose threads in the fabric of 
reality -- bugs to be squashed in the data-set that was the Order's 
universe. Starting with the pre-Order sysadmin who'd tracked a \$0.75 
billing anomaly back to a foreign spy-ring that was using his systems 
to hack his military, these morality tales were object lessons to the 
Order's monks: pick at the seams and the world will unravel in useful 
and interesting ways.

Lawrence had reached the end of his personal picking capacity, though. 
It was time to talk it over with Gerta.

He stood up and walked away from his cubicle, touching his belt to let 
his sensor array know that he remembered it was there. It counted his 
steps and his heartbeats and his EEG spikes as he made his way out into 
the compound.

It's not like Gerta was in charge -- the Order worked in autonomous 
little units with rotating leadership, all coordinated by some 
groupware that let them keep the hierarchy nice and flat, the way that 
they all liked it. Authority sucked.

But once you instrument every keystroke, every click, every erg of 
productivity, it soon becomes apparent who knows her shit and who just 
doesn't. Gerta knew the shit cold.

“Question,” he said, walking up to her. She liked it brusque. No 
nonsense.

She batted her handball against the court wall three more times, making 
long dives for it, sweaty grey hair whipping back and forth, body 
arcing in graceful flows. Then she caught the ball and tossed it into 
the basket by his feet. “Lawrence, huh? All right, surprise me.”

“It's this,” he said, and tossed the file at her pan. She caught it 
with the same fluid gesture and her computer gave it to her on the 
handball court wall, which was the closest display for which she 
controlled the lockfile. She peered at the data, spinning the graph 
this way and that, peering intently.

She pulled up some of her own instruments and replayed the bitstream, 
recalling the logfiles from many network taps from the moment at which 
the file grew by the anomalous 68 bytes.

“You think it's an Anomaly, don't you?” She had a fine blond 
mustache that was beaded with sweat, but her breathing had slowed to 
normal and her hands were steady and sure as she gestured at the wall.

“I was kind of hoping, yeah. Good opportunity for personal growth, 
your Anomalies.”

“Easy to say why you'd call it an Anomaly, but look at this.” She 
pulled the checksum of the injected bytes, then showed him her network 
taps, which were playing the traffic back and forth for several minutes 
before and after the insertion. The checksummed block moved back 
through the routers, one hop, two hops, three hops, then to a terminal. 
The authentication data for the terminal told them who owned its 
lockfile then: Zbigniew Krotoski, login zbigkrot. Gerta grabbed his 
room number.

“Now, we don't have the actual payload, of course, because that gets 
flushed. But we have the checksum, we have the username, and look at 
this, we have him typing 68 unspecified bytes in a pattern consistent 
with his biometrics five minutes and eight seconds prior to the 
injection. So, let's go ask him what his 68 characters were and why 
they got added to the Securitat's data-stream.”

He led the way, because he knew the corner of the campus where zbigkrot 
worked pretty well, having lived there for five years when he first 
joined the Order. Zbigkrot was probably a relatively recent inductee, 
if he was still in that block.

His belt gave him a reassuring buzz to let him know he was being logged 
as he entered the building, softer haptic feedback coming as he was 
logged to each floor as they went up the clean-swept wooden stairs. 
Once, he'd had the work-detail of re-staining those stairs, stripping 
the ancient wood, sanding it baby-skin smooth, applying ten coats of 
varnish, polishing it to a high gloss. The work had been incredible, 
painful and rewarding, and seeing the stairs still shining gave him a 
tangible sense of satisfaction.

He knocked at zbigkrot's door twice before entering. Technically, any 
brother or sister was allowed to enter any room on the campus, though 
there were norms of privacy and decorum that were far stronger than any 
law or rule.

The room was bare, every last trace of its occupant removed. A fine 
dust covered every surface, swirling in clouds as they took a few steps 
in. They both coughed explosively and stepped back, slamming the door.

“Skin,” Gerta croaked. “Collected from the ventilation filters. 
DNA for every person on campus, in a nice, even, Gaussian distribution. 
Means we can't use biometrics to figure out who was in this room before 
it was cleaned out.”

Lawrence tasted the dust in his mouth and swallowed his gag reflex. 
Technically, he knew that he was always inhaling and ingesting other 
people's dead skin-cells, but not by the mouthful.

“All right,” Gerta said. “\emph{Now} you've got an Anomaly. 
Congrats, Lawrence. Personal growth awaits you.”

\tb

The campus only had one entrance to the wall that surrounded it. 
“Isn't that a fire-hazard?” Lawrence asked the guard who sat in the 
pillbox at the gate.

“Naw,” the man said. He was old, with the serene air of someone 
who'd been in the Order for decades. His beard was combed and shining, 
plaited into a thick braid that hung to his belly, which had only the 
merest hint of a little pot. “Comes a fire, we hit the panic button, 
reverse the magnets lining the walls, and the foundations destabilize 
at twenty sections. The whole thing'd come down in seconds. But no 
one's going to sneak in or out that way.”

“I did \emph{not} know that,” Lawrence said.

“Public record, of course. But pretty obscure. Too tempting to a 
certain prankster mindset.”

Lawrence shook his head. “Learn something new every day.”

The guard made a gesture that caused something to depressurize in the 
gateway. A primed \emph{hum} vibrated through the floorboards. “We 
keep the inside of the vestibule at 10 atmospheres, and it opens inward 
from outside. No one can force that door open without us knowing about 
it in a pretty dramatic way.”

“But it must take forever to re-pressurize?”

“Not many people go in and out. Just data.”

Lawrence patted himself down.

“You got everything?”

“Do I seem nervous to you?”

The old timer picked up his tea and sipped at it. “You'd be an idiot 
if you weren't. How long since you've been out?”

“Not since I came in. Sixteen years ago. I was twenty one.”

“Yeah,” the old timer said. “Yeah, you'd be an idiot if you 
weren't nervous. You follow politics?”

“Not my thing,” Lawrence said. “I know it's been getting worse 
out there --”

The old timer barked a laugh. “Not your thing? It's probably time you 
got out into the wide world, son. You might ignore politics, but it 
won't ignore \emph{you}.”

“Is it dangerous?”

“You going armed?”

“I didn't know that was an option.”

“Always an option. But not a smart one. Any weapon you don't know how 
to use belongs to your enemy. Just be circumspect. Listen before you 
talk. Watch before you act. They're good people out there, but they're 
in a bad, bad situation.”

Lawrence shuffled his feet and shifted the straps of his bindle. 
“You're not making me very comfortable with all this, you know.”

“Why are you going out anyway?”

“It's an Anomaly. My first. I've been waiting sixteen years for this. 
Someone poisoned the Securitat's data and left the campus. I'm going to 
go ask him why he did it.”

The old man blew the gate. The heavy door lurched open, revealing the 
vestibule. “Sounds like an Anomaly all right.” He turned away and 
Lawrence forced himself to move toward the vestibule. The man held his 
hand out before he reached it. “You haven't been outside in sixteen 
years, it's going to be a surprise. Just remember, we're a noble 
species, all appearances to the contrary notwithstanding.”

Then he gave Lawrence a little shove that sent him into the vestibule. 
The door slammed behind him. The vestibule smelled like machine oil and 
rubber, gaskety smells. It was dimly lit by rows of white LEDs that 
marched up the walls like drunken ants. Lawrence barely had time to 
register this before he heard a loud \emph{thunk} from the outer door 
and it swung away.

\tb

Lawrence walked down the quiet street, staring up at the same sky he'd 
lived under, breathing the same air he'd always breathed, but marveling 
at how \emph{different} it all was. His heartbeat and respiration were 
up -- the tips of the first two fingers on his right hand itched 
slightly under his feedback gloves -- and his thoughts were doing that 
race-condition thing where every time he tried to concentrate on 
something he thought about how he was trying to concentrate on 
something and should stop thinking about how he was concentrating and 
just concentrate.

This was how it had been sixteen years before, when he'd gone into the 
Order. He'd been so \emph{angry} all the time then. Sitting in front of 
his keyboard, looking at the world through the lens of the network, 
suffering all the fools with poor grace. He'd been a bright 14 year 
old, a genius at 16, a rising star at 18, and a failure by 21. He was 
depressed all the time, his weight had ballooned to nearly 300 pounds, 
and he had been fired three times in two years.

One day he stood up from his desk at work -- he'd just been hired at a 
company that was selling learning, trainable vision-systems for 
analyzing images, who liked him because he'd retained his security 
clearance when he'd been fired from his previous job -- and walked out 
of the building. It had been a blowing, wet, grey day, and the streets 
of New York were as empty as they ever got.

Standing on Sixth Avenue, looking north from midtown, staring at the 
buildings the cars and the buses and the people and the tallwalkers, 
that's when he had his realization: \emph{He was not meant to be in 
this world.}

It just didn't suit him. He could \emph{see} its workings, see how its 
politics and policies were flawed, see how the system needed debugging, 
see what made its people work, but he couldn't touch it. Every time he 
reached in to adjust its settings, he got mangled by its gears. He 
couldn't convince his bosses that he knew what they were doing wrong. 
He couldn't convince his colleagues that he knew best. Nothing he did 
succeeded -- every attempt he made to right the wrongs of the world 
made him miserable and made everyone else angry.

Lawrence knew about humans, so he knew about this: this was the exact 
profile of the people in the Order. Normally he would have taken the 
subway home. It was forty blocks to his place, and he didn't get around 
so well anymore. Plus there was the rain and the wind.

But today, he walked, huffing and limping, using his cane more and more 
as he got further and further uptown, his knee complaining with each 
step. He got to his apartment and found that the elevator was out of 
service -- second time that month -- and so he took the stairs. He 
arrived at his apartment so out of breath he felt like he might vomit.

He stood in the doorway, clutching the frame, looking at his sofa and 
table, the piles of books, the dirty dishes from that morning's 
breakfast in the little sink. He'd watched a series of short videos 
about the Order once, and he'd been struck by the little monastic cells 
each member occupied, so neat, so tidy, everything in its perfect 
place, serene and thoughtful.

So unlike his place.

He didn't bother to lock the door behind him when he left. They said 
New York was the burglary capital of the developed world, but he didn't 
know anyone who'd been burgled. If the burglars came, they were welcome 
to everything they could carry away and the landlord could take the 
rest. He was not meant to be in this world.

He walked back out into the rain and, what the hell, hailed a cab, and, 
hail mary, one stopped when he put his hand out. The cabbie grunted 
when he said he was going to Staten Island, but, what the hell, he 
pulled three twenties out of his wallet and slid them through the glass 
partition. The cabbie put the pedal down. The rain sliced through the 
Manhattan canyons and battered the windows and they went over the 
Verrazano bridge and he said goodbye to his life and the outside world 
forever, seeking a world he could be a part of.

Or at least, that's how he felt, as his heart swelled with the drama of 
it all. But the truth was much less glamorous. The brothers who 
admitted him at the gate were cheerful and a little weird, like his 
co-workers, and he didn't get a nice clean cell to begin with, but a 
bunk in a shared room and a detail helping to build more quarters. And 
they didn't leave his stuff for the burglars -- someone from the Order 
went and cleaned out his place and put his stuff in a storage locker on 
campus, made good with his landlord and so on. By the time it was all 
over, it all felt a little...ordinary. But in a good way, Ordinary was 
good. It had been a long time since he'd felt ordinary. Order, 
ordinary. They went together. He needed ordinary.

\tb

The Securitat van played a cheerful engine-tone as it zipped down the 
street towards him. It looked like a children's drawing -- a perfect 
little electrical box with two seats in front and a meshed-in lockup in 
the rear. It accelerated smoothly down the street towards him, then 
braked perfectly at his toes, rocking slightly on its suspension as its 
doors gull-winged up.

“Cool!” he said, involuntarily, stepping back to admire the smart 
little car. He reached for the lifelogger around his neck and aimed it 
at the two Securitat officers who were debarking, moving with stiff 
grace in their armor. As he raised the lifelogger, the officer closest 
to him reached out with serpentine speed and snatched it out of his 
hands, power-assisted fingers coming together on it with a loud, 
plasticky \emph{crunk} as the device shattered into a rain of 
fragments. Just as quickly, the other officer had come around the 
vehicle and seized Lawrence's wrists, bringing them together in a 
painful, machine-assisted grip.

The one who had crushed his lifelogger passed his palms over Lawrence's 
chest, arms and legs, holding them a few millimeters away from him. 
Lawrence's pan went nuts, intrusion detection sensors reporting 
multiple hostile reads of his identifiers, millimeter-wave radar scans, 
HERF attacks, and assorted shenanigans. All his feedback systems went 
to full alert, going from itchy, back-of-the-neck liminal sensations 
into high intensity pinches, prods and buzzes. It was a deeply alarming 
sensation, like his internal organs were under attack.

He choked out an incoherent syllable, and the Securitat man who was 
hand-wanding him raised a warning finger, holding it so close to his 
nose he went cross-eyed. He fell silent while the man continued to wand 
him, twitching a little to let his pan know that it was all OK.

“From the cult, then, are you?” the Securitat man said, after he'd 
kicked Lawrence's ankles apart and spread his hands on the side of the 
truck.

“That's right,” Lawrence said. “From the Order.” He jerked his 
head toward the gates, just a few tantalizing meters away. “I'm out 
--”

“You people are really something, you know that? You could have been 
\emph{killed}. Let me tell you a few things about how the world works: 
when you are approached by the Securitat, you stand still with your 
hands stretched straight out to either side. You do \emph{not} raise 
unidentified devices and point them at the officers. Not unless you're 
trying to commit suicide by cop. Is that what you're trying to do?”

“No,” Lawrence said. “No, of course not. I was just taking a 
picture for --”

“And you do \emph{not} photograph or log our security procedures. 
There's a war on, you know.” The man's forehead bunched together. 
“Oh, for shit's sake. We should take you in now, you know it? Tie up 
a dozen people's day, just to process you through the system. You could 
end up in a cell for, oh, I don't know, a month. You want that?”

“Of course not,” Lawrence said. “I didn't realize --”

“You didn't, but you \emph{should have}. If you're going to come 
walking around here where the real people are, you have to learn how to 
behave like a real person in the real world.”

The other man, who had been impassively holding Lawrence's wrists in a 
crushing grip, eased up. “Let him go?” he said.

The first officer shook his head. “If I were you, I would turn right 
around, walk through those gates, and never come out again. Do I make 
myself clear?”

Lawrence wasn't clear at all. Was the cop ordering him to go back? Or 
just giving him advice? Would he be arrested if he didn't go back in? 
It had been a long time since Lawrence had dealt with authority and the 
feeling wasn't a good one. His chest heaved, and sweat ran down his 
back, pooling around his ass, then moving in rivulets down the backs of 
his legs.

“I understand,” he said. Thinking: \emph{I understand that asking 
questions now would not be a good idea}.

\tb

The subway was more or less as he remembered it, though the long line 
of people waiting to get through the turnstiles turned out to be a line 
to go through a security checkpoint, complete with bag-search and 
X-ray. But the New Yorkers were the same -- no one made eye contact 
with anyone else, but if they did, everyone shared a kind of bitter 
shrug, as if to say, \emph{Ain't it the fuckin' truth?}

But the smell was the same -- oil and damp and bleach and the 
indefinable, human smell of a place where millions had passed for 
decades, where millions would pass for decades to come. He found 
himself standing before a subway map, looking at it, comparing it to 
the one in his memory to find the changes, the new stations that must 
have sprung up during his hiatus from reality.

But there weren't new stations. In fact, it seemed to him that there 
were a lot \emph{fewer} stations -- hadn't there been one at Bleeker 
Street and another at Cathedral Parkway? Yes, there had been -- but 
look now, they were gone, and... And there were stickers, white 
stickers over the places where the stations had been. He reached up and 
touched the one over Bleeker Street.

“I still can't get used to it, either,” said a voice at his side. 
“I used to change for the F Train there every day when I was a 
kid.” It was a woman, about the same age as Gerta, but more beaten 
down by the years, deeper creases in her face, a stoop in her stance. 
But her face was kind, her eyes soft.

“What happened to it?”

She took a half-step back from him. “Bleeker Street,” she said. 
“You know, Bleeker Street? Like 9/11? Bleeker Street?” Like the 
name of the station was an incantation.

It rang a bell. It wasn't like he didn't ever read the news, but it had 
a way of sliding off of you when you were on campus, as though it was 
some historical event in a book, not something happening right there, 
on the other side of the wall.

“I'm sorry,” he said. “I've been away. Bleeker Street, yes, of 
course.”

She gave him a squinty stare. “You must have been \emph{very} far 
away.”

He tried out a sheepish grin. “I'm a monk,” he said. “From the 
Order of Reflective Analytics. I've been out of the world for sixteen 
years. Until today, in fact. My name is Lawrence.” He stuck his hand 
out and she shook it like it was made of china.

“A monk,” she said. “That's very interesting. Well, you enjoy 
your little vacation.” She turned on her heel and walked quickly down 
the platform. He watched her for a moment, then turned back to the map, 
counting the missing stations.

\tb

When the train ground to a halt in the tunnel between 42nd and 50th 
street, the entire car let out a collective groan. When the lights 
flickered and went out, they groaned louder. The emergency lights came 
on in sickly green and an incomprehensible announcement played over the 
loudspeakers. Evidently, it was an order to evacuate, because the press 
of people began to struggle through the door at the front of the car, 
then further and further. Lawrence let the press of bodies move him too.

Once they reached the front of the train, they stepped down onto the 
tracks, each passenger turning silently to help the next, again with 
that \emph{Ain't it the fuckin' truth?} look. Lawrence turned to help 
the person behind him and saw that it was the woman who'd spoken to him 
on the platform. She smiled a little smile at him and turned with 
practiced ease to help the person behind her.

They walked single file on a narrow walkway beside the railings. 
Securitat officers were strung out at regular intervals, wearing night 
scopes and high, rubberized boots. They played flashlights over the 
walkers as they evacuated.

“Does this happen often?” Lawrence said over his shoulder. His 
words were absorbed by the dead subterranean air and he thought that 
she might not have heard him but then she sighed.

“Only every time there's an anomaly in the head-count -- when the 
system says there's too many or too few people in the trains. Maybe 
once a week.” He could feel her staring at the back of his head. He 
looked back at her and saw her shaking her head. He stumbled and went 
down on one knee, clanging his head against the stone walls made soft 
by a fur of condensed train exhaust, cobwebs and dust.

She helped him to his feet. “You don't seem like a snitch, Lawrence. 
But you're a monk. Are you going to turn me in for being suspicious?”

He took a second to parse this out. “I don't work for the 
Securitat,” he said. It seemed like the best way to answer.

She snorted. “That's not what we hear. Come on, they're going to 
start shouting at us if we don't move.”

They walked the rest of the way to an emergency staircase together, and 
emerged out of a sidewalk grating, blinking in the remains of the 
autumn sunlight, a bloody color on the glass of the highrises. She 
looked at him and made a face. “You're filthy, Lawrence.” She 
thumped at his sleeves and great dirty clouds rose off them. He looked 
down at the knees of his pants and saw that they were hung with boogers 
of dust.

The New Yorkers who streamed past them ducked to avoid the dirty 
clouds. “Where can I clean up?” he said.

“Where are you staying?”

“I was thinking I'd see about getting a room at the Y or a 
backpacker's hostel, somewhere to stay until I'm done.”

“Done?”

“I'm on a complicated errand. Trying to locate someone who used to be 
in the Order.”

Her face grew hard again. “No one gets out alive, huh?”

He felt himself blushing. “It's not like that. Wow, you've got 
strange ideas about us. I want to find this guy because he disappeared 
under mysterious circumstances and I want to --” How to explain 
Anomalies to an outsider? “It's a thing we do. Unravel mysteries. It 
makes us better people.”

“Better people?” She snorted again. “Better than what? Don't 
answer. Come on, I live near here. You can wash up at my place and be 
on your way. You're not going to get into any backpacker's hostel 
looking like you just crawled out of a sewer -- you're more likely to 
get detained for being an `indigent of suspicious character.'”

He let her steer him a few yards uptown. “You think that I work for 
the Securitat but you're inviting me into your home?”

She shook her head and led him around a corner, along a long crosstown 
block, and then turned back uptown. “No,” she said. “I think 
you're a confused stranger who is apt to get himself into some trouble 
if someone doesn't take you in hand and help you get smart, fast. It 
doesn't cost me anything to lend a hand, and you don't seem like the 
kind of guy who'd mug, rape and kill an old lady.”

\tb

“The discipline,” he said, “is all about keeping track of the way 
that the world is, and comparing it to your internal perceptions, all 
the time. When I entered the Order, I was really big. Fat, I mean. The 
discipline made me log every bit of food I ate, and I discovered a few 
important things: first, I was eating about 20 times a day, just 
grazing on whatever happened to be around. Second, that I was consuming 
about 4,000 calories a day, mostly in industrial sugars like 
high-fructose corn syrup. Just \emph{knowing} how I ate made a gigantic 
difference. I felt like I ate sensibly, always ordering a salad with 
lunch and dinner, but I missed the fact that I was glooping on half a 
cup of sweetened, high-fat dressing, and having a cookie or two every 
hour between lunch and dinner, and a half-pint of ice-cream before bed 
most nights.

“But it wasn't just food -- in the Order, we keep track of 
\emph{everything}; our typing patterns, our sleeping patterns, our 
moods, our reading habits. I discovered that I read faster when I've 
been sleeping more, so now, when I need to really get through a lot of 
reading, I make sure I sleep more. Used to be I'd try to stay up all 
night with pots of coffee to get the reading done. Of course, the more 
sleep-deprived I was, the slower I read; and the slower I read the more 
I needed to stay up to catch up with the reading. No wonder college was 
such a blur.

“So that's why I've stayed. It's empiricism, it's as old as Newton, 
as the Enlightenment.” He took another sip of his water, which tasted 
like New York tap water had always tasted (pretty good, in fact), and 
which he hadn't tasted for sixteen years. The woman was called Posy, 
and her old leather sofa was worn but well-loved, and smelled of saddle 
soap. She was watching him from a kitchen chair she'd brought around to 
the living room of the tiny apartment, rubbing her stockinged feet over 
the good wool carpet that showed a few old stains hiding beneath 
strategically placed furnishings and knick-knacks.

He had to tell her the rest, of course. You couldn't understand the 
Order unless you understood the rest. “I'm a screwup, Posy. Or at 
least, I was. We all were. Smart and motivated and promising, but just 
a wretched person to be around. Angry, bitter, all those smarts turned 
on biting the heads off of the people who were dumb enough to care 
about me or employ me. And so smart that I could talk myself into 
believing that it was all everyone else's fault, the idiots. It took 
instrumentation, empiricism, to get me to understand the patterns of my 
own life, to master my life, to become the person I wanted to be.”

“Well, you seem like a perfectly nice young man now,” Posy said.

That was clearly his cue to go, and he'd changed into a fresh set of 
trousers, but he couldn't go, not until he picked apart something she'd 
said earlier. “Why did you think I was a snitch?”

“I think you know that very well, Lawrence,” she said. “I can't 
imagine someone who's so into measuring and understanding the world 
could possibly have missed it.”

\emph{Now} he knew what she was talking about. “We just do contract 
work for the Securitat. It's just one of the ways the Order sustains 
itself.” The founders had gone into business refilling toner 
cartridges, which was like the 21st century equivalent of keeping bees 
or brewing dark, thick beer. They'd branched out into remote IT 
administration, then into data-mining and security, which was a natural 
for people with Order training. “But it's all anonymized. We don't 
snitch on people. We report on anomalous events. We do it for lots of 
different companies, too -- not just the Securitat.”

Posy walked over to the window behind her small dining room table, 
rolling away a couple of handsome old chairs on castors to reach it. 
She looked down over the billion lights of Manhattan, stretching all 
the way downtown to Brooklyn. She motioned to him to come over, and he 
squeezed in beside her. They were on the twenty-third floor, and it had 
been many years since he'd stood this high and looked down. The world 
is different from high up.

“There,” she said, pointing at an apartment building across the 
way. “There, you see it? With the broken windows?” He saw it, the 
windows covered in cardboard. “They took them away last week. I don't 
know why. You never know why. You become a person of interest and they 
take you away and then later, they always find a reason to keep you 
away.”

Lawrence's hackles were coming up. He found stuff that didn't belong in 
the data -- he didn't arrest people. “So if they always find a reason 
to keep you away, doesn't that mean --”

She looked like she wanted to slap him and he took a step back. 
“We're all guilty of something, Lawrence. That's how the game is 
rigged. Look closely at anyone's life and you'll find, what, a little 
black-marketeering, a copyright infringement, some cash economy 
business with unreported income, something obscene in your Internet 
use, something in your bloodstream that shouldn't be there. I bought 
that sofa from a \emph{cop}, Lawrence, bought it ten years ago when he 
was leaving the building. He didn't give me a receipt and didn't 
collect tax, and technically that makes us offenders.” She slapped 
the radiator. “I overrode the governor on this ten minutes after they 
installed it. Everyone does it. They make it easy -- you just stick a 
penny between two contacts and hey presto, the city can't turn your 
heat down anymore. They wouldn't make it so easy if they didn't expect 
everyone to do it -- and once everyone's done it, we're all guilty.

“The people across the street, they were Pakistani or maybe Sri 
Lankan or Bangladeshi. I'd see the wife at the service laundry. Nice 
professional lady, always lugging around a couple kids on their way to 
or from day-care. She --” Posy broke off and stared again. “I once 
saw her reach for her change and her sleeve rode up and there was a 
number tattooed there, there on her wrist.” Posy shuddered. “When 
they took her and her husband and their kids, she stood at the window 
and pounded at it and screamed for help. You could hear her from 
here.”

“That's terrible,” Lawrence said. “But what does it have to do 
with the Order?”

She sat back down. “For someone who is supposed to know himself, 
you're not very good at connecting the dots.”

Lawrence stood up. He felt an obscure need to apologize. Instead, he 
thanked her and put his glass in the sink. She shook his hand solemnly.

“Take care out there,” she said. “Good luck finding your 
escapee.”

\tb

Here's what Lawrence knew about Zbigniew Krotoski. He had been inducted 
into the Order four years earlier. He was a native-born New Yorker. He 
had spent his first two years in the Order trying to coax some of the 
elders into a variety of pointless flamewars about the ethics of 
working for the Securitat, and then had settled into being a very 
productive member. He spent his 20 percent time -- the time when each 
monk had to pursue non-work-related projects -- building aerial 
photography rigs out of box-kites and tiny cameras that the Monks 
installed on their systems to help them monitor their body mechanics 
and ergonomic posture.

Zbigkrot performed in the eighty-fifth percentile of the Order, which 
was respectable enough. Lawrence had started there and had crept up and 
down as low as 70 and as high as 88, depending on how he was doing in 
the rest of his life. Zbigkrot was active in the gardens, both the big 
ones where they grew their produce and a little allotment garden where 
he indulged in baroque cross-breeding experiments, which were in vogue 
among the monks then.

The Securitat stream to which he'd added 68 bytes was long gone, but it 
was the kind of thing that the Order handled on a routine basis: given 
the timing and other characteristics, Lawrence thought it was probably 
a stream of purchase data from hardware and grocery stores, to be 
inspected for unusual patterns that might indicate someone buying bomb 
ingredients. Zbigkrot had worked on this kind of data thousands of 
times before, six times just that day. He'd added the sixty-eight bytes 
and then left.

Zbigkrot once had a sister in New York -- that much could be 
ascertained. Anja Krotoski had lived on 23d Street in a co-op near 
Lexington. But that had been four years previous, when he'd joined the 
Order, and she wasn't there anymore. Her numbers all rang dead.

The apartment building had once been a pleasant, middle-class sort of 
place, with a red awning and a niche for a doorman. Now it had become 
more run down, the awning's edges frayed, one pane of lobby glass 
broken out and replaced with a sheet of cardboard. The doorman was long 
gone.

It seemed to Lawrence that this fate had befallen many of the City's 
buildings. They reminded him of the buildings he'd seen in Belgrade one 
time, when he'd been sent out to brief a gang of outsource programmers 
his boss had hired -- neglected for years, indifferently patched by 
residents who had limited access to materials.

It was the dinner hour, and a steady trickle of people were letting 
themselves into Anja's old building. Lawrence watched a couple of them 
enter the building and noticed something wonderful and sad: as they 
approached the building, their faces were the hard masks of 
city-dwellers, not meeting anyone's eye, clipping along at a fast pace 
that said, “Don't screw with me.” But once they passed the 
threshold of their building and the door closed behind them, their 
whole affect changed. They slumped, they smiled at one another, they 
leaned against the mailboxes and set down their bags and took off their 
hats and fluffed their hair and turned back into people.

He remembered that feeling from his life before, the sense of having 
two faces: the one he showed to the world and the one that he reserved 
for home. In the Order, he only wore one face, one that he knew in 
exquisite detail.

He approached the door now, and his pan started to throb ominously, 
letting him know that he was enduring hostile probes. The building 
wanted to know who he was and what business he had there, and it was 
attempting to fingerprint everything about him from his pan to his gait 
to his face.

He took up a position by the door and dialed back the pan's response to 
a dull pulse. He waited for a few minutes until one of the residents 
came down: a middle-aged man with a dog, a little sickly-looking 
schnauzer with grey in its muzzle.

“Can I help you?” the man said, from the other side of the security 
door, not unlatching it.

“I'm looking for Anja Krotoski,” he said. “I'm trying to track 
down her brother.”

The man looked him up and down. “Please step away from the door.”

He took a few steps back. “Does Ms Krotoski still live here?”

The man considered. “I'm sorry, sir, I can't help you.” He waited 
for Lawrence to react.

“You don't know, or you can't help me?”

“Don't wait under this awning. The police come if anyone waits under 
this awning for more than three minutes.”

The man opened the door and walked away with his dog.

\tb

His phone rang before the next resident arrived. He cocked his head to 
answer it, then remembered that his lifelogger was dead and dug in his 
jacket for a mic. There was one at his wrist pulse-points used by the 
health array. He unvelcroed it and held it to his mouth.

“Hello?”

“It's Gerta, boyo. Wanted to know how your Anomaly was going.”

“Not good,” he said. “I'm at the sister's place and they don't 
want to talk to me.”

“You're walking up to strangers and asking them about one of their 
neighbors, huh?”

He winced. “Put it that way, yeah, OK, I understand why this doesn't 
work. But Gerta, I feel like Rip Van Winkle here. I keep putting my 
foot in it. It's so different.”

“People are people, Lawrence. Every bad behavior and every good one 
lurks within us. They were all there when you were in the world -- in 
different proportion, with different triggers. But all there. You know 
yourself very well. Can you observe the people around you with the same 
keen attention?”

He felt slightly put upon. “That's what I'm trying --”

“Then you'll get there eventually. What, you're in a hurry?”

Well, no. He didn't have any kind of timeline. Some people chased 
Anomalies for \emph{years}. But truth be told, he wanted to get out of 
the City and back onto campus. “I'm thinking of coming back to Campus 
to sleep.”

Gerta clucked. “Don't give in to the agoraphobia, Lawrence. Hang in 
there. You haven't even heard my news yet, and you're already ready to 
give up?”

“What news? And I'm not giving up, just want to sleep in my own bed 
--”

“The entry checkpoints, Lawrence. You can\emph{not} do this job if 
you're going to spend four hours a day in security queues. Anyway, the 
news.

“It wasn't the first time he did it. I've been running the logs back 
three years and I've found at least a dozen streams that he tampered 
with. Each time he used a different technique. This was the first time 
we caught him. Used some pretty subtle tripwires when he did it, so 
he'd know if anyone ever caught on. Must have spent his whole life 
living on edge, waiting for that moment, waiting to bug out. Must have 
been a hard life.”

“What was he doing? Spying?”

“Most assuredly,” Gerta said. “But for whom? For the enemy? The 
Securitat?”

They'd considered going to the Securitat with the information, but why 
bother? The Order did business with the Securitat, but tried never to 
interact with them on any other terms. The Securitat and the Order had 
an implicit understanding: so long as the Order was performing 
excellent data-analysis, it didn't have to fret the kind of overt 
scrutiny that prevailed in the real world. Undoubtedly, the Securitat 
kept satellite eyes, data-snoopers, wiretaps, millimeter radar and 
every other conceivable surveillance trained on each Campus in the 
world, but at the end of the day, they were just badly socialized geeks 
who'd left the world, and useful geeks at that. The Securitat treated 
the Order the way that Lawrence's old bosses treated the company 
sysadmins: expendable geeks who no one cared about -- so long as 
nothing went wrong.

No, there was no sense in telling the Securitat about the 68 bytes.

“Why would the Securitat poison its own data-streams?”

“You know that when the Soviets pulled out of Finland, they found 40 
\emph{kilometers} of wire-tapping wire in KGB headquarters? The 
building was only 12 storeys tall! Spying begets spying. The worst, 
most dangerous enemy the Securitat has is the Securitat.”

There were Securitat vans on the street around him, going past every 
now and again, eerily silent engines, playing their cheerful music. He 
stepped back into shadow, then thought better of it and stood under a 
pool of light.

“OK, so it was a habit. How do I find him? No one in the sister's 
building will talk to me.”

“You need to put them at their ease. Tell them the truth, that often 
works.”

“You know how people feel about the Order out here?” He thought of 
Posy. “I don't know if the truth is going to work here.”

“You've been in the Order for sixteen years. You're not just some 
fumble-tongued outcast anymore. Go talk to them.”

“But --”

“Go, Lawrence. Go. You're a smart guy, you'll figure it out.”

He went. Residents were coming home every few minutes now, carrying 
grocery bags, walking dogs, or dragging their tired feet. He almost 
approached a young woman, then figured that she wouldn't want to talk 
to a strange man on the street at night. He picked a guy in his 
thirties, wearing jeans and a huge old vintage coat that looked like it 
had come off the eastern front.

“Scuse me,” he said. “I'm trying to find someone who used to live 
here.”

The guy stopped and looked Lawrence up and down. He had a handsome 
sweater on underneath his coat, design-y and cosmopolitan, the kind of 
thing that made Lawrence think of Milan or Paris. Lawrence was keenly 
aware of his generic Order-issued suit, a brown, rumpled, ill-fitting 
thing, topped with a polymer coat that, while warm, hardly flattered.

“Good luck with that,” he said, then started to move past.

“Please,” Lawrence said. “I'm -- I'm not used to how things are 
around here. There's probably some way I could ask you this that would 
put you at your ease, but I don't know what it is. I'm not good with 
people. But I really need to find this person, she used to live here.”

The man stopped, looked at him again. He seemed to recognize something 
in Lawrence, or maybe it was that he was disarmed by Lawrence's honesty.

“Why would you want to do that?”

“It's a long story,” he said. “Basically, though: I'm a monk from 
the Order of Reflective Analytics and one of our guys has disappeared. 
His sister used to live here -- maybe she still does -- and I wanted to 
ask her if she knew where I could find him.”

“Let me guess, none of my neighbors wanted to help you.”

“You're only the second guy I've asked, but yeah, pretty much.”

“Out here in the real world, we don't really talk about each other to 
strangers. Too much like being a snitch. Lucky for you, my sister's in 
the Order, out in Oregon, so I know you're not all a bunch of snoops 
and stoolies. Who're you looking for?”

Lawrence felt a rush of gratitude for this man. “Anja Krotosky, 
number 11-J?”

“Oh,” the man said. “Well, yeah, I can see why you'd have a hard 
time with the neighbors when it comes to old Anja. She was well-liked 
around here, before she went.”

“Where'd she go? When?”

“What's your name, friend?”

“Lawrence.”

“Lawrence, Anja \emph{went}. Middle of the night kind of thing. No 
one heard a thing. The CCTVs stopped working that night. Nothing on the 
drive the next day. No footage at all.”

“Like she skipped out?”

“They stopped delivering flyers to her door. There's only one power 
stronger than direct marketing.”

“The Securitat took her?”

“That's what we figured. Nothing left in her place. Not a stick of 
furniture. We don't talk about it much. Not the thing that it pays to 
take an interest in.”

“How long ago?”

“Two years ago,” he said. A few more residents pushed past them. 
“Listen, I approve of what you people do in there, more or less. It's 
good that there's a place for the people who don't -- you know, who 
don't have a place out here. But the way you make your living. I told 
my sister about this, the last time she visited, and she got very angry 
with me. She didn't see the difference between watching yourself and 
being watched.”

Lawrence nodded. “Well, that's true enough. We don't draw a really 
sharp distinction. We all get to see one another's stats. It keeps us 
honest.”

“That's fine, if you have the choice. But --” He broke off, looking 
self-conscious. Lawrence reminded himself that they were on a public 
street, the cameras on them, people passing by. Was one of them a 
snitch? The Securitat had talked about putting him away for a month, 
just for logging them. They could watch him all they wanted, but he 
couldn't look at them.

“I see the point.” He sighed. He was cold and it was full autumn 
dark now. He still didn't have a room for the night and he didn't have 
any idea how he'd find Anja, much less zbigkrot. He began to understand 
why Anomalies were such a big deal.

\tb

He'd walked 18,453 steps that day, about triple what he did on campus. 
His heart rate had spiked several times, but not from exertion. Stress. 
He could feel it in his muscles now. He should really do some 
biofeedback, try to calm down, then run back his lifelogger and make 
some notes on how he'd reacted to people through the day.

But the lifelogger was gone and he barely managed 22 seconds his first 
time on the biofeedback. His next ten scores were much worse.

It was the hotel room. It had once been an office, and before that, it 
had been half a hotel-room. There were still scuff-marks on the floor 
from where the wheeled office chair had dug into the scratched lino. 
The false wall that divided the room in half was thin as paper and 
Lawrence could hear every snuffle from the other side. The door to 
Lawrence's room had been rudely hacked in, and weak light shone through 
an irregular crack over the jamb.

The old New Yorker Hotel had seen better days, but it was what he could 
afford, and it was central, and he could hear New York outside the 
window -- he'd gotten the half of the hotel room with the window in it. 
The lights twinkled just as he remembered them, and he still got a 
swimmy, vertiginous feeling when he looked down from the great height.

The clerk had taken his photo and biometrics and had handed him a 
tracker-key that his pan was monitoring with tangible suspicion. It 
radiated his identity every few yards, and in the elevator. It even 
seemed to track which part of the minuscule room he was in. What the 
hell did the hotel do with all this information?

Oh, right -- it shipped it off to the Securitat, who shipped it to the 
Order, where it was processed for suspicious anomalies. No wonder there 
was so much work for them on campus. Multiply the New Yorker times a 
hundred thousand hotels, two hundred thousand schools, a million cabs 
across the nation -- there was no danger of the Order running out of 
work.

The hotel's network tried to keep him from establishing a secure 
connection back to the Order's network, but the Order's countermeasures 
were better than the half-assed ones at the hotel. It took a lot of 
tunneling and wrapping, but in short measure he had a strong private 
line back to the Campus -- albeit a slow line, what with all the 
jiggery-pokery he had to go through.

Gerta had left him with her file on zbigkrot and his activities on the 
network. He had several known associates on Campus, people he ate with 
or played on intramural teams with, or did a little extreme programming 
with. Gerta had bulk-messaged them all with an oblique query about his 
personal life and had forwarded the responses to Lawrence. There was a 
mountain of them, and he started to plow through them.

He started by compiling stats on them -- length, vocabulary, number of 
paragraphs -- and then started with the outliers. The shortest ones 
were polite shrugs, apologies, don't have anything to say. The long 
ones -- whew! They sorted into two categories: general whining, mostly 
from noobs who were still getting accustomed to the way of the Order; 
and protracted complaints from old hands who'd worked with zbigkrot 
long enough to decide that he was incorrigible. Lawrence sorted these 
quickly, then took a glance at the median responses and confirmed that 
they appeared to be largely unhelpful generalizations of the sort that 
you might produce on a co-worker evaluation form -- a proliferation of 
null adjectives like “satisfactory,” “pleasant,” “fine.”

Somewhere in this haystack -- Lawrence did a quick word-count and came 
back with 140,000 words, about two good novels' worth of reading -- was 
a needle, a clue that would show him the way to unravel the Anomaly. It 
would take him a couple days at least to sort through it all in depth. 
He ducked downstairs and bought some groceries at an all-night grocery 
store in Penn Station and went back to his room, ready to settle in and 
get the work done. He could use a few days' holiday from New York, 
anyway.

\tb

> About time Zee Big Noob did a runner. He never had a moment's 
happiness here, and I never figured out why he'd bother hanging around 
when he hated it all so much.

> Ever meet the kind of guy who wanted to tell you just how much you 
shouldn't be enjoying the things you enjoy? The kind of guy who could 
explain, in detail, \emph{exactly} why your passions were stupid? That 
was him.

> “Brother Antony, why are you wasting your time collecting tin toys? 
They're badly made, unlovely, and represent, at best, a history of 
slave labor, starting with your cherished `Made in Occupied Japan,' 
tanks. Christ, why not collect rape-camp macrame while you're at it?” 
He had choice words for all of us about our passions, but I was singled 
out because I liked to extreme program in my room, which I'd spent a 
lot of time decorating. (See pic, below, and yes, I built and sanded 
and mounted every one of those shelves by hand) (See magnification shot 
for detail on the joinery. Couldn't even drive a nail when I got here) 
(Not that there are any nails in there, it's all precision-fitted 
tongue and groove) (holy moley, lasers totally rock)

> But he reserved his worst criticism for the Order itself. You know 
the litany: we're a cult, we're brainwashed, we're dupes of the 
Securitat. He was convinced that every instrument in the place was 
feeding up to the Securitat itself. He'd mutter about this constantly, 
whenever we got a new stream to work on -- “Is this your lifelog, 
Brother Antony? Mine? The number of flushes per shitter in the west 
wing of campus?”

> And it was no good trying to reason with him. He just didn't 
acknowledge the benefit of introspection. “It's no different from 
them,” he'd say, jerking his thumb up at the ceiling, as though there 
was a Securitat mic and camera hidden there. “You're just flooding 
yourself with useless information, trying to find the useful parts. Why 
not make some predictions about which part of your life you need to pay 
attention to, rather than spying on every process? You're a spy in your 
own body.

> So why did I work with him? I'll tell you: first, he was a shit-hot 
programmer. I know his stats say he was way down in the 78th 
percentile, but he could make every line of code that \emph{I} wrote 
smarter. We just don't have a way of measuring that kind of effect 
(yes, someone should write one; I've been noodling with a framework for 
it for months now).

> Second, there was something dreadfully fun about listening him light 
into \emph{other} people, \emph{their} ridiculous passions and 
interests. He could be incredibly funny, and he was incisive if not 
insightful. It's shameful, but there you have it. I am imperfect.

> Finally, when he wasn't being a dick, he was a good guy to have in 
your corner. He was our rugby team's fullback, the baseball team's 
shortstop, the tank on our MMOG raids. You could rely on him.

> So I'm going to miss him, weirdly. If he's gone for good. I wouldn't 
put it past him to stroll back onto campus someday and say, “What, 
what? I just took a little French Leave. Jesus, overreact much?”

Plenty of the notes ran in this direction, but this was the most 
articulate. Lawrence read it through three times before adding it to 
the file of useful stuff. It was a small pile. Still, Gerta kept 
forwarding him responses. The late responders had some useful things to 
say:

> He mentioned a sister. Only once. A whole bunch of us were talking 
about how our families were really supportive of our coming to the 
Order, and after it had gone round the whole circle, he just kind of 
looked at the sky and said, “My sister thought I was an idiot to go 
inside. I asked her what she thought I should do and she said, `If I 
was you, kid, I'd just disappear before someone disappeared me.'” 
Naturally we all wanted to know what he meant by that. “I'm not very 
good at bullshitting, and that's a vital skill in today's world. She 
was better at it than me, when she worked at it, but she was the kind 
of person who'd let her guard slip every now and then.”

Lawrence noted that zbigkrot had used the past-tense to describe his 
sister. He'd have known about her being disappeared then.

He stared at the walls of his hotel room. The room next door was now 
occupied by at least four people and he couldn't even imagine how you'd 
get that many people inside -- he didn't know how four people could all 
\emph{stand} in the room, let alone lie down and sleep. But there were 
definitely four voices from next door, talking in Chinese.

New York was outside the window and far below, and the sun had come up 
far enough that everything was bright and reflective, the cars and the 
buildings and the glints from sunglasses far below. He wasn't getting 
anywhere with the docs, the sister, the data streams. And there was New 
York, just outside the window.

He dug under the bed and excavated his boots, recoiling from them with 
soft, dust-furred old socks and worse underneath the mattress.

\tb

The Securitat man pointed to Lawrence as he walked past Penn Station. 
Lawrence stopped and pointed at himself in a who-me? gesture. The 
Securitat man pointed again, then pointed to his alcove next to the 
entrance.

Lawrence's pan didn't like the Securitat man's incursions and tried to 
wipe itself.

“Sir,” he said. “My pan is going nuts. May I put down my arms so 
I can tell it to let you in?”

The Securitat man acted as though he hadn't heard, just continued to 
wave his hands slowly over Lawrence's body.

“Come with me,” the Securitat man said, pointing to the door on the 
other side of the alcove that led into a narrow corridor, into the 
bowels of Penn Station. The door let out onto the concourse, thronged 
with people shoving past each other, disgorged by train after train. 
Though none made eye contact with them or each other, they parted 
magically before them, leaving them with a clear path.

Lawrence's pan was not helping him. Every inch of his body itched as it 
nagged at him about the depredations it was facing from the station and 
the Securitat man. This put him seriously on edge and made his heart 
and breathing go crazy, triggering another round of warnings from his 
pan, which wanted him to calm down, but wouldn't help. This was a bad 
failure mode, one he'd never experienced before. He'd have to file a 
bug report.

Some day.

The Securitat's outpost in Penn Station was as clean as a dentist's 
office, but with mesh-reinforced windows and locks that made three 
distinct clicks and a soft hiss when the door closed. The Securitat man 
impersonally shackled Lawrence to a plastic chair that was bolted into 
the floor and then went off to a check-in kiosk that he whispered into 
and prodded at. There was no one else in evidence, but there were huge 
CCTV cameras, so big that they seemed to be throwbacks to an earlier 
era, some paleolithic ancestor of the modern camera. These cameras were 
so big because they were meant to be seen, meant to let you know that 
you were being watched.

The Securitat man took him away again, stood him in an interview room 
where the cameras were once again in voluble evidence.

“Explain everything,” the Securitat man said. He rolled up his mask 
so that Lawrence could see his face, young and hard. He'd been in 
diapers when Lawrence went into the Order.

And so Lawrence began to explain, but he didn't want to explain 
everything. Telling this man about zbigkrot tampering with Securitat 
data-streams would not be good; telling him about the disappearance of 
Anja Krotoski would be even worse. So -- he lied. He was already so 
stressed out that there was no way the lies would register as 
extraordinary to the sensors that were doubtless trained on him.

He told the Securitat man that he was in the world to find an Order 
member who'd taken his leave, because the Order wanted to talk to him 
about coming back. He told the man that he'd been trying to locate 
zbigkrot by following up on his old contacts. He told the Securitat man 
that he expected to find zbigkrot within a day or two and would be 
going back to the Order. He implied that he was crucial to the Order 
and that he worked for the Securitat all the time, that he and the 
Securitat man were on the same fundamental mission, on the same team.

The Securitat man's face remained an impassive mask throughout. He 
touched an earbead from time to time, cocking his head slightly to 
listen. Someone else was listening to Lawrence's testimony and feeding 
him more material.

The Securitat man scooted his chair closer to Lawrence, leaned in 
close, searching his face. “We don't have any record of this Krotoski 
person,” he said. “I advise you to go home and forget about him.”

The words were said without any inflection at all, and that was 
scariest of all -- Lawrence had no doubt about what this meant. There 
were no records because Zbigniew Krotoski was erased.

Lawrence wondered what he was supposed to say to this armed child now. 
Did he lay his finger alongside of his nose and wink? Apologize for 
wasting his time? Everyone told him to listen before he spoke here. 
Should he just wait?

“Thank you for telling me so,” he said. “I appreciate the 
advice.” He hoped it didn't sound sarcastic.

The Securitat man nodded. “You need to adjust the settings on your 
pan. It reads like it's got something to hide. Here in the world, it 
has to accede to lawful read attempts without hesitation. Will you 
configure it?”

Lawrence nodded vigorously. While he'd recounted his story, he'd 
imagined spending a month in a cell while the Securitat looked into his 
deeds and history. Now it seemed like he might be on the streets in a 
matter of minutes.

“Thank you for your cooperation.” The man didn't say it. It was a 
recording, played by hidden speakers, triggered by some unseen agency, 
and on hearing it, the Securitat man stood and opened the door, waiting 
for the three distinct clicks and the hiss before tugging at the handle.

They stood before the door to the guard's niche in front of Penn 
Station and the man rolled up his mask again. This time he was smiling 
an easy smile and the hardness had melted a little from around his 
eyes. “You want a tip, buddy?”

“Sure.”

“Look, this is New York. We all just want to get along here. There's 
a lot of bad guys out there. They got some kind of beef. They want to 
fuck with us. We don't want to let them do that. You want to be safe 
here, you got to show New York that you're not a bad guy. That you're 
not here to fuck with us. We're the city's protectors, and we can spot 
someone who doesn't belong here the way your body can spot a cold-germ. 
The way you're walking around here, looking around, acting -- I could 
tell you didn't belong from a hundred yards. You want to avoid trouble, 
you get less strange, fast. You get me?”

“I get you,” he said. “Thank you, sir.” Before the Securitat 
man could say any more, Lawrence was on his way.

\tb

The man from Anja's building had a different sweater on, but the new 
one -- bulky wool the color of good chocolate -- was every bit as 
handsome as the one he'd had on before. He was wearing some kind of 
citrusy cologne and his hair fell around his ears in little waves that 
looked so natural they had to be fake. Lawrence saw him across the 
Starbucks and had a crazy urge to duck away and change into better 
clothes, just so he wouldn't look like such a fucking hayseed next to 
this guy. \emph{I'm a New Yorker,} he thought, \emph{or at least I was. 
I belong here.}

“Hey, Lawrence, fancy meeting you here!” He shook Lawrence's hand 
and gave him a wry, you-and-me-in-it-together smile. “How's the 
vision quest coming?”

“Huh?”

“The Anomaly -- that's what you're chasing, aren't you? It's your 
little rite of passage. My sister had one last year. Figured out that 
some guy who travelled from Fort Worth to Portland, Oregon, every week 
was actually a fictional construct invented by cargo smugglers who used 
his seat to plant a series of mules running heroin and cash. She was so 
proud afterwards that I couldn't get her to shut up about it. You had 
the holy fire the other night when I saw you.”

Lawrence felt himself blushing. “It's not really `holy' -- all that 
religious stuff, it's just a metaphor. We're not really spiritual.”

“Oh, the distinction between the spiritual and the material is pretty 
arbitrary anyway. Don't worry, I don't think you're a cultist or 
anything. No more than any of us, anyway. So, how's it going?”

“I think it's over,” he said. “Dead end. Maybe I'll get an easier 
Anomaly next time.”

“Sounds awful! I didn't think you were allowed to give up on 
Anomalies?”

Lawrence looked around to see if anyone was listening to them. “This 
one leads to the Securitat,” he said. “In a sense, you could say 
that I've solved it. I think the guy I'm looking for ended up with his 
sister.”

The man's expression froze, not moving one iota. “You must be 
disappointed,” he said, in neutral tones. “Oh well.” He leaned 
over the condiment bar to get a napkin and wrestled with the dispenser 
for a moment. It didn't cooperate, and he ended up holding fifty 
napkins. He made a disgusted noise and said, “Can you help me get 
these back into the dispenser?”

Lawrence pushed at the dispenser and let the man feed it his excess 
napkins, arranging them neatly. While he did this, he contrived to hand 
Lawrence a card, which Lawrence cupped in his palm and then ditched 
into his inside jacket pocket under the pretense of reaching in to 
adjust his pan.

“Thanks,” the man said. “Well, I guess you'll be going back to 
your campus now?”

“In the morning,” Lawrence said. “I figured I'd see some New York 
first. Play tourist, catch a Broadway show.”

The man laughed. “All right then -- you enjoy it.” He did nothing 
significant as he shook Lawrence's hand and left, holding his paper 
cup. He did nothing to indicate that he'd just brought Lawrence into 
some kind of illegal conspiracy.

Lawrence read the note later, on a bench in Bryant Park, holding a 
paper bag of roasted chestnuts and fastidiously piling the husks next 
to him as he peeled them away. It was a neatly cut rectangle of card 
sliced from a health-food cereal box. Lettered on the back of it in 
pencil were two short lines:

Wednesdays 8:30PM Half Moon Cafe 164 2nd Ave

The address was on the Lower East Side, a neighborhood that had been 
scorchingly trendy the last time Lawrence had been there. More 
importantly: it was Wednesday.

\tb

The Half Moon Cafe turned out to be one of those New York places that 
are so incredibly hip they don't have a sign or any outward indication 
of their existence. Number 164 was a frosted glass door between a 
dry-cleaner's and a Pakistani grocery store, propped open with a 
squashed Mountain Dew can. Lawrence opened the door, heart pounding, 
and slipped inside. A long, dark corridor stretched away before him, 
with a single door at the end, open a crack, dim light spilling out of 
it. He walked quickly down the corridor, sure that there were cameras 
observing him.

The door at the end of the hallway had a sheet of paper on it, with 
HALF MOON CAFE laser-printed in its center. Good food smells came from 
behind it, and the clink of cutlery, and soft conversation. He nudged 
it open and found himself in a dim, flickering room lit by candles and 
draped with gathered curtains that turned the walls into the proscenia 
of a grand and ancient stage. There were four or five small tables and 
a long one at the back of the room, crowded with people, with wine in 
ice-buckets at either end.

A very pretty girl stood at the podium before him, dressed in a 
conservative suit, but with her hair shaved into a half-inch brush of 
electric blue. She lifted an eyebrow at him as though she was sharing a 
joke with him and said, “Welcome to the Half Moon. Do you have a 
reservation?”

Lawrence had carefully shredded the bit of cardboard and dropped its 
tatters in six different trash cans, feeling like a real spy as he did 
so (and realizing at the same time that going to all these different 
cans was probably anomalous enough in itself to draw suspicion).

“A friend told me he'd meet me here,” he said.

“What was your friend's name?”

Lawrence stuck his chin in the top of his coat to tell his pan to stop 
warning him that he was breathing too shallowly. “I don't know,” he 
said. He craned his neck to look behind her at the tables. He couldn't 
see the man, but it was so dark in the restaurant --

“You made it, huh?” The man had yet another fantastic sweater on, 
this one with a tight herringbone weave and ribbing down the sleeves. 
He caught Lawrence sizing him up and grinned. “My weakness -- the 
world's wool farmers would starve if it wasn't for me.” He patted the 
greeter on the hand. “He's at our table.” She gave Lawrence a 
knowing smile and the tiniest hint of a wink.

“Nice of you to come,” he said as they threaded their way slowly 
through the crowded tables, past couples having murmured conversations 
over candlelight, intense business dinners, an old couple eating in 
silence with evident relish. “Especially as it's your last night in 
the city.”

“What kind of restaurant is this?”

“Oh, it's not any kind of restaurant at all. Private kitchen. Ormund, 
he owns the place and cooks like a wizard. He runs this little place 
off the books for his friends to eat in. We come every Wednesday. 
That's his vegan night. You'd be amazed with what that guy can do with 
some greens and a sweet potato. And the cacao nib and avocado chili 
chocolate is something else.”

The large table was crowded with men and women in their thirties, 
people who had the look of belonging. They dressed well in fabrics that 
draped or clung like someone had thought about it, with jewelry that 
combined old pieces of brass with modern plastics and heavy clay beads 
that clicked like pool-balls. The women were beautiful or at least 
handsome -- one woman with cheekbones like snowplows and a jawline as 
long as a ski-slope was possibly the most striking person he'd ever 
seen up close. The men were handsome or at least craggy, with three-day 
beards or neat, full mustaches. They were talking in twos and threes, 
passing around overflowing dishes of steaming greens and oranges and 
browns, chatting and forking by turns.

“Everyone, I'd like you to meet my guest for the evening.” The man 
gestured at Lawrence. Lawrence had told the man his name, but he made 
it seem like he was being gracious and letting Lawrence introduce 
himself.

“Lawrence,” he said, giving a little wave. “Just in New York for 
one more night,” he said, still waving. He stopped waving. The 
closest people -- including the striking woman with the cheekbones -- 
waved back, smiling. The furthest people stopped talking and tipped 
their forks at him or at least cocked their heads.

“Sara,” the cheekbones woman said, pronouncing the first “a” 
long, “Sah-rah,” and making it sound unpretentious. The low-key 
buzzing from Lawrence's pan warned him that he was still overwrought, 
breathing badly, heart thudding. Who were these people?

“And I'm Randy,” the man said. “Sorry, I should have said that 
sooner.”

The food was passed down to his end. It was delicious, almost as good 
as the food at the campus, which was saying something -- there was a 
dedicated cadre of cooks there who made gastronomy their 20 percent 
projects, using elaborate computational models to create dishes that 
were always different and always delicious.

The big difference was the company. These people didn't have to retreat 
to belong, they belonged right here. Sara told him about her job 
managing a specialist antiquarian bookstore and there were a hundred 
stories about her customers and their funny ways. Randy worked at an 
architectural design firm and he had done some work at Sara's 
bookstore. Down the table there were actors and waiters and an 
insurance person and someone who did something in city government, and 
they all ate and talked and made him feel like he was a different kind 
of man, the kind of man who could live on the outside.

The coals of the conversation banked over port and coffees as they 
drifted away in twos and threes. Sara was the last to leave and she 
gave him a little hug and a kiss on the cheek. “Safe travels, 
Lawrence.” Her perfume was like an orange on Christmas morning, 
something from his childhood. He hadn't thought of his childhood in 
decades.

Randy and he looked at each other over the litter on the table. The 
server brought a check over on a small silver tray and Randy took a 
quick look at it. He drew a wad of twenties in a bulldog clip out of 
his inside coat pocket and counted off a large stack, then handed the 
tray to the server, all before Lawrence could even dig in his pocket.

“Please let me contribute,” he managed, just as the server 
disappeared.

“Not necessary,” Randy said, setting the clip down on the table. 
There was still a rather thick wad of money there. Lawrence hadn't been 
much of a cash user before he went into the Order and he'd seen hardly 
any spent since he came back out into the world. It seemed rather 
antiquarian, with its elaborate engraving. But the notes were crisp, as 
though freshly minted. The government still pressed the notes, even if 
they were hardly used any longer. “I can afford it.”

“It was a very fine dinner. You have interesting friends.”

“Sara is lovely,” he said. “She and I -- well, we had a thing 
once. She's a remarkable person. Of course, you're a remarkable person, 
too, Lawrence.”

Lawrence's pan reminded him again that he was getting edgy. He shushed 
it.

“You're smart, we know that. 88th percentile. Looks like you could go 
higher, judging from the work we've evaluated for you. I can't say as 
your performance as a private eye is very good, though. If I hadn't 
intervened, you'd still be standing outside Anja's apartment building 
harassing her neighbors.”

His pan was ready to call for an ambulance. Lawrence looked down and 
saw his hands clenched into fists. “You're Securitat,” he said.

“Let me put it this way,” the man said, leaning back. “I'm not 
one of Anja's neighbors.”

“You're Securitat,” Lawrence said again. “I haven't done anything 
wrong --”

“You came here,” Randy said. “You had every reason to believe 
that you were taking part in something illegal. You lied to the 
Securitat man at Penn Station today --”

Lawrence switched his pan's feedback mechanisms off altogether. Posy, 
at her window, a penny stuck in the governor of her radiator, rose in 
his mind.

“Everyone was treating me like a criminal -- from the minute I 
stepped out of the Order, you all treated me like a criminal. That made 
me act like one -- everyone has to act like a criminal here. That's the 
hypocrisy of the world, that honest people end up acting like crooks 
because the world treats them like crooks.”

“Maybe we treat them like crooks because they act so crooked.”

“You've got it all backwards,” Lawrence said. “The causal arrow 
runs the other direction. You treat us like criminals and the only way 
to get by is to act criminal. If I'd told the Securitat man in Penn 
Station the truth --”

“You build a wall around the Order, don't you? To keep us out, 
because we're barbarians? To keep you in, because you're too fragile? 
What does that treatment do, Lawrence?”

Lawrence slapped his hand on the table and the crystal rang, but no one 
in the restaurant noticed. They were all studiously ignoring them. 
“It's to keep \emph{you} out! All of you, who treated us --”

Randy stood up from the table. Bulky figures stepped out of the shadows 
behind them. Behind their armor, the Securitat people could have been 
white or black, old or young. Lawrence could only treat them as 
Securitat. He rose slowly from his chair and put his arms out, as 
though surrendering. As soon as the Securitat officers relaxed by a 
tiny hair -- treating him as someone who was surrendering -- he dropped 
backwards over the chair behind him, knocking over a little two-seat 
table and whacking his head on the floor so hard it rang like a gong. 
He scrambled to his feet and charged pell-mell for the door, sweeping 
the empty tables out of the way as he ran.

He caught a glimpse of the pretty waitress standing by her podium at 
the front of the restaurant as he banged out the door, her eyes wide 
and her hands up as though to ward off a blow. He caromed off the wall 
of the dark corridor and ran for the glass door that led out to Second 
Avenue, where cars hissed by in the night.

He made it onto the sidewalk, crashed into a burly man in a Mets cap, 
bounced off him, and ran downtown, the people on the sidewalk leaping 
clear of him. He made it two whole storefronts -- all the running 
around on the Campus handball courts had given him a pretty good pace 
and wind -- before someone tackled him from behind.

He scrambled and squirmed and turned around. It was the guy in the Mets 
hat. His breath smelled of onions and he was panting, his lips pulled 
back. “Watch where you're going --” he said, and then he was lifted 
free, jerked to his feet.

The blood sang in Lawrence's ears and he had just enough time to 
register that the big guy had been lifted by two blank, armored 
Securitat officers before he flipped over onto his knees and used the 
posture like a runner's crouch to take off again. He got maybe ten feet 
before he was clobbered by a bolt of lightning that made every muscle 
in his body lock into rigid agony. He pitched forward face-first, not 
feeling anything except the terrible electric fire from the taser-bolt 
in his back. His pan died with a sizzle up and down every haptic point 
in his suit, and between that and the electricity, he flung his arms 
and legs out in an agonized X while his neck thrashed, grating his face 
over the sidewalk. Something went horribly \emph{crunch} in his nose.

\tb

The room had the same kind of locks as the Securitat room in Penn 
Station. He'd awakened in the corner of the room, his face taped up and 
aching. There was no toilet, but there was a chair, bolted to the 
floor, and three prominent video cameras.

They left him there for some time, alone with his thoughts and the 
deepening throb from his face, his knees, the palms of his hands. His 
hands and knees had been sanded raw and there was grit and glass and 
bits of pebble embedded under the skin, which oozed blood.

His thoughts wanted to return to the predicament. They wanted to fill 
him with despair for his situation. They wanted to make him panic and 
weep with the anticipation of the cells, the confession, the life he'd 
had and the life he would get.

He didn't let them. He had spent sixteen years mastering his thoughts 
and he would master them now. He breathed deeply, noticing the places 
where his body was tight and trembling, thinking each muscle into 
tranquility, even his aching face, letting his jaw drop open.

Every time his thoughts went back to the predicament, he scrawled their 
anxious message on a streamer of mental ribbon which he allowed to slip 
through his mental fingers and sail away.

Sixteen years of doing this had made him an expert, and even so, it was 
not easy. The worries rose and streamed away as fast as his mind's hand 
could write them. But as always, he was finally able to master his 
mind, to find relaxation and calm at the bottom of the thrashing, 
churning vat of despair.

When Randy came in, Lawrence heard each bolt click and the hiss of air 
as from a great distance, and he surfaced from his calm, watching Randy 
cross the floor bearing his own chair.

“Innocent people don't run, Lawrence.”

“That's a rather self-serving hypothesis,” Lawrence said. The cool 
ribbons of worry slithered through his mind like satin, floating off 
into the ether around them. “You appear to have made up your mind, 
though. I wonder at you -- you don't seem like an idiot. How've you 
managed to convince yourself that this --” he gestured around at the 
room “-- is a good idea? I mean, this is just --”

Randy waved him silent. “The interrogation in this room flows in one 
direction, Lawrence. This is not a dialogue.”

“Have you ever noticed that when you're uncomfortable with something, 
you talk louder and lean forward a little? A lot of people have that 
tell.”

“Do you work with Securitat data-streams, Lawrence?”

“I work with large amounts of data, including a lot of material from 
the Securitat. It's rarely in cleartext, though. Mostly I'm doing 
sigint -- signals intelligence. I analyze the timing, frequency and 
length of different kinds of data to see if I can spot anomalies. 
That's with a lower-case `a', by the way.” He was warming up to the 
subject now. His face hurt when he talked, but when he thought about 
what to say, the hurt went away, as did the vision of the cell where he 
would go next. “It's the kind of thing that works best when you don't 
know what's in the payload of the data you're looking at. That would 
just distract me. It's like a magician's trick with a rabbit or a glass 
of water. You focus on the rabbit or on the water and what you expect 
of them, and are flummoxed when the magician does something unexpected. 
If he used pebbles, though, it might seem absolutely ordinary.”

“Do you know what Zbigniew Krotoski was working on?”

“No, there's no way for me to know that. The streams are enciphered 
at the router with his public key, and rescrambled after he's done with 
them. It's all zero-knowledge.”

“But you don't have zero knowledge, do you?”

Lawrence found himself grinning, which hurt a lot, and which caused a 
little more blood to leak out of his nose and over his lips in a hot 
trickle. “Well, signals intelligence being what it is, I was able to 
discover that it was a Securitat stream, and that it wasn't the first 
one he'd worked on, nor the first one he'd altered.”

“He altered a stream?”

Lawrence lost his smile. “I hadn't told you that part yet, had I?”

“No.” Randy leaned forward. “But you will now.”

\tb

The blue silk ribbons slid through Lawrence's mental fingers as he sat 
in his cell, which was barely lit and tiny and padded and utterly 
devoid of furniture. High above him, a ring of glittering red LEDs cast 
no visible light. They would be infrared lights, the better for the 
hidden cameras to see him. It was dark, so he saw nothing, but for the 
infrared cameras, it might as well have been broad daylight. The 
asymmetry was one of the things he inscribed on a blue ribbon and 
floated away.

The cell wasn't perfectly soundproof. There was a gaseous hiss that 
reverberated through it every forty six to fifty three breaths, which 
he assumed was the regular opening and shutting of the heavy door that 
led to the cell-block deep within the Securitat building. That would be 
a patrol, or a regular report, or someone with a weak bladder.

There was a softer, regular grinding that he felt more than heard -- a 
subway train, running very regular. That was the New York rumble, and 
it felt a little like his pan's reassuring purring.

There was his breathing, deep and oceanic, and there was the sound in 
his mind's ear, the sound of the streamers hissing away into the ether.

He'd gone out in the world and now he'd gone back into a cell. He 
supposed that it was meant to sweat him, to make him mad, to make him 
make mistakes. But he had been trained by sixteen years in the Order 
and this was not sweating him at all.

“Come along then.” The door opened with a cotton-soft sound from 
its balanced hinges, letting light into the room and giving him the 
squints.

“I wondered about your friends,” Lawrence said. “All those people 
at the restaurant.”

“Oh,” Randy said. He was a black silhouette in the doorway. 
“Well, you know. Honor among thieves. Rank hath its privileges.”

“They were caught,” he said.

“Everyone gets caught,” Randy said.

“I suppose it's easy when everybody is guilty.” He thought of Posy. 
“You just pick a skillset, find someone with those skills, and then 
figure out what that person is guilty of. Recruiting made simple.”

“Not so simple as all that,” Randy said. “You'd be amazed at the 
difficulties we face.”

“Zbigniew Krotoski was one of yours.”

Randy's silhouette -- now resolving into features, clothes (another 
sweater, this one with a high collar and squared-off shoulders) -- made 
a little movement that Lawrence knew meant yes. Randy was all tells, no 
matter how suave and collected he seemed. He must have been really up 
to something when they caught him.

“Come along,” Randy said again, and extended a hand to him. He 
allowed himself to be lifted. The scabs at his knees made crackling 
noises and there was the hot wet feeling of fresh blood on his calves.

“Do you withhold medical attention until I give you what you want? Is 
that it?”

Randy put an affectionate hand on his shoulder. “You seem to have it 
all figured out, don't you?”

“Not all of it. I don't know why you haven't told me what it is you 
want yet. That would have been simpler, I think.”

“I guess you could say that we're just looking for the right way to 
ask you.”

“The way to ask me a question that I can't say no to. Was it the 
sister? Is that what you had on him?”

“He was useful because he was so eager to prove that he was smarter 
than everyone else.”

“You needed him to edit your own data-streams?”

Randy just looked at him calmly. Why would the Securitat need to change 
its own streams? Why couldn't they just arrest whomever they wanted on 
whatever pretext they wanted? Who'd be immune to --

Then he realized who'd be immune to the Securitat: the Securitat would 
be.

“You used him to nail other Securitat officers?”

Randy's blank look didn't change.

Lawrence realized that he would never leave this building. Even if his 
body left, now he would be tied to it forever. He breathed. He tried 
for that oceanic quality of breath, the susurration of the blue silk 
ribbons inscribed with his worries. It wouldn't come.

“Come along now,” Randy said, and pulled him down the corridor to 
the main door. It hissed as it opened and behind it was an old 
Securitat man, legs crossed painfully. Weak bladder, Lawrence knew.

\tb

“Here's the thing,” Randy said. “The system isn't going to go 
away, no matter what we do. The Securitat's here forever. We've treated 
everyone like a criminal for too long now -- everyone's really a 
criminal now. If we dismantled tomorrow, there'd be chaos, bombings, 
murder sprees. We're not going anywhere.”

Randy's office was comfortable. He had some beautiful vintage circus 
posters -- the bearded lady, the sword swallower, the hoochi-coochie 
girl -- framed on the wall, and a cracked leather sofa that made 
amiable exhalations of good tobacco smell mixed with years of saddle 
soap when he settled into it. Randy reached onto a tall mahogany 
bookcase and handed him down a first-aid kit. There was a bottle of 
alcohol in it and a lot of gauze pads. Gingerly, Lawrence began to 
clean out the wounds on his legs and hands, then started in on his 
face. The blood ran down and dripped onto the slate tiled floor, almost 
invisible. Randy handed him a waste-paper bin and it slowly filled with 
the bloody gauze.

“Looks painful,” Randy said.

“Just skinned. I have a vicious headache, though.”

“That's the taser hangover. It goes away. There's some codeine 
tablets in the pill-case. Take it easy on them, they'll put you to 
sleep.”

While Lawrence taped large pieces of gauze over the cleaned-out 
corrugations in his skin, Randy tapped idly at a screen on his desk. It 
felt almost as though he'd dropped in on someone's hot-desk back at the 
Order. Lawrence felt a sharp knife of homesickness and wondered if 
Gerta was OK.

“Do you really have a sister?”

“I do. In Oregon, in the Order.”

“Does she work for you?”

Randy snorted. “Of course not. I wouldn't do that to her. But the 
people who run me, they know that they can get to me through her. So in 
a sense, we both work for them.”

“And I work for you?”

“That's the general idea. Zbigkrot spooked when you got onto him, so 
he's long gone.”

“Long gone as in --”

“This is one of those things where we don't say. Maybe he disappeared 
and got away clean, took his sister with him. Maybe he disappeared into 
our...operations. Not knowing is the kind of thing that keeps our other 
workers on their game.”

“And I'm one of your workers.”

“Like I said, the system isn't going anywhere. You met the gang 
tonight. We've all been caught at one time or another. Our little cozy 
club manages to make the best of things. You saw us -- it's not a bad 
life at all. And we think that all things considered, we make the world 
a better place. Someone would be doing our job, might as well be us. At 
least we manage to weed out the real retarded sadists.” He sipped a 
little coffee from a thermos cup on his desk. “That's where Zbigkrot 
came in.”

“He helped you with `retarded sadists'?”

“For the most part. Power corrupts, of course, but it attracts the 
corrupt, too. There's a certain kind of person who grows up wanting to 
be a Securitat officer.”

“And me?”

“You?”

“I would do this too?”

“You catch on fast.”

\tb

The outside wall of Campus was imposing. Tall, sheathed in seamless 
metal painted uniform grey. Nothing grew for several yards around it, 
as though the world was shrinking back from it.

\emph{How did Zbigkrot get off campus?}

That's a question that should have occurred to him when he left the 
campus. He was embarrassed that it took him this long to come up with 
it. But it was a damned good question. Trying to force the gate -- what 
was it the old Brother on the gate had said? Pressurized, blowouts, the 
walls rigged to come down in an instant.

If zbigkrot had left, he'd walked out, the normal way, while someone at 
the gate watched him go. And he'd left no record of it. Someone, 
working on Campus, had altered the stream of data fountaining off the 
front gate to remove the record of it. There was more than one forger 
there -- it hadn't just been zbigkrot working for the Securitat.

He'd \emph{belonged} in the Order. He'd learned how to know himself, 
how to see himself with the scalding, objective logic that he'd 
normally reserved for everyone else. The Anomaly had seemed like such a 
bit of fun, like he was leveling up to the next stage of his progress.

He called Gerta. They'd given him a new pan, one that had a shunt that 
delivered a copy of all his data to the Securitat. Since he'd first 
booted it, it had felt strange and invasive, every buzz and warning 
coming with the haunted feeling, the \emph{watched} feeling.

“You, huh?”

“It's very good to hear your voice,” he said. He meant it. He 
wondered if she knew about the Securitat's campus snitches. He wondered 
if she was one. But it was good to hear her voice. His pan let him know 
that whatever he was doing was making him feel great. He didn't need 
his pan to tell him that, though.

“I worried when you didn't check in for a couple days.”

“Well, about that.”

“Yes?”

If he told her, she'd be in it too -- if she wasn't already. If he told 
her, they'd figure out what they could get on her. He should just tell 
her nothing. Just go on inside and twist the occasional data-stream. He 
could be better at it than zbigkrot. No one would ever make an Anomaly 
out of him. Besides, so what if they did? It would be a few hours, 
days, months or years more that he could live on Campus.

And if it wasn't him, it would be someone else.

It would be someone else.

“I just wanted to say good bye, and thanks. I suspect I'm not going 
to see you again.”

Off in the distance now, the sound of the Securitat van's happy little 
song. His pan let him know that he was breathing quickly and shallowly 
and he slowed his breathing down until it let up on him.

“Lawrence?”

He hung up. The Securitat van was visible now, streaking toward the 
Campus wall.

He closed his eyes and watched the blue satin ribbons tumble, like 
silky water licking over a waterfall. He could get to the place that 
took him to anywhere. That was all that mattered.

\section{Afterword:}

I wrote this story for the launch of tor.com in 2008, at the behest of 
Patrick Nielsen Hayden, my friend and longstanding editor (Patrick also 
initially published the story “Power Punctuation!” which appears 
later in this volume; and, of course, he bought my first novel and my 
novels thereafter). Like “Scroogled” (also in this volume), this 
story considers the problem with losing sight of the ethical dimensions 
of hard and satisfying technical challenges, like data-mining.

I got the inspiration for this story while driving from Martha's 
Vineyard to New York with Patrick and his wife Teresa (Teresa 
copy-edited my next novel, the young adult book “For the Win”). We 
were talking about people we knew from science fiction fandom who had 
started out bright and promising but who had met their match in the 
real world's difficulties and sunk into a ferocious curmudgeonliness 
that would be comical if it wasn't so tragic. I wondered aloud, 
“Where do you suppose those people would have gone in ages past?” 
and Patrick immediately answered, “To a monastery.” It was so 
obviously true and weird that I knew I had to write this story.

Today, there's a monkish order that makes its living refurbishing toner 
cartridges, just as other orders make honey or beer (mmm, Chimay!). 
It's not such a stretch to imagine a future order that provides IT 
services to totalitarian governments.

\end{document}
