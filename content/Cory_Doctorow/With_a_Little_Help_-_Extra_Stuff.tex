\hyphenation{mo-no-poly car-ne-gie pro-ject pro-gress mo-dem rou-lette
  browse-wrap Use-net mon-as-tery mo-dems}
\hyphenation{co-me-dic polt-roon stove-pipe Ma-dame scru-ta-ble star-tling}
\hyphenation{heal-thily lim-ou-sines wrest-lers tan-trum push-over un-asked
  bras-siere bro-th-er}
\hyphenation{Can-a-da Fred-rick teen-agers wrest-ler Cha-vez Tho-mas 
  a-nom-a-lies sur-veil-lance ar-mies ref-u-gee ref-u-gees bris-tling
  eve-ning man-chu-ria man-chu-ri-an mid-terms me-di-um jap-a-nese}
\hyphenation{spend-ers googl-ing tour-ist tour-ists leg-end-ary}
\hyphenation{Dan-iel Van-essa Doc-to-row Ste-phen-son}
\hyphenation{de-cade sur-veilled rout-ers Wol-fen-stein teen-ager to-night}
\hyphenation{his-to-gram an-o-nym-ize Ga-la-xy sym-pa-the-tic}
\hyphenation{ar-phid ar-phids Found-ers}
\hyphenation{stran-ger stran-gers shoul-der-blades dump-ling dump-lings}
\hyphenation{ice-pack guard-rail Sep-tem-ber boot-able e-co-nom-ist}
\hyphenation{grown-ups roos-ter shoe-laces li-quid-i-ty}
\hyphenation{side-arm}
\hyphenation{wo-man wo-men tan-trum tan-trums Le-nin-grad zom-bie bunk-house}
\hyphenation{up-tick bio-mass}
\hyphenation{of-fi-cial of-fi-cial-ly gov-ern-ment}
\hyphenation{heal-thy Or-ville spark-ling}
\hyphenation{ves-ti-bule Law-rence au-to-no-mous}
\hyphenation{sau-sage door-step staf-fer}
\hyphenation{tree-trunk}
\hyphenation{to-ron-to}
\hyphenation{qua-dril-lion-aire qua-dril-lion-aires}
\hyphenation{sports-jack-et sports-jack-ets}
\hyphenation{work-space skunk-works}
\hyphenation{kings-ton}


\begin{document}
%\setlength{\emergencystretch}{1ex}
\raggedbottom

\title{With a Little Help}
\subtitle{Extra Stuff}
\author{Cory Doctorow}
\date{\bigskip\normalsize Copyright CorDoc-Co, Ltd (UK), 2010.}
\publishers{
\begin{flushleft}
\normalsize
%Copyright CorDoc-Co, Ltd (UK), 2010.; \\  
License: Some Rights Reserved
under a Creative Commons Attribution-NonCommercial-ShareAlike 3.0
license:\\
\texttt{http://creativecommons.org/licenses/by-nc-sa/}
\end{flushleft}
}
\maketitle

\vfill

\begin{flushleft}
This volume contains parts of Cory Doctorow’s short story collection
“With a Little Help”, that haven't been included into
\href{http://ccbib.org}{Creative Commons Bibliothek}
as separate stories.

The original file is available at:
\texttt{http://craphound.com/walh/}
\end{flushleft}

%Copyright CorDoc-Co, Ltd (UK), 2010.; \\  License: Some Rights Reserved 
%under a Creative Commons Attribution-NonCommercial-ShareAlike 3.0 
%license: 
%‹\begin{scriptsize}\url{http://creativecommons.org/licenses/by-nc-sa/}
%\end{scriptsize}›

\tableofcontents

\section{Dedication:}

For my friends, past, present and future. No man is an island.

\section{Publication history}
\begin{flushleft}
\setlength{\parskip}{0.5\baselineskip}
“Introduction,” written by Jonathan Coulton for this volume

“The Things that Make Me Weak and Strange Get Engineered Away” 
originally published on Tor.com, August 2008

“The Right Book” originally published in \emph{The Bookseller}, 
June 2008

“Other People's Money” originally published in \emph{Forbes}, 
November 2007

“Scroogled” originally published in \emph{Radar}, September 2007

“Human Readable” originally published in \emph{Future Washington}, 
Ernest Lilley, editor (WSFA Press, 2005)

“Liberation Spectrum” originally published on Salon.com, 2003

“Power Punctuation!” originally published in \emph{Starlight 3}, 
Patrick Nielsen Hayden, editor (Tor, 2001)

“Visit the Sins,” originally published in \emph{Asimov's Science 
Fiction Magazine}, June 1999

“Constitutional Crisis,” produced for the Future of the Book 
Project, 2009

“Pester Power,” originally published in \emph{Communications of the 
Association for Computing Machinery}, December 2008

“Chicken Little,” written for \emph{Gateways,} forthcoming from Tor 
Books, 2010

“Epoch,” commissioned by Mark Shuttleworth for this volume

“I'm Only In It For the Money,” written by Russell Galen for this 
volume
\end{flushleft}

\section{A note about typos and other errors}
\begin{flushleft}
Every book has typos. \emph{Every} book. But this book is different. 
This book isn't perfect, but it \emph{fails well.}

If you spot a typo in this book, send it to
\texttt{walh\_typos@craphound.com} 
(that's me) and I'll correct it in the electronic editions and in the 
next copy of the print-on-demand book that's printed -- 
nigh-instantaneously.

What's more, as a thank-you, I'll include your name as a footnote on 
the page you fixed for me, and at the bottom of the ebooks.
\end{flushleft}

\section{Introduction by Jonathan Coulton}

It turns out the future doesn't really care about space travel. It used 
to, or at least when I was growing up all the science fiction I read 
promised that space travel would someday be commonplace. That was what 
made it the future: we would all be so bored with flying to other 
planets that we wouldn't even really talk about it anymore, it would 
just become a dull backdrop to our daily lives. There would be aliens, 
obviously. Probably there would be some sort of intergalactic governing 
body, maybe a war involving a trade federation, some asteroid mines. At 
the very least, a mission to Mars. But it doesn't seem to be shaping up 
that way.

There's always something that science fiction gets charmingly wrong 
about the future. The problem is, every now and then there's an 
unanticipated seismic shift in the world, something that changes 
everything and creates a corner we can't see around. The most recent of 
these was the potent combination of digital information and global 
connectivity that transformed the end of the 20th century. I like to 
call it “The Internet,” and mark my words, it's going to be very 
big. The struggling record industry, the death of the newspaper, the 
rise of LOLCats - these are just warning shots. Everything is going to 
get swallowed up eventually, and it's all going to get loud and messy 
and complicated. Forget space travel, this is the future we need to 
imagine now, and quickly, before it overtakes us.

Luckily, we have Cory Doctorow; he thinks about the Internet, a lot. 
And so his stories are especially compelling because they are so 
relevant to our immediate future. “Pester Power” tells us how the 
Internet has been watching us and learning, and how will finally, one 
day, wake up. “Scroogled” warns us of what might happen if Google 
someday decides that yes, actually, they would like to be evil after 
all. For a future-lover like me it's easy to get caught up in rosy 
visions of a world where we're all connected, and everything is free, 
and our in-brain iPods have every Beatles album with all the correct 
metadata. Cory's fiction reminds us that we have quite a few thorny 
issues to sort out before we get there, not least of which is the 
question of how people like Cory are going to make a living when books 
and publishing companies disappear. But of course he's thinking about 
that too.

With a Little Help is an experiment of sorts, an attempt to re-imagine 
what it means to publish, market and sell a book. It will be 
self-published, and like all of Cory's books it will be released under 
a Creative Commons license that allows for non-commercial sharing and 
remixing. There will be a number of price-points, ranging from free 
ebook and audiobook downloads, to print-on-demand paperbacks, to 
hardcover special editions with all sorts of extra goodies. The highest 
price-point comes with an opportunity to commission a brand new story 
based on a mutually agreeable premise (hence, “Epoch”). Throughout 
the process Cory will hold weekly public production meetings on Twitter 
in an effort to share information about the success or failure of these 
strategies. The plan combines a lot of different new ideas - audience 
participation, free culture, long tail economics - and it will test a 
few hypotheses about what it might mean to be an author in the future. 
It's a shotgun approach to innovation; as the old business models 
become quaint antiques from a not-so-distant past, sometimes the best 
way forward is simply to try a bunch of stuff and see what works.

At least, that was my theory when I finally decided to become a 
full-time musician. I had spent years avoiding a career in the music 
business because it seemed impossible. How do people discover you if 
you're not famous? And how do you get famous if nobody ever discovers 
you? Then I heard about Creative Commons, a brilliant licensing hack 
that sits on top of the complicated and antiquated copyright system. It 
allows creators to specify ahead of time what sorts of uses they'd like 
to allow for the things they create. For me and for Cory this means 
allowing people to share our work freely, and to re-use it to create 
new things. The first time the concept was explained to me I felt as 
though someone had set my brain on fire - it was the most exciting idea 
I had ever heard.

In my head, songs became little autonomous vehicles that I could 
release into the wild, letting them bounce around and find their way to 
the people who would enjoy them. It was a way to let this new 
“Internet” thing do all the heavy lifting, an organic and efficient 
method of targeting an audience of fans who did not yet know they were 
fans. On top of that, it was a perfect expression of what I had always 
felt about art, this idea that everything ever created owes its 
existence to something that came before. To be sure, there is a 
boundary between inspiration and theft, but it's a thick and mushy one. 
When we create, we borrow, we build, we steal. Declaring my intentions 
to allow this sort of thing, indeed to encourage it, made perfect 
sense. I didn't have it all figured out, but I started licensing my 
music with Creative Commons that very day. It became the first piece of 
the puzzle, and it remains an essential component of the mysterious 
machinery that now allows me to make my living as a musician. It was 
just one of those ideas that resonated, the buzzing end of a long wire 
stretching off into the distance, perhaps even around a corner or two.

Speaking of which, it's not unreasonable to ask: as a science fiction 
author, what is it that Cory is getting wrong about the future? What is 
the corner that he can't see around? Certainly there's something big 
coming, and we'll know it once we've gotten past it. But until then, 
we've got our own rather sharp corner to turn, and we're just now 
getting a glimpse of some of the possible futures that might be in 
store for us. Here in the real world, where constant change seems to be 
the new status quo, he's hedging against what we don't know, not just 
thinking about the future, but trying to take us there.

\section{I'm Only In It For the Money, by Russell Galen}

I'm Cory Doctorow's literary agent. I advise him on various business 
and editorial aspects of his writing career and I negotiate licenses to 
commercial organizations like book publishers and movie studios. When I 
sell a piece of writing to such an organization, I receive 15\% of what 
the author makes, and that's how I make a living.

Like any literary agent, I need to keep my clients happy and loyal, 
which means selling whatever they need me to sell. A small press deal 
could earn me as little as \$150 on a \$1,000 advance, but the same 
client might have a national bestseller that earns me enough to put a 
kid through private college. I handle all these deals, large and small, 
because it's what you do when taking care of a writer's entire body of 
work. But there's always \emph{something}.

In the case of this volume, we're not licensing any rights to any 
organization, and so there are no fees or royalties due the author and 
therefore nothing for me to take 15\% of. I'm getting bupkes.

And yet I've happily been one of Cory's “friends” putting the 
project together: giving him some advice and ideas on various aspects 
of the project; composing a little contract regarding one aspect of his 
relationship to lulu.com; and donating some of my office staff time to 
maintaining certain records relating to the venture.

Why would I do such a thing? It's not \emph{pro bono}. I'm only in it 
for the money.

Publishing economics are ridiculous. Suppose we'd sold this volume to a 
conventional publisher, and let's call it a \$15.95 trade paperback. 
The author's share would be 7.5\% of the cover price, or \$1.17. We're 
leaving 92.5\% -- \$14.78 -- to bookstores and publishers. There's no 
room to negotiate a better share for the author because most of that 
\$14.78 goes to people who won't reduce their shares: bookstores, paper 
suppliers, printers, warehouses, shippers, the owner of the publisher's 
office building, taxes, and so on.

It's a business model that enables only a tiny fraction of authors to 
make even a modest living, while employees of publishing houses enjoy 
solid middle class lives with salaries, health care, pensions, and 
expense accounts. Most authors need another job or a supportive spouse 
to be able to create books. Or, if they can support themselves from 
writing, their income is frighteningly erratic.

Bad as that model is, we'll soon look back at it as a Golden Age, 
because the future will be worse. I'm old enough to remember when the 
choice was between NBC, CBS, ABC, whatever was at the local movie 
theater, or a book. Those days are long gone, of course, and we ain't 
seen nothin' yet in terms of the share of our minds devoted to new 
media. Book sales have been trending down for a generation, and haven't 
bottomed.

As a result, authors' incomes are way down. Those of us who work on the 
business side of writing have always tried to think of new ways for 
writers to make money, but now that quest has taken on a new 
desperation.

New technologies make it theoretically possible to cut out some of the 
middlemen, enabling writers to derive more income from each copy sold, 
and reach more readers. We need to find out if this can work.

I've built a career within the conventional publishing industry. Would 
I care if that industry were destroyed by a new model? Editors out of 
work, bookstores taken over by rats and squatters? Not really. As long 
as there's a way for writers to get their work to readers and be paid a 
fair amount for what they provide, I'm content. All of us in the 
publishing industry are a mere support structure: we don't have 
intrinsic value. We're there to help writers find readers. Anyone who 
finds that he's no longer doing that, and has in fact become an 
obstacle, needs to get a new job.

It's not just that there might be a way for writers to earn more from 
each copy of a book that they sell. The current model ensures that only 
certain kinds of books can be published commercially in the first place.

For instance, like most people I have my own little area of passionate 
interest. In my case it happens to be nature, wildlife, and the 
environment. I like to experience wild nature and I like to read about 
it. Sometimes I can help an author create a commercial book about that 
subject, but all too often the material is just too narrow and 
specialized. Commercial publishers can't touch such books, and so they 
either don't get published or are published by university presses, 
which often means very expensive books that are under-marketed and 
never find their audience.

When I daydream about a new publishing model, this is what I secretly 
dream of. Not my million dollar clients becoming five million dollar 
clients: that would be nice but I don't daydream about it. What I dream 
of is the book about slime-mold that can't be created because it would 
only sell 2,500 copies and earn the author less than \$5,000. I dream 
that using new technologies and distribution methods, we could find a 
way to sell 10,000 copies, earning the author \$20,000, making the book 
doable for him.

You don't have to share my interest in slime-mold -- which is actually 
fascinating, but don't get me started on that -- to see that there are 
millions of worthy books that will never find a mass audience and never 
interest the stressed-out commercial publishers of today.

That's what we're trying to do. Experiment with a new publishing model 
that might show a way that we could reach more people, and keep more of 
the selling price for the author, than the old way.

Where do I fit in? I can imagine a world where there's no bookstore and 
no publisher, but I can't imagine a world in which authors -- or any 
other kind of freelance artist -- operate alone, without a business 
partner, advisor, editorial consultant, and manager. I know Cory needs 
me. I just can't figure out how he's going to pay me. I'm in this 
project to explore that question.

For Cory to pay me, he first must make money for himself. Literary 
agents can and should only be paid on a commission basis: that is, as a 
percentage of what the author earns. I would not accept an hourly fee 
or a flat sum. A lawyer can be nicely paid for setting up a nonprofit 
or for otherwise working on a venture that doesn't make money. But not 
a literary agent. For me to prosper, my clients must prosper.

That means that if I'm to benefit from this mode of publishing, it has 
to be profitable, so that I can receive some share of the profits. 
There's no media conglomerate forking over fees; profits are the only 
money that might be available to be divided up. I don't know if “With 
A Little Help” will be profitable, but it has the potential to earn a 
profit. If it doesn't, it won't be because the model is wrong, but 
because we did it wrong, or because the time isn't right for it. We'll 
learn; change the formula; wait for the culture or the technology to 
evolve; and then come back and try it again.

By volunteering some of my time and talent to the project, I'm in a 
position to educate myself about what does and doesn't work in this 
crucial area: generating a profit in this new publishing model. If you 
want to say, “Typical agent, only cares about money, has no soul and 
no taste for art,” go right ahead. Lack of money is art's 
abortionist, killing off too many God-possibled projects that are 
conceived but never born. I'm here to help my clients make money. They, 
and the slime-molds, need me.

I hope to take what I learn and do a better job next time, advising 
Cory (or other clients one day) on how to make money from this mode of 
publishing. I'm trying to be as useful to him as I can be, but I'm 
really preparing for a time when this might be my primary source of 
commissions. I feel it coming.

In this new model, the term “literary agent” might fall out of use, 
since the word “agent” means “the person who gets you the 
money.” Already some literary agents like to use the word 
“manager” to provide a broader sense of our role as literary 
counselors and not mere deal-brokers.

But I like the word “agent” because it is my job to find you the 
money and that shouldn't change. Managing is simply one of the skills 
an agent uses to get that money in the first place, like a used-car 
salesman who details the car and changes the oil before showing it to 
customers. What will change is where the money comes from and what kind 
of management skills I'll need to help get it. It could be something 
like being a partner in a small business -- with each book its own 
individual small business -- and receiving a share of profits, if any. 
That is, of course, already what I do, helping to run the day-to-day 
business of our agency, and so it won't be that radical a change.

It might not be as much fun as calling up some international media 
conglomerate, wresting a few million dollars out of its hide, and 
keeping 15\% for myself. I'll miss that. In exchange, I won't have to 
listen to them, either, and I won't have to devote my life to selling 
them only what they want to buy, which is not necessarily what I want 
to represent, or what people want to read.

\section{Acknowledgments}

I would like to gratefully acknowledge all the editors who commissioned 
or bought the stories in this volume. Without you, this book wouldn't 
exist.

Similarly, I would like to thank all the writers who contributed their 
paper ephemera for the premium hardcover, as well as the readers who 
read these stories aloud for the audiobook.

Thanks also to the production staff: John D Berry, John Taylor 
Williams, Roz Doctorow and Pablo Defendini; and the cover artists: Rick 
Leider, Rudy Rucker, Pablo Defendini, Frank Wu and Randall Munroe.

Thanks to my agent, Russell Galen, to \emph{Publishers Weekly} and to 
Jonathan Coulton.

Thanks, finally, to my wife and family, who make it all worth doing.

\section{About the author}

Cory Doctorow (doctorow@craphound.com) is a writer, activist, 
journalist and blogger. He is co-owner/editor of Boing Boing 
(boingboing.net), a popular blog. He is also the author of several 
novels for adults (\emph{Makers} from Tor Books in the USA and 
HarperCollins in the UK and Commonwealth is the most recent) and two 
novels for young adults (the \emph{New York Times} bestseller 
\emph{Little Brother}, 2008 and \emph{For the Win}, 2010, also from Tor 
and HarperCollins), as well as three short story collections (including 
this one) and a book of essays (\emph{Content}, Tachyon, 2008).

His writing has won the Locus Award, the White Pine Award, the Sunburst 
Award, the Golden Duck Award, the Prometheus Award, and the John W 
Campbell Award for Best New Writer, as well as the \emph{other} John W 
Campbell Award for best book. He has also been nominated for several 
Hugo and Nebula Awards, including Best Novel.

He writes for several magazines, newspapers and websites, including 
\emph{Make}, \emph{The Guardian} (London), and \emph{Locus Magazine}.

Born in Toronto, Canada, he now lives in London, England, with his wife 
and daughter. He is the former European Director of the Electronic 
Frontier Foundation and he co-founded the UK Open Rights Group. He is a 
Forbes “25 Top Web Influencers” and a World Economic Forum “Young 
Global Leader” and served as the Fulbright Chair in Public Diplomacy 
at the University of Southern California. He is also a Visiting Senior 
Lecturer at Open University (UK) and Scholar in Virtual Residence at 
the University of Waterloo Independent Studies Program (Canada).

His personal blog is at craphound.com.

The entire text of this book can be downloaded from\\
\texttt{http://craphound.com/walh}

\section{Typo corrections}

Thanks to my readers for catching the following typos (page numbers 
refer to print edition):

\begin{flushleft}
\setlength{\parskip}{.5\baselineskip}
OldMiser: “Peoples'”, \emph{Title page}, P6, Dec 11, 2010.

OldMiser: “Peoples'”, \emph{Contents}, P9, Dec 11, 2010.

Eunah Choi: missing sentence, \emph{Introduction}, P11, Jan 1, 2011

Ralph Broom: “overlayed”, \emph{The Things That Make Me Weak and 
Strange Get Engineered Away}, P17, Dec 7, 2010.

Ralph Broom: “heirarchy”, \emph{The Things That Make Me Weak and 
Strange Get Engineered Away}, P18, Dec 7, 2010.

Alex Ghitza: “other peoples'”, \emph{The Things That Make Me Weak 
and Strange Get Engineered Away}, P20, Dec 8, 2010.

Eric Halin: missing quote, \emph{The Things That Make Me Weak and 
Strange Get Engineered Away}, P27, Dec 31, 2010.

Andrew Crocker: “understandingthe”, \emph{The Things That Make Me 
Weak and Strange Get Engineered Away}, P31, Nov 1, 2010.

Set Humme: “my cue to go”, \emph{The Things That Make Me Weak and 
Strange Get Engineered Away}, P31, Dec 9, 2010.

Roger van der Horst: “order”, \emph{The Things That Make Me Weak 
and Strange Get Engineered Away}, P38, Jan 5, 2011.

Ralph Broom: “Analysitcs”, \emph{The Things That Make Me Weak and 
Strange Get Engineered Away}, P38, Dec 7, 2010.

Roger van der Horst: “datastreams”, \emph{The Things That Make Me 
Weak and Strange Get Engineered Away}, P38, Jan 5, 2011.

Joe Burch: “playing on”, \emph{The Things That Make Me Weak and 
Strange Get Engineered Away}, P41, Jan 5, 2011.

Bertrand Lorentz: “backto”, \emph{The Things That Make Me Weak and 
Strange Get Engineered Away}, P42, Dec 7, 2010.

Roger van der Horst: missing comma, \emph{The Things That Make Me Weak 
and Strange Get Engineered Away}, P48, Jan 5, 2011.

Tycho Clendenny: “hadn't told the man his name”, \emph{The Things 
That Make Me Weak and Strange Get Engineered Away}, P51, Jan 12, 2011.

Roger van der Horst: “data streams”, \emph{The Things That Make Me 
Weak and Strange Get Engineered Away}, P56, Jan 5, 2011.

Ryan Junk: “sauve”, \emph{The Things That Make Me Weak and Strange 
Get Engineered Away}, P58, Dec 7, 2010.

Ryan Bernacki: “a lot gauze pads”, \emph{The Things That Make Me 
Weak and Strange Get Engineered Away}, P60, Dec 7, 2010.

Alex Ghitza: “what whatever”, \emph{The Things That Make Me Weak 
and Strange Get Engineered Away}, P62, Dec 8, 2010.

Alex Ghitza: “If Zbigkrot”, \emph{The Things That Make Me Weak and 
Strange Get Engineered Away}, P62, Dec 8, 2010.

Roger van der Horst: “Gerta”, \emph{The Things That Make Me Weak 
and Strange Get Engineered Away}, P62, Jan 5, 2011.

Paul Renault: “hand made”, \emph{The Right Book}, P65, Dec 13, 2010.

Anthony Sheetz: “said. getting”, \emph{The Right Book}, P71, Dec 8, 
2010.

OldMiser: “Peoples'”, \emph{Other People's Money}, P73, Dec 11, 
2010.

Katelyn Eads: Extra quote, \emph{Other Peoples' Money}, P75, Dec 8, 
2010.

Haakon Nilsen: “momento'”, \emph{Other People's Money}, P76, Dec 
13, 2010.

OldMiser: “Peoples'”, \emph{Other People's Money}, P79, Dec 11, 
2010.

Katelyn Eads: Missing period in opening quote, \emph{Scroogled}, P81, 
Dec 8, 2010.

Alex Ghitza: “starting googling”, \emph{Scroogled}, P81, Dec 8, 
2010.

Alex Ghitza: “Al Quaeda”, \emph{Scroogled}, P84, Dec 8, 2010.

Carl Rigney: “Greg has just”, \emph{Scroogled}, P84, Dec 20, 2010.

Chris Kovacs: em-dash, \emph{Scroogled}, P88, Dec 27, 2010.

Chris Kovacs: missing quote, \emph{Scroogled}, P89, Dec 27, 2010.

Steve Clark: “unplumable”, \emph{Scroogled}, P91, Dec 13, 2010.

Alex Ghitza: “hissed a laughed”, \emph{Scroogled}, P95, Dec 8, 2010.

Alex Ghitza: “that'd he'd”, \emph{Human Readable}, P103, Dec 8, 
2010.

Chris Kovacs: misformatted title, \emph{Human Readable}, P103, Dec 27, 
2010.

Ralph Broom: “yarlmulke”, \emph{Human Readable}, P113, Dec 11, 2010.

Ralph Broom: “paen”, \emph{Human Readable}, P120, Dec 11, 2010.

Steve Clark: “Nickle”, \emph{Human Readable}, P120, Dec 14, 2010.

Alex Ghitza: “found its'”, \emph{Human Readable}, P125, Dec 8, 2010.

Katelyn Eads: “twice times a day”, \emph{Human Readable}, P108, Dec 
8, 2010.

Carl Rigney: “supercalafragilistic'”, \emph{Human Readable}, P113, 
Dec 20, 2010.

Johannes Payr: “sheltters”, \emph{Human Readable}, P113, Jan 14, 
2011.

Mike McConnell: “a solutions”, \emph{Human Readable}, P123, Dec 8, 
2010.

Mike McConnell: “and rainer and” \emph{Human Readable}, P135, Dec 
8, 2010

Carl Rigney: “media-training'”, \emph{Human Readable}, P145, Dec 
20, 2010.

Alex Ghitza: “berzerk”, \emph{Human Readable}, P148, Dec 8, 2010.

Alex Ghitza: “uncorruptable”, \emph{Human Readable}, P151, Dec 8, 
2010.

Paul Renault: “Akwesahsne”, \emph{Liberation Spectrum}, P157, Dec 
13, 2010.

Alex Ghitza: “astrong”, \emph{Liberation Spectrum}, P157, Dec 8, 
2010.

Alex Ghitza: “probabalistically”, \emph{Liberation Spectrum}, P159, 
Dec 8, 2010.

Alex Ghitza: “syndacalist”, \emph{Liberation Spectrum}, P161, Dec 
8, 2010.

Alex Ghitza: “awe-stuck”, \emph{Liberation Spectrum}, P161, Dec 8, 
2010.

Alex Ghitza: “thana”, \emph{Liberation Spectrum}, P166, Dec 8, 2010.

Alex Ghitza: “do that have”, \emph{Liberation Spectrum}, P168, Dec 
8, 2010.

Lise Andreasen: “got a good job”, \emph{Liberation Spectrum}, P168, 
Jan 7, 2011.

Alex Ghitza: “ina”, \emph{Liberation Spectrum}, P168, Dec 8, 2010.

Alex Ghitza: “ahand”, \emph{Liberation Spectrum}, P169, Dec 8, 2010.

Alex Ghitza: “altimiter”, \emph{Liberation Spectrum}, P176, Dec 8, 
2010.

Paul Renault: “not ever”, \emph{Liberation Spectrum}, P182, Dec 13, 
2010.

Phil Wellings: badly encoded punctuation, \emph{Power Punctuation!}, 
P189, Dec 12, 2010.

Lise Andreasen: to be move, \emph{Power Punctuation!}, P189, Jan 7, 
2011.

Phil Wellings: badly encoded punctuation, \emph{Power Punctuation!}, 
P206, Dec 12, 2010.

OldMiser: “non compis”, \emph{Visit the Sins}, P218, Dec 11, 2010.

Jacob Lewis: “over educated”, \emph{Visit the Sins}, P220, Jan 17, 
2011.

Yamandu Ploskonka: “principle ogre”, \emph{Visit the Sins}, P226, 
Jan 5, 2011.

Cindie Flannigan: stray asterisk, \emph{Visit the Sins}, P226, Jan 5, 
2011.

Mike Harris: “here down”, \emph{Visit the Sins}, P226, Jan 21, 2011.

Mike McConnell: “Hebredies”, \emph{Constitutional Crisis}, P231, 
Dec 10, 2010.

Mike McConnell: “levelingjob”, \emph{Constitutional Crisis}, P236, 
Dec 10, 2010.

Chris Pepper: “to waged”, \emph{Constitutional Crisis}, P238, Dec 
17, 2010.

Lise Andreasen: “Guild-enemy”, \emph{Constitutional Crisis}, P239, 
Jan 7, 2011.

Lise Andreasen: “clause 2”, \emph{Constitutional Crisis}, P239, Jan 
7, 2011.

Lise Andreasen: “Guild-enemy”, \emph{Constitutional Crisis}, P240, 
Jan 7, 2011.

OldMiser: “non compis”, \emph{Pester Power}, P242, Dec 11, 2010.

Alex Ghitza: “botnet `s”, \emph{Pester Power}, P243, Dec 9, 2010.

Alex Ghitza: “econopocalytpic”, \emph{Chicken Little}, P256, Dec 9, 
2010.

Rodney Hoffman: “Hamilton Beech”, \emph{Chicken Little}, P254, Dec 
7, 2010.

Rodney Hoffman: “made pay every bill”, \emph{Chicken Little}, P255, 
Dec 7, 2010.

Rodney Hoffman: “Gretelprize”, \emph{Chicken Little}, P255, Dec 7, 
2010.

Blake Girardot: “herin”, \emph{Chicken Little}, P257, Dec 8, 2010.

Lise Andreasen: “was that you could”, \emph{Chicken Little}, P257, 
Jan 7, 2011.

Peter Hollo: extra space, \emph{Chicken Little}, P258, Jan 12, 2011.

Blake Girardot: “he's say”, \emph{Chicken Little}, P272, Dec 8, 
2010.

Chris Pepper: “smiled stopped”, \emph{Chicken Little}, P273, Dec 
17, 2010.

David Picón Álvarez: “caprihina'”, \emph{Chicken Little}, P274, 
Dec 28, 2010.

Tom Johnson: “buen”, \emph{Chicken Little}, P280, Jan 4, 2011.

Lise Andreasen: “Ria is that”, \emph{Chicken Little}, P282, Jan 7, 
2011.

Alex Ghitza: “cromatograph”, \emph{Chicken Little}, P287, Dec 9, 
2010.

Alex Ghitza: “he when he sat”, \emph{Chicken Little}, P288, Dec 9, 
2010.

Alex Ghitza: “peoples' gardens”, \emph{Chicken Little}, P289, Dec 
9, 2010.

Alex Ghitza: “metalling”, \emph{Chicken Little}, P295, Dec 9, 2010.

Chris Pepper: “part Buhle's”, \emph{Chicken Little}, P297, Dec 17, 
2010.

Blake Girardot: “bullshit Statistically”, \emph{Chicken Little}, 
P297, Dec 7, 2010.

Alex Ghitza: “Even she you”, \emph{Chicken Little}, P302, Dec 9, 
2010.

Matthew Reames: “September 30”, \emph{Chicken Little}, P302, Jan 
17, 2011.

OldMiser: “Peoples'”, \emph{Epoch}, P318, Dec 12, 2010.

Steve Clark: “principle”, \emph{Epoch}, P323, Dec 11, 2010.

Chris Kovacs: “he said”, \emph{Epoch}, P323, Dec 28, 2010.

Mike McConnell: “PEBCAK”, \emph{Epoch}, P331, Dec 10, 2010.

Mike McConnell: “PEBCAK”, \emph{Epoch}, P332, Dec 10, 2010.

Lise Andreasen: missing comma, \emph{Epoch}, P337, Jan 11, 2011.

Alex Ghitza: “Lobjan”, \emph{Epoch}, P345, Dec 9, 2010.

Chris Kovacs: extra parenthesis, \emph{Epoch}, P349, Dec 28, 2010.

Alex Ghitza: “uninterruptable”, \emph{Epoch}, P350, Dec 9, 2010.

Mike McConnell: “the some”, \emph{I'm Only In It For the Money}, 
P354 Dec 10, 2010.

Chris Kovacs: “Acknowledegments,” \emph{Acknowledgments}, P359, Dec 
28, 2010.
\end{flushleft}

Thanks also to the following readers for spotting bugs in the markup of 
the ebook: Jeff Cohen, OldMiser, Floyd Gecko, Steve Clark, Chris 
Pepper, @frogworth, Eunah Choi, Yama Ploskonka, Lise Andreasen, Tycho 
Clendenny

\end{document}

