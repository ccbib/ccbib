\usepackage[ngerman]{babel}
\usepackage[T1]{fontenc}
\hyphenation{wa-rum Fracht-raum}
\hyphenation{schien}
\hyphenation{Tief-ebe-ne Tief-ebe-ne gro-ßen}


%\setlength{\emergencystretch}{1ex}

\renewcommand*{\tb}{\begin{center}* \quad * \quad *\end{center}}

\newcommand\bigpar\medskip
\newcommand\gedanke\textit

\begin{document}
\raggedbottom
\begin{center}
\textbf{\huge\textsf{Die letzte Grenze}}

\bigskip

Dieter Bohn
\end{center}

\bigskip

\begin{flushleft}
Dieser Text wurde erstmals veröffentlicht in:
\begin{center}
Die Steampunk-Chroniken\\
Band I -- Æthergarn
\end{center}

\bigskip

Der ganze Band steht unter einer
\href{http://creativecommons.org/licenses/by-nc-nd/2.0/de/}{Creative-Commons-Lizenz.} \\
(CC BY-NC-ND)

\bigskip

Spenden werden auf der
\href{http://steampunk-chroniken.de/download}{Downloadseite}
des Projekts gerne entgegen genommen.

\vfill

Dieter Bohn, geboren 1963 in Trier, wohnt derzeit in Dormagen und
arbeitet im Marketing eines großen Automobilzulieferers.

Von »Raumpatrouille«, »Zack« und »Perry Rhodan« geprägt, ist er
seit fast 25 Jahren im Fandom mit Risszeichnungen, Datenblättern
und Illustrationen aktiv.

Die letzten Jahre nahm dabei das Schreiben von Stories einen immer
breiteren Raum ein. Er hat sich mit seinen Geschichten bei einer
Reihe von Schreibwettbewerben platziert, darunter 2009 beim
William-Voltz-Award.

Neben einer Storysammlung und einem Hörbuch hat er in verschiedenen
Anthologien veröffentlicht, sowie drei STELLARIS-Gaststories zum
»Perryversum« beigesteuert.

Zurzeit schreibt er an seinem ersten Roman.

\bigpar

\texttt{http://www.dieterbohn.de/}
\end{flushleft}

\section{Die letzte Grenze}

»Eine Frau? Mir kommt keine Frau an Bord!« Nicolas John McGuire III
hielt es nicht mehr in seinem Sessel. Sein Gesicht war puterrot.
»Und schon gar nicht eine Lady Thornton! Wenn sich diese
Suffragette beweisen will, soll sie Polo spielen!«

Die missbilligenden Blicke einiger anderer Mitglieder des
Hershley-Clubs in Oxford richteten sich auf die beiden Männer, die
sich bis gerade noch in angemessenem Tonfall in einer Ecke des
blauen Salons unterhalten hatten.

»Gemach, lieber Nicolas! Ihr vergesst Eure gute Erziehung!« Lord
Ralph Cumberland richtete das Mundstück seiner Pfeife anklagend auf
seinen Freund. »Lord Thornton ist ein Earl des Commonwealth, ein
Mitglied des Unterhauses … und unser wichtigster Geldgeber. Dem
schlägt man nicht so einfach einen Wunsch ab.«

»Einen Wunsch? Nur weil seine ungezogene Tochter den \emph{Wunsch}
nach etwas Aufregung verspürt, kommt mir keine Frau an Bord der
\emph{Explorer}!«

»Nun, da hinter diesem \emph{Wunsch} unser größter Mäzen steht, ist
es wohl eher ein \emph{Befehl}.« Das Mundstück der Pfeife
verschwand wieder in der gepflegten Fülle seines üppigen Barts.

»Dies ist kein Sommernachtsausflug zu den Scharrbildern von
Avebury. Wir setzen unser Leben aufs Spiel, um dorthin zu gehen, wo
noch nie ein Mensch zuvor gewesen ist! Die Chancen stehen fünfzig
zu fünfzig, dass wir nicht zurückkehren. Was sagt denn Lord
Thornton dazu? Hat er keine Angst um sein Töchterchen?« McGuire
warf entschuldigende Blicke zu den Männern im Salon. Dann richtete
der Mittdreißiger seinen Gehrock und ließ sich wieder in seinem
Sessel nieder.

»Nun, Lady Ellen ist ein wenig … geradlinig in ihrem Temperament.
Wenn sie erst einmal Fahrt aufgenommen hat, ist sie wie ein
Dampfross, das niemand so schnell stoppen kann. Außerdem munkelt
man, dass Lord Thornton ohnedies nicht viel zu sagen hat. Sie ist
die Älteste von sechs Schwestern. Nach dem frühen Tode ihrer Mutter
hat sie das Regiment im Hause Thornton übernommen. Letztes Jahr hat
sie Kimberley auf seiner Expedition durch den Schwarzen Kontinent
begleitet und davor war sie mit Lambermont in der Antarktis. Ihr
seht, mein Freund, Master Thornton ist die Angst um sein
Töchterlein gewohnt.«

»Mit einem Schlitten durch das Ewige Eis oder auf Dromedaren durch
Dschungel und Wüsten ist eines, aber hier reden wir von einer Reise
mit einer Rakete zu den Planetenräumen!«

\tb

Nicolas John McGuire III starrte wie ein verschrecktes Kaninchen
auf die feingeschnittene Hand, die sich ihm forsch
entgegenstreckte.

»Gehe ich recht in der Annahme, dass auch Sie ein großer Liebhaber
der Werke von Jules Verne sind?«, fragte der Rest, der irgendwo an
der Hand hing und der Nicolas nur als weißes Kleid ins Bewusstsein
drang. Diese Mädchen – diese Frau – legte ein reichlich
ungebührliches Verhalten an den Tag. Nicolas’ Augen wanderte den
rüschenbesetzten Arm hinauf, bis er in zwei grüne Augen blickte,
die ihn im Schein der Gaslaternen frech ansahen. Sie hatte ein
schmales Gesicht. Zu schmal für Damen ihres Standes, deren
Lebenszweck sich üblicherweise scheinbar darin erschöpfte, auf
Gesellschaften zu promenieren und sich mit erlesenen Speisen zu
mästen.

›Natürlich!‹, wollte er sagen, aber es kamen nur gegurgelte Laute
aus seinem Mund. Die Luft im Lesezimmer seines Landsitzes kam ihm
plötzlich heiß und stickig vor. Diese Frau entsprach ganz und gar
nicht den Ladies, die er im Allgemeinen mit Verachtung strafte.

»Ah, da stehen ja seine Bücher!« Sie drehte sich scheinbar entzückt
zu einer Bücherwand um und überspielte damit den Fauxpas McGuires.
»Sie haben bestimmt eine Menge Anregungen in seinen Büchern
gefunden.«

»Mylady«, fand McGuire endlich seine Stimme wieder. »Ich kann nicht
verhehlen, dass es mir widerstrebt, Sie auf diese ungewisse Reise
mitzunehmen. Aber nachdem Ihr werter Herr Vater uns keine Wahl
lässt, muss ich wohl in den sauren Apfel beißen.«

»Ich danke Ihnen für ihre Offenheit, Lord McGuire. Aber ich bin nun
einmal neugierig. Zu neugierig, um mein Leben damit zu vergeuden,
an der Seite eines Ehemannes \emph{nur} schön auszusehen. Wenn man
fünf widerborstige Schwestern hat, lernt man schnell sich
durchzusetzen, oder man findet sich mit irgendeiner langweiligen
Lordschaft verheiratet wieder … Oh! Verzeihung, Eure
\emph{Lordschaft}.« Sie klimperte kokett mit den Wimpern.

\bigpar

In diesem Augenblick kam Cumberland zur Tür herein. »Die Kutsche
ist vorgefahren.«

McGuire bot der knapp zehn Jahre jüngeren Frau galant den Arm.
Gemeinsam schlenderten sie zur Kutsche.

»Ja, ich bewundere Verne. Aber auch all die anderen Utopisten. Ohne
solche Menschen – Menschen mit Visionen … mit Phantasie – würden
wir vielleicht noch auf den Bäumen hocken.«

»Sie sind doch nicht etwas auch ein Anhänger von diesem Darwin?
Lassen Sie das nicht den Erzbischof hören!«

»Verne hat sich nur in einem wichtigen Punkt geirrt. Man kann eine
Rakete nicht mit einer Kanone \emph{abschießen}. Die Besatzung
würde durch die Kräfte des Abschusses regelrecht – verzeihen Sie
mir den ordinäre Ausdruck – zerquetscht werden. Aber warten Sie ab,
bis Sie es sehen!«

Sie stiegen in die Kutsche und machten sich auf den langen Weg
durch das hügelige Hinterland. Nach einer Stunde, in der Lady
Thornton die beiden Männer mit fundiertem Wissen und
ausgezeichneter Bildung überraschte, erreichten sie einen Durchlass
in einer schwer bewachten Zaunanlage.

»Nationales Interesse! Sie verstehen?«, sagte Cumberland, als die
Soldaten Ihrer Majestät sie durchließen. »Wer den Æther beherrscht,
beherrscht die Welt.«

Schließlich bog ihre Kutsche um einen letzten Hügel … und da lag
sie vor ihnen.

»Dies ist die \emph{Explorer}!«, Nicolas deutet voller Stolz auf
das Gebirge aus Kupfer, Messing und Holz. Die \emph{Explorer} war
beileibe nicht schön, aber beeindruckend in ihren Ausmaßen. Im
Grunde sah sie aus wie eine Dampflokomotive, die garstige Riesen
zur Form einer Mörsergranate verknetet hatten.

»Sie liegt ja auf Schienen!«, sagte Lady Thornton verblüfft,
nachdem sie eine Weile die gigantischen Ausmaße der Gefährtes
bestaunt hatte.

»Wie ich bereits sagte: Verne lag falsch mit seiner Kanone. Wie Sie
sehen, laufen die Schienen von der Spitze dieses Hügels dort durch
das Tal bis zum Gipfel jenes Berges dort in der Ferne. Wir
verwenden ein Dampfkatapult, das die \emph{Explorer} den ganzen Weg
\emph{kontinuierlich} beschleunigt. Erst wenn wir dort hinten die
Schienen verlassen, setzt unser eigener Antrieb ein.« Nikolas war
in seinem Element. »Wir verwenden eine gewichtsreduzierte WATSON
231 Dampfmaschine mit fliehkraftgeregelter Druckkontrolle und eine
spezielle Übersetzung mit verminderter Reibung. Eine automatische
Zufuhr schiebt spezielle Kohlepellets nach einem fein abgestimmten
Programm in den Brennofen ein. Die Kessel werden bereits
aufgeheizt. Sehen Sie diese Holzluken dort, Mylady? Sobald uns das
Dampfkatapult aus dem Bereich der Erdanziehungskraft herausgeworfen
hat, werfen wir diese Luken ab – das spart Gewicht – und dann
fahren vier Propeller aus, die im Moment noch hinter den Luken im
Rumpf der \emph{Explorer} versenkt sind.«

»Wenn man bedenkt, dass vor wenigen Jahrzehnten noch dieser Newton
behauptete, dass das Schwerefeld der Erde zu stark sei, um es mit
dampfgetriebenen Luftschiffen zu verlassen.« Auch Cumberland stand
die Begeisterung in Gesicht geschrieben. »Mit diesem Gefährt werden
wir Geschichte schreiben. In wenigen Jahren werden hunderte solcher
Fahrzeuge unterwegs sein, um neue Welten zu erobern.«

»Und was geschieht, wenn wir diese neuen Welten erobert haben?«,
fragte die junge Frau, die ihren Blick nur mühsam von der
\emph{Explorer} losreißen konnte. »Werden wir dann das Leid, mit
dem wir unseren eigenen Planeten zu Grunde richten, auch auf neue
Planeten exportieren? Ich \emph{war} in Afrika und ich habe
gesehen, was wir \emph{zivilisierten} Weißen mit den schwarzen
Eingeborenen gemacht haben! Sie haben uns mit offenen Armen
empfangen – und wir machten Leibeigene aus ihnen!«

»Mich dünkt, Eure Ansichten sind recht … \emph{liberal}, Mylady«,
sagte Lord Cumberland mit einem missbilligenden Heben der
Augenbraue.

»Bei Ihnen klingt das wie ein Schimpfwort! Spricht es denn gegen
die Würde unseres Standes, die Würde anderer Menschen … die Würde
\emph{aller} Menschen zu achten?«

»Bitte, Mylady! Lord Cumberland! Echauffieren Sie sich bitte nicht!
Vielleicht ist es dem Menschen ja gar nicht gegeben, die Erde zu
verlassen. Vielleicht überstehen unsere schwächlichen Körper den
Start ja gar nicht. Vielleicht ist die Luft dort draußen im Æther
doch nicht atembar? Vielleicht wimmelt es dort oben von Meteoren,
die jeden Verkehr zwischen den Planeten unmöglich machen.«

»Wenn Sie mich fragen«, flüsterte Lord Cumberland mit einem
Seitenblick auf Lady Thornton, »halte ich Meteoriten nicht
unbedingt für die größte Gefahr, die uns auf der Reise droht.«

\tb

„Verehrter Lord McGuire, zuerst einmal danke ich Ihnen dafür, dass
Sie sich für dieses exklusive Interview bereit erklärt haben. Der
Start Ihrer … Wie nennen Sie es? \emph{Weltenrakete}? … wird
\emph{das} Ereignis von nationaler Bedeutung sein. Und deshalb ist
es mir eine besondere Ehre, dass Sie den Lesern der
\emph{New London Gazette} diesen Logenplatz in der Geschichte
zugedacht haben!«

»Es ist mir ein Vergnügen, Mister Conolly! Ich möchte Sie aber noch
einmal nachdrücklich auf unsere Vereinbarung hinweisen, dass dieses
Interview erst am Tage vor dem Start am 19. Mai erscheinen darf!
Wie ich Ihnen bereits mitgeteilt habe, wissen wir, dass eine Gruppe
französischer Offiziere an einem ähnlichen Vorhaben arbeitet. Es
ist eine Frage der nationalen Ehre, dass es ein Gefährt des
Britischen Empires sein \emph{muss}, das zuerst den Weltenæther
zwischen den Planeten erobert.«

»Das bringt mich auf die Frage nach Ihrer Motivation. Unbeachtet
davon, dass es eine Frage der Ehre ist. Was bringt einen Gentleman
dazu, dieses Risiko auf sich zu nehmen?«

»Sehen Sie, der Reichtum des Commonwealth liegt in seinen Kolonien
begründet. Um seinen Reichtum zu bewahren, muss ein Reich
expandieren. Nun, da die weißen Flecken der \emph{Terra Incognita}
immer kleiner werden, wird eine weitere Expansion nicht ohne
militärische Konflikte ablaufen. Aber \emph{da draußen} erwarten
uns genügend unerforschte Gestade – wie seinerzeit Australien – um
neue Kolonien zu gründen.«

»Aber Sie wissen doch überhaupt nicht, was Sie dort draußen
erwarten wird.«

»Da muss ich Ihnen widersprechen: Heute wissen wir sehr wohl, wie
es dort draußen aussieht. Zumindest \emph{glauben} wir es zu
wissen. Sehen Sie, früher glaubte man, die Erde stehe im Zentrum
des Universums. Dabei war das nur die Erklärung für das, was die
Menschen sahen: Sterne gehen im Osten auf, im Westen unter – also
dreht sich der Kosmos offenbar um die Erde. Sternbilder verändern
nie ihren Umriss – also müssen ihre Lichtpunkte wohl an der
Innenseite einer weiten Kugel fixiert sein, in deren Zentrum die
Erde ruht. Jede andere Konfiguration würde nämlich im Lauf der
Nacht den Abstand der Fixsterne verändern und perspektivische
Verzerrungen verursachen. Und auf dieser Fixsternsphäre kreisen die
Wandelsterne wie Mond, Sonne, Planeten um die Erde auf ihren
verschiedenen Bahnen.«

»Sie sprechen von der Vorzeit.«

»Durchaus nicht! Es sind gerade mal 500 Jahre her, dass Kopernikus
die Erde von ihrem Thron im Zentrum des Universums gestoßen hat.
Zumindest der hinreichend gebildete Mensch weiß dies. Aber gehen
Sie mal aufs Land und fragen Sie einen einfachen Bauern … Dort
spricht man immer noch von \emph{Himmelsgewölbe}. Heute wissen wir,
dass die Erde eine Welt unter vielen ist und der Blick durch immer
bessere Teleskope zeigt uns, wie es auf diesen Welten aussieht.
Zumindest \emph{glauben} wir es zu wissen. Aus diesem
\emph{Glauben} wird jedoch nur \emph{Wissen}, wenn jemand hinfährt
und nachsieht.«

»Stimmt es, dass Sie den Mond nur umrunden wollen? Wenn Sie schon
einmal diesen weiten Weg zurückgelegt haben, warum können Sie dann
nicht landen?«

»Landen könnten wir schon! Aber wir könnten nicht mehr starten!
Auch wenn die Schwerkraft auf dem Mond geringer als auf unserer
Erde ist, es fehlt uns auf dem Mond ein geeignetes Dampfkatapult,
das uns hilft, diese Schwerkraft zu überwinden. Aus eigener Kraft
ist unser Weltenschiff nicht dazu in der Lage. Noch nicht! Wir
legen \emph{jetzt} den Grundstein für eine Mondlandung, indem wir
ein mögliches Landeterrain sondieren. Es bleibt späteren
Expeditionen überlassen, eine permanente Kolonie auf dem Mond zu
errichten und ein geeignetes Rückkehrkatapult zu installieren.«

»Würden denn die Menschen einer solchen Kolonie auf dem Mond
überleben können?«

»Nun, spätestens seit den Höhenexpeditionen der Gebrüder
Montgolfier wissen wir, dass der gesamte Weltenæther mit atembarer
Luft gefüllt ist. Bis man Wasser auf dem Mond selbst findet, müsste
man es von der Erde oder den Eisplanetoiden herbeischaffen. Ob der
Mondstaub allerdings ausreichend fruchtbar für den Anbau von
Nahrung ist, ist ein Restrisiko, das ich persönlich bereit wäre
einzugehen.«

»Apropos \emph{Risiko}: Ist es nicht zu riskant, eine Frau mit auf
die Expedition zu nehmen?«

»In der Tat: Es \emph{ist} riskant! Aber Lady Ellen Thornton hat in
den letzten Jahren gezeigt, dass Sie das Risiko nicht scheut!«

»Aber ist es denn schicklich – gerade für eine Dame ihres Standes –
Wochen zusammen mit zwei \emph{Männern} auf engstem Raum zu
verbringen?«

»Ich bitte Sie! \emph{Gerade} Angehörige unseres Standes verfügen
über die nötige Distinguiertheit, um \emph{über} solchen Gelüste zu
stehen, auf die Sie offenbar anspielen. Wir treten diese Expedition
als Vertreter der Menschheit an – der \emph{gleichberechtigten}
Menschheit.«

»Sie sprachen Australien an. Erwarten Sie, auf neue Aboriginies zu
treffen?«

»Ganz auszuschließen ist dieser Gedanke nicht. Aber das Ausmalen
dieses Bildes überlasse ich lieber den hochverehrten Herrschaften
Shelley, Conan-Doyle, Wells und Verne.«

\tb

Nicolas John McGuire III stellte eine hölzerne Kiste auf den Tisch,
die ein wenig aussah wie ein dickes Grammophon samt Schalltrichter,
mit einer milchig, gläsernen Platte an der Stirnseite. »Dies ist
eine Erfindung eines befreundeten preußischen Obersts. Er nennt sie
\emph{Fernzeichner}. Sie sendet geheimnisvolle Strahlen aus – er
nennt sie Z-Strahlen. Wenn die Strahlen auf einen Gegenstand
treffen, werden sie zurückgeworfen und hier in diesem Trichter
gesammelt. Auf irgendeine Weise, fragt mich nicht \emph{wie}, wird
hier …«, er klopfte auf die milchige Glasplatte, »… ein Bild
daraus, wie der Æther vor uns aussieht. Wenn ein Hindernis vor uns
ist, sollte es hier erscheinen.«

Sein Freund Ralph Cumberland trat näher und beäugte den Kasten
misstrauisch. »Hat das irgendetwas mit dieser neumodischen
Elektrizitätskraft zu tun, die auf so geisterhafte Weise Dinge
bewegen soll?«

Nicolas zuckte mit den Schultern. »Es funktioniert! Wie … ist mir
egal.«

»Ich habe davon in einer Gazette gelesen. Das ist unheimlich! Wenn
nicht sogar \emph{unheilig}!«

»Ich habe auch davon gelesen«, sagte Lady Thornton. »Das ist die
Zukunft!«

»Das ist Magie! \emph{Schwarze} Magie!« Cumberland fuchtelte mit
seiner Pfeife herum. »Die Welt wird vom Dampf bewegt. Das reicht!«

»In ein paar Jahrzehnten wird die Dampfkraft von der Elektrizität
verdrängt werden!«, fauchte die Lady.

»Papperlapapp! Das Britische Empire ruht auf den Säulen Kohle,
Dampf und den Ressourcen seiner Kolonien! Und so wird es – mögen
Gott und die Königin uns beistehen – immer sein!«

Nicolas blickte zwischen seinem Freund und dieser streitbaren
\emph{Frau} hin und her. Auf der einen Seite sein Freund Ralph, den
er seit Jahren kannte und schätzte und der ihm nun erschreckend
blasiert und borniert vorkam. Und auf der anderen Seite dieses
aufgeschlossene, scharfzüngige, quirlige, streitbare, starke … und
zarte Wesen.

\tb

Der Start war schlimmer gewesen als erwartet. Tonnen schienen ihre
Körper in die Spezialsessel zu pressen. Die unhandlichen Anzüge aus
feuerhemmenden Materialien taten das ihre, die Andruckphase zu
einer Tortur zu machen. In ihren Rücken fauchte der Kessel,
gurgelten die Leitungen, stampften die Kolben – wie ein
vorsintflutliches Ungeheuer. Lady Thornton hielt sich besser als
erwartet. Beim Einsetzen der Schwerelosigkeit verlor sie kurz das
Bewusstsein, doch schon bald nahm sie ihren Platz vor den
Backbordinstrumenten ein. McGuire saß festgeschnallt vor dem Ruder
und Cumberland hielt mit einer Karbidlampe und einem System von
Parabolspiegeln Kontakt mit Beobachtern auf der Erde.

»Ich wünschte, mehr Menschen könnten diesen Blick auf den Erdball
werfen.« Lady Thornton konnte ihren Blick kaum vom Backbordbullauge
abwenden. »Es würde ihnen zeigen, wie klein sie und ihre
kleinlichen Zänkereien um die Länder Anderer sind.«

McGuire unterdrückte ein Grinsen als er die Schnute sah, die Lady
Thornton ganz undamenhaft zog. Diese Frau – und dass sie eine Frau
war, konnte auch die eher funktionale als ästhetische Schutzmontur
nicht verbergen – benahm sich manches Mal eher wie ein Lausejunge,
als eine Lady des Commonwealths. McGuire ertappte sich immer öfter
dabei, dass er sich, wie ein Bildhauer vor einem unbehauenen Block
Marmor, die Figur \emph{darin} vorstellte.

»Sie spielen doch nicht etwa die Kolonialpolitik unseres
ruhmreichen Landes an?«, brauste Cumberland auf.

»Wo denken Sie hin, Lord Cumberland?«, begann sie in zuckersüßem
Tonfall, um just, als ein befriedigter Ausdruck auf seinem Gesicht
erschien, nachzuschieben: »Frankreich, Preußen, Spanien sind keinen
Deut besser! Überall regiert die Gier auf die Habe des anderen …
auf das vermeintlich grünere Gras auf der anderen Seite.«

»Warum sind Sie überhaupt mitgekommen? Auf einer Reise, die
möglicherweise das Fundament für \emph{neue} Kolonien legt?«

»Um zu verhindern, dass Menschen wie Sie
\emph{für Gott und den König} Besitz von anderen Wesen und deren
Ländereien ergreifen und zugrunde richten – wie es in Afrika oder
der Neuen Welt geschehen ist!«

»Sie beleidigen Ihr Land, \emph{Mylady}! Wenn Sie ein Mann wären
…«

»Was dann? Würden Sie mir dann in männlichem Imponiergehabe den
Fehdehandschuh hinwerfen? Um einen Menschen, der anderer Meinung
ist als Sie, mundtot zu machen? – Ja nicht auf die Argumente der
anderen hören, lieber gleich diese Meinung \emph{tilgen}?«

»Ralph! Mylady! Sie vergessen sich!«, fuhr McGuire empört
dazwischen. »Unsere Reise ist gefährlich genug. Auch ohne Ihre
Zwistigkeiten! Ralph, darf ich Sie an unsere Gespräche im Club
erinnern, in denen auch wir zwei über gewissen Schattenseiten
unserer Politik diskutiert haben. Wohl gemerkt: distinguiert
diskutiert. Wie es sich für Gentlemen geziemt! – Und Sie, meine
Dame, werfen meinem Freund das Gleiche vor, was Sie selbst
praktizieren: Auch \emph{Sie} sehen \emph{Ihre} Meinung als die
allein gültige an und wollen Cumberland seinen Standpunkt nicht
zugestehen. – Also bitte: Mäßigen Sie sich! – Beide!«

Cumberland starrte ihn einen Augenblick aufgebracht an, dann drehte
er sich brüsk um und widmete sich wieder seinen Geräten. Ellen
Thornton zog wieder ihre Schnute, doch diesmal wie ein gescholtenes
Mädchen. Man sah ihr an, dass sie nicht gewohnt war, ihre eigenen
Argumente um die hübschen Ohren gehauen zu bekommen.

\tb

»Mit dem Fernzeichner scheint etwas nicht zu stimmen.« Ellen
Thorntons Stimme klang beunruhigt von ihrem Platz herüber. »Er
zeigt ein massives Hindernis voraus!«

»Hab ich nicht gesagt, dass dieses Ding suspekt ist!« Cumberland
sah nicht von seinem Platz am Teleskop auf. »Wir schneiden gleich
zum ersten Mal die Mondbahn. Noch ein paar tausend Meilen und wir
werden als erste Menschen einen Blick auf die unbekannte Rückseite
des Mondes werfen können! Er ist das einzig Massive weit und
breit!«

Die letzten Tage hatte es nur kleinere Reibereien zwischen Lady
Thornton und Cumberland gegeben, denen Nicolas stets wie ein Puffer
zwischen ihnen die Spitzen genommen hatte. Zu seinem Erstaunen
musste McGuire feststellen, dass die Frau auf Bemerkungen von ihm
weit weniger patzig reagierte als auf solche seines Freundes.

Nicolas verriegelte das Steuerrad in seiner Position, schnallte
sich los und schwebte zu der jungen Frau hinüber.

Am Rande der Milchglasscheibe zeigte sich ein schnurgerader
Strich.

»Hier! Sehen Sie, Lord McGuire! Es scheint, als hätten wir eine
massive Mauer vor uns. So weit das Auge reicht … beziehungsweise
die Strahlen des Gerätes.«

»Hmm? Vielleicht hat das Gerät tatsächlich einen Defekt.«

»Diese … Mauer \emph{hat} sich bewegt!«, sagte sie eindringlich.
»Viel mehr: \emph{Wir} bewegen uns … mit voller Geschwindigkeit
darauf zu! – Wir \emph{müssen} beidrehen!«

Nicolas schwebte zum Bullauge am Bug und hielt sich links und
rechts an den Messinggriffen fest. So sehr er auch in die
sternendurchwirkte Schwärze des Himmelsæthers starrte, er konnte
kein Hindernis erkennen. Nun kam auch Cumberland herbeigeschwebt
und blickte angestrengt nach draußen. »Nichts!«, brummte er nach
einer Weile. »Ich sehe nur Sterne. Die sehen zwar merkwürdig nah
aus, aber das liegt wohl …«, er klopfte mit dem Knöchel seines
Zeigefingers an das Glas des Bullauges, »… an diesem deutschen
Spezialglas, dass Sie unbedingt haben wollten, mein Freund.« Damit
räumte er den Platz und schwebte zu seinem Posten zurück.

Nicolas sah nach draußen. Plötzlich spürte er einen Luftzug an
seiner Seite. Dann prallte Lady Thornton, die herangeschwebt war,
sanft gegen ihn und krallte sich an den Handgriffen fest. Ihre
Hände berührten sich. Diese Nähe verwirrte ihn. Der dezente Duft
ihres Parfüms drang in seine Nase, konnte jedoch \emph{ihren} Duft
nicht ganz überdecken. »Nick, bitte … Lord McGuire, bitte vertrauen
Sie mir. Vertrauen Sie meiner weiblichen Intuition. Wir müssen
beidrehen.«

»Mein liebes Kind!«, donnerte Cumberland von achtern. »Wenn wir
jetzt beidrehen, haben wir vielleicht nicht mehr genügend Kohle für
eine Rückkehr! Wollen Sie uns dem Tode überantworten?«

»Wenn wir nicht beidrehen, kollidieren wir mit voller
Geschwindigkeit mit … mit … dem da!« Sie deutete auf den Strich auf
der Scheibe des Fernzeichners. Nun sah auch McGuire, dass der
Strich weiter zur Mitte gewandert war. Irgendjemand musste eine
Entscheidung treffen! Noch einmal spähte er durch das Bullauge. Mit
beiden Händen schirmte er die störenden Reflexe der Gaslaternen ab.
Nichts! Nur das Meer der Sterne. So fern, und doch so …
\emph{nah}.

Er drehte sich herum, stieß sich an der Täfelung ab und schoss quer
durch den Raum. Mit beiden Händen fing er sich ab, schwang sich vor
die Steuerung, löste die Arretierung und warf das Ruder herum.

»Was tun Sie da?«, brüllte Cumberland. »Hat diese Furie Ihren
Verstand geraubt? Selbst wenn da etwas wäre, wir bekämen unser
Vehikel niemals bis dahin zum Stillstand.«

Wortlos arretierte McGuire das Ruder in seiner Endlage. Dann warf
er sich herum in Richtung Heizungsraum. An der Stahltüre fing er
sich ab. Mit fliegenden Fingern riss er Kohlepakete aus den
Zuführmagazinen und stopfte sie per Hand in den Brennofen.
»Richtig! Zum Abbremsen sind wir zu schnell, aber mit etwas Glück
können wir daran vorbeischrammen!«

»Vorbeischrammen? Woran vorbeischrammen? Kerl! Sind Sie von Sinnen?
Sie verheizen unseren Vorrat für die Rückreise!«

»Der Kohlenvorrat wird reichen. Wir müssen nur etwas langsamer
fahren … und den Gürtel enger schnallen.«

»Unser Kurs ändert sich! Wir können es schaffen!« Lady Thornton
zeigte auf den Fernzeichner. Ihr Finger zitterte. Die Linie quer
auf dem Glas hatte sich nach rechts geneigt. Bald würde sie
parallel zur Flugrichtung stehen … wenn sie nicht vorher mit diesem
\emph{Etwas} kollidierten.

»Cumberland! Ralph! Fahren Sie um Himmels Willen die Räder aus!«,
schrie Nicolas. »Ellen! Schnallen Sie sich an!« Er sah noch, wie
sein Freund an den wuchtigen Hebeln hantierte.

\bigpar

Dann war nichts mehr.

\tb

Er erwachte in einem Bad aus Düften. Als er blinzelnd die Augen
öffnete, blickte er in zwei grüne Seen über sich … und ein Meer aus
Kopfschmerzen trieb ihm das Wasser in die Augen. Er lag auf dem
Boden und sein Kopf ruhte weich im Schoß von Lady Thornton. Sein
Körper fühlte sich so leicht an, doch eigentlich wollte er nie
wieder aufstehen.

»Du hast dir den Kopf angestoßen, Nicolas«, sagte die sanfte Stimme
über ihm. Etwas Kühles drückte sich auf seine Stirn. »Lord
Cumberland läuft draußen herum und schaut sich die Schäden der
Bruchlandung an.«

\bigpar

Nicolas richtete sich auf und hielt sich stöhnend den Kopf.
»\emph{Läuft} draußen herum? Auf \emph{wem} oder \emph{was} läuft
er?«

»Das siehst du … das sehen Sie sich besser selbst an, Lord
McGuire«, verbesserte sie sich rasch.

\bigpar

Er griff nach ihrer Hand und drückte sie sanft. »Es klingt schöner,
wenn du \emph{Nicolas} sagst!« Sie hielten in einem Moment
zärtlicher Übereinkunft inne. Dann rappelte sich Nicolas ächzend
auf und kletterte vor ihr nach draußen. Es war nicht nur die
Gegenwart Ellens, die ihn beschwingte – er \emph{hatte} nur ein
Bruchteil des Gewichtes, das er auf der Erde wog. Am Heck des
Schiffes tänzelte die Karbidlaterne in der Hand von Lord Cumberland
in der Dunkelheit. Es knirschte, als Nicolas seinen Fuß vorsichtig
von der Leiter auf den nachtschwarzen Boden setze.

»Ralph! Lord Cumberland! Wie sieht es aus?«

»Ah! Sind Sie wieder auf den Beinen?« Der Lichtkreis der Laterne
wanderte auf sie zu und riss die Holzbeplankung des Schiffes aus
der allgegenwärtigen Finsternis. »Ich muss Ihnen danken! Ihnen
beiden! Ohne Sie wären wir auf diesem Himmelskörper zerschellt. Das
Schiff hat nur wenig abbekommen. Und wie Sie sicher bemerkt haben,
ist die Schwerkraft so gering, dass ein Abheben eigentlich möglich
sein sollte. Und wir haben sogar noch soviel Kohlen, dass wir das
hier …« Er drehte seinen Fuß auf dem Fußballen. Es knirschte. »…
gar nicht benötigen! Wär’ eine ziemliche Plackerei, das
zusammenzukratzen.«

McGuire hockte sich nieder und strich mit der Hand über eine
hauchdünne Staubschicht auf dem Boden. »Ist das Ruß?«

»In der Tat! Dieser Boden ist über und über damit bedeckt. Und
darunter ist etwas, das ich in Ermangelung eines besseren Begriffes
\emph{Stahl} nennen möchte.«

»Wollen Sie damit sagen, dass diese Welt \emph{künstlich} ist?«,
keuchte Lady Thornton.

»Fällt Ihnen eine andere Antwort ein, Mylady?«

\tb

»Nicolas, schau doch, dort drüben!« Ellen deutete auf einen schwach
leuchtenden Fleck in einiger Entfernung oberhalb des unsichtbaren
Horizonts. »Was mag das sein?«

\bigpar

Cumberland hob eine Augebraue ob dieser ungebührlichen
Vertrautheit.

\bigpar

»Und dort sind noch mehr! Und dort … und dort!« Sie schienen in
einem Talkessel gelandet zu sein. Nun, da sich seine Augen an die
Finsternis gewöhnt hatten, schälten sich immer mehr schwache
Lichtflecken an den ansteigenden Hängen des Talkessels aus der
Schwärze der Nacht. Zwar standen der grelle Ball der Sonne und die
blaue Murmel der Erde hoch am Himmel, und auch der Mond hing hell
strahlend rund vierzig Grad über dem Horizont, doch alle drei
schafften es nicht, die Schwärze des Untergrundes zu erhellen. Es
schien, als ob dieser \emph{Ruß} jedes Licht verschluckte.

»Lasst uns nachsehen!« Nicolas packte zwei Karbidlaternen aus. Dann
machten sich die drei mit weit ausholenden Sprüngen schweigend auf
den Weg zur nächstgelegenen Lichtquelle.

»Man merkt gar nicht, dass wir bergauf gehen«, sagte Lady Thornton
nach einer Weile.

»Warum sollten wir bergauf gehen?, fragte Cumberland.

»Weil es so aussieht, als ob dieses leuchtende Oval auf das wir zu
marschieren, im Hang eines kleinen Hügels steckt – den man in
dieser Schwärze nicht sehen kann. Und wenn wir uns auf etwas
zubewegen, das \emph{über} uns liegt, muss es nun mal
\emph{bergauf} gehen!«, meinte sie schnippisch.

»Ich habe den Eindruck, dass das Licht nicht mehr ganz so hoch
hängt«, bemerkte McGuire. In der Tat erschien es ihm, dass der
Lichtfleck immer weiter zu Boden sank, je mehr sie sich ihm
näherten, dafür schienen die Lichter an der anderen Seite des
Talkessels angestiegen zu sein. Schließlich standen sie vor einem
mannsgroßen, blendend hellen Fleck im flachen Boden. Sein Licht
warf scharfe Kontraste auf ihre Kleidung.

»Vorsicht!«, rief Lady Thornton, als Cumberland seine Hand langsam
dem Licht näherte, die Augen mit dem anderen Arm beschattet.

»Kalt! Strahlt keine Wärme aus!«, sagte er. Dann zog er einen
Schraubendreher aus der Tasche und tippte damit auf die leuchtende
Fläche. »Sieht nach so etwas wie Glas aus. Fühlt sich aber viel
weicher an.« Nun wagten sich auch die anderen näher. Die strahlend
helle Fläche war übergangslos in den Boden eingelassen.

»Keine Fuge!«, sagte Cumberland, der mit seinem Schraubendreher
über den Rand der Fläche kratzte. »Das ist mir zu hell!« Er erhob
sich und drehte sich um. »Merkwürdig, keine Sterne zu sehen«,
brummte Cumberland als er wieder klar sehen konnte. »Nur im Zenit
sind die Sterne klar zu sehen. Zum Horizont hin werden sie immer
schwächer. Ob die Luft dieser Planetenkugel so diesig ist?«

»Sie standen gerade vor einem, mein Freund!«

»Bitte?«

»Sie standen gerade vor einem Stern!«

»Sind Sie toll, Nicolas?«

\bigpar

McGuire lachte höhnisch. »Ich wollte, ich wäre es! Ich wünschte,
dies wäre eine Ausgeburt meines kranken Geistes … und nicht der
Ruin unseres Weltbildes! Das ist keine Planetenkugel auf der wir
hier stehen … sondern das Himmelsgewölbe. Und das da …«, er zeigte
auf den Lichtfleck im Boden, »… ist einer der vielen Sterne, die
man am Nachthimmel sieht.«

Sein Lachen nahm einen irren Tonfall an. »Wir sind nicht auf den
Hügel eines kugelförmigen Körper gestiegen, sondern wir bewegen uns
an der Innenwand einer gigantischen \emph{Hohlkugel} entlang!«

Er deutete hinauf zum Mond. »Schauen Sie sich den Mond an. Sollten
wir seine Rückseite nicht als konvexe Form wahrnehmen? Doch sieht
die Rückseite nicht eher wie … abgeschnitten aus? Wie eine
Halbkugel, die auf der riesigen Fläche der Himmelsphäre festgemacht
ist? – Keppler, Galilei, Kopernikus! Alle haben sie falsch gelegen!
Die Erde steht im Zentrum ihres eigenen, kleinen Universums, und
alles um Sie herum – Sterne, Planeten, Monde, die Sonne – alles nur
Fälschungen! Vorspiegelungen eines zynischen Wesens, das uns in
einem Terrarium hält!«

»Ihr seid lästerlich, Nicolas!«

»Nein, er hat Recht, Lord Cumberland«, sagte Ellen Thornton tonlos.
»Die Welt, wie wir sie kennen – zu kennen glaubten – ist mit einem
gewaltigen Getöse eingestürzt!«

Sie wandte sich dem Schiff zu. »Wir müssen die französische
Expedition warnen, dass sie nicht wie eine Fliege an die Wand
klatscht.«

\tb

»Es wird Kriege geben«, meinte Nicolas nach dem geglückten Start.
Die geringe Schwerkraft hatte in der Tat kein Problem dargestellt.
»Kriege zwischen religiösen Fundamentalisten, die in diesem
\emph{Terrarium} den Beweis für eine Schöpfung Gottes sehen werden,
und Aufklärern, die nicht ruhen werden, einen Blick \emph{hinter}
die Scheibe des Terrariums zu werfen.«

\bigpar

»\emph{Terrarium}«, sagte Cumberland nun tonlos, ohne aufzublicken.
Seit Stunden hatte er nichts mehr gesagt, sondern nur in seinem
Sitz gekauert und vor sich hin gestiert. »Das ist der richtige
Ausdruck. Wir sind das Spielzeug von … was auch immer. – Es wird
eine beispiellose Welle von Suiziden geben.«

\bigpar

»Wird man uns überhaupt glauben, Nicolas?« Seit dem Start hatte sie
seine Nähe gesucht, so als bräuchte sie in einen Halt für ihr
Leben. »Dürfen wir das den Menschen überhaupt sagen?«

»Die Wahrheit lässt sich nicht verschweigen! Eine Weile kann man
sie vielleicht unterdrücken, aber es werden andere kommen.« Nicolas
legte seinen Arm um sie … und sie ließ es sich gefallen. Die Zeiten
des demonstrativen Zurschaustellens ihrer Unabhängigkeit waren
vorbei. Sie hatten vom Baum der Erkenntnis gegessen und die
Erkenntnis war zu groß für einen Einzelnen.

»Andere, die herausfinden werden, wie diese Himmelssphäre
funktioniert. Wie sich die falschen Planeten auf ihren Bahnen –
oder vielleicht \emph{Schienen} – bewegen. Und was jenseits der
Sphäre ist.«

»Andere? \emph{Du} möchtest wieder kommen und die Geheimnisse
lösen.

\bigpar

Er drückte seine Stirn an ihre. »\emph{Ich} werde bei Dir sein und
wir werden die Unruhen, die da kommen mögen, überstehen.
Gemeinsam!«

\bigpar

Sie sah ihn mit einem Augenaufschlag an und ein klein wenig blitzte
in ihren Augen der Lausejunge auf. »Dann können wir die Geheimnisse
vielleicht Nicolas John McGuire, dem Vierten, überlassen?«

\end{document}

