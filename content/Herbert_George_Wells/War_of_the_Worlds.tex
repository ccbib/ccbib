\hyphenation{mo-no-poly car-ne-gie pro-ject pro-gress mo-dem rou-lette
  browse-wrap Use-net mon-as-tery mo-dems}
\hyphenation{co-me-dic polt-roon stove-pipe Ma-dame scru-ta-ble star-tling}
\hyphenation{heal-thily lim-ou-sines wrest-lers tan-trum push-over un-asked
  bras-siere bro-th-er}
\hyphenation{Can-a-da Fred-rick teen-agers wrest-ler Cha-vez Tho-mas 
  a-nom-a-lies sur-veil-lance ar-mies ref-u-gee ref-u-gees bris-tling
  eve-ning man-chu-ria man-chu-ri-an mid-terms me-di-um jap-a-nese}
\hyphenation{spend-ers googl-ing tour-ist tour-ists leg-end-ary}
\hyphenation{Dan-iel Van-essa Doc-to-row Ste-phen-son}
\hyphenation{de-cade sur-veilled rout-ers Wol-fen-stein teen-ager to-night}
\hyphenation{his-to-gram an-o-nym-ize Ga-la-xy sym-pa-the-tic}
\hyphenation{ar-phid ar-phids Found-ers}
\hyphenation{stran-ger stran-gers shoul-der-blades dump-ling dump-lings}
\hyphenation{ice-pack guard-rail Sep-tem-ber boot-able e-co-nom-ist}
\hyphenation{grown-ups roos-ter shoe-laces li-quid-i-ty}
\hyphenation{side-arm}
\hyphenation{wo-man wo-men tan-trum tan-trums Le-nin-grad zom-bie bunk-house}
\hyphenation{up-tick bio-mass}
\hyphenation{of-fi-cial of-fi-cial-ly gov-ern-ment}
\hyphenation{heal-thy Or-ville spark-ling}
\hyphenation{ves-ti-bule Law-rence au-to-no-mous}
\hyphenation{sau-sage door-step staf-fer}
\hyphenation{tree-trunk}
\hyphenation{to-ron-to}
\hyphenation{qua-dril-lion-aire qua-dril-lion-aires}
\hyphenation{sports-jack-et sports-jack-ets}
\hyphenation{work-space skunk-works}
\hyphenation{kings-ton}


%\setlength{\emergencystretch}{1ex}

\newcommand{\headline}[1]{\begin{center}#1\end{center}}
\newcommand\Book\part
\newcommand\Chapter\chapter

\begin{document}
\raggedbottom

\begin{Verbatim}[fontsize=\footnotesize]
The Project Gutenberg EBook of The War of the Worlds, by H. G. Wells

This eBook is for the use of anyone anywhere at no cost and with
almost no restrictions whatsoever.  You may copy it, give it away or
re-use it under the terms of the Project Gutenberg License included
with this eBook or online at www.gutenberg.net


Title: The War of the Worlds

Author: H. G. Wells

Release Date: July 1992 [EBook #36]
[Most recently updated October 1, 2004]

Language: English

Character set encoding: ASCII

*** START OF THIS PROJECT GUTENBERG EBOOK THE WAR OF THE WORLDS ***
\end{Verbatim}

\title{The War of the Worlds}
\author{by H. G. Wells}
\titlehead{
`But who shall dwell in these worlds if they be
inhabited? \ldots{} Are we or they Lords of the
World? \ldots{} And how are all things made for man?'

\hfill \textsc{Kepler} (quoted in The Anatomy of Melancholy)
}
\date{1898}
\maketitle

\Book{BOOK ONE\\THE COMING OF THE MARTIANS}
\Chapter{CHAPTER ONE\\THE EVE OF THE WAR}
No one would have believed in the last years of the nineteenth
century that this world was being watched keenly and closely by
intelligences greater than man's and yet as mortal as his own; that
as men busied themselves about their various concerns they were
scrutinised and studied, perhaps almost as narrowly as a man with a
microscope might scrutinise the transient creatures that swarm and
multiply in a drop of water. With infinite complacency men went to
and fro over this globe about their little affairs, serene in their
assurance of their empire over matter. It is possible that the
infusoria under the microscope do the same. No one gave a thought
to the older worlds of space as sources of human danger, or thought
of them only to dismiss the idea of life upon them as impossible or
improbable. It is curious to recall some of the mental habits of
those departed days. At most terrestrial men fancied there might be
other men upon Mars, perhaps inferior to themselves and ready to
welcome a missionary enterprise. Yet across the gulf of space,
minds that are to our minds as ours are to those of the beasts that
perish, intellects vast and cool and unsympathetic, regarded this
earth with envious eyes, and slowly and surely drew their plans
against us. And early in the twentieth century came the great
disillusionment.

The planet Mars, I scarcely need remind the reader, revolves about
the sun at a mean distance of 140,000,000 miles, and the light and
heat it receives from the sun is barely half of that received by
this world. It must be, if the nebular hypothesis has any truth,
older than our world; and long before this earth ceased to be
molten, life upon its surface must have begun its course. The fact
that it is scarcely one seventh of the volume of the earth must
have accelerated its cooling to the temperature at which life could
begin. It has air and water and all that is necessary for the
support of animated existence.

Yet so vain is man, and so blinded by his vanity, that no writer,
up to the very end of the nineteenth century, expressed any idea
that intelligent life might have developed there far, or indeed at
all, beyond its earthly level. Nor was it generally understood that
since Mars is older than our earth, with scarcely a quarter of the
superficial area and remoter from the sun, it necessarily follows
that it is not only more distant from time's beginning but nearer
its end.

The secular cooling that must someday overtake our planet has
already gone far indeed with our neighbour. Its physical condition
is still largely a mystery, but we know now that even in its
equatorial region the midday temperature barely approaches that of
our coldest winter. Its air is much more attenuated than ours, its
oceans have shrunk until they cover but a third of its surface, and
as its slow seasons change huge snowcaps gather and melt about
either pole and periodically inundate its temperate zones. That
last stage of exhaustion, which to us is still incredibly remote,
has become a present-day problem for the inhabitants of Mars. The
immediate pressure of necessity has brightened their intellects,
enlarged their powers, and hardened their hearts. And looking
across space with instruments, and intelligences such as we have
scarcely dreamed of, they see, at its nearest distance only
35,000,000 of miles sunward of them, a morning star of hope, our
own warmer planet, green with vegetation and grey with water, with
a cloudy atmosphere eloquent of fertility, with glimpses through
its drifting cloud wisps of broad stretches of populous country and
narrow, navy-crowded seas.

And we men, the creatures who inhabit this earth, must be to them
at least as alien and lowly as are the monkeys and lemurs to us.
The intellectual side of man already admits that life is an
incessant struggle for existence, and it would seem that this too
is the belief of the minds upon Mars. Their world is far gone in
its cooling and this world is still crowded with life, but crowded
only with what they regard as inferior animals. To carry warfare
sunward is, indeed, their only escape from the destruction that,
generation after generation, creeps upon them.

And before we judge of them too harshly we must remember what
ruthless and utter destruction our own species has wrought, not
only upon animals, such as the vanished bison and the dodo, but
upon its inferior races. The Tasmanians, in spite of their human
likeness, were entirely swept out of existence in a war of
extermination waged by European immigrants, in the space of fifty
years. Are we such apostles of mercy as to complain if the Martians
warred in the same spirit?

The Martians seem to have calculated their descent with amazing
subtlety\dash{}their mathematical learning is evidently far in excess of
ours\dash{}and to have carried out their preparations with a well-nigh
perfect unanimity. Had our instruments permitted it, we might have
seen the gathering trouble far back in the nineteenth century. Men
like Schiaparelli watched the red planet\dash{}it is odd, by-the-bye,
that for countless centuries Mars has been the star of war\dash{}but
failed to interpret the fluctuating appearances of the markings
they mapped so well. All that time the Martians must have been
getting ready.

During the opposition of 1894 a great light was seen on the
illuminated part of the disk, first at the Lick Observatory, then
by Perrotin of Nice, and then by other observers. English readers
heard of it first in the issue of \emph{Nature} dated August 2. I
am inclined to think that this blaze may have been the casting of
the huge gun, in the vast pit sunk into their planet, from which
their shots were fired at us. Peculiar markings, as yet
unexplained, were seen near the site of that outbreak during the
next two oppositions.

The storm burst upon us six years ago now. As Mars approached
opposition, Lavelle of Java set the wires of the astronomical
exchange palpitating with the amazing intelligence of a huge
outbreak of incandescent gas upon the planet. It had occurred
towards midnight of the twelfth; and the spectroscope, to which he
had at once resorted, indicated a mass of flaming gas, chiefly
hydrogen, moving with an enormous velocity towards this earth. This
jet of fire had become invisible about a quarter past twelve. He
compared it to a colossal puff of flame suddenly and violently
squirted out of the planet, ``as flaming gases rushed out of a
gun.''

A singularly appropriate phrase it proved. Yet the next day there
was nothing of this in the papers except a little note in the
\emph{Daily Telegraph}, and the world went in ignorance of one of
the gravest dangers that ever threatened the human race. I might
not have heard of the eruption at all had I not met Ogilvy, the
well-known astronomer, at Ottershaw. He was immensely excited at
the news, and in the excess of his feelings invited me up to take a
turn with him that night in a scrutiny of the red planet.

In spite of all that has happened since, I still remember that
vigil very distinctly: the black and silent observatory, the
shadowed lantern throwing a feeble glow upon the floor in the
corner, the steady ticking of the clockwork of the telescope, the
little slit in the roof\dash{}an oblong profundity with the stardust
streaked across it. Ogilvy moved about, invisible but audible.
Looking through the telescope, one saw a circle of deep blue and
the little round planet swimming in the field. It seemed such a
little thing, so bright and small and still, faintly marked with
transverse stripes, and slightly flattened from the perfect round.
But so little it was, so silvery warm\dash{}a pin's-head of light! It
was as if it quivered, but really this was the telescope vibrating
with the activity of the clockwork that kept the planet in view.

As I watched, the planet seemed to grow larger and smaller and to
advance and recede, but that was simply that my eye was tired.
Forty millions of miles it was from us\dash{}more than forty millions of
miles of void. Few people realise the immensity of vacancy in which
the dust of the material universe swims.

Near it in the field, I remember, were three faint points of light,
three telescopic stars infinitely remote, and all around it was the
unfathomable darkness of empty space. You know how that blackness
looks on a frosty starlight night. In a telescope it seems far
profounder. And invisible to me because it was so remote and small,
flying swiftly and steadily towards me across that incredible
distance, drawing nearer every minute by so many thousands of
miles, came the Thing they were sending us, the Thing that was to
bring so much struggle and calamity and death to the earth. I never
dreamed of it then as I watched; no one on earth dreamed of that
unerring missile.

That night, too, there was another jetting out of gas from the
distant planet. I saw it. A reddish flash at the edge, the
slightest projection of the outline just as the chronometer struck
midnight; and at that I told Ogilvy and he took my place. The night
was warm and I was thirsty, and I went stretching my legs clumsily
and feeling my way in the darkness, to the little table where the
siphon stood, while Ogilvy exclaimed at the streamer of gas that
came out towards us.

That night another invisible missile started on its way to the
earth from Mars, just a second or so under twenty-four hours after
the first one. I remember how I sat on the table there in the
blackness, with patches of green and crimson swimming before my
eyes. I wished I had a light to smoke by, little suspecting the
meaning of the minute gleam I had seen and all that it would
presently bring me. Ogilvy watched till one, and then gave it up;
and we lit the lantern and walked over to his house. Down below in
the darkness were Ottershaw and Chertsey and all their hundreds of
people, sleeping in peace.

He was full of speculation that night about the condition of Mars,
and scoffed at the vulgar idea of its having inhabitants who were
signalling us. His idea was that meteorites might be falling in a
heavy shower upon the planet, or that a huge volcanic explosion was
in progress. He pointed out to me how unlikely it was that organic
evolution had taken the same direction in the two adjacent
planets.

``The chances against anything manlike on Mars are a million to
one,'' he said.

Hundreds of observers saw the flame that night and the night after
about midnight, and again the night after; and so for ten nights, a
flame each night. Why the shots ceased after the tenth no one on
earth has attempted to explain. It may be the gases of the firing
caused the Martians inconvenience. Dense clouds of smoke or dust,
visible through a powerful telescope on earth as little grey,
fluctuating patches, spread through the clearness of the planet's
atmosphere and obscured its more familiar features.

Even the daily papers woke up to the disturbances at last, and
popular notes appeared here, there, and everywhere concerning the
volcanoes upon Mars. The seriocomic periodical \emph{Punch}, I
remember, made a happy use of it in the political cartoon. And, all
unsuspected, those missiles the Martians had fired at us drew
earthward, rushing now at a pace of many miles a second through the
empty gulf of space, hour by hour and day by day, nearer and
nearer. It seems to me now almost incredibly wonderful that, with
that swift fate hanging over us, men could go about their petty
concerns as they did. I remember how jubilant Markham was at
securing a new photograph of the planet for the illustrated paper
he edited in those days. People in these latter times scarcely
realise the abundance and enterprise of our nineteenth-century
papers. For my own part, I was much occupied in learning to ride
the bicycle, and busy upon a series of papers discussing the
probable developments of moral ideas as civilisation progressed.

One night (the first missile then could scarcely have been
10,000,000 miles away) I went for a walk with my wife. It was
starlight and I explained the Signs of the Zodiac to her, and
pointed out Mars, a bright dot of light creeping zenithward,
towards which so many telescopes were pointed. It was a warm night.
Coming home, a party of excursionists from Chertsey or Isleworth
passed us singing and playing music. There were lights in the upper
windows of the houses as the people went to bed. From the railway
station in the distance came the sound of shunting trains, ringing
and rumbling, softened almost into melody by the distance. My wife
pointed out to me the brightness of the red, green, and yellow
signal lights hanging in a framework against the sky. It seemed so
safe and tranquil.

\Chapter{CHAPTER TWO\\THE FALLING STAR}
Then came the night of the first falling star. It was seen early in
the morning, rushing over Winchester eastward, a line of flame high
in the atmosphere. Hundreds must have seen it, and taken it for an
ordinary falling star. Albin described it as leaving a greenish
streak behind it that glowed for some seconds. Denning, our
greatest authority on meteorites, stated that the height of its
first appearance was about ninety or one hundred miles. It seemed
to him that it fell to earth about one hundred miles east of him.

I was at home at that hour and writing in my study; and although my
French windows face towards Ottershaw and the blind was up (for I
loved in those days to look up at the night sky), I saw nothing of
it. Yet this strangest of all things that ever came to earth from
outer space must have fallen while I was sitting there, visible to
me had I only looked up as it passed. Some of those who saw its
flight say it travelled with a hissing sound. I myself heard
nothing of that. Many people in Berkshire, Surrey, and Middlesex
must have seen the fall of it, and, at most, have thought that
another meteorite had descended. No one seems to have troubled to
look for the fallen mass that night.

But very early in the morning poor Ogilvy, who had seen the
shooting star and who was persuaded that a meteorite lay somewhere
on the common between Horsell, Ottershaw, and Woking, rose early
with the idea of finding it. Find it he did, soon after dawn, and
not far from the sand pits. An enormous hole had been made by the
impact of the projectile, and the sand and gravel had been flung
violently in every direction over the heath, forming heaps visible
a mile and a half away. The heather was on fire eastward, and a
thin blue smoke rose against the dawn.

The Thing itself lay almost entirely buried in sand, amidst the
scattered splinters of a fir tree it had shivered to fragments in
its descent. The uncovered part had the appearance of a huge
cylinder, caked over and its outline softened by a thick scaly
dun-coloured incrustation. It had a diameter of about thirty yards.
He approached the mass, surprised at the size and more so at the
shape, since most meteorites are rounded more or less completely.
It was, however, still so hot from its flight through the air as to
forbid his near approach. A stirring noise within its cylinder he
ascribed to the unequal cooling of its surface; for at that time it
had not occurred to him that it might be hollow.

He remained standing at the edge of the pit that the Thing had made
for itself, staring at its strange appearance, astonished chiefly
at its unusual shape and colour, and dimly perceiving even then
some evidence of design in its arrival. The early morning was
wonderfully still, and the sun, just clearing the pine trees
towards Weybridge, was already warm. He did not remember hearing
any birds that morning, there was certainly no breeze stirring, and
the only sounds were the faint movements from within the cindery
cylinder. He was all alone on the common.

Then suddenly he noticed with a start that some of the grey
clinker, the ashy incrustation that covered the meteorite, was
falling off the circular edge of the end. It was dropping off in
flakes and raining down upon the sand. A large piece suddenly came
off and fell with a sharp noise that brought his heart into his
mouth.

For a minute he scarcely realised what this meant, and, although
the heat was excessive, he clambered down into the pit close to the
bulk to see the Thing more clearly. He fancied even then that the
cooling of the body might account for this, but what disturbed that
idea was the fact that the ash was falling only from the end of the
cylinder.

And then he perceived that, very slowly, the circular top of the
cylinder was rotating on its body. It was such a gradual movement
that he discovered it only through noticing that a black mark that
had been near him five minutes ago was now at the other side of the
circumference. Even then he scarcely understood what this
indicated, until he heard a muffled grating sound and saw the black
mark jerk forward an inch or so. Then the thing came upon him in a
flash. The cylinder was artificial\dash{}hollow\dash{}with an end that
screwed out! Something within the cylinder was unscrewing the top!

``Good heavens!'' said Ogilvy. ``There's a man in it\dash{}men in it! Half
roasted to death! Trying to escape!''

At once, with a quick mental leap, he linked the Thing with the
flash upon Mars.

The thought of the confined creature was so dreadful to him that he
forgot the heat and went forward to the cylinder to help turn. But
luckily the dull radiation arrested him before he could burn his
hands on the still-glowing metal. At that he stood irresolute for a
moment, then turned, scrambled out of the pit, and set off running
wildly into Woking. The time then must have been somewhere about
six o'clock. He met a waggoner and tried to make him understand,
but the tale he told and his appearance were so wild\dash{}his hat had
fallen off in the pit\dash{}that the man simply drove on. He was equally
unsuccessful with the potman who was just unlocking the doors of
the public-house by Horsell Bridge. The fellow thought he was a
lunatic at large and made an unsuccessful attempt to shut him into
the taproom. That sobered him a little; and when he saw Henderson,
the London journalist, in his garden, he called over the palings
and made himself understood.

``Henderson,'' he called, ``you saw that shooting star last night?''

``Well?'' said Henderson.

``It's out on Horsell Common now.''

``Good Lord!'' said Henderson. ``Fallen meteorite! That's good.''

``But it's something more than a meteorite. It's a cylinder\dash{}an
artificial cylinder, man! And there's something inside.''

Henderson stood up with his spade in his hand.

``What's that?'' he said. He was deaf in one ear.

Ogilvy told him all that he had seen. Henderson was a minute or so
taking it in. Then he dropped his spade, snatched up his jacket,
and came out into the road. The two men hurried back at once to the
common, and found the cylinder still lying in the same position.
But now the sounds inside had ceased, and a thin circle of bright
metal showed between the top and the body of the cylinder. Air was
either entering or escaping at the rim with a thin, sizzling
sound.

They listened, rapped on the scaly burnt metal with a stick, and,
meeting with no response, they both concluded the man or men inside
must be insensible or dead.

Of course the two were quite unable to do anything. They shouted
consolation and promises, and went off back to the town again to
get help. One can imagine them, covered with sand, excited and
disordered, running up the little street in the bright sunlight
just as the shop folks were taking down their shutters and people
were opening their bedroom windows. Henderson went into the railway
station at once, in order to telegraph the news to London. The
newspaper articles had prepared men's minds for the reception of
the idea.

By eight o'clock a number of boys and unemployed men had already
started for the common to see the ``dead men from Mars.'' That was
the form the story took. I heard of it first from my newspaper boy
about a quarter to nine when I went out to get my
\emph{Daily Chronicle}. I was naturally startled, and lost no time
in going out and across the Ottershaw bridge to the sand pits.

\Chapter{CHAPTER THREE\\ON HORSELL COMMON}
I found a little crowd of perhaps twenty people surrounding the
huge hole in which the cylinder lay. I have already described the
appearance of that colossal bulk, embedded in the ground. The turf
and gravel about it seemed charred as if by a sudden explosion. No
doubt its impact had caused a flash of fire. Henderson and Ogilvy
were not there. I think they perceived that nothing was to be done
for the present, and had gone away to breakfast at Henderson's
house.

There were four or five boys sitting on the edge of the Pit, with
their feet dangling, and amusing themselves\dash{}until I stopped
them\dash{}by throwing stones at the giant mass. After I had spoken to
them about it, they began playing at ``touch'' in and out of the
group of bystanders.

Among these were a couple of cyclists, a jobbing gardener I
employed sometimes, a girl carrying a baby, Gregg the butcher and
his little boy, and two or three loafers and golf caddies who were
accustomed to hang about the railway station. There was very little
talking. Few of the common people in England had anything but the
vaguest astronomical ideas in those days. Most of them were staring
quietly at the big table like end of the cylinder, which was still
as Ogilvy and Henderson had left it. I fancy the popular
expectation of a heap of charred corpses was disappointed at this
inanimate bulk. Some went away while I was there, and other people
came. I clambered into the pit and fancied I heard a faint movement
under my feet. The top had certainly ceased to rotate.

It was only when I got thus close to it that the strangeness of
this object was at all evident to me. At the first glance it was
really no more exciting than an overturned carriage or a tree blown
across the road. Not so much so, indeed. It looked like a rusty gas
float. It required a certain amount of scientific education to
perceive that the grey scale of the Thing was no common oxide, that
the yellowish-white metal that gleamed in the crack between the lid
and the cylinder had an unfamiliar hue. ``Extra-terrestrial'' had no
meaning for most of the onlookers.

At that time it was quite clear in my own mind that the Thing had
come from the planet Mars, but I judged it improbable that it
contained any living creature. I thought the unscrewing might be
automatic. In spite of Ogilvy, I still believed that there were men
in Mars. My mind ran fancifully on the possibilities of its
containing manuscript, on the difficulties in translation that
might arise, whether we should find coins and models in it, and so
forth. Yet it was a little too large for assurance on this idea. I
felt an impatience to see it opened. About eleven, as nothing
seemed happening, I walked back, full of such thought, to my home
in Maybury. But I found it difficult to get to work upon my
abstract investigations.

In the afternoon the appearance of the common had altered very
much. The early editions of the evening papers had startled London
with enormous headlines:

\headline{``A MESSAGE RECEIVED FROM MARS.''}
\headline{``REMARKABLE STORY FROM WOKING,''}
and so forth. In addition, Ogilvy's wire to the Astronomical
Exchange had roused every observatory in the three kingdoms.

There were half a dozen flies or more from the Woking station
standing in the road by the sand pits, a basket-chaise from
Chobham, and a rather lordly carriage. Besides that, there was
quite a heap of bicycles. In addition, a large number of people
must have walked, in spite of the heat of the day, from Woking and
Chertsey, so that there was altogether quite a considerable
crowd\dash{}one or two gaily dressed ladies among the others.

It was glaringly hot, not a cloud in the sky nor a breath of wind,
and the only shadow was that of the few scattered pine trees. The
burning heather had been extinguished, but the level ground towards
Ottershaw was blackened as far as one could see, and still giving
off vertical streamers of smoke. An enterprising sweet-stuff dealer
in the Chobham Road had sent up his son with a barrow-load of green
apples and ginger beer.

Going to the edge of the pit, I found it occupied by a group of
about half a dozen men\dash{}Henderson, Ogilvy, and a tall, fair-haired
man that I afterwards learned was Stent, the Astronomer Royal, with
several workmen wielding spades and pickaxes. Stent was giving
directions in a clear, high-pitched voice. He was standing on the
cylinder, which was now evidently much cooler; his face was crimson
and streaming with perspiration, and something seemed to have
irritated him.

A large portion of the cylinder had been uncovered, though its
lower end was still embedded. As soon as Ogilvy saw me among the
staring crowd on the edge of the pit he called to me to come down,
and asked me if I would mind going over to see Lord Hilton, the
lord of the manor.

The growing crowd, he said, was becoming a serious impediment to
their excavations, especially the boys. They wanted a light railing
put up, and help to keep the people back. He told me that a faint
stirring was occasionally still audible within the case, but that
the workmen had failed to unscrew the top, as it afforded no grip
to them. The case appeared to be enormously thick, and it was
possible that the faint sounds we heard represented a noisy tumult
in the interior.

I was very glad to do as he asked, and so become one of the
privileged spectators within the contemplated enclosure. I failed
to find Lord Hilton at his house, but I was told he was expected
from London by the six o'clock train from Waterloo; and as it was
then about a quarter past five, I went home, had some tea, and
walked up to the station to waylay him.

\Chapter{CHAPTER FOUR\\THE CYLINDER OPENS}
When I returned to the common the sun was setting. Scattered groups
were hurrying from the direction of Woking, and one or two persons
were returning. The crowd about the pit had increased, and stood
out black against the lemon yellow of the sky\dash{}a couple of hundred
people, perhaps. There were raised voices, and some sort of
struggle appeared to be going on about the pit. Strange imaginings
passed through my mind. As I drew nearer I heard Stent's voice:

``Keep back! Keep back!''

A boy came running towards me.

``It's a-movin','' he said to me as he passed; ``a-screwin' and
a-screwin' out. I don't like it. I'm a-goin' 'ome, I am.''

I went on to the crowd. There were really, I should think, two or
three hundred people elbowing and jostling one another, the one or
two ladies there being by no means the least active.

``He's fallen in the pit!'' cried some one.

``Keep back!'' said several.

The crowd swayed a little, and I elbowed my way through. Every one
seemed greatly excited. I heard a peculiar humming sound from the
pit.

``I say!'' said Ogilvy; ``help keep these idiots back. We don't know
what's in the confounded thing, you know!''

I saw a young man, a shop assistant in Woking I believe he was,
standing on the cylinder and trying to scramble out of the hole
again. The crowd had pushed him in.

The end of the cylinder was being screwed out from within. Nearly
two feet of shining screw projected. Somebody blundered against me,
and I narrowly missed being pitched onto the top of the screw. I
turned, and as I did so the screw must have come out, for the lid
of the cylinder fell upon the gravel with a ringing concussion. I
stuck my elbow into the person behind me, and turned my head
towards the Thing again. For a moment that circular cavity seemed
perfectly black. I had the sunset in my eyes.

I think everyone expected to see a man emerge\dash{}possibly something a
little unlike us terrestrial men, but in all essentials a man. I
know I did. But, looking, I presently saw something stirring within
the shadow: greyish billowy movements, one above another, and then
two luminous disks\dash{}like eyes. Then something resembling a little
grey snake, about the thickness of a walking stick, coiled up out
of the writhing middle, and wriggled in the air towards me\dash{}and
then another.

A sudden chill came over me. There was a loud shriek from a woman
behind. I half turned, keeping my eyes fixed upon the cylinder
still, from which other tentacles were now projecting, and began
pushing my way back from the edge of the pit. I saw astonishment
giving place to horror on the faces of the people about me. I heard
inarticulate exclamations on all sides. There was a general
movement backwards. I saw the shopman struggling still on the edge
of the pit. I found myself alone, and saw the people on the other
side of the pit running off, Stent among them. I looked again at
the cylinder, and ungovernable terror gripped me. I stood petrified
and staring.

A big greyish rounded bulk, the size, perhaps, of a bear, was
rising slowly and painfully out of the cylinder. As it bulged up
and caught the light, it glistened like wet leather.

Two large dark-coloured eyes were regarding me steadfastly. The
mass that framed them, the head of the thing, was rounded, and had,
one might say, a face. There was a mouth under the eyes, the
lipless brim of which quivered and panted, and dropped saliva. The
whole creature heaved and pulsated convulsively. A lank tentacular
appendage gripped the edge of the cylinder, another swayed in the
air.

Those who have never seen a living Martian can scarcely imagine the
strange horror of its appearance. The peculiar V-shaped mouth with
its pointed upper lip, the absence of brow ridges, the absence of a
chin beneath the wedgelike lower lip, the incessant quivering of
this mouth, the Gorgon groups of tentacles, the tumultuous
breathing of the lungs in a strange atmosphere, the evident
heaviness and painfulness of movement due to the greater
gravitational energy of the earth\dash{}above all, the extraordinary
intensity of the immense eyes\dash{}were at once vital, intense,
inhuman, crippled and monstrous. There was something fungoid in the
oily brown skin, something in the clumsy deliberation of the
tedious movements unspeakably nasty. Even at this first encounter,
this first glimpse, I was overcome with disgust and dread.

Suddenly the monster vanished. It had toppled over the brim of the
cylinder and fallen into the pit, with a thud like the fall of a
great mass of leather. I heard it give a peculiar thick cry, and
forthwith another of these creatures appeared darkly in the deep
shadow of the aperture.

I turned and, running madly, made for the first group of trees,
perhaps a hundred yards away; but I ran slantingly and stumbling,
for I could not avert my face from these things.

There, among some young pine trees and furze bushes, I stopped,
panting, and waited further developments. The common round the sand
pits was dotted with people, standing like myself in a
half-fascinated terror, staring at these creatures, or rather at
the heaped gravel at the edge of the pit in which they lay. And
then, with a renewed horror, I saw a round, black object bobbing up
and down on the edge of the pit. It was the head of the shopman who
had fallen in, but showing as a little black object against the hot
western sun. Now he got his shoulder and knee up, and again he
seemed to slip back until only his head was visible. Suddenly he
vanished, and I could have fancied a faint shriek had reached me. I
had a momentary impulse to go back and help him that my fears
overruled.

Everything was then quite invisible, hidden by the deep pit and the
heap of sand that the fall of the cylinder had made. Anyone coming
along the road from Chobham or Woking would have been amazed at the
sight\dash{}a dwindling multitude of perhaps a hundred people or more
standing in a great irregular circle, in ditches, behind bushes,
behind gates and hedges, saying little to one another and that in
short, excited shouts, and staring, staring hard at a few heaps of
sand. The barrow of ginger beer stood, a queer derelict, black
against the burning sky, and in the sand pits was a row of deserted
vehicles with their horses feeding out of nosebags or pawing the
ground.

\Chapter{CHAPTER FIVE\\THE HEAT-RAY}
After the glimpse I had had of the Martians emerging from the
cylinder in which they had come to the earth from their planet, a
kind of fascination paralysed my actions. I remained standing
knee-deep in the heather, staring at the mound that hid them. I was
a battleground of fear and curiosity.

I did not dare to go back towards the pit, but I felt a passionate
longing to peer into it. I began walking, therefore, in a big
curve, seeking some point of vantage and continually looking at the
sand heaps that hid these new-comers to our earth. Once a leash of
thin black whips, like the arms of an octopus, flashed across the
sunset and was immediately withdrawn, and afterwards a thin rod
rose up, joint by joint, bearing at its apex a circular disk that
spun with a wobbling motion. What could be going on there?

Most of the spectators had gathered in one or two groups\dash{}one a
little crowd towards Woking, the other a knot of people in the
direction of Chobham. Evidently they shared my mental conflict.
There were few near me. One man I approached\dash{}he was, I perceived,
a neighbour of mine, though I did not know his name\dash{}and accosted.
But it was scarcely a time for articulate conversation.

``What ugly \emph{brutes}!'' he said. ``Good God! What ugly brutes!''
He repeated this over and over again.

``Did you see a man in the pit?'' I said; but he made no answer to
that. We became silent, and stood watching for a time side by side,
deriving, I fancy, a certain comfort in one another's company. Then
I shifted my position to a little knoll that gave me the advantage
of a yard or more of elevation and when I looked for him presently
he was walking towards Woking.

The sunset faded to twilight before anything further happened. The
crowd far away on the left, towards Woking, seemed to grow, and I
heard now a faint murmur from it. The little knot of people towards
Chobham dispersed. There was scarcely an intimation of movement
from the pit.

It was this, as much as anything, that gave people courage, and I
suppose the new arrivals from Woking also helped to restore
confidence. At any rate, as the dusk came on a slow, intermittent
movement upon the sand pits began, a movement that seemed to gather
force as the stillness of the evening about the cylinder remained
unbroken. Vertical black figures in twos and threes would advance,
stop, watch, and advance again, spreading out as they did so in a
thin irregular crescent that promised to enclose the pit in its
attenuated horns. I, too, on my side began to move towards the
pit.

Then I saw some cabmen and others had walked boldly into the sand
pits, and heard the clatter of hoofs and the gride of wheels. I saw
a lad trundling off the barrow of apples. And then, within thirty
yards of the pit, advancing from the direction of Horsell, I noted
a little black knot of men, the foremost of whom was waving a white
flag.

This was the Deputation. There had been a hasty consultation, and
since the Martians were evidently, in spite of their repulsive
forms, intelligent creatures, it had been resolved to show them, by
approaching them with signals, that we too were intelligent.

Flutter, flutter, went the flag, first to the right, then to the
left. It was too far for me to recognise anyone there, but
afterwards I learned that Ogilvy, Stent, and Henderson were with
others in this attempt at communication. This little group had in
its advance dragged inward, so to speak, the circumference of the
now almost complete circle of people, and a number of dim black
figures followed it at discreet distances.

Suddenly there was a flash of light, and a quantity of luminous
greenish smoke came out of the pit in three distinct puffs, which
drove up, one after the other, straight into the still air.

This smoke (or flame, perhaps, would be the better word for it) was
so bright that the deep blue sky overhead and the hazy stretches of
brown common towards Chertsey, set with black pine trees, seemed to
darken abruptly as these puffs arose, and to remain the darker
after their dispersal. At the same time a faint hissing sound
became audible.

Beyond the pit stood the little wedge of people with the white flag
at its apex, arrested by these phenomena, a little knot of small
vertical black shapes upon the black ground. As the green smoke
arose, their faces flashed out pallid green, and faded again as it
vanished. Then slowly the hissing passed into a humming, into a
long, loud, droning noise. Slowly a humped shape rose out of the
pit, and the ghost of a beam of light seemed to flicker out from
it.

Forthwith flashes of actual flame, a bright glare leaping from one
to another, sprang from the scattered group of men. It was as if
some invisible jet impinged upon them and flashed into white flame.
It was as if each man were suddenly and momentarily turned to
fire.

Then, by the light of their own destruction, I saw them staggering
and falling, and their supporters turning to run.

I stood staring, not as yet realising that this was death leaping
from man to man in that little distant crowd. All I felt was that
it was something very strange. An almost noiseless and blinding
flash of light, and a man fell headlong and lay still; and as the
unseen shaft of heat passed over them, pine trees burst into fire,
and every dry furze bush became with one dull thud a mass of
flames. And far away towards Knaphill I saw the flashes of trees
and hedges and wooden buildings suddenly set alight.

It was sweeping round swiftly and steadily, this flaming death,
this invisible, inevitable sword of heat. I perceived it coming
towards me by the flashing bushes it touched, and was too astounded
and stupefied to stir. I heard the crackle of fire in the sand pits
and the sudden squeal of a horse that was as suddenly stilled. Then
it was as if an invisible yet intensely heated finger were drawn
through the heather between me and the Martians, and all along a
curving line beyond the sand pits the dark ground smoked and
crackled. Something fell with a crash far away to the left where
the road from Woking station opens out on the common. Forth-with
the hissing and humming ceased, and the black, dome-like object
sank slowly out of sight into the pit.

All this had happened with such swiftness that I had stood
motionless, dumbfounded and dazzled by the flashes of light. Had
that death swept through a full circle, it must inevitably have
slain me in my surprise. But it passed and spared me, and left the
night about me suddenly dark and unfamiliar.

The undulating common seemed now dark almost to blackness, except
where its roadways lay grey and pale under the deep blue sky of the
early night. It was dark, and suddenly void of men. Overhead the
stars were mustering, and in the west the sky was still a pale,
bright, almost greenish blue. The tops of the pine trees and the
roofs of Horsell came out sharp and black against the western
afterglow. The Martians and their appliances were altogether
invisible, save for that thin mast upon which their restless mirror
wobbled. Patches of bush and isolated trees here and there smoked
and glowed still, and the houses towards Woking station were
sending up spires of flame into the stillness of the evening air.

Nothing was changed save for that and a terrible astonishment. The
little group of black specks with the flag of white had been swept
out of existence, and the stillness of the evening, so it seemed to
me, had scarcely been broken.

It came to me that I was upon this dark common, helpless,
unprotected, and alone. Suddenly, like a thing falling upon me from
without, came\dash{}fear.

With an effort I turned and began a stumbling run through the
heather.

The fear I felt was no rational fear, but a panic terror not only
of the Martians, but of the dusk and stillness all about me. Such
an extraordinary effect in unmanning me it had that I ran weeping
silently as a child might do. Once I had turned, I did not dare to
look back.

I remember I felt an extraordinary persuasion that I was being
played with, that presently, when I was upon the very verge of
safety, this mysterious death\dash{}as swift as the passage of
light\dash{}would leap after me from the pit about the cylinder and
strike me down.

\Chapter{CHAPTER SIX\\THE HEAT-RAY IN THE CHOBHAM
ROAD}
It is still a matter of wonder how the Martians are able to slay
men so swiftly and so silently. Many think that in some way they
are able to generate an intense heat in a chamber of practically
absolute non-conductivity. This intense heat they project in a
parallel beam against any object they choose, by means of a
polished parabolic mirror of unknown composition, much as the
parabolic mirror of a lighthouse projects a beam of light. But no
one has absolutely proved these details. However it is done, it is
certain that a beam of heat is the essence of the matter. Heat, and
invisible, instead of visible, light. Whatever is combustible
flashes into flame at its touch, lead runs like water, it softens
iron, cracks and melts glass, and when it falls upon water,
incontinently that explodes into steam.

That night nearly forty people lay under the starlight about the
pit, charred and distorted beyond recognition, and all night long
the common from Horsell to Maybury was deserted and brightly
ablaze.

The news of the massacre probably reached Chobham, Woking, and
Ottershaw about the same time. In Woking the shops had closed when
the tragedy happened, and a number of people, shop people and so
forth, attracted by the stories they had heard, were walking over
the Horsell Bridge and along the road between the hedges that runs
out at last upon the common. You may imagine the young people
brushed up after the labours of the day, and making this novelty,
as they would make any novelty, the excuse for walking together and
enjoying a trivial flirtation. You may figure to yourself the hum
of voices along the road in the gloaming. \ldots{}

As yet, of course, few people in Woking even knew that the cylinder
had opened, though poor Henderson had sent a messenger on a bicycle
to the post office with a special wire to an evening paper.

As these folks came out by twos and threes upon the open, they
found little knots of people talking excitedly and peering at the
spinning mirror over the sand pits, and the newcomers were, no
doubt, soon infected by the excitement of the occasion.

By half past eight, when the Deputation was destroyed, there may
have been a crowd of three hundred people or more at this place,
besides those who had left the road to approach the Martians
nearer. There were three policemen too, one of whom was mounted,
doing their best, under instructions from Stent, to keep the people
back and deter them from approaching the cylinder. There was some
booing from those more thoughtless and excitable souls to whom a
crowd is always an occasion for noise and horse-play.

Stent and Ogilvy, anticipating some possibilities of a collision,
had telegraphed from Horsell to the barracks as soon as the
Martians emerged, for the help of a company of soldiers to protect
these strange creatures from violence. After that they returned to
lead that ill-fated advance. The description of their death, as it
was seen by the crowd, tallies very closely with my own
impressions: the three puffs of green smoke, the deep humming note,
and the flashes of flame.

But that crowd of people had a far narrower escape than mine. Only
the fact that a hummock of heathery sand intercepted the lower part
of the Heat-Ray saved them. Had the elevation of the parabolic
mirror been a few yards higher, none could have lived to tell the
tale. They saw the flashes and the men falling and an invisible
hand, as it were, lit the bushes as it hurried towards them through
the twilight. Then, with a whistling note that rose above the
droning of the pit, the beam swung close over their heads, lighting
the tops of the beech trees that line the road, and splitting the
bricks, smashing the windows, firing the window frames, and
bringing down in crumbling ruin a portion of the gable of the house
nearest the corner.

In the sudden thud, hiss, and glare of the igniting trees, the
panic-stricken crowd seems to have swayed hesitatingly for some
moments. Sparks and burning twigs began to fall into the road, and
single leaves like puffs of flame. Hats and dresses caught fire.
Then came a crying from the common. There were shrieks and shouts,
and suddenly a mounted policeman came galloping through the
confusion with his hands clasped over his head, screaming.

``They're coming!'' a woman shrieked, and incontinently everyone was
turning and pushing at those behind, in order to clear their way to
Woking again. They must have bolted as blindly as a flock of sheep.
Where the road grows narrow and black between the high banks the
crowd jammed, and a desperate struggle occurred. All that crowd did
not escape; three persons at least, two women and a little boy,
were crushed and trampled there, and left to die amid the terror
and the darkness.

\Chapter{CHAPTER SEVEN\\HOW I REACHED HOME}
For my own part, I remember nothing of my flight except the stress
of blundering against trees and stumbling through the heather. All
about me gathered the invisible terrors of the Martians; that
pitiless sword of heat seemed whirling to and fro, flourishing
overhead before it descended and smote me out of life. I came into
the road between the crossroads and Horsell, and ran along this to
the crossroads.

At last I could go no further; I was exhausted with the violence of
my emotion and of my flight, and I staggered and fell by the
wayside. That was near the bridge that crosses the canal by the
gasworks. I fell and lay still.

I must have remained there some time.

I sat up, strangely perplexed. For a moment, perhaps, I could not
clearly understand how I came there. My terror had fallen from me
like a garment. My hat had gone, and my collar had burst away from
its fastener. A few minutes before, there had only been three real
things before me\dash{}the immensity of the night and space and nature,
my own feebleness and anguish, and the near approach of death. Now
it was as if something turned over, and the point of view altered
abruptly. There was no sensible transition from one state of mind
to the other. I was immediately the self of every day again\dash{}a
decent, ordinary citizen. The silent common, the impulse of my
flight, the starting flames, were as if they had been in a dream. I
asked myself had these latter things indeed happened? I could not
credit it.

I rose and walked unsteadily up the steep incline of the bridge. My
mind was blank wonder. My muscles and nerves seemed drained of
their strength. I dare say I staggered drunkenly. A head rose over
the arch, and the figure of a workman carrying a basket appeared.
Beside him ran a little boy. He passed me, wishing me good night. I
was minded to speak to him, but did not. I answered his greeting
with a meaningless mumble and went on over the bridge.

Over the Maybury arch a train, a billowing tumult of white, firelit
smoke, and a long caterpillar of lighted windows, went flying
south\dash{}clatter, clatter, clap, rap, and it had gone. A dim group of
people talked in the gate of one of the houses in the pretty little
row of gables that was called Oriental Terrace. It was all so real
and so familiar. And that behind me! It was frantic, fantastic!
Such things, I told myself, could not be.

Perhaps I am a man of exceptional moods. I do not know how far my
experience is common. At times I suffer from the strangest sense of
detachment from myself and the world about me; I seem to watch it
all from the outside, from somewhere inconceivably remote, out of
time, out of space, out of the stress and tragedy of it all. This
feeling was very strong upon me that night. Here was another side
to my dream.

But the trouble was the blank incongruity of this serenity and the
swift death flying yonder, not two miles away. There was a noise of
business from the gasworks, and the electric lamps were all alight.
I stopped at the group of people.

``What news from the common?'' said I.

There were two men and a woman at the gate.

``Eh?'' said one of the men, turning.

``What news from the common?'' I said.

``?Ain't yer just \emph{been} there?'' asked the men.

``People seem fair silly about the common,'' said the woman over the
gate. ``What's it all abart?''

``Haven't you heard of the men from Mars?'' said I; ``the creatures
from Mars?''

``Quite enough,'' said the woman over the gate. ``Thenks''; and all
three of them laughed.

I felt foolish and angry. I tried and found I could not tell them
what I had seen. They laughed again at my broken sentences.

``You'll hear more yet,'' I said, and went on to my home.

I startled my wife at the doorway, so haggard was I. I went into
the dining room, sat down, drank some wine, and so soon as I could
collect myself sufficiently I told her the things I had seen. The
dinner, which was a cold one, had already been served, and remained
neglected on the table while I told my story.

``There is one thing,'' I said, to allay the fears I had aroused;
``they are the most sluggish things I ever saw crawl. They may keep
the pit and kill people who come near them, but they cannot get out
of it. \ldots{} But the horror of them!''

``Don't, dear!'' said my wife, knitting her brows and putting her
hand on mine.

``Poor Ogilvy!'' I said. ``To think he may be lying dead there!''

My wife at least did not find my experience incredible. When I saw
how deadly white her face was, I ceased abruptly.

``They may come here,'' she said again and again.

I pressed her to take wine, and tried to reassure her.

``They can scarcely move,'' I said.

I began to comfort her and myself by repeating all that Ogilvy had
told me of the impossibility of the Martians establishing
themselves on the earth. In particular I laid stress on the
gravitational difficulty. On the surface of the earth the force of
gravity is three times what it is on the surface of Mars. A
Martian, therefore, would weigh three times more than on Mars,
albeit his muscular strength would be the same. His own body would
be a cope of lead to him. That, indeed, was the general opinion.
Both \emph{The Times} and the \emph{Daily Telegraph}, for instance,
insisted on it the next morning, and both overlooked, just as I
did, two obvious modifying influences.

The atmosphere of the earth, we now know, contains far more oxygen
or far less argon (whichever way one likes to put it) than does
Mars. The invigorating influences of this excess of oxygen upon the
Martians indisputably did much to counterbalance the increased
weight of their bodies. And, in the second place, we all overlooked
the fact that such mechanical intelligence as the Martian possessed
was quite able to dispense with muscular exertion at a pinch.

But I did not consider these points at the time, and so my
reasoning was dead against the chances of the invaders. With wine
and food, the confidence of my own table, and the necessity of
reassuring my wife, I grew by insensible degrees courageous and
secure.

``They have done a foolish thing,'' said I, fingering my wineglass.
``They are dangerous because, no doubt, they are mad with terror.
Perhaps they expected to find no living things\dash{}certainly no
intelligent living things.''

``A shell in the pit'' said I, ``if the worst comes to the worst will
kill them all.''

The intense excitement of the events had no doubt left my
perceptive powers in a state of erethism. I remember that dinner
table with extraordinary vividness even now. My dear wife's sweet
anxious face peering at me from under the pink lamp shade, the
white cloth with its silver and glass table furniture\dash{}for in those
days even philosophical writers had many little luxuries\dash{}the
crimson-purple wine in my glass, are photographically distinct. At
the end of it I sat, tempering nuts with a cigarette, regretting
Ogilvy's rashness, and denouncing the shortsighted timidity of the
Martians.

So some respectable dodo in the Mauritius might have lorded it in
his nest, and discussed the arrival of that shipful of pitiless
sailors in want of animal food. ``We will peck them to death
tomorrow, my dear.''

I did not know it, but that was the last civilised dinner I was to
eat for very many strange and terrible days.

\Chapter{CHAPTER EIGHT\\FRIDAY NIGHT}
The most extraordinary thing to my mind, of all the strange and
wonderful things that happened upon that Friday, was the
dovetailing of the commonplace habits of our social order with the
first beginnings of the series of events that was to topple that
social order headlong. If on Friday night you had taken a pair of
compasses and drawn a circle with a radius of five miles round the
Woking sand pits, I doubt if you would have had one human being
outside it, unless it were some relation of Stent or of the three
or four cyclists or London people lying dead on the common, whose
emotions or habits were at all affected by the new-comers. Many
people had heard of the cylinder, of course, and talked about it in
their leisure, but it certainly did not make the sensation that an
ultimatum to Germany would have done.

In London that night poor Henderson's telegram describing the
gradual unscrewing of the shot was judged to be a canard, and his
evening paper, after wiring for authentication from him and
receiving no reply\dash{}the man was killed\dash{}decided not to print a
special edition.

Even within the five-mile circle the great majority of people were
inert. I have already described the behaviour of the men and women
to whom I spoke. All over the district people were dining and
supping; working men were gardening after the labours of the day,
children were being put to bed, young people were wandering through
the lanes love-making, students sat over their books.

Maybe there was a murmur in the village streets, a novel and
dominant topic in the public-houses, and here and there a
messenger, or even an eye-witness of the later occurrences, caused
a whirl of excitement, a shouting, and a running to and fro; but
for the most part the daily routine of working, eating, drinking,
sleeping, went on as it had done for countless years\dash{}as though no
planet Mars existed in the sky. Even at Woking station and Horsell
and Chobham that was the case.

In Woking junction, until a late hour, trains were stopping and
going on, others were shunting on the sidings, passengers were
alighting and waiting, and everything was proceeding in the most
ordinary way. A boy from the town, trenching on Smith's monopoly,
was selling papers with the afternoon's news. The ringing impact of
trucks, the sharp whistle of the engines from the junction, mingled
with their shouts of ``Men from Mars!'' Excited men came into the
station about nine o'clock with incredible tidings, and caused no
more disturbance than drunkards might have done. People rattling
Londonwards peered into the darkness outside the carriage windows,
and saw only a rare, flickering, vanishing spark dance up from the
direction of Horsell, a red glow and a thin veil of smoke driving
across the stars, and thought that nothing more serious than a
heath fire was happening. It was only round the edge of the common
that any disturbance was perceptible. There were half a dozen
villas burning on the Woking border. There were lights in all the
houses on the common side of the three villages, and the people
there kept awake till dawn.

A curious crowd lingered restlessly, people coming and going but
the crowd remaining, both on the Chobham and Horsell bridges. One
or two adventurous souls, it was afterwards found, went into the
darkness and crawled quite near the Martians; but they never
returned, for now and again a light-ray, like the beam of a
warship's searchlight swept the common, and the Heat-Ray was ready
to follow. Save for such, that big area of common was silent and
desolate, and the charred bodies lay about on it all night under
the stars, and all the next day. A noise of hammering from the pit
was heard by many people.

So you have the state of things on Friday night. In the centre,
sticking into the skin of our old planet Earth like a poisoned
dart, was this cylinder. But the poison was scarcely working yet.
Around it was a patch of silent common, smouldering in places, and
with a few dark, dimly seen objects lying in contorted attitudes
here and there. Here and there was a burning bush or tree. Beyond
was a fringe of excitement, and farther than that fringe the
inflammation had not crept as yet. In the rest of the world the
stream of life still flowed as it had flowed for immemorial years.
The fever of war that would presently clog vein and artery, deaden
nerve and destroy brain, had still to develop.

All night long the Martians were hammering and stirring, sleepless,
indefatigable, at work upon the machines they were making ready,
and ever and again a puff of greenish-white smoke whirled up to the
starlit sky.

About eleven a company of soldiers came through Horsell, and
deployed along the edge of the common to form a cordon. Later a
second company marched through Chobham to deploy on the north side
of the common. Several officers from the Inkerman barracks had been
on the common earlier in the day, and one, Major Eden, was reported
to be missing. The colonel of the regiment came to the Chobham
bridge and was busy questioning the crowd at midnight. The military
authorities were certainly alive to the seriousness of the
business. About eleven, the next morning's papers were able to say,
a squadron of hussars, two Maxims, and about four hundred men of
the Cardigan regiment started from Aldershot.

A few seconds after midnight the crowd in the Chertsey road,
Woking, saw a star fall from heaven into the pine woods to the
northwest. It had a greenish colour, and caused a silent brightness
like summer lightning. This was the second cylinder.

\Chapter{CHAPTER NINE\\THE FIGHTING BEGINS}
Saturday lives in my memory as a day of suspense. It was a day of
lassitude too, hot and close, with, I am told, a rapidly
fluctuating barometer. I had slept but little, though my wife had
succeeded in sleeping, and I rose early. I went into my garden
before breakfast and stood listening, but towards the common there
was nothing stirring but a lark.

The milkman came as usual. I heard the rattle of his chariot and I
went round to the side gate to ask the latest news. He told me that
during the night the Martians had been surrounded by troops, and
that guns were expected. Then\dash{}a familiar, reassuring note\dash{}I heard
a train running towards Woking.

``They aren't to be killed,'' said the milkman, ``if that can possibly
be avoided.''

I saw my neighbour gardening, chatted with him for a time, and then
strolled in to breakfast. It was a most unexceptional morning. My
neighbour was of opinion that the troops would be able to capture
or to destroy the Martians during the day.

``It's a pity they make themselves so unapproachable,'' he said. ``It
would be curious to know how they live on another planet; we might
learn a thing or two.''

He came up to the fence and extended a handful of strawberries, for
his gardening was as generous as it was enthusiastic. At the same
time he told me of the burning of the pine woods about the Byfleet
Golf Links.

``They say,'' said he, ``that there's another of those blessed things
fallen there\dash{}number two. But one's enough, surely. This lot'll
cost the insurance people a pretty penny before everything's
settled.'' He laughed with an air of the greatest good humour as he
said this. The woods, he said, were still burning, and pointed out
a haze of smoke to me. ``They will be hot under foot for days, on
account of the thick soil of pine needles and turf,'' he said, and
then grew serious over ``poor Ogilvy.''

After breakfast, instead of working, I decided to walk down towards
the common. Under the railway bridge I found a group of
soldiers\dash{}sappers, I think, men in small round caps, dirty red
jackets unbuttoned, and showing their blue shirts, dark trousers,
and boots coming to the calf. They told me no one was allowed over
the canal, and, looking along the road towards the bridge, I saw
one of the Cardigan men standing sentinel there. I talked with
these soldiers for a time; I told them of my sight of the Martians
on the previous evening. None of them had seen the Martians, and
they had but the vaguest ideas of them, so that they plied me with
questions. They said that they did not know who had authorised the
movements of the troops; their idea was that a dispute had arisen
at the Horse Guards. The ordinary sapper is a great deal better
educated than the common soldier, and they discussed the peculiar
conditions of the possible fight with some acuteness. I described
the Heat-Ray to them, and they began to argue among themselves.

``Crawl up under cover and rush 'em, say I,'' said one.

``Get aht!'' said another. ``What's cover against this 'ere 'eat?
Sticks to cook yer! What we got to do is to go as near as the
ground'll let us, and then drive a trench.''

``Blow yer trenches! You always want trenches; you ought to ha' been
born a rabbit Snippy.''

``Ain't they got any necks, then?'' said a third, abruptly\dash{}a little,
contemplative, dark man, smoking a pipe.

I repeated my description.

``Octopuses,'' said he, ``that's what I calls 'em. Talk about fishers
of men\dash{}fighters of fish it is this time!''

``It ain't no murder killing beasts like that,'' said the first
speaker.

``Why not shell the darned things strite off and finish 'em?'' said
the little dark man. ``You carn tell what they might do.''

``Where's your shells?'' said the first speaker. ``There ain't no
time. Do it in a rush, that's my tip, and do it at once.''

So they discussed it. After a while I left them, and went on to the
railway station to get as many morning papers as I could.

But I will not weary the reader with a description of that long
morning and of the longer afternoon. I did not succeed in getting a
glimpse of the common, for even Horsell and Chobham church towers
were in the hands of the military authorities. The soldiers I
addressed didn't know anything; the officers were mysterious as
well as busy. I found people in the town quite secure again in the
presence of the military, and I heard for the first time from
Marshall, the tobacconist, that his son was among the dead on the
common. The soldiers had made the people on the outskirts of
Horsell lock up and leave their houses.

I got back to lunch about two, very tired for, as I have said, the
day was extremely hot and dull; and in order to refresh myself I
took a cold bath in the afternoon. About half past four I went up
to the railway station to get an evening paper, for the morning
papers had contained only a very inaccurate description of the
killing of Stent, Henderson, Ogilvy, and the others. But there was
little I didn't know. The Martians did not show an inch of
themselves. They seemed busy in their pit, and there was a sound of
hammering and an almost continuous streamer of smoke. Apparently
they were busy getting ready for a struggle. ``Fresh attempts have
been made to signal, but without success,'' was the stereotyped
formula of the papers. A sapper told me it was done by a man in a
ditch with a flag on a long pole. The Martians took as much notice
of such advances as we should of the lowing of a cow.

I must confess the sight of all this armament, all this
preparation, greatly excited me. My imagination became belligerent,
and defeated the invaders in a dozen striking ways; something of my
schoolboy dreams of battle and heroism came back. It hardly seemed
a fair fight to me at that time. They seemed very helpless in that
pit of theirs.

About three o'clock there began the thud of a gun at measured
intervals from Chertsey or Addlestone. I learned that the
smouldering pine wood into which the second cylinder had fallen was
being shelled, in the hope of destroying that object before it
opened. It was only about five, however, that a field gun reached
Chobham for use against the first body of Martians.

About six in the evening, as I sat at tea with my wife in the
summerhouse talking vigorously about the battle that was lowering
upon us, I heard a muffled detonation from the common, and
immediately after a gust of firing. Close on the heels of that came
a violent rattling crash, quite close to us, that shook the ground;
and, starting out upon the lawn, I saw the tops of the trees about
the Oriental College burst into smoky red flame, and the tower of
the little church beside it slide down into ruin. The pinnacle of
the mosque had vanished, and the roof line of the college itself
looked as if a hundred-ton gun had been at work upon it. One of our
chimneys cracked as if a shot had hit it, flew, and a piece of it
came clattering down the tiles and made a heap of broken red
fragments upon the flower bed by my study window.

I and my wife stood amazed. Then I realised that the crest of
Maybury Hill must be within range of the Martians? Heat-Ray now
that the college was cleared out of the way.

At that I gripped my wife's arm, and without ceremony ran her out
into the road. Then I fetched out the servant, telling her I would
go upstairs myself for the box she was clamouring for.

``We can't possibly stay here,'' I said; and as I spoke the firing
reopened for a moment upon the common.

``But where are we to go?'' said my wife in terror.

I thought perplexed. Then I remembered her cousins at Leatherhead.

``Leatherhead!'' I shouted above the sudden noise.

She looked away from me downhill. The people were coming out of
their houses, astonished.

``How are we to get to Leatherhead?'' she said.

Down the hill I saw a bevy of hussars ride under the railway
bridge; three galloped through the open gates of the Oriental
College; two others dismounted, and began running from house to
house. The sun, shining through the smoke that drove up from the
tops of the trees, seemed blood red, and threw an unfamiliar lurid
light upon everything.

``Stop here,'' said I; ``you are safe here''; and I started off at once
for the Spotted Dog, for I knew the landlord had a horse and dog
cart. I ran, for I perceived that in a moment everyone upon this
side of the hill would be moving. I found him in his bar, quite
unaware of what was going on behind his house. A man stood with his
back to me, talking to him.

``I must have a pound,'' said the landlord, ``and I've no one to drive
it.''

``I'll give you two,'' said I, over the stranger's shoulder.

``What for?''

``And I'll bring it back by midnight,'' I said.

``Lord!'' said the landlord; ``what's the hurry? I'm selling my bit of
a pig. Two pounds, and you bring it back? What's going on now?''

I explained hastily that I had to leave my home, and so secured the
dog cart. At the time it did not seem to me nearly so urgent that
the landlord should leave his. I took care to have the cart there
and then, drove it off down the road, and, leaving it in charge of
my wife and servant, rushed into my house and packed a few
valuables, such plate as we had, and so forth. The beech trees
below the house were burning while I did this, and the palings up
the road glowed red. While I was occupied in this way, one of the
dismounted hussars came running up. He was going from house to
house, warning people to leave. He was going on as I came out of my
front door, lugging my treasures, done up in a tablecloth. I
shouted after him:

``What news?''

He turned, stared, bawled something about ``crawling out in a thing
like a dish cover,'' and ran on to the gate of the house at the
crest. A sudden whirl of black smoke driving across the road hid
him for a moment. I ran to my neighbour's door and rapped to
satisfy myself of what I already knew, that his wife had gone to
London with him and had locked up their house. I went in again,
according to my promise, to get my servant's box, lugged it out,
clapped it beside her on the tail of the dog cart, and then caught
the reins and jumped up into the driver's seat beside my wife. In
another moment we were clear of the smoke and noise, and spanking
down the opposite slope of Maybury Hill towards Old Woking.

In front was a quiet sunny landscape, a wheat field ahead on either
side of the road, and the Maybury Inn with its swinging sign. I saw
the doctor's cart ahead of me. At the bottom of the hill I turned
my head to look at the hillside I was leaving. Thick streamers of
black smoke shot with threads of red fire were driving up into the
still air, and throwing dark shadows upon the green treetops
eastward. The smoke already extended far away to the east and
west\dash{}to the Byfleet pine woods eastward, and to Woking on the
west. The road was dotted with people running towards us. And very
faint now, but very distinct through the hot, quiet air, one heard
the whirr of a machine-gun that was presently stilled, and an
intermittent cracking of rifles. Apparently the Martians were
setting fire to everything within range of their Heat-Ray.

I am not an expert driver, and I had immediately to turn my
attention to the horse. When I looked back again the second hill
had hidden the black smoke. I slashed the horse with the whip, and
gave him a loose rein until Woking and Send lay between us and that
quivering tumult. I overtook and passed the doctor between Woking
and Send.

\Chapter{CHAPTER TEN\\IN THE STORM}
Leatherhead is about twelve miles from Maybury Hill. The scent of
hay was in the air through the lush meadows beyond Pyrford, and the
hedges on either side were sweet and gay with multitudes of
dog-roses. The heavy firing that had broken out while we were
driving down Maybury Hill ceased as abruptly as it began, leaving
the evening very peaceful and still. We got to Leatherhead without
misadventure about nine o'clock, and the horse had an hour's rest
while I took supper with my cousins and commended my wife to their
care.

My wife was curiously silent throughout the drive, and seemed
oppressed with forebodings of evil. I talked to her reassuringly,
pointing out that the Martians were tied to the Pit by sheer
heaviness, and at the utmost could but crawl a little out of it;
but she answered only in monosyllables. Had it not been for my
promise to the innkeeper, she would, I think, have urged me to stay
in Leatherhead that night. Would that I had! Her face, I remember,
was very white as we parted.

For my own part, I had been feverishly excited all day. Something
very like the war fever that occasionally runs through a civilised
community had got into my blood, and in my heart I was not so very
sorry that I had to return to Maybury that night. I was even afraid
that that last fusillade I had heard might mean the extermination
of our invaders from Mars. I can best express my state of mind by
saying that I wanted to be in at the death.

It was nearly eleven when I started to return. The night was
unexpectedly dark; to me, walking out of the lighted passage of my
cousins' house, it seemed indeed black, and it was as hot and close
as the day. Overhead the clouds were driving fast, albeit not a
breath stirred the shrubs about us. My cousins' man lit both lamps.
Happily, I knew the road intimately. My wife stood in the light of
the doorway, and watched me until I jumped up into the dog cart.
Then abruptly she turned and went in, leaving my cousins side by
side wishing me good hap.

I was a little depressed at first with the contagion of my wife's
fears, but very soon my thoughts reverted to the Martians. At that
time I was absolutely in the dark as to the course of the evening's
fighting. I did not know even the circumstances that had
precipitated the conflict. As I came through Ockham (for that was
the way I returned, and not through Send and Old Woking) I saw
along the western horizon a blood-red glow, which as I drew nearer,
crept slowly up the sky. The driving clouds of the gathering
thunderstorm mingled there with masses of black and red smoke.

Ripley Street was deserted, and except for a lighted window or so
the village showed not a sign of life; but I narrowly escaped an
accident at the corner of the road to Pyrford, where a knot of
people stood with their backs to me. They said nothing to me as I
passed. I do not know what they knew of the things happening beyond
the hill, nor do I know if the silent houses I passed on my way
were sleeping securely, or deserted and empty, or harassed and
watching against the terror of the night.

From Ripley until I came through Pyrford I was in the valley of the
Wey, and the red glare was hidden from me. As I ascended the little
hill beyond Pyrford Church the glare came into view again, and the
trees about me shivered with the first intimation of the storm that
was upon me. Then I heard midnight pealing out from Pyrford Church
behind me, and then came the silhouette of Maybury Hill, with its
tree-tops and roofs black and sharp against the red.

Even as I beheld this a lurid green glare lit the road about me and
showed the distant woods towards Addlestone. I felt a tug at the
reins. I saw that the driving clouds had been pierced as it were by
a thread of green fire, suddenly lighting their confusion and
falling into the field to my left. It was the third falling star!

Close on its apparition, and blindingly violet by contrast, danced
out the first lightning of the gathering storm, and the thunder
burst like a rocket overhead. The horse took the bit between his
teeth and bolted.

A moderate incline runs towards the foot of Maybury Hill, and down
this we clattered. Once the lightning had begun, it went on in as
rapid a succession of flashes as I have ever seen. The
thunderclaps, treading one on the heels of another and with a
strange crackling accompaniment, sounded more like the working of a
gigantic electric machine than the usual detonating reverberations.
The flickering light was blinding and confusing, and a thin hail
smote gustily at my face as I drove down the slope.

At first I regarded little but the road before me, and then
abruptly my attention was arrested by something that was moving
rapidly down the opposite slope of Maybury Hill. At first I took it
for the wet roof of a house, but one flash following another showed
it to be in swift rolling movement. It was an elusive vision\dash{}a
moment of bewildering darkness, and then, in a flash like daylight,
the red masses of the Orphanage near the crest of the hill, the
green tops of the pine trees, and this problematical object came
out clear and sharp and bright.

And this Thing I saw! How can I describe it? A monstrous tripod,
higher than many houses, striding over the young pine trees, and
smashing them aside in its career; a walking engine of glittering
metal, striding now across the heather; articulate ropes of steel
dangling from it, and the clattering tumult of its passage mingling
with the riot of the thunder. A flash, and it came out vividly,
heeling over one way with two feet in the air, to vanish and
reappear almost instantly as it seemed, with the next flash, a
hundred yards nearer. Can you imagine a milking stool tilted and
bowled violently along the ground? That was the impression those
instant flashes gave. But instead of a milking stool imagine it a
great body of machinery on a tripod stand.

Then suddenly the trees in the pine wood ahead of me were parted,
as brittle reeds are parted by a man thrusting through them; they
were snapped off and driven headlong, and a second huge tripod
appeared, rushing, as it seemed, headlong towards me. And I was
galloping hard to meet it! At the sight of the second monster my
nerve went altogether. Not stopping to look again, I wrenched the
horse's head hard round to the right and in another moment the dog
cart had heeled over upon the horse; the shafts smashed noisily,
and I was flung sideways and fell heavily into a shallow pool of
water.

I crawled out almost immediately, and crouched, my feet still in
the water, under a clump of furze. The horse lay motionless (his
neck was broken, poor brute!) and by the lightning flashes I saw
the black bulk of the overturned dog cart and the silhouette of the
wheel still spinning slowly. In another moment the colossal
mechanism went striding by me, and passed uphill towards Pyrford.

Seen nearer, the Thing was incredibly strange, for it was no mere
insensate machine driving on its way. Machine it was, with a
ringing metallic pace, and long, flexible, glittering tentacles
(one of which gripped a young pine tree) swinging and rattling
about its strange body. It picked its road as it went striding
along, and the brazen hood that surmounted it moved to and fro with
the inevitable suggestion of a head looking about. Behind the main
body was a huge mass of white metal like a gigantic fisherman's
basket, and puffs of green smoke squirted out from the joints of
the limbs as the monster swept by me. And in an instant it was
gone.

So much I saw then, all vaguely for the flickering of the
lightning, in blinding highlights and dense black shadows.

As it passed it set up an exultant deafening howl that drowned the
thunder\dash{}``Aloo! Aloo!''\dash{}and in another minute it was with its
companion, half a mile away, stooping over something in the field.
I have no doubt this Thing in the field was the third of the ten
cylinders they had fired at us from Mars.

For some minutes I lay there in the rain and darkness watching, by
the intermittent light, these monstrous beings of metal moving
about in the distance over the hedge tops. A thin hail was now
beginning, and as it came and went their figures grew misty and
then flashed into clearness again. Now and then came a gap in the
lightning, and the night swallowed them up.

I was soaked with hail above and puddle water below. It was some
time before my blank astonishment would let me struggle up the bank
to a drier position, or think at all of my imminent peril.

Not far from me was a little one-roomed squatter's hut of wood,
surrounded by a patch of potato garden. I struggled to my feet at
last, and, crouching and making use of every chance of cover, I
made a run for this. I hammered at the door, but I could not make
the people hear (if there were any people inside), and after a time
I desisted, and, availing myself of a ditch for the greater part of
the way, succeeded in crawling, unobserved by these monstrous
machines, into the pine woods towards Maybury.

Under cover of this I pushed on, wet and shivering now, towards my
own house. I walked among the trees trying to find the footpath. It
was very dark indeed in the wood, for the lightning was now
becoming infrequent, and the hail, which was pouring down in a
torrent, fell in columns through the gaps in the heavy foliage.

If I had fully realised the meaning of all the things I had seen I
should have immediately worked my way round through Byfleet to
Street Cobham, and so gone back to rejoin my wife at Leatherhead.
But that night the strangeness of things about me, and my physical
wretchedness, prevented me, for I was bruised, weary, wet to the
skin, deafened and blinded by the storm.

I had a vague idea of going on to my own house, and that was as
much motive as I had. I staggered through the trees, fell into a
ditch and bruised my knees against a plank, and finally splashed
out into the lane that ran down from the College Arms. I say
splashed, for the storm water was sweeping the sand down the hill
in a muddy torrent. There in the darkness a man blundered into me
and sent me reeling back.

He gave a cry of terror, sprang sideways, and rushed on before I
could gather my wits sufficiently to speak to him. So heavy was the
stress of the storm just at this place that I had the hardest task
to win my way up the hill. I went close up to the fence on the left
and worked my way along its palings.

Near the top I stumbled upon something soft, and, by a flash of
lightning, saw between my feet a heap of black broadcloth and a
pair of boots. Before I could distinguish clearly how the man lay,
the flicker of light had passed. I stood over him waiting for the
next flash. When it came, I saw that he was a sturdy man, cheaply
but not shabbily dressed; his head was bent under his body, and he
lay crumpled up close to the fence, as though he had been flung
violently against it.

Overcoming the repugnance natural to one who had never before
touched a dead body, I stooped and turned him over to feel for his
heart. He was quite dead. Apparently his neck had been broken. The
lightning flashed for a third time, and his face leaped upon me. I
sprang to my feet. It was the landlord of the Spotted Dog, whose
conveyance I had taken.

I stepped over him gingerly and pushed on up the hill. I made my
way by the police station and the College Arms towards my own
house. Nothing was burning on the hillside, though from the common
there still came a red glare and a rolling tumult of ruddy smoke
beating up against the drenching hail. So far as I could see by the
flashes, the houses about me were mostly uninjured. By the College
Arms a dark heap lay in the road.

Down the road towards Maybury Bridge there were voices and the
sound of feet, but I had not the courage to shout or to go to them.
I let myself in with my latchkey, closed, locked and bolted the
door, staggered to the foot of the staircase, and sat down. My
imagination was full of those striding metallic monsters, and of
the dead body smashed against the fence.

I crouched at the foot of the staircase with my back to the wall,
shivering violently.

\Chapter{CHAPTER ELEVEN\\AT THE WINDOW}
I have already said that my storms of emotion have a trick of
exhausting themselves. After a time I discovered that I was cold
and wet, and with little pools of water about me on the stair
carpet. I got up almost mechanically, went into the dining room and
drank some whiskey, and then I was moved to change my clothes.

After I had done that I went upstairs to my study, but why I did so
I do not know. The window of my study looks over the trees and the
railway towards Horsell Common. In the hurry of our departure this
window had been left open. The passage was dark, and, by contrast
with the picture the window frame enclosed, the side of the room
seemed impenetrably dark. I stopped short in the doorway.

The thunderstorm had passed. The towers of the Oriental College and
the pine trees about it had gone, and very far away, lit by a vivid
red glare, the common about the sand pits was visible. Across the
light huge black shapes, grotesque and strange, moved busily to and
fro.

It seemed indeed as if the whole country in that direction was on
fire\dash{}a broad hillside set with minute tongues of flame, swaying
and writhing with the gusts of the dying storm, and throwing a red
reflection upon the cloud-scud above. Every now and then a haze of
smoke from some nearer conflagration drove across the window and
hid the Martian shapes. I could not see what they were doing, nor
the clear form of them, nor recognise the black objects they were
busied upon. Neither could I see the nearer fire, though the
reflections of it danced on the wall and ceiling of the study. A
sharp, resinous tang of burning was in the air.

I closed the door noiselessly and crept towards the window. As I
did so, the view opened out until, on the one hand, it reached to
the houses about Woking station, and on the other to the charred
and blackened pine woods of Byfleet. There was a light down below
the hill, on the railway, near the arch, and several of the houses
along the Maybury road and the streets near the station were
glowing ruins. The light upon the railway puzzled me at first;
there were a black heap and a vivid glare, and to the right of that
a row of yellow oblongs. Then I perceived this was a wrecked train,
the fore part smashed and on fire, the hinder carriages still upon
the rails.

Between these three main centres of light\dash{}the houses, the train,
and the burning county towards Chobham\dash{}stretched irregular patches
of dark country, broken here and there by intervals of dimly
glowing and smoking ground. It was the strangest spectacle, that
black expanse set with fire. It reminded me, more than anything
else, of the Potteries at night. At first I could distinguish no
people at all, though I peered intently for them. Later I saw
against the light of Woking station a number of black figures
hurrying one after the other across the line.

And this was the little world in which I had been living securely
for years, this fiery chaos! What had happened in the last seven
hours I still did not know; nor did I know, though I was beginning
to guess, the relation between these mechanical colossi and the
sluggish lumps I had seen disgorged from the cylinder. With a queer
feeling of impersonal interest I turned my desk chair to the
window, sat down, and stared at the blackened country, and
particularly at the three gigantic black things that were going to
and fro in the glare about the sand pits.

They seemed amazingly busy. I began to ask myself what they could
be. Were they intelligent mechanisms? Such a thing I felt was
impossible. Or did a Martian sit within each, ruling, directing,
using, much as a man's brain sits and rules in his body? I began to
compare the things to human machines, to ask myself for the first
time in my life how an ironclad or a steam engine would seem to an
intelligent lower animal.

The storm had left the sky clear, and over the smoke of the burning
land the little fading pinpoint of Mars was dropping into the west,
when a soldier came into my garden. I heard a slight scraping at
the fence, and rousing myself from the lethargy that had fallen
upon me, I looked down and saw him dimly, clambering over the
palings. At the sight of another human being my torpor passed, and
I leaned out of the window eagerly.

``Hist!'' said I, in a whisper.

He stopped astride of the fence in doubt. Then he came over and
across the lawn to the corner of the house. He bent down and
stepped softly.

``Who's there?'' he said, also whispering, standing under the window
and peering up.

``Where are you going?'' I asked.

``God knows.''

``Are you trying to hide?''

``That's it.''

``Come into the house,'' I said.

I went down, unfastened the door, and let him in, and locked the
door again. I could not see his face. He was hatless, and his coat
was unbuttoned.

``My God!'' he said, as I drew him in.

``What has happened?'' I asked.

``What hasn't?'' In the obscurity I could see he made a gesture of
despair. ``They wiped us out\dash{}simply wiped us out,'' he repeated
again and again.

He followed me, almost mechanically, into the dining room.

``Take some whiskey,'' I said, pouring out a stiff dose.

He drank it. Then abruptly he sat down before the table, put his
head on his arms, and began to sob and weep like a little boy, in a
perfect passion of emotion, while I, with a curious forgetfulness
of my own recent despair, stood beside him, wondering.

It was a long time before he could steady his nerves to answer my
questions, and then he answered perplexingly and brokenly. He was a
driver in the artillery, and had only come into action about seven.
At that time firing was going on across the common, and it was said
the first party of Martians were crawling slowly towards their
second cylinder under cover of a metal shield.

Later this shield staggered up on tripod legs and became the first
of the fighting-machines I had seen. The gun he drove had been
unlimbered near Horsell, in order to command the sand pits, and its
arrival it was that had precipitated the action. As the limber
gunners went to the rear, his horse trod in a rabbit hole and came
down, throwing him into a depression of the ground. At the same
moment the gun exploded behind him, the ammunition blew up, there
was fire all about him, and he found himself lying under a heap of
charred dead men and dead horses.

``I lay still,'' he said, ``scared out of my wits, with the fore
quarter of a horse atop of me. We'd been wiped out. And the
smell\dash{}good God! Like burnt meat! I was hurt across the back by the
fall of the horse, and there I had to lie until I felt better. Just
like parade it had been a minute before\dash{}then stumble, bang,
swish!''

``Wiped out!'' he said.

He had hid under the dead horse for a long time, peeping out
furtively across the common. The Cardigan men had tried a rush, in
skirmishing order, at the pit, simply to be swept out of existence.
Then the monster had risen to its feet and had begun to walk
leisurely to and fro across the common among the few fugitives,
with its headlike hood turning about exactly like the head of a
cowled human being. A kind of arm carried a complicated metallic
case, about which green flashes scintillated, and out of the funnel
of this there smoked the Heat-Ray.

In a few minutes there was, so far as the soldier could see, not a
living thing left upon the common, and every bush and tree upon it
that was not already a blackened skeleton was burning. The hussars
had been on the road beyond the curvature of the ground, and he saw
nothing of them. He heard the Martians rattle for a time and then
become still. The giant saved Woking station and its cluster of
houses until the last; then in a moment the Heat-Ray was brought to
bear, and the town became a heap of fiery ruins. Then the Thing
shut off the Heat-Ray, and turning its back upon the artilleryman,
began to waddle away towards the smouldering pine woods that
sheltered the second cylinder. As it did so a second glittering
Titan built itself up out of the pit.

The second monster followed the first, and at that the artilleryman
began to crawl very cautiously across the hot heather ash towards
Horsell. He managed to get alive into the ditch by the side of the
road, and so escaped to Woking. There his story became ejaculatory.
The place was impassable. It seems there were a few people alive
there, frantic for the most part and many burned and scalded. He
was turned aside by the fire, and hid among some almost scorching
heaps of broken wall as one of the Martian giants returned. He saw
this one pursue a man, catch him up in one of its steely tentacles,
and knock his head against the trunk of a pine tree. At last, after
nightfall, the artilleryman made a rush for it and got over the
railway embankment.

Since then he had been skulking along towards Maybury, in the hope
of getting out of danger Londonward. People were hiding in trenches
and cellars, and many of the survivors had made off towards Woking
village and Send. He had been consumed with thirst until he found
one of the water mains near the railway arch smashed, and the water
bubbling out like a spring upon the road.

That was the story I got from him, bit by bit. He grew calmer
telling me and trying to make me see the things he had seen. He had
eaten no food since midday, he told me early in his narrative, and
I found some mutton and bread in the pantry and brought it into the
room. We lit no lamp for fear of attracting the Martians, and ever
and again our hands would touch upon bread or meat. As he talked,
things about us came darkly out of the darkness, and the trampled
bushes and broken rose trees outside the window grew distinct. It
would seem that a number of men or animals had rushed across the
lawn. I began to see his face, blackened and haggard, as no doubt
mine was also.

When we had finished eating we went softly upstairs to my study,
and I looked again out of the open window. In one night the valley
had become a valley of ashes. The fires had dwindled now. Where
flames had been there were now streamers of smoke; but the
countless ruins of shattered and gutted houses and blasted and
blackened trees that the night had hidden stood out now gaunt and
terrible in the pitiless light of dawn. Yet here and there some
object had had the luck to escape\dash{}a white railway signal here, the
end of a greenhouse there, white and fresh amid the wreckage. Never
before in the history of warfare had destruction been so
indiscriminate and so universal. And shining with the growing light
of the east, three of the metallic giants stood about the pit,
their cowls rotating as though they were surveying the desolation
they had made.

It seemed to me that the pit had been enlarged, and ever and again
puffs of vivid green vapour streamed up and out of it towards the
brightening dawn\dash{}streamed up, whirled, broke, and vanished.

Beyond were the pillars of fire about Chobham. They became pillars
of bloodshot smoke at the first touch of day.

\Chapter{CHAPTER TWELVE\\WHAT I SAW OF THE
DESTRUCTION OF WEYBRIDGE AND SHEPPERTON}
As the dawn grew brighter we withdrew from the window from which we
had watched the Martians, and went very quietly downstairs.

The artilleryman agreed with me that the house was no place to stay
in. He proposed, he said, to make his way Londonward, and thence
rejoin his battery\dash{}No. 12, of the Horse Artillery. My plan was to
return at once to Leatherhead; and so greatly had the strength of
the Martians impressed me that I had determined to take my wife to
Newhaven, and go with her out of the country forthwith. For I
already perceived clearly that the country about London must
inevitably be the scene of a disastrous struggle before such
creatures as these could be destroyed.

Between us and Leatherhead, however, lay the third cylinder, with
its guarding giants. Had I been alone, I think I should have taken
my chance and struck across country. But the artilleryman dissuaded
me: ``It's no kindness to the right sort of wife,'' he said, ``to make
her a widow''; and in the end I agreed to go with him, under cover
of the woods, northward as far as Street Cobham before I parted
with him. Thence I would make a big detour by Epsom to reach
Leatherhead.

I should have started at once, but my companion had been in active
service and he knew better than that. He made me ransack the house
for a flask, which he filled with whiskey; and we lined every
available pocket with packets of biscuits and slices of meat. Then
we crept out of the house, and ran as quickly as we could down the
ill-made road by which I had come overnight. The houses seemed
deserted. In the road lay a group of three charred bodies close
together, struck dead by the Heat-Ray; and here and there were
things that people had dropped\dash{}a clock, a slipper, a silver spoon,
and the like poor valuables. At the corner turning up towards the
post office a little cart, filled with boxes and furniture, and
horseless, heeled over on a broken wheel. A cash box had been
hastily smashed open and thrown under the debris.

Except the lodge at the Orphanage, which was still on fire, none of
the houses had suffered very greatly here. The Heat-Ray had shaved
the chimney tops and passed. Yet, save ourselves, there did not
seem to be a living soul on Maybury Hill. The majority of the
inhabitants had escaped, I suppose, by way of the Old Woking
road\dash{}the road I had taken when I drove to Leatherhead\dash{}or they had
hidden.

We went down the lane, by the body of the man in black, sodden now
from the overnight hail, and broke into the woods at the foot of
the hill. We pushed through these towards the railway without
meeting a soul. The woods across the line were but the scarred and
blackened ruins of woods; for the most part the trees had fallen,
but a certain proportion still stood, dismal grey stems, with dark
brown foliage instead of green.

On our side the fire had done no more than scorch the nearer trees;
it had failed to secure its footing. In one place the woodmen had
been at work on Saturday; trees, felled and freshly trimmed, lay in
a clearing, with heaps of sawdust by the sawing-machine and its
engine. Hard by was a temporary hut, deserted. There was not a
breath of wind this morning, and everything was strangely still.
Even the birds were hushed, and as we hurried along I and the
artilleryman talked in whispers and looked now and again over our
shoulders. Once or twice we stopped to listen.

After a time we drew near the road, and as we did so we heard the
clatter of hoofs and saw through the tree stems three cavalry
soldiers riding slowly towards Woking. We hailed them, and they
halted while we hurried towards them. It was a lieutenant and a
couple of privates of the 8th Hussars, with a stand like a
theodolite, which the artilleryman told me was a heliograph.

``You are the first men I've seen coming this way this morning,''
said the lieutenant. ``What's brewing?''

His voice and face were eager. The men behind him stared curiously.
The artilleryman jumped down the bank into the road and saluted.

``Gun destroyed last night, sir. Have been hiding. Trying to rejoin
battery, sir. You'll come in sight of the Martians, I expect, about
half a mile along this road.''

``What the dickens are they like?'' asked the lieutenant.

``Giants in armour, sir. Hundred feet high. Three legs and a body
like 'luminium, with a mighty great head in a hood, sir.''

``Get out!'' said the lieutenant. ``What confounded nonsense!''

``You'll see, sir. They carry a kind of box, sir, that shoots fire
and strikes you dead.''

``What d'ye mean\dash{}a gun?''

``No, sir,'' and the artilleryman began a vivid account of the
Heat-Ray. Halfway through, the lieutenant interrupted him and
looked up at me. I was still standing on the bank by the side of
the road.

``It's perfectly true,'' I said.

``Well,'' said the lieutenant, ``I suppose it's my business to see it
too. Look here''\dash{}to the artilleryman\dash{}``we're detailed here clearing
people out of their houses. You'd better go along and report
yourself to Brigadier-General Marvin, and tell him all you know.
He's at Weybridge. Know the way?''

``I do,'' I said; and he turned his horse southward again.

``Half a mile, you say?'' said he.

``At most,'' I answered, and pointed over the treetops southward. He
thanked me and rode on, and we saw them no more.

Farther along we came upon a group of three women and two children
in the road, busy clearing out a labourer's cottage. They had got
hold of a little hand truck, and were piling it up with
unclean-looking bundles and shabby furniture. They were all too
assiduously engaged to talk to us as we passed.

By Byfleet station we emerged from the pine trees, and found the
country calm and peaceful under the morning sunlight. We were far
beyond the range of the Heat-Ray there, and had it not been for the
silent desertion of some of the houses, the stirring movement of
packing in others, and the knot of soldiers standing on the bridge
over the railway and staring down the line towards Woking, the day
would have seemed very like any other Sunday.

Several farm waggons and carts were moving creakily along the road
to Addlestone, and suddenly through the gate of a field we saw,
across a stretch of flat meadow, six twelve-pounders standing
neatly at equal distances pointing towards Woking. The gunners
stood by the guns waiting, and the ammunition waggons were at a
business-like distance. The men stood almost as if under
inspection.

``That's good!'' said I. ``They will get one fair shot, at any rate.''

The artilleryman hesitated at the gate.

``I shall go on,'' he said.

Farther on towards Weybridge, just over the bridge, there were a
number of men in white fatigue jackets throwing up a long rampart,
and more guns behind.

``It's bows and arrows against the lightning, anyhow,'' said the
artilleryman. ``They 'aven't seen that fire-beam yet.''

The officers who were not actively engaged stood and stared over
the treetops southwestward, and the men digging would stop every
now and again to stare in the same direction.

Byfleet was in a tumult; people packing, and a score of hussars,
some of them dismounted, some on horseback, were hunting them
about. Three or four black government waggons, with crosses in
white circles, and an old omnibus, among other vehicles, were being
loaded in the village street. There were scores of people, most of
them sufficiently sabbatical to have assumed their best clothes.
The soldiers were having the greatest difficulty in making them
realise the gravity of their position. We saw one shrivelled old
fellow with a huge box and a score or more of flower pots
containing orchids, angrily expostulating with the corporal who
would leave them behind. I stopped and gripped his arm.

``Do you know what's over there?'' I said, pointing at the pine tops
that hid the Martians.

``Eh?'' said he, turning. ``I was explainin' these is vallyble.''

``Death!'' I shouted. ``Death is coming! Death!'' and leaving him to
digest that if he could, I hurried on after the artillery-man. At
the corner I looked back. The soldier had left him, and he was
still standing by his box, with the pots of orchids on the lid of
it, and staring vaguely over the trees.

No one in Weybridge could tell us where the headquarters were
established; the whole place was in such confusion as I had never
seen in any town before. Carts, carriages everywhere, the most
astonishing miscellany of conveyances and horseflesh. The
respectable inhabitants of the place, men in golf and boating
costumes, wives prettily dressed, were packing, river-side loafers
energetically helping, children excited, and, for the most part,
highly delighted at this astonishing variation of their Sunday
experiences. In the midst of it all the worthy vicar was very
pluckily holding an early celebration, and his bell was jangling
out above the excitement.

I and the artilleryman, seated on the step of the drinking
fountain, made a very passable meal upon what we had brought with
us. Patrols of soldiers\dash{}here no longer hussars, but grenadiers in
white\dash{}were warning people to move now or to take refuge in their
cellars as soon as the firing began. We saw as we crossed the
railway bridge that a growing crowd of people had assembled in and
about the railway station, and the swarming platform was piled with
boxes and packages. The ordinary traffic had been stopped, I
believe, in order to allow of the passage of troops and guns to
Chertsey, and I have heard since that a savage struggle occurred
for places in the special trains that were put on at a later hour.

We remained at Weybridge until midday, and at that hour we found
ourselves at the place near Shepperton Lock where the Wey and
Thames join. Part of the time we spent helping two old women to
pack a little cart. The Wey has a treble mouth, and at this point
boats are to be hired, and there was a ferry across the river. On
the Shepperton side was an inn with a lawn, and beyond that the
tower of Shepperton Church\dash{}it has been replaced by a spire\dash{}rose
above the trees.

Here we found an excited and noisy crowd of fugitives. As yet the
flight had not grown to a panic, but there were already far more
people than all the boats going to and fro could enable to cross.
People came panting along under heavy burdens; one husband and wife
were even carrying a small outhouse door between them, with some of
their household goods piled thereon. One man told us he meant to
try to get away from Shepperton station.

There was a lot of shouting, and one man was even jesting. The idea
people seemed to have here was that the Martians were simply
formidable human beings, who might attack and sack the town, to be
certainly destroyed in the end. Every now and then people would
glance nervously across the Wey, at the meadows towards Chertsey,
but everything over there was still.

Across the Thames, except just where the boats landed, everything
was quiet, in vivid contrast with the Surrey side. The people who
landed there from the boats went tramping off down the lane. The
big ferryboat had just made a journey. Three or four soldiers stood
on the lawn of the inn, staring and jesting at the fugitives,
without offering to help. The inn was closed, as it was now within
prohibited hours.

``What's that?'' cried a boatman, and ``Shut up, you fool!'' said a man
near me to a yelping dog. Then the sound came again, this time from
the direction of Chertsey, a muffled thud\dash{}the sound of a gun.

The fighting was beginning. Almost immediately unseen batteries
across the river to our right, unseen because of the trees, took up
the chorus, firing heavily one after the other. A woman screamed.
Everyone stood arrested by the sudden stir of battle, near us and
yet invisible to us. Nothing was to be seen save flat meadows, cows
feeding unconcernedly for the most part, and silvery pollard
willows motionless in the warm sunlight.

``The sojers'll stop 'em,'' said a woman beside me, doubtfully. A
haziness rose over the treetops.

Then suddenly we saw a rush of smoke far away up the river, a puff
of smoke that jerked up into the air and hung; and forthwith the
ground heaved under foot and a heavy explosion shook the air,
smashing two or three windows in the houses near, and leaving us
astonished.

``Here they are!'' shouted a man in a blue jersey. ``Yonder! D'yer see
them? Yonder!''

Quickly, one after the other, one, two, three, four of the armoured
Martians appeared, far away over the little trees, across the flat
meadows that stretched towards Chertsey, and striding hurriedly
towards the river. Little cowled figures they seemed at first,
going with a rolling motion and as fast as flying birds.

Then, advancing obliquely towards us, came a fifth. Their armoured
bodies glittered in the sun as they swept swiftly forward upon the
guns, growing rapidly larger as they drew nearer. One on the
extreme left, the remotest that is, flourished a huge case high in
the air, and the ghostly, terrible Heat-Ray I had already seen on
Friday night smote towards Chertsey, and struck the town.

At sight of these strange, swift, and terrible creatures the crowd
near the water's edge seemed to me to be for a moment
horror-struck. There was no screaming or shouting, but a silence.
Then a hoarse murmur and a movement of feet\dash{}a splashing from the
water. A man, too frightened to drop the portmanteau he carried on
his shoulder, swung round and sent me staggering with a blow from
the corner of his burden. A woman thrust at me with her hand and
rushed past me. I turned with the rush of the people, but I was not
too terrified for thought. The terrible Heat-Ray was in my mind. To
get under water! That was it!

``Get under water!'' I shouted, unheeded.

I faced about again, and rushed towards the approaching Martian,
rushed right down the gravelly beach and headlong into the water.
Others did the same. A boatload of people putting back came leaping
out as I rushed past. The stones under my feet were muddy and
slippery, and the river was so low that I ran perhaps twenty feet
scarcely waist-deep. Then, as the Martian towered overhead scarcely
a couple of hundred yards away, I flung myself forward under the
surface. The splashes of the people in the boats leaping into the
river sounded like thunderclaps in my ears. People were landing
hastily on both sides of the river. But the Martian machine took no
more notice for the moment of the people running this way and that
than a man would of the confusion of ants in a nest against which
his foot has kicked. When, half suffocated, I raised my head above
water, the Martian's hood pointed at the batteries that were still
firing across the river, and as it advanced it swung loose what
must have been the generator of the Heat-Ray.

In another moment it was on the bank, and in a stride wading
halfway across. The knees of its foremost legs bent at the farther
bank, and in another moment it had raised itself to its full height
again, close to the village of Shepperton. Forthwith the six guns
which, unknown to anyone on the right bank, had been hidden behind
the outskirts of that village, fired simultaneously. The sudden
near concussion, the last close upon the first, made my heart jump.
The monster was already raising the case generating the Heat-Ray as
the first shell burst six yards above the hood.

I gave a cry of astonishment. I saw and thought nothing of the
other four Martian monsters; my attention was riveted upon the
nearer incident. Simultaneously two other shells burst in the air
near the body as the hood twisted round in time to receive, but not
in time to dodge, the fourth shell.

The shell burst clean in the face of the Thing. The hood bulged,
flashed, was whirled off in a dozen tattered fragments of red flesh
and glittering metal.

``Hit!'' shouted I, with something between a scream and a cheer.

I heard answering shouts from the people in the water about me. I
could have leaped out of the water with that momentary exultation.

The decapitated colossus reeled like a drunken giant; but it did
not fall over. It recovered its balance by a miracle, and, no
longer heeding its steps and with the camera that fired the
Heat-Ray now rigidly upheld, it reeled swiftly upon Shepperton. The
living intelligence, the Martian within the hood, was slain and
splashed to the four winds of heaven, and the Thing was now but a
mere intricate device of metal whirling to destruction. It drove
along in a straight line, incapable of guidance. It struck the
tower of Shepperton Church, smashing it down as the impact of a
battering ram might have done, swerved aside, blundered on and
collapsed with tremendous force into the river out of my sight.

A violent explosion shook the air, and a spout of water, steam,
mud, and shattered metal shot far up into the sky. As the camera of
the Heat-Ray hit the water, the latter had immediately flashed into
steam. In another moment a huge wave, like a muddy tidal bore but
almost scaldingly hot, came sweeping round the bend upstream. I saw
people struggling shorewards, and heard their screaming and
shouting faintly above the seething and roar of the Martian's
collapse.

For a moment I heeded nothing of the heat, forgot the patent need
of self-preservation. I splashed through the tumultuous water,
pushing aside a man in black to do so, until I could see round the
bend. Half a dozen deserted boats pitched aimlessly upon the
confusion of the waves. The fallen Martian came into sight
downstream, lying across the river, and for the most part
submerged.

Thick clouds of steam were pouring off the wreckage, and through
the tumultuously whirling wisps I could see, intermittently and
vaguely, the gigantic limbs churning the water and flinging a
splash and spray of mud and froth into the air. The tentacles
swayed and struck like living arms, and, save for the helpless
purposelessness of these movements, it was as if some wounded thing
were struggling for its life amid the waves. Enormous quantities of
a ruddy-brown fluid were spurting up in noisy jets out of the
machine.

My attention was diverted from this death flurry by a furious
yelling, like that of the thing called a siren in our manufacturing
towns. A man, knee-deep near the towing path, shouted inaudibly to
me and pointed. Looking back, I saw the other Martians advancing
with gigantic strides down the riverbank from the direction of
Chertsey. The Shepperton guns spoke this time unavailingly.

At that I ducked at once under water, and, holding my breath until
movement was an agony, blundered painfully ahead under the surface
as long as I could. The water was in a tumult about me, and rapidly
growing hotter.

When for a moment I raised my head to take breath and throw the
hair and water from my eyes, the steam was rising in a whirling
white fog that at first hid the Martians altogether. The noise was
deafening. Then I saw them dimly, colossal figures of grey,
magnified by the mist. They had passed by me, and two were stooping
over the frothing, tumultuous ruins of their comrade.

The third and fourth stood beside him in the water, one perhaps two
hundred yards from me, the other towards Laleham. The generators of
the Heat-Rays waved high, and the hissing beams smote down this way
and that.

The air was full of sound, a deafening and confusing conflict of
noises\dash{}the clangorous din of the Martians, the crash of falling
houses, the thud of trees, fences, sheds flashing into flame, and
the crackling and roaring of fire. Dense black smoke was leaping up
to mingle with the steam from the river, and as the Heat-Ray went
to and fro over Weybridge its impact was marked by flashes of
incandescent white, that gave place at once to a smoky dance of
lurid flames. The nearer houses still stood intact, awaiting their
fate, shadowy, faint and pallid in the steam, with the fire behind
them going to and fro.

For a moment perhaps I stood there, breast-high in the almost
boiling water, dumbfounded at my position, hopeless of escape.
Through the reek I could see the people who had been with me in the
river scrambling out of the water through the reeds, like little
frogs hurrying through grass from the advance of a man, or running
to and fro in utter dismay on the towing path.

Then suddenly the white flashes of the Heat-Ray came leaping
towards me. The houses caved in as they dissolved at its touch, and
darted out flames; the trees changed to fire with a roar. The Ray
flickered up and down the towing path, licking off the people who
ran this way and that, and came down to the water's edge not fifty
yards from where I stood. It swept across the river to Shepperton,
and the water in its track rose in a boiling weal crested with
steam. I turned shoreward.

In another moment the huge wave, well-nigh at the boiling-point had
rushed upon me. I screamed aloud, and scalded, half blinded,
agonised, I staggered through the leaping, hissing water towards
the shore. Had my foot stumbled, it would have been the end. I fell
helplessly, in full sight of the Martians, upon the broad, bare
gravelly spit that runs down to mark the angle of the Wey and
Thames. I expected nothing but death.

I have a dim memory of the foot of a Martian coming down within a
score of yards of my head, driving straight into the loose gravel,
whirling it this way and that and lifting again; of a long
suspense, and then of the four carrying the debris of their comrade
between them, now clear and then presently faint through a veil of
smoke, receding interminably, as it seemed to me, across a vast
space of river and meadow. And then, very slowly, I realised that
by a miracle I had escaped.

\Chapter{CHAPTER THIRTEEN\\HOW I FELL IN WITH THE
CURATE}
After getting this sudden lesson in the power of terrestrial
weapons, the Martians retreated to their original position upon
Horsell Common; and in their haste, and encumbered with the debris
of their smashed companion, they no doubt overlooked many such a
stray and negligible victim as myself. Had they left their comrade
and pushed on forthwith, there was nothing at that time between
them and London but batteries of twelve-pounder guns, and they
would certainly have reached the capital in advance of the tidings
of their approach; as sudden, dreadful, and destructive their
advent would have been as the earthquake that destroyed Lisbon a
century ago.

But they were in no hurry. Cylinder followed cylinder on its
interplanetary flight; every twenty-four hours brought them
reinforcement. And meanwhile the military and naval authorities,
now fully alive to the tremendous power of their antagonists,
worked with furious energy. Every minute a fresh gun came into
position until, before twilight, every copse, every row of suburban
villas on the hilly slopes about Kingston and Richmond, masked an
expectant black muzzle. And through the charred and desolated
area\dash{}perhaps twenty square miles altogether\dash{}that encircled the
Martian encampment on Horsell Common, through charred and ruined
villages among the green trees, through the blackened and smoking
arcades that had been but a day ago pine spinneys, crawled the
devoted scouts with the heliographs that were presently to warn the
gunners of the Martian approach. But the Martians now understood
our command of artillery and the danger of human proximity, and not
a man ventured within a mile of either cylinder, save at the price
of his life.

It would seem that these giants spent the earlier part of the
afternoon in going to and fro, transferring everything from the
second and third cylinders\dash{}the second in Addlestone Golf Links and
the third at Pyrford\dash{}to their original pit on Horsell Common. Over
that, above the blackened heather and ruined buildings that
stretched far and wide, stood one as sentinel, while the rest
abandoned their vast fighting-machines and descended into the pit.
They were hard at work there far into the night, and the towering
pillar of dense green smoke that rose therefrom could be seen from
the hills about Merrow, and even, it is said, from Banstead and
Epsom Downs.

And while the Martians behind me were thus preparing for their next
sally, and in front of me Humanity gathered for the battle, I made
my way with infinite pains and labour from the fire and smoke of
burning Weybridge towards London.

I saw an abandoned boat, very small and remote, drifting
down-stream; and throwing off the most of my sodden clothes, I went
after it, gained it, and so escaped out of that destruction. There
were no oars in the boat, but I contrived to paddle, as well as my
parboiled hands would allow, down the river towards Halliford and
Walton, going very tediously and continually looking behind me, as
you may well understand. I followed the river, because I considered
that the water gave me my best chance of escape should these giants
return.

The hot water from the Martian's overthrow drifted downstream with
me, so that for the best part of a mile I could see little of
either bank. Once, however, I made out a string of black figures
hurrying across the meadows from the direction of Weybridge.
Halliford, it seemed, was deserted, and several of the houses
facing the river were on fire. It was strange to see the place
quite tranquil, quite desolate under the hot blue sky, with the
smoke and little threads of flame going straight up into the heat
of the afternoon. Never before had I seen houses burning without
the accompaniment of an obstructive crowd. A little farther on the
dry reeds up the bank were smoking and glowing, and a line of fire
inland was marching steadily across a late field of hay.

For a long time I drifted, so painful and weary was I after the
violence I had been through, and so intense the heat upon the
water. Then my fears got the better of me again, and I resumed my
paddling. The sun scorched my bare back. At last, as the bridge at
Walton was coming into sight round the bend, my fever and faintness
overcame my fears, and I landed on the Middlesex bank and lay down,
deadly sick, amid the long grass. I suppose the time was then about
four or five o'clock. I got up presently, walked perhaps half a
mile without meeting a soul, and then lay down again in the shadow
of a hedge. I seem to remember talking, wanderingly, to myself
during that last spurt. I was also very thirsty, and bitterly
regretful I had drunk no more water. It is a curious thing that I
felt angry with my wife; I cannot account for it, but my impotent
desire to reach Leatherhead worried me excessively.

I do not clearly remember the arrival of the curate, so that
probably I dozed. I became aware of him as a seated figure in
soot-smudged shirt sleeves, and with his upturned, clean-shaven
face staring at a faint flickering that danced over the sky. The
sky was what is called a mackerel sky\dash{}rows and rows of faint
down-plumes of cloud, just tinted with the midsummer sunset.

I sat up, and at the rustle of my motion he looked at me quickly.

``Have you any water?'' I asked abruptly.

He shook his head.

``You have been asking for water for the last hour,'' he said.

For a moment we were silent, taking stock of each other. I dare say
he found me a strange enough figure, naked, save for my
water-soaked trousers and socks, scalded, and my face and shoulders
blackened by the smoke. His face was a fair weakness, his chin
retreated, and his hair lay in crisp, almost flaxen curls on his
low forehead; his eyes were rather large, pale blue, and blankly
staring. He spoke abruptly, looking vacantly away from me.

``What does it mean?'' he said. ``What do these things mean?''

I stared at him and made no answer.

He extended a thin white hand and spoke in almost a complaining
tone.

``Why are these things permitted? What sins have we done? The
morning service was over, I was walking through the roads to clear
my brain for the afternoon, and then\dash{}fire, earthquake, death! As
if it were Sodom and Gomorrah! All our work undone, all the
work\ldots{} What are these Martians?''

``What are we?'' I answered, clearing my throat.

He gripped his knees and turned to look at me again. For half a
minute, perhaps, he stared silently.

``I was walking through the roads to clear my brain,'' he said. ``And
suddenly\dash{}fire, earthquake, death!''

He relapsed into silence, with his chin now sunken almost to his
knees.

Presently he began waving his hand.

``All the work\dash{}all the Sunday schools\dash{}What have we done\dash{}what has
Weybridge done? Everything gone\dash{}everything destroyed. The church!
We rebuilt it only three years ago. Gone! Swept out of existence!
Why?''

Another pause, and he broke out again like one demented.

``The smoke of her burning goeth up for ever and ever!'' he shouted.

His eyes flamed, and he pointed a lean finger in the direction of
Weybridge.

By this time I was beginning to take his measure. The tremendous
tragedy in which he had been involved\dash{}it was evident he was a
fugitive from Weybridge\dash{}had driven him to the very verge of his
reason.

``Are we far from Sunbury?'' I said, in a matter-of-fact tone.

``What are we to do?'' he asked. ``Are these creatures everywhere? Has
the earth been given over to them?''

``Are we far from Sunbury?''

``Only this morning I officiated at early celebration\ldots{}''

``Things have changed,'' I said, quietly. ``You must keep your head.
There is still hope.''

``Hope!''

``Yes. Plentiful hope\dash{}for all this destruction!''

I began to explain my view of our position. He listened at first,
but as I went on the interest dawning in his eyes gave place to
their former stare, and his regard wandered from me.

``This must be the beginning of the end,'' he said, interrupting me.
``The end! The great and terrible day of the Lord! When men shall
call upon the mountains and the rocks to fall upon them and hide
them\dash{}hide them from the face of Him that sitteth upon the
throne!''

I began to understand the position. I ceased my laboured reasoning,
struggled to my feet, and, standing over him, laid my hand on his
shoulder.

``Be a man!'' said I. ``You are scared out of your wits! What good is
religion if it collapses under calamity? Think of what earthquakes
and floods, wars and volcanoes, have done before to men! Did you
think God had exempted Weybridge? He is not an insurance agent.''

For a time he sat in blank silence.

``But how can we escape?'' he asked, suddenly. ``They are
invulnerable, they are pitiless.''

``Neither the one nor, perhaps, the other,'' I answered. ``And the
mightier they are the more sane and wary should we be. One of them
was killed yonder not three hours ago.''

``Killed!'' he said, staring about him. ``How can God's ministers be
killed?''

``I saw it happen.'' I proceeded to tell him. ``We have chanced to
come in for the thick of it,'' said I, ``and that is all.''

``What is that flicker in the sky?'' he asked abruptly.

I told him it was the heliograph signalling\dash{}that it was the sign
of human help and effort in the sky.

``We are in the midst of it,'' I said, ``quiet as it is. That flicker
in the sky tells of the gathering storm. Yonder, I take it are the
Martians, and Londonward, where those hills rise about Richmond and
Kingston and the trees give cover, earthworks are being thrown up
and guns are being placed. Presently the Martians will be coming
this way again.''

And even as I spoke he sprang to his feet and stopped me by a
gesture.

``Listen!'' he said.

From beyond the low hills across the water came the dull resonance
of distant guns and a remote weird crying. Then everything was
still. A cockchafer came droning over the hedge and past us. High
in the west the crescent moon hung faint and pale above the smoke
of Weybridge and Shepperton and the hot, still splendour of the
sunset.

``We had better follow this path,'' I said, ``northward.''

\Chapter{CHAPTER FOURTEEN\\IN LONDON}
My younger brother was in London when the Martians fell at Woking.
He was a medical student working for an imminent examination, and
he heard nothing of the arrival until Saturday morning. The morning
papers on Saturday contained, in addition to lengthy special
articles on the planet Mars, on life in the planets, and so forth,
a brief and vaguely worded telegram, all the more striking for its
brevity.

The Martians, alarmed by the approach of a crowd, had killed a
number of people with a quick-firing gun, so the story ran. The
telegram concluded with the words: ``Formidable as they seem to be,
the Martians have not moved from the pit into which they have
fallen, and, indeed, seem incapable of doing so. Probably this is
due to the relative strength of the earth's gravitational energy.''
On that last text their leader-writer expanded very comfortingly.

Of course all the students in the crammer's biology class, to which
my brother went that day, were intensely interested, but there were
no signs of any unusual excitement in the streets. The afternoon
papers puffed scraps of news under big headlines. They had nothing
to tell beyond the movements of troops about the common, and the
burning of the pine woods between Woking and Weybridge, until
eight. Then the \emph{St.\ James's Gazette}, in an extra-special
edition, announced the bare fact of the interruption of telegraphic
communication. This was thought to be due to the falling of burning
pine trees across the line. Nothing more of the fighting was known
that night, the night of my drive to Leatherhead and back.

My brother felt no anxiety about us, as he knew from the
description in the papers that the cylinder was a good two miles
from my house. He made up his mind to run down that night to me, in
order, as he says, to see the Things before they were killed. He
dispatched a telegram, which never reached me, about four o'clock,
and spent the evening at a music hall.

In London, also, on Saturday night there was a thunderstorm, and my
brother reached Waterloo in a cab. On the platform from which the
midnight train usually starts he learned, after some waiting, that
an accident prevented trains from reaching Woking that night. The
nature of the accident he could not ascertain; indeed, the railway
authorities did not clearly know at that time. There was very
little excitement in the station, as the officials, failing to
realise that anything further than a breakdown between Byfleet and
Woking junction had occurred, were running the theatre trains which
usually passed through Woking round by Virginia Water or Guildford.
They were busy making the necessary arrangements to alter the route
of the Southampton and Portsmouth Sunday League excursions. A
nocturnal newspaper reporter, mistaking my brother for the traffic
manager, to whom he bears a slight resemblance, waylaid and tried
to interview him. Few people, excepting the railway officials,
connected the breakdown with the Martians.

I have read, in another account of these events, that on Sunday
morning ``all London was electrified by the news from Woking.'' As a
matter of fact, there was nothing to justify that very extravagant
phrase. Plenty of Londoners did not hear of the Martians until the
panic of Monday morning. Those who did took some time to realise
all that the hastily worded telegrams in the Sunday papers
conveyed. The majority of people in London do not read Sunday
papers.

The habit of personal security, moreover, is so deeply fixed in the
Londoner's mind, and startling intelligence so much a matter of
course in the papers, that they could read without any personal
tremors: ``About seven o'clock last night the Martians came out of
the cylinder, and, moving about under an armour of metallic
shields, have completely wrecked Woking station with the adjacent
houses, and massacred an entire battalion of the Cardigan Regiment.
No details are known. Maxims have been absolutely useless against
their armour; the field guns have been disabled by them. Flying
hussars have been galloping into Chertsey. The Martians appear to
be moving slowly towards Chertsey or Windsor. Great anxiety
prevails in West Surrey, and earthworks are being thrown up to
check the advance Londonward.'' That was how the Sunday \emph{Sun}
put it, and a clever and remarkably prompt ``handbook'' article in
the \emph{Referee} compared the affair to a menagerie suddenly let
loose in a village.

No one in London knew positively of the nature of the armoured
Martians, and there was still a fixed idea that these monsters must
be sluggish: ``crawling,'' ``creeping painfully''\dash{}such expressions
occurred in almost all the earlier reports. None of the telegrams
could have been written by an eyewitness of their advance. The
Sunday papers printed separate editions as further news came to
hand, some even in default of it. But there was practically nothing
more to tell people until late in the afternoon, when the
authorities gave the press agencies the news in their possession.
It was stated that the people of Walton and Weybridge, and all the
district were pouring along the roads Londonward, and that was
all.

My brother went to church at the Foundling Hospital in the morning,
still in ignorance of what had happened on the previous night.
There he heard allusions made to the invasion, and a special prayer
for peace. Coming out, he bought a \emph{Referee}. He became
alarmed at the news in this, and went again to Waterloo station to
find out if communication were restored. The omnibuses, carriages,
cyclists, and innumerable people walking in their best clothes
seemed scarcely affected by the strange intelligence that the news
venders were disseminating. People were interested, or, if alarmed,
alarmed only on account of the local residents. At the station he
heard for the first time that the Windsor and Chertsey lines were
now interrupted. The porters told him that several remarkable
telegrams had been received in the morning from Byfleet and
Chertsey stations, but that these had abruptly ceased. My brother
could get very little precise detail out of them.

``There's fighting going on about Weybridge'' was the extent of their
information.

The train service was now very much disorganised. Quite a number of
people who had been expecting friends from places on the
South-Western network were standing about the station. One
grey-headed old gentleman came and abused the South-Western Company
bitterly to my brother. ``It wants showing up,'' he said.

One or two trains came in from Richmond, Putney, and Kingston,
containing people who had gone out for a day's boating and found
the locks closed and a feeling of panic in the air. A man in a blue
and white blazer addressed my brother, full of strange tidings.

``There's hosts of people driving into Kingston in traps and carts
and things, with boxes of valuables and all that,'' he said. ``They
come from Molesey and Weybridge and Walton, and they say there's
been guns heard at Chertsey, heavy firing, and that mounted
soldiers have told them to get off at once because the Martians are
coming. We heard guns firing at Hampton Court station, but we
thought it was thunder. What the dickens does it all mean? The
Martians can't get out of their pit, can they?''

My brother could not tell him.

Afterwards he found that the vague feeling of alarm had spread to
the clients of the underground railway, and that the Sunday
excursionists began to return from all over the South-Western
``lung''\dash{}Barnes, Wimbledon, Richmond Park, Kew, and so forth\dash{}at
unnaturally early hours; but not a soul had anything more than
vague hearsay to tell of. Everyone connected with the terminus
seemed ill-tempered.

About five o'clock the gathering crowd in the station was immensely
excited by the opening of the line of communication, which is
almost invariably closed, between the South-Eastern and the
South-Western stations, and the passage of carriage trucks bearing
huge guns and carriages crammed with soldiers. These were the guns
that were brought up from Woolwich and Chatham to cover Kingston.
There was an exchange of pleasantries: ``You'll get eaten!'' ``We're
the beast-tamers!'' and so forth. A little while after that a squad
of police came into the station and began to clear the public off
the platforms, and my brother went out into the street again.

The church bells were ringing for evensong, and a squad of
Salvation Army lassies came singing down Waterloo Road. On the
bridge a number of loafers were watching a curious brown scum that
came drifting down the stream in patches. The sun was just setting,
and the Clock Tower and the Houses of Parliament rose against one
of the most peaceful skies it is possible to imagine, a sky of
gold, barred with long transverse stripes of reddish-purple cloud.
There was talk of a floating body. One of the men there, a
reservist he said he was, told my brother he had seen the
heliograph flickering in the west.

In Wellington Street my brother met a couple of sturdy roughs who
had just been rushed out of Fleet Street with still-wet newspapers
and staring placards. ``Dreadful catastrophe!'' they bawled one to
the other down Wellington Street. ``Fighting at Weybridge! Full
description! Repulse of the Martians! London in Danger!'' He had to
give threepence for a copy of that paper.

Then it was, and then only, that he realised something of the full
power and terror of these monsters. He learned that they were not
merely a handful of small sluggish creatures, but that they were
minds swaying vast mechanical bodies; and that they could move
swiftly and smite with such power that even the mightiest guns
could not stand against them.

They were described as ``vast spiderlike machines, nearly a hundred
feet high, capable of the speed of an express train, and able to
shoot out a beam of intense heat.'' Masked batteries, chiefly of
field guns, had been planted in the country about Horsell Common,
and especially between the Woking district and London. Five of the
machines had been seen moving towards the Thames, and one, by a
happy chance, had been destroyed. In the other cases the shells had
missed, and the batteries had been at once annihilated by the
Heat-Rays. Heavy losses of soldiers were mentioned, but the tone of
the dispatch was optimistic.

The Martians had been repulsed; they were not invulnerable. They
had retreated to their triangle of cylinders again, in the circle
about Woking. Signallers with heliographs were pushing forward upon
them from all sides. Guns were in rapid transit from Windsor,
Portsmouth, Aldershot, Woolwich\dash{}even from the north; among others,
long wire-guns of ninety-five tons from Woolwich. Altogether one
hundred and sixteen were in position or being hastily placed,
chiefly covering London. Never before in England had there been
such a vast or rapid concentration of military material.

Any further cylinders that fell, it was hoped, could be destroyed
at once by high explosives, which were being rapidly manufactured
and distributed. No doubt, ran the report, the situation was of the
strangest and gravest description, but the public was exhorted to
avoid and discourage panic. No doubt the Martians were strange and
terrible in the extreme, but at the outside there could not be more
than twenty of them against our millions.

The authorities had reason to suppose, from the size of the
cylinders, that at the outside there could not be more than five in
each cylinder\dash{}fifteen altogether. And one at least was disposed
of\dash{}perhaps more. The public would be fairly warned of the approach
of danger, and elaborate measures were being taken for the
protection of the people in the threatened southwestern suburbs.
And so, with reiterated assurances of the safety of London and the
ability of the authorities to cope with the difficulty, this
quasi-proclamation closed.

This was printed in enormous type on paper so fresh that it was
still wet, and there had been no time to add a word of comment. It
was curious, my brother said, to see how ruthlessly the usual
contents of the paper had been hacked and taken out to give this
place.

All down Wellington Street people could be seen fluttering out the
pink sheets and reading, and the Strand was suddenly noisy with the
voices of an army of hawkers following these pioneers. Men came
scrambling off buses to secure copies. Certainly this news excited
people intensely, whatever their previous apathy. The shutters of a
map shop in the Strand were being taken down, my brother said, and
a man in his Sunday raiment, lemon-yellow gloves even, was visible
inside the window hastily fastening maps of Surrey to the glass.

Going on along the Strand to Trafalgar Square, the paper in his
hand, my brother saw some of the fugitives from West Surrey. There
was a man with his wife and two boys and some articles of furniture
in a cart such as greengrocers use. He was driving from the
direction of Westminster Bridge; and close behind him came a hay
waggon with five or six respectable-looking people in it, and some
boxes and bundles. The faces of these people were haggard, and
their entire appearance contrasted conspicuously with the
Sabbath-best appearance of the people on the omnibuses. People in
fashionable clothing peeped at them out of cabs. They stopped at
the Square as if undecided which way to take, and finally turned
eastward along the Strand. Some way behind these came a man in
workday clothes, riding one of those old-fashioned tricycles with a
small front wheel. He was dirty and white in the face.

My brother turned down towards Victoria, and met a number of such
people. He had a vague idea that he might see something of me. He
noticed an unusual number of police regulating the traffic. Some of
the refugees were exchanging news with the people on the omnibuses.
One was professing to have seen the Martians. ``Boilers on stilts, I
tell you, striding along like men.'' Most of them were excited and
animated by their strange experience.

Beyond Victoria the public-houses were doing a lively trade with
these arrivals. At all the street corners groups of people were
reading papers, talking excitedly, or staring at these unusual
Sunday visitors. They seemed to increase as night drew on, until at
last the roads, my brother said, were like Epsom High Street on a
Derby Day. My brother addressed several of these fugitives and got
unsatisfactory answers from most.

None of them could tell him any news of Woking except one man, who
assured him that Woking had been entirely destroyed on the previous
night.

``I come from Byfleet,'' he said; ``man on a bicycle came through the
place in the early morning, and ran from door to door warning us to
come away. Then came soldiers. We went out to look, and there were
clouds of smoke to the south\dash{}nothing but smoke, and not a soul
coming that way. Then we heard the guns at Chertsey, and folks
coming from Weybridge. So I've locked up my house and come on.''

At the time there was a strong feeling in the streets that the
authorities were to blame for their incapacity to dispose of the
invaders without all this inconvenience.

About eight o'clock a noise of heavy firing was distinctly audible
all over the south of London. My brother could not hear it for the
traffic in the main thoroughfares, but by striking through the
quiet back streets to the river he was able to distinguish it quite
plainly.

He walked from Westminster to his apartments near Regent's Park,
about two. He was now very anxious on my account, and disturbed at
the evident magnitude of the trouble. His mind was inclined to run,
even as mine had run on Saturday, on military details. He thought
of all those silent, expectant guns, of the suddenly nomadic
countryside; he tried to imagine ``boilers on stilts'' a hundred feet
high.

There were one or two cartloads of refugees passing along Oxford
Street, and several in the Marylebone Road, but so slowly was the
news spreading that Regent Street and Portland Place were full of
their usual Sunday-night promenaders, albeit they talked in groups,
and along the edge of Regent's Park there were as many silent
couples ``walking out'' together under the scattered gas lamps as
ever there had been. The night was warm and still, and a little
oppressive; the sound of guns continued intermittently, and after
midnight there seemed to be sheet lightning in the south.

He read and re-read the paper, fearing the worst had happened to
me. He was restless, and after supper prowled out again aimlessly.
He returned and tried in vain to divert his attention to his
examination notes. He went to bed a little after midnight, and was
awakened from lurid dreams in the small hours of Monday by the
sound of door knockers, feet running in the street, distant
drumming, and a clamour of bells. Red reflections danced on the
ceiling. For a moment he lay astonished, wondering whether day had
come or the world gone mad. Then he jumped out of bed and ran to
the window.

His room was an attic and as he thrust his head out, up and down
the street there were a dozen echoes to the noise of his window
sash, and heads in every kind of night disarray appeared. Enquiries
were being shouted. ``They are coming!'' bawled a policeman,
hammering at the door; ``the Martians are coming!'' and hurried to
the next door.

The sound of drumming and trumpeting came from the Albany Street
Barracks, and every church within earshot was hard at work killing
sleep with a vehement disorderly tocsin. There was a noise of doors
opening, and window after window in the houses opposite flashed
from darkness into yellow illumination.

Up the street came galloping a closed carriage, bursting abruptly
into noise at the corner, rising to a clattering climax under the
window, and dying away slowly in the distance. Close on the rear of
this came a couple of cabs, the forerunners of a long procession of
flying vehicles, going for the most part to Chalk Farm station,
where the North-Western special trains were loading up, instead of
coming down the gradient into Euston.

For a long time my brother stared out of the window in blank
astonishment, watching the policemen hammering at door after door,
and delivering their incomprehensible message. Then the door behind
him opened, and the man who lodged across the landing came in,
dressed only in shirt, trousers, and slippers, his braces loose
about his waist, his hair disordered from his pillow.

``What the devil is it?'' he asked. ``A fire? What a devil of a row!''

They both craned their heads out of the window, straining to hear
what the policemen were shouting. People were coming out of the
side streets, and standing in groups at the corners talking.

``What the devil is it all about?'' said my brother's fellow lodger.

My brother answered him vaguely and began to dress, running with
each garment to the window in order to miss nothing of the growing
excitement. And presently men selling unnaturally early newspapers
came bawling into the street:

``London in danger of suffocation! The Kingston and Richmond
defences forced! Fearful massacres in the Thames Valley!''

And all about him\dash{}in the rooms below, in the houses on each side
and across the road, and behind in the Park Terraces and in the
hundred other streets of that part of Marylebone, and the
Westbourne Park district and St.\ Pancras, and westward and
northward in Kilburn and St.\ John's Wood and Hampstead, and
eastward in Shoreditch and Highbury and Haggerston and Hoxton, and,
indeed, through all the vastness of London from Ealing to East
Ham\dash{}people were rubbing their eyes, and opening windows to stare
out and ask aimless questions, dressing hastily as the first breath
of the coming storm of Fear blew through the streets. It was the
dawn of the great panic. London, which had gone to bed on Sunday
night oblivious and inert, was awakened, in the small hours of
Monday morning, to a vivid sense of danger.

Unable from his window to learn what was happening, my brother went
down and out into the street, just as the sky between the parapets
of the houses grew pink with the early dawn. The flying people on
foot and in vehicles grew more numerous every moment. ``Black
Smoke!'' he heard people crying, and again ``Black Smoke!'' The
contagion of such a unanimous fear was inevitable. As my brother
hesitated on the door-step, he saw another news vender approaching,
and got a paper forthwith. The man was running away with the rest,
and selling his papers for a shilling each as he ran\dash{}a grotesque
mingling of profit and panic.

And from this paper my brother read that catastrophic dispatch of
the Commander-in-Chief:

``The Martians are able to discharge enormous clouds of a black and
poisonous vapour by means of rockets. They have smothered our
batteries, destroyed Richmond, Kingston, and Wimbledon, and are
advancing slowly towards London, destroying everything on the way.
It is impossible to stop them. There is no safety from the Black
Smoke but in instant flight.''

That was all, but it was enough. The whole population of the great
six-million city was stirring, slipping, running; presently it
would be pouring \emph{en masse} northward.

``Black Smoke!'' the voices cried. ``Fire!''

The bells of the neighbouring church made a jangling tumult, a cart
carelessly driven smashed, amid shrieks and curses, against the
water trough up the street. Sickly yellow lights went to and fro in
the houses, and some of the passing cabs flaunted unextinguished
lamps. And overhead the dawn was growing brighter, clear and steady
and calm.

He heard footsteps running to and fro in the rooms, and up and down
stairs behind him. His landlady came to the door, loosely wrapped
in dressing gown and shawl; her husband followed ejaculating.

As my brother began to realise the import of all these things, he
turned hastily to his own room, put all his available money\dash{}some
ten pounds altogether\dash{}into his pockets, and went out again into
the streets.

\Chapter{CHAPTER FIFTEEN\\WHAT HAD HAPPENED IN
SURREY}
It was while the curate had sat and talked so wildly to me under
the hedge in the flat meadows near Halliford, and while my brother
was watching the fugitives stream over Westminster Bridge, that the
Martians had resumed the offensive. So far as one can ascertain
from the conflicting accounts that have been put forth, the
majority of them remained busied with preparations in the Horsell
pit until nine that night, hurrying on some operation that
disengaged huge volumes of green smoke.

But three certainly came out about eight o'clock and, advancing
slowly and cautiously, made their way through Byfleet and Pyrford
towards Ripley and Weybridge, and so came in sight of the expectant
batteries against the setting sun. These Martians did not advance
in a body, but in a line, each perhaps a mile and a half from his
nearest fellow. They communicated with one another by means of
sirenlike howls, running up and down the scale from one note to
another.

It was this howling and firing of the guns at Ripley and St.\ George's
Hill that we had heard at Upper Halliford. The Ripley
gunners, unseasoned artillery volunteers who ought never to have
been placed in such a position, fired one wild, premature,
ineffectual volley, and bolted on horse and foot through the
deserted village, while the Martian, without using his Heat-Ray,
walked serenely over their guns, stepped gingerly among them,
passed in front of them, and so came unexpectedly upon the guns in
Painshill Park, which he destroyed.

The St.\ George's Hill men, however, were better led or of a better
mettle. Hidden by a pine wood as they were, they seem to have been
quite unsuspected by the Martian nearest to them. They laid their
guns as deliberately as if they had been on parade, and fired at
about a thousand yards' range.

The shells flashed all round him, and he was seen to advance a few
paces, stagger, and go down. Everybody yelled together, and the
guns were reloaded in frantic haste. The overthrown Martian set up
a prolonged ululation, and immediately a second glittering giant,
answering him, appeared over the trees to the south. It would seem
that a leg of the tripod had been smashed by one of the shells. The
whole of the second volley flew wide of the Martian on the ground,
and, simultaneously, both his companions brought their Heat-Rays to
bear on the battery. The ammunition blew up, the pine trees all
about the guns flashed into fire, and only one or two of the men
who were already running over the crest of the hill escaped.

After this it would seem that the three took counsel together and
halted, and the scouts who were watching them report that they
remained absolutely stationary for the next half hour. The Martian
who had been overthrown crawled tediously out of his hood, a small
brown figure, oddly suggestive from that distance of a speck of
blight, and apparently engaged in the repair of his support. About
nine he had finished, for his cowl was then seen above the trees
again.

It was a few minutes past nine that night when these three
sentinels were joined by four other Martians, each carrying a thick
black tube. A similar tube was handed to each of the three, and the
seven proceeded to distribute themselves at equal distances along a
curved line between St.\ George's Hill, Weybridge, and the village
of Send, southwest of Ripley.

A dozen rockets sprang out of the hills before them so soon as they
began to move, and warned the waiting batteries about Ditton and
Esher. At the same time four of their fighting machines, similarly
armed with tubes, crossed the river, and two of them, black against
the western sky, came into sight of myself and the curate as we
hurried wearily and painfully along the road that runs northward
out of Halliford. They moved, as it seemed to us, upon a cloud, for
a milky mist covered the fields and rose to a third of their
height.

At this sight the curate cried faintly in his throat, and began
running; but I knew it was no good running from a Martian, and I
turned aside and crawled through dewy nettles and brambles into the
broad ditch by the side of the road. He looked back, saw what I was
doing, and turned to join me.

The two halted, the nearer to us standing and facing Sunbury, the
remoter being a grey indistinctness towards the evening star, away
towards Staines.

The occasional howling of the Martians had ceased; they took up
their positions in the huge crescent about their cylinders in
absolute silence. It was a crescent with twelve miles between its
horns. Never since the devising of gunpowder was the beginning of a
battle so still. To us and to an observer about Ripley it would
have had precisely the same effect\dash{}the Martians seemed in solitary
possession of the darkling night, lit only as it was by the slender
moon, the stars, the afterglow of the daylight, and the ruddy glare
from St.\ George's Hill and the woods of Painshill.

But facing that crescent everywhere\dash{}at Staines, Hounslow, Ditton,
Esher, Ockham, behind hills and woods south of the river, and
across the flat grass meadows to the north of it, wherever a
cluster of trees or village houses gave sufficient cover\dash{}the guns
were waiting. The signal rockets burst and rained their sparks
through the night and vanished, and the spirit of all those
watching batteries rose to a tense expectation. The Martians had
but to advance into the line of fire, and instantly those
motionless black forms of men, those guns glittering so darkly in
the early night, would explode into a thunderous fury of battle.

No doubt the thought that was uppermost in a thousand of those
vigilant minds, even as it was uppermost in mine, was the
riddle\dash{}how much they understood of us. Did they grasp that we in
our millions were organized, disciplined, working together? Or did
they interpret our spurts of fire, the sudden stinging of our
shells, our steady investment of their encampment, as we should the
furious unanimity of onslaught in a disturbed hive of bees? Did
they dream they might exterminate us? (At that time no one knew
what food they needed.) A hundred such questions struggled together
in my mind as I watched that vast sentinel shape. And in the back
of my mind was the sense of all the huge unknown and hidden forces
Londonward. Had they prepared pitfalls? Were the powder mills at
Hounslow ready as a snare? Would the Londoners have the heart and
courage to make a greater Moscow of their mighty province of
houses?

Then, after an interminable time, as it seemed to us, crouching and
peering through the hedge, came a sound like the distant concussion
of a gun. Another nearer, and then another. And then the Martian
beside us raised his tube on high and discharged it, gunwise, with
a heavy report that made the ground heave. The one towards Staines
answered him. There was no flash, no smoke, simply that loaded
detonation.

I was so excited by these heavy minute-guns following one another
that I so far forgot my personal safety and my scalded hands as to
clamber up into the hedge and stare towards Sunbury. As I did so a
second report followed, and a big projectile hurtled overhead
towards Hounslow. I expected at least to see smoke or fire, or some
such evidence of its work. But all I saw was the deep blue sky
above, with one solitary star, and the white mist spreading wide
and low beneath. And there had been no crash, no answering
explosion. The silence was restored; the minute lengthened to
three.

``What has happened?'' said the curate, standing up beside me.

``Heaven knows!'' said I.

A bat flickered by and vanished. A distant tumult of shouting began
and ceased. I looked again at the Martian, and saw he was now
moving eastward along the riverbank, with a swift, rolling motion.

Every moment I expected the fire of some hidden battery to spring
upon him; but the evening calm was unbroken. The figure of the
Martian grew smaller as he receded, and presently the mist and the
gathering night had swallowed him up. By a common impulse we
clambered higher. Towards Sunbury was a dark appearance, as though
a conical hill had suddenly come into being there, hiding our view
of the farther country; and then, remoter across the river, over
Walton, we saw another such summit. These hill-like forms grew
lower and broader even as we stared.

Moved by a sudden thought, I looked northward, and there I
perceived a third of these cloudy black kopjes had risen.

Everything had suddenly become very still. Far away to the
southeast, marking the quiet, we heard the Martians hooting to one
another, and then the air quivered again with the distant thud of
their guns. But the earthly artillery made no reply.

Now at the time we could not understand these things, but later I
was to learn the meaning of these ominous kopjes that gathered in
the twilight. Each of the Martians, standing in the great crescent
I have described, had discharged, by means of the gunlike tube he
carried, a huge canister over whatever hill, copse, cluster of
houses, or other possible cover for guns, chanced to be in front of
him. Some fired only one of these, some two\dash{}as in the case of the
one we had seen; the one at Ripley is said to have discharged no
fewer than five at that time. These canisters smashed on striking
the ground\dash{}they did not explode\dash{}and incontinently disengaged an
enormous volume of heavy, inky vapour, coiling and pouring upward
in a huge and ebony cumulus cloud, a gaseous hill that sank and
spread itself slowly over the surrounding country. And the touch of
that vapour, the inhaling of its pungent wisps, was death to all
that breathes.

It was heavy, this vapour, heavier than the densest smoke, so that,
after the first tumultuous uprush and outflow of its impact, it
sank down through the air and poured over the ground in a manner
rather liquid than gaseous, abandoning the hills, and streaming
into the valleys and ditches and watercourses even as I have heard
the carbonic-acid gas that pours from volcanic clefts is wont to
do. And where it came upon water some chemical action occurred, and
the surface would be instantly covered with a powdery scum that
sank slowly and made way for more. The scum was absolutely
insoluble, and it is a strange thing, seeing the instant effect of
the gas, that one could drink without hurt the water from which it
had been strained. The vapour did not diffuse as a true gas would
do. It hung together in banks, flowing sluggishly down the slope of
the land and driving reluctantly before the wind, and very slowly
it combined with the mist and moisture of the air, and sank to the
earth in the form of dust. Save that an unknown element giving a
group of four lines in the blue of the spectrum is concerned, we
are still entirely ignorant of the nature of this substance.

Once the tumultuous upheaval of its dispersion was over, the black
smoke clung so closely to the ground, even before its
precipitation, that fifty feet up in the air, on the roofs and
upper stories of high houses and on great trees, there was a chance
of escaping its poison altogether, as was proved even that night at
Street Cobham and Ditton.

The man who escaped at the former place tells a wonderful story of
the strangeness of its coiling flow, and how he looked down from
the church spire and saw the houses of the village rising like
ghosts out of its inky nothingness. For a day and a half he
remained there, weary, starving and sun-scorched, the earth under
the blue sky and against the prospect of the distant hills a
velvet-black expanse, with red roofs, green trees, and, later,
black-veiled shrubs and gates, barns, outhouses, and walls, rising
here and there into the sunlight.

But that was at Street Cobham, where the black vapour was allowed
to remain until it sank of its own accord into the ground. As a
rule the Martians, when it had served its purpose, cleared the air
of it again by wading into it and directing a jet of steam upon
it.

This they did with the vapour banks near us, as we saw in the
starlight from the window of a deserted house at Upper Halliford,
whither we had returned. From there we could see the searchlights
on Richmond Hill and Kingston Hill going to and fro, and about
eleven the windows rattled, and we heard the sound of the huge
siege guns that had been put in position there. These continued
intermittently for the space of a quarter of an hour, sending
chance shots at the invisible Martians at Hampton and Ditton, and
then the pale beams of the electric light vanished, and were
replaced by a bright red glow.

Then the fourth cylinder fell\dash{}a brilliant green meteor\dash{}as I
learned afterwards, in Bushey Park. Before the guns on the Richmond
and Kingston line of hills began, there was a fitful cannonade far
away in the southwest, due, I believe, to guns being fired
haphazard before the black vapour could overwhelm the gunners.

So, setting about it as methodically as men might smoke out a
wasps' nest, the Martians spread this strange stifling vapour over
the Londonward country. The horns of the crescent slowly moved
apart, until at last they formed a line from Hanwell to Coombe and
Malden. All night through their destructive tubes advanced. Never
once, after the Martian at St.\ George's Hill was brought down, did
they give the artillery the ghost of a chance against them.
Wherever there was a possibility of guns being laid for them
unseen, a fresh canister of the black vapour was discharged, and
where the guns were openly displayed the Heat-Ray was brought to
bear.

By midnight the blazing trees along the slopes of Richmond Park and
the glare of Kingston Hill threw their light upon a network of
black smoke, blotting out the whole valley of the Thames and
extending as far as the eye could reach. And through this two
Martians slowly waded, and turned their hissing steam jets this way
and that.

They were sparing of the Heat-Ray that night, either because they
had but a limited supply of material for its production or because
they did not wish to destroy the country but only to crush and
overawe the opposition they had aroused. In the latter aim they
certainly succeeded. Sunday night was the end of the organised
opposition to their movements. After that no body of men would
stand against them, so hopeless was the enterprise. Even the crews
of the torpedo-boats and destroyers that had brought their
quick-firers up the Thames refused to stop, mutinied, and went down
again. The only offensive operation men ventured upon after that
night was the preparation of mines and pitfalls, and even in that
their energies were frantic and spasmodic.

One has to imagine, as well as one may, the fate of those batteries
towards Esher, waiting so tensely in the twilight. Survivors there
were none. One may picture the orderly expectation, the officers
alert and watchful, the gunners ready, the ammunition piled to
hand, the limber gunners with their horses and waggons, the groups
of civilian spectators standing as near as they were permitted, the
evening stillness, the ambulances and hospital tents with the
burned and wounded from Weybridge; then the dull resonance of the
shots the Martians fired, and the clumsy projectile whirling over
the trees and houses and smashing amid the neighbouring fields.

One may picture, too, the sudden shifting of the attention, the
swiftly spreading coils and bellyings of that blackness advancing
headlong, towering heavenward, turning the twilight to a palpable
darkness, a strange and horrible antagonist of vapour striding upon
its victims, men and horses near it seen dimly, running, shrieking,
falling headlong, shouts of dismay, the guns suddenly abandoned,
men choking and writhing on the ground, and the swift
broadening-out of the opaque cone of smoke. And then night and
extinction\dash{}nothing but a silent mass of impenetrable vapour hiding
its dead.

Before dawn the black vapour was pouring through the streets of
Richmond, and the disintegrating organism of government was, with a
last expiring effort, rousing the population of London to the
necessity of flight.

\Chapter{CHAPTER SIXTEEN\\THE EXODUS FROM LONDON}
So you understand the roaring wave of fear that swept through the
greatest city in the world just as Monday was dawning\dash{}the stream
of flight rising swiftly to a torrent, lashing in a foaming tumult
round the railway stations, banked up into a horrible struggle
about the shipping in the Thames, and hurrying by every available
channel northward and eastward. By ten o'clock the police
organisation, and by midday even the railway organisations, were
losing coherency, losing shape and efficiency, guttering,
softening, running at last in that swift liquefaction of the social
body.

All the railway lines north of the Thames and the South-Eastern
people at Cannon Street had been warned by midnight on Sunday, and
trains were being filled. People were fighting savagely for
standing-room in the carriages even at two o'clock. By three,
people were being trampled and crushed even in Bishopsgate Street,
a couple of hundred yards or more from Liverpool Street station;
revolvers were fired, people stabbed, and the policemen who had
been sent to direct the traffic, exhausted and infuriated, were
breaking the heads of the people they were called out to protect.

And as the day advanced and the engine drivers and stokers refused
to return to London, the pressure of the flight drove the people in
an ever-thickening multitude away from the stations and along the
northward-running roads. By midday a Martian had been seen at
Barnes, and a cloud of slowly sinking black vapour drove along the
Thames and across the flats of Lambeth, cutting off all escape over
the bridges in its sluggish advance. Another bank drove over
Ealing, and surrounded a little island of survivors on Castle Hill,
alive, but unable to escape.

After a fruitless struggle to get aboard a North-Western train at
Chalk Farm\dash{}the engines of the trains that had loaded in the goods
yard there \emph{ploughed} through shrieking people, and a dozen
stalwart men fought to keep the crowd from crushing the driver
against his furnace\dash{}my brother emerged upon the Chalk Farm road,
dodged across through a hurrying swarm of vehicles, and had the
luck to be foremost in the sack of a cycle shop. The front tire of
the machine he got was punctured in dragging it through the window,
but he got up and off, notwithstanding, with no further injury than
a cut wrist. The steep foot of Haverstock Hill was impassable owing
to several overturned horses, and my brother struck into Belsize
Road.

So he got out of the fury of the panic, and, skirting the Edgware
Road, reached Edgware about seven, fasting and wearied, but well
ahead of the crowd. Along the road people were standing in the
roadway, curious, wondering. He was passed by a number of cyclists,
some horsemen, and two motor cars. A mile from Edgware the rim of
the wheel broke, and the machine became unridable. He left it by
the roadside and trudged through the village. There were shops half
opened in the main street of the place, and people crowded on the
pavement and in the doorways and windows, staring astonished at
this extraordinary procession of fugitives that was beginning. He
succeeded in getting some food at an inn.

For a time he remained in Edgware not knowing what next to do. The
flying people increased in number. Many of them, like my brother,
seemed inclined to loiter in the place. There was no fresh news of
the invaders from Mars.

At that time the road was crowded, but as yet far from congested.
Most of the fugitives at that hour were mounted on cycles, but
there were soon motor cars, hansom cabs, and carriages hurrying
along, and the dust hung in heavy clouds along the road to St.\ Albans.

It was perhaps a vague idea of making his way to Chelmsford, where
some friends of his lived, that at last induced my brother to
strike into a quiet lane running eastward. Presently he came upon a
stile, and, crossing it, followed a footpath northeastward. He
passed near several farmhouses and some little places whose names
he did not learn. He saw few fugitives until, in a grass lane
towards High Barnet, he happened upon two ladies who became his
fellow travellers. He came upon them just in time to save them.

He heard their screams, and, hurrying round the corner, saw a
couple of men struggling to drag them out of the little pony-chaise
in which they had been driving, while a third with difficulty held
the frightened pony's head. One of the ladies, a short woman
dressed in white, was simply screaming; the other, a dark, slender
figure, slashed at the man who gripped her arm with a whip she held
in her disengaged hand.

My brother immediately grasped the situation, shouted, and hurried
towards the struggle. One of the men desisted and turned towards
him, and my brother, realising from his antagonist's face that a
fight was unavoidable, and being an expert boxer, went into him
forthwith and sent him down against the wheel of the chaise.

It was no time for pugilistic chivalry and my brother laid him
quiet with a kick, and gripped the collar of the man who pulled at
the slender lady's arm. He heard the clatter of hoofs, the whip
stung across his face, a third antagonist struck him between the
eyes, and the man he held wrenched himself free and made off down
the lane in the direction from which he had come.

Partly stunned, he found himself facing the man who had held the
horse's head, and became aware of the chaise receding from him down
the lane, swaying from side to side, and with the women in it
looking back. The man before him, a burly rough, tried to close,
and he stopped him with a blow in the face. Then, realising that he
was deserted, he dodged round and made off down the lane after the
chaise, with the sturdy man close behind him, and the fugitive, who
had turned now, following remotely.

Suddenly he stumbled and fell; his immediate pursuer went headlong,
and he rose to his feet to find himself with a couple of
antagonists again. He would have had little chance against them had
not the slender lady very pluckily pulled up and returned to his
help. It seems she had had a revolver all this time, but it had
been under the seat when she and her companion were attacked. She
fired at six yards' distance, narrowly missing my brother. The less
courageous of the robbers made off, and his companion followed him,
cursing his cowardice. They both stopped in sight down the lane,
where the third man lay insensible.

``Take this!'' said the slender lady, and she gave my brother her
revolver.

``Go back to the chaise,'' said my brother, wiping the blood from his
split lip.

She turned without a word\dash{}they were both panting\dash{}and they went
back to where the lady in white struggled to hold back the
frightened pony.

The robbers had evidently had enough of it. When my brother looked
again they were retreating.

``I'll sit here,'' said my brother, ``if I may''; and he got upon the
empty front seat. The lady looked over her shoulder.

``Give me the reins,'' she said, and laid the whip along the pony's
side. In another moment a bend in the road hid the three men from
my brother's eyes.

So, quite unexpectedly, my brother found himself, panting, with a
cut mouth, a bruised jaw, and bloodstained knuckles, driving along
an unknown lane with these two women.

He learned they were the wife and the younger sister of a surgeon
living at Stanmore, who had come in the small hours from a
dangerous case at Pinner, and heard at some railway station on his
way of the Martian advance. He had hurried home, roused the
women\dash{}their servant had left them two days before\dash{}packed some
provisions, put his revolver under the seat\dash{}luckily for my
brother\dash{}and told them to drive on to Edgware, with the idea of
getting a train there. He stopped behind to tell the neighbours. He
would overtake them, he said, at about half past four in the
morning, and now it was nearly nine and they had seen nothing of
him. They could not stop in Edgware because of the growing traffic
through the place, and so they had come into this side lane.

That was the story they told my brother in fragments when presently
they stopped again, nearer to New Barnet. He promised to stay with
them, at least until they could determine what to do, or until the
missing man arrived, and professed to be an expert shot with the
revolver\dash{}a weapon strange to him\dash{}in order to give them
confidence.

They made a sort of encampment by the wayside, and the pony became
happy in the hedge. He told them of his own escape out of London,
and all that he knew of these Martians and their ways. The sun
crept higher in the sky, and after a time their talk died out and
gave place to an uneasy state of anticipation. Several wayfarers
came along the lane, and of these my brother gathered such news as
he could. Every broken answer he had deepened his impression of the
great disaster that had come on humanity, deepened his persuasion
of the immediate necessity for prosecuting this flight. He urged
the matter upon them.

``We have money,'' said the slender woman, and hesitated.

Her eyes met my brother's, and her hesitation ended.

``So have I,'' said my brother.

She explained that they had as much as thirty pounds in gold,
besides a five-pound note, and suggested that with that they might
get upon a train at St.\ Albans or New Barnet. My brother thought
that was hopeless, seeing the fury of the Londoners to crowd upon
the trains, and broached his own idea of striking across Essex
towards Harwich and thence escaping from the country altogether.

Mrs. Elphinstone\dash{}that was the name of the woman in white\dash{}would
listen to no reasoning, and kept calling upon ``George''; but her
sister-in-law was astonishingly quiet and deliberate, and at last
agreed to my brother's suggestion. So, designing to cross the Great
North Road, they went on towards Barnet, my brother leading the
pony to save it as much as possible. As the sun crept up the sky
the day became excessively hot, and under foot a thick, whitish
sand grew burning and blinding, so that they travelled only very
slowly. The hedges were grey with dust. And as they advanced
towards Barnet a tumultuous murmuring grew stronger.

They began to meet more people. For the most part these were
staring before them, murmuring indistinct questions, jaded,
haggard, unclean. One man in evening dress passed them on foot, his
eyes on the ground. They heard his voice, and, looking back at him,
saw one hand clutched in his hair and the other beating invisible
things. His paroxysm of rage over, he went on his way without once
looking back.

As my brother's party went on towards the crossroads to the south
of Barnet they saw a woman approaching the road across some fields
on their left, carrying a child and with two other children; and
then passed a man in dirty black, with a thick stick in one hand
and a small portmanteau in the other. Then round the corner of the
lane, from between the villas that guarded it at its confluence
with the high road, came a little cart drawn by a sweating black
pony and driven by a sallow youth in a bowler hat, grey with dust.
There were three girls, East End factory girls, and a couple of
little children crowded in the cart.

``This'll tike us rahnd Edgware?'' asked the driver, wild-eyed,
white-faced; and when my brother told him it would if he turned to
the left, he whipped up at once without the formality of thanks.

My brother noticed a pale grey smoke or haze rising among the
houses in front of them, and veiling the white facade of a terrace
beyond the road that appeared between the backs of the villas. Mrs.
Elphinstone suddenly cried out at a number of tongues of smoky red
flame leaping up above the houses in front of them against the hot,
blue sky. The tumultuous noise resolved itself now into the
disorderly mingling of many voices, the gride of many wheels, the
creaking of waggons, and the staccato of hoofs. The lane came round
sharply not fifty yards from the crossroads.

``Good heavens!'' cried Mrs. Elphinstone. ``What is this you are
driving us into?''

My brother stopped.

For the main road was a boiling stream of people, a torrent of
human beings rushing northward, one pressing on another. A great
bank of dust, white and luminous in the blaze of the sun, made
everything within twenty feet of the ground grey and indistinct and
was perpetually renewed by the hurrying feet of a dense crowd of
horses and of men and women on foot, and by the wheels of vehicles
of every description.

``Way!'' my brother heard voices crying. ``Make way!''

It was like riding into the smoke of a fire to approach the meeting
point of the lane and road; the crowd roared like a fire, and the
dust was hot and pungent. And, indeed, a little way up the road a
villa was burning and sending rolling masses of black smoke across
the road to add to the confusion.

Two men came past them. Then a dirty woman, carrying a heavy bundle
and weeping. A lost retriever dog, with hanging tongue, circled
dubiously round them, scared and wretched, and fled at my brother's
threat.

So much as they could see of the road Londonward between the houses
to the right was a tumultuous stream of dirty, hurrying people,
pent in between the villas on either side; the black heads, the
crowded forms, grew into distinctness as they rushed towards the
corner, hurried past, and merged their individuality again in a
receding multitude that was swallowed up at last in a cloud of
dust.

``Go on! Go on!'' cried the voices. ``Way! Way!''

One man's hands pressed on the back of another. My brother stood at
the pony's head. Irresistibly attracted, he advanced slowly, pace
by pace, down the lane.

Edgware had been a scene of confusion, Chalk Farm a riotous tumult,
but this was a whole population in movement. It is hard to imagine
that host. It had no character of its own. The figures poured out
past the corner, and receded with their backs to the group in the
lane. Along the margin came those who were on foot threatened by
the wheels, stumbling in the ditches, blundering into one another.

The carts and carriages crowded close upon one another, making
little way for those swifter and more impatient vehicles that
darted forward every now and then when an opportunity showed itself
of doing so, sending the people scattering against the fences and
gates of the villas.

``Push on!'' was the cry. ``Push on! They are coming!''

In one cart stood a blind man in the uniform of the Salvation Army,
gesticulating with his crooked fingers and bawling, ``Eternity!
Eternity!'' His voice was hoarse and very loud so that my brother
could hear him long after he was lost to sight in the dust. Some of
the people who crowded in the carts whipped stupidly at their
horses and quarrelled with other drivers; some sat motionless,
staring at nothing with miserable eyes; some gnawed their hands
with thirst, or lay prostrate in the bottoms of their conveyances.
The horses' bits were covered with foam, their eyes bloodshot.

There were cabs, carriages, shop cars, waggons, beyond counting; a
mail cart, a road-cleaner's cart marked ``Vestry of St.\ Pancras,'' a
huge timber waggon crowded with roughs. A brewer's dray rumbled by
with its two near wheels splashed with fresh blood.

``Clear the way!'' cried the voices. ``Clear the way!''

``Eter-nity! Eter-nity!'' came echoing down the road.

There were sad, haggard women tramping by, well dressed, with
children that cried and stumbled, their dainty clothes smothered in
dust, their weary faces smeared with tears. With many of these came
men, sometimes helpful, sometimes lowering and savage. Fighting
side by side with them pushed some weary street outcast in faded
black rags, wide-eyed, loud-voiced, and foul-mouthed. There were
sturdy workmen thrusting their way along, wretched, unkempt men,
clothed like clerks or shopmen, struggling spasmodically; a wounded
soldier my brother noticed, men dressed in the clothes of railway
porters, one wretched creature in a nightshirt with a coat thrown
over it.

But varied as its composition was, certain things all that host had
in common. There were fear and pain on their faces, and fear behind
them. A tumult up the road, a quarrel for a place in a waggon, sent
the whole host of them quickening their pace; even a man so scared
and broken that his knees bent under him was galvanised for a
moment into renewed activity. The heat and dust had already been at
work upon this multitude. Their skins were dry, their lips black
and cracked. They were all thirsty, weary, and footsore. And amid
the various cries one heard disputes, reproaches, groans of
weariness and fatigue; the voices of most of them were hoarse and
weak. Through it all ran a refrain:

``Way! Way! The Martians are coming!''

Few stopped and came aside from that flood. The lane opened
slantingly into the main road with a narrow opening, and had a
delusive appearance of coming from the direction of London. Yet a
kind of eddy of people drove into its mouth; weaklings elbowed out
of the stream, who for the most part rested but a moment before
plunging into it again. A little way down the lane, with two
friends bending over him, lay a man with a bare leg, wrapped about
with bloody rags. He was a lucky man to have friends.

A little old man, with a grey military moustache and a filthy black
frock coat, limped out and sat down beside the trap, removed his
boot\dash{}his sock was blood-stained\dash{}shook out a pebble, and hobbled
on again; and then a little girl of eight or nine, all alone, threw
herself under the hedge close by my brother, weeping.

``I can't go on! I can't go on!''

My brother woke from his torpor of astonishment and lifted her up,
speaking gently to her, and carried her to Miss Elphinstone. So
soon as my brother touched her she became quite still, as if
frightened.

``Ellen!'' shrieked a woman in the crowd, with tears in her
voice\dash{}``Ellen!'' And the child suddenly darted away from my brother,
crying ``Mother!''

``They are coming,'' said a man on horseback, riding past along the
lane.

``Out of the way, there!'' bawled a coachman, towering high; and my
brother saw a closed carriage turning into the lane.

The people crushed back on one another to avoid the horse. My
brother pushed the pony and chaise back into the hedge, and the man
drove by and stopped at the turn of the way. It was a carriage,
with a pole for a pair of horses, but only one was in the traces.
My brother saw dimly through the dust that two men lifted out
something on a white stretcher and put it gently on the grass
beneath the privet hedge.

One of the men came running to my brother.

``Where is there any water?'' he said. ``He is dying fast, and very
thirsty. It is Lord Garrick.''

``Lord Garrick!'' said my brother; ``the Chief Justice?''

``The water?'' he said.

``There may be a tap,'' said my brother, ``in some of the houses. We
have no water. I dare not leave my people.''

The man pushed against the crowd towards the gate of the corner
house.

``Go on!'' said the people, thrusting at him. ``They are coming! Go
on!''

Then my brother's attention was distracted by a bearded,
eagle-faced man lugging a small handbag, which split even as my
brother's eyes rested on it and disgorged a mass of sovereigns that
seemed to break up into separate coins as it struck the ground.
They rolled hither and thither among the struggling feet of men and
horses. The man stopped and looked stupidly at the heap, and the
shaft of a cab struck his shoulder and sent him reeling. He gave a
shriek and dodged back, and a cartwheel shaved him narrowly.

``Way!'' cried the men all about him. ``Make way!''

So soon as the cab had passed, he flung himself, with both hands
open, upon the heap of coins, and began thrusting handfuls in his
pocket. A horse rose close upon him, and in another moment, half
rising, he had been borne down under the horse's hoofs.

``Stop!'' screamed my brother, and pushing a woman out of his way,
tried to clutch the bit of the horse.

Before he could get to it, he heard a scream under the wheels, and
saw through the dust the rim passing over the poor wretch's back.
The driver of the cart slashed his whip at my brother, who ran
round behind the cart. The multitudinous shouting confused his
ears. The man was writhing in the dust among his scattered money,
unable to rise, for the wheel had broken his back, and his lower
limbs lay limp and dead. My brother stood up and yelled at the next
driver, and a man on a black horse came to his assistance.

``Get him out of the road,'' said he; and, clutching the man's collar
with his free hand, my brother lugged him sideways. But he still
clutched after his money, and regarded my brother fiercely,
hammering at his arm with a handful of gold. ``Go on! Go on!''
shouted angry voices behind.

``Way! Way!''

There was a smash as the pole of a carriage crashed into the cart
that the man on horseback stopped. My brother looked up, and the
man with the gold twisted his head round and bit the wrist that
held his collar. There was a concussion, and the black horse came
staggering sideways, and the carthorse pushed beside it. A hoof
missed my brother's foot by a hair's breadth. He released his grip
on the fallen man and jumped back. He saw anger change to terror on
the face of the poor wretch on the ground, and in a moment he was
hidden and my brother was borne backward and carried past the
entrance of the lane, and had to fight hard in the torrent to
recover it.

He saw Miss Elphinstone covering her eyes, and a little child, with
all a child's want of sympathetic imagination, staring with dilated
eyes at a dusty something that lay black and still, ground and
crushed under the rolling wheels. ``Let us go back!'' he shouted, and
began turning the pony round. ``We cannot cross this\dash{}hell,'' he said
and they went back a hundred yards the way they had come, until the
fighting crowd was hidden. As they passed the bend in the lane my
brother saw the face of the dying man in the ditch under the
privet, deadly white and drawn, and shining with perspiration. The
two women sat silent, crouching in their seat and shivering.

Then beyond the bend my brother stopped again. Miss Elphinstone was
white and pale, and her sister-in-law sat weeping, too wretched
even to call upon ``George.'' My brother was horrified and perplexed.
So soon as they had retreated he realised how urgent and
unavoidable it was to attempt this crossing. He turned to Miss
Elphinstone, suddenly resolute.

``We must go that way,'' he said, and led the pony round again.

For the second time that day this girl proved her quality. To force
their way into the torrent of people, my brother plunged into the
traffic and held back a cab horse, while she drove the pony across
its head. A waggon locked wheels for a moment and ripped a long
splinter from the chaise. In another moment they were caught and
swept forward by the stream. My brother, with the cabman's whip
marks red across his face and hands, scrambled into the chaise and
took the reins from her.

``Point the revolver at the man behind,'' he said, giving it to her,
``if he presses us too hard. No!\dash{}point it at his horse.''

Then he began to look out for a chance of edging to the right
across the road. But once in the stream he seemed to lose volition,
to become a part of that dusty rout. They swept through Chipping
Barnet with the torrent; they were nearly a mile beyond the centre
of the town before they had fought across to the opposite side of
the way. It was din and confusion indescribable; but in and beyond
the town the road forks repeatedly, and this to some extent
relieved the stress.

They struck eastward through Hadley, and there on either side of
the road, and at another place farther on they came upon a great
multitude of people drinking at the stream, some fighting to come
at the water. And farther on, from a lull near East Barnet, they
saw two trains running slowly one after the other without signal or
order\dash{}trains swarming with people, with men even among the coals
behind the engines\dash{}going northward along the Great Northern
Railway. My brother supposes they must have filled outside London,
for at that time the furious terror of the people had rendered the
central termini impossible.

Near this place they halted for the rest of the afternoon, for the
violence of the day had already utterly exhausted all three of
them. They began to suffer the beginnings of hunger; the night was
cold, and none of them dared to sleep. And in the evening many
people came hurrying along the road nearby their stopping place,
fleeing from unknown dangers before them, and going in the
direction from which my brother had come.

\Chapter{CHAPTER SEVENTEEN\\THE ``THUNDER CHILD''}
Had the Martians aimed only at destruction, they might on Monday
have annihilated the entire population of London, as it spread
itself slowly through the home counties. Not only along the road
through Barnet, but also through Edgware and Waltham Abbey, and
along the roads eastward to Southend and Shoeburyness, and south of
the Thames to Deal and Broadstairs, poured the same frantic rout.
If one could have hung that June morning in a balloon in the
blazing blue above London every northward and eastward road running
out of the tangled maze of streets would have seemed stippled black
with the streaming fugitives, each dot a human agony of terror and
physical distress. I have set forth at length in the last chapter
my brother's account of the road through Chipping Barnet, in order
that my readers may realise how that swarming of black dots
appeared to one of those concerned. Never before in the history of
the world had such a mass of human beings moved and suffered
together. The legendary hosts of Goths and Huns, the hugest armies
Asia has ever seen, would have been but a drop in that current. And
this was no disciplined march; it was a stampede\dash{}a stampede
gigantic and terrible\dash{}without order and without a goal, six
million people unarmed and unprovisioned, driving headlong. It was
the beginning of the rout of civilisation, of the massacre of
mankind.

Directly below him the balloonist would have seen the network of
streets far and wide, houses, churches, squares, crescents,
gardens\dash{}already derelict\dash{}spread out like a huge map, and in the
southward \emph{blotted}. Over Ealing, Richmond, Wimbledon, it
would have seemed as if some monstrous pen had flung ink upon the
chart. Steadily, incessantly, each black splash grew and spread,
shooting out ramifications this way and that, now banking itself
against rising ground, now pouring swiftly over a crest into a
new-found valley, exactly as a gout of ink would spread itself upon
blotting paper.

And beyond, over the blue hills that rise southward of the river,
the glittering Martians went to and fro, calmly and methodically
spreading their poison cloud over this patch of country and then
over that, laying it again with their steam jets when it had served
its purpose, and taking possession of the conquered country. They
do not seem to have aimed at extermination so much as at complete
demoralisation and the destruction of any opposition. They exploded
any stores of powder they came upon, cut every telegraph, and
wrecked the railways here and there. They were hamstringing
mankind. They seemed in no hurry to extend the field of their
operations, and did not come beyond the central part of London all
that day. It is possible that a very considerable number of people
in London stuck to their houses through Monday morning. Certain it
is that many died at home suffocated by the Black Smoke.

Until about midday the Pool of London was an astonishing scene.
Steamboats and shipping of all sorts lay there, tempted by the
enormous sums of money offered by fugitives, and it is said that
many who swam out to these vessels were thrust off with boathooks
and drowned. About one o'clock in the afternoon the thinning
remnant of a cloud of the black vapour appeared between the arches
of Blackfriars Bridge. At that the Pool became a scene of mad
confusion, fighting, and collision, and for some time a multitude
of boats and barges jammed in the northern arch of the Tower
Bridge, and the sailors and lightermen had to fight savagely
against the people who swarmed upon them from the riverfront.
People were actually clambering down the piers of the bridge from
above.

When, an hour later, a Martian appeared beyond the Clock Tower and
waded down the river, nothing but wreckage floated above
Limehouse.

Of the falling of the fifth cylinder I have presently to tell. The
sixth star fell at Wimbledon. My brother, keeping watch beside the
women in the chaise in a meadow, saw the green flash of it far
beyond the hills. On Tuesday the little party, still set upon
getting across the sea, made its way through the swarming country
towards Colchester. The news that the Martians were now in
possession of the whole of London was confirmed. They had been seen
at Highgate, and even, it was said, at Neasden. But they did not
come into my brother's view until the morrow.

That day the scattered multitudes began to realise the urgent need
of provisions. As they grew hungry the rights of property ceased to
be regarded. Farmers were out to defend their cattle-sheds,
granaries, and ripening root crops with arms in their hands. A
number of people now, like my brother, had their faces eastward,
and there were some desperate souls even going back towards London
to get food. These were chiefly people from the northern suburbs,
whose knowledge of the Black Smoke came by hearsay. He heard that
about half the members of the government had gathered at
Birmingham, and that enormous quantities of high explosives were
being prepared to be used in automatic mines across the Midland
counties.

He was also told that the Midland Railway Company had replaced the
desertions of the first day's panic, had resumed traffic, and was
running northward trains from St.\ Albans to relieve the congestion
of the home counties. There was also a placard in Chipping Ongar
announcing that large stores of flour were available in the
northern towns and that within twenty-four hours bread would be
distributed among the starving people in the neighbourhood. But
this intelligence did not deter him from the plan of escape he had
formed, and the three pressed eastward all day, and heard no more
of the bread distribution than this promise. Nor, as a matter of
fact, did anyone else hear more of it. That night fell the seventh
star, falling upon Primrose Hill. It fell while Miss Elphinstone
was watching, for she took that duty alternately with my brother.
She saw it.

On Wednesday the three fugitives\dash{}they had passed the night in a
field of unripe wheat\dash{}reached Chelmsford, and there a body of the
inhabitants, calling itself the Committee of Public Supply, seized
the pony as provisions, and would give nothing in exchange for it
but the promise of a share in it the next day. Here there were
rumours of Martians at Epping, and news of the destruction of
Waltham Abbey Powder Mills in a vain attempt to blow up one of the
invaders.

People were watching for Martians here from the church towers. My
brother, very luckily for him as it chanced, preferred to push on
at once to the coast rather than wait for food, although all three
of them were very hungry. By midday they passed through Tillingham,
which, strangely enough, seemed to be quite silent and deserted,
save for a few furtive plunderers hunting for food. Near Tillingham
they suddenly came in sight of the sea, and the most amazing crowd
of shipping of all sorts that it is possible to imagine.

For after the sailors could no longer come up the Thames, they came
on to the Essex coast, to Harwich and Walton and Clacton, and
afterwards to Foulness and Shoebury, to bring off the people. They
lay in a huge sickle-shaped curve that vanished into mist at last
towards the Naze. Close inshore was a multitude of fishing
smacks\dash{}English, Scotch, French, Dutch, and Swedish; steam launches
from the Thames, yachts, electric boats; and beyond were ships of
large burden, a multitude of filthy colliers, trim merchantmen,
cattle ships, passenger boats, petroleum tanks, ocean tramps, an
old white transport even, neat white and grey liners from
Southampton and Hamburg; and along the blue coast across the
Blackwater my brother could make out dimly a dense swarm of boats
chaffering with the people on the beach, a swarm which also
extended up the Blackwater almost to Maldon.

About a couple of miles out lay an ironclad, very low in the water,
almost, to my brother's perception, like a water-logged ship. This
was the ram \emph{Thunder Child}. It was the only warship in sight,
but far away to the right over the smooth surface of the sea\dash{}for
that day there was a dead calm\dash{}lay a serpent of black smoke to
mark the next ironclads of the Channel Fleet, which hovered in an
extended line, steam up and ready for action, across the Thames
estuary during the course of the Martian conquest, vigilant and yet
powerless to prevent it.

At the sight of the sea, Mrs. Elphinstone, in spite of the
assurances of her sister-in-law, gave way to panic. She had never
been out of England before, she would rather die than trust herself
friendless in a foreign country, and so forth. She seemed, poor
woman, to imagine that the French and the Martians might prove very
similar. She had been growing increasingly hysterical, fearful, and
depressed during the two days' journeyings. Her great idea was to
return to Stanmore. Things had been always well and safe at
Stanmore. They would find George at Stanmore.

It was with the greatest difficulty they could get her down to the
beach, where presently my brother succeeded in attracting the
attention of some men on a paddle steamer from the Thames. They
sent a boat and drove a bargain for thirty-six pounds for the
three. The steamer was going, these men said, to Ostend.

It was about two o'clock when my brother, having paid their fares
at the gangway, found himself safely aboard the steamboat with his
charges. There was food aboard, albeit at exorbitant prices, and
the three of them contrived to eat a meal on one of the seats
forward.

There were already a couple of score of passengers aboard, some of
whom had expended their last money in securing a passage, but the
captain lay off the Blackwater until five in the afternoon, picking
up passengers until the seated decks were even dangerously crowded.
He would probably have remained longer had it not been for the
sound of guns that began about that hour in the south. As if in
answer, the ironclad seaward fired a small gun and hoisted a string
of flags. A jet of smoke sprang out of her funnels.

Some of the passengers were of opinion that this firing came from
Shoeburyness, until it was noticed that it was growing louder. At
the same time, far away in the southeast the masts and upperworks
of three ironclads rose one after the other out of the sea, beneath
clouds of black smoke. But my brother's attention speedily reverted
to the distant firing in the south. He fancied he saw a column of
smoke rising out of the distant grey haze.

The little steamer was already flapping her way eastward of the big
crescent of shipping, and the low Essex coast was growing blue and
hazy, when a Martian appeared, small and faint in the remote
distance, advancing along the muddy coast from the direction of
Foulness. At that the captain on the bridge swore at the top of his
voice with fear and anger at his own delay, and the paddles seemed
infected with his terror. Every soul aboard stood at the bulwarks
or on the seats of the steamer and stared at that distant shape,
higher than the trees or church towers inland, and advancing with a
leisurely parody of a human stride.

It was the first Martian my brother had seen, and he stood, more
amazed than terrified, watching this Titan advancing deliberately
towards the shipping, wading farther and farther into the water as
the coast fell away. Then, far away beyond the Crouch, came
another, striding over some stunted trees, and then yet another,
still farther off, wading deeply through a shiny mudflat that
seemed to hang halfway up between sea and sky. They were all
stalking seaward, as if to intercept the escape of the
multitudinous vessels that were crowded between Foulness and the
Naze. In spite of the throbbing exertions of the engines of the
little paddle-boat, and the pouring foam that her wheels flung
behind her, she receded with terrifying slowness from this ominous
advance.

Glancing northwestward, my brother saw the large crescent of
shipping already writhing with the approaching terror; one ship
passing behind another, another coming round from broadside to end
on, steamships whistling and giving off volumes of steam, sails
being let out, launches rushing hither and thither. He was so
fascinated by this and by the creeping danger away to the left that
he had no eyes for anything seaward. And then a swift movement of
the steamboat (she had suddenly come round to avoid being run down)
flung him headlong from the seat upon which he was standing. There
was a shouting all about him, a trampling of feet, and a cheer that
seemed to be answered faintly. The steamboat lurched and rolled him
over upon his hands.

He sprang to his feet and saw to starboard, and not a hundred yards
from their heeling, pitching boat, a vast iron bulk like the blade
of a plough tearing through the water, tossing it on either side in
huge waves of foam that leaped towards the steamer, flinging her
paddles helplessly in the air, and then sucking her deck down
almost to the waterline.

A douche of spray blinded my brother for a moment. When his eyes
were clear again he saw the monster had passed and was rushing
landward. Big iron upperworks rose out of this headlong structure,
and from that twin funnels projected and spat a smoking blast shot
with fire. It was the torpedo ram, \emph{Thunder Child}, steaming
headlong, coming to the rescue of the threatened shipping.

Keeping his footing on the heaving deck by clutching the bulwarks,
my brother looked past this charging leviathan at the Martians
again, and he saw the three of them now close together, and
standing so far out to sea that their tripod supports were almost
entirely submerged. Thus sunken, and seen in remote perspective,
they appeared far less formidable than the huge iron bulk in whose
wake the steamer was pitching so helplessly. It would seem they
were regarding this new antagonist with astonishment. To their
intelligence, it may be, the giant was even such another as
themselves. The \emph{Thunder Child} fired no gun, but simply drove
full speed towards them. It was probably her not firing that
enabled her to get so near the enemy as she did. They did not know
what to make of her. One shell, and they would have sent her to the
bottom forthwith with the Heat-Ray.

She was steaming at such a pace that in a minute she seemed halfway
between the steamboat and the Martians\dash{}a diminishing black bulk
against the receding horizontal expanse of the Essex coast.

Suddenly the foremost Martian lowered his tube and discharged a
canister of the black gas at the ironclad. It hit her larboard side
and glanced off in an inky jet that rolled away to seaward, an
unfolding torrent of Black Smoke, from which the ironclad drove
clear. To the watchers from the steamer, low in the water and with
the sun in their eyes, it seemed as though she were already among
the Martians.

They saw the gaunt figures separating and rising out of the water
as they retreated shoreward, and one of them raised the camera-like
generator of the Heat-Ray. He held it pointing obliquely downward,
and a bank of steam sprang from the water at its touch. It must
have driven through the iron of the ship's side like a white-hot
iron rod through paper.

A flicker of flame went up through the rising steam, and then the
Martian reeled and staggered. In another moment he was cut down,
and a great body of water and steam shot high in the air. The guns
of the \emph{Thunder Child} sounded through the reek, going off one
after the other, and one shot splashed the water high close by the
steamer, ricocheted towards the other flying ships to the north,
and smashed a smack to matchwood.

But no one heeded that very much. At the sight of the Martian's
collapse the captain on the bridge yelled inarticulately, and all
the crowding passengers on the steamer's stern shouted together.
And then they yelled again. For, surging out beyond the white
tumult, drove something long and black, the flames streaming from
its middle parts, its ventilators and funnels spouting fire.

She was alive still; the steering gear, it seems, was intact and
her engines working. She headed straight for a second Martian, and
was within a hundred yards of him when the Heat-Ray came to bear.
Then with a violent thud, a blinding flash, her decks, her funnels,
leaped upward. The Martian staggered with the violence of her
explosion, and in another moment the flaming wreckage, still
driving forward with the impetus of its pace, had struck him and
crumpled him up like a thing of cardboard. My brother shouted
involuntarily. A boiling tumult of steam hid everything again.

``Two!'' yelled the captain.

Everyone was shouting. The whole steamer from end to end rang with
frantic cheering that was taken up first by one and then by all in
the crowding multitude of ships and boats that was driving out to
sea.

The steam hung upon the water for many minutes, hiding the third
Martian and the coast altogether. And all this time the boat was
paddling steadily out to sea and away from the fight; and when at
last the confusion cleared, the drifting bank of black vapour
intervened, and nothing of the \emph{Thunder Child} could be made
out, nor could the third Martian be seen. But the ironclads to
seaward were now quite close and standing in towards shore past the
steamboat.

The little vessel continued to beat its way seaward, and the
ironclads receded slowly towards the coast, which was hidden still
by a marbled bank of vapour, part steam, part black gas, eddying
and combining in the strangest way. The fleet of refugees was
scattering to the northeast; several smacks were sailing between
the ironclads and the steamboat. After a time, and before they
reached the sinking cloud bank, the warships turned northward, and
then abruptly went about and passed into the thickening haze of
evening southward. The coast grew faint, and at last
indistinguishable amid the low banks of clouds that were gathering
about the sinking sun.

Then suddenly out of the golden haze of the sunset came the
vibration of guns, and a form of black shadows moving. Everyone
struggled to the rail of the steamer and peered into the blinding
furnace of the west, but nothing was to be distinguished clearly. A
mass of smoke rose slanting and barred the face of the sun. The
steamboat throbbed on its way through an interminable suspense.

The sun sank into grey clouds, the sky flushed and darkened, the
evening star trembled into sight. It was deep twilight when the
captain cried out and pointed. My brother strained his eyes.
Something rushed up into the sky out of the greyness\dash{}rushed
slantingly upward and very swiftly into the luminous clearness
above the clouds in the western sky; something flat and broad, and
very large, that swept round in a vast curve, grew smaller, sank
slowly, and vanished again into the grey mystery of the night. And
as it flew it rained down darkness upon the land.

\Book{BOOK TWO\\THE EARTH UNDER THE MARTIANS}
\Chapter{CHAPTER ONE\\UNDER FOOT}
In the first book I have wandered so much from my own adventures to
tell of the experiences of my brother that all through the last two
chapters I and the curate have been lurking in the empty house at
Halliford whither we fled to escape the Black Smoke. There I will
resume. We stopped there all Sunday night and all the next day\dash{}the
day of the panic\dash{}in a little island of daylight, cut off by the
Black Smoke from the rest of the world. We could do nothing but
wait in aching inactivity during those two weary days.

My mind was occupied by anxiety for my wife. I figured her at
Leatherhead, terrified, in danger, mourning me already as a dead
man. I paced the rooms and cried aloud when I thought of how I was
cut off from her, of all that might happen to her in my absence. My
cousin I knew was brave enough for any emergency, but he was not
the sort of man to realise danger quickly, to rise promptly. What
was needed now was not bravery, but circumspection. My only
consolation was to believe that the Martians were moving
London-ward and away from her. Such vague anxieties keep the mind
sensitive and painful. I grew very weary and irritable with the
curate's perpetual ejaculations; I tired of the sight of his
selfish despair. After some ineffectual remonstrance I kept away
from him, staying in a room\dash{}evidently a children's
schoolroom\dash{}containing globes, forms, and copybooks. When he
followed me thither, I went to a box room at the top of the house
and, in order to be alone with my aching miseries, locked myself
in.

We were hopelessly hemmed in by the Black Smoke all that day and
the morning of the next. There were signs of people in the next
house on Sunday evening\dash{}a face at a window and moving lights, and
later the slamming of a door. But I do not know who these people
were, nor what became of them. We saw nothing of them next day. The
Black Smoke drifted slowly riverward all through Monday morning,
creeping nearer and nearer to us, driving at last along the roadway
outside the house that hid us.

A Martian came across the fields about midday, laying the stuff
with a jet of superheated steam that hissed against the walls,
smashed all the windows it touched, and scalded the curate's hand
as he fled out of the front room. When at last we crept across the
sodden rooms and looked out again, the country northward was as
though a black snowstorm had passed over it. Looking towards the
river, we were astonished to see an unaccountable redness mingling
with the black of the scorched meadows.

For a time we did not see how this change affected our position,
save that we were relieved of our fear of the Black Smoke. But
later I perceived that we were no longer hemmed in, that now we
might get away. So soon as I realised that the way of escape was
open, my dream of action returned. But the curate was lethargic,
unreasonable.

``We are safe here,'' he repeated; ``safe here.''

I resolved to leave him\dash{}would that I had! Wiser now for the
artilleryman's teaching, I sought out food and drink. I had found
oil and rags for my burns, and I also took a hat and a flannel
shirt that I found in one of the bedrooms. When it was clear to him
that I meant to go alone\dash{}had reconciled myself to going alone\dash{}he
suddenly roused himself to come. And all being quiet throughout the
afternoon, we started about five o'clock, as I should judge, along
the blackened road to Sunbury.

In Sunbury, and at intervals along the road, were dead bodies lying
in contorted attitudes, horses as well as men, overturned carts and
luggage, all covered thickly with black dust. That pall of cindery
powder made me think of what I had read of the destruction of
Pompeii. We got to Hampton Court without misadventure, our minds
full of strange and unfamiliar appearances, and at Hampton Court
our eyes were relieved to find a patch of green that had escaped
the suffocating drift. We went through Bushey Park, with its deer
going to and fro under the chestnuts, and some men and women
hurrying in the distance towards Hampton, and so we came to
Twickenham. These were the first people we saw.

Away across the road the woods beyond Ham and Petersham were still
afire. Twickenham was uninjured by either Heat-Ray or Black Smoke,
and there were more people about here, though none could give us
news. For the most part they were like ourselves, taking advantage
of a lull to shift their quarters. I have an impression that many
of the houses here were still occupied by scared inhabitants, too
frightened even for flight. Here too the evidence of a hasty rout
was abundant along the road. I remember most vividly three smashed
bicycles in a heap, pounded into the road by the wheels of
subsequent carts. We crossed Richmond Bridge about half past eight.
We hurried across the exposed bridge, of course, but I noticed
floating down the stream a number of red masses, some many feet
across. I did not know what these were\dash{}there was no time for
scrutiny\dash{}and I put a more horrible interpretation on them than
they deserved. Here again on the Surrey side were black dust that
had once been smoke, and dead bodies\dash{}a heap near the approach to
the station; but we had no glimpse of the Martians until we were
some way towards Barnes.

We saw in the blackened distance a group of three people running
down a side street towards the river, but otherwise it seemed
deserted. Up the hill Richmond town was burning briskly; outside
the town of Richmond there was no trace of the Black Smoke.

Then suddenly, as we approached Kew, came a number of people
running, and the upperworks of a Martian fighting-machine loomed in
sight over the housetops, not a hundred yards away from us. We
stood aghast at our danger, and had the Martian looked down we must
immediately have perished. We were so terrified that we dared not
go on, but turned aside and hid in a shed in a garden. There the
curate crouched, weeping silently, and refusing to stir again.

But my fixed idea of reaching Leatherhead would not let me rest,
and in the twilight I ventured out again. I went through a
shrubbery, and along a passage beside a big house standing in its
own grounds, and so emerged upon the road towards Kew. The curate I
left in the shed, but he came hurrying after me.

That second start was the most foolhardy thing I ever did. For it
was manifest the Martians were about us. No sooner had the curate
overtaken me than we saw either the fighting-machine we had seen
before or another, far away across the meadows in the direction of
Kew Lodge. Four or five little black figures hurried before it
across the green-grey of the field, and in a moment it was evident
this Martian pursued them. In three strides he was among them, and
they ran radiating from his feet in all directions. He used no
Heat-Ray to destroy them, but picked them up one by one. Apparently
he tossed them into the great metallic carrier which projected
behind him, much as a workman's basket hangs over his shoulder.

It was the first time I realised that the Martians might have any
other purpose than destruction with defeated humanity. We stood for
a moment petrified, then turned and fled through a gate behind us
into a walled garden, fell into, rather than found, a fortunate
ditch, and lay there, scarce daring to whisper to each other until
the stars were out.

I suppose it was nearly eleven o'clock before we gathered courage
to start again, no longer venturing into the road, but sneaking
along hedgerows and through plantations, and watching keenly
through the darkness, he on the right and I on the left, for the
Martians, who seemed to be all about us. In one place we blundered
upon a scorched and blackened area, now cooling and ashen, and a
number of scattered dead bodies of men, burned horribly about the
heads and trunks but with their legs and boots mostly intact; and
of dead horses, fifty feet, perhaps, behind a line of four ripped
guns and smashed gun carriages.

Sheen, it seemed, had escaped destruction, but the place was silent
and deserted. Here we happened on no dead, though the night was too
dark for us to see into the side roads of the place. In Sheen my
companion suddenly complained of faintness and thirst, and we
decided to try one of the houses.

The first house we entered, after a little difficulty with the
window, was a small semi-detached villa, and I found nothing
eatable left in the place but some mouldy cheese. There was,
however, water to drink; and I took a hatchet, which promised to be
useful in our next house-breaking.

We then crossed to a place where the road turns towards Mortlake.
Here there stood a white house within a walled garden, and in the
pantry of this domicile we found a store of food\dash{}two loaves of
bread in a pan, an uncooked steak, and the half of a ham. I give
this catalogue so precisely because, as it happened, we were
destined to subsist upon this store for the next fortnight. Bottled
beer stood under a shelf, and there were two bags of haricot beans
and some limp lettuces. This pantry opened into a kind of wash-up
kitchen, and in this was firewood; there was also a cupboard, in
which we found nearly a dozen of burgundy, tinned soups and salmon,
and two tins of biscuits.

We sat in the adjacent kitchen in the dark\dash{}for we dared not strike
a light\dash{}and ate bread and ham, and drank beer out of the same
bottle. The curate, who was still timorous and restless, was now,
oddly enough, for pushing on, and I was urging him to keep up his
strength by eating when the thing happened that was to imprison
us.

``It can't be midnight yet,'' I said, and then came a blinding glare
of vivid green light. Everything in the kitchen leaped out, clearly
visible in green and black, and vanished again. And then followed
such a concussion as I have never heard before or since. So close
on the heels of this as to seem instantaneous came a thud behind
me, a clash of glass, a crash and rattle of falling masonry all
about us, and the plaster of the ceiling came down upon us,
smashing into a multitude of fragments upon our heads. I was
knocked headlong across the floor against the oven handle and
stunned. I was insensible for a long time, the curate told me, and
when I came to we were in darkness again, and he, with a face wet,
as I found afterwards, with blood from a cut forehead, was dabbing
water over me.

For some time I could not recollect what had happened. Then things
came to me slowly. A bruise on my temple asserted itself.

``Are you better?'' asked the curate in a whisper.

At last I answered him. I sat up.

``Don't move,'' he said. ``The floor is covered with smashed crockery
from the dresser. You can't possibly move without making a noise,
and I fancy \emph{they} are outside.''

We both sat quite silent, so that we could scarcely hear each other
breathing. Everything seemed deadly still, but once something near
us, some plaster or broken brickwork, slid down with a rumbling
sound. Outside and very near was an intermittent, metallic rattle.

``That!'' said the curate, when presently it happened again.

``Yes,'' I said. ``But what is it?''

``A Martian!'' said the curate.

I listened again.

``It was not like the Heat-Ray,'' I said, and for a time I was
inclined to think one of the great fighting-machines had stumbled
against the house, as I had seen one stumble against the tower of
Shepperton Church.

Our situation was so strange and incomprehensible that for three or
four hours, until the dawn came, we scarcely moved. And then the
light filtered in, not through the window, which remained black,
but through a triangular aperture between a beam and a heap of
broken bricks in the wall behind us. The interior of the kitchen we
now saw greyly for the first time.

The window had been burst in by a mass of garden mould, which
flowed over the table upon which we had been sitting and lay about
our feet. Outside, the soil was banked high against the house. At
the top of the window frame we could see an uprooted drainpipe. The
floor was littered with smashed hardware; the end of the kitchen
towards the house was broken into, and since the daylight shone in
there, it was evident the greater part of the house had collapsed.
Contrasting vividly with this ruin was the neat dresser, stained in
the fashion, pale green, and with a number of copper and tin
vessels below it, the wallpaper imitating blue and white tiles, and
a couple of coloured supplements fluttering from the walls above
the kitchen range.

As the dawn grew clearer, we saw through the gap in the wall the
body of a Martian, standing sentinel, I suppose, over the still
glowing cylinder. At the sight of that we crawled as circumspectly
as possible out of the twilight of the kitchen into the darkness of
the scullery.

Abruptly the right interpretation dawned upon my mind.

``The fifth cylinder,'' I whispered, ``the fifth shot from Mars, has
struck this house and buried us under the ruins!''

For a time the curate was silent, and then he whispered:

``God have mercy upon us!''

I heard him presently whimpering to himself.

Save for that sound we lay quite still in the scullery; I for my
part scarce dared breathe, and sat with my eyes fixed on the faint
light of the kitchen door. I could just see the curate's face, a
dim, oval shape, and his collar and cuffs. Outside there began a
metallic hammering, then a violent hooting, and then again, after a
quiet interval, a hissing like the hissing of an engine. These
noises, for the most part problematical, continued intermittently,
and seemed if anything to increase in number as time wore on.
Presently a measured thudding and a vibration that made everything
about us quiver and the vessels in the pantry ring and shift, began
and continued. Once the light was eclipsed, and the ghostly kitchen
doorway became absolutely dark. For many hours we must have
crouched there, silent and shivering, until our tired attention
failed. \ldots{}

At last I found myself awake and very hungry. I am inclined to
believe we must have spent the greater portion of a day before that
awakening. My hunger was at a stride so insistent that it moved me
to action. I told the curate I was going to seek food, and felt my
way towards the pantry. He made me no answer, but so soon as I
began eating the faint noise I made stirred him up and I heard him
crawling after me.

\Chapter{CHAPTER TWO\\WHAT WE SAW FROM THE RUINED
HOUSE}
After eating we crept back to the scullery, and there I must have
dozed again, for when presently I looked round I was alone. The
thudding vibration continued with wearisome persistence. I
whispered for the curate several times, and at last felt my way to
the door of the kitchen. It was still daylight, and I perceived him
across the room, lying against the triangular hole that looked out
upon the Martians. His shoulders were hunched, so that his head was
hidden from me.

I could hear a number of noises almost like those in an engine
shed; and the place rocked with that beating thud. Through the
aperture in the wall I could see the top of a tree touched with
gold and the warm blue of a tranquil evening sky. For a minute or
so I remained watching the curate, and then I advanced, crouching
and stepping with extreme care amid the broken crockery that
littered the floor.

I touched the curate's leg, and he started so violently that a mass
of plaster went sliding down outside and fell with a loud impact. I
gripped his arm, fearing he might cry out, and for a long time we
crouched motionless. Then I turned to see how much of our rampart
remained. The detachment of the plaster had left a vertical slit
open in the debris, and by raising myself cautiously across a beam
I was able to see out of this gap into what had been overnight a
quiet suburban roadway. Vast, indeed, was the change that we
beheld.

The fifth cylinder must have fallen right into the midst of the
house we had first visited. The building had vanished, completely
smashed, pulverised, and dispersed by the blow. The cylinder lay
now far beneath the original foundations\dash{}deep in a hole, already
vastly larger than the pit I had looked into at Woking. The earth
all round it had splashed under that tremendous impact\dash{}``splashed''
is the only word\dash{}and lay in heaped piles that hid the masses of
the adjacent houses. It had behaved exactly like mud under the
violent blow of a hammer. Our house had collapsed backward; the
front portion, even on the ground floor, had been destroyed
completely; by a chance the kitchen and scullery had escaped, and
stood buried now under soil and ruins, closed in by tons of earth
on every side save towards the cylinder. Over that aspect we hung
now on the very edge of the great circular pit the Martians were
engaged in making. The heavy beating sound was evidently just
behind us, and ever and again a bright green vapour drove up like a
veil across our peephole.

The cylinder was already opened in the centre of the pit, and on
the farther edge of the pit, amid the smashed and gravel-heaped
shrubbery, one of the great fighting-machines, deserted by its
occupant, stood stiff and tall against the evening sky. At first I
scarcely noticed the pit and the cylinder, although it has been
convenient to describe them first, on account of the extraordinary
glittering mechanism I saw busy in the excavation, and on account
of the strange creatures that were crawling slowly and painfully
across the heaped mould near it.

The mechanism it certainly was that held my attention first. It was
one of those complicated fabrics that have since been called
handling-machines, and the study of which has already given such an
enormous impetus to terrestrial invention. As it dawned upon me
first, it presented a sort of metallic spider with five jointed,
agile legs, and with an extraordinary number of jointed levers,
bars, and reaching and clutching tentacles about its body. Most of
its arms were retracted, but with three long tentacles it was
fishing out a number of rods, plates, and bars which lined the
covering and apparently strengthened the walls of the cylinder.
These, as it extracted them, were lifted out and deposited upon a
level surface of earth behind it.

Its motion was so swift, complex, and perfect that at first I did
not see it as a machine, in spite of its metallic glitter. The
fighting-machines were coordinated and animated to an extraordinary
pitch, but nothing to compare with this. People who have never seen
these structures, and have only the ill-imagined efforts of artists
or the imperfect descriptions of such eye-witnesses as myself to go
upon, scarcely realise that living quality.

I recall particularly the illustration of one of the first
pamphlets to give a consecutive account of the war. The artist had
evidently made a hasty study of one of the fighting-machines, and
there his knowledge ended. He presented them as tilted, stiff
tripods, without either flexibility or subtlety, and with an
altogether misleading monotony of effect. The pamphlet containing
these renderings had a considerable vogue, and I mention them here
simply to warn the reader against the impression they may have
created. They were no more like the Martians I saw in action than a
Dutch doll is like a human being. To my mind, the pamphlet would
have been much better without them.

At first, I say, the handling-machine did not impress me as a
machine, but as a crablike creature with a glittering integument,
the controlling Martian whose delicate tentacles actuated its
movements seeming to be simply the equivalent of the crab's
cerebral portion. But then I perceived the resemblance of its
grey-brown, shiny, leathery integument to that of the other
sprawling bodies beyond, and the true nature of this dexterous
workman dawned upon me. With that realisation my interest shifted
to those other creatures, the real Martians. Already I had had a
transient impression of these, and the first nausea no longer
obscured my observation. Moreover, I was concealed and motionless,
and under no urgency of action.

They were, I now saw, the most unearthly creatures it is possible
to conceive. They were huge round bodies\dash{}or, rather, heads\dash{}about
four feet in diameter, each body having in front of it a face. This
face had no nostrils\dash{}indeed, the Martians do not seem to have had
any sense of smell, but it had a pair of very large dark-coloured
eyes, and just beneath this a kind of fleshy beak. In the back of
this head or body\dash{}I scarcely know how to speak of it\dash{}was the
single tight tympanic surface, since known to be anatomically an
ear, though it must have been almost useless in our dense air. In a
group round the mouth were sixteen slender, almost whiplike
tentacles, arranged in two bunches of eight each. These bunches
have since been named rather aptly, by that distinguished
anatomist, Professor Howes, the \emph{hands}. Even as I saw these
Martians for the first time they seemed to be endeavouring to raise
themselves on these hands, but of course, with the increased weight
of terrestrial conditions, this was impossible. There is reason to
suppose that on Mars they may have progressed upon them with some
facility.

The internal anatomy, I may remark here, as dissection has since
shown, was almost equally simple. The greater part of the structure
was the brain, sending enormous nerves to the eyes, ear, and
tactile tentacles. Besides this were the bulky lungs, into which
the mouth opened, and the heart and its vessels. The pulmonary
distress caused by the denser atmosphere and greater gravitational
attraction was only too evident in the convulsive movements of the
outer skin.

And this was the sum of the Martian organs. Strange as it may seem
to a human being, all the complex apparatus of digestion, which
makes up the bulk of our bodies, did not exist in the Martians.
They were heads\dash{}merely heads. Entrails they had none. They did not
eat, much less digest. Instead, they took the fresh, living blood
of other creatures, and \emph{injected} it into their own veins. I
have myself seen this being done, as I shall mention in its place.
But, squeamish as I may seem, I cannot bring myself to describe
what I could not endure even to continue watching. Let it suffice
to say, blood obtained from a still living animal, in most cases
from a human being, was run directly by means of a little pipette
into the recipient canal. \ldots{}

The bare idea of this is no doubt horribly repulsive to us, but at
the same time I think that we should remember how repulsive our
carnivorous habits would seem to an intelligent rabbit.

The physiological advantages of the practice of injection are
undeniable, if one thinks of the tremendous waste of human time and
energy occasioned by eating and the digestive process. Our bodies
are half made up of glands and tubes and organs, occupied in
turning heterogeneous food into blood. The digestive processes and
their reaction upon the nervous system sap our strength and colour
our minds. Men go happy or miserable as they have healthy or
unhealthy livers, or sound gastric glands. But the Martians were
lifted above all these organic fluctuations of mood and emotion.

Their undeniable preference for men as their source of nourishment
is partly explained by the nature of the remains of the victims
they had brought with them as provisions from Mars. These
creatures, to judge from the shrivelled remains that have fallen
into human hands, were bipeds with flimsy, silicious skeletons
(almost like those of the silicious sponges) and feeble
musculature, standing about six feet high and having round, erect
heads, and large eyes in flinty sockets. Two or three of these seem
to have been brought in each cylinder, and all were killed before
earth was reached. It was just as well for them, for the mere
attempt to stand upright upon our planet would have broken every
bone in their bodies.

And while I am engaged in this description, I may add in this place
certain further details which, although they were not all evident
to us at the time, will enable the reader who is unacquainted with
them to form a clearer picture of these offensive creatures.

In three other points their physiology differed strangely from
ours. Their organisms did not sleep, any more than the heart of man
sleeps. Since they had no extensive muscular mechanism to
recuperate, that periodical extinction was unknown to them. They
had little or no sense of fatigue, it would seem. On earth they
could never have moved without effort, yet even to the last they
kept in action. In twenty-four hours they did twenty-four hours of
work, as even on earth is perhaps the case with the ants.

In the next place, wonderful as it seems in a sexual world, the
Martians were absolutely without sex, and therefore without any of
the tumultuous emotions that arise from that difference among men.
A young Martian, there can now be no dispute, was really born upon
earth during the war, and it was found attached to its parent,
partially \emph{budded} off, just as young lilybulbs bud off, or
like the young animals in the fresh-water polyp.

In man, in all the higher terrestrial animals, such a method of
increase has disappeared; but even on this earth it was certainly
the primitive method. Among the lower animals, up even to those
first cousins of the vertebrated animals, the Tunicates, the two
processes occur side by side, but finally the sexual method
superseded its competitor altogether. On Mars, however, just the
reverse has apparently been the case.

It is worthy of remark that a certain speculative writer of
quasi-scientific repute, writing long before the Martian invasion,
did forecast for man a final structure not unlike the actual
Martian condition. His prophecy, I remember, appeared in November
or December, 1893, in a long-defunct publication, the
\emph{Pall Mall Budget}, and I recall a caricature of it in a
pre-Martian periodical called \emph{Punch}. He pointed out\dash{}writing
in a foolish, facetious tone\dash{}that the perfection of mechanical
appliances must ultimately supersede limbs; the perfection of
chemical devices, digestion; that such organs as hair, external
nose, teeth, ears, and chin were no longer essential parts of the
human being, and that the tendency of natural selection would lie
in the direction of their steady diminution through the coming
ages. The brain alone remained a cardinal necessity. Only one other
part of the body had a strong case for survival, and that was the
hand, ``teacher and agent of the brain.'' While the rest of the body
dwindled, the hands would grow larger.

There is many a true word written in jest, and here in the Martians
we have beyond dispute the actual accomplishment of such a
suppression of the animal side of the organism by the intelligence.
To me it is quite credible that the Martians may be descended from
beings not unlike ourselves, by a gradual development of brain and
hands (the latter giving rise to the two bunches of delicate
tentacles at last) at the expense of the rest of the body. Without
the body the brain would, of course, become a mere selfish
intelligence, without any of the emotional substratum of the human
being.

The last salient point in which the systems of these creatures
differed from ours was in what one might have thought a very
trivial particular. Micro-organisms, which cause so much disease
and pain on earth, have either never appeared upon Mars or Martian
sanitary science eliminated them ages ago. A hundred diseases, all
the fevers and contagions of human life, consumption, cancers,
tumours and such morbidities, never enter the scheme of their life.
And speaking of the differences between the life on Mars and
terrestrial life, I may allude here to the curious suggestions of
the red weed.

Apparently the vegetable kingdom in Mars, instead of having green
for a dominant colour, is of a vivid blood-red tint. At any rate,
the seeds which the Martians (intentionally or accidentally)
brought with them gave rise in all cases to red-coloured growths.
Only that known popularly as the red weed, however, gained any
footing in competition with terrestrial forms. The red creeper was
quite a transitory growth, and few people have seen it growing. For
a time, however, the red weed grew with astonishing vigour and
luxuriance. It spread up the sides of the pit by the third or
fourth day of our imprisonment, and its cactus-like branches formed
a carmine fringe to the edges of our triangular window. And
afterwards I found it broadcast throughout the country, and
especially wherever there was a stream of water.

The Martians had what appears to have been an auditory organ, a
single round drum at the back of the head-body, and eyes with a
visual range not very different from ours except that, according to
Philips, blue and violet were as black to them. It is commonly
supposed that they communicated by sounds and tentacular
gesticulations; this is asserted, for instance, in the able but
hastily compiled pamphlet (written evidently by someone not an
eye-witness of Martian actions) to which I have already alluded,
and which, so far, has been the chief source of information
concerning them. Now no surviving human being saw so much of the
Martians in action as I did. I take no credit to myself for an
accident, but the fact is so. And I assert that I watched them
closely time after time, and that I have seen four, five, and
(once) six of them sluggishly performing the most elaborately
complicated operations together without either sound or gesture.
Their peculiar hooting invariably preceded feeding; it had no
modulation, and was, I believe, in no sense a signal, but merely
the expiration of air preparatory to the suctional operation. I
have a certain claim to at least an elementary knowledge of
psychology, and in this matter I am convinced\dash{}as firmly as I am
convinced of anything\dash{}that the Martians interchanged thoughts
without any physical intermediation. And I have been convinced of
this in spite of strong preconceptions. Before the Martian
invasion, as an occasional reader here or there may remember, I had
written with some little vehemence against the telepathic theory.

The Martians wore no clothing. Their conceptions of ornament and
decorum were necessarily different from ours; and not only were
they evidently much less sensible of changes of temperature than we
are, but changes of pressure do not seem to have affected their
health at all seriously. Yet though they wore no clothing, it was
in the other artificial additions to their bodily resources that
their great superiority over man lay. We men, with our bicycles and
road-skates, our Lilienthal soaring-machines, our guns and sticks
and so forth, are just in the beginning of the evolution that the
Martians have worked out. They have become practically mere brains,
wearing different bodies according to their needs just as men wear
suits of clothes and take a bicycle in a hurry or an umbrella in
the wet. And of their appliances, perhaps nothing is more wonderful
to a man than the curious fact that what is the dominant feature of
almost all human devices in mechanism is absent\dash{}the \emph{wheel}
is absent; among all the things they brought to earth there is no
trace or suggestion of their use of wheels. One would have at least
expected it in locomotion. And in this connection it is curious to
remark that even on this earth Nature has never hit upon the wheel,
or has preferred other expedients to its development. And not only
did the Martians either not know of (which is incredible), or
abstain from, the wheel, but in their apparatus singularly little
use is made of the fixed pivot or relatively fixed pivot, with
circular motions thereabout confined to one plane. Almost all the
joints of the machinery present a complicated system of sliding
parts moving over small but beautifully curved friction bearings.
And while upon this matter of detail, it is remarkable that the
long leverages of their machines are in most cases actuated by a
sort of sham musculature of the disks in an elastic sheath; these
disks become polarised and drawn closely and powerfully together
when traversed by a current of electricity. In this way the curious
parallelism to animal motions, which was so striking and disturbing
to the human beholder, was attained. Such quasi-muscles abounded in
the crablike handling-machine which, on my first peeping out of the
slit, I watched unpacking the cylinder. It seemed infinitely more
alive than the actual Martians lying beyond it in the sunset light,
panting, stirring ineffectual tentacles, and moving feebly after
their vast journey across space.

While I was still watching their sluggish motions in the sunlight,
and noting each strange detail of their form, the curate reminded
me of his presence by pulling violently at my arm. I turned to a
scowling face, and silent, eloquent lips. He wanted the slit, which
permitted only one of us to peep through; and so I had to forego
watching them for a time while he enjoyed that privilege.

When I looked again, the busy handling-machine had already put
together several of the pieces of apparatus it had taken out of the
cylinder into a shape having an unmistakable likeness to its own;
and down on the left a busy little digging mechanism had come into
view, emitting jets of green vapour and working its way round the
pit, excavating and embanking in a methodical and discriminating
manner. This it was which had caused the regular beating noise, and
the rhythmic shocks that had kept our ruinous refuge quivering. It
piped and whistled as it worked. So far as I could see, the thing
was without a directing Martian at all.

\Chapter{CHAPTER THREE\\THE DAYS OF IMPRISONMENT}
The arrival of a second fighting-machine drove us from our peephole
into the scullery, for we feared that from his elevation the
Martian might see down upon us behind our barrier. At a later date
we began to feel less in danger of their eyes, for to an eye in the
dazzle of the sunlight outside our refuge must have been blank
blackness, but at first the slightest suggestion of approach drove
us into the scullery in heart-throbbing retreat. Yet terrible as
was the danger we incurred, the attraction of peeping was for both
of us irresistible. And I recall now with a sort of wonder that, in
spite of the infinite danger in which we were between starvation
and a still more terrible death, we could yet struggle bitterly for
that horrible privilege of sight. We would race across the kitchen
in a grotesque way between eagerness and the dread of making a
noise, and strike each other, and thrust and kick, within a few
inches of exposure.

The fact is that we had absolutely incompatible dispositions and
habits of thought and action, and our danger and isolation only
accentuated the incompatibility. At Halliford I had already come to
hate the curate's trick of helpless exclamation, his stupid
rigidity of mind. His endless muttering monologue vitiated every
effort I made to think out a line of action, and drove me at times,
thus pent up and intensified, almost to the verge of craziness. He
was as lacking in restraint as a silly woman. He would weep for
hours together, and I verily believe that to the very end this
spoiled child of life thought his weak tears in some way
efficacious. And I would sit in the darkness unable to keep my mind
off him by reason of his importunities. He ate more than I did, and
it was in vain I pointed out that our only chance of life was to
stop in the house until the Martians had done with their pit, that
in that long patience a time might presently come when we should
need food. He ate and drank impulsively in heavy meals at long
intervals. He slept little.

As the days wore on, his utter carelessness of any consideration so
intensified our distress and danger that I had, much as I loathed
doing it, to resort to threats, and at last to blows. That brought
him to reason for a time. But he was one of those weak creatures,
void of pride, timorous, anaemic, hateful souls, full of shifty
cunning, who face neither God nor man, who face not even
themselves.

It is disagreeable for me to recall and write these things, but I
set them down that my story may lack nothing. Those who have
escaped the dark and terrible aspects of life will find my
brutality, my flash of rage in our final tragedy, easy enough to
blame; for they know what is wrong as well as any, but not what is
possible to tortured men. But those who have been under the shadow,
who have gone down at last to elemental things, will have a wider
charity.

And while within we fought out our dark, dim contest of whispers,
snatched food and drink, and gripping hands and blows, without, in
the pitiless sunlight of that terrible June, was the strange
wonder, the unfamiliar routine of the Martians in the pit. Let me
return to those first new experiences of mine. After a long time I
ventured back to the peephole, to find that the new-comers had been
reinforced by the occupants of no fewer than three of the
fighting-machines. These last had brought with them certain fresh
appliances that stood in an orderly manner about the cylinder. The
second handling-machine was now completed, and was busied in
serving one of the novel contrivances the big machine had brought.
This was a body resembling a milk can in its general form, above
which oscillated a pear-shaped receptacle, and from which a stream
of white powder flowed into a circular basin below.

The oscillatory motion was imparted to this by one tentacle of the
handling-machine. With two spatulate hands the handling-machine was
digging out and flinging masses of clay into the pear-shaped
receptacle above, while with another arm it periodically opened a
door and removed rusty and blackened clinkers from the middle part
of the machine. Another steely tentacle directed the powder from
the basin along a ribbed channel towards some receiver that was
hidden from me by the mound of bluish dust. From this unseen
receiver a little thread of green smoke rose vertically into the
quiet air. As I looked, the handling-machine, with a faint and
musical clinking, extended, telescopic fashion, a tentacle that had
been a moment before a mere blunt projection, until its end was
hidden behind the mound of clay. In another second it had lifted a
bar of white aluminium into sight, untarnished as yet, and shining
dazzlingly, and deposited it in a growing stack of bars that stood
at the side of the pit. Between sunset and starlight this dexterous
machine must have made more than a hundred such bars out of the
crude clay, and the mound of bluish dust rose steadily until it
topped the side of the pit.

The contrast between the swift and complex movements of these
contrivances and the inert panting clumsiness of their masters was
acute, and for days I had to tell myself repeatedly that these
latter were indeed the living of the two things.

The curate had possession of the slit when the first men were
brought to the pit. I was sitting below, huddled up, listening with
all my ears. He made a sudden movement backward, and I, fearful
that we were observed, crouched in a spasm of terror. He came
sliding down the rubbish and crept beside me in the darkness,
inarticulate, gesticulating, and for a moment I shared his panic.
His gesture suggested a resignation of the slit, and after a little
while my curiosity gave me courage, and I rose up, stepped across
him, and clambered up to it. At first I could see no reason for his
frantic behaviour. The twilight had now come, the stars were little
and faint, but the pit was illuminated by the flickering green fire
that came from the aluminium-making. The whole picture was a
flickering scheme of green gleams and shifting rusty black shadows,
strangely trying to the eyes. Over and through it all went the
bats, heeding it not at all. The sprawling Martians were no longer
to be seen, the mound of blue-green powder had risen to cover them
from sight, and a fighting-machine, with its legs contracted,
crumpled, and abbreviated, stood across the corner of the pit. And
then, amid the clangour of the machinery, came a drifting suspicion
of human voices, that I entertained at first only to dismiss.

I crouched, watching this fighting-machine closely, satisfying
myself now for the first time that the hood did indeed contain a
Martian. As the green flames lifted I could see the oily gleam of
his integument and the brightness of his eyes. And suddenly I heard
a yell, and saw a long tentacle reaching over the shoulder of the
machine to the little cage that hunched upon its back. Then
something\dash{}something struggling violently\dash{}was lifted high against
the sky, a black, vague enigma against the starlight; and as this
black object came down again, I saw by the green brightness that it
was a man. For an instant he was clearly visible. He was a stout,
ruddy, middle-aged man, well dressed; three days before, he must
have been walking the world, a man of considerable consequence. I
could see his staring eyes and gleams of light on his studs and
watch chain. He vanished behind the mound, and for a moment there
was silence. And then began a shrieking and a sustained and
cheerful hooting from the Martians.

I slid down the rubbish, struggled to my feet, clapped my hands
over my ears, and bolted into the scullery. The curate, who had
been crouching silently with his arms over his head, looked up as I
passed, cried out quite loudly at my desertion of him, and came
running after me.

That night, as we lurked in the scullery, balanced between our
horror and the terrible fascination this peeping had, although I
felt an urgent need of action I tried in vain to conceive some plan
of escape; but afterwards, during the second day, I was able to
consider our position with great clearness. The curate, I found,
was quite incapable of discussion; this new and culminating
atrocity had robbed him of all vestiges of reason or forethought.
Practically he had already sunk to the level of an animal. But as
the saying goes, I gripped myself with both hands. It grew upon my
mind, once I could face the facts, that terrible as our position
was, there was as yet no justification for absolute despair. Our
chief chance lay in the possibility of the Martians making the pit
nothing more than a temporary encampment. Or even if they kept it
permanently, they might not consider it necessary to guard it, and
a chance of escape might be afforded us. I also weighed very
carefully the possibility of our digging a way out in a direction
away from the pit, but the chances of our emerging within sight of
some sentinel fighting-machine seemed at first too great. And I
should have had to do all the digging myself. The curate would
certainly have failed me.

It was on the third day, if my memory serves me right, that I saw
the lad killed. It was the only occasion on which I actually saw
the Martians feed. After that experience I avoided the hole in the
wall for the better part of a day. I went into the scullery,
removed the door, and spent some hours digging with my hatchet as
silently as possible; but when I had made a hole about a couple of
feet deep the loose earth collapsed noisily, and I did not dare
continue. I lost heart, and lay down on the scullery floor for a
long time, having no spirit even to move. And after that I
abandoned altogether the idea of escaping by excavation.

It says much for the impression the Martians had made upon me that
at first I entertained little or no hope of our escape being
brought about by their overthrow through any human effort. But on
the fourth or fifth night I heard a sound like heavy guns.

It was very late in the night, and the moon was shining brightly.
The Martians had taken away the excavating-machine, and, save for a
fighting-machine that stood in the remoter bank of the pit and a
handling-machine that was buried out of my sight in a corner of the
pit immediately beneath my peephole, the place was deserted by
them. Except for the pale glow from the handling-machine and the
bars and patches of white moonlight the pit was in darkness, and,
except for the clinking of the handling-machine, quite still. That
night was a beautiful serenity; save for one planet, the moon
seemed to have the sky to herself. I heard a dog howling, and that
familiar sound it was that made me listen. Then I heard quite
distinctly a booming exactly like the sound of great guns. Six
distinct reports I counted, and after a long interval six again.
And that was all.

\Chapter{CHAPTER FOUR\\THE DEATH OF THE CURATE}
It was on the sixth day of our imprisonment that I peeped for the
last time, and presently found myself alone. Instead of keeping
close to me and trying to oust me from the slit, the curate had
gone back into the scullery. I was struck by a sudden thought. I
went back quickly and quietly into the scullery. In the darkness I
heard the curate drinking. I snatched in the darkness, and my
fingers caught a bottle of burgundy.

For a few minutes there was a tussle. The bottle struck the floor
and broke, and I desisted and rose. We stood panting and
threatening each other. In the end I planted myself between him and
the food, and told him of my determination to begin a discipline. I
divided the food in the pantry, into rations to last us ten days. I
would not let him eat any more that day. In the afternoon he made a
feeble effort to get at the food. I had been dozing, but in an
instant I was awake. All day and all night we sat face to face, I
weary but resolute, and he weeping and complaining of his immediate
hunger. It was, I know, a night and a day, but to me it seemed\dash{}it
seems now\dash{}an interminable length of time.

And so our widened incompatibility ended at last in open conflict.
For two vast days we struggled in undertones and wrestling
contests. There were times when I beat and kicked him madly, times
when I cajoled and persuaded him, and once I tried to bribe him
with the last bottle of burgundy, for there was a rain-water pump
from which I could get water. But neither force nor kindness
availed; he was indeed beyond reason. He would neither desist from
his attacks on the food nor from his noisy babbling to himself. The
rudimentary precautions to keep our imprisonment endurable he would
not observe. Slowly I began to realise the complete overthrow of
his intelligence, to perceive that my sole companion in this close
and sickly darkness was a man insane.

From certain vague memories I am inclined to think my own mind
wandered at times. I had strange and hideous dreams whenever I
slept. It sounds paradoxical, but I am inclined to think that the
weakness and insanity of the curate warned me, braced me, and kept
me a sane man.

On the eighth day he began to talk aloud instead of whispering, and
nothing I could do would moderate his speech.

``It is just, O God!'' he would say, over and over again. ``It is
just. On me and mine be the punishment laid. We have sinned, we
have fallen short. There was poverty, sorrow; the poor were trodden
in the dust, and I held my peace. I preached acceptable folly\dash{}my
God, what folly!\dash{}when I should have stood up, though I died for
it, and called upon them to repent—repent! \ldots{} Oppressors of the
poor and needy \ldots{} ! The wine press of God!''

Then he would suddenly revert to the matter of the food I withheld
from him, praying, begging, weeping, at last threatening. He began
to raise his voice\dash{}I prayed him not to. He perceived a hold on
me\dash{}he threatened he would shout and bring the Martians upon us.
For a time that scared me; but any concession would have shortened
our chance of escape beyond estimating. I defied him, although I
felt no assurance that he might not do this thing. But that day, at
any rate, he did not. He talked with his voice rising slowly,
through the greater part of the eighth and ninth days\dash{}threats,
entreaties, mingled with a torrent of half-sane and always frothy
repentance for his vacant sham of God's service, such as made me
pity him. Then he slept awhile, and began again with renewed
strength, so loudly that I must needs make him desist.

``Be still!'' I implored.

He rose to his knees, for he had been sitting in the darkness near
the copper.

``I have been still too long,'' he said, in a tone that must have
reached the pit, ``and now I must bear my witness. Woe unto this
unfaithful city! Woe! Woe! Woe! Woe! Woe! To the inhabitants of the
earth by reason of the other voices of the trumpet\ldots{}''

``Shut up!'' I said, rising to my feet, and in a terror lest the
Martians should hear us. ``For God's sake\ldots{}''

``Nay,'' shouted the curate, at the top of his voice, standing
likewise and extending his arms. ``Speak! The word of the Lord is
upon me!''

In three strides he was at the door leading into the kitchen.

``I must bear my witness! I go! It has already been too long
delayed.''

I put out my hand and felt the meat chopper hanging to the wall. In
a flash I was after him. I was fierce with fear. Before he was
halfway across the kitchen I had overtaken him. With one last touch
of humanity I turned the blade back and struck him with the butt.
He went headlong forward and lay stretched on the ground. I
stumbled over him and stood panting. He lay still.

Suddenly I heard a noise without, the run and smash of slipping
plaster, and the triangular aperture in the wall was darkened. I
looked up and saw the lower surface of a handling-machine coming
slowly across the hole. One of its gripping limbs curled amid the
debris; another limb appeared, feeling its way over the fallen
beams. I stood petrified, staring. Then I saw through a sort of
glass plate near the edge of the body the face, as we may call it,
and the large dark eyes of a Martian, peering, and then a long
metallic snake of tentacle came feeling slowly through the hole.

I turned by an effort, stumbled over the curate, and stopped at the
scullery door. The tentacle was now some way, two yards or more, in
the room, and twisting and turning, with queer sudden movements,
this way and that. For a while I stood fascinated by that slow,
fitful advance. Then, with a faint, hoarse cry, I forced myself
across the scullery. I trembled violently; I could scarcely stand
upright. I opened the door of the coal cellar, and stood there in
the darkness staring at the faintly lit doorway into the kitchen,
and listening. Had the Martian seen me? What was it doing now?

Something was moving to and fro there, very quietly; every now and
then it tapped against the wall, or started on its movements with a
faint metallic ringing, like the movements of keys on a split-ring.
Then a heavy body\dash{}I knew too well what\dash{}was dragged across the
floor of the kitchen towards the opening. Irresistibly attracted, I
crept to the door and peeped into the kitchen. In the triangle of
bright outer sunlight I saw the Martian, in its Briareus of a
handling-machine, scrutinizing the curate's head. I thought at once
that it would infer my presence from the mark of the blow I had
given him.

I crept back to the coal cellar, shut the door, and began to cover
myself up as much as I could, and as noiselessly as possible in the
darkness, among the firewood and coal therein. Every now and then I
paused, rigid, to hear if the Martian had thrust its tentacles
through the opening again.

Then the faint metallic jingle returned. I traced it slowly feeling
over the kitchen. Presently I heard it nearer\dash{}in the scullery, as
I judged. I thought that its length might be insufficient to reach
me. I prayed copiously. It passed, scraping faintly across the
cellar door. An age of almost intolerable suspense intervened; then
I heard it fumbling at the latch! It had found the door! The
Martians understood doors!

It worried at the catch for a minute, perhaps, and then the door
opened.

In the darkness I could just see the thing\dash{}like an elephant's
trunk more than anything else\dash{}waving towards me and touching and
examining the wall, coals, wood and ceiling. It was like a black
worm swaying its blind head to and fro.

Once, even, it touched the heel of my boot. I was on the verge of
screaming; I bit my hand. For a time the tentacle was silent. I
could have fancied it had been withdrawn. Presently, with an abrupt
click, it gripped something\dash{}I thought it had me!\dash{}and seemed to go
out of the cellar again. For a minute I was not sure. Apparently it
had taken a lump of coal to examine.

I seized the opportunity of slightly shifting my position, which
had become cramped, and then listened. I whispered passionate
prayers for safety.

Then I heard the slow, deliberate sound creeping towards me again.
Slowly, slowly it drew near, scratching against the walls and
tapping the furniture.

While I was still doubtful, it rapped smartly against the cellar
door and closed it. I heard it go into the pantry, and the
biscuit-tins rattled and a bottle smashed, and then came a heavy
bump against the cellar door. Then silence that passed into an
infinity of suspense.

Had it gone?

At last I decided that it had.

It came into the scullery no more; but I lay all the tenth day in
the close darkness, buried among coals and firewood, not daring
even to crawl out for the drink for which I craved. It was the
eleventh day before I ventured so far from my security.

\Chapter{CHAPTER FIVE\\THE STILLNESS}
My first act before I went into the pantry was to fasten the door
between the kitchen and the scullery. But the pantry was empty;
every scrap of food had gone. Apparently, the Martian had taken it
all on the previous day. At that discovery I despaired for the
first time. I took no food, or no drink either, on the eleventh or
the twelfth day.

At first my mouth and throat were parched, and my strength ebbed
sensibly. I sat about in the darkness of the scullery, in a state
of despondent wretchedness. My mind ran on eating. I thought I had
become deaf, for the noises of movement I had been accustomed to
hear from the pit had ceased absolutely. I did not feel strong
enough to crawl noiselessly to the peephole, or I would have gone
there.

On the twelfth day my throat was so painful that, taking the chance
of alarming the Martians, I attacked the creaking rain-water pump
that stood by the sink, and got a couple of glassfuls of blackened
and tainted rain water. I was greatly refreshed by this, and
emboldened by the fact that no enquiring tentacle followed the
noise of my pumping.

During these days, in a rambling, inconclusive way, I thought much
of the curate and of the manner of his death.

On the thirteenth day I drank some more water, and dozed and
thought disjointedly of eating and of vague impossible plans of
escape. Whenever I dozed I dreamt of horrible phantasms, of the
death of the curate, or of sumptuous dinners; but, asleep or awake,
I felt a keen pain that urged me to drink again and again. The
light that came into the scullery was no longer grey, but red. To
my disordered imagination it seemed the colour of blood.

On the fourteenth day I went into the kitchen, and I was surprised
to find that the fronds of the red weed had grown right across the
hole in the wall, turning the half-light of the place into a
crimson-coloured obscurity.

It was early on the fifteenth day that I heard a curious, familiar
sequence of sounds in the kitchen, and, listening, identified it as
the snuffing and scratching of a dog. Going into the kitchen, I saw
a dog's nose peering in through a break among the ruddy fronds.
This greatly surprised me. At the scent of me he barked shortly.

I thought if I could induce him to come into the place quietly I
should be able, perhaps, to kill and eat him; and in any case, it
would be advisable to kill him, lest his actions attracted the
attention of the Martians.

I crept forward, saying ``Good dog!'' very softly; but he suddenly
withdrew his head and disappeared.

I listened\dash{}I was not deaf\dash{}but certainly the pit was still. I
heard a sound like the flutter of a bird's wings, and a hoarse
croaking, but that was all.

For a long while I lay close to the peephole, but not daring to
move aside the red plants that obscured it. Once or twice I heard a
faint pitter-patter like the feet of the dog going hither and
thither on the sand far below me, and there were more birdlike
sounds, but that was all. At length, encouraged by the silence, I
looked out.

Except in the corner, where a multitude of crows hopped and fought
over the skeletons of the dead the Martians had consumed, there was
not a living thing in the pit.

I stared about me, scarcely believing my eyes. All the machinery
had gone. Save for the big mound of greyish-blue powder in one
corner, certain bars of aluminium in another, the black birds, and
the skeletons of the killed, the place was merely an empty circular
pit in the sand.

Slowly I thrust myself out through the red weed, and stood upon the
mound of rubble. I could see in any direction save behind me, to
the north, and neither Martians nor sign of Martians were to be
seen. The pit dropped sheerly from my feet, but a little way along
the rubbish afforded a practicable slope to the summit of the
ruins. My chance of escape had come. I began to tremble.

I hesitated for some time, and then, in a gust of desperate
resolution, and with a heart that throbbed violently, I scrambled
to the top of the mound in which I had been buried so long.

I looked about again. To the northward, too, no Martian was
visible.

When I had last seen this part of Sheen in the daylight it had been
a straggling street of comfortable white and red houses,
interspersed with abundant shady trees. Now I stood on a mound of
smashed brickwork, clay, and gravel, over which spread a multitude
of red cactus-shaped plants, knee-high, without a solitary
terrestrial growth to dispute their footing. The trees near me were
dead and brown, but further a network of red thread scaled the
still living stems.

The neighbouring houses had all been wrecked, but none had been
burned; their walls stood, sometimes to the second story, with
smashed windows and shattered doors. The red weed grew tumultuously
in their roofless rooms. Below me was the great pit, with the crows
struggling for its refuse. A number of other birds hopped about
among the ruins. Far away I saw a gaunt cat slink crouchingly along
a wall, but traces of men there were none.

The day seemed, by contrast with my recent confinement, dazzlingly
bright, the sky a glowing blue. A gentle breeze kept the red weed
that covered every scrap of unoccupied ground gently swaying. And
oh! the sweetness of the air!

\Chapter{CHAPTER SIX\\THE WORK OF FIFTEEN DAYS}
For some time I stood tottering on the mound regardless of my
safety. Within that noisome den from which I had emerged I had
thought with a narrow intensity only of our immediate security. I
had not realised what had been happening to the world, had not
anticipated this startling vision of unfamiliar things. I had
expected to see Sheen in ruins\dash{}I found about me the landscape,
weird and lurid, of another planet.

For that moment I touched an emotion beyond the common range of
men, yet one that the poor brutes we dominate know only too well. I
felt as a rabbit might feel returning to his burrow and suddenly
confronted by the work of a dozen busy navvies digging the
foundations of a house. I felt the first inkling of a thing that
presently grew quite clear in my mind, that oppressed me for many
days, a sense of dethronement, a persuasion that I was no longer a
master, but an animal among the animals, under the Martian heel.
With us it would be as with them, to lurk and watch, to run and
hide; the fear and empire of man had passed away.

But so soon as this strangeness had been realised it passed, and my
dominant motive became the hunger of my long and dismal fast. In
the direction away from the pit I saw, beyond a red-covered wall, a
patch of garden ground unburied. This gave me a hint, and I went
knee-deep, and sometimes neck-deep, in the red weed. The density of
the weed gave me a reassuring sense of hiding. The wall was some
six feet high, and when I attempted to clamber it I found I could
not lift my feet to the crest. So I went along by the side of it,
and came to a corner and a rockwork that enabled me to get to the
top, and tumble into the garden I coveted. Here I found some young
onions, a couple of gladiolus bulbs, and a quantity of immature
carrots, all of which I secured, and, scrambling over a ruined
wall, went on my way through scarlet and crimson trees towards
Kew\dash{}it was like walking through an avenue of gigantic blood
drops\dash{}possessed with two ideas: to get more food, and to limp, as
soon and as far as my strength permitted, out of this accursed
unearthly region of the pit.

Some way farther, in a grassy place, was a group of mushrooms which
also I devoured, and then I came upon a brown sheet of flowing
shallow water, where meadows used to be. These fragments of
nourishment served only to whet my hunger. At first I was surprised
at this flood in a hot, dry summer, but afterwards I discovered
that it was caused by the tropical exuberance of the red weed.
Directly this extraordinary growth encountered water it straightway
became gigantic and of unparalleled fecundity. Its seeds were
simply poured down into the water of the Wey and Thames, and its
swiftly growing and Titanic water fronds speedily choked both those
rivers.

At Putney, as I afterwards saw, the bridge was almost lost in a
tangle of this weed, and at Richmond, too, the Thames water poured
in a broad and shallow stream across the meadows of Hampton and
Twickenham. As the water spread the weed followed them, until the
ruined villas of the Thames valley were for a time lost in this red
swamp, whose margin I explored, and much of the desolation the
Martians had caused was concealed.

In the end the red weed succumbed almost as quickly as it had
spread. A cankering disease, due, it is believed, to the action of
certain bacteria, presently seized upon it. Now by the action of
natural selection, all terrestrial plants have acquired a resisting
power against bacterial diseases\dash{}they never succumb without a
severe struggle, but the red weed rotted like a thing already dead.
The fronds became bleached, and then shrivelled and brittle. They
broke off at the least touch, and the waters that had stimulated
their early growth carried their last vestiges out to sea.

My first act on coming to this water was, of course, to slake my
thirst. I drank a great deal of it and, moved by an impulse, gnawed
some fronds of red weed; but they were watery, and had a sickly,
metallic taste. I found the water was sufficiently shallow for me
to wade securely, although the red weed impeded my feet a little;
but the flood evidently got deeper towards the river, and I turned
back to Mortlake. I managed to make out the road by means of
occasional ruins of its villas and fences and lamps, and so
presently I got out of this spate and made my way to the hill going
up towards Roehampton and came out on Putney Common.

Here the scenery changed from the strange and unfamiliar to the
wreckage of the familiar: patches of ground exhibited the
devastation of a cyclone, and in a few score yards I would come
upon perfectly undisturbed spaces, houses with their blinds trimly
drawn and doors closed, as if they had been left for a day by the
owners, or as if their inhabitants slept within. The red weed was
less abundant; the tall trees along the lane were free from the red
creeper. I hunted for food among the trees, finding nothing, and I
also raided a couple of silent houses, but they had already been
broken into and ransacked. I rested for the remainder of the
daylight in a shrubbery, being, in my enfeebled condition, too
fatigued to push on.

All this time I saw no human beings, and no signs of the Martians.
I encountered a couple of hungry-looking dogs, but both hurried
circuitously away from the advances I made them. Near Roehampton I
had seen two human skeletons\dash{}not bodies, but skeletons, picked
clean\dash{}and in the wood by me I found the crushed and scattered
bones of several cats and rabbits and the skull of a sheep. But
though I gnawed parts of these in my mouth, there was nothing to be
got from them.

After sunset I struggled on along the road towards Putney, where I
think the Heat-Ray must have been used for some reason. And in the
garden beyond Roehampton I got a quantity of immature potatoes,
sufficient to stay my hunger. From this garden one looked down upon
Putney and the river. The aspect of the place in the dusk was
singularly desolate: blackened trees, blackened, desolate ruins,
and down the hill the sheets of the flooded river, red-tinged with
the weed. And over all\dash{}silence. It filled me with indescribable
terror to think how swiftly that desolating change had come.

For a time I believed that mankind had been swept out of existence,
and that I stood there alone, the last man left alive. Hard by the
top of Putney Hill I came upon another skeleton, with the arms
dislocated and removed several yards from the rest of the body. As
I proceeded I became more and more convinced that the extermination
of mankind was, save for such stragglers as myself, already
accomplished in this part of the world. The Martians, I thought,
had gone on and left the country desolated, seeking food elsewhere.
Perhaps even now they were destroying Berlin or Paris, or it might
be they had gone northward.

\Chapter{CHAPTER SEVEN\\THE MAN ON PUTNEY HILL}
I spent that night in the inn that stands at the top of Putney
Hill, sleeping in a made bed for the first time since my flight to
Leatherhead. I will not tell the needless trouble I had breaking
into that house\dash{}afterwards I found the front door was on the
latch\dash{}nor how I ransacked every room for food, until just on the
verge of despair, in what seemed to me to be a servant's bedroom, I
found a rat-gnawed crust and two tins of pineapple. The place had
been already searched and emptied. In the bar I afterwards found
some biscuits and sandwiches that had been overlooked. The latter I
could not eat, they were too rotten, but the former not only stayed
my hunger, but filled my pockets. I lit no lamps, fearing some
Martian might come beating that part of London for food in the
night. Before I went to bed I had an interval of restlessness, and
prowled from window to window, peering out for some sign of these
monsters. I slept little. As I lay in bed I found myself thinking
consecutively\dash{}a thing I do not remember to have done since my last
argument with the curate. During all the intervening time my mental
condition had been a hurrying succession of vague emotional states
or a sort of stupid receptivity. But in the night my brain,
reinforced, I suppose, by the food I had eaten, grew clear again,
and I thought.

Three things struggled for possession of my mind: the killing of
the curate, the whereabouts of the Martians, and the possible fate
of my wife. The former gave me no sensation of horror or remorse to
recall; I saw it simply as a thing done, a memory infinitely
disagreeable but quite without the quality of remorse. I saw myself
then as I see myself now, driven step by step towards that hasty
blow, the creature of a sequence of accidents leading inevitably to
that. I felt no condemnation; yet the memory, static,
unprogressive, haunted me. In the silence of the night, with that
sense of the nearness of God that sometimes comes into the
stillness and the darkness, I stood my trial, my only trial, for
that moment of wrath and fear. I retraced every step of our
conversation from the moment when I had found him crouching beside
me, heedless of my thirst, and pointing to the fire and smoke that
streamed up from the ruins of Weybridge. We had been incapable of
co-operation\dash{}grim chance had taken no heed of that. Had I
foreseen, I should have left him at Halliford. But I did not
foresee; and crime is to foresee and do. And I set this down as I
have set all this story down, as it was. There were no
witnesses\dash{}all these things I might have concealed. But I set it
down, and the reader must form his judgment as he will.

And when, by an effort, I had set aside that picture of a prostrate
body, I faced the problem of the Martians and the fate of my wife.
For the former I had no data; I could imagine a hundred things, and
so, unhappily, I could for the latter. And suddenly that night
became terrible. I found myself sitting up in bed, staring at the
dark. I found myself praying that the Heat-Ray might have suddenly
and painlessly struck her out of being. Since the night of my
return from Leatherhead I had not prayed. I had uttered prayers,
fetish prayers, had prayed as heathens mutter charms when I was in
extremity; but now I prayed indeed, pleading steadfastly and
sanely, face to face with the darkness of God. Strange night!
Strangest in this, that so soon as dawn had come, I, who had talked
with God, crept out of the house like a rat leaving its hiding
place\dash{}a creature scarcely larger, an inferior animal, a thing that
for any passing whim of our masters might be hunted and killed.
Perhaps they also prayed confidently to God. Surely, if we have
learned nothing else, this war has taught us pity\dash{}pity for those
witless souls that suffer our dominion.

The morning was bright and fine, and the eastern sky glowed pink,
and was fretted with little golden clouds. In the road that runs
from the top of Putney Hill to Wimbledon was a number of poor
vestiges of the panic torrent that must have poured Londonward on
the Sunday night after the fighting began. There was a little
two-wheeled cart inscribed with the name of Thomas Lobb,
Greengrocer, New Malden, with a smashed wheel and an abandoned tin
trunk; there was a straw hat trampled into the now hardened mud,
and at the top of West Hill a lot of blood-stained glass about the
overturned water trough. My movements were languid, my plans of the
vaguest. I had an idea of going to Leatherhead, though I knew that
there I had the poorest chance of finding my wife. Certainly,
unless death had overtaken them suddenly, my cousins and she would
have fled thence; but it seemed to me I might find or learn there
whither the Surrey people had fled. I knew I wanted to find my
wife, that my heart ached for her and the world of men, but I had
no clear idea how the finding might be done. I was also sharply
aware now of my intense loneliness. From the corner I went, under
cover of a thicket of trees and bushes, to the edge of Wimbledon
Common, stretching wide and far.

That dark expanse was lit in patches by yellow gorse and broom;
there was no red weed to be seen, and as I prowled, hesitating, on
the verge of the open, the sun rose, flooding it all with light and
vitality. I came upon a busy swarm of little frogs in a swampy
place among the trees. I stopped to look at them, drawing a lesson
from their stout resolve to live. And presently, turning suddenly,
with an odd feeling of being watched, I beheld something crouching
amid a clump of bushes. I stood regarding this. I made a step
towards it, and it rose up and became a man armed with a cutlass. I
approached him slowly. He stood silent and motionless, regarding
me.

As I drew nearer I perceived he was dressed in clothes as dusty and
filthy as my own; he looked, indeed, as though he had been dragged
through a culvert. Nearer, I distinguished the green slime of
ditches mixing with the pale drab of dried clay and shiny, coaly
patches. His black hair fell over his eyes, and his face was dark
and dirty and sunken, so that at first I did not recognise him.
There was a red cut across the lower part of his face.

``Stop!'' he cried, when I was within ten yards of him, and I
stopped. His voice was hoarse. ``Where do you come from?'' he said.

I thought, surveying him.

``I come from Mortlake,'' I said. ``I was buried near the pit the
Martians made about their cylinder. I have worked my way out and
escaped.''

``There is no food about here,'' he said. ``This is my country. All
this hill down to the river, and back to Clapham, and up to the
edge of the common. There is only food for one. Which way are you
going?''

I answered slowly.

``I don't know,'' I said. ``I have been buried in the ruins of a house
thirteen or fourteen days. I don't know what has happened.''

He looked at me doubtfully, then started, and looked with a changed
expression.

``I've no wish to stop about here,'' said I. ``I think I shall go to
Leatherhead, for my wife was there.''

He shot out a pointing finger.

``It is you,'' said he; ``the man from Woking. And you weren't killed
at Weybridge?''

I recognised him at the same moment.

``You are the artilleryman who came into my garden.''

``Good luck!'' he said. ``We are lucky ones! Fancy \emph{you}!'' He put
out a hand, and I took it. ``I crawled up a drain,'' he said. ``But
they didn't kill everyone. And after they went away I got off
towards Walton across the fields. But\ldots{} It's not sixteen days
altogether\dash{}and your hair is grey.'' He looked over his shoulder
suddenly. ``Only a rook,'' he said. ``One gets to know that birds have
shadows these days. This is a bit open. Let us crawl under those
bushes and talk.''

``Have you seen any Martians?'' I said. ``Since I crawled out\ldots{}''

``They've gone away across London,'' he said. ``I guess they've got a
bigger camp there. Of a night, all over there, Hampstead way, the
sky is alive with their lights. It's like a great city, and in the
glare you can just see them moving. By daylight you can't. But
nearer\dash{}I haven't seen them\dash{}'' (he counted on his fingers) ``five
days. Then I saw a couple across Hammersmith way carrying something
big. And the night before last''\dash{}he stopped and spoke
impressively\dash{}``it was just a matter of lights, but it was something
up in the air. I believe they've built a flying-machine, and are
learning to fly.''

I stopped, on hands and knees, for we had come to the bushes.

``Fly!''

``Yes,'' he said, ``fly.''

I went on into a little bower, and sat down.

``It is all over with humanity,'' I said. ``If they can do that they
will simply go round the world.''

He nodded.

``They will. But\ldots{} It will relieve things over here a bit. And
besides\ldots{}'' He looked at me. ``Aren't you satisfied it \emph{is} up
with humanity? I am. We're down; we're beat.''

I stared. Strange as it may seem, I had not arrived at this fact\dash{}a
fact perfectly obvious so soon as he spoke. I had still held a
vague hope; rather, I had kept a lifelong habit of mind. He
repeated his words, ``We're beat.'' They carried absolute
conviction.

``It's all over,'' he said. ``They've lost \emph{one}\dash{}just
\emph{one}. And they've made their footing good and crippled the
greatest power in the world. They've walked over us. The death of
that one at Weybridge was an accident. And these are only pioneers.
They kept on coming. These green stars\dash{}I've seen none these five
or six days, but I've no doubt they're falling somewhere every
night. Nothing's to be done. We're under! We're beat!''

I made him no answer. I sat staring before me, trying in vain to
devise some countervailing thought.

``This isn't a war,'' said the artilleryman. ``It never was a war, any
more than there's war between man and ants.''

Suddenly I recalled the night in the observatory.

``After the tenth shot they fired no more\dash{}at least, until the first
cylinder came.''

``How do you know?'' said the artilleryman. I explained. He thought.
``Something wrong with the gun,'' he said. ``But what if there is?
They'll get it right again. And even if there's a delay, how can it
alter the end? It's just men and ants. There's the ants builds
their cities, live their lives, have wars, revolutions, until the
men want them out of the way, and then they go out of the way.
That's what we are now\dash{}just ants. Only\ldots{}''

``Yes,'' I said.

``We're eatable ants.''

We sat looking at each other.

``And what will they do with us?'' I said.

``That's what I've been thinking,'' he said; ``that's what I've been
thinking. After Weybridge I went south\dash{}thinking. I saw what was
up. Most of the people were hard at it squealing and exciting
themselves. But I'm not so fond of squealing. I've been in sight of
death once or twice; I'm not an ornamental soldier, and at the best
and worst, death\dash{}it's just death. And it's the man that keeps on
thinking comes through. I saw everyone tracking away south. Says I,
`Food won't last this way,' and I turned right back. I went for the
Martians like a sparrow goes for man. All round''\dash{}he waved a hand
to the horizon\dash{}``they're starving in heaps, bolting, treading on
each other. \ldots{}''

He saw my face, and halted awkwardly.

``No doubt lots who had money have gone away to France,'' he said. He
seemed to hesitate whether to apologise, met my eyes, and went on:
``There's food all about here. Canned things in shops; wines,
spirits, mineral waters; and the water mains and drains are empty.
Well, I was telling you what I was thinking. `Here's intelligent
things,' I said, `and it seems they want us for food. First,
they'll smash us up\dash{}ships, machines, guns, cities, all the order
and organisation. All that will go. If we were the size of ants we
might pull through. But we're not. It's all too bulky to stop.
That's the first certainty.' Eh?''

I assented.

``It is; I've thought it out. Very well, then\dash{}next; at present
we're caught as we're wanted. A Martian has only to go a few miles
to get a crowd on the run. And I saw one, one day, out by
Wandsworth, picking houses to pieces and routing among the
wreckage. But they won't keep on doing that. So soon as they've
settled all our guns and ships, and smashed our railways, and done
all the things they are doing over there, they will begin catching
us systematic, picking the best and storing us in cages and things.
That's what they will start doing in a bit. Lord! They haven't
begun on us yet. Don't you see that?''

``Not begun!'' I exclaimed.

``Not begun. All that's happened so far is through our not having
the sense to keep quiet\dash{}worrying them with guns and such foolery.
And losing our heads, and rushing off in crowds to where there
wasn't any more safety than where we were. They don't want to
bother us yet. They're making their things\dash{}making all the things
they couldn't bring with them, getting things ready for the rest of
their people. Very likely that's why the cylinders have stopped for
a bit, for fear of hitting those who are here. And instead of our
rushing about blind, on the howl, or getting dynamite on the chance
of busting them up, we've got to fix ourselves up according to the
new state of affairs. That's how I figure it out. It isn't quite
according to what a man wants for his species, but it's about what
the facts point to. And that's the principle I acted upon. Cities,
nations, civilisation, progress\dash{}it's all over. That game's up.
We're beat.''

``But if that is so, what is there to live for?''

The artilleryman looked at me for a moment.

``There won't be any more blessed concerts for a million years or
so; there won't be any Royal Academy of Arts, and no nice little
feeds at restaurants. If it's amusement you're after, I reckon the
game is up. If you've got any drawing-room manners or a dislike to
eating peas with a knife or dropping aitches, you'd better chuck
?em away. They ain't no further use.''

``You mean\ldots{}''

``I mean that men like me are going on living\dash{}for the sake of the
breed. I tell you, I'm grim set on living. And if I'm not mistaken,
you'll show what insides \emph{you've} got, too, before long. We
aren't going to be exterminated. And I don't mean to be caught
either, and tamed and fattened and bred like a thundering ox. Ugh!
Fancy those brown creepers!''

``You don't mean to say\ldots{}''

``I do. I'm going on, under their feet. I've got it planned; I've
thought it out. We men are beat. We don't know enough. We've got to
learn before we've got a chance. And we've got to live and keep
independent while we learn. See! That's what has to be done.''

I stared, astonished, and stirred profoundly by the man's
resolution.

``Great God!'' cried I. ``But you are a man indeed!'' And suddenly I
gripped his hand.

``Eh!'' he said, with his eyes shining. ``I've thought it out, eh?''

``Go on,'' I said.

``Well, those who mean to escape their catching must get ready. I'm
getting ready. Mind you, it isn't all of us that are made for wild
beasts; and that's what it's got to be. That's why I watched you. I
had my doubts. You're slender. I didn't know that it was you, you
see, or just how you'd been buried. All these\dash{}the sort of people
that lived in these houses, and all those damn little clerks that
used to live down that way\dash{}they'd be no good. They haven't any
spirit in them\dash{}no proud dreams and no proud lusts; and a man who
hasn't one or the other\dash{}Lord! What is he but funk and precautions?
They just used to skedaddle off to work\dash{}I've seen hundreds of 'em,
bit of breakfast in hand, running wild and shining to catch their
little season-ticket train, for fear they'd get dismissed if they
didn't; working at businesses they were afraid to take the trouble
to understand; skedaddling back for fear they wouldn't be in time
for dinner; keeping indoors after dinner for fear of the back
streets, and sleeping with the wives they married, not because they
wanted them, but because they had a bit of money that would make
for safety in their one little miserable skedaddle through the
world. Lives insured and a bit invested for fear of accidents. And
on Sundays\dash{}fear of the hereafter. As if hell was built for
rabbits! Well, the Martians will just be a godsend to these. Nice
roomy cages, fattening food, careful breeding, no worry. After a
week or so chasing about the fields and lands on empty stomachs,
they'll come and be caught cheerful. They'll be quite glad after a
bit. They'll wonder what people did before there were Martians to
take care of them. And the bar loafers, and mashers, and singers\dash{}I
can imagine them. I can imagine them,'' he said, with a sort of
sombre gratification. ``There'll be any amount of sentiment and
religion loose among them. There's hundreds of things I saw with my
eyes that I've only begun to see clearly these last few days.
There's lots will take things as they are\dash{}fat and stupid; and lots
will be worried by a sort of feeling that it's all wrong, and that
they ought to be doing something. Now whenever things are so that a
lot of people feel they ought to be doing something, the weak, and
those who go weak with a lot of complicated thinking, always make
for a sort of do-nothing religion, very pious and superior, and
submit to persecution and the will of the Lord. Very likely you've
seen the same thing. It's energy in a gale of funk, and turned
clean inside out. These cages will be full of psalms and hymns and
piety. And those of a less simple sort will work in a bit of\dash{}what
is it?\dash{}eroticism.''

He paused.

``Very likely these Martians will make pets of some of them; train
them to do tricks\dash{}who knows?\dash{}get sentimental over the pet boy who
grew up and had to be killed. And some, maybe, they will train to
hunt us.''

``No,'' I cried, ``that's impossible! No human being\ldots{}''

``What's the good of going on with such lies?'' said the
artilleryman. ``There's men who'd do it cheerful. What nonsense to
pretend there isn't!''

And I succumbed to his conviction.

``If they come after me,'' he said; ``Lord, if they come after me!''
and subsided into a grim meditation.

I sat contemplating these things. I could find nothing to bring
against this man's reasoning. In the days before the invasion no
one would have questioned my intellectual superiority to his\dash{}I, a
professed and recognised writer on philosophical themes, and he, a
common soldier; and yet he had already formulated a situation that
I had scarcely realised.

``What are you doing?'' I said presently. ``What plans have you
made?''

He hesitated.

``Well, it's like this,'' he said. ``What have we to do? We have to
invent a sort of life where men can live and breed, and be
sufficiently secure to bring the children up. Yes\dash{}wait a bit, and
I'll make it clearer what I think ought to be done. The tame ones
will go like all tame beasts; in a few generations they'll be big,
beautiful, rich-blooded, stupid\dash{}rubbish! The risk is that we who
keep wild will go savage\dash{}degenerate into a sort of big, savage
rat. \ldots{} You see, how I mean to live is underground. I've been
thinking about the drains. Of course those who don't know drains
think horrible things; but under this London are miles and
miles\dash{}hundreds of miles\dash{}and a few days rain and London empty will
leave them sweet and clean. The main drains are big enough and airy
enough for anyone. Then there's cellars, vaults, stores, from which
bolting passages may be made to the drains. And the railway tunnels
and subways. Eh? You begin to see? And we form a band\dash{}able-bodied,
clean-minded men. We're not going to pick up any rubbish that
drifts in. Weaklings go out again.''

``As you meant me to go?''

``Well\dash{}I parleyed, didn't I?''

``We won't quarrel about that. Go on.''

``Those who stop obey orders. Able-bodied, clean-minded women we
want also\dash{}mothers and teachers. No lackadaisical ladies\dash{}no
blasted rolling eyes. We can't have any weak or silly. Life is real
again, and the useless and cumbersome and mischievous have to die.
They ought to die. They ought to be willing to die. It's a sort of
disloyalty, after all, to live and taint the race. And they can't
be happy. Moreover, dying's none so dreadful; it's the funking
makes it bad. And in all those places we shall gather. Our district
will be London. And we may even be able to keep a watch, and run
about in the open when the Martians keep away. Play cricket,
perhaps. That's how we shall save the race. Eh? It's a possible
thing? But saving the race is nothing in itself. As I say, that's
only being rats. It's saving our knowledge and adding to it is the
thing. There men like you come in. There's books, there's models.
We must make great safe places down deep, and get all the books we
can; not novels and poetry swipes, but ideas, science books. That's
where men like you come in. We must go to the British Museum and
pick all those books through. Especially we must keep up our
science\dash{}learn more. We must watch these Martians. Some of us must
go as spies. When it's all working, perhaps I will. Get caught, I
mean. And the great thing is, we must leave the Martians alone. We
mustn't even steal. If we get in their way, we clear out. We must
show them we mean no harm. Yes, I know. But they're intelligent
things, and they won't hunt us down if they have all they want, and
think we're just harmless vermin.''

The artilleryman paused and laid a brown hand upon my arm.

``After all, it may not be so much we may have to learn before\dash{}Just
imagine this: four or five of their fighting machines suddenly
starting off\dash{}Heat-Rays right and left, and not a Martian in 'em.
Not a Martian in 'em, but men\dash{}men who have learned the way how. It
may be in my time, even\dash{}those men. Fancy having one of them lovely
things, with its Heat-Ray wide and free! Fancy having it in
control! What would it matter if you smashed to smithereens at the
end of the run, after a bust like that? I reckon the Martians'll
open their beautiful eyes! Can't you see them, man? Can't you see
them hurrying, hurrying\dash{}puffing and blowing and hooting to their
other mechanical affairs? Something out of gear in every case. And
swish, bang, rattle, swish! Just as they are fumbling over it,
\emph{swish} comes the Heat-Ray, and, behold! man has come back to
his own.''

For a while the imaginative daring of the artilleryman, and the
tone of assurance and courage he assumed, completely dominated my
mind. I believed unhesitatingly both in his forecast of human
destiny and in the practicability of his astonishing scheme, and
the reader who thinks me susceptible and foolish must contrast his
position, reading steadily with all his thoughts about his subject,
and mine, crouching fearfully in the bushes and listening,
distracted by apprehension. We talked in this manner through the
early morning time, and later crept out of the bushes, and, after
scanning the sky for Martians, hurried precipitately to the house
on Putney Hill where he had made his lair. It was the coal cellar
of the place, and when I saw the work he had spent a week upon\dash{}it
was a burrow scarcely ten yards long, which he designed to reach to
the main drain on Putney Hill\dash{}I had my first inkling of the gulf
between his dreams and his powers. Such a hole I could have dug in
a day. But I believed in him sufficiently to work with him all that
morning until past midday at his digging. We had a garden barrow
and shot the earth we removed against the kitchen range. We
refreshed ourselves with a tin of mock-turtle soup and wine from
the neighbouring pantry. I found a curious relief from the aching
strangeness of the world in this steady labour. As we worked, I
turned his project over in my mind, and presently objections and
doubts began to arise; but I worked there all the morning, so glad
was I to find myself with a purpose again. After working an hour I
began to speculate on the distance one had to go before the cloaca
was reached, the chances we had of missing it altogether. My
immediate trouble was why we should dig this long tunnel, when it
was possible to get into the drain at once down one of the
manholes, and work back to the house. It seemed to me, too, that
the house was inconveniently chosen, and required a needless length
of tunnel. And just as I was beginning to face these things, the
artilleryman stopped digging, and looked at me.

``We're working well,'' he said. He put down his spade. ``Let us knock
off a bit'' he said. ``I think it's time we reconnoitred from the
roof of the house.''

I was for going on, and after a little hesitation he resumed his
spade; and then suddenly I was struck by a thought. I stopped, and
so did he at once.

``Why were you walking about the common,'' I said, ``instead of being
here?''

``Taking the air,'' he said. ``I was coming back. It's safer by
night.''

``But the work?''

``Oh, one can't always work,'' he said, and in a flash I saw the man
plain. He hesitated, holding his spade. ``We ought to reconnoitre
now,'' he said, ``because if any come near they may hear the spades
and drop upon us unawares.''

I was no longer disposed to object. We went together to the roof
and stood on a ladder peeping out of the roof door. No Martians
were to be seen, and we ventured out on the tiles, and slipped down
under shelter of the parapet.

From this position a shrubbery hid the greater portion of Putney,
but we could see the river below, a bubbly mass of red weed, and
the low parts of Lambeth flooded and red. The red creeper swarmed
up the trees about the old palace, and their branches stretched
gaunt and dead, and set with shrivelled leaves, from amid its
clusters. It was strange how entirely dependent both these things
were upon flowing water for their propagation. About us neither had
gained a footing; laburnums, pink mays, snowballs, and trees of
arbor-vitae, rose out of laurels and hydrangeas, green and
brilliant into the sunlight. Beyond Kensington dense smoke was
rising, and that and a blue haze hid the northward hills.

The artilleryman began to tell me of the sort of people who still
remained in London.

``One night last week,'' he said, ``some fools got the electric light
in order, and there was all Regent Street and the Circus ablaze,
crowded with painted and ragged drunkards, men and women, dancing
and shouting till dawn. A man who was there told me. And as the day
came they became aware of a fighting-machine standing near by the
Langham and looking down at them. Heaven knows how long he had been
there. It must have given some of them a nasty turn. He came down
the road towards them, and picked up nearly a hundred too drunk or
frightened to run away.''

Grotesque gleam of a time no history will ever fully describe!

From that, in answer to my questions, he came round to his
grandiose plans again. He grew enthusiastic. He talked so
eloquently of the possibility of capturing a fighting-machine that
I more than half believed in him again. But now that I was
beginning to understand something of his quality, I could divine
the stress he laid on doing nothing precipitately. And I noted that
now there was no question that he personally was to capture and
fight the great machine.

After a time we went down to the cellar. Neither of us seemed
disposed to resume digging, and when he suggested a meal, I was
nothing loath. He became suddenly very generous, and when we had
eaten he went away and returned with some excellent cigars. We lit
these, and his optimism glowed. He was inclined to regard my coming
as a great occasion.

``There's some champagne in the cellar,'' he said.

``We can dig better on this Thames-side burgundy,'' said I.

``No,'' said he; ``I am host today. Champagne! Great God! We've a
heavy enough task before us! Let us take a rest and gather strength
while we may. Look at these blistered hands!''

And pursuant to this idea of a holiday, he insisted upon playing
cards after we had eaten. He taught me euchre, and after dividing
London between us, I taking the northern side and he the southern,
we played for parish points. Grotesque and foolish as this will
seem to the sober reader, it is absolutely true, and what is more
remarkable, I found the card game and several others we played
extremely interesting.

Strange mind of man! that, with our species upon the edge of
extermination or appalling degradation, with no clear prospect
before us but the chance of a horrible death, we could sit
following the chance of this painted pasteboard, and playing the
``joker'' with vivid delight. Afterwards he taught me poker, and I
beat him at three tough chess games. When dark came we decided to
take the risk, and lit a lamp.

After an interminable string of games, we supped, and the
artilleryman finished the champagne. We went on smoking the cigars.
He was no longer the energetic regenerator of his species I had
encountered in the morning. He was still optimistic, but it was a
less kinetic, a more thoughtful optimism. I remember he wound up
with my health, proposed in a speech of small variety and
considerable intermittence. I took a cigar, and went upstairs to
look at the lights of which he had spoken that blazed so greenly
along the Highgate hills.

At first I stared unintelligently across the London valley. The
northern hills were shrouded in darkness; the fires near Kensington
glowed redly, and now and then an orange-red tongue of flame
flashed up and vanished in the deep blue night. All the rest of
London was black. Then, nearer, I perceived a strange light, a
pale, violet-purple fluorescent glow, quivering under the night
breeze. For a space I could not understand it, and then I knew that
it must be the red weed from which this faint irradiation
proceeded. With that realisation my dormant sense of wonder, my
sense of the proportion of things, awoke again. I glanced from that
to Mars, red and clear, glowing high in the west, and then gazed
long and earnestly at the darkness of Hampstead and Highgate.

I remained a very long time upon the roof, wondering at the
grotesque changes of the day. I recalled my mental states from the
midnight prayer to the foolish card-playing. I had a violent
revulsion of feeling. I remember I flung away the cigar with a
certain wasteful symbolism. My folly came to me with glaring
exaggeration. I seemed a traitor to my wife and to my kind; I was
filled with remorse. I resolved to leave this strange undisciplined
dreamer of great things to his drink and gluttony, and to go on
into London. There, it seemed to me, I had the best chance of
learning what the Martians and my fellowmen were doing. I was still
upon the roof when the late moon rose.

\Chapter{CHAPTER EIGHT\\DEAD LONDON}
After I had parted from the artilleryman, I went down the hill, and
by the High Street across the bridge to Fulham. The red weed was
tumultuous at that time, and nearly choked the bridge roadway; but
its fronds were already whitened in patches by the spreading
disease that presently removed it so swiftly.

At the corner of the lane that runs to Putney Bridge station I
found a man lying. He was as black as a sweep with the black dust,
alive, but helplessly and speechlessly drunk. I could get nothing
from him but curses and furious lunges at my head. I think I should
have stayed by him but for the brutal expression of his face.

There was black dust along the roadway from the bridge onwards, and
it grew thicker in Fulham. The streets were horribly quiet. I got
food\dash{}sour, hard, and mouldy, but quite eatable\dash{}in a baker's shop
here. Some way towards Walham Green the streets became clear of
powder, and I passed a white terrace of houses on fire; the noise
of the burning was an absolute relief. Going on towards Brompton,
the streets were quiet again.

Here I came once more upon the black powder in the streets and upon
dead bodies. I saw altogether about a dozen in the length of the
Fulham Road. They had been dead many days, so that I hurried
quickly past them. The black powder covered them over, and softened
their outlines. One or two had been disturbed by dogs.

Where there was no black powder, it was curiously like a Sunday in
the City, with the closed shops, the houses locked up and the
blinds drawn, the desertion, and the stillness. In some places
plunderers had been at work, but rarely at other than the provision
and wine shops. A jeweller's window had been broken open in one
place, but apparently the thief had been disturbed, and a number of
gold chains and a watch lay scattered on the pavement. I did not
trouble to touch them. Farther on was a tattered woman in a heap on
a doorstep; the hand that hung over her knee was gashed and bled
down her rusty brown dress, and a smashed magnum of champagne
formed a pool across the pavement. She seemed asleep, but she was
dead.

The farther I penetrated into London, the profounder grew the
stillness. But it was not so much the stillness of death\dash{}it was
the stillness of suspense, of expectation. At any time the
destruction that had already singed the northwestern borders of the
metropolis, and had annihilated Ealing and Kilburn, might strike
among these houses and leave them smoking ruins. It was a city
condemned and derelict. \ldots{}

In South Kensington the streets were clear of dead and of black
powder. It was near South Kensington that I first heard the
howling. It crept almost imperceptibly upon my senses. It was a
sobbing alternation of two notes, ``Ulla, ulla, ulla, ulla,'' keeping
on perpetually. When I passed streets that ran northward it grew in
volume, and houses and buildings seemed to deaden and cut it off
again. It came in a full tide down Exhibition Road. I stopped,
staring towards Kensington Gardens, wondering at this strange,
remote wailing. It was as if that mighty desert of houses had found
a voice for its fear and solitude.

``Ulla, ulla, ulla, ulla,'' wailed that superhuman note\dash{}great waves
of sound sweeping down the broad, sunlit roadway, between the tall
buildings on each side. I turned northwards, marvelling, towards
the iron gates of Hyde Park. I had half a mind to break into the
Natural History Museum and find my way up to the summits of the
towers, in order to see across the park. But I decided to keep to
the ground, where quick hiding was possible, and so went on up the
Exhibition Road. All the large mansions on each side of the road
were empty and still, and my footsteps echoed against the sides of
the houses. At the top, near the park gate, I came upon a strange
sight\dash{}a bus overturned, and the skeleton of a horse picked clean.
I puzzled over this for a time, and then went on to the bridge over
the Serpentine. The voice grew stronger and stronger, though I
could see nothing above the housetops on the north side of the
park, save a haze of smoke to the northwest.

``Ulla, ulla, ulla, ulla,'' cried the voice, coming, as it seemed to
me, from the district about Regent's Park. The desolating cry
worked upon my mind. The mood that had sustained me passed. The
wailing took possession of me. I found I was intensely weary,
footsore, and now again hungry and thirsty.

It was already past noon. Why was I wandering alone in this city of
the dead? Why was I alone when all London was lying in state, and
in its black shroud? I felt intolerably lonely. My mind ran on old
friends that I had forgotten for years. I thought of the poisons in
the chemists' shops, of the liquors the wine merchants stored; I
recalled the two sodden creatures of despair, who so far as I knew,
shared the city with myself. \ldots{}

I came into Oxford Street by the Marble Arch, and here again were
black powder and several bodies, and an evil, ominous smell from
the gratings of the cellars of some of the houses. I grew very
thirsty after the heat of my long walk. With infinite trouble I
managed to break into a public-house and get food and drink. I was
weary after eating, and went into the parlour behind the bar, and
slept on a black horsehair sofa I found there.

I awoke to find that dismal howling still in my ears, ``Ulla, ulla,
ulla, ulla.'' It was now dusk, and after I had routed out some
biscuits and a cheese in the bar\dash{}there was a meat safe, but it
contained nothing but maggots\dash{}I wandered on through the silent
residential squares to Baker Street\dash{}Portman Square is the only one
I can name\dash{}and so came out at last upon Regent's Park. And as I
emerged from the top of Baker Street, I saw far away over the trees
in the clearness of the sunset the hood of the Martian giant from
which this howling proceeded. I was not terrified. I came upon him
as if it were a matter of course. I watched him for some time, but
he did not move. He appeared to be standing and yelling, for no
reason that I could discover.

I tried to formulate a plan of action. That perpetual sound of
``Ulla, ulla, ulla, ulla,'' confused my mind. Perhaps I was too tired
to be very fearful. Certainly I was more curious to know the reason
of this monotonous crying than afraid. I turned back away from the
park and struck into Park Road, intending to skirt the park, went
along under the shelter of the terraces, and got a view of this
stationary, howling Martian from the direction of St.\ John's Wood.
A couple of hundred yards out of Baker Street I heard a yelping
chorus, and saw, first a dog with a piece of putrescent red meat in
his jaws coming headlong towards me, and then a pack of starving
mongrels in pursuit of him. He made a wide curve to avoid me, as
though he feared I might prove a fresh competitor. As the yelping
died away down the silent road, the wailing sound of ``Ulla, ulla,
ulla, ulla,'' reasserted itself.

I came upon the wrecked handling-machine halfway to St.\ John's Wood
station. At first I thought a house had fallen across the road. It
was only as I clambered among the ruins that I saw, with a start,
this mechanical Samson lying, with its tentacles bent and smashed
and twisted, among the ruins it had made. The forepart was
shattered. It seemed as if it had driven blindly straight at the
house, and had been overwhelmed in its overthrow. It seemed to me
then that this might have happened by a handling-machine escaping
from the guidance of its Martian. I could not clamber among the
ruins to see it, and the twilight was now so far advanced that the
blood with which its seat was smeared, and the gnawed gristle of
the Martian that the dogs had left, were invisible to me.

Wondering still more at all that I had seen, I pushed on towards
Primrose Hill. Far away, through a gap in the trees, I saw a second
Martian, as motionless as the first, standing in the park towards
the Zoological Gardens, and silent. A little beyond the ruins about
the smashed handling-machine I came upon the red weed again, and
found the Regent's Canal, a spongy mass of dark-red vegetation.

As I crossed the bridge, the sound of ``Ulla, ulla, ulla, ulla,''
ceased. It was, as it were, cut off. The silence came like a
thunderclap.

The dusky houses about me stood faint and tall and dim; the trees
towards the park were growing black. All about me the red weed
clambered among the ruins, writhing to get above me in the dimness.
Night, the mother of fear and mystery, was coming upon me. But
while that voice sounded the solitude, the desolation, had been
endurable; by virtue of it London had still seemed alive, and the
sense of life about me had upheld me. Then suddenly a change, the
passing of something\dash{}I knew not what\dash{}and then a stillness that
could be felt. Nothing but this gaunt quiet.

London about me gazed at me spectrally. The windows in the white
houses were like the eye sockets of skulls. About me my imagination
found a thousand noiseless enemies moving. Terror seized me, a
horror of my temerity. In front of me the road became pitchy black
as though it was tarred, and I saw a contorted shape lying across
the pathway. I could not bring myself to go on. I turned down
St.\ John's Wood Road, and ran headlong from this unendurable stillness
towards Kilburn. I hid from the night and the silence, until long
after midnight, in a cabmen's shelter in Harrow Road. But before
the dawn my courage returned, and while the stars were still in the
sky I turned once more towards Regent's Park. I missed my way among
the streets, and presently saw down a long avenue, in the
half-light of the early dawn, the curve of Primrose Hill. On the
summit, towering up to the fading stars, was a third Martian, erect
and motionless like the others.

An insane resolve possessed me. I would die and end it. And I would
save myself even the trouble of killing myself. I marched on
recklessly towards this Titan, and then, as I drew nearer and the
light grew, I saw that a multitude of black birds was circling and
clustering about the hood. At that my heart gave a bound, and I
began running along the road.

I hurried through the red weed that choked St.\ Edmund's Terrace (I
waded breast-high across a torrent of water that was rushing down
from the waterworks towards the Albert Road), and emerged upon the
grass before the rising of the sun. Great mounds had been heaped
about the crest of the hill, making a huge redoubt of it\dash{}it was
the final and largest place the Martians had made\dash{}and from behind
these heaps there rose a thin smoke against the sky. Against the
sky line an eager dog ran and disappeared. The thought that had
flashed into my mind grew real, grew credible. I felt no fear, only
a wild, trembling exultation, as I ran up the hill towards the
motionless monster. Out of the hood hung lank shreds of brown, at
which the hungry birds pecked and tore.

In another moment I had scrambled up the earthen rampart and stood
upon its crest, and the interior of the redoubt was below me. A
mighty space it was, with gigantic machines here and there within
it, huge mounds of material and strange shelter places. And
scattered about it, some in their overturned war-machines, some in
the now rigid handling-machines, and a dozen of them stark and
silent and laid in a row, were the Martians\dash{}\emph{dead}!\dash{}slain by
the putrefactive and disease bacteria against which their systems
were unprepared; slain as the red weed was being slain; slain,
after all man's devices had failed, by the humblest things that
God, in his wisdom, has put upon this earth.

For so it had come about, as indeed I and many men might have
foreseen had not terror and disaster blinded our minds. These germs
of disease have taken toll of humanity since the beginning of
things\dash{}taken toll of our prehuman ancestors since life began here.
But by virtue of this natural selection of our kind we have
developed resisting power; to no germs do we succumb without a
struggle, and to many\dash{}those that cause putrefaction in dead
matter, for instance\dash{}our living frames are altogether immune. But
there are no bacteria in Mars, and directly these invaders arrived,
directly they drank and fed, our microscopic allies began to work
their overthrow. Already when I watched them they were irrevocably
doomed, dying and rotting even as they went to and fro. It was
inevitable. By the toll of a billion deaths man has bought his
birthright of the earth, and it is his against all comers; it would
still be his were the Martians ten times as mighty as they are. For
neither do men live nor die in vain.

Here and there they were scattered, nearly fifty altogether, in
that great gulf they had made, overtaken by a death that must have
seemed to them as incomprehensible as any death could be. To me
also at that time this death was incomprehensible. All I knew was
that these things that had been alive and so terrible to men were
dead. For a moment I believed that the destruction of Sennacherib
had been repeated, that God had repented, that the Angel of Death
had slain them in the night.

I stood staring into the pit, and my heart lightened gloriously,
even as the rising sun struck the world to fire about me with his
rays. The pit was still in darkness; the mighty engines, so great
and wonderful in their power and complexity, so unearthly in their
tortuous forms, rose weird and vague and strange out of the shadows
towards the light. A multitude of dogs, I could hear, fought over
the bodies that lay darkly in the depth of the pit, far below me.
Across the pit on its farther lip, flat and vast and strange, lay
the great flying-machine with which they had been experimenting
upon our denser atmosphere when decay and death arrested them.
Death had come not a day too soon. At the sound of a cawing
overhead I looked up at the huge fighting-machine that would fight
no more for ever, at the tattered red shreds of flesh that dripped
down upon the overturned seats on the summit of Primrose Hill.

I turned and looked down the slope of the hill to where, enhaloed
now in birds, stood those other two Martians that I had seen
overnight, just as death had overtaken them. The one had died, even
as it had been crying to its companions; perhaps it was the last to
die, and its voice had gone on perpetually until the force of its
machinery was exhausted. They glittered now, harmless tripod towers
of shining metal, in the brightness of the rising sun.

All about the pit, and saved as by a miracle from everlasting
destruction, stretched the great Mother of Cities. Those who have
only seen London veiled in her sombre robes of smoke can scarcely
imagine the naked clearness and beauty of the silent wilderness of
houses.

Eastward, over the blackened ruins of the Albert Terrace and the
splintered spire of the church, the sun blazed dazzling in a clear
sky, and here and there some facet in the great wilderness of roofs
caught the light and glared with a white intensity.

Northward were Kilburn and Hampsted, blue and crowded with houses;
westward the great city was dimmed; and southward, beyond the
Martians, the green waves of Regent's Park, the Langham Hotel, the
dome of the Albert Hall, the Imperial Institute, and the giant
mansions of the Brompton Road came out clear and little in the
sunrise, the jagged ruins of Westminster rising hazily beyond. Far
away and blue were the Surrey hills, and the towers of the Crystal
Palace glittered like two silver rods. The dome of St.\ Paul's was
dark against the sunrise, and injured, I saw for the first time, by
a huge gaping cavity on its western side.

And as I looked at this wide expanse of houses and factories and
churches, silent and abandoned; as I thought of the multitudinous
hopes and efforts, the innumerable hosts of lives that had gone to
build this human reef, and of the swift and ruthless destruction
that had hung over it all; when I realised that the shadow had been
rolled back, and that men might still live in the streets, and this
dear vast dead city of mine be once more alive and powerful, I felt
a wave of emotion that was near akin to tears.

The torment was over. Even that day the healing would begin. The
survivors of the people scattered over the country\dash{}leaderless,
lawless, foodless, like sheep without a shepherd\dash{}the thousands who
had fled by sea, would begin to return; the pulse of life, growing
stronger and stronger, would beat again in the empty streets and
pour across the vacant squares. Whatever destruction was done, the
hand of the destroyer was stayed. All the gaunt wrecks, the
blackened skeletons of houses that stared so dismally at the sunlit
grass of the hill, would presently be echoing with the hammers of
the restorers and ringing with the tapping of their trowels. At the
thought I extended my hands towards the sky and began thanking God.
In a year, thought I\dash{}in a year\ldots{}

With overwhelming force came the thought of myself, of my wife, and
the old life of hope and tender helpfulness that had ceased for
ever.

\Chapter{CHAPTER NINE\\WRECKAGE}
And now comes the strangest thing in my story. Yet, perhaps, it is
not altogether strange. I remember, clearly and coldly and vividly,
all that I did that day until the time that I stood weeping and
praising God upon the summit of Primrose Hill. And then I forget.

Of the next three days I know nothing. I have learned since that,
so far from my being the first discoverer of the Martian overthrow,
several such wanderers as myself had already discovered this on the
previous night. One man\dash{}the first\dash{}had gone to St.\ Martin's-le-Grand,
and, while I sheltered in the cabmen's hut, had
contrived to telegraph to Paris. Thence the joyful news had flashed
all over the world; a thousand cities, chilled by ghastly
apprehensions, suddenly flashed into frantic illuminations; they
knew of it in Dublin, Edinburgh, Manchester, Birmingham, at the
time when I stood upon the verge of the pit. Already men, weeping
with joy, as I have heard, shouting and staying their work to shake
hands and shout, were making up trains, even as near as Crewe, to
descend upon London. The church bells that had ceased a fortnight
since suddenly caught the news, until all England was bell-ringing.
Men on cycles, lean-faced, unkempt, scorched along every country
lane shouting of unhoped deliverance, shouting to gaunt, staring
figures of despair. And for the food! Across the Channel, across
the Irish Sea, across the Atlantic, corn, bread, and meat were
tearing to our relief. All the shipping in the world seemed going
Londonward in those days. But of all this I have no memory. I
drifted\dash{}a demented man. I found myself in a house of kindly
people, who had found me on the third day wandering, weeping, and
raving through the streets of St.\ John's Wood. They have told me
since that I was singing some insane doggerel about ``The Last Man
Left Alive! Hurrah! The Last Man Left Alive!'' Troubled as they were
with their own affairs, these people, whose name, much as I would
like to express my gratitude to them, I may not even give here,
nevertheless cumbered themselves with me, sheltered me, and
protected me from myself. Apparently they had learned something of
my story from me during the days of my lapse.

Very gently, when my mind was assured again, did they break to me
what they had learned of the fate of Leatherhead. Two days after I
was imprisoned it had been destroyed, with every soul in it, by a
Martian. He had swept it out of existence, as it seemed, without
any provocation, as a boy might crush an ant hill, in the mere
wantonness of power.

I was a lonely man, and they were very kind to me. I was a lonely
man and a sad one, and they bore with me. I remained with them four
days after my recovery. All that time I felt a vague, a growing
craving to look once more on whatever remained of the little life
that seemed so happy and bright in my past. It was a mere hopeless
desire to feast upon my misery. They dissuaded me. They did all
they could to divert me from this morbidity. But at last I could
resist the impulse no longer, and, promising faithfully to return
to them, and parting, as I will confess, from these four-day
friends with tears, I went out again into the streets that had
lately been so dark and strange and empty.

Already they were busy with returning people; in places even there
were shops open, and I saw a drinking fountain running water.

I remember how mockingly bright the day seemed as I went back on my
melancholy pilgrimage to the little house at Woking, how busy the
streets and vivid the moving life about me. So many people were
abroad everywhere, busied in a thousand activities, that it seemed
incredible that any great proportion of the population could have
been slain. But then I noticed how yellow were the skins of the
people I met, how shaggy the hair of the men, how large and bright
their eyes, and that every other man still wore his dirty rags.
Their faces seemed all with one of two expressions\dash{}a leaping
exultation and energy or a grim resolution. Save for the expression
of the faces, London seemed a city of tramps. The vestries were
indiscriminately distributing bread sent us by the French
government. The ribs of the few horses showed dismally. Haggard
special constables with white badges stood at the corners of every
street. I saw little of the mischief wrought by the Martians until
I reached Wellington Street, and there I saw the red weed
clambering over the buttresses of Waterloo Bridge.

At the corner of the bridge, too, I saw one of the common contrasts
of that grotesque time\dash{}a sheet of paper flaunting against a
thicket of the red weed, transfixed by a stick that kept it in
place. It was the placard of the first newspaper to resume
publication\dash{}the \emph{Daily Mail}. I bought a copy for a blackened
shilling I found in my pocket. Most of it was in blank, but the
solitary compositor who did the thing had amused himself by making
a grotesque scheme of advertisement stereo on the back page. The
matter he printed was emotional; the news organisation had not as
yet found its way back. I learned nothing fresh except that already
in one week the examination of the Martian mechanisms had yielded
astonishing results. Among other things, the article assured me
what I did not believe at the time, that the ``Secret of Flying,''
was discovered. At Waterloo I found the free trains that were
taking people to their homes. The first rush was already over.
There were few people in the train, and I was in no mood for casual
conversation. I got a compartment to myself, and sat with folded
arms, looking greyly at the sunlit devastation that flowed past the
windows. And just outside the terminus the train jolted over
temporary rails, and on either side of the railway the houses were
blackened ruins. To Clapham Junction the face of London was grimy
with powder of the Black Smoke, in spite of two days of
thunderstorms and rain, and at Clapham Junction the line had been
wrecked again; there were hundreds of out-of-work clerks and
shopmen working side by side with the customary navvies, and we
were jolted over a hasty relaying.

All down the line from there the aspect of the country was gaunt
and unfamiliar; Wimbledon particularly had suffered. Walton, by
virtue of its unburned pine woods, seemed the least hurt of any
place along the line. The Wandle, the Mole, every little stream,
was a heaped mass of red weed, in appearance between butcher's meat
and pickled cabbage. The Surrey pine woods were too dry, however,
for the festoons of the red climber. Beyond Wimbledon, within sight
of the line, in certain nursery grounds, were the heaped masses of
earth about the sixth cylinder. A number of people were standing
about it, and some sappers were busy in the midst of it. Over it
flaunted a Union Jack, flapping cheerfully in the morning breeze.
The nursery grounds were everywhere crimson with the weed, a wide
expanse of livid colour cut with purple shadows, and very painful
to the eye. One's gaze went with infinite relief from the scorched
greys and sullen reds of the foreground to the blue-green softness
of the eastward hills.

The line on the London side of Woking station was still undergoing
repair, so I descended at Byfleet station and took the road to
Maybury, past the place where I and the artilleryman had talked to
the hussars, and on by the spot where the Martian had appeared to
me in the thunderstorm. Here, moved by curiosity, I turned aside to
find, among a tangle of red fronds, the warped and broken dog cart
with the whitened bones of the horse scattered and gnawed. For a
time I stood regarding these vestiges. \ldots{}

Then I returned through the pine wood, neck-high with red weed here
and there, to find the landlord of the Spotted Dog had already
found burial, and so came home past the College Arms. A man
standing at an open cottage door greeted me by name as I passed.

I looked at my house with a quick flash of hope that faded
immediately. The door had been forced; it was unfast and was
opening slowly as I approached.

It slammed again. The curtains of my study fluttered out of the
open window from which I and the artilleryman had watched the dawn.
No one had closed it since. The smashed bushes were just as I had
left them nearly four weeks ago. I stumbled into the hall, and the
house felt empty. The stair carpet was ruffled and discoloured
where I had crouched, soaked to the skin from the thunderstorm the
night of the catastrophe. Our muddy footsteps I saw still went up
the stairs.

I followed them to my study, and found lying on my writing-table
still, with the selenite paper weight upon it, the sheet of work I
had left on the afternoon of the opening of the cylinder. For a
space I stood reading over my abandoned arguments. It was a paper
on the probable development of Moral Ideas with the development of
the civilising process; and the last sentence was the opening of a
prophecy: ``In about two hundred years,'' I had written, ``we may
expect\ldots{}'' The sentence ended abruptly. I remembered my inability
to fix my mind that morning, scarcely a month gone by, and how I
had broken off to get my \emph{Daily Chronicle} from the newsboy. I
remembered how I went down to the garden gate as he came along, and
how I had listened to his odd story of ``Men from Mars.''

I came down and went into the dining room. There were the mutton
and the bread, both far gone now in decay, and a beer bottle
overturned, just as I and the artilleryman had left them. My home
was desolate. I perceived the folly of the faint hope I had
cherished so long. And then a strange thing occurred. ``It is no
use,'' said a voice. ``The house is deserted. No one has been here
these ten days. Do not stay here to torment yourself. No one
escaped but you.''

I was startled. Had I spoken my thought aloud? I turned, and the
French window was open behind me. I made a step to it, and stood
looking out.

And there, amazed and afraid, even as I stood amazed and afraid,
were my cousin and my wife\dash{}my wife white and tearless. She gave a
faint cry.

``I came,'' she said. ``I knew\dash{}knew\ldots{}''

She put her hand to her throat\dash{}swayed. I made a step forward, and
caught her in my arms.

\Chapter{CHAPTER TEN\\THE EPILOGUE}
I cannot but regret, now that I am concluding my story, how little
I am able to contribute to the discussion of the many debatable
questions which are still unsettled. In one respect I shall
certainly provoke criticism. My particular province is speculative
philosophy. My knowledge of comparative physiology is confined to a
book or two, but it seems to me that Carver's suggestions as to the
reason of the rapid death of the Martians is so probable as to be
regarded almost as a proven conclusion. I have assumed that in the
body of my narrative.

At any rate, in all the bodies of the Martians that were examined
after the war, no bacteria except those already known as
terrestrial species were found. That they did not bury any of their
dead, and the reckless slaughter they perpetrated, point also to an
entire ignorance of the putrefactive process. But probable as this
seems, it is by no means a proven conclusion.

Neither is the composition of the Black Smoke known, which the
Martians used with such deadly effect, and the generator of the
Heat-Rays remains a puzzle. The terrible disasters at the Ealing
and South Kensington laboratories have disinclined analysts for
further investigations upon the latter. Spectrum analysis of the
black powder points unmistakably to the presence of an unknown
element with a brilliant group of three lines in the green, and it
is possible that it combines with argon to form a compound which
acts at once with deadly effect upon some constituent in the blood.
But such unproven speculations will scarcely be of interest to the
general reader, to whom this story is addressed. None of the brown
scum that drifted down the Thames after the destruction of
Shepperton was examined at the time, and now none is forthcoming.

The results of an anatomical examination of the Martians, so far as
the prowling dogs had left such an examination possible, I have
already given. But everyone is familiar with the magnificent and
almost complete specimen in spirits at the Natural History Museum,
and the countless drawings that have been made from it; and beyond
that the interest of their physiology and structure is purely
scientific.

A question of graver and universal interest is the possibility of
another attack from the Martians. I do not think that nearly enough
attention is being given to this aspect of the matter. At present
the planet Mars is in conjunction, but with every return to
opposition I, for one, anticipate a renewal of their adventure. In
any case, we should be prepared. It seems to me that it should be
possible to define the position of the gun from which the shots are
discharged, to keep a sustained watch upon this part of the planet,
and to anticipate the arrival of the next attack.

In that case the cylinder might be destroyed with dynamite or
artillery before it was sufficiently cool for the Martians to
emerge, or they might be butchered by means of guns so soon as the
screw opened. It seems to me that they have lost a vast advantage
in the failure of their first surprise. Possibly they see it in the
same light.

Lessing has advanced excellent reasons for supposing that the
Martians have actually succeeded in effecting a landing on the
planet Venus. Seven months ago now, Venus and Mars were in
alignment with the sun; that is to say, Mars was in opposition from
the point of view of an observer on Venus. Subsequently a peculiar
luminous and sinuous marking appeared on the unillumined half of
the inner planet, and almost simultaneously a faint dark mark of a
similar sinuous character was detected upon a photograph of the
Martian disk. One needs to see the drawings of these appearances in
order to appreciate fully their remarkable resemblance in
character.

At any rate, whether we expect another invasion or not, our views
of the human future must be greatly modified by these events. We
have learned now that we cannot regard this planet as being fenced
in and a secure abiding place for Man; we can never anticipate the
unseen good or evil that may come upon us suddenly out of space. It
may be that in the larger design of the universe this invasion from
Mars is not without its ultimate benefit for men; it has robbed us
of that serene confidence in the future which is the most fruitful
source of decadence, the gifts to human science it has brought are
enormous, and it has done much to promote the conception of the
commonweal of mankind. It may be that across the immensity of space
the Martians have watched the fate of these pioneers of theirs and
learned their lesson, and that on the planet Venus they have found
a securer settlement. Be that as it may, for many years yet there
will certainly be no relaxation of the eager scrutiny of the
Martian disk, and those fiery darts of the sky, the shooting stars,
will bring with them as they fall an unavoidable apprehension to
all the sons of men.

The broadening of men's views that has resulted can scarcely be
exaggerated. Before the cylinder fell there was a general
persuasion that through all the deep of space no life existed
beyond the petty surface of our minute sphere. Now we see further.
If the Martians can reach Venus, there is no reason to suppose that
the thing is impossible for men, and when the slow cooling of the
sun makes this earth uninhabitable, as at last it must do, it may
be that the thread of life that has begun here will have streamed
out and caught our sister planet within its toils.

Dim and wonderful is the vision I have conjured up in my mind of
life spreading slowly from this little seed bed of the solar system
throughout the inanimate vastness of sidereal space. But that is a
remote dream. It may be, on the other hand, that the destruction of
the Martians is only a reprieve. To them, and not to us, perhaps,
is the future ordained.

I must confess the stress and danger of the time have left an
abiding sense of doubt and insecurity in my mind. I sit in my study
writing by lamplight, and suddenly I see again the healing valley
below set with writhing flames, and feel the house behind and about
me empty and desolate. I go out into the Byfleet Road, and vehicles
pass me, a butcher boy in a cart, a cabful of visitors, a workman
on a bicycle, children going to school, and suddenly they become
vague and unreal, and I hurry again with the artilleryman through
the hot, brooding silence. Of a night I see the black powder
darkening the silent streets, and the contorted bodies shrouded in
that layer; they rise upon me tattered and dog-bitten. They gibber
and grow fiercer, paler, uglier, mad distortions of humanity at
last, and I wake, cold and wretched, in the darkness of the night.

I go to London and see the busy multitudes in Fleet Street and the
Strand, and it comes across my mind that they are but the ghosts of
the past, haunting the streets that I have seen silent and
wretched, going to and fro, phantasms in a dead city, the mockery
of life in a galvanised body. And strange, too, it is to stand on
Primrose Hill, as I did but a day before writing this last chapter,
to see the great province of houses, dim and blue through the haze
of the smoke and mist, vanishing at last into the vague lower sky,
to see the people walking to and fro among the flower beds on the
hill, to see the sight-seers about the Martian machine that stands
there still, to hear the tumult of playing children, and to recall
the time when I saw it all bright and clear-cut, hard and silent,
under the dawn of that last great day. \ldots{}

And strangest of all is it to hold my wife's hand again, and to
think that I have counted her, and that she has counted me, among
the dead.


\begin{Verbatim}[fontsize=\footnotesize]
End of the Project Gutenberg EBook of The War of the Worlds, by H. G. Wells

*** END OF THIS PROJECT GUTENBERG EBOOK THE WAR OF THE WORLDS ***

***** This file should be named 36-h.htm or 36-h.zip *****
This and all associated files of various formats will be found in:
        http://www.gutenberg.net/3/36/



Updated editions will replace the previous one--the old editions
will be renamed.

Creating the works from public domain print editions means that no
one owns a United States copyright in these works, so the Foundation
(and you!) can copy and distribute it in the United States without
permission and without paying copyright royalties.  Special rules,
set forth in the General Terms of Use part of this license, apply to
copying and distributing Project Gutenberg-tm electronic works to
protect the PROJECT GUTENBERG-tm concept and trademark.  Project
Gutenberg is a registered trademark, and may not be used if you
charge for the eBooks, unless you receive specific permission.  If you
do not charge anything for copies of this eBook, complying with the
rules is very easy.  You may use this eBook for nearly any purpose
such as creation of derivative works, reports, performances and
research.  They may be modified and printed and given away--you may do
practically ANYTHING with public domain eBooks.  Redistribution is
subject to the trademark license, especially commercial
redistribution.



*** START: FULL LICENSE ***

THE FULL PROJECT GUTENBERG LICENSE
PLEASE READ THIS BEFORE YOU DISTRIBUTE OR USE THIS WORK

To protect the Project Gutenberg-tm mission of promoting the free
distribution of electronic works, by using or distributing this work
(or any other work associated in any way with the phrase "Project
Gutenberg"), you agree to comply with all the terms of the Full Project
Gutenberg-tm License (available with this file or online at
http://gutenberg.net/license).


Section 1.  General Terms of Use and Redistributing Project Gutenberg-tm
electronic works

1.A.  By reading or using any part of this Project Gutenberg-tm
electronic work, you indicate that you have read, understand, agree to
and accept all the terms of this license and intellectual property
(trademark/copyright) agreement.  If you do not agree to abide by all
the terms of this agreement, you must cease using and return or destroy
all copies of Project Gutenberg-tm electronic works in your possession.
If you paid a fee for obtaining a copy of or access to a Project
Gutenberg-tm electronic work and you do not agree to be bound by the
terms of this agreement, you may obtain a refund from the person or
entity to whom you paid the fee as set forth in paragraph 1.E.8.

1.B.  "Project Gutenberg" is a registered trademark.  It may only be
used on or associated in any way with an electronic work by people who
agree to be bound by the terms of this agreement.  There are a few
things that you can do with most Project Gutenberg-tm electronic works
even without complying with the full terms of this agreement.  See
paragraph 1.C below.  There are a lot of things you can do with Project
Gutenberg-tm electronic works if you follow the terms of this agreement
and help preserve free future access to Project Gutenberg-tm electronic
works.  See paragraph 1.E below.

1.C.  The Project Gutenberg Literary Archive Foundation ("the Foundation"
or PGLAF), owns a compilation copyright in the collection of Project
Gutenberg-tm electronic works.  Nearly all the individual works in the
collection are in the public domain in the United States.  If an
individual work is in the public domain in the United States and you are
located in the United States, we do not claim a right to prevent you from
copying, distributing, performing, displaying or creating derivative
works based on the work as long as all references to Project Gutenberg
are removed.  Of course, we hope that you will support the Project
Gutenberg-tm mission of promoting free access to electronic works by
freely sharing Project Gutenberg-tm works in compliance with the terms of
this agreement for keeping the Project Gutenberg-tm name associated with
the work.  You can easily comply with the terms of this agreement by
keeping this work in the same format with its attached full Project
Gutenberg-tm License when you share it without charge with others.

1.D.  The copyright laws of the place where you are located also govern
what you can do with this work.  Copyright laws in most countries are in
a constant state of change.  If you are outside the United States, check
the laws of your country in addition to the terms of this agreement
before downloading, copying, displaying, performing, distributing or
creating derivative works based on this work or any other Project
Gutenberg-tm work.  The Foundation makes no representations concerning
the copyright status of any work in any country outside the United
States.

1.E.  Unless you have removed all references to Project Gutenberg:

1.E.1.  The following sentence, with active links to, or other immediate
access to, the full Project Gutenberg-tm License must appear prominently
whenever any copy of a Project Gutenberg-tm work (any work on which the
phrase "Project Gutenberg" appears, or with which the phrase "Project
Gutenberg" is associated) is accessed, displayed, performed, viewed,
copied or distributed:

This eBook is for the use of anyone anywhere at no cost and with
almost no restrictions whatsoever.  You may copy it, give it away or
re-use it under the terms of the Project Gutenberg License included
with this eBook or online at www.gutenberg.net

1.E.2.  If an individual Project Gutenberg-tm electronic work is derived
from the public domain (does not contain a notice indicating that it is
posted with permission of the copyright holder), the work can be copied
and distributed to anyone in the United States without paying any fees
or charges.  If you are redistributing or providing access to a work
with the phrase "Project Gutenberg" associated with or appearing on the
work, you must comply either with the requirements of paragraphs 1.E.1
through 1.E.7 or obtain permission for the use of the work and the
Project Gutenberg-tm trademark as set forth in paragraphs 1.E.8 or
1.E.9.

1.E.3.  If an individual Project Gutenberg-tm electronic work is posted
with the permission of the copyright holder, your use and distribution
must comply with both paragraphs 1.E.1 through 1.E.7 and any additional
terms imposed by the copyright holder.  Additional terms will be linked
to the Project Gutenberg-tm License for all works posted with the
permission of the copyright holder found at the beginning of this work.

1.E.4.  Do not unlink or detach or remove the full Project Gutenberg-tm
License terms from this work, or any files containing a part of this
work or any other work associated with Project Gutenberg-tm.

1.E.5.  Do not copy, display, perform, distribute or redistribute this
electronic work, or any part of this electronic work, without
prominently displaying the sentence set forth in paragraph 1.E.1 with
active links or immediate access to the full terms of the Project
Gutenberg-tm License.

1.E.6.  You may convert to and distribute this work in any binary,
compressed, marked up, nonproprietary or proprietary form, including any
word processing or hypertext form.  However, if you provide access to or
distribute copies of a Project Gutenberg-tm work in a format other than
"Plain Vanilla ASCII" or other format used in the official version
posted on the official Project Gutenberg-tm web site (www.gutenberg.net),
you must, at no additional cost, fee or expense to the user, provide a
copy, a means of exporting a copy, or a means of obtaining a copy upon
request, of the work in its original "Plain Vanilla ASCII" or other
form.  Any alternate format must include the full Project Gutenberg-tm
License as specified in paragraph 1.E.1.

1.E.7.  Do not charge a fee for access to, viewing, displaying,
performing, copying or distributing any Project Gutenberg-tm works
unless you comply with paragraph 1.E.8 or 1.E.9.

1.E.8.  You may charge a reasonable fee for copies of or providing
access to or distributing Project Gutenberg-tm electronic works provided
that

- You pay a royalty fee of 20% of the gross profits you derive from
     the use of Project Gutenberg-tm works calculated using the method
     you already use to calculate your applicable taxes.  The fee is
     owed to the owner of the Project Gutenberg-tm trademark, but he
     has agreed to donate royalties under this paragraph to the
     Project Gutenberg Literary Archive Foundation.  Royalty payments
     must be paid within 60 days following each date on which you
     prepare (or are legally required to prepare) your periodic tax
     returns.  Royalty payments should be clearly marked as such and
     sent to the Project Gutenberg Literary Archive Foundation at the
     address specified in Section 4, "Information about donations to
     the Project Gutenberg Literary Archive Foundation."

- You provide a full refund of any money paid by a user who notifies
     you in writing (or by e-mail) within 30 days of receipt that s/he
     does not agree to the terms of the full Project Gutenberg-tm
     License.  You must require such a user to return or
     destroy all copies of the works possessed in a physical medium
     and discontinue all use of and all access to other copies of
     Project Gutenberg-tm works.

- You provide, in accordance with paragraph 1.F.3, a full refund of any
     money paid for a work or a replacement copy, if a defect in the
     electronic work is discovered and reported to you within 90 days
     of receipt of the work.

- You comply with all other terms of this agreement for free
     distribution of Project Gutenberg-tm works.

1.E.9.  If you wish to charge a fee or distribute a Project Gutenberg-tm
electronic work or group of works on different terms than are set
forth in this agreement, you must obtain permission in writing from
both the Project Gutenberg Literary Archive Foundation and Michael
Hart, the owner of the Project Gutenberg-tm trademark.  Contact the
Foundation as set forth in Section 3 below.

1.F.

1.F.1.  Project Gutenberg volunteers and employees expend considerable
effort to identify, do copyright research on, transcribe and proofread
public domain works in creating the Project Gutenberg-tm
collection.  Despite these efforts, Project Gutenberg-tm electronic
works, and the medium on which they may be stored, may contain
"Defects," such as, but not limited to, incomplete, inaccurate or
corrupt data, transcription errors, a copyright or other intellectual
property infringement, a defective or damaged disk or other medium, a
computer virus, or computer codes that damage or cannot be read by
your equipment.

1.F.2.  LIMITED WARRANTY, DISCLAIMER OF DAMAGES - Except for the "Right
of Replacement or Refund" described in paragraph 1.F.3, the Project
Gutenberg Literary Archive Foundation, the owner of the Project
Gutenberg-tm trademark, and any other party distributing a Project
Gutenberg-tm electronic work under this agreement, disclaim all
liability to you for damages, costs and expenses, including legal
fees.  YOU AGREE THAT YOU HAVE NO REMEDIES FOR NEGLIGENCE, STRICT
LIABILITY, BREACH OF WARRANTY OR BREACH OF CONTRACT EXCEPT THOSE
PROVIDED IN PARAGRAPH F3.  YOU AGREE THAT THE FOUNDATION, THE
TRADEMARK OWNER, AND ANY DISTRIBUTOR UNDER THIS AGREEMENT WILL NOT BE
LIABLE TO YOU FOR ACTUAL, DIRECT, INDIRECT, CONSEQUENTIAL, PUNITIVE OR
INCIDENTAL DAMAGES EVEN IF YOU GIVE NOTICE OF THE POSSIBILITY OF SUCH
DAMAGE.

1.F.3.  LIMITED RIGHT OF REPLACEMENT OR REFUND - If you discover a
defect in this electronic work within 90 days of receiving it, you can
receive a refund of the money (if any) you paid for it by sending a
written explanation to the person you received the work from.  If you
received the work on a physical medium, you must return the medium with
your written explanation.  The person or entity that provided you with
the defective work may elect to provide a replacement copy in lieu of a
refund.  If you received the work electronically, the person or entity
providing it to you may choose to give you a second opportunity to
receive the work electronically in lieu of a refund.  If the second copy
is also defective, you may demand a refund in writing without further
opportunities to fix the problem.

1.F.4.  Except for the limited right of replacement or refund set forth
in paragraph 1.F.3, this work is provided to you ?AS-IS?, WITH NO OTHER
WARRANTIES OF ANY KIND, EXPRESS OR IMPLIED, INCLUDING BUT NOT LIMITED TO
WARRANTIES OF MERCHANTIBILITY OR FITNESS FOR ANY PURPOSE.

1.F.5.  Some states do not allow disclaimers of certain implied
warranties or the exclusion or limitation of certain types of damages.
If any disclaimer or limitation set forth in this agreement violates the
law of the state applicable to this agreement, the agreement shall be
interpreted to make the maximum disclaimer or limitation permitted by
the applicable state law.  The invalidity or unenforceability of any
provision of this agreement shall not void the remaining provisions.

1.F.6.  INDEMNITY - You agree to indemnify and hold the Foundation, the
trademark owner, any agent or employee of the Foundation, anyone
providing copies of Project Gutenberg-tm electronic works in accordance
with this agreement, and any volunteers associated with the production,
promotion and distribution of Project Gutenberg-tm electronic works,
harmless from all liability, costs and expenses, including legal fees,
that arise directly or indirectly from any of the following which you do
or cause to occur: (a) distribution of this or any Project Gutenberg-tm
work, (b) alteration, modification, or additions or deletions to any
Project Gutenberg-tm work, and (c) any Defect you cause.


Section  2.  Information about the Mission of Project Gutenberg-tm

Project Gutenberg-tm is synonymous with the free distribution of
electronic works in formats readable by the widest variety of computers
including obsolete, old, middle-aged and new computers.  It exists
because of the efforts of hundreds of volunteers and donations from
people in all walks of life.

Volunteers and financial support to provide volunteers with the
assistance they need, is critical to reaching Project Gutenberg-tm's
goals and ensuring that the Project Gutenberg-tm collection will
remain freely available for generations to come.  In 2001, the Project
Gutenberg Literary Archive Foundation was created to provide a secure
and permanent future for Project Gutenberg-tm and future generations.
To learn more about the Project Gutenberg Literary Archive Foundation
and how your efforts and donations can help, see Sections 3 and 4
and the Foundation web page at http://www.pglaf.org.


Section 3.  Information about the Project Gutenberg Literary Archive
Foundation

The Project Gutenberg Literary Archive Foundation is a non profit
501(c)(3) educational corporation organized under the laws of the
state of Mississippi and granted tax exempt status by the Internal
Revenue Service.  The Foundation's EIN or federal tax identification
number is 64-6221541.  Its 501(c)(3) letter is posted at
http://pglaf.org/fundraising.  Contributions to the Project Gutenberg
Literary Archive Foundation are tax deductible to the full extent
permitted by U.S. federal laws and your state's laws.

The Foundation's principal office is located at 4557 Melan Dr. S.
Fairbanks, AK, 99712., but its volunteers and employees are scattered
throughout numerous locations.  Its business office is located at
809 North 1500 West, Salt Lake City, UT 84116, (801) 596-1887, email
business@pglaf.org.  Email contact links and up to date contact
information can be found at the Foundation's web site and official
page at http://pglaf.org

For additional contact information:
     Dr. Gregory B. Newby
     Chief Executive and Director
     gbnewby@pglaf.org

Section 4.  Information about Donations to the Project Gutenberg
Literary Archive Foundation

Project Gutenberg-tm depends upon and cannot survive without wide
spread public support and donations to carry out its mission of
increasing the number of public domain and licensed works that can be
freely distributed in machine readable form accessible by the widest
array of equipment including outdated equipment.  Many small donations
($1 to $5,000) are particularly important to maintaining tax exempt
status with the IRS.

The Foundation is committed to complying with the laws regulating
charities and charitable donations in all 50 states of the United
States.  Compliance requirements are not uniform and it takes a
considerable effort, much paperwork and many fees to meet and keep up
with these requirements.  We do not solicit donations in locations
where we have not received written confirmation of compliance.  To
SEND DONATIONS or determine the status of compliance for any
particular state visit http://pglaf.org

While we cannot and do not solicit contributions from states where we
have not met the solicitation requirements, we know of no prohibition
against accepting unsolicited donations from donors in such states who
approach us with offers to donate.

International donations are gratefully accepted, but we cannot make
any statements concerning tax treatment of donations received from
outside the United States.  U.S. laws alone swamp our small staff.

Please check the Project Gutenberg Web pages for current donation
methods and addresses.  Donations are accepted in a number of other
ways including including checks, online payments and credit card
donations.  To donate, please visit: http://pglaf.org/donate


Section 5.  General Information About Project Gutenberg-tm electronic
works.

Professor Michael S. Hart is the originator of the Project Gutenberg-tm
concept of a library of electronic works that could be freely shared
with anyone.  For thirty years, he produced and distributed Project
Gutenberg-tm eBooks with only a loose network of volunteer support.

Project Gutenberg-tm eBooks are often created from several printed
editions, all of which are confirmed as Public Domain in the U.S.
unless a copyright notice is included.  Thus, we do not necessarily
keep eBooks in compliance with any particular paper edition.

Most people start at our Web site which has the main PG search facility:

     http://www.gutenberg.net

This Web site includes information about Project Gutenberg-tm,
including how to make donations to the Project Gutenberg Literary
Archive Foundation, how to help produce our new eBooks, and how to
subscribe to our email newsletter to hear about new eBooks.
\end{Verbatim}
\end{document}
