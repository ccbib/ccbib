\usepackage[german,ngerman]{babel}
\usepackage[T1]{fontenc}
\hyphenation{wa-rum}


%\setlength{\emergencystretch}{1ex}

\renewcommand*{\tb}[1]{\begin{center}#1\end{center}}
\newcommand*{\latein}[1]{#1}
\newcommand*{\unterschrift}[1]{\begin{flushright}#1\end{flushright}}

%% Kommentiere folgende Zeile, um Originalseitenzahlen zu deaktivieren.
\renewcommand*{\pagenum}[1]{\marginline{#1}}

\newcommand\Zwickerl{\discretionary{Zwik-}{kerl}{Zwickerl}}

\begin{document}
\raggedbottom

\begin{Verbatim}[fontsize=\footnotesize]
The Project Gutenberg EBook of Die Stadt ohne Juden, by Hugo Bettauer

This eBook is for the use of anyone anywhere at no cost and with
almost no restrictions whatsoever.  You may copy it, give it away or
re-use it under the terms of the Project Gutenberg License included
with this eBook or online at www.gutenberg.org


Title: Die Stadt ohne Juden
       Ein Roman von übermorgen

Author: Hugo Bettauer

Illustrator: Martha von Wagner-Schidrowitz

Release Date: March 13, 2011 [EBook #35569]

Language: German

Character set encoding: UTF-8

*** START OF THIS PROJECT GUTENBERG EBOOK DIE STADT OHNE JUDEN ***


Produced by Jana Srna, Norbert H. Langkau and the Online
Distributed Proofreading Team at http://www.pgdp.net.
\end{Verbatim}

\section{Anmerkungen zur Transkription}

Schreibweise und Interpunktion des Originaltextes wurden
übernommen; lediglich offensichtliche Druckfehler wurden
korrigiert. Änderungen sind im Source-Code
\erratum{so wie hier}{gekennzeichnet}.

\section{Typographische Anpassungen}
Dieser Text wurde für die Creative Commons Bibliothek als 
\LaTeX-Dokument aufbereitet. Dabei wurden gegenüber der 
Projekt Gutenberg Version einige Änderungen vorgenommen, um moderne
typographische Standards zu erfüllen.

Dieses Dokument versucht den Text wiederzugeben und nicht eine bestimmte
Ausgabe in allen Details zu reproduzieren, mit einer Ausnahme: Die
Zahlen im äußeren Rand der Buchseiten entsprechen den Seitenzahlen
in der Ausgabe vom Gloriette-Verlag, Wien -- Dritte Auflage. 11.-15. Tausend.

\title{Die\\Stadt ohne Juden}

\subtitle{Ein Roman von übermorgen}

\author{Von\\Hugo Bettauer}

%Gloriette-Verlag, Wien

%Alle Rechte vorbehalten

\date{Copyright by Gloriette-Verlag, Vienna 1922}

%Umschlag-Entwurf von Martha v. Wagner-Schidrowitz

%Dritte Auflage. 11.–15. Tausend

\maketitle

\chapter{Erster Teil.}
\pagenum{3}

Von der Universität bis zur Bellaria umlagerte das schöne, ruhige
und vornehme Parlamentsgebäude eine einzige Menschenmauer. Ganz
Wien schien sich an diesem Junitag um die zehnte Vormittagsstunde
versammelt zu haben, um dort zu sein, wo sich ein historisches
Ereignis von unabsehbarer Tragweite abspielen sollte. Bürger und
Arbeiter, Damen und Frauen aus dem Volke, halbwüchsige Burschen und
Greise, junge Mädchen, kleine Kinder, Kranke in Rollwagen, alles
quoll durcheinander, schrie, politisierte und schwitzte. Und immer
wieder fand sich ein Begeisterter, der plötzlich an den Kreis um
ihn herum eine Ansprache hielt und immer wieder brauste der Ruf
auf:

„Hinaus mit den Juden!“

Sonst pflegten bei ähnlichen Demonstrationen hier und dort Leute
mit gebogener Nase oder besonders schwarzem Haar weidlich
verprügelt zu werden; diesmal kam es zu keinem solchen
Zwischenfall, denn Jüdisches war weit und breit nicht zu sehen, und
zudem hatten die Kaffeehäuser und Bankgeschäfte am Franzens- und
Schottenring, in weiser Erkenntnis aller Möglichkeiten, ihre
Pforten geschlossen und die Rollbalken herabgezogen.

\pagenum{4}Plötzlich zerriß ein einziges Aufbrüllen die Luft.

„Hoch Doktor Karl Schwertfeger, hoch, hoch, hoch! Hoch der Befreier
Österreichs!“

Ein offenes Auto fuhr langsam mitten durch die Menschenmassen
hindurch, die zurückdrängten und Bahn machten. Im Auto saß ein
großer älterer Herr, dessen mächtiger Schädel mit willkürlichen
Büscheln weißer Haare bedeckt war.

Er nahm den grauen, weichen Schlapphut ab, nickte der jubelnden
Menschenmenge zu und verzerrte das Gesicht zu einem Lächeln. Aber
es war ein saures Lächeln, das von den zwei Falten, die von den
Mundwinkeln abwärts liefen, gewissermaßen dementiert wurde. Und die
tiefliegenden grauen Augen blickten eher finster als vergnügt
drein.

Lachende Mädchen drängten sich vor, schwangen sich auf das
Trittbrett, die eine warf dem Gefeierten Blumen zu, eine andere war
noch dreister, schlang ihren Arm um seinen Hals und küßte den
Doktor Schwertfeger auf die Wange. Als ob der Chauffeur ahnte, wie
seinem Herrn bei solchen Gefühlsausbrüchen zumute wurde, ließ er
das Auto vorwärts springen, so daß die Mädchen mit jähem Ruck nach
rückwärts fielen. Sie taten sich dabei nicht wehe, denn die
Menschenmauer fing sie auf.

Im Parlamentsgebäude herrschte nicht die laute Begeisterung der
Straße, sondern fieberhafte Erregung, zu stark, um Ausdruck nach
außen zu finden. Die Abgeordneten, die sich bis zum letzten Mann
eingefunden hatten, die Minister, die Saaldiener gingen schweigend
und unruhig umher, sogar die überfüllten Galerien verhielten sich
lautlos.

\pagenum{5}In der Journalistenloge, in der es sonst am
ungeniertesten zuzugehen pflegte, wurde nur im Flüsterton
gesprochen. Und eine bemerkenswerte räumliche Spaltung hatte sich
eingestellt. Die kompakte jüdische Majorität der Berichterstatter
drängte ihre Stühle zusammen, die Referenten der christlichsozialen
und deutschnationalen Blätter bildeten ihrerseits eine Gruppe.
Sonst mischten sich die jüdischen und christlichen Journalisten
fröhlich durcheinander, im Berufskreis war man nicht Parteigänger,
sondern nur der Herr Kollege, und da die jüdischen Journalisten
gewöhnlich mehr Neuigkeiten wußten und sie besser verwerten
konnten, standen die antisemitischen zu ihnen in einem starken
Abhängigkeitsverhältnis. Heute aber flogen hämische Blicke von der
christlichen Ecke in die jüdische, und als der kleine Karpeles von
der „Weltpost“, der eben erst eingetreten war, den Doktor Wiesel
von der „Wehr“ mit „Servus Herr Kollege!“ begrüßte, wandte ihm
dieser ohne Erwiderung den \discretionary{Rük-}{ken}{Rücken}.

Es drängten immer noch Journalisten herein, darunter Vertreter
ausländischer Zeitungen, die heute in Wien angekommen waren.

„Nicht rühren kann man sich“, brummte der Herglotz vom christlichen
„Tag“, worauf ihm ein Kollege mit kleinem, bärtigem Kopf und
mächtigem Bierbauch erwiderte:

„Na, ein paar Tage noch und wir werden hier Platz genug haben!“

Hüsteln, Lächeln, Lachen auf der einen Seite, gegenseitige
bedeutungsvolle Blicke auf der anderen.

\pagenum{6}Ein junger blonder Herr mit roten Backen machte nach
links und rechts eine leichte Verbeugung.

„Holborn vom \glq{}London Telegraph\grq{}! Bin eben vor einer Stunde
angekommen und kenne mich wahrhaftig nicht aus. Vorgestern kam ich
aus Sidney nach halbjähriger Abwesenheit in London an, eine Stunde
später saß ich wieder im Zug, um nach Wien zu fahren. Unser
Managing-Editor, das Kamel, hat mir nichts gesagt, als: In Wien
wird es jetzt lustig, da schmeißen sie die Juden hinaus! Fahren Sie
hin und berichten Sie, daß das Kabel reißt! Also bitte, wäre sehr
nett von Ihnen, wenn Sie mich rasch instruieren wollten.“

Das alles war in so drolligem Englisch-deutsch herausgekommen, daß
sich die Spannung ein wenig löste. Minkus vom „Tagesboten“
bemächtigte sich, heftig gestikulierend, des englischen Kollegen
und begann mit den Worten:

„Also, ich werde Ihnen alles genau erklären~–.“ Aber Doktor Wiesel
ließ ihn nicht weitersprechen. „Sie verzeihen, aber diese
Aufklärung wird besser von \emph{uns} ausgehen.“

Tonfall drohend, das „uns“ bedeutungsvoll unterstrichen.

Und schon befand sich Holborn in der christlichen Ecke, wo Wiesel
kurz und sachlich erklärte:

„Was geschehen soll, werden Sie sofort aus dem Munde unseres
Bundeskanzlers Dr. Karl Schwertfeger erfahren, der das Gesetz zur
Ausweisung aller Nichtarier aus Österreich eingehend begründen
wird. Die Vorgeschichte ist, kurz gesagt, folgende: Als die
österreichische Krone auf den Wert \pagenum{7} eines fünfzigstel
Centimes herabgesunken war, begann das Chaos einzutreten. Ein
Ministerium nach dem anderen mußte gehen, es entstanden Unruhen,
täglich kam es zu Plünderungen der Geschäfte, zu Pogroms, die Wut
und Verzweiflung der Bevölkerung kannte keine Grenzen mehr und
schließlich mußte zu Neuwahlen geschritten werden. Die
Sozialdemokraten traten ohne neues Programm in den Wahlkampf, die
Christlichsozialen hingegen scharten sich um ihren geistvollen
Führer Dr. Karl Schwertfeger, dessen Losungswort lautete: Hinaus
mit den Juden aus Österreich! Nun, vielleicht ist es Ihnen
bekannt,“ – Holborn nickte, obwohl er keine Ahnung hatte – „daß die
Wahlen den völligen Zusammenbruch der Sozialdemokraten, Kommunisten
und Liberalen brachten. Selbst die Arbeitermassen wählten unter der
Parole „Hinaus mit den Juden!“, und die sozialistische Partei,
vordem relativ die stärkste, konnte knapp elf Mandate retten. Die
Großdeutschen aber, die gut abschnitten, hatten sich ebenfalls auf
das „Hinaus mit den Juden!“ eingestellt.

Nun, der Genialität des Doktor Schwertfeger, seiner unerschrockenen
Energie, seiner kühnen Impetuosität und Beredsamkeit gelang es, dem
Völkerbund, der vor die Alternative Anschluß Österreichs an
Deutschland oder Gewährenlassen gestellt war, die Zustimmung zur
großen Judenausweisung abzuringen. Und jetzt wird Schwertfeger
selbst das Gesetz einbringen, das sicher angenommen werden wird.
Sie sind also Zeuge eines historischen~–.“

„Pst!“-Rufe wurden laut. Wiesel konnte nicht weiterreden, denn der
Präsident des Hauses, ein Tiroler mit \pagenum{8} rötlichem
Vollbart, schwang die Glocke und erteilte dem Bundeskanzler das
Wort.

Grabesstille, in die das Surren der Ventilatoren unheimlich klang.
Das leiseste Räuspern, das Rascheln der Papiere in der
Journalistenloge wurde gehört und empfunden.

Übergroß, trotz des vorgebeugten Schädels und gewölbten 
\discretionary{Rük-}{kens}{Rückens},
stand der Bundeskanzler auf der Rednertribüne, die Hände, zu
Fäusten geballt, stützten sich auf das Pult, unter den grauen,
buschigen Brauen glitzerten die scharfen Augen über den Saal
hinweg. So stand er bewegungslos, bis er plötzlich den Schädel ins
Genick warf und mit seiner mächtigen Stimme, die sich in den
turbulentesten Versammlungen immer hatte Gehör erzwingen können,
begann.

„Verehrte Damen und Herren! Ich lege Ihnen jenes Gesetz und jene
Änderungen unserer Bundesverfassung vor, die gemeinsam nichts
weniger bezwecken, als die Ausweisung der nichtarischen, deutlicher
gesagt, der jüdischen Bevölkerung aus Österreich. Bevor ich das
tue, möchte ich aber einige rein persönliche Bemerkungen machen.

Seit fünf Jahren bin ich der Führer der christlichsozialen Partei,
seit einem Jahr durch den Willen der überwiegenden Mehrheit dieses
Hauses Bundeskanzler. Und durch diese fünf Jahre hindurch haben
mich die sogenannten liberalen Blätter wie die
sozialdemokratischen, mit einem Wort alle von Juden geschriebenen
Zeitungen, als eine Art Popanz dargestellt, als einen wütenden
\pagenum{9} Judenfeind, als einen fanatischen Hasser des
Judentums und der Juden. Nun, gerade heute, wo die Macht dieser
Presse ihrem unwiderruflichen Ende entgegengeht, drängt es mich, zu
erklären, daß das alles nicht so ist. Ja, ich habe den Mut, heute
von dieser Tribüne aus zu sagen, daß ich viel eher Judenfreund als
Judenfeind bin!“

Ein Murmeln und Surren ging durch den Saal, als flöge eine Schar
Vögel aus dem Felde auf.

„Ja, meine Damen und Herren, ich bin ein Schätzer der Juden, ich
habe, als ich noch nicht den heißen Boden der Politik betreten,
jüdische Freunde gehabt, ich saß einst in den Hörsälen unserer
\latein{Alma mater} zu Füßen jüdischer Lehrer, die ich verehrte und
noch immer verehre, ich bin jederzeit bereit, die autochthonen
jüdischen Tugenden, ihre außerordentliche Intelligenz, ihr Streben
nach aufwärts, ihren vorbildlichen Familiensinn, ihre
Internationalität, ihre Fähigkeit, sich jedem Milieu anzupassen,
anzuerkennen, ja zu bewundern!“

„Hört! Hört!“-Rufe wurden laut, sensationelle Spannung bemächtigte
sich der Abgeordneten und des Auditoriums, und der englische
Journalist Holborn, der nicht alles verstanden hatte, fragte
interessiert den Doktor Wiesel, ob der Mann da unten der Vertreter
der Judenschaft sei.

Der Kanzler fuhr fort.

„Trotzdem, ja gerade deshalb wuchs im Laufe der Jahre in mir immer
mehr und stärker die Überzeugung, daß wir Nichtjuden nicht länger
mit, unter und neben den Juden leben können, daß es entweder Biegen
oder \pagenum{10} Brechen heißt, daß wir entweder uns, unsere
christliche Art, unser Wesen und Sein oder aber die Juden aufgeben
müssen. Verehrtes Haus! Die Sache ist einfach die, daß wir
österreichische Arier den Juden nicht gewachsen sind, daß wir von
einer kleinen Minderheit beherrscht, unterdrückt, vergewaltigt
werden, weil eben diese Minderheit Eigenschaften besitzt, die uns
fehlen! Die Romanen, die Angelsachsen, der Yankee, ja sogar der
Norddeutsche wie der Schwabe – sie alle können die Juden verdauen,
weil sie an Agilität, Zähigkeit, Geschäftssinn und Energie den
Juden gleichen, oft sie sogar übertreffen. Wir aber können sie
nicht verdauen, uns bleiben sie Fremdkörper, die unsern Leib
überwuchern und uns schließlich versklaven. Unser Volk kommt zum
überwiegenden Teil aus den Bergen, unser Volk ist ein naives,
treuherziges Volk, verträumt, verspielt, unfruchtbaren Idealen
nachhängend, der Musik und stiller Naturbetrachtung ergeben, fromm
und bieder, gut und sinnig! Das sind schöne, wunderbare
Eigenschaften, aus denen eine herrliche Kultur, eine wunderbare
Lebensform sprießen kann, wenn man sie gewähren und sich entwickeln
läßt. Aber die Juden unter uns duldeten diese stille Entwicklung
nicht. Mit ihrer unheimlichen Verstandesschärfe, ihrem von
Tradition losgelösten Weltsinn, ihrer katzenartigen
Geschmeidigkeit, ihrer blitzschnellen Auffassung, ihren durch
jahrtausendelange Unterdrückung geschärften Fähigkeiten haben sie
uns überwältigt, sind unsere Herren geworden, haben das ganze
wirtschaftliche, geistige und kulturelle Leben unter ihre Macht
bekommen.“

\pagenum{11}Brausende „Bravo!“-Rufe; „Sehr richtig!“ „So ist
es!“

Doktor Schwertfeger führte mit der knochigen Rechten das Glas zu
den dünnen Lippen und sein halb spöttischer, halb befriedigter
Blick kreiste im Saal.

„Sehen wir dieses kleine Österreich von heute an. Wer hat die
Presse und damit die öffentliche Meinung in der Hand? Der Jude! Wer
hat seit dem unheilvollen Jahre 1914 Milliarden auf Milliarden
gehäuft? Der Jude! Wer kontrolliert den ungeheuren Banknotenumlauf,
sitzt an den leitenden Stellen in den Großbanken, wer steht an der
Spitze fast sämtlicher Industrieen? Der Jude! Wer besitzt unsere
Theater? Der Jude! Wer schreibt die Stücke, die aufgeführt werden?
Der Jude! Wer fährt im Automobil, wer praßt in den Nachtlokalen,
wer füllt die Kaffeehäuser, wer die vornehmen Restaurants, wer
behängt sich und seine Frau mit Juwelen und Perlen? Der Jude!

Verehrte Anwesende! Ich habe gesagt, daß ich den Juden, an sich und
objektiv betrachtet, für ein wertvolles Individuum halte und ich
bleibe dabei. Aber ist nicht auch der Rosenkäfer mit seinen
schimmernden Flügeln ein an sich schönes, wertvolles Geschöpf und
wird er von dem sorgsamen Gärtner nicht trotzdem vertilgt, weil ihm
die Rose näher steht als der Käfer? Ist nicht der Tiger ein
herrliches Tier, voll von Kraft, Mut und Intelligenz? Und wird er
nicht doch gejagt und verfolgt, weil es der Kampf um das eigene
Leben erfordert? Von diesem und nur von diesem Standpunkt kann bei
uns die Judenfrage \pagenum{12} betrachtet werden. Entweder wir
oder die Juden! Entweder wir, die wir neun Zehntel der Bevölkerung
ausmachen, müssen zugrunde gehen oder die Juden müssen
verschwinden! Und da wir jetzt endlich die Macht in den Händen
haben, wären wir Toren, nein, Verbrecher an uns und unseren
Kindern, wenn wir von dieser Macht nicht Gebrauch machen und die
kleine Minderheit, die uns vernichtet, nicht vertreiben wollten.
Hier handelt es sich nicht um Schlagworte und Phrasen, wie
Menschlichkeit, Gerechtigkeit, Toleranz, sondern um unsere
Existenz, unser Leben, das Leben der kommenden Generationen! Die
letzten Jahre haben unser Elend vertausendfacht, wir stehen mitten
im vollen Staatsbankrott, wir gehen der Auflösung entgegen, ein
paar Jahre noch und unsere Nachbarn werden unter dem Vorwand, bei
uns Ordnung schaffen zu müssen, über uns herfallen und unser
kleines Land auf Stücke zerreißen – unberührt von allen
Geschehnissen aber werden die Juden blühen, gedeihen, die Situation
beherrschen und, da sie ja nie Deutsche im Herzen und im Blut
waren, unter den geänderten Verhältnissen Herren bleiben, wenn wir
Sklaven sind!“

Das ganze Haus geriet jetzt in furchtbare Aufregung. Wilde Rufe
wurden ausgestoßen. „Das darf nicht sein! Retten wir uns und unsere
Kinder!“ Und als Echo klang es von der Straße her aus zehntausend
Kehlen: „Hinaus mit den Juden!“

Doktor Schwertfeger ließ die Erregung auslaufen, nahm von den
Ministerkollegen Händedrücke entgegen und sprach dann über die
Durchführung des Gesetzes. \pagenum{13} Gemäß den Forderungen
der Menschlichkeit und den Bedingungen des Völkerbundes würde mit
größter Milde und Gerechtigkeit vorgegangen werden. Jeder habe das
Recht, sein Vermögen mitzunehmen, soweit es aus Bargeld und
Wertpapieren oder Juwelen bestehe, Immobilien zu veräußern, sein
Geschäft freihändig zu verkaufen. Unternehmungen, die nicht
veräußerlich seien, würden vom Staat übernommen werden, und zwar
derart, daß nach dem Steuerbekenntnis des letzten Jahres der
Reinertrag fünfprozentig kapitalisiert werden würde. Hätte also zum
Beispiel ein Unternehmen im vergangenen Jahr eine halbe Million
Reinertrag aufgewiesen, so würde es mit zehn Millionen abgelöst
werden. Ein boshaftes Lächeln kräuselte die Lippen des Kanzlers.

„Natürlich sind sowohl bei diesen Ablösungen als auch bei der
Erlaubnis zur Mitnahme von Bargeld lediglich die Steuerbekenntnisse
maßgebend. Hat sich jemand als Vermögensloser bekannt, so darf er
kein Geld ausführen, besitzt er trotzdem Vermögen, so wird dieses
natürlich konfisziert. Hat jemand den Reinertrag seines Geschäftes
mit einer halben Million beziffert, so darf er zehn Millionen
mitnehmen, auch wenn sich herausstellen sollte, daß sein wirkliches
Einkommen zehnmal so groß war. Auf diese Art wird sich manche Sünde
bitter rächen~–“, bemerkte der Redner unter schallender Heiterkeit
der Anwesenden. Er fuhr dann fort:

„Festbesoldete und geistige Arbeiter, die tatsächlich vermögenslos
sind, wie zum Beispiel Ärzte, erhalten vom Staat den Betrag zur
Fortreise, den sie als Jahreseinkommen \pagenum{14} versteuert
hatten. Gab also ein Arzt sein Einkommen mit dreihunderttausend an,
so erhält er diese Summe. Um jede anderweitige Steuerflucht zu
verhüten, enthält das Gesetz die drakonische Bestimmung, daß der
Versuch, größere als erlaubte Summen fortzuschleppen, mit dem Tode
zu bestrafen sei. Ebenso ist die Todesstrafe über die Juden oder
Judenstämmlinge verhängt, die den Versuch machen, sich auch
weiterhin heimlich in Österreich aufzuhalten.

Das Gesetz soll in folgender Weise durchgeführt werden:

„Nichtprotokollierte Kaufleute, Händler und sogenannte Agenten
müssen innerhalb dreier Monate nach Annahme des Gesetzes die
Grenzen verlassen, protokollierte Firmeninhaber, Angestellte,
Beamte und manuelle Arbeiter innerhalb von vier Monaten, Künstler,
Gelehrte, Ärzte, Rechtsanwälte und so weiter innerhalb von fünf
Monaten. Direktoren von Aktienunternehmungen, Banken und
Industrien, die im letzten Jahre ein Einkommen von mehr als sechs
Millionen versteuert haben, ist eine Frist von einem halben Jahr
\erratum{gegeben.}{gegeben.“}

Und nun komme ich zu einem wichtigen Punkt, dem ich die volle
Aufmerksamkeit zu schenken bitte. Wie Sie wissen, bezieht sich das
Ausweisungsgesetz nicht nur auf Juden und getaufte Juden, sondern
auch auf Judenstämmlinge. Als Judenstämmling gelten die Kinder aus
Mischehen. Hat also zum Beispiel eine Christin rein deutscharischer
Abstammung einen Juden geheiratet, so trifft die Ausweisung ihn und
die Kinder aus dieser Ehe, während es \pagenum{15} der Frau
unbenommen bleibt, in Österreich zu verweilen. Nach reiflicher
Überlegung hat die Regierung beschlossen, die Kindeskinder aus
Mischehen nicht mehr als Judenstämmlinge, sondern als Arier zu
betrachten. Hat also ein Christ eine Jüdin geheiratet, so werden
wohl die Kinder ausgewiesen, die Kindeskinder aber, vorausgesetzt,
daß die Eltern sich nicht wieder mit Juden gemischt haben, können
im Lande bleiben. Dies ist aber auch die absolut einzige
Konzession, die das Gesetz macht. Andere Ausnahmen sind nicht
zulässig. Von vielen Seiten wurde uns nahegelegt, gewisse Ausnahmen
gelten zu lassen. So sollte das Gesetz Leute über ein gewisses
Alter hinaus, Kranke, Schwächliche und solche Juden, die besondere
Verdienste um den Staat haben, nicht treffen.

Meine Damen und Herren! Hätte ich diesen Ratgebern nachgegeben, so
würde das ganze Gesetz zur Posse geworden sein. Das jüdische Geld,
jüdischer Einfluß hätten Tag und Nacht gearbeitet, zehntausende von
Ausnahmsfällen würden konstruiert werden und in fünfzig Jahren
wären wir genau so weit wie heute. Nein, es gibt keine Ausnahme, es
gibt keine Protektion, es gibt kein Mitleid und kein
Augenzudrücken! Für Hinfällige und Kranke wird die Regierung
prachtvolle Spitalzüge zur Verfügung stellen, und nur solche Juden,
die nach gerichtsärztlichem Gutachten absolut nicht transportfähig
sind, werden hier ihre Genesung oder ihren Tod abwarten dürfen.“

Doktor Schwertfeger verbeugte sich leicht und ließ sich
schwerfällig auf seinem Sitz nieder. Die Wirkung seiner letzten
Eröffnung war aber ganz eigenartig gewesen. \pagenum{16} Nur
vereinzelte Bravo-Rufe waren laut geworden, eine gewisse
Beklommenheit machte sich fast körperlich fühlbar, auf vielen
Gesichtern malte sich deutlich Schrecken und Angst, auf der Galerie
entstand Unruhe, eine Frau fiel mit dem Ruf: „Meine Kinder!“
ohnmächtig zusammen, und als der Kanzler geendet, erdröhnte zwar
starker Beifall, aber die kleine Gruppe der Sozialdemokraten schrie
unisono „Unerhört! Pfui! Skandal!“

Und nun erteilte der Präsident mit dem roten Bart dem
Finanzminister Professor Trumm das Wort. Trumm war klein, verhuzelt
wie eine halbgedörrte Pflaume, er sprach im Diskant und mußte sich
jedesmal unterbrechen, wenn seine Zunge zwischen dem Gaumen und dem
oberen Rand des falschen Gebisses stecken blieb. Unter großer
Spannung erörterte er die finanzielle Seite des
Ausweisungsgesetzes. Natürlich würde die Ablösung der jüdischen
Geschäfte und Immobilien nicht nur das christliche Privatkapital,
sondern auch die Mittel des Staates stark in Anspruch nehmen.
Hunderte von Milliarden Kronen würden kaum ausreichen, und man
dürfe sich nicht verhehlen, daß die Ausweisung der Juden zunächst
allerlei finanzielle Schwierigkeiten im Gefolge haben werde.

„Aber, gottlob,“ – der Finanzminister bekreuzigte sich – „wir
werden in den kommenden schweren Tagen nicht allein stehen! Ich
kann dem hohen Hause die erfreuliche Mitteilung machen, daß sich
das echte wahre Christentum der ganzen Welt gesammelt hat, um uns
zu helfen. Nicht nur, daß die österreichische Regierung seit
Monaten internationale Verhandlungen führt, auch der Piusverein hat
\pagenum{17} in aller Stille eine mächtige Agitation entfaltet,
die glänzende Früchte trägt. Der Verband des erwachten Christentums
der skandinavischen Länder, dem viele große Bankiers und Kaufleute
angehören, stellt uns einen gewaltigen Kredit in dänischer,
schwedischer und norwegischer Valuta zur Verfügung, der
amerikanische Industriekönig Jonathan Huxtable, einer der reichsten
Männer der Welt und ein begeisterter Streiter in Christo, hat sich
bereit erklärt, zwanzig Millionen Dollars in Österreich anzulegen,
der französische Christenbund macht hundert Millionen Francs mobil
– kurzum, es werden Milliarden Kronen ins Ausland wandern müssen
und dafür Milliarden in Gold einströmen!“

Riesige Begeisterung im ganzen Hause. Einige Dutzend Abgeordnete
verließen fluchtartig den Sitzungssaal und stürmten die Telephone,
um ihren Banken Verkaufsorders für fremde Valuten zu geben. Die
Hauszentrale konnte das stürmische Begehren nach Verbindungen mit
„Karpeles \& Co.“, „Veilchenfeld \& Sohn“, „Rosenstrauch \&
Butterfaß“, „Kohn, Cohn \& Kohen“ und wie alle die großen
Bankhäuser hießen, kaum bewältigen. Während aber der
Finanzminister, der eine volle Minute gebraucht hatte, um seine
eingeklemmte Zunge zu befreien, fortfuhr, erzählte der Engländer
Holborn in der Journalistenloge grinsend:

„Jonathan Huxtable ist ein frommer Kerl! Er spuckt Gift und Galle
gegen die Juden, seitdem ihm seine Frau mit einem jüdischen
Preisboxer durchgegangen ist. Er ist ein strenger Temperenzler,
aber er besauft sich jeden Tag mit Magentropfen, die er aus der
Apotheke bezieht. Einmal \pagenum{18} hat man gesehen, wie er
eine ganze Flasche Eau de Cologne auf einen Zug austrank. Und wenn
er hier zwanzig Millionen investieren wird, will er sicher fünfzig
daran verdienen.“

Doktor Wiesel schnitt ein abweisendes Gesicht, während die
jüdischen Journalisten sich rasch Notizen machten, um letzte
Bosheiten zu publizieren.

Die Pro- und Kontra-Redner meldeten sich zum Wort. Die
Sozialdemokraten sprachen gegen das Gesetz. Als aber ihr Führer
Weitherz in ruhigen und sachlichen Worten seiner Entrüstung
Ausdruck gab und den Gesetzentwurf als ein Dokument menschlicher
Schmach bezeichnete, entstand ein furchtbarer Tumult, die Galerie
warf mit Schlüsseln und Papierknäueln nach den Sozialdemokraten, es
kam zu einer Prügelei und die kleine Opposition verließ unter
Protest den Saal. Der christlichsoziale Abgeordnete Pfarrer
Zweibacher pries Doktor Schwertfeger als modernen Apostel, der
würdig sei, dereinst heilig gesprochen zu werden, die großdeutschen
Abgeordneten Wondratschek und Jiratschek aber beleuchteten das
Gesetz lediglich vom Rassenstandpunkt, und Jiratschek, der stark
mit böhmischem Akzent sprach, schluchzte vor Ergriffenheit und
schloß mit den Worten:

„Wotan weilt unter uns!“

Als letzter Redner ergriff unter Hepp! Hepp!-Rufen und höhnischem
Aih-Wai!-Geschrei der einzige zionistische Abgeordnete, Ingenieur
Minkus Wassertrilling, das Wort. Der schlanke, große und hübsche
junge Mann wartete \pagenum{19} mit verschränkten Armen ab, bis
Ruhe eintrat, dann sagte er:

„Verehrte Jünger jenes Juden, der sich, um die Menschheit zu
erlösen, törichterweise ans Kreuz hatte schlagen lassen!“

Stürmische Unterbrechung: „Hinaus mit den Juden!“

„Jawohl, meine Herren, ich stimme mit Ihnen in den Ruf: „Hinaus mit
den Juden!“ ein und werde mit freudigem Herzen dem Gesetz meine
Stimme geben. Wir Zionisten begrüßen dieses Gesetz, das ganz
unseren Zielen und Tendenzen entspricht. Von der halben Million
Juden, die das Gesetz trifft, wird sich wohl die Hälfte unter dem
zionistischen Banner vereinigen, die anderen werden, wie ich weiß,
in Frankreich und England, in Italien und Amerika, in Spanien und
den Balkanländern willig Aufnahme finden. Mir ist um das Schicksal
meines Volkes nicht bange, zum Segen wird das werden, was hier
gehässige Bosheit und Dummheit als Fluch gedacht hat.“

Der Tumult, der sich erhob, verschlang die weiteren Worte und
schließlich wurde auch der Zionist aus dem Saal gedrängt.

So ergab denn die Abstimmung, die namentlich erfolgte, die
einstimmige Annahme des Gesetzes, das noch am selben Tag durch den
Ausschuß und die zweite und dritte Lesung gepeitscht wurde.

Als die Abgeordneten spät abends endlich das Haus verlassen
konnten, sahen sie ein festlich beleuchtetes Wien. Von allen
öffentlichen Gebäuden wehten die weiß-roten Fahnen, Feuerwerke
wurden abgebrannt, bis lange nach \pagenum{20} Mitternacht
dauerten die Umzüge der Menschenmassen, die immer vor das
Kanzlerpalais marschierten, um Doktor Schwertfeger hoch leben zu
lassen und als Befreier Österreichs zu preisen~–~–~–

\tb{* * *}
Als der Nationalrat, Gemeinderat, Armenrat und Gewerberat Antonius
Schneuzel am nächsten Vormittag – es war ein Sonntag – infolge der
endlosen Siegesfeier arg verkatert am häuslichen Frühstückstisch
erschien, fand er eine recht unbehagliche Stimmung vor. Seine
Gattin hatte eine nadelspitze Nase, was auf Sturm deutete, seine
Tochter, Frau Corroni, saß mit verquollenen Augen da, ihr Gatte,
der Prokurist Alois Corroni, lächelte den Schwiegervater
impertinent und verächtlich an, und die beiden Enkelkinder Lintschi
und Hansl stießen ein furchtbares Geheul aus, als Herr Schneuzel
seine kleinen Äuglein verwirrt und ängstlich um den Tisch kreisen
ließ.

„Ja, was is denn da los?“

Frau Schneuzel stemmte die Arme in die Seite.

„Was los is, du Fallot, du? Gar nichts is los, als daß du alter
Tepp geholfen hast, deine Tochter und die Enkelkinder aus dem Land
zu treiben!“

„Ja, wieso denn?“ stammelte Herr Schneuzel, aber schon dämmerte ihm
grauenhafte Wahrheit. Richtig, er hatte im Laufe der Jahrzehnte
total vergessen, daß sein Schwiegersohn, Herr Alois Corroni, in
frühester Jugend Sami Cohn geheißen und erst stehend und aufrecht
die \pagenum{21} Taufe empfangen. Also mußte er ja hinaus und
mit ihm die beiden Kinder, die Judenstämmlinge waren!

„So eine Gemeinheit,“ schluchzte Frau Corroni in ihr Taschentuch
hinein, „was soll ich jetzt mit den Kindern anfangen? Nach Zion
auswandern vielleicht, du Rabenvater, du?“

„Jawohl, es ist ein starkes Stückchen,“ erklärte nun Herr Corroni
mit scharfer Betonung jedes Wortes, „einen Mann wie ich, der
behaupten darf, mindestens ein ebenso guter Christ zu sein als
tausend andere, die den ganzen Tag im Wirtshaus herumsitzen, einen
Mann wie ich, dessen Kinder im christlichen Glauben groß geworden
sind, aus dem Lande zu jagen wie einen tollen Hund!“

Herr Schneuzel wollte eine Erwiderung machen und murmelte etwas von
großer, heiliger Sache, Prinzipien, die auf Einzelfälle keine
Rücksicht nehmen können. Aber schon saß die Hand der Gattin in
seinen spärlichen Haaren und ließ nicht locker, bevor sie sich mit
einem ganzen Büschel des immer rarer werdenden Gewächses
zurückziehen konnte.

„Viecher seids Ihr alle zusammen! Gestohlen könnts Ihr mir werden
mit eurem Christentum! Hat der Loisl unser Annerl nicht immer gut
behandelt? Hat sie nicht einen Bisampelz von ihm bekommen, läßt er
die Kinder nicht aufwachsen wie die Prinzen? Dem lieben Gott sollst
du danken, daß sie einen Juden bekommen hat und nicht einen Kerl,
wie dich, einen Saufbruder und Skandalmacher!“

\pagenum{22}„I geh? net nach Zion“, heulte Lintscherl, während
Hans die Gelegenheit benützte, von Großvaters Teller weg den
Sonntagsgugelhupf zu grapsen.

Im Moment höchster Aufregung kam die Köchin Pepi herein, räumte
resolut den Tisch ab und erklärte seelenruhig:

„I geh?! I heirat? mein? Isidor, der was Kommis im Konsumverein is,
und wann er auswandern muß, wander? i mit ihm aus! Von mir aus
können sich die Herrn Nationenräte mitsamt dem Kränzler alle
zusammen aufhängen.“

Nachdem sich die Aufregung gelegt, erörterte Herr Corroni sachlich
die Situation.

„Ich denke natürlich gar nicht daran, nach Palästina auszuwandern,
schon deshalb nicht, weil man mich als getauften Juden gar nicht
hineinließe. Nein, ich habe einen Bruder in Hamburg, den Onkel
Eduard, wie Ihr wißt, und wenn er auch eben meiner Taufe halber bös
mit mir ist, so wird er mich jetzt nicht im Stich lassen – Juden
haben ja, gottlob, Familiensinn“ – diese Worte begleitete ein
stechender Blick gegen Schneuzel – „und ich werde eben dort für
mich und meine Familie eine neue Zukunft aufbauen. Es sei denn, daß
Annerl lieber bei euch bleiben will“.

Worauf Frau Anna, müde und verblüht, wie man es nach
fünfzehnjähriger Ehe zu sein pflegt, rosige Wangen bekam, ihre Arme
zärtlich um den Hals des Alois Corroni, rekte Sami Cohn, schlang,
ihn küßte wie eine Braut ihren Bräutigam und wirklich wie ein
junges Mädchen \pagenum{23} aussah. Und schließlich mußte sich
Herr Schneuzel völlig verstört und verzweifelt verpflichten, dem
Schwiegersohn so gewissermaßen als Fundament für die neue Zukunft
eine Million mit nach Hamburg zu geben.

Nachmittags ging der National-, Gemeinde- und Armenrat Schneuzel
allein zum Heurigen nach Sievering, fing dort mit einer
Gesellschaft, die noch immer „Hinaus mit den Juden!“ schrie, Streit
an, zerbrach seine Flasche an dem Schädel des einen Schreiers und
wurde furchtbar verprügelt.

\tb{* * *}
Gespräch in einer Fensternische des Kaffee Wögerer, gegenüber der
Börse, zwischen Herrn Strauß, Inhaber eines Bankhauses, und seinem
Neffen, dem Mediziner Siegfried Steiner. Solche und ähnliche
Gespräche fanden aber an allen Tischen statt, es wurde an diesem
Tage nicht lärmend, sondern fast lautlos mit Zuhilfenahme der Hände
geredet.

Der Neffe schüttelte dem Onkel die Hand.

„Lieber Onkel, ich danke dir dafür, daß du mich mit nach London
nehmen wirst. Das ist ein großer Trost für mich, denn unter uns
gesagt – Zion – ne, ist nichts für mich! Nur Juden, nicht
auszudenken!“

Der Onkel lächelte behaglich. „Zion kann mir gestohlen werden. In
London werde ich mich mit meinem alten Freunde Moe Seegward, der
dort eine Wechselstube in bester Lage hat, associieren.“

Siegfried Steiner beugte sich vor und flüsterte:

\pagenum{24}„Aber sag? mir eines, Onkel, du hast doch sicher
nicht der Steuerbehörde dein wirkliches Vermögen und Einkommen
angegeben. Wie wirst du nun dein Geld herüberkriegen, da doch seit
gestern Briefzensur eingeführt ist?“

Der Onkel ließ die Zigarrenasche auf seine Weste fallen.

„Chammer! Wozu hat man christliche Freunde? Ich war heute schon bei
dem Fabrikanten Schuster, habe ihm, unter uns gesagt, zwanzig
Millionen in Effekten und Bargeld gebracht und dafür von ihm eine
Anweisung auf eine Londoner Bank bekommen. Natürlich tut es der
Ganef nicht umsonst, sondern er verdient eine koschere Million
dabei.“

Der Neffe nickt befriedigt und an dreißig anderen Tischen endigten
verschiedene Gespräche ebenfalls mit einem zufriedenen Nicken.

Ein alter Hebräer mit Kaftan und Lockerln kam herein und sagte von
Tisch zu Tisch sein Sprüchlein auf: „Ein Almosen für einen alten
Juden, der beim Pogrom in Lemberg um Hab und Gut gekommen ist.“

Von einem Tisch wurde er angerufen: „Na, Alter, wohin werden Sie
auswandern?“

Der Jude wackelte mit dem Kopf. „Herrleben, wenn ich aus dem
brennenden Ghetto von Lemberg nach Wien gekommen bin, wer? ich auch
aus Wien wieder irgendwohin kommen. Ob ich schnorr? in Wien oder in
Berlin oder Paris, ist gleichgültig. Nur wer? ich dann nichts
erzählen mehr vom Pogrom, sondern davon, daß man hat mich alten
Juden ausgewiesen. Aber sagen Sie, Herrleben, \pagenum{25}
glauben Sie, man soll noch kaufen vor Torschluß Julisüd oder is
besser Siemens?“

\tb{* * *}
In der Villa des Schriftstellers Herbert Villoner in Alt-Aussee war
der Freundeskreis versammelt. Literaten von bekanntem Namen, Maler,
Bildhauer, Musiker, Verleger. Sonst pflegten sie erst im Hochsommer
die Sommerfrische aufzusuchen, diesmal hatten sie schon im Juni die
Stadtflucht ergriffen, um von den politischen Schmutzwellen
wenigstens nicht unmittelbar bespritzt zu werden.

Es war nach dem Abendessen, man saß in Korbstühlen auf der
Terrasse, blickte auf den lieblichen See, in dem sich der Mond
spiegelte, der Rauch der Zigaretten kräuselte in der unbeweglichen
Luft empor, jeder war in seine Gedanken versunken. Villoner
unterbrach das tiefe Schweigen.

„So ist denn kein Zweifel mehr, daß die meisten von uns zum
letztenmal den Sommer in Aussee verbringen werden und daß wir wie
vagabundierende Strolche den Staub von unseren Stiefeln werden
schütteln und in die Fremde gehen müssen. Wie seltsam! Mein Vater,
ein berühmter Kliniker, der nicht wenig zum Ruhm der Wiener
medizinischen Schule beitrug, mein Großvater, schon ein
erbangesessener Kaufmann vom Mariahilfer Grund und ich selbst –~–
Nun, man behauptet, daß ich in meinen Dramen und Romanen das Wiener
Wesen tief erfaßt und wie kein anderer die Wiener Jugend, das süße
Mädel erkannt und geschildert habe. Und nun ist das alles nichts
gewesen, ich bin einfach ein fremder Jude, der hinaus \pagenum{26}{
} muß wie irgend ein galizischer Flüchtling, den eine
Spekulationswelle nach Wien verschlagen!“

„Immerhin,“ sagte der junge Lyriker Max Seider leise mit zitternder
Stimme, „immerhin, Sie werden auch fern von der undankbaren Heimat
sich wohl fühlen können. Berlin wird Sie mit offenen Armen
aufnehmen, schon sind dort unter den Intellektuellen besondere
Ehrungen für Sie geplant, und Sie sind so reif und stark, daß Sie
mächtige Zweige werden treiben können, wo immer Sie sind. Aber was
soll ich tun? Ich bin erst am Anfang, und ich kann nur leben und
arbeiten, wenn ich durch das grüne Gelände des Wienerwaldes
schlendere, wenn ich als Wegweiser die zierliche Silhouette des
Kahlenberges vor mir sehe. Aus Ihnen strömt des Lebens Quelle in
unerschöpflichem Maß, ich muß um jede Zeile, um jeden Vers mit mir
ringen und kämpfen und das kann ich nur in Wien.“

„Ach was,“ schrie der Komponist Wallner ergrimmt, „der Teufel soll
dieses Wien mit seiner vertrottelten Bevölkerung holen! Ich geh?
nach Süddeutschland, miete mir ein Häuschen im Schwarzwald und
werde dort mit meiner Lene herrlich leben. Was, Schatz?“

Seine blonde junge Frau ließ es ruhig geschehen, daß der Gatte ihr
Madonnenköpfchen an seine Schulter zog, aber ein boshaftes Lächeln
huschte über den üppigen Mund und ihre Blicke kreuzten sich
verständnisvoll mit denen des Schriftstellers Walter Haberer.
Diesem schwellte Triumph die Brust. Er wußte, die Frau des
Komponisten blieb hier, niemand konnte sie zwingen, mit ihrem
Gatten ins Exil zu gehen, und verabredetermaßen würde sie endlich,
wenn \pagenum{27} der Mann erst fort, sein werden. Sein würde
aber nicht nur sie werden, sondern ganz Wien, ganz Österreich!
Denn sie alle, hinter denen er zurückstehen mußte, sie alle, deren
Theaterstücke aufgeführt wurden, während die seinen jahrelang in
den Schubladen der Dramaturgen schliefen, sie alle, die gestern
noch die großen Modeschriftsteller gewesen waren, sie alle, der
Villoner und der Seider, der Hoff und der Thal, der Meier und der
Marich, sie alle mußten fort und er blieb allein als Herrscher im
Reiche der Musen!

Frau Lene nickte ihm lächelnd zu, während der Gatte ihr liebkosend
die Wangen streichelte.

Donnernd und polternd lachte der große Schauspieler Armin Horch
auf.

„Meine Herrschaften, nun muß es heraus! Auch ich werde Österreich
verlassen müssen! Denn ich, den die „Wehr“ und andere Zeitungen
immer als den Verkörperer des christlichen Schönheitsideals
gepriesen haben, ich bin ein ganz gewöhnlicher Judenstämmling! Mein
Vater stammte aus Brody und hieß nicht Horch, sondern Storch!“

Schallendes Gelächter ringsumher, Galgenhumor quoll auf, Scherze,
die zur Situation paßten, wurden erzählt.

„Na und Sie, Herr Pinkus, wohin werden Sie Ihren Buchverlag
transferieren?“ fragte einer den dicken, kleinen Verleger mit den
krummen Beinen und dem prononciert jüdischen Gesicht.

„Ich? Ich bleibe! Ich bin doch Urchrist!“

Und als alles lachte, sagte er behaglich schmunzelnd:

\pagenum{28}„Spaß beiseite, ich bin ein waschechter Goi! Mein
Großvater Amsel Pinkus war ein Tuchhändler in Frankfurt am Main und
ein braver, frommer Jude. Als er sich aber in meine Großmutter,
Christine Haberle, eine kleine Sängerin aus Stuttgart, verliebte,
ließ er sich, da sie anders nicht die Seine werden wollte, taufen.
Nun, mein Vater heiratete wieder eine Christin und so bin ich
Christ in dritter Generation, also werde ich nicht ausgewiesen,
obwohl ich in Art und Äußerem ganz entschieden ein Duplikat meines
Großvaters bin.“

„Es lebe der Urchrist Pinkus,“ rief der Hausherr belustigt und alle
hoben lachend die Gläser. Da klang vom See her ein Knall wie ein
Peitschenhieb. Und von seltsamer Ahnung ergriffen, rief Villoner:
„Wo ist Seider?“

Aber schon brachten Leute die Leiche des jungen Lyrikers. Er hatte
sich unten am See erschossen, um seine müde, empfindsame Seele
nicht in der Fremde frieren lassen zu müssen.

\tb{* * *}
Bei der Lona in der Gumpendorferstraße herrschte geradezu
Panikstimmung. Acht junge Damen, eine schöner als die andere, waren
schon versammelt und immer wieder mußte die dicke Wirtschafterin,
Frau Kathi Schoberlechner, die Wohnungstür öffnen und ein Fräulein
hereinlassen. Im Salon roch es außerordentlich kräftig nach
Houbigant, Ambre, Coty, Rouge und Zigaretten, und es leuchtete und
funkelte von hellblonden, rotblonden, schwefelgelben und schwarzen
Haaren, Diamanten und Perlen. Alle waren in \pagenum{29} Spitzen
und Seide gekleidet, nur die Lona trug einen duftigen Schlafrock,
der vorn offen war, so daß ihr der schneeweiße Busen fast entquoll,
und ihre nackten Füße steckten in roten Pantöffelchen.

Die schwarze Yvonne weinte zum Herzzerbrechen, die rote Margit aber
schlug auf den Tisch und schrie erbost:

„Mir müssen demonschtrieren! Wann i? so an Nationalpülcher
derwisch, kratz? i eahm die scheangleten Augen aus!“

„A so a Gemeinheit! Was soll?n mir denn machen, wann s? die Juden
hinausschmeißen?“

Yvonne weinte noch heftiger. „Und grad jetzt, wo mir der Fredi
Pollak a neuches Automobil bestellt hat.“

„Mir gibt der Reizes, mit dem was ich seit zwei Wochen geh?,
fünfhundert Fetzen im Monat! Möcht? wissen, ob die Herren Christen
auch so splendid sein wer?n?“

„Ihr wißt ja eh, ich hab? den Zwitterbauch aus Mährisch-Ostrau, der
mich ganz aushält und nur amal im Monat auf a Wochen nach Wien
kummt!“

Eine üppige Juno mit gelben Haaren schlug die starken, aber schönen
Beine übereinander, daß man die blauseidenen Strumpfhalter sah,
leerte ein Gläschen Cointreau und sagte mit klingender Altstimme:

„Kinder, am meisten Erfahrung habe wohl ich im Leben! Und ich kann
nur sagen, wenn die Juden verschwunden sind, müssen wir alle
verhungern oder uns um Stellen als Klosettfrauen in Kaffeehäusern
umsehen. Geld lassen tun nur die Juden, die anderen wollen alle
viel Liebe und wenig Spesen! Zehn Jahre bin ich mit dem
\pagenum{30} Baron Stummerl vom Auswärtigen Amt gegangen, und in
diesen zehn Jahren hat er mir ein goldenes Armband, einen
Pelzkragen und tausend Gulden geschenkt. Ein Glück, daß ich dabei
noch den Herschmann von der Anglobank gehabt habe, sonst hätte ich
am Ende noch arbeiten müssen. Seither flieg? ich nur auf die
Israeliten!“

Claire spielte nervös mit dem goldenen, diamantbesetzten Kreuz, das
sie an einer Platinkette trug. „Was wohl der Karl sagen wird, wenn
ich vom Doktor Baruch nichts mehr bekomm?!“

Neue Klagen erhoben sich, Wehrufe wurden laut. Daran hatte man im
Drange der Geschehnisse noch gar nicht gedacht! Was sollte mit den
Freunden werden, die man liebte und aushielt, wenn die Freunde, die
zahlten, nicht mehr waren?

Da führte die Frau Kathi einen dieser Freunde herein. Pepi war das
Ideal eines feschen Kerls. Tiptop vom staubgrauen Samthut über die
gestrickte Krawatte hinweg bis zu den gelben Halbschuhen, über
denen man sanft getönte, blaue Seidenstrümpfe sah.

Schluchzend warf sich die reizende schwarze Yvonne in die Arme
ihres Herzensfreundes. Alle begrüßten ihn stürmisch, ein Hagel von
Rufen und Fragen ergoß sich über ihn. Pepi ließ sich ruhig in einen
Fauteuil fallen, zog Yvonne auf seine Knie, zwickte die neben ihm
sitzende Lona in die nackten Waden und sagte, nachdem er sich eine
Zigarette hatte in den Mund stecken lassen:

„Kinder, da kann man halt nichts machen, als auch auswandern!“

\pagenum{31}„Ja, woher wirst an? Auslandspaß kriegen und wer
laßt dich denn hinein?“, entgegnete die kluge goldblonde Carola.

„Sehr einfach“, lachte Pepi. „Morgen geh? ich aufs Rathaus, werde
konfessionslos, übermorgen geh? ich zur israelitischen
Kultusgemeinde, erkläre mich solidarisch mit dem mißhandelten
Judentum und werde Israelit. Hoffentlich ohne Operation. Dann
heiraten wir, bekommen unser Ablösegeld vom Staat und können nach
den Bestimmungen des Völkerbundes uns anderswo ansiedeln. Wir gehen
nach Paris oder nach Brüssel oder sonst wohin, wo was los ist.“

Yvonne lachte unter Tränen. „Geh?, was soll ich denn in Paris als
verheiratete Frau machen?“

„Tschapperl! Braucht ja niemand zu erfahren, daß wir verheiratet
sind! Nimmst dir eine Wohnung, suchst einen Freund, der dich
ordentlich aushält und ich bin so wie jetzt fürs Herz da!“

In den nächsten Tagen wußten die liberalen Blätter zu berichten,
daß hunderte von wackeren christlichen Jünglingen, empört über das
den Juden angetane Unrecht, demonstrativ ihren Übertritt zum
Judentum beschlossen hätten, um das Schicksal dieses schwer
geprüften Volkes zu teilen.

\tb{* * *}
\pagenum{32}Der Bundeskanzler, der auch Minister für auswärtige
Angelegenheiten war und seine Wohnung im Auswärtigen Amte hatte,
stand an einem milden Septembertag an der offenen Balkontüre und
sah über die Straße hinweg auf das Getriebe des Volksgartens. Aber
dieses Treiben schien ihm weniger lebhaft zu sein als in den
vergangenen Jahren, die weißlackierten Kinderwägelchen rollten nur
vereinzelt durch die Alleen, die Sesselreihen und Bänke waren trotz
des warmen Wetters nur spärlich besetzt.

Es klopfte, der Kanzler rief scharf: „Herein!“ und stand nun seinem
Präsidialchef, dem Doktor Fronz, gegenüber.

Schwertfeger war Ende Juni, kurz nach der Annahme des
Ausweisungsgesetzes, nach Tirol gefahren, um seine unter der Last
der Verantwortung und Arbeit fast zusammengebrochenen Nerven zu
erholen. In einem Dorf am Arlberg blieb er mehr als zwei Monate
inkognito, niemand außer seinem Präsidialchef kannte seinen
Aufenthalt, er ließ sich weder Briefe noch Akten nachschicken,
kümmerte sich nicht um die Zeitereignisse, und nur von ganz eminent
wichtigen Vorfällen durfte ihm Fronz schriftlich Mitteilung machen.
Tatsächlich war ja für alles vorgesorgt, der Wiener
Polizeipräsident wie die Bezirkshauptleute hatten ihre genauen
Instruktionen, das Parlament war bis zum Herbst vertagt, also
fühlte sich Doktor Schwertfeger entbehrlich, ja er hielt es für
seine Pflicht, neue Kräfte zu sammeln, um der kommenden Arbeit
frisch und stark gegenübertreten zu können. Heute vormittag war er
nach Wien zurückgekehrt und nun mußte ihm Fronz gründlich
referieren. \pagenum{33} Nachdem verschiedene
Personalangelegenheiten erledigt waren, ließ sich Schwertfeger
schwer und wuchtig vor seinem Schreibtisch nieder, nahm Papier und
Feder, um sich stenographische Notizen zu machen und sagte
äußerlich ruhig und kalt, während vor Spannung jeder Nerv in ihm
vibrierte:

„Nun, lieber Freund, berichten Sie mir über den bisherigen Vollzug
des neuen Gesetzes und seine sichtbaren Folgen. Wie ist unsere
Finanzlage? Sie wissen, ich bin völlig unorientiert.“

Doktor Fronz räusperte sich und begann:

„Finanztechnisch verläuft nicht alles so glatt, wie wir hofften.
Zuerst stieg unsere Krone in Zürich sprunghaft bis auf ein
Zwanzigstel Centime, dann traten leise, wenn auch unbedeutende
Schwankungen ein, seit Ende Juli rührt sich trotz des starken
Goldzustromes aus den Tresors der großen christlichen Vereine und
des Bankiers Huxtable unsere Krone nicht, sie beharrt auf dem Kurs
von 0.02. Merkwürdigerweise erfüllen sich vorläufig unsere
Hoffnungen auf enorme Geldabgaben seitens der Ausgewiesenen nicht.
Es fließen den Steuerämtern weder große Beträge in Kronen noch in
fremden Währungen zu. Es scheint, daß sich unter unseren
christlichen Mitbürgern tausende von Parasiten befinden, die in
gewissenloser Weise die überschüssigen, der Besteuerung
hinterzogenen Vermögen der Juden an sich nehmen und den Juden dafür
Abstandsummen in Gestalt von Anweisungen an ausländische Banken
geben.“

\pagenum{34}„Das war nicht anders zu erwarten“, sagte der
Kanzler, während ein verächtliches Lächeln um seine
zusammengekniffenen Lippen spielte. „Ob Jud? oder Christ –
habgierig und selbstsüchtig sind sie alle!“

Das dürften die Judenblätter nicht erfahren, dachte Fronz und fuhr
fort:

„Wie ich aus dem sehr pessimistischen Referat des Finanzministers
Professor Trumm folgern darf, wird uns die Ausweisung der Juden mit
ungeheuren Schulden, in Gold rückzahlbar, belasten, unseren
Banknotenumlauf aber in keiner nennenswerten Weise vermindern.“

„Geht die Liquidierung und Übergabe der Finanzinstitute, Banken
und Aktiengesellschaften glatt vor sich?“

„In dieser Beziehung ist alles in vollem Gange, aber leider zeigt
es sich, daß unsere einheimischen Kapitalisten entweder nicht
willens oder nicht in der Lage sind, die großen Unternehmungen an
sich zu reißen, so daß überwiegend Ausländer als Übernehmer in
Betracht kommen. Die Länderbank, die Kreditanstalt, die Anglobank,
die Escompte-Gesellschaft und andere Großbanken gehören bereits
Italienern, Engländern, Franzosen, Tschechoslowaken und so weiter,
desgleichen unsere großen Industrieunternehmungen. Eben hat ein
holländisches Konsortium die Simmeringer Lokomotivfabrik
übernommen. Wir passen natürlich höllisch auf, daß sich auf solchem
Umweg nicht ausländische Juden hier einnisten, und jeder
Kaufvertrag weist nachdrücklich auf die Klausel hin, wonach auch
ausländische Juden keinerlei Aufenthaltsrecht in Österreich
genießen, weder dauerndes noch vorübergehendes. Daß \pagenum{35}
die Aktionäre und Direktoren der fremden Gesellschaften, die hier
aufkaufen, zum Teile Juden sind, läßt sich aber nicht vermeiden.“

Der Kanzler stützte die mächtige, gewölbte Stirne in die knochige
Hand, wischte dann peinliche Gedanken mit einer Handbewegung fort
und sagte gleichmütig:

„Übergangserscheinungen, denen späterhin abzuhelfen sein wird! Wie
vollzieht sich die Ausweisung?“

„Genau nach den Durchführungsbestimmungen des Gesetzes! Sowohl die
Polizei als auch das Verkehrsamt arbeiten vortrefflich, täglich
verlassen ungefähr zehn Züge mit Ausgewiesenen Österreich nach
allen Richtungen und bis heute haben etwa vierhunderttausend Juden
das Land verlassen.“

Schwertfeger blickte überrascht auf. „Wie ist das möglich? Wir
haben an ungefähr eine halbe Million Auszuweisender gedacht! Also
waren jetzt, nach einem Drittel der präliminierten Zeit, vier
Fünftel erledigt?“

Doktor Fronz lächelte dünn. „Wir haben eben die große Zahl der
Konvertiten und Judenstämmlinge unterschätzt! Heute hat die
Staatspolizei mehr Überblick und sie rechnet nun nicht mehr mit
einer halben Million, sondern mit achthunderttausend, vielleicht
sogar mit einer Million Menschen, die unter das Gesetz fallen! Bei
dieser Gelegenheit möchte ich bemerken, daß sich gewisse
devastierende, oft sehr peinliche oder auch nur groteske Folgen der
Ausweisung zeigen. Zehn christlichsoziale Nationalräte müssen als
Judenstämmlinge landesverwiesen werden, beinahe ein Drittel der
christlichen Journalisten wird entweder \pagenum{36} direkt oder
in seinen Familienmitgliedern betroffen, es stellt sich heraus, daß
unsere besten christlichen Bürger vom Judentum durchtränkt sind,
uralte Familien werden auseinandergerissen, ja es hat sich etwas
ereignet, was schallendes Gelächter nicht nur in den Judenblättern,
die ja noch bis zum letzten Augenblick hetzen werden, erregt,
sondern auch in der Presse des Auslandes. Eine Schwester des
Fürsterzbischofs von Österreich, Kardinal Rößl, ist mit einem
Juden verheiratet, sein Bruder aber mit einer Jüdin, so daß seine
Eminenz durch das Gesetz sämtlicher Neffen, Nichten und Geschwister
beraubt wird. Vielleicht wird es sich doch empfehlen, unter solchen
Umständen der Nationalversammlung ein Amendement zu dem Gesetz zu
unterbreiten, durch das die Ausweisung von Judenstämmlingen unter
gewissen Umständen unterbleiben darf~–~–.“

Der Bundeskanzler sprang in die Höhe und schlug mit der geballten
Faust auf den Schreibtisch, daß die Tinte hochspritzte.

„Nie und nimmer, wenigstens nicht, solang ich im Amte bin! Eine
solche Ausnahmebestimmung würde das ganze Gesetz zum Weltwitz
machen, wir wären bis auf die Knochen blamiert, das internationale
Judentum würde triumphieren wie noch nie in seiner Geschichte, der
Korruption, der Bestechlichkeit wäre Tür und Tor geöffnet! Sie
kennen ja die gewissen Herren Hof- und Sektions- und Regierungsräte
mit den offenen Händen und leeren Taschen! Nein, es darf keine
Ausnahmen geben, das Leid und der Kummer einzelner Familien darf an
den Grundmauern des Gesetzes nicht rütteln! Im Namen der Habsburger
\pagenum{37} wurde ein Krieg geführt, der einer Million Männer
das Leben gekostet hat und man hat nicht zu mucksen gewagt! Was ist
im Vergleich dazu die Tatsache, daß ein paar tausend oder
vielleicht hunderttausend Menschen Unbequemlichkeit und Ärger
verursacht wird? Ich bitte Sie, in diesem Sinne die christlichen
Blätter zu instruieren. Besser noch, wenn die politische
Korrespondenz sofort eine diesbezügliche Enunziation der Regierung
den Blättern zugehen läßt. Und Sie bitte ich dringend, sich nicht
mehr zum Sprachrohr solcher Einflüsterungen machen zu lassen!“

Doktor Fronz verbeugte sich erblassend.

„Dann ist es ja auch überflüssig, wenn ich Eurer Exzellenz von
furchtbaren Jammerszenen berichte, die sich täglich bei der Abfahrt
der Evakuierungszüge beobachten lassen und die oft solche
Dimensionen annehmen, daß selbst der Straßenpöbel, der sich zur
Abfahrt der Züge mit der Absicht einzufinden pflegt, die
Ausgewiesenen zu beschimpfen, ergriffen schweigt und Tränen
vergießt~–~–.“

„Solche Szenen waren vorhergesehen und sind unvermeidlich!
Instruieren Sie sofort die Polizei dahin, daß die Bahnhöfe
abgesperrt werden, die Abfahrt der Züge tunlichst nur zur Nachtzeit
erfolgt und nicht von den Hauptbahnhöfen, sondern von den außerhalb
der Stadt gelegenen Rangierbahnhöfen. Und nun nur noch eine Frage:
Wie nimmt die Bevölkerung im allgemeinen die Durchführung des
Gesetzes auf?“

„Mit größter Begeisterung natürlich! Die Polizei läßt hundert
geschickte Agenten sich anonym in die Volksmengen mischen und
Beobachtungen sammeln. Nun, die Berichte \pagenum{38} gehen
übereinstimmend dahin, daß die christliche Bevölkerung sich
geradezu in einem Freudentaumel befindet, eine baldige Sanierung
der Verhältnisse, Verbilligung der Lebensmittel und gleichmäßigere
Verbreitung des Wohlstandes erwartet. Auch innerhalb der noch
sozialdemokratisch organisierten Arbeiterschaft ist die
Befriedigung über den Fortzug der Juden groß. Aber anderseits läßt
sich nicht verhehlen, daß die Bevölkerung erregt und unsicher ist.
Niemand weiß, was die Zukunft bringen wird, die Massen leben in den
Tag hinein, eine ganz staunenswerte Verschwendungssucht in den
unteren Klassen macht sich bemerkbar und die Zahl der
Trunkenheitsexzesse mehrt sich von Tag zu Tag.

Zur Gehobenheit der Stimmung trägt aber sehr wesentlich der Umstand
bei, daß die Wohnungsnot mit einem Schlage aufgehört hat. Allein in
Wien sind seit Beginn des Monates Juli vierzigtausend Wohnungen,
die bisher Juden inne hatten, frei geworden. Eine direkte Folge
davon ist, daß eine wahre Hochflut von Trauungen eingesetzt hat und
die Priester zehn und zwanzig Paare gleichzeitig einsegnen
müssen.“

Schwertfeger, der Junggeselle geblieben war, nickte befriedigt
lächelnd. „Damit wären wir also für heute fertig. Ich bin nun
halbwegs im Bilde und werde jetzt die Referate der einzelnen
Bundesministerien durchstudieren.“

Ein Kopfnicken und der Präsidialchef war entlassen. Fronz blieb
aber noch stehen und lenkte die Aufmerksamkeit \pagenum{39} des
Kanzlers, der schon ein Aktenfaszikel aufgeschlagen hatte, durch
diskretes Räuspern auf sich.

„Ich möchte Exzellenz noch darauf aufmerksam machen, daß der Wiener
Gemeinderat mit großer Stimmenmehrheit beschlossen hat, den
Schottenring in Dr. Karl Schwertfeger-Ring umzutaufen und daß
seitens dreihundert österreichischer Gemeinden ähnliche Umtaufungen
von Plätzen und Straßen beschlossen wurden. In Innsbruck hat sich
sogar ein Denkmalkomitee gebildet, das Eurer Exzellenz im nächsten
Jahr schon ein Denkmal aus Laaser Marmor errichten will.“

Der Kanzler stand auf, ging zum Balkon, sah wieder auf den
Volksgarten hinab, schritt mit wuchtigen Tritten schwer und plump
zweimal durch den großen Raum und sagte dann:

„Inhibieren Sie alle solchen Ehrungen! Sie sollen verschoben werden
bis zum zehnjährigen Jubiläum der Befreiung Wiens von den Juden!“

\tb{* * *}
Weihnachtsabend im Hause des Hofrates Franz Spineder. Weit draußen
in Grinzing, außerhalb der Endstation der Straßenbahn, lag das
kleine, gelbe Backsteinhäuschen, das der Hofrat noch von seinem
Großvater ererbt hatte. Von außen sah das einstöckige Haus mit dem
großen grün gestrichenen Holztor und den grünen Jalousien fast
primitiv aus, aber wenn man das Tor öffnete und in den Hof mit dem
altertümlichen Ziehbrunnen trat, \pagenum{40} blieb man
überrascht und entzückt stehen. Der Hof ging in einen sanft
ansteigenden Garten über, der schier endlos war. Im Sommer
leuchteten die Levkojen, Tulpen, Rosen und Nelken in südlicher
Pracht, hinter dem Ziergarten kamen Hunderte von Bäumen, die unter
der Last der Äpfel, Birnen, Aprikosen, Pflaumen und Kirschen sich
tief zur Erde beugten, und wenn man auch die Obstbäume hinter sich
hatte, so war man noch immer nicht am Ende des Gartens, sondern
ging steil durch einen Weinberg, um endlich ganz oben auf ein
altwienerisches Lusthäuschen mit bunten Scheiben zu stoßen.

Köstlich wie der unvermutete Garten war auch die Einrichtung der
Wohnzimmer. Uralte, behagliche, steife und graziöse Möbel aus der
Barock-, Kongreß- und Biedermeierzeit, kostbare Stiche und Bilder
an den Wänden, zwei echte Waldmüller, ein Schwind im Salon, bunte,
schöne Gläser, Altwiener Porzellan, funkelndes Silbergerät in den
Vitrinen und Kredenzen, und man brauchte nur die Augen zu
schließen, um die Männer und Frauen im Kostüm der Maria
Theresianischen Zeit und Biedermeierrock vor sich zu sehen.

Franz Spineder war Beamter, wie es sein Vater und sein Großvater
gewesen, aber er war auf den Gehalt eines Hofrates im
Unterrichtsministerium nicht angewiesen, sondern recht vermögend,
und schon das Haus mit dem riesigen Garten und der kostbaren
Einrichtung repräsentierte heute einen nach vielen Millionen
zählenden Wert. Außerdem aber war seine Frau eine geborene
Halbhuber, deren Urgroßväter schon als Gerber und Lederfabrikanten
soliden \pagenum{41} Reichtum erworben hatten. Und da das
Ehepaar Spineder nur mehr ein Kind, die jetzt knapp achtzehnjährige
Lotte, besaß, so konnte es inmitten der Wirrnisse einer zerrissenen
Zeit und aller Teuerung zum Trotz sein behagliches Leben führen.

Schweigend schmückten Lotte und Frau Spineder den Weihnachtsbaum,
befestigten an den duftenden Zweigen die Schokoladekringel,
Bonbons, Glaskugeln und Kerzen. Frau Spineder, noch immer eine
hübsche, runde Frau, sah die blonde, schlanke, auffallend schöne
und liebreizende Tochter von der Seite an.

„Lotte, nun hast du schon wieder Tränen in den Augen! Bedenk? doch,
daß Papa heute wenigstens fröhliche Gesichter sehen will und mach?
dem armen Leo das Herz nicht noch schwerer.“

Lotte ließ einen kleinen Rauchfangkehrer aus Schokolade fallen, daß
sein Kopf fortrollte, schlug die Hände vor das Gesicht, lehnte sich
an die Schulter der Mutter und begann bitterlich zu schluchzen.

„Mutter, mir bricht das Herz! Du wirst sehen, ich werde es nicht
überleben, daß Leo in die Fremde fort muß! Mutter, laßt mich doch
mit ihm ziehen!“

Frau Spineder, der selbst das Wasser in den Augen stand,
streichelte zärtlich das weiche, wie Gold leuchtende Haar der
Tochter.

„Lotte, es geht nicht! Bedenk? doch, Papa ist sechzig und er hat,
seit uns der unselige Krieg den Sohn genommen, niemanden als dich.
Du kannst es ihm nicht \pagenum{42} zumuten, daß er dich in die
ungewisse Zukunft ziehen läßt, so gern er ja auch den Leo hat.
Schau nur, Leo wird nach Paris ziehen; bei der Entwertung der Krone
könnten wir euch unmöglich mit Francs unterstützen und ihr würdet
vielleicht ins Elend kommen, ohne daß Papa helfen kann. Leo wird
sich allein schon durchschlagen und ihr seid ja noch beide so jung,
daß ihr auf andere, bessere Zeiten warten könnt?. Still jetzt, der
Vater kommt! Und es klingelt, der Leo wird auch schon da sein.“

Herr Spineder, der jetzt eintrat, um die Kerzen anzuzünden, war der
Typus des alten österreichischen Hofrates in seiner besten Art.
Musik liebend und ausübend, voll innerlicher Kultur, gepflegt von
außen und innen, ein Schönheitssucher, Lebensfreund und
Lebensbejaher, rechtlich, gewissenhaft, tolerant und dabei doch ein
wenig beschränkt, bedächtig und zögernd. Er trug auch jetzt noch
den veralteten Kaiserbart, weil er es unter seiner Würde hielt, dem
Umschwung der Verhältnisse an seiner Person Konzessionen zu machen,
er war Demokrat durch und durch, ein treuer Diener der Republik,
aber das schöne Kaiserbild von Angeli hing noch immer über seinem
Schreibtisch. Wie er jetzt eintrat, war der alte Herr mit den
schlohweißen Haaren und den milden, graublauen Augen der echte
Altösterreicher, den man bald nur mehr aus Büchern kennen wird.

„Leo ist draußen und kratzt sich den Schnee von den Sohlen ab“,
sagte Hofrat Spineder, während er die Kerzen bedächtig anzündete.
„Geht hinauf zu ihm, ich werde die Bescherung machen und klingeln,
wenn es so weit \erratum{ist.“.}{ist.“}

\pagenum{43}Frau Spineder sah noch rasch in die Küche nach dem
Karpfen, der Sachertorte und den Krapfen; Lotte hing aber schon am
Halse Leos und schluchzte wortlos an seiner Brust.

Leo Strakosch, schlank, dunkelhaarig, glattrasiert, mit lebhaften
braunen Augen, aus denen Klugheit und Humor blitzten, war um zehn
Jahre älter als Lotte. Im letzten Kriegsjahre war er als
Einjähriger eingerückt und im Felde hatte er den gleichaltrigen
Rudolf Spineder, den Sohn des Hofrates, kennen und als Freund
lieben und schätzen gelernt. In der letzten Piaveschlacht hatte
Rudolf einen Kopfschuß bekommen und in den Armen des Freundes seine
junge Seele ausgehaucht, nachdem er ihn gebeten, die Eltern und das
Schwesterchen zu grüßen. So war Leo in das Haus des Hofrates
gekommen, der arme Sohn eines kleinen Agenten, fühlte sich in dem
vornehm-bürgerlichen Milieu unendlich wohl, und als Lotte aus einem
Kinde ein blühendes, schönes Mädchen wurde, stand es in ihm fest:
Diese oder keine! Lotte erwiderte die Liebe des lebhaften,
geistvollen, begabten jungen Mannes von ganzem Herzen.

Hofrat Spineder sah die Entwicklung dieser Liebe und hatte nichts
einzuwenden. Leo Strakosch war Radierer, in jungen Jahren schon
ganz außerordentlich erfolgreich, man begann sich um seine
Zeichnungen zu reißen, eine vor einem Jahr erschienene Leo
Strakosch-Mappe erregte Aufsehen auch im Ausland, und der Hofrat
wie seine Frau sagten sich mit Recht, daß sie ihr Kind in keine
besseren Hände würden geben können, als in die Leos, den sie nach
und \pagenum{44} nach liebten wie ihren eigenen Sohn. Daß Leo
Jude war, focht den Hofrat nicht im mindesten an. In seinem Hause
verkehrten viele Musiker, Literaten, Maler, die Mehrzahl von ihnen
waren Juden, und der verstorbene Rechtsanwalt Viktor Rosen war
sogar der intimste Freund Spineders gewesen.

Als vor Jahresfrist zuerst in politischen Kreisen von dem Plan des
Führers der Christlichsozialen, ein Antijudengesetz durchzubringen,
geraunt wurde, hatte Hofrat Spineder daran nicht glauben wollen und
können. Und als er daran glauben mußte, war seine Empörung maßlos
gewesen. Und noch größer sein Schmerz über den Schicksalsschlag,
den die bevorstehende Ausweisung Leos für seine Tochter bedeutete.
Den Gedanken aber, seine Lotte mit Leo ins Exil ziehen zu lassen,
wies er weit von sich, die Liebe zu seinem einzigen Kind und der
Egoismus des Alternden vereinigten sich hier und
\erratum{machte}{machten} ihn absolut unerbittlich.

\tb{* * *}
Die Bescherung war sehr reichlich ausgefallen, Lotte von den Eltern
freigebig bedacht worden, aber sie schenkte dem Pelzkragen, den
Seidenstrümpfen, den Büchern und Noten kaum einen Blick, sondern
preßte immer wieder das kleine Bild Leos, das er ihr in einem
goldenen Medaillon geschenkt, an die zuckenden Lippen. Man saß nun
beim festlich geschmückten Tisch, aber es herrschte eher Trauer als
Feststimmung und vergeblich versuchte der Hofrat ein leichtes
Gespräch zu entwickeln. Als dann der \pagenum{45}
selbstgekelterte goldgelbe Wein kredenzt wurde, erhob Hofrat
Spineder sein Glas und sagte mit bewegter Stimme:

„Dein Wohl, Leo! Möge das Glück dich auch in der Fremde begleiten,
möge das Schicksal in absehbarer Zeit uns alle wieder vereinigen!
Kinder, ich weiß, daß ihr mir grollt und ich kann doch nichts tun,
als mit euch leiden. Seht, Mutter und ich haben den besten Teil des
Lebens hinter uns, ich stehe an der Schwelle des Greisenalters, und
so ist es doch nur natürlich, wenn wir uns mit allen Fasern dagegen
sträuben, den letzten Sonnenstrahl, der uns noch leuchtet,
fortziehen zu lassen. Aber selbst wenn wir solcher schier
übermenschlicher Selbstlosigkeit fähig wären, würde mich das
Pflichtgefühl davon abhalten. Lebten wir in normalen Zeiten, so
ließ ich euch ziehen und würde sagen, daß wir ja schließlich
alljährlich ein paar Monate bei euch in Paris zubringen können.
Aber das ist heute unmöglich, da die Krone fast wertlos ist. Nur
Spekulanten können sich noch solchen Luxus leisten, und ihr wißt,
daß wir in guten, geordneten Verhältnissen leben, aber doch mit
jedem Tausendkronenschein rechnen müssen. Würde Lotte jetzt mit dir
in die Fremde gehen, so müßte sie das Elternhaus für immer
verlieren. Und nicht nur sie, sondern auch euere Kinder wären
entwurzelt, vaterlandslos, würden nicht wissen, wo ihre Großeltern
in der Erde ruhen. Und wer weiß, es würde der Tag vielleicht
kommen, wo du, Lotte, von solcher Heimatssehnsucht erfüllt wärest,
daß sie deine Liebe zum Gatten verdrängen und dein ganzes Wesen
sich in einen bitteren Vorwurf gegen den, dem du in die Verbannung
gefolgt, wandeln würde. Ihr seid beide \pagenum{46} jung, du,
Lotte, bist fast noch ein Kind, du Leo, ein Jüngling und das ganze
Leben liegt vor euch. Lasset ein paar Jahre vergehen, vielleicht
seid ihr dann voneinander losgekommen oder aber es traten
Entwicklungen ein, die euch doch noch vereinigen.“

Während Lotte fassungslos weinte und mit ihr ihre Mutter, hob nun
auch Leo sein Glas.

„Vater, so darf ich dich ja doch wohl noch nennen, ich muß die
Gründe deiner Weigerung, Lotte mit mir ziehen zu lassen, würdigen,
wahrscheinlich würde ich an deiner Stelle nicht anders handeln.
Aber eines sage ich dir, sage ich Lotte, die ich nie aufhören werde
zu lieben: Mein Leben wird von nun an ein einziger Kampf werden!
Man sagt meinem Volke Zähigkeit nach – nun so will ich die ganze
Fähigkeit meines Volkes in mir vereinigen. Mit Kopf und Herz, mit
meinem ganzen Können und Wollen werde ich darauf hinarbeiten, Lotte
zu gewinnen, so oder so! Man kann mich vertreiben wie einen
räudigen Hund, man kann aber den Willen in mir nicht töten! Und ich
leere mein Glas auf euer Wohl und auf unsere Vereinigung, die
früher kommen wird als wir alle heute zu hoffen wagen!“

Am nächsten Tage fuhr Leo Strakosch mit einem Zuge fort, der sich
zum großen Teil aus geistigen Arbeitern und Künstlern
zusammensetzte. Hofrat Spineder, Frau Spineder und Lotte gaben ihm
das Geleite. Außer ihnen ließ Leo nichts zurück, was ihm wert war,
da seine Eltern längst nicht mehr lebten.

\tb{* * *}
\pagenum{47}Der letzte Jahrestag wurde für Wien zu einem
Festtag, wie ihn die lustige und leichtsinnige Stadt noch nie
erlebt hatte. Unter Aufbietung aller Verkehrsmittel, mit Hilfe von
Lokomotiven, die aus den Nachbarstaaten entliehen waren, bei
Einstellung jedes sonstigen Personen- und Güterverkehrs war es
gelungen, an diesem Tag in dreißig riesigen Trains die letzten
Juden fortzubringen. Vormittags fuhren die Direktoren und leitenden
Funktionäre der Großbanken, mittags die jüdischen Journalisten mit
ihren Familien. Sie hatten bis zum letzten Augenblicke ausgeharrt,
noch die Abendblätter waren von ihnen geschrieben und redigiert
worden, und erst als die feuchten Blätter aus den
Rotationsmaschinen flogen, rückten die neuen Herren in die
Redaktionsstuben ein. Die Mehrzahl der Wiener Journalisten hatte
Engagements bei reichsdeutschen und deutschböhmischen Blättern
gefunden, viele wanderten nach Amerika aus, einige wenige
beschlossen, sich anderen Berufen zuzuwenden. Der Herausgeber der
großen „Weltpresse“ aber übersiedelte mit einem kleinen Stabe von
Mitarbeitern nach London, um dort unter dem Titel „Im Exil“ eine
deutsche Wochenschrift, die sich in erster Linie mit Österreich
befassen sollte, erscheinen zu lassen.

Um ein Uhr mittags verkündeten Sirenentöne, daß der letzte Zug mit
Juden Wien verlassen, um sechs Uhr abends läuteten sämtliche
Kirchenglocken zum Zeichen, daß in ganz Österreich kein Jude mehr
weilte.

In diesem Augenblicke begann Wien sein großes Befreiungsfest zu
feiern. Von hunderttausend Häusergiebeln wurden die rot-weiß-roten
Fahnen gehißt, Tücher in diesen \pagenum{48} Farben schmückten
alle Geschäfte, Lampions vor allen Fenstern wurden entzündet, und
bei sternenheller Frostnacht zog eine Million Menschen über den
knisternden Schnee, um sich zu Zügen zu vereinigen. Männer, Frauen
und Kinder trugen Lampions, Musikkapellen marschierten den
einzelnen Bezirksgruppen voran, ein Jauchzen und Jubeln ertönte,
und immer wieder zerriß der Ruf: „Es lebe das christliche Wien“,
die Luft!

Treffpunkt aller Züge war das Rathaus. In feenhafter Pracht lag der
schöne, gotische Bau Meister Schmidts da. Millionen elektrischer
Lichter ließen ihn wie eine einzige Flamme leuchten. Auf einer
Tribüne spielten die unvergleichlichen Wiener Philharmoniker, von
Juden gesäubert und daher ein wenig reduziert, volkstümliche
Weisen, und der Wiener Männergesangverein bot seine besten Lieder
dar. Die Volkshalle, der große Platz vor dem Rathaus, der Ring vom
Schottentor bis zur Bellaria bildeten eine einzige Menschenmauer,
und um acht Uhr war es kein Rufen mehr, sondern ein Heulen aus
einer Million Kehlen, das immer wieder erdröhnte.

Endlich kam der große Moment. Bürgermeister Karl Maria Laberl
erschien mit dem Bundeskanzler Doktor Schwertfeger auf dem Balkon.
Der Bundeskanzler ergriff zuerst mit machtvoller Stimme, die sich
bis jenseits des Ringes Gehör verschaffte, das Wort. Er sprach
kurz, trocken, aber um so wirkungsvoller:

„Mitbürger, ein ungeheures Werk ist vollendet! Alles das, was in
seinem innersten Wesen nicht österreichisch ist, \pagenum{49}
hat die Grenzen unseres kleinen, aber schönen Vaterlandes
verlassen! Wir sind nun allein unter uns, eine einzige Familie, wir
sind fürderhin auf uns und unsere Eigenart gestellt, mit eigener
Kraft werden wir unser gesäubertes Haus frisch bestellen, morsche
Mauern stützen, geborstene Pfeiler aufbauen. Wiener und Brüder aus
dem ganzen Bundesstaat! Wir feiern heute ein Fest, wie es noch nie
gefeiert wurde. Morgen beginnt ein neues Jahr und für uns alle ein
neues Leben. Morgen dürfen wir noch ruhen und uns beschaulich
besinnen. Dann aber müssen wir arbeiten, wie wir noch nie
gearbeitet haben. Unser ganzes Können müssen wir unserem Vaterlande
widmen, jede Stunde muß genützt werden. Wir werden der ganzen Welt
zeigen müssen, daß Österreich auch ohne Juden leben kann, ja daß
wir eben deshalb gesunden, weil wir das Fremde aus unserem
Blutkreislauf entfernt haben. Mitbürger, schwört es mir in dieser
feierlichen Stunde in die Hand, daß wir alle nicht mehr schwelgend
in den Tag hineinleben wollen, sondern arbeiten, arbeiten und
nichts als arbeiten, bis uns die Früchte unserer Arbeit erblüht
sind.“

Und der Ruf: „Wir schwören es!“ brauste auf, fremde Menschen
schüttelten einander die Hände, Männer und Frauen sanken einander
weinend und lachend in die Arme, die neue Volkshymne wurde
intoniert und mitgesungen und dann erklang ohne Verabredung und
doch wie aus einem Munde das „Hoch unser Doktor Schwertfeger, der
Befreier Österreichs!“

\pagenum{50}Als sich der Jubel und Tumult ein wenig gelegt
hatte, kam endlich auch Bürgermeister Herr Karl Maria Laberl zum
Wort. Er begann seine Ansprache mit den Worten:

„Meine lieben Christen!~–~–“

Aber viel mehr vernahm die Menge nicht, denn dem warmen Föhn, der
seit Minuten durch die vorher noch so kalte Nacht fegte, folgte in
diesem Augenblick ein Regenguß, und schreiend, kreischend
zerstreute sich die Menschenmasse, um durch ein Meer von Kot und
zerflossenem Schnee zu den Straßenbahnen zu eilen.

\chapter{Zweiter Teil.}
\pagenum{51}

\begin{center}
\textit{Lotte Spineder an Leo Strakosch, Paris, Rue Foch 22.}

\end{center}
Mein Lieber, nun ist genau ein Jahr vergangen, seitdem ich Dir auf
dem Westbahnhofe mit meinem von Tränen ganz durchnäßten Taschentuch
nachgewinkt habe. Und das erste Weihnachtsfest, das ich als Deine
Braut ohne Dich verbringen mußte, liegt hinter mir. Es war wieder
recht traurig, und Papa meinte sehr besorgt, daß ich noch ganz
krank und elend werden würde, wenn ich mich meinem Schmerz so
hingebe. Ich bin jetzt nämlich immer sehr blaß, schlafe schlecht,
habe viel Kopfschmerz und werde gleich so müde. Unser Hausarzt
meint, es sei Bleichsucht und hat mir Guberquelle verordnet, aber
ich weiß, daß es nur meine Sehnsucht nach Dir ist, die mich schwach
und krank macht.

Unsagbare Freude hat mir Deine wundervolle Mappe bereitet, die
gerade am Weihnachtsabend eingetroffen ist. Du bist jetzt, wie man
aus diesen herrlichen Stichen sieht, ein ganz großer Künstler;
Papa, der doch so viel davon versteht, meint, daß Du schon zu den
ersten Meistern gehörst und hat furchtbar auf unsere Regierung
geschimpft, die solche Männer, statt sie zu ehren, aus dem Lande
jagt. \pagenum{52} Dein Brief, in dem Du von Deinen großen
Erfolgen berichtest, hat mich natürlich sehr beglückt, und Papa hat
umgerechnet, daß die dreißigtausend Francs, die Du für diese Mappe
bekommen hast, viele Millionen österreichischer Kronen sind. Die
Krone ist nämlich wieder riesig gefallen. Nur als ich las, daß Du
so viel in Gesellschaft verkehrst und dich der Einladungen in die
feinsten Häuser kaum erwehren kannst, bekam ich ordentlich
Herzklopfen. Wirst Du bei den schönen Pariserinnen nicht Deine
arme, kleine Lotte ganz vergessen? O Leo, was soll nur aus uns
werden, wann werde ich wieder meinen Kopf an Deine Schulter legen
können? Weißt Du, Leo, neulich flog ein großer Äroplan über den
Kahlenberg westwärts, und da habe ich gedacht, daß ich, wenn ich
die Möglichkeit dazu hätte, gleich zu Dir nach Paris fliegen würde,
ob meine Eltern es nun erlauben oder nicht. Überhaupt, wenn ich
wüßte, wie man, ohne daß es jemand erfährt, einen Paß bekommt,
würde ich mir von Dir Geld schicken lassen und heimlich zu Dir
kommen. Ich weiß, daß ich Papa und die Mutter damit furchtbar
kränken würde, aber meine Sehnsucht nach Dir ist so groß, daß ich
ganz schlecht und grausam geworden bin.

Du bittest mich, ich möge Dir in großen Zügen die Entwicklung der
Dinge schildern, seitdem die Israeliten fort sind, da Du aus den
farblosen und langweiligen Wiener Zeitungen kein richtiges Bild
bekommen kannst. Nun, ich will versuchen, Dir alles zu erzählen,
was ich selbst sehe oder von den anderen weiß; aber wenn es dumm
wird, so darfst Du mich nicht auslachen.

\pagenum{53}Also, von dem großen Jubel und den Festzügen am
Silvestertage, als die letzten Israeliten Wien und Österreich
verlassen hatten, wirst Du ja ohnedies alles aus den Zeitungen
ersehen haben. Nun, den ganzen Januar hielt diese Stimmung an, die
Leute machten alle fröhliche Gesichter, ein Festkonzert folgte dem
anderen und immer wieder zogen die Massen vor das Rathaus oder das
Kanzlerpalais, um dem Bürgermeister Laberl und dem Doktor
Schwertfeger zu huldigen. Mir selbst ist es aufgefallen, daß die
Wiener in der Elektrischen viel freundlicher und netter waren als
vorher, und der Hofrat Tumpel, der bei uns verkehrt, Du weißt, der
mit dem blonden Vollbart, den Du nie leiden mochtest, sagte
triumphierend zu uns:

„Sehen Sie, das Wiener sonnige Gemüt, das so lange von all dem
Fremden überschattet worden war, bricht sich wieder Bahn.“

„Ja, Schnecken,“ brummte der Vater, „das ist nur, weil den Wienern
das Ganze eine Riesenhetz ist und weil die Lebensmittel billiger
und wieder Wohnungen zu haben sind.“ Tumpel meinte aber: „Oho,
lieber Freund, das ist es nicht allein, sondern die indogermanische
Naivität unseres Volkes wagt sich wieder heraus!“

Die Lebensmittel waren wirklich viel billiger geworden, weil unsere
Krone damals sehr gut, nämlich auf 0·02 Centime stand. Ich erinnere
mich, daß Mama im Winter einmal ganz froh nach Hause kam und sagte,
man könne jetzt wieder existieren, das Schweineschmalz kostet nur
mehr zehntausend Kronen per Kilogramm. Und das mit den
\pagenum{54} Wohnungen hat den Wienern wirklich so viel Freude
gemacht. Stelle Dir nur vor: Plötzlich hingen fast an allen
Haustoren Zettel, auf denen Wohnungen und möblierte Zimmer
angeboten wurden. Die Leute gingen rein zum Zeitvertreib von Haus
zu Haus, um die Wohnungen zu besichtigen. Und den ganzen Tag sah
man Möbelwagen durch die Straßen fahren.

Das dauerte so bis zum Fasching, aber dann war die gute Laune weg.
Plötzlich begann große Arbeitslosigkeit zu herrschen. Die ganze
Konfektionsindustrie stand still, und jeden Augenblick hörte man,
daß dieses oder jenes Geschäft abgekracht sei. Die Blätter
schrieben, man müsse die ehrlichen christlichen Kaufleute, die die
alten jüdischen Geschäfte übernommen hatten und ihrer Aufgabe noch
nicht gewachsen seien, von Staats wegen unterstützen. Die
Arbeitslosen machten aber großen Krawall, zogen über den Ring,
demolierten ein paar Geschäfte, schlugen Fensterscheiben ein und
setzten es durch, daß ihnen der Staat zehntausend Kronen täglich
Arbeitslosenunterstützung zahlte. Da begann die Krone zu fallen,
weil, wie Papa mir erklärte, der Banknotenumlauf enorm stieg. Auf
ja und nein stand die Krone wieder auf 0·01 Centime und die
Lebensmittel wurden wieder so teuer und noch teurer als früher.
Heute erzählte Mama ganz aufgeregt, daß die Butter schon
dreißigtausend Kronen kostet. Seit dem Frühjahr sind die Leute
wieder sehr mürrisch und in der Elektrischen wird viel geschimpft.
Hauptsächlich auf die Schieber, die alles verteuern, aber man
spricht nicht von jüdischen Schiebern, sondern nur so im
allgemeinen.

\pagenum{55}Du fragst, ob ich viel ins Theater gehe? Ach nein,
lieber Leo, wenn man die Oper ausnimmt, so ist in den Theatern gar
nichts mehr los. In den Schauspielhäusern wird ununterbrochen
Ganghofer und Anzengruber gespielt, weil man von Israeliten nichts
aufführen darf und die Klassiker ja doch nicht ziehen. Eine
Zeitlang hat man auch viel von Shaw gegeben; seitdem er aber in
einer englischen Zeitung erklärt hat, Wien sei ein internationales
Dummheitsmuseum geworden, ist er verpönt. Hauptsächlich aber
deshalb, weil er auch gesagt hat, ein gescheiter Jude sei ihm
lieber als zehn dumme Christen. Die Operettentheater sind alle
pleite. (Erinnerst Du Dich, wie ich lachen mußte, als ich von Dir
zum erstenmal das Wort pleite hörte?) Es hat sich nämlich
herausgestellt, daß sämtliche alte und neue Operetten von Juden
entweder komponiert oder geschrieben sind, meistens beides. Auch
fehlt es an Kräften, denn fast alle Tenore mußten ja auswandern.
Wohl sind rasch ein paar ganz arische Operetten herausgebracht
worden, aber das Publikum hat gezischt, weil es ein furchtbarer
Schmarren war. Der Hofrat Tumpel meinte, daß sich die christliche
Kunst eben nur für seriöse Sachen eigne, nicht für frivoles Zeug.
Worauf Papa schmunzelte und sagte, man würde bald einsehen, daß
sich die Juden und Christen hierzulande sehr gut ergänzt haben.

Neulich ist mir mittags am Graben aufgefallen, daß man heuer viel
weniger elegante Herren und Damen sieht als früher. Es wird eben
gar kein Modeluxus mehr getrieben. Allerdings muß ich sagen, daß
mir die widerlichen jüdischen Schiebergesichter, über die Du Dich
auch immer so \pagenum{56} geärgert hast, gar nicht fehlen.
Dafür machen sich auf dem Korso sehr viele junge Lackeln, die wie
Bauern aussehen und unmöglich angezogen sind, mit mächtigen
Uhrketten und Diamantringen an den dicken Fingern, breit.
Überhaupt scheint unser ganzer Fremdenverkehr nur mehr aus Bauern
zu bestehen. Der Besitzer vom Hotel Imperial hat neulich in einer
Zeitung geklagt, daß er jetzt Gäste habe, die sich mit den
genagelten Schuhen ins Bett legen und ihre Jägerwäsche in der
Badewanne waschen. Wenn Du durch die Kärntnerstraße gehen würdest,
so würdest Du schauen, wie wenig elegant die Geschäfte jetzt sind!
Nun muß ich aber schließen, weil es schon ein Uhr nachts ist und
ich auch nichts Besonderes mehr weiß. Lebe wohl, mein Geliebter,
und denke was aus, damit wir bald wieder beisammen sind, weil ich
sonst gar nicht mehr leben mag. Es küßt Dich vieltausendmal Deine
ganz verzagte

\unterschrift{
Lotte.“
}

\tb{* * *}
Herr Habietnik ging düster, schweigend, mit gerunzelter Stirne
durch die prunkvollen Verkaufsräume des großen Modehauses in der
Kärntnerstraße, das einst Zwieback geheißen und jetzt den Namen
Wilhelm Habietnik trug. Herr Habietnik war der erste Verkäufer in
der Damenmaßabteilung gewesen, und mit Hilfe der Mittelbank
deutscher Sparkassen war es ihm gelungen, bei der großen
Judenvertreibung das Haus an sich zu bringen. Herr Habietnik ging
nun, wie gesagt, von Saal zu Saal, wechselte in jedem ein paar
Worte mit dem Rayonchef, sein \pagenum{57} Antlitz wurde immer
finsterer und er stieß unwillige Rufe aus. Durch die ganz in Weiß
und Rosa gehaltene Abteilung für Babywäsche schritt er, ohne sich
aufzuhalten, in den entzückenden Konditoreisalon, der vollständig
leer war, warf er nur einen schiefen Blick, dann stürmte er in sein
Privatkontor und ließ sich den Prokuristen Smetana kommen.

„Sie, Herr Smetana, so geht das nicht weiter, da muß etwas
geschehen! Wir stehen vor Ostern, früher war das die Hochsaison und
man konnte vor Gedränge gar nicht durch das Haus gehen, und jetzt
habe ich auf meinem Rundgang drei alte Weiber gefunden, von denen
zwei zusammen eine Chenillepelerine, wie sie gar nicht mehr
existieren, kaufen wollten und eine einen Barchentunterrock. Wenn
wir so weitermachen, können wir sperren. Sagen Sie, wie groß ist
das Betriebsdefizit, seitdem ich die Firma übernommen habe?“

Der Prokurist Smetana lächelte sauer:

„Na, so an die hundert Millionen, das wird wohl reichen!“

Herr Habietnik ging aufgeregt auf und ab. „Ich versteh? das nicht!
Wir haben früher, wie die Juden noch da waren, doch auch eine Menge
christliche Käuferinnen gehabt! Wo sind denn die hingekommen?“

Smetana, der früher in der Buchhaltung gesessen und die Rechnungen
ausgeschrieben hatte, lächelte.

„Herr Habietnik, mit den christlichen Kundschaften war es nie weit
her, und wenn es schon Christinnen waren, so hatte ihr Christentum
doch irgendwo ein Klampferl. Entweder sie waren die Frauen oder die
Maitressen von Juden. \pagenum{58} Bitt? Sie, da erinnere ich
mich an die schöne Gräfin Wurmdorf, die was zuletzt noch eine
Redoutentoilette für eineinhalb Millionen bei uns hat machen
lassen. Na, wer aber hat sie gezahlt? Der Herr Gemahl vielleicht?
Keine Spur! Der reiche Eisler von der Firma Eisler und Breisler!
Und die Manoni von der Oper, die was die Tochter von einer
waschechten christlichen Waschfrau ist und zehn gute Millionen im
Jahr bei uns gelassen hat? Na, bei der hat die ganze israelitische
Kultusgemeinde herhalten müssen! Und die~–“

Herr Habietnik winkte ab. „Trotzdem, es gab genug Damen ohne
Liebhaber, die ganz schön eingekauft haben. Ich weiß das besser,
weil ich doch gerade die Maßabteilung unter mir hatte.“

„Ja, sehen Sie, Herr Habietnik, wenn es schon keine Jüdinnen waren,
so war es eben die Konkurrenz der Judenfrauen, die uns geholfen
hat. Wenn die Jüdinnen fein und elegant gekleidet waren, so wollten
die christlichen Damen der guten Gesellschaft nicht zurückstehen.“

„Da können Sie recht haben“, meinte der Chef nachdenklich. „Neulich
habe ich selbst gehört, wie die Frau Artander die Preise
bekrittelte und ohne zu bestellen mit den Worten wegging: „Ach was,
heutzutage hat man es ja gottlob nicht mehr notwendig, sich so
aufzutackeln und jede Modedummheit mitzumachen. Ich werde eben die
alten Sachen herrichten lassen.“

Herr Habietnik bekam einen roten Kopf und schlug mit der Hand auf
den Tisch. „Ich habe Sie aber nicht gerufen, um mit Ihnen zu
schmusen, sondern weil ich einen \pagenum{59} Rat von Ihnen
will! Dazu zahl? ich Ihnen ja den hohen Gehalt!“

Smetana knickte zusammen. „Eine Idee hätt? ich schon, Herr von
Habietnik. Die Leut? tragen jetzt so viel Loden und andere solide
Sachen. Sie haben es ja selbst gesehen, sogar nach Barchent ist
Nachfrage. Wie wäre es, wenn wir ein paar Fenster mit Lodenstoffen,
Lodenröcken, Barchent- und Flanellwäsche füllen würden? Und dazu
eine schöne Tafel und viel Inserate mit der Ankündigung: Loden,
Barchent, Baumwolle und Flanell – die hohe Pariser Mode!“

Herr Habietnik bekam einen Lachkrampf und krümmte sich so lange,
bis ihm die Tränen über die Backen liefen. „Flanell und Loden – die
große Pariser Mode! Sie, wenn das die Frau Ella Zwieback, die jetzt
in Brüssel lebt, erfährt, so glaubt sie, daß wir in Wien alle
zusammen verrückt geworden sind! Aber meinethalben, mich ekelt die
ganze Geschichte schon an, ich bekomme Platzangst, wenn ich durch
das leere Haus gehe! Gut, machen Sie Lodenfenster! Und
Steirerhüteln dazu nicht vergessen und genagelte Schuhe! Und die
Konditorei verwandeln wir langsam, aber sicher in eine
Stehbierhalle mit heißen Würsteln. Mir ist schon alles egal, so
kapores oder so!“

Zehn Tage später sah man richtig hinter einem der Fenster rote,
blaue und gemusterte Flanellröcke, Hosen, gestrickte
Miederleibchen, hinter einem anderen Baumwollstrümpfe und solides
Schuhzeug und in einer der Auslagen türmten sich Lodenstoffe in
Braun, Grau und Schwarz zu Bergen. Und die Verkaufsräume füllten
sich, bis der \pagenum{60} Bedarf der weitesten Kreise gedeckt
war und die Verkäuferinnen wieder verstohlen hinter ihren schwarzen
Seidenschürzen gähnten oder Engelhornromane lasen.

\tb{* * *}
Im Kaffee Imperial saß der Rechtsanwalt Dr. Haberfeld und schob die
Zeitungen, die ihm der alte Zahlmarkör Josef gebracht hatte,
unwirsch beiseite.

„Sie, Josef, leer ist es jetzt bei euch, daß man neben dem Ofen
friert! Früher hat man mühsam sein Platzerl ergattern können und
jetzt, jetzt könnt? man bei euch das Traberderby abhalten, weil eh?
kein Mensch im Weg steht!“

Josef strich die ergrauten Bartkoteletten, machte tieftraurige
Augen, wischte mit der Serviette über den Tisch und sagte
sorgenvoll:

„Es geht eh? ein Ringkaffee nach dem andern ein, ich glaub?, lang?
wer?n mir?s auch net mehr machen. Wissen S?, Herr Doktor, was die
Herren Israeliten – pardon, die Juden, waren, die sind halt alle
gern in die feinen Lokale gegangen, wo was los ist und man was
sieht. Aber die christlichen Herrschaften, die geh?n ins
Vorstadtkaffeehaus und spielen ihr Tarock oder machen eine
Billardpartie und gehen sonst lieber zum Heurigen oder ins
Wirtshaus. ?s ist halt eine andere Zeit jetzt!“

„Das merkt ein Blinder, der taubstumm ist“, brummte der Anwalt.
„Sie, Josef, wir zwei kennen einander ja schon lange genug und
brauchen uns keine Komödie vorzuspielen. Mir g?fallt halt die ganze
G?schicht net! Wien versumpert ohne Juden!“

\pagenum{61}Josef fuhr erschreckt zusammen und sah sich
ängstlich um.

„Ah was, es hört uns eh? niemand! Wien versumpert, sag? ich Ihnen,
und wenn ich als alter, graduierter Antisemit das sag?, so ist es
wahr, sag? ich Ihnen! Ich wer? Ihnen was sagen, Josef. Wenn ich
gegessen hab?, muß ich, Sie wissen?s ja am besten, immer mein
Soda-Bikarbonat nehmen, um die elendige Magensäure zu bekämpfen.
Wenn ich aber gar keine Magensäure hätt?, so könnt? ich überhaupt
nichts verdauen und müßt? krepieren. Und wissen S?, der
Antisemitismus, der war das Soda zur Bekämpfung der Juden, damit
sie nicht lästig werden! Jetzt haben wir aber keine Magensäure, das
heißt, keine Juden, sondern nur Soda, und ich glaub?, daran wer?n
wir noch zugrund? geh?n!“

Josef, der mit atemloser und ehrfürchtiger Spannung gelauscht
hatte, schlug verzweifelt mit der Serviette auf einen Stuhl und
flüsterte beklommen:

„Recht haben S?, Herr Doktor, wenn man sich auch net traut, es laut
zu sagen. Mit dem Zugrundegehen aber fang? ich schon an! Ich habe
im letzten Halbjahr die Hälfte von meinem Ersparten aufgebraucht.
Herr Doktor, unter uns gesagt, und weil Sie selbst ein nobler Herr
sein, den was es nicht treffen tut: Die Herren Israeliten, pardon,
ich mein? die Juden, waren nobel im Trinkgeldgeben!“

Josef räumte die Zeitungen fort, die dem Doktor Haberfeld zu
langweilig waren, brachte auf seinen Wunsch das Prager und das
Berliner Tagblatt und wandte sich \pagenum{62} anderen, eben
eingetretenen Gästen zu, die sich je ein Viertel Wein bestellten.

„Wie in einem Beisel“, raunte Josef dem Rechtsanwalt im
Vorübergehen zu. Und dieser nickte verständnisvoll, zündete sich
eine Zigarre an und gedachte der Zeiten, da er allabendlich im
Kreise jüdischer Kollegen hier gesessen und trotz aller politischen
Gegnerschaft manch? klugen und guten Gedanken mit ihnen
ausgetauscht hatte~\ldots{}

\tb{* * *}
Der Frühlingsbeginn, der seit jeher als politisch aufgeregte Zeit
gegolten hat, brachte auch diesmal den Wienern unruhige Tage. Die
Arbeitslosigkeit griff erschreckend um sich, eine Fabrik nach der
anderen stellte den Betrieb ein, aber auch die Konkurse der
Detailgeschäfte häuften sich und allenthalben gab es lärmende
Kundgebungen, nicht nur der Arbeiter, für die der Staat halbwegs
sorgte, sondern auch der entlassenen Kommis und Verkäuferinnen,
Buchhalter und Tippmädels, bis in bewegter Ministerratssitzung
beschlossen wurde, auch diesen Kategorien für die Zeit ihrer
Stellenlosigkeit Zuschüsse zu gewähren. Der Finanzminister hatte
sich mit Händen und Füßen dagegen gesträubt, der Kanzler, Doktor
Schwertfeger, aber schließlich seinen Willen durchgesetzt. Doktor
Schwertfeger, der noch starrer, knochiger, härter geworden war,
erklärte, daß auch diese neue Belastung getragen werden müsse.

„Wir dürfen es nicht dazu kommen lassen, daß eines Tages der
Ausweisung der Juden die Schuld an Not und \pagenum{63} Elend
gegeben wird. Wir haben bis heute die „Arbeiter-Zeitung“, die jetzt
zwar von Christen, aber doch noch im jüdischen Geiste geschrieben
wird, bewegen können, jede Kritik des Antijuden-Gesetzes zu
unterlassen. Erfüllen wir die Forderungen der Stellungslosen im
kaufmännischen Betriebe nicht, so wird ihr die Geduld reißen und
sie wird, schon um diese Leute in ihr Lager zu drängen, eine
Polemik eröffnen, die verderblich werden kann, weil wir die
Übergangszeit von der Judenherrschaft zur Befreiung noch nicht
hinter uns haben.“

„Und unsere Krone?“ wandte der Finanzminister Professor Trumm
höhnisch ein.

„Wir müssen uns an unsere christlichen Freunde im Auslande wenden
und ihnen unsere Bedrängnis klar machen. Am besten, Sie fahren
gleich nach Paris und London.“

Trumm lachte laut auf. „Ganz vergeblich! Schon von der ersten
Bittfahrt vor drei Monaten bin ich mit leeren Händen gekommen! Die
Leute geben nichts mehr, haben ja sogar ihre festen Versprechungen
nicht ganz gehalten. Sie unterschätzen den Einfluß unserer früheren
Konnationalen, der österreichischen Juden, die zum Teil heute in
den ausländischen Banken sitzen! Und abgesehen davon, der
christliche Begeisterungstaumel ist vorbei und man steht wieder auf
dem kalt-geschäftlichen Standpunkt. Sogar Mister Huxtable hat
abgewinkt. Also meinethalben, bewilligen wir die Forderungen der
stellenlosen kaufmännischen Angestellten! Aber ich wasche meine
Hände in Unschuld.“

\pagenum{64}Am nächsten Tag wurde der Kabinettsbeschluß
verlautbart, es trat wieder Ruhe ein, aber am zweitnächsten Tag
fiel die Krone an der Züricher Börse um dreißig Prozent. Und die
„Neue Züricher Zeitung“ veröffentlichte einen Artikel, in dem sie
ziffernmäßig nachwies, daß Wien langsam aber sicher aufhöre,
irgendwelche Bedeutung für den mitteleuropäischen Handelsverkehr zu
haben und der Rivalität Prags und Budapests unterliege. „In Ungarn
ist man nach dem Ende des Horthy-Regimes ebenso schlau wie in Prag
gewesen. Man hat gewisse Kategorien von anständigen Juden mit
offenen Armen aus Wien aufgenommen und dadurch den Handel an sich
gerissen. Die Einkäufer der ganzen Welt können, weil sie zum großen
Teil Juden sind, ohnedies Wien nicht mehr besuchen, sie gehen nach
Prag, Brünn und Budapest, in erster Linie natürlich nach Berlin,
das reißt die christlichen Einkäufer mit, die österreichischen
Erzeuger von Fertigfabrikaten, wie Ledergalanterie, Schuhe, Keramik
und so weiter, müssen, statt die Einkäufer bei sich zu empfangen,
mit dem Musterkoffer nach dem Ausland reisen, kurzum, es werden
trotz des beispiellos niedrigen Standes der Krone in Wien keine
nennenswerten Geschäfte gemacht. Damit hat naturgemäß in Wien auch
das Schiebertum in Valuten sein Ende erreicht, aber wie es scheint,
auf Kosten des österreichischen Organismus. Der geniale
Bundeskanzler Doktor Schwertfeger hat mit seinem Gesetz keine
große, sondern eine allzugroße Tat getan!“

Und wie zur Bekräftigung der Wahrheit dieses Artikels begann sich
in Wien eine völlige Deroutierung des \pagenum{65} Bankenwesens
einzustellen. Die ausländischen Konsortien, die die Wiener
Großbanken übernommen hatten, sahen sich in ihren Hoffnungen bitter
enttäuscht. Ihr Umsatz wurde immer geringer, mit dem Fortgang der
Juden hatte auch das Börsenspiel einen beträchtlichen Rückgang
aufzuweisen, und die Banken waren genötigt, wenn sie nicht mit
Verlust arbeiten wollten, eine der Tausenden von Bankfilialen, mit
denen Wien überfüllt war, nach der anderen aufzulassen. Vergebens
legte die Organisation der Bankbeamten dagegen Protest ein, daß ein
Teil ihrer Mitglieder brotlos gemacht wurde. Die Banken steckten
sich hinter ihre Gesandtschaften, es kam zu peinlichen
diplomatischen Interventionen, die damit endeten, daß die
österreichische Regierung, statt ihre eigenen Beamten abzubauen,
noch die stellenlosen Bankangestellten in ihren Dienst nehmen
mußte. Und die Krone fiel auf ein Tausendstel Centime.

\tb{* * *}
An einem schönen, sommerlich warmen Maimorgen kam vom Westbahnhof
her ein Automobil vor das Hotel Bristol gefahren, dem ein
eleganter, schlanker, dunkelhaariger Herr entstieg. Der
Hoteldirektor musterte mit geübtem Blick den schweren Lederkoffer
und das Handgepäck und dann erst den Fremden, dem ein kleines
Knebelbärtchen im Verein mit dem aufgezwirbelten und in Wien sehr
unmodernen Schnurrbart einen exotischen Anstrich verlieh.
Südfranzose! taxierte der Direktor, rechnete rasch im Kopf
französische Franken in Kronen um, und beschloß, dem erstaunlichen
Resultat gemäß, den Zimmerpreis zu \pagenum{66} stellen. Auf die
französisch vorgebrachte Frage, ob ein Zimmer frei sei, erwiderte
er, ein ironisches Lächeln mühsam unterdrückend:

„Jawohl, Monsieur, ein einzelnes Zimmer gefällig oder ein
Appartement mit Bad? Mit Aussicht auf den Ring oder nach
rückwärts?“

Der Passagier ließ vor Erstaunen das eingeklemmte Monokel fallen.

„Ja, wie ist denn das? Früher konnte man doch ohne vorherige
Anmeldung nirgends unterkommen!“

„Mein Herr,“ seufzte der Direktor jetzt tief und ehrlich, „Sie
waren wahrscheinlich anderthalb Jahre oder länger nicht mehr in
Wien! Seither hat sich viel verändert!“

Der Fremde war sofort im Bilde, nickte verständnisvoll, forderte
ein Appartement auf die Ringstraße hinaus und füllte dann den
Meldezettel aus.

„Henry Dufresne, Kunstmaler aus Paris, 29 Jahre alt, katholisch,
ledig.“

Monsieur Dufresne nahm ein Bad, kleidete sich um, pfiff dabei
vergnügt einen Pariser Gassenhauer vor sich hin, ließ sich ein
vorzügliches Frühstück auf dem Zimmer servieren und verließ dann so
gegen zehn Uhr vormittags ersichtlich aufgeräumt das Hotel.

Der Franzose mit dem Knebelbärtchen kannte sich in Wien entschieden
gut aus, denn er schwang sich ohne jemanden zu fragen, auf einen
Straßenbahnwagen, und er mußte auch die deutsche Sprache vorzüglich
beherrschen, denn man sah ihm an, daß er den Gesprächen der
Umstehenden interessiert lauschte. Als eine alte Frau über die
Teuerung zu \pagenum{67} jammern begann und arg auf die hohe
Obrigkeit schimpfte, klopfte Herr Dufresne sie auf die Schulter und
meinte in tadellosem Deutsch und wienerischem Akzent besänftigend:

„Wie kann man nur so was sagen, Mutterl, wir müssen doch alle froh
und glücklich sein, weil wir die Juden losgeworden sind.“

Aber das Mutterl begehrte jetzt erst recht auf.

„Mir ham? die Juden nie was g?tan! Wegen meiner hätten s? in Wien
bleiben können. A so a gute Bedienung hab? i bei an jüdischen Herrn
g?habt und alleweil, wann er a Madl mit nach Haus g?bracht und an
Unordnung g?macht hat, hat er mir an Hunderter extra g?schenkt.
Leben und leben lassen, hat er immer g?sagt und recht hat er
g?habt!“

Die Leute auf der Plattform lachten und ein biederer Mann mit einer
weinselig funkelnden Nase meinte bestätigend:

„Ja, das derf man schon sagen, es hat auch anständige Leut? unter
den Juden ?geben!“

Ein eigenartiges Lächeln spielte um den Mund des Franzosen, der nun
ausstieg und langsam zu Fuß die Währingerstraße entlang
schlenderte, dann in die Nußdorferstraße einbog, mitunter vor einer
Auslage kopfschüttelnd stehen blieb, die Preise der ausgestellten
Waren zur Kenntnis nahm und so schließlich in die Billrothstraße
kam, die im weiteren Verlauf nach den rebenreichen Vororten
Sievering und Grinzing führt.

Ein Zettel am Haustor eines modernen Zinspalastes in der
Billrothstraße fesselte seine Aufmerksamkeit.

\pagenum{68}„Kleine, elegant möblierte Wohnung mit Atelier
sofort zu vermieten. Auskunft erteilt der Portier.“

Kurz entschlossen betrat Herr Dufresne das Haus und suchte den
Portier auf, der ihn mittelst Lift nach dem fünften Stock führte
und die Wohnung zeigte. Sie bestand aus einem Schlafzimmer, einem
als Herrenzimmer eingerichteten Salon, an den sich ein
atelierartiger, großer Raum mit Glasdach schloß. Auch ein
Badezimmer war vorhanden.

„Wie kommt es, daß die Wohnung leer steht?“

„I, du meine Güte,“ rief der Portier, „in Wien stehen jetzt an die
zwanzigtausend Wohnungen leer! Diese da hat ein Architekt, ein Herr
Rosenbaum, gehabt, der mit den anderen Juden fort mußte. Der
Hausherr hat ihm die Möbel abgekauft, konnte aber bis heute keinen
Mieter finden, weil keine Küche dabei ist.“

Nach weiteren fünf Minuten hielt der Portier einen
Zehntausendkronenschein als Angabe in der Hand, und Herr Dufresne
war Besitzer der Wohnung. Als er jetzt mit beschleunigten Schritten
gegen Grinzing ging, wirbelte er vergnügt sein Spazierstöckchen in
der Luft und murmelte vor sich hin: „Der Anfang ist gut, besser
hätte ich es mit der Wohnung gar nicht treffen können.“ Je näher er
aber Grinzing kam, desto erregter wurde er, seine Wangen färbten
sich rot und seine braunen lustigen Augen leuchteten wie im Fieber.
Nun hatte er die Kobenzlgasse erreicht und seine Schritte wurden
langsam, fast schleppend, wie die eines Mannes, der einem
schicksalsschweren Augenblick entgegengeht. Vor dem Hause des
Hofrates Spineder blieb \pagenum{69} er tiefatmend stehen und
zog sich den grauen Kalabreserhut in die Stirne, daß man nur mehr
seinen Knebelbart und das Kinn sah. Unschlüssig ging er auf und ab,
mitunter nervös auf die Armbanduhr sehend, die auf halb zwölf wies.
Gerade als er wieder vor dem grünen Tor stand, ging dieses auf und
ein Dienstmädchen verließ das Haus. Und eben in diesem Augenblick,
als das Tor offen stand, sah Herr Dufresne, wie von der links im
Hofe gelegenen Wohnungstür ein junges, weißgekleidetes Mädchen mit
goldblonden Haaren, die kein Hut verdeckte, in der Hand ein Buch,
den Hof nach rückwärts durchschritt und den Garten aufwärts ging.

„Hurra!“ schrie der Mann mit dem Knebelbart in sich hinein und sein
Kriegsplan war fertig. Rechts vom Spinederschen Grundstück lag, von
ihm durch einen Holzzaun getrennt, ein langer, leerer Bauplatz,
seit dem Kriege provisorisch in einen riesigen Gemüsegarten
verwandelt. Der Länge nach zog sich dieser Gemüsegarten bis hoch
hinauf zum Lusthäuschen auf der höchsten Stelle des
Spinedergartens. Auf der anderen Längsseite war das Grundstück
ebenfalls durch einen Holzzaun von einer Nebengasse der
Kobenzlgasse getrennt, aber dieser Zaun war verwahrlost und wies
mehrfach Unterbrechungen auf. Durch eines der Löcher kroch nun der
Franzose, eilte mit langen Sätzen den Gemüsegarten aufwärts, wobei
er links von sich das blonde Mädchen gehen sah und es bald hinter
sich ließ. Nun war Herr Dufresne ganz oben, mit einem Ruck schwang
er sich über den Zaun in den Garten des Hofrates Spineder hinüber
und versteckte sich hinter einem mächtigen Lindenbaum,
\pagenum{70} der mitten im Weingarten stand. Einige Minuten
später war das Mädchen beim Baum angelangt, aber es konnte den Mann
hinter dem Baum nicht sehen. Bis plötzlich Unerwartetes geschah.
Herr Dufresne rief halblaut: „Lotte!“ Und als Lotte Spineder
erschreckt und verwirrt stehen blieb und sich umsah, rief er
wieder: „Lotte! Ich bin es, um Himmelswillen erschrick nicht!“

Im nächsten Augenblick hielt der Herr mit dem Knebelbart Lotte, die
schneeweiß geworden war und zu schwanken begonnen hatte, in seinem
Arm. Und immer wieder preßte er seinen Mund auf ihre kalten Lippen,
bis sich ihre Wangen färbten und sie ihn, am ganzen Körper bebend,
fest umklammerte, als wollte man ihn ihr entreißen.

Und nun saßen sie im Lusthäuschen, Leo Strakosch hielt Lotte auf
seinem Schoß und erzählte in fliegenden Worten:

„Ja, Lottchen, ich bin es, und dir zuliebe habe ich mir diesen
entsetzlichen Napoleonbart plus Schnurrbart wachsen lassen. Ich
habe es einfach vor Sehnsucht nach dir nicht mehr ausgehalten, und
als mir dein Vater schrieb, daß er ernstlich um deine Gesundheit
besorgt sei und es für richtiger halte, wenn wir den Briefwechsel,
der in dir alle Wunden immer wieder aufreiße, einstellen würden,
war mein Plan gefaßt. Ich vertraute mich einem lieben, guten
Kameraden, Henry Dufresne, der für mich ins Feuer gehen würde, an,
ließ mir den Knebelbart, wie er ihn hat, stehen und bekam von ihm
sämtliche Papiere, als da sind: Taufschein, Heimatschein,
Militärzeugnis und den ordnungsgemäß von der österreichischen
Gesandtschaft in Paris \pagenum{71} vidierten Paß. Wir sahen
durch den Bart einander so ziemlich ähnlich, so daß er es riskieren
konnte, sich seinen Paß mit meiner Photographie zu besorgen. Und
meine Unterschrift hat er nachgemacht und nicht ich seine. Der gute
Junge hat natürlich allen Freunden und Bekannten erzählt, daß er
nach Wien fährt, in Wirklichkeit ist er auf das Gut seines Onkels
in Südfrankreich gegangen, wo er ein Jahr bleibt. Und genau so
lange, als er dort ist, kann ich hier in Wien als Henry Dufresne
leben.“

Lotte schluchzte und lachte in einem Atem.

„Leo, ich bin ja so namenlos glücklich! Aber ich habe auch solche
Angst um dich! Du weißt, es steht die Todesstrafe auf die verbotene
Rückkehr – was, wenn sie dich erwischen?!“

„Ganz ausgeschlossen, mein Lieb! Die wenigen Freunde, die ich
hatte, sind Juden und mußten so wie ich das Land verlassen.
Außerdem bin ich tatsächlich durch den Bart unkenntlich, besonders,
wenn ich ein Monokel trage. Und selbst wenn jemand käme und
behaupten würde, daß ich Leo Strakosch bin – ich würde einfach
leugnen und niemand könnte mich überführen, denn mein Paß ist echt,
die Unterschrift ist echt, und wenn man bei der Polizei in Paris
anfragen sollte, so würde man die Auskunft bekommen, daß Henry
Dufresne mit Reisepaß nach Wien abgereist sei.“

„Aber Papa und Mama?“ fragte Lotte nach etlichen herzhaften Küssen,
die ihr trotz Schnurrbart und Mouche wohl taten.

\pagenum{72}„Die dürfen natürlich nicht ein Sterbenswörtchen
erfahren, Lotte“, meinte Leo ernst. „Nicht, daß sie mich anzeigen
würden! Aber dein Papa ist zu sehr Beamter und Hofrat, um mir eine
solche Mystifikation nicht übel zu nehmen, und außerdem würde er
unter keinen Umständen dulden, daß wir zusammenkommen, sondern mich
beschwören, wieder fortzufahren. So aber werden wir uns täglich
sehen, nicht wahr, Lotte?“

Und Leo erzählte ihr von der behaglichen, kleinen Wohnung, die er
eben gemietet und schilderte, wie sie dort täglich ein paar
Stunden, so lange Lotte sich eben würde freimachen können, zusammen
verbringen würden. Lotte war nur über und über rot geworden, aber
sie sah in die ehrlichen und treuen Augen ihres Bräutigams und
wußte, daß sie auch ganz allein mit ihm in guter Hut sein würde.

Lotte konnte jeden Augenblick im Garten gesucht werden und Leo
mußte verschwinden. Bevor sie aber Abschied nahmen, bewölkte sich
wieder die weiße Stirne des Mädchens.

„Leo, du hast nun deine glänzende Karriere in Paris aufgegeben! Was
aber willst du hier in Wien tun, wie bei dieser schrecklichen
Teuerung, über die nun auch Papa zu klagen beginnt, deinen
Unterhalt bestreiten?“

Leo lachte so vergnügt und laut, daß ihm Lotte erschreckt die
Finger auf den Mund legte. Was er für eine Aufforderung nahm, die
kleinen rosigen Finger zu küssen. Er tat es reichlich und sagte
dann:

„Mein Liebes, was ich hier tun werde? Arbeiten, und zwar tüchtig,
und ungeheuer viel Geld sparen, weil diese \pagenum{73} Wiener
Teuerung, in Franken umgerechnet, lächerlich billig ist. Du mußt
nämlich wissen, daß ich von der größten Pariser Verlagsfirma den
Auftrag bekommen habe, eine neue Gesamtausgabe der Werke Zolas zu
illustrieren. Glänzende Bedingungen, sage ich dir. Sechzigtausend
Francs, wovon ich die Hälfte bei Abschluß der Vertrages bekommen
habe. Die andere Hälfte erhalte ich, wenn ich die zweihundert
Zeichnungen abliefere, und das muß in einem Jahr sein. Also, du
siehst wieder einmal: Wir Juden sind schlau und wissen, wo unser
Vorteil bleibt!“

Leo kroch über den Zaun zurück und Herr Dufresne besorgte noch am
selben Tag seinen Umzug nach der Billrothstraße. Hofrat Spineder
und seine Gattin stellten aber mit Befriedigung fest, daß ihr
Töchterchen zum erstenmal seit Jahr und Tag guter Laune war und
heiter vor sich hinsang.

„Du wirst sehen,“ sagte der Hofrat seiner Gattin, „Lotte schlägt
sich nach und nach die ganze traurige Geschichte aus dem Kopf! Der
arme Bursch? tut mir ja leid, aber es ist besser so. Übrigens hat
er mir ja auch ganz vernünftig geschrieben und versprochen, den
Briefwechsel mit Lotte aufzugeben.“

Die Frau Hofrätin schüttelte verwundert das Haupt und dachte: Wie
doch die Mädchen von heute ganz anders sind! Ich würde an Lottes
Stelle meine Liebe nicht überwunden haben!

\tb{* * *}
\pagenum{74}Die „Weltpresse“, einst das Blatt des liberalen
Bürgertums, jetzt das Hauptorgan der christlichsozialen Partei,
erhielt eine Zuschrift von dem Besitzer des Hauses Billrothstraße
19, in der in scharfer und logischer Weise gegen den Fortbestand
des Mieterschutzgesetzes Stellung genommen wurde. „Das
Mieterschutzgesetz“, hieß es in der Zuschrift, „hatte Zweck und
Sinn, als Wohnungsnot herrschte und die Bevölkerung davor geschützt
werden mußte, durch die Habgier einzelner Hausbesitzer obdachlos
gemacht zu werden. Heute gibt es keinen Mangel an Wohnungen mehr;
dank dem segensreichen Antijudengesetz unseres hochverehrten
Bundeskanzlers sind wieder normale Verhältnisse eingetreten, es ist
der notwendige Überschuß an Wohnungen vorhanden, und so erübrigt
sich dieses Mieterschutzgesetz, das nur mehr einen brutalen
Eingriff in die Rechte der Hausbesitzer bildet, ja sogar einen
Verfassungsbruch. Sicher werden nach Aufhebung des Gesetzes
Steigerungen der Mietzinse eintreten, was nur gerechtfertigt wäre
und schließlich der Allgemeinheit zugute käme, denn von den höheren
Mietzinsen sind höhere Steuern zu zahlen und mit höheren
Mietpreisen steigt der Wert der Häuser. Es ist charakteristisch,
daß es ein in meinem Hause wohnender, vornehmer französischer
Künstler ist, der mir sein Entsetzen über dieses Mieterschutzgesetz
ausdrückte. Er erklärte, daß man sich in französischen
Kapitalistenkreisen über dieses Gesetz lustig mache, das unter
anderem auch verhindert, daß Ausländer ihr Geld in Wiener Häusern
anlegen. Also fort mit dem Mieterschutzgesetz! Die vornehme
christliche Gesinnung der Wiener Hausbesitzer, vor allem aber das
Gesetz \pagenum{75} von Angebot und Nachfrage werden automatisch
ein allzu starkes Hinaufschnellen der Mietpreise verhindern.“

Die Zuschrift erschien an auffallender Stelle in der „Weltpresse“
mit einem redaktionellen Zusatz, in dem sehr vorsichtig die Ansicht
des geehrten Einsenders gebilligt, ihr aber gleichzeitig auch sanft
widersprochen wurde. Denn man wollte weder die Hausbesitzer noch
die Mieter vor den Kopf stoßen.

Von da an begann ein lebhafter öffentlicher Gedankenaustausch, es
hagelte von Zuschriften und immer stürmischer wurde der Ruf der
Hausbesitzer nach Aufhebung des Mieterschutzgesetzes, Einräumung
des Kündigungsrechtes und der individuellen Mietsteigerung. Herr
Windholz, der Besitzer des Hauses in der Billrothstraße, war
plötzlich eine gewichtige Persönlichkeit geworden, der Verein der
Hausbesitzer wählte ihn zum Vorstand und täglich kam er zu seinem
vornehmen französischen Mieter, Herrn Dufresne, um sich bei ihm Rat
zu holen. Herr Strakosch, \latein{alias} Dufresne, aber hetzte
munter weiter und sagte eines Tages mit Emphase:

„Wenn sich die Hausbesitzer noch weiter diese Versklavung gefallen
lassen, so halte ich sie alle zusammen für alberne Waschlappen und
ich werde eine Stadt verlassen, in der solche Zustände möglich
sind.“

„Ja, was sollen wir nur tun,“ meinte Herr Windholz kleinmütig,
„wenn die Regierung absolut unseren Wünschen nicht entsprechen
will?“

„Was Sie tun sollen? Ich werde es Ihnen sagen! Heute noch trommeln
Sie Ihren Verein zusammen und \pagenum{76} fassen den Beschluß,
der Regierung ein dreitägiges Ultimatum zu stellen. Stellt sie bis
dahin die Freizügigkeit im Wohnungsverkehr nicht wieder her, so
wird von den Hausbesitzern gestreikt! Sie führen keine Steuern ab,
unterlassen die Hausbeleuchtung und Reinigung, verweigern die
Bezahlung der Hypothekarzinsen, kurzum, Sie sabotieren den Staat!“

Herr Windholz war begeistert, umarmte den Franzosen und versicherte
ihm, daß er keinesfalls im Zinse gesteigert werden würde.

Es geschah ganz nach dem Programm des Herrn Dufresne. Der Verein
der Wiener Hausbesitzer beschloß einstimmig das Ultimatum und die
Regierung fiel um. Vergebens versicherte Doktor Schwertfeger, daß
die Aufhebung des Mieterschutzgesetzes die unheilvollsten Folgen
haben werde, er wurde von seinen Ministerkollegen überstimmt. Wie
die „Arbeiter-Zeitung“ boshaft behauptete, in erster Linie deshalb,
weil der Finanzminister, der Unterrichtsminister und der
Handelsminister mehrfache Hausbesitzer waren.

Das Mieterschutzgesetz, das den Hausbesitzern sowohl die Kündigung
der Mieter als die willkürliche Erhöhung der Mietpreise untersagte,
fiel also, und vierundzwanzig Stunden später fand eine stürmische
Generalversammlung der Hausbesitzer statt, in der beschlossen
wurde, die derzeitigen Mietpreise der Teuerung halbwegs
entsprechend auf das Tausendfache zu erhöhen. Eine Art Rütlischwur
verpflichtete zur unbedingten Einhaltung dieses Beschlusses.

\pagenum{77}Die Bevölkerung, die ja nur zum geringsten Teile aus
Hausbesitzern besteht, geriet in Tobsucht. Arbeiterfamilien mußten
nunmehr eine halbe Million im Jahr für ihre Wohnung bezahlen, eine
kleine Mittelstandswohnung kostete nicht unter einer ganzen
Million! Die Organisation der Hausfrauen, die Gewerkschaften, der
Verband der Festangestellten, die Kriegsinvaliden und Kriegswitwen,
der Bund der Gewerbetreibenden, sie alle veranstalteten
Massendemonstrationen, und durch volle acht Tage wurde in Wien und
den Provinzstädten überhaupt nicht gearbeitet, sondern vom Morgen
bis in die Nacht demonstriert. Die Zahl der eingeschlagenen
Fensterscheiben wuchs erschreckend, und zum erstenmal seit einer
geraumen Anzahl von Jahren hörte man auf der Straße den Ruf:

„Nieder mit der Regierung!“

Die christlichen Blätter ebenso wie die deutschnationalen verloren
massenhaft Leser, während der Weizen der „Arbeiter-Zeitung“ wieder
zu blühen begann.

\tb{* * *}
Herr \Zwickerl{} war schlechter Laune und stocherte wütend in seinem
Kirschenstrudel umher, der auf dem Teller vor ihm lag. Frau
\Zwickerl{} sah Sturm kommen und beugte vor.

„Anton, was is dir denn wieder über die Leber gelaufen? Geht das
Geschäft nicht?“

Das war für Herrn \Zwickerl{} zu viel. Er schob den Kirschenstrudel
fort, wurde röter im Gesicht als die Kirschen im Strudel und
brüllte:

\pagenum{78}„Oh ja, das G?schäft geht! Zum Teufel nämlich geht
es! Damit du nur weißt, Konkurs muß ich ansagen!“

„Jessasmariandjosef!“ kreischte Frau \Zwickerl{} auf. „Wie ist denn
das möglich?! Es ist doch immer g?steckt voll im Laden und alle
Leut? glauben, daß du eine Goldgruben von dem Juden, dem Leßner,
übernommen hast!“

„Ja,“ höhnte \Zwickerl{}, „eine Goldgruben voll mit Dreck! Je mehr die
Leut? kaufen, desto mehr verlier? ich! Weißt was? Daran san die
verfluchten Valuten schuld! Kronen, schäbige Kronen krieg? ich
herein und Mark und tschechische Kronen und Franken fliegen hinaus.
Zehntausend Meter Batist kauf? ich in Reichenberg und nach acht
Tagen kommt der Verkäufer von der Abteilung und strahlt über das
ganze blöde Gesicht und sagt: „Herr \Zwickerl{}, die Ware fliegt einem
nur so aus der Hand! Morgen haben wir nicht mehr einen Meter im
Haus!“

„Schön, denk? ich mir und geh? in die Buchhaltung, und wie wir
nachrechnen, sehen wir, daß ich, weil die tschechische Krone wieder
gestiegen ist, bei jedem Meter tausend Kronen verloren hab?. Und
das ist nur ein Beispiel von hunderten. Ich schlag? eh? bei jeder
War? schon dreihundert Prozent auf und trotzdem, die Krone fällt
rascher, als ich aufschlagen kann, Verluste, nichts als Verluste,
und die Länderbank, die mir das Kapital zur Übernahme gegeben hat,
fordert Rückzahlung und ich kann nicht zahlen, weil ich ein
riesiges Defizit habe. Im Gegenteil, ich brauche wieder hundert
Millionen, weil ich sonst nicht einkaufen kann!“

\pagenum{79}Herr \Zwickerl{} hatte sich Luft gemacht und war
besänftigt. Er zog den Kirschenstrudel an sich heran und machte ein
pfiffiges Gesicht:

„Weißt, Alte, wir braucheten einfach ein paar jüdische Banken, das
ist alles! Früher, als ich noch mein kleines Geschäft in der
Stumpergassen gehabt habe, da bin ich alleweil, wenn ich im Ausland
kaufen mußte, zum krummen Kohn von der Hermesbank gegangen, wo mein
Konto war, und der hat gesagt: Herr \Zwickerl{}, hat er gesagt, Sie
müssen sich jetzt mit Mark eindecken, weil die Mark steigen wird;
oder: die Krone wird fester kommen, hat er gesagt, kaufen Sie
Kronen. Und immer ist es richtig so gewesen und ich hab? nicht nur
an der Ware, sondern auch noch an der Valuta verdient! Aber jetzt –
die Affen, die jetzt in der Bank beieinandersitzen, kennen sich
selber net aus und i kenn? mi? auch net aus und alles geht kaput,
sag? ich dir!“

Herr \Zwickerl{} gehörte zu den vielen kleinen Geschäftsleuten, die
durch das Antijudengesetz mächtig in die Höhe gekommen waren. Mit
Hilfe der urchristlich gewordenen Länderbank hatte er, der kleine
Dutzendkaufmann, das große Warenhaus in der Mariahilferstraße an
sich bringen können, und das erste Halbjahr war alles eitel Wonne
gewesen. Wenn Herr \Zwickerl{} auf der Galerie des Kaufhauses stand
und auf den Menschenschwarm hinabsah, kam er sich wie ein kleiner
König vor und er berauschte sich ordentlich an dem Klingeln der
Registrierkassen, dem Knistern der Seide und dem Stimmengewirr. Und
allabendlich leerte er beim Nachtessen sein Weinglas auf das
\pagenum{80} Wohl des Schwertfeger, und immer wieder sagte er zu
seiner Frau, die jetzt nur mehr in Glacéhandschuhen kochte:

„Alte, da sieht man es am besten, wie uns die Juden ausgesaugt
haben! Die Juden haben die großen Geschäfte gehabt und wir Christen
konnten im finsteren Laden schuften und darben. Gottlob, daß das
aufgehört hat!“

Aber schon die erste Semestralbilanz brachte dem Herrn \Zwickerl{}
arge Enttäuschung. Trotz der enormen Umsätze und des gefüllten
Kaufhauses war von einem Gewinn keine Rede, immer wieder hatte man
sich beim Einkauf im Ausland so oder so verspekuliert. Und mehr als
einmal hatte Herr \Zwickerl{} in sich hineingeseufzt: An ordentlichen
Juden, wenn ich hätt?, der was mich beraten tät?!

Herr \Zwickerl{} mußte tatsächlich Konkurs anmelden, das Geschäft
wurde geschlossen und von einem Grundbesitzer aus der
Gumpoldskirchner Gegend übernommen, der aus dem großen Haus eine
riesige Stehweinhalle machte.

In den Jahren, die dem Kriegsende und dem Umsturz gefolgt waren,
hatte sich Wien immer mehr zur Zentrale des mitteleuropäischen
Luxus entwickelt und das Leben gewisser Schichten eine Üppigkeit
angenommen, die in der ganzen Welt als beispiellos besprochen
wurde. Die breiten Massen der Wiener Bevölkerung aber, nicht nur
die Arbeiter, sondern auch das mittlere Bürgertum, hatten
zähneknirschend gesehen, wie sich die fremden Elemente, vor allem
die Juden aus Galizien, Rumänien und Ungarn, als Herren Wiens
aufspielten, mit dem für sie fast wertlosen österreichischen Geld
um sich warfen, Champagner tranken, wo der kleine Mann kaum noch
das Glas Bier \pagenum{81} zahlen konnte, ihre Weiber mit Perlen
und Pelzen behängten, während die wirklich gute Gesellschaft den
alten Familienschmuck stückweise verkaufen mußte, in prachtvollen
Luxusautomobilen durch die Straßen rasten, den bodenständigen
Wienern die Wohnungen wegnahmen und mit ihrem lärmenden protzigen
Gehaben die alte kultivierte Stadt erfüllten.

Als die Juden fortgetrieben waren, änderte sich das alles von Tag
zu Tag auf das gründlichste. Der sinnbetörende Luxus verschwand,
der Wiener Ausverkauf stockte, man mußte sich nicht mehr anstellen,
um einen Platz im Opernhaus zu ergattern, das Leben wurde stiller,
solider, einfacher. Bis es sich zeigte, daß eine Stadt wie Wien
ohne Luxus nicht leben kann. Zuerst hatten die christlichen
Geschäftsleute, die die Kaufläden der Juden übernahmen, sich auch
deren Automobile bemächtigt, es schien der Wohlstand derselbe
geblieben zu sein und nur eine Umgruppierung erfahren zu haben, und
der Jubel, mit dem die Wiener es begrüßten, daß sie nicht bei jedem
Schritt auf jüdische Schieber stoßen mußten, war ebenso ehrlich als
begreiflich. Als dann aber bald die Krone wieder ins Uferlose fiel
und die Teuerung neue Wellen zog, als alles das, was eben auf
äußersten Luxus eingestellt war, wie die vornehmen Geschäfte, die
Kabaretts, die Theater, die fürstlichen Restaurants und Bars,
einging, als die Arbeitslosigkeit um sich griff und der Export nach
dem Ausland immer geringer wurde, da begann auch das äußere Leben
flügellahm zu werden. Die Zehntausende von Automobilen, die aus
jüdischen Händen in christliche \pagenum{82} übergegangen waren,
wurden für eine Handvoll Lire oder Franken ins Ausland verkauft,
weil bei dem schlechten Geschäftsgang das Benzin unerschwinglich
wurde, die Kunsthändler klagten über völlige Geschäftslosigkeit,
das Defizit der Staatstheater wuchs riesenhaft, christliche
Künstler und Gelehrte von Ruf, vor allem aber die großen Ärzte,
zogen ins Ausland, weil das Inland ihnen nicht mehr die Honorare
bezahlen konnte und wollte, die sie von den jüdischen Zeiten her
gewohnt waren.

Und unaufhaltsam griffen Mißmut, Unzufriedenheit und die
Erkenntnis, auf einer abschüssigen Bahn zu gehen, um sich.

\tb{* * *}
An einem herrlichen Junitag ging Leo Strakosch als Franzose
Dufresne nach dem Stadtpark, um wieder einmal Fühlung zum Wien von
heute zu bekommen. Sonst verließ er den neunzehnten Bezirk kaum, da
er entweder in seinem Atelier arbeitete oder aber mit Lotte
ausgedehnte Spaziergänge im Wienerwald unternahm. Als er heute nun
zwischen den dichtbesetzten Tischen um den Kursalon herum
spazierte, war er so belustigt, daß er laut auflachte.

„Um Himmels willen, was ist aus meinem schönen eleganten Wien
geworden!“

Die Mode des Alpenkleides und Touristenanzuges schien allgemein
geworden zu sein; so weit das Auge reichte, sah er alte und junge
Herren in Loden, Kniehosen und mit dem grünen Steirerhütl auf dem
Kopf. Und die \pagenum{83} Damen! Die Mehrzahl trug
Dirndlkostüme, die ja im freien Gelände sehr nett und anmutend
wirken, hier aber wie Karikaturen, wie schlechte Witze erschienen.
Man war eben sehr bescheiden geworden, und vor allem bildete man ja
eine einzige große Familie, war unter sich und hatte es nicht
notwendig, sich „herzurichten“.

Hie und da sah man auch noch elegant gekleidete Damen und Herren;
sie fielen aber auf, man konnte von den Älpler-Tischen bissige
Bemerkungen über sie hören, und Strakosch wurde es fast unheimlich
zumute, als er sah, wie ihn dieses oder jenes „Dirndl“ durch ein
Lorgnon anstierte, wahrscheinlich nur deshalb, weil sein
dunkelblauer Anzug, die Lackstiefel und die kostbare Seidenkrawatte
auffielen.

Die elektrische Straßenbahn, städtische Musik und Dirndln, die ein
Lorgnon tragen – Leo schüttelte sich. Er eilte aus dem Stadtpark
fort über die Ringstraße, fand auch das Bild, das die Kaffeehäuser
boten, trostlos, grinste, als er wahrnahm, daß die meisten Leute
einander mit „Heil“ begrüßten und mußte lange suchen, bis er ein
Autotaxi fand. Denn auch diese Mietwagen waren ein Luxus geworden,
der so wenig Benutzer hatte, daß die meisten ihr Geschäft
aufgaben.

Spät abends, als die Sonne schon langsam unterging, traf er Lotte
verabredetermaßen am Rande des Kobenzlwaldes. Sie ließen sich auf
einer Bank nieder, und nachdem sie sich sattgeküßt, erzählte Lotte,
daß ihre Eltern beschlossen hatten, schon in der nächsten Woche
nach ihrer kleinen Villa am Wolfgangsee zu übersiedeln.

\pagenum{84}„Was soll nur aus uns werden,“ klagte Lotte, „wie
soll ich es ertragen, dich den ganzen Sommer nicht zu sehen?“

„Davon kann auch keine Rede sein, Lieb. Ich werde eben auch
ausspannen, und wenn du in St. Gilgen bist, werde ich in Wolfgang
wohnen und jeden Tag wirst du herüberkommen und wir werden
wenigstens eine Stunde beisammen sein.“

„Hm,“ meinte Lotte vergnügt, „das läßt sich ja hören! Aber jetzt
muß ich dir auch sagen, daß ich gestern eine Auseinandersetzung mit
Papa hatte. Stelle dir nur vor, plötzlich sah mich Papa scharf an
und sagte sehr ernst: Lotte, wo treibst du dich eigentlich
neuerdings immer stundenlang allein herum? Du weißt, wir lassen dir
alle mögliche Freiheit, aber was zu viel ist, ist zu viel!

Also, ich fühlte, wie ich blutrot wurde und dachte, das beste ist,
ich beichte.“

„Was,“ unterbrach sie Leo entsetzt, „du hast deinem Vater
erzählt\ldots{}?“

„Ausreden lassen, Aff?“, lachte Lotte und zwickte ihn in das Ohr.
„Ich beichtete also, aber natürlich nur das, was mir paßte. Ich
sagte dem Papa, daß ich bei der Erna einen sehr feinen jungen Mann
kennen gelernt habe, den ich ebenso gut leiden mag, wie er mich und
daß ich ihn oft treffe, um mit ihm spazieren zu gehen. Er sei ein
Franzose, namens Henry Dufresne, der hier große Geschäfte mache.

Der Papa war zuerst ganz sprachlos, dann fragte er mich, warum ich
den Franzosen nicht zu uns einlade. \pagenum{85} Darauf
erwiderte ich, daß ich meiner Gefühle noch nicht sicher sei und
deshalb der Sache keinen offiziellen Anstrich geben wolle. Und zum
Schlusse meinte ich ganz empört:

Papa, du weißt doch, daß du dich auf mich verlassen kannst! Ich tue
sicher nichts Unrechtes, und wenn ich es für gut und notwendig
halten werde, so wird Henry schon zu euch kommen! Jetzt aber laßt
mich meine Wege allein gehen.

Papa war darauf sehr lieb und nett und Mama auch, und später hörte
ich, wie der Papa der Mama sagte: „Ich hätte nicht gedacht, daß
Lotte den armen Leo so rasch und gründlich vergessen würde. Aber
ich bin sehr glücklich darüber, daß sie eine neue Neigung gefaßt
hat und wir wollen ihr nichts in den Weg legen.“

Und Mama, die dich doch so gerne hat, meinte kopfschüttelnd: „Ich
versteh? das Mädel gar nicht! Sie hat wirklich schon wieder rote
Wangen bekommen und trällert den ganzen Tag umher, als wäre ihr nie
ein Herzleid widerfahren.“

Weißt du, Leo, es ist sicher nicht schön von uns, daß wir meine
Eltern so an der Nase herumführen, aber ich bin ja so glücklich,
daß du hier in Wien bist!“

Leo zog Lotte an sich, küßte sie gründlich ab und sagte dann mit
wichtiger Miene:

„Jetzt gehen wir aufs Land, und wenn ich dann wieder hier bin, dann
werde ich die ganze Stadt an der Nase herumzerren, aber tüchtig,
sage ich dir! Mehr kann \pagenum{86} ich dir heute noch nicht
verraten, aber du wirst deine Wunder erleben!“

Dieser Sommer tröstete die Wiener zum zweitenmal für das viele
Ungemach und die argen Enttäuschungen, die sie erleben mußten.
Gerade die schönsten Plätze und Orte in dem klein gewordenen
Österreich waren in den früheren Jahren zum Pachtgut der Juden
geworden. Das ganze herrliche Salzkammergut, das Semmeringgebiet,
sogar Tirol, soweit es einigen Komfort bot, waren von
österreichischen, tschechoslowakischen und ungarischen Juden
überflutet gewesen; in Ischl, Gmunden, Wolfgang, Gilgen, Strobl, am
Attersee und in Aussee erregte es direkt Aufsehen, wenn Leute
auftauchten, die im Verdacht standen, Arier zu sein. Die
christliche Bevölkerung, zum Teil weniger im Überfluß schwelgend,
zum Teil auch großen Geldausgaben konservativer gegenüberstehend,
fühlte sich nicht ohne Unrecht verdrängt und mußte mit den
billigeren, aber auch weniger schönen Gegenden in Niederösterreich,
Steiermark oder in entlegenen Tiroler Dörfern vorlieb nehmen. Das
war seit der Judenvertreibung anders geworden. Es gab in den
schönsten Sommerfrischen keine Überfüllung, die Städter bekamen
auf ihre Nachfragen höfliche und eilige Antworten, und trotz der
sonstigen Teuerung waren die Wohnungs- und Zimmerpreise erheblich
billiger als vor zwei Jahren. Und so schwärmte denn alles, was Geld
und Zeit hatte, in jene Gegenden, die dem bodenständigen Wiener
früher verleidet worden waren.

Die Besitzer der großen Etablissements, Kuranstalten und
sogenannten Sanatorien schnitten allerdings sauere \pagenum{87}
Mienen. Sie hatten immer von dem internationalen Judentum gelebt,
ihr ganzer Betrieb war auf jene Menschen eingestellt, die nicht
rechnen, wenn es sich um ihre Behaglichkeit handelt, und nun fanden
sie, da sie auch bei gutem Willen nicht billig sein konnten, nicht
genügend Gäste. Die großen Semmeringhotels eröffneten ihre Betriebe
überhaupt nicht mehr und viele Hotels im Salzkammergut und Tirol
sahen sich mitten im Sommer genötigt, zu sperren und ihr Personal
zu entlassen. Das war ein Wermuttropfen im Becher der Freude und
machte böses Blut unter der Landbevölkerung, die gewohnt war, ihre
Produkte zu enormen Preisen den großen Hotels zu verkaufen und ihre
Töchter und Söhne im Sommer ein schweres Stück Geld als
Stubenmädchen und Hausdiener verdienen zu lassen.

Der Bürgermeister von Semmering hatte den Mut, es in einer
Gemeinderatssitzung offen herauszusagen:

„Mit den Juden hat man bei uns den Wohlstand vertrieben, ein paar
Jahre noch und wir werden zwar gute Christen, aber bettelarm
sein!“

\tb{* * *}
Als der Sommer vorüber war und der Herbst die Blätter färbte,
begann in fast schon gewohnter Weise die Krone neuerlich zu fallen
und die Teuerung anzusteigen. Die Preise wurden phantastisch,
selbst reiche Leute scheuten die Anschaffung eines neuen
Kleidungsstückes, die Arbeiter, die Angestellten, ja auch die
Arbeitslosen stellten neue Forderungen, eine Fahrt auf der
Straßenbahn kostete \pagenum{88} schon tausend Kronen und ein
Kilogramm Butter fünfzigtausend.

Unter allgemeiner Verbitterung, Nervosität und Unruhe trat im
Oktober die Nationalversammlung zusammen, und das Gesicht des
Kanzlers Doktor Schwertfeger sah zerklüftet, durchfurcht, vergrämt
aus. Als er sprach, herrschte nicht jene weihevolle Ruhe wie
früher, sondern es wurden Rufe, Zwischenbemerkungen laut, sogar die
Galerie machte sich durch Oho-Rufe bemerkbar und die kleine
Opposition der Sozialdemokraten ließ sich nicht mehr einschüchtern,
sondern griff immer wieder in die Debatte ein.

Schwertfeger gab einen Überblick über die trostlose finanzielle
Lage des Landes und fuhr dann fort:

„Ich muß es rund heraussagen: Große und schwere Opfer stehen der
christlichen Bevölkerung Österreichs bevor. (Zwischenruf von der
Galerie: Natürlich nur den Christen, da wir ja die Juden
hinausgeschmissen haben!) Opfer, die mit Mannesmut und Bürgertreue
geleistet werden müssen! Die Regierung braucht zur Fortführung der
Geschäfte Geld, und da wir vom Auslande keine weiteren Kredite
bekommen können, müssen wir die Unsummen, die die Verwaltung, die
Verzinsung der Schulden und die Unterstützung der Arbeitslosen
verschlingt, durch neue Steuern, direkte und indirekte,
hereinbringen. (Große Unruhe im ganzen Hause.)

\erratum{„Meine}{Meine} Herren und Damen, ich weiß, daß die
Bevölkerung schwer enttäuscht ist und ich bin es mit ihr. Wir alle
haben eben die Schwierigkeit der Übergangswirtschaft unterschätzt,
wir alle dachten, daß die christlichen \pagenum{89} Bürger sich
besser auf die Beherrschung der Finanzen und des Geschäftslebens
einstellen würden, die ganz in Händen der Juden waren. Aber was
sind solche Enttäuschungen gegenüber dem ungeheuren Ziel, das wir
uns gesteckt haben, dem Ziel, Österreich seiner arischen
Bevölkerung wiederzugeben, ein Land aufzurichten, das frei von
Wuchergeist, frei von jüdischem Skeptizismus, frei von jenen
zersetzenden Eigenschaften und Elementen ist, die das Judentum
repräsentieren!“

Zum Schluß stellte der Kanzler mit erhobener Stimme die
Vertrauensfrage.

Im Namen der kleinen sozialistischen Fraktion sprach Doktor Wolters
gegen die Kreditgewährung, gegen die Gutheißung der
Regierungspläne, gegen das Vertrauensvotum. In krassen Farben
schilderte er die zunehmende Verelendung, die Gefahr des
unmittelbar bevorstehenden Staatsbankerottes, die Verödung des
wirtschaftlichen und geistigen Lebens. Er sagte unter anderem:

„Der Herr Bundeskanzler hat vor mehr als zwei Jahren, als er sein
Antijudengesetz begründete, unsere Bevölkerung bieder, einfältig
und ehrlich genannt und behauptet, daß sie der Konkurrenz der
überlegenen Juden nicht gewachsen sei. Er hat nur eines übersehen:
Daß wir biederen, ehrlichen und einfachen Österreicher auch ohne
Juden von Völkern umgeben sein werden, die uns jetzt, wo wir die
Juden nicht mehr haben, erst recht überlegen sind. Wo ist der
mitteleuropäische Handel hingekommen, seitdem die Juden weg sind?
Wir haben ihn verloren, denn die Juden haben ihn nach Prag und
Budapest mitgenommen. \pagenum{90} Was ist aus der blühenden
Konfektions-, Galanterie- und Mode-Industrie geworden? Sie ist fast
spurlos verschwunden, weil sie von der Biederkeit und Ehrlichkeit
allein nicht leben kann, sondern den jüdischen Konsumenten aus
aller Herren Länder braucht, der das leicht verdiente Geld auch
leicht wieder ausgibt. Heute zeigt es sich, daß wir der Juden nicht
entraten können~–~–.“

Stürmische Rufe unterbrachen den sozialistischen Führer. Die
Christlichsozialen und Deutschnationalen tobten, schrien „Hinaus
mit dem gekauften Judenknecht“ und der Tumult wurde so groß, daß
der Präsident, der Tiroler mit dem roten Bart, die Sitzung
unterbrechen mußte. Als er sie wieder eröffnete, erteilte er dem
Doktor Wolters eine Rüge, weil er durch seine Worte das christliche
Gefühl der Abgeordneten schwer verletzt und den Versuch gemacht
habe, die Grundfesten des neuen Staates zu erschüttern.

Schließlich wurden alle Regierungsanträge gegen die Stimmen der
Sozialisten angenommen. Aber viele Abgeordnete hatten sich vor der
Abstimmung entfernt und Schwertfeger sagte später seinem
Präsidialisten mit grimmigem Lächeln:

„Diesmal sind sie davongelaufen, das nächstemal werden sie gegen
mich stimmen, die Erfolghascher, Konjunkturisten, die gestern
Hosianna schrieen und morgen \latein{crucifige} rufen werden!“

\tb{* * *}
\pagenum{91}Seltsame, mysteriöse Dinge ereigneten sich. Eines
Morgens standen am Schottentor vor einer Litfaßsäule, desgleichen
vor der Oper, am Stubenring und an anderen Plätzen Hunderte von
Männern und Frauen vor kleinen, mit einem Reisnagel befestigten
Plakaten im Oktavformat, die folgende Inschriften enthielten:

„Wiener, Österreicher! Rafft euch auf, bevor Ihr alle zugrunde
gegangen seid! Mit den Juden habt Ihr den Wohlstand, die Hoffnung,
die Zukunftsmöglichkeit ausgewiesen! Fluch den Volksverführern, die
euch irregeleitet haben!

\unterschrift{
Der Bund wahrhaftiger Christen.“
}

Die Menschen lasen einander die frechen Worte vor, viele schimpften
und behaupteten, daß Freimaurer das getan haben mußten, andere
entfernten sich wortlos, wieder andere hatten den Mut, zustimmende
Äußerungen zu tun und die Anderssprechenden trotzig anzusehen.

Nach einigen Tagen erschienen an verschiedenen Plätzen neue Plakate
mit den Worten:

„Wien verdorft! Wiener, seht Ihr es denn nicht? Noch ein paar Jahre
und aus der alten, ehemaligen Kaiserstadt wird ein schäbiges,
vergessenes Nest geworden sein!“

Das ging den Leuten, die nun den Inhalt des Plakates auch aus der
„Arbeiter-Zeitung“ vernahmen, auf die Nerven, allenthalben wurde
man unruhig. War nicht etwas Wahres an dieser neuen Behauptung des
mysteriösen Bundes wahrhaftiger Christen? Leidenschaftliche
Diskussionen wurden darüber in Versammlungen, im Wirtshaus, in der
Straßenbahn geführt, aber das Wort \pagenum{92} von der
Verdorfung Wiens blieb irgendwie in der Luft hängen, wurde
geflügelt, man bekam es überall zu hören, ja sogar die christliche
„Weltpresse“ schrieb am Schluß eines Leitartikels ganz
unwillkürlich: „Wir müssen alles tun, um der Verdorfung zu
entgehen!“

Die Polizei wurde von der erbosten Regierung aufgefordert, den
Übeltäter aufzuspüren, der die Plakate anschlug. Vergebliche Mühe!
Alle paar Tage kamen neue zum Vorschein, immer an anderen Plätzen,
an Haustoren, Kirchenportalen, ja einmal hing je eines an den Toren
des Kanzlerpalais, des Polizeipräsidiums und des Parlamentes. Und
immer enthielt das kleine Plakat in wenigen Worten eine wirksame
Polemik gegen die Regierung, eine suggestive Aufhetzung der
Bevölkerung. Die „Arbeiter-Zeitung“ war jedesmal in der Lage, schon
in ihrer Morgenausgabe den Inhalt des Pamphlets, das heute
angeschlagen werden würde, zu veröffentlichen, weil ihr ein
Exemplar schon am Tage vorher mit der Post gebracht wurde.

Schließlich geriet ganz Wien in Aufregung, man sprach fast von
nichts anderem, zerbrach sich den Kopf darüber, wer hinter diesem
geheimnisvollen Bund wohl stecken möge, die Zahl derer, die dem
Inhalte der kleinen Aufrufe zustimmten, wuchs von Woche zu Woche,
die sozialdemokratischen Versammlungen bekamen wieder einen
ungeheuren Zulauf und der Nimbus des Kanzlers sank ersichtlich.

Lotte war eines Nachmittags früher zu Leo gekommen, als er sie
erwartet hatte. Da sie einen eigenen \pagenum{93} Schlüssel zu
der Wohnung besaß und Leo sie nicht wie sonst im Wohnzimmer
erwartete, ging sie direkt in das Atelier. Leo warf rasch ein Tuch
über einen kleinen Holztisch und begrüßte sie dann ein wenig
verlegen.

Lotte zog ihn beim Knebelbärtchen, sah ihm in die braunen Augen und
sagte dann:

„Du, Leo, du hast da soeben etwas vor mir verbergen wollen! Was
befindet sich dort unter dem Tuch?“

Leo lachte herzlich.

„Mädel, du hast Augen wie ein Luchs! Also, dann will ich dir mein
Geheimnis eben schon heute anvertrauen.“

Er zog das Tuch fort und Lotte erblickte neben einem Typenkasten
und einer Miniatur-Handpresse einen Stoß frisch gedruckter Zettel.
Erstaunt las sie:

„Wiener, geht es euch heute besser oder schlechter als zur Zeit der
Juden? Überlegt in Ruhe und Ihr werdet euch die richtige Antwort
geben! Wir alle haben einst geschrien: „Hinaus mit den Juden!“ So
schreien wir heute: „Herein mit jenen Juden, die ehrlich und treu
mit uns arbeiten \erratum{wollen.}{wollen.“}

\unterschrift{
Der Bund der wahrhaftigen Christen.“
}

Verblüfft, verwirrt, verständnislos ließ Lotte das Papier fallen
und ergriff einen anderen Zettel, auf dem gedruckt stand:

„Wir sehnen uns nicht nach den kulturfernen Ostjuden. Aber die
intelligenten, klugen, wertvollen Juden, die schon vor dem Jahre
1914 unsere Mitbürger waren, müssen \pagenum{94} wir wieder mit
offenen Armen aufnehmen, wenn wir nicht rettungslos verelenden
wollen! Auf zur Tat, bevor es zu spät ist!

\unterschrift{
Der Bund der wahrhaftigen Christen.“
}

Fragend sah Lotte ihren Bräutigam an.

Dieser hob sie zu sich empor, küßte sie auf die Nasenspitze und
lachte wieder aus vollem Halse.

„Na, Tschapperl, verstehst du noch immer nicht? Der Bund der
wahrhaftigen Christen, der seit Wochen Wien verrückt macht, bin
ich! Und ich werde nicht aufhören, bevor nicht der große Wirbel
eingetreten ist. Die zwei neuen Plakate werden wirken, sag ich dir!
Das sind meine Gas-, Stink- und Leuchtbomben, mit denen ich töte,
ersticke und erleuchte.“

Lotte zitterte.

„Leo, wenn du dabei erwischt wirst, so ist es um dich geschehen!“

„Wenn, wenn! Aber man wird nicht! Ich habe eine wunderbare Technik
beim Befestigen der Zettel! Ich schlendere morgens an einem Tor
oder einer Wand vorbei, und im Gehen, ohne auch nur eine Sekunde
mich aufzuhalten, treibe ich den Nagel ein, an dem der Zettel schon
hängt! Und selbst, wenn die Polizei die Zettel wenige Minuten
später wieder abreißt, so schadet das nicht, weil die
„Arbeiter-Zeitung“ den Inhalt schon abgedruckt hat. Verlaß dich auf
mich, mein Lieb, es muß das geschehen, ich gehe einen genau
vorgezeichneten Weg und nehme mich ohnedies höllisch in acht.“

\pagenum{95}Lotte saß auf dem großen Zeichentisch, baumelte mit
den schlanken Beinen und sagte nachdenklich:

„Weißt du, Leo, du hast schon sehr viel erreicht, glaube ich.
Gestern war bei uns größere Gesellschaft. Zehn Herren und Damen
waren da und es wurde fast ununterbrochen von der Judenausweisung
und ihren Folgen gesprochen. Und alle, darunter auch der Hofrat
Tumpel, waren darin einig, daß man sich mit der Ausweisung eines
Teiles der Ostjuden, und zwar jenes Teiles, der eine anständige
Beschäftigung nicht nachweist, hätte begnügen müssen. Hofrat
Tumpel, der vor einem Jahr noch wütend wurde, wenn man mit dem
Bundeskanzler nicht ganz einverstanden war, sagte schließlich:

„Ja, ja, es scheint, als wenn man da in einen höchst komplizierten
Mechanismus allzu brutal eingegriffen hätte! Gewisse nicht zu
unterschätzende jüdische Eigenschaften fehlen uns ganz
bedenklich!“

\erratum{„Dazu}{Dazu} ist allerdings zu bemerken, daß der Bruder
des Hofrates die Buchhandlung in der Seilergasse besitzt, die sich
nur mit dem Vertrieb von Luxusbüchern und Kunstdrucken befaßt. Seit
die Juden weg sind, macht er gar keine Geschäfte mehr und sein
Bruder, der Hofrat, hat schon zweimal große Summen opfern müssen,
um ihn vor dem Bankerott zu bewahren. Und noch etwas, Leo: Ich
halte doch immer, in der Früh?, wenn ich einkaufe, und im Konzert
und in der Oper und der Straßenbahn die Augen und Ohren offen. Und
ich höre, wie die Leute immer mehr mit Wehmut an die Vergangenheit
zurückdenken und von ihr wie von etwas sehr Schönem sprechen.
\pagenum{96} „Damals, wie die Juden noch da waren“, das kann man
täglich zehnmal in allen Tonarten nur in keiner gehässigen, hören.
\erratum{„Weißt}{Weißt} du, ich glaub?, die Leute bekommen
ordentlich Sehnsucht nach den Juden!“

Leo preßte das kluge Mädchen an sich. „Und ich will das Meinige
tun, um diese Sehnsucht unwiderstehlich zu machen.“

„Aber sei recht vorsichtig, Leo, bedenk?, daß, wenn man dich
umbringt, es auch mein Leben kostet!“

\tb{* * *}
Traurigere Weihnachten hatte Wien noch nie erlebt. Der
ungeheuerlichen Teuerung stand der vollständige Stillstand des
Lebens gegenüber. Die Teuerung allein hätte die guten Phäaken nicht
anfechten können. Sie waren sie ja schon seit einem Dezennium
gewöhnt, und ob das Viertel Wein nun zehntausend oder fünftausend
Kronen kostete, war schließlich egal, wenn man genug verdiente,
wenn der Arbeiter hohen Lohn bekam und der Kaufmann abends die
Kasse voll mit Zehntausendern hatte. Jetzt war das aber nicht mehr
der Fall. Die enormen Banknotenmassen blieben bei den Bauern
liegen, in den Städten herrschte vollständige Kaufunlust, ein
großer Teil der Arbeiter feierte und war auf die staatliche
Unterstützung angewiesen, und in der Weihnachtsnummer
veröffentlichten die Zeitungen Statistiken, aus denen hervorging,
daß seit zwei Jahren allein in Wien an die fünftausend
Bankfilialen, Kaffeehäuser, Restaurants und Geschäfte geschlossen
hatten. Neuerdings trat ein Riesenkrach nach dem anderen in der
\pagenum{97} Industrie ein, Aktiengesellschaften, die man noch
vor kurzem für bombensicher gehalten hatte, erklärten sich
insolvent und man sprach sogar von dem baldigen Zusammenbruch
zweier Großbanken.

Was nutzte es den Wienern unter solchen Umständen, daß sie überall
Platz hatten, sogar an den Weihnachtsfeiertagen die Theater nicht
ausverkauft waren und man nicht mehr den aufreizenden Judennasen
begegnete? Was nutzte es, daß man zur christlichen Einfachheit
zurückgekehrt war und sich den Vollbart wachsen ließ, wenn die
Friseurgehilfen massenhaft entlassen werden mußten, weil es keine
Arbeit mehr für sie gab?

Am schlimmsten waren die Juweliere daran. Die meisten waren Juden
gewesen und hatten auswandern müssen, und nun führten diese
Geschäfte ehemalige kleine Uhrmacher und andere sicher sehr
ehrenwerte Leute, die aber zum holländischen Edelsteinmarkt, der
fast ausschließlich in jüdischen Händen liegt, keinerlei
Beziehungen hatten und bei jedem Einkauf über die Ohren gehauen
wurden. Schließlich hatte der Einkauf im Ausland ganz aufgehört,
weil niemand mehr Schmuck wollte, wohl aber der Andrang derer, die
verkaufen mußten, immer stärker wurde. Langsam aber sicher wanderte
ein großer Teil des inländischen Juwelenbesitzes in die
Nachbarstaaten, nach England, Frankreich und Amerika, und auch
dabei waren die Juweliere, die diesen Export betrieben, die
Leidtragenden. Wenn ein Juwelier heute eine Perlenschnur für zehn
Millionen aus privatem Besitz kaufte und sie bald darauf für
dreißig Millionen einem Amerikaner anhängte, so \pagenum{98}
bildete er sich ein, ein glänzendes Geschäft gemacht zu haben und
begoß seine Freude mit Wein, lobte den Doktor Schwertfeger und
kaufte eine Fettgans, die nun nicht mehr das Privilegium der Juden
war. Bevor er aber noch die schwere Gansleber verdauen hatte
können, waren seine dreißig Millionen nicht einmal die zehn wert,
die er ausgegeben und er besaß kein Geld mehr zu neuen Ankäufen.

So war es wahrhaftig kein Wunder, wenn zu Weihnachten eine Welle
der Erbitterung und Unzufriedenheit durch Wien ging und die
Silvesternacht nicht mit Jubel und Radau wie sonst, sondern in
Verdrossenheit und Mutlosigkeit gefeiert wurde.

Und wenn der Bundeskanzler das Gespräch mitangehört hätte, das in
der Weihnachtswoche der Herr Habietnik, Besitzer des großen
Modehauses in der Kärntnerstraße, und der Herr Mauler, Inhaber des
großen Juweliergeschäftes am Graben, miteinander führten, so wäre
sein Ingrimm noch größer gewesen, als er es ohnedies war.

Herr Habietnik und Herr Mauler saßen im Grabenkaffee und klagten
beide über das elende Weihnachtsgeschäft, das den Ruin Tausender
von Geschäftsleuten besiegeln mußte. Plötzlich beugte sich Herr
Habietnik zu Herrn Mauler und erzählte ihm von einem Traum, den er
in der vergangenen Nacht gehabt.

„Stellen Sie sich vor, Herr Mauler, i hab? g?träumt, daß plötzlich
zu mir ins Geschäft lauter Juden und Jüdinnen gekommen san. Alle
waren hochelegant und haben Banknotenbündel in den Händen gehalten
und es ist ein Riesenwirbel entstanden. Die Madeln konnten die
Pelze \pagenum{99} und Stoffe, die Mäntel und Kostüme gar nicht
schnell genug herbeibringen und die ganze Modeabteilung war von
Seide und Samt, von Spitzen und Stickereien gefüllt. Und nichts war
den Jüdinnen gut genug und eine sehr eine fesche jüdische Dame hat
immer geschrien: „Das ist gar nichts! Wir kommen aus Paris und
Palästina, wo die neuesten Moden sind, zeigen Sie das Beste, was
Sie haben.“ Und da hat meine erste Verkäuferin plötzlich eine
Barchenthose gebracht und hat gesagt: „Aber meine verehrte gnädige
Israelitin, das ist doch das Neueste aus Paris!“ Und da ist ein
furchtbares Gelächter entstanden, so daß ich aufgewacht bin!
Glauben \erratum{'s}{S?} nicht, Herr Mauler, daß der Traum was zu
bedeuten \erratum{hat?}{hat?“}

Herr Mauler aber meinte grinsend:

„Ja, er hat zu bedeuten, daß bald die ganze Welt über uns lachen
wird und wir uns in Flanell und Barchent einwickeln werden, bevor
wir begraben werden. Aber das eine weiß ich, Herr Habietnik, wenn
so plötzlich vor meinem Laden ein Automobil vorfahren würde mit
einem jüdischen Ehepaar, so tät ich sie beide abküssen und hätt?
noch einmal eine Freude am Leben! Wissen Sie, Herr Habietnik, wie
ich früher noch Kommis beim Herrn Zwirner war, der mein Geschäft
gehabt hat, da hab? ich mir oft gedacht, daß es eigentlich eine
Schand? ist, daß fast nur die Juden Geld genug haben, um Brillanten
und Perlen zu kaufen. Und einmal habe ich das auch laut gesagt. Da
hat mich der Herr Zwirner angelacht und gesagt: „Herr Mauler, sein
Sie kein Narr, sondern froh darüber, daß die Juden kaufen und das
Geld unter die Leute bringen. Oder möchten \pagenum{100} Sie es
lieber haben, daß auch die Juden ihr Geld vergraben und verstecken
wie die Bauern? Sie werden sehen, wenn das mit dem Antisemitismus
so weitergeht, so werden die reichen Juden auswandern und dann
können die Geschäftsleute sperren!“

Na und jetzt sind nicht nur die reichen, sondern auch die armen
Juden ausgewandert und wir sind richtig alle
\erratum{kapores!}{kapores!“}

\tb{* * *}
Bei Spineders war der heilige Abend in der gewohnten
patriarchalischen Weise gefeiert worden. Die Stimmung war aber
nicht die beste. Der Hofrat begann ernstliche Sorgen materieller
Art zu haben, die ihm die Entwertung seines Vermögens bereitete;
Frau Spineder konnte sich noch immer von dem Schrecken nicht
erholen, den ihr die Tatsache eingejagt, daß sie für den
Weihnachtskarpfen fünfzigtausend Kronen und für die Weihnachtsgans
hunderttausend hatte zahlen müssen, und Lotte war unruhig, weil sie
ohne Nachricht von Leo war und doch gehofft hatte, daß er sich
irgendwie wenigstens mit einem Glückwunsch melden würde.

Gerade als mit Andacht der kostbare Fisch verzehrt wurde, läutete
die Haustorglocke und das Stubenmädchen meldete, ein Mann sei da,
der dem gnädigen Fräulein etwas persönlich zu überbringen habe.
Lotte stürzte hinaus, und der in einen Pelz gehüllte Mann, der ihr
etwas zu übergeben hatte, küßte sie im dunklen Hausflur wie
\pagenum{101} verrückt ab, um ihr dann ein winziges Päckchen in
die Hand zu drücken und eilends wieder zu verschwinden.

Im Speisezimmer wickelte Lotte das kleine Paket aus und entnahm
einem Lederetui einen Ring mit einer köstlichen, haselnußgroßen
Perle.

„Ein Weihnachtsgeschenk von Herrn Henry Dufresne“, sagte Lotte, die
purpurrot geworden war, und ein unendliches Glücksgefühl
durchströmte ihr junges Herz, als sie den Ring über den Finger
zog.

Der Herr Hofrat aber war betreten und erklärte kategorisch:

„Lotte, nun aber muß dieser Herr Dufresne sich uns doch endlich
vorstellen und um deine Hand anhalten. Denn ein solcher Ring, den
man einem Mädchen schenkt, ist einfach ein Verlobungsring.“

Lachend küßte Lotte ihren Vater.

„Habt noch ein wenig Geduld! Leo – Henry sagt, daß er sehr bald zu
euch kommen werde.“

Die Mama aber schüttelte wieder den Kopf und dachte:

„Seltsame Zeiten, seltsame Jugend! Liebt einen, vergißt ihn und
verwechselt dann seinen Namen mit dem des Nachfolgers!“

Im Januar vereinigten sich mehrere große Konsumentenorganisationen
zu einer Massenversammlung in der Volkshalle des Rathauses unter
der Devise: „Wir können nicht weiter!“ Zehntausende von Menschen
waren der Einladung gefolgt und trotz der außerordentlichen Kälte
standen vor dem Rathaus ungeheure Menschenmassen, die in der
Volkshalle nicht mehr Platz gefunden hatten.

\pagenum{102}Die Versammlung bot ein merkwürdiges Bild. Leo
Strakosch, der sich ebenfalls eingefunden hatte, konstatierte, noch
niemals so viele vollbärtige Männer gesehen und noch nie so viele
Heilrufe gehört zu haben. Eine andere Staffage und man hätte an
eine Tiroler Bauernversammlung zur Zeit des Andreas Hofer denken
können. Auch Weiblichkeit war massenhaft vertreten, aber wahrhaftig
nicht die lieblichste, die Wien aufzuweisen hat. Unter allgemeinem
Heil-Gebrüll eröffnete der Apotheker Doktor Njedestjenski die
Versammlung mit der Feststellung, daß es so nicht weitergehen
könne. Er vermied es sorgfältig, die Notlage und Teuerung mit der
Judenausweisung in Zusammenhang zu bringen, sondern gab sich höchst
deutschnational und behauptete, nur die Tatsache, daß Österreich
sich nicht an Deutschland anschließen könne, sei schuld an dem
jammervollen Niedergang Wiens. Worauf ein Arbeiter unter
schallender Heiterkeit dazwischen rief:

„Wir können uns ja gar nicht mehr anschließen, oder glauben Sie,
daß die Deutschen auch solche Trotteln wie wir sind und ihre Juden
hinausschmeißen werden?“

Das brachte den Apotheker aus dem Konzept, er stammelte noch etwas
von deutscher Einheit und deutschem Volksbewußtsein, schrie „Heil“
und gab den Rednern das Wort. Worauf fast nur mehr über die Juden
gesprochen wurde. Und zwar so, daß ein Unkundiger hätte glauben
müssen, Wien sei die judenfreundlichste Stadt der Welt. Als ein
Weinhändler antisemitische Töne anschlug, wurde er direkt
niedergeschrieen und ein Zwischenruf: „Hätten wir lieber von den
Juden gelernt, als sie hinauszujagen!“ \pagenum{103} fand großen
Beifall. Leo konnte sich nicht länger beherrschen. Mit bedenklichem
Herzklopfen meldete er sich bei dem Vorsitzenden zum Wort und
bestieg die Rednertribüne, während er dachte: Nun, Frechheit, steh?
mir bei! Er tat, als würde er die deutsche Sprache nur unvollkommen
beherrschen, betonte immer wieder, daß er als Franzose eigentlich
nicht befugt sei, sich in die Angelegenheiten Österreichs zu
mischen, aber von Wohlwollen für diese unvergleichlich schöne und
liebreizende Stadt, der schönsten nach oder mit Paris, erfüllt,
doch nicht umhin könne, seiner Meinung Ausdruck zu geben. Worauf
die anwesenden Vollbärte geschmeichelt und die Frauen, von dem
schlanken, hübschen Mann trotz des Knebelbartes entzückt „Heil!“
schrieen. Und dann fuhr Leo mit französischem Akzent fort:

„Auch wir in Paris haben sehr viele Juden, gute und schlechte,
wertvolle und schädliche. Jedenfalls sind viele darunter, die alle
Hochachtung verdienen und dem Land von großem Nutzen sind.
Niemandem aber würde es bei uns einfallen, die Juden ausweisen zu
wollen, sondern jeder versucht, ihre guten Eigenschaften
auszunützen. Ich bin hier nicht zu Hause und kenne daher die Wiener
Juden nicht so genau, kann aber sagen, daß ich in Paris mit sehr
vielen aus Wien Ausgewiesenen verkehrt habe, die einen
vortrefflichen Eindruck gemacht haben und sicher sehr bald gute
Franzosen sein werden. Es ist möglich, daß zwischen den
österreichischen Christen und den Juden ein größerer Unterschied
ist, als zwischen den leichtbeweglichen und temperamentvollen
Franzosen und den Juden. Aber gerade deshalb müßte doch eine gute
Ergänzung möglich sein. \pagenum{104} Ich höre, daß man den
Juden hierzulande den Vorwurf gemacht hat, das Kapital zu
beherrschen und relativ mehr Geld zu besitzen als die christlichen
Bürger. Ja, meine Verehrten, daraus geht doch nur hervor, daß sie
rascher im Denken und Handeln sind, und eine kluge Regierung müßte
solche Eigenschaften für die Allgemeinheit zu benutzen verstehen.“

Stürmische Zurufe von allen Seiten: „Jawohl, eine gescheite
Regierung, aber wir haben eben eine blöde! Recht hat er! Heil!
Heil!“

„Meine Verehrten,“ sagte Leo lächelnd, „ob einem die Juden
sympathisch sind oder nicht, ist eigentlich gleichgültig. Der
Sauerteig, der dem Brotmehl beigegeben wird, schmeckt an sich recht
abscheulich und doch kann ohne ihn kein Brot gemacht werden. So
müßte man auch die Juden betrachten. Sauerteig, an sich wenig
erfreulich und in zu großen Quantitäten schädlich, aber in der
richtigen Mischung unentbehrlich für das tägliche Brot. Und ich
glaube, daß Ihr Brot sitzen bleibt, weil ihm der Sauerteig fehlt!

Nun heißt es aber nicht räsonieren und das, was geschehen ist,
beklagen, sondern zusehen, wie Abhilfe geschaffen werden kann. Wie
das in Österreich möglich sein wird, weiß ich nicht. In Frankreich
würde in solchem Falle die Bevölkerung auf Neuwahlen dringen, die
zeigen müßten, ob das Volk mit den herrschenden Zuständen zufrieden
ist oder sie ändern will!“

Damit trat Leo ab, um rasch in der Menge zu verschwinden. Der
Versammlung hatte sich eine ungeheure Aufregung bemächtigt. Wie ein
Funke in ein Dynamitfaß, so \pagenum{105} hatte das Wort
„Neuwahlen“ in die Menschenmassen eingeschlagen, die riesige Halle
erdröhnte von diesem aus dreißigtausend Kehlen geschrieenen Wort,
das sich auf die Straße fortpflanzte und zum Schlagwort der
kommenden Zeit wurde.

Am folgenden Tage fand in der Redaktion der „Arbeiter-Zeitung“ eine
Konferenz der Hauptredakteure und der Vertrauensmänner der Partei
statt, in der zum erstenmal seit Jahren wieder beschlossen wurde,
aktive, energische Politik zu machen und mit dieser Politik aus den
geschlossenen Räumen auf die Straße zu gehen. Der Chefredakteur der
„Arbeiter-Zeitung“, der ehemalige Federnschmücker Wunderlich, der
nach bestem Gewissen das Erbe Viktor Adlers verwaltete, kam zu
folgender Konklusion:

„Wir müssen das Schlagwort dieses merkwürdigen französischen
Malers, der unmöglich Diefreß heißen kann, wie ihn der Trottel von
Vorsitzenden niedergeschrieben hat, aufgreifen. Von heute an werden
wir in unseren Blättern, in unseren Versammlungen und Beratungen
immer wieder Neuwahlen fordern. Und nun werden wir unsere Freunde
in Frankreich, Holland, der Tschechoslowakei, in England und
Amerika in Aktion setzen und sie veranlassen, alles zu tun, damit
große Kronenbeträge auf den Markt geworfen werden. Fällt die Krone
neuerdings empfindlich, steigt die Teuerung, die derzeit stagniert,
wieder an, so ist die Lage reif für uns und wir werden, wenn es
sein muß, die Auflösung der Nationalversammlung mit Gewalt
erzwingen.“

\tb{* * *}
\pagenum{106}In den nächsten Tagen ereignete sich noch etwas,
was in den stramm-christlichsozialen Kreisen große Bestürzung
erregte. Der Bürgermeister von Wien, nach Schwertfeger der
mächtigste Mann im Reiche, Herr Karl Maria Laberl, fiel sozusagen
um. Nicht aus eigenem Willen allerdings, sondern weil ihm sein
Präsidialist Herr Kallop ein Bein stellte. Von diesem Herrn Kallop
wußte man längst im Rathause, daß er eigentlich umgekehrt, das
heißt Pollak, heißen müßte, weil dies der Name seines Großvaters
war. Und als die Juden noch in Wien gewesen, erzählte man in ihren
Kreisen, daß der alte Pollak ein aus Galizien eingewanderter
Getreidehändler wäre, der eine Christin geheiratet habe und sich
deshalb taufen ließ. Sein Sohn habe schon den Namen Kallop
angenommen, war ein in christlichen Kreisen angesehener Advokat,
der wieder eine Christin heiratete, so daß die Enkelkinder des
alten Pollak nach dem Schwertfegerschen Gesetz als Vollarier
anzusehen waren. Josef Kallop, der Sohn des Advokaten, taugte in
seiner Jugend nichts, konnte seine juristischen Studien nicht
beenden und wurde daher mit Erfolg Magistratsbeamter. An Schlauheit
den meisten seiner Kollegen turmhoch überlegen, brachte er es bald
zum Präsidialisten und seit geraumer Zeit war er die rechte Hand
des Bürgermeisters Laberl.

Herr Kallop also war es, der den Bürgermeister zum Umfallen
brachte. Er machte ihm klar, daß ein großer Umschwung bevorstehe.

„So geht es nicht weiter, Herr Laberl, das ist Ihnen doch ganz
klar. Es wird demnächst Unruhen geben, ernste \pagenum{107}
Unruhen sogar, und eines Tages wird die Regierung sozusagen
flötengehen. Wenn Sie nicht mit flötengehen wollen, so müssen Sie
sich beizeiten ein wenig umdrehen. 
\discretionary{Rük-}{ken}{Rücken} Sie von Schwertfeger ab,
geben Sie zu, daß man bei der Judenausweisung zu weit gegangen ist,
und ganz Wien wird plötzlich inmitten des Rummels, der kommen muß
und wird, sagen: Unser Bürgermeister, das ist ein Gescheiter, der
lenkt ein und wird uns noch herausreißen.“

Herr Karl Maria Laberl nickte, strich sich den schönen, weißen
Bart, war von seinem überlegenen Verstand schon ganz durchdrungen,
fragte aber einigermaßen ängstlich:

„Lieber Kallop, das ist ja ganz richtig, was Sie da sagen und
entspricht dem, was ich mir schon längst gedacht habe. Aber wie
soll ich denn das machen?“

„Sehr einfach, Herr Bürgermeister. Wir berufen eine Versammlung der
christlichsozialen Bürgervereinigung des, na, sagen wir ersten
Bezirkes ein, weil dort unter den Geschäftsleuten geradezu eine
Panikstimmung herrscht. Und dann halten Sie eben eine Rede, die wir
zusammen ausarbeiten werden.“

Und so geschah es, nur daß das „Zusammenausarbeiten“ darin bestand,
daß Herr Laberl die Rede, die sein Präsidialist niederschrieb,
auswendig lernen mußte. Als dann die Versammlung der
Bürgervereinigung abgehalten wurde, begrüßte sie Herr Laberl sehr
feierlich, sprach von dem Ernst der Zeiten, von den Zuständen, die
man nicht mehr ertragen könne und sagte schließlich:

„Der Ruf nach Neuwahlen wird immer ungestümer und ich bin der
letzte, der den Ruf nicht hören will. Im \pagenum{108}
Gegenteil, ich persönlich bin dafür, daß man tut, was das Volk will
und durch Neuwahlen feststellt, ob die Bevölkerung Österreichs
auch jetzt noch gutheißt, was die Regierung vor mehr als zwei
Jahren getan, oder ob sie eine radikale Änderung wünscht. Ich und
wohl mit mir Sie alle, meine Herren, haben nur ein Ziel vor Augen:
Den Wiederaufbau möglich zu machen, das unglückliche Volk aus dem
Labyrinth, in das die Entente aber vielleicht auch schwerwiegende
eigene Irrtümer es gestoßen haben, wieder ans Licht des Tages zu
führen. Keine Dogmatik, kein Fanatismus, keine persönliche
Antipathie oder Sympathie darf uns leiten, meine Herren, sondern
lediglich der Nützlichkeitsgedanke!“

Kallop sorgte dafür, daß die Rathauskorrespondenz noch in derselben
Nacht die Rede des Bürgermeisters im Wortlaut den Zeitungen
übermittelte, und am nächsten Tag wußte es sogar der dümmste Kerl
von Wien, daß Karl Maria Laberl den Bundeskanzler im geeigneten
Moment \erratum{in}{im} Stich lassen werde.

Als Doktor Schwertfeger in den Morgenblättern die nur von der
„Arbeiter-Zeitung“ entsprechend kommentierte Rede des
Bürgermeisters las, stieg ihm gallbitterer Speichel in den Mund und
er spie aus. Dann warf er einen langen, verlorenen, glanzlosen
Blick vom Fenster über den Volksgarten, den jetzt ein weißes
Leichentuch bedeckte.

Herr Kallop aber rieb sich im Rathaus vergnügt die Hände. Und
nachdem er sich vergewissert, daß weder ein Kollege noch ein
Amtsdiener im Zimmer war, sagte er laut und vernehmlich:
„Maseltoff!“ und klopfte dreimal unter \pagenum{109} den Tisch.
Wobei zu bemerken ist, daß Herr Kallop eine üppige, zwar schon
zweimal geschiedene, aber dafür mit zahlreichen Millionen gesegnete
Jüdin verehrte, die in Prag im Exil lebte. Und er wünschte nichts
sehnlicher, als ihre und ihrer Millionen Rückkehr ins teure
Vaterland, schon deshalb, weil er mit seinem Gehalt als
Präsidialchef unmöglich die Teuerung länger aushalten konnte und
außerdem falsch in polnischer Mark spekuliert hatte.

\tb{* * *}
Der Fasching dieses Jahres konnte die Laune der Wiener nicht
verbessern. Grimmige Kälte, viel Schnee, ungeheizte Zimmer, weil
der Meterzentner Kohle hunderttausend Kronen kostete, eine Pleite
nach der anderen, der Zusammenbruch eines großen Bankkonzerns, bei
dem viele ihr Geld liegen hatten.

Die Bälle und Redouten standen vollständig unter dem Zeichen des
Dirndlkostüms. Da der Toilettenluxus fehlte, machte man aus der Not
eine Tugend, veranstaltete fast nur Bauernbälle, so daß Wien eher
einem „Kirtag“ glich als einer Großstadt.

Dazu kam, daß Wien vollständig aufgehört hatte, eine Theaterstadt
zu sein. Die ersten Kräfte der Staatsoper gastierten unaufhörlich
im Ausland, die Philharmoniker absolvierten eben eine Tournee in
Südamerika, die Privattheater hatten sich in Provinzschmieren mit
unzulänglicher Regie, minderen Kräften und veralteten Spielplänen
verwandelt, von auswärts kamen längst keine Konzertgäste mehr, weil
ihnen Wien die großen Gagen nicht zahlen \pagenum{110} konnte,
Zeitungen waren neuerdings eingegangen, weil die Zahl der Leser
immer mehr abnahm und plötzlich ertönte wieder der Alarmruf: „Die
Krone fällt!“

An den ausländischen Börsen fanden enorme Kronenabgaben statt, so
daß Zürich sie bald nur mehr auf ein Dreißigtausendstel Centime
bewertete. Demgemäß stiegen alle Preise und die Bevölkerung begann
in Verzweiflung zu geraten. Als das Kilogramm Fett eine
Viertelmillion Kronen kostete, erschien wieder das geheimnisvolle
kleine Plakat des Bundes der wahrhaftigen Christen mit den Worten:

„Wie lange noch, Wiener, werdet Ihr diese Regierung dulden? Wann
endlich wollt Ihr die Nationalversammlung auseinandertreiben und
Neuwahlen erzwingen?“

In den Morgenstunden des nächsten Tages kam es zu Plünderungen auf
den Märkten, die erbitterten Hausfrauen stürmten die Stände,
verprügelten die Marktfrauen und bemächtigten sich der Waren. In
Favoriten nahm der Tumult einen revolutionären Charakter an, es
mußte die Reichswehr aufgeboten werden, die sich aber weigerte,
gegen die Frauen vorzugehen.

In der Nationalversammlung, die eben tagte, richteten nicht nur die
Sozialdemokraten, sondern auch einzelne Christlichsoziale und
Großdeutsche Interpellationen an die Regierung, in denen gefragt
wurde, was man zu tun gedenke, um der verzweifelten Bevölkerung zu
helfen. Die Sozialdemokraten stellten einen Dringlichkeitsantrag,
die Regierung möge sofort Neuwahlen ausschreiben, damit das
\pagenum{111} Volk selbst entscheiden könne, ob es bereit sei,
die herrschenden Zustände noch länger zu dulden.

Totenbleich erhob sich der Bundeskanzler zu einer Entgegnung:

„In diesem Augenblick der allgemeinen Verwirrung Neuwahlen
ausschreiben, hieße das Geschick des Landes den radikalen Elementen
ausliefern und den Juden wieder Tor und Türe öffnen! Das stolzeste
und größte Werk, das die österreichische Legislatur jemals
geschaffen, würde zusammenbrechen, weil wir nicht genug Geduld und
Aufopferungsfähigkeit haben, um auszuhalten und die gegenwärtigen
Schwierigkeiten zu überwinden. Ich weiß, daß das internationale
Judentum am Werke ist und sicher arbeiten Agitatoren, von jüdischem
Gelde bestochen, daran~–“

Die weiteren Worte des Kanzlers gingen in dem ungeheuren Tumult
verloren, der nun folgte. Die Sozialdemokraten klopften mit den
Pultdeckeln, die Galerie tobte und schrie, sogar aus den Reihen der
Gesinnungsgenossen kamen Zurufe, wie: „Haben Sie Beweise für Ihre
Behauptungen?“

Um sechs Uhr abends wurde noch immer über den Dringlichkeitsantrag
der Sozialdemokraten gesprochen, die ersichtlicherweise alles
taten, um die Sitzung in die Länge zu ziehen. Jeder Redner sprach
stundenlang; hatte der eine geendet, so meldete sich ein anderer
zum Wort, die meisten Abgeordneten hörten längst nicht mehr zu,
sondern stärkten sich am Büfett, auch die Ministerbank war leer
geworden, nur Schwertfeger saß mit verschränkten Armen starr und
düster auf seinem Sitz.

\pagenum{112}Plötzlich kam neues Leben in das Haus. Das Gerücht
verbreitete sich, daß Arbeitermassen im Anzuge seien, gleich darauf
hörte man aus weiter Ferne die Klänge des Arbeiterliedes, das
Jauchzen und Toben erregter Menschenmassen, bis plötzlich ein
einziger Ruf von ungeheurer Stärke durch die geschlossenen Fenster
drang:

Nieder mit der Regierung! Fort mit der Nationalversammlung! Wir
wollen Neuwahlen!

Und schon umzingelten dichte Menschenmassen mit ihren Fahnen und
Standarten das Abgeordnetenhaus und immer neue Züge kamen an, die
gesamte Arbeiterschaft Groß-Wiens, die Angestellten und Beamten
waren von den Fabriken und Werkstätten, Bureaus und Ämtern in
geschlossenen Gruppen anmarschiert.

Schon donnerten mächtige Schläge gegen die Tore des Hauses, die
rasch geschlossen worden waren, schon prasselte ein Steinhagel
gegen die Fenster, schon hatte sich eine Deputation der Arbeiter
gewaltsam Einlaß verschafft. Ihr Führer, ein Eisenarbeiter namens
Stürmer, ein gewaltiger Kerl mit klugen Augen und riesigem Schädel,
stellte sich mitten unter die Abgeordneten, die, von Panik
ergriffen, wie die Schafe beim Gewitter einen geschlossenen Haufen
bildeten, und erklärte kurz und bündig:

„Das Militär hält zu uns, die Jungmannschaft unter den Polizisten
ebenfalls! Entweder die Regierung löst innerhalb zehn Minuten das
Haus auf und erklärt, daß sofort Neuwahlen ausgeschrieben werden,
oder die Massen gehen mit Gewalt vor. Die Erbitterung der Leute
kennt \pagenum{113} keine Grenzen, hinter den Arbeitern steht
diesmal das Bürgertum, es handelt sich um keine politische
Angelegenheit, sondern um Taten der Verzweiflung. Am wildesten sind
die Frauen, hören \erratum{sie}{Sie} nur, wie sie schreien, man
möge das Parlament anzünden! Gibt die Regierung nicht nach, so
können wir für nichts garantieren!“

Und es geschah, was geschehen mußte. Die Minister erklärten nach
kurzer Beratung mit den christlichsozialen und großdeutschen
Parteiführern, sich dem Terror zu fügen, das Haus auflösen und
Neuwahlen sofort ausschreiben zu wollen. Der Bundeskanzler bot
gleich seine Demission an, aber seine Kollegen und die Parteigrößen
beschworen ihn, sie in diesem kritischen Augenblick nicht zu
verlassen und so willigte er denn ein, die Zügel der Regierung noch
bis zu den Wahlen in seinen Händen zu behalten.

Als dem erregten Volke Mitteilung von der Auflösung der
Nationalversammlung gemacht wurde, löste sich die Spannung in
ungeheuren Jubel auf und in der kommenden Nacht wurden die
Weinvorräte Wiens ganz erheblich gelichtet.

Sogar der Franzose Henry Dufresne, der der denkwürdigen Sitzung auf
der Galerie beigewohnt hatte, trank sich allein in seinem Atelier
einen ordentlichen Rausch an. Am nächsten Morgen aber war er wieder
frisch und munter, entwarf eine geniale Skizze, die das Titelbild
des Warenhausromanes von Zola bilden sollte und schwenkte Lotte,
die vormittags schneebedeckt mit kalten roten Backen zu ihm kam, in
seinen Armen durch die Luft.

\pagenum{114}Lotte war in ausgelassener Laune wie er, denn ihr
Papa hatte nach der Lektüre der Morgenblätter sehr ernst gesagt:

„Mein Kind, ich sehe schwere Konflikte für dich kommen! Wenn nicht
alles trügt, so wird Leo Strakosch bald die Möglichkeit haben, nach
Wien zurückzukehren und dann wirst du dich entscheiden müssen:
Entweder er, den du so sehr geliebt hast und der mir ein
willkommener Sohn wäre oder dieser mysteriöse Franzose, den wir
noch immer nicht kennen gelernt haben!“

Als Lotte darauf lächelnd erwidert hatte, sie würde am liebsten
beide, Leo und den Franzosen nehmen, da war Hofrat Spineder
ernstlich böse geworden und hatte sie für frivol und unmoralisch
erklärt. Sie mußte ihre ganze Verführungskunst aufwenden, um ihn zu
besänftigen.

Und nun saß sie auf dem Schoß ihres Geliebten und küßte Henry
Dufresne und Leo Strakosch in einer Person mit Feuereifer ab.

\tb{* * *}
Leo, der fast nie Gelegenheit fand, mit irgend jemandem außer mit
Lotte und seiner Aufwartefrau zu sprechen, hatte in der letzten
Zeit zwei Bekanntschaften gemacht, die ihm wichtig dünkten. Die
eine bestand in der Person des Nationalrates Wenzel Krötzl, die
andere war der Inhaber des großen Modehauses in der Kärntnerstraße,
Herr Habietnik.

Mit Krötzl war Leo auf folgende Weise bekannt geworden: Als er
einmal spät nachts aus dem Kaffeehaus, \pagenum{115} in dem er
die Zeitungen und Zeitschriften zu lesen pflegte, nach Hause
gekommen war, fand er auf dem letzten Treppenabsatz einen
stockbesoffenen Mann liegen, der jämmerlich weinte und sich
vergeblich bemühte, aufzustehen. Leo half ihm in die Wohnung, die
unterhalb seines Ateliers gelegen war und erfuhr bei dieser
Gelegenheit, daß er den ehrsamen Nationalrat Wenzel Krötzl vor sich
hatte, seines Zeichens im Nebenberuf Häuserschieber. Nicht nur, daß
dies auf dem Türschild vermerkt stand, Herr Krötzl schrie auch,
während er hin- und hertaumelte, immerzu:

„Wann aner sagt, daß i b?soffen bin, so is er a jüdischer Gauner! I
bin a g?wählter Nationalrat, an Abgeordneter und hab? fufzich
Häuser zum verkaufen, die was früher denen Saujuden g?hört ham!“

Leo hatte dann im Laufe der Zeit Gelegenheit, zu erfahren, daß Herr
Krötzl nicht nur einer der wütendsten Antisemiten sei, sondern auch
ein notorischer Trunkenbold, der sich gewöhnlich schon am Büfett
des Parlaments seinen Frühstücksrausch kaufte. Nebenbei hatte er
eine gewisse Beredsamkeit und genoß infolge seiner derben
Ausdrucksweise viel Popularität unter seinen Wählern. Er war Witwer
und beherbergte von Zeit zu Zeit eine angebliche Wirtschafterin bei
sich, mitunter solche, die knapp das straffreie Alter von vierzehn
Jahren besaßen.

Die Bekanntschaft des Herrn Habietnik hatte Leo auf wesentlich
bürgerlichere Art gemacht. Leo pflegte seinen Bedarf an Krawatten
und Wäschestücken in dem Modehaus zu decken, das trotz seiner
„Verloderung“ noch immer die besten Waren führte, und bei solcher
Gelegenheit war er \pagenum{116} einmal mit Herrn Habietnik ins
Gespräch gekommen. Herr Habietnik war entzückt, einen Franzosen von
Distinktion zu bedienen, der sich tadellos trug und genau wußte,
daß zu einem blauen Cheviotanzug eine perlengraue Seidenkrawatte am
besten paßte, es kam zu einem angeregten Gespräch, im Verlaufe
dessen Leo erkannte, wie sehr der intelligente Kaufmann unter den
herrschenden Verhältnissen litt, und von da an trafen sich die
beiden öfters in dem Laden, schließlich vereinbarten sie sogar hie
und da eine Zusammenkunft im Graben-Café.

Nach der Auflösung der Nationalversammlung beeilte sich Leo, mit
Herrn Habietnik wieder in Fühlung zu kommen, und im Laufe der
Unterhaltung fragte er ihn um seine Meinung über die künftige
Entwicklung.

Herr Habietnik schüttelte sorgenvoll das Haupt:

„Also die Sozis arbeiten wieder mit Volldampf und werden die
Stimmen, die sie das letztemal verloren hatten, zurückgewinnen. Die
Christlichsozialen und Großdeutschen haben den Kopf verloren, sind
mit ihrem Programm noch nicht herausgekommen, aber schließlich wird
jeder, der nicht Sozialdemokrat ist, doch für eine der beiden
Parteien stimmen müssen.“

„So daß also vielleicht gar das Judengesetz in Kraft bleiben
wird?“

„Kann sein, wenn die Sozialisten nicht die Zweidrittelmehrheit, die
zu jeder Verfassungsänderung notwendig ist, bekommen. Denn ich
fürchte, daß die Christlichsozialen und Großdeutschen doch nicht
den Mut haben werden, das Ausnahmsgesetz gegen die Juden
aufzuheben. Das heißt, \pagenum{117} eigentlich müßte ich sagen,
ich hoffe, denn wenn die Juden wieder kommen, so wird man mir am
Ende gar das Geschäft wieder nehmen~–~–.“

„Unsinn“, erklärte Leo energisch. „Was Sie haben, kann man Ihnen
nicht mehr nehmen! Vielleicht, daß man es Ihnen abkaufen oder daß
der frühere Firmeninhaber sich mit Ihnen zu einer Teilhaberschaft
bequemen würde. Die Hauptsache ist aber doch wohl, daß Sie die
Jagerhütln und die Lodenröcke wieder hinausschmeißen und Ihre
Auslagen so arrangieren können, wie sie einst waren.“

Begeisterung glomm in den Augen Habietniks auf und mit warmem,
ehrlichem Ton erwiderte er:

„Jawohl! Das ist die Hauptsache! Wenn ich daran denke, daß hier
wieder einmal Leben und Luxus herrschen könnte, wie einst – nein,
das ist ein zu schöner Traum, um wahr zu sein.“

„Hören Sie, Herr Habietnik,“ sagte Leo, indem er seine Hand auf den
Arm des Kaufmannes legte, „Sie sind der Mann, um den Traum wahr zu
machen! Noch trennen uns Wochen von den Neuwahlen. Das genügt, um
eine bürgerliche Partei, bestehend aus den fortgeschrittenen
Elementen, den angesehenen Kaufleuten, den Gelehrten,
Rechtsanwälten, Künstlern und Fabrikanten zu bilden, mit der
offenen und ungeschminkten Parole: Aufhebung des Ausnahmegesetzes
gegen die Juden! Nehmen Sie das heute noch in Angriff, bilden Sie
ein zwölfgliedriges Komitee, in dem drei Kaufleute, drei
Industrielle, drei Festangestellte und drei Leute mit freiem,
akademischem Beruf sitzen, lassen Sie, da Sie noch keine Zeitung
zur Verfügung haben, zehntausend \pagenum{118} Plakate drucken,
gründen Sie dann Bezirkskomitees, betreiben Sie Propaganda von
Straße zu Straße, von Haus zu Haus und der Erfolg kann nicht
ausbleiben. Ich bin ein Fremder, kenne die Verhältnisse nicht so
genau wie Sie, aber dafür bin ich objektiver und ich weiß ganz
sicher, daß ein erheblicher Teil der Bevölkerung die neue Partei
stürmisch begrüßen wird.“

Herr Habietnik war Feuer und Flamme. Am selben Abend noch trommelte
er ein halbes Hundert Kaufleute aus der Inneren Stadt, Fabrikanten,
Rechtsanwälte zusammen, und um ein Uhr morgens war das Komitee
konstituiert, dem ein gemeinsam gezeichnetes Millionenkapital zur
Verfügung stand.

Die neue Partei hieß „Partei der tätigen Bürger Österreichs“,
stellte sich auf ein absolut bürgerlich-freisinniges Programm und
begann mit einer lebhaften und temperamentvollen Agitation. Daß der
Franzose Dufresne die Flugzettel und Aufrufe verfaßte, das wußte
niemand als Herr Habietnik.

Der Erfolg übertraf die kühnsten Erwartungen. Früher war die
Bevölkerung jedem Versuch, eine demokratische Bürgerpartei zu
gründen, mit größtem Mißtrauen entgegengetreten, weil sich in
solcher Partei immer wieder die Juden vordrängten. Diesmal war das
eine rein christliche Angelegenheit, die Namen der Parteiführer
bürgten dafür, daß es sich nicht um eine von auswärtigen Juden
angezettelte Verschwörung handelte, und alle die Leute, die durch
das Judengesetz geschädigt worden waren, drängten sich in die
Komiteelokale, um Mitglieder der neuen Partei zu werden.
\pagenum{119} In hellen Scharen kamen die Kaufleute, die
Juweliere, die Stückmeister der großen Schneider, die brotlos
gewordenen Chauffeure, sie brachten ihre Frauen mit, immer größer
wurde der Ansturm, trotz des Zeter- und Mordiogeschreies der
christlichsozialen Blätter. Die „Arbeiter-Zeitung“ verhielt sich
zurückhaltend und durchaus nicht aggressiv. Man sagte sich dort,
daß zweifellos die Partei der tätigen Bürger den Sozialdemokraten
Tausende von Stimmen entziehen würde, andererseits aber dorthin
alle jene Stimmen strömen würden, die sonst sich der Wahl
enthielten oder doch wieder den Christlichsozialen oder
Großdeutschen zuliefen. Also beschränkte sie sich darauf, hier und
dort gegen das Programm der Bürgerlichen zu polemisieren, im
geheimen aber wurden in zweifelhaften Bezirken sogar Vereinbarungen
geschlossen.

Und der Tag der Wahlen, die auf den 3. April festgesetzt worden
waren, rückte näher und näher, die ganze Welt begann sich für sie
zu interessieren, die fremden Börsen nahmen eine abwartende Haltung
ein und ließen die Krone auf ihrem Tiefstand ruhen, und Wiens
bemächtigte sich zunehmende Aufregung, die wiederholt zu Exzessen
und bösartigen Tumulten führte. Denn alle Parteien arbeiteten mit
jedem verfügbaren Mittel: die antisemitischen schrien „Verrat!“ und
erzählten Schauergeschichten von der Verschwörung des
internationalen Judentums; die Sozialdemokraten hetzten gegen die
Bauern, die die arbeitende Stadtbevölkerung ausplündern und gegen
die christliche Demagogie, die sich nur selbst durch die Ausweisung
der Juden hatte bereichern wollen; die neue Bürgerpartei aber
führte immer wieder \pagenum{120} auf riesengroßen Plakaten
Ziffern auf, die bewiesen, wie furchtbar die Verelendung Wiens seit
der Ausweisung der Juden, wie Wien tatsächlich zu einem Riesendorf
geworden, wie jeder Schwung und Zug ins Große geschwunden. Und
immer wieder versicherte sie in allen Variationen und Tonarten:

„Das Ausnahmsgesetz gegen die Juden muß aufgehoben werden, aber
gleichzeitig wird es Sache einer klugen, gewissenhaften Regierung
sein, alle jene Elemente, die nicht schon vor dem Weltkrieg in Wien
seßhaft waren, fern zu halten, es sei denn, sie können vor einem
zuständigen, aus Bürgern und Arbeitern zusammengesetzten
Gerichtshof nachweisen, daß sie willens und fähig sind, in
Österreich nutzbringende, produktive, werterzeugende, dem
Gesamtwohl notwendige Arbeit zu leisten.“

Beim Bundeskanzler fanden täglich bis in die Nacht währende
Sitzungen statt, in denen beraten wurde, wie man am besten der
neuen Partei und dem wieder erstarkten Sozialismus entgegenarbeiten
könnte. Schwertfeger hatte die richtige Empfindung gehabt. Es mußte
ein neuer, mächtiger Geldkredit aufgebracht werden, die Krone mußte
steigen, die Bevölkerung erfahren, daß das Christentum der ganzen
Welt mit ihr solidarisch sei – dann würde die Regierung den Sieg
erringen. Und der Finanzminister Professor Trumm hatte sich gleich
nach der Auflösung des Hauses auf die Beine gemacht und war nach
Berlin, Paris und London gefahren, um zu betteln und zu beschwören.
Vergebens! Die großen christlichen Vereinigungen im Ausland, die
französischen Antisemiten, die holländischen Christen
\pagenum{121} – sie alle hatten Worte des Mitempfindens und der
Sympathie, erkundigten sich lebhaft nach dem Schicksal der vielen
Millionen, die sie der guten Sache schon
\erratum{geopfert}{geopfert,} und hielten die Taschen fest zu. Die
größte Enttäuschung bildete das Verhalten des amerikanischen
Billionärs Mister Huxtable, auf den man am sichersten gerechnet
hatte. Er ließ alle Telegramme und Bittschriften unbeantwortet, und
zehn Tage vor den Wahlen kam ein Kabeltelegramm des
Vertrauensmannes der österreichischen Regierung in Newyork, das
folgenden niederschmetternden Wortlaut hatte:

„Huxtable unnahbar. Hat sich heimlich mit einer jungen Jüdin aus
Chicago vermählt. Beabsichtigt, den der österreichischen Regierung
vor drei Jahren eingeräumten Kredit der jüdischen Großbank „Kuhn
und Loeb“ um ein Viertel zu verkaufen.“

Schwertfeger begann in Düsterkeit zu erstarren, die antisemitischen
Häuptlinge verloren vollends den Kopf. Bürgermeister Laberl aber
tat etwas, was die ungeheuerste Sensation erregte. Drei Tage vor
den Wahlen trat er aus dem christlichsozialen Bürgerklub aus und
der Partei der tätigen Bürger bei. Und seinem Beispiel folgte mehr
als die Hälfte der Gemeinderäte.

An diesem Tage wehte ein warmer Wind die letzten Schneemassen von
den Abhängen der Wiener Berge fort und oben im Atelier in der
Billrothstraße hielten sich zwei junge Menschenkinder heiß und
sehnsuchtsvoll umfangen. Und er flüsterte:

„Oh, wärst du schon mein!“

Und sie erwiderte traumverloren:

\pagenum{122}„Wenn du dir schon den Knebelbart abnehmen
könntest; er kitzelt so arg!“

\tb{* * *}
Die Wahlen vollzogen sich unter einer Beteiligung, wie sie kaum
jemals auf der Welt erlebt worden. Greise, Kranke, Lahme kamen zu
den Urnen, und nachmittags, als die Wahllokale geschlossen wurden,
wußte man, daß in Wien 99 Prozent der Wahlberechtigten ihre
Bürgerpflicht getan. Dann begann im ganzen Lande die Zählung der
Stimmen, die bis in die frühen Morgenstunden währte, und vormittags
verkündeten Extra-Ausgaben der „Arbeiter-Zeitung“ und der
„Weltpresse“ das staunenswerte Resultat.

Den Christlichsozialen und Großdeutschen waren nur die Landbewohner
treu geblieben, Wien hatte fast ausschließlich die Kandidaten der
Sozialisten und der Bürgervereinigung gewählt, ebenso die kleinen
Städte und das österreichische Industriegebiet. Und so setzte sich
denn das neue Parlament folgendermaßen zusammen: Siebzig
Sozialdemokraten, sechsunddreißig Mitglieder der Vereinigung der
tätigen Bürger, dreißig Christlichsoziale und vierundzwanzig
Großdeutsche. Das ergab \erratum{160}{106} Stimmen für die
Aufhebung des Ausnahmsgesetzes gegen die Juden, vierundfünfzig für
die Aufrechterhaltung. Und damit schien der schöne Traum Leos, der
freisinnigen Bürger und Sozialdemokraten zerstört, denn es fehlte
ihnen genau eine Stimme zur Zweidrittelmajorität, ohne die eine
Änderung der Verfassung nicht vorgenommen werden konnte. Trotz
ihrer vernichtenden Niederlage, trotz der Tatsache, \pagenum{123}{
} daß die Regierung sofort demissionieren und einer
sozialistisch-demokratischen weichen mußte, jubelten die
Antisemiten, sie veranstalteten Kundgebungen unter der Parole „Die
Juden bleiben draußen!“

Eine einzige Angst beherrschte die besiegten Sieger: Die Mehrheit
hatte verkündet, daß sie schon in der zweiten Sitzung des
neugewählten Hauses, die in acht Tagen stattzufinden hatte, den
Dringlichkeitsantrag auf Aufhebung des Judengesetzes und
Wiederherstellung der Freizügigkeit für jedermann stellen würde.
Wie nun, wenn ein Christlichsozialer oder großdeutscher Nationalrat
der Sitzung fernbleiben würde? An ein beabsichtigtes Fernbleiben
war nicht zu denken, aber schließlich konnte einer der Abgeordneten
vom Lande krank werden oder einen Unfall erleiden und dieser eine
würde den Gegnern die Zweidrittelmajorität sichern. Die
unterlegenen Parteien ließen daher für sämtliche gewählte
Nationalräte aus ihrem Lager am Tage vor dem Zusammentritt des
Hauses Extrazüge mit je einem begleitenden Arzt bereitstellen. Auf
diese Weise glaubten sie sich vor jedem verhängnisvollen
Zwischenfall sicher. Für Wien selbst waren Vorsichtsmaßregeln nicht
notwendig, denn in Wien war einzig und allein der Häuseragent Herr
Wenzel Krötzl von den Weinbauern und Wirten des neunzehnten
Bezirkes, denen es in dem judenreinen Wien sehr gut ging, gewählt
worden. Seiner war man in jeder Beziehung sicher und er
\erratum{erfreut}{erfreute} sich einer vorzüglichen Gesundheit.

Dieser Herr Krötzl bildete nun die einzige und letzte Hoffnung
Leos, während Lotte unter der schweren Enttäuschung \pagenum{124}{
} fast zusammenbrach. Sie weinte den ganzen Tag, kaum daß sie noch
die Energie aufbrachte, täglich zu Leo zu eilen, der sich vergebens
bemühte, ihr Mut und Hoffnung einzuflößen. Hofrat Spineder, der
selbst durch den Fortbestand des Judengesetzes schwer gekränkt und
enttäuscht wurde, kannte sich in seiner Tochter nicht mehr aus und
begann ernstlich an ihrem Verstand zu zweifeln. Sorgenvoll besprach
er ihr merkwürdiges Verhalten mit seiner Gattin.

„Was soll das alles heißen? Hat Leo vergessen, verbringt halbe Tage
mit einem neuen Verlobten, diesem Franzosen, den ich zu hassen
beginne, ohne ihn zu kennen, läßt sich von ihm beschenken, erklärt
plötzlich, daß sie am liebsten beide, den Leo und den Dufresne,
nehmen würde, und nun, da Leo nicht zurückkommen kann, sitzt sie da
und weint sich die Augen aus dem Kopf. Ich glaube, das Mädel ist
übergeschnappt!“

Frau Spineder seufzte tief.

„Mein Lieber, ich kenne selbst mein Kind nicht mehr und habe keine
Ahnung, was in seinem Herzen vorgeht. Jedenfalls müssen wir, wenn
sich zeigt, daß das Judengesetz bestehen bleibt, darauf dringen,
diesen Herrn Dufresne kennen zu lernen.“

Hofrat Spineder nickte.

„Jawohl! Und sollte sich Lotte abermals weigern oder die Sache
hinauszuschieben versuchen, so schicken wir sie zu Tante Minna nach
Klagenfurt!“

Leo überlegte Tag und Nacht und hatte schließlich einen festen Plan
gefaßt, einen Plan, der entscheiden sollte, ob er \pagenum{125}
weiterhin mit offenem Visier in Wien bleiben konnte oder zurück
nach Paris mußte. Fiel das Gesetz nicht, so wurde seine Rückreise
zwingende Notwendigkeit, da sein Freund Henry Dufresne, dessen
Namen er führte, jetzt selbst aus Südfrankreich wieder nach Paris
übersiedeln wollte und von da an die Gefahr einer Aufdeckung seines
verwegenen Spiels vorlag.

\tb{* * *}
Am Tage der Eröffnung der Nationalversammlung, also einen Tag vor
der ersten entscheidenden Sitzung, besorgte Leo Strakosch, mit
einem Handkoffer bewaffnet, allerlei Einkäufe. Bei Sacher kaufte er
für einen phantastischen Preis, für den man einmal ein ganzes
Ringstraßenhaus bekommen hätte, eine Straßburger Gänseleberpastete
in der Terrine, im Hotel Imperial ließ er sich drei Flaschen eines
köstlichen weißen Burgunders, drei Flaschen des schwersten und
kostbarsten Bordeauxweines geben, außerdem eine Flasche uralten
französischen Kognaks. Abends lauerte er dann vor dem Haustor dem
Herrn Krötzl auf, der sich gerade nach der feierlichen
Eröffnungssitzung des Hauses ins Wirtshaus begeben wollte,
gratulierte ihm herzlich zu seiner Wiederwahl und sagte:

„Lieber Herr Nationalrat, ich möchte morgen auch der historischen
Tagung des Hauses beiwohnen. Um elf ist der Beginn der Sitzung,
also werde ich auf zehn Uhr mein Auto bestellen und Sie, wenn es
Ihnen recht ist, mitnehmen.“

Herr Krötzl fühlte sich durch die Liebenswürdigkeit des vornehmen
und, wie es schien, sehr reichen jungen Franzosen \pagenum{126}
höchst geschmeichelt, er nahm die Einladung dankend an und fügte
hinzu:

„Bin Ihnen sogar sehr verbunden, wenn Sie um zehn Uhr zu mir
kommen, weil i? dann net riskier?, zu verschlafen. Meine
Wirtschafterin, das dumme Luder, vergißt am End? noch, mich zu
wecken, und i? hab? an so schweren Schlaf, daß i die Weckuhr net
hör?. Dös wär? aber a schöne G?schicht?, wann i morgen verschlafen
tät. Nachher hätten mir in vierundzwanzig Stunden die Saujuden, die
verfluchten, wieder in Wien!“

Henry Dufresne nahm die übernommene Pflicht, Österreich vor den
Juden zu schützen, sehr ernst, denn er läutete schon um halb zehn
Uhr bei Herrn Krötzl an. Ein schlumpiges, zwar ungewaschenes, aber
noch geschminktes junges Ding öffnete ihm und ließ den ihr
wohlbekannten hübschen Franzosen, der eine mächtige Schachtel trug,
ohneweiters ein, ein wenig enttäuscht, daß er ihr und ihren
reichlichen Blößen nicht die geringste Aufmerksamkeit schenkte,
sondern sich damit begnügte, ihr eine Banknote zu geben und sie zu
bitten, gleich die Morgenblätter aus der Trafik zu holen.

Leo packte im Vorzimmer umständlich die Schachtel aus, dann, als
das Mädchen gegangen war, um seinen Auftrag auszuführen, begab er
sich rasch in die Küche, rückte den Stundenzeiger der Kuckucksuhr
um eine volle Stunde zurück, schlich sich auf den Zehenspitzen in
das Wohnzimmer, bearbeitete dort die große Pendeluhr in gleicher
Weise und öffnete schließlich, ohne anzuklopfen, leise die Türe zum
Schlafzimmer des Herrn Nationalrates. Richtig lag dieser mit
offenem Maul sägend und schnarchend in seinem Bett \pagenum{127}
und auf dem Nachtkästchen erblickte Leo sofort die goldene
Taschenuhr, die eben auf ein viertel vor zehn wies. Blitzschnell
war auch sie auf ein viertel vor neun gestellt und dann machte sich
der Franzose an die unerquickliche Arbeit, Herrn Krötzl, das Wiener
Postament der christlichsozialen Partei, zu wecken. Es dauerte
geraume Zeit, bevor Krötzl endlich die verquollenen Äuglein
aufschlug und die Situation begriff.

„Jessas, der Herr Dufresne, is? schon so spät?“ Und dann, mit einem
Blick auf die Taschenuhr, brummend: „Noch net amal Neun is?! Da
hätt? i? noch a ganze Stund? schlafen können!“

„Jawohl,“ sagte Leo lachend, „wenn ich nicht eine bessere
Unterhaltung für Sie und mich wüßte. Stellen Sie sich nur vor, wie
ich gestern nacht nach Hause komme, finde ich ein Postpaket aus
Paris vor mit den besten Weinen, die Frankreich besitzt. Na, und
weil ich mich wirklich über Ihren Sieg von ganzem Herzen freue,
denke ich, daß wir, bevor wir ins Parlament fahren, noch eine
kleine Siegesfeier unter uns veranstalten können. Sie sind ja
Kenner, Herr Nationalrat, und werden sehr bald zugeben, einen
solchen Wein, wie ich ihn Ihnen kredenze, im Leben noch nicht
genossen zu haben.“

Wie elektrisiert sprang Herr Krötzl aus dem Bett, zog sich
notdürftig an und streichelte dann bewundernd die eine der sechs
Weinflaschen nach der anderen, die mit allen Zeichen des
ehrwürdigen Alters vor ihm standen. Weißbrot war vorhanden, die
Straßburger Pastete entlockte Herrn Krötzl ein rülpsendes Grunzen,
das sich in einen Jubelhymnus \pagenum{128} verwandelte, als das
erste Glas des goldgelben Burgunders durch seine Kehle rann.

„A so a Weinerl! Wann man den immer hätt?, dann tät? man an anderer
Mensch wer?n! Ka Wunder, wenn die Franzosen so an Schick zum Leben
haben, wo ?s so an Wein bei ihnen gibt!“

Das zweite Glas wurde auf den Sieg des Herrn Krötzl geleert, das
dritte auf „Nieder mit den Juden“, das vierte auf „Hoch die schöne,
judenreine Stadt Wien“. Dann wurde einer Flasche des blutroten
Bordeaux der Hals gebrochen, und als sie zur Neige ging und Leo die
dritte Flasche entkorkte, trug ihm Krötzl die Bruderschaft an. Bei
der vierten Flasche machte er den Franzosen mit den Geheimnissen
seines Sexuallebens bekannt und erklärte, daß Frauenzimmer über
vierzehn eigentlich alte Weiber seien. Die sechste Flasche wurde
von Leo, ohne daß Krötzl, dem sich die Welt vor den Augen zu drehen
begann, es merkte, zur Hälfte mit Kognak gemischt, und nun hieß es
– Schluß machen, weil der Herr Nationalrat sonst überhaupt nicht
mehr die Treppen hinuntergebracht hätte werden können und die
richtiggehende Uhr auf zwölf ging, also die Gefahr bestand, daß
jeden Augenblick die Parteigenossen Krötzls nach ihm fahnden
würden. Daß Leo bei solcher Zecherei selbst vollständig nüchtern
geblieben war, verdankte er lediglich dem Umstand, daß er den
Inhalt seines Glases regelmäßig unter den Tisch auf den schönen
Perserteppich gegossen hatte.

Mit ungeheurer Anstrengung beendigte Leo die Toilettierung des
Nationalrates, dann trug er ihn fast die vielen Treppen hinunter
und beförderte ihn mit Hilfe des \pagenum{129} Chauffeurs in das
Innere des geschlossenen Automobils. Grinsend hatte der Chauffeur
dem Franzosen, den er oft zu führen pflegte, zugenickt. Leo stieg
ein, setzte sich neben Krötzl, der schon als halbe Weinleiche in
der Ecke lag, und in mäßigem Tempo ging es vorwärts.

Am Tage vorher hatte Leo mit dem Chauffeur eine wichtige
Unterredung gehabt, die mit der Frage begann:

„Wollen Sie hundert französische Francs verdienen?“

Der Chauffeur hatte ungeheure Augen gemacht, war blutrot geworden
und erwiderte keuchend:

„Herr, für hundert Francs führ? ich Sie auf den Mond!“

Aber der Franzose erwies sich als wesentlich bescheidener. Er
erklärte, daß es sich um eine Wette handle und er nichts weiter zu
tun habe, als vor dem Haus in der Billrothstraße zu warten, bis er,
Monsieur Dufresne, mit einem voraussichtlich schwergeladenen Herrn
einsteigen werde. Daraufhin habe das Auto stadtwärts bis zur
Volksoper zu fahren, wo er aussteigen werde. Nunmehr müsse die
Fahrt weiter bis zur großen Irrenanstalt am Steinhof, die weit
außerhalb im Südwesten der Stadt liegt, gehen. Dort müsse der
Chauffeur so lange stehen bleiben, bis sein betrunkener Gast sich
melde. Und dann folgten weitere ausführliche Instruktionen für den
intelligenten, lustigen Chauffeur.

Alles wickelte sich programmäßig ab. Bevor noch das Auto bei der
Volksoper angelangt war, schlief Herr Krötzl, nachdem er sich
heftig übergeben hatte, den Schlaf des gerechten Säufers und Leo
konnte ungestört ausspringen. \pagenum{130} Während Leo nach dem
Parlament eilte, setzte der Chauffeur die fast halbstündige Fahrt
nach Steinhof fort, wo er auf offener Straße seelenruhig stehen
blieb und eine der guten Zigaretten Leos nach der anderen rauchte.
So wurde es schließlich nahezu zwei Uhr, als endlich Herr Krötzl
mit schmerzendem Schädel erwachte. Minuten vergingen, bevor er die
Situation begriff und sich endlich klar darüber war, daß er sich in
total verunreinigtem Zustande allein in einem Automobil befand.
Schließlich, nach weiteren Minuten, erkannte er sogar, daß er sich
durchaus nicht vor dem Parlament, sondern in der unmittelbaren Nähe
der Irrenanstalt am Steinhof aufhielt. Er sah verwirrt auf seine
Uhr. Da sie zurückgerichtet war, wies sie auf eins. Entsetzt riß
Krötzl den Wagenschlag auf, schimpfend und tobend drang er auf den
Chauffeur ein, der gleichmütig erklärte, er habe als Fahrtziel
Steinhof verstanden und der andere Herr sei unterwegs ausgestiegen.
Mit den Fäusten fuhr sich Krötzl in die Haare, er weinte, schrie,
bekam fast einen Tobsuchtsanfall, nannte den Chauffeur einen
Staatsverbrecher, sprach von einer furchtbaren Verschwörung und
Rache und flehte schließlich den Wagenlenker, der auch grob zu
werden begann, an, er möge mit Windeseile nach dem Parlament
fahren.

Tausend Meter etwa fuhr dann auch das Auto, dann blieb es weit und
breit von jeder Behausung entfernt stehen, und achselzuckend
erklärte der Chauffeur, daß etwas am Motor in Unordnung sei und er
nicht weiter könne.

Im Galopp rannte der nüchtern gewordene Krötzl die tausend Meter
nach der Irrenanstalt zurück. Dort benahm \pagenum{131} er sich
dem Pförtner gegenüber so aufgeregt, daß dieser ihn für einen
entsprungenen Insassen hielt und Wärter herbeirief. Es verging eine
weitere halbe Stunde, bevor Krötzl zu einem Fernsprecher geführt
wurde, er bekam natürlich keine Verbindung mit dem Parlament, da
dort alle Nummern besetzt waren, und als er endlich die Verbindung
hatte und der Parteisekretär zur Stelle gebracht war, schrie ihm
dieser in die Ohren, daß er ein besoffenes Schwein sei; ein von den
Juden gekaufter Gauner und bereits alles vorbei wäre.

„Das Judengesetz ist gefallen!“ Mit diesen Worten läutete er dem
unglücklichen Nationalrat in die Ohren, der daraufhin in eine
lange, wohltätige Ohnmacht fiel.

\tb{* * *}
Als Leo das Parlamentsgebäude betrat, hatte der neugewählte
Präsident eben die schon am Tage vorher an Stelle des
zurückgetretenen Kabinetts gewählten Minister begrüßt und
mitgeteilt, daß zwei Dringlichkeitsanträge eingebracht worden
seien, dahingehend, den Paragraph 11 der Bundesverfassung, der den
Juden und Judenabkömmlingen den Aufenthalt in Österreich
untersagt, zu streichen.

Ein sozialdemokratischer Nationalrat erhob sich und stellte den
Antrag, über die gestellten Dringlichkeitsanträge sofort zu
verhandeln. Trotz des tosenden Lärmens der Christlichsozialen und
Großdeutschen pflichtete die Mehrheit bei, worauf der Präsident dem
Führer der Sozialdemokraten, Doktor Wolters, als erstem Proredner
das Wort erteilte.

\pagenum{132}Wolters wies darauf hin, daß er und seine
Parteikollegen schon vor fast drei Jahren gegen das Gesetz gewesen
seien, das einen Faustschlag gegen die Menschenrechte, einen
Rückfall in das finstere Mittelalter bedeutete. Damals sei die
Opposition niedergeschrieen, beschimpft und aus dem Saal gedrängt
worden, heute aber habe das verführte und berauschte Volk sie in
solcher Zahl zurückgeführt, daß nunmehr die Macht in ihren und den
Händen anderer freisinniger Männer liege. Wolters entwickelte dann
die Ereignisse der letzten Jahre, wies den furchtbaren
Zusammenbruch Österreichs nach, führte schlagende Ziffern an und
schloß mit den Worten:

„Das kühne, allzukühne Werk des Mannes, der sich göttliche Macht
anmaßte und nun nicht einmal mehr einen Sitz in diesem Hause
bekommen konnte, ist zusammengebrochen, und draußen warten
hunderttausend Arbeitslose und mit ihnen alle tätigen, zur
Verzweiflung getriebenen Kräfte, daß das neue Haus einer neuen
Zukunft die Tore öffne und unseren jüdischen Mitbürgern die
Möglichkeit gebe, wieder an unserer Seite nicht gegen uns, sondern
mit uns ihre Intelligenz, ihre Emsigkeit und schöpferische
Arbeitskraft im Interesse des schwergeprüften und fast ruinierten
Landes zu betätigen.“

Nachdem der Beifallssturm, an dem sich auch die Galerie beteiligte,
verklungen war, ergriff der zweite Pro-Redner, Herr Habietnik, der
von den Geschäftsleuten der Inneren Stadt sein Mandat bekommen
hatte, das Wort. In launiger, oft durch schallende Heiterkeit
unterbrochener Rede schilderte \pagenum{133} er das verarmte,
verdorfte Wien von heute, gab die Erfahrungen im eigenen Betriebe
zum besten und sagte:

„Posemukel ist eine Großstadt im Vergleiche zu Wien von heute. Wien
ist ein ungeheures Dorf mit anderthalb Millionen Einwohnern
geworden, und wenn wir die Juden nicht wieder hereinlassen, so
werden wir es demnächst erleben, daß statt vornehmer Geschäfte in
der Kärntnerstraße Jahrmarktsbuden stehen und auf dem Stephansplatz
Viehmärkte werden abgehalten werden. Die Wiener sind in ihrem
Tiefinnersten in Verzweiflung über diese Rückentwicklung, die sie
nicht aufhalten können und nicht zuletzt haben die Wiener Frauen
und Mädchen, indem sie die christlichsoziale Partei im Stich
ließen, gezeigt, daß sie wieder ein blühendes, lustiges Wien voll
Luxus, auch wenn es mitunter einen orientalischen Anstrich hat,
haben wollen.“

Die weiteren Ausführungen Habietniks gingen in einer seltsamen
Unruhe verloren, die sich über das Haus verbreitete. Was war
geschehen? Nun, man hatte endlich auf der rechten Seite des Hauses
entdeckt, daß der Nationalrat Krötzl nicht anwesend war, und eine
Katastrophenstimmung bemächtigte sich der Christlichsozialen und
Großdeutschen. Sie hörten nicht einmal ihren eigenen Kontra-Redner
an, die Diener wurden mit Automobilen ausgeschickt, um Krötzl aus
seinem Bureau in der Inneren Stadt oder aus der Wohnung in der
Billrothstraße zu holen.

Noch wäre vielleicht die Situation zu retten gewesen, wenn man die
Geistesgegenwart gehabt hatte, den Kontra-Redner zu veranlassen,
stundenlang bis zum Eintreffen Krötzls zu sprechen. Aber man hatte
total den Kopf verloren, \pagenum{134} der christlichsoziale
Redner, Herr Wurm, kürzte, als er die Unruhe bemerkte und seine
Genossen verschwinden sah, seine Rede sogar ab, und schon war ein
bürgerlicher Antrag auf Schluß der Debatte und Abkürzung der
weiteren Redezeiten auf fünf Minuten mit der erforderlichen
Zweidrittelmehrheit angenommen.

Vergebens schrieen die überrumpelten Antisemiten Zeter und Mordio,
der sozialistische Präsident waltete mit eiserner Energie seines
Amtes, entzog jedem der wenigen schon vorgemerkten Redner nach fünf
Minuten das Wort und unter enormer Spannung und allgemeiner
Aufregung strömten die Abgeordneten wieder in den Saal, um bei der
kommenden namentlichen Abstimmung anwesend zu sein.

Herr Krötzl war noch immer nicht da, die Diener konnten nur
berichten, daß er in seinem Bureau überhaupt nicht gewesen und sein
Wohnhaus in Begleitung eines anderen Herrn vormittags, ersichtlich
angeheitert, verlassen habe.

Ein Großdeutscher machte den letzten Rettungsversuch. Er erbat und
erhielt das Wort, um zur Geschäftsordnung zu sprechen und sagte:

„Der Nationalrat Herr Krötzl ist nicht anwesend und wir haben
Anzeichen dafür, daß er mit Gewalt ferne gehalten wird, ja wir
haben begründeten Anlaß zur Befürchtung, daß er das Opfer eines
Verbrechens geworden ist. Unter solchen Umständen kann unmöglich
über ein Gesetz abgestimmt werden, das über das Schicksal des
Landes entscheiden wird. Wenn auf Seite der neuen Mehrheit dieses
Hauses auch nur ein Funken Anstandsgefühl herrscht, so wird sie mit
mir \pagenum{135} darin übereinstimmen, daß wir uns zunächst auf
zwei Stunden vertagen. Bis dahin werden wir wohl Klarheit darüber
haben, ob unser hochverehrter Kollege, Herr Nationalrat Krötzl,
überhaupt noch unter den Lebenden weilt.“

Totenstille entstand nach diesen Worten, die nicht zurückzuweisen
waren.

Sollte Krötzl wirklich mit Gewalt verhindert worden sein, an der
Sitzung teilzunehmen, so mußte man wohl oder übel warten.

In diesem höchst kritischen Augenblick schlich sich ein Herr mit
Knebelbart unbeobachtet in den Sitzungssaal, winkte Herrn Habietnik
zu sich heran und flüsterte vor Aufregung keuchend mit ihm, worauf
sich Herr Habietnik zum Worte meldete.

„Ich kann dem Hohen Haus auf Ehr? und Gewissen versichern, daß Herr
Krötzl nicht ermordet und auf keinerlei gewaltsame Weise verhindert
wurde, dieser so überaus wichtigen Sitzung beizuwohnen. Herr Krötzl
\erratum{befinde}{befindet} sich irgendwo in einem Automobil, in
dem er einen Kanonenrausch, von dem ihn der Chauffeur nicht
erwecken kann, ausschläft. Der sehr ehrenwerte Herr Krötzl, diese
einzige Wiener Zierde der christlichsozialen Partei, hat nämlich
schon am frühen Morgen in Gesellschaft eines lustigen Kumpanen,
seines Wohnungsnachbars, eine kleine Siegesfeier begangen und
entschieden mehr getrunken, als er verträgt. Sein Nachbar, der mir
diese Mitteilung macht und den ich persönlich als zuverlässigen
Ehrenmann kenne, fuhr dann mit Krötzl in einem Autotaxi hieher,
mußte aber vorzeitig aussteigen, weil er den Gestank im Wagen nicht
aushielt. Herr Krötzl \pagenum{136} gehört nämlich zu jener
alten Garde, die sich lieber übergibt als stirbt. Wo sich in diesem
Augenblick die springlebendige Leiche des Herrn Krötzl befindet,
weiß ich nicht, aber das geht uns auch nichts an und man wird
unmöglich verlangen, daß wir uns vertagen, bis Herr Krötzl nüchtern
geworden ist.“

Tosende Heiterkeit erfüllte das Haus und es wurde nunmehr nach der
Anordnung des Präsidenten zur Abstimmung geschritten.
Hundertundsechs Nationalräte stimmten für die Eliminierung des
Ausnahmsgesetzes, dreiundfünfzig dagegen – das Gesetz war gefallen!
Und die hunderttausend Menschen, die sich auf der Straße vor dem
Parlament angesammelt hatten, riefen diesmal nicht „Heil!“, sondern
„Hurra!“ Sie waren nicht so begeistert wie vor drei Jahren, sondern
ein wenig beschämt, hatten aber wieder ihren Humor gefunden und
schon begannen Witze in der Luft zu schwirren.

Leo hatte nur die Abstimmung abgewartet, dann stürzte er aus dem
Parlamentsgebäude, warf sich in ein Autotaxi und fuhr nach der
Linken Wienzeile zur „Arbeiter-Zeitung“. Dort ließ er sich in
dringender Angelegenheit beim Chefredakteur melden, mit dem er eine
halbstündige Unterredung ohne Zeugen hatte. Als er sich
verabschiedete, schüttelte ihm der Redakteur kräftig beide Hände
und sagte lachend:

„Sie haben Außerordentliches geleistet und ich freue mich mit Ihnen
von ganzem Herzen! Ihre Frechheit bewundere ich einfach! Man kann
da wirklich nicht umhin, von~–“

„Jüdischer Frechheit zu sprechen“, ergänzte Leo vergnügt und eilte
die Treppen hinab.

\tb{* * *}
\pagenum{137}Kaum waren die Extra-Ausgaben der Zeitungen
erschienen, die das Ende der Judenverbannung verkündeten, als auch
schon eine zweite Extraausgabe der „Arbeiter-Zeitung“ ausgerufen
wurde:

\begin{center}
\textit{Die Krone steigt!}

\end{center}
Zürich. Auf der hiesigen Börse wurden die drahtlich und
telephonisch einlangenden Nachrichten von der entscheidenden
Sitzung der Wiener Nationalversammlung mit fieberhaftem Interesse
verfolgt. Kaum war das Fallen des Antijudengesetzes zur Gewißheit
geworden, als auch schon umfangreiche Kronenankäufe, darunter
solche von amerikanischen und englischen Finanzgruppen, erfolgten.
Die österreichische gestempelte Krone ging sprunghaft auf das
Doppelte, zum Börsenschluß sogar auf das Dreifache hinauf.

Um sechs Uhr abends erschien eine dritte Extra-Ausgabe, die in ganz
Wien Aufsehen und mit Galgenhumor gemischte Heiterkeit hervorrief.
Die Nachricht lautete:

\begin{center}
\textit{Ankunft des ersten Juden in Wien.}

\end{center}
Wie wir mitteilen können, ist soeben der erste Jude aus dem Exil
nach Wien zurückgekehrt. Es ist dies der junge, aber bereits
weltberühmte Maler und Radierer Leo Strakosch, der die ganze Zeit
von Heimweh erfüllt in Paris verbracht und sich vorgestern von dort
an die österreichisch-mährische Grenze nach Lundenburg begeben
hatte. Als er telephonisch von der Nichtigkeitserklärung des
Ausweisungsgesetzes erfuhr, begab er sich sofort per Automobil nach
seiner Vaterstadt Wien. Er hält sich derzeit im Hause seines
zukünftigen Schwiegervaters, des Hofrates Spineder, in der
\pagenum{138} Kobenzlgasse auf, wo er nach jahrelanger bitterer
Trennung die in Treue und Liebe seiner harrende Braut umarmt.

Diese Extra-Ausgabe bildete einen wohlwollend-boshaften Scherz des
Chefredakteurs der „Arbeiter-Zeitung“. Gleich nach ihr erschien
aber eine Extraausgabe der „Weltpresse“ mit zwei sensationellen
Nachrichten. In der einen wurde angekündigt, daß sich der ehemalige
Bundeskanzler Doktor Schwertfeger in Verzweiflung über das
Scheitern seines so groß und ehrlich gedachten Werkes durch einen
Revolverschuß entleibt habe. Anknüpfend daran machte die
„Weltpresse“ die Mitteilung, daß sie, dem Willen der
überwältigenden Mehrheit der Bevölkerung Wiens folgend, vom
heutigen Tage an als das Organ der neuen Partei der tätigen Bürger
erscheinen werde.

\tb{* * *}
Leo war von der Redaktion der „Arbeiter-Zeitung“ aus tatsächlich
direkt nach Grinzing gefahren. Lotte, die ebenso wie ihre Eltern
von dem Verlauf der Parlamentssitzung bereits unterrichtet war,
erwartete ihren Bräutigam am offenen Fenster im Parterregeschoß.
Und als das Auto vorgefahren war und Leo sie erblickte, erschien
ihm der Weg durch den Hausflur zu weitläufig, mit einem Satz
schwang er sich auf das Fensterbrett und schon hielten die beiden
jungen Leute einander lachend und weinend umschlungen. Da Leo aber
trotz seiner turnerischen Gewandtheit bei seinem abgekürzten
Eintrittsverfahren eine Fensterscheibe eingeschlagen hatte, was ein
hörbares Klirren und Schmettern verursachte, kamen der Hofrat und
seine Gattin aus dem \pagenum{139} nebengelegenen Wohnzimmer
bestürzt herbei und blieben angesichts ihrer Tochter, die von einem
fremden, knebelbärtigen Herrn unaufhörlich abgeküßt wurde,
überrascht stehen. Bis der Hofrat so energisch zu husten begann,
daß Lotte es vernahm und sich blutrot aus den Armen des Geliebten
befreite, um ihn ihren Eltern vorzustellen:

„Papa, Mama, dies ist mein Bräutigam, Henry Dufresne\ldots{}!“

„\latein{Recte} Leo Strakosch“, lautete die Ergänzung und Leo warf
sich auch schon dem Hofrat und dann seiner zukünftigen
Schwiegermutter in die Arme.

Nachdem sich die erste Freude und Verwirrung gelegt, tat Herr
Spineder das, was ein Hofrat in solcher Situation zu tun hatte. Er
sagte:

„Nun, Kinder, erzählt mir einmal alles ordentlich der Reihe nach.“

Frau Spineder aber tat das, was jede andere ordentliche Hausfrau an
ihrer Stelle getan hätte. Sie weinte, erklärte vor Aufregung nicht
stehen und gehen zu können und lief nach der Küche, um für ein
ordentliches Souper zu sorgen.

Die Unterhaltung zwischen dem Hofrat, Lotte und Leo spielte sich
indessen im Badezimmer ab, wo Leo sich zuerst mit einer
Papierschere den Knebelbart abschnitt, um sich dann zu rasieren und
gleichzeitig zu erzählen. Und das war sehr gut so, denn gerade als
er rasiert und wieder ein schöner, glatter junger Mann war,
ereignete sich ganz Unerwartetes.

Ein Automobil mit Herrn Habietnik, einem sozialdemokratischen
Nationalrat und einem bekehrten Gemeinderat \pagenum{140} fuhr
vor und die Herren teilten Leo mit, daß er unbedingt mit ihnen zum
Rathause fahren müsse, um sich der dort versammelten Menschenmenge
zu zeigen und eine Ansprache des Bürgermeisters zu erdulden.

Sträuben nützte nichts, Leo mußte mit, aber Lotte, die die Garantie
dafür übernahm, daß sie rechtzeitig zum Abendessen zurück sein
würden, fuhr mit ihm.

Bis zum Schottentor verlief die Fahrt ganz glatt, dann stellte sich
ein Hindernis ein. Die Menschenmassen standen hier so dicht
aneinandergedrängt, daß das Auto nicht vorwärts kam. Worauf sich
der Gemeinderat hinausbeugte und in bester Absicht, wenn auch mit
wenig Zartgefühl den Leuten zuschrie:

„Laßt?s uns durch! Der Herr Leo Strakosch, der erste Jud, der
wieder in Wien ist, muß zum Rathaus!“

Diese \erratum{Worten}{Worte} waren das Signal zu einem stürmischen
Jubelschrei, und das Auto konnte zwar nicht durch, sondern mußte
hier mit Lotte warten, aber Leo saß auch schon auf den Schultern
zweier handfester Männer und wurde unter dem Jauchzen und Johlen
und Hurra-Geschrei der Massen zum Rathaus getragen.

Das schöne Rathaus war wieder illuminiert, sah wieder wie eine
brennende Fackel aus und mühsam nur konnten sich die Männer mit Leo
auf den Schultern Bahn machen. Fanfarenklänge, Trompetentöne, der
Bürgermeister von Wien, Herr Karl Maria Laberl, betrat den Balkon,
streckte segnend seine Arme aus und hielt eine zündende Ansprache,
die mit den Worten begann:

„Mein lieber Jude!~–~–“

\begin{center}
\textit{Ende.}

\end{center}
%„Corona“-Druck (G. Davis \& Co.), Wien IX.

\begin{Verbatim}[fontsize=\footnotesize]


End of the Project Gutenberg EBook of Die Stadt ohne Juden, by Hugo Bettauer

*** END OF THIS PROJECT GUTENBERG EBOOK DIE STADT OHNE JUDEN ***

***** This file should be named 35569-h.htm or 35569-h.zip *****
This and all associated files of various formats will be found in:
        http://www.gutenberg.org/3/5/5/6/35569/

Produced by Jana Srna, Norbert H. Langkau and the Online
Distributed Proofreading Team at http://www.pgdp.net. Cover
image cleaned up by Sharon Joiner


Updated editions will replace the previous one--the old editions
will be renamed.

Creating the works from public domain print editions means that no
one owns a United States copyright in these works, so the Foundation
(and you!) can copy and distribute it in the United States without
permission and without paying copyright royalties.  Special rules,
set forth in the General Terms of Use part of this license, apply to
copying and distributing Project Gutenberg-tm electronic works to
protect the PROJECT GUTENBERG-tm concept and trademark.  Project
Gutenberg is a registered trademark, and may not be used if you
charge for the eBooks, unless you receive specific permission.  If you
do not charge anything for copies of this eBook, complying with the
rules is very easy.  You may use this eBook for nearly any purpose
such as creation of derivative works, reports, performances and
research.  They may be modified and printed and given away--you may do
practically ANYTHING with public domain eBooks.  Redistribution is
subject to the trademark license, especially commercial
redistribution.



*** START: FULL LICENSE ***

THE FULL PROJECT GUTENBERG LICENSE
PLEASE READ THIS BEFORE YOU DISTRIBUTE OR USE THIS WORK

To protect the Project Gutenberg-tm mission of promoting the free
distribution of electronic works, by using or distributing this work
(or any other work associated in any way with the phrase "Project
Gutenberg"), you agree to comply with all the terms of the Full Project
Gutenberg-tm License (available with this file or online at
http://gutenberg.org/license).


Section 1.  General Terms of Use and Redistributing Project Gutenberg-tm
electronic works

1.A.  By reading or using any part of this Project Gutenberg-tm
electronic work, you indicate that you have read, understand, agree to
and accept all the terms of this license and intellectual property
(trademark/copyright) agreement.  If you do not agree to abide by all
the terms of this agreement, you must cease using and return or destroy
all copies of Project Gutenberg-tm electronic works in your possession.
If you paid a fee for obtaining a copy of or access to a Project
Gutenberg-tm electronic work and you do not agree to be bound by the
terms of this agreement, you may obtain a refund from the person or
entity to whom you paid the fee as set forth in paragraph 1.E.8.

1.B.  "Project Gutenberg" is a registered trademark.  It may only be
used on or associated in any way with an electronic work by people who
agree to be bound by the terms of this agreement.  There are a few
things that you can do with most Project Gutenberg-tm electronic works
even without complying with the full terms of this agreement.  See
paragraph 1.C below.  There are a lot of things you can do with Project
Gutenberg-tm electronic works if you follow the terms of this agreement
and help preserve free future access to Project Gutenberg-tm electronic
works.  See paragraph 1.E below.

1.C.  The Project Gutenberg Literary Archive Foundation ("the Foundation"
or PGLAF), owns a compilation copyright in the collection of Project
Gutenberg-tm electronic works.  Nearly all the individual works in the
collection are in the public domain in the United States.  If an
individual work is in the public domain in the United States and you are
located in the United States, we do not claim a right to prevent you from
copying, distributing, performing, displaying or creating derivative
works based on the work as long as all references to Project Gutenberg
are removed.  Of course, we hope that you will support the Project
Gutenberg-tm mission of promoting free access to electronic works by
freely sharing Project Gutenberg-tm works in compliance with the terms of
this agreement for keeping the Project Gutenberg-tm name associated with
the work.  You can easily comply with the terms of this agreement by
keeping this work in the same format with its attached full Project
Gutenberg-tm License when you share it without charge with others.

1.D.  The copyright laws of the place where you are located also govern
what you can do with this work.  Copyright laws in most countries are in
a constant state of change.  If you are outside the United States, check
the laws of your country in addition to the terms of this agreement
before downloading, copying, displaying, performing, distributing or
creating derivative works based on this work or any other Project
Gutenberg-tm work.  The Foundation makes no representations concerning
the copyright status of any work in any country outside the United
States.

1.E.  Unless you have removed all references to Project Gutenberg:

1.E.1.  The following sentence, with active links to, or other immediate
access to, the full Project Gutenberg-tm License must appear prominently
whenever any copy of a Project Gutenberg-tm work (any work on which the
phrase "Project Gutenberg" appears, or with which the phrase "Project
Gutenberg" is associated) is accessed, displayed, performed, viewed,
copied or distributed:

This eBook is for the use of anyone anywhere at no cost and with
almost no restrictions whatsoever.  You may copy it, give it away or
re-use it under the terms of the Project Gutenberg License included
with this eBook or online at www.gutenberg.org

1.E.2.  If an individual Project Gutenberg-tm electronic work is derived
from the public domain (does not contain a notice indicating that it is
posted with permission of the copyright holder), the work can be copied
and distributed to anyone in the United States without paying any fees
or charges.  If you are redistributing or providing access to a work
with the phrase "Project Gutenberg" associated with or appearing on the
work, you must comply either with the requirements of paragraphs 1.E.1
through 1.E.7 or obtain permission for the use of the work and the
Project Gutenberg-tm trademark as set forth in paragraphs 1.E.8 or
1.E.9.

1.E.3.  If an individual Project Gutenberg-tm electronic work is posted
with the permission of the copyright holder, your use and distribution
must comply with both paragraphs 1.E.1 through 1.E.7 and any additional
terms imposed by the copyright holder.  Additional terms will be linked
to the Project Gutenberg-tm License for all works posted with the
permission of the copyright holder found at the beginning of this work.

1.E.4.  Do not unlink or detach or remove the full Project Gutenberg-tm
License terms from this work, or any files containing a part of this
work or any other work associated with Project Gutenberg-tm.

1.E.5.  Do not copy, display, perform, distribute or redistribute this
electronic work, or any part of this electronic work, without
prominently displaying the sentence set forth in paragraph 1.E.1 with
active links or immediate access to the full terms of the Project
Gutenberg-tm License.

1.E.6.  You may convert to and distribute this work in any binary,
compressed, marked up, nonproprietary or proprietary form, including any
word processing or hypertext form.  However, if you provide access to or
distribute copies of a Project Gutenberg-tm work in a format other than
"Plain Vanilla ASCII" or other format used in the official version
posted on the official Project Gutenberg-tm web site (www.gutenberg.org),
you must, at no additional cost, fee or expense to the user, provide a
copy, a means of exporting a copy, or a means of obtaining a copy upon
request, of the work in its original "Plain Vanilla ASCII" or other
form.  Any alternate format must include the full Project Gutenberg-tm
License as specified in paragraph 1.E.1.

1.E.7.  Do not charge a fee for access to, viewing, displaying,
performing, copying or distributing any Project Gutenberg-tm works
unless you comply with paragraph 1.E.8 or 1.E.9.

1.E.8.  You may charge a reasonable fee for copies of or providing
access to or distributing Project Gutenberg-tm electronic works provided
that

- You pay a royalty fee of 20% of the gross profits you derive from
     the use of Project Gutenberg-tm works calculated using the method
     you already use to calculate your applicable taxes.  The fee is
     owed to the owner of the Project Gutenberg-tm trademark, but he
     has agreed to donate royalties under this paragraph to the
     Project Gutenberg Literary Archive Foundation.  Royalty payments
     must be paid within 60 days following each date on which you
     prepare (or are legally required to prepare) your periodic tax
     returns.  Royalty payments should be clearly marked as such and
     sent to the Project Gutenberg Literary Archive Foundation at the
     address specified in Section 4, "Information about donations to
     the Project Gutenberg Literary Archive Foundation."

- You provide a full refund of any money paid by a user who notifies
     you in writing (or by e-mail) within 30 days of receipt that s/he
     does not agree to the terms of the full Project Gutenberg-tm
     License.  You must require such a user to return or
     destroy all copies of the works possessed in a physical medium
     and discontinue all use of and all access to other copies of
     Project Gutenberg-tm works.

- You provide, in accordance with paragraph 1.F.3, a full refund of any
     money paid for a work or a replacement copy, if a defect in the
     electronic work is discovered and reported to you within 90 days
     of receipt of the work.

- You comply with all other terms of this agreement for free
     distribution of Project Gutenberg-tm works.

1.E.9.  If you wish to charge a fee or distribute a Project Gutenberg-tm
electronic work or group of works on different terms than are set
forth in this agreement, you must obtain permission in writing from
both the Project Gutenberg Literary Archive Foundation and Michael
Hart, the owner of the Project Gutenberg-tm trademark.  Contact the
Foundation as set forth in Section 3 below.

1.F.

1.F.1.  Project Gutenberg volunteers and employees expend considerable
effort to identify, do copyright research on, transcribe and proofread
public domain works in creating the Project Gutenberg-tm
collection.  Despite these efforts, Project Gutenberg-tm electronic
works, and the medium on which they may be stored, may contain
"Defects," such as, but not limited to, incomplete, inaccurate or
corrupt data, transcription errors, a copyright or other intellectual
property infringement, a defective or damaged disk or other medium, a
computer virus, or computer codes that damage or cannot be read by
your equipment.

1.F.2.  LIMITED WARRANTY, DISCLAIMER OF DAMAGES - Except for the "Right
of Replacement or Refund" described in paragraph 1.F.3, the Project
Gutenberg Literary Archive Foundation, the owner of the Project
Gutenberg-tm trademark, and any other party distributing a Project
Gutenberg-tm electronic work under this agreement, disclaim all
liability to you for damages, costs and expenses, including legal
fees.  YOU AGREE THAT YOU HAVE NO REMEDIES FOR NEGLIGENCE, STRICT
LIABILITY, BREACH OF WARRANTY OR BREACH OF CONTRACT EXCEPT THOSE
PROVIDED IN PARAGRAPH 1.F.3.  YOU AGREE THAT THE FOUNDATION, THE
TRADEMARK OWNER, AND ANY DISTRIBUTOR UNDER THIS AGREEMENT WILL NOT BE
LIABLE TO YOU FOR ACTUAL, DIRECT, INDIRECT, CONSEQUENTIAL, PUNITIVE OR
INCIDENTAL DAMAGES EVEN IF YOU GIVE NOTICE OF THE POSSIBILITY OF SUCH
DAMAGE.

1.F.3.  LIMITED RIGHT OF REPLACEMENT OR REFUND - If you discover a
defect in this electronic work within 90 days of receiving it, you can
receive a refund of the money (if any) you paid for it by sending a
written explanation to the person you received the work from.  If you
received the work on a physical medium, you must return the medium with
your written explanation.  The person or entity that provided you with
the defective work may elect to provide a replacement copy in lieu of a
refund.  If you received the work electronically, the person or entity
providing it to you may choose to give you a second opportunity to
receive the work electronically in lieu of a refund.  If the second copy
is also defective, you may demand a refund in writing without further
opportunities to fix the problem.

1.F.4.  Except for the limited right of replacement or refund set forth
in paragraph 1.F.3, this work is provided to you ?AS-IS? WITH NO OTHER
WARRANTIES OF ANY KIND, EXPRESS OR IMPLIED, INCLUDING BUT NOT LIMITED TO
WARRANTIES OF MERCHANTIBILITY OR FITNESS FOR ANY PURPOSE.

1.F.5.  Some states do not allow disclaimers of certain implied
warranties or the exclusion or limitation of certain types of damages.
If any disclaimer or limitation set forth in this agreement violates the
law of the state applicable to this agreement, the agreement shall be
interpreted to make the maximum disclaimer or limitation permitted by
the applicable state law.  The invalidity or unenforceability of any
provision of this agreement shall not void the remaining provisions.

1.F.6.  INDEMNITY - You agree to indemnify and hold the Foundation, the
trademark owner, any agent or employee of the Foundation, anyone
providing copies of Project Gutenberg-tm electronic works in accordance
with this agreement, and any volunteers associated with the production,
promotion and distribution of Project Gutenberg-tm electronic works,
harmless from all liability, costs and expenses, including legal fees,
that arise directly or indirectly from any of the following which you do
or cause to occur: (a) distribution of this or any Project Gutenberg-tm
work, (b) alteration, modification, or additions or deletions to any
Project Gutenberg-tm work, and (c) any Defect you cause.


Section  2.  Information about the Mission of Project Gutenberg-tm

Project Gutenberg-tm is synonymous with the free distribution of
electronic works in formats readable by the widest variety of computers
including obsolete, old, middle-aged and new computers.  It exists
because of the efforts of hundreds of volunteers and donations from
people in all walks of life.

Volunteers and financial support to provide volunteers with the
assistance they need, are critical to reaching Project Gutenberg-tm?s
goals and ensuring that the Project Gutenberg-tm collection will
remain freely available for generations to come.  In 2001, the Project
Gutenberg Literary Archive Foundation was created to provide a secure
and permanent future for Project Gutenberg-tm and future generations.
To learn more about the Project Gutenberg Literary Archive Foundation
and how your efforts and donations can help, see Sections 3 and 4
and the Foundation web page at http://www.pglaf.org.


Section 3.  Information about the Project Gutenberg Literary Archive
Foundation

The Project Gutenberg Literary Archive Foundation is a non profit
501(c)(3) educational corporation organized under the laws of the
state of Mississippi and granted tax exempt status by the Internal
Revenue Service.  The Foundation?s EIN or federal tax identification
number is 64-6221541.  Its 501(c)(3) letter is posted at
http://pglaf.org/fundraising.  Contributions to the Project Gutenberg
Literary Archive Foundation are tax deductible to the full extent
permitted by U.S. federal laws and your state?s laws.

The Foundation?s principal office is located at 4557 Melan Dr. S.
Fairbanks, AK, 99712., but its volunteers and employees are scattered
throughout numerous locations.  Its business office is located at
809 North 1500 West, Salt Lake City, UT 84116, (801) 596-1887, email
business@pglaf.org.  Email contact links and up to date contact
information can be found at the Foundation?s web site and official
page at http://pglaf.org

For additional contact information:
     Dr. Gregory B. Newby
     Chief Executive and Director
     gbnewby@pglaf.org


Section 4.  Information about Donations to the Project Gutenberg
Literary Archive Foundation

Project Gutenberg-tm depends upon and cannot survive without wide
spread public support and donations to carry out its mission of
increasing the number of public domain and licensed works that can be
freely distributed in machine readable form accessible by the widest
array of equipment including outdated equipment.  Many small donations
($1 to $5,000) are particularly important to maintaining tax exempt
status with the IRS.

The Foundation is committed to complying with the laws regulating
charities and charitable donations in all 50 states of the United
States.  Compliance requirements are not uniform and it takes a
considerable effort, much paperwork and many fees to meet and keep up
with these requirements.  We do not solicit donations in locations
where we have not received written confirmation of compliance.  To
SEND DONATIONS or determine the status of compliance for any
particular state visit http://pglaf.org

While we cannot and do not solicit contributions from states where we
have not met the solicitation requirements, we know of no prohibition
against accepting unsolicited donations from donors in such states who
approach us with offers to donate.

International donations are gratefully accepted, but we cannot make
any statements concerning tax treatment of donations received from
outside the United States.  U.S. laws alone swamp our small staff.

Please check the Project Gutenberg Web pages for current donation
methods and addresses.  Donations are accepted in a number of other
ways including checks, online payments and credit card donations.
To donate, please visit: http://pglaf.org/donate


Section 5.  General Information About Project Gutenberg-tm electronic
works.

Professor Michael S. Hart is the originator of the Project Gutenberg-tm
concept of a library of electronic works that could be freely shared
with anyone.  For thirty years, he produced and distributed Project
Gutenberg-tm eBooks with only a loose network of volunteer support.


Project Gutenberg-tm eBooks are often created from several printed
editions, all of which are confirmed as Public Domain in the U.S.
unless a copyright notice is included.  Thus, we do not necessarily
keep eBooks in compliance with any particular paper edition.


Most people start at our Web site which has the main PG search facility:

     http://www.gutenberg.org

This Web site includes information about Project Gutenberg-tm,
including how to make donations to the Project Gutenberg Literary
Archive Foundation, how to help produce our new eBooks, and how to
subscribe to our email newsletter to hear about new eBooks.


\end{Verbatim}

\end{document}
