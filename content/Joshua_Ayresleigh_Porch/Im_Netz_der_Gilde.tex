\usepackage[ngerman]{babel}
\usepackage[T1]{fontenc}
\hyphenation{wa-rum Fracht-raum}
\hyphenation{schien}
\hyphenation{Tief-ebe-ne Tief-ebe-ne gro-ßen}


%\setlength{\emergencystretch}{1ex}

\newcommand\bigpar\medskip
\newcommand\schiff\textit
\newcommand\latein\textit

\begin{document}
\raggedbottom
\begin{center}
\textbf{\huge\textsf{Im Netz der Gilde}}

\medskip
Aus den Tagebüchern des \\
Joshua Ayresleigh Porch
\end{center}

\bigskip

\begin{flushleft}
Dieser Text wurde erstmals veröffentlicht in:
\begin{center}
Die Steampunk-Chroniken\\
Geschichten aus dem Æther
\end{center}

\bigskip

Der ganze Band steht unter einer 
\href{http://creativecommons.org/licenses/by-nc-nd/2.0/de/}{Creative-Commons-Lizenz.} \\ 
(CC BY-NC-ND)

\bigskip

Spenden werden auf der 
\href{http://steampunk-chroniken.de/download}{Downloadseite}
des Projekts gerne entgegen genommen. 
\end{flushleft}

\newpage

»Nun macht schon auf, ihr Satansbraten! Meine Füße frieren fest!«
Pünktlich zum sechsten Glockenschlag hämmerte ich gegen die Tür der
Kaufmannsgilde von Nieuw-Delft auf dem Saturnmond Dione, in meine
dicksten Pelze gehüllt und mit klappernden Zähnen. Es war ein
prächtiges Gebäude, eigens zu dem Zweck erbaut, einen einfachen
Kapitän wie mich vor Ehrfurcht auf die Knie sinken zu lassen. Beide
Teile, das Verwaltungsgebäude im Vordergrund und das Lagerhaus
versetzt dahinter, ragten mattrot über vier Stockwerke in den
blassgrünen Himmel hinauf, und jedes einzelne dieser Stockwerke
prunkte mit Zinnen, Rosetten, Türmchen und kleinen Statuen. Die
Front des Lagerhauses zierte eine goldene Uhr, deren Sekundenzeiger
vermutlich mehr wert war als die gesamte Fracht, die in den letzten
zwanzig Jahren durch meine Hände gegangen war, und sie war es auch,
die den Glockenschlag hatte erklingen lassen.

Wer klug war, der stellte sich gut mit der Kaufmannsgilde, denn sie
besaß nicht nur das Geld von Nieuw-Delft, sondern sie ernannte auch
den Magistrat und bezahlte die kleine, effiziente Bürgerwehr, die
jeglichen Ärger von weitem roch und ihn unauffällig vor die
Stadttore beförderte, hinaus in Diones unwirtliche Eiswüsten. Die
Gilde vergab und kontrollierte zudem die Hafenlizenzen, jene
lebenswichtigen Papiere, die mir und meiner Mannschaft einen
lukrativen Auftrag bescheren konnten, vorzugsweise an einem
freundlicheren Ort weit weg von Dione, dem schmutzigen Schneeball;
und deshalb stand ich nun hier und verfluchte meine eigene Klugheit
und Umsicht, um derentwillen ich das beheizte Quartier im
\emph{Ring and Moon} so zeitig verlassen hatte, um nur ja nicht
verspätet zu erscheinen.

Noch einmal hämmerte ich gegen das dunkle Fraxinusholz der
Türflügel. »Lasst mich ein! Der Gildenmeister erwartet mich!«

Eine kleine Klappe schwang auf und hätte mir beinahe die Nase
eingeschlagen. Dahinter erschien das hochnäsige Gesicht eines zu
gut bezahlten Domestiken.

»Der Gildenmeister? Was sollte Mijnheer de Mulder wohl zu so früher
Stunde mit einer Teerjacke wie Euch zu besprechen haben?«

Anstelle einer Antwort zeigte ich den Briefkopf mit dem privaten
Wappen der de Mulders vor, vier schneeweiße Mühlenflügel auf
mattrotem Grund. Dem Domestiken fiel der Kiefer herunter.

»Ihr habt einen Brief von Mijnheer bekommen? Was steht drin?«

»Das«, erwiderte ich mit breitem Grinsen, »wird ein Lakai wie Ihr
von einer Teerjacke wie mir ganz gewiss nicht erfahren. Und jetzt
öffnet die Tür, ehe der Gildenmeister persönlich herunterkommt und
nachsieht, was mich aufhält!«

\bigpar

Unter Dutzenden von Kratzfüßen führte der Mann mich nun endlich
hinein, durch die hohe, düstere Halle, in der nur ein paar
Gaslampen ein mattes Licht gaben, die ausgetretene Stiege hinauf in
die oberen Stockwerke, in denen die Stufen mit immer üppigeren
Teppichen ausgelegt waren, die jeden Schritt dämpften. Endlich, als
ich schon das Gefühl hatte, in dem weichen Flor zu ersticken,
hielten wir vor einer mit reichem Schnitzwerk verzierten Tür, und
der Domestik klopfte an.

»Was gibt es?«, ertönte posaunenhaft eine Stimme von drinnen.

»Hier ist ein \ldots{}« – der Mann ließ einen unsicheren Blick über mich
gleiten – »ein Kapitän der Handelsschifffahrt offensichtlich, der
einen Brief von dero Gnaden bei sich zu führen scheint.«

»Ah – Kapitän Porch! Ihr seid beinahe auf die Minute gekommen.
Tretet ein!«

Der Domestik ließ die Tür weit aufschwingen, und ich folgte der
Aufforderung, nicht ohne ihm ein schadenfrohes Lächeln zu gönnen.
Das verging mir allerdings rasch, als hinter mir die Tür ins
Schloss fiel und ich mich allein im Zentrum der Macht wiederfand.

Der Raum war größer als die Gaststube im \emph{Ring and Moon} und
vollständig in den Farben weiß und rot gehalten. Auf dem dicken,
rahmweißen Teppich stand ein Schreibtisch aus rötlichem
Magondani-Holz, dahinter ein thronartiger Sessel. Die beiden
schlichteren Stühle vor dem Tisch waren mit rotem Samt bezogen.
Auch die Vorhänge vor dem großen Fenster leuchteten rot und
verliehen dem blassen Morgenlicht einen unwirklichen warmen
Schimmer. Zaghaft trat ich näher, bis ich hinter dem Tisch die
riesenhafte, fleischige Gestalt Mijnheer de Mulders ausmachen
konnte, in einem roten Wams mit modischen Zipfeln, den
\emph{Flasques}, einer weißen Kniehose und roten Becherstiefeln,
eine knochenweiße seidene Schärpe um den Leib geschlungen. Er
neigte sich ein wenig nach vorn und reichte mir die bleiche, fette
Hand, die ich sehr behutsam ergriff, um mich nicht an den zahllosen
Ringen zu verletzen.

\bigpar

»Setzt Euch, Kapitän Porch!«, dröhnte er.

Ich nahm auf einem der Stühle Platz.

»Für einen Seemann habt Ihr einen bemerkenswert guten Ruf«, fuhr de
Mulder fort. »Ihr geltet als vertrauenswürdig, gewissenhaft und
verschwiegen. Habt Ihr schon einmal eine Ladung an den Mars
verloren?«

Die Handelsstationen auf dem Mars waren mit denen der Saturnringe
seit Jahrzehnten verfeindet und jagten jedes saturnische
Ætherschiff, das ihren Einflussbereich auf dem Weg zur Erde
passierte. Sie verteilten sogar Kaperbriefe und machten auf diese
Weise die Strecke von Zeit zu Zeit nahezu unpassierbar.

»Noch nie, Euer Gnaden. Aber ich befahre nicht die Route zur Erde,
sondern nur die Binnenlinien innerhalb der Saturn \ldots{}«

Ungeduldig wedelte de Mulder mit der Hand. »Das spielt keine Rolle.
Jedenfalls kennt man Euch dort nicht, es gibt keine Steckbriefe,
keine Beschreibung Eures Schiffes oder dergleichen.«

»Nein, Euer Gnaden.«

»Gut.« Er trommelte mit den Fingern einen schnellen Rhythmus auf
die Tischplatte und zog dann ein mehrfach gefaltetes Pergament aus
der Wamstasche. »Ihr werdet einen Auftrag für die Gilde übernehmen.
Ein Frachtstück soll von Dione aus zur Erde, zum Ætherhafen
Amsterdam, transportiert werden. Hier sind die notwendigen Papiere,
der Hafenmeister wird Euch die Fracht übergeben.«

»Euer Gnaden, ich befahre nur die Binnen \ldots{}«

»Euer Schiff ist aber doch tauglich für den offenen Æther«, ging de
Mulder dazwischen. »Eine Brigg, soviel ich gehört habe. Ihr werdet
heute Vormittag ablegen. Besorgt Euch einen fähigen Navigator, dann
werdet Ihr die Reise schon meistern.«

Er legte das Pergament vor mir auf den Tisch, winkte mit der Hand
zum Zeichen, dass die Unterredung beendet war, und wandte sich
seinen Akten zu. Ich blieb, wo ich war. Nach einer Weile schaute er
auf und zog fragend die Brauen hoch.

»Um was für eine Fracht handelt es sich?«, forschte ich. »Wenn ich
den Auftrag übernehme, muss ich zumindest wissen, was ich
transportiere und wie hoch das Risiko für mein Schiff und meine
Mannschaft ist.«

»Ihr seid mir als umsichtig geschildert worden«, erwiderte de
Mulder, »deswegen bin ich geneigt, Eure Frage nicht als Frechheit
zu werten, sondern als Ausdruck Eurer Fürsorge für das Eigentum der
Gilde, das Euch anvertraut werden wird. Bei dem Frachtstück handelt
es sich um die sterblichen Überreste unseres Gildenmitgliedes
Cornel de Fries. Er verschied vor zwei Monaten, hochbetagt, und
hinterließ der Gilde umfangreiche Ländereien auf Japetus und
Hyperion sowie ein nicht unbeträchtliches Barvermögen. Sein
Testament ist allerdings an die Bedingung geknüpft, dass sein
Körper in der Heimat seiner Vorfahren, also in Amsterdam auf der
Erde, bestattet wird.«

»Unmöglich!«, stieß ich hervor. »Euer Gnaden, diesen Auftrag kann
ich nicht durchführen. Niemand kann das. Ein Toter an Bord bringt
Unglück, die Mannschaft wird meutern. Und selbst wenn man die
Fracht geheim halten könnte – er ist seit zwei Monaten tot! In
dieser Zeit ist er doch schon völlig verfault und verwest. Er wird
stinken. Und er wird eine Seuche auslösen. Wenn er seit zwei
Monaten verpackt in dieser Lagerhalle verwahrt wird, dann ist er
möglicherweise schon geplatzt.«

»Gewiss nicht«, sagte de Mulder. »Wir haben unsere Vorkehrungen
getroffen. Wenn Ihr zukünftig mit der Gilde zusammenarbeiten wollt,
Kapitän Porch, dann solltet Ihr Euch daran gewöhnen, das Denken uns
zu überlassen. Die Fracht wird weder stinken noch platzen, dafür
haben wir Sorge getragen.«

Ich schüttelte den Kopf. »Wer tot ist, verwest. Das immerhin lernt
man auf Ætherreisen.«

»Was man aber offensichtlich nicht lernt, ist Erdgeschichte«,
erklärte de Mulder mit einem Hauch von Ungeduld. »Nun – auf der
Erde war es in sehr alter Zeit üblich, die Körper der Toten zu
konservieren, indem man sie mumifizierte. Die Ägypter
beispielsweise sind so verfahren, und ihre Vorgehensweise haben wir
in diesem Falle übernommen. Um den Zerfallsprozess aufzuhalten,
haben wir die inneren Organe und das Gehirn des Toten entnommen,
sie präpariert und gesondert in gut verschlossenen Krügen verstaut.
Solche Krüge nennt man Kanopen. Anschließend haben wir seinen
Körper für fünfunddreißig Tage in Natron eingelegt, um ihm die
Flüssigkeit zu entziehen, ihn dann in Stoffbinden gewickelt und in
einen Sarkophag gebettet.«

Bei der Vorstellung, so mit einer Leiche zu verfahren, wurde mir
übel, aber ich musste anerkennen, dass diese Methode einen
Transport erlaubte. »Es bleibt also noch der Aberglaube der
Mannschaft«, wandte ich ein. »Sobald bekannt wird, dass ein Toter
an Bord ist, habe ich eine Meuterei am Hals. Meiner Haut kann ich
mich wohl erwehren, aber die Æthermänner könnten auf die Fracht
losgehen. Wer weiß von diesem Transport?«

»Niemand!«, erklärte de Mulder prompt.

Ich lächelte, obwohl mir nicht gerade fröhlich zumute war. »Ein
Niemand ist Euer Gnaden bestimmt nicht«, zählte ich an den Fingern
vor, »ebenso wenig wie die Dienerschaft, die gewiss alles weiß, was
in diesem Hause vor sich geht. Dann sind da noch diejenigen, die
den Toten präpariert und konserviert haben. Sind sie
vertrauenswürdig? Und wie steht es mit denen, die den Sarkophag und
die Kanopen gefertigt haben? Sie werden sich gewiss zusammenreimen
können, wofür diese Werkstücke bestimmt waren. Weiß der
Hafenmeister, worum es sich bei der Fracht handelt?«

De Mulders feistes Gesicht war bei dieser Aufzählung immer länger
geworden. »Nun«, sagte er schließlich, »immerhin erscheint mir die
Möglichkeit gering, dass einer Eurer Æthermänner zu den
Eingeweihten gehört, und nur darauf kommt es ja an. Sobald Ihr die
Fracht in Amsterdam abgeliefert habt, werdet Ihr von unserem
Kontaktmann entlohnt werden.«

»Und wenn ich mich weigere, den Auftrag zu übernehmen?«, fragte
ich.

Der Gildenmeister zuckte die Achseln. »Dann erlöschen mit
sofortiger Wirkung Eure Hafenlizenz und Eure
Aufenthaltserlaubnis.«

Ich brauchte eine Weile, um mir über die Konsequenzen klar zu
werden.

»Das bedeutet, ich dürfte weder in der Stadt bleiben noch sie
verlassen. Aber was \ldots{}?«

»Ich vergaß zu erwähnen«, schloss der Kaufmann mit einem kalten
Lächeln, »dass binnen kurzer Zeit auch Eure Erlaubnis zu atmen
hinfällig würde. Und nun trollt Euch zum Hafenmeister, ich habe zu
arbeiten.«

\bigpar

Als die Tür des Gildenhauses hinter mir ins Schloss fiel, kochte
ich vor Zorn, vor allem über das süffisante Grinsen, das mir der
Lakai nachgesandt hatte. Natürlich hatte er gelauscht, das konnte
ich ihm nicht einmal verdenken. Ich hoffte nur, dass er sein Wissen
zumindest so lange für sich behielt, bis ich einen Navigator für
die interstellare Fahrt gefunden hatte und mit meiner Ætherbrigg,
der \emph{Fancy}, im offenen Weltall war.

Bevor ich den Hafenmeister aufsuchte, ging ich zunächst zum
\emph{Ring and Moon}, um mein Zimmer zu kündigen und einen Diener
mit meinem Gepäck zum Hafen zu schicken, scheuchte dann meine
Mannschaft hoch, indem ich meinem Ersten Maat James Mulligan
anwies, die Brigg zum Ablegen bereit zu machen, ohne freilich die
Art des Auftrags oder der Fracht zu erwähnen, und lenkte
anschließend meine Schritte zum \emph{Alesia}, einer der
Hafenkaschemmen, in denen vor allem Steuerleute und Lotsen
verkehrten.

Die Tür klemmte. Ich warf mich dagegen und trat ein. Heiße,
verräucherte Luft schlug mir entgegen und nahm mir den Atem. Im
Gastraum schien ein Streit im Gange zu sein.

»Das ist eine verdammte Lüge!«, brüllte ein stiernackiger Kerl,
dessen dunkelrotes Gesicht anzeigte, dass er deutlich mehr
getrunken hatte, als ihm gut tat. »Niemand kann so etwas, beim
Phaeton!«

Sein Gegner, einen halben Kopf kleiner als er und mit dem
weißblonden Haar und der kupfergetönten Haut eines befahrenen
Æthermannes, ließ die Provokation mit einem Achselzucken an sich
abperlen. »Es ist mein Æther, und er spielt nach meinen Regeln«,
behauptete er kühn.

Ich wandte mich an meinen Nebenmann, der sich an einem Humpen
Dünnbier festhielt und gespannt von einem Kontrahenten zum anderen
schaute. »Worum geht es?«

»Er hat behauptet, er könne am Ruder einer Galley einen Sonnensturm
abreiten«, gab dieser flüsternd zurück.

»Unmöglich!«, entfuhr es mir. »Niemand kann so etwas!«

»Der da schon! Das ist Adam Hardie, der verdammt beste Steuermann
des Universums.«

»Und wer seid Ihr, dass Ihr darüber urteilen könnt?«

Ohne seinen Blick von dem Schauspiel abzuwenden, reichte er mir die
Hand. »Tomlinson, Schiffsarzt. Ich bin mit Hardie gefahren. Wenn er
am Ruder steht, fragen die Sterne allerhöflichst nach der
Erlaubnis, um die Sonne kreisen zu dürfen.«

\bigpar

Nun hatten die beiden Streithähne meine ungeteilte Aufmerksamkeit,
und ich rückte so nahe wie möglich heran. Hardie schien genau der
Mann zu sein, den ich brauchte, aber anscheinend stand er kurz
davor, sich wegen einer Nichtigkeit den Schädel einschlagen zu
lassen.

»\ldots{} wollen doch mal sehen, ob du mit den Fäusten ebenso schnell
bist wie mit dem Mundwerk!«, brüllte der Stiernacken eben und holte
mit einer ungeschickten Bewegung Schwung.

Weit kam er nicht, denn ich streckte den Arm aus und blockierte
damit sein Handgelenk. Er drehte sich um und starrte mich
überrumpelt an. Aus dem Augenwinkel bemerkte ich, wie einige der
Zecher näher rückten, die Hände zu Fäusten geballt oder mit
Flaschen und Gläsern bewaffnet. Sie hatten auf einen Zweikampf
gehofft und nun, da ihnen dieses Vergnügen zu entgehen drohte,
kochte das Gift in ihren Adern. Andere, die an den Tischen weiter
hinten im Gastraum hockten, wurden auf das bevorstehende Spektakel
aufmerksam und begannen sich durch ermunternde Rufe bemerkbar zu
machen.

Wenn ich meinen zukünftigen Steuermann an einem Stück bekommen
wollte, musste ich wohl oder übel zu seinen Gunsten eingreifen. Die
Axt, meine bevorzugte Waffe, ruhte in meinem Seesack im
\emph{Ring and Moon}, und so hatte ich außer einem zu eher
repräsentativen Zwecken gedachten, zierlichen Auriferrum-Dolch im
Stiefel nichts, was ich den Angreifern entgegensetzen konnte.

»Doktor Tomlinson«, rief ich, während ich an Hardies Seite sprang,
»wollt Ihr für die nächste Reise auf meinem Schiff anheuern? Dann
könntet Ihr schon jetzt Eure Loyalität unter Beweis stellen!«

»Mit Vergnügen!«, erwiderte Tomlinson, stellte den Humpen ab und
krempelte die spitzenbesetzten Manschetten seiner Ärmel auf. »Wenn
Hardie uns begleitet, bin ich mit von der Partie.«

Zu dritt und Rücken an Rücken standen wir im Zentrum eines rasch
enger werdenden Kreises. Bedrohlich wie ein Kraterrand ragte der
Stiernacken vor mir auf. »Hast du deine Kindermädchen
herbeigerufen, du Mondpeiler?«, brüllte er in Hardies Richtung.

»Ich bin auf diese Herrschaften nicht angewiesen«, gab der zurück.
»Mit dir werde ich allein fertig, \emph{teredo galactis}.«

Ich wandte mich zum Doktor, die Augenbraue fragend erhoben.

»Der lateinische Name des Schiffsbohrwurms«, erteilte er mir
Auskunft, während der Stiernacken vor Wut aufbrüllte. Kein Wunder,
denn der Schiffsbohrwurm war nicht nur ein übler Schädling, der
seinen Körper in kürzester Zeit selbst durch die kupferbeschlagene
Beplankung eines Ætherschiffes trieb und es in ein
manövrierunfähiges Sieb verwandelte, er lebte außerdem nur in der
engen Umlaufbahn von Planeten und Monden und kam nicht im offenen
Æther vor.

»Lass dir das nicht gefallen, Stafford!«, hetzte einer seiner
Parteigänger. »Gib ihm eins auf die Nase, dass es nur so kracht!«

Stafford nickte und riss erneut die Fäuste hoch. Bedauerlicherweise
stand er immer noch vor mir anstatt vor Hardie, so dass ich mich
gezwungen sah, seinen Angriff anstelle des Steuermanns zu parieren.
Rasch ergriff ich Tomlinsons abgestellten Bierkrug und schüttete
Stafford die blassgelbe Flüssigkeit in die Augen. Während er wütend
aufschrie und sich die Lider rieb, schickte ich einen Hieb mit dem
Krug hinterher, der ihn am Kinn traf und ihn von den Beinen holte.
Wie ein Sack mit Hafermehl plumpste er vor meine Füße.

Seine Kampfgenossen brüllten auf und stürzten alle gleichzeitig auf
uns zu, verfolgt von denen, die aus unerfindlichen Gründen auf
unserer Seite zu stehen schienen oder auch einfach nur auf eine
Keilerei aus waren. »Kommt nur her!«, rief Hardie und tänzelte
leichtfüßig vor und zurück. »Mit euch nehme ich es auch im Dutzend
auf!«

Ich hörte ein splitterndes Geräusch. Doktor Tomlinson hatte nach
einem der blank gesessenen Hocker gegriffen und ihn gegen den
nächstgelegenen Pfeiler geschmettert. Die Sitzfläche zerbrach, und
er hielt die drei Stuhlbeine in der Hand, die er als Keulen an uns
austeilte. Als müssten wir Korn dreschen, hieben der Doktor und ich
nun auf den aufgebrachten, grölenden Haufen ein, während Hardie,
weitgehend unbehelligt, sein Stuhlbein wie ein Zepter schwang.
»Wenn ihr mir zu nahe kommt«, schwadronierte er, »lasse ich einen
Asteroidenhagel auf euch los! Ich bin der Herr der Sterne, ich kann
die Gezeiten lenken, ihr æthernautischen Schwachköpfe!«

»Doktor«, rief ich und wich mit knapper Not dem Hieb eines
scharfkantigen Flaschenhalses aus, »können Sie den Mann nicht zum
Schweigen bringen?«

»Bedaure, das wird nicht möglich sein!« Tomlinson hielt sich
wacker, wenn auch mittlerweile ein wenig atemlos, gegen zwei
Faustkämpfer. »Er ist gebürtiger Ire und obendrein furchtbar
besoffen.«

»Er stammt von der Erde?« Nun war ich mir erst recht sicher, dass
Hardie mein Mann war, aber es wurde eng für uns. Schon vernahm ich
von der Gasse aus die schrillen Pfeifen der Bürgerwehr, die dem
Kampf ein Ende bereiten und sämtliche Beteiligten in Diones
tödliche Eiswüsten verfrachten würde. »Wir müssen auf der Stelle
durch das Fenster dort hinaus zum Hafen!«, brüllte ich und vollzog
eine halbe Wendung.

Doktor Tomlinson verstand sofort. Er versetzte den beiden
Raufbolden vor ihm einen kräftigen Stoß und rammte uns damit die
Fluchtbahn frei. Hardie indessen, mein designierter Steuermann,
schien nicht gewillt, den Platz zu räumen und fuchtelte weiter mit
dem Stuhlbein herum. »Bei meiner Ehre \ldots{}«

»Die kann warten, einstweilen geht es um Euer Leben!«, erwiderte
ich kurz, packte ihn an der Schulter und zerrte ihn hinter mir her
zu dem hinteren Fenster, das den Hafen und die ankernden
Ætherschiffe überblickte. Tomlinson hatte unterdessen eine Bank in
die Höhe gestemmt und ließ sie durch den Rahmen krachen. Das
Dickicht der aufragenden Masten zerstob in einem bunten
Scherbenregen. Der Doktor hielt sich einen Ärmel vor das Gesicht
und sprang hindurch. Ich versuchte ihm zu folgen, wurde aber durch
Hardie daran gehindert, der ein heroisches »Ich weiche nicht!«
ausrief, sich an meinen Arm klammerte und mich zurück in das
Getümmel der Kämpfenden zu ziehen versuchte.

»Ich weiche für Euch mit!«, erklärte ich, nahm ihm die Keule aus
der Hand, versetzte ihm einen wohlgezielten Hieb über den
starrsinnigen irischen Schädel, der ihm fürs erste die Sinne
raubte, und lud ihn mir auf die Schulter. Dann sprang ich ebenfalls
hinunter in die Gasse, drückte mich in die Schatten und ließ den
Kampf und die Bürgerwehr so schnell wie möglich hinter mir.

\bigpar

»Wohin?«, keuchte Doktor Tomlinson an meiner Seite.

»Wisst Ihr, wo Hardie logiert?«, fragte ich, ohne innezuhalten.
»Wir müssen seine astronautischen Instrumente holen, denn ohne die
kann er nicht arbeiten. Wenn die Zeit reicht, sollten wir auch noch
sein übriges Gepäck mitnehmen, das wird seine Stimmung verbessern,
wenn er im offenen Æther zu sich kommt.«

»Was ist mit meinen Sachen?«, protestierte der Doktor und blieb
abrupt stehen. »Soll ich etwa alles zurücklassen? Wenigstens meinen
Arzneikoffer brauche ich, und auch der mit dem chirurgischen
Besteck wäre von Vorteil, wenn ich Euch und Eure Mannschaft nicht
mit der Ausrüstung der Zimmerleute zusammenflicken soll.«

Zähneknirschend bremste ich ab und wandte mich zu ihm. Hardie
schien auf meinen Schultern bei jedem Schritt schwerer zu werden,
und wir waren noch lange nicht in Sicherheit, denn die Pfeifen der
Bürgerwehr klangen bedrohlich nahe und schienen im Gewirr der
Gassen aus allen Richtungen zu kommen.

»Nun gut«, entschied ich schließlich, »holt Euer Gepäck, so schnell
es geht, und meldet Euch an Bord der \emph{Fancy}. Sollte Euch auf
dem Weg jemand aufhalten, dann erklärt ihm, dass Ihr für Kapitän
Joshua Porch arbeitet und Mijnheer de Mulder persönlich sein
Auftraggeber ist. Na los, nun lauft schon! Wollt Ihr Euch wegen
Unruhestifterei vor den Toren von Nieuw-Delft wiederfinden?«

Aber Tomlinson stand wie festgefroren und starrte mich an, aschfahl
im grellen Licht des Vormittags, die Augen schwarz wie Kohlen. »De
Mulder! Ihr seid ein Handlanger der Gilde! In diesem Falle, werter
Herr, trennen sich hier unsere Wege. Macht Eure Fahrt allein, lasst
mich hier und setzt auch Hardie ab. Lieber wollen wir im Eis Diones
verrecken als Euch zu begleiten. Warum habt Ihr mir nicht gleich
gesagt, dass Ihr für die Gilde arbeitet? Dann hätte ich mir an
unserem Handschlag nicht die Finger beschmutzt.«

Mühsam unterdrückte ich einen Fluch. Für derartige
Grundsatzdiskussionen fehlte uns eindeutig die Zeit. Andererseits
kannte ich genügend Menschen vom Schlag des Doktors, die ihre
Prinzipientreue jederzeit nicht nur über das eigene Leben, sondern
auch bedenkenlos über das Leben anderer stellen würden, deswegen
war ich mir sicher, dass er meinen Steuermann und Navigator mit
sich in den Untergang reißen würde, wenn es mir nicht gelänge, ihn
von meiner Integrität zu überzeugen. Ich entsann mich des Gesprächs
mit de Mulder, seiner Ausflüchte und seiner Drohungen, ich dachte
auch an meinen Widerwillen, was die Fracht anging. Aber stets kam
mir auch die verlockende Entlohnung in den Sinn, die de Mulder mir
in Amsterdam in Aussicht gestellt hatte. Erst als mir die
schadenfrohe Visage des Domestiken wieder vor Augen stand, fielen
mir die passenden Worte ein.

»Es ist nicht so, wie Ihr meint«, sagte ich bitter. »Ich werde von
der Gilde erpresst. Wenn es nur um mich selbst ginge, dann wäre
meine Entscheidung klar. Aber ich bin für meine Mannschaft
verantwortlich, und auch für ihre Familien.«

Doktor Tomlinson zögerte noch immer. Ich hörte die Pfeifen immer
näher kommen und spannte meine Muskeln, bereit, den Doktor und den
Navigator sich selbst zu überlassen, um die Gassen hinunter zum
Hafen zu springen und mein eigenes Leben zu retten. Man möge mir
verzeihen, ich bin nun einmal nicht zum Märtyrer geboren und sehe
deswegen auch keinen Gewinn darin, zu einem zu werden. Aber gerade
als meine Füße sich beinahe aus eigener Kraft vom Pflaster
losreißen wollten, kam Tomlinson zur Besinnung.

»Erpresst also!«, stieß er hervor. »Ja, das hätte ich mir denken
müssen, dass ein Ætherkapitän wie Ihr nicht aus freien Stücken den
Machenschaften der Gilde dient. Also gut, ich habe bei Euch
angeheuert und stehe zu meinem Wort. Bringt Hardie an Bord Eures
Schiffes. Ich werde mein eigenes Gepäck holen, meinen Diener nach
seiner Ausrüstung senden und mich so schnell wie möglich mit Euch
am Hafen treffen. Habt Ihr Geld bei Euch? Ich denke nicht, dass
Hardie seine Unterkunft im Voraus bezahlt haben wird, und meine
eigenen Taschen sind bedauerlicherweise leer.«

Ich reichte zögernd meine schmale Börse herüber. Tomlinson ergriff
sie und eilte durch die engen Gassen davon. Hardie schien
mittlerweile eine Tonne zu wiegen. Seufzend rückte ich ihn auf
meinen Schultern zurecht, um ihn auf schnellstem Weg zur
\emph{Fancy} zu schaffen.

Die Truppen der Bürgerwehr zogen inzwischen weiter, ihre Pfeifen
entfernten sich und wurden leiser. Sicher hatten sie im
\emph{Alesia} reichliche Beute gemacht, wie immer in der Form von
armen Teufeln, die verzweifelt ihre letzten Silberstücke hergaben,
um verschont zu werden, und dann doch ohne jedes Erbarmen vor die
Stadttore geworfen wurden. Die Gilde duldete keine Hungerleider in
ihren Mauern.

\bigpar

Ich schlüpfte unter den Vorsprüngen der verwinkelten Häuser
hindurch, drückte mich in Eingänge und spähte um Ecken herum. Auch
wenn ich mich auf den Auftrag der Gilde berufen konnte, war ich
doch nicht unantastbar. Kapitäne wie mich gab es wie Sand im
Hafenbecken, auch Ætherbriggs konnte man zur Genüge finden, und ich
war nicht sicher, ob ich auf de Mulder tatsächlich einen
vertrauenswürdigen Eindruck gemacht hatte. Was, wenn er mir seine
kurzlebige Gunst inzwischen schon wieder entzogen hatte?

Keuchend, verschwitzt und atemlos unter meiner Last erreichte ich
schließlich den Ætherhafen. Hier endete die Glaskuppel, die
Niew-Delft bedeckte und für eine künstliche Atmosphäre und
erträgliche Temperaturen sorgte. Der Himmel über mir war nicht mehr
blassgrün, sondern von eisiger Klarheit. Die Schiffe ankerten dicht
an dicht, die Masten mit Eis glasiert, die Planken von Raureif
bedeckt. Scharen von Hafenarbeitern in ihren winzigen Nussschalen
wimmelten im Wasser herum und schlugen mit Spitzhacken darauf ein,
um es am Zufrieren zu hindern.

»Atemmasken in allen Größen! Sir, kaufen Sie eine Atemmaske!« Ein
fliegender Händler zupfte mich am Ärmel. Ich drehte mich zu ihm um.
Er war kaum älter als ein Knabe, an seinem Gürtel hingen ein halbes
Dutzend lederner Masken mit langen, rüsselartigen Mundstücken.
»Eine Atemmaske für die Hafenrundfahrt, Sir?«

Ich schüttelte ihn ab. »Verschwinde! Ich habe meine eigene
Atemmaske an Bord meiner Brigg.«

»Und was ist mit Ihrem Freund, Sir?« Mit einem verschlagenen
Lächeln deutete der Junge auf Hardie. »Oder schmeißen Sie den ins
Hafenbecken, Sir? Weiß die Bürgerwehr davon?« In der Erwartung
eines saftigen Schmiergeldes streckte er mir die Hand entgegen.

»Ich bin im Auftrag der Gilde unterwegs!«, donnerte ich. »Wo ist
deine Handelslizenz?«

Er wurde blass. »Sir, ich \ldots{}«

»Die Atemmasken sind beschlagnahmt – Diebesgut, nicht wahr? Wie
viele von ihnen sind beschädigt und hätten brave Bürger von
Niew-Delft getötet? Ich werde das überprüfen.« Mit einem raschen
Griff nahm ich ihm den Gürtel ab. »Jetzt lauf, sonst muss ich dich
festnehmen und zu Mijnheer de Mulder bringen.«

Der Junge taumelte ein paar Schritte rückwärts, drehte sich dann um
und verschwand mit ein paar Sprüngen im Gewirr der Gassen.

Während ich Hardie noch immer auf der Schulter balancierte, prüfte
ich im Weitergehen meinen Fang. Drei der Masken waren erkennbar
defekt, ihre Mundstücke waren verbogen, die Filter verstopft. Die
anderen schienen brauchbar zu sein. Eine hatte etwa die richtige
Größe für das Gesicht des Doktors, eine andere passte
möglicherweise über Hardies irischen Dickschädel. Gut gelaunt
hängte ich mir den Gürtel über. Den Verlust hatte sich der Junge
selbst zuzuschreiben. Weshalb hatte er auch versucht, mich zu
erpressen?

\bigpar

Endlich erreichte ich den Pier, an dem die \emph{Fancy} vor Anker
lag. Um sie herum herrschte ein wildes Getümmel, als sei sie die
Königin in einem Bienenstock. Mulligan hatte ganze Arbeit geleistet
und jeden verfügbaren Hafenarbeiter aufgescheucht. Pinassen
brachten die Vorräte an Bord, große Fässer mit getautem
Dione-Schnee, kleinere mit dem dunkelbraunen, scharfen Rum von
Pasiphae, gepökeltem Fleisch, blassblauen getrockneten
Amalthea-Erbsen, Käse, Zwieback und Hafermehl. Die zähen,
violettgefiederten Tethis-Hühner wurden in ihren Käfigen an Bord
gehievt, meckernde, gescheckte Ziegen auf Bootsmannssitzen empor
gezerrt und unter Deck zu den riesigen Sumpfschildkröten von
Epimetheus gebracht. Der Kessel war schon vorgeheizt, aus dem
hohen, schlanken Schornstein mittschiffs stieg kerzengerade eine
feine Rauchfahne empor. Auch an Deck herrschte aufgeregter Betrieb.
Die Männer überprüften die Ausrüstung, schossen die Taue noch
einmal auf oder verstauten hastig ihre Æthersäcke und die kleinen
Kisten mit den Waren, die sie auf eigene Rechnung zu verkaufen
hofften.

Als Jasper, der Bootsmann, mich auf dem Steg bemerkte, gab er ein
Pfeifensignal und baute sich salutierend an der Reling auf.
»Willkommen an Bord, Sir, Kapitän Porch, Sir! Um vier Glasen der
zweiten Tagwache sind wir klar zum Ablegen! Was ist denn das da –
mit Verlaub, Sir?« Mit der langen Bootsmannspfeife wies er auf
Hardie, der nach wie vor über meiner Schulter hing.

»Das ist unser neuer Navigator, Mister Adam Hardie«, gab ich zur
Antwort. »Er weiß es allerdings noch nicht. Sein Gepäck wird gleich
nachkommen, zusammen mit unserem neuen Schiffsarzt Doktor
Tomlinson. Mulligan soll den beiden ein Quartier herrichten lassen.
Einstweilen helfen Sie mir doch, Hardie ins Galion zu
verfrachten.«

Jasper grinste von einem Ohr zum anderen. »In Ketten, Sir, wegen
Trunkenheit im Dienst?«

»Aber nicht doch, Mann!« Missbilligend schüttelte ich den Kopf.
»Als er sich betrunken hat, wusste er ja noch nichts von seinem
neuen Posten. Ich will nur, dass er ausnüchtert, ohne sich unter
Deck zu übergeben und ein Tohuwabohu zu verursachen. Wir werden ihn
sichern müssen, aber nur mit Tauen, und nicht zu straff.«

Einmal mehr hob Jasper die Hand an die Stirn, seine Augen blitzten
vor unterdrücktem Vergnügen. »Aye, aye, Sir! Ich werde ihn so sanft
einwickeln wie ein Baby!«

\bigpar

Endlich konnte ich den streitbaren Iren von meinen Schultern laden.
Mit Jaspers Hilfe schleppte ich ihn zum Bug der \emph{Fancy} und
ließ ihn sacht in das Tauwerk unter dem Bugspriet gleiten. Jasper
sprang hinunter und band meinen zukünftigen Navigator gewissenhaft
fest. Anschließend stülpte er ihm die Atemmaske über, die
glücklicherweise passte, und kletterte geschickt wie ein Affe
zurück an Deck.

»Sir«, meldete er, »der wird im Æther sein blaues Wunder erleben!«

»Wenn er erwacht, Jasper«, erwiderte ich kühl, »dann ist er einer
Ihrer Vorgesetzten. Sie werden ihm in allen Belangen gehorchen. Ist
das klar?«

Jasper sackte in sich zusammen, als sei er ein Sphärenballon im
Vakuum. »Aye, aye, Sir!«, rief er und trollte sich unter Deck, um
das Verstauen des zusätzlichen Sonnensegeltuchs zu überwachen.

Inzwischen war Mulligan eingetroffen, der hoch aufgeschossene,
hagere erste Maat mit den beinahe durchsichtigen Augen. Er
salutierte vorschriftsmäßig und hob dann fragend die Brauen. »Ein
sehr kurzfristiger Aufbruch, Kapitän Porch, Sir.«

»Sonderauftrag der Gilde«, seufzte ich. »Aber wir sollten froh
darüber sein. Oder hatten Sie vor, den Winter auf Dione zu
verbringen?«

»Gewiss nicht, Sir. Wozu benötigen wir einen Navigator?«

»Der Sonderauftrag wird uns ein wenig weiter verschlagen als
gewohnt. Wir werden Kurs auf die Erde nehmen.«

»Die Erde?«, wiederholte er verblüfft. »Aber Sir, die \emph{Fancy}
befährt nur die Binnenlinien innerhalb der Saturnringe! Und wir
haben außer ein paar Coulombgewehren keinerlei Bewaffnung. Wie
sollen wir an den Marskontrollen vorbeikommen?«

»Mit Höflichkeit, Mulligan«, erwiderte ich. »Etwas Besseres fällt
mir jedenfalls nicht ein. Immerhin werden wir einen Schiffsarzt an
Bord haben – ich habe ihn sozusagen als Gratisbeigabe zum Navigator
bekommen. Ah – da vorn ist er ja!«

Im Laufschritt näherte sich Doktor Tomlinson dem Pier, beladen mit
hölzernen Kästen und hastig zusammengerollten Ætherkarten. Hinter
ihm drein stolperte sein Diener, der eine Reisetasche und ein
Medizinschränkchen schleppte. Als der Doktor uns sah, winkte er
hektisch und machte noch längere Schritte.

Ich schickte den beiden zwei Æthermänner entgegen, die ihnen einen
Teil des Gepäcks abnahmen und ihnen an Bord halfen. Kurz darauf
stand Tomlinson schwer atmend vor mir. »Wann können wir ablegen?«,
keuchte er. »Ich glaube, mir ist die halbe Stadt auf den Fersen!«

»Du lieber Himmel!«, rief ich. »Die Rauferei war doch längst
vorüber! Was haben Sie denn nur in dieser knappen halben Stunde
angestellt?«

»Bezahlt«, erwiderte er, »oder vielmehr: nicht ausreichend bezahlt.
Der Inhalt Ihrer Börse reichte bedauerlicherweise nicht für beide
Unterkünfte. Hardies Wirt wollte die Karten und die
Navigationsinstrumente als Pfand zurückbehalten, da musste ich ihn
aus dem Weg schieben und ihn bezüglich des noch ausstehenden Geldes
auf unsere Rückkehr vertrösten. Sobald er wieder auf die Beine
gekommen war, alarmierte er die Bürgerwehr, die allerdings zu
meinem und unserem Glück noch mit den Raufbolden aus dem
\emph{Alesia} beschäftigt war, sodass Hopkins und ich uns einen
gewissen Vorsprung verschaffen konnten. Dennoch sollten wir meiner
Meinung nach so bald wie möglich in den Æther aufsteigen.«

Ich blickte zu Mulligan. »Die Fracht ist verstaut«, meldete der,
»ebenso wie die Vorräte. Der Kessel ist auf Starttemperatur. Wir
können ablegen, sobald Sie befehlen.«

Diesen Moment ließ ich mir nicht nehmen. Ich trat auf das
Achterdeck.

»Klar zum Hieven des Ankers!«, rief ich schallend über die Decks.

Der Bootsmann Jasper gab meinen Befehl weiter. Vorn am Bug steckten
acht Æthermänner lange hölzerne Stangen, die sogenannten Spaken, in
das Spill und schoben sie nach Kräften an, um den Anker zu bergen
und ihn raumfest zu vertäuen.

»Setzt die Segel!«

Von vorn nach achtern wurde die Segelfläche gesetzt, präzise auf
einander abgestimmt. Jeder der Toppgasten oben auf den Rahen
reagierte auf den Mann vor sich, jeder der Æthermänner darunter auf
den Mann über sich, in einem eingespielten, perfekten Rhythmus, der
die \emph{Fancy} in eine beinahe tänzerische Bewegung versetzte.
Ohne auszubrechen, zu »gieren«, legte sie in elegantem Schwung ab,
verließ das Hafenbecken und nahm immer mehr Fahrt auf.

Der Kessel unter mir stampfte beharrlich und gleichmäßig wie ein
Herzschlag. »Maschinen auf volle Kraft!«, brüllte ich. Noch während
mein Kommando weitergegeben wurde, nahm das Dröhnen aus dem
Maschinendeck zu. Aus dem schmalen Schornstein löste sich ein
leuchtender Funkenregen der aufgrund der künstlichen Atmosphäre
Diones in alle Regenbogenfarben zerstob. Erst leise rauschend, dann
schneller und lauter, lief der Auriferrum-Propeller an, ohne dessen
zusätzliche Kraft sich kein Schiff aus dem Wasser in den Æther
erheben konnte, und trug die \emph{Fancy} hinauf in den Himmel.

\bigpar

»Kurs ringauswärts!«, kommandierte ich und übersah geflissentlich
die ungläubigen Blicke der Mannschaft, die von der Fahrt Richtung
Erde noch nicht unterrichtet worden war. Natürlich war ihnen
aufgefallen, dass wir nur wenig Fracht und dafür umso mehr Vorräte
geladen hatten, dennoch hatten sie wohl eher an eine Reise entlang
der Binnenlinien geglaubt, wie wir sie häufiger unternahmen. Ich
zuckte die Achseln. Sobald wir weit genug draußen im offenen Æther
waren, um den Gedanken an eine Umkehr unmöglich zu machen, mochte
Mulligan sie über unser Ziel unterrichten. Bis dahin waren mir die
abenteuerlichsten Gerüchte an Bord lieber als die Wahrheit, die
vermutlich eine Panik auslösen würde.

Mulligan neben mir hatte meine Gedanken gelesen und wiegte
zweifelnd den Kopf hin und her. »Was für ein Auftrag ist das,
Sir?«, fragte er. »Was für eine Fracht transportieren wir für die
Gilde?«

»Nichts, das Sie etwas anginge«, erwiderte ich viel zu schroff, und
fuhr gleich darauf besänftigend fort: »Je weniger Menschen von der
Art dieser Fracht wissen, umso besser ist es für alle Beteiligten –
glauben Sie mir.«

»Sir, nichts für ungut«, – Mulligan wand sich unbehaglich – »aber
sollte nicht wenigstens Ihr Erster Maat wissen, um was für eine
Fracht es sich handelt? Schon allein, damit ich den Gerüchten
Einhalt gebieten könnte, wenn die Rede auf die illegalen Geschäfte
der Gilde kommt.«

»Ihr Kapitän wird erpresst«, warf Tomlinson ungefragt ein. »Ihm
sind die Hände und die Zunge gebunden, und das ist wohl der Grund
für sein Stillschweigen. Am Besten wird es also sein, wenn wir
nicht weiter in ihn dringen und sowohl die Mannschaft als auch die
Fracht so gut wie möglich schützen. Immerhin müssen wir an den
Marsstationen vorbei.«

Ich wünschte, ich hätte ihm beizeiten das Maul stopfen können, denn
bei der Erwähnung der Marsstationen wurde nicht nur Mulligan
hellhörig, sondern auch mein Zweiter Maat Wilkins, der dem Gespräch
mit halbem Ohr gefolgt war. Er war beinahe noch ein Knabe, auch
wenn er seit seinem siebenten Geburtstag zur See gefahren war und
sicherlich weit mehr gesehen hatte, als er in seinem jugendlichen
Alter verstehen konnte.

»Geht es gegen die Marsianer, Sir?«, fragte er atemlos. »Hätte ich
das nur eher gewusst, dann hätte ich meine neue Coulomb-Pistole
mitgenommen. Diesen miesen Ratten werden wir es zeigen, nicht wahr,
Sir?«

Ich beugte mich ein Stück hinunter, um ihm in die Augen sehen zu
können. Gerade verloren sie ihr jugendliches Blau und nahmen die
eisige Farbe des Æthers an. Die Kehle wurde mir eng, aber es gelang
mir, meinen schroffen Tonfall beizubehalten. »Sie, Wilkins«,
donnerte ich, »werden sich im Falle eines Angriffs der Marsianer
unter Deck halten, verstanden? – Keine Widerrede, das ist ein
Befehl, und er wird die ganze Fahrt über bestehen bleiben. Wenn es
zu einem Kampf gegen die Marsianer kommt, werden Sie allein für die
Sicherheit der Fracht gerade stehen müssen. Trauen Sie sich das
zu?«

Wilkins salutierte mit Tränen in den Augen. »Ja, Sir, Kapitän
Porch, Sir!«

\bigpar

In den nächsten Stunden verließen wir die vertrauten Saturnringe.
Noch immer vernahm ich keinen Laut aus dem Galion, in dem mein
Navigator selig unter der Atemmaske schlief. Erst als wir Rhea
passierten und ich mich bereit machte, mit den beiden Maaten und
dem Doktor das Abendessen einzunehmen, klopfte Jasper an meine
Kajütentür.

»Sir«, meldete er salutierend, »der Navigator, den Sie da haben –
der ist mittlerweile aufgewacht, und er schwört Stein und Bein,
dass er jeden, der an seiner Entführung beteiligt war, zu Mus
stampfen wird.«

»Danke, Jasper«, erwiderte ich abwesend. »Ich werde mich nach dem
Essen darum kümmern.«

Aber Jasper hielt seine Stellung. »Sir«, insistierte er, »der Mann
macht Ernst. Er hat seine Æthermaske abgesetzt und versucht, damit
das Galion zu zertrümmern.«

»Bei allen Marsmonden!« Fluchend drängte ich mich an ihm vorbei und
eilte voraus.

Ich hörte Hardies Stimme über das ganze Deck. »Lasst mir hier raus,
ihr feigen Mondmaarfischer! Tretet der Reihe nach gegen mich an,
wenn ihr Männer seid! Dann werfe ich euch allesamt über Bord und
bringe dieses ausgemusterte Teesieb von einer Brigg allein zurück
in den Hafen!«

»Wenn Ihr weiterhin so tobt, lasse ich Eure Arme in Ketten legen!«,
rief ich ungerührt von oben hinunter.

»Dann erledige ich Euch mit den Füßen, Asteroidenkrake!«,
erschallte es prompt zur Antwort.

Ich schüttelte vorwurfsvoll den Kopf. »Mister Hardie, redet man so
mit einem Mann, dem man das Leben verdankt? Ihr erinnert Euch
vielleicht nicht, aber ich habe Euch aus dem \emph{Alesia}
gerettet, gegen die Übermacht, mit der Ihr Euch angelegt hattet.«

»Mit denen wäre ich auch allein fertig geworden!«

»Nein, das wäret Ihr nicht!«, widersprach ich fest. »Die Bürgerwehr
war schon vor der Tür. Hätte ich nicht eingegriffen, dann wäret Ihr
fröstelnd vor den Toren von Niew-Delft aufgewacht anstatt in
relativer Behaglichkeit im Galion meiner \emph{Fancy}.«

Er bedachte sich eine Weile. Schließlich zuckte er die Achseln. »Es
wäre mir lieb, wenn Ihr diese relative Behaglichkeit in eine
absolute verwandeln könntet, indem Ihr mich an Deck hieven würdet.
Wenn ich allerdings als Navigator für Euch tätig werden soll,
brauche ich meine Instrumente.«

»Die stehen Euch zur Verfügung, ebenso wie Euer restliches Gepäck.
Doktor Tomlinson war so umsichtig, daran zu denken.«

»Tomlinson ist auch an Bord? Nun gut – dann bin ich wenigstens
nicht als Einziger shanghait worden und habe in meinem Unglück
einen Kameraden. Hievt mich hoch, Kapitän! Ich unterwerfe mich
Eurer Befehlsgewalt.«

Einmal an Deck, schien Hardie die misslichen Umstände seines
Anheuerns vollständig vergessen zu haben. Er erwies sich als ein
umsichtiger Astrallotse und Navigator, aber auch als angenehmer
Gesellschafter, und unterhielt bei den Mahlzeiten in der
Kapitänskajüte Mulligan, Tomlinson und mich, vor allem aber den
jungen Wilkins mit dem wildesten Æthergarn. Der Junge hing an
seinen Lippen und war völlig verzaubert, ein Umstand, der mich mit
Sorge erfüllte, den ich aber nicht ändern konnte.

\bigpar

So passierten wir ereignislos Titan und Japetus und schlugen den
Kurs Richtung Jupiter ein.

\bigpar

Auf der Höhe des äußersten Jupitermondes Kallisto schlug die
Stimmung an Bord um. Eine zunehmende Widerspenstigkeit lag in der
Luft, die Zeichen standen auf Meuterei, ohne dass ich begreifen
konnte, weshalb. Wir hatten guten, stetigen Sonnenwind vor dem wir
kreuzen konnten, die \emph{Fancy} machte dank der Navigationskünste
Hardies bessere Fahrt denn je. Die Vorräte würden ausreichen,
sodass ich die Rationen nicht kürzen musste, und da die Mannschaft
daher weder übermüdet noch entkräftet war, blieben wir auch von
Unfällen verschont.

Dennoch war eine seltsame Unruhe eingezogen. Die Gespräche der
Mannschaft verstummten, wenn ich an Deck ging, ihre Blicke folgten
mir und klebten an meinem Rücken. Peter Flynt, der abergläubische
Schiffszimmermann, hielt den Atem an und machte ein kompliziertes
Handzeichen, wenn er mir unvermutet begegnete, und sogar der
altgediente Rogers, der seine Hosenbeine teerte, weil ein
Ætheranzug ihm zu neumodisch erschien, zuckte zusammen, sobald er
meiner angesichtig wurde.

Es hätte keinen Sinn gehabt, die Männer direkt auf ihr Benehmen
anzusprechen, also gewöhnte ich mir einen leisen Tritt an und
versuchte ihre Gespräche zu überhören, bis ich mir ein vages Bild
machen konnte.

Als mir schließlich die Zusammenhänge klar wurden, stockte mir der
Atem. Ich ließ Mulligan und Doktor Tomlinson in meine Kajüte
bitten, besprach mich mit ihnen, holte den jungen Wilkins hinzu und
brachte ihm schonend, aber unmissverständlich die Sachlage bei, und
schickte schließlich Jasper, der als einziger der Mannschaft seinen
klaren Kopf behalten hatte, um unseren Navigator Adam Hardie
herbeizuschaffen.

»In Ketten, Sir?«, scherzte Jasper.

Ich überdachte diesen Vorschlag. »Was meinen Sie, Jasper«, fragte
ich, »wie viele von der Mannschaft wären im Falle eines
Kurswechsels bereit, sich Hardie anzuschließen?«

Dem Bootsmann fiel der Kinnladen herunter. »Sir, ich \ldots{}«

»Sie, Jasper, würden bis zum bitteren Ende auf meiner Seite stehen,
dessen bin ich sicher«, erklärte ich im Brustton der Überzeugung
und hoffte, dass er mir diese faustdicke Lüge abnahm. Bis zum
bitteren Ende würde nicht einmal ich selbst auf meiner Seite
stehen, da mir an meinem Leben stets mehr gelegen war als an meinen
Prinzipien. »Sie haben Mister Hardie von Anfang an durchschaut, und
Sie haben den Worten Ihres Kapitäns mehr getraut als den Gerüchten,
die dieser dahergelaufene Ire ausstreut. Aber auf wen außer Ihnen
kann ich zählen?«

Jasper straffte die Schultern, salutierte, dachte nach und ließ den
Arm wieder sinken. »Also, dann wohl besser nicht in Ketten, Sir.
Ich werde Mister Hardie aufsuchen und ihn bitten, sich in der
Kapitänskajüte einzufinden, Sir.«

Er wieselte in Richtung Achterdeck, überzeugt davon, dass er sich
sauber aus der Affäre gezogen hatte. Kurz darauf kam er mit Hardie
im Schlepptau wieder zurück.

\bigpar

Der Navigator lächelte breit, als er mich erblickte. »Kapitän
Porch! Es ist noch nicht Essenszeit. Was verschafft mir also die
Ehre Ihrer Gegenwart?«

»Hardie, Sie sind der Meuterei angeklagt«, erwiderte ich leise und
ohne mit der Wimper zu zucken. »Ich werde Sie vierteilen, Ihre
Gliedmaßen an die Rahen des Hauptmastes nageln, sie anschließend
verbrennen und Ihre Überreste dem Sonnenwind überantworten. Folgen
Sie mir unter Deck, dort werden die Offiziere der \emph{Fancy} über
Sie zu Gericht sitzen.«

Er blieb noch immer vollkommen unbesorgt, denn Doktor Tomlinson war
auf seiner letzten Fahrt in seiner Messe gewesen, und mit Wilkins
glaubte er wegen dessen Jugend leichtes Spiel zu haben, deswegen
folgte er mir in die Kapitänskajüte. Als er sich dort den ernsten
Mienen gegenüber sah, schien er sich zwar anders zu besinnen, aber
nun war an eine Flucht nicht mehr zu denken.

»Mister Hardie«, begann Mulligan, »da Kapitän Porch selbst die
Anklage gegen Sie erhebt, werde ich diesem Gericht vorsitzen.
Möchten Sie sich selbst verteidigen oder wünschen Sie, dass wir
einen Verteidiger für Sie bestimmen?«

Verunsichert blickte Hardie umher, in die erschreckten Augen
Wilkins’ und die versteinerte Miene Tomlinsons. Er räusperte sich.
»Da das Urteil dieser ehrenwerten Versammlung schon festzustehen
scheint, ziehe ich es vor, mich selbst zu verteidigen.«

Mulligan zuckte gleichgültig die Achseln. »Kapitän Porch, wollen
Sie bitte in Gegenwart des Angeklagten wiederholen, was Sie uns
vorhin mitgeteilt haben?«

»Der hier anwesende Adam Hardie hat auf meinem Schiff als Navigator
angeheuert \ldots{}«

»Was heißt hier angeheuert?«, unterbrach der Ire scharf. »Ich wurde
shanghait!«

»Als ich Sie aus dem Galion zog, haben Sie sich meiner
Befehlsgewalt unterworfen.«

»Unter Zwang! Sie hätten mich sonst dort unten verrecken lassen!«

Mulligan klopfte auf den Tisch. »Angeklagter, lassen Sie den
Kapitän ausreden. Sie werden später noch Gelegenheit haben, sich zu
rechtfertigen.«

»Adam Hardie«, fuhr ich fort, »hat seine Vertrauensstellung als
Navigator missbraucht, um gegenüber der Mannschaft Gerüchte
auszustreuen, die geeignet waren, die Moral zu zerstören und eine
Meuterei herbeizuführen. Im Einzelnen sind dies die Gerüchte, die
\emph{Fancy} sei ein Totenschiff \ldots{}«

»Aber es ist doch wahr! Unter Deck liegt ein Toter! Es war
Tagesgespräch im \emph{Alesia}, dass die Gilde eine Leiche zur Erde
transportieren wollte!«

» \ldots{} des weiteren, alle, die sich längere Zeit unter Deck
aufgehalten hätten, könnten bereits infiziert sein \ldots{}«

»Wer tot ist, verwest. Und dabei entsteht ein Gift, das alle
anderen, die damit in Berührung kommen, mit in den Untergang reißt.
Jeder Æthermann weiß das!«

» \ldots{} und schließlich, die Offiziere hätten von dieser tödlichen
Bedrohung Kenntnis und hätten sie bewusst in Kauf genommen, um der
Mannschaft keine Heuer zahlen zu müssen.«

»Das war ein Scherz. Ich \ldots{} ich dachte nicht, dass irgendjemand
diese Worte wirklich glaubt. Ich wollte Unruhe stiften, nicht mehr.
Zum Teufel, ich war wütend über die Art und Weise, wie über mich
verfügt worden ist. Ich bin doch kein Seesack, den man an Bord
schafft und ins Galion wirft. Und wenn ich als Navigator angeheuert
werde, habe ich ein Recht darauf, die wesentlichen Fakten über
Fracht und Zielort zu erfahren. Ich wollte \ldots{} Kapitän Porch, bitte
glauben Sie mir: Ich wollte, dass die Mannschaft aufmerksam wird
und Sie zur Rede stellt. Aber eine Meuterei hatte ich nicht im
Sinn.«

Unversehens war ich in die Defensive geraten. Mulligans beinahe
durchsichtige Augen betrachteten mich forschend, der junge Wilkins
bebte beinahe vor Furcht, und Doktor Tomlinson schüttelte
vorwurfsvoll den Kopf.

»Ein Toter an Bord \ldots{} Bei allen Göttern, haben Sie denn die
Seuchengefahr nicht bedacht? Ganz gleich, womit die Gilde Sie
erpresst hat – «

»Es besteht keine Seuchengefahr«, erwiderte ich knapp. »Die Leiche
ist konserviert und verpackt. Dieser Transport ist sicher.«

»Warum wurde die Art der Fracht dann vor der Mannschaft geheim
gehalten?«, fragte Mulligan.

»Eben wegen dieses verfluchten Aberglaubens! Hätte ich gesagt, dass
wir einen Toten transportieren, dann wäre doch jeder zerbrochene
Eimer, jedes gescheuerte Tau und überhaupt jede noch so winzige
Unannehmlichkeit darauf geschoben worden, dass ein Fluch auf diesem
Schiff liegt. Bei allen Heliaden, Sie wissen doch selbst, wie die
Ætherleute sind. Die Fahrten sind nun einmal gefährlich, die
Sonnenwinde unberechenbar, da klammert man sich eben an Talismane
und Vorzeichen und Omen aller Art. Wir transportieren einen Toten –
na und? Seine Organe stecken in gut verkorkten Krügen, der
restliche Körper ist zu Dörrfleisch verarbeitet und eingepackt. Es
besteht keinerlei Gefahr, dass er aus seinem Sarg steigt und unter
Deck herumwandert. Wer sich ihm nähert, kann sich ebenso wenig
infizieren wie Woodes, der Smutje, wenn er einen Schinken
anschneidet. Aber nun, nach diesen wilden Gerüchten, wird es
vermutlich schwer sein, die Mannschaft davon zu überzeugen. Ich
habe die Verantwortung für diesen Transport, Mister Hardie. Ich
kann es mir weder leisten, meine Mannschaft zu verlieren, noch
diese Fracht über Bord zu werfen.«

Hardie senkte reumütig den Kopf, und mein Zorn verrauchte.

»Wir sitzen alle auf derselben Ætherbrigg«, fuhr ich fort, »und wir
haben noch eine lange Reise vor uns. Sobald wir den Jupiter
passiert haben, gibt es bis zur Erde keinerlei Hafen mehr für uns.
Möglicherweise werden die Marsianer uns aufs Korn nehmen. Eine
Chance haben wir auf dieser Fahrt nur, wenn wir zusammenhalten.
Mister Hardie, ich glaube Ihnen, dass Sie keine Meuterei anstiften
wollten. Darüber hinaus sind wir auf Ihre Fähigkeiten als Navigator
angewiesen und können es uns schlicht nicht leisten, Sie zu
verlieren. Daher beantrage ich vor diesem Gericht, Sie ohne weitere
Einschränkungen freizusprechen.«

Die anderen wechselten einen Blick und nickten.

»Aber wie soll es nun weitergehen?«, fragte Tomlinson. »Die
Gerüchte sind nun einmal in der Welt. Wie sollen wir sie zurück in
Pandoras Büchse stopfen?«

Hardie straffte die Schultern. »Da ich den Ärger verursacht habe,
werde ich ihn auch wieder ausräumen. Ich werde mit jedem einzelnen
Æthermann sprechen und erklären, dass ich mich über die Art der
Fracht geirrt habe. Ich kann sehr überzeugend sein, wissen Sie?«

Mit einem halben Lächeln und einem angedeuteten Salutieren verließ
er die Kajüte.

Mulligan schüttelte den Kopf. »Was haben wir uns da nur
aufgeladen«, murmelte er, und es war nicht klar, ob er die Fracht
oder unseren Navigator meinte.

\bigpar

Immerhin, im weiteren Verlauf unserer Reise, nun durch den offenen
Æther, wurde die Mannschaft wieder zuversichtlicher, und die Gefahr
einer Meuterei schien gebannt. Das erleichterte mich ungemein, denn
nicht nur ich selbst, sondern auch der größte Teil meiner
Mannschaft war noch nie zuvor so weit von unserer Heimat auf den
Saturnmonden entfernt gewesen. Wenn ich in den Æther hinaus
schaute, erblickte ich keinerlei vertraute Konstellationen, kein
Mond erhellte unseren Weg, die Sterne waren nur blässliche
Schatten. Allein Adam Hardie studierte gut gelaunt die Ætherkarten
und stand mit Vorliebe selbst am Ruder. Die Abhängigkeit, in der
wir uns alle von ihm befanden, war mir schmerzlich bewusst, aber
Hardie schien sie seit der Gerichtsverhandlung nicht mehr
auszunutzen. Er leistete gewissenhafte Arbeit, schloss sich mir und
den anderen Offizieren enger an und führte uns sicher durch den
Asteroidengürtel bis in den Machtbereich der Marskolonie.

\bigpar

Der Mars und seine beiden Monde Deimos und Phobos waren die ersten
Kolonien der Erde gewesen, von Pionieren besiedelt, von
Ausgestoßenen und Verbrechern, die dort in harter Arbeit den Boden
urbar machten, Rohstoffe förderten und Nutztiere züchteten. Ihr
Leben war schwer und kurz, ihre Gemeinschaft verschworen. Als dann
die großen Ætherflotten an ihren Häfen vorbeizogen, bis zum Jupiter
und zum Saturn vorstießen und aufgrund der reichen Bodenschätze
rasch zu Wohlstand und Einfluss gelangten, fühlten die Marsianer
sich übergangen. Sie schlossen sich nach alter Sitte zu Matelotagen
zusammen, zu engen Partnerschaften, bei denen der eine Partner sein
Glück in Ackerbau und Viehzucht versuchte, während der andere auf
Raubzüge ging und die vorüberfahrenden Ætherschiffe überfiel, um
einen Teil des neuen Reichtums für den eigenen Bedarf abzuzweigen.

In den darauffolgenden Jahrzehnten waren die Fronten zunehmend
verhärtet und unversöhnlich geworden. Vor allem seit die Gilde
ihren Machtbereich bis zum Pluto ausgedehnt hatte und auf dessen
Mond Charon mit dem Abbau des seltenen Edelmetalls Auriferrum
begonnen hatte, ließen die Marsianer keinen Transport mehr
ungeschoren, und darüber hinaus hieß es, dass sie mit den Kapitänen
der Ætherschiffe, deren Ladung nicht ihren Wünschen entsprach,
nicht eben zimperlich umgingen.

Mein eigener Wunsch war es, sowohl die \emph{Fancy} als auch mein
Fell zu retten, weil ich an beiden sehr hing. Daher gab ich Hardie
die Anweisung, einen gehörigen Haken um den Mars mitsamt seinen
Monden und alle in der Umlaufbahn befindlichen Zollkutter zu
schlagen, ließ alle Segel setzen und den Propeller auf voller Kraft
laufen, und hoffte, dass es mir gelingen würde, den Machtbereich
der Marsianer so rasch und unentdeckt wie eine Schiffsratte zu
passieren. Und es hätte auch gelingen können, wären wir nicht
längst verraten gewesen.

\bigpar

Als Mulligan die Kajüte betrat und einen Zollkutter an Lee meldete,
begriff ich, was für ein Narr ich gewesen war. Hatte ich nicht
selbst lachen müssen, als Mijnheer de Mulder behauptete, unsere
Order sei geheim? Hatte nicht außerdem Hardie erwähnt, dass unser
Transport das Tagesgespräch im \emph{Alesia} gewesen sei? Natürlich
hatten dort neben den Steuerleuten, den Lotsen und den
Æther-Navigatoren auch alle übrigen zugehört – die klatschsüchtigen
Weiber, ihre verdreckten, krummbeinigen Kinder und schließlich die
hoffnungslosen Trunkenbolde, deren einzige Aussicht auf eine volle
Börse darin bestand, die Gerüchte aus den Spelunken weiterzutragen.
Wer von all diesen Eingeweihten hatte die Route und die Ladung der
\emph{Fancy} ausgeplaudert? Hatte er wenigstens einen Beutel voller
Silberlinge in Empfang nehmen können, oder war sein einziger Lohn
ein Dolch im Rücken gewesen?

\bigpar

So rasch ich konnte, folgte ich Mulligan hinauf zum Achterdeck und
beobachtete durch mein Perspektiv den Kutter, der sich an unserer
Seite knapp außerhalb der Schussweite hielt. Für ein Zollschiff war
er bemerkenswert gut bewaffnet. Ich zählte acht Stückpforten, dazu
eine Reihe von Drehbassen, die an der Brüstung befestigt waren. Das
Achterdeck des Kutters war voll besetzt. Einer der Offiziere hob
sein Sprachrohr und rief uns an.

»Ætherbrigg \emph{Fancy}, drehen Sie bei!«

Ich entriss Mulligan das Sprachrohr und brüllte: »Den Teufel werde
ich tun, ihr verdammten Piraten! Meine Ware ist für die Erde
bestimmt, ich werde sie erst dort verzollen!«

Ȯtherbrigg \emph{Fancy}, drehen Sie bei und lassen Sie sich ins
Schlepptau nehmen! Sie befinden sich im Hoheitsgebiet der
marsianischen Kolonie. Folgen Sie uns zur Zollstelle Phobos und
deklarieren Sie Ihre Ware!«

»Nein, zum Henker! Ich habe Transit-Papiere der Gilde bei mir!«

Ein höhnisches Gelächter war die Antwort.

»Wenn Sie uns nicht freiwillig folgen, werden wir Sie dazu
zwingen!«

Ich wollte eine angemessene Antwort geben, aber in diesem Moment
erscholl ein Schreckensschrei von Sonnenluv.

»Feind an Deck! Kapitän Porch, die Marsianer entern auf!«

Mit einem unterdrückten Fluch fuhr ich herum. Natürlich hatten die
Marsianer von Anfang an im Sinn gehabt, uns zu entern. Das ganze
Gespräch hatte nur der Ablenkung gedient, um der Sturmtruppe in
einer kleinen Barkasse die Gelegenheit zu geben, uns unbemerkt zu
überraschen und das Deck zu stürmen.

\bigpar

»Alle Mann klar zum Gefecht!«, befahl ich. »Mulligan, öffnen Sie
die Waffenkammer und geben Sie die Coulomb-Gewehre und die
Entermesser aus. Wilkins!« Ich schnappte meinen zweiten Maat, der
sich gerade ins Getümmel stürzen wollte, bei der Schulter. »Sie
haben strikte Order, im Falle eines Gefechtes unter Deck zu gehen
und die Fracht zu schützen, haben Sie das vergessen?«

»Aber Sir, gegen die Marsianer brauchen Sie doch jeden Mann!«

»Verteidigen Sie die Fracht der Gilde mit Ihrem Leben!«, schnauzte
ich, »sonst werde ich Sie persönlich zur Rechenschaft ziehen!«

»Aye, aye, Sir!« Der Junge stand stramm, zog seine Axt aus dem
Gürtel und kletterte unter Deck.

\bigpar

Die Marsianer kamen über uns wie Ameisen. Als wir uns nach Luv
wandten, um die Besatzung der Barkasse abzuwehren, enterte eine
zweite Gruppe aus mehreren Jollen über unser Heck und schwärmte
aus. Ich schickte einige befahrene Æthermänner nach achtern, aber
nun kam auch der Kutter näher und tauchte unter unserer Artillerie
hindurch. Die Drehbassen feuerten und zwangen uns in Deckung.

»Scharfschützen in die Wanten!«, brüllte ich. »Nehmt die Marsianer
an den Drehbassen aufs Korn!«

Mulligan tauchte neben mir auf und drückte mir meine zweischneidige
Axt in die Hand. Er selbst hatte sich mit einem Auriferrumsäbel und
einer halbautomatischen Coulomb ausgerüstet. »Immerhin sind sie
nicht an die Waffenkammer herangekommen, Sir. Aber sie rücken von
allen Seiten an.«

»Sagen Sie Hardie, dass er uns unauffällig ein Stück von dem Kutter
wegbringen soll, außerhalb der Reichweite ihrer Drehbassen«,
entschied ich. »Dann können wir unsere Kanonen nutzen. Lassen Sie
auf dem Geschützdeck klar zum Feuern machen. Wenn der Kutter
aufgibt oder abdreht, sind sie von der Nachhut abgeschnitten.«

\bigpar

Mulligan gab die Befehle weiter, und die \emph{Fancy} begann kaum
merklich nach Luv zu driften. Inzwischen wurde auf dem gesamten
Deck gekämpft. Die Marsianer waren mit großkalibrigen
Strahlengewehren ausgerüstet, furchterregenden Waffen, die
allerdings auf dem begrenzten Raum des Schiffes schwer zu benutzen
waren. Meine Männer verwickelten sie in Zweikämpfe und versuchten
sie in die Enge zu treiben. Blitze erhellten den Æther, immer
wieder sirrte der hochfrequente Ton eines abgefeuerten
Strahlengewehres über das Deck, Splitter von getroffenen Masten und
Aufbauten flogen vorbei. Doktor Tomlinson jagte seine Helfer herum,
um die Verletzten zu bergen. Aber auch wir hatten Erfolg, denn drei
der sechs Drehbassen des Kutters feuerten nicht mehr. Vermutlich
hatten die Scharfschützen unter dem Kommando von Rogers gute Arbeit
geleistet.

Drei Marsianer fielen mir besonders auf, sie schienen überall zu
sein und lichteten die Reihen meiner Männer. Einer war sehr groß,
mit der kupferfarbenen Haut der Ætherleute. In seiner Nähe hielt
sich eine blasse, beinahe weißhäutige Gestalt, die geschickt mit
einem Auriferrumsäbel kämpfte. Der Dritte im Bunde schließlich war
ein finsterer Geselle mit pechschwarzem Bart, der mehrere
Coulombpistolen in seinem Gürtel trug und mit beiden Händen
feuerte.

»Mulligan!«, rief ich und deutete auf das Trio. »Wenn wir diese
Drei aufhalten, können wir ihnen den Schwung nehmen!«

Er nickte und stürzte sich mitten zwischen die Kämpfer. Ich hob die
Axt und machte mich ebenfalls auf den Weg.

Ein furchtbares Krachen riss mich von den Füßen und raubte mir
beinahe die Sinne. Die \emph{Fancy} schwankte wie eine Fahne im
Sonnenwind, dichter Qualm zog über das Deck. Anscheinend waren wir
außerhalb der Reichweite des Kutters gelangt, und die
Geschützmannschaften hatten gefeuert. Ich zog mich hoch, um das
Ergebnis der Breitseite ins Auge zu fassen – und blickte direkt in
den Lauf eines Strahlengewehrs. Der riesenhafte Marsianer stand vor
mir und legte auf mich an. Wer jemals in den Lauf einer Waffe
gestarrt hat, weiß, dass vor diesem Anblick alle anderen Gedanken
schwinden. Ich dachte weder an das Schicksal meines Schiffes und
meiner Männer, noch an die Fracht der Gilde. Alles, was in meinem
Kopf noch Platz fand, war das nachtschwarze Rohr und das bösartig
grinsende Gesicht dahinter. Dann zog der Riese den Abzug durch.

\bigpar

Ein Elmsfeuer tanzte über den Kupferdraht oberhalb des Laufs,
behände wie ein Iltis. Es sprang auf die Hand des Riesen und
versetzte ihm einen schmerzhaften Schlag. Für einen winzigen
Augenblick starrten wir beide fassungslos auf die nun unbrauchbare
Waffe, die von dem zufälligen magnetischen Phänomen lahmgelegt
worden war. Dann schüttelte sich der Riese. Mit einem Fluch ließ er
sein Gewehr fallen und griff nach dem Entermesser in seinem Gürtel,
aber nun hatte ich den Schock ebenfalls überwunden und hob meine
Axt.

Einige Kämpfer schoben sich zwischen uns, wir wurden auseinander
gedrückt und verloren uns aus den Augen. Ich schlug mich zur Reling
durch und hinderte die Mannschaft eines weiteren Beibootes am
Aufentern, indem ich ihre Enterhaken kappte.

Rogers schlitterte die Takelage hinunter und landete vor meinen
Füßen, seine Hand blutete. »Der Kutter ist schwer getroffen, Sir«,
meldete er dennoch guter Dinge. »Die Breitseite hat seine Steuerung
beschädigt, und die Besatzung der Drehbassen haben wir sauber
erwischt. Ich denke, der hat genug.«

Ich zog mich ein Stück in die Wanten hoch, um mir einen Überblick
zu verschaffen.

\bigpar

Tatsächlich, wir schienen allmählich im Vorteil zu sein. Die
Marsianer kämpften mit dem Rücken zur Reling und konnten ihre
Strahlenkarabiner nicht wirkungsvoll einsetzen. Außerdem deuteten
einige von ihnen besorgt auf den Kutter, der tatsächlich Anstalten
machte, beizudrehen, zumindest soweit die demolierte Steuerung es
zuließ.

Erneut schickte die \emph{Fancy} eine Breitseite hinüber, die den
Kutter heftig schaukeln ließ. Nun gaben die marsianischen
Hauptleute Befehl zum Rückzug. Die Männer sammelten, so gut es
ging, ihre verletzten Kameraden ein, ließen sich über Bord fallen
und wurden von den Beibooten aufgesammelt.

Mulligan erschien an meiner Seite. »Sieg auf der ganzen Linie,
Sir!«, meldete er mit verhaltenem Triumph.

»Sie fliehen!«, jubelte auch Doktor Tomlinson. »Wir haben sie
tatsächlich zurückgeschlagen!«

Nachdenklich schüttelte ich den Kopf. Ich suchte nach dem
kupferhäutigen Riesen, der auf mich angelegt hatte, und nach seinem
blassen Gefährten. Auch der Schwarzbärtige war nicht unter den
Flüchtenden.

»Das sind nicht alle«, stellte ich fest. »Es müssen noch welche an
Bord sein. Aber wo \ldots{}«

Meine Beine setzten sich in Bewegung, noch bevor ich das Ende des
Gedankenganges erreicht hatte.

»Unter Deck!«, rief Mulligan und schloss sich mir mit erhobenem
Säbel an. »Sie haben sich nach unten durchgeschlagen, zu den
Lagerräumen. Wahrscheinlich haben sie es auf die Fracht der Gilde
abgesehen.«

»Wilkins ist dort unten allein!«

Ich hörte das Jaulen eines abgefeuerten Coulomb-Gewehrs, nahm die
letzten Sprossen des Niedergangs mit einem Sprung und riss die Axt
in die Höhe. Hinter mir hörte ich Mulligan aufsetzen. Drei Köpfe
fuhren zu uns herum, ein kupferhäutiger, ein blasser, einer mit
schwarzem Bart, genau wie ich vermutet hatte. Hinter ihnen auf den
Planken krümmte sich mein zweiter Maat.

Der Blasse packte seine rauchende Strahlenbüchse am Lauf und
schwang sie gegen mich. Ich tauchte darunter hindurch und schlug
mit der Axt nach seinen Beinen. Er fiel um wie ein gefällter Baum,
und ich versetzte ihm den Rest. Neben ihm stürzte auch der Riese,
getroffen von Mulligans halbautomatischer Pistole. Ich empfand eine
grimmige Genugtuung bei dem Gedanken, dass sein letzter Blick einem
nachtschwarzen Lauf gegolten hatte.

Nun war nur noch der Schwarzbärtige übrig. Seine Pistolen hatte er
im Verlauf des Kampfes eingebüßt, bis auf eine, die er nun gezogen
hatte und abwechselnd auf Mulligan und mich richtete. »Was meint
ihr«, fragte er lauernd, »warum ich euch nicht schon längst
erschossen habe?«

Es war Mulligan, der schneller reagierte. »Vermutlich, weil die
Pistole leer ist!«, stellte er fest und schoss den Marsianer über
den Haufen.

\bigpar

Ich schob den Körper des Schwarzbarts mit dem Fuß aus dem Weg und
kniete mich neben Wilkins. Der Junge war blass, auf seiner Stirn
stand öliger Schweiß, und seine Lippen schimmerten bläulich.

»Kapitän Porch, Sir«, murmelte er. »Zu Befehl, ich habe die Fracht
der Gilde verteidigt und keinen der Marsianer vorbei gelassen.«

Ein dünner Faden Blut rann aus seinem Mundwinkel. Ich tupfte ihn
mit dem Ärmel fort.

»Gut gemacht, Wilkins! Sie haben meinen Befehl vorbildlich
ausgeführt und hier die Stellung gehalten. Damit haben Sie sich
eine Belobigung verdient.«

»Wirklich, Sir? Ich dachte \ldots{} Ach, bitte verzeihen Sie mir, Sir,
ich war so dumm! Für eine Weile dachte ich, Sie hätten mich nur
unter Deck geschickt, damit ich aus dem Weg sei.«

»Aber gewiss nicht, Wilkins«, erwiderte ich sanft. »Ich hätte doch
im Kampf nicht auf meinen besten Mann verzichtet, wenn ich Sie
nicht dringend für diese Sonderaufgabe gebraucht hätte!«

Er lächelte matt und verzog gleich darauf schmerzlich das Gesicht,
als ob ihn diese Bewegung zu viel Kraft gekostet hätte.

»Sir? Wenn ich es nicht schaffe – wenn ich nicht bis zur Erde und
wieder zurück zum Saturn durchhalte – werden Sie dann meinen Eltern
von der Belobigung erzählen?«

»Natürlich, Wilkins. Und auch dem Gildenmeister de Mulder. Ich bin
sicher, er wird sich Ihren beispielhaften Einsatz für seine Fracht
Einiges kosten lassen. Ihre Eltern werden stolz auf Sie sein,
ebenso stolz wie ich es bin. Aber nun versuchen Sie zu schlafen.«

»Das würde ich gern, Sir, aber es tut so weh. Wenn ich nur meine
neue Coulomb-Pistole mitgenommen hätte \ldots{}«

Seine Stimme verebbte. Ein paar Mal holte er noch rasselnd Atem,
dann sackte sein Kopf zur Seite, seine Augen brachen.

»Wilkins!« Ich horchte vergebens an seiner Brust, zog ihn hoch,
ließ seinen leblosen Körper wieder zu Boden sinken und schlug in
wildem Zorn mit den Fäusten auf die Planken ein, bis sie
schmerzten.

»Verdammt, was wird hier gespielt? Was ist an dieser Fracht nur so
wichtig, dass der Junge dafür draufgehen musste?«

Mulligan neben mir legte sacht die Hand auf meine Schulter.
»Kapitän Porch, Sir. Ich weiß, wie Sie sich fühlen, denn mir geht
es nicht anders. Aber wir haben keine Zeit. Wir müssen hier weg.
Bisher hat der Kutter nicht um Hilfe heliographiert, weil sie die
Beute für sich allein wollten. Doch jetzt werden sie uns ihre
Kameraden auf den Hals hetzen. Innerhalb kürzester Zeit haben wir
alle verfügbaren marsianischen Zollkutter auf den Fersen.«

Mühsam erkämpfte ich meine Fassung zurück. »Sie haben Recht,
Mulligan«, sagte ich. »Zuerst müssen wir an die Lebenden denken.
Aber danach, das schwöre ich, werfen wir einen genauen Blick auf
diese Mumie.«

\bigpar

Ich erhob mich und eilte den Niedergang hinauf, zum Achterdeck, wo
Doktor Tomlinson und der Bootsmann Jasper mich erwarteten. »Hardie,
sind Sie noch dienstfähig?«, rief ich zu der zerkratzten,
zerschundenen Gestalt am Steuerrad hinüber.

Der Navigator salutierte. »Kapitän Porch, Sir!«

»Da draußen sind noch mehr Marsianer unterwegs. Können Sie uns so
schnell wie möglich hier herausbringen?«

»Aye, aye, Kapitän Porch! Mit dem allergrößten Vergnügen!«

Er drehte am Steuerrad, bis die \emph{Fancy} protestierend
quietschte.

»Was macht er denn da?«, fragte Mulligan beunruhigt. »Das Schiff
liegt viel zu hart am Sonnenwind. Die Segel werden zurückschlagen,
und dann machen wir keine Fahrt mehr!«

Ohne seinen Blick vom Steuerrad zu wenden, streckte Doktor
Tomlinson die Hand aus. »Was wetten wir, dass es gut geht?«

»Ich wette nicht«, erwiderte Mulligan.

»Weil Sie verlieren würden, mein Freund!«

»Segel in Sicht!«, meldete in diesem Moment der Ausguck. »Fünf
Kutter in Lee!«

»Gleich fünf von dieser Sorte \ldots{}«, murmelte Mulligan. »Gegen die
kommen wir nicht an!«

»Das müssen wir auch nicht.« Der Doktor war mehr und mehr guter
Dinge. »Die Marsianer bekommen nur noch unser Heck zu Gesicht, Sie
werden es ja erleben.«

Ich beobachtete Hardie. Alles Großsprecherische war von ihm
gewichen. Er neigte sich tief über das Steuerrad, die Lippen halb
geöffnet, als halte er Zwiesprache mit dem Schiff. Dann korrigierte
er den Kurs um einen halben Strich. Die \emph{Fancy} ächzte unter
dieser Zumutung und schwankte, als wolle sie sich auf die Seite
legen. Dann, ganz langsam, füllten sich die Segel. Wir nahmen Fahrt
auf.

Ich bemerkte, dass ich die Luft angehalten hatte, und atmete aus.
Als sei ihr Kiel eine Kufe, ritt die \emph{Fancy} den Sonnenwind
ab. Sie glitt durch den Æther, wurde schneller und schneller.
Zischend tanzten Elmsfeuer über ihre Masten und Rahen, der
Propeller sprühte Funken.

»Kapitän Porch, Sir«, flüsterte Jasper neben mir, »ich bin kein
Ætherküken, aber so etwas habe ich noch nie gesehen. Was tut Hardie
da?«

»Soll mich der Teufel holen, wenn ich das weiß, Jasper«, erwidert
ich ebenso leise, »aber jedenfalls hat er nicht zuviel versprochen.
Er ist tatsächlich der Herr der Sterne.«

\bigpar

Bald darauf hatten wir das vom Mars kontrollierte Gebiet verlassen.
Die \emph{Fancy} wurde wieder langsamer, und wir nahmen direkten
Kurs auf die Erde.

Hardie, der von der Anstrengung des Steuerns zitterte, gesellte
sich zu den übrigen Offizieren auf das Achterdeck.

»Ein Meisterstück!«, lobte Doktor Tomlinson, aber der Navigator
schüttelte nur abweisend den Kopf.

»Ich habe ja gesagt, dass ich es kann. Kapitän Porch, Sir, mich
würde brennend interessieren, was in unserem Laderaum liegt. Ist es
wirklich die Mumie eines verstorbenen Kaufmanns? Und warum ist sie
so wertvoll, dass wir ihretwegen die gesamte Marsflotte auf den
Fersen haben?«

»Das würde ich auch gern wissen, Hardie«, pflichtete ich ihm bei.
»Jasper, sorgen Sie dafür, dass die \emph{Fancy} auf Kurs bleibt.
Mulligan, Doktor Tomlinson – kommen Sie mit, wir werden dem
Geheimnis dieser Fracht auf den Grund gehen.«

Gemeinsam stiegen wir den Niedergang hinunter. Die Marsianer waren
inzwischen über Bord geschafft worden und Wilkins lag aufgebahrt in
den Räumen des Doktors, aber über den Planken schwebte noch immer
der metallische Geruch des Todes. Meine Hände zitterten ein wenig,
als ich das schwere Vorhängeschloss öffnete und die Tür zum
Laderaum aufstieß.

Der Raum war nur schwach durch ein paar Luken und Grätinge erhellt.
In dem matten Licht erkannte ich gleich neben der Tür die kleinen
Kisten mit den privaten Habseligkeiten der Æthermänner. Dahinter
lagerten einige Fässer mit Vorräten, die in der Kombüse keinen
Platz gefunden hatten.

\bigpar

»Dort hinten in der Ecke steht die Fracht der Gilde«, sagte
Mulligan. »Der Hafenmeister und seine Gesellen haben sie selbst
dort aufgestellt, die Plane darüber befestigt und alles
versiegelt.«

Das Siegel de Mulders mit den vier Mühlenflügeln prangte auf dem
Öltuch. Es war unzerstört.

Ich blickte in die Runde. »Meine Herren, dies ist unsere letzte
Möglichkeit, die Angelegenheit auf sich beruhen zu lassen. Wir
können die Fracht abliefern und die Bezahlung dafür kassieren, ohne
Fragen zu stellen.«

Hardie biss sich auf die Lippen und schüttelte den Kopf.

»Jetzt sind wir schon so weit gegangen«, murmelte der Doktor, »da
sollten wir es auch zu Ende bringen.«

»Wir sind es Wilkins schuldig«, fügte Mulligan hinzu.

»Nun gut.« Ich zog den schmalen Auriferrum-Dolch aus meinem
Stiefel, durchtrennte die Plane und zerrte sie zur Seite. Ein
schmuckloser Holzsarkophag lag darunter, mit einem ebenfalls
versiegelten Band verschlossen, zu beiden Seiten lehnten die vier
Kanopen. Mit klopfendem Herzen ergriff ich das nächstgelegene Gefäß
und schnitt den mit Harz verklebten Deckel herunter. Dann drehte
ich sie um.

Eine undefinierbare, weißliche Masse ergoss sich auf die Planken
und spritzte bis auf meine Schuhe.

»Das Hirn«, kommentierte Doktor Tomlinson trocken, »nicht ganz nach
den Regeln der Kunst einbalsamiert.«

Ich unterdrückte ein Würgen und wischte die Schuhe verstohlen an
der Plane ab. Nur halb entschlossen griff ich nach der nächsten
Kanope. »Leber, Magen oder Gedärme? Es werden noch Wetten
angenommen.«

»Ich wette nicht«, erwiderte Mulligan mechanisch. Er war sehr blass
geworden.

Hardie machte einen Schritt zurück. »Na los, bringen wir es hinter
uns!«

Dieses Mal wendete ich die Kanope behutsamer. Nichts geschah. Ich
schüttelte sie vorsichtig, schließlich griff ich mit spitzen
Fingern hinein, tastete herum und zog den Inhalt heraus. Es war ein
eng beschriebenes Pergament mit dem Briefkopf de Mulders. »Treffer!
Wir sind auf der richtigen Spur.«

Ich durchtrennte das Siegel am Sarkophag. Hardie und Mulligan
fassten den Deckel und hoben ihn zur Seite. Eine in Leinenbinden
gehüllte Gestalt lag darunter. Wir lösten die Binden, und uns fiel
eine bräunliche, vertrocknete Mumie entgegen, dürr wie ein Zweig,
mit überscharfen Zügen. Ihre Bauchhöhle war mit Leinensäckchen
vollgestopft. Als ich eines davon aufschnürte, entdeckte ich ein
leuchtend gelbes Pulver, das ich behutsam durch meine Finger
rieseln ließ.

»Was zum Henker ist das?«, fragte Tomlinson.

»Dies, werter Doktor«, erklärte ich, »ist die Antwort auf unsere
Fragen.«

Bald darauf erreichten wir die Erde und unseren Zielhafen
Amsterdam. Der dortige Hafenmeister war über unseren Auftrag
offenkundig informiert, er nahm die Transitpapiere der Gilde
entgegen, gab die Anweisung, unsere Fracht mitsamt der darum
gehüllten Plane ins Lagerhaus zu schaffen, und überreichte mir
einen Passierschein, mit dem ich mich zum Amsterdamer
Gildenoberhaupt, Mijnheer van Felderen zu begeben hatte. Der
schroffe Ton seiner Order missfiel mir, aber ich hatte nun schon
mehrfach erfahren, dass es unklug war, in den Angelegenheiten der
Gilde allzu lautstarken Protest einzulegen, deswegen nickte ich nur
verbindlich und verfügte mich zum Gildehaus.

Das Gebäude, vor dem ich bald darauf stand, war nicht weniger
prächtig als sein Gegenstück auf Dione. Nur die Farben waren
anders, hier herrschten Grün- und Brauntöne vor. Der Domestik, der
mich einließ, war nicht weniger hochnäsig, die Teppiche eher noch
dicker, und der Raum, in dem Mijnheer van Felderen mich schließlich
empfing, fast so groß wie ein Kirchenschiff.

\bigpar

»Nun, Kapitän Porch«, sagte der schmale, hochgewachsene Kaufmann
und streckte mir seine schlanke Hand entgegen, »ich höre, Sie haben
die Überreste unseres verstorbenen Gildenmitgliedes Cornel de Fries
beim Hafenmeister abgeliefert. Hoffentlich hatten Sie eine
angenehme Reise.«

Ich schüttelte den Kopf. »Leider nicht so angenehm, wie ich es mir
gewünscht hätte. Die Marsianer waren ganz außerordentlich
interessiert an dem Verstorbenen, und im weiteren Verlauf der Fahrt
habe ich auch herausgefunden, woran es lag. Der Tote hatte eine
sehr interessante Füllung.«

Van Felderen zuckte nicht mit der Wimper. »Saphroniglia, mein
Bester, das teuerste Gewürz des Universums. Es gedeiht nur auf
Japetus und Hyperion, und die Aufbereitung ist außerordentlich
kompliziert. Dennoch ist es der Mühe wert, denn eine einzige Prise
davon genügt, selbst dem verwöhntesten und überdrüssigsten Gaumen
das Vergnügen am Speisen zurückzugeben.«

»Aber es ist auch ein Nervengift«, sagte ich. »Sein Verzehr
schränkt den freien Willen ein.«

»Selbstverständlich – was meinen Sie, weshalb wir uns in der
glücklichen Lage sehen, für die nächsten Jahrzehnte die exklusiven
Importrechte zu besitzen. Das ist in den Verträgen festgehalten,
die Sie unzweifelhaft in den Kanopen entdeckt haben.«

»Sie haben mich und meine Mannschaft als Schmuggler missbraucht!«

Leichthin zuckte er die Achseln. »Wenn Sie mögen, können Sie als
Wiedergutmachung eines der Saphroniglia-Säckchen behalten. Es ist
gewiss mehr wert als Ihr Schiff mitsamt der Mannschaft. Und nun –
leben Sie wohl.«

\bigpar

Ich verließ das Gildehaus und kehrte zum Hafen zurück. Dort gab ich
Order, die \emph{Fancy} auf der Stelle zum Ablegen klar zu machen.
Wir nahmen nur den nötigsten Proviant an Bord und heizten den
Kessel vor, bis er glühte. Hardie stand am Steuerrad, bereit, den
kleinsten Sonnenwind abzureiten, denn uns allen war klar, was
passieren würde, sobald van Felderen feststellte, was aus seiner
Fracht geworden war. Zwischen Mars und Erdenmond hatten wir den
Inhalt jedes einzelnen Säckchens gewissenhaft in den Æther geleert
und zum guten Schluss die Verträge in winzige Fetzen gerissen.

\bigpar

Mulligan erwartete mich auf dem Achterdeck. »Klar zum Ablegen!«,
meldete er. »Welcher Kurs, Kapitän Porch?«

Ich blickte über das Deck. Zwanzig Mann sahen zu mir auf, denen in
wenigen Stunden nicht nur sämtliche Marskutter, sondern auch jedes
einzelne Ætherschiff der Gilde auf den Fersen sein würde. »Zu den
Freihandelsstationen auf den Saturnringen«, entschied ich. »Ab
jetzt sind wir auf uns selbst gestellt, und auf Dione haben wir
noch ein paar Rechnungen offen. Hievt die Anker! Setzt die Segel!

\bigpar

Und unsere Losung soll sein: PHAETONS FREUND UND ALLEM KOSMOS
FEIND!«

\end{document}
