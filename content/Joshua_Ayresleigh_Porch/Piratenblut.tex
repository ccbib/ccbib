\usepackage[ngerman]{babel}
\usepackage[T1]{fontenc}
\hyphenation{wa-rum}


%\setlength{\emergencystretch}{1ex}

\newcommand\bigpar\medskip
\newcommand\schiff\textit
\newcommand\latein\textit
\newenvironment{dekret}{\itshape}{}

\begin{document}
\raggedbottom
\begin{center}
\textbf{\huge\textsf{Piratenblut}}

\medskip
Aus den Tagebüchern des \\
Joshua Ayresleigh Porch
\end{center}

\bigskip

\begin{flushleft}
Dieser Text wurde erstmals veröffentlicht in:
\begin{center}
Die Steampunk-Chroniken\\
Geschichten aus dem Æther
\end{center}

\bigskip

Der ganze Band steht unter einer 
\href{http://creativecommons.org/licenses/by-nc-nd/2.0/de/}{Creative-Commons-Lizenz.} \\ 
(CC BY-NC-ND)

\bigskip

Spenden werden auf der 
\href{http://steampunk-chroniken.de/download}{Downloadseite}
des Projekts gerne entgegen genommen. 
\end{flushleft}

\newpage


»Eure Ausreden haben das hohe Gericht nicht überzeugt. Einmal Pirat
– immer Pirat! Joshua A. Porch, genannt Säbel-Josh, Ihr werdet
verurteilt, hinaus zum Richtplatz gebracht und dort am Halse
aufgehängt zu werden, bis Ihr tot, tot, tot seid. Danach soll Euer
Körper in Ketten gelegt und im All ausgebracht werden, auf dass der
Sonnenwind ihn verdorrt. Und Gott sei Eurer armen Seele gnädig!«

\bigpar

Zwei Gerichtsdiener packten mich fest am Halseisen, als
befürchteten sie einen Ausbruch, aber ich stand still und verzog
keine Miene. Dies war das Urteil, das einem Piraten zukam, auch
wenn er sich längst geläutert fühlte, und zumindest würde ich
aufrecht und in meinen Stiefeln sterben, anders als die armen
Teufel, die bereits vor dem Prozess in den dunklen Verliesen von
Cape Ivory Castle an Fieber und Hunger krepiert waren.

\bigpar

Der Richter, ein kleiner fetter Mann mit weißen Händchen und
stechend schwarzen Augen, hatte mich während seiner Worte genau
beobachtet und blickte nun beifallheischend um sich. »Dies ist das
offizielle Urteil des Gerichts. Ich habe hier allerdings«, fuhr er
fort und zog ein vielfach gefaltetes Pergament aus dem Ärmel, »ein
Dekret von unserem allergnädigsten König George.«

Ich hielt den Atem an und spürte, wie sich meine Nackenhaare
aufstellten. Gab es etwa eine Generalamnestie, von der ich noch
nichts wusste? Aber warum dann dieser Blick, so schlau und voller
Genugtuung?

»Möchtet Ihr, Säbel-Josh, Pirat und Feind der Menschheit, dass ich
dieses Dekret verlese?«

Niemand hatte mich je Säbel-Josh genannt, auch nicht ich selbst. Es
war ein Name, den die Gazetten erfunden hatten. Außerdem hatte ich
mich nie mit Säbeln abgegeben. Im Nahkampf war eine Axt
praktischer, und darüber hinaus konnte man sie benutzen, um
störrische Seekisten aufzusprengen. Dennoch protestierte ich nicht
gegen diese Anrede, sondern nickte nur mit trockener Kehle.

»Unser allergnädigster König George«, sagte der Richter, stand auf
und entfaltete das Pergament gänzlich, »tut Folgendes kund und zu
wissen:

\bigpar

\begin{dekret}
Zur Mehrung Unserer Güter und zum größeren Ruhme des Commonwealth
der Planeten sollen die Todesurteile, die ein Ordentliches Gericht
in Unseren Gebieten aufgrund der Verbrechen der Piraterie, des
Raubes, des Straßenraubes und der Desertion fällt, von der
Vollstreckung ausgesetzt werden. Die Verurteilten sollen vielmehr
auf eines der Ætherschiffe verbracht werden, die zu Unseren Kolonien
auf dem Pluto reisen, um dort für den Rest ihres wertlosen Lebens
Dienst in den Auriferrum-Minen des Charon zu tun.

Gegeben am siebenten November zu Buckingham Palace,
\begin{flushright}
George VIII. Rex.«
\end{flushright}
\end{dekret}

Er legte das Pergament wieder in ordentliche Falten und schob es
zurück in den Ärmel.

»Nun, was denkt Ihr darüber, Säbel-Josh? Ihr dürft leben! Werdet
Ihr die Gnade Unseres Königs voller Dankbarkeit annehmen?«

Sehr langsam sickerte der Inhalt des Dekrets in mein Bewusstsein,
aber als er dort angekommen war, bäumte ich mich auf wie ein Tier.

»Nein! Nicht die Minen! Lasst mich hängen, aber nicht dort unten in
den Minen verrotten!«

»Wäret Ihr bereit, mich auf Knien darum zu bitten?«

Ich starrte den Richter an. Wie eine Klapperschlange kam er mir
vor, und ich hätte alles darum gegeben, ihn wie solch eine Schlange
zertreten zu können. Aber meine Arme und Beine waren mit Ketten
gefesselt, um meinen Hals hing ein schweres Eisen und daran zwei
zappelnde Gerichtsdiener, die sich mühten, mich wieder unter
Kontrolle zu bringen.

\bigpar

»Wäret Ihr dazu bereit?«

\bigpar

Die Auriferrum-Minen des Charon. Das dort geförderte Metall, so
leicht zu bearbeiten wie Gold und nach dem Aushärten doch so stabil
und fest wie Eisen, war das kostbarste Gut des Commonwealth und
wurde eifersüchtig gehütet. Seit seiner Entdeckung vor gerade
einmal zwanzig Jahren war es der unentbehrliche Bestandteil beinahe
jedes Maschinenantriebs geworden. Es förderte unsere Kohle, bewegte
unsere Dampfmaschinen und trieb sogar unsere Ætherschiffe an, wenn
der Sonnenwind nicht ausreichte.

Aber die Arbeit in den Minen dieses kalten, unbarmherzigen Mondes,
von dem aus die Sonne nur als ein blasser Punkt am Himmel erschien,
war weitaus schlimmer als der Tod. Niemand hatte je länger als ein
halbes Jahr in den Minen verbracht, bevor die fallenden Wetter ihn
erschlugen, die Kälte ihm das Leben aus den Knochen sog oder die
dünne Luft ihn den letzten Tropfen Blut in den Æther speien ließ.
Die Vorstellung, auf allen Vieren über den grauen Steinstaub zu
kriechen und röchelnd Blut zu spucken, ließ mich zu Boden sinken.

Demütiger als ein Kind kniete ich nieder, soweit die Ketten an
meinen Beinen es zuließen, und hob flehend die Hände.

»Ich bitte Euch, lasst mich hängen, aber schickt mich nicht in die
Minen.«

»Genug!« Der Richter wedelte mit seiner fetten weißen Hand. »Lieber
wollt Ihr gehängt werden, als Eurem König zu dienen? Schämt Euch,
Säbel-Josh! Euch wird die königliche Gnade widerfahren. Abführen!
Schafft ihn zurück in die Zelle. Gleich morgen geht es für ihn zum
Pluto, auf der \schiff{Mary Jane}. Und nun bringt mir den nächsten
von diesem Pack zur Verhandlung. Diese Welt werden wir von Euch und
Euresgleichen säubern, mein Freund. Eure Zeit ist endgültig
vorbei!«

\bigpar

Die Gerichtsdiener schleiften mich aus dem Saal, so sehr ich auch
brüllte und um mich schlug. Sie schleppten mich zurück durch die
hundertfach verzweigten Eingeweide von Cape Ivory Castle, so tief
hinunter, dass meine Schreie im Gerichtssaal nicht mehr zu hören
waren. Dann stießen sie mich in eine Zelle und schlossen klirrend
hinter mir ab.

Ich schaute mich um. War es die Zelle, in der ich die Wochen zuvor
verbracht hatte? Ich hätte es nicht sagen können. Die Wände waren
grau und feucht, auf dünnen Pritschen oder auf dem nackten
Steinboden hockten zusammengesunkene, verlorene Gestalten. Eine
blickte auf, die Augen von Fieber verdunkelt.

»Joshua Porch? Du lebst?«

Ich versuchte dem bleichen Gespenst einen Namen zuzuordnen. »Adam
Hardie? Bist du nicht ertrunken? Ich habe doch gesehen, wie du über
Bord gegangen bist, als die Navy uns enterte.«

»Sie haben mich rausgefischt«, murmelte er bitter. »Sie haben mich
sogar zusammengeflickt, nur um mich auf dem Charon verrecken zu
lassen. Skirmer ist tot. Als das Urteil verkündet wurde, hat er
sich losgerissen und ist geradewegs in das Bajonett eines der
Türwächter gerannt. Dem blieb nicht einmal mehr die Zeit, ›Halt,
oder ich schieße‹ zu rufen, da hing Skirmer schon an seinem
Bajonett wie ein Schwein am Haken und hatte sich selbst die Kehle
zerfetzt. Wenn du mich fragst: Das war der sauberste Abgang. Warum
sind wir nicht auf diese Idee gekommen?«

Ich hockte mich neben Hardie und dachte nach. Man möge mir
verzeihen, ich bin sehr langsam in meinem Denken. Ich brauche meine
Zeit. Erst nach einer ganzen Weile wusste ich, warum ich zwar
bereit gewesen war, vor dem Richter zu knien, nur um gehängt zu
werden, aber doch in letzter Konsequenz weder den Tod in den
Bajonetten noch den durch einen raschen Sprung durch die hohen
Fenster gewählt hatte. Ich wollte leben. Ein kleiner Teil von mir,
vielleicht meine angeborene Halsstarrigkeit, hing mit allen Fasern
an diesem kläglichen menschlichen Dasein und zog es der Ewigkeit in
der Hölle vor. So beschloss ich also, weiterzuleben und die
Hoffnung erst dann aufzugeben, wenn ich in den Minen des Charon
zugrunde ging. Ich war ein Mann des Æthers und hatte ihn mein Leben
lang befahren, zuerst als Matrose, dann als Kapitän und schließlich
als Pirat. Solange die Fahrt durch den Æther dauerte, würden sich
Wege und Möglichkeiten ergeben, mein Schicksal doch noch zum
Besseren zu wenden.

Hardie hustete rasselnd und holte mich aus meinen Gedanken zurück.
»Ich denke, ich werde es gar nicht bis zum Charon schaffen«,
keuchte er, »vielleicht nicht einmal mehr bis auf das Schiff.
Versprich mir nur, dass sie mich nicht auf die Krankenstation
bringen und mit ihren Wässerchen und Tinkturen quälen. Ich will in
meinen Stiefeln sterben!«

Ich legte den Arm um seine Schultern und richtete ihn ein wenig
auf, um ihm das Atmen zu erleichtern. »Was redest du dauernd vom
Sterben! Du bist mein Steuermann, und ich werde dich da draußen
noch brauchen.«

Er blickte zu mir auf, und in seinen fiebrigen Augen glomm ein
Hoffnungsfunke. »Du glaubst tatsächlich, dass ich noch einmal am
Steuerrad stehen werde?«

»So wahr ich hier im Dreck hocke, Kamerad«, erwiderte ich mit weit
mehr Gewissheit, als ich empfand. »Einmal Pirat, immer Pirat. Wir
werden einen Weg finden – wenn du nicht länger darauf bestehst zu
sterben. Ein toter Steuermann nützt mir nichts.«

Mühsam rappelte er sich auf. »Aye, aye, Captain Porch!«, wisperte
er.

Mit meinem zerlumpten Hemdsärmel wischte ich den kalten Schweiß von
seiner Stirn. »Halte durch, Kamerad. Noch baumeln wir nicht am
Galgen, das ist doch immerhin etwas!«

\bigpar

Die Zellentür öffnete sich, und zwei weitere Verurteilte wurden
abgeladen. Der eine, dunkel, mit blitzenden schwarzen Augen und
schmalen beweglichen Gliedern, trug seinen rechten Arm in der
Schlinge und stützte sich schwer auf seinen Begleiter. Dennoch
grinste er von einem Ohr zum anderen.

»\latein{Morituri salutant} – die dem Tode Geweihten grüßen!«, rief
er in den Raum. »Nun, immerhin sind wir Freund Hein ein weiteres
Mal vom Sichelblatt gehüpft, und das ist doch weitaus mehr, als ich
noch vor einer halben Stunde erhofft hätte!«

»Minoe, kannst du nicht ein einziges Mal dein dummes Maul halten!«,
knurrte der andere. Er war ein Riese von einem Mann,
grobschlächtig, mit Händen wie Schaufeln. Sein Gesicht war durch
eine grausame Narbe entstellt, die sich vom linken Ohr unter der
platten Nase hindurch bis zum rechten Mundwinkel zog.

»\latein{Cogito, ergo sum}«, war die flinke Antwort des anderen. »Ich
denke, also bin ich. Und wenn ich meine Gedanken nicht mehr äußern
kann – nun, so darfst du daraus getrost schließen, dass ich tot
bin, und mir ein Grab ausheben. Aber noch ist es nicht soweit, mein
lieber Alcester, und solange noch Hoffnung besteht, weigere ich
mich zu verzweifeln.«

Das rasche Mundwerk Minoes und noch mehr Alcesters Körperkräfte
waren genau nach meinem Geschmack, und so klopfte ich auf den
freien Platz neben Hardie und mir. »Setzt euch, Kameraden! Was hat
euch in diesen bunten Kreis geführt?«

»Wir sind ehrliche Straßenräuber«, erwiderte Minoe prompt und ließ
sich an meiner Seite nieder, »denn wir nehmen den Reichen und geben
den Armen, wobei durch einen reinen Zufall die bedachten Armen wir
selber sind. Auch Deserteure waren wir einmal, aber das lag nur
daran, dass die Armee darauf beharrte, die Saturn-Kolonien zu
unterwerfen, ein Unterfangen, das wir zu Recht für aussichtslos
hielten. Immerhin gab es bei der Schlacht um den Japetus
zehntausend Tote, unter denen wir uns nicht befanden – das spricht
für meinen Weitblick und Alcesters Folgequalitäten.

Was nun allerdings diese Fahrt zum Plutotrabanten betrifft – da
muss ich gestehen, dass es mir an Kenntnissen mangelt, denn als
Æthermann war ich bisher noch nie unterwegs. Wenn ich mir aber Eure
Hände anschaue, Euer sonnengebleichtes Haar und Eure Kleidung mit
den weiten Hosen und dem schmal geschnittenen Hemd, so denke ich,
Ihr werdet meinen Mangel an Kenntnissen gewiss ausgleichen können,
oder, mit einfacheren Worten ausgedrückt: Ihr seid ein Æthermann,
wie er im Buche steht, und gewiss in der Lage, uns armen Ætherküken
auf die Sprünge zu helfen.«

»Er meint«, dolmetschte der Riese, während er sich neben Hardie zu
Boden sinken ließ und damit eine leichte Erschütterung verursachte,
»dass wir froh wären, wenn ein Æthermann uns zur Seite stünde.«

Einer allzu raschen Antwort auf diese Anfrage wurde ich enthoben,
denn wieder schwang die Tür auf, und ein Verurteilter wurde
zwischen uns geworfen.

Der Richter schien mit seiner Ankündigung Ernst zu machen und
sämtliches Gesindel des Empires an einem einzigen Arbeitstag in
Richtung Charon entsorgen zu wollen, denn so rasch öffnete und
schloss sich die Tür und so viele Verdammte traten in unsere enge
Zelle ein, dass weitere Gespräche rasch unmöglich wurden. Nicht
anders als Vieh fanden wir uns zusammengedrängt, die Luft wurde
dick, und Hardie, der wieder zu husten begonnen hatte, sackte
schwer gegen meine Schulter.

\bigpar

Minoe wies auf ihn. »Ist dein Freund überhaupt in der Lage, den
Gang zum Wächter der griechischen Unterwelt anzutreten? Wird er es
bis aufs Schiff schaffen?«

»Niemand wird zurückgelassen«, schnappte ich, »merk dir das ein für
allemal. Wenn wir eine Messe bilden, dann zu viert, nicht zu
dritt.«

Begütigend streckte Minoe die Hand aus. »Ganz wie du willst,
Kamerad. An mir soll’s nicht liegen. Aber vor der Abreise wird der
Schiffsarzt ihn begutachten, und ich glaube kaum, dass er eine
Seuche an Bord einschleppen will.«

In diesem Punkt hatte er unzweifelhaft Recht, aber noch ehe ich mir
Gedanken über das Problem machen konnte, war es auch schon soweit.
Erneut wurde die Tür aufgerissen, aber dieses Mal war es kein
Unglücklicher in Halseisen, für den sie sich öffnete. Statt dessen
blieb sie weit offen, eine Reihe von Soldaten stellte sich im
Spalier davor auf, die Gewehre im Anschlag, und eine barsche Stimme
rief: »Einzeln heraustreten!«

Diejenigen, die der Tür am nächsten standen, hatten Mühe, diesem
Befehl zu folgen, denn aus der überfüllten Zelle wurden sie beinahe
mit Gewalt herausgepresst. Aber die Soldaten schoben sie wieder
zurück, und nun begann die Prozedur des Verladens. Einer nach dem
anderen musste seinen Gang durch das Spalier antreten, wurde an
dessen Ende vom Schiffsarzt in Empfang genommen und auf ansteckende
Erkrankungen untersucht und fand sich schließlich ausgesondert und
zurück in die Eingeweide von Cape Ivory Castle verbannt oder mit
einer Fußfessel an seinen Nachbarn gekettet und auf dem Weg zum
Ætherhafen. Ich schob Hardie, Minoe und Alcester so weit wie
möglich nach hinten. Je länger diese Prozedur dauerte, desto
ungeduldiger würde der Kapitän der \schiff{Mary Jane} werden, der den
morgendlichen Sonnenwind ausnutzen wollte, und desto flüchtiger
müssten die Untersuchungen des Schiffsarztes ausfallen.

Das Abwarten war riskant, denn Hardie hatte wieder begonnen zu
fiebern, er hing beinahe besinnungslos in meinen Armen und würde
auf keinen Fall den Gang durch das Spalier bewältigen. Aber in
meinen beinahe dreißig Ætherjahren hatte ich alle Arten von
Kapitänen kennen gelernt, fähige und unfähige, und in einem Punkt
waren sie alle gleich ...

\bigpar

Als die Zelle beinahe leer war, hastete ein Unteroffizier herbei,
ein rotwangiges Jüngelchen, das vom Laufen erhitzt war. Mühsam kam
er vor dem Schiffsarzt zum Stehen und salutierte. »Unteroffizier
Dunleavy, Sir, Doktor Wilson, Sir, und der erste Maat Scanolon
lässt fragen, wann Sie mit der Untersuchung fertig sind, Sir, denn
die \schiff{Mary Jane} ist zum Ablegen bereit und die Sonnenflut
wartet auf niemanden, nicht einmal auf Schiffsärzte, Sir.«

Ich spitzte die Ohren. Weshalb führte auf diesem Schiff der Erste
Maat das Kommando? War der Kapitän erkrankt, oder gehörte er zu der
bequemen Sorte, die alle Befehle so lange delegierte, bis niemand
mehr gehorchte? Immerhin ließ sich mit dieser Information etwas
anfangen. Aber während ich noch darüber nachdachte, hatte der
Schiffsarzt, ein nervöser, rundlicher Mann mit flinken Augen, seine
Entscheidung getroffen.

»Bringt den Rest zu mir!«, ordnete er an, und das letzte Häuflein,
außer uns Vieren noch eine Handvoll Æthermänner, wurde von den
Soldaten zu ihm getrieben.

Aufmerksam ging er die Reihe ab und blieb, wie ich befürchtet
hatte, vor Hardie stehen. »Was ist mit dem?«

»Der markiert!«, sagte Minoe sofort und grinste von einem Ohr zum
anderen. »Eben war er noch munter wie ein Täubchen und wollte mit
mir wetten, dass er ausgesondert wird. Aber so leicht lässt sich
James Minoe nicht um seine Gürtelschnalle bringen, Sir, da muss er
schon früher aufstehen, Sir, und nicht nur ein Stück Seife
hinunterwürgen, damit seine Stirn sich heiß anfühlt.«

Misstrauisch musterte der Schiffsarzt Hardie und legte ihm die Hand
auf die Stirn. »Er ist tatsächlich sehr heiß. Nun gut, wir werden
sehen. Wenn er tatsächlich Seife gegessen hat, um sich krank zu
stellen, dann wird er im Verlauf des Tages wieder auf die Beine
kommen, und dann blüht ihm die Peitsche wegen des
Täuschungsversuchs. Bringt ihn an Bord!«

Beinahe hätte ich gelacht. Wie konnte der Schiffsarzt eine so dumme
Geschichte glauben? Seit Monaten war kein Stück Seife mehr in
unsere Hände gelangt, und dass man einen Fieberanfall markieren
konnte, indem man Seife aß, hielt ich ohnehin nur für ein Gerücht.
Aber Minoe mit seinem offenen Gesicht und seinen flinken Worten
konnte die Menschen anscheinend dazu bringen, alles Mögliche zu
glauben. Auch das merkte ich mir gut und hoffte, dass es mir später
zugute kommen würde.

Ich hielt mich dicht an Hardies Seite, und es gelang mir, mit ihm
zusammengekettet zu werden. Das war gewiss kein Spaß, denn von nun
an musste ich selbst die kleinste Bewegung mit Rücksicht auf meinen
Kettengefährten ausführen. Schon an einen Gesunden gekettet zu
sein, der die Rücksicht erwidert, ist nicht einfach, aber einen
Kranken mit sich zu schleppen, der mal taumelt und einen beinahe
von den Füßen reißt, mal stehen bleibt, sodass man aus dem Tritt
kommt, und kurz darauf jäh zur Seite ausbricht, ist beinahe
unmöglich. Aber Hardie war nun einmal mein Steuermann, ich selbst
beherrschte diese Kunst nur unzureichend, und deswegen konnte ich
es mir für den Fall, dass ich noch einmal davonkommen würde, nicht
leisten, auf ihn zu verzichten. Ganz gewiss geschah es nicht aus
Selbstlosigkeit oder Menschenliebe, denn so eine Empfindung vergeht
einem leicht auf dem endlosen Weg durch die unterirdischen Gänge,
und erst recht wenn man die engen Treppen hinauf stolpert. Drei
Versuche benötigten wir für die erste, weil Hardie immer wieder
hinunterstürzte, mich im Fallen mit sich riss und alle, die hinter
uns gingen, ebenfalls zum Stolpern brachte. Danach war ich klüger
geworden, ich packte Hardie, schob ihn, so gut es ging, vor mir her
und stieß ihm mein Knie ins Kreuz, wenn wir die Stufen hinauf
mussten.

\bigpar

Durch eine Seitentür ging es aus dem Fort hinaus, unvermittelt
traten wir ans Tageslicht, das erste, das wir seit fast vier
Monaten sahen. Die plötzliche Helligkeit blendete uns, einige
brachen in die Knie, andere tappten herum, als seien sie blind,
jeder zog dabei seinen Kettengenossen mit und der ganze Haufen
geriet in ein solches Durcheinander, dass ich lachen musste. Die
Soldaten sortierten uns mühsam mit ihren Bajonetten und fluchten
über die verlorene Zeit, der junge Unteroffizier bekam einen roten
Kopf und der Schiffsarzt wischte sich den Schweiß von der Stirn.

Endlich ging es weiter, durch das Tor und den abschüssigen Weg zum
Strand hinunter. Die \schiff{Mary Jane} war kein großes Linienschiff,
sondern eine der kleinen, dreimastigen Æthergalleys mit
dampfbetriebenem Heckpropeller, die man für die Transporte von
Gefangenen und Sklaven benutzte. Dennoch konnte sie nicht direkt am
Strand ankern, da das Wasser zu seicht war. Daher wurden wir in
Boote verladen, und das gab uns schon einen kleinen Vorgeschmack
auf Kapitän und Mannschaft. Damit die Matrosen nur ja nicht einmal
häufiger rudern mussten als unbedingt notwendig, wurden wir in
Lagen in die Boote gepfercht, dass sie tief im Wasser lagen und
beinahe kenterten. So viel sie auch an uns verdienen würden –
offensichtlich war niemand an Bord dieses Schiffes dazu bereit, auf
unser Wohlergehen zu achten. Auch das war eine wichtige
Information, die zum Bösen oder zum Guten ausschlagen konnte, je
nachdem, wie man sie nutzte.

Den Landratten unter uns war schon der Wellengang beim Übersetzen
zu viel, mehr als einer wurde bleich und begann zu schlucken. Auch
Minoe gehörte dazu, ihm hatte es die Worte verschlagen, und das war
ein schlechtes Zeichen. So hatte ich in meiner Messe gleich zwei
Kranke am Hals und keine Ahnung, was ich mit ihnen anfangen
sollte.

Wir dockten an und wurden über Bootsmannssitze an Bord gehievt,
eine schaukelnde Konstruktion aus Tampen und Brettern, aus der wir
wie Stückgut an Bord plumpsten. Noch ehe wir uns aufraffen konnten,
trieben uns die Matrosen schon weiter unter Deck, steile
Niedergänge hinab, die wir aneinander gekettet bewältigen mussten,
und bis ins stickige Orlopdeck, das unterhalb der Wasserlinie lag.
Der Boden bestand zum größten Teil aus hölzernen Gitterrosten, die
halb verfault waren und nach Schimmel stanken. Dies würde für die
Reise zum Charon unsere Unterkunft sein.

Nachdenklich betrachtete ich die versprengte Schar. Es handelte
sich um ungefähr fünfzig Gefangene, die mit mir in Ketten lagen.
Etwa zwei Drittel von ihnen hatten das weißblonde Haar und die
kupfergetönte Haut von befahrenen Æthermännern. Die übrigen waren
blasshäutig und mehr oder weniger dunkelhaarig. Ihnen stand die
Ætherkrankheit bevor. Das einzige mir bekannte Mittel dagegen war
viel Bewegung an Deck oder in den Wanten. Hier unten im Orlop,
zusammengepfercht wie die Heringe im Fass, hatten die Frischlinge
keine Chance und würden sterben wie die Fliegen.

\bigpar

Der erste Maat schien es mit dem Aufbruch tatsächlich sehr eilig zu
haben, denn kaum waren wir mehr oder weniger untergebracht, verriet
mir ein Rucken tief in den Eingeweiden des Schiffes und das dumpfe
Grollen der frühzeitig anlaufenden Kolben, dass wir ablegten. Die
Befehle und Bewegungsabläufe an Bord kannte ich so gut, dass ich
sie im Geiste als Bilderbogen vor mir sah.

»Klar zum Hieven des Ankers!«

Auf das Kommando des Bootsmanns steckten die Ætherleute lange
hölzerne Stangen, die Spaken, in die Aussparungen an den Seiten des
Spills und lehnten sich dann mit aller Kraft dagegen, um es in eine
drehende Bewegung zu versetzen. Der Fiedler, beweglicher als ein
Äffchen, sprang obenauf und gab mit einem Shanty den Takt an.

»Anker Ætherfest machen und amperisieren!«

War der Auriferrumanker erst einmal aus dem Wasser und in der
magnetischen Aufladestation, mussten die Æthermänner ihn so
befestigen, dass er mit wenigen Griffen zum Fallen gebracht werden
konnte. Bei einem plötzlichen Sonnensturm, nicht unüblich auf
unserer Route und zu dieser Jahreszeit, konnten Schiff und
Besatzung davon abhängen, dass der Anker magnetisch geladen und
ohne jede Verzögerung einsatzbereit war. Ich fragte mich, wie
gewissenhaft diese Mannschaft arbeitete, und hätte beinahe alles
dafür gegeben, um einen Blick auf den Ankerstropp und die
angeschlossenen Kupferkabel werfen zu dürfen.

»Setzt die Segel!«

Ein heftiger Schlag ging durch den Rumpf und verriet mir, dass der
Matrose im Vortopp geschlafen und zu wenig Segelfläche gesetzt
hatte. Als daraufhin im Großtopp die Segel gesetzt wurden, war die
\schiff{Mary Jane} luvgierig geworden. Sie brach aus und musste
gegengesteuert werden. Ich schnalzte mit der Zunge. Dafür würde
dieser Pfuscher die neunschwänzige Katze bekommen, soviel war
sicher. Nun trug der Wind uns rasch aufs offene Meer hinaus.

»Maschine auf volle Kraft! Kurs fünf Strich sonnenauswärts!«

Das Stampfen der Kolben in den Zylindern der Dampfmaschine nahm zu.
Ein Ächzen und Vibrieren ging durch das ganze Schiff, als eine
dichte Qualmwolke, durchsetzt mit einem glühenden Funkenregen, aus
dem Schornstein stob. Dröhnend lief der aus Auriferrum gefertigte
Propeller an und gab den zusätzlichen Impuls für den Aufstieg.

Um das Schiff sauber aus der Erdanziehung zu lösen, mussten wir
zunächst in eine elliptische Umlaufbahn gelangen und genügend
Schwung für den Absprung holen, bevor wir direkten Kurs auf das
äußere Sonnensystem nahmen. Die durch den Wellengang
hervorgerufenen Schaukelbewegungen endeten abrupt, als wir uns aus
dem Wasser lösten und uns in die Lüfte erhoben. Der Aufwind griff
nach uns und trug uns in einem sanften Wiegen empor.

Ich atmete tief durch. Mit jeder Meile, die wir uns dem Æther
näherten, wurde mir wohler zumute. Die Angst und Beklemmung, unter
der ich auf der Erde und zumal in ihren Eingeweiden in den
Verliesen von Cape Ivory gelitten hatte, lösten sich, und ich
spürte, wie mein Mut und meine Zuversicht zurückkehrten. Hardie
neben mir atmete ebenfalls auf. Er schien nun leichter zu atmen,
das Rasseln in seiner Brust ließ nach und seine Stirn wurde
merklich kühler.

Minoe, der kreideweiß an meiner Seite hockte, erbrach sich neben
meine Füße.

»Hat dieser Sonnensturm bald ein Ende?«, würgte er hervor.

»Was für ein Sturm?«

»Der Sonnensturm! Ich habe davon gehört, aber bei den Göttern
meiner Vorfahren, ich hätte nie gedacht, dass er so heftig sein
könnte. Der Boden wankt unter meinen Füßen, ich finde keinen
sicheren Stand mehr. Ihr seid ein Æthermann, Ihr müsst doch wissen,
wie lange so etwas dauert!«

»Nicht mehr lange«, erwiderte ich mitleidig und legte den Arm um
ihn. »Ganz gewiss nicht mehr lange.«

\bigpar

Bald darauf kam der schlimmste Moment für die Unbefahrenen. Die
\schiff{Mary Jane} durchbrach die Atmosphäre und trat in den Æther
ein. Als ich Minoe zitternd zu meinen Füßen kauern sah, erinnerte
ich mich nur allzu deutlich an meinen eigenen ersten Ætherritt. Der
Boden war mir entgegengekommen und hatte sich auf die unmöglichste
Weise umgestülpt. Die dumpfen Geräusche aus dem Schiffsrumpf hatten
mir erschreckende Trugbilder vorgegaukelt. Dazu kam die
allgegenwärtige Übelkeit, als alle mir bekannten Dimensionen ihre
Verbindlichkeit verloren.

»Porch, in welche Richtung sind wir unterwegs?«, flüsterte Minoe,
den Kopf in Alcesters Schoß gebettet. Den entstellten Riesen hatte
die Ætherkrankheit weitaus weniger heftig erwischt als seinen
Freund, und gerade das schien ihm über alle Maßen zu schaffen zu
machen, da er es gewohnt war, dass Minoe ihm das Denken abnahm.

»Wenn ich nur wüsste«, murmelte Minoe, »ob es nun bergauf oder
bergab geht, dann könnte ich meine Sinne entsprechend ausrichten,
wenn Ihr versteht, was ich meine. Aber so, wie es nun einmal ist,
kann ich weder mich selbst noch irgendeinen Gedanken justieren.«

Er driftete in Fieberphantasien davon, und ich konnte nicht mehr
für ihn tun als nur zu hoffen. Ein Blick in die Runde überzeugte
mich davon, dass es nicht nur ihm allein so erging. Ungefähr
zwanzig andere lagen bleich auf dem feuchten Holz, stöhnten und
gaben das Wenige, was ihr Magen enthielt, unter Krämpfen von sich.
Die Befahrenen fluchten und rückten zur Seite, soweit ihre Ketten
es zuließen, aber sie hatten die Ætherkrankheit auf ihrer ersten
Reise selbst durchlitten und wussten, dass es kein wirksames Mittel
dagegen gab. Ein Großteil der Kranken würde sterben, es sei denn,
man schaffte sie an Deck.

Wieder warf ich einen Blick auf Minoe. Seine Haut war wächsern,
sein Puls flatterte wie ein gefangener venusischer Schmetterling.

\bigpar

Am Niedergang gab es einige Bewegung. Ein hagerer Priester hatte
das Orlop betreten und schritt zwischen unseren Reihen herum wie
ein Storch auf einer Feuchtwiese. Zielstrebig steuerte er die
Kranken an und begann leise auf sie einzureden, wobei seine Nase
auf und ab nickte wie ein langer Schnabel. Aber er kam zu früh,
außer Minoe war noch niemand nahe genug an der Schwelle des Todes,
um ihm Aufmerksamkeit zu schenken. Die Unbefahrenen wandten sich
ab, die Æthermänner grinsten nur und winkten ihn weiter. So stakste
er schließlich zu unserem kleinen Grüppchen, faltete die Hände und
neigte sich zu Minoe und Hardie herunter.

»Mein Name ist Vater Cassidy«, begann er mit unangenehm drängender
Stimme. »Ich sehe, meine Kinder, dass die Stunde naht, in der ihr
vor eurem Schöpfer stehen werdet.«

Hardie hustete und lächelte schwach, Minoe aber schrak im
Fieberschlaf zurück und klammerte sich an Alcesters Arme.

»Worin siehst du das, Cassock?«, fuhr ich dazwischen. »Im
Kaffeesatz vielleicht? Oder in deiner Kristallkugel?«

»Mein Name ist Cassidy«, berichtigte er ungerührt, »und diese
Männer, das erkennst du selbst, sind dem Tode geweiht.«

»Das sind wir alle, früher oder später«, schnappte ich.

Er lächelte ölig. »Mein Sohn, noch ist es Zeit, zu bereuen und
umzukehren.«

Aus den Augenwinkeln bemerkte ich, dass wir mittlerweile die
Aufmerksamkeit des gesamten Orlopdecks auf uns gezogen hatten. Das
war nun eine Situation ganz nach meinem Geschmack, und ich war
entschlossen, meinen Kameraden eine hübsche Szene zu bieten.

»Umkehren, Cassock?«, brüllte ich. »Was für eine grandiose Idee!
Sag dem Ersten Maat, er solle das Schiff wenden und uns nach Hause
zurück bringen.«

Die Männer grinsten, aber so leicht war der Priester nicht aus dem
Konzept zu bringen.

»Umkehren und euch zu Gott bekehren, das ist es, was ihr tun
sollt«, fuhr er fort. »Die Qualen Charons sind nichts verglichen
mit den ewigen Qualen der Hölle!«

»Woher weißt du das, Cassock?«, stichelte ich. »Warst du schon
einmal da?«

»Mein Name ist Cassidy«, wiederholte er mit all der frommen Geduld,
die man ihn im Priesterseminar gelehrt hatte. Noch immer hielt er
die Hände gefaltet, aber zu meiner Genugtuung bemerkte ich, dass
seine Knöchel langsam weiß wurden. »Mein Sohn, du hast große Schuld
auf dich geladen, und nur der Glaube und die Reue können dich davon
befreien.«

»Du nennst mich ... deinen Sohn, Vater Cassock?«, fragte ich und
ließ meine Stimme ein wenig schwanken. »Kann das ... kann es die
Wahrheit sein?«

Er neigte sich vor, glaubte mich endlich am Haken zu haben, da fuhr
ich fort: »Nun, undenkbar ist es immerhin nicht, denn meine Mutter
war eine Hure im besten Bordell von Kilkenny!«

\bigpar

Unter unserem dröhnenden Gelächter floh er aus dem Orlop.

\bigpar

Sogar Hardie wischte sich eine Lachträne aus dem Augenwinkel, bevor
er wieder trocken zu husten begann. »Dem hast du es sauber gegeben,
Porch, der macht sich hier nicht noch einmal lästig. Aber was ist
mit unserem neuen Freund, Minoe? Er sieht schlecht aus, die
Ætherkrankheit hat ihn schlimm erwischt.«

Ich nickte. »Er muss dringend an Deck, sonst überlebt er es
nicht.«

»An Deck?«, rief Hardie. »Du hast wohl selbst Fieber, sonst
wüsstest du, dass wir Gefangene sind. Schau dir die Ketten an! Wir
können nicht an Deck wandern, wie es uns passt! Und der dort ist
nicht mehr zu retten, schau ihn doch an!«

Minoes Stirn glänzte noch immer wie eine Opferkerze, er warf sich
unruhig hin und her.

Ich zog Hardie hoch, bis sein Gesicht dicht an meinem war.
»Niemand«, knurrte ich, »niemand aus meiner Messe wird
zurückgelassen! Er hat deinen Hals gerettet, als dieser Schiffsarzt
Wilson dich in Cape Ivory Castle ausmustern wollte, und ich habe
dich an Bord geschleppt, obwohl du ein rechter Klotz am Bein warst.
Wir müssen ihn an Deck schaffen, auf irgendeine Weise.«

»Der Schiffsarzt«, sagte Hardie, ungerührt von meinem Ausbruch.
»Wenn wir den überzeugen können ... Was meinst du, hat er schon
einige Ætherreisen hinter sich?«

Ich rief mir die Erinnerung an den rundlichen, dunkelhaarigen Mann
ins Gedächtnis zurück. »Keine einzige«, erwiderte ich dann. »Der
Kerl ist genauso ein Weltraumfrischling wie Minoe hier, und es
würde mich nicht wundern, wenn ihn die Ætherkrankheit ebenfalls
erwischt hätte. Glaubst du, man kann mit ihm reden?«

Hardie wiegte den Kopf hin und her. »Meistens ist der Schiffsarzt
am Erlös einer Sklavenladung beteiligt, nicht wahr?«, überlegte er.
»Es kann also nur in seinem Sinne sein, möglichst viele der
Gefangenen wohlbehalten ans Ziel zu bringen. Doch, ich denke, man
kann mit ihm reden. Wir sollten den Matrosen, der uns das Essen
bringt, nach dem Arzt schicken. Einen Versuch ist es immerhin
wert.«

\bigpar

Auf unser Essen mussten wir lange warten, beinahe so, als hätten
die Offiziere sich nicht einigen können, was sie uns geben sollten.
Endlich erschien ein übler Geselle mit dem weißblonden Haar und der
dunklen Haut der Ætherleute, verdreckt und schlampig gekleidet. Er
trug einen dampfenden Kessel vor dem Bauch, der zu meiner
Enttäuschung nicht besonders schwer zu sein schien.

»Stellt euch ordentlich hintereinander auf, dann bekommt ihr etwas
zu essen!«, witzelte der Æthermann. Aber wir waren ausgehungert und
nicht in der Stimmung, auf seine Scherze einzugehen.

»Nun gut«, legte er nach, »dann reicht wenigstens eure Teller nach
vorn durch. Ach, Teller habt ihr wohl auch nicht? Ihr Charonsklaven
habt eben keine Manieren!«

Er schöpfte mit einer kleinen Kelle einen undefinierbaren,
widerlich stinkenden Brei aus dem Kessel und ließ ihn einem der
Gefangenen in die erhobenen Hände rinnen. Der leckte ohne Protest
seine Finger ab und streckte die Arme nach mehr aus.

»Nichts da!« Der Æthermann versetzte ihm einen Tritt. »Hier kommen
alle der Reihe nach dran, und gebettelt wird nicht.«

Wir verharrten in eisigem Schweigen. Er hatte es in der Hand, uns
leben oder verhungern zu lassen, und er genoss seine Macht. Einem
nach dem anderen löffelte er den Brei in die Finger und ging
Schritt für Schritt durch die Reihen.

Ich wartete, bis Hardie, Minoe und Alcester ihren Anteil bekommen
hatten, dann meldete ich mich zu Wort. »Bitte, Sir, wann schaut der
Arzt hier vorbei?«

Der Æthermann wandte sich um und blinzelte mich mit seinen
Schweinsäuglein an. »Den Arzt möchtest du sprechen? Befindest du
dich nicht wohl?« Er trat gegen meine Ketten und wollte sich
ausschütten vor Lachen.

»Bitte, Sir«, beharrte ich, »soweit ich weiß, muss jeden Tag der
Arzt nach uns sehen, um ein Ausbreiten von Seuchen zu verhindern.
Wann schaut er hier vorbei?«

Er zuckte die Achseln. »Vielleicht morgen. Vielleicht nächsten
Monat. Er hat die Ætherkrankheit und kann sich nicht rühren. Du
wirst dich gedulden müssen. Wenn du allerdings darauf bestehst,
kann ich dir gern das Fell gerben – dann kommst du gleich vorn auf
die Liste!«

Ich hob den Fuß und trat kräftig gegen seinen Kessel, dass es nur
so schepperte und der Brei über den Rand schwappte.

»Das war ein tätlicher Angriff eines Gefangenen auf ein Mitglied
der Mannschaft«, sagte ich mit Bedacht. »So ein Vorfall muss auf
der Stelle dem Kapitän gemeldet werden. Die Schuld trifft allein
mich, ich habe keine Mitverschwörer, und ich bin bereit, die
Bestrafung auf mich zu nehmen. Wenn ich also bitten dürfte!«

Fassungslos glotzte der Æthermann mich an und wischte die
breiverschmierten Hände an seiner Hose ab. »Scanolon wird dir zwölf
Hiebe aufzählen, ist dir das klar?«

»Sir«, erwiderte ich, »ein tätlicher Angriff eines Gefangenen muss
dem Kapitän gemeldet werden – nicht dem Ersten Maat.«

»Wie du meinst! Der alte Brassbow wird dich nicht einmal ansehen,
und für diese Frechheit bekommst du von Scanolon volle
vierundzwanzig. Die Kette kann ich allerdings nicht aufschließen,
deswegen musst du deinen Kumpan wohl oder übel mitnehmen. Also los,
vorwärts!«

\bigpar

Der Æthermann trieb uns beide mit dem Kessel vor sich her, die
steilen Stufen hinauf, die wir vor wenigen Stunden so mühsam
hinuntergeklettert waren, und zur Kapitänskajüte. Dort klopfte er
an und meldete: »Tompkins, Sir, Kapitän Brassbow. Ich muss einen
tätlichen Angriff melden.«

Es verging eine ganze Weile, bevor ein Bediensteter die Tür
öffnete. »Kapitän Brassbow ist beschäftigt. Wenden Sie sich an den
Ersten Maat.«

Tompkins zuckte die Achseln. »Na, siehst du. Abmarsch.«

»Kapitän Brassbow, Sir!«, rief ich durch die geöffnete Tür. »Hören
Sie mir zu, oder Sie werden mit einer Ladung Leichen auf dem Charon
ankommen!«

Etwas knarrte, als stemme der Kapitän sich mühsam aus der
Hängematte hoch.

»Dein Name, Gefangener!«

»Porch, Sir, Joshua Porch. Ich bin über dreißig Jahre im Æther
gefahren, Sir. Ich weiß, was die Ætherkrankheit anrichtet. Unten im
Orlop haben wir beinahe zwanzig Unbefahrene. Wenn sie keinen
Ausgang an Deck bekommen, werden sie sterben.«

»Dann werden wir ihre Körper über Bord werfen«, erwiderte der
Kapitän mit träger Stimme, »und mit dem elenden Rest von euch
weiterfahren.«

»Sir, mit Verlaub, es ist schon jetzt unerträglich dort unten. Es
werden Seuchen ausbrechen, und Ihr Schiffsarzt ist unpässlich und
kann niemanden behandeln. Jeden Morgen und jeden Abend zehn Minuten
an Deck, in Ketten, wenn Sie darauf bestehen – das ist die einzige
Möglichkeit, den Tod Ihrer gesamten Ladung zu verhindern. Und den
Arzt sollten Sie auch aufscheuchen, damit er sich draußen bewegt.«

»Genug!«, rief Brassbow. »Bringen Sie ihn zurück unter Deck!«

»Aber der tätliche Angriff ...« protestierte Tompkins.

»Sind Sie etwa verletzt? – Na also, dann lassen Sie mich mit diesen
Kinkerlitzchen zufrieden und bringen Sie die Gefangenen ins Orlop,
wo sie hingehören!«

Die Tür wurde zugeschlagen, genau vor Tompkins’ Nase. Der Æthermann
schaute einigermaßen verdattert drein. Aus dem Augenwinkel sah ich
Hardie breit grinsen und musste mir selbst das Lachen verkneifen.

»Aufgeschoben ist nicht aufgehoben«, tröstete ich. »Bestimmt ergibt
sich noch eine andere Gelegenheit, mir die Hiebe aufzuzählen.«

»Da kannst du sicher sein!«, knurrte Tompkins und stieß uns die
Treppe hinunter, während wir lauthals zu johlen begannen.

Meine Argumente schienen Brassbow überzeugt zu haben, auch wenn er
es niemals zugegeben hätte. Gleich am folgenden Morgen erschien der
Schiffsarzt Wilson bei uns. Er hatte die Nacht an Deck verbracht,
das war unverkennbar, denn sein Haar war deutlich heller geworden
und die Ætherkrankheit war vergangen, er schien nur noch ein wenig
blass um die Nase zu sein. Dennoch war er missgelaunt, denn die
neue Order bürdete ihm deutlich mehr Arbeit auf.

»Ah, der Seifenesser!«, brummte er und stieß Hardie mit dem Fuß in
die Seite. »Erinnere mich daran, dass ich dich dem Ersten Maat
melde, damit du eine Abreibung bekommst, weil du dich krank
gestellt hast.«

Dann wandte er sich an Minoe. »Dafür markiert jetzt dieser hier!«

»Mit Verlaub, Sir«, wandte ich ein, »er hat die Ætherkrankheit.
Daran sollten Sie sich noch gut erinnern. Er muss an Deck.«

»Wie soll das gehen? Er kann nicht mehr laufen!«

»Aber ich kann ihn tragen«, erklärte Alcester und lud sich Minoe
trotz der Ketten ohne weitere Umstände auf die Schultern. Sehr
langsam, aber unaufhaltsam wie eine Naturgewalt, schritt er zu den
Stufen und begann sie zu erklimmen. Wilson schaute ihm mit offenem
Mund nach, bis ich ihn in die Seite stupste.

»Gilt der Ausgang für alle?«, fragte ich.

»Immer nur fünf Paare zugleich«, erwiderte Wilson, während er die
Augen nicht von Alcester und Minoe wandte, »und immer nur für zehn
Minuten, und die Ketten werden nicht gelöst. Anordnung von
Scanolon.«

\bigpar

Schon wieder dieser Erste Maat!

\bigpar

»Seit wann«, fragte ich, »hat auf der \schiff{Mary Jane} der Erste
Maat das Kommando?«

Diese direkte Frage brachte ihn ein wenig aus dem Gleichgewicht,
denn er war ein schlechter Lügner. »Mir ist zu Ohren gekommen,«,
sagte er schließlich, »dass Kapitän Brassbow vor einem Jahr während
einer Zwischenlandung auf dem Ganymed schwer erkrankt ist,
vermutlich war es das marsianische Sumpffieber. Als die
\schiff{Mary Jane} ablegte, gab er sinnlose Anweisungen, fuchtelte
mit seiner Pistole herum und hätte das Schiff unzweifelhaft auf
einen Asteroiden auflaufen lassen, wenn nicht Scanolon gewesen
wäre. Er war der Einzige, der mit dem Kapitän reden konnte. Nach
Brassbows Genesung ist es anscheinend dabei geblieben: Wann immer
er etwas anordnet, fragt die Mannschaft zur Sicherheit bei Scanolon
nach.«

Wilson seufzte. »Das wäre in Ordnung, wenn Scanolon ein guter Kerl
wäre. Aber leider ist er eine wahre Plage des Æthers. Er ist
geltungssüchtig, ehrgeizig, verlogen und falsch. Lieber hätte ich
eine Schiffsratte als Befehlshaber als ausgerechnet ihn. Nun,
immerhin ist er ein tüchtiger Æthermann. Wir werden also nicht
havarieren, das ist wenigstens etwas.«

Ich schnaubte hörbar. Ein schwacher Kapitän, ein ehrgeiziger Erster
Maat, der die Mannschaft beinahe vollständig hinter sich wusste –
diese Konstellation war mir schon häufiger untergekommen, und sie
hatte nie zu einem guten Ende geführt, weder für das Schiff noch
für die Æthermänner noch für die Ladung, zu der ich mich in diesem
Falle selbst zählen musste.

\bigpar

»Vermutlich wäre es besser, Wilson«, sagte ich schließlich, »wenn
du die Paare für den Ausgang einteilen würdest. Oder möchtest du
zur Sicherheit Scanolon befragen?«

Der rundliche Schiffsarzt riss sich zusammen. »Auf mein Kommando
also!«, rief er. »Zuerst werden die fünf Paare neben der Treppe an
Deck gehen!«

Damit hatte ich zumindest für ein wenig Bewegung im Orlop gesorgt
und mir selbst die Möglichkeit eröffnet, mich an Deck umzusehen.
Hardie und ich ließen den anderen großmütig den Vortritt. Alcester
kam zurück, mit Minoe neben sich, den er zwar noch stützen musste,
dessen Fieber aber deutlich gesunken war.

»Eine glänzende Aussicht dort draußen«, wisperte er, ein wenig
heiser, aber guter Dinge. »Nichts als kohlenschwarzer Æther
ringsherum, und hie und dort ein paar dottergelbe Sterne. Porch,
ich sage dir, wenn die Betten und die Bedienung besser wären, würde
ich den kleinen Ausflug genießen.«

»Was nicht ist, kann noch werden«, tröstete ich und machte mich
dann mit Hardie auf den Weg zum Niedergang.

\bigpar

An Deck war es kühl. Die Dampfmaschine lief nur auf halber
Leistung, denn die Sonnenwinde hatten aufgefrischt und trugen uns
rasch sonnenauswärts. Ich streckte meine Glieder. Auch Hardie neben
mir atmete tief durch.

»Es tut gut, wieder im Æther zu sein«, sagte er.

Ich nickte abwesend. »Ist das dort auf dem Achterdeck Kapitän
Brassbow?«

Der Mann wirkte gedrückt, unansehnlich, und hielt sich unbehaglich
im Hintergrund.

»Dann muss der neben ihm der allmächtige Scanolon sein«, fuhr
Hardie fort.

Ich drehte ein wenig den Kopf und musterte den Ersten Maat. Seine
Kleidung war mehr als ordentlich, frisch gestärkt und mit präzisen
Bügelfalten. Er hatte das Aussehen und die Haltung eines
langjährigen Æthermannes, seine Züge waren hart und mitleidlos, und
seine Augen schienen überall zu sein. Eben jetzt rüffelte er
unbarmherzig den jungen Dunleavy wegen einer geringfügigen
Unaufmerksamkeit.

\bigpar

Hardie schüttelte den Kopf. »Wenn der hier zu sagen hat, geht es
für uns nicht gut aus.«

»Du meinst also«, überlegte ich laut, »im Zweifel sollten wir uns
auf die Seite des Kapitäns schlagen?«

Er blickte mich verdutzt an. »In welchem Zweifel?«

\bigpar

Am Niedergang entstand Unruhe. Tompkins stieg herauf und schleppte
ausgerechnet den kaum genesenen Minoe am Kragen mit sich.

»Tätlicher Angriff eines Gefangenen auf ein Mitglied der
Mannschaft!«, rief er laut, mit einem triumphierenden Seitenblick
auf mich. Ich blieb stehen, als habe mich der Blitz getroffen. Dass
er sich auf diese Weise rächen würde, indem er das schwächste Glied
meiner Messe attackieren würde, hatte ich nicht vorausgesehen.

Tompkins wandte sich direkt an Scanolon. »Sir, dieser Hundesohn hat
nach mir geschlagen!«

»Das hätte ich gern«, erwiderte Minoe ohne Zögern, »allein, meine
Rechte liegt immer noch in der Schlinge und die Linke hat nicht
genügend Kraft, um dem Abschaum der Menschheit dort, der sich uns
gegenüber als Demeter präsentiert, die Zähne einzuschlagen!«

Scanolon blickte ihn nicht einmal an. »Zwölf Hiebe!«, befahl er
kalt.

»Aber Sir!«, wandte Tompkins ein. »Es war ein tätlicher Angriff auf
meine Person!«

Hätte ich irgendeine Empfindung in Scanolons Augen gesehen, mir
wäre wohler gewesen. Aber er winkte nur ab. »Zwölf, sagte ich.«

»Sollte man nicht vielleicht überprüfen ...«, murmelte Brassbow.

»Mit Verlaub, nein, Sir. Der Gefangene war vorlaut und muss
bestraft werden.«

Brassbow senkte den Kopf und schwieg, und Tompkins zog mit seiner
Beute ab.

Was hätte ich dafür gegeben, an Minoes Stelle abgestraft zu werden!
Als Tompkins ihn hinunter brachte und Alcester mir einen
vorwurfsvollen Blick zuwarf, wurde mir übel vor Scham und Kummer.

Aber die Fahrt ging weiter, wir passierten die Marsmonde und
wandten uns dem Jupiter zu, und schließlich heilte Minoes Rücken,
wenn auch seine Augen verdunkelt blieben von der Demütigung.

\bigpar

Wieder einmal drehten wir unsere abendliche Runde auf dem Deck. Ich
schnalzte missbilligend mit der Zunge, als ich ein nachlässig
aufgeschossenes Falltau bemerkte. Wer in diese Schlinge trat, brach
sich unweigerlich den Hals. Auch ein umgestülpter Eimer, wohl von
der letzten Decksreinigung zurückgeblieben, zeugte von der laschen
Schiffsführung.

»Du würdest am liebsten wohl selbst das Kommando übernehmen«,
spottete Hardie, dem meine Blicke nicht entgingen.

»Die \schiff{Mary Jane} ist eine hübsche Galley«, stimmte ich zu.
»Sie hätte etwas Besseres verdient als diesen nachlässigen
Haufen.«

Hinter dem Mast passierten wir den Zweiten und den Dritten Maat.
Sie standen dicht beieinander und unterhielten sich flüsternd.
Irgendetwas fiel mir an ihrer Haltung auf, an der Wachsamkeit, mit
denen sie um sich schauten, und der betonten Unauffälligkeit, die
sie an den Tag legten, wenn Brassbow vom Achterdeck zu ihnen
herüberschaute.

Die beiden Offiziere waren verstummt, als der junge Dunleavy an
ihnen vorbeiging und sie vorschriftsmäßig grüßte. Jetzt nahmen sie
ihr halblautes Gespräch wieder auf, unbekümmert um die Gefangenen,
die in ihren Augen offensichtlich keinerlei Gefahr darstellten. Ich
lenkte Hardie vorsichtig ein wenig näher zum Mast und spitzte die
Ohren.

» ... der Schlüssel zur Waffenkammer?«, fragte der Zweite Maat.

»Den hat seit dem Ablegen Scanolon. Ich habe gesehen, wie er ihn
von Brassbow bekommen hat.«

»Wie sieht es mit den Matrosen aus?«

»Vernon ist auf Brassbows Seite, weil Scanolon ihn hat prügeln
lassen.«

»Jeder vernünftige Offizier hätte ihn prügeln lassen nach dem, was
er sich beim Ablegen geleistet hat! Die \schiff{Mary Jane} war
luvgierig wie eine besoffene alte Jungfer!«

Sie warfen Hardie und mir einen Seitenblick zu, und ich schwenkte
sofort wieder in die Runde ein. Ungeduldig zog ich Hardie voran,
bis wir wieder am Mast vorbeikamen.

» ... Faulkner, Mayhew, Spehan«, zählte der Dritte Maat auf. »Aber
auch nur, weil sie zu faul sind. Alle anderen stehen bereit und
warten nur auf Scanolons Kommando.«

»Dann also während der Abendwache. Das erste Kommando nimmt das
Kanonendeck ein, das ... Vorsicht! Glauben Sie, die Gefangenen dort
können uns hören?«

Der andere zuckte die Achseln. »Und wenn schon! Was könnten sie
wohl unternehmen? Der Charon ist ihnen sicher, verkauft werden sie
so oder so, ob nun von Brassbow oder von uns.«

Ich zog Hardie wieder eine halbe Runde weiter, dann ließ ich mich
unvermittelt fallen.

\bigpar

»Ich sterbe!«, brüllte ich und brach wie ein getroffener Stier
zusammen. »Diese Krämpfe! Diese entsetzlichen Krämpfe!«

Aus den Augenwinkeln bemerkte ich, wie Hardie einen erschrockenen
Blick mit Minoe und Alcester wechselte und Anstalten machte, sich
über mich zu beugen. Rasch verdrehte ich die Augen und warf mich
auf die andere Seite.

»Ich sterbe! Holt Cassock, ich will die Beichte ablegen! Ich will
die Letzte Ölung! Schickt nach Cassock, ich will als guter Katholik
sterben!«

»Er ist nicht bei Sinnen!«, rief Hardie. »Holt Doktor Wilson,
rasch!«

»Aber wenn er doch nach Cassidy verlangt ...«, wandte Minoe ein.

Hardie schüttelte den Kopf. »Das ist der beste Beweis dafür, dass
er nicht bei sich ist! Los, schickt nach dem Doktor. Er soll sich
beeilen.«

Besorgt kniete er sich neben mich und versuchte meinen Kragen zu
öffnen, um mir Luft zu verschaffen. »Porch, was ist denn nur los?
Kann ich helfen?«

Ich schlang beide Arme um den Leib und warf mich wie ein Besessener
hin und her. »Die Flammen der Hölle!«, keuchte ich dabei. »Sie
lecken schon an meinem Körper! Ich will beichten! Vater, ich habe
gesündigt!«

»Wo bleibt der verdammte Doktor!«, brüllte Hardie verzweifelt.

Er wurde zur Seite geschoben, und ich fühlte die kühle Hand Wilsons
auf meiner Stirn. Stöhnend drehte ich den Kopf zur Seite und
bewegte die Lippen. Wilson beugte sich näher, um mich verstehen zu
können. »Meuterei!«, wisperte ich. »Wilson, ich muss den Kapitän
sprechen, so schnell wie möglich. Scanolon darf nichts davon
mitbekommen!«

Der Doktor begriff sofort. »Zurück!«, befahl er. »Er hat das
marsianische Sumpffieber, es ist hoch ansteckend.«

Alle prallten zurück, selbst der getreue Hardie neigte sich so weit
von mir weg, wie die Ketten es zuließen.

»Ich werde ihn in Quarantäne schaffen«, sagte Wilson, während er
sich schon bückte, um meine Eisen aufzuschließen. »Und ich muss den
Kapitän davon verständigen, denn es könnte bereits zu spät sein,
und dann droht uns eine Epidemie. Hardie, Ihr kommt mit mir. Als
sein Kettengenosse seid Ihr ohnehin so gut wie sicher angesteckt
und müsst ebenfalls in Quarantäne. Fasst ihn unter dem anderen
Arm!«

Während sie mich unterfassten, mich vom Deck schleiften und die
schmalen Treppen hinab schoben, verdrehte ich unablässig die Augen,
bewegte die Lippen und stöhnte leise. Erst in den Räumen des
Doktors wagte ich mich auf wundersame Weise zu erholen.

»Hardie, es tut mir leid, dass ich dir diese Komödie vorspielen
musste!«, sagte ich, während Wilson davon eilte, um Kapitän
Brassbow zu verständigen. »Vorhin an Deck habe ich eine Verabredung
zur Meuterei belauscht, die ich dem Kapitän melden muss. Ich weiß
nicht, wem ich vertrauen kann, außer dir und dem Doktor
natürlich.«

Hardie starrte mich einen Augenblick begriffsstutzig an, dann
begann er breit zu grinsen. »Ich hätte es merken müssen, als du
nach Cassock gebrüllt hast. Verdammt, du hast mir einen schönen
Schrecken eingejagt! Ich dachte schon, du müsstest den Löffel
abgeben.«

»Noch nicht!«, erwiderte ich, »aber es könnte knapp werden. Auf der
Seite des Kapitäns sind höchstens fünf oder sechs aus der
Mannschaft und außer Dunleavy kein einziger Offizier.«

Hardie ließ die Augenbrauen nach oben wandern. »Warum willst du es
dann melden?«

»Weil ...«, begann ich, aber in diesem Moment schwang die Tür auf,
und Brassbow höchstselbst stob herein, Wilson in seinem
Schlepptau.

»Wie ist dein Name, Gefangener?«, donnerte er.

»Joshua Porch, Sir.«

«Ah ja, ich erinnere mich. Ein besonders vorlautes Stück der
Ladung. Warum sollte ich dich überhaupt anhören?«

»Mit Verlaub, Sir, weil ich beim ersten Mal Recht hatte und nun
wieder Recht habe.«

Er schnappte nach Luft und starrte mich an. Nun, da ich ihm
persönlich gegenüber stand, bemerkte ich die Anzeichen von Alter
und Erschöpfung in seiner Erscheinung. Er war ebenso groß wie ich,
wirkte aber kleiner, denn er ging ein wenig gebeugt. In seinem
schlecht gepuderten Haar sah man einige graue Strähnen, seine
Kinnlinie begann schwammig zu werden, am Hals und um die Augen
zeigten sich die ersten Falten. Offensichtlich war sein Diener
nicht bei der Sache, denn das Hemd war nur nachlässig gebügelt und
der Rock nicht ausgebürstet.

»Meuterei also?« Brassbow kreuzte die Arme auf dem Rücken und
begann in dem engen Raum auf und ab zu gehen. »Wie kommst du
darauf?«

»Mit Verlaub, Sir«, sagte ich, »weil ich selbst lange Jahre Kapitän
meiner eigenen Ætherbrigg war und die Anzeichen einer Meuterei nur
allzu gut kenne. Ich habe ein Gespräch zwischen zwei Offizieren
gehört, und es ließ keinen anderen Schluss zu, als dass die
Meuterei noch heute Nacht beginnen soll.«

Er musterte mich abschätzig. »Wie pflegtest du eine drohende
Meuterei zu unterdrücken, als Kapitän deiner eigenen Ætherbrigg?«

»Ich hatte stets genügend Getreue auf meiner Seite«, schnappte ich,
»und außerdem habe ich den Schlüssel zur Waffenkammer niemals aus
der Hand gegeben.«

»Aber der Erste Maat Scanolon ...«

»Ist der Kopf der Meuterer, Sir«, unterbrach ich ihn. »Er hat die
meisten Offiziere hinter sich, nur Dunleavy scheint nicht
eingeweiht zu sein.«

»Kein Wunder«, murmelte Brassbow. »Ich war zweiter Maat unter
seinem Vater, damals auf der Phillimore – er würde mir niemals in
den Rücken fallen. Aber – alle anderen? Bist du dir sicher? Wer von
der Mannschaft ist auf unserer Seite?«

Ich zuckte die Achseln. »Die wenigsten. Sir, Mit Verlaub, Ihr habt
das Achterdeck Scanolon überlassen. Da ist es kein Wunder, dass er
nun auch die Mannschaft in der Hand hat.«

Brassbow presste die Lippen zusammen und ging noch schneller auf
und ab, wie ein gefangenes Tier in einem Käfig. »Welche Möglichkeit
habe ich dann, diese Meuterei zu verhindern? Wer würde für mich
eintreten?«

»Die Gefangenen, Sir«, sagte ich leise, »wenn Ihr uns an Eurer
Seite wollt.«

Abrupt blieb er stehen. »Das ist Teil eines perfiden Plans, nicht
wahr? Du gaukelst mir eine Meuterei vor, damit ich dich und deine
Spießgesellen losmachen lasse. Warum solltest gerade du mich warnen
sollen? Einmal Pirat – immer Pirat. Käme so etwas dir nicht gerade
zupass?«

»Nicht in diesem Falle, Sir.« Ich schüttelte den Kopf. »Die
Meuterer planen, uns zum Charon zu bringen und dort auf eigene
Rechnung zu verkaufen. Da kämen wir nur vom Regen in die Traufe.
Und Ihr habt uns bisher auf der Überfahrt anständig behandelt. Wer
weiß, ob es unter Scanolon genauso wäre? Aber ich will Euch nicht
überreden. Ihr habt Recht, im Grunde ist es mir gleich, wer das
Kommando auf diesem Schiff führt. Entscheidet also, wie Ihr wollt,
nur lasst es mich beizeiten wissen.«

Damit drehte ich ihm den Rücken zu und machte einen Schritt zur
Tür.

\bigpar

»Halt!« Brassbow war mir nachgesetzt und riss mich herum. Jetzt
stand er kerzengerade. Ich hatte es tatsächlich geschafft, ihn aus
seiner Lethargie zu reißen, allerdings war ich nicht sicher, ob ich
mir dazu gratulieren sollte.

»Für deine Frechheit sollte man dich auspeitschen, bis du liegen
bleibst!«, knurrte er. »Erinnere mich beizeiten daran, dass ich es
nachhole. Aber zunächst ist es wichtiger, die Meuterei aufzuhalten.
Scanolon hat den Schlüssel zur Waffenkammer, daran ist nun nichts
zu ändern. Was können wir ihm entgegensetzen?«

»Ketten«, schlug ich vor. »Eimer, Belegnägel, Staken. Die gesamte
Zimmermannsausrüstung und alle spitzen Gegenstände der Segelmacher,
wenn wir schnell genug sind. Und wie wäre es mit Naturgewalten?«

»Wie das? Kannst du dem Sonnenwind befehlen?«

Ich lächelte nur. »Hardie, sag dem Kapitän, welchen Kurs wir
einschlagen sollten.«

»Drei Grad sonnenauswärts«, erwiderte er prompt. »Das bringt uns in
den Asteroidengürtel, da werden uns ordentlich Gesteinsbrocken um
die Ohren fliegen.«

»Aber das ist viel zu riskant!«, protestierte Brassbow.

»Lassen Sie Hardie ans Ruder, Sir. Dann können Sie gewiss sein,
dass die Steine nur auf die Meuterer regnen.«

Brassbow seufzte. »Ich weiß nicht, wie ich der Admiralität erklären
soll, dass ich mich gerade mit verurteilten Sträflingen und
ehemaligen Piraten einlasse.«

»Mit Verlaub, Sir«, wandte ich ein, »darüber können Sie sich
Gedanken machen, wenn der Kampf gewonnen ist. Bis dahin haben wir
noch eine Menge zu tun. Ich schlage vor, dass Sie mit Dunleavy
sprechen, Sir, und Hardie und ich ins Orlopdeck zurückkehren, um
die Gefangenen einzuweihen.«

Damit hatte ich natürlich den übleren Teil erwischt. Schon bei
unserer Rückkehr erntete ich misstrauische Blicke, obwohl Wilson
lautstark verkündete, ich sei nur von einer harmlosen Variante des
marsianischen Fiebers befallen gewesen, die sich mit einigen
Tinkturen habe heilen lassen.

Nicht alle Gefangenen waren auf meiner Seite, und manche wären
sogar in Versuchung, meinen Plan an Scanolon und die übrigen
Meuterer zu verraten, weil sie sich selbst Vorteile davon
erhofften. Daher musste ich behutsam vorgehen und zunächst eine
Mehrheit für mich gewinnen.

Wir nahmen wieder unseren Platz neben Minoe und Alcester ein. Das
war ein guter Anfang. »Ich bin eben ohne Ketten durch das Schiff
gewandert«, sagte ich leichthin. »Beinahe hatte ich vergessen, wie
angenehm sich das anfühlt.«

Minoe spitzte die Ohren und beugte sich ein wenig näher. »Wilson
sprach davon, den Kapitän zu verständigen. Hast du ihm auch von
deinen Empfindungen erzählt? Was meinte Brassbow dazu?«

»Nun – du weißt, wie überzeugend ich reden kann. Er ist durchaus
Willens, allen hier unter Deck dieselbe Freude zu machen.
Allerdings nur unter der Bedingung, dass wir jemanden für ihn
aufhalten.« Ich lehnte mich so dicht zu ihm, dass meine Lippen
seine Schläfen streiften, und wisperte: »Scanolon plant eine
Meuterei. Wenn Brassbow sie verhindern kann, dann nur mit unserer
Hilfe.«

Entschlossen schüttelte Minoe den Kopf. »Nenne mir einen Grund,
warum ich mich auf die Seite desjenigen schlagen sollte, der mich
in die Minen des Charon verschleppt.«

»Wer hat dafür gesorgt, dass dir das Fell gegerbt wurde?«, fragte
ich. »Brassbow war es nicht.«

»Nein, er selbst hat es nicht angeordnet. Er hat auf dieser Reise
noch gar nichts angeordnet. Sein Erster Maat war es, der hier
ohnehin alles zu sagen hat.«

»Und genau dem, Minoe, kannst du es jetzt heimzahlen«, erwiderte
ich. »Was glaubst du denn, was mit uns passiert, wenn Scanolon und
seine Getreuen dieses Schiff übernehmen? Wir werden auf dem Charon
verkauft, so wie es von Anfang an vorgesehen war, und Scanolon
selbst wird mit einem breiten Grinsen das Geld für dich
einstreichen! Vermutlich schenkt er deinem neuen Eigentümer noch
die neunschwänzige Katze dazu, damit er dich gleich richtig
behandeln kann.«

»Und was geschieht«, fragte er, »wenn Brassbow das Schiff behält?
Glaubst du denn, vor lauter Dankbarkeit wird er uns laufen
lassen?«

Ich zuckte die Achseln. »Kommt Zeit, kommt Rat. Jedenfalls wird er
dich nicht vor lauter Dankbarkeit prügeln lassen, und das ist
immerhin etwas. Wenn wir nichts unternehmen, gehört den Meuterern
noch heute Nacht das Schiff und wir sitzen weiterhin gefesselt im
Rumpf. Wenn wir aber für Brassbow kämpfen, lösen sich diese Ketten
von deinen Beinen. Nur daran solltest du denken und die übrigen
Gedanken auf die Zukunft verschieben. Bist du dabei?«

\bigpar

Er streckte die Glieder und zuckte zusammen, als ihn sein
schmerzender Rücken an Scanolons Behandlung erinnerte. »Für
Brassbow – niemals. Aber gegen Scanolon, das kommt schon eher in
Betracht. Der Feind meines Feindes kann mein Freund nicht sein,
aber doch immerhin ein Verbündeter auf Zeit. Hast du das gemeint,
Porch? Genügt es dir, wenn ich mit solchen Gefühlen an deiner Seite
stehe?«

»Solange du nur an meiner Seite stehst«, erwiderte ich herzlich,
»werde ich mich glücklich schätzen. Glaubst du, auch die anderen
werden sich überzeugen lassen?«

Zweifelnd blickte Minoe über die Köpfe im Orlopdeck hinweg. »Alle?
Das wird nicht leicht. Und ein einziger Verräter würde den Plan
zunichte machen. Aber sei’s drum, eine andere Chance werden wir
wohl nicht bekommen, um ohne Ketten durch das Schiff zu wandern.
Nun gut, Kamerad, du kannst dich auf mich verlassen. Ich vertraue
dir. Und wenn der Plan misslingt, werden wir immerhin kämpfend
sterben und Brassbows Gewinn mindern. Auch das ist eine gewisse
Genugtuung, möchte ich meinen.«

\bigpar

Damit wandte er sich an Alcester und begann leise auf ihn
einzureden. An der Art, wie der Riese die Augen aufriss, heftig den
Kopf schüttelte und schließlich zögernd zu nicken begann, konnte
ich den Verlauf des Gesprächs nur allzu gut ablesen. Daher wandte
ich mich selbst an meinen Nachbarn zur Linken, einer schlichten
Ætherjacke, die zwar meiner fein gesponnenen Argumentation nicht zu
folgen vermochte, aber immerhin ein »Drauf und dran gegen Scanolon
und seine Bande!« akzeptierte. Hardie und ich rollten ein Stück
weiter in eine neue Gruppe, und erneut begannen wir vorsichtig ein
Gespräch.

So ging es den ganzen Tag über. Einige der gefesselten Kameraden
empfanden genügend Hass auf Scanolon, um allein deswegen den
Schritt zu wagen. Andere ließen sich überzeugen. Einen Gutteil
vermochte ich einzuschüchtern. Und der Rest, die eingeschworenen
Glücksritter an Bord, brauchte ohnehin nur einen kleinen Anstoß, um
sich auf eine Rauferei einzulassen.

Ein Blick zu Minoe und Alcester, die unsere Botschaft in die andere
Hälfte des Orlopdecks trugen, zeigte mir, dass es bei ihnen ähnlich
stand.

Gegen acht Glasen der vierten Tagwache war ich mir sicher, dass die
meisten meinem Kommando folgen würden und den wenigen anderen der
Mut fehlte, unseren Plan an Scanolon zu verraten. Daher ließ ich
nach Wilson schicken, der unfreiwillig die Rolle des
Verbindungsmannes zwischen Brassbow und mir übernommen hatte.

»Wie steht es?«, fragte ich knapp.

Ihm stand der Schweiß auf der Stirn, er gab einen denkbar
schlechten Verschwörer ab.

»Dunleavy ist tatsächlich der einzige Offizier, der nicht zu den
Meuterern gehört«, berichtete er. »Der arme Junge hat sich schon
gewundert, dass er von allen Gesprächen ausgeschlossen wurde. Und
nun kann man ihn kaum daran hindern, auf eigene Faust die
Offiziersmesse zu stürmen. Unter den Matrosen sind nur etwa fünf
oder sechs Getreue, ganz wie du es vorausgesagt hast, Porch.«

»Diejenigen, die Scanolon ungerechterweise hat auspeitschen
lassen«, bestätigte ich. »Zu schade, dass er nicht noch weitaus
ungerechter war, denn dann hätten wir mehr auf unserer Seite. Wie
steht es mit den Waffen? Hast du die Werkzeuge der Zimmerleute und
Segelmacher gesichert?«

»In dieser Hinsicht steht es leider schlecht!« Betrübt zuckte
Wilson die Achseln. »Sämtliche Gerätschaften sind gut verschlossen,
und sie herauszuverlangen würde zu viel Aufmerksamkeit erregen.
Brassbow befürchtet, dass die Zimmerleute und Segelmacher ebenfalls
zu der Verschwörung gehören. Da bleiben als Waffen also nicht mehr
als nur die Ketten.«

»Wilson«, tadelte ich, »du hast keine Phantasie! Öffne uns den Weg
zum Kanonendeck, dort gibt es Kugeln, Stopfer und Ladeschaufeln. Im
Maschinenraum finden wir Kohleschaufeln und Richtspaken. Und wie
steht es mit der Kombüse? Auf keinem Ætherschiff der Welt wirst du
einen meuternden Koch finden, diese Kerle haben doch immer nur
Angst davor, dass ihr Essensplan über den Haufen geworfen wird, und
werden sich ohne jede Überlegung auf die Seite des Kapitäns
schlagen. In der Kombüse gibt es Messer, Töpfe, Pfannen, ja, sogar
Hammelkeulen und Schweinshaxen, um die Meuterer in Schach zu
halten. Und da wir gerade von der Verpflegung sprechen: Schlage
Brassbow vor, der gesamten Mannschaft noch eine Extraportion Rum zu
spendieren. Vielleicht überlegt der eine oder andere es sich
daraufhin noch einmal mit der Meuterei, und falls nicht, dann
werden sie doch ein wenig schläfriger sein, als es ihnen gut tut.«

Wilson blickte mich bewundernd an. »Dich möchte ich nicht zum Feind
haben, Porch!«

»Dann gib dir Mühe. Wann bekommen wir die Schlüssel für die
Fußfesseln?«

»Sobald Tompkins seine Runde gedreht und die Abendmahlzeit
ausgeteilt hat.« Nachdenklich blickte er über das Deck. »Bist du
sicher, dass dieser Haufen deinem Kommando folgen wird?«

Ich musste lachen. »Wir sind im offenen Æther. Außer Hardie könnte
keiner von ihnen gut genug navigieren, um die \schiff{Mary Jane}
sicher in den nächsten Hafen zu bringen. Zum Teufel, für beinahe
ein Drittel ist dies die erste Ætherfahrt ihres Lebens. Was sollten
sie wohl anderes tun, als meinem Kommando zu folgen?«

\bigpar

Tompkins betrat mit dem Essen unser Deck, und alle Gespräche
verstummten. Aus seinem Kessel drang ein noch undefinierbarer
Gestank als sonst, und an seiner gehässigen Miene konnten wir
ablesen, dass nicht einmal diese Mahlzeit uns gegönnt sein sollte.

Als er Wilson bemerkte, stutzte er kurz. »Immer noch Kranke zu
versorgen? Sie werden in der Messe erwartet.«

Mit einem gemurmelten Abschiedsgruß verschwand Wilson, und
Tompkins’ bösartiges Grinsen wurde breiter, da er uns nun völlig in
seiner Gewalt wusste.

»Seid ihr hungrig?«, rief er. »Dann habe ich eine gute Nachricht
für euch: Heute werdet ihr alle gleichzeitig bedient!«

Damit versetzte er dem Kessel einen Tritt und stieß ihn um.

Diejenigen, die ihm am nächsten lagen, wurden von dem heißen Sud
übergossen und jämmerlich verbrüht. Sie schrien auf und versuchten
sich rollend und stoßend in Sicherheit zu bringen. Aber die
hinteren, von wölfischem Hunger getrieben, strebten nach vorn und
begannen das graue, widerliche Gebräu vom Boden aufzulecken. Alles
brüllte und tobte durcheinander, die zusammengeketteten Paare
verfingen sich in den Ketten der Nachbarn, sodass sie schließlich
verknotet waren wie ein Rattenkönig, vom Essen verklebt,
ausgehungert, verbrannt und verzweifelt, während Tompkins aus
vollem Halse lachte.

Es war unser Glück, dass wir die Schlüssel für die Ketten noch
nicht besaßen, denn auch ich hätte die Männer nicht davon abhalten
können, Tompkins auf der Stelle zu erschlagen, wenn sie nur die
Macht dazu besessen hätten, und damit wäre der Plan unzweifelhaft
fehlgeschlagen.

\bigpar

So aber gerieten sie nur in rasende Wut, das Gift strömte durch
ihre Adern, und als Tompkins mit dem leeren Kessel verschwunden
war, brannten sie auf Rache und Vergeltung. Dies machte ich mir für
einen letzten Appell zunutze.

»Kameraden!«, brüllte ich, und gleich ein wenig leiser: »Kameraden,
hört mich an! Dieser letzte Tropfen hat das Fass zum Überlaufen
gebracht. Wir wissen jetzt, was wir von Scanolon und seinen
Günstlingen zu erwarten haben. Sie lassen uns hungern, schikanieren
uns und behandeln uns schlimmer als Tiere. Und auch wenn ich nicht
im Geringsten Lust darauf verspüre, als Minensklave auf dem Charon
zu landen, so will ich doch noch weniger ihren gemeinen Launen
ausgeliefert sein. Tompkins wird büßen!«

Viele nahmen meinen Schlachtruf auf, darunter auch einige, die ich
zuvor nicht recht von meinem Plan hatte überzeugen können. Und so
war Tompkins’ allerletzte boshafte Tat zumindest für mich von
Nutzen.

\bigpar

Kurz darauf kam Wilson mit den Schlüsseln und löste unsere Ketten.
Ein geborener Verschwörer war er noch immer nicht, aber doch
zumindest für etwas tauglich, denn er versorgte notdürftig die
Verbrühungen und Verletzungen, während ich endgültig das Kommando
auf der \schiff{Mary Jane} übernahm, unauffällig, wie es in solchen
Situationen stets meine Art gewesen ist.

Beinahe unbemerkt ernannte ich meine Offiziere, indem ich Alcester
mit einem Trupp zum Kanonendeck schickte, Minoe mit einigen
weiteren Männer zur Kombüse orderte, einem der Glücksritter namens
Sutton den Befehl erteilte, die Offiziersmesse zu stürmen, und
schließlich den getreuen Hardie mit einem reichlich bemessenen
Geleitschutz zum Achterdeck entsandte, damit er dort den Kurs drei
Grad sonnenauswärts steuern sollte.

Behutsamer als Bilgeratten machten sich meine Kameraden auf den
Weg, huschten in die verschiedenen Richtungen davon. Nur eine
kleine Gruppe von Ætherfrischlingen war noch verblieben, die
Verbrühten zumeist, die nicht in der Lage gewesen waren, sich einem
der Kommandos anzuschließen.

Kopfschüttelnd betrachtete ich nun mein eigenes Kommando. Für
gewöhnlich zog ich es vor, Meutereien und Rebellionen zu
verschlafen und mich am nächsten Morgen dem Sieger anzuschließen.
Aber in diesem Falle war ein solches Vorgehen offensichtlich nicht
möglich, und so suchte ich nach einem geeigneten Ort, an dem weder
mir noch meinen versprengten Truppen ein größeres Unheil drohte.

»Wir werden das Deck klarieren«, ordnete ich schließlich an. »Jeder
Meuterer, der zu entkommen versucht, wird von uns in Empfang
genommen und über Bord geworfen. Haltet euch alle vor dem Mast, und
nehmt die Ketten als Waffen mit.«

Ein Aufprall und ein ferner Schrei signalisierten, dass der Kampf
begonnen hatte. So wie wir uns den Tag über gerüstet hatten, hatten
sich auch die Meuterer vorbereitet. Scanolon hatte einen Trupp zum
Kanonendeck geschickt, um dort die Handwaffen zu sichern, aber die
Männer stießen auf unerwarteten Widerstand. Alcester selbst zählte
in der Enge des Decks für ein halbes Dutzend Kämpfer, denn schon
wenn er sich nur umdrehte, stieß er die Gegner reihenweise von den
Füßen. Aber auch die Kameraden an seiner Seite schienen sich mit
Stopfern, Ladeschaufeln und Doppelkugeln gewaltig zu schlagen, denn
die Decke schwankte über unseren Köpfen, als schwere Gegenstände
und Matrosen auf sie niederschlugen.

Wir huschten die Decks hinauf, vorbei an der Kombüse, in der Minoe
das große Wort schwang und den Koch und die Küchenjungen in sein
Heer aufnahm. Was ich erlauschte, überzeugte mich einmal mehr von
seiner brauchbaren und vorausschauenden Phantasie, denn er ließ Öl
erhitzen und getrocknete Marserbsen über die Gänge streuen.

Behutsam stiegen wir weiter empor, bis wir an Deck waren. Auch dort
war die Schlacht schon entbrannt. Hardie hatte mit seinen Getreuen
das Achterdeck gestürmt und drehte das Steuerrad, während der
vollkommen überraschte Steuermann versuchte, ihm in den Arm zu
fallen und gleichzeitig Alarm zu schlagen.

»Die Gefangenen sind entkommen!«, schrie er.

»Das ist schon in Ordnung, guter Mann«, informierte ich ihn im
Vorbeilaufen. »Der Kapitän weiß darüber Bescheid.«

\bigpar

Ich sammelte meine Männer vor dem Mast und vergewisserte mich, dass
sie ihre Ketten bereit hielten. Im nächsten Moment brach die Hölle
los. Inzwischen hatten die Meuterer bemerkt, dass ihr Plan verraten
worden war. Für sie war es zu spät, sich auf einen besseren
Zeitpunkt zu vertagen, deswegen stürmten sie nun mit aller Gewalt
gegen uns los. Sie hatten die Waffenkammer geöffnet und waren mit
Coulombmusketen und Pistolen, Auriferrum-Säbeln und -Messern
bewaffnet. Wir hatten ihnen nur unsere große Zahl und unsere wilde
Entschlossenheit entgegenzusetzen. Schon knallten die ersten
Schüsse, der scharfe Geruch elektrischer Entladungen zog durch den
Æther. Mit einem Aufschrei brach einer meiner Männer zusammen. Er
wurde nach hinten durchgereicht, wo hoffentlich Wilson auf seinem
Posten war und ihn zusammenflicken konnte. Ich lächelte grimmig. Es
war immerhin besser als nichts, den Schiffsarzt auf der eigenen
Seite zu haben.

Die Meuterer drangen gegen uns vor, wir wehrten uns mit Ketten,
Belegnägeln, Eimern, Schaufeln und allem, was uns in die Hände
fiel.

»Haltet das Achterdeck klar!«, brüllte ich gegen Schüsse, Klirren
und Geschrei an. »Haltet sie vom Steuerrad weg!«

Es gelang mir einen flüchtigen Blick hinter mich zu werfen. Noch
standen Hardie und sein Trupp auf dem Achterdeck, er hatte den Kurs
sonnenauswärts genommen und hielt auf den Asteroidengürtel zu. Aber
Tompkins, der die Meuterer dort anführte, hatte seinen Plan
durchschaut und versuchte erbittert, das Deck zu stürmen und ihn
vom Steuerrad zu drängen. Meine Männer, die es durch einen
verwünschten Zufall auf sich genommen hatten, den Schwung der
nachrückenden Meuterer abzufangen und so Hardie den Rücken frei zu
halten, standen auf verlorenem Posten. Einer nach dem anderen
wurden sie aus der Linie geschossen, wir konnten nicht mehr lange
standhalten.

Da hörten wir wildes Gebrüll aus dem Aufgang. Minoe hatte seine
Truppen geordnet und kam uns nun zur Hilfe. Die Meuterer wurden
unversehens in die Zange genommen und von hinten mit Töpfen,
Pfannen und Bratgeschirr überrollt. Nun waren sie es, die sich
verzweifelt ihrer Haut erwehren mussten. Und Minoe hatte ganze
Arbeit geleistet. Auf sein Kommando »Hievt an!« wurde ein Topf mit
heißem Fett in die vorderste Linie geschafft und gegen die Meuterer
geschleudert. Sie sprangen zurück, um nicht verbrüht zu werden, und
liefen geradewegs in unsere Ketten hinein. Unter lautem
Triumphgeheul gaben wir ihnen den Rest.

Minoe hielt vor mir an und salutierte. »Wie geht es weiter,
Porch?«

Rasch zog ich mich an den Wanten hoch, um mir einen Überblick zu
verschaffen. »Kannst du mit deinen Leuten das Deck übernehmen?«,
fragte ich dann. »Ich habe zu viele Männer verloren, mit dem
kläglichen Rest kann ich Hardie nicht verteidigen.«

Minoe wiegte den Kopf. »Es sind Köche«, wandte er ein. »Sie fühlen
sich an Deck nicht wohl und wollen lieber wieder hinunter in ihre
Kombüse.«

»Haltet nur eine kleine Weile durch!«, sagte ich ermutigend. »Bald
werden wir im Asteroidengürtel angekommen sein, und dann gibt es
hier an Deck ohnehin nichts mehr zu tun. Uns werden die Steine nur
so um die Ohren fliegen. Hardie kann die \schiff{Mary Jane} auf
diesem Kurs halten, solange er muss – und in dem ausbrechenden
Chaos werden wir die Oberhand bekommen. Treibt die Meuterer immer
weiter nach unten, bis wir sie im Orlop einsammeln können.«

»Zu Befehl!«, erklärte Minoe. »Womit ich ausdrücken möchte, dass es
an meinem guten Willen jedenfalls nicht scheitern soll, auch wenn
ich kein Prophet bin und deswegen das Ende nicht vorhersagen
kann.«

Ich legte kurz die Hand auf seine Schulter. »Das können wir alle
nicht. Versuche nur das Achterdeck freizuhalten, damit es Hardie
nicht an den Kragen geht.«

Er nickte und stürzte sich zurück ins Gewimmel.

\bigpar

Meine eigene versprengte Schar sammelte ich mühsam, schickte
diejenigen unter Deck, die aufgrund ihrer Verletzungen nicht mehr
von Nutzen sein konnten, und zog mit dem Rest zur Kapitänskajüte,
um Brassbow zu unterstützen, der dort allein mit dem jungen
Dunleavy die Stellung hielt.

Die Tür fanden wir verrammelt, drei der Meuterer hielten davor
Wache.

»Zurück!«, rief Clavers, einer der Vertrauten Scanolons, und hob
seine wuchtige Pistole, »wir haben den Kapitän in unserer Gewalt!«

»Eben um dies zu ändern, sind wir ja hier!«, erwiderte ich und
befahl den Angriff.

Clavers zog hektisch den Abzug durch, zu rasch, um genau zu zielen.
Die Coulombpistoleentlud sich mit einer gewaltigen Detonation, aber
die Kugel ging ins Leere, und der Rauch kam uns obendrein noch
zupass, weil er den anderen die Sicht nahm, sodass sie ebenfalls
nicht zielen konnten. Wir stürmten auf Clavers und seine Genossen
zu und rangen sie nieder.

Eben wollte ich sie in Ketten legen lassen, da hörte ich die
Posaunenstimme Scanolons:

»Säbel-Josh, du bist einmal mehr auf der falschen Seite! Das Schiff
ist unser, wir halten das Kanonendeck und gerade haben wir das
Achterdeck eingenommen!«

Ich wirbelte herum. »Das ist ein Trick!«, entfuhr es mir. »Hardie
hätte euch das Deck niemals lebendig überlassen!«

»Hardie? Ach ja, dein Kettenbruder. Es ist wahr, lebendig wäre er
nicht gewichen. Aber über Bord werfen ließ er sich ganz leicht –
als er tot war.«

Mit einem Aufschrei stürmte ich auf Scanolon zu und schwang die
Kette gegen ihn. Er wich mir aus, mit der Eleganz eines Matadors
beim Stierkampf, und zog das glänzende, aus Auriferrum geschmiedete
Entermesser.

»Einmal Pirat – immer Pirat!«, spottete er. »Ungeschickt
dreinschlagen, das ist alles, was du vermagst.«

\bigpar

Mir war klar, dass er mich in Wut zu bringen versuchte, damit ich
mich noch einmal zu einem leichtsinnigen Ausfall gegen ihn
hinreißen ließ. Daher biss ich die Zähne zusammen und zog mich ein
Stück zurück, sodass ich besser mit der Kette ausholen konnte. Ich
wirbelte sie herum und schlug zu. Er parierte mit dem Entermesser
und riss es so heftig zurück, dass mir die Kettenglieder durch die
Hand rutschten und meine Handfläche ansengten. Mit der linken Hand
griff ich nach und verschaffte mir neuen Halt.

Nun machte Scanolon einen Ausfall, tauchte unter der Kette durch
und versuchte mir die Klinge in die Seite zu stoßen, aber ich
rammte ihn mit der Schulter und stolperte zurück. Es war ein
bizarrer Tanz, den wir aufführten, immer darauf bedacht, nicht
gegen die Wände des schmalen Durchgangs zu geraten. Weder seine
noch meine Männer konnten eingreifen, weil zu wenig Platz blieb,
und so standen sie nur da, hüben und drüben, und schauten unseren
Verrenkungen zu.

»Genug gespielt!«, schrie Scanolon schließlich. »Lass es uns zu
Ende bringen, Säbel-Josh!«

Er griff in die Kette und zog mich mit aller Gewalt zu sich heran.
Ich stemmte mich zunächst dagegen, dann gab ich ihm plötzlich nach
und sauste wie eine Kanonenkugel auf ihn zu. Mit der linken Hand
drückte ich sein Handgelenk nach unten, um nicht ins Messer zu
geraten, den rechten Ellbogen rammte ich ihm ungebremst ins
Gesicht. Blutend ging er zu Boden.

Ich riss ihm das Messer aus der Hand, richtete mich auf und rang
nach Atem.

»Bringt ihn ins Orlop!«, stieß ich hervor. »Nehmt meine Kette mit
und schließt ihn sicher fest.«

Unvermittelt bebte der Boden unter meinen Füßen. Die
\schiff{Mary Jane} wurde hin- und hergeschleudert. Krachend schlug
etwas auf das Deck. Über uns brach Panik aus.

»Der Asteroidengürtel!«, schrie ich. »Hardie lebt, er hat es
geschafft! Den Meuterern fliegen die Gesteinsbrocken um die Ohren!
Jetzt müssen alle unter Deck flüchten, und wir können sie hier am
Niedergang einsammeln.«

Wir hörten schnelle Schritte, die sich der Treppe näherten. Hals
über Kopf kletterten die Meuterer die steilen Stufen hinunter,
verfolgt von Minoes triumphierenden Küchengeschwader. Wir
bereiteten ihnen den passenden Empfang und schafften sie hinunter
ins Orlop. Dann schickte ich meine Leute zum Kanonendeck, um dort
nach dem Rechten zu schauen, und trat selbst zur Kapitänskajüte.

Ich hämmerte gegen das Holz. »Kapitän Brassbow, Sir, ich bin es,
Porch!«

Die Tür flog auf, ich starrte in die Mündung einer elektrischen
Pistole. Brassbow stand vor mir, hinter ihm lugte verschreckt der
junge Dunleavy durch die Tür.

»Porch, der Gefangene«, sagte Brassbow mit schwerer Zunge. Man
merkte ihm an, dass er getrunken hatte, dem Geruch nach zu
urteilen, Sherry vom Ganymed. »Hast du die Meuterei unter
Kontrolle?«

»Mit Verlaub, Sir, ja, ich denke, das haben wir«, erwiderte ich.
»Allerdings wäre es Ihre Aufgabe gewesen, uns anzuführen. Anstatt
sich hier wie ein Ætherküken gefangen setzen zu lassen, hätten Sie
sich an Deck zeigen und Scanolon in die Schranken weisen sollen.
Warum haben Sie uns nicht gesagt, dass Sie eine Pistole in der
Kajüte hatten? Die hätte uns weitergeholfen.«

Brassbow lachte hässlich. »So weit kommt es noch, dass ich einem
verurteilten Piraten eine Pistole in die Hand gebe. Das könnte ich
niemals vor der Admiralität erklären. Nun, anscheinend haben ja
auch Ketten, Belegnägel und ein günstiger Sonnenwind ausgereicht,
um dieses bisschen Meuterei in den Griff zu bekommen. Gib deinen
Leuten Bescheid, dass sie wieder den alten Kurs in Richtung Pluto
einschlagen, und damit ist die Angelegenheit ausgestanden.
Wegtreten!«

Ich blieb, wo ich war. »Sir, wir haben Ihnen den Hals gerettet und
der Admiralität ein Schiff. Darf ich annehmen, dass Sie sich auf
dem Pluto für uns verwenden werden?«

»Soweit es in meiner Macht steht, durchaus«, erwiderte er
verbindlich, ohne mir in die Augen zu sehen.

Mit einer solchen Antwort hatte ich gerechnet. Ich salutierte und
lief die schmale Stiege hinauf auf das Deck.

\bigpar

Noch immer pfiffen dort die Asteroiden dichter als Artilleriefeuer.
Ich huschte im Schutz der Reling voran zum Achterdeck. Dort am
Steuerrad klebte Hardie, blutend und zerkratzt, doch zweifellos
lebendig, rechts und links gestützt von zwei Æthermännern seiner
Truppe.

Ich spürte, wie ich über das ganze Gesicht zu strahlen begann, als
ich zu ihm lief.

»Scanolon hat behauptet, es sei aus mit dir!«

Er nickte matt. »Scanolon hat auch behauptet, er habe dich von oben
bis unten aufgeschlitzt. Aber ich wusste, dass es nicht wahr sein
konnte. Dieser Mistkerl kommt doch niemals auf drei Schritt an dich
heran! Sind die Meuterer festgesetzt?«

»Sie hocken allesamt im Orlop«, erwiderte ich, »sofern sie nicht
gerade von dem armen Wilson zusammengeflickt werden. Allerdings
rechne ich damit, dass Kapitän Brassbow sie in den nächsten Stunden
wieder freilassen wird. Uns traut er nicht, und ohne Mannschaft
kann er nicht segeln.«

Leise pfiff Hardie durch die Zähne.

»Die \schiff{Mary Jane} ist ein hübsches Schiff. Ein Jammer, dass
weder der Kapitän noch die Mannschaft irgendetwas taugen.«

»Was ist mit der Ladung für den Charon?«, fragte ich. »Meinst du,
die taugt etwas?«

»Man könnte sie sich heranziehen.«

»Und du wärst mein Steuermann?«

Hardie grinste von einem Ohr zum anderen. »Aye, aye, Captain!«

\bigpar

Wir steuerten die \schiff{Mary Jane} aus dem Asteroidengürtel hinaus
und machten die beiden Beiboote bereit zum Ablegen. Dann schleppten
wir die Meuterer aus dem Orlop und stapelten sie in die Barkasse.
Brassbow, den kleinen Dunleavy, der seinen Kapitän nicht im Stich
lassen wollte, Cassidy mitsamt seinen Gebetbüchern und die wenigen
Matrosen, die nicht an der Meuterei beteiligt gewesen waren, sich
aber auch nicht bei der Verteidigung des Schiffes ausgezeichnet
hatten, bekamen die komfortableren Plätze in der Pinasse.

»Ich wusste es!«, brüllte Brassbow, als wir ihn über die Reling
hievten. »Porch, du bist verfault bis in die Wurzel hinein! Dir
kann man nicht trauen!«

Ich zuckte die Achseln. »Da wir nun einmal aus der anständigen
Gesellschaft verstoßen sind, bleibt uns nichts anderes übrig als
auf eigene Rechnung zu arbeiten. Aber wie auch immer – gute
Überfahrt und sichere Heimkehr!«

\bigpar

Er blieb mir die Antwort schuldig.

\bigpar

Wilson bat darum, sich uns anschließen zu dürfen, und da er
zusätzlich zu seinen medizinischen Künsten leidlich zeichnen
konnte, entwarf er gleich unsere neue Flagge, weiß auf schwarzem
Grund: einen Æthermann und ein Skelett, die einander zuprosteten.

\bigpar

Hardie stand am Steuerrad. »Welchen Kurs, Captain Porch?«

»Zu den Freihandelsstationen auf den Saturnringen«, erwiderte ich,
»fünf Grad sonnenauswärts!«

»Und welche Losung?«

Ich blickte über das Deck. Ich hatte fünfzig Mann, etwa dreißig
davon befahrene Ætherleute. Wir besaßen ein feines Schiff mit
vierzehn hydraulischen Kanonen, das sich mit geblähten Segeln und
einem geschmeidig laufendem Auriferrum-Propeller aufmachte, die
fetten Pfeffersäcke vom Saturn in Angst und Schrecken zu
versetzen.

\bigpar

»Unsere Losung soll sein«, rief ich, »EINMAL PIRAT – IMMER PIRAT!«

\end{document}
