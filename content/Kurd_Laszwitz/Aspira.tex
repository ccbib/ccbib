\usepackage[german,ngerman]{babel}
\usepackage[T1]{fontenc}
\hyphenation{wa-rum Fracht-raum}
\hyphenation{schien}
\hyphenation{Tief-ebe-ne Tief-ebe-ne gro-ßen}


% Diese Datei verwendet das \modernize Makro. Kommentarzeichen löschen
% für moderne/überarbeitete Darstellung.
%\renewcommand{\modernize}[2]{#2}

% Im originalen Buch wurden in Gedichten lange Zeilen umgebrochen
% und mit einem Großbuchstaben begonnen. Das wurde hier rückgängig
% gemacht. Durch entsprechende Definition dieses Makros kann das
% originale Aussehen wiedererzeugt werden.
\newcommand\rejoined{}

%\setlength{\emergencystretch}{1ex}

%\renewcommand*{\tb}[1]{\begin{center}#1\end{center}}

\newcommand*\ausaspirastagebuch{\newline\normalsize\normalfont (Aus Aspiras Tagebuch)}
\newcommand*\verseindent\hfill
\newenvironment{liebesgedicht}%
  {\setlength{\leftmargini}{0pt}\begin{verse}}%
%  {\end{verse} \nopagebreak \begin{center}\textsf{<3}\end{center}\medskip}
  {\par \smallskip \nopagebreak \centering\textsf{<3}\end{verse} \smallskip}


\hyphenation{As-pi-ra}

\begin{document}
\raggedbottom

\author{Kurd Laßwitz}
\title{Aspira}
\subtitle{Roman einer Wolke}
\date{}
\maketitle

\section{Wolken}

Heiß hatte die Sonne des langen Sommertags über dem Hochtal
gebrütet. Nun krochen die ersten Bergschatten über die grüne
Rasenfläche, und die bunten Blumenköpfchen schlossen ihre Blüten.

Es war schwül. Denn die Luft lag feucht über den
quelldurchrieselten Matten. Dort weilte Aspira, die Wolke, ganz
aufgelöst in die unsichtbaren Teilchen ihres Elements. So ruhte sie
am liebsten, noch ungestaltet, regungslos, in Luft zerflossen. So
schlummern die Wolken und träumen.

Aspira träumte.

Vom Firnfeld am Blankhorn zog eine wache Wolke herüber. Weiß
leuchtend streckte sie ihr kugeliges Haupt in den Sonnenschein,
dunkel und grau schwebte ihr breiter Fuß in dem Tale.

„Komm herauf, Aspira!“ rief sie hinunter. „Schon lange genug
verbirgst du dich dort auf dem Wiesengrund und am Waldhang. Komm
herauf und laß uns spielen. Die feurigen Bälle werfen wir hinab und
lachen dazu, daß Onkel Blankhorn dröhnend den Ton uns zurückgibt.“

„Laß die Torheiten, Turgula,“ antwortete es aus dem Tale. „Ich habe
keine Lust zum Spiel. Soll ich mich sammeln und aufsteigen, nur um
wieder herabzuregnen?“

„Du bist langweilig. Was ist's denn mit dir? Aber wenn du nicht
magst – sieh, wie ich wachse, wie ich mich dehne! Bald kann ich
allein blitzen. Sieh dort die frechen Menschen. Sie ärgern mich
schon lange mit ihrem Hämmern am Felsen. Ich will nach ihnen
zielen.“

„Das wirst du nicht! Ich will weiter sehen, was sie tun,“ rief
Aspira.

„Haha! Was geht's uns an? Warte nur, bald will ich sie
vertreiben.“

Da stieg's von Wald und Wiese wie leichte Nebel und zog sich empor
und wurde dichter. Aspira erhob sich in raschem Schweben und
breitete ihren Wolkenschleier schützend über das Tal. Schwärzer
aber ballte Turgula sich darüber und warf den feurigen Ball lachend
hinab. Und ihr Lachen rollte zurück von der Felswand am Langberg
und hinüber bis zum Blankhorn.

Doch Aspira fing den Blitz auf, der sich in ihrem feuchten
Wolkenleib unschädlich verteilte.

„Ich will es nicht,“ rief sie. Und schneller dehnte sie sich und
stieg, bis sie Turgula erreichte und umschlang. Die Wolken
durchdrangen sich. Da entwich Turgula ihre Kraft.

„Warum hinderst du mich?“ fragte sie.

Zürne mir nicht, Turgula. In deinem Spiel will ich dich nicht
stören. Ziehe hinauf in die wilden Gründe oder wo es dir sonst
gefällt. Aber sieh, die Menschen dort unten, gerade diese, habe ich
schon im vorigen Jahre beobachtet, und nun wieder, und ich will
sehen, was da wird.“

„Du willst sehen was wird? Das kommt ja von selbst, das wirst du
doch sehen. Ob ich hier blitze oder nicht, es kommt irgend etwas.
Und was, das ist doch gleich. Das ist eben da. Und dann kommt
wieder etwas.“

„Du verstehst mich nicht. Es ist wohl so, wie du sagst; aber bei
den Menschen ist es anders. Es kommt etwas, jedoch – wie soll ich
es dir erklären? Es kommt etwas Bestimmtes.“

„Bestimmtes? Erklären? Ja, ich verstehe dich wirklich nicht,
Aspira. Was soll das heißen?“

„Wenn ich es wüßte, so wär' ich froh. Dann zög' ich wieder umher in
der weiten Welt und freute mich. Dann läg' ich nicht hier und
wartete des Kommenden. Ich weiß nur dies. Die Menschen sind nicht
bloß die kurzlebigen, kriechenden Wesen, deren wir lachen. Es muß
etwas anderes in ihnen sein. Sie tun etwas, und damit können sie
bewirken, daß etwas anderes geschieht, was sie wollen.“

„Aber das können wir doch auch?“

„Nicht so. Sieh, wir steigen jetzt empor, und die Sonne erwärmt
uns, und unsere Tröpfchen lösen sich auf. Klar ist die Luft und wir
schweben darin unsichtbar weiter und wollen, was wir tun. Und was
wir wollen, das tun wir. Aber können wir bewirken, daß außer in uns
selbst etwas geschieht, was wir wollen? Daß der Felsen dort
zerbricht? Daß er sich wieder aufbaut zur Gestalt eines Hauses? Und
das kann der Mensch.“

„Aber du konntest doch machen, daß ich nicht mehr blitze, wie du es
wolltest. Du kannst auch den Felsen zerbrechen, wenn ihn dein Blitz
trifft, und kannst die Trümmer häufen, wenn du den Gießbach
anschwellst.“

„ \emph{Wenn} ich ihn treffe! Wenn es so kommt! Wollen kann ich es
schon und vielleicht treffen. Der Mensch jedoch kann es bestimmt,
genau so, wie er will. Eben dies Bestimmen, daß es so sein muß, das
kann ich nicht verstehen. Das ist eine Wunderkraft. Höre, was ich
sah. Jene Menschen hatten ein Papier bei sich, darauf war die
Gegend abgemalt. Sie maßen alles nach und steckten Stangen in den
Boden. Und genau, wie sie es zeichneten, geht jetzt der Weg durch
den Wald, wird der Fels durchbohrt, legt sich die Brücke über den
Fluß. Woher wußten sie im voraus, daß dies so kommen mußte, genau
so? Das können wir nicht.“

„Haben's auch nicht nötig. Wir sind freie Wolken. Laß die Tierchen
da unten. Quäle dich nicht, Aspira, mit solchem Zeug!“

„Ich muß dahinter kommen. Denn es muß etwas Großes sein. Denke
daran –vor tausend, tausend Jahren – nichts war hier als Schnee und
Eis und drunter der Wald, den das Wasser brach im Frühjahr und den
der Schutt vermurte, und immer wieder sprossen aus der Wildnis Gras
und Blumen. Und jetzt sieh, wie sie uns die Berge eingeengt haben,
wie sie immer näher herandrängen. Dort glänzen die hohen Häuser von
Schmalbrück. Schon ist weit drüben der Mittelstein durchbohrt, und
nun arbeiten sie hier am Langberg. Bald werden die schnellen Wagen
bis dicht an das Blankhorn laufen. Wer sagt dir, was sie beginnen,
ob sie nicht in das Blankhorn selbst hineinhacken? Und wir können
nichts dagegen tun. Wir stürzen die Lawinen hinunter, aber sie
errichten Mauern und Dächer, und vergebens rütteln unsre Stürme an
ihren festen Bauten. Wer gibt ihnen diese Macht über die
Elemente?“

„Wenn sie wirklich so etwas haben, dann wird's ihnen wohl der Hohe
geben. Was geht's mich an?“ sagte Turgula.

„Das denk' ich wohl auch,“ fuhr Aspira fort, „was geht's mich an?
Aber es ist in mir wie ein Schmerz, eine Qual, die mit den Atem
nimmt, daß es eine Welt gibt, die mir fremd ist, in die ich nicht
schweben und dringen kann. Über das Blankhorn steig' ich empor und
bis in die Höhen des Äthers, zur Wohnung meines königlichen Vaters,
dehn' ich mich in kleinster Kristalle unbegrenzter Feinheit. In die
tiefsten Höhlen der Erde schleich' ich hinein und dampfe
unzerstörbar in der Glut der Gesteine. Über Länder und Meere zieh'
ich, ich löse mich auf und bin wieder da, ich zerteile und balle
mich, unverletzlich in meiner Freiheit. Und hier gibt es ein
Übermächtiges, Unerreichliches, das Vergangenheit und Zukunft
zusammenbindet –~– Was künftig geschieht, ist schon hier im Schoße
der Gegenwart, und doch unbemerkbar für mich, die Lebendige. Die
Menschen müssen das sehen können, was ich nicht sehe, die geheimen
Fäden des Lichts, mit denen der Hohe die Sterne zusammenzieht, daß
sie morgen gehen wie heute, daß sie dem folgen, was er sich
aussinnt. Und wenn es so wäre, und wenn auch ich folgen müßte einer
Bestimmung~–~–“

„Unsinn, ich folge nicht,“ rief Turgula. „Sieh, wie ich glühe im
Abendstrahl –~– und nun verschwimme ich. Leb wohl, auf Wiedersehen
am Morgen!“

Zarter und zarter wurde Turgula, ein schmaler, rosafarbener
Schleier umschlang sie die höchsten Spitzen, und dann war sie
verschwunden.

Von der Spitze des Blankhorns zuletzt unter all den andern
Eisgipfeln des höchsten Gebirgs war der letzte Dämmerstrahl
verschwunden. Über den Höhen der Erde, in den Tiefen des Weltraums
schritten leuchtende Sterne den gewohnten Weg.

Aspira blickte zu ihnen empor, wie sie es immer getan, in der
heiligen Scheu einer unbewußten Ehrfurcht. Aber auch unter ihr
glommen jetzt helle Sterne auf. Sie waren nicht so fern, nicht so
alt. Aspira konnte sie erreichen, umhüllen, daß sie nur in die Nähe
hineinleuchteten. Ja früher konnte sie sogar ihr Licht ausblasen,
wenn sie einherstürmte in der Nebelnacht; jetzt ging das nicht
mehr.

Waren diese Sterne mächtiger geworden? Sie wußte, daß es die Sterne
der Menschen waren, die sie in Schmalbrück anzündeten und überall,
wo sie sich niederließen. Durfte sie auf die Sterne verächtlich
hinabblicken? Oder verdienten sie auch Ehrfurcht? Waren sie nicht
vielleicht den hohen Himmelsleuchten näher verwandt als sie selbst,
die bewegliche Tochter der Luft, des Höhenkönigs Migro frei
wallendes, schwebendes Kind? Denn das war das Wunderbare, worüber
sie nicht hinwegkam, seit ihr einmal Menschenwerk die holde
Freiheit des Spiels gestört hatte: Es gab etwas, das sich ganz
genau im voraus bestimmte, das dann gerade so kommen mußte und
nicht anders. Nur noch die Sterne waren so unbegreiflich bestimmt.

Der Fels und das Wasser stürzten und der Gletscher zerbarst, aber
niemals wußte man genau, wann und wo und wie und in welcher Form.
Jedes Frühjahr ergrünte das Tal und die bunten Blumenaugen schauten
in die Sonne und badeten sich in den Tropfen des Taus. Aber nie
konnte man wissen, wo und wieviele Augen sie im nächsten Jahre
aufschlagen würden, und kein Blättchen und kein Tröpfchen war genau
so groß und genau so gelagert wie vorher. Aber Menschenwerk das
konnte bestimmt sein wie die Sterne.

Menschen! Sollten sie wirklich den Sternen verwandter sein als sie,
die freie Wolke?

Und langsam zog Aspira dem mächtigen Bergriesen zu und schmiegte
sich an das Massiv des Blankhorns. Mit weichen Armen umschmeichelte
sie des Schlummernden eisige Schultern. Da träumte er von warmem
Sonnenschein, und von seinem Schneehaupt löste sich eine weiße
Locke, die rollte abwärts, und donnernd stürzte eine Lawine in die
Schwarzschlucht.

Unwillig erwachte er aus seiner Ruhe und brummte:

„Nun, nun? Was soll's? Was hat sich die Lawine zu rühren?“

„Zürne nicht,“ sagte Aspira bittend, „daß ich dich störte. Ich barg
mich bei dir.“

„Du bist es, Aspira? Du weißt doch, daß ich jetzt keinen Besuch
empfange. Komm morgen wieder.“

Da löste Aspira ihren Wolkenarm von seinen Schultern, und große
Tropfen fielen auf seinen Scheitel.

„Nun, nun!“ raunte der Berg sanfter. „So weine nur nicht gleich!
Was hast du denn, daß du jetzt nicht Ruhe hältst?“

„Mir ist so bang.“

„Dummes Zeug. Das mag ich schon gar nicht. Nur keine Aufregung. Ich
bin zu breit unten. Wenn ich mich aufrege, steigt mir der Druck
nach dem Kopfe.“

„Aber ich fürchte mich.“

„Nun, nun! Und du willst eine Wolkenprinzessin sein? Fürchten? Das
geht ja auf was Künftiges. So was kümmert doch Wolken nicht?“

„Das ist's ja eben, \emph{daß} es mich kümmert. O sag mir,
Gewaltiger, der du in der Erde wurzelst bis weit unter die
Wohnungen der Menschen, sag mir, gibt es etwas, das über das
Kommende bestimmen kann?“

„Bestimmen? Nun, nun! Das gibt es vielleicht, das weiß der Hohe
allein. Aber was hast du deswegen zu fürchten, du freie Wolke? Du
bist, wie du bist. Was kümmert dich das Kommende?“

„Wenn's aber etwas gibt, das auch mich bestimmen kann? Wenn eine
Macht wäre, in mir zu wirken, daß ich müßte, wie andere wollen?
Wenn mich etwas zwingen könnte, zusammenzufließen zum See, daß ich
nicht wieder hinaufzugelangen vermöchte zur Höhe? Wenn ich tun
müßte, was vorher bestimmt ist und nicht anders sein kann? Wenn sie
mich einsperrten in die großen Fässer, die drunten liegen, und
hinwegführten in den engen Wagen?“

„Was redest du da für Zeug? Was wollten sie mit dir anfangen? Und
wer denn?“

„Wer? Die Menschen.“

„Die Menschen? Glimmer und Schwefelkies! Die Menschen?“

„Ich denke nur so, ich weiß nicht, was sie können. Ich möchte nur
von dir erfahren, was das für eine Macht in ihnen ist, schon heute
zu wissen, was sie morgen tun werden; zu wissen, was an dieser
Stelle werden wird, ein Haus, eine Brücke, ein Weg, genau so, wie
es dann wirklich da ist Was ist das für eine Macht?“

„Nun, nun!“ sagte das Blankhorn. Dann schwieg es lange. Aber am
leisen Rieseln des Schnees von seinem Haupte merkte Aspira, daß
etwas in ihm vorging.

„Nun, nun!“ sagte es noch einmal. „Die Menschen! Spalten und
Gletscherschliff! Da könntest du recht haben. Gestern sind mir
wieder drei auf dem Kopfe herumgekrabbelt. Haben auf mir
herumgehackt. Sieh mal da links an der Eiswand, dicht über dem
Sprung im Felsen, da müssen noch die Stufen sein. Bitte, wasche sie
mir nachher ein bißchen ab. Und jetzt erinnere ich mich. Der
Nachtwind sagte mir's, als er vom Gletscher heraufkam und sich bei
mir empfahl. An der Hütte am Schmalstein hatte er sie sitzen sehen
– es war noch Nacht – mit ihren Lichtern und hatte sie belauscht.
Der Blonde war wieder dabei, der mich schon lange ärgert, weil er
mir immer mit seinem Strick und Beil auf dem Leibe herumkriecht.“

„Über den Westgrat kommen wir leicht,“ hat er gesagt, „aber hinter
dem Kamin ist eine Eiswand, da müssen wir Stufen hauen.“ „Viele?“
fragte einer. „Nun, so einige vierzig können's werden.“ Und sieh
einmal nach, Aspira, wieviel es sind. Doch – du kannst so viel
nicht zählen. Aber ich – es regt mich nur auf, das ist dumm! Doch
ich tu's! Gleite darüber hin mit deinem Nebelhändchen, mein
Aspirchen, da kann ich's zählen. So – so – noch ein paar, weiter
oben! Es sind zweiundvierzig! Bei allen Zirbelkiefern! Woher konnte
diese Fleischklümpchen, diese Schmiernasen, diese benagelten
Zweibeiner das wissen, das doch noch gar nicht war, als sie dort
sprachen? Das ich gar nicht erlaubt hatte?“

„Du verstehst es nicht?“

„Nein, ich versteh's nicht. Aber du hast schon recht. Bedenklich
ist es. Wir dürfen uns nicht alles gefallen lassen. Mir paßt das
nicht. Unter meinen Beinen mögen sie meinetwegen in ihren Gruben
herumzappeln, das ist ein kleiner Spaß, das ist gesund für unser
einen. Aber hier oben auf mein ehrwürdiges Haupt sollen sie mir
keine Mauern bauen, noch Türmchen setzen.“

„Aber was willst du tun?“

„Nun, nun!“

„Du weißt es auch nicht?“

„Sieh, Aspira, wie sollen wir freien Geister der Elemente so was
wissen? Nicht wahr, das brauchen wir nicht? War auch sonst nicht –
das ganze Menschengesindel ist was Neumodisches.“

„Aber dann können wir auch nichts dagegen tun.“

Das Blankhorn überlegte. „Wenn es nun mal so ist, so wird es wohl
von dem Hohen so bestimmt sein. Dann ist es ein großes Geheimnis,
das nur der Hohe weiß. Dann weiß er vielleicht auch, was uns gegen
die Menschen zusteht. Es gibt ja allerlei Mittel. Man kann Schnee
und Eis und Felsen hinabstürzen. Du kannst die ganze Gesellschaft
ertränken, wenn du mit deinen Geschwistern dich über sie hermachst.
Ich könnte mit meinem Fuße wackeln. Aber das regt auf. Man hält's
nicht lange aus. Es hilft nicht für die Dauer, das Gewürm kommt
wieder. Du hast schon recht, man muß hinter ihre Kniffe kommen, man
muß sehen, aus welch breiigem Stoff der Hohe sie gemacht hat. Nur
der Hohe weiß es.“

„Der Hohe! Was nutzt das uns?“

„Ja, der Hohe. Du mußt ihn fragen.“

Aspira schauerte zusammen und zog sich dicht um den Berggipfel.

„Wie kann ich das?“ fragte sie in ängstlicher Spannung.

„Du kannst es. Nur eine Wolke kann's. Wir vermögen's nicht, die wir
im Boden wurzeln. Aber du schwebst.“

„Was hab' ich zu tun?“

„Steige! Steige, meine Aspira, bis du ausgedehnt bist aufs
allerfeinste. Bis deine kleinsten Teilchen an der Grenze des
Luftmeeres zittern, wo König Migro herrscht, dein erhabener Vater.
Du allein kannst straflos sein Reich durchschweben und
hineintauchen in den unendlichen Äther.“

„Und dann?“

„Der Hohe wird es dir offenbaren, wenn er will. Ich kann nichts
weiter dazu tun. Es regt mich auch auf. Nun, nun! Fürchte dich
nicht. Es ist alles, wie es ist. Laß mich noch ein wenig schlafen,
ehe die Sonne kommt. Leb' wohl, Aspira. Steige! Steige!“

Das Blankhorn entschlummerte, und droben im Äther wandelten die
Sterne.

\section{Legende}

Aspira stieg. Im Schattenkegel der Erde stieg sie hinauf in den
unendlichen Weltraum. Immer weiter zerstreuten sich die feinen
Eiskristalle. Die sich sonst im glitzernden Streifen der Federwolke
geschart hatten, waren schon meilenweit voneinander. Aber ihre
Einheit war nicht getrennt. Noch war sie Aspira, noch fühlte sie
die gemeinsame Ordnung, die ihr Wolken-Ich durchzog und band. Schon
lag das Reich des Vaters hinter ihr, wo Migro die Grenzen des
Planeten hütete. Immer leichter und freier wurde ihr zumute. Wann
kam sie zum Hohen?

Niemand kann ihn sehen, den Urewigen, den Unendlichen, dessen
Wohnung der dunkle Äther ist. Würde er reden? Konnte eine Wolke ihn
verstehen?

Und nun trat sie aus dem Erdschatten hinaus. Der Lichtdruck der
Sonnenstrahlen trieb ihre Teilchen hinweg vom umfassenden Arm der
Erde. Aufgelöst war sie in die Spannung des freien Äthers, eins mit
dem unergründlichen Weltraum. Nur wer den höchsten Lüften verwandt
ist vom Geschlecht der Planetengeister, vermag dahin zu gelangen.
Da schwinden die Grenzen des Stoffs, da strahlt das Geheimnis des
Unbegriffenen in die ungeschiedenen Seelen des Allebendigen. –

Und die heilige Sehnsucht der ewigen Frage durchschauerte Aspiras
Seele. In ihrem bebenden Herzen sprach die Bitte nach Erleuchtung:

\begin{verse}
Uralte Weisheit wag' ich zu suchen.\\ Uralte Weisheit, fühl' ich,
umfließt mich.\\ Uralte Weisheit künde mir, Hoher!
\end{verse}
Ehrfurchtsvoll lauschte sie auf eine Antwort.

Da klang es zu ihr mit einer Stimme, die keine Stimme war, aber
unwiderleglich wie der Gang der Zeit und das Licht der Sonne:

\begin{verse}
Uralte Weisheit gibt es nirgend.\\ Uralte Weisheit kann man nicht
fühlen.\\ Uralte Weisheit läßt sich nicht künden.
\end{verse}
Und Aspira war es, als sollte sie in nichts verschwinden,
niedergeschmettert vom dreifachen Nein des Unbegreiflichen. Aber
wieder vernahm sie die Stimme milder:

„Habe Mut, Aspira, denn ich weiß, daß du Redliches willst. So höre!
Weisheit ist nicht alt, denn Weisheit ist zeitlos. Sie muß immer
neu werden für jeden, der sie sucht. Weisheit kann man nicht
fühlen, denn fühlen kann man nur sich selbst; Weisheit aber ist
gegründet in dem, was außer dir ist und die Welt bedingt und dich
selbst. Weisheit kann man niemand verkündigen, denn sie muß
errungen sein und erobert mit mühevoller Arbeit, weil ein jeder sie
schöpfen muß aus dem Quell der Welt.“

Aspira schwieg und versuchte das Gehörte, das ihr noch zu fremd
war, anzuknüpfen an das, was sie ganz erfüllte. Dann fragte sie
schüchtern: „Und diese Weisheit erringen, vermögen das die
Menschen? Ist das ihre Macht, zu wissen, was noch nicht ist? Zu
bestimmen, was erst sein wird?“

„Diese Macht ist noch nicht die Weisheit, aber sie ist der Weg zu
ihrem Tempel.“

„Nur diese Macht der Menschen möcht' ich erkunden, nur dieser Weg
ist's, den ich schauen möchte. O, wie kann man diesen Weg finden?“

„Das Wort muß man verstehen, das ihn den Menschen weist, euch
freien Geistern der Elemente aber unbekannt ist. Ein Wort, das zu
der Welt Geheimnis leitet, öffnet den Weg.“

„O nenne das Wort!“

„Wenn ich es nenne, so wird die Sehnsucht dich ergreifen, den Weg
zu wandeln. Und die Sehnsucht wird deinen leichten Wolkenleib
hemmen und dein glückliches Spiel in den freien Lüften. Darum
erbitte es nicht.“

„Die Sehnsucht hemmt mich schon, die Furcht ergriff mich vor der
Macht der Menschen. Kann ich, eine freie Wolke, dazu gelangen, das
Geheimnis des Menschen zu erfassen, so nenne das Wort!“

„Das Wort weist nur den Weg, noch nicht die Weisheit.“

„Nur den Weg suche ich.“

„Es ist ein schwerer Weg und ein schweres Wort, ein heiliges Wort,
das die Macht der Welt in die Seele legt~–“

„Nenne das Wort!“

„Aber mit der Macht auch das Leid, das erhabene Leid des Schöpfers
um sein Werk, daß seine Seele erbebt in ihrer Tiefe, die zur Tiefe
der Welt wird.“

„Nenne das Wort!“

„Das Gesetz!“

Als dieses Wort in der Leere des Raumes vernommen ward, da war kein
Ton und keine Stimme, und dennoch war das Wort und schauerte durch
alles Leben der Kreatur. Denn es war das Wort, vom dem gesagt ist,
daß es im Anfang war – das Gesetz – die Bestimmung des Gesetzes.

Und ehe Aspira sich bewußt wurde, was ihr Neues gegeben ward, klang
in ihr wieder milde die Stimme des Hohen, die keine Stimme war:
„Vernimm, Aspira, das Geheimnis des Gesetzes, soweit du vermagst.
Frei in deiner Seele entstand die Frage nach dem, was anders ist
als das Jetzt, entstand der Zweifel. So hast du ein Recht errungen
zu hören vom Gesetz. Denn nur wer zweifeln kann, der kann erkennen.
Ihr Geister der Natur kennt nicht den Zweifel, darum bedürft ihr
nicht der Erkenntnis. Du aber hast nun eine Macht gewonnen, die
über das Wolkenreich hinausführt.

Ich künde dir die \emph{Legende}.

Das Gesetz war vereint mit Zeit und Raum. Und in ihnen war die
Fülle der Welt.

Da entsprossen zwei Kinder. \emph{Weil} hieß der eine, und
\emph{Will} der andre.

Weil konnte nur rückwärts blicken und nahm alles, was gewesen war,
das hielt er fest mit der Kraft seines Vaters, des Gesetzes. Denn
er sah nur, wie alles Geschehene in ihm bedingt war.

Will schaute nur voran und forderte alles, was in Zukunft geschehen
sollte, das bestimmte er als Ziel mit der Macht seines Vaters, des
Gesetzes. Wie aber etwas geschah, das wußte er nicht. Denn er
kannte nur, was geschehen solle nach dem Gebot.

Da entstand Streit zwischen den Brüdern, und sie gaben ihrem Vater,
dem Gesetze, verschiedene Namen. „Notwendigkeit“ nannte es Weil,
„Freiheit“ aber nannte es Will. Und jeder sagte, daß die Welt nur
ihm allein gehorche. Und sie klagten beim Vater.

Der Vater schied, um den Streit zu schlichten, alles Lebendige in
zwei Teile, in das Reich des \emph{Werdens} und in das Reich des
\emph{Sollens}.

Das Reich des Werdens nannte er „Natur“ und gab es dem Weil, der
nur das Geschehene sah, wie es ward, und nicht das Ziel, wohin es
strebte. Darum nannte er das Gesetz die „Notwendigkeit“, und alles
mußte so sein, wie es ist.

Das Reich des Sollens aber nannte er „Idee“ und gab es dem Will,
der nur das Ziel sah, aber nicht verstand, wie es werden mußte.
Darum nannte Will das Gesetz „Freiheit“, und alles wollte anders
sein, als es ist.

Der Vater aber sprach weiter zu den Geistern alles Lebendigen:
„Gehet hin und wählet das Reich, darin ihr leben möget; in jedem
von ihnen findet ihr das Glück; weil nur \emph{ein} Gesetz in euch
ist, werdet ihr nichts vom Gesetze mehr merken. Geht ihr in das
Reich der Natur, so lebt ihr nach dem Gesetze der Notwendigkeit.
Aber da ihr nicht wißt, daß ihr auch anders sein könntet, so nehmt
ihr das Leben ohne zu fragen hin, wie es eben kommt. Nach dem
eignen Wesen lebt ihr glücklich, ohne Enttäuschung, sorglos und
frei. Euer Leben ist ein Spiel, denn das Gesetz ist in euch, ohne
daß ihr es wißt.

Geht ihr in das Reich der Idee, so werdet ihr euch stets erblicken,
wie ihr wollt, und ihr werdet auch glücklich sein. Denn euer Ziel
setzt ihr euch nach Gefallen und wißt nichts von den Hindernissen,
die entgegenstehen. Eure Wünsche sind euer Leben. Ihr lebt in der
Idee und seht das Gewollte als erfüllt, das Erfüllte wie ein
Gewolltes. So seid auch ihr glücklich und sorglos ohne
Enttäuschung. Euer Leben ist ein Spiel, denn das Gesetz ist euer
Wille.

Aber wenn ihr glücklich bleiben wollt, so hütet euch, in beide
Reiche zu dringen, in beiden leben zu wollen. Von einem Reiche zum
andern führt der Weg der Macht, doch er ist eine Brücke. Wer sie
betritt, der unterliegt dem Gesetze beider Reiche.

Wer aus dem Reiche der Natur kommt in das der Idee, der lernt die
Freiheit kennen, der fordert für sich, was er nicht hat. Da strebt
er nach dem Unerreichbaren. Da sieht er, daß er notwendig bedingt
ist, seine Freiheit däucht ihm ein Schein, und seine Macht sinkt
zusammen im Leide der Sehnsucht.

Wer aber aus dem Reiche der Idee in das der Natur tritt, dem
scheint es, daß er seiner Freiheit beraubt werde. Denn will er die
Macht erreichen über das, was geschieht, so muß er sich dem Gesetze
der Notwendigkeit unterwerfen. Sich selbst muß er ansehen als
bedingt durch das, was war, und vorausbestimmt für alles, was sein
wird. Und der Stolz der Freiheit sinkt ihm zusammen im Leide der
Sehnsucht.

Darum hütet euch, die Brücke zu betreten.

Die Brücke aber heißt: Erkenntnis.

Wer nur lebt in der Reiche einem, der weiß nichts vom Geheimnis der
Welt. Er lebt nur unter \emph{einem} Gesetz, und so weiß er nichts
davon, daß es zwei Seiten hat, die Notwendigkeit und die Freiheit.
Er kennt das Leben nur in einem Gefühl ohne Grund, ohne
Widerstand.

Wer aber die Brücke der Erkenntnis betrat, dem hallt das Wort
entgegen, das am Anfang war, dem wird die Seele erschüttert vom
Zweifel und der Sehnsucht in der Wonne des Stolzes und der Macht.
Das ist das erhabene Leid des Schöpfers um sein Werk.

Du stehst auf der Brücke, Aspira, du hast das Wort vom Gesetze
vernommen. Aber noch steht dir frei, zurückzutreten in dein Reich.

Du weißt nicht, welches dein Reich ist? Freilich kannst du es nicht
wissen, da du nur das eine kennst. Und die nur das eine kennen,
sind glücklich im freien Spiel. Denn unglücklich macht nur der
Zwang. Der Zwang aber entsteht für den, der beide kennt, die
Notwendigkeit und die Freiheit.

So vernimm, was weiter geschah.

Die Geister des Lebendigen trennten sich voneinander, die einen
zogen zu Weil, die andern zu Will.

Die zu Weil zogen, waren die Gewaltigen des Raumes. Das waren die
Riesen des Äthers, die von Sonnen zu Sonnen ihre Strahlenleiber
strecken. Das waren die Sonnen selbst und die Planeten. Und auf der
Erde waren es die Geister der Elemente, die im Innern des Erdballs
glühn und die sich in den Bergen zur Höhe steifen, die in den
Schluchten rauschen und in den Lüften ziehen. Und ihr Wolken seid
die feinsten, die höchsten, die beweglichsten von ihnen. Daher kann
es geschehen, daß eine von euch sich über ihr Reich verliert bis
zur Brücke der Erkenntnis.

Die aber zu Will zogen, das waren die Geister, die sich die
kleinen, mühsamen Zellenleiber bauten. Die schwimmen im Meer und
wachsen am Fels, die kriechen und summen und fliegen und klettern,
und teilen sich und sterben, aber leben wieder auf in ihren Teilen.
Und auch sie sind verschieden, und viele von ihnen reichen bis an
die Grenze, wo die Brücke der Erkenntnis steht. Und die höchsten
von ihnen sind die Menschen.“

Lange schwieg Aspira. Dann wagte sie zu sagen:

„Die Menschen! Um ihretwillen kam ich zu dir, ihr Geheimnis zu
vernehmen. So ist es ihr Geheimnis und ihre Macht, daß sie nicht
zum Reiche der Notwendigkeit gehören, wie wir Wolken, sondern zum
Reiche der Freiheit?“

„Nein, das ist es nicht. Der Menschen gibt es zweierlei. Solange
sie jung sind und Kinder, ob nun die einzelnen, ob ganze Völker, so
leben sie noch allein im Reiche der Freiheit und willen nichts von
den beiden Reichen. Und sie sind glücklich, aber machtlos. Wenn sie
jedoch älter werden~–“

„So gewinnen sie die Macht?“

„Die Macht und das Leid. Denn sie treten auf die Brücke der
Erkenntnis und können nicht wieder zurück.“

„Was ist Erkenntnis?“

„Das eben ist die Macht, aus dem Reiche der Freiheit in das der
Notwendigkeit zu treten und das Gesetz zu verstehen, wie aus dem
Geschehenen das Künftige werden muß.“

„Und das ist das Geheimnis der Menschen?“

„Das ist eines der Geheimnisse, es ist ihre Macht.“

„Ich will auf die Brücke der Erkenntnis treten. Ich will ihre Macht
gewinnen. Denn das ist das Geheimnis, das ich suche.“

„Erinnerst du dich, was ich von der Weisheit sagte? Weisheit kann
nicht geschenkt werden, sie kann nur erworben werden in mühevoller
Arbeit. Und Erkenntnis ist ein Teil der Weisheit. Sie zu erringen
kann dir nicht gelingen mit deinem geschmeidigen, dehnsamen
Wolkenleibe. Dazu müßtest du ein Mensch werden.“

„Ein Mensch! Ich? Kann ich das?“

„Es gibt auserwählte Geister in beiden Reichen, denen es gestattet
ist, aus dem einen in das andre Reich zu tauchen. Dann enthüllt
sich ihnen das Gesetz in seiner doppelten Gestalt als der
\emph{eine} Vater alles Lebendigen. Mitunter können auch Menschen
hineinschweben in die Tiefe der Natur und das Gesetz in seiner
Einheit erblicken.“

„So könne sie Wolken werden?“

„Nein. Ihr Leib muß ein Menschenleib bleiben. Denn nur die
Zellenwesen haben im Bau ihres Hirns die Fähigkeit, nicht nur
mitzuschwingen und mitzuerleben, was die Welt im Verborgenen
durchflutet, sondern auch es nachzubilden und mitzuteilen. So könne
sie mit euch schweben in allen Höhen und Tiefen und hinaus bis in
die Unendlichkeit des Äthers und mitfühlen das Geheimnis der Welt.
Dichter heißen sie unter den Menschen. Ihr aber, die ihr nicht
schon wohnt im Reiche der Freiheit, könnt in der Menschen
Gedankenreich nur hineintauchen, wenn ihr einen Menschenleib
gewinnt.“

„O gibt ihn mir, Hoher, o laß mich ein Mensch werden! Wie ist das
möglich?“

„Der die Bewegung kennt aller Strahlen und Atome der Erde, wird es
dir sagen. Schwebe zurück, Aspira, aus den Höhen des Raumes und
frage deinen Vater. Will er's gestatten, so magst du ein Mensch
sein.“

\section{Berggeister}

Noch lange zögerte Aspira, aufgelöst in den verschwindenden Weiten.
Sie lebte in Erwartung des Zukünftigen, in Stolz und zagender
Ungewißheit, und hoffte auf ein weiteres Wort der Erklärung. Aber
die Stimme des Hohen war nicht mehr zu vernehmen.

Da zog sie ihre Teilchen zusammen und ward wieder schwer und sank
hinab zur Erde. Und sie kam in das Reich des Vaters.

Wo das Atem des Erdballs sich verdünnt zur Leere des Weltraums, an
den Grenzen des Luftmeers wohnt König Migro. Dort umspannt er das
Rund der Erde, der Pförtner der Strahlung, die Erde und Sonne und
all die Welten des weiten Himmels verbindet. Von dort wandern
schwingende Boten zum dauernden Austausch der Kräfte rastlos ums
Erdenrund. Tief unter ihm schweben Dünste und Wolken, unter ihm
zucken von den Polen die glimmenden Lichter, ihn trifft
ungeschwächt die Fülle der Sonnenbotschaft. Und er verteilt die
Gaben des beherrschenden Glutballs an die Stoffe des Planeten, daß
sie wallen und wandern in der Elemente Gewalt und sich einen und
lösen im feinsten Gliederbau geordneter Zellen.

Der Vater empfing Aspira mit freundlichem Ernste.

„Vom Hohen kommst du“, so sprach er, „mit wundersamer Bitte, mit
seltener Erlaubnis.“

„Ja, Vater, der Hohe erlaubt mir, in das Reich der Freiheit zu
treten aus meinem Reiche der Notwendigkeit, wenn du mir den
Menschenleib gewähren willst, dessen ich bedarf.“

„Wisse, mein Kind, was mir von den Menschen bekannt ist, das ist
nur das Spiel der Stoffe und Kräfte in den Zellen ihres Leibes. So
weiß ich auch nur, wie du dich mit diesem Leibe verschmelzen
kannst. Dein Wolkenkörper ist durchstrahlt von einem schwingenden
Äther, dessen Spannung deine Teile zu einer Einheit verbindet, und
mögen sie durch die Räume der Welt verstreut sein. Ihr nennt ihn
das Wolkenherz, und in ihm liegt dein Leben als Wolke. Dieses
Wolkenherz kann sich verschmelzen mit dem Menschenhirn, das des
Menschen Leib und Leben regiert. Und ich weiß, wie sie sich wieder
trennen können, ja auch, daß sie noch vereint bleiben können, wenn
dein Leib wieder die Form der Wolke angenommen hat.“

„So kann ich auch eine Wolke sein mit der Seele des Menschen?“

„Was das bedeutet, weiß ich nicht. Was in der Seele des Menschen
vorgeht, kann ich nicht verstehen und es geht mich nichts an. So
weiß ich auch nicht, was deine Seele fühlen mag, wenn du ein Mensch
bist. Denn mein Reich ist das der Notwendigkeit und nicht der
Freiheit. Und ich weiß nur, daß die getrennten Reiche verbunden
sind durch die Brücke der Erkenntnis, wie der Hohe dir offenbarte.
Die Menschen können sie betreten und dadurch Macht gewinnen über
unser Reich. Dann zürnen ihnen unsre Geister. Was aber die Menschen
dabei erleben, ist uns unbekannt. Vielleicht will der Hohe, daß
auch wir erfahren von dem unbekannten Leben der Menschenseele. Dazu
muß einer der Unsern Mensch werden. Das aber können nur die
Königswolken. Und auch diese nur, wenn in ihnen die Sehnsucht rege
geworden ist nach dem Reiche der Freiheit, wenn sie mutig genug
sind die Brücke der Erkenntnis zu betreten. Es ist eine seltene
Gnade, die der Hohe dir gewährt. Doch was dann an Leid oder Lust
die Seele durchbebt, das weiß niemand von uns Geistern der Natur.“

„Der Hohe selbst hat mich gewarnt. Doch, Vater, wenn ich es darf,
so will ich es. Ich habe den Mut.“

„Zwei Tage mußt du warten und still ruhen, um dich zu prüfen. Und
beharrst du dann in deinem Entschlusse, so magst du am Morgen des
dritten Tage hinabgehen als Mensch zu den Menschen.“

„Ich werde gehorchen. Doch sage mir, wie alles geschehen soll!“

„Ich will es dir künden. Mit deinem Wolkenherzen wirst du
hineinziehen in den Leib eines Menschen, der im Schlafe der
Erstarrung dich aufnimmt. Dann wird der Leib des Menschen dein sein
mit alles, was der Mensch durch sein Leben erworben hat, soweit
dein Wolkenherz ihn zu durchdringen vermag. Und soweit wird auch
seine Seele die deine sein. Dann bist du ein Mensch unter
Menschen.“

„Und kann ich wieder zurückkehren als Wolke?“

„Du kannst es, so oft du willst, doch nur auf kurze Zeit. Dann mußt
du dich niederlegen mit deinem Menschenleibe an verborgener Stelle
dieses Bergreichs, das deine Heimat ist, und in den Schlaf der
Erstarrung versinken. Und dein Atem wird entweichen als Wolke
zugleich mit der Menschenseele. Und du wirst sein eine Wolke mit
der Seele einer Wolke und dem Geiste eines Menschen zugleich. Der
Leib des Menschen aber liegt inzwischen erstarrt in den Bergen.
Darum hüte dich, daß du nicht zu lange als Wolke säumst, damit
nicht etwa inzwischen der Leib des Menschen Schaden nimmt; sonst
kannst du nicht wieder Mensch werden. Und hüte dich, daß von deinem
Wolkenherzen nichts verloren geht; sonst kannst du nicht wieder
eine reine Wolke werden. Denn wenn du wieder frei sein willst in
unserm Reiche, so mußt du dem Menschen alles zurückgeben, was du
mit deinem Wolkenherzen vermischt hattest. Kannst du das nicht, so
stirbt der Mensch und du bleibst eine Menschenwolke.“

„O mein Vater, es ist ein gefährliches Wagnis.“

„Darum hast du Bedenkzeit. Und wisse noch dies: Es ist dir
verboten, den Menschen zu verraten, daß du eine Wolke bist.“

„Warum das?“

„Der Hohe will's.“

„Und was für ein Mensch werde ich sein, o Vater?“

„Einer von denen, die in der Frühe des dritten Tages heraufkommen
in die Berge. Ich sage dir, daß ich die Seelen der Menschen so
wenig kenne, wie irgend einer von uns Geistern der Natur. Darüber
vermag ich nichts. Doch seinen Leib können wir beschützen. Auch
dich, meine Tochter, werden wir hüten mit unsrer Macht, solange du
in Menschengestalt wandelst. Du wirst einen andern Namen haben als
Mensch, uns aber bleibst du Aspira. Migros Segen geleitet dich. Und
faßt dich ein Leid um der Menschen Not, so fliehe zurück in unser
Reich. Und nun ziehe hin, mein Kind, und entschließe dich. Am
Morgen des dritten Tages werde ich dich wieder hören.“

Da sank Aspira hinab in die Schlucht am Blankhorn und barg sich vor
den ersten Strahlen der Sonne, die das weiße Haupt des Bergriesen
vergoldete.

Wieder leuchteten die hohen Schneeberge im Sonnenschein.

An der dunkeln Felswand, die zum Firnfeld des Blankhorns abstürzt,
erwachte der Nachtreif und begann mit feuchtem Augenaufschlag ins
Tal hinabzuglänzen. Unter den Schneeresten wurde ein kleines
Rinnsal lebendig; das tropfte leise auf die Streifen der
gelbbraunen Flechten hernieder.

„Sehr ihr nichts?“ knurrte der Fels.

„Es sind Wolken über dem Gletscher,“ sagte der Nachtreif
schüchtern.

„Das weiß ich selbst. Aber ob Aspira darunter ist, das kann ich
nicht unterscheiden. Wozu seid ihr denn naß, wenn ihr nicht
zwischen dem Wolkengewimmel hindurch sehen könnt?“

„Wen meinen Sie denn mit dem „ihr“,“ tönte es von den Flechten.
„Wir sind nämlich auch hier und wir sind auch naß, aber etwas mehr
Höflichkeit möchten wir uns doch ausbitten.“

„Sie meine ich freilich nicht,“ polterte der Fels. „Das weiß ich
schon, daß Sie nichts Rechtes sehen können mit Ihrem gefärbten
Zellenleibe. Jetzt tun Sie groß, aber warten Sie nur, bis die
Schneereste fort sind und Sie austrocknen.“

„Das tut uns nichts, höchstens ruhen wir dann ein Weilchen. Es ist
ja hier dafür gesorgt, daß es Feuchtigkeit genug gibt. Der Nebel
und der Herr Nachtreif genügen uns vollständig.“

„O bitte,“ sagte der Nachtreif geschmeichelt, „ich tue es gern.“

„Sieh dich lieber um,“ brummte der Fels, „ob du nicht Aspira
erblicken kannst.“

„Ich sehe nichts. Vielleicht ist schon alles vorüber.“

„Was soll denn vorübersein?“ fragte das Rinnsal.

„Das weißt du nicht?“ rief der getaute Nachtreif.

„Das weißt du nicht?“ rief der Schnee entrüstet.

„Das weißt du nicht?“ brummte sogar der Fels verächtlich.

„Aber ich bin doch eben erst ganz neu geboren,“ klagte das Rinnsal
zaghaft.

„Geschmolzen, meinen Sie,“ sprachen die Flechten. „Sie scheinen ein
kurzes Gedächtnis zu haben. Sie waren doch vorher Schnee, und da
wußten Sie sicher, wer Aspira ist.“

„Ach, Aspira? Des Königs Migro kluge Tochter? Die kenn ich
freilich. Was ist mir ihr?“

„Na, das könnten Sie doch auch wissen,“ sagten die Flechten. „Das
ganze Bergland redet ja darüber, vom Blankhorn bis unten zu den
Maulwurfshaufen. Jeder Tropfen kümmert sich darum. Wir sind genug
mit der Geschichte gelangweilt worden, und wir glauben kein Wort
davon.“

„Seien Sie nicht zu vorlaut,“ knurrte der Fels die Flechten an.

„Der Schnee, von dem das Rinnsal kommt, liegt noch vom vorigen
Herbst hier ganz unten in der Spalte, da ist von den neuesten
Geschichten noch nichts hingedrungen. Und wenn Sie's nicht glauben
wollen, so können Sie mir leid tun.“

„So sagt mir doch, was mit Aspira ist,“ bat das Rinnsal.

„Mensch wird sie!“ riefen Fels und Schnee und Nachtreif.

„Unsinn ist es, das geht gar nicht,“ sagten die Flechten.

„Unsinn mag's sein, aber es geht,“ entschied der Fels. „Das weiß
ich genau. Der Morgenwind hat mir's gestern erzählt.“

„Da haben wir's auch gehört,“ riefen die Flechten.

„Sie werden's eben nicht verstanden haben.“

„Erlauben Sie. Wir wohnen zwar schon über tausend Jahre bei Ihnen,
aber Sie haben unsre Leistungen und Verdienste noch immer nicht
schätzen gelernt.“

„Bitte sehr, ziehen Sie ruhig aus!“ brummte der Fels. „Ich brauche
Ihre Zellkriecherei und die ganze Familienwirtschaft nicht.“

„Familienwirtschaft ist gut. Damit gestehen Sie eben, daß wir etwas
Höheres sind. Wir sind ein wirtschaftlicher Verein. Wir sind Pilze,
die Algenzucht treiben, damit wir leben können. Und dafür wollten
Sie uns dankbar sein. Denn nur durch unsre gemeinsame Arbeit können
wir Sie annagen und zerlegen. Und wenn wir das nicht täten, würden
Sie niemals Humus werden und zu etwas nütze sein. Hätten Sie sich
nicht so hoch oben plaziert mit so steilem Abfall, so könnten Sie
längst eine anständige Viehweide oder gar ein Wald geworden sein.“

„Sie sind wirklich eine unverschämte Gesellschaft. Ich werde Sie
mit dem nächsten Bergsturz abschieben.“

„Das ist uns eben recht, da kommen Sie Ihrem Ziele näher.“

„Ziele? Was ist das für ein Unsinn. Kenne ich nicht.“

„Das glauben wir! Das verstehen nur die Zellenwesen. Sie wissen
nicht einmal, daß Sie Humus werden sollen, Dammerde, fruchtbarer
Talboden. Dazu wollen wir Ihnen verhelfen.“

„Ich will nichts. Ich bin hier oben und bleibe hier, das Hinab und
Hinauf können die Wolken besorgen. Liegen sie denn immer noch da
unten?“

„Ich kann noch nichts sehen,“ sagte das Rinnsal.

„Vielleicht ist doch schon alles vorüber,“ wiederholte der
Nachtreif. „Aber ich werde es bald wissen. Ich werde gleich
verdunstet sein, und dann kann ich hinüberfliegen.“

„Aber bei mir dauert's noch lange, bis ich hinunterkomme, und dann
sehe ich nichts mehr, ich möchte doch hören, wie den Aspira Mensch
werden kann.“

„Solange die Wolken unten sind, wird's noch nicht vorübersein. Der
Morgenwind hat mir versprochen, sogleich heraufzukommen und mir
alles zu erzählen, sobald's vorbei ist.“

„Wie soll denn eins von Ihnen Mensch werden?“ riefen die Flechten.
„Wir wissen doch, wie das bei den Zellenwesen ist, denn wir sind
selbst welche. Der Mensch ist auch nichts als eine höhere Flechte –
was man so höher nennt. Eigentlich sind wir die höheren, aber weil
er herumlaufen kann, so – na, darüber will ich nicht streiten. Aber
das weiß ich, daß bei ihm zwei dazu gehören, wenn ein neuer Mensch
werden soll. Das ist also gerade wie bei uns. Unsere grünen Algen
allein bringen keine Flechte zustande, und unsere Pilzsporen allein
auch nicht. Erst wenn sie sich zusammentun, wird wieder eine neue
Flechte daraus. Beide aber sind Zellenwesen. Wie soll denn von
Ihnen ein Mensch kommen, da Sie gar keine Zellen haben?“

„Das mag bei Ihnen so sein,“ sagte der Fels. „Wir können vieles
allein, wozu Sie sich erst zusammentun müssen. Aber das muß ich
zugeben: für gewöhnlich können wir keine Menschen werden. Wir
Felsen und Berge, und die Gletscher und der Wind und das fließende
Wasser, die können's überhaupt nicht. Nur die Wolken können's, und
auch von denen bloß die Königswolken. Und der Morgenwind hat
einiges gehört, wie Migro mit seiner Tochter sprach, die seit zwei
Tagen in der Schlucht liegt, denn er mußte eine Botschaft zwischen
ihnen tragen. Ich will es dem Rinnsalchen mitteilen, damit es
Bescheid weiß. Und wenn Sie's schon einmal gehört haben, so ist mir
das egal.“

Der Fels nahm eine noch steifere Haltung an und fuhr dann fort:
„Ich bin zwar nur ein kleiner Teil vom Blankhorn, das alles weiß~–“
die Flechten räusperten sich höhnisch – „aber ich weiß doch auch
allerlei. Die Wolken haben nämlich etwas voraus vor uns andern
Erdgeistern, sie sind so etwas für sich, ja, na, so wie die
Menschen auch sein sollen, – sie haben auch einen Namen dafür~–“

„Den haben Sie aber vergessen,“ mischten sich die Flechten ein.
„Wir Zellenwesen sind nämlich Individuen. Sie, meine Herrschaften,
sind immer nur ein Stück von etwas anderem. Sie sind ein Stück
Fels, und Sie ein Stück Wasser, und der Wind ist ein Stück Luft.
Wir aber, die Pflanzen und die Tiere, sind etwas für sich, jeder
was Eigenes. Und das nennt man ein Individuum. So was ist der
Mensch auch.“

„Das weiß ich natürlich,“ rief der Fels. „Seien Sie doch schon
still! Also die Wolken sind solche Dinger. Sie können wachsen und
können abnehmen und sich so dehnen und so, und sich auflösen und
wieder sammeln, aber sie bleiben immer eins dabei. Darum kann eine
zu einem Menschen werden, wenn sie einen lebendigen Menschen findet
und ihr unsichtbares Wolkenherz mit dem Menschenherzen mischt.“

„Ich weiß nur nicht,“ begann der Nachtreif, „was eine Wolke davon
haben kann, ein Mensch zu werden. Denn sie ist doch etwas
Höheres.“

„Freilich ist sie das,“ fuhr der Fels fort. „Die Menschen sind eben
traurige Maschinen wie die Tiere – unterbrechen Sie mich nicht, Sie
Flechtenwesen – sie sind alle aus einer Form gebacken wie die
Käfer. Und dabei wollen sie den Langberg anbohren, vielleicht sogar
uns selbst, das Blankhorn. Ich kann ja freilich nichts dagegen tun,
ich bin nur ein Felsen und kann mich wenig rühren, ich kann nur
schimpfen. Aber der Gletscher, der sich fortwährend wandelt und
wandert, und die Wolken, die schweben und weben in tausend
Gestalten, sollen sie sich den Menschen nahe kommen lassen, der
immer derselbe ist? Sieh den Toni, wenn er mit dem Strick und
Eisbeil herumsteigt, immer hat er zwei Arme und zwei Beine und
denselben Kopf, und der Hans und der Peter drüben auf der Alp sieht
ebenso aus. Hast du jemals gesehen, daß ein Mensch sich in drei
Beine gespalten hat, oder in Streifen in der Luft zerflossen ist?
Ist's nicht so?“

„Das ist schon richtig,“ fielen die Flechten ein. „Darin sind wir
eben die Höheren, wir brauchen uns nicht so einzuschränken.“

„Na, also!“ fuhr der Fels fort. „Maschinen sind sie. Die Nebeltante
erzählt uns ja immer, so oft sie drüben zwischen den großen Häusern
in Schmalbrück liegt. Mittags klingelt es und abends wieder; dann
müssen sie allerlei Zeug in sich hineinstecken, sonst können sie
nicht mehr kriechen.“

„Aber,“ fragte das Rinnsal, „wenn die Menschen so närrisch sind,
was kümmert ihr euch um sie?“

„Närrisch sind sie wohl nicht bloß, es muß noch etwas hinter ihnen
stecken. Mir kann's ja gleich sein, aber den Wolken ist's wohl
nicht gleich. Sonst würde doch der Hohe Aspira nicht zu den
Menschen schicken.“

„Na,“ meinte der Nachtreif, „das ist vielleicht eine Art Strafe.“

„Warum denn?“ fragte das Rinnsal.

„Nimm dich in acht, Kleines,“ sprach der Nachtreif im Verdunsten,
„daß dir's nicht auch so geht. Wegen des „Warum“ wurde Aspira
weggeschickt, weil sie immer diese dumme Frage hatte. Sie wollte
allemal einen Grund wissen. Und ein Grund, das ist so was
Menschliches, das wollen die Menschen auch überall haben. Eine
richtige Wolke hat keinen Grund.“

„Es ist auch Blödsinn,“ brummte der Fels wieder. „Man ist eben, und
damit ist's genug. Aber Aspira fragte immer: „Warum steigen die
Menschen auf die Berge? Warum macht sich der Vetter Kumulus unten
so breit und oben so rund? Warum kann ich mich ausdehnen und
zusammenziehen? Warum können die Menschen nicht fliegen?“ Da wurde
Migro einmal ärgerlich und rief: „Wenn du jetzt noch einmal Warum
fragst, so kannst du unter die Menschen gehen, dann da paßt du
hin.“~ „Warum?“ fragte Aspira. Und da mußte sie fort, da mußte sie
ein Mensch werden.“

„Na, na, na!“ riefen die Flechten. „Da reden Sie wieder einmal
rechten Unsinn. Der Morgenwind hat es ganz anders erzählt. Eine
Belohnung ist's für Aspira durch den Hohen selbst. Sie soll einmal
erfahren, was sonst nur die Menschen wissen. Und daß die allerlei
verstehen, das haben Sie ja selbst zugegeben. Da fragen Sie doch
einmal den Nachtreif.“

„Wo ist denn der Nachtreif hin?“ rief der Fels.

„Verdunstet! Er hat sich dünn gemacht.“

„Wirklich! Doch seht, da lichten sich die Wolken. Da muß es
geschehen sein. Nun wird der Morgenwind gleich erscheinen. Drüben
auf der Alp wogt schon das Gras. Da werden wir ja hören, was mit
Aspira geworden ist.“

Klar im Sonnenschein lagen jetzt die Gletscher und das Tal mit den
grünen Matten und den dunklen Fichtenwäldern und dem blauen See.
Und der Morgenwind strich über Flechten und Fels.

„Kommt spät,“ brummte der Fels. „Die Sonne steht schon hoch.“

„Ich mußte mich solange verweilen, wenn ich alles selbst sehen
wollte. So schnell geht das nicht, wie man sonst wohl einmal einen
Menschen abstürzen läßt. Sämtlichen Luftgeistern hatte es Migro
selbst eingeschärft, sie mußten den Menschen aufs zarteste
behandeln. Denn wenn ihm irgend ein Schaden zugestoßen wäre, so
hätte Aspira darunter zu leiden gehabt.“

„Hier der Fels behauptet,“ mischten die Flechten sich ein, „Aspira
sei zur Strafe Mensch geworden. Wir hatten Sie aber dahin
verstanden, daß es zur Belohnung sei.“

„Unterbrechen Sie gefälligst den Morgenwind nicht,“ polterte der
Fels. „Er spricht zu mir.“

„Lassen Sie mich darüber hinweg fächeln,“ säuselte der Morgenwind.
„Keins von beiden ist ganz zutreffend. Es handelte sich vielmehr um
eine Bitte von Aspira. Was aber diesem etwas wolkenhaften Wunsche
zugrunde liegt~–“ der Morgenwind zog einen kleinen graziösen
Wirbel, der alles sagen sollte, was er nicht wußte – „so darf man
sich darüber nicht aussprechen. Es handelt sich da um Verhältnisse,
die nur den höchsten Kreisen bekannt sind. Kommen wir lieber zur
Sache.“

„Das denk' ich auch,“ brummte der Fels. „Wie war's denn?“

„Zwei Tage hatte Aspira Bedenkzeit, aber sie blieb dabei, ein
Mensch zu werden. Heute war der Morgen des dritten Tages. Vom
Blankhorn bis zum Großblick lauerte die ganze Wolkensippe schon vor
Sonnenaufgang, daß ein Mensch sich blicken ließe, der König Migro
für Aspira gut genug schien. Aufgelöst hatten sich alle in der
Luft, daß es ganz klar war und das erste Frührot glückverheißend
heraufdämmerte.“

„Werden Sie nicht zu poetisch,“ murmelten die Flechten leise.

„Wenn Sie es gesehen hätten, das schöne Menschenweib! Dort drüben
kam sie heraufgestiegen auf dem Wege von Schmalbrück am Abhange des
Langbergs. Dann betrat sie das Band über dem Gletscher, wo die
weißen Blumen stehen. Kräftig schritt sie dahin, den Bergstock in
der rechten, in der linken Hand trug sie eine Mappe. Und wo das
Band endet, ist eine Höhle im Felsen, da drinnen ist es dämmrig und
kühl. Sie aber setzte sich vor die Höhle auf einen Stein und hüllte
sich in ihr Tuch. Und mit den großen braunen Augen blickte sie
hinüber zum Blankhorn und den Eisriesen, ganz allein im weiten
Wolkenreich. Und die Sonne ging auf und strahlte auf die Kette, und
all die Gipfel leuchteten rosig, und die Augen des Mädchens
glänzten noch mächtiger und herrlicher als die schimmernden Riesen
der Luft.

Da winkte Migro. Und plötzlich stieg die Wolkensippe vom Gletscher
herauf als ein dichter Nebel. Und mir winkte er. Da mußte ich
hineinblasen mit meinem kältesten Atem und die Nebel vor die Höhle
treiben, daß die Schneeflocken und das schöne Menschenkind
wirbelten. Da stand sie auf und zog sich in die Höhle zurück und
lehnte sich an die Steine. Aber Aspira flog ihr nach und hüllte sie
ein, daß sie in Kälte erschauerte. Sie schloß die Augen und sank
langsam um, ganz sanft, denn Aspira schützte sie und trug sie. So
lag sie erstarrt, wie schlafend, in ihr Tuch gehüllt. Die Mappe lag
neben ihr, und die schönen, dunklen Locken quollen unter dem Hut
hervor, und ich habe sie bewegt, leise, ganz leise, als wären sie
lebendig.

Da floß Aspira in sie hinein und mischte ihr Wolkenherz mit dem
Menschenherzen.

Und Migro winkte wieder. Da mußte wir alle hinweg, eilends. Die
Wolken lösten sich auf und die Sonne schien wieder klar und die
Luft in der Höhle erwärmte sich.

Ich aber flog zuletzt fort und sah noch, rückwärts blickend, wie
die Gestalt des schönen Mädchens aus der Höhle trat und sich
umschaute. Sie sah aus wie der Mensch, der sich vorhin
hineinbegeben hatte, aber ich weiß, daß es Aspira war.

Ich grüßte sie mit meinem mildesten Hauche und eilte hierher.“

„Und Aspira –?“

„Blickt dort hin! Weiter unten! Schon am Waldrand – da steht sie
noch~–“

„Sie winkt!“

„Und jetzt, jetzt verschwindet sie hinter den Bäumen.“

„Horcht, horcht!“

„Es schwebt von ferne wie eine Menschenstimme~–“

„Es hallt wider von den Bergen – die Alten rufen Aspira den letzten
Gruß~–~–“

„Lebt wohl! Ich fliege.“

Und der Morgenwind hob sich aufwärts zum Schneehaupt des
Blankhorns.

Aspira aber wandelte zu Tal.

\section{Menschenstolz \ausaspirastagebuch}

%\subsection{Aus Aspiras Tagebuch}

Ein merkwürdiges Ding, so ein Menschenleib! Als ich sie da liegen
sah in der Höhle – mich, muß ich eigentlich jetzt sagen, Wera
Lentius, denn so heiße ich ja nun – da überkam mich eine große
Angst. In diesen kleinen, fest umgrenzten Körper sollte ich hinein!
Wie mochte ich dann über die Berge und Täler kommen? Wie sollte ich
dieses regelmäßige Hin und Her der Beine erlernen? Aber als ich nun
einmal darin war, seltsam, da war ich eben dieses Menschen-Ich. Da
verstand sich alles von selbst.

Wie es kam, weiß ich nicht, aber ich lehnte plötzlich aufgerichtet
am Höhleneingang. Nun stand ich eine Weile ganz still. Es sah alles
anders aus, als ich's gewohnt war, und doch wußte ich gleich, wie
alles zusammengehörte und was es war, der Gletscher und die Felsen
und drüben der Wald. Aber es waren auch nicht bloß der Gletscher
und die Felsen und der Wald, es war so unendlich, so verwirrend
vieles, das bei dem vertrauten Anblick in mir zugleich als etwas
Neues vorging. Neu nur für Aspira, bekannt mir schon als Wera
Lentius. Ich wußte nicht bloß, wie ich hier geschwebt und geregnet
und mich aufgelöst hatte, ich wußte auch, mit wem ich als Wera dort
unten gewandelt war und gesprochen hatte, und daß ich nun nach der
Pension Leberecht gehen wollte.

Freilich, wie sollte ich das machen? Aufschweben – ja da war kein
Ausdehnungsorgan da, kein Schwebemittel. Aber das war nur so ein
ganz flüchtiges Bedenken. Ich wollte hin, und da bewegten sich
meine Glieder, zogen sich zusammen und streckten sich und – ich
ging, den richtigen Weg auf dem schmalen Bande, sicheren Schritts.
Was ich dabei tat, ich wußte es nicht, und als ich darüber
nachdachte, begann ich zu straucheln. Nun verstand ich auch gleich,
daß die Menschen das alles machen ohne selbst zu wissen wie.
Merkwürdig! Und doch, ist es denn bei uns anders? Wissen wir denn,
wie wir es anfangen, uns aufzulösen oder zu schweben? Also in
diesen einfachen Verrichtungen des gewohnten Lebens ist kein
Unterschied zwischen Wolke und Mensch. Dazu brauchte ich nicht
Mensch zu werden. Es versteht sich alles von selbst. Alles?

Als ich weiter hinab auf den Fußweg gekommen war, begegnete mir der
erste Mensch. Es war ein altes Mütterchen mit einem Korbe. Sie
sagte Worte, die ich nicht verstand, doch ich wußte, daß es ein
Gruß sei. Und auf einmal klang es laut, daß ich zusammenschrak:

„Guten Morgen.“

Es war meine eigene Stimme, das wurde mir jetzt erst klar. Zum
ersten Male hörte ich meine Stimme. Ich habe eine Menschenstimme!
Wie sonderbar! Das wußte ich ja, ich wußte alles, was Wera wußte,
aber doch nur als Erinnerung. Nun das wirklich zum ersten Male zu
erleben in der Wahrnehmung! Das war etwas unbeschreiblich Neues.
„Ich will mehr hören! Ich will reden! Was soll ich denn sprechen?“
Alles das sagte ich laut vor mich hin.

Zu meinen Füßen lag das Tal. Drüben im Grünen die hellen Häuser von
Schmalbrück, zur Linken davor der sonnenbestrahlte See, und von
meinen Lippen klang es:

\begin{verse}
„Auf der Welle blinken\\ Tausend schwebende Sterne;\\ Weiche Nebel
trinken\\ Rings die türmende Ferne;\\ Morgenwind umflügelt\\ Die
beschattete Bucht,\\ Und im See bespiegelt\\ Sich die reifende
Frucht.“
\end{verse}
Ich berauschte mich am Wohllaut der eigenen Stimme. Und dann rief
ich jubelnd hinaus:

\begin{verse}

„Wie im Morgenglanze\\ Du rings mich anglühst,\\ Frühling,
Geliebter!\\ Mit tausendfacher Liebeswonne\\ Sich an mein Herz
drängt\\ Deiner ewigen Wärme\\ Heilig Gefühl,\\ Unendliche Schöne!“
\end{verse}
Was diese Wera alle wußte! Das konnte sie erklingen lassen! O, es
ist doch schön, ein Mensch zu sein und eine Stimme zu haben. Und
das hatte ich nun alles, ich war ja Wera. Eines nach dem andern
fiel mir ein, was zu den Versen gehörte. Ein großer Menschendichter
hatte sie zuerst gesagt. Ich wußte, wie er hieß und wann und wo er
gelebt hatte; ich wußte wo das Buch in meinem Zimmer stand und wie
es aussah. Wie oft hatte ich darin gelesen! Ich sagte mir die Verse
noch einmal. Aber ich weiß nicht – als ich nun nicht bloß im Klange
schwelgte, als ich mir überlegte, was das bedeute und sagen wolle,
da war es, als stockte etwas in mir.

„Frühling, Geliebter!“ Gab es in Weras Seele Dinge, die mir noch
nicht zugänglich waren? Daß jetzt im letzten Drittel Juni nicht
Frühling war, störte mich nicht, das war Phantasie, das verstand
ich. Aber „Geliebter“ und „Mit tausendfacher Liebeswonne“? Das
waren Worte, Klänge, zu denen mir ein innerer Nachhall fehlte. Es
war mir wie mit dem See, als ich jetzt durch den Wald schritt. Ich
wußte, dort hinter den Bäumen lag er, aber ich sah nicht der Welle
Blinken und die tausend schwebenden Sterne –~– Wera mußte etwas
gesehen haben, als sie jene Worte sich einprägte, mir aber war hier
eine Leitung unterbrochen. War ich noch zu sehr Aspira, noch zu
wenig mit Weras Seele gemischt? Doch das mußte sich ja finden.

In meiner Weraseele wirkten die Verse fort, Erinnerungen zogen
herauf, Gedichte hörte ich erklingen, die mir galten. An fremdem
Ort sah ich mich Hand in Hand gehen mit einem andern Menschen, ich
wußte jedes Wort, das er gesagt, und – nein, was die Menschen für
seltsame Sitten hatten! Es mußte vermutlich sehr schön sein, und
doch – es war wie ein Bild ohne Farbe. Ich fühlte nichts dabei, ich
fand keinen Sinn darin, keinen Zusammenhang mit meinem Denken. Aber
das kam wohl daher, weil ich nur die Erinnerung kannte. Wenn ich's
einmal erlebe in der Wahrnehmung erfasse, dann werd' ich's schon
verstehen. So war's ja auch mit meiner Stimme. Erst als ich sie
gehört hatte, freute sie mich.

Eins aber hatte ich doch gelernt. In Verlegenheit würde ich nicht
kommen. Als Wolke hatte ich mich gefürchtet, wie das sein würde,
wenn ich mit Menschen zusammenträfe, ob ich mich richtig würde
benehmen können. Jetzt wußte ich, daß ich gehen kann, mich bewegen,
sprechen, daß ich wohl für die Menschen genau bin wie Wera. Ich
bin's ja doch auch.

Ach, ich weiß ganz furchtbar viel! Vorlesungen habe ich gehört und
Bücher gelesen und Versuche gemacht – was fällt mir nicht alles
ein! Gut, daß Wera so fleißig war, ich hätte es sicher nicht
zustande gebracht.

Sollte ich jetzt gleich unter die Menschen gehen? Es war noch früh
am Tage. Meine Weraseele sagte mir, daß dies die Zeit nicht sei, in
der ich nach Hause zu kommen pflegte. Ich wußte, daß ich meine
Zeichenmappe mitgenommen hatte und ein Buch, um mich im Freien zu
beschäftigen. Alles dies floß mir als Erinnerung durcheinander,
eine Vorstellung verdrängte die andre. Wir traumhaft schritt ich
auf dem schmalen Fußsteige dahin. Da klang das Rauschen des
Weißbachs mir ans Ohr. Immer näher ging ich hinan, schon stand ich
auf einem Felsstück dicht über dem weißen Schaum. Es war mir, als
sollte ich hineingleiten, ich mußte mich erst wieder besinnen, daß
ich Wera sei.

Nein, so ging das nicht weiter. Ehe ich zu den Menschen hinabstieg,
mußte ich mein Bewußtsein selbst in festere Ordnung bringen. Ich
mußte erst einmal versuchen, die Welt mit Weras Augen anzusehen.
Denn bis jetzt war mir ja alles nur wie zufällig entgegengekommen.
Ich wollte in Weras Seele lesen wie in einem Buche. Sie war ja
jetzt die meine.

Ich streckte mich auf das weiche Moos. Der Bach rauschte weiter.
Sonnenlichter fielen durch die Zweige der alten Arven und spielten
auf den zarten grünen Blättchen des Mooses neben mir. Was taten
sie? Was hatten sie mit mir zu tun? Sie spielten?

Nein! Plötzlich fiel es wie ein Schleier von meinen Augen. Das
Menschenhirn arbeitete in mir. Was es sich erarbeitet hatte durch
zahllose Geschlechter in Millionen von Jahren, auf einmal ging es
in mir auf, stieg empor als Gedanke, groß, unendlich, klar und
folgerecht, das Geheimnis des Gesetzes! Und wußte nichts mehr von
Wera noch Aspira, nichts von Menschen- und Wolkenseelen. Ich war
nur ein Teil dieses machtvollen Zusammenhangs, dieses gewaltigen
Werdens, das in meinem Menschenhirn sich ordnete.

Ich war die Welt, die sich selbst erkennt; der Teil der Welt, darin
sie sich erkennt.

Ein Neues, ein Ungeahntes erfüllte mich.

Die Sonnenlichter spielten? Nein, sie spielten ja nicht, sie
arbeiteten.

Von den fernen, fernen Sonne, wo glühende Gase wogten und sich
preßten, drängten sich die Schwingungen durch den Weltraum und in
die Zellen der zierlichen Blättchen. Und die grünen Körnchen des
Chlorophylls schwangen mit ihnen im Takte und ihre Atome tanzten
den geregelten Reigen. So erhielten sie die Kraft, die Kohlensäure
zu spalten. Da riß der Zellsaft die Kohle an sich, da eroberte sich
die Pflanze den Stoff, aus dem sie sich aufbaute, das kleine Moos
wie die hohe Arve. Ich sah die Werkstatt des Lebens.

Hier war das Reich meines Vaters Migro zum Quell geworden alles
Lebens, das auf dieser Erde erwachsen war bis zu diesem
verwickelten Organismus meines Leibes. Das sangen die kleinen
Körnchen des Blattgrüns leise meiner Aspiraseele, für mein Hirn
aber arbeiteten sie, getrieben von den Schwingungen des Äthers, nur
als Maschine, die allein auf der ganzen Erde imstande war, die
Elemente zum lebendigen Plasma zu verbinden. Hier liegt der
Ursprung aller Nahrung, durch die erst das wimmelnde Heer der Tiere
seinen künstlichen Nervenleib sich aufzubauen vermag.

Wie oft waren meine Tropfen niedergesunken auf das Moos und hatten
sich im Boden verloren. Und was der Pflanzenkörper nicht aufsog und
zurückhielt, das entwich wieder in die Mutterluft und stieg als
Wasserdampf empor. Und je höher ich stieg, um so dichter ward ich,
bis ich in der Kühle mich wieder sammelte in feinen Tröpfchen und
als Wolke über die Berge zog. So hatte ich's getrieben durch
ungezählte Jahrtausende in ewigem Kreislauf. Und ich hatte gemeint
zu spielen.

Eine freie Wolke war ich, König Migros Tochter Aspira. Und wenn ich
zur Höhe zog und in den zierlichen Eiskristallen erstarrte, und
wenn ich mich sammelte im weißen Firn und nach Jahren zu Tale glitt
im schimmernden Gletscher, und wenn ich im Bache entrauschte und im
See mich wiegte und wieder im Sonnenstrahl mich aufschwang zur
Wetterwolke, die den Blitzstrahl entsandte, immer wußte ich nur,
daß ich spielte, daß ich tat, was mir der Augenblick eingab –~– Was
auch sollte ich sonst sein?

Aber nun ich dich habe, du liebes, kluges Weraköpfchen, nun weiß
ich's besser. Ach, ich bin ja so unendlich viel mehr! Ich glaubte
zu spielen, nun jedoch weiß ich's, auch ich arbeitete, arbeitete
wie die leuchtenden Strahlen auf dem Moos, wie die
hämmerschwingenden Männer am Felsen. Was in mir waltete, das warst
du, Schöpferin Natur, war dein heiliges Gesetz zwingender
Gestaltung.

Alles berechnest du wie eine weise Verwalterin in deinem ungeheuren
Haushalt, auch wo du zu verschwenden scheinst. Nicht länger konnte
ich unsichtbar bleiben in meiner Gasform, als Temperatur und
Dichtigkeit der Luft es zuließ. Nicht eher konnte ich mich zu
Tropfen ballen, bis der Raum mit meinem Dampfe gesättigt war, oder
ich mich anklammern konnte an die Ionen der Luft. Dann zwang mich
die Oberflächenspannung zur Kugelform. Nicht höher konnte ich
schweben, als der aufsteigende warme Strom der Luft mich hob, und
mit jedem Meter Steigung verbrauchte ich mein bestimmtes Maß an
Wärme, um mich auszudehnen. Und nicht eher vermochte ich den
Glutball zu schleudern, bis nicht die Ladung meiner Tröpfchen die
vorgeschriebene Spannung erreichte. Und nicht früher durfte ich
niederregnen, bis meine Tropfen zur angemessenen Größe
zusammengeflossen waren~–~–

Aber nicht zwecklos stürzt' ich hinab im geglaubten Spiele des
Ergusses. Von den Felsen wusch ich das lose Gestein, den
fruchtbaren Boden des Tales zu ebnen. Ich tränkte das Moos und die
grünenden Matten, die Bäume des Waldes und drüben im Lande das
wogende Feld. Den Druck der Schwere sammelte ich in Bach und Fluß
und hob die Lasten und drehte die Räder der Menschen – ich
arbeitete.

Arbeiten! Versteh' ich dich ganz, du königliches Wort? Soll ich
trauern, daß ich nun weiß, warum ich spielte? Nein, dieses Gesetz
der Arbeit ist kein Zwang, um den ich klage, es ist mein Wille, den
ich achte. Wie wärest du möglich geworden, Wera, du schönes, kluges
Menschenkind, wenn nicht du, wenn nicht vor dir und um dich die
Millionen und aber Millionen gearbeitet hätten mit ihrem Gehirn,
mit ihren Händen, um dieses große Werk weiter und weiter zu
fördern, das man die Menschheit nennt? Wenn ihr nicht
aufgespeichert hättet in immer neuem Mühen, was ihr saht, hörtet
und faßtet?

Was ist so ein Menschenhirn doch für ein köstliches Ding! Ein Ding
ist zu wenig gesagt. Es ist ja die Welt, die ganze Welt, es ist
ihre Einheit. Was zurückliegt in jenen Zeiten, da selbst wir Wolken
noch nicht unsre Einheit als gestaltete Wesen gewonnen hatte, das
vermag solch ein Gehirn heraufzuführen und zu erforschen in der
Arbeit seiner Zellen. Die Geschichte des Erdballs liest es aus den
Spuren des Vergangenen und aus dem ergründeten Gesetz des ewig
Lebendigen. Was da war und was da sein wird, verbindet es zur
mächtigen Gegenwart.

Das ist eine ganz andre Verbindung, die sich hier erlebt, als wir
Wolken sie erleben in unsrer Elementenseele, in unsrer
Selbstdurchdringung.

Die Welt hat sich um Menschenhirn noch einmal sich selbst
gegenübergestellt. Sie liest sich darin wie in ihrem Tagebuch,
geschrieben in der Sprache des Gesetzes, nicht so umfassend, wie
sie lebt, aber um soviel klarer, nicht so verschwimmend, aber sich
selbst bestimmend.

Nun weiß ich, warum ich in diesen festumgrenzten Leib hineinwandern
mußte mit meinem Wolkenherzen, um zu lernen, was der Menschen Macht
und Wesen ist. Nur ein solcher Zellenleib konnte sich diese
Eigenwelt erbauen. Jede Wirkung von außen erkämpft sich ihren
eigenen Weg hinein in das Zentrum, schließ sich mit allen andern
zusammen und wirkt wieder hinaus in die Welt. Und all die einzelnen
Menschenhirne, wie ich nun so stolz eines in mir trage, ich, Wera
Lentius, die wirken ebenso eins aufs andre, alles zusammen, die
bilden die große unvertilgbare Einheit, ein Volk, die Menschheit.
Nun sammeln wir da alle Kräfte der Natur zu unserm Werkzeug. Durch
sie arbeiten wir, denn wir allein wissen, warum wir es tun. Was auf
gut Glück gelang hier und da im Spiel der Elemente, das Gehirn
weist ihm die Bahn, daß es das Nützliche leiste, das gewußte Ziel
bewußt erreiche.

Ein Satz fällt mir ein aus dem Buche, das dort in meiner Mappe
liegt. Ich brauche es nicht aufzuschlagen.

„Die Verwandlung des blinden Naturgeschehens in bewußtes Schaffen
ist nichts anderes als die Kulturentwicklung selbst. Das Mittel,
sich selbst zu verwirklichen, ist der Vernunft allein in der Natur
gegeben.“

Ja, das ist's. Ich trat in die Menschheit hinein, nicht um die
Natur zu verlieren, sondern sie herrlicher zu gewinnen auf der
höheren Stufe, die man Vernunft nennt. Ein Sinnenwesen war ich und
ich bleibe es, aber nun bin ich auch noch ein vernünftiges Wesen.
Bis hinein in deine unendlichen Fernen, hoher Äther, der um die
zahllosen Welten wirbelt, greife ich mit den Armen der Natur, um
dich zu schließen an mein mutvolles Herz, dich zu durchdringen mit
dem heißen Atem meines vernünftigen Willens. Mein bist du! Mit dir
leb' ich, von dir fordre ich mich selbst!

Ziehet hin, ihr Wolkenschwestern, ich kann nicht mehr mit euch
ziehen und regnen und blitzen, aber ihr zieht und regnet und blitzt
für mich als die Werkzeuge meiner Arbeit. Eure Kräfte lenke ich,
daß geschehe, was dem Werke der Vernunft dient. Ihr ehrwürdigen
Häupter der Berge, schaut nicht zürnend herab auf eure entflohene
Tochter! Wir kränken euch nicht, wenn wir den siegenden Fuß auf
eure Häupter, in das Geheimnis eurer Tiefen setzen. Wir gliedern
euch dem großen Ziele des Planeten nur auf eine edlere Weise an,
als ihr es vermögt in eurer erhabenen Ruhe. Rauschender Bach,
spiegelnder See, geliebte Gewässer, spielet weiter im Wechseltanz
von Luft und Sonne! Ich arbeite in euch, ich lenke euer Spiel, ich
hauche heilige Schöpfung des Gesetzes in euern wirren Reigen.

Ich trat auf die Brücke der Erkenntnis.

Was warntest du, Hoher, vor dem Leide des Schöpfers um sein Werk?
Das Glück soll der verlieren, der auf die Brücke tritt?

Was war mein Glück im Spiel der Elemente?

Jetzt kenne ich das Glück. Das Glück ist der Stolz, ist das Wissen
um die Macht. Ich wandle auf dieser festen Erde Schritt für
Schritt, und dennoch liegt die Welt zu meinen Füßen, Bergeshäupter
und Wolken und ewige Sterne! Denn ich umfasse euch in meines
Menschenhaupts gebietendem Gesetze. Ich habe mehr als ihr alle,
mehr als ihr Geister der Natur.

Ich bin glücklich, denn ich bin ein Mensch!

Ich ehre die Menschen!

\section{Nach dem Tunnel \ausaspirastagebuch}

%\subsection{Aus Aspiras Tagebuch}

Ich war hochgesprungen. Ein unbeschreibliches Gefühl des Glückes
und des Stolzes erfüllte mich. Ich wollte zu den Menschen, zu den
andern Menschen, zu meinen Schwestern und Brüdern. So schritt ich
rasch durch den Wald und weiter auf dem Wege.

Da stand ein Wegweiser. Was bedurfte ich seiner? Ich war noch
Aspira genug, um mich zurechtzufinden. Aber ich konnte auch lesen.
Das mußte ich doch gleich probieren.

Der Wegweiser hatte drei Arme. Auf dem einen stand: „Zum
Weißbachfall und Gletscherblick.“ Aus dieser Gegend kam ich. Der
zweite zeigte die Inschrift: „Promenadenweg nach Hotel Leberecht.“
Dort wohnte ich. Ich wollte in den Weg einbiegen, da fiel mein
Blick auf den dritten Arm:

„Zum Langbergtunnel. Unbefugten verboten.“

Da stutzte ich. Warum? Doch es fiel mir gleich ein. Der Weg führte
nach der Stelle, wo ich als Wolke die arbeitenden Menschen
beobachtet hatte. Von dort dröhnten die Sprengschüsse, erklangen
die Hammerschläge. Dort sollte man nicht hingehen, weil
Steintrümmer herabrollten~–~–

Und es fiel mir noch mehr ein. Es war bei der Mittagstafel im
Hotel. An meinem Tische, an der Ecke mir schräg gegenüber, sitzt
ein Herr mit dunklem Haar und Bart. Er sieht klug aus, sehr ernst,
aber gutmütig. Er kommt immer erst, wenn wir schon beim zweiten
Gange sind, und wenn das Dessert herumgeht, steht er schon wieder
auf und verschwindet. Niemals richtet er das Wort an seine
Nachbarn, schweigend sitzt er da und sagt höchstens höflich „bitte“
oder „danke“, wenn eine Schüssel weitergereicht wird. Fräulein
Bertilde von Okeley, die neben ihm sitzt, ärgert sich darüber. Den
„Schwätzer“ nennt sie ihn. Es ist der Ingenieur Martin, der die
Tunnelarbeiten leitet. Die Gäste wissen wohl, wer er ist, aber
niemand kennt ihn näher, denn er hat keine Zeit mit ihnen zu
verkehren; er hat sehr viel zu tun und ist abends ermüdet, wenn
sich die andern unterhalten. So oft ich gelegentlich zu ihm
hinübersah, bemerkte ich, daß seine großen, etwas träumerischen
Augen auf mir ruhten.

Neulich wollte ihn Fräulein von Okeley offenbar zum Reden bringen.
Sie rief ganz laut zu mir über den Tisch herüber:

„Heute war ich ein ganzes Stück auf dem verbotenen Wege nach dem
Tunnel zu. Dort sollten Sie einmal hingehen, Fräulein Lentius. Da
ist es wunderschön. Schade, daß man nicht weiterkann. Ich möchte so
gern einmal die Arbeiten sehen.“

Ich nickte ihr nur zu, denn ich rede nicht gern mit ihr. Man kommt
so schwer wieder los.

Auf einmal aber zu aller Erstaunen erhob der Ingenieur seine Stimme
und sagte mit Nachdruck:

„Verzeihen Sie, meine Damen und Herren, ich bitte Sie dringend,
vermeiden Sie den Weg, jedenfalls hinter der Stelle an der
Barriere, wo er aus dem Wald tritt. Oben im Steinbruch hinter der
Schiefklippe wird gesprengt. Darüber liegt brüchiges Gestein und es
kommen manchmal ganz unvermutet Felsblöcke nachgestürzt. Auch die
Förderbahn ist nicht ohne Gefahr zu überschreiten.“

Dann schwieg er wieder hartnäckig, und als das Fräulein von Okeley
ihn direkt fragte, ob er ihr nicht einmal den Tunnel zeigen wollte,
sagte er höflich aber bestimmt:

„Bedaure, gnädiges Fräulein, aber es ist nicht möglich.“

Seltsam, wie mir das alles so einfiel! Die Leute werde ich nun alle
zu Gesicht bekommen. Ich wollte also nach dem Hotel gehen. Und nun,
ich weiß nicht, auf einmal fühlte ich in mir einen merkwürdigen
Widerspruch. Meine Wera-Erinnerung und meine Aspiraseele wollten
nicht recht stimmen. Gerade diese Arbeiten am Tunnel wollte ich
doch sehen, sie hatten mich ja vornehmlich zu dem Wunsche
angetrieben, die Menschen und ihr Werk genauer kennen zu lernen.
Und das gerade sollte ich als Wera nicht dürfen? Und ich bekam eine
unwiderstehliche Lust, nun doch hinzugehen.

Hatte ich noch Zeit vor Tische? Aber, ich besaß ja eine Uhr. Ich
zog sie hervor. Sie zeigte 9~Uhr 15~Minuten. O, das war noch viel
Zeit, wenn ich auch mit meinen Werabeinen bis zum Tunnel noch eine
gute halbe Stunde brauchte. Um 12~Uhr mußte ich zu Hause sein, denn
ich mußte mich noch umziehen – Komisch, was man als Mensch für
Rücksichten zu nehmen hat! Als Mensch!

Wie gleichgültig sind all die kleinen Mühen gegen das große, große
Glück! Ich hielt die Uhr noch in der Hand. Was ist das für ein
Wunderwerk! Ich führte sie ans Ohr und lauschte ihrem leisen
Ticken. Ich drückte die Lippen auf das kleine, glatte, glänzende
Ding –~– Es war mit ein heiliges Symbol der Menschenmacht! Ich
konnte denken, an was ich wollte, ich konnte stehen und gehen und
schaffen, ich konnte schlafen – das kleine Ding an meinem Gürtel
arbeitete und wachte für mich, es zählte den unaufhaltsamen Schritt
der unendlichen Zeit. Den gleichmäßigen Umschwung des Erdballs trug
ich in der zierlichen Kapsel, das Maß des Weltalls gehörte mir –~–
Losgelöst von allem Schwanken wogender Wolkenseelen und flüchtiger
Menschenneigung zeigte mir der kleine unbestechliche Richter das
große, milde Antlitz des Gesetzes.

Ihr glücklichen Menschen! Wohin ihr blickt, was ihr berührt, die
Falte des Kleides, die haftende Nadel, hier der stützende Stock –
spricht euch nicht jede Kleinigkeit zu jeder Minute von eurer
Größe, von dem kunstvollen Werk der gemeinsamen Arbeit, das euch
entlastet von der Anstrengung der eignen Hand, euch frei macht für
immer neues Schaffen? Denkt ihr nicht daran? Und ich bin ein
Mensch!

In meiner Seligkeit hatte ich kaum gemerkt, wie der Weg unter
meinen Füßen dahinflog. Ich hatte den Ausläufer des Langbergs
überquert, der Schmalbrück vom Tal der Festina trennt. Plötzlich
stand ich am Ausgang des Waldes.

Ja, es war schön. Dicht vor mir stürzte der Langberg steil ab zur
Schlucht, zwischen den braunen und grauen Felsen schimmerte
blühendes Gesträuch, unten donnerte die Festina schäumend zu Tale.
Und gegenüber aus dem Walde des Berghangs trat eine hellgraue,
gerade Linie und zog sich aufwärts am Rande der Schlucht,
verschwand hinter einem Felsvorsprung und lief dann ebenso glatt
quer über den tosenden Fluß, getragen von einem zierlichen Bogen,
dessen Gitter in der Sonne glänzte wie von Elfenbein geschnitzt.
Als gäbe es für sie kein Hindernis, so drang diese Linie durch Wald
und Fels und freie Luft, und unter ihr in der Tiefe brauste
machtlos die wilde Festina.

Ich hatte den hellen Streifen schon oft von oben erblickt, ich war
unter der Brücke hinweggerast in den Wogen des schäumenden Flusses,
aber erst hier von der Seite sah ich die gewaltige Größe und die
schlanke Eleganz des wunderbaren Baus – und ich sah mit andern
Augen – denn du warst ja jetzt auch mein Werk, du eisernes
Menschheitsband – ich bin ein Mensch!

Oder bin ich vielleicht noch mehr? Was hindert mich dieser
Balkenverschlag und diese Warnungstafel? Links läuft der Weg an der
trümmerbesäten Berghalde talaufwärts, schräg vor mir auf dem
steilen Abfall des Langbergs grüßt mich die morsche Schiefklippe.
Dort, wo die Brücke an der diesseitigen Bergwand endet, sah ich
Menschen beschäftigt. Hammerschläge, Maschinenklappern hallten
herüber. Dort ist der Eingang zum Tunnel. Schmal nur ist der Pfad.
Ich eilte ihn aufwärts. Ich hörte nicht auf den Warnruf meiner
Weraseele. Ich war doch wohl mehr als ein Mensch, denn ich fühlte
keine Schwere. Es war, als ob Aspira mich auf Wolkenschwingen trüge
und meinen Schritt beflügelte.

Da ein schmales Schienenband, steil vom Berge kommend, quer über
den Weg. Wieder eine Warnungstafel. Was tut's? Hinüber! Ich schritt
vor. Da sah ich, wie sich unmittelbar vor meinen Füßen zwischen den
Schienen ein Draht vom Boden hob, und zugleich donnerte und
rasselte es vom Berge, ein kleiner, schwer mit Steinen beladener
Wagen kam herabgeschossen – ich springe noch über den Draht und
will vorwärts eilen, aber das Kleid hat sich irgendwo verfangen –
ich sinke ins Knie – ein Menschenschrei hallt von drüben – und sehe
den Wagen unmittelbar vor mir – unvergeßlich prägt sich der Moment
mir ein – ein großer, breiter Felsblock vorn auf dem Wagen, der
mich zermalmen mußte – aber dieser Felsblock~–~–

Es mußte wohl alles in mir, was Wera war, in diesem Augenblick
bewußtlos sein vor Schreck, aber ich war ganz Aspira –~– Dieser
Felsblock grinste vergnüglich und blies mich an wie ein Sturmwind;
der riß mein Kleid los und warf mich vorwärts über die Schiene
hinaus – hinter mir polterte der Wagen vorüber, es klang mir wie
ein Lachen aus dem Rasseln und Knattern: „Das war ich, das war ich!
Dein Freund vom Langberg! Gib acht, Aspira! Das war ich, das war
ich!“

Und im Augenblick darauf hatte ich mich emporgerafft. Ich war
wieder Wera und mußte mich erst besinnen, was geschehen war. Ich
hatte mir keinen Schaden getan, nur ein Stück vom Saum des Gewandes
war abgerissen. Der Hut lag neben mir auf dem Wege. Ich hob ihn
auf.

Vom Tunnel her auf dem steinigen Pfade kam in großen Sprüngen ein
Mann angerannt –unbekümmert um seine gesunden Glieder schwang er
sich bergabwärts – er trug einen staubigen Arbeitskittel und einen
unbeschreiblich verbogenen Strohhut – und noch ehe er mich erreicht
hatte, rief er atemlos:

„Um Gotteswillen, Fräulein Lentius, sind Sie verletzt?“

„Nein, nein, ich danke, es ist gar nichts.“

Nun stand er vor mir und hatte den Hut abgenommen. Ich sah, wie er
tief atmete und das bleiche Gesicht sich rötete. Jetzt erst
erkannte ich in ihm den Ingenieur Martin.

„Gott sei Dank!“ sagte er. „Ich sah, wie Sie hängen blieben. Es ist
mir fast unbegreiflich, daß Sie sich noch losreißen konnten –~–
Gott sei Dank,“ sagte er noch einmal leise.

Nun fühlte ich mich beschämt. „Ich war sehr unvorsichtig,“ sprach
ich. „Es tut mir furchtbar leid, wenn ich Sie erschreckte.“

Er hatte sich gebückt und untersuchte das Gleis. Ich sah, daß er
ein Stück Stoff ablöste, aber ich weiß nicht, wo es hinkam. Dann
drehte er sich wieder nach mir um und begann:

„Sie sind zum Glück nicht an dem Draht hängen geblieben, sondern
nur an der zweiten Schiene. Das Kleid hatte sich in dem Stoß
festgeklemmt. Immerhin, es ist mir unbegreiflich, woher Sie die
Kraft nehmen konnten, sich noch weit genug zur Seite zu werfen –
die Wagen laden breit über die Schiene hinaus – Sie hätten
mindestens gestreift werden müssen.“

Ich war wieder ganz ruhig und mußte jetzt lächeln. „Sie nehmen's
mir hoffentlich nicht übel, daß ich noch hinüberkam?“

„O, gnädiges Fräulein, – ich war nur so sehr erschrocken, als ich
Sie stürzen sah – man muß doch versuchen, sich klar zu machen~–“

Er sah ganz rührend au, als müßte er sich entschuldigen. Ich gab
ihm die Hand und sagte:

„Ich muß wirklich um Entschuldigung bitten. Sie haben uns noch
neulich Mittag gewarnt, und ich habe mich eigentlich straffällig
gemacht~–“

„O, ich bitte, ich bin ja nur froh, daß –“

Er schwieg verlegen. Da sagte ich:

„Und wissen Sie, ich bin gar nicht durch eigne Kraft losgekommen.
Wenn nicht der Langberg~–“ Fast hätte ich mich verplaudert, aber
ich hielt noch inne.

„Wie meinten Sie?“

„Ich glaube – ich denke, es war der Luftdruck –“ wie war ich froh,
daß mir das einfiel – „der Luftdruck vor dem Wagen hat mich noch
rechtzeitig zur Seite geschleudert.“

Ich sah seinem Gesicht an, daß er dieser Erklärung nicht
beistimmte. Da er aber nichts sagte, so sprach ich weiter:

„Da ich nun doch einmal so leichtsinnig bis in Ihr Gebiet
vorgedrungen bin, wollen Sie da nicht die Güte haben, mich nun auch
bis an das Ziel zu lassen und mir die Arbeiten im Tunnel zu
zeigen?“

Er schwieg wieder eine Weile. Es tat mir eigentlich leid, daß ich
etwa verlangte, was er offenbar nicht gern tun wollte. Aber ich war
hartnäckig und schwieg auch. Endlich begann er:

„In den Tunnel können wir jetzt wirklich nicht hineingelangen. Die
erste Attacke ist noch nicht beendet, wir sind noch beim Schottern
– d.\,h. beim Herausschaffen des gesprengten Gesteins – so lange ist
es für Sie nicht möglich in den Tunnel zu gehen, die Förderwagen
versperren den Weg. Vor Mittag werden wir mit dem Ausräumen nicht
fertig.“

„Ich hätte so gern einmal die Bohrmaschinen gesehen und die ganze
Einrichtung.“

„Nun,“ sprach er zögernd, „bis zum Eingang könnte ich Sie schon
führen. Da könnten Sie immerhin allerlei sehen, obwohl –
eigentlich~–“

Er machte eine Pause. Auf einmal sah er mich mit einem leuchtenden
Blicke an und sagte treuherzig:

„Ich bring's halt nicht übers Herz, daß ich Ihnen schon wieder
Adieu sagen soll. Aber werden Sie nicht zu müde sein?“

„Müde?“ Ich lachte. „Das kenne ich nicht.“

„Aber der Sturz, der Schreck?“

„Kommen Sie nur!“ rief ich übermütig und sprang den Weg hinauf, den
er herabgekommen war. Zum Glück fiel mir gleich ein, daß ich ein
gelehrtes Menschenfräulein bin. Ich blieb stehen und drehte mich
um, war aber doch ein wenig erschrocken, als ich sah, wieder hoch
ich schon über ihm stand.

Er hatte inzwischen meinen Bergstock herbeigeholt und kam mir
schnell nach. Ganz ernsthaft schüttelte er den Kopf und sagte:

„Sie sollten nicht so eilen. Hier ist der Stock. Geben Sie mir Ihre
Mappe.“

Ich wollte nicht. Den Stock brauchte ich nicht und mit der Mappe
wollte ich ihn nicht belasten. Aber ich weiß nicht – er nahm mir
die Mappe aus der Hand, gab mir den Stock und schritt rüstig neben
mir her, und ich – sagte gar nichts. Es versteht sich alles von
selbst, dachte ich wieder – oder dachte ich gar nichts?

So gingen wir schweigend weiter. Immer lauter dröhnten die
Hammerschläge am Brückenkopf und das Klopfen auf den Steinen,
daneben ein tiefes Summen –~–Der Weg wurde breit und zertreten. Wir
schritten über Schienengeleise, zwischen Bauhütten und Schuppen
hindurch, an Arbeitern vorbei, die mich verwundert ansahen – vor
uns lag die Tunnelöffnung.

Plötzlich blieb der Ingenieur stehen und fragte mich:

„Kennen Sie die Körnbachschlucht?“

„Freilich,“ rief ich. „Sie haben ja die ganze Schlucht abgedämmt,
es ist ein See entstanden, und das Wasser kann nur noch durch den
Spalt im Körnstein, wenn es nicht durch Ihr finstres Abflußrohr~–“

„Sie kennen die Gegend so genau? Sie waren wohl schon häufig hier?
Aber ich glaubte nicht, daß Touristen je bis an den Körnstein
kämen. Von einem Spalt weiß ich übrigens nichts.“

„Er ist ja auch –“ ich wollte sagen, „unterirdisch“. Doch zum Glück
besann ich mich. Ich hatte mich schon wieder verplaudert. Was
sollte ich über meine Lokalkenntnis sagen? Wera war ja erst seit
zwei Wochen hier und noch nie am Körnstein gewesen.

„Ich hörte nur davon reden,“ behauptete ich kühn. „Bitte, erklären
Sie mir~–“

„Am Körnstein ist unser Sammelbassin für die Druckleitung. Sehen
Sie, durch dieses 110~Zentimeter starke Rohr strömt das Wasser aus
dem Körnbach zu uns herab. Und hier“ – er öffnete die Tür zu einem
Hause – „sehen Sie die Turbinen, die dadurch getrieben werden.
Diese hier erzeugt die Preßluft, die durch jene Röhrenleitung
gedrückt wird. Durch die Preßluft treiben wir unsre Bohrmaschinen
und lüften zugleich den Tunnel. Die Turbine im Nebenraum – hier
können Sie ruhige herantreten – ist mit einer Drehstrommaschine
gekuppelt und erleuchtet uns den Tunnel.“

\section{Der Ingenieur\ausaspirastagebuch}

%\subsection{Aus Aspiras Tagebuch}

Der Ingenieur führte mich von einer Einrichtung zur andern, durch
mehrere Arbeitsschuppen hindurch, und knüpfte überall seine
Erklärungen an. Und je länger er sprach, um so sicherer klang seine
Rede. Ich kannte den schweigsamen Mann kaum wieder. Er lebte
förmlich mit seinen Röhren und Turbinen, Hebewerken und
Bohrmaschinen – nein, das alles lebte in ihm, das waren ihm Kinder
und Kameraden, die bei der Arbeit halfen. Wenn mir etwas besonders
gefiel, freute er sich wie ein Vater, dem der Junge ein gutes
Zeugnis nach Hause bringt. Wie froh war ich wieder, daß Wera ihre
Zeit auf der Hochschule so gut benutzt hatte. Ich entdeckte in mir
so treffliche Kenntnisse in technischen Dingen, daß ich alles ohne
Schwierigkeit verstand. Wenn ich ihn etwas fragte, oder wenn er ich
etwas fragte – ich glaube, er wollte manchmal sondieren, ob ich
auch seinen Worten folge – dann sah er mich ganz glücklich an, daß
ich nichts Dümmeres sagte. Als ich mich nach den Sprengpatronen
erkundigte, da imponierte ich ihm sichtlich. Da war ich ganz in
meinem chemischen Fache und verstand mehr davon als er selbst. Ich
sah ihm sein Erstaunen an, aber er war zu bescheiden, mich
auszufragen.

„Wissen Sie,“ sagte er dann, „wenn die Sprengschüsse krachen, das
ist eigentlich mein schönster Moment.“

„Nun, ästhetisch ist es gerade nicht,“ bemerkte ich.

„Nein, aber ich möchte sagen – ethisch. Das wird Ihnen sonderbar
vorkommen. Sehen Sie, was wir hier tun, ist doch im Grunde eine Art
Kriegführung gegen diese ehrwürdige Ruhe des Gebirges, und wenn man
einmal Krieg führen muß, soll es auch ehrlich sein, mit eignen
Waffen, nicht mit solchen, die wir dem Feinde heimlich gestohlen
haben. Und diese chemische Energie des Sprengschusses, das ist
außer dem bißchen Handarbeit die einzige Wirkung, die wir aus
eignen Mitteln in den Berg hineinbringen. Die Sprenggelatine haben
wir hergestellt, die haben wir nicht aus diesem Gebirge. Aber alle
andre Energie haben wir ja aus dem Berge selbst entwendet. Sehen
Sie, diese Turbinen liefern uns sechzehnhundert Pferdestärken zum
Bohren und Transportieren. Wer gibt sie uns? Der Körnbach. Und wer
gibt uns den Körnbach? Der Berg. Ohne ihn hätten wir kein Gefälle.
Und so zwingen wir eigentlich den Berg, sich selbst zu durchbohren.
Ist das nicht Verrat?“

„O nein!“ rief ich sehr lebhaft. „Das erscheint mir ganz anders.
Wie kommt denn das Wasser auf den Berg? Durch die Wolke. Und wer
hebt die Wolke hinauf? Die Sonne. Das Bild vom Kampfe mit den
Elementen mag ich nicht hören. Die Elemente mögen freilich den
Menschen als Feind betrachten, aber sie sollten es nicht. Und der
Mensch darf nun schon gar nicht so denken. Nenne Sie diesen Kampf
lieber eine Erziehung, eine Leitung der Natur durch den Menschen.
Der Berg und das Wasser und die Wolken wissen nicht, was sie
können, der Mensch weiß es, und so zwingt er sie, seinen Willen
auszuführen. Er steht über diesem scheinbaren Kampfe, er vertritt
die hohe Macht, um derentwillen Berg und Bach und Sonne da sind und
dort dieser Zellenbau des Menschenleibes und die Vielheit der
Individuen. Daß Sie den Berg durchbohren, ist doch nicht Zweck; das
ist doch auch nur ein Mittel zu einem höheren Ziele, damit die
Verbindung geschaffen werde von Volk zu Volk, von Geist zu Geist;
damit das werde, was wir Kultur nennen und Freiheit. Der Berg weiß
davon nichts. Aber er soll es lernen. Er muß es lernen –– ich will
– ich meine – wenn er es wüßte, würde er sich nicht beklagen. Er
hätte kein Recht dazu. Wir leihen ihm nur eine neue Fähigkeit, eine
neue Macht, sowie die ganze, große, unendliche Natur auch jedem
kleinen Menschlein ihre Macht leiht~–“ Ich brach plötzlich ab. Ich
hatte mich so in Eifer geredet, – es war doch die Frage, die meine
ganze Existenz erfüllte. Aber was sollte er denken? Er stand da und
sah mich mit großen Augen verwundert an, als fände er keine Worte.
Ich wollte von dem Thema abkommen. So fragte ich ganz unvermittelt,
ob der Tunnel nicht auch von der andern Seite in Angriff genommen
sei.

Martin schwieg noch immer. Dann sprach er etwas von
Terrainschwierigkeiten, gleichmäßiger Steigung und Wasserabfluß,
und sagte schließlich, das könne er mir eigentlich nur an den
Zeichnungen deutlich machen. Ob ich nicht einen Augenblick in sein
Bureau eintreten wolle. Dort könne er mir die Pläne vorlegen.

Ich stand in dem hellen Zimmer eines schmucklosen
Bretterverschlages, auf einem großen Tische waren Pläne
ausgebreitet –~– Noch vor wenigen Tagen konnte ich nicht verstehen,
wie es möglich sei im voraus zu bestimmen, daß etwas gerade so und
nicht anders geschehen werde, und heute lagen alle diese Entwürfe
vor mir – und nun begriff ich ganz deutlich, warum dieser Mann
sagen konnte: „Sehen Sie, hier kommen wir heraus.“

Ich beugte mich einige Zeit über ein Blatt, und als ich mich
aufrichtete, bemerkte ich, daß der Ingenieur verschwunden war. Aber
schon trat er wieder herein – jetzt in seinem Straßenrock – und
fuhr fort, als wenn er sich gar nicht unterbrochen hätte:

„Im Herbst, denke ich, schlagen wir durch. Das Gestein ist günstig
und wir können schnell vorstoßen. Vorausgesetzt freilich, daß eine
Befürchtung nicht eintrifft~–“ Er brach ab.

„Was ist das für eine Befürchtung?“

„Ich sollte darüber vielleicht nicht sprechen. Es ist möglich, daß
wir Erfahrungen machen, die uns noch zu einer Änderung der
Tunnelachse am andern Ende zwingen. Deshalb haben wir dort erst
Vorarbeiten. Doch etwas Bestimmtes läßt sich noch nicht sagen. Die
Herren Geologen beruhigen uns ja. Dennoch bin ich manchmal in
Zweifel – aber ich langweile Sie~–“

„Nein, wirklich nicht! Im Gegenteil, ich bitte Sie recht sehr~–“

Er suchten einen großen bunten Bogen hervor.

„Sehen Sie,“ sagte er, „das ist ein geologisches Profil. Soweit Sie
hier die Farben durchgeführt sehen, beruhen die Angaben auf den
Ergebnissen der Bohrungen und sind sicher. Aber dort bemerken Sie
einige Stellen, wo wir nur auf Vermutungen angewiesen sind. In
diesen verzwickten Bergen kommen nämlich manchmal Verwerfungen vor,
auf die kein Mensch gefaßt sein kann. So ein Berg hat sich bei
irgend einem Jugendübermut eine innere Quetschung zugezogen und
alles ist durcheinandergedrückt.“

Ich lachte. „Sie müssen doch den Bergen auch ihr Vergnügen gönnen,“
sagte ich. „Glauben Sie mir, das tut den harten Gesellen nicht weh,
das ist ihnen eine angenehme Abwechslung, so als wenn Sie zwischen
Ihren Berechnungen ein heiteres Märchen lesen.“

„Oder es erzählt bekommen,“ antwortete er lustig. „Das freut mich
sehr, daß die Verwerfungen den Bergen keine Schmerzen machen. Dann
werden sie hoffentlich auch das Anbohren nicht übel nehmen und
keine unangenehmen Überraschungen bereiten.“

Darüber dachten die Berge freilich anders. Ich unterdrückte aber
natürlich meine Meinung.

„Ja,“ fuhr er fort, „wenn ich ein Berg wäre, würde ich mir aus
einer alten Schramme im Innern wohl auch nichts machen. Der Mensch
darf es ja auch nicht. Aber – wenn ich nun durch einen solchen Berg
hindurch soll und an eine Bruchstelle komme, und statt des festen
Gesteins preßt mir der Berg einen Brei von Schlamm und heißem
Wasser entgegen und ersäuft mir meine Maschinen – dagegen hilft mir
keine Charakterstärke. Wie Sie hier sehen, besitzt dieser Teil des
Langbergs, unter dem wir in unserm letzten Tunneldrittel hindurch
müssen, die Marotte, mitten zwischen seinen Gneislagern ein
Kalkgeschiebe eingepreßt zu haben. Es tritt oben zutage, wo es die
bizarren Felsgruppen bildet, die man die Gamssteine nennt. Sie
kennen Sie? Die Bohrungen zeigen zwar, daß die Mächtigkeit der
Schicht nach der Tiefe schnell abnimmt, und so hoffen wir, daß sie
nicht bis in unser Niveau herabreicht. Aber wir können nicht
dahinter kommen, ob sie nicht noch allerlei Verwerfungen hat, die
von unten her uns stören könnten. Schneiden wir eine solche an, so
müßten wir bei dem hier herrschenden Druck darauf gefaßt sein, daß
die Schicht zermalmt und mit heißem Wasser durchtränkt ist, und
dann – dann kann uns der Spaß des Herrn Langberg im besten Falle
ein Jahr Arbeit und ein paar Millionen mehr kosten.“

„O, nein, nein!“ rief ich. „Das darf nicht sein, das darf der
Langberg nicht.“

Verlegen brach ich ab. Ich kannte ja das Kalkband. Wenn wir oben
auf dem Langberg regneten, dann huschten wir immer durch die
Spalten bei den Gamssteinen hinab und kamen unten an einer Stelle
des Silbertobels wieder heraus. Und dabei hatten wir die schönsten
Kanäle und Grotten ausgewaschen. Tiefer hinab hatte ich mich
freilich nur einmal gelegentlich hinverirrt, da war allerdings ein
grauenhafter heißer Brei von zerquetschtem Gestein. Aber ob sich
dieser Teil überhaupt nach der Seite des Tunnels und wie tief er
sich ins Innre des Berges hinzog, das wußte ich nicht. Ich wußte
nur, wo man sich in der zerklüfteten Kalkschicht wieder ans
Tageslicht hinauspressen konnte – die Stelle kannte ich – da würde
ich ja jetzt wohl feststellen können, ob sie über oder unter dem
Tunnelniveau läge. Ich wußte noch nicht, was zu tun sei, aber der
nette Ingenieur sollte keine Not haben, wenn ich ihm helfen konnte
– das nahm ich mir vor.

Während ich mir das überlegte, hatte ich gar nicht daran gedacht,
wo ich saß. Ich war erst lebhaft erschrocken und dann auf einmal
wieder sehr froh geworden in dem Gedanken, den Menschen vielleicht
nützen zu können. Aber freilich, sagen durfte ich ja nichts. Und
nun sprang ich plötzlich auf und rief ganz vergnügt:

„O, bitte, bitte, schreiben Sie mir doch einmal genau die Höhenlage
des Tunnels auf.“

Ich war mit meinen Gedanken so ganz im Berg-Innern gewesen, daß ich
den Ingenieur gar nicht beachtet hatte. Jetzt war ich erstaunt, daß
er ein Glas mit Wasser gefüllt hatte und zu mir sagte:

„Ich freue mich, daß Sie wieder vergnügt sind. Sie waren plötzlich
ganz bleich geworden, ich fürchtete, Sie hätten sich vorhin doch
Schaden getan. Aber jetzt haben Sie wieder Farbe~–“

Ich fühlte, wie mir das Blut ins Gesicht stieg.

„Nein, nein,“ rief ich, „ich bin ganz wohl – ich war nur wirklich
erschrocken bei dem Gedanken, daß Sie – daß der schöne Tunnel in
den Schlamm geraten könnte. Doch es wird ja nicht sein, und da bin
ich wieder froh. Aber das brauchte man mir nicht gleich
anzusehen.“

„O nicht doch,“ fiel er lebhaft ein. „Beklagen Sie das doch nicht.
Wir Geschäftsleute müssen ja leider so oft unsre Mienen in acht
nehmen, aber Sie –~– das ist gerade so schön, Sie sind das Leben
selbst, Sie sind wie die Natur, die sich nicht verstecken kann, wie
die Wolken um das Blankhorn, die in jedem Augenblick den Zauber der
Beleuchtung wechseln, und das macht so froh –~– ja – doch –
verzeihen Sie – Sie wollten die Höhenzahlen für den Tunnel – darf
ich fragen, warum?“

„Würden Sie es nicht ohne Frage tun, Herr Martin?“

Er antwortete nicht, sondern kramte in den Papieren. Er legte einen
großen Bogen Pauspapier über das Tunnelprofil, skizzierte mit
schnellen Strichen die Hauptlinien und schrieb ein paar Zahlen
daran.

Ich stand etwas verlegen dabei und schielte nach dem Glase Wasser.
Ich wußte, daß ich Durst hatte; gar zu gern hätte ich getrunken.
Aber das hatte ich noch nie wirklich probiert. Wenn ich es nun
ungeschickt machte? Zu dumm, daß man sich genieren kann! Und doch
paßte ich den Moment ab, wo er nur auf die Zeichnung sah, da griff
ich schnell nach dem Glase und trank es aus wie ein richtiges
Menschenfräulein – das hätte ich mir denken können, und doch freute
mich's so, daß ich leise lachte.

Er war gerade fertig und blickte auf. Eigentlich mußte er mich doch
für sehr dumm halten. Aber als er mit den Bogen überreichte, sah er
gerade so vergnügt aus wie ich.

„Ich danken Ihnen herzlich,“ sagte ich erfreut. „Ganz besonders für
Ihr Vertrauen, das freut mich so. Aber wo soll ich nun mit dem
großen Bogen hin?“

„Den legen wir in die Mappe. Ich darf doch?“

Und schon hatte er die Mappe geöffnet. Das hatte ich ja selbst noch
gar nicht getan, und wie das aussah, was darin war, darüber hatte
ich auch nur meine Wera-Erinnerung.

Er schlug die Mappe ganz auf, um das Blatt sorgfältig
hineinzulegen, und dabei kam eine fast fertige Zeichnung zum
Vorschein – ich erkannte gleich das Blankhorn über dem Gletscher,
mit einem Wolkenstreifen, den ich auch kannte, denn – das war meine
Lieblingsstellung –~– Er fragte gar nicht, ob er das Blatt ansehen
durfte, er betrachtete es einfach, lange, genau – Ich wagte gar
nichts zu sagen.

„Sind Sie Malerin?“ fragte er dann kurz.

„Nein, ich habe Chemie studiert.“

„Als Fach?“

„Ja, ich arbeite im Laboratorium von Rötelein.“

„O, da müssen Sie sehr glücklich sein.“

„Warum?“

„Daß Sie so etwas machen können. Einfach mit dem Stift. Es ist eine
große Stimmung darin.“

„Und wenn ich Malerin wäre?“

„Dann hätte ich Ihnen noch Einiges gesagt.“

„Was denn?“

„Wenn ich nun auch bäte, nicht zu fragen?“

„Aber gefällt Ihnen das Blatt? Würden Sie es annehmen, wenn ich mir
erlaubte, es Ihnen als Gegengabe anzubieten?“

„Sie könnten mir keine größere Freude machen.“

„Gern! Ich bin Ihnen sehr zu Dank verpflichtet.“

„Ich bitte, ich weiß nicht, wie ich Ihnen danken soll. Nun
vollenden Sie Ihre Güte, schreiben Sie Ihren Namen hierher.“

Er reichte mir einen Bleistift. Meinen Namen schreiben! Welche
Idee! Ich hatte noch nie geschrieben. Ich nahm den Stift, hielt ihn
richtig und machte schnell einmal die Bewegung meines Namenzuges
mit den Fingern. Dann brachte ich die Spitze auf das Papier, noch
einmal dieselbe Bewegung, und da stand es: Wera Lentius.

Als ich den Bleistift hinlegte, ergriff er meine Hand und küßte
sie. Und dann noch einmal. Aber anders, lange. Ich schauerte
zusammen. Es war mir nicht angenehm. Und – ich weiß nicht – es
hatte etwas Lächerliches für mich. Ich trat zurück.

Er sah mich erschrocken an. Ich wollte ihn nicht kränken. Er meinte
es gewiß sehr gut, nach Menschenart. Und ich dachte einen
Augenblick, ich müßte doch eigentlich auch das probieren, wie es
weitergeht. Aber er packte jetzt seine Pläne zusammen und ich meine
Mappe. Wir sprachen nichts. Dann überwand ich mich und trat auf ihn
zu. Ich gab ihm die Hand und sagte etwas von gehen müssen und Dank.
Er hielt sie lange fest und antwortete dann:

„Sie müssen mir erlauben, Sie bis an die Seilbahn zu begleiten. Ich
fühle mich verantwortlich.“

So gingen wir hinaus. Er trug wieder die Mappe. Am Turbinenhaus
kamen wir dicht an der großen Druckleitung vorbei. Ich klopfte mit
der Hand auf das Rohr und murmelte: „Wackerer Körnbach.“ Dann ging
es schnell bergab.

An der Seilbahn hielt er mich zurück.

„Wir müssen ein paar Minuten warten,“ sagte er. „Sehen Sie dort
oben die Fahne? Das ist das Zeichen, daß sofort ein Wagen
abgelassen werden wird. Und wenn er vorüber ist, dann muß und
zurück.“

\section{Zu den Menschen \ausaspirastagebuch}

%\subsection{Aus Aspiras Tagebuch}

Wir setzten uns an den Rand des Weges. Ich hielt es nicht aus, daß
wir so stumm waren, daß wir scheiden sollten wie in einer
Verstimmung. Er gefiel mir doch so gut, und ich war so glücklich.
Ich sah ihn freundlich an und sagte:

„Sie meinten vorhin, ich müßte recht glücklich sein. Ja, ich bin's
auch. Nicht wegen des bißchen Talent. Das freut mich höchstens,
wenn's andre freut. Ich bin glücklich, weil ich ein Mensch bin, der
sich als Herr der Natur weiß und doch mit ihr lebt, wir ihr
getreues Kind. Und darum bin ich Ihnen auch so dankbar für diese
Stunde. Denn ich hab' es wohlgemerkt, Sie denken auch so. Lassen
Sie die Elemente feindlich sein – Sie lieben Sie doch – nicht wahr?
Nicht bloß Ihre Maschinen sind Ihnen ans Herz gewachsen, auch der
alte Langberg und der Körnbach; und jedes Stückchen Schotter, das
Sie die Halde hinunterstürzen, ist Ihnen ein Stück von dem großen
Werke, an dem Sie arbeiten, und Sie lieben es, weil Sie's von einem
Ort zum andern schaffen können. Und so müssen Sie doch ebenso
glücklich sein? Das würde mich so freuen.“

Ich reichte ihm meine Hand hinüber, und er hielt sie fest. Er
blickte still vor sich hin.

„Ja,“ sprach er leise, „Sie sind glücklich, denn Sie sind
beglückend. Aber ich~–“

„Sie sollten es nicht sein? Ich das nicht Glück, was Sie in Ihrer
Arbeit um sich verbreiten? Wie Sie Ihre Mitarbeiter führen und
fördern? Wie Sie für andre schaffen und sorgen? Wie Sie heute
selbst diese unvorsichtige Touristin gütig aufnahmen? Und Ihr Werk,
wie es gelingt? Glück ist doch die Macht, wirken zu können, was man
wirken will. Und das können Sie.“

Er lächelte schmerzlich. „Das könnte ich? O~mein liebes, verehrtes
Fräulein! Gewiß ist das Glück, was Sie da nennen. Aber es ist nur
die eine Seite. Sie sind glücklich, weil Sie die andere noch nicht
kennen. Das höchste Glück des Menschen nannten Sie nicht – es
ist~–~–“

Sagte er nichts weiter oder verstand ich es nur nicht? Rasselnd
erhob sich der Draht zwischen den Gleisen. Von oben kam es dröhnend
herab, vorüber donnerte der Wagen. Wir blickten ihm beide nach.

Mit einem freundlichen Druck löste ich meine Hand und stand auf.

Allmählich verhallte der Lärm der Bahn. Ich überschritt die
Schienen. Er folgte mir hinüber und blieb noch einmal stehen.

Was kannte ich nicht? Was lag da noch im Grunde des
Menschenherzens? Meinte er es, wie der Hohe warnte? Das Leid des
Schöpfers um sein Werk? Ich mußte es wissen.

„Ich habe vorhin nicht verstanden,“ sagte ich. „Was meinten Sie mit
dem höchsten Glück?“

Er schüttelte den Kopf. „Es gibt kein reines Glück für den
Menschen. Aber das ist auch nicht einmal nötig, nicht nötig zur
Zufriedenheit. Doch den Menschen bindet nicht bloß die Natur, ihn
bindet vor allem der andre Mensch. Und das höchste Glück – das
wissen Sie doch – das ist zugleich das größte Leid – dieweil sich
Liebe doch von Leid nicht trennen läßt noch scheiden~–“

Er sagte das so traurig. Aber ich konnte mir nicht helfen, ich
mußte lachen.

„Wenn es nur das ist,“ rief ich, „dann machen wir uns keine Sorgen.
Was mit Leid verknüpft ist, kann unmöglich das größte Glück sein.
Darum also braucht man sich nicht zu kümmern. Ich fürchtete ganz
etwas anderes. Nochmals herzlichen Dank! Wir sehen uns doch heute
noch im Hotel?“

Er nickte nur mit dem Kopfe. Ich war schon einige Schritte fort, da
rief er erst: „Leben Sie wohl!“

„Adieu, adieu!“ rief ich zurück. Ich flog den Berg hinab. An der
Barriere erst hielt ich still. Ich drehte mich zurück und wehte mit
dem Tuche. Er stand noch immer an derselben Stelle und sah mir
nach. Jetzt lüftete er den Hut. Und dann war ich hinter der
Barriere und zwischen den Bäumen.

Es war Zeit, daß ich nach der Pension kam. Wie schnell verschwand
der Weg unter mir! Ich war selig, mehr noch als auf dem Hinweg. Ich
jubelte: Ein Mensch! Ein Mensch!

Ich hatte gefürchtet, er würde mir ein furchtbares Geheimnis
enthüllen von der Warnung des Hohen, vom Leide des Schöpfers um
sein Werk. Aber davon sprach er nicht, das mußte also doch so
schlimm nicht sein für den Menschen. Und so war meine letzte Angst
gehoben. Ich war ein Mensch, ich kannte die Welt, und dieser liebe
Mensch, den ich gefunden hatte, war gut zu mir.

Die Liebe? Mochte sie Glück oder Leid sein, was ging es mich an?
Die suchte ich nicht bei den Menschen. Die brauchte ich nicht.
Davon wollten auch die Wolken nichts erfahren.

Ich aber habe jetzt, was sie wollen: Erkenntnis!

Da war der Wegweiser. Glückauf! Zu den Menschen!

Und da waren die Menschen auch schon. Sie gingen auf den
Promenaden, sie saßen auf den Bänken. Unangenehm ist das Geschrei
und Gequiek der Kinder, das sie bei ihren Spielen verführen. Ich
eilte vorwärts. Meine Toilette war doch etwas mitgenommen, und
manche von den Menschen kannte ich, einige grüßten mich, die Herren
sehr höflich. Keiner sah so nett aus wie mein Ingenieur.

Zu seltsam, daß es zwei Arten von Menschen gibt, Mann und Weib. Ich
weiß es natürlich, theoretisch, aber wie ich sie nun vor mir sah,
verwunderte es ich doch. Und ich bin ein Weib? Warum? Wäre es nicht
besser gewesen, ich wäre in eines Mannes Körper gefahren? Ist es
nur Zufall gewesen? Doch ich habe es gut getroffen; es ist mir sehr
klar, was es für mich bedeutet, daß ich Wera Lentius bin. Ich habe
alles, was Menschen zum Glücke brauchen, ich bin gesund und jung –
ach, so furchtbar jung! Fünfundzwanzig Jahre! Was ist das gegen
mein ehrwürdiges Wolkenalter, das ich gar nicht kenne! Jedenfalls
lebte Aspira schon, ehe es noch Menschen gab. Und gelernt habe ich
so viel, und dumm bin ich auch nicht. Das darf ich sagen, denn es
ist ja nicht mein Verdienst, sondern das der schönen Wera. Wäre es
irgend ein verirrter Ziegenhirt oder Holzhauer gewesen, in den ich
gefahren wäre, so wäre ich jetzt der Peter oder der Hans. Und was
würde ich dann von den Menschen und der Wissenschaft für einen
Begriff bekommen? Höchstens der Ingenieur, der wäre ich vielleicht
noch lieber – aber ich bin's nun nicht.

Es ist wunderbar, daß man überhaupt ein „Dieser gerade“, ein
besonderes Ich ist. Wie steht es denn nun mit meinem Aspira-Ich?
Habe ich der armen Wera ihr Ich gestohlen? Nein, das ist gewiß
nicht richtig. Sie lebt ja weiter, sie ist es selbst, die hier ihre
Aufzeichnungen macht. Ich muß mich nur erst ganz daran gewöhnen,
daß ich Wera bin, vielmehr, daß es Wera ist, die mich aufgenommen
hat. Es ist nur etwas zu Wera hinzugekommen. Ein Stück Aspira ist
jetzt in ihr. Sie weiß nicht, daß es Wolkenseelen gibt und ein
Reich der Höhe, eine Welt des sonnigen Spiels, wovon sich das
kurzlebige Menschengeschlecht nichts träumen läßt, und daß eine
Wolkenseele sich sehnt, das Menschengemüt ganz zu verstehen, und
glücklich und stolz ist, nun Menschenleib und -geist zu besitzen.

Ja, ich bin Wera, und kam gerade so regelmäßig zur Mittagszeit nach
Hause, wie Wera es immer tat. Ich ging sofort auf mein Zimmer und
kleidete mich um. Das geschah ohne weiteres Nachdenken, ich wußte
überall Bescheid. Nun konnte ich's nicht verhindern, daß mich
manchmal ein komisches Staunen ankam, einen so eingekapselten
Gliederleib zu besitzen. Ich weiß ja, daß ich ohne ihn nicht der
denkende Herr der Natur sein könnte, aber einige Unbequemlichkeiten
bringt er doch mit sich, zum Beispiel das Waschen.

Ich war kaum mit meiner Toilette fertig, als die Tischglocke
ertönte. Nun sollte ich zum ersten Male das Essen aus Erfahrung
kennen lernen, denn bisher kannte ich's nur aus der Erinnerung. So
muß es den Menschen mit dem gehen, was sie nur in Büchern gelesen
und noch nicht selbst erlebt haben. Ich freute mich auf das Essen,
denn ich spürte den Hunger jetzt als wirkliche Wahrnehmung. So
beeilte ich mich, in den Speisesaal zu gehen.

Da ich sehr pünktlich kam, war ich die erste an unserm Tische. Die
Wirtin, Frau Leberecht, trat an mich heran und begrüßte mich sehr
freundlich. Ob ich denn nicht müde wäre? Ich wäre doch so früh,
schon vor Sonnenaufgang, weggegangen, als alle andern noch
schliefen. Ob ich denn auch richtig Frühstück bekommen hätte? Ich
sähe ja wunderbar frisch aus! Und dann – ich möchte entschuldigen,
es wären Herrschaften abgereist und neue angekommen, da hätte sie
mein Gedeck um zwei Plätze hinaufrücken müssen.

Ich warf schnell einen Blick auf die Schildchen an den Servietten
der Nachbarn und sah, daß ich jetzt Bertilde von Okeley gerade
gegenüber gekommen war, links saß der Alpinist und rechts –der
Ingenieur. Das freute mich. Jetzt kam auch eilig der Alpinist und
begrüßte mich mit vielen Worten und Verbeugungen. Eigentlich heißt
er Dr.~Habendorf und ist von Beruf Arzt, speziell Zahnarzt. Im
Sommer praktiziert er hier, d.\,h. nur bei schlechtem Wetter; bei
gutem ist er höchstens gegen abend zu sprechen. Wir nennen ihn
immer nur den Alpinisten. Er ist ein stattlicher Mann mit sehr
hellen Haaren und Schnurrbart und einer großen Narbe über die
Wange. Wenn er nur nicht so schrecklich in Liebenswürdigkeiten
zerflösse! Fräulein von Okeley mir gegenüber grüßte steif, als sie
sich hinsetzte, und begann gleich mit ihrer Nachbarin, ihrer um
mindestens zehn Jahre älteren Schwester Beate, ein lebhaftes
Gespräch. Mein Nachbar zur Rechten, Herr Martin, war natürlich noch
nicht da. Ich war also auf den Alpinisten angewiesen, der mich
fragte, welche Tour ich heute unternommen hätte. Aber ich brauchte
gar nicht zu antworten, denn er fuhr gleich fort:

„Habe großartige Traversierung des Langbergs gemacht. Gewöhnlicher
Aufstieg zu den Gamssteinen, Fräulein Doktor werden wissen. Man
rechnet drei Stunden. War in 1~Stunde 52~Minuten oben, hier vom
Hotel. Aber dann, neuer Abstieg. Von mir heute kreiert. Großartig,
aber sehr schwierig. Gefährliche Kletterei, zum Teil Felsen mit
brüchigen Griffen, über Silberwand abgeseilt, dann durch den
Silbertobel in die Festinaschlucht. Das heißt, bis zum letzten Fall
in der Klamm. Mußte aber wieder hinauf bis zur Waldgrenze, da Klamm
tatsächlich unpassierbar. Fräulein Doktor werden wissen – War keine
schlechte Arbeit! Aber bei meinem Fassungsvermögen!“

Dabei ballte er lachend seine kräftige Hand zusammen.

Ich lachte, denn das war einer seiner stehenden Witze. Er machte
ihn, wie man mir erzählte, seinen Patienten gegenüber, wenn sie
befürchteten, er würde eine Zahnwurzel nicht entfernen können. Und
ich wußte, daß es ihn beglückte, wenn ich ihn für geistreich hielt.
Warum sollte ich das nicht?

Während ich ihn weiter so sprechen hörte und zugleich die Nachbarn,
ließ ich den Blick über die Tafeln des Saales mit den essenden und
schluckenden Menschen gleiten. Wie gräßlich fade kam mir doch
eigentlich diese ganze Gesellschaft vor! Und ich merkte zugleich
wie prächtig trotz meiner Seelenmischerei der psychophysische
Apparat arbeitete, der sich Wera Lentius nannte. Ich hatte ich ja
auch vorher gewöhnlich bei Tische gelangweilt, obwohl ich mir
nichts merken ließ. So benahm ich mich auch jetzt gewohnheitsmäßig
ganz korrekt. Ich fragte den Alpinisten nach seinem neuen Abstieg
aus, daß er vor Glück über meine Liebenswürdigkeit strahlte. Ich
erkundigte mich nach dem Laufe des Baches im Silbertobel. Ob dort
nicht eine Quelle direkt aus dem Berge bräche? Davon war ihn nichts
bekannt.

Wie sollte ich jetzt als Mensch die Stelle finden, wo ich aus dem
Langberg im Silbertobel herauszukommen pflegte? Ich sagte: „Können
Sie mir nicht einmal Ihren Weg auf Ihrer topographischen Karte
zeigen?“

Er griff sofort in die Tasche.

„Nein, nein,“ bat ich, „nicht jetzt. Ich möchte mir das in aller
Muße ansehen. Würden Sie mir einen großen Gefallen erweisen
wollen?“

„O, Fräulein Doktor haben nur zu befehlen.“

„Wenn Sie die Güte hätten, Ihren Abstieg in die Karte einzutragen –
sie enthält doch die Isohypsen?“

„Gewiß, Fräulein Doktor. Von zehn zu zehn Meter. Sie können daran
alles genau sehen, auch die Stelle, wo ich mich abseilen mußte.“

„Die müssen Sie mir anmerken, und dann leihen Sie mir die Karte auf
einen Tag.“

„Selbstverständlich! Ich werde sogleich – wann wünschen Sie die
Karte?“

„Wenn ich sie heute Abend bekommen könnte?“

„Mit Vergnügen. Fräulein Doktor beherrschen, wie ich sehe, alle
Gebiete der Wissenschaft. Die Kartographie ist eine
Lieblingsbeschäftigung von mir.“

Ich hörte seinen weiteren Ausführungen nur scheinbar zu. Zunächst
freute ich mich über meine Schlauheit. Da hatte ich ja gleich die
beste Gelegenheit, meine Erfahrungen als Wolke auf
wissenschaftlichen Boden zu stellen. Ich beschloß, so lange im
Silbertobel an Ort und Stelle zu suchen, bis ich den Austritt des
Wassers aus der Kalkschicht, – ich nannte ihn jetzt bei mir im
stillen die „Silberquelle“ – gefunden und auf der Karte festgelegt
hatte. Dann mußte sich ja zeigen, wie er zum Tunnel lag.

Das Tellerklappern, dabei das Schwirren der Stimmen und die etwas
schreiende Rede meines Nachbars störten mich in meinen Gedanken –
mein Widerwille gegen diese ganze Wirtstafel stieg aufs neue in mir
auf. Der Alpinist sprach jetzt von Geologie und der Tektonik des
Langbergs – und auf einmal sah ich mich in einem andern Saale, an
einem ganz andern Tische, der mit Schalen und Retorten bedeckt war,
ich hörte die ruhige, klare Stimme eines andern Mannes und sah zwei
leuchtende Augen, die mich so seltsam anblickten – da durchzuckte
mich das Gefühl einer unbeschreiblichen Sehnsucht, die ich nicht
verstand. Es war, als wäre meine Aspiraseele auf ihrer Wanderung
durch Weras Erinnerungen an einen verschlossenen Garten gekommen,
zu dem sie noch keinen Eingang fand – und schon war das flüchtige
Bild verschwunden. Das Rücken eines Stuhles weckte mich aus meinem
Traume und die Stimme des Ingenieurs, der höflich „Guten Tag“
wünschte. Die helle Wirklichkeit war da und leuchtete~–~–

Warum mich Bertilde so prüfend ansah, als ich seinen Gruß
erwiderte? Nun ja, ich fühlte, daß ich dabei lebhaft war – In
diesem Augenblicke wurde mir präsentiert und ich hatte Zeit, mich
zu sammeln.

Genieren aber wollte ich mich nicht. Ich fragte Herrn Martin ganz
ruhig, ob er nachmittags wieder im Tunnel zu tun habe, und zu alles
Erstaunen antwortete er ganz ausführlich.

„Nur einige Stunden. Jetzt werden die Bohrer wieder angesetzt, und
ich warte nur, bis die Sprengung vorüber ist. Dann bin ich dort
nicht mehr nötig, aber ich habe noch im Zentralbureau zu tun.“

Und so entwickelte sich ein ruhiges Gespräch, dem die andern mit
einem gewissen Neide zuhörten. Keines von uns beiden erwähnte, daß
wir den ganzen Vormittag am Tunnel zusammengewesen waren. Es war,
als wäre dies unsre erste Unterredung. So hatten wir ein Geheimnis
miteinander. Aber es war wohl gut so. Ich hörte ganz deutlich, wie
Bertilde zu ihrer Schwester sagte:

„Du, der Mann kann ja auf einmal reden wie ein Buch.“

Und der Alpinist brummte: „Ein unerträglicher Schwätzer.“

Ich besänftigte ihn, indem ich mich wieder zu ihm wandte.

Zum allgemeinen Erstaunen blieb der Ingenieur diesmal während des
Nachtisches sitzen. Erst als alle Gäste aufbrachen, verabschiedeten
wir uns. Ich war schlau genug, dann den Alpinisten noch einmal an
die Karte zu erinnern, was er so hoch aufnahm, daß er meine Hand an
die Lippen führte.

Dann lief ich auf mein Zimmer. Ich mußte meine Erfahrungen zu
Papier bringen, das war Wera- Gewohnheit. Da lagen meine Tagebücher.
Ich wollte schon weiter schreiben, wo ich einmal aufgehört hatte,
aber da fiel mir ein – wenn ich doch einmal wieder aufhören sollte,
Wera zu sein – wenn ein Mensch dann das lesen sollte, was ich nun
zu schreiben hatte –~– nein, das durfte nicht sein!

Bald wußte ich, wie ich's einrichte. Ich nahm nur lose Briefbogen.
Und heute noch kaufe ich mir wasserdichtes Papier zum Einschlagen
und eine Blechbüchse, da tu ich meine Aufzeichnungen hinein, und
ich weiß schon, wo ich sie droben am Langberg verstecke, daß kein
Mensch sie finden kann.

Und so habe ich den ganzen Nachmittag geschrieben – die Finger tun
mir weh~–~–

\section{Die Silberquelle}

Als Wera ihre Aufzeichnungen geschlossen hatte, war es so spät
geworden, daß sie vor dem Abendessen gerade nur Zeit fand, einmal
durch den Ort zu gehen und einige Einkäufe zu machen. Bei Tische
nahm sie die Karte von dem Alpinisten in Empfang und hörte mit
Vergnügen auf seine begeisterten Schilderungen aus der Bergwelt.
Der Platz des Ingenieurs war leer geblieben. Er hatte absagen
lassen. So wechselte Wera nach Tische nur einige gleichgültige
Gespräche mit verschiedenen Gästen und zog sich bald zurück.
Eigentlich wollte sie noch die Karte für ihren Zweck studieren,
aber die Beleuchtung des Zimmers war dazu nicht geeignet. Ermüdet
von all den neuen Eindrücken zog sie es vor, ihren ersten
Menschenschlaf zu tun.

Am andern Morgen weckte sie der helle Tag. Schnell huschte sie ans
Fenster und warf einen Blick durch die Gardine. Über das grüne Tal
und die schimmernden Gletscher herüber leuchtete das weiße Haupt
des Blankhorns im Sonnenglanz. Wera winkte ihm grüßend zu. Ob der
Alte sie sehen konnte? Aber eiligst trat sie zurück. Sie war ja
keine Wolke mehr. Seltsame Seelenmischung – das war wohl ein
Menscheninstinkt?

Doch die Frage interessierte sie nicht. Wie steht es im Langberg?
Das war der Gedanke, der sie jetzt so ausschließlich beschäftigte,
daß von den vielen Erinnerungen, die noch in ihrer Weraseele
schlummerten, keine einzige irgendwie deutlich emporstieg. Hier war
eine Aufgabe für ihren jungen Menschenstolz. Die Natur mit
Menschenaugen zu betrachten, die Macht der Erkenntnis nun selbst zu
üben, das hob und erfüllte ihre ganze Seele – alles, was weiter
zurück lag, war im Augenblick nicht für sie vorhanden.

Bald saß sie am Tische und hatte das Tunnelprofil und die Karte des
Alpinisten vor sich ausgebreitet. Das Kartenbild sah freilich ganz
anders aus als die Landschaft aus der Wolkenhöhe. Das also war der
Langberg! Er trennte das Tal des Schaumbachs, an dem Schmalbrück
lag, von der Schlucht der Festina, die unten bei St.~Florentin den
Schaumbach in sich aufnahm. Nach der Seite der Festina hin besitzt
der Langberg einen starken seitlichen Vorsprung, den der Fluß in
weitem Bogen umgehen mußte. Dabei hatte er eine enge Klamm
ausgewaschen. Die Bahn kam von St.~Florentin her im Festinatal
herauf, überschritt den Fluß dicht vor der Klamm und mußte nun den
Vorsprung des Langbergs in dem bewußten Tunnel durchbohren.

Dieser Vorsprung war durch eine Reihe von Gießbächen zerrissen, die
in die Klamm niederstürzten. Von ihnen war der Silbertobel der
tiefste und wildeste, und an einer Stelle dieser Schlucht trat die
Silberquelle – wie Aspira den gesuchten Ausfluß nannte – aus dem
Berge hervor. Aber wo lag diese Stelle? Lag sie tiefer als der
Tunnel, so mußte der Weiterbau notwendig in die Kalkschicht
geraten, und dann trat die befürchtete Gefahr ein. Die Karte konnte
darüber keine Auskunft geben, denn die Menschen wußten von dem
unterirdischen Zufluß nichts, weil ihn der durch den ganzen
Silbertobel rauschende Bach verdeckte. Aspira kannte die Stelle nur
aus den Merkmalen der Umgebung. Sie mußte sie also, um ihre
Höhenlage festzustellen, selbst aufsuchen.

Aber wie dahin gelangen? In ihrem Wolkenbewußtsein war ihr das sehr
einfach vorgekommen, jetzt vor der Karte sagte sie sich, welche
Schwierigkeit und Gefahr darin lag, diese Berge auf ungebahnten
Wegen zu durchstreifen. Wenn sie nicht wie der Alpinist den
Langberg überqueren wollte, mußte sie die wilden Schluchten
überklettern, die vom Langberg nach der Festinaklamm abstürzen.

Wera wurde ärgerlich und legte die Karte zusammen. Was sollte sie
lange grübeln? Versuchen wir's, sagte sie sich. Sie versah ihren
Handbeutel mit den erforderlichen Instrumenten, einem
Taschenbarometer und einem Thermometer zur Bestimmung der
Wassertemperatur. Beim eiligen Frühstück nötigte ihr Frau Leberecht
noch einen wohlverpackten Imbiß auf. Vor dem Hotel begegnete ihr
der Hausdiener, der eben von der Post zurückkam, und händigte ihr
einen Brief und eine Drucksache aus. Sie nahm sich gar keine Zeit,
die Papiere anzusehen, sondern steckte sie eilends in die Tasche
ihres Kleides. Denn es war für ihre Expedition schon spät
geworden.

Um die andere Seite des Langbergs zu gewinnen, schlug Wera zunächst
den Weg nach dem Tunnel ein. Von dort führte sie ein schmaler
Holzfällerpfad durch den Wald bis auf die Matten. Schon sah sie die
Schiefklippe und den Steinbruch unter sich liegen. Nun mußte sie
oberhalb der Waldgrenze bis an den Silbertobel gelangen, um ihn
seiner ganzen Erstreckung nach im Walde zu verfolgen, denn sie
wußte ja nicht, in welcher Höhe die Silberquelle lag. Da aber kam
der erste Einriß. Sie mußte sich einen Weg durch die Felstrümmer
der Schlucht suchen, über den Bach springen und auf der andern
Seite hinaufklettern. Obwohl ihr das alles leichter gelang, als es
einem Menschen ohne Wolkenerbteil möglich gewesen wäre, verging
doch ziemlich viel Zeit darüber. Und solcher Schluchten gab es vier
bis zum Silbertobel! Sie sah nach der Uhr und blickte sich dann
ziemlich ratlos um.

„Ach, wär ich nur auf ein paar Minuten eine Wolke, wie schnell wäre
ich drüben! Aber das geht nun nicht – Ich bin nur ein Mensch und
will es sein! Hier müssen Menschenmittel helfen. Ich will denken!“
Sie sammelte alle ihre Erinnerungen an ihre letzte Regenfahrt von
den Gamssteinen in die Festinaschlucht. Wenn es ihr gelang, die
ungefähre Höhe der Silberquelle zu ermitteln, so brauchte sie nicht
den ganzen Tobel hinabzuklettern, sondern konnte ihren Weg
bedeutend kürzen und zugleich die Übergangsstellen über die
Einschnitte sich aussuchen, indem sie bergab stieg. Von einer ins
Tal hervorspringenden Klippe gewann sie einen umfassenden Einblick
in die Schlucht. Und dort, an der gegenüberliegenden Seite der
Schlucht, etwa hundertfünfzig Meter unter ihrem Standpunkt,
erkannte sie eine Felspartie, oben abgeplattet, und auf dieser
Platte lagen die verkrümmten, dicken Wurzeln einer umgestürzten und
abgestorbenen Kiefer, die wie die silbergrauen Glieder eines
riesigen Drachens im Sonnenschein schillerten. Da blitzte es in ihr
auf. Diese grotesken Windungen waren ihr jedesmal in dem
Augenblicke aufgefallen, wenn sie aus dem Berge hervorquellend das
Licht wieder erblickt hatte. Aber sie erinnerte sich deutlich, daß
der Fels ausgesehen hatte wie ein Tierkopf, der dieses wunderliche
Gehörn auf dem Scheitel trug. Die Platte und der untere Teil des
Holzes waren nicht sichtbar gewesen. Also hatte sie den Felsen von
unten gesehen, und die Silberquelle mußte somit tiefer liegen als
jener Felsen.

Sie konnte von hier auch den nächsten Einschnitt überblicken und
erkannte, daß er ein Stück tiefer unten zu einer schmalen Spalte
wurde. Also dorthin! Sie sprang den Berg hinab. Dabei bemerkte sie
zu ihrem Entzücken, daß sie mit voller Sicherheit wie von der Luft
getragen über die Steine und durch das Gestrüpp des Bodens flog.
Und so setzte sie auch mit einem kühnen Sprung über den Spalt. Nun
kam sie in den Wald, der aber hier nur aus vereinzelten,
verkümmerten Bergkiefern bestand.

Sie glühte in Lebenslust und Kraft. Es war ihr, als summte es um
sie: „Wohin, Aspira, wohin so geschwind?“ Und übermütig rief sie:
„Zur Silberquelle!“ Plötzlich ein breiter Riß, den sie nicht zu
überspringen wagte. Ärgerlicher Aufenthalt! Mußte sie hier wieder
klettern?

Da kam's hinter ihr von oben polternd herab, ein großer Felsblock.
Sie erschrak nicht, sie wußte, daß er sie nicht trifft. Er sauste
vorbei in den Einschnitt und riß dabei einen schon halb
entwurzelten größeren Stamm um, daß er sich über den Erdriß wie
eine Brücke legte. Jauchzend lief sie ohne Besinnen hinüber. Ein
Stück des Kleides blieb an einem Aste – was tat's? Ihre Freund vom
Berge schützten sie. Unter ihr verbarg das Moos Löcher und Spalten
des Gesteins, lagen Trümmer von Ästen und Wurzeln, aber ihr
eilender Fuß traf stets den festen Boden, die sperrigen Zweige der
Bäume wichen zurück, der letzte Wildbach ward übersprungen, noch
ein Stück Wald durch hochstämmige, bemooste Fichten – sie stand am
Silbertobel.

Und im Augenblick erkannte sie die Stelle. Da aus jenem Spalt
rauschte eine Quelle direkt im Bette des Bachs. Und drüben durch
die Wipfel der Bäume blickte von der andern Talseite der
Widderfelsen mit seiner Schlangenkrone.

Sie kletterte bis hinab an das Wasser, von dem sie schlürfte. Dann
saß sie ausruhend auf breitem Felsstück. Schnell waren Barometer u
Notizbuch hervorgeholt. Das Tunnelniveau stieg bis 1952~Meter. Was
würde sich ergeben? Wie hoch war sie hier?

Sie mußte einen Augenblick die Augen schließen. Das Herz pochte.
Dann las sie das Instrument ab.

Neunzehnhundert und – siebzig! Die Korrektion betrug noch 6~Meter.
Also zwanzig Meter über der Tunneldecke! Die Angaben des
Instruments konnten um zehn Meter nach oben oder unten ungenau
sein. Aber im schlimmsten Falle blieb doch immer noch ein Mehr von
zehn Meter, und im Innern des Berges mußte der Unterschied viel
größer sein. Freilich – es konnten auch noch tiefer Verwerfungen
liegen, aber von diesem Zweige des Kalkbandes drohten jedenfalls
keine Störungen. Um sicher zu gehen, nahm Wera noch eine Bestimmung
oberhalb des Zuflusses und in diesem selbst vor. Letztere war nur
um 4~Grad höher als die des Baches. Kein Wunder, daß noch niemand
auf diese warme Quelle aufmerksam geworden war. Der geringe
Unterschied bewies, daß dieser Wasserlauf mit dem tiefer liegenden,
zerdrückten und schlammigen Teile im Innern des Berges nicht
zusammenhing.

Nun atmete Wera auf. Sie war glücklich und stolz. Ihre Freunde vom
Berge hatte sie nicht verlassen, sie fühlte, wie sie rings umher
für sie sorgten. Und doch hatte sie sich ihres Menschenseins würdig
erwiesen. Ihr Denken hatte ihr geholfen, die Mittel der Natur hatte
sie zu benutzen gewußt. Sie hatte die erste Probe abgelegt, daß sie
sich einen Menschen nennen konnte, der auf der Brücke der
Erkenntnis stand. Sie hatte eine wichtige und erfreuliche
Feststellung gemacht.

Es litt sie nicht in der düstern Schlucht, sie wollte ins Licht,
ins Freie. Aber wie? Nach dem Tunnelausgang, der in nahezu gleicher
Höhe lag, wäre es nicht weit gewesen, aber da hätte sie über
sämtliche, kaum zugängliche Einschnitte klettern müssen. Die
Festinaklamm selbst war unpassierbar. So blieb ihr nichts übrig,
als, wie der Alpinist, den Berg wieder hinaufzuklimmen. Nach zwei
Stunden war die Höhe am Flügel des Berges gewonnen, von wo sie auf
beide Täler und weit hinaus über den See und die Bergzacken
jenseits blicken konnte. Von hier vermochte sie bequem nach
Schmalbrück hinabzusteigen.

Wera lagerte sich behaglich auf dem grünen Rasen und stärkte sich
an dem willkommenen Imbiß.

Und nun konnte sie träumen. Kühl war die Luft, und doch ruhte
sich's so warm im leuchtenden Sonnenglanz. Wie wundervoll still und
einsam lag die blumenbunte Alpenwiese! Die dunklen Nigritellen
durchdufteten die Luft, um die gelbe Arnika, den orangelichten
Senecio summten leise die Hummeln. Sonst kein Ton als mitunter der
Pfiff eines Murmeltiers. Und weit, ganz weit, ein verhallendes
Herdengeläut.

Ihre Hand strich leise über die dunkelblauen Sterne des Enzians an
ihrer Seite.

Da – in der Ferne ein dumpfer Knall, noch einmal, und dann
schwächer und schwächer der rollende Widerhall an den Bergwänden.

Das ist der Mensch, der felsensprengend in die Stille der Berge
bricht. Der Ton stört sei nicht. Er ist ihr ein froher Gruß einer
befreundeten Macht. Wie töricht, darin den Feind der Ruhe, des
heiligen Friedens der Natur zu sehen! Donnert nicht auch die
blitzende Wolke, stürmt nicht der Bergsturz zerschmetternd ins Tal,
rauscht nicht der Wildbach vernichtend hernieder? Und sie wissen
nicht, warum sie schädigen. Der Mensch aber zerstört, um Höheres
aufzubauen. Aspira weiß es jetzt. Die Elemente sträuben sich
vergebens. Ihr Frieden ist ein Schein. Es gibt einen höhern
Frieden, es gibt einen Zweck des Schaffens, denn es gibt
Menschenherzen. Ihrem Frieden die Stätte zu bereiten müssen die
Elemente zum Dienste gezwungen werden. Sie wollen sich sträuben,
den Menschen vertreiben – deshalb sollte sie ihn ja studieren. Aber
nun hat sie erkannt, daß der Mensch im Rechte ist. Und sie, die
Tochter der Natur, sie wird den Menschen schützen, sie wird ihm
helfen bei seinem Werke. Denn sie beide, Natur und Mensch, haben
ein gemeinsames Ziel –~– der Mensch kennt es, er ist ihr Führer; so
darf er herrschen, so darf er zwingen!

Und warum \emph{zwingen?} Wera richtete sich empor. Warum zwingen?
Warum nicht bloß führen?

Warum konnte sie die Frage beantworten, die den Ingenieur
bedrängte? Warum hatte sie in die Kalkschicht blicken können? Weil
sie Aspira war, die Wolke, die durch die Klüfte zu rauschen
vermochte und ins Innere des Berges mit Leichtigkeit zu schauen.
Aber freilich, als Wolke wußte sie nichts von den Beziehungen der
Natur zum Menschen. Erst da sie ein Mensch geworden, konnte sie die
Frage verstehen und mit besonnener Überlegung die Mittel ergreifen,
sie zu lösen. Auch der Mensch mochte schließlich das Innere des
Berges ergründen, wenn er Arbeit und Kosten und Zeit erschwang.
Aber wie umständlich und langwierig war dies Beginnen! Jetzt, da
sie ein Mensch geworden, konnte sie dem Menschen direkt helfen. Sie
verband das unmittelbare Leben der Natur mit dem mittelbaren
Erkennen des Menschen. Sie ersparte ihm unsägliche Mühe, wie heute
auf ihrem Wege befreundete Geister des Berges sie großer
Anstrengung überhoben hatten. Wieviel Zeit hätte ein gewöhnlicher
Mensch zur Auffindung der Silberquelle verwenden müssen!

Und nun stieg es in Wera auf wie ein neues, heiliges Licht, und wie
ein unendliches Glück.

Nein, nicht zwingen mehr sollte der Mensch die Elemente. Etwas
Größeres noch gab es als diese Arbeit. Helfen sollten sich beide
gemeinsam. Wie sie ihr freies Naturleben heute willig in den Dienst
der Menschenzwecke gestellt hatte, so sollten ihre Geschwister
alle, so sollten die Geister der Erde, des Wassers und der Luft
sich vereinen aus freiem Entschlusse mit den Menschen. So sollte
der Mensch unmittelbar in die Natur hineinschauen und in raschem
Fluge von ihr gewinnen, was ihm, dem Erforscher des Gesetzes, sonst
erst in der Arbeit der Jahrtausende gelang.

Das war ihre Aufgabe! Versöhnen die Reiche der Natur und des
Menschen, die sich getrennt hatten. Mitarbeiten sollten die
Elemente am Kulturzweck aus eignem Trieb. Dazu war sie aufgestiegen
zum hohen Äther, dazu war sie herabgestiegen aus dem Reiche der
Luft zu den Wohnungen der Menschen. Diese Einsicht mußte sie den
Elementen wie den Menschen vermitteln. Das war die Aufgabe ihres
Daseins, ihre Arbeit. Das war das Glück auf der Brücke der
Erkenntnis!

Und nun an das Werk! Sie mußte damit beginnen, dem Ingenieur ihr
Wissen mitzuteilen. Aber wie? Was durfte sie ihm sagen?

Doch es würde sich schon ein Weg finden. Hier eben mußte der
Menschenverstand einsetzen.

Da war es ja! Die beiden Gewässer, das von oben kommende und das
aus der Kalkschicht, mußten sich chemisch unterscheiden. Die
brauchte also nur – wozu hatte sie denn so viele~–~–

Damit dachte sie zum ersten Male klar zurück an ihre Arbeit als
Wera. Und ein innerer Schauer, eine dunkle Angst lief durch ihre
Seele und zerriß den einfachen Gedanken~–~–

Warum hatte sie so viele – Quell-Analysen gemacht?

Mit der Erinnerung an ihre Arbeit tauchte eine zweite auf – ein
unbestimmter Schreck durchzuckte sie, wie ein dumpfer Schrei klang
ein Name~–~–

Paul! Paul!

Sie griff an ihr Kleid und fühlte das Knittern des Papiers. Sie riß
die Postsendungen heraus. Zuerst fiel ihr die Drucksache in die
Augen. Jetzt wußte sie sogleich, was es war. Die Korrektur eines
kleinen Artikels aus dem Journal für Geologische Chemie – ihre
letzte Arbeit vor ihrer Abreise aus der Universitätsstadt in die
Alpen. Und die erste, die sie mit Paul zusammen gemacht hatte und
hier veröffentlichte. Da stand es ja: Über den Gasgehalt der
Quellen von Hellborn. Von P.~Sohm und W.~Lentius.

Paul Sohm! O wie hatte dieser Name ihr auch nur auf einen Tag
entschwinden können! Der doch seit Monaten ihr alles war, ihr
ganzes Leben erfüllte! Ach, nicht seit Monaten, im stillen schon
viel länger~–~–

Ihre zitternden Finger öffneten den Brief. Von ihm!

„Meine geliebte Wera! – –“

Sie konnte nicht weiter lesen. Es wurde so dunkel. Die Sonne, die
Eisgipfel, die Wiese verschwanden –~– alles so wirr und leer in ihr
– wer war sie?

Die freie Wolke, die vom Äther kam, die Weisheit der Menschen in
sich aufzunehmen, das Gesetz zu ergänzen durch die Fülle des Lebens
und den Schritt der Kultur zu beflügeln durch die Schwingen den
Natur?

Oder der mühsam tastende, langsam fortarbeitende Einzelmensch,
gefesselt durch zahllose Bande an die andern, gehemmt durch
unbekannte Gewalten rings umher und im Innern der Menschenbrust?

Wer war sie?

\section{Die Braut}

Noch immer ruhte Wera auf dem sonnenwarmen Rasen in der Stille der
Alpenwiese. Noch immer war ihr das weite Bild der leuchtenden
Bergwelt hinweggesunken. Ihre Augen waren geschlossen. Noch einmal
suchte sie in ihrer Weraseele zu lesen. Nicht mehr die große Frage
nach dem Gesetz der Erkenntnis, das die Menschenherrschaft gewährt
über die Reiche des Lebendigen. Sie war auf ein Kapitel gestoßen,
das den Menschen allein anging, und das so viel schwerer für sie zu
verstehen war~–~–

Wohin sie in die letzten Jahre zurückblickte, überall fand sie den
Namen Paul Sohm, überall fühlte sie ihr innigstes Leben daran
geknüpft, und dennoch lag es jetzt darüber wie ein lichtloser
Fleck, ein undurchsichtiger Schatten, der sie den eigentlichen Sinn
dieses Seeleninhaltes, dieses Lebensmittelpunktes nicht begreifen
ließ. Es gab eine Stelle in Weras Bewußtsein, wohin Aspiras
Verständnis nicht zu dringen vermochte. Alle Vorgänge, alle
gemeinsamen Erlebnisse, was sie taten und sprachen, konnte sie klar
ins Gedächtnis rufen, und immer wieder fragte sie: Warum? Warum
mußte ich das tun? Warum war ich so glücklich?

Seit dem ersten Tage, an dem sie Sohms Vorlesung über Meteorologie
gehört hatte, fühlte sie sich von diesem Manne gefesselt. Zuerst,
ja lange nur, von dem Lehrer. Da war alles so klar und einfach
gesagt und doch nicht trocken; selbst in den mathematischen
Entwicklungen verschwand nie der Endzweck des Ganzen; überall
eröffnete sich ihr der Blick in die gesetzliche Einheit der
Naturerscheinungen. Die Strömungen der Luft, der Kreislauf des
Wassers, die große Arbeit der Sonne an der Atmosphäre enthüllten
sich in strengem Zusammenhang, und doch lag in allem, was er sagte,
ein so feiner, freier Geist, von dem sie zuerst lernte, wie ein
ernsthafter, kritischer Gelehrter den Inhalt seiner Forschung
zugleich mit warmem Herzen umfassen und an den lebendigen
Weltzusammenhang knüpfen konnte. Und das verstand Aspira wohl, wie
der Einfluß dieses Mannes Wera begeistern und in ihrer Arbeit
fördern und beglücken mußte.

Später waren sie im Hause von Geheimrat Rötelein gesellig
zusammengetroffen, wo sie als Freundin von Suse und Else Rötelein
viel verkehrte. Warum sie nur immer so heiter war? Und warum sie
mitten zwischen ihren ernsthaften Untersuchungen so fröhlich lachen
konnte wie ein Kind? Denn damals hatte sie mit ihren Analysen der
natürlichen Gewässer begonnen – sie wußte wohl, daß ihre chemischen
Studien diese Beziehungen zur Geologie gerade durch Sohm gewonnen
hatten.

Und dann, als Sohm seine große Arbeit über die geologische
Bedeutung der Gase plante –~– Sie sah ihn vor sich auf dem
Spaziergang mit Röteleins nach dem Erdsturz am Klippberg, wie er
mit ihr so vertrauensvoll über seine Grundgedanken sprach und sie
fragte, ob sie nicht einen Teil der Analysen übernehmen wolle. Wie
sie zögerte –~– Es war keine falsche Bescheidenheit von ihr, daß
sie sich bedachte. Sie wußte genau, was sie damals gesagt und getan
hatte, sie wußte, daß sie der Aufgabe gewachsen war und daß jede
kleinliche Ziererei ihr fern lag. Und welches ehrenvollere
Anerbieten konnte ihr gemacht werden, als von einem so bedeutenden
Gelehrten zur Mitarbeiterschaft aufgefordert zu werden, welche
bessere Einführung in die wissenschaftliche Welt konnte sie, die
junge Doktorin, gewinnen? Und dennoch hatte sie gezögert? Sie
verstand sich nicht. Ganz deutlich sah sie sein ernstes, männliches
Gesicht vor sich und den traurigen Blick der treuen Augen, mit dem
er die Verlegene anschaute, und wieder verstand sie sich nicht,
warum dieser Blick auf einmal alle ihre Bedenken umwarf, und sie
ihm die Hand entgegenstreckte und dankte. Und wie glücklich er nun
aussah, und wie warm er von seiner großen Freude sprach, daß sie
annahm, während doch sie allein es war, die ihm zu danken
hatte~–~–

Und dann weiter, wie die Arbeit sie täglich zusammenführte – Sie
wollte an etwas andres denken, aber es lag ein Zwang in dem Namen
Paul Sohm – sie mußte in Weras Gedächtnis nachlesen Woche für
Woche, wie die Arbeit fortschritt, gemeinsam beiden bald im
Zweifel, bald in der Freude des Erfolgs – wie jede längere
Unterbrechung ihr fast unerträglich schien –~– bis jener Abend kam,
da sie einander gegenüberstanden und such ansahen und nichts
sagten, und er auf einmal ihre Hände faßte und sie an sich zog, und
sie in seinen Armen lag –~– Und diese Küsse und diese heißen Worte
–~– Wie war das möglich? Wie konnte sie das dulden und erwidern?
Was war das?

Wer war sie?

Und schluchzend warf sich Wera auf den Rasen, und Träne auf Träne
rann aus ihren Augen. Sie kam sich so namenlos elend vor, so ganz
vernichtet, zerrissen in ihrem innersten Wesen.

Gewiß, das war die Liebe, ja, das hatten sie sich ja tausendmal
glückstrahlend gestanden – aber was war das jetzt? Ein Wort, ein
leeres Wort, das ihr nur von unverständlichen Handlungen, von
unglaublichen Erlebnissen sprach, die sie vor sich selbst
erniedrigten – Sie verstand es nicht und konnte es nicht verstehen,
daß sie das gesagt und getan und versprochen hatte~–

Versprochen hatte! Wera hatte versprochen, und Aspira, die
unglückliche, mußte sich mit dieser gebundenen Seele verschmelzen –
mußte gefesselt sein an ein Menschenschicksal, das sie nicht
verstand – das sie nicht wollte. Nein, nicht wollte!

Ja, er war ihr Freund, dem sie ihr Bestes verdankte, den sie
hochhielt in einem aufrichtigen Gefühle der Verehrung – im übrigen
konnte sie sich nur vorstellen, daß er ihr gefiel wie – wie der
Ingenieur. Aber die Liebe? Sie hatte eine einzige Erfahrung, den
Handkuß des Ingenieurs, und das war ihr wie ein körperliches
Unbehagen – die Erinnerung an all die Stunden, die Wera so
glücklich gemacht hatten, hätte sie jetzt aus dem Gedächtnis reißen
mögen –~– Und doch – Sie war Wera, sie hatte diesen Leib und diese
Seele auf sich genommen, sie hatte mit Stolz und Seligkeit dieses
Menschenwesen erfaßt, sie verdankte ihm ihre Erhebung, ihre
Erkenntnis, ihre Macht, sie konnte diese Lebenseinheit nicht
entbehren – sie war Wera und mußte es bleiben.

Was hatte sie gestern von der Liebe geschrieben? Mochte sie Glück
sein oder Leid, was ging es sie an? Sie suchte sie nicht bei den
Menschen, sie brauchte sie nicht. Und wenn die Wolken etwas von
Aspira wissen wollten, so war es nicht dies.

Und wenn nun Liebe Leid war? So mußte sie es wohl auf sich nehmen
um ihrer großen Aufgabe willen? Diese Aufgabe war ihr heute, hier,
in diesem wundersamen goldenen Lichte einer erhabenen Sendung
aufgegangen. Sie suchte ja auch nicht das Glück. Sonst hätte sie
nicht auf die Brücke der Erkenntnis treten dürfen. Da aber stand
sie nun, da wollte sie stehen. Würde sie das jetzt noch vermögen?

Sie raffte sich empor und schritt langsam ihren Weg nach Hause. Die
Sonne glänzte und die Nigritellen dufteten und die Eisgipfel
grüßten, bis der dunkle Wald seine Schatten über ihr Haupt
breitete. Ach, es war doch schwer, so schwer ein Mensch zu sein!
Aber sie war es nun und wollte es sein – Wera Lentius! Ja, wenn es
nur das wäre! Aber Wera – Sohm! Das – o~Gott – wie würde das sein?
Freiheit und Macht wollte sie erringen, und nun gab es nur diesen
Weg durch Sklavenfesseln?

Sie setzte sich auf eine einsame Bank, denn sie war inzwischen bis
auf den Weg in der Nähe von Schmalbrück gekommen. Die Pfade waren
jetzt verlassen, alle Welt befand sich schon bei der Toilette zum
Mittagessen – nur sie hatte die Zeit versäumt.

Doch noch ein anderer – sie vernahm rasche Schritte.

Es war der Ingenieur Martin, der nach seiner Gewohnheit so spät zu
Tische ging. Jetzt erkannte er sie.

„Sie noch hier, Fräulein Lentius?“ rief er fröhlich. „Guten Tag!“
Sie versuchte freundlich zu danken. Aber jetzt sah er sie in der
Nähe. Erschrocken trat er auf sie zu und blickte sie teilnehmend
an.

„O,“ sagte er, „was ist Ihnen? Sie haben geweint? Sie sind nicht
wohl?“

Die Tränen traten ihr wieder in die Augen, aber sie zwang sich zu
einem Lächeln.

„Es ist nichts,“ sagte sie. „Vielleicht habe ich mich doch etwas
überanstrengt. Wissen Sie, wo ich war? Bei der Silberquelle.“

„Silberquelle? Wo ist das?“

„Das – das nenne ich nur so. Ich habe sie nämlich heute entdeckt,
und Entdecker können doch den Namen geben, nicht? Die Stelle liegt
im Silbertobel über Ihrem Tunnel. Und die Quellen stammen
wahrscheinlich aus Ihrem Kalkbande, natürlich dort, wo es aufhört,
wo wieder undurchlässiges Gestein darunter lagert. Ich wollte
sehen, ob das hoch genug über dem Tunnel liegt. Und ich glaube es
bestimmt. Morgen will ich eine Probe holen, da wird ja die Analyse
ergeben, ob das Wasser aus dem Kalk kommt~–“

Sie sprach, wie es ihr einfiel, um ihre Gedanken zu betäuben. Er
hörte verwundert zu.

„Verstehe ich recht?“ sagte er. „Sie sind dort gewesen? Das ist ja
eine halsbrecherische Partie! Gestatten Sie, daß ich mich einen
Augenblick hersetze. Das müssen Sie mir noch erklären. Aber Sie
sind in der Tat angegriffen.“

„Es wird schon vergehen. Ich wollte doch wissen, ob Ihr Tunnel
wirklich in Gefahr ist.“

Er kam aus seinem Erstaunen nicht heraus.

„Wie gütig Sie sind! Und wie umsichtig!“

Er wollte eigentlich von dem Tunnel und der Kalkschicht sprechen.
Aber er konnte nicht anders als sie ansehen. Die tiefe Erregung
ihrer Seele verklärte ihr Antlitz, und die Augen leuchteten so
wunderbar in dem verhaltenen Schmerze. Wie schön sie war, und lieb,
und klug! Und Martin sprach nicht vom Tunnel und der Silberquelle,
er sprach von ihr und von sich, und wie er mit seinen Gedanken ohne
Unterlaß bei ihr gewesen wäre, und wie er im stillen hoffe~–~–

Sie hatte eigentlich nur halb auf seine Worte gehört. Sie wußte
schon, das war wieder die Liebe, von der er sprach, und die sie
nichts anging. Und doch wieder, es ging sie an. Sie mußte ja nun
doch sehen, ob denn diese Liebe wirklich so unerträglich ist, wie
sie ihr schien. Sie mußte sich abfinden mit ihrem
Menschenschicksal. Wer sollte ihr raten? Wen konnte sie fragen?
Niemand.

Auf einmal hörte sie seine Frage:

„Wera, geliebte Wera, können Sie mir keine Hoffnung geben?“

Sie fühlte, wie er ihre Hände faßte und sie an sich zog. Was in ihr
vorging, bildete ein undurchdringliches Gewirr von Gefühlen und
Vorstellungen. Sie wollte nicht nachgeben, sie konnte nicht – aber
was da geschah, war ein so Fremdes, Neues, das sie doch erfahren
mußte, das doch wieder instinktiv in ihr wirkte~–

Sie wollte sagen: „Ich liebe einen andern.“ Aber das war ja nicht
wahr.

Sie wollte sagen: „Ich bin Braut.“ Aber sie brachte es nicht über
die Lippen.

Sie fühlte die Küsse des Mannes auf Wange und Mund – und es war
wieder dieses Abstoßende, Unerträgliche~–

„Ich kann nicht,“ stöhnte sie sich ihm entziehend. „Es kann nicht
sein, es ist unmöglich!“

„Wera!“

„Ich kann nicht dafür! Ich bitte Sie! Es darf nicht sein!“

„Wera! Mein Glück! Mein Leben!“

„Nein, nein! Nicht so! Ich kann Ihnen nicht zürnen, nein! Aber ich,
ich kann nicht – nicht Ihnen gehören – nein – niemand!“

Es war ein Stammeln, nicht in Entrüstung; es war wie eine Klage.
Sie hatten sich beide erhoben.

„So muß ich gehen,“ sagte er mit erstickter Stimme.

Die Tränen traten wieder in ihre Augen. Sie raffte sich zusammen
und schüttelte den Kopf. Martin wußte nicht mehr, was er denken
sollte.

„Sie sind mein lieber Freund,“ sagte sie. „Ich möchte Sie
begleiten. Ich bin so unendlich traurig, daß ich Ihnen wehe tue.
Aber ich kann nicht – fragen Sie mich nicht.“

„Ich weiß nicht, ob ich das versprechen kann. Wenn du nicht mein
sein kannst,“ rief er leidenschaftlich, „dann ist es besser, ich
sehe Sie nicht wieder. Von gleichgültigen Dingen kann ich jetzt
nicht reden.“

„Dann muß ich allein gehen,“ sagte sie zurücktretend. „Aber gehen
Sie voran. Ich habe Zeit. Gehen Sie!“

Er blieb unschlüssig stehen. Er wußte, wenn er sie jetzt nicht
verließ, würde er doch wieder von seiner Liebe sprechen, und das
würde heute vergeblich sein. Und ach, es war doch so unendlich
schwer, sich zu trennen!

„Bitte, gehen Sie!“ sagte sie noch einmal.

„Leben Sie wohl,“ sprach Martin traurig. „Zürnen Sie mir nicht,
wenn ich jetzt fliehe. Wie kann ich Ihnen Ruhe versprechen, wenn
ich weiß, daß ich sie jetzt nicht halten kann? Sie würden mich
verstehen, wenn Sie wüßten, was Ihre Nähe bedeutet und was – was
Liebe ist.“

„Nein, das weiß ich allerdings nicht,“ rief sie bitter. „Das ist
ja~–“ Sie brach ab. Aber ihre Augen flammten, ihre Brust wogte. Der
Zorn über ihr Schicksal überwältigte sie. So trat sie auf ihn zu,
hochaufgerichtet, auf den unschuldigen Repräsentanten der
Menschheit~–~–

„Liebe! Ich habe niemand, dem ich sagen kann, was mich quält,
niemand, den ich fragen kann, was in meiner Seele Not ein warmes
Menschenherz mir raten würde. Und der einzige Mensch, den ich hier
für meinen Freund hielt, der verläßt mich in dem Augenblick, in dem
ich nach ihm greife, und – warum? Warum? Aus Liebe? Weil er mich
liebt!“ Sie lachte höhnisch und streckte den Arm gegen ihn aus. Da
donnerte es dumpf über den Bergen. Eine dicke graue Wolke war
unbemerkt über den Kamm herübergequollen, und die Sonne verschwand
hinter ihr. Weder Wera noch Martin achteten darauf.

„Liebe! Das ist das schöne Wort, hinter dem ihr euch versteckt, ihr
klugen Menschen! Auch ein Gesetz, aber nicht von denen, die euch
die Herrschaft geben – nein, die euch in die Sklaverei werfen – die
euch feig und klein machen! Aus Liebe müßt ihr quälen und hemmen
und vernichten, was wonnig und frei und groß in euch emporgeglüht
ist, was ich suchte und fand. Aber ich will nicht diesem törichten
Worte weichen – ich will den Kampf aufnehmen~–“

Plötzlich ein krachender Donnerschlag. Drüben am Berghang stürzte
eine Fichte im Blitzstrahl. Martin schrak zusammen. Er hatte auf
Weras leidenschaftliche Worte gehört ohne recht zu verstehen, warum
sie so empört war – sie, der er alles, sein ganzes Ich geboten
hatte, die ihn fortschickte – jetzt klagte sie ihn an?

Wera aber stand unerschüttert. Sie warf nur einen Blick nach dem
qualmenden Baum und nach der Wolke. Dann schüttelte sie leicht
abweisend den Kopf, als wollte sie sagen: „Laß das, Turgula.“

Auch Martin sah nach dem Wetter aus. Aber schon gehorchte die runde
Haufenwolke dem Wink. Sie schrumpfte ein und zog sich zurück. Über
den Rand lugte die Sonne, und der Himmel klarte auf.

„Es kommt nicht herauf,“ sagte Wera ruhiger. „Leben Sie wohl!“

Aber Martin ging nicht. Wera machte eine Bewegung, als wollte sie
sich wieder auf die Bank setzen. Da begann der Ingenieur:

„Verzeihen Sie mir, wenn ich jetzt nicht gehe, wenn ich bitte, Sie
nun begleiten zu dürfen. Ich kann Sie so nicht verlassen. In Trauer
konnte ich wohl von Ihnen scheiden, nicht unter Ihrem Zorn. Sie
standen da wie eine zürnende Göttin, die über Blitz und Donner
gebietet. Und so sprachen Sie zu uns Menschen, als gehörten Sie
nicht selbst zu uns. Es ist da ein Geheimnis in Ihrer Seele, das
ich nicht verstehe. Ich habe kein Recht Sie zu fragen. Aber darin
tun Sie mir Unrecht, wenn Sie glauben, ich könnte Ihnen versagen in
Teilnahme, in Freundschaft, in Hilfe, falls ich es vermag – dann
verkennen Sie meine Liebe. Vielleicht drückte ich mich in meiner
Erregung nicht richtig aus. Ich wollte nur offen sein. An Ihrer
Seite zu gehen in konventionellen Gesprächen, die Ruhe zu bewahren
neben Ihnen Wera, die ich – Verzeihung! Gemütlich mit Ihnen
plaudern, das kann ich jetzt so wenig, wie Sie – das andre können.
Aber wenn Sie eines Freundes bedürfen, wenn Ihnen ein Leid
widerfahren ist, wenn es eine Möglichkeit gibt, Ihnen zu dienen,
glauben Sie mir, dann können Sie auf mich vertrauen.“

Wera antwortete nicht. Sie schritt den Weg hinab. Martin ging an
ihrer Seite.

„Wenn ich noch eines sagen darf –“ begann er wieder. „Ich bin gewiß
nicht der Ansicht, daß die Menschen zu andern über das reden
sollen, was sie im tiefsten Herzen bewegt. Vieles läßt sich nur in
der Stille und im schweigenden Kampfe der Seele bewältigen. Aber
wenn zwei Menschen, die sich sonst verstehen, an eine Stelle
gekommen sind, wo sie merken, daß eine Scheidewand sich zwischen
ihnen aufrichtet durch ihr Schweigen, dann sollen sie nicht in
stolzem Trotze verharren. Ich meine, daß uns die Sprache gegeben
ist, Klarheit zu schaffen. Sie sagten mir, daß Sie mich nicht leben
– nicht lieben können~–“

„Ich kann es nicht,“ sagte Wera, ihn unterbrechend. „Und so wehe es
mir tut, ich muß es Ihnen sagen, weil ich es Ihnen schuldig bin –
hoffen Sie nicht, daß die Zeit daran etwas ändern könne. Wenn
jemals das, was Sie Liebe nennen, in mir lebendig werden könnte, so
würde auch das uns nichts helfen.“

Martin zuckte schmerzlich zusammen. Und doch gab ihm ihr Wort auch
eine Ermutigung –~– Es \emph{konnte} ja doch lebendig werden. Und
dann konnte es sich doch nur um ein äußeres Hindernis handeln. Und
ein äußeres Hindernis kann überwunden werden – er würde es
überwinden.

Aber durfte er sie fragen? Er suchte nach einem Worte. Und
schließlich sagte er nur halblaut:

„Äußere Hindernisse kann man überwinden.“

Wera schüttelte den Kopf.

„Sie sagten mir auch,“ fuhr Martin fort, „daß Sie des Rates eines
Freundes bedürften~–“

„Ich hätte es nicht sagen sollen. Raten kann mir eigentlich
niemand. Ein Verhängnis, von dem ich nicht sprechen kann, hat mich
in einen Zwiespalt gebracht, den ich nicht zu lösen weiß. Und wenn
ich, gegen mein Schicksal mich empörend, Ihnen gegenüber mich zu
einer Klage hinreißen ließ, so war es eben deshalb, weil ich keiner
Menschenseele mich offenbaren kann. Und dennoch, das Eine muß ich
Ihnen sagen, Sie würden es ja doch hören, sobald Sie fragten, wer
ich sei. Denn – das wissen Sie ja gar nicht.“

„Mir genügt zu wissen, \emph{wie} Sie sind. Das andere ist ja jetzt
gleichgültig. Und wenn Sie eine Prinzessin wären~–“

„Sie haben Recht,“ sagte Wera. „Es ist jetzt gleichgültig. Und wenn
ich eine Prinzessin wäre, – wenn ich wollte, würde ich aufhören, es
zu sein. Das kümmert mich nicht.“

Martin blickte sie von der Seite an. Sie sah so stolz und ernst
aus, daß er fühlte, in ihren Worten lag mehr als ein Bild, da lag
ein Erlebnis. Und er wußte nicht, was er denken sollte. Er
schauderte in dem Gedanken, daß in diesem klaren Geiste ein Punkt
sein könnte – nein, nein – das war nicht möglich!

Plötzlich blieb Wera stehen und sagte unvermittelt:

„Kennen Sie Paul Sohm? In Weidburg?“

„Den Geologen?“ fragte Martin erstaunt. „Ich kenne einige seiner
Schriften.“

„Und ich – ich bin seine Braut.“

Wenn ihm Wera gesagt hätte, ich bin eine Prinzessin, ich bin
Aspira, die Wolke, König Migros Tochter, des Beherrschers der
Erdstrahlung, – er hätte nicht überraschter dastehen können. Wera
gehörte schon einem andern, – daran hatte er noch nicht gedacht.
Endlich stammelte er:

„Sie sagten doch – ich verstand, Sie kennen die Liebe nicht, Sie
liebten niemand~–“

„Aber ich habe sie einst gekannt. Ich habe sie nur verlernt~–“

Ein Hoffnungsstrahl zuckte durch Martins Züge. Wera fühlte, was in
ihm vorging.

„Hoffen Sie nichts,“ sagte sie schnell. „Das eben war's, was ich
vorhin meinte, was Sie auf ein äußeres Hindernis deuteten. Ihnen
kann es nichts helfen. Könnte ich meine Liebe wiedergewinnen, so
müßte sie dem gehören. Ich will, ich muß es versuchen. Denken Sie
nichts Falsches. Keinen von uns trifft eine Schuld. Noch weiß Paul
nichts davon, daß ich –~– Haben Sie je Ritterromane gelesen? Da
gibt es wundersame Quellen in den Wäldern; wenn die verliebte Dame
dahin kommt und trinkt von dem klaren Bronnen, so verwandelt sich
plötzlich ihr Herz, alle Liebe verschwindet, Gleichgültigkeit und
Kälte tritt an ihre Stelle. Ich ging, nur um mich von angestrengter
Arbeit zu erholen, glücklich und liebend hier hinauf in die Berge.
Paul war mein Gedanke, Paul meine Hoffnung. Gestern am frühen
Morgen noch stieg ich, das Herz von seligen Träumen erfüllt, hinauf
zum Gletscher. Dort muß ich wohl vom Zauberbronnen getrunken haben
–~– Seitdem vergaß ich ihn – nein, ich weiß ja alles, aber ich
verlor mein Gefühl, ich verstehe mich nicht mehr. Schon gestern,
ehe ich Sie traf, war es geschehen. Ich sehe, Sie verstehen mich
auch nicht. Niemand kann mir helfen. Ich habe nur die Bitte – ich
mußte zu einer Menschenseele sprechen – wenn ich Ihnen etwas gelte,
vergessen Sie, war ich sagte, verraten Sie es niemand~–“

Wera schluchzte tief auf und brach in Tränen aus.

Martin stand ratlos.

„Mein Schweigen versteht sich von selbst,“ murmelte er nur. Aber er
dachte, sie ist doch krank, das unglückliche schöne Weib. Er
ergriff ihren Arm und führte sie sanft weiter.

„Weinen Sie doch nicht, teuerste Freundin. Das kann ja nur
Einbildung sein. Das muß vorübergehen. Sie werden bald wieder
glücklich sein.“

Wera löste sich sanft von ihm und trocknete ihre Tränen. Sie
schüttelte den Kopf. Sie konnte ihm ja nicht sagen, warum sie keine
Hoffnung habe. Sie fürchtete, daß das Wolkenherz in ihr nicht
Menschenliebe lernen könne, daß ihres Verlobten Glück daran
scheitern werde. Aber das Wolkenherz ausreißen, das hieß die
Sendung vernichten, zu der sie sich erkoren fühlte; und Paul
aufgeben, hieß ihr Menschentum mit einem Wortbruch beginnen.

Martin redete tröstend zu ihr: „Gönnen Sie sich Zeit. Es kann sich
nur um eine nervöse Überreizung handeln. Schonen Sie sich. Und
beschäftigen Sie sich mit Ihrer Liebe. Lesen Sie seine Briefe,
schreiben Sie. Halten Sie sich ruhig, wahrscheinlich schadet Ihnen
dieses Bergsteigen in der dünnen Luft. Sobald Sie sich kräftiger
fühlen“ – er seufzte leise – „kehren Sie nach Weidburg zurück –~–
Und – glauben Sie, daß ich stets nicht anderes will als Ihr Glück,
Ihr Glück~–“

Sie reichte ihm die Hand.

„Sie sind ein Mann,“ sagte sie, „ich danke Ihnen. Ich will es
versuchen. Verzeihen Sie mir – ich war egoistisch. Es kam so über
mich, so erdrückend, es war mir so neu – ich kann Ihnen ja nicht
alles sagen. Aber ich mußte sprechen – Es ist mir jetzt leichter. –
Und was tun wir nun?“ fragte sie schwermütig lächelnd. „Dort ist
das Hotel.“

„Jetzt gehen Sie voran,“ antwortete er. „Sagen Sie, Sie fühlten
sich nicht wohl. Gehen Sie auf Ihr Zimmer, lassen Sie sich dort
servieren. Vergessen Sie ja nicht, sich ordentlich zu pflegen. Und
dann – lesen Sie, schreiben Sie~–“

„Und Sie?“

„Ich – ich komme später.“

Sie sah ihn ängstlich an.

„Ich habe zwei Tröster – Gott und die Arbeit. Gehen Sie, geliebte
Freundin!“

Er wandte sich schnell um und schlug einen Nebenweg ein.

\section{Werbung}

Wera saß in ihrem stillen Zimmer. Sie las in dem Tagebuch aus
Weidburg, sie las Sohms Briefe. Und nun nochmals die Gedichte, die
sie längst auswendig wußte~–~–

So fühlte er, ehe sie es ahnte – –

\minisec{Paul Sohm an Wera.}

\begin{liebesgedicht}
Ausgelöscht in Dämmerungen\\
Liegt mein Leben, liegt mein Denken.\\
Nimmermehr vom Glücke fordr' ich\\
Neue Tage mir zu schenken.

Und doch glühn durch meiner Seele\\
Rätselvolle schwüle Nächte\\
Wundersame Mädchenaugen\\
Wie geheime Schicksalsmächte.

Ob mir goldne Zukunftssonnen\\
Nahes Morgenrot verbreiten?\\
Ob nur fern die Wetter leichten\\
In den unnahbaren Weiten?

Diese lieben dunkeln Sterne,\\
Ach, ich weiß nicht, was sie sagen – –\\
Ob sie Schweigen mir gebieten?\\
Ob sie mich verstohlen fragen?
\end{liebesgedicht} 

\begin{liebesgedicht}
Das holde Glück, bei dir zu weilen,\\
Zwei Stimmen ruft es in mir wach –\\
Nur eine darf dein Ohr ereilen,\\
Doch heimlich tönt die andre nach.

Die eine wird dir höflich sagen,\\
Wie deine Nähe mich erfreut, –\\
Die andre stürmt in wirren Fragen,\\
Vom Herzen tausendfach erneut.

Die eine spricht von weisen Dingen,\\
Und klug und freundlich stimmst du zu, –\\
Die andre möchte jauchzend klingen:\\
Geliebtes Weib, wie hold bist du!

Und muß die erste plötzlich stocken,\\
Wenn mich dein Auge leuchtend mahnt,\\
Frag' ich im stillen tief erschrocken,\\
Ob du die zweite wohl geahnt?
\end{liebesgedicht}

\begin{liebesgedicht}
Auf meine Hand stütz' ich das heiße Haupt\\
Und achtlos laß ich die Minuten rinnen.\\
Wieviel der Stunden hast du mir geraubt,\\
Wieviel der Tage, träumerisches Sinnen!

In Plänen, schon verworfen beim Entstehen,\\
In Wünschen, die ich auszudenken schaudre,\\
So muß das Leben nutzlos mir vergehn,\\
Und ach, so leb' ich nur, indem ich zaudre.
\end{liebesgedicht}

\begin{liebesgedicht}
Von diesem Haupte nimm die Last der Jahre\\
Und was sie lehrten nimm mir, Herr der Zeit,\\
Daß ich den Frühlingssegen ganz erfahre,\\
Mit dem ihr Atem meine Tage weiht.

Nimm all die Zweifel, die das Herz berauben,\\
Nimm mir das Wissen um die neue Qual,\\
Laß mich noch einmal an die Liebe glauben\\
Und an ihr Glück – noch dieses eine Mal!
\end{liebesgedicht}

\begin{liebesgedicht}
Ich kenne dich und die verborgnen Wege,\\ 
Wo deine Seele wandert – – \rejoined durch die Höhn\\
Des eisigen Äthers, wo den irren Schein\\ Die letzten Sterne
wärmelos versprühn,\\ Führt ihre Straße sie empor ins Reich\\ Des
ewigen Traumes. Eine Fremde Welt\\ Durchstrahlt mit seltsam mildem
Eigenlicht\\ Die Seele, die sich durch die Nacht gewagt.\\ Doch
einsam schwebt sie, ach, unendlich einsam.\\ Tief unter ihr
verloren liegt die Erde,\\ Wo Menschen wohnen – Menschen, die sie
rufen,\\
Und die sie flieht – – \rejoined ich aber kenne dich\\
Und die verborgnen Wege deiner Seele.\\ Ich
bin sie selbst gewandert, endlos, stumm –\\ Denn keine Sprache
dringt aus Menschenmund\\ In jene Götterhöhn, – und Götter
schweigen.\\ Nur der ist frei, den niemand fragen kann.\\ Es ist so
süß zu leben ungefragt,\\ So hingegeben ganz dem eignen Herzen\\
Und dem Gefühl, das seine Wege sucht.\\ Und weil die Menschen
fragen, immer fragen,\\ Floh ich hinauf, wo keine Neugier wohnt,\\
Und eine Welt nichts weiß von andern Welten – –\\ Dort traf ich
dich, und darum kenn' ich dich.
\end{liebesgedicht}

\begin{liebesgedicht}
Die Stunde des Schaffens, die segnende, schwebt\\ Leisatmend durchs
stille Gemach,\\ Und der Schein ist wahr, und der Traum, er lebt,\\
Und das Schweigen des Ewigen sprach.\\ Wenn die Fessel des
Endlichen klingend zerspringt,\\ Wenn das lösende Wort von der
Seele sich ringt,\\ Und die Erde vergeht, und der Himmel ist mein
–\\ In der heiligen Stunde gedenk' ich dein.

Wenn die Göttin des Sieges den seltenen Kranz\\ Auf die Stirn dem
Zögernden drückt,\\ Und das Aug' erglüht in kühnerem Glanz\\ Und
der Mut den Verzagten beglückt,\\ Wenn der Funke gezündet im weiten
Land\\ Und freudiger Dank mir die Geister verband,\\ Dir möcht' ich
den Lohn, den errungenen, weihn – –\\ In der Stunde des Stolzes
gedenk' ich dein.

Und der Tag entschläft, und der Abend naht –\\ Von den Gärten
duftet es weich,\\ Und zärtliche Pärchen auf dunkelndem Pfad\\
Durchwandeln ihr glückliches Reich.\\ Und es legt sich der Neid um
die irdische Lust\\ Mit Sehnsuchtsqual auf seufzende Brust,\\ Und
die Schatten flüstern: Allein – allein – –\\ In der Stunde der
Tränen gedenk' ich dein.
\end{liebesgedicht}

\begin{liebesgedicht}
\quad Wenn im letzten Dämmerlichte\\
Näh' und Ferne matt verschwimmen,\\
\qquad\quad Klingt es nicht\\
Dir ins Ohr wie leise Stimmen?

\quad Dann in meinem wachen Traum\\
Sehn' ich mich zu deinen Füßen – –\\
\qquad\quad Durch den Raum\\
Schweben Schatten, uns zu grüßen.

\quad Aus den Höhen, erdenfern,\\
Wo sich unsre Seelen finden,\\
\qquad\quad Fällt ein Stern,\\
Und ein Lied zieht mit den Winden.
\end{liebesgedicht}

\begin{liebesgedicht}
Ein Tag, da ich dich nicht gesehn,\\
Ist wie ein Aug' in tiefer Nacht,\\
Das starr in wesenlosem Spähn\\
Durch müde Finsternisse wacht.

Ein Tag, da ich dich nicht gesehn,\\
Ist wie der atemlose Gang,\\
Umblendet von der Nebel Wehn,\\
Auf wegverlornem Felsenhang.

Ein Tag, da ich dich nicht gesehn,\\
Ist wie des Büßers frommer Tod\\
Im Glauben an ein Auferstehn\\
Zu neuem, seligem Morgenrot!
\end{liebesgedicht}

\begin{liebesgedicht}
Oft in Mühen des Tags, wenn die engen Gewalten des Lebens\\
Unmut senken und Zorn in die bewegliche Brust,

Dein gedenk' ich, und ob du mich siehst; und die düsteren Falten\\
Glätten sich über der Stirn, und es bezwingt sich das Herz.

Leicht umfächeln mit segnendem Hauch mich freundliche Geister,\\
Boten der Liebe, von dir ohne dein Wissen entsandt,

Leuchtende Blicke, ein deutsames Wort, ein leichtes Berühren –\\
In der Erinnerung Glanz schließt sich der Reigen des Glücks.

Auf dem lichten Gebild entschwebt die getröstete Seele\\
Mit der deinen geeint über die Erde hinaus.

Hand in Hand, so steigen wir auf zum Reiche der Freiheit,\\
Und die Herrscher der Höhn neigen sich freundlich herab.

Denn den Göttern vertraut zu leben ist einziges Vorrecht\\
Dem belächelten Mann, der von der Menge sich schied. –

Hohe Gewalten, die ihr wohl sonst den Bittenden hörtet,\\
Sinne und Wort mit geschenkt, wenn ich euch ehrlich gesucht,

Heißt sie willkommen, die teure Gestalt, in der ewigen Schönheit\\
Wunderpalast! Nicht fremd geht sie die Stufen hinan.

Nimmermehr nun komm' ich allein; in ihrem Geleite\\
Meinem zagenden Fuß öffnen die Tore sich weit,

Öffnen dem kühneren Blicke sich tief die unendlichen Fernen,\\
Und in reinerem Glanz schau' ich die heilige Form.
\end{liebesgedicht}

\begin{liebesgedicht}
Durch die herbstgebräunten Bäume\\
Fließt der graue Nebel hin.\\
Nasse Tage, kalte Räume –\\
Sagt, warum ich fröhlich bin?

Still die Blicke senk' ich nieder,\\
Und die fremde Störung fällt,\\
Und durch die geschlossnen Lider\\
Rosig leuchtet mir die Welt.

Nicht mehr schwebt es wirr vorüber\\ Was der rasche Traum erfand,\\
Denn ein holdes Gegenüber\\ Hält die Bilder festgebannt.

Leiser Wünsche Spiel und Regung\\ Blitzt ein Auge hell zurück,\\
Und die stürmische Bewegung\\ Löst sich in gewährtem Glück.

Traute Sonne meiner Träume,\\ Weile an des Winters Tor,\\ Und mit
deinem Golde säume\\ Wettergrau und Nebelflor.
\end{liebesgedicht}

\begin{liebesgedicht}
Träume steigen zur Gestalt\\
Wieder auf aus dunklem Schwanken –\\
All die formende Gewalt\\
Hab' ich deiner Huld zu danken.

Was in fahlem Abendgrau'n\\
Mir für immer schien verloren,\\
Hat dein rettendes Vertraun,\\
Glück und Welt, mir neu geboren.

Hingegeben deinem Bann\\
Flehen meine Lippen leise:\\
Schütze mich, mein Talisman,\\
Im Geheimnis deiner Kreise.
\end{liebesgedicht}

\begin{liebesgedicht}
Da draußen aus grauer Wolkenschicht\\
Eintönig rieselt die Regenflut,\\
Doch hell aus seliger Augen Licht\\
Strömt mir die goldene Himmelsglut.

Von deinen Lippen entgegenlacht\\
Verschwiegene Wonne der Lenzeszeit.\\
Mit Rosenwangen das Glück erwacht\\
Verschämt zu leuchtender Wirklichkeit.
\end{liebesgedicht}

\begin{liebesgedicht}
Wohl sind im Weltenschoß der finstern Nacht\\
Viel tausend Sonnen rings im Raum erwacht,

Doch eine nur zieht machtvoll zu sich hin\\
Den Erdenball als stille Herrscherin.

Nur eine leuchtet, daß der Tag erglüht,\\ Nur eine wärmt, daß neu
der Frühling blüht.

Und zu der einen nur vertrauend fleht\\ Der ferne Träumer selig im
Gebet.

Lauscht sie der Stimme dann im weiten All,\\ Vernimmt sie ihres
Namens Widerhall,

Und schickt sie suchend ihre Strahlen aus,\\ Der eignen Farbe Licht
kennt sie heraus,

Das Sonnengold, das seine Welt verklärt,\\ Der Wärme Glut, die
seine Lieder nährt.
\end{liebesgedicht}

\begin{liebesgedicht}
O Tag des Findens, siegend wirf herein\\
In dunkle Herzen deine Flammenzeichen!\\
Groß wie der Morgen, dem die Schatten weichen,\\
Groß laß und klar das neue Leben sein!

Nicht jenen Halben, die so arm und klein\\
Um das Geheimnis heiligen Feuers schleichen,\\
Darf die Geliebte, die ich kenne, gleichen;\\
Denn ihre Liebe haßt den eitlen Schein.

Ich will in spielerischem Zeitvertreib\\
Um Worte nicht, um Küsse nicht mehr werben,\\
Ein Stückchen Seele und ein Stückchen Leib –

Sei es die Rettung, sei es das Verderben!\\
Nichts oder alles – Leben oder Sterben!\\
Gib, lichter Tag, mein alles, mir, mein Weib!
\end{liebesgedicht}

\smallskip

Wera legte die Blätter beiseite. Alles stand so deutlich vor ihren
Augen, der ganze glückliche Winter, von jenem ersten
unausgesprochenen Bewußtsein bis zu jenem Tage des Findens, da er
ihr noch abends die Lieder sandte –~– Und dieses Frühjahr! Aber das
Glücksgefühl selbst, das Glück, das wollte die Erinnerung nicht mit
sich bringen!

Daß sich all dies Nehmen und Geben wiederholen sollte, immer enger
und heißer wiederholen sollte! Wie sollte sie das ertragen! Nicht
jenen Halben wollte sie gleichen, sie wollte ihm schreiben~–~–

Nein, das ging nicht an. Wie sollte er das ertragen? Sie wußte
doch, was sie ihm war. Sie wußte, wie eng nun auch sein Schaffen
mit ihrem gemeinsamen Erlebnis verknüpft war. Mit welchem Rechte
durfte sie das stören, was sich in der freiwilligen Hingabe ihrer
Persönlichkeiten aufgebaut hatte zu einem neuen, mächtigen
Menschendasein?

Und sie selbst, wie stolz war sie auf diesen Mann, den sie den
ihren nennen durfte. Mußte denn dies alles zusammenbrechen um
dieses eines Mangels willen, der durch Aspiras Eintritt in ihre
Menschenseele geschaffen war? Wera, der freie Mensch, empörte sich
in ihr, sie zürnte Aspira, der Wolke, und ach, das war ja wieder
sie selbst! Sie war nicht imstande, völlig in Weras Seele zu
dringen. Ein unlöslicher Widerspruch in ihrem Innern!

Der Kampf mußte ausgestritten werden. Aber das sah sie jetzt ein,
da die ganze Fülle ihres Lebens mit Sohm wieder in ihr wirksam
geworden war, auf diese Probe zweier Tage durfte sie keinen
entscheidenden Schritt tun. Wer kennt die wunderbare Wirkung
solcher Seelenmischung? Wer weiß, wie mit der Zeit eine Anpassung
des Gefühls sich gestaltet? Das ist keine reine Gehirnarbeit, wie
ihre Erkenntnis, das ist eine Inanspruchnahme des ganzen
Organismus~–~–

Gewiß, sie war zu vorschnell in ihrer Verzweiflung. Zwei Tage und
eine Nacht erst war sie Mensch, und da wollte sie sogleich
ergriffen haben und beherrschen, was die Menschen als ihr tiefstes
Geheimnis preisen, die Liebe?

Und sie hatte in ihrer neuen Gestalt den Mann selbst noch nicht
einmal gesehen und gesprochen, den sie als Wera liebte –~– Das war
doch ihre erste Pflicht, nun zu versuchen, Weras Erbteil auch nach
dieser Seite mit gutem, ehrlichem Willen auf sich zu nehmen!

Ja, Martin hatte Recht: „Sobald Sie sich kräftiger fühlen, kehren
Sie zurück nach Weidburg.“

O, kräftig fühlte sie sich. Das war ja wohl eine Wirkung dieser
Seelenmischerei. Von der nervösen Abspannung, die sich Wera durch
ihre angestrengte Arbeit und durch die seelische Erregung ihres
Brautstandes zugezogen hatte, war vom ersten Augenblicke an nach
ihrem Erwachen am Gletscher nichts zu spüren gewesen. Nur die Angst
vor der Liebe, die sie nicht verstand, die ihr Widerwillen
einflößte, hatte sie heute in Verwirrung gesetzt. Aber nun war sie
ruhiger. Sie mußte versuchen, auch diese Schwierigkeit zu
überwinden.

Und wieder leuchtete vor ihr groß und strahlend das hohe Ziel.
Menschendenken und Gewalt der Elemente wollte sie vereinen, nicht
so, daß die besiegte Natur dem Gesetze gezwungen diene, sondern so,
daß sie es verstehe als eine wohltuende Macht und sich dem Menschen
willig offenbare, damit sie beide eins werden im Schauen und
Schaffen der ewigen Bestimmung alles Seienden.

Nun glaubte sie zu verstehen, was es heißt: Das Leid des Schöpfers
um sein Werk. Der Schöpfer sieht sein Werk als ein Ideal, aber der
spröde Stoff hemmt den bildenden Willen. Das Werk sträubt sich
gegen sein Werden, und der Schöpfer erfährt das Leid seiner
Ohnmacht – bis er doch endlich die Macht gewinnt. Sie wollte sie
gewinnen. Mit Weras Leben war ihr das Mittel gegeben, das große
Versöhnungswerk von Natur und Menschheit zu vollbringen. Aber
dieses Mittel war zunächst hier im Gebirge nicht anzuwenden. Sie
mußte suchen, Menschen für sich zu gewinnen. In jedem Falle mußte
sie nach Weidburg, zu ihrem – Verlobten.

Wie leer klang dieses Wort. Es mußte Leben gewinnen!

Sie nahm einen Briefbogen und schrieb:

„Geliebter Paul.“

Da stand es. Sie wollte das Blatt wegwerfen. Das konnte sie ja doch
nicht schreiben, das war ja eine Lüge. Noch einmal überkam sie der
ganze Jammer ihrer Doppelseele. Doch sie überwand sich. Es sollte
ja keine Lüge sein, denn es sollte Wahrheit werden. Also weiter!

„Du erhältst nur diese Zeilen, doch ich hoffe, Du wirst mir nicht
zürnen, denn übermorgen bin ich selbst bei Dir. Ich habe mich so
prachtvoll erholt, und es geht mir so vorzüglich, daß ich mich zu
Tode langweile. Ich halte es nicht mehr aus und reise morgen. Auf
Wiedersehen! Mit tausend Küssen Deine Wera.“

Den Schluß hatte sie ganz mechanisch hingeworfen, wie ihn Wera
schon so oft geschrieben. Als sie ihre Zeilen durchlas, begriff sie
ihn selbst nicht. Aber sie ließ es dabei. Mochte doch Wera das
Mögliche tun!

Sie traf ihre Vorbereitungen zur Reise. Den morgigen Vormittag
brauchte sie noch, um ihre Aufzeichnungen als Aspira an sicherm
Orte draußen zu verbergen. Deswegen wollte sie erst am Nachmittag
abreisen. Und in der Nacht war sie in Weidburg.

\section{Undine}

Es war Anfang September. Ein warmer Abend senkte sich über die
ausgedehnten Parkanlagen von Weidburg. Das neue, schloßartige
Gebäude, worin das geologische und das chemische Institut
untergebracht waren, stand verwaist. Die Studenten befanden sich in
den Ferien, auch die Assistenten hatten sich jetzt auf die Reise
begeben. Nur im zweiten Stockwerk, wo Professor Sohm, der Leiter
des geologischen Instituts, wohnte, gönnten die geöffneten Fenster
der Abendluft freien Zutritt.

Sohm lehnte an dem Eckfenster seines Studierzimmers. Die kräftig
gebräunte Farbe seines Antlitzes, der flotte Schnurrbart und der
freie Blick seiner grau-blauen Augen ließen ihn jünger erscheinen,
als seinem Alter entsprach. Schon als er vor fünf Jahren nach
Weidburg berufen wurde, hatte er gehofft, sich hier eine
Häuslichkeit zu gründen. Aber eine schwere Enttäuschung in der
Person seiner erwählten Braut, einer Ausländerin, die er auf seinen
wissenschaftlichen Reisen im fernen Osten kennen gelernt hatte,
warf einen tiefen Schatten auf die ersten Jahre in Weidburg. Erst
allmählich hatte er sich wieder zur Ruhe durchgearbeitet und die
alte Heiterkeit seines Gemüts zurückgewonnen. Da lernte er Wera
Lentius kennen. Das junge, schöne Mädchen, das, nach dem Verlust
ihrer Eltern ganz alleinstehend, still und zurückgezogen sich mit
eisernem Fleiße ihren Studien widmete, hatte bald seine aufrichtige
Wertschätzung errungen. Seitdem er dann Gelegenheit gefunden, ihr
persönlich näher zu treten, eroberte die Liebenswürdigkeit ihres
harmonischen Wesens sein ganzes Herz. Es lag über ihm wie ein
stilles Glück, das er sich nicht auszumalen wagte. Und als er ihr
einen Anteil an seiner Arbeit anbot, wußten beide, daß dies zu
einer Entscheidung führen mußte.

Nun erblühte sein Leben in einer neuen Jugend. Neben der besonnenen
Strenge seines theoretischen Schaffens barg sein Inneres ein warmes
Poetenherz. Das bewahrte ihn vor der Einseitigkeit, die dem
Gelehrten in seiner notwendigen Beschränkung auf spezielle Aufgaben
so leicht droht. Das glühte in einem unaussprechlichen Glücksgefühl
auf in der holden Zärtlichkeit, die ihm Weras Nähe gewährte. Und
die Beobachtung, wie bei dieser leidenschaftlichen Liebe sie
zugleich eine gemeinsame Lebensarbeit selbstlos verband, erfüllte
ihn mit tiefem Vertrauen in ihre Zukunft. Im Herbst wollten sie
sich in ihrem gemeinsamen Heim vereinigen.

Und nun war doch wieder eine unklare Sorge in seiner Seele
aufgestiegen.

Sohm spähte über den freien Platz, über den Wera kommen mußte. Es
war die Stunde ihres gemeinsamen Spazierganges, der sie fast
regelmäßig durch die Anlagen zu Geheimrat Röteleins führte. In der
am Hügelabhang reizend gelegenen Villa, die Rötelein mit seiner
Familie bewohnte, pflegten sie als ständige Gäste die Abende
zuzubringen.

Sohm sah nach der Uhr. Dann setzte er sich in den Lehnstuhl ans
Fenster und stützte den Kopf in die Hand. In den letzten Tagen
hatte sich seine Besorgnis um Wera ernstlich gesteigert.

Es war ihm unbegreiflich – seit zehn Wochen, seitdem Wera von ihrem
kurzen Erholungsaufenthalt in den Alpen körperlich gekräftigt
zurückgekehrt war, hatte er in ihrem Verhalten eine eigentümliche
Veränderung bemerkt.

In dem ersten Augenblick des Wiedersehens, als er sie
leidenschaftlich in seine Arme zog, war es ihm aufgefallen, wie sie
seine Liebkosungen erwiderte. Fast als wenn sie sich einen Zwang
antun müßte! Es ließ sich nicht definieren, es ließ sich nur
fühlen.

Er hatte sie mit ihrer Schwerfälligkeit geneckt. Sie hätte wohl zu
viel Gletscherluft geschluckt? Die müßte wieder herausgekocht
werden.

„Ach, wenn wir das könnten!“ hatte sie einmal lächelnd gesagt. Aber
in diesem Lächeln lag ein Hauch von Schwermut. Und dieser Zug, der
ihr sonst fremd gewesen war, kehrte von Zeit zu Zeit wieder, ja er
war immer stärker geworden. Wenn er mit Fragen in sie drang, ob sie
etwas beschwere, wich sie mit Scherzen aus. Er bemerkte, daß sie es
möglichst vermied, mit ihm unbeobachtet allein zu sein. Und doch
konnte er nicht einen Augenblick daran denken, daß ihr Liebe gegen
ihn erkaltet sei. Im Gegenteil, durch hundert kleine
Aufmerksamkeiten bewies sie ihm, daß sie nur ihm zum Gefallen und
zur Freude leben wollte. Sie war liebenswürdiger und freundlicher
wie je, ihre geistige Regsamkeit und Lebendigkeit hatte sich noch
gesteigert, und in Gesellschaft erregte sie das allgemeine
Entzücken. Wie unbeschreiblich anmutig war sie bei jenem Reigen
gewesen, an dem sie mitwirkte, als Röteleins ihr kleines Sommerfest
gaben. Das war ein Schweben, das fast schwerelos erschien. Und dann
wieder – an andern Tagen erschien sie hastig und unstet, als ob sie
von einer geheimen Unruhe umhergetrieben würde~–~–

Aber alles dies, sagte sich Sohm zum Troste, war ja wohl nur eine
vorübergehende Erregung des Brautstandes, das wird sich wohl geben.
Eine andere Veränderung in Weras Gedankenleben verursachte ihm viel
ernstere Bedenken.

Ihrer chemischen wissenschaftlichen Arbeit zwar widmete sich Wera
mit voller Hingabe, und ihre Analysen schritten glücklich vorwärts.
Daneben aber hatte sie sich jetzt mit philosophischen Spekulationen
beschäftigt, die nicht selten zu lebhaften Debatten zwischen den
Verlobten führten. Sie las mit Vorliebe naturphilosophische
Schriften, die Sohm zu stark in das Gebiet der Phantasie
hinüberzuschweifen schienen. So hatte sie sich ganz in Fechners
„Zend-Avesta“ hineingedacht. Und wenn Sohm mit seinem wohlwollenden
Humor über die Belebung und Beseelung der Erde scherzte und sie nur
als einen tiefsinnigen poetischen Einfall gelten lassen wollte, so
vertrat Wera hartnäckig den Standpunkt, daß die ganze Natur in
Wirklichkeit lebe, fühle und empfinde.

Jetzt, während Sohm auf Wera wartete, zogen viele ihrer kühnen
Behauptungen an seiner Erinnerung vorüber. Wollte sie ihn mit ihren
Aufstellungen nur necken, wenn sie allen Ernstes davon sprach, daß
in Fels und Berg, in Fluß und Meer, in Wolken und Wind nicht bloß
ein allgemeines Naturleben webe, sondern daß ein individuelles
Bewußtsein diese Natureinheiten beseele? Aber warum kam sie so oft
und eigensinnig auf diese phantastischen Vorstellungen zurück,
denen sie doch früher nicht nachgehangen hatte? Er konnte sich
eines Unbehagens bei dem Gedanken nicht erwehren, daß sie nicht
mehr so vollständig in allen Fragen eines Sinnes seien. Und bei
aller Achtung vor der Freiheit der Überzeugung fürchtete er doch
eine Gefahr für ihr gegenseitiges Verständnis, wenn sich Wera
wirklich in jene mystische Gedankenwelt mehr und mehr einspinnen
sollte.

Seine Stirn verdüsterte sich.

Da schlug die Klingel zweimal an. Weras Zeichen an der Haustür.

Sofort hatte Sohm seinen Hut ergriffen und war die Treppen
hinabgeeilt.

Als er die anmutige Gestalt erblickte, als die leuchtenden dunkeln
Augen unter dem großen Hute hervor ihn freundlich grüßten und er
die zierliche Hand in der seinen hielt, waren alle verdrießlichen
Grübeleien verflogen. Wie konnte man ihr zürnen?

Das Brautpaar schlug den gewohnten Weg zur Röteleinschen Villa ein,
wo es schon von der Familie zur Abendtafel erwartet wurde.

Nach Tische, die Dunkelheit war längst angebrochen, saß die kleine
Gesellschaft auf der Veranda am Ende des Gartens. Der Blick reichte
über die Anpflanzungen der Vorstadt und den Fluß bis an das ferne
Gebirge, das die weite Ebene begrenzte. Darüber am Horizonte hatten
sich Wolken getürmt, in denen das Wetter leuchtete. Mitunter sprang
ein stärkerer Blitz mit feuriger Spur blendend hervor.

Das muntere Gespräch war verstummt. Man beobachtete das wunderbare
Feuerwerk der Natur und die langsam sich verändernden bizarren
Formen der Wolken. Wera hatte ihren Platz verlassen und sich dicht
an das Geländer gestellt. Ihr Schattenriß hob sich deutlich von dem
Hintergrunde des Himmels ab. Bei jedem Blitze bewegte sich ihr Kopf
wie mit einem leichten Gruße.

„Schauen Sie nur, wie entzückend Ihre Braut aussieht,“ bemerkte
Frau Rötelein leise zu Sohm.

Er nickte glücklich, während seine Augen auf Weras Gestalt ruhten.
Ein doppelter Blitz züngelte mächtiger als die früheren aus der
dunkeln Wolkenmasse.

„Geduld, Geduld!“ murmelte Wera leise vor sich hin. „Wartet nur,
ich besuche euch wieder. Dann sollt ihr merkwürdige Dinge hören.“
Sie hob unwillkürlich den Arm, wie man jemand Lebewohl winkt, und
schritt zur Gesellschaft zurück.

Sohm ergriff ihre Hand und zog sie auf den Stuhl an seine Seite.
Alle saßen still und warteten auf einen neuen Blitz. Aber die
Erscheinung wiederholte sich nicht. Die hochgetürmte Wolke wurde
sichtlich kleiner.

„Ich glaube, Fräulein Lentius,“ sagte Rötelein scherzend, „Sie
haben das Wetter beschworen. Mir war es gerade, als wenn Sie vorhin
etwas geraunt hätten, gewiß einen Wetterspruch.“

„Halten Sie mich für eine Hexe, Herr Geheimrat?“ fragte Wera
lachend.

„Wer weiß?“ neckte Rötelein weiter. „Natürlich im besten Sinne.
Vielleicht sind Sie so ein Elementargeist, eine Sylphide oder eine
Undine.“

„An so etwas glauben Sie ja gar nicht,“ antwortete Wera.

„Aber Wera glaubt daran,“ rief Suse Rötelein dazwischen. „Denke
Papa, neulich hatte sie gesagt, die Berge und Flüsse und Wolken und
so weiter hätten auch Seelen. Nicht wahr, Herr Sohm?“

Wera schwieg, da sie wußte, daß Paul das Thema nicht liebte.

Sohm hätte auch die Frage lieber überhört, aber da die lebhafte
Suse sie nochmals wiederholte, so sagte er:

„Ich meine, Wera denkt nur an eine Allgemeinbeseelung der Natur im
Sinne Fechners, dagegen läßt sich doch höchstens einwenden, daß sie
Sache des Glaubens bleibt.“

„Nein, nein, sie sprach von Elementargeistern. Nicht wahr, Wera?
Weißt du, ich komme nur darauf, weil Papa dich eine Undine nannte.
Es war neulich, als du dazu kamst, wie ich Fouqués Undine las.“

„Und da habe ich gerade gesagt, daß diese Undine eine ganz
unhaltbare Figur ist und daß es solche Elementargeister gar nicht
gibt.“

„Ja, aber andere,“ behauptete Suse hartnäckig.

„Ach,“ sagte Frau Rötelein, „schelten Sie mir nicht die Undine, ich
mag das Büchlein so gern.“

„Verzeihen Sie, Frau Geheimrat, ich will weiter nichts gegen das
Buch sagen, obwohl es nicht mein Geschmack ist. Der Dichter mag
meinetwegen auch solche Produkte des Volksaberglaubens beleben.
Aber diese Art Romantik kann uns doch heute nicht genügen, wir
leben nun einmal alle in einer ganz andern Naturauffassung, wir
wissen zu viel von der Natur und den physischen Bedingungen des
Menschen. Ich wiederhole, das Märchen braucht sich darum natürlich
nicht zu kümmern. Wenn wir aber von Elementargeistern sprechen, an
deren wirkliche Existenz wir glauben sollen – und ich will gar
nicht leugnen, daß ich es tue~–“

„Aha!“

„Ja, aber doch nicht an Sylphiden und Undinen, nicht an Wesen mit
menschlichen Leibern, die im Wasser leben sollen, und so weiter.
Solchen Koboldspuk gibt es nicht in der Natur. Und dann diese
unmögliche Psychologie! Undine soll keine Seele haben! Was soll man
sich darunter vorstellen? Das Wesen lebt und denkt und will und
fühlt doch, es ist eine Bewußtseinseinheit, also muß es auch eine
Seele haben, wenn man dem Worte überhaupt einen Sinn geben will.
Meine Elementargeister haben selbst Seelen, aber sie haben keinen
Menschenkörper. Sie sind mehr oder weniger einheitliche
Naturformen, wenngleich anders organisiert als unser Nervensystem.
Aber dadurch bedeuten ihre Veränderungen auch für sie ein bewußtes
Erlebnis. In diesem Sinne rede ich von Elementargeistern als von
einem Bewußtsein der Existenz bei elementaren Gewalten. Und da
sollte auch der Dichter einsetzen. Da könnte er die Natur, die wir
in Erkenntnis und Technik entgöttern müssen, wieder im Gefühle
lebendig machen.“

„Nun,“ sagte Rötelein, „wenn Undine behauptet, sie hätte keine
Seele, so müssen Sie ihr diesen Verstoß gegen die psychologische
Terminologie nicht so übel nehmen. Sie hatte doch keine
Reifeprüfung abgelegt. Sie versteht eben unter Seele nur das, was
sie eine unsterbliche Seele nennt, und was sie bei den Menschen
gewinnen will.“

„Und was ist das für eine Seele? Wie zeigt sich das? Daß sie recht
mitfühlsam und zärtlich und geduldig und gehorsam ist, so recht
unterwürfig fügsam und so recht langweilig – dazu lohnt es sich ein
Mensch zu werden mit einer unsterblichen Seele? Diese ganze
mittelalterlich-kirchliche Auffassung vom Seelenleben kann mich
nicht interessieren.“

„Sie können doch,“ erwiderte Rötelein, „vom Dichter nicht mehr
verlangen, als der Zeit entspricht, in die er sein Märchen verlegt.
Wenn Ihnen aber die Auffassung dieser konventionellen Ritterzeit
nicht zusagt, so hindert Sie nichts, sich diese sogenannte
unsterblichen Seele als das zu deuten, was wir uns heute darunter
denken. Ich meine als das, was das Menschenbewußtsein über alle
Natur hinaushebt, jene zeitlose Bestimmung, ein Selbstzweck zu
sein, eine sittliche Idee zu vertreten.“

„Eine Persönlichkeit also,“ sagte Wera nach kurzem Schweigen. Sie
nickte langsam mit dem Kopfe und fuhr dann fort: „Das läßt sich
hören. Ein Wesen, das sich selbst bestimmt, das sich seiner
Selbstbestimmung und Verantwortung bewußt wird. Das wäre freilich
eine herrliche Aufgabe für einen Elementargeist, eine höhere Stufe,
die zu erreichen ich ihm gönnen möchte.“

„Na, Schatz,“ sagte Sohm lachend, „werde nur nicht gar zu
feierlich. Ich bekomme schon Angst. Bei der nächsten Analyse redet
am Ende das Quellwasser aus deiner Flasche: Erlauben Sie mal, was
destillieren Sie mich da? Ich bin eine freie Persönlichkeit und
wünsche meinen Selbstzweck in flüssigem Zustande zu erfüllen.“

„Das brauchst du gar nicht zu befürchten,“ antwortete Wera
ernsthaft. „Als Wasser oder Dampf bleiben die Elemente eben
Elementargeister; wenn sie Persönlichkeiten werden sollen, müßten
sie erst einen Menschenleib erwerben.“

„Aber ich bitte Sie,“ sagte Rötelein, „was wir am Menschen seine
Persönlichkeit nennen, kommt doch nicht den Elementen zu, die
seinen Leib zusammensetzen, sondern dass ist die Bestimmung, um
derentwillen dieser Leib eine Einheit bildet.“

„Ja,“ verteidigte sich Wera, „aber diese Einheit ist an die
physische Einheit des Nervensystems gebunden und kann sich nur in
dieser als Selbstbestimmung bewußt werden. Die Elemente erleben
sich freilich selbst und mögen dabei in ihrer Art Gefühle und
Vorstellungen haben, die nur der Märchendichter in unsre Sprache
übersetzen kann. Sollen sie aber nicht bloß symbolisch, sondern in
Wirklichkeit zum Verständnis des Menschenwesens kommen, so gehört
dazu ein Zellenleib mit seinem Gehirn~–“

„Aber Wera,“ fiel Sohm ein, „nimm mir's nicht übel – man darf doch
diese Phantasien nicht zu ernsthaft nehmen. Ich fürchte, du
verlierst dich da in Spekulationen, die wir als Naturforscher
besser beiseite lassen.“

Wera schwieg verstimmt.

„Stören Sie doch Ihr Fräulein Braut nicht,“ sagte Rötelein heiter.
„Sie wollen sicherlich nicht sagen, Fräulein Lentius, daß so ein
Quantum Wasserdampf sich in unser Nervensystem schleichen und an
unserm Bewußtsein teilhaben könne, und sich dabei doch noch
erinnern, daß es einmal eine Wolke gewesen sei? Vorausgesetzt, –
was ich ja für meine Person zugebe – daß diese Naturgebilde
überhaupt ein Bewußtsein besitzen.“

„Verzeihen Sie, Herr Geheimrat,“ entgegnete Wera, „das will ich
gerade sagen. Ich meine, daß so ein Elementargeist unter besondern
Umständen wirklich einmal zur Vernunft gelangen könne, ohne seine
Zugehörigkeit zum Reiche der Naturgewalten zu verlieren.“

„Um Himmelswillen, Wera!“ rief Sohm entsetzt. „Da riskiert man
schon eher, daß der Mensch seine Vernunft verliert!“

„Paul!“

„Schatz, sei nicht böse – aber aus deinen Worten kann man wirklich
nicht entnehmen, daß du uns zum Besten haben willst.“

„Das will ich auch nicht. Ich spreche im Ernste – Paul, du brauchst
kein so finsteres Gesicht zu machen. Denn sieh mal, das weißt du
ja, daß wir im Ziele einige sind. Wir alle betrachten die Natur als
ein Mittel, die unendlichen Zwecke der Vernunft mehr und mehr zu
verwirklichen. Wir wollen die Natur unterwerfen, damit Kultur
herrsche; wir wollen die Naturnotwendigkeit in den Dienst der
Freiheit stellen. Nicht wahr?“

„Gewiß, Wera. Aber dazu brauchen dir deine Elementargeister nicht.
Der Weg geht durch die Naturwissenschaft und Technik. Das einzige
Mittel ist die Erkenntnis der Gesetze.“

„Siehst du, Paul,“ sagte Wera und faßte seine Hand, „auch darin
sind wir völlig einig. Aber nun kommt das, was du nicht leiden
kannst – und da wage ich augenblicklich gar nicht mehr, es zu
sagen.“

„Nun haben Sie mich aber neugierig gemacht,“ mischte Rötelein sich
ein. „Jetzt müssen Sie schon Farbe bekennen! Ich weiß nämlich
nicht, wozu Sie noch Ihre biedern Elementargeister brauchen, wenn
Sie das Recht der Naturwissenschaft voll anerkennen.“

„Wenn sie nun aber einmal da sind –“

Sohm machte eine ungeduldige Bewegung.

„Na, na,“ sagte Rötelein gemütlich, „wir glauben ja freilich nicht
daran. Vielleicht meine Frau manchmal so ein bißchen, wenn's gerade
paßt. Aber wir können ja einmal ganz hypothetisch sprechen.
Angenommen, es gäbe solche geheimnisvollen Naturseelen – was wollen
Sie damit erreichen?“

„Eine Abkürzung des Weges zur Kultur.“

Röteleins sahen sie erstaunt an. Sohm wußte, worauf Wera
hinauswollte. Er spottete:

„Abrichten will sie Wera zu wissenschaftlichen Haustierchen,
Kultureseln!“

„Nenn's, wie du willst,“ erwiderte Wera heftig. „Ich sage, warum
soll das Wissen um die großen Ziele der Kultur allein im Menschen
lebendig sein, warum sollen nicht auch Geister andrer Art daran
teilnehmen und die Arbeit fördern lernen?“

„Aber, liebes Fräulein,“ entgegnete Rötelein ernsthafter, „das ist
eigentlich gar nichts so Neues, sondern nur ein modernisierter
Ausdruck für einen längst überwundenen Standpunkt aus der
Kinderzeit der Naturforschung. Das geht in die Zeit vor Paracelsus
zurück, etwa in die Magie des Agrippa von Nettesheim. Und alles,
was wir errungen haben, verdanken wir den Männern des 16. und
17.\,Jahrhunderts, die uns gezeigt haben, daß die Erkenntnis nur von
außen ansetzen kann und sich mit psychologischen Träumen nicht
abgeben darf. Sie werden uns doch nicht um vier Jahrhunderte in der
Kulturgeschichte zurückschrauben wollen.“

„Nein, Herr Geheimrat. Damals wäre meine heutige Idee nur ein Traum
gewesen wie der der Magie. Damals wagte man sich an das Beginnen,
ohne eine Ahnung, wie es durchzuführen sei. Aber ganz ebenso
unfruchtbar tastete man damals auf experimentellem Wege. Man setzte
vertrauensvoll einen Blumentopf auf die Waage, um die
Nahrungsaufnahme der Pflanze zu beobachten, ohne zu wissen, daß die
damaligen Mittel der Messung in keiner Weise für die erforderliche
Präzision ausreichen konnten. Man wußte überhaupt nicht, was man
messen sollte; und wenn man imstande gewesen wäre, die
Elementargeister in Dienst zu nehmen, man hätte nicht gewußt, was
man ihnen auftragen sollte. Aber heute wissen wir Bescheid und
können ein Erkenntnismittel zu Hilfe nehmen, das damals unbrauchbar
war. Heute haben wir die experimentelle und mathematische
Naturwissenschaft und laufen nicht mehr Gefahr, der Mystik in die
Arme zu fallen. Wir können uns aber neue Hilfsarbeiter für die
Erkenntnis heranziehen.“

„Nun sagen Sie mir bloß, verehrtestes Fräulein, wie Sie das machen
wollen?“

„Nun, beschwören lassen sie sich nicht, weder von Faust noch von
sonst jemand. Auch als Intelligenzen sind sie nicht über- sondern
untermenschlich. Daß man Kommendes vorausberechnen kann, verstehen
sie nicht. Aber sie sind individuelle, bewußte Wesen und vermögen
dabei rein physisch vieles, was der Mensch nicht leisten kann. Wenn
ihnen nun der Mensch einen direkten Auftrag gäbe, z.\,B. einer
Luftströmung, ihren Weg über unzugängliche Landstrecken zu
beschreiben~–“

Jetzt brach Rötelein in ein herzliches Lachen aus. Sohm hatte, um
sich durch seinen Unwillen nicht zu einer unfreundlichen
Unterbrechung hinreißen zu lassen, seinen Platz verändert. Jetzt
trat er wieder hinzu.

Rötelein rief lustig:

„Sie sind wirklich kostbar, Fräulein Lentius! Jetzt möchte ich bloß
noch wissen, in welcher Sprache Sie den Wind befragen wollen.“

Wera erhob sich und sagte trocken: „Dazu brauche ich natürlich
einen Dolmetscher. Das wird eben der Elementargeist sein, der ein
Mensch geworden ist. Und der wird sich finden.“

Rötelein lachte noch immer.

Sohm bezwang sich und ergriff Weras Hände.

„Weißt du, Schatz,“ sagte er, „bis dahin wollen wir doch lieber
Frieden schließen, denn so lange kann ich unmöglich mit dir
schmollen, obwohl du uns gründlich zum Besten gehabt hast. Und ich
fürchte, wir werden auch hier nicht auf deinen Dolmetscher warten
können, denn es ist ziemlich spät geworden.“

Wera stand stumm. Sie hatte sich zu weit hinreißen lassen – und
dennoch, sie mußte doch einmal die Menschen zu überzeugen
suchen~–~–

„Nein, nein,“ rief Fräulein Rötelein. „So dürfen Sie noch nicht
gehen. Auf Ihre Geister lassen Sie uns noch ein menschliches
Gläschen trinken. Wir sind ja ganz von unsrer Undine abgekommen.“

Suse streichelte der Freundin die Hand. Wera zwang sich zu einem
Lächeln. Sie sah ein, daß es vergeblich war, von dem zu sprechen,
was sie im Innersten bewegte.

„Ach,“ sagte sie, indem sie sich wieder setzte, „ich wünschte, wir
wären gar nicht auf diese unglückliche Wassertante geraten.“

„O, im Gegenteil,“ rief Rötelein, indem er Wera sein Glas
entgegenhielt, „ich möchte Ihre köstlichen Fingerzeige nicht
vermissen. Es wäre doch nett, wenn heutzutage so ein Elementargeist
der Undine nachahmte, um, falls nicht zu einer unsterblichen Seele,
so doch zu einem Menschenhirn zu kommen.“

„Nun laß einmal unsre Wera zufrieden,“ sagte Frau Rötelein, indem
sie Wera zärtlich die Wange klopfte. „Wenn jetzt eine Undine käme,
würdest du sie doch nicht als Elementargeist anerkennen, und einen
Ritter können wir ihr auch nicht zum Gemahl verschaffen.“

„Na, heutzutage,“ scherzte Rötelein weiter, „würde ihr der Ritter
auch nichts nutzen; es müßte mindestens ein Professor sein. Sie
soll ja wissenschaftliche Erkenntnis gewinnen. Am besten wäre ein
Geologe. Aber freilich, unsrer ist schon vergeben.“

„Wenn Sie mich meinen,“ sagte Sohm jetzt ebenfalls heiter, „ich
habe Gott sei Dank mein Elementargeisterchen; um des Gehirns willen
braucht das nicht zu heiraten.“

Er drückte Weras Hand und sah sie zärtlich an. Wera erwiderte den
Druck mechanisch. Es durchzuckte sei ein Gedanke, der sie im
Augenblick alles vergessen ließ. Mit Gewalt versetzte sie sich
wieder in ihre Umgebung. Und halblaut sagte sei vor sich hin:

„Es ist ja doch Unsinn.“

„Das Heiraten?“ neckte sie Frau Rötelein, die ihre Worte verstanden
hatte.

„Ja, von der Undine, meine ich.“

„Ach so!“ sagte Sohm lachend.

„Wie soll sie dadurch zu einer Seele kommen? Das könnten doch
höchstens~–“ Wera brach ab.

„Ihre Nachkommen, meinen Sie?“ rief Rötelein. „Das sollte ich auch
meinen. Der Romantiker hat natürlich an irgend eine mystische
Einwirkung durch die Ehe gedacht.“

„Da will ich doch einmal den Dichter in Schutz nehmen,“ mischte
sich die Hausfrau ein. „Mann und Frau leben sich eben ineinander
ein. Manchmal kann es ziemlich lange dauern. Aber nach und nach
stellt sich durch die Gewohnheit eine Übereinstimmung des ganzen
Seelenlebens ein, und eine solche Anpassung ist sicherer als eine
Vererbung. Man lernt sich verstehen.“

„Und Sie glauben,“ fragte Wera etwas zögernd, „daß sich in der Ehe
manche Fähigkeit, ich will sagen, ein Verständnis für gewisse
Tiefen des menschlichen Bewußtseins erst ausbildet, das nur durch
eine solche Verbindung zu gewinnen ist?“

„Ganz sicher.“

„Es ist ein zu kompliziertes Ding, so ein Menschenhirn. Schlimmer
als alle Schluchten des Hochgebirgs. Man denkt, man hat in jedes
Winkelchen geguckt, und dann gibt's immer noch Windungen, wo man
doch nicht hineinschlüpfen kann.“

Sie erhob sich und reichte Rötelein die Hand.

„Ja, Schatz,“ sagte Sohm, „wir wollen gehen. Du bist heute dunkel
wie unser Heimweg.“

Und der Heimweg war nicht nur dunkel, er war auch schweigsam. Arm
in Arm schritten sie dahin und wechselten doch nur wenige Worte.
Beide waren mit ihren Gedanken beschäftigt.

Sohm war unzufrieden mit Weras lebhaften Auseinandersetzungen. Ihn
ängstigte die eingehende Hingabe Weras an diese Vorstellungen, die
sie sichtlich zu einer ganzen Theorie ausgesponnen hatte. Er wollte
es vermeiden, darauf zurückzukommen.

Wera rang innerlich schwer mit ihrer Aspiraseele. Sie fühlte, daß
sie binnen kurzem eine Entscheidung treffen müsse. Aber wie?

So waren sie bis an Weras Haustür gelangt.

Sohm öffnete und neigte sich dann zu Wera zu einem Abschiedskusse.

Da schlang sie plötzlich beide Arme um seinen Hals und küßte ihn
heiß und leidenschaftlich wie noch nie seit ihrer Rückkehr aus dem
Gebirge.

Sie lehnte das Haupt an seine Wange und schluchzte:

„Sei nicht traurig, Geliebter, behalte mich lieb!“

Noch ein inniger Kuß, und sie war im Hause verschwunden.

\section{Im Laboratorium}

Vergeblich suchte Wera den Schlummer.

Sie war ratlos. Sie schämte sich dieser Liebkosung beim Abschiede,
– das war eine Willensanstrengung, aus langem inneren Ringen
hervorgegangen, und doch, sie fühlte es, ein leeres Spiel, ein
künstliches Feuer, das ihr keine innere Wärme gab. War es nicht ein
Betrug?

Sie wußte ja genau, wie glücklich er jetzt sein würde. Und wie
gönnte sie ihm dies Glück, wie innig wünschte sie, es ihm so zu
geben, wie sie es früher als Wera gekonnt hatte. Aber nun! Es war
ja doch eine Verstellung. Würde sie es durchsetzen können, um
seines Glückes willen sie zu üben? Durfte sie das?

Nein! Nein! So oft sie sich diese Frage vorgelegt hatte, immer
zwingender schien ihr dieses Nein zu werden. Hinweg von hier,
hinweg! klang es in ihr.

Sie war nahe daran gewesen, ihn zu bitten: „Gibt mich frei!“ Aber
warum? Was sollte sie ihm sagen? Daß sie sich in ihrer Liebe
getäuscht habe? Um seinetwillen konnte sie den Mut nicht finden.
Und sie selbst?

Zehn Wochen täglichen, vertrauten Zusammenseins, und doch keine
Spur in ihren Adern, kein Hauch in ihrem Herzen von der Glut der
Leidenschaft, von der verzehrenden Seligkeit, die Liebeswonne heißt
–~– Gebärden ohne Gefühl!

Und trotzdem hatte sie heute noch einmal den Versuch gemacht –~–
Eine neue Hoffnung hatte sich in ihr geregt. Woran sie gezweifelt
hatte, daß sie die Liebe gewinnen konnte, die er verdiente, das
erschien ihr nun nicht mehr unmöglich; das war der Gedanke, der ihr
bei der letzten Wendung des Gesprächs über Undine aufgeblitzt war.
Vielleicht konnte dieser Teil der Seele Weras wirklich erst von ihr
errungen werden, wenn sie Pauls Frau geworden war. Vielleicht
gehörte dazu dieses unverständliche Zusammenleben, das die Menschen
Ehe nannten. Und wenn es Jahre dauerte – was sind Jahre für ein
Wolkenleben? Sie mußten daran gegeben werden, wenn sie dadurch
ihrer Aufgabe leben konnte, ohne sein Glück zu zerstören. Und wenn
es dann gelang, wenn sie sich beiden nun wirklich ganz finden
lernten, auch noch in diesem Innersten des Menschenseins, dann war
es ja kein Unrecht, keine Lüge, keine Entwürdigung mehr, wenn sie
bis dahin sich zwang, das zu scheinen, was sie werden wollte,
werden würde. Sie mußte den Schmerz der Zärtlichkeit darangeben,
bis er sich in Lust und Glück verwandle, solange noch eine Hoffnung
des Gelingens war.

Und darum hatte sie ihn heute so glücklich gemacht. Und sie barg
das Gesicht in ihre Hände und weinte.

Auf dem stummen Heimwege hatte sie sich das ausgedacht. Aber nun –
nun kamen ihr doch wieder Zweifel, und sie fuhr empor von neuen
Fragen durchwühlt. Die Unwahrheit! Was halfen da Beschwichtigungen!
Selbst in dem Vertrauen, Paul alles zu werden, was er von Wera
erhoffte, durfte sie diesen Bund eingehen, wenn sie ihrem Manne das
Geheimnis ihrer Herkunft für immer verschweigen mußte? Menschen
können sich ja lieben und ein gemeinsames Leben führen und doch
ganz verschiedene Ansichten über Welt und Dinge haben, aber sie
müssen es voneinander wissen, sie müssen sich verstehen, sie müssen
es sich sagen können und ihre Gründe achten. Sie aber konnte
niemals sagen: „Du hast eine Wolke geheiratet.“

Warum nicht? Es war ihr verboten. Dennoch, wenn sie sich entschloß,
ein Mensch zu bleiben und nie wieder mit dem Wolkenreich in Verkehr
zu treten, so war sie auch an das Verbot des Vaters nicht gebunden.
Aber das durfte sie nicht eher, bis sie ihre Sendung erfüllt hatte,
bis den Elementen das Verständnis für die Aufgabe der Menschen
erschlossen war. Das konnte sie nur durch diese Seelenmischung
erreichen. Sonst verlor sie die Macht des Zusammenhangs und der
Vermittlung. Also mußte sie hinaus zu den Geistern der Berge, ehe
sie das Band der Menschheit unauflöslich um sich legte.

Aber wenn es wirklich einmal dazu käme, daß sie die Wahrheit sagen
durfte um ihrer Aufgabe willen – würde denn das jemals möglich sein
um der Menschen willen? Das hatte sie eben vorläufig erproben
wollen. Und heute hatte sie ja deutlicher wie je gesehen – man
würde ihr nie glauben. Diese Menschen konnten sie nicht verstehen.
Man würde sie für geistesgestört halten, für wahnsinnig!

Und Paul! O Gott! Auch für ihn wäre sie die Unzurechnungsfähige,
die sich an eine fixe Idee klammert. Ja, das war die Angst, die
manchmal leise in seinem Auge, in seinen verständnislosen Mienen
zuckte, wenn sie von den Elementargeistern sprach. Und diese Angst,
diese Qual – würde sie ihn nicht vernichten? Zum mindesten ihr
gemeinsames Leben würde sie zerstören. Er würde sie ertragen, wie
man einen unheilbaren Kranken erträgt – aus mitleidiger Liebe –
doch das war kein Leben. Dazu wird man nicht ein Mensch. Und das
durfte sie um seinetwillen nicht tun.

Sie grübelte verzweifelt. Gab es keinen Ausweg?

Nur dann, wenn sich Paul von der Wahrheit überzeugen ließ, daß ihre
Sendung zu den Menschen kein Spiel wahnwitziger Phantasie sei, wenn
er begriff, daß es Elementargeister gibt, daß ein Verkehr zwischen
ihnen und den Menschen möglich ist. Aber das wußte sie schon jetzt,
das würde bei ihm niemals eintreten.

Sollte sie ihr Menschsein aufgeben, ihre Sendung verleugnen? Statt
des Leides lieber das freie, sorglose Spiel der Wolke wieder
wählen? Das durfte sie doch nicht, ohne zuvor sich den Rat des
Vaters geholt und versucht zu haben, die Geister der Berge für ihr
Werk zu gewinnen.

Ja, sie mußte hinauf in die Heimat. Vielleicht konnte sie von dort
mit neuen Hoffnungen zurückkehren.

Aber eines mußte vorher noch hier geschehen. Noch eine Hoffnung
mußte sie verfolgen. Sie war es auch ihm schuldig, dem sie so
vieles zu rauben drohte. Wenigstens die Sorge sollte er nicht
haben, daß seine Braut ihres klaren Verstandes nicht mehr mächtig
wäre. Die Liebe, die sie ihm gab, sollte ihn nicht nur zur
mitleidigen Duldung ihres Glaubens führen, er sollte ihr Recht
anerkennen, er wollte gewiß werden, daß sie bewußt und pflichttreu
wie er selbst eine geprüfte Überzeugung vertrat.

Und dann, dann mochte die letzte Entscheidung fallen nicht hier,
sondern dort, wo ihre letzte Zuflucht war~–~–

Aus angstvollen Träumen durch ihre Gedanken immer wieder
aufgeschreckt, versank Wera erst gegen Morgen in einen wohltätigen
Schlaf. Später als gewöhnlich erwachte sie.

Auf dem Frühstückstisch fand sie einen Brief mit dem Poststempel
„Schmalbrück“. Neugierig öffnete sie. Von Martin, dem Ingenieur.
Sie war von Schmalbrück abgereist, ohne ihn seit ihrer Unterredung
im Walde wiedergesehen zu haben. Von Weidburg aus hatte sie ihm
einen Abschiedsgruß geschickt, aber nichts mehr von ihm gehört.

Sie las:

„Hochverehrtes Fräulein! Das Interesse, das Sie bei unserer letzten
Unterredung über den Tunnelbau bekundeten, und die wichtige
Anregung, die ich durch Ihre Bemerkung über die Quelle im
Silbertobel empfing, geben mir den Mut, mich in dieser
Angelegenheit an Sie zu wenden. Auch fühle ich mich verpflichtet,
Ihnen meinen innigsten Dank auszusprechen und zugleich über den
Verlauf der Angelegenheit zu berichten.

Die genauere örtliche Untersuchung und die chemische Prüfung des
Wassers an verschiedenen Stellen des Tobels haben Ihre Vermutung
durchweg bestätigt. Doch hat sich gezeigt, daß in einer kleinen
Parallelschlucht etwa 200~Meter weiterhin eine zweite, schwächere
Quelle ähnlicher Art auf ein nochmaliges Auftreten von Kalk
hinweist. Bedenklicher sind gewisse Erscheinungen beim Fortschritt
des Tunnels, die es wahrscheinlich machen, daß noch andere
Verwerfungen in der Tiefe bei weiterem Vordringen von unten her
unsere Arbeit bedrohen. Obgleich wir alle technischen
Vorsichtsmaßregeln getroffen haben, um einem etwaigen Einbruch
heißer Quellen sofort zu begegnen, hat sich die Direktion doch
entschlossen, noch ein geologisches Obergutachten einzufordern, und
beabsichtigt, sich zu diesem Zwecke an Herrn Professor Sohm zu
wenden.

Dieses Ansuchen und der Bericht mit den erforderlichen Zeichnungen
ging heute früh an Ihren Herrn Bräutigam ab. Vielleicht darf ich
hoffen, daß Ihre Teilnahme an dem Schicksal des Tunnels dazu
beitragen könnte, Herrn Professor Sohm zur Annahme des Antrages der
Direktion geneigt zu machen.

Möchten Sie, hochverehrtes Fräulein, in erwünschter völliger
Wiederherstellung Ihres Befindens diese Zeilen empfangen, mit denen
sich Ihnen empfiehlt Ihr aufrichtig ergebener Theodor Martin.“

Die Nachricht konnte Wera nur halb befriedigen. Sie fühlte sich
aufs neue beunruhigt, und ihr Entschluß erstarkte nur um so mehr,
das Innere des Berges so bald wie möglich selbst in Augenschein zu
nehmen, wenn – ja wenn sie wieder die Macht dazu hatte.

Sie beeilte sich, in das chemische Laboratorium zu kommen, wo sie
jetzt die einzige noch Arbeitende war. Kaum hatte sie ihre gewohnte
Beschäftigung aufgenommen, als Sohm bei ihr eintrat. Er legte
schnell einen Stoß Papiere aus der Hand und zog Wera in seine Arme.
Von gestern sprachen sie nicht mehr.

„Ich weiß schon, was du da bringst,“ sagte sie endlich. „Ich habe
auch einen Brief erhalten.“

„Von der Direktion?“

„Nein, von dem Oberingenieur am Tunnel. Ich habe dir ja erzählt,
daß ich dort war und dann die Silberquelle aufgesucht habe.“

„Ja, und deine Vermutung war richtig, wie die Analysen in Zürich
zeigen. Da weißt du wohl schon, daß ich zu einem Gutachten
aufgefordert bin?“

„Tu's doch!“ sagte sie, seine Hand ergreifend und sich an ihn
lehnend.

„Wenn du's willst, so wird wohl nichts anderes übrig bleiben,“
antwortete er glücklich. „Aber es müßte dann sofort sein. Bekomme
ich denn so leicht Urlaub von dir?“

„Es ist ja nicht auf lange.“

„Siehst du,“ sagte er lächelnd, „das kommt nun davon. Wärest du
schon mein Frauchen, wie ich eigentlich wollte, so könnten wir
jetzt zusammen hingehen.“

„Wer weiß, wo wir da jetzt wären,“ sagte sie. „Aber“ – sie stand
nachdenklich – „vielleicht – vielleicht könnte ich dich einmal
besuchen. Ich könnte ja auf ein paar Tage nach St.~Florentin gehen,
da kommst du leicht hinüber~–“

„Abgemacht!“ rief er fröhlich. „Das muß besiegelt werden, du
geliebtes~–“

Auf dem Arbeitstisch zischte etwas.

„Himmel, mein Apparat!“ rief Wera und sprang hinzu, um zum Rechen
zu sehen.

Sohm holte die Papiere. „Ich habe mich schon so ziemlich
orientiert,“ sagte er, während Wera sich vor den Tisch setzte. „Ich
werde heute telegraphieren und den Nachtzug benutzen.“ Er blätterte
in den Plänen.

„Und was denkst du denn über den Fall?“ fragte Wera gespannt.

„Soweit ich sehe, glaube ich eigentlich nicht an eine Gefahr. Nach
der Stärke deiner Silberquelle muß der Gneis das brüchige Kalkband
ganz abschließen. Die zweite Quelle ist schwach und viel weniger
kalkhaltig, sie kann also garnicht mit der ersten zusammenhängen.
Dort liegt wohl nur ein versprengtes Stück der Schicht. Aber
freilich – die Temperaturzunahme macht Bedenken – da muß ich erst
einmal an Ort und Stelle sehen, wie die Schichten liegen.“

„Es würde mich wirklich sehr freuen,“ rief Wera, „wenn alles gut
ginge!“

„Wohl um des Herrn – wie heißt er?“

„Martin. Ja. Er würde mir so leid tun. Und er ist so nett.“

Sohm drohte lächelnd mit dem Finger. „Da muß ich schon einmal hin,
um mit den Mann anzusehen. Übrigens – so sicher ist die Sache
keineswegs. Da der Kalk offenbar in der Tiefe zermalmt ist, kann
von dort aus irgendwo ein Schlammerguß nach oben gedrückt werden.
Kein Mensch kann so einem Berge in den Leib sehen.“

„Ein Mensch freilich nicht, aber –“

„Wera, du willst doch nicht wieder von dem unglücklichen Thema
anfangen? Lassen wir das doch nun~–“

„Aber Paul, du siehst doch, hier ist einmal ein Fall, wo der
Geologe fast machtlos ist. Wenn es jedoch ein Wesen gäbe, das nun
wirklich durch die Kalkschicht in ihren verschiedenen Teilen
hindurchstreichen könnte~–“

„Das gibt's auch. Wasser oder Luft.“

„Ja. Wenn dir aber dann das Wesen genau sagen könnte, wie die
Schicht verläuft und in welchem Zustande~–“

„Hm! Ja! Das wäre ganz schön. Das Wasser brauchte nur Kompaß,
Barometer, Thermometer und Geschwindigkeitsmesser mitzunehmen und
seine Route kartenmäßig festzulegen, und dann müßte es sich etwa
noch ein menschliches Sprachorgan anschaffen – oder meinst du, daß
eine Schreibmaschine genügen würde?“

„Du bist unausstehlich!“

„Aber, lieber Schatz, du kannst doch nicht verlangen, daß ich
solche Reden ernst nehme?“

„Warum nicht? Wenn das Wasser oder die Luft nun Bewußtsein besitzt
und ein so feines Orientierungsvermögen, daß es seinen Weg ohne
Instrumente kennt, und wenn es dann sein Bewußtsein in das eines
Menschenhirns umsetzen könnte~–“

„Wenn und wenn! Wenn du nichts Gescheiteres weißt, so schließe
wenigstens deine Waage ab, damit man dir einmal in die
nichtsnutzigen Augen sehen kann.“

„Du sollst sehen, daß ich ganz vernünftig bin,“ sagte Wera, indem
sie den Glasverschluß der Waage zuschob und aufstand. Dann trat sie
dicht an Sohm heran und legte die Arme um seinen Hals.

„Nun sieh mir in die Augen,“ rief sie, „und sage, ob ich verrückt
bin.“

„Die Pupillen sind groß –“

„Weil ich in deine schwarze Seele schaue. Aber die Lippen sind ganz
kühl, nicht wahr? Bist du mir noch böse?“

Er war machtlos. Sie lehnte sich an ihn und flüsterte:

„Weißt du, was ich heute Nacht geträumt habe? Wir waren irgendwo
zusammen, und du – so wie jetzt – und auf einmal war ich ein
kleines, ganz kleines Wölkchen, und mit einem Atemzuge sogst du
mich ein, ohne es zu wissen. Und nun war ich in dir, ganz, in
diesem furchtbar klugen Kopfe, und wußte alle deine Gedanken. Aber
auch du – du kanntest mich nun ganz und verstandest, warum ich an
die Beseelung der Elemente glaube. Und da sagtest du: Da hat die
Wera in ihrer Art doch auch recht, und sie ist wirklich nicht
verrückt. Das will ich ihr doch gleich sagen. Da sahst du dich nach
mir um und suchtest mich überall, aber ich war nicht mehr da, ich
war ja in dir, ganz in dir. Und ich sah, wie du dich um mich
ängstigtest und dich quältest, und deine Qual war mein Qual, und
ich fühlte ich so namenlos elend, daß ich laut aufschrie: Zu spät!
Da wachte ich auf und war so froh, daß es nur ein Traum war.“

„Du mein geliebtes Glück, du bist ja meine gute, verständige Wera!
Aber wenn du immer diesen mystischen Gedanken nachhängst, so darfst
du dich nicht wundern, wenn sie dich schließlich bis in den Traum
verfolgen und ängstigen.“

Sie schüttelte traurig den Kopf. „Meine Gedanken ängstigen mich ja
gar nicht, nur die Sorge, daß ich dich kränken, daß du um
meinetwillen dich beunruhigst. Das ist es, was mich quält. Was der
Traum zeigt, ist sinnlos, aber die Stimmung, aus der er kommt, ist
echt. Und Sorgen wollen wir uns doch nicht machen. Deshalb bitte
ich dich, Paul, vertraue mir, glaube mir, daß ich weiß, was ich
denke, daß meine Gedanken nicht verworren sind, wenn sie dir auch
wunderlich erscheinen, und – ängstige dich nicht um mich und mein
bißchen Verstand!“

„Um Gotteswillen, Wera, was sagst du da!“

„Sei gut. Ich weiß doch, daß es dich beunruhigt, wenn ich solche
Ansichten äußere. Ich will sie ja auch möglichst unterdrücken. Aber
ich kann mich doch nicht selbst verleugnen. Meinen Glauben mußt du
mir schon lassen.“

„Aber Herz, das versteht sich doch von selbst, daß du deine
Freiheit hast, und daß ich sie achte. Glauben magst du ja, was du
willst. Und ich bin auch sicher, daß wir in dem Wege
übereinstimmen, den unsere Arbeit zu nehmen hat; wir wissen beide,
daß es keine andere Erkenntnis gibt, als durch die Erforschung des
Gesetzes.“

„Ja, Paul.“

„Und durch die Mittel der Wissenschaft.“

„Ja, aber diese Mittel können erweitert werden unbeschadet der
wissenschaftlichen Methode.“

„Unbeschadet? Siehst du, Wera, das eben ist die Frage. Das ist die
Stelle, wo wir auseinandergehen. Der Weg, den du im Auge hast, ist
ein Phantasma, ein subjektiver Glaube, ich sage eine Illusion. Und
wer sich solchen Einbildungen hingibt, dem droht eine ungeheure
Gefahr. Der überschreitet die Grenzen der Wissenschaft, der gewöhnt
sich an eine spielende Beschäftigung des Geistes, die ihn verführt,
den langsamen, mühsamen Weg aufzugeben und sein Glück auf seinem
Sprunge ins Leere zu versuchen.“

„Nein Liebster. Was ich im Auge habe, ist nichts anderes, als es
die Erfindung eines neuen Instrumentes wäre, ein neues Mittel, das
wir noch nicht kennen.“

„Eben das ist die Gefahr. Mit diesem Suchen nach unmöglichen
Erfindungen haben schon Unzählige ihr Leben vergeudet und~–“

„Sprich es nur aus“ – rief Wera, sich aus seinen Armen lösend.

„Nein, Wera! Ich will dich nur bitten, spiele nicht mit deinen
Phantasien. Oder ja, spiele, aber eben nur da, wo dieses freie
Spiel sein Recht hat, in der Dichtung. Doch in der Forschung, wo
wir handeln müssen, wo wir nicht Gefühle formen, sondern Schlüsse,
da bringe nichts hinein von dem, was im Reiche des schönen Scheins
seine ewige Wahrheit hat, aber im Reiche der Wahrheit ewiger Schein
bleibt und darum verwerflich ist. Wera! Wenn dir unser gemeinsames
Werk heilig ist, so rühre nicht an dem Grunde, worin wir wurzeln.“

Wera ließ sich auf einen der hölzernen Schemel fallen und stützte
ihre Arme auf den Arbeitstisch, das Gesicht in ihre Hände
vergrabend. Sie hätte es hinausschreien mögen: Aber es ist keine
Phantasie, ich weiß es besser! Ich bin selbst das Instrument, von
dem ich rede. Ich bin das Wasser, das durch die Kalkschicht fließen
kann. Und doch kann ich denken und reden wie die Menschen! Doch sie
mußte sich bezwingen. So saß sie zusammengebrochen am Tische.

Sohm machte einige Schritte durchs Zimmer. Er konnte nicht
verstehen, was Wera so heftig bewegte. Wie konnte sie so
eigensinnig sein? Wie konnte ihr eine solche Marotte so tief gehen?
War er denn unfreundlich gewesen? Hatte er etwas Heftiges gesagt?

Er trat an ihre Seite und legte den Arm um sie. Er versuchte ihren
Kopf aufzurichten. Sie bewegte sich nicht.

„Habe ich dir wehe getan, Liebste?“ fragte er sanft. „Sei doch
nicht traurig. Du weißt doch, daß ich dir nur meine ehrliche
Ansicht sagte, wie du mir. Und ich bin dir dankbar für dein
Vertrauen. Ich habe dir gesagt, warum ich deine Ansicht für
gefährlich halte, – weil sie nämlich bloß auf dem Gefühle beruht.
Hast du ernste Beweise, so teile sie mit. Gründe kann man erwägen,
Gefühle beweisen hier nichts. Kannst du mir etwas Faßbares sagen?“

„Ich will nicht,“ murmelte sie.

Sohm stand ratlos, verzweifelnd. Er durchmaß das Zimmer und trat
wieder neben sie. Er streichelte ihr Haar und küßte es.

„Wera,“ begann er wieder, „es ist doch gar kein Grund, so trostlos
zu sein. Es hat sich doch nichts zwischen uns geändert. Ich wollte
dich nur warnen. Ich glaube ja an dich!“

Da richtete sie sich auf und sah ihn groß an.

„Du glaubst an mich? Und dieser Glaube genügt dir für unser Leben?
Warum dann nicht der meine für unsre Arbeit? Kannst du Gründe
angeben, die beweisen?“

„Ich könnte sagen, daß es sich hier um etwas ganz anderes handelt.
Hier gilt es das Vertrauen zwischen Personen, und das beruht allein
auf dem guten Glauben. Die Liebe ist kein Erkenntnisproblem. Aber
ich kann dir auch Gründe angeben. Ich kenne dich seit Jahren. Ich
weiß, wie sich deine wissenschaftliche Überzeugung gebildet hat,
ich weiß, wie gewissenhaft und umsichtig deine Arbeit ist, ich
kenne ihren Wert und den ehrlichen Ernst, der sie leitet. Diese
Gründe beweisen mir, daß du deines Weges sicher bleiben und die
Lockung der Phantasie überwinden wirst.“

„Nun denn,“ antwortete Wera, indem sie sich erhob, „wenn du diese
Überzeugung hast, dann mußt du auch wissen, daß ich ebenfalls
Gründe für meine Ansicht besitzen werde. Dann mußt du wissen, daß
ich, deine Wera, eine von der deinen so stark abweichende
Anschauung nicht auf einem phantastischen Einfall werde gebaut
haben, nicht auf ein Spiel der Einbildungskraft, und daß sie nicht
eine Ausgeburt des Wahnsinns ist~–“

Sie wehrte seinen Versuch, sich ihr zu nähern, mit einer
hoheitsvollen Bewegung ab. Noch nie hatte er sie so gesehen, in
stolzem Selbstbewußtsein, in heiligem Ernste, ihr Ich gegen das
seine.

„Diese Gründe sind so gewiß,“ fuhr sie fort, „wie ich hier vor dir
stehe. Du aber verlange nicht, sie zu hören. Das mußte ich sagen.
Mehr kann ich nicht. Und nun – ich bitte dich – sorge dich nie
wieder um mich. Ich weiß, was ich tue.“

Beide sahen sich in die Augen. Vergeblich hoffte Sohm auf ein
milderes Wort, auf einen versöhnenden Schluß. Wera schwieg.

Er fühlte sich verletzt. Er konnte nicht begreifen, warum sie ihm
so feierlich begegnete.

„Aber ich glaube ja an dich,“ sagte er endlich befremdet.

„Und ich wollte nur sagen,“ antwortete sie ruhig, „daß an mich
glauben nichts anderes bedeuten darf, als an den Ernst und die
Klarheit meiner Überzeugung zu glauben.“

Er schüttelte den Kopf. Aber ihr Blick wurde so finster, daß er sie
nicht aufs neue erzürnen wollte. Er schwieg.

Nun trat sie langsam auf ihn zu. Sie streckte ihm die Hand entgegen
und sprach:

„Wir wollen uns Lebewohl sagen, Paul. Reise glücklich und hilf dem
Tunnel.“

„Jetzt Lebewohl? Und so? Wir sehen uns doch noch am Nachmittag?
Holst du mich denn nicht ab?“

„Ich glaube nicht. Ich bin müde, ich habe schlecht geschlafen und
du hast noch viel zu tun. Laß mich lieber jetzt allein.“

„Aber heute abend bei Röteleins? Mein Zug geht erst um halb
zwölf.“

„Ich kann es nicht versprechen.“

„Ich werde dich schon noch finden. Und es bleibt dabei, du kommst
nach St.~Florentin?“

„Wir können uns ja schreiben oder telegraphieren.“

„Ich rechne bestimmt auf dich. Also auf Wiedersehen.“

Sie waren inzwischen bis an die Tür gelangt. Er hielt ihre Hand
fest und sah ihr angstvoll in die Augen, denn es war ihm, als wolle
sie sich ihm entziehen.

Da fühlte er sich plötzlich wieder heiß umschlungen.

„Lebewohl“, flüsterte sie noch einmal und riß sich los.

Die Tür hatte sich geschlossen. Sohm stand auf dem Korridor. Er
hatte seine Papiere liegen lassen. Sollte er noch einmal umkehren?
Schon griff er nach der Klinke, da hörte er die Klingel, durch die
Wera den Institutsdiener rief.

Sohm ging weiter. Er wollte später noch einmal nach ihr sehen.

Wera war auf einen Stuhl neben der Tür gesunken. Mit Gewalt hatte
sie sich aufrecht gehalten. Jetzt war ihre Kraft gebrochen. Wie
hatte sie sich zu dieser zärtlichen Hingebung überwinden müssen, um
ihn nicht in seinem Glücke zu kränken, auf das er so volles Recht
hatte. Und wie wenig hatte es doch genügt, ihn über das Recht ihrer
eignen Überzeugung zu beruhigen! War das die Achtung vor der
inneren Klarheit ihres Wesens? Wo war die Hoffnung~–~–?

Da vernahm sie die Schritte des Dieners.

Sie raffte sich zusammen und erhob sich. Sie wußte, was sie
wollte.

„Guten Morgen, Fräulein Doktor.“

„Guten Tag, Herr Walter. Ich wollte Ihnen nur sagen, daß ich jetzt
gehe und in den nächsten Tagen nicht herkommen kann. Sie können die
Sachen dort forträumen. Und dann – Herr Professor Sohm hat diese
Papiere liegen lassen. Sie sind wohl so gut und tragen sie nachher
hinüber. Und einen freundlichen Gruß von mir, bitte.“

„Sehr wohl, Fräulein Doktor.“

Wera verschloß ihren Schrank, setzte den Hut auf und verließ noch
vor dem Diener das Zimmer.

\section{Wieder zur Höhe}

Durch den regnerischen Tag, in engen Kurven an schwindelnden
Abgründen entlang, auf kühnen Viadukten über schäumende Wildwasser,
in langen Tunneln die Berge durchsetzend, wand sich der Schnellzug
zur Höhe und donnerte seinem Ziele entgegen, dem Endpunkte der Bahn
in St.~Florentin. In zwanzig Minuten mußte es erreicht sein; eben
war die letzte Station vorüber. Das graue Licht des herannahenden
Abends mischte sich mit dem Lampenschein in den Wagen.

Allein in ihrem Abteil erster Klasse hatte Wera das Fenster
geöffnet. Kalt sprühte ihr der Nebel entgegen, der Wind preßte ihr
den Schleier gegen das Gesicht. Sie schlug ihn zurück und atmete
mit Wohlbehagen die frische Bergluft ein. Hinauf! hinauf! Bald
wollte sie dort oben bei den Geschwistern sein. Sogleich vom
Bahnhof hinan zum Gletscher! Und dann~–~–

Sie vernahm das Geräusch der Schiebetür, die zum Seitengange des
Wagens führte, und wandte sich um. Ein Herr war eingetreten und
verbeugte sich. Im Augenblick erkannten sich beide.

„Herr Martin?“

Das bleiche Antlitz des Ingenieurs färbte sich leicht.

„Wenn ich störe – gnädiges Fräulein sind allein? Oder~–“

Wera war in Verlegenheit. Was sollte sie sagen? Sie wollte sich ja
gar nicht mehr sehen lassen – und doch – hier war der einzige
Mensch, mit dem sie noch reden konnte~–~–

„Ich will Bekannte besuchen,“ sagte sie. „Mein Bräutigam kommt
morgen nach Schmalbrück, wir werden uns später treffen. Sie werden
sein Telegramm erhalten haben? Herzlichen Dank übrigens für Ihren
Brief.“

Sie reichte ihm die Hand, er nahm Platz.

„Ja,“ antwortete er. „Es ist sehr liebenswürdig, daß der Herr
Professor unsern Antrag annahm. Gerade in diesen Tagen muß sich die
Frage entscheiden. Ich hatte heute Nachmittag hier zu tun, wo ich
eben einstieg, und erwartet in St.~Florentin Nachricht über das
Ergebnis der heutigen Sprengung~–“

Er sprach unsicher. Seine traurigen Augen ruhten auf Wera mit einer
Frage, die er nicht auszusprechen wagte. Sie fühlte es. Nach ein
paar Bemerkungen über den Tunnelbau stockte das Gespräch.

Endlich begann Wera, ohne ihn anzublicken:

„Sie sehen nicht so froh aus, wie – wie ich es Ihnen wünschte.
Macht Ihnen die Arbeit so viel Sorten? Ich denke, es wird doch
alles glücklich ablaufen~–“

Martin schüttelte leise den Kopf. „Sorge um die Arbeit gehört zum
Kampfe, sie greift an, aber sie fördert auch und stählt. Es gibt
ein andres Leid – ach, und Sie kennen es ja selbst, teuerste
Freundin! Verzeihen Sie mir – auch um Sie sorge ich mich, um Ihr
Glück – das ist es, was mich jetzt so – befangen macht. Ich weiß
nicht, ob ich fragen darf, ob Sie es wiedergefunden haben.“

Wera atmete tief. Dann sagte sie leise: „Ich weiß es selbst
nicht.“

Jetzt schlug sie die Augen voll zu ihm auf:

„Die Zeit drängt. Wir werden bald voneinander scheiden. Mein lieber
Freund, wir werden uns wohl nicht mehr wiedersehen. Sie sind der
einzige Mensch, dem ich damals mein Leid verriet, und Ihnen danke
ich großherzigen Rat, den ich befolgt habe. Darum verdienen Sie
Offenheit. Mit aller Macht des Willens bekämpfte ich die Störung,
die mich hier in den Bergen überfiel, noch habe ich sie nicht
überwunden. Noch aber gebe ich es nicht auf, mein Gefühl
wiederzugewinnen. O, wenn es auf mich ankäme! Ich wollte verzichten
auf alles Menschenglück, nichts mehr sehen von den Menschen,
hinausfliehen in die Einsamkeit, in die Berge – dahin treibt es
mich jetzt – und ich kehrte am liebsten nimmer zurück. Aber ich
gehöre ja nicht mir allein. Ich vernichte ja zugleich das Glück,
das Leben eines anderen.“

„Nicht nur des einen,“ murmelte Martin stöhnend.

Wera schlug die Hände vor ihr Gesicht. „Ich weiß es! Ich weiß es!“

„Doch der zweite ist nicht zu retten,“ sagte Martin gefaßt. „Aber
sich selbst müssen Sie retten und damit den einen.“

„Meine Hoffnung ist gering. O dieses Abscheuliche, was Sie Liebe
nennen! Was den Menschen so elend macht und klein!“

„Nicht immer klein, auch groß, am größten!“

„Sie dürfen das sagen. Aber elend sind Sie doch! Ich sah es Ihnen
an, als Sie hereintraten. Was werden Sie tun, mein Freund, sagen
Sie es mir. Ich will es wissen! Was kann ein Mensch tun, dem jede
Hoffnung verloren ist?“

„Fragen Sie um meinetwillen?“

„Ich will es wissen! Sagen Sie nicht wieder, die Arbeit! Nein,
nein, die täuscht die Zeit fort, aber das Elend kommt, und wenn es
nur Sekunden findet, es quält für Jahre. Sagen Sie mir, gibt es für
den Menschen eine Erlösung vom Leide?“

„Das hängt von der Art des Menschen ab, das wissen Sie ja.

„So gibt es Menschen, die verloren sind –“

„Der Mensch kann sterben.“

Wera lachte bitter. „Ha! Sie glauben an den Tod? Ja, wenn es der
Hohe will! Wenn die Natur diesen Leib zerstört, da verfließt der
Schmerz in nichts, der an diesem zeitlichen Zellenbau hängt. Aber
der Wille des Menschen, der seinen Leib bewußt vernichtet, glauben
Sie, daß der das Leid mittrifft, das im ewigen Leide wurzelt? Kann
der freiwillige Tod töten? Dann müßte sich der Mensch auch
freiwillig das Leben geben können. O, Sie wissen nicht, wie das
einzelne Leben im unendlichen Zusammenhange verknüpft ist – der
Schmerz, der aus den unzugänglichen Tiefen des Weltleibes hervor in
das Bewußtsein zuckt, den kann der Eigenwille nicht erreichen –
denn der Weltleib lebt weiter~–“

„Nun denn, wer diesen Glauben hat, was ist dem der Tod, was ist ihm
dieses Leben? Wer teil hat am Weltleibe, der hat auch teil am
Weltwillen. Der wird das Leid aufnehmen willig und groß als seinen
Anteil, der ihm im ewigen Weltprozeß geworden ist, und seinem Gott
gehorchend vertrauen, daß er ein Größeres hinaufführen will, um
dessentwillen dieses Menschen-Ich leiden muß. Wer diesen Glauben
hat, um den mögen Sie nicht sorgen, den mögen Sie beneiden. Er wird
seine Pflicht tun und sagen: Herr, Dein Wille geschehe.“

Wera sah ihn mit großen, leuchtenden Augen an. Dann sprach sie:
„Der sind Sie! Der sind Sie! O, ihr großen, reichen Menschen! Das
ist die Erlösung, die euch gegeben ist. Ja, ihr könnt auf die
Brücke der Erkenntnis treten, denn ihr habt noch eine andere Macht,
die selbst das Leid des Schöpfers um sein Werk bezwingt!“

Sie brach ab, denn sie las in Martins Augen, daß er nicht wußte,
was ihre letzten Worte bedeuten sollten. Sie war nahe daran, ihr
Geheimnis zu verraten. Aber sie mußte schweigen. Ihre Augen füllten
sich mit Tränen, und sie sagte:

„Und dennoch – es muß etwas Schönes sein um das Glück! O, daß Sie
es noch gewinnen möchten!“

Er lächelte wehmütig. „Das ist zunächst mein Wunsch für Sie.“

„Um mich sorgen Sie sich nicht. Ich werde danach streben, aber –
wenn es anders bestimmt sein sollte, es gibt noch ein anderes
Glück, als es die Menschen kennen – und ich~–~–“

Sie hatten gar nicht bemerkt daß der Zug hielt. Die Fahrgäste
drängten sich im Gange. Wera sprang auf und ergriff Martins Hände.

„Leben Sie wohl, mein teurer, lieber Freund!“ rief sie. „Zürnen Sie
mir nicht, daß ich in Ihr Leben getreten bin~–“

Ein Schaffner öffnete die Tür. Sie trat zurück. Martin griff
mechanisch nach ihrem Gepäck.

„Haben Sie weiter nichts?“ fragte er. „Wo wollen Sie wohnen?“

„O fragen Sie nicht,“ bat sie. „Dies kleine Bündel nehme ich
selbst. Lassen Sie mich allein – ich weiß meinen Weg. Leben Sie
wohl!“

Er folgte ihr aus dem Wagen, aber er kam nicht mehr zu Worte. Ein
Beamter trat an ihn heran. Wera hörte noch, daß Martin rief: „Im
Tunnel? Ich komme sofort.“ Er wandte sich Wera zu. Sie verschwand
im Gedränge.

Eilig wand sie sich in der Bahnhofshalle durch das Gewühl der
Reisenden, Bahnbediensteten und Hoteldiener. Sie kümmerte sich
nicht um die Angebote der Träger, nicht um die Verweisung auf die
Gepäckausgabe. Ihr Koffer mochte dort stehen, bis sie ihn brauchte.
Fürs nächste – ah! Sie trat aus dem hellen Vorplatz in den dunklen,
rauschenden Regen. Das war Luft! Das war Heimat! Nun hinein in die
sinkende Nacht! Schnell schritt sie an der Reihe der Hotelwagen
vorüber und bog in einen schmalen Gang ein, der nach dem See
hinabführte.

Sie hatte ihren Weg abkürzen wollen, aber es sollte ihr nicht
sogleich gelingen. Wo der Weg die Fahrstraße wieder kreuzte, hatte
sich zum Unglück ein mit langen Baumstämmen beladener Wagen
festgefahren, und nun stockte dort die ganze Reihe der Hotelwagen,
die vom Bahnhofe kamen. Wera konnte nicht hinüber. Sie stand dicht
an der Straße unter dem Dunkel eines Baumes, bald sammelten sich
noch andere Passanten des Weges. Auf der Straße versuchte ein mit
einem Herrn und zwei Damen besetzter Einspänner sich an der
Wagenburg vorbeizudrängen, wurde aber bald durch den allgemeinen
Unwillen zum Halten gezwungen, unmittelbar vor Wera, die weder
vorwärts noch zurück konnte.

Unter dem Verdeck des Wagens hervor klang eine schneidende Stimme,
die ihr bekannt vorkam, und vom Rücksitz unter einem Regenschirm
erwiderte eine ebenfalls laute Männerstimme. Und jetzt glaubte Wera
ihren eigenen Namen zu vernehmen, so daß sie aufmerkte.

„Hättest du dich nicht solange nach der Lentius umgeguckt, so wären
wir hier noch rechtzeitig vorbeigekommen.“

„Aber sie war's ja gar nicht, wie soll sie jetzt hierhin kommen?“

Das waren Bertile von Okeley und der Alpinist, die sich da stritten
– kein Zweifel. Und jetzt erinnerte sich erst Wera, daß sie vor
einigen Wochen die Verlobungsanzeige von Dr.~Haberdorf erhalten
hatte.

„Sie war's ganz bestimmt,“ klang eine dritte Stimme, in der Wera
diejenige Beatens erkannte. „Ich hab' sie ja vom Gange aus in der
ersten Klasse mit dem Ingenieur allein sitzen sehen.“

„Der ist ja erst auf der letzten Station eingestiegen,“ sagte der
Alpinist.

„Ja, und gerade in das Abteil. Merkwürdig!“

„Das ist doch ein Zufall.“

„Natürlich, du wirst sie auch noch verteidigen! Das war natürlich
auch ein Zufall, damals auf der Bank im Walde, wo sie gesehen
worden sind –~– Und die Person soll verlobt sein!“

„Und wie sie dann plötzlich verschwunden ist!“

„Aber was soll sie denn hier? Was ihr immer redet!“

„Das wird der Martin schon wissen, wenn du's nicht weißt!“

„Bertildchen!“

„Du hast dich doch auch mit ihr kompromittiert! Es war eigentlich
ein Skandal! Nun, sie soll sich nur nicht in Schmalbrück sehen
lassen!“

„O Gott, o Gott! Nun werdet ihr euch noch zanken!“ jammerte Beate.
„Und euertwegen sind wir so schrecklich eingeregnet. Ich war ja
dagegen, daß wir heute die Partie machten. Und nun soll uns noch
die gräßliche Person dazwischen kommen!“

Wera hatte sich gleich bemerkbar machen wollen, aber das Geräusch
des Regens und das Rufen der Fuhrleute hatten ihren Versuch
vereitelt. Dann dachte sie daran, lieber unbemerkt zu verschwinden,
nur sah sie keinen Ausweg. Neben dem Wege zwar rauschte das
Regenwasser gewaltig bergab dem See zu, und es lockte ihre
Aspiraseele nicht wenig, einfach hineinzuspringen und auf dieser
ungewöhnlichen Bahn dem Gedränge zu entfliehen. Aber das wäre doch
aufgefallen, sie wäre erkannt worden und durfte Wera nicht durch
einen solchen Geniestreich unmöglich machen. Länger aber hielt sie
es nicht aus, hier zuzuhören. Sie entschloß sich, in den
Straßenschlamm an die Wagentür heranzutreten, und die Hand darauf
legend sagte sie laut:

„Guten Abend, meine Herrschaften! Kennen Sie mich noch?“

Beate brach in ihrer eben neu angefangenen Rede ab, und die
Insassen des Wagens sahen sich einen Augenblick starr an. Aber
sofort faßte sich Bertilde und rief:

„Ach, Fräulein Lentius! Das ist aber eine reizende Überraschung!
Sie sind wohl auch eingeregnet.“

„Wie Sie sehen. Doch es tut mir nichts.“

„Ich habe mich schrecklich erkältet,“ fiel Berta ein. „Wir waren
drüben in Passurn, wir haben meine Tante, Excellenz Wieden, besucht
– aber wollen Sie nicht in unsern Wagen kommen, liebes Fräulein?
Sie müssen ja klatschnaß sein. Wir rücken ein Stückchen zusammen.“

„Emil, du kannst dich auf den Bock setzen!“ sagte Bertilde. „Nein,
wie ich mich freue, Sie wiederzusehen. Wie entzückend Sie in der
Kapuze aussehen! Natürlich fahren Sie mit uns.“

„Ich danke sehr,“ lehnte Wera ab. „Ich will gar nicht nach
Schmalbrück. Und der Regen tut mir nichts. Ich konnte nur nicht
weiter, weil der Weg gesperrt war. Aber jetzt scheint es ja
vorwärts zu gehen.“

In der Tat kam Bewegung in die Masse, Wera trat an ihren Platz
zurück. Der Wagen rückte an.

„Aber Sie werden uns doch besuchen? Nicht wahr?“ rief Bertilde
zurück.

„Wir haben Sie so vermißt, jeden Tag haben wir von Ihnen
gesprochen. Auf Wiedersehen, auf Wiedersehen!“

Wera sah dem Wagen achselzuckend nach. Sie runzelte die Stirn.

„Das sind auch Menschen!“ murmelte sie. „O, fort! Fort!“

Bald war die Straße passierbar. Wera eilte vorwärts.

Jetzt hatte sie das unmittelbare Bereich der Häuser verlassen, der
Blick ging ins Freie. Aber sie sah nichts als einen Kranz von
Lichtern, der sich allmählich im Dunkel der Regennacht verlor,
während nur hier und da die Glasveranden der großen Hotels
hindurchschimmerten. Bald lag der erleuchtete Fußweg am Seeufer
hinter ihr.

Sie trat in die Finsternis des Waldes. Der Pfad, obwohl hier noch
ein sorgfältig gepflegter Promenadenweg, war nicht zu erkennen.
Dennoch zögerte ihr Fuß keinen Augenblick. Der unfehlbare Raumsinn
Aspiras leitete sie.

Spärlicher wurde der Wald. Der Wind warf den Regen von der Seite
rücksichtslos gegen die unbekannte Nachtwandlerin. Er heulte um die
Felsecken und knarrte und klapperte in den dürren Ästen der
Kiefern. Der Gießbach tobte neben ihr in der Seitenschlucht der
Festina zu, und wilde Sturzwässer benetzten hier und da ihren Fuß.
Immer dichter hüllten die Wolkenmassen sie ein, je höher sie kam.
Ihr Mantel triefte. Sie kümmerte sich nicht darum.

„Nur zu! Nur zu!“ klang es in ihr. „Bald bin ich bei euch im
Wettertanz!“

Nacht, wohin sie blickte, Nacht, Wasser und Wind. Wo waren die
Menschen und ihrer Wohnungen Lichterglanz? Verschwunden drüben im
Dunkel. Doch nein. Vorübergehend teilt sich der Nebel. Da drüben
schimmert noch ein trübes Leuchten von fern – dort liegt die
Arbeitsstätte am Tunneleingang – ein Abschiedsgruß! Verschwunden.
Wieder Nacht und Nebel!

Und es war gut so! Hui! Heulte der Wind – es gibt keine
Naturgeister – ich bin Bewegungsenergie der Luft! – Krrr! prasselte
der Regen – ich fühle nichts, ich bin kondensierter Wasserdampf. –
Weh, Weh! seufzte die Wolke – ich kann nicht denken, ich habe kein
Menschenhirn!

„Sollst es lernen, sollst es lernen,“ rief es in Weras Seele. „Ich
komme, ich komme! Sei gegrüßt, umfassende Nacht, nimm mich in den
bergenden Arm deines Dunkels! Ich wandle gehüllt in den schwarzen
Schleier der Verneinung. Nein! riefen sie mir zu drunten im Tale,
du bist nicht, du lebst nicht. Nein riefe sie, du kannst nichts
beweisen. Nein und immer nein! Und ich stand ohnmächtig und durfte
nicht rufen: „Hier bin ich, Aspira! Seht ihr mich nicht? Und ich
denke, ihr liebt mich?“~ „Wer ist Aspira? Wir kennen nur Wera
Lentius, der Philosophie Doktorin, die schöne, kluge Braut~–~–“

„Ach! Aber ich komme!“

Der Wald liegt unten. Über die Halde fliegt der eilende Fuß. Der
Wind saust vom Rücken, muß treiben, muß heben. „Ja, ich bin Wera,
ich zwinge die Kräfte der Natur. Doch ich bin auch Aspira, ich
verbünde mir ihre Gewalten. Vorwärts! Hinauf! Wie er rasend
geflogen kommt, Bruder Sturm. Erkennst und mich noch nicht? Ich
kenne dich. Kommst vom Wirbel her, den Vetter Drehbold ansaugte
über der Biskayasee. Ich lehne mich an deine starken, weichen
Schultern, und du hebst mich, trägst mich. Ja, ich komme!

Hab's euch gestern hinübergerufen vom milden Abendgarten unter den
Kastanien, doch ich dachte nicht, daß ich so bald kommen würde.
Heute schon, nach vierundzwanzig Stunden. Was sollt' ich noch dort?
Wieder und wieder verhandeln, ohne sich zu verstehen? Gegenseitig
sich kränken? Nein, ich muß Gewißheit finden hier droben bei den
Meinen~–~–

Liebe! Ich suchte sie nicht. Doch da ich sie fand, hab' ich sie
ehrlich gegrüßt mit leuchtendem Menschenauge. Aber sie zerknirscht
mir den Körper, sie verzittert in Schmerz –~– Wußt' ich denn, daß
einer mich schon liebte, ehe ich hinabstieg ins Menschenreich?
Gestern, ach, wie er mich umfaßte! Und heute! Und nun wird er mich
suchen im Arbeitssaal, er wird durch den Draht klingeln und
sprechen, er wird draußen fragen bei den Freunden im Gartenhaus.
Jetzt hat er meine Botschaft, doch ich bin fort. Und nun sitzt er
geängstet im rasenden Wagen und saust über dieselben Schienen, die
mich hergeführt haben. Ach, ich will nicht, daß er leidet! Auch der
andre nicht –~– Ich werde sie wiedersehen. Und dann? Dann – wird
eine Tat geschehen – doch wie? Er soll doch glücklich sein, nicht
wahr, Wera? Zürne deiner Aspira nicht. Gedulde dich noch ein
wenig.

Nun? Ist dir der Atem ausgegangen, mein wackerer Träger? Du läßt
mich fallen? Ach, ich bin über die Höhe. Was leuchtet da droben?
Ein Stern? Verwaschen sieht er aus, doch ein Stern ist's. Es wird
heller, die Luft ist still. Und dort, ein silberner Streifen. Sei
mir gegrüßt, alter Mond!

Sieh doch, die leuchtende Zacke drüben! Bist du es nicht, getreuer
Oheim, Blankhorn, der alte? O~wie herrlich! Das weite, weite
Nebelmeer zu meinen Füßen! Unter mir alles eine einzige, weiße
Decke. Und nur darüber die höchsten Spitzen der Freunde. Ja, ich
komme!

Glück? Ich suchte es nicht, so miss' ich es nicht. Bleib' es
drunten bei euch, ihr glücklichen Menschen! O~daß i es keinem zu
trüben brauchte! Fand ich doch das Glück bei euch, das Wissen und
die Macht – sollt' ich euch nicht danken?

Und ihr, geliebte Geschwister im hohen Äther, ihr weißen Eiszacken,
ihr ragenden Berge, du fliehende Luft, du rieselnder Bach, ich
komme, komme und bringe euch, was ihr noch nicht kennt, ein Glück,
da die Macht ist! Ich komme.

Erkenntnis! Mitwirken sollt ihr am gemeinsamen Glück der Menschen
und Elemente, zu treten auf die Brücke der Erkenntnis. Versöhnen
will ich euch, ihr scheinbaren Feinde. Und das wird auch meine
Versöhnung sein!

Hinab in die weißen Nebel, die den Gletscher verdecken! Kommst du
wieder, fliegender Wind? Ah, du schaffest mir Bahn. Die Nebel
weichen vor meinem niedersteigenden Fuß. Da ist die Moräne – da
liegt der weiße, bläulich schimmernde Riese – im Mondlicht glitzern
die eisigen Glieder. Sieh doch, Neuschnee, wie eine frischweiße
Hülle, den Nachtgast zu empfangen.

Ja, ich komme, bei dir ich zu bergen. Willkommen, zerklüfteter
Kristall – willkommen, Phosphorglanz der Spalte! Ich bin da!

Nun mögt ihr mich kennen! Ich bin da. Aspira ist da. Doch leise,
leise. Ruft's noch nicht hinaus. Nur die Nächsten sollen mich
hören. Da, ihr gespenstigen Nebelstreifen, die ihr noch träge am
Grunde der Spalte hinschleicht, bleibt hier, ihr sollt mich
herausheben. Ich suche mein Lager. Da, wo der ausgewaschene Sand
die Ecke füllt, da ist gut ruhen. Haltet brav Wache, daß mir der
Gletscher die Tür nicht sperrt. Unversehrt muß ich dich finden,
Weras Körper, wenn ich zurückkomme und wieder davon gehen will als
Mensch, du, Wera, oder ich, Aspira.

Einmal darf ich's schon wagen, den Wolkenleib wieder mit dem
Menschenleibe zu vertauschen. Nicht auf lange, dann kehr' ich
zurück. Dann werd' ich wissen, was werden soll.

Eiswand, die du so geheimnisvoll schillerst im verirrten
Mondstrahl, sickerndes Wasser, und ihr, unsichtbar spielende Ionen
der Luft, ich rufe euch, ich banne euch im Namen Migros, des
mächtigen Strahlenkönigs! Wahret den Schatz, den ich euch hier
niederlege, Weras Menschenleib! Und empfanget mich, daß ich euch
versammle und dehne zur alten, freien Wolke Aspira!“

Wera kniete nieder in der Ecke der Nebenspalte. Sie öffnete ihr
Bündel und hüllte sich in die dichte Decke, dann streckte sie sich
auf dem Boden aus. Sie faltete die Hände über der Brust und atmete
langsam. Nur nichts versäumen von den Vorschriften des Vaters,
dachte sie, damit das Wolkenherz nichts zurücklasse von allem, was
es durch Wera gewonnen hat. So lag sie still und regungslos wie ein
schönes Wachsbild mit geschlossenen Augen. Ein matter Reflex des
Mondlichts spielte um ihr Antlitz.

„Ich komme.“ Das sprach sie nicht mehr, sie dachte es nur. Immer
langsamer und schwächer wurden die Atemzüge. Jetzt bewegte sich
nichts mehr. Ein leichter Nebel legte sich über sie und wurde
dichter und dichter.

Und nun hob es sich lautlos empor und stieg in die Höhe und
erreichte den Rand der Spalte.

Und da schoß es heran von allen Seiten, unsichtbare Teilchen, und
wandelte sich zu kleinsten Tröpfchen, und es quoll in die
Gletscherluft hinaus im Mondlicht schimmernd und sich dehnend
weiter und weiter, Aspira, die lebendige Wolke.

Weras Leib aber lag eisig und starr in der Gletschergruft.

\section{Bei Gneis und Kalk}

Höher und höher stieg Aspira, eine einsame Wolke in der lichten
Mondnacht. Denn hier droben war jetzt alles klar. Deutlich
strahlten die Eishäupter der ragenden Gipfel, und die feuchten
Felsenschroffen der mittleren Berghöhen, an den flacheren Stellen
silberbestäubt vom Neuschnee, schimmerten im Widerschein des
Mondes. Nur drunten in den Tälern zogen noch schwere Wolkenmassen
und verhüllten die Wohnungen der Menschen.

Wie wohlig sie sich dehnte in der Freiheit des lange entbehrten
Schwebens! Sie blickte ringsum und freute sich der vertrauten
Höhen. Sie nickte ihrem eigenen Schatten zu, der am Firnfelde des
Blankhorns emporstieg. Sollte sie hinüber und den Alten aus seinem
Schlummer wecken?

Ach nein, sie hatte zunächst Dringenderes zu tun.

O, es war schön, eine freie Wolke zu sein, aber – es war doch alles
anders als zuvor. Es war eine andere Freiheit als die frühere des
spielenden Kindes. All ihre Würde und all ihr Wissen, das sie in
Weras Seele gefunden, hatte sie nun mit hinaufgebracht und, ach,
auch ein Neues war ihr dazu erstanden – das Leid! Eine denkende
Wolke! Sie wußte um sich selbst. Wie sie wuchs und sich bewegte,
begleitete sie ihr eignes Sein mit dem Interesse der Erkenntnis. Es
reizte sie, über alle diese atmosphärischen Veränderungen Studien
zu machen, wie menschliche Wissenschaft sie erheischte.

Und mit dem Menschenwissen war auch der hohe Menschenwille ihr
eigen geblieben. Jenseits von Gut und Böse spielen die Geister der
Elemente, das ist ihre Freiheit. Der Menschen Freiheit aber ist die
Bestimmung über ihre sittliche Aufgabe. Und dieser Aufgabe war sich
nun Aspira bewußt.

Dort lagen die mächtigen Glieder, die der Erdball in den Äther
streckt. Über ihnen wogten ihre Schwestern, die ewig hastenden und
umschaffenden Wolken und Winde. Drunten im Boden schlugen die Pulse
des Erdkörpers. Dazwischen aber arbeitet der Mensch mit einem
höheren Bewußtsein, diese Kreatur zu zwingen und zu wandeln zu
einem Werkzeug der Kultur. Es genügt nicht, daß die blauen Wasser
spielen im Lichtgewog, daß die holden Blumen blühen und duften, daß
die Falter kosen und die millionenfältige Kreatur sich des Daseins
freut und in tausend Schmerzen sich aufzehrt und vernichtet. Dieses
unendliche Leben ist mehr als sein Dasein, es hat eine Aufgabe. Ein
Höheres soll werden. Und das weiß auf diesem Planeten nur der
Mensch. Das weiß jetzt Aspira. Darum trat sie auf die Brücke der
Erkenntnis.

Seitdem zerfloß ihr in nichts das sorglose Spiel des Wolkenseins,
und es wob sich in ihr das Geheimnis des persönlichen Willens, der
Plan. Und mit dem Plan das Gefühl der Verantwortung und Furcht und
Hoffnung um das Gelingen.

Noch war sie nicht wieder erkannt von all den Freunden in Luft und
Boden. Wie sollte sie ihnen begegnen? Wie sich ihnen entdecken,
eine denkende Wolke? Was ihr als Mensch so leicht erschien, das den
Genossen zu sagen, was sie selbst in Weras Seele gelesen hatte, was
sie bei den Menschen so leicht verstanden, wie sollte sie es denen
verständlich machen, die nicht wie sie der Menschen Denkart in sich
aufgenommen hatten? Und riesenschwer erschien ihr das Werk. Doch es
mußte gewagt sein.

Hier streckte der Langberg seinen massigen Leib in die Luft, den
die zerklüfteten Gamssteine krönten. In ihm lag das Rätsel, das sie
zunächst interessierte. Eine deutlich begrenzte Aufgabe verband
diesen Ort mit dem Schicksal der Menschen, die sie kannte. Da mußte
sie beginnen, zu beobachten und auf die Elemente zu achten. Der
alte Langberg mit seinen Kalkschichten und die verborgenen Wasser
darin, die mußten verständigt werden. Die sollten zuerst Vernunft
annehmen!

Aspira erhob sich über die Gamssteine und zog aus der gesamten
Umgebung die Nebeltröpfchen zusammen. Sie wollte als starker
Regenguß herabfallen und in flüssiger Gestalt sich durch das
zerklüftete Gestein verteilen.

Als sie so von oben auf den Berg herabschaute, dessen Vorsprünge
und Schroffen, Schluchten und Wälder im Mondlicht mit
scharfgezeichneten Schatten dalagen, kam ihr eine recht
menschlichen Erinnerung. Sah er nicht ganz aus wie eine riesige
Karikaturzeichnung des Professor Strümpler, des Dekans der
Fakultät, als sie mit ihm wegen ihrer Promotion unterhandelte? Der
große viereckige Kopf, aus dessen Gesicht die Gamssteine als
gewaltige Nase ragten? Der breite Vorsprung nach der
Festinaschlucht mit seinen Einrissen glich den über die Brust
verschränkten Armen, wie er, das Kinn auf die Hand gestützt, mit
äußerster Wichtigkeit dazustehen pflegte. Den Rücken bildete der
Abfall ins Tal von Schmalbrück, und nach unten verjüngte sich die
Figur immermehr, bis sich die dünnen Beinchen in den Nebeln des
Tales verloren.

Und es war ihr, als hörte sie seine etwas gequetschte Stimme
heraufklingen: „Wenn ich allerdings Ihre Auffassung nicht ganz
zurückweisen kann, so muß ich doch immerhin sagen, ja betonen, daß
ich die allgemeinen Gesichtspunkte bei der Aufrechterhaltung der
Bestimmung zwar in Betracht zu ziehen, den Wortlaut der Vorschrift
aufs gewissenhafteste zu wahren aber verpflichtet bin, wiewohl ich
persönlich immerhin nicht abgeneigt wäre, einer Auffassung mich
anzuschließen, die mit der Ihrigen im Prinzip übereinkommt, mich
indessen zu einem gerade entgegengesetzten Schlusse führt.“

Wie kamen diese unendlichen Einschränkungen aus dem Amtszimmer in
ihre Höhe? Oder sprach gar nicht der Dekan Strümpler? War es nicht
der eintönige Aufprall der Tropfen auf das nasse Haupt des
Langbergs? Ja, sie regnete gründlich in den Langberg hinein. Jetzt
war sie im Innern. Das war also die Kalkschicht. Noch nie hatte sie
sich die Zeit genommen, genauere Umschau zu halten. Der nächste
Ausweg ins Freie war ihr der liebste gewesen. Jetzt achtete sie auf
alles, hier als sickerndes Wasser, dort als aufsteigender Dampf,
hier zog sie durch enge Spalten, dort durch ausgewaschene
Höhlräume.

Eine breite Höhle tat sich auf. Von Decke und Boden wuchsen sich
phantastische Tropfsteingebilde entgegen. Alles war von
Feuchtigkeit bedeckt. Am untern Ende der Höhle stürzte das Wasser
in einer Kaskade zur Tiefe. Hier war der Kalk fortgewaschen, der
geglättete Gneis bildete einen festen Untergrund. Allerlei enge
Gänge verloren sich nach oben. Von dort vernahm Aspira jetzt
deutlich eine Stimme. Sie klang weinerlich und etwas keifend
dabei:

„Wenn ich dir's sage, an drei Stellen! So ein hohles stählernes
Ding bohren sie mir in den Leib. Du müßtest es längst gemerkt
haben, sind sie doch auch ein Stück durch dich hindurchgegangen.
Aber natürlich, in deinen alten Knochen spürst du so was gar nicht
mehr. Aber ich! Ich kann mir das nicht gefallen lassen. Ich hab' es
nicht nötig.“

„Aber erlaube, mein liebes Atollchen, diese kleinen Dinger sind
doch gar nicht der Rede wert, sie können dir unmöglich Unbehagen
verursachen. Da hab' ich schon ganz anderes erlitten, wenn man auch
immerhin in Zweifel sein könnte, ob eine dringende Notwendigkeit
für diese Operation vorliegt.“

„Ach was, Notwendigkeit! Ich verlange, daß du dafür sorgst, die
Bohrnadeln aus meinem Rücken fortzuschaffen. Gerade in meine
schönsten Strukturen stechen sie mir hinein. Ich bin doch kein
gewöhnlicher Bergklotz, ich bin organischen Ursprungs, ich bin ein
echtes Korallenriff.“

„Aber erlaube, das stelle ich durchaus nicht in Abrede. Immerhin
bist du durch langjährige Gewöhnung in gewissem Sinne umgestaltet
und dadurch in uns verschoben worden, so daß es schwer sein würde,
für dich eine Ausnahmestellung zu konstruieren~–“

„Ich bin ein Korallenriff und bleibe es –“

„Gewiß, gewiß, obwohl bei deinem etwas zerdrückten Zustande Zweifel
an deiner Struktur auftreten könnten, die mich jedoch nicht
abhalten würden, deiner Beschwerde eine gewisse Berechtigung
zuzugestehen, insofern nämlich~–“

„Insofern nämlich hier überhaupt niemand hineinzustechen hat~–“

„Das könnte noch nach den Bestimmungen der Schichtlagerung erwogen
werden~–“

„Gar nichts ist zu erwägen. Du bist nur rücksichtslos gegen mich
geworden. So was hättest du nicht gesagt, als du noch horizontal
lagertest und die blauen Fluten des Korallenmeeres an deinem Ufer
spielten. O, wie schön war es, als meine süßen Polypchen Röhrchen
auf Röhrchen in ihren feinen Kalkmustern aufsetzten! Wäre nicht
dieser abscheuliche Granit gekommen!“

„Erlaube, mein Atollchen! Das ist noch in keiner Weise sicher
gestellt, ob wirklich der Granit uns in diese schiefe Lage gebracht
hat. Es wäre erst zu erwägen~–“

„Nanu! Fünf Millionen Jahre liegen wir schon so schief, und du
wirst noch erwägen! Kein Sonnenstrahl ist in der ganzen Zeit bis
hier herabgedrungen. Wo sind die bunten Nautileen hin, die mich
umschwammen und meine Schönheit priesen? Ach, sie starben schon,
als die ersten Strömungen unsern lieblichen Meerbusen
erschütterten. Ich armes Korallenriff! Soll ich in meiner
Zurückgezogenheit auch noch durch diese frechen Sticheleien meine
große Vergangenheit erniedrigen lassen?“

„Immerhin könnten ja diese sich einbohrenden Wesen eine neue
Botschaft der Außenwelt sein. Vielleicht sind wir wieder einmal
Meeresboden, und es handelt sich um eine neue Art Bohrmuschel. Dann
wäre allerdings zu erwägen~–“

„Dummes Zeug. Ich habe doch meine Gamssteine oben am Langberg.
Soviel kann ich damit immer noch sehen, daß von Meer keine Rede
ist. Das könntest du doch auch wissen, wozu hast du denn deine
Schichtenköpfe? Hast du noch nicht gemerkt, daß die neuen
Kriechtiere da oben hausen, die sich Menschen nennen?“

„Erlaube, mein liebes Atollchen, diese Felsen und Steine und
Verwitterungen habe ich als unsolide aus meinem innern Bau
ausgeschlossen. Was sie da draußen anfangen, ist ihre Sache. Leider
höre ich, daß sie mit Wolken, Luft, Wasser und sogar den neuen
organischen Individuen kokettieren und Kindereien treiben. Ich habe
mich ganz auf mein Innenleben zurückgezogen. Ich bin einer der
ältesten Gneise, die es gibt, und habe so viel erlebt, daß ich es
kaum noch zu fassen vermag. Was die draußen tun, ist mir ganz
gleichgültig, natürlich, vorausgesetzt, innerhalb der Grenzen, die
durch die allgemeinen Erwägungen immerhin gezogen sind.“

„Wenn du nicht so ein alter Zaudergneis wärest, müßtest du doch
sagen, daß diese stählernen Bohrspitzen etwas ganz Neues sind, die
dich möglicherweise in deiner Ruhe stören wollen. Hast du denn
vergessen, daß sie da unten schon ein großes, tiefes Loch in dich
gebohrt haben?“

„Erlaube, dieses Loch hat, meiner Ansicht nach, mit der Frage gar
nichts zu tun. Ich habe dir das schon öfters erklärt. Nach meinen
sorgfältigen Erwägungen ist dies durchaus eine spontane Bildung
meiner Natur. Indem nämlich aus deinem leicht zerstörbaren
Kalkleibe größere Höhlungen ausgewaschen werden, dringen die
atmosphärischen Gase auch in mein Inneres, und es ist daher sehr
wohltuend, daß sich für diese eine Abzugsöffnung bildet. Die
kräftigen Explosionen, die ich jetzt täglich wahrnehme, sind
offenbar eine Folge der angesammelten Luft, die sich damit Bahn
bricht. Du siehst, meine Erwägungen sind geeignet, mein
harmonisches Weltbild befriedigend abzurunden. Immerhin könnte man
jedoch zweifeln~–“

„Still! Still!“ schrie auf einmal die Kalkschicht in die Rede des
Gneises hinein. „Au, au, au! Was ist denn das? Da zwickt mich auf
einmal etwas ganz unten. Das ist doch noch gar nicht dagewesen. Au!
Und da zieht etwas von unten hinauf, etwas Beißendes. Findest du
nicht, daß es hier höchst unangenehm riecht?“

„Hahaha!“ lachte da eine fremde Stimme. „Kommt nur alle her,
Kinder! Ganz famos hier, da kann man sich doch ordentlich
ausdehnen. Es riecht etwas, meinen Sie? Das tut nichts, das
verliert sich. Guten Abend übrigens, oder guten Morgen meine
Herrschaften! Wir stören doch nicht? Paßt auf Kinder! Da redet der
höchst ehrwürdige Herr Gneis und das reizende, graziöse Kindchen,
die Kalkschicht.“

„Was ist denn das? Wer seid ihr? Was wollt ihr hier? Ich kenne euch
gar nicht.“

„Tut nichts. Man darf doch mitreden? Wir stellen uns vor. Wir sind
nämlich die Sprenggase.“

„Die Sprenggase?“

„Ja, haben endlich das Vergnügen, beim letzten Schusse ein Stück
Kalkschicht angesprengt zu haben. Früher konnten wir durch den
Herrn Gneis nicht hindurch nach oben, mußten immer zum Abzugsloch
hinaus. Aber heute ist's uns geglückt, durch die feine poröse
Gegend zu kommen, und da sind wir. 's ist riesig gemütlich hier.
Dehnt euch aus, Kinder, das tut wohl!“

„Aber was sind Sie denn eigentlich?“

„Eigentlich sind wir ein Kunstprodukt, Dynamit, Sprenggelatine –
d.\,h. das waren wir. Da kamen wir endlich zur Explosion, und nun
sind wir wieder Natur, aber zivilisiert, so zu sagen. Freie
Sprenggase! Freuen uns kräftigst Kollegen zu treffen, mit denen man
einmal reden kann. Hörten schon unten, daß der Herr Gneis mit dem
Tunnel zufrieden ist. Ja, wir haben tüchtig gearbeitet.“

„Aber erlauben Sie,“ sagte der Gneis, „Sie haben~–“

„Ja natürlich, wir. Haben Sie uns nicht knallen hören? Sie sprangen
ja vor Freuden auseinander, daß es eine Lust war.“

„Immerhin wäre doch erst zu erwägen, mit welchem Rechte~–“

„Ach was Recht! Expansion, das ist die Hauptsache.
Volumenentwicklung. Das haben wir weg. Das gibt eine
Geschwindigkeit, der nichts widerstehen kann.“

„Mit welchem Recht, muß ich doch fragen, auf Grund~–“

„Auf Grund der freiwerdenden Energie. Ja, wir sind nun so. Wir
können ja nichts dafür, aber wenn uns der Mensch nun einmal
entzündet~–“

„Der Mensch?“

„Na, das wissen Sie doch? Wir sprengen Sie hier auseinander auf
Ansuchen des Menschen.“

„Ha! Siehst du?“ rief die Kalkschicht. „Sie sprengen uns
auseinander! Ahnte ich es nicht? Mich auseinander! Ein
Korallenriff! Durch mich hindurch, von unten kommen sie! Wissen
Sie, daß das eine Unverschämtheit ist? Da stecken Sie wohl auch in
den Bohrern?“

„Nein, die probieren bloß so ein bißchen, wo wir hinsollen.“

„Entsetzlich! Und das sollen wir uns gefallen lassen?“

„Aber warum nicht? Nach und nach werden Sie eben abgetragen. Dann
kommen Sie wieder an die frische Luft. Hier ist es ohnehin etwas
dumpfig.“

„Das will er Mensch? Mich abtragen? Hilfe, Hilfe! Gneis, du mußt
etwas tun! Der ganze Langberg muß zusammenhalten. Zu Hilfe!“

„Was gibt's? Was gibt's?“ tönte es dumpf vom Wasserfall.

„Was gibt's? Was gibt's?“ pfiff die Luft in der Höhle. „Ich kam von
draußen. Ich wittre fremdes Gesindel.“

„Gesindel? Was? Wir haben mehr Kohlensäure als Sie! Wie haben einen
chemischen Kursus durchgemacht!“ schrieen die Sprenggase.

„Was gibt's?“ zischte es von unten. „Hier sind noch mehr Leute. Ich
bin die Erdwärme. Wenn ihr was braucht, ich will's euch weich
sieden.“

„Ruhe, Ruhe!“ gebot der Tropfstein. „Bitte mich nicht zu stören.“

„Aber sie wollen uns sprengen!“

„Wer?“

„Die Menschen!“

„Das wäre! Man läßt sich viel gefallen, aber schließlich sind wir
auch noch da.“

„Ich komme von draußen,“ pfiff die Luft wieder. „Ich weiß es, das
Blankhorn hat's selbst gesagt. Der Mensch feindet uns an. Er will
uns zersprengen, zerstören, unterjochen!“

„Der Mensch muß hinaus aus dem Berge!“

„Ich will ihn verbrühen,“ zischte die Erdwärme.

„Ich will ihn ersäufen,“ rauschte der Wasserfall.

„Wir zerquetschen den Tunnel!“

„Aber erlauben Sie,“ rief da der Gneis. „Das ist mein Tunnel. Und
da wäre doch erst zu erwägen, ob ich zu der Entschließung kommen
kann, meine Einwilligung zu geben. Immerhin muß ich gestehen, daß
nach der Aussage der Sprenggase ein gewisser Eingriff seitens des
sogenannten Menschen nicht ganz zu leugnen ist, jedoch~–“

„Was jedoch!“ schrie die Kalkschicht. „Das ist gar keine Frage. Das
ist geradezu empörend. Das ist rohe Gewalt! Hinaus mit dem
Menschen!“

Da quoll es aus der Ecke, wo sich Aspira verborgen hielt, als ein
dichter Nebel, der die finstere Höhle mit einem phosphoreszierenden
Schimmer erfüllte.

„Was ist das? Was ist das?“ fragte die Kalkschicht erschrocken.

„Vielleicht sind wir's?“ sagten die Sprenggase.

„Nein, nein!“ pfiff die Luft. „Das ist eine starke elektrische
Ladung. Ich fühl's! Ich werde leitend. Es muß eine echte Wolke hier
sein.“

„Ja,“ sprach Aspira. „Es ist so. Ich bin hier, Aspira.“

„Aspira, Aspira!“ Tönte es von allen Seiten. „Aspira ist wieder
hier, die ein Mensch war.“

„Ein Mensch?“ fragte der Gneis. „Das ist mir neu.“

„Er weiß es nicht! Unglaublich!“ pfiff die Luft.

„Er weiß es nicht!“ sagte die Kalkschicht. „Die Gamssteine haben
doch davon gesprochen! Sie war doch auch an der Silberquelle!“

„Hört mich!“ sagte Aspira.

„Was ist das für eine Geschichte mit Aspira?“ zischelten die
Sprenggase.

„Pst! Pst!“ pfiff die Luft. „Wenn sie nur von den Menschen
erzählte! Das muß eine gefährliche Gesellschaft sein. Mich hätte
beinahe einmal einer eingeatmet.“

„Ruhe!“ gebot der Tropfstein. „König Migros Tochter spricht.“

\section{Widerspenstige Geister}

„Hört auf mich, ihr Geschwister aus Luft und Wasser, aus Feuer und
Erde! Und ihr vornehmlich, Stützten des Langbergs, ehrwürdiger
Gneis, zürnender Korallenkalk, all ihr Geister der Elemente, hört
mich!

Ja, es ist wahr, ein Mensch war ich, noch vor wenigen Stunden.
Gestern noch weilte ich in der großen Stadt unter den Menschen und
dachte und sprach mit ihnen in ihrer Sprache. Und wie kam ich so
schnell hierher? Nicht langsamer, als wenn ich im Wirbel als Wolke
gezogen wäre, nicht die Füße rührte ich und kam doch herauf in die
Höhe unserer Berge, schlummernd auf weichen Kissen trug mich der
Menschen Werk durch unsre Felsen, über unsre Klüfte.“

„Wie? Wie geschah das? Wie können das die Menschen? Haben sie einen
Zauber?“ so flüsterte es rings in der Höhle.

„Ja, sie haben einen Zauber. Damit können sie bestimmen, was werden
soll. Seit Jahrtausenden bereden sie sich und teilen einander mit,
was sie von euern Kräften und Wirkungen wahrnehmen. Und dann
stellen sie es zusammen nach ihrem Willen.

Aus der Erde graben sie die Kohle und das Eisenerz und mischen sie
und glühen sie im Feuer und blasen die Luft hinein, und sie
gewinnen das Eisen und den harten Stahl und geben ihnen die Form,
die sie brauchen. Das Wasser schließen sie in den festen Kessel und
wandeln es in Dampf durch die brennende Kohle. Und der Dampf dehnt
sich aus und schiebt ihre Kolben und dreht damit ihre Räder. Auf
den glatten Eisenstangen, die sie über Tal und Berg legen, wollen
ihre Wagen mit der Eile des Windes. Was zehntausend Menschen nicht
zusammenbrächten mit ihren Zwergenarmen, das tut für sie eure
Riesenkraft, die sie sich leihen. Das ist die Macht der Menschen.“

„Und die unsre, vergessen Sie das nicht,“ riefen die Sprenggase.
„Wir sprengen nicht bloß Felsen, wir können auch Maschinen
treiben.“

„Wenn der Mensch euch leitet.“

„Seid ihr immer noch da, Gesindel?“ pfiff die Luft. „Ihr habt hier
gar nichts zu suchen, ich will euch hinausblasen.“

„Na, na, na,“ klang es von den Sprenggasen, aber schon schwächer,
denn sie hatten sich bereits stark zerstreut. „Wir sind doch auch
Elementargeister.“

„Gar nichts seid ihr! Bildet euch doch nicht ein, lebendige Luft zu
sein. Elende Reste seid ihr von Kraftprodukten. Ein kurzes Weilchen
seid ihr noch munter, weil ihr solange zusammengepreßt waret und
nun die Bewegung in euch gekommen ist. Auflösen könnt ihr euch,
aber nicht wieder zusammenziehen. Versucht's doch. Ich zerstreue
euch, und das Wasser saugt euch ein, und gewesen seid ihr! Nahrung
für neues Leben, weiter nichts. Habe ich nicht recht, Aspira?“
fragte die Luft.

„Du hast recht. Es gibt auch solche vorübergehende Gebilde in
unserm Reiche. Aber ihnen fehlt die Einheit der lebendigen Natur,
sie schaffen nicht am eigenen Leibe. Doch ihr, Elemente, die ihr im
dauernden Wechsel kreist und euch immer wieder zusammenschließt,
ihr haltet die mächtigen Kräfte des Erdballs und leiht sie dem
Menschen. Er braucht sie zu seinen Zwecken und gibt sie dann an
euch zurück, wie die Sprenggase. Ihr nehmt sie wieder in euern
Kreislauf auf, doch ihr wißt nicht, was aus ihnen wird. Ihr lebt
dahin, wie es gerade kommt. Der Mensch aber weiß eure Kräfte zu
ordnen, er sucht sich aus euch heraus, was ihm dienlich ist. Und so
vermag er euch zu zwingen nach seinem Willen. Aus euern
Bestandteilen zog er die Kraft, mit der er jetzt den harten Felsen
zersprengt.“

„Das ist sehr gut, daß du uns das sagst,“ fiel die Kalkschicht ein.
„Jetzt wissen wir erst recht, woran wir mit dem gefährlichen
Menschen sind. Wir kennen seinen Zauber. Es fällt mir gar nicht
ein, ihm meine Bestandteile zu leihen, damit er mich auseinander
sprengt. Da wäre ihm wohl mein feiner Kalk gerade recht. Du weißt
ja, daß ich ein Korallenriff bin.“

„War –“ brummte der Wasserfall.

„Warum willst du ihm den Kalk nicht leihen, den du entbehren
kannst?“ fragte Aspira. „Den Kalk braucht er nicht zum Sprengen,
sondern gerade zum Festhalten und Aufbauen.“

„Das konnte ich mir denken,“ bemerkte der Kalk. „Ich halte mich
gut. Aber eben darum muß ihm das Zerstörungswerk gelegt werden.“

„Jawohl,“ dröhnte der Wasserfall. „Ganz gleich, was er tut. Ich
habe keine Lust ihm zu dienen. Wer weiß, was ich ihm dann
zerklopfen muß.“

„Und hier im Berge hat er gar nichts zu suchen. Hinaus mit ihm!“
sprach der Tropfstein.

Aspira gab die Hoffnung noch nicht auf.

„Aber liebe Freunde,“ begann sie wieder, „ihr kennt den Menschen
nicht. Ihr beurteilt ihn falsch. Ich habe auch so gedacht. Darum
bat ich den Hohen, mich zu den Menschen zu schicken, damit ich
erfahre, wie wir uns gegen sie verhalten sollen.“

„Sag' es uns, sag' es uns!“ klang es im Chorus.

„Nicht bekämpfen sollt ihr ihn, ihr sollt ihm helfen.“

„Was? Wie?“ schrie es von allen Seiten.

„Wenn das nicht Aspira sagte, so pfiffe ich darauf,“ meinte die
Luft.

„Hört mich in Ruhe! Der Mensch will uns ja nicht zerstören, um euch
zu schaden. Er will Nutzen stiften, er will etwas Höheres aus uns
aufbauen.“

„Etwas Höheres?“ schnaufte die Luft. „Etwas Höheres als die Berge
und die Wolken und die Luft gibt es ja gar nicht.“

„Vielleicht eben den Menschen,“ keifte der Kalk. „Natürlich, zum
Aufbauen braucht er mich. Sich selbst will er ausstaffieren mit
meinen feinen Korallen. Aber dazu bin ich nicht zu haben.“

„Ja, ja, wir sollen ihm dienen! Aspira hat's ja gesagt.“

„Nicht ihm allein, sondern euch selbst. Der Mensch ist nicht euer
Feind, und ihr sollt nicht seine Feinde sein. Bundesgenossen sollt
ihr werden, damit ihr ihm an seinem Werke helft und gemeinsam das
Höhere errichtet.“

„Immer das Höhere,“ zischte die Luft. „Was soll das sein? Wir
brauchen nichts Höheres! Wir brauchen nicht den Menschen! Was wir
brauchen, das haben wir, und was wir nicht haben, darauf pfeifen
wir.“

„Aber es gibt ein Höheres, wozu ihr dem Menschen helfen sollt.“

„Dem Menschen? Sollt? Das eben wollen wir nicht. Der habt seinen
Zauber.“

„An diesem Zauber könnt ihr teilnehmen. Ich sage euch, es gibt ein
gemeinsames Ziel, wozu ihr da seid und auch der Mensch. Um es zu
erreichen, bedarf er euer. Darum sollt ihr ihm freiwillig eure
Kräfte leihen und ihn nicht an seiner Arbeit hindern.“

„Das Ziel, das Ziel, höre ich immer,“ dröhnte es unten von der
Erdwärme. „Was ist das, ein Ziel? Was meinst du? Ist es vielleicht
das, was kommen soll? Dann helfe ich nicht. Denn mir ist gesagt vor
namenlosen Zeiten, als ich noch auf der Sonne war, es wird etwas
kommen am Ende, das ist die Kälte.“

„Nein, nein, das meine ich nicht,“ rief Aspira jetzt verzweifelt.
Wie sollte sie diesen Geistern sagen, was sie wollte? „Das Ziel, zu
dem ihr alle helfen sollte, das ist das Gute.“

„Das Gute?“ sagte die Kalkschicht. „Was ist das Gute? Das Gute war
da, als meine Polypen im blauen Meeresgolf ihre Fangärmlein
bewegten.“

„Das Gute?“ dröhnte der Fall. „Das Gute ist, daß immer frisches
Wasser hier herabläuft Dazu brauche ich die Menschen nicht.“

„Das Gute,“ pfiff die Luft, „ist überall. Das brauchen wir nicht,
das haben wir.“

„Das Gute ist die Ruhe,“ bemerkte der Tropfstein. „Alle Achtung vor
unserem hohen Besuche, aber ich bin etwas ermüdet.“

„Was das Gute ist, wie soll ich's euch sagen?“ seufzte Aspira. „Das
Gute, meint ihr, sei das, was ihr seid. Nur der Kalk hat eine
Ahnung, daß es auch etwas andres geben kann, als was gerade ist.
Das kommt wohl daher, weil er von einem Zellenwesen stammt.“

„Das will ich meinen,“ sagte die Kalkschicht geschmeichelt.

„Ihr müßt aber wissen, das Gute ist das, was werden soll. Es ist
das, was die Menschen suchen, die Macht, sich zu verbinden und zu
helfen auf der ganzen Erde, damit alle Wesen Freunde sind. Dann
sind sie gut. Ihr könnt es noch nicht verstehen, weil ihr den
Menschen noch nicht kennt. Aber versucht nur, ihm nicht feindlich
zu sein, so werdet ihr schon merken, was gut ist.“

„So sind also die Menschen gut?“ fragte die Kalkschicht, um ihr
höheres Verständnis zu zeigen. „Ist das ihr Zauber, daß sie gut
sind?“

„Sie wollen es werden. Manche sind es auch.“

„Nun,“ pfiff die Luft, „warum bist du denn nicht bei den Menschen
geblieben?“

„Weil ich auch euch das Gute bringen wollte.“

„Wir brauchen es nicht,“ dröhnte es aus der Erde.

„Ihr wißt nicht, wie es um euch steht,“ klagte Aspira. „Ihr meint,
was ihr tut, das Strömen der Luft, das Rauschen des Wassers, das
Ruhen des Steins, das Glühen des Erdinnern, Das Wogen der Wolken,
das sei weiter nichts, als daß ihr lebt, wie es euch gefällt. Ihr
wißt nicht, daß ihr so sein müßt. Doch es gibt ein Reich, wo es
anders ist, wo die Wesen nach einem Ziele des Lebens streben~–“

„Schweig' von deinem Ziele,“ dröhnte es jetzt stärker von unten.
„Ich habe nun lange genug davon gehört. Wir wollen nichts wissen
von dem andern Reiche. Schwebe zu deinen Menschen, Aspira. Sie
sollen sich an der Sonne wärmen, von mir haben sie nichts zu
erwarten. Kommen sie aber zu mir herunter, so will ich ihnen einen
Brei kochen, der ihnen nicht schmecken wird. Ich kann nämlich auch
wollen.“

„Nimm es der Erdwärme nicht übel, Aspira,“ sagte der Kalk. „Aber
sie ist nun einmal auf die Oberen nicht gut zu sprechen. Sie will
nichts von sich abgeben. Und das muß ich gestehen, für den Menschen
habe ich auch nichts übrig. Er sticht von oben in mich hinein, und
von unten – es ist eine Schande – hat er mich angesprengt. Und wir
wollen es uns nun einmal nicht gefallen lassen.“

„Nein!“ brüllte der Wasserfall.

„Nein!“ zischte die Luft.

„Nein!“ brummte der Tropfstein.

Aspira glühte auf in Zorn und Trauer, daß von den Zacken der Höhle
die Elmsfeuer leuchteten.

„O ihr Toren!“ rief sie. „Wenn ihr mich auch nicht versteht, so
glaubt mir doch, was ich sage. Ich will ja nichts von euch, als daß
ihr den Menschen nicht stören sollt. Laßt ihn in Frieden in seinem
Tunnel arbeiten, was schadet es euch? Tut nur gar nichts, dann ist
es schon gut. Was zu tun ist zu des Menschen Heil, das will ich
selbst besorgen. Ich gleite jetzt hinab – laß mich hindurch,
Kalkschicht.“

„Ich kann dich nicht hindern.“

„Ihr könnt überhaupt nichts hindern. Nützen könntet ihr, aber das
wollt ihr nicht. Den Zauber des Menschen wollt' ich euch bringen,
aber ihr wollt nicht.“

„Nein, nein, nein! Wir wollen nichts von dem Menschen. Aus dem
Berge soll er! Zerdrücken werden wir ihn! Kochen! Vernichten!“ So
rief es im Chorus durcheinander.

„Das werdet ihr nicht!“ drohte Aspira. „König Migro wird es euch
verbieten! Fürchtet seinen Zorn!“

„Hahaha!“ höhnten die Geister durcheinander. „Das werden wir
abwarten. König Migro hat uns noch nie etwas vom Menschen gesagt.“

Aspira bezwang sich. Die Höhle versank wieder im Dunkel.

„Übrigens,“ begann sie noch einmal ruhiger, „ihr seid gar nicht
maßgebend. Ihr seid nicht der Langberg. Hier bestimmt der
ehrwürdige Gneis, und der hat noch nicht gesprochen.“

„Erlaube,“ sagte der Gneis bedächtig, „ich habe mich von der
Außenwelt zurückgezogen, lange bevor von Menschen die Rede war, ich
kann mich daher in diese Angelegenheit nicht mischen.“

„Aber, ich bitte, der Langberg hat mich doch so freundlich
unterstützt, er hat mich vor dem herabrollenden Wagen gerettet, er
hat mir den Weg nach der Silberquelle erleichtert~–“

„Erlaube, davon weiß ich nichts. Indessen können die Felsen und
Wälder an meiner Außenseite selbständig gehandelt haben. Immerhin
wäre zu erwägen, ob die Beteiligung des sogenannten Menschen an der
Öffnung meines Tunnels eine berechtigte ist. Da ich nun der Ansicht
bin, daß dieser Tunnel ein freiwilliges Erzeugnis meiner innern
Natur im Interesse meines Wohlbefindens ist, so liegt von meinem
Standpunkte aus in der fördernden Tätigkeit des Menschen keine
Veranlassung, ihm feindlich gesinnt zu sein.“

„Gneis!“ rief die Kalkschicht.

„Jedoch,“ fuhr der Gneis fort, „kann dieser Gesichtspunkt nicht
allein maßgebend werden. Es ist vielmehr auch die Schädigung meiner
lieben Kalkschicht durch den sogenannten Menschen zu erwägen, die
unter Umständen Veranlassung geben könnte, ihr im Interesse des
gesamten Berges entgegenzutreten. Bei dieser nach beiden Seiten hin
wohl zu erwägenden Gegensätzlichkeit scheint es mir den allgemeinen
Bestimmungen meiner Organisation zu entsprechen, wenn ich, wie
bisher, die Existenz des Menschen als nicht in Betracht zu ziehen
erachte und bei meiner wohlbewährten Neutralität verharre, wobei
ich immerhin nicht abgeneigt bin, mich der Ansicht von Prinzessin
Aspira anzuschließen, aber meinerseits mich den Wünschen des
Korallenkalks nicht entgegenzusetzen denke.“

„Strümpler!“ murmelte Aspira, indem sie sich zusammenzog und mit
dem Wasserfall in die Tiefe stürzte.

Nach allen Richtungen verteilte sie sich durch die feinen Risse der
Kalkschicht, um dem Wege nachzuspüren, auf dem die Sprenggase in
die Höhle gelangt waren. Denn hier mußte eine Verbindung mit dem
Tunnel sein, von der die Gefahr drohte.

Lange irrte sie so durch die ausgedehnten und verschobenen
Schichten. Dann stieß sie auf eine zweite geräumige Aushöhlung, von
der aus sich die Kalkschicht in zwei Hauptteile spaltete. Durch
welchen waren die Gase gekommen? Sie wußte, daß der obere, weniger
geneigte Teil nach der Silberquelle führte, der andere aber tief
hinab in jene heiße und zerdrückte Region, wohin sich Aspira bisher
nicht gewagt hatte.

Sie entschloß sich, zunächst den oberen Teil der Kalkschicht zu
untersuchen. Auch hier galt es, genau zu prüfen. Aber immer
schmaler wurde die poröse, unten wie an den Seiten von
undurchlässigem Gestein eingeschlossene Schicht, ohne daß sich
irgendwo die Möglichkeit gezeigt hätte, daß Wasser oder gar Schlamm
eine Verbindung nach dem Tunnel finden könnten. Zugleich hatte sie
sich überzeugt, daß es keinerlei Anzeichen gab, die auf einen
Zusammenhang dieser Hauptschichten mit der neu entdeckten Quelle
hinter der Silberquelle, von der sie der Ingenieur benachrichtigt
hatte, hinwiesen. So gelangte Aspira nach dem Ausgange am Tobel und
sprudelte als Silberquelle ins Freie.

Aber was war das? Morgendämmerung? Das hatte sie erwartet. Doch
schnell erkannte sie, daß es die Nacht war, die hereinbrach. So
hatte sie nicht nur die Nacht, sondern auch den ganzen Tag im Berge
zugebracht. Es waren vierundzwanzig Stunden seit ihrer Ankunft in
St.~Florentin vergangen.

Wo mochte Paul sein? Was mochte er im Tunnel gefunden haben? Von
hier aus, so viel war wenigstens festgestellt, drohte keine Gefahr.
Aber um so sicherer schien es, daß die Sprenggase durch die tiefe,
zerdrückte Schicht gekommen waren und daß dort ein Einbruch
drohte.

Nachsinnend ruhte Aspira als Nebel in der Festinaschlucht. Der
Himmel war mit Wolken bedeckt, das Wetter war regnerisch und
unfreundlich im Tale wie bei ihrer Ankunft. Sie mußte jetzt in den
Berg zurück, um den tieferen Zweig der Schicht zu erforschen.
Sollte sie wieder durch die Gamssteine als Wasser? Oder sollte sie
von der Silberquelle bergauf? Das konnte sie nur in Dampfform, und
auch das nahm längere Zeit in Anspruch. Doch sie entschloß sich
dazu. Es mochte wohl draußen im Lande schon der Morgen angebrochen
sein, als sie sich wieder in der Höhlung befand, von der sie sich
nun in die unbekannte Tiefe hinabwagte.

Je weiter Aspira abwärts kam, um so höher stieg die Temperatur.
Gewaltige heiße Wassermassen erfüllten die Kalkschicht, die sich
tief unter die Talsohle von Schmalbrück hinabsenkte. Dann bog sie
sich wieder steil nach oben und bildete so eine Art Heber, dessen
Inhalt in der Tiefe erhitzt wurde. Nach unten zu war kein Ausgang.
Eine dünne Tonschieferlage über dem Gneis schloß den großen
Hexenkessel fest ab, dessen größerer Teil eine schlammige Masse
enthielt. Die obere Grenze des aufsteigenden Armes zog sich nach
Aspiras Schätzung ungefähr wieder bis zum Tunnelniveau hinauf. Wenn
die Sprengungen oder die Gewalt der Bergeslasten zwischen diesen
Schlammassen und dem Tunnel eine Verbindung herstellten, dann mußte
der Druck, unter dem sie standen, den Brei in den Tunnel drängen
und diesen verwüsten.

Wo lag der Tunnel, genau genommen? Aspira mußte an die spöttischen
Worte Sohms denken von dem Elementargeist, der sich Barometer,
Kompaß und Geschwindigkeitsmesser mitnehmen sollte. Es war doch
etwas anderes, gedankenlos durch Luft und Erdreich zu ziehen, als
behaftet mit der Aufgabe der Erkenntnis, die Dinge nach Zahl und
Maß festzustellen. Und diese dummen Berggeister!

Während Aspira sorgenvoll nachsann, vernahm sie wieder, und
allmählich immer deutlicher, die Stimmen der unterirdischen
Gewalten, die noch weiter beratschlagten, was sie dem Menschen
antun wollten. Von unten polterte die Erdwärme: „Kochen! Kochen!
Kochen!“

Sollte sei noch einmal versuchen, ihren Einfluß geltend zu machen?

Ungeduldig ließ sie ihre Stimme vernehmen:

„Warte doch einmal noch ein wenig mit dem Kochen! Ich bin hier noch
im Berge, ich kann nicht so schnell durch eure breiigen Kanäle
hindurch. Da könntest du wohl so liebenswürdig sein, dein Kochen
aufzuschieben, bis ich hinaus bin.“

„Bin kein Freund von Rücksichten,“ klang es von unten. „Muß heizen,
heizen, heizen. Wir wollen den Menschen im Tunnel kochen.“

„Aber was hast du denn davon? Du weißt ja gar nicht, ob er darin
ist?“

„Haha, das werden wir wissen.“

„Wie denn? Wie kommst du darauf?“

„Hast es ja selbst gesagt, wie wir es machen müssen. Oben in der
Höhle! Die Luft hat's uns erklärt und die Kalkschicht und die neuen
Sprenggase, die gekommen sind. Wie der Mensch machen wir's. Bereden
uns unter einander. Das ist der Zauber. Das können wir auch. Denkst
wohl, die Wolken sind allein klug?“

„Ja,“ sagte Aspira, „da seid ihr freilich sehr klug. Wie mögt ihr
das nur machen wollen?“

„Haha! Das möchtest du wissen? Kennst du die Schiefklippe draußen?
Die Luft hat uns gesagt, sie hat früher einmal gehört, daß die
Menschen Angst haben, die Klippe könnte herabstürzen. Das soll sie
nun tun. Gerade dahin wird sie fallen, wo es in den Tunnel
hineingeht. Die Sprenggase berichten uns, wann die Menschen darin
sind. Und die Luft sagt es der Schiefklippe. Dann fällt sie herab,
und die Menschen können nicht heraus. Und wir drücken gegen den
Tunnel, und ich koche, koche, koche! Haha! Das hättest du nicht
gedacht?“

„Nein!“ antwortete Aspira innerlich erschauernd. „Was du da
erzählst, das ist ja ein Plan. Und ein Plan hat ein Ziel. Das freut
mich, daß du nun auch für Ziele bist, da wirst du dich schon noch
mit mir vereinigen, daß alles gut wird.“

„Das ist ein Ziel? Ist das gewiß?“

„Freilich.“

„Hm! Hm! Aber kein höheres! Gekocht wird doch.“

„Wenn ich nun aber selbst im Tunnel bin?“

„Gekocht wird doch. Dir kann es ja nichts schaden.“

„Wann soll denn das sein?“

„Das wird wohl nicht mehr lange dauern. Wir werden's schon merken,
wenn die Schiefklippe herabsaust.“

„Aber wenn ich dich bitte –“

„Gekocht wird doch!“

Aspira schwieg. Es war ja völlig vergebens zu verhandeln. Was
sollte sie tun?

Da ein Knall, ein Zittern des Gesteins. Es kam aus dem Innern des
Berges. Man sprengte also weiter im Tunnel. Und es konnte nicht
fern sein. Und jetzt, jetzt merkte sie ganz deutlich den Geruch der
Sprenggase, die bis hierher drangen –~– es mußten demnach Spalten
im Gestein sein~–

Aspira spürte wieder umher. Dort aus jenem Seitengang mußten die
Gase kommen. Ja, da stiegen sogar kleine Blasen auf.

Aspira drängte sich durch die Spalten, die unregelmäßig hier das
einschließende feste Gestein durchsetzten. Sie hatten sich wohl
erst durch die neuen Sprengungen so erweitert, daß auch das Wasser
hindurchkonnte; wenigstens hatte sie bisher nichts davon bemerkt.
Und jetzt, Aspira erschauderte ängstlich – jetzt vernahm sie durch
den Fels den Schlag von Hacken, das Scharren von Schaufeln –~– Da
trifft sie auf eine zweite Spalte, durch die ein Wasserstrahl
quillt, sie fühlt sich mitgerissen und wie ein Springbrunnen
schießt das Wasser in einen weiten Raum, den eine elektrische Lampe
erhellt. Wilde Gesteinstrümmer werfen zackige Schatten in der
grellen Beleuchtung, in der eine Anzahl Männer den Schutt in eine
Karre schaufelt~–

Angstvoll späht Aspira in den Tunnel, worin an der Seitenwand, nahe
am Boden eine lebhafte Quelle entspringt.

Sie sammelt sich im Schutt des Bodens, sie verdampft und hält sich
in der Luft, – sie will sehen, was im Tunnel vorgeht~–~–

Die Arbeiter hatten aufgehört zu schaufeln. Zwei Männer traten
heran und beleuchteten die Wand, sie maßen die Temperatur und die
Wassermenge der Quelle. Aspira kannte sie nicht. Balken, Röhren,
eisernes Gerät wurde herangebracht. Aspira mußte weiter in den
Tunnel hineinschweben, denn an der Wand wurde gearbeitet.

Also drang wirklich Wasser in den Tunnel, warmes Wasser, das wußte
sie wohl. Wenn es nun doch den Gewalten der Tiefe gelang, mit ihren
Schlammassen sich den Eingang zu erzwingen? Wenn der Plan der
Erdgeister zur Ausführung kam?

Martin und Sohm konnte sie nicht erblicken, sie waren nicht hier.
Aber sie konnten jeden Augenblick kommen, und dann –~– Sie
fürchtete, den Donner der stürzenden Felsmassen vom Tunneleingang
her zu vernehmen. Dann waren die Menschen eingeschlossen, dann
hatten die Erdkräfte Zeit, in den Tunnel einzubrechen~–

Was sollte sie tun? Momentan dachte sie daran, so schnell wie
möglich aus dem Tunnel zu eilen, nach dem Gletscher zu stürzen,
wieder Weras Gestalt anzunehmen und Sohm zu warnen – aber das
dauerte viel zu lange, und – was hätte sie auch sagen sollen? Wer
hätte ihr geglaubt?

Und als Wolke konnte sie nicht eingreifen. Ihr Plan den
Berggeistern gegenüber war gescheitert. Ja sie hatte nur Unheil
angerichtet. Was den Menschen zunutzen gereichen sollte, das wurde
ihnen nun zum Verderben. Ihre Lehren waren es, die von den
Berggeistern in ihrer täppischen Manier ausgenutzt wurden. Und sie,
sie war ohnmächtig.

Verzweifelnd zog Wera langsam durch den Tunnel.

\section{Im Tunnel}

Sohm hatte bestimmt erwartet, Wera noch am Abend bei Röteleins zu
treffen, nachdem er nachmittags vergeblich an ihrer Wohnung
angeklingelt hatte. So lebhaft ihn der Gedanke an ihr wunderliches
Lebewohl beschäftigte, fand er doch keine Zeit, ihm lange
nachzuhängen. Es gab noch vielerlei zu ordnen, die Pläne zu
studieren, Fachwerke nachzuschlagen, Mitzunehmendes zu bestimmen.
Er hatte sich nach einem neuen Verfahren Apparate bestellt, um
atmosphärische Luft an den Stellen, wo er ihre Beschaffenheit
untersuchen wollte, aufzufangen, ohne daß die unmittelbare Nähe des
Beobachters auch nur die geringste Verunreinigung verursachen
konnte. Die Einsaugung der Luft in die Glasflasche und der
momentane Abschluß konnten dabei aus einigen Metern Entfernung
bewirkt werden. Diese Apparate wurden ihm nachmittags gebracht.

Anfänglich wollte er die Kiste unbeachtet stehen lassen. Dann fiel
es ihm ein, daß sein Aufenthalt in den Alpen, der sich
möglicherweise länger ausdehnen konnte, die beste Gelegenheit gebe,
die Apparate zu erproben, und daß er so den Zufall, der ihn ins
Gebirge führte, für seine gegenwärtige Untersuchung fruchtbar
machen könne. Er freute sich, die neue Einrichtung Wera zu zeigen,
der er erst eine flüchtige Andeutung darüber gegeben hatte. Also
beschloß er, diese Kiste seinem Gepäck beizufügen.

Es war schon spät geworden, als er, eben im Begriff sich zu
Röteleins zu begeben, ein Telegramm erhielt:

„Ängstige dich nicht um mich. Hatte keine Ruhe, mußte ins Freie.
Ich bleibe einige Tage im Gebirge, möglichst einsam. Nachricht
erhältst du nach Schmalbrück, Hotel Leberecht. Auf Wiedersehen.
Wera.“

Das Telegramm war unterwegs von Wera aufgegeben. Er hoffte nun, um
so eher wieder mit ihr zusammenzutreffen. Die Einsamkeit würde sie
beruhigen.

Am nächsten Morgen gegen neun Uhr traf Sohm in St.~Florentin ein.
Martin und ein Vertreter der Direktion erwarteten ihn am Bahnhof.
Der Ingenieur, der sofort nach seiner Rückkehr am Abend vorher in
den Tunnel gerufen worden war, hatte dort bis spät in die Nacht zu
tun gehabt. Jetzt berichtete er noch in der Bahnhofshalle in seiner
kurzen, sachlichen Weise über den neuesten Befund. Danach war man
bei der letzten Sprengung auf eine warme Quelle geraten, die aber
so schwach war, daß sie keine weitere Beachtung erforderte. Er
erwartete, daß man auf noch mehr Wasser stoßen werde; es seien aber
alle Vorbereitungen getroffen, auch eine stärkere Quelle sofort zu
fassen, so daß er die Arbeit vorläufig weiter fortsetzen lasse.

Sohm wurde nachdenklich.

„Könnten wir nicht sogleich hingehen?“ fragte er.

„Wenn Sie nicht zu ermüdet sind, Herr Professor.“

Sie standen vor der Gepäckausgabe. Sohm warf einen Blick auf die
Koffer, um sich von der Ankunft seines Gepäcks zu überzeugen. Da
fiel ihm ein Koffer auf, den er kannte. Ein deutliches W.\,L. und
der Ortsname Weidburg auf dem Deckel ließ keinen Zweifel aufkommen,
daß er Wera gehörte. Er stutzte.

„Eine Frage, entschuldigen Sie,“ sagte er zu Martin. „Sie kennen ja
meine Braut, Fräulein Lentius. Sie wollte zu Freunden in der Nähe;
nun sehe ich aber dort noch ihren Koffer. Wissen Sie vielleicht, ob
sie noch hier ist?“

„Ihr Fräulein Braut,“ erwiderte Martin, indem er fühlte, daß ihm
das Blut in das Gesicht stieg, „ist gestern abend hier angekommen.
Ich hatte die Ehre, sie zu sprechen. Sie ist zu Fuß fortgegangen
und wird wohl ihr Gepäck noch holen lassen. Da ich sofort in den
Tunnel gerufen wurde, konnte ich ihr leider nicht weiter behilflich
sein und vermag keine Auskunft zu geben.“

„Ich erwarte Nachricht in Hotel Leberecht in Schmalbrück.“

„Da wünschen Sie natürlich zunächst nach Schmalbrück?“

„Ich hätte freilich gern gewußt – jedoch die Arbeit geht vor,
selbstverständlich. Nur – eine Kleinigkeit zu frühstücken müssen
Sie mir erlauben.“

„Wir haben für alles gesorgt, Herr Professor. Der Bahnhofswirt ist
schon angewiesen. Während Sie sich restaurieren, werde ich bei
Leberecht telephonisch anfragen, ob Fräulein Lentius dort ist, oder
Nachricht für Sie.“

„Das ist sehr gut, ich danke Ihnen herzlich. So sparen wir Zeit.“

Als Martin nach einer Viertelstunde zurückkam, hatte er zu
berichten, daß bei Leberecht keine Nachricht für Sohm eingetroffen
sei und daß man von Fräulein Lentius nichts wisse. Es wurde
bestimmt, daß Sohm in St.~Florentin Quartier nehme, weil der Tunnel
jetzt von hier am schnellsten auf der Maschine zu erreichen war.
Auch Martin wohnte deshalb nicht mehr in Schmalbrück, wo ihm die
Tischgesellschaft ohnehin verleidet war.

Sie müssen es sich freilich gefallen lassen, ein wenig gerüttelt zu
werden,“ sagte Martin entschuldigend zu Sohm.

„Kommen Sie, kommen Sie, meine Herren!“

An der Brücke über die Festinaschlucht stieg man aus. Sohm
betrachtete aufmerksam die Gegend. Vor sich, jenseits der Schlucht,
hatte er den steilen Abfall des Langbergs, der die hinter ihm
liegenden Schneeberge verdeckte, talabwärts blickte man nach
St.~Florentin hinab und auf einen Zipfel des schönen Sees.

„Darf ich fragen,“ sagte er zu Martin, „warum Sie nicht gleich
hinter dem Ort die Festina übersetzt haben und am Langberg selbst
hinaufgegangen sind? Sie hätten da dabei diesen kostbaren Viadukt
sparen können. Oder hatten Sie dann nicht Raum genug, um die Höhe
zu gewinnen?“

„Wir hätten es gern getan, die Steigung wäre auch herauszubringen
gewesen. Aber auf diesem kahlen Abhang des Langbergs ist es nicht
geheuer. Wir wären nicht bloß der Lawinengefahr und Steinschlägen
ausgesetzt, die ganze Unterlage ist so unsicher, daß uns die Mauer-
und Schutzarbeiten mehr gekostet hätten als der Viadukt. Wir mußten
hier auf diesem Ufer entlang gehen und zwar bis in den Wald
hinein.“

„Und das ein ganzes Stück, wie ich sehe.“

„Nur bis hinter die Schiefklippe, so heißt der Felsen dort
drüben.“

„Ah, das sieht seltsam aus. Das ist Kalk, offenbar. Sie trauten ihm
wohl nicht? Geht die Schicht tiefer hinein?“

„Nein. Wir haben natürlich aufs genaueste untersucht. Es ist nur
ein stehen gebliebener Rest von ein paar tausend Kubikmeter. Er
liegt aber so gefährlich auf geneigtem Gneise und ist so stark
unterwaschen, daß die Klippe eines schönen Tages herabkommen wird.
Durch die fortgesetzten Sprengungen im Steinbruch ist sie merklich
erschüttert worden. Jetzt brauchen wir glücklicherweise den
Steinbruch nicht mehr. Vor ein paar Wochen habe ich die Seilbahn
abbrechen und den Weg vollständig sperren lassen. Wären wir weiter
unten über das Tal gegangen, wie ja zuerst geplant war, so wäre der
Tunneleingang gerade unter die Schiefklippe gekommen, so daß sie
uns vorkommendenfalls darauf gestürzt wäre.“

„Aber sie kann Ihnen auch so noch den Fluß abdämmen.“

„Das ist nicht zu befürchten. Das Gestein ist so brüchig, daß es
bei einem etwaigen Abgleiten sich in Trümmern über den ganzen
Abhang zerstreuen muß und nicht sehr viel bis in das Flußbett
gelangen würde. Aber natürlich haben wir auch dort Vorsorge
getroffen, daß der Abfluß auf jeden Fall gesichert ist.“

Man überschritt die Brücke und fuhr auf einer Draisine in den
Tunnel bis an die Arbeitsstelle. Der Wasserzufluß hatte zugenommen,
doch war das Wasser durchaus klar und die Menge in keiner Weise
bedenklich.

„Wegen des Wassers,“ erklärte Martin, „habe ich überhaupt keine
Sorge. Es ist ja bei uns nicht wie beim Simplontunnel, wir haben
keinen Fall nach dem toten Ende zu, sondern unser Tunnel steigt
fortwährend, bis wir durch sind. Deswegen brauchen wir gar keine
Pumpen, das Wasser läuft von selbst ab. Die Gefahr liegt nur darin,
daß wir einen Schlammeinbruch bekommen. Aber meine anfänglichen
Besorgnisse sind auch geringer geworden, da unser Versuchsstollen
genügend vorgetrieben ist, um zu zeigen, daß das Gestein weiterhin
wieder ganz fest ist.“

Stunden vergingen mit der sorgfältigen Untersuchung aller
Einzelheiten in der Beschaffenheit der Gesteine und ihrer Lagerung.
Sohm war mit dem Ergebnis sehr zufrieden.

„Soviel sich vom geologischen Standpunkte aus sagen läßt,“ faßte er
seine Ansicht zusammen, „können Sie unbesorgt weiter arbeiten. Es
ist zweifellos, daß eine zerdrückte Stelle, aus der ein
Schlammeinbruch erfolgen könnte, nicht über oder vor Ihnen liegen
kann. Wenn sie vorhanden ist, liegt sie tiefer, und Sie sind
entweder schon darüber hinweg oder sind wenigstens im Begriff,
darüber hinwegzukommen. Es kann sich dann nur darum handeln, daß
durch eine Spalten von unten her breiige Massen heraufgedrückt
würden. Aber wie ich sehe, haben Sie ja für alle Eventualitäten
gesorgt.“

„Das freut mich sehr zu hören,“ sagte Martin. „Wenn wir nicht in
den Brei hineinkommen, sondern der Schlamm nur zu uns hereingepreßt
wird, so wird er kein Glück haben. Sie sehen, daß wir jetzt
vorsichtshalber sofort die eisernen Rahmen einbauen, bis sie durch
Mauerwerk ersetzt werden können. Außerdem können wir durch dieses
Tor im Notfalle jederzeit die Arbeitsstelle absperren. Gegen die
Rahmen kann der ganze Langberg drücken, sie würden nicht nachgeben.
Solche einzelne Spalten können uns also nichts tun. Nur in einen
ganzen Breikessel hinein können wir nicht bauen, dann müßten wir
ausweichen.“

„Nun, in dieser Hinsicht dürfen Sie diesmal der Geologie trauen.“

„Dann, darf ich sagen, können wir uns auf die Technik verlassen.“

Am Nachmittage fand noch eine Besichtigung der später entdeckten
sekundären Quelle statt, die zu keiner Besorgnis Anlaß gab. Abends
konferierte man in St.~Florentin und beschloß, den Bau ohne
Aufenthalt fortzusetzen. Am nächsten Morgen wollte Sohm den
Ingenieur noch einmal bis zur Arbeitsstelle begleiten, um zu sehen,
welche Veränderung etwa infolge der neuen Sprengungen eingetreten
wäre.

An diesem Morgen war es, an dem Aspira mit einer neuen, stärkeren
Quelle in den Tunnel gedrungen war und, geängstet von dem Vorhaben
der Berggeister, das Schlimmste fürchtete, falls Sohm mit den
Ingenieuren im Tunnel erscheinen sollte. Sie hatte ja die Absicht,
hinaus ins Freie zu schweben und bei ihrem Vater in den Höhen des
Äthers Trost und Rat zu holen. Aber ihr Menschenherz hing noch zu
sehr an Menschenwerk und Menschenschicksal, als daß sie sich
entschließen konnte den Tunnel eher zu verlassen, bis sie den
nächsten Erfolg der drohenden Pläne beobachtet hatte.

Jetzt vernahm sie vom Tunneleingang her das Rollen der Draisine.
Bald war der Wagen an der Arbeitsstelle angelangt, wohin auch
Aspira sich wieder zurückgezogen hatte. Martin, Sohm und einige
andere Herren stiegen ab und näherten sich der Stelle, wo Arbeiter,
nachdem die neue Quelle gefaßt war, durch Einsetzung weiterer
Rahmen die Verlängerung des Tunnels sicherten.

In diesem Augenblick fühlte man ein leichtes Zittern des Bodens,
dann kam vom Tunneleingang her ein prasselndes und rollendes
Geräusch, das alle aufhorchen ließ.

„Ist das Donner?“ fragte Sohm.

„Nein, den pflegt man hier kaum zu hören. Es müßte denn direkt vor
dem Tunneleingang eingeschlagen haben. Aber es klang eher wie ein
Einsturz. Ich werde sofort fragen.“

Er ging ein Stück zurück bis an das Telephon, das zum Tunneleingang
führte, Sohm und die andern betrachteten die Arbeitsstätte.

Aspira zitterte in ohnmächtiger Furcht für die Menschen, für ihr
Werk. Sie wußte, was geschehen war – die Schiefklippe war
niedergegangen. Der Eingang verschüttet! Lebendig begraben!
Rettungslos dem schrecklichsten Tode verfallen alle diese Männer,
darunter Martin und er, ach, um dessentwillen sie diese Sorge und
Qual auf sich genommen hatte, dem sie sein Glück wiedergeben wollte
–~– Und was gab sie ihm mit all ihrer klugen Überlegung, mit ihrem
Mute der Erkenntnis? Das Verderben! Sie, ja sie trug im Grunde die
Schuld an dem Komplott der Elemente! Und sie wußte, was nun
geschehen würde. Wohl würden die Menschen an den Eingang eilen, von
innen wie von außen würden sie ihre Schaufeln ansetzen um den
Durchgang zu erzwingen, aber schneller als ihre Arbeit würde die
der Erdwärme sein und der gewaltige Druck der innern Massen.
Zermalmen würden sie diese Gesteine, und hereinquellen wird der
heiße Schlamm, wird den Tunnel erfüllen und die Menschen ersticken
– o~Gott!

Und sie konnte nicht helfen? War sie nicht König Migros mächtige
Tochter? Konnte sie nicht die Wassermassen zwingen, einen Ausgang
zu öffnen? Aber was nutzte dies? Auch die Hilfe wäre dem Menschen
todbringend! O~daß sie nie in Menschenwerke sich eingemischt
hätte!

Kurze Minuten waren es, in denen sich die angstvollen Gedanken in
ihr jagten. Da scholl es von der Arbeitsstelle her wie ein dumpfes
Dröhnen, und gleich darauf rief die Stimme eines der Ingenieure:

„Hier unten am Boden quillt heißer Schlamm hervor!“

„Wo?“

„Unter dem letzten Rahmen drängt er sich heraus. Es muß sich ein
Spalt im Boden gebildet haben. Weiter vor uns ist nichts zu
sehen.“

„Die Abdichtungen her! Gleich den nächsten Rahmen! Wir zwingen
es!“

Die eiserne Platte senkte sich auf den Boden. Kalte Wasserstrahlen
säuberten die Stelle.

Martin kam vom Telephon zurück. Er sprach nichts, er beugte sich
nur herab.

„Es schließt luftdicht,“ sagte er, als er sich aufrichtete. „Hier
kommt nichts mehr durch. Wir können ruhig weiterarbeiten.“

Aspira lauschte erstaunt. War das möglich? Mit welcher Ruhe konnte
er das sagen? Und wußte er denn nicht, daß draußen~–

„Hing diese Pressung mit dem Geräusch draußen zusammen?“ fragte
Sohm.

„Ich weiß es nicht,“ antwortete Martin. „Draußen ist allerdings
etwas passiert.“ Alle horchten auf und Martin fuhr fort, so daß es
alle vernahmen:

„Durchaus nichts Schlimmes! Die Schiefklippe hat das Zeitliche
gesegnet, sie ist den Langberg hinabgestürzt. Aber zum Glück, ohne
irgend einen Schaden anzurichten, außer an etwas Wiese und
Brombeersträuchern. Niemand ist getroffen worden, die Trümmer sind
genau die Bahn gegangen, die wir vorausgesehen haben. Wir wollen
nun hinausfahren und zum Rechten sehen,“ wandte er sich an Sohm und
die Herren, die mit ihm gekommen waren, „hier kann alles ruhig
weitergehen. Ich komme nachmittags wieder.“

„Ich gratuliere Ihnen,“ sagte Sohm. „Der Tunnel ist gerettet, und
das dürfen Sie Ihrer Voraussicht zuschreiben.“

„Das war nur selbstverständliche Arbeit. Glück haben wir aber dabei
gehabt, daß der Schlammkessel in der Tiefe liegt und nicht im
Niveau. Die Beruhigung danken wir Ihnen.“

Der Wagen entfernte sich.

Aspira zog durch den Tunnel, dem Ausgange zu, an den festen
Gewölben entlang. Dumpf vernahm sie im Berge die Stimmen der
Geister, das „Kochen, Kochen, Kochen“ der Erdwärme und das Schelten
der Kalkschicht. „Es gibt nicht nach! Es gibt nicht nach! Wir
können nirgends hinein! Sie haben das Loch ausgepanzert!“~ „Warum
ist auch die Schiefklippe so dumm heruntergestürzt!“Sie hörte die
Luft durch den Tunnel pfeifen: „Nichts zu machen, nichts zu
machen!“

Und sie selbst, mußte sie nicht froh und glücklich sein, daß die
Gefahr abgewendet, daß der täppische Plan der Berggeister an der
klugen Vorsicht der Menschen gescheitert war? Gewiß, eine
entsetzliche Angst, eine quälende Sorge war von ihr genommen.
Gewiß, stolz war sie auf den Sieg der Menschenmacht, auf die
Erkenntnis, die da weiß, was kommen wird. Sie selbst aber fühlte
sich gedemütigt~–~–

Ihre Hilfe war machtlos gewesen – die Menschen hatten ihrer gar
nicht bedurft!

\section{Über der Erde}

Auf dem Berggebiet lag wieder die Nacht. Dichter Nebel verstärkte
die Dunkelheit. Darüber aber in König Migros Reich leuchten die
Sterne in ungestörtem Gleichmaß am tiefschwarzen Himmel, und um die
Wölbung liegt's wie ein schimmernder Schleier; denn ringsum strahlt
das unendliche All.

Aspira war beim Vater.

„Was führt dich herauf in die eisige Nacht, mein liebes Kind, so
früh schon zurück von ersehnter Ausfahrt? Reut dich der Weg in der
Freiheit Reich? Fandest du nicht bei den Menschen das Geheimnis
ihrer Macht? Kränkte dich das Leid um der Menschen Not?“

„O Vater, ich fand! Groß und gewaltig fand ich das Reich der
Notwendigkeit, mutig stand ich auf der Brücke der Erkenntnis.“

„So willst du berichten, was freut und taugt?“

„Ich weiß nicht, Vater, wie mir's gelinge. Ich fand zu viel, und
ich fand zu wenig. Und ich komme um Rat, vielleicht um Hilfe.“

„Sprich deutlicher, Aspira.“

„Mit dem Menschenleibe, den ich gewann, war mir ein unendliches
Glück gegeben. Ich sah den Zusammenhang der Dinge, soweit irgend
Menschenverstand ihn zu durchdringen vermag. Ich sah den Weg, der
die Menschen zur Macht führt. Vergeblich ist der Kampf der
Elemente, der Mensch wird sie bezwingen, muß sie bezwingen, denn er
kämpft für das Ganze der Welt, damit wir alle aufsteigen ins
herrliche Reich des Gesetzes, auf zu jener Freiheit, die sie die
Idee nennen, die Idee des Guten.“

„Das hilft uns nichts, meine Tochter. Für die Kinder der Natur sind
das Worte ohne Sinn. Das kann nur verstehen, wer selbst Teil hat an
einem Menschenleibe.“

„Höre mich weiter, Vater. Weil ich sah, daß die Geister der Berge
und des Wassers, der Luft und der Tiefen den Kampf gegen die
Menschen vergeblich kämpfen; weil ich sah, daß sie dem großen Ziele
besser dienen würden, wenn sie die Feindschaft in Beistand
verwandelten; weil ich meinte, daß sie an der Arbeit der Kultur
teilnehmen können, wenn sie sich ihr nur nicht entgegenstemmen, –
so fand ich mehr, als ich suchte. Ich fand eine neue Aufgabe. Wir
freilich können des Menschen Werk nicht verstehen, wir können ihm
auch nicht helfen. Aber wenn, so wie ich, noch viele von den Wolken
zu den Menschen hinabstiegen, so könnten wir Dolmetscher werden
zwischen beiden Reichen, wir könnten die Unsern belehren und die
Menschenarbeit unterstützen. Das ist das Ziel, das ich mir setzte,
zu vermitteln~–“

„Aspira, das neu gewonnene Licht berauschte dich und blendete dein
scharfes Auge. Was dir aufging als junger Tag in wenigen Stunden,
das fiel dir zu als die glückliche Erbschaft des Menschenleibes, in
den du einzogst; die Menschheit aber hat es erarbeitet in vielen
Jahrtausenden ohne unsere Hilfe, wenn anders du wahr berichtet bist
über ihre Macht. Meinst du, daß sie den erprobten Weg verlassen
wird und dir glauben die neue Kunde? Meinst du, daß meine Geister
dir glauben werden? Sie hoffen, du würdest ihnen ein Mittel
bringen, sich des Menschen zu erwehren. Druck und Gegendruck mögen
sie verstehen. Beistand und Unterstützung sind ihnen
unverständliche Forderungen.“

„O mein Vater, es ist wahr, was du sagst. Ich versuchte es. Die
Menschen verlachen meinen Rat und halten mich für wahnwitzig. Die
Elemente verlachen meine Bitte und nützen meine Worte zum
Gegenteil, soweit sie vermögen. Darum bin ich hier und suche deine
Hilfe.“

„Wie kann ich sie gewähren, Aspira? Den Elementen kann ich wohl
gebieten durch meine Mittel, aber nur in ihrem Treiben
untereinander. Was sie mit den Menschen tun, reicht über die
Grenzen meiner Macht. Ich bin der Pförtner dieses Planeten. Nichts
kommt herein von den Himmelsräumen, nichts geht hinaus, das nicht
durch mein Gebiet strömte und strahlte. Darum konnte ich dir einen
Menschenleib verschaffen. Von da ab warst du auf die Mittel der
Menschen gestellt für Macht und Glück. Die Wolkenseele, die du mit
dir führtest, mochte dir nützen, mochte dir schaden; sie verband
dich mit uns, aber mit den anderen Menschen kann sie uns nicht
verbinden. Du magst erzählen den Wolken von den Wundern der
Menschen. Doch Menschenwillen und Menschenkönnen zu verpflanzen in
das Reich der Elemente über dich selbst hinaus, das vermagst du
nicht, das vermag niemand.“

„In schwere Zweifel werde ich durch dein Wort versetzt, schwerer
noch, als es durch meine Tat schon geschehen ist. Denn ich gestehe
es, ich versuchte dem Menschen zu helfen, und ich erkannte, daß er
meiner nicht bedurfte. Aber vielleicht war dieser Versuch nur nicht
richtig angestellt, vielleicht könnte ein anderer besser gelingen.
Darum wollte ich an meiner Aufgabe noch nicht verzweifeln, darum
hoffte ich auf dein Wort. Denn siehe, Vater, wäre mir nicht eine
solche Aufgabe gestellt, die ein Neues in die Welt bringt, was kein
geborener Mensch vermag, wozu brauchte ich dann ein Mensch zu
werden, deren es so viele gibt? Nur, weil ich darum bat? Weil ich
es wollte? Dann wäre ich's nur geworden um meinetwillen, um für
mich den Stolz zu gewinnen, ein Mensch zu sein mit der Macht der
Erkenntnis. Dann aber, o~Vater, verzeihe mir, daß ich es sage, dann
hast du deine Wahl schlecht getroffen! Dann habt ihr mich in den
falschen Menschenleib gesandt!“

„Was sprichst du da, Aspira? So faßte das Leid dich aber nicht um
der Menschen Not, sondern um deine eigene?“

„Das meinte ich, als ich sagte, ich fand zu wenig! Zu wenig für
mich und für den, dem ich zugehöre.“

„Wie versteh' ich das? Wem gehörst du zu?“

„Ich ward ein Weib, und das gehört nach Menschensitte zu einem
Manne in gegenseitiger Liebe. Wohl fand meine Wolkenseele in allem
sich zurecht, was des einzelnen Menschen Leben ausmacht. Aber
zwischen Mann und Weib ist noch ein Band, ein Gefühl, das mir fremd
blieb, das ich nicht gewann, das ist: Liebe zu geben und Liebe zu
nehmen nach Menschenart. Weil es mir fehlt, raube ich dem Manne
Glück und Vertrauen des Lebens. Damit stürzte ich in den Zwang des
Leides.“

„Ich verstehe dich nicht, Aspira.“

„Ich glaube, daß du das nicht verstehen kannst. Ich aber sage dir,
ich will das haben, was mir mangelt. Ein ganzer Mensch will ich
sein, wenn ich ein Mensch bleiben soll, nicht ein halber, dem die
Gabe fehlt, sich zu ergänzen. Wenn ich nicht alles Menschliche
haben kann, nicht auch das, was die Menschen das Mächtigste nennen
in Wonne und leid, so will ich gar nichts haben. So wandle Wera
zurück in des Menschen Arm, ich aber will vergessen des leidvollen
Trugs – So schwebe Aspira durch die Jahrtausende um die eisigen
Höhen, und hinter ihr fern stürze die Brücke der Erkenntnis!“

„Mein liebes Kind, das sind Geheimnisse, die ich nicht kenne; doch
mich erschüttert deine Klage. Ich weiß keinen Rat als den des
Hohen, der dich begnadete mit dem seltenen Vorrecht der
Königwolken. Ihn magst du suchen, ob er zu dir rede in der Enge
deines Herzens oder in der Ferne seines unendlichen Reiches.“

Da klang aus der unerschöpflichen Nacht die Stimme, die keine
Stimme war:

„Ich vernahm deiner Tochter Klage. Sende sie herauf, Migro, in die
Leere des Raums über dein Erdenreich. Denn nur zwischen den
Sternenwelten, wo die Sonnen versinken wie Atome, redet das
Geheimnis des Unbegreiflichen.“

Aspira flog durch ungemessne Weiten mit einer Geschwindigkeit, die
ihr nicht bekannt war. Denn sie wußte nicht um die Zeit. Bis die
Stimme erklang, die keine Stimme war:

„Blicke um dich, Aspira. Wo ist die Sonne?“

Sie schaute auf. Ringsum strahlten die Sterne. Wohl erkannte sie
noch die alten Sternbilder, doch manches hatte sich verändert. Und
ein Stern schimmerte mit besonders hellem Glanze, den sie noch nie
gesehen hatte. Zögernd sagte sie:

„Die Sonne – sie ist ein Stern unter Sternen geworden. Ich vermute,
daß sie dort drüben glänzt.“

„So ist es. Nun bist du fern genug von der Heimat, um sie mitten
zwischen zahllosen Welten zu sehen, die gleiche zwischen gleichen
und ähnlichen. Hier ist die Stätte, zu blicken auf das Leid des
Schöpfers um sein Werk.“

Aspira schwieg, und der Hohe sprach weiter:

„Die Legende habe ich dir gekündet vom Reiche der Notwendigkeit und
der Freiheit. Du wolltest die Reiche versöhnen, die auf der Brücke
der Erkenntnis zusammenstoßen. Wohl ist es bestimmt, daß das Reich
der Natur ein Mittel werde für die Idee des Guten. Darum eben ist
der Mensch. Ihm ist die Aufgabe gestellt, das Reich der
Notwendigkeit umzuwandeln zum Reiche der Freiheit. Dies aber kann
nur geschehen von der Arbeit des Menschenhirns her. Daran
teilzunehmen ward dir gestattet. Du aber hast deine Sendung
verkannt. Das Reich der Notwendigkeit kann niemals frei handeln aus
sich heraus. Du kannst es nicht belehren, du kannst es nur
erobern.“

„So hab' ich vergeblich nach Erkenntnis gerungen?“ klagte Aspira.
„So war ich nicht würdig auf die Brücke zu treten?“

„Wer sagt das? Du bist jetzt im Reiche der Freiheit. Nicht mehr
nimmst du wahllos das Gegebene hin, wie es kommt. Ein Wille lebt in
dir, zu erstreben, zu verwerfen. Dafür aber kann dir auch niemand
helfen als du selbst. Denn Freiheit ist Selbstverantwortung. Du
allein hast zu entscheiden, was du willst und wollen kannst, du
allein bestimmst, was du hoffen darfst, du allein aber trägst auch
die Folgen des Irrtums. Du bist nicht mehr bloß Geschöpf, du bist
ein Schöpfer und hast nun teil am Leide des Schöpfers um sein Werk.
Das ist es, wo vor ich dich warnte. Du kennst nun dieses Leid. Und
wisse, es heißt: Unvollkommenheit! Die Freiheit fordert stets, die
Notwendigkeit kann nicht immer geben. Daher erfährst du das Leid im
Streben nach Dingen, die unerreichbar sind.“

„Unerreichbar? Unerreichbar?“ stammelte Aspira zitternd.

„Unerreichbar ist nur das Ziel, nicht das Streben danach. Willst du
freiwillig das Leid auf dich nehmen, so bleibt dir das Streben,
wenn du dadurch ein Höheres zu gewinnen glaubst als in der Lust des
Besitzes. Des Menschen Leben ist nicht Erfüllung, sondern Bemühen,
und auch sein Leiden ist ein Mittel zum letzten Ziele. Bei dir aber
steht es, zu wählen, wie dein Weg sich schlinge, durch enge Gärten
mit süßen Früchten oder durch weite Steppen mit ferne leuchtenden
Bildern des Ziels. Und griffst du einmal fehl in deinem Entschluß,
so kannst du einen anderen erfassen. Zahllos sind die Aufgaben, die
den Menschen gestellt sind.“

„Welch eine arme Welt“, begann Aspira klagend, „ist dann dies große
All, wenn der Erde mächtigste Wesen sich begnügen müssen mit dem
Genusse des Unerreichbaren! Warum empören sie sich nicht gegen
diese Trennung der Reiche? Ich aber, Hoher, ich klage, denn das
Unerreichbare ward mir nicht nur hingestellt, wie den Kindern der
Freiheit, als lockendes Ziel meines Ringens! Mir ist auch versagt,
was der Menschen Geringstem sonst als Besitz geschenkt ward, auch
der engen Gärten süße Früchte vermocht' ich nicht zu finden~–~–“

„Sprich nicht weiter, Aspira! Liebe willst du geben nach
Menschenart~–“

„Ja, und ich will es lernen, ich will es erarbeiten – ich will mich
fügen den Gebräuchen und Gewohnheiten der Menschen, ich will mich
unterwerfen fremdartigem Verlangen und geforderten Diensten! Aber
das will ich wissen, wenn ich mich in der Menschen Recht begebe, ob
ich dann auch den Besitz gewinne, nicht für mich, aber für ihn!
Wissen will ich, ob ich das Glück ihm rette, das ihm geraubt ward,
als des Weibes Körper für meine Wolkenseele gewählt ward. Denn
damals war ich noch nicht im Reiche der Freiheit.“

„Still, still, Aspira. Empöre dich nicht – du wolltest ein Mensch
werden, schon mit dieser Bitte standest du auf der Brücke zur
Freiheit. Aber du kamst aus dem Reiche der Notwendigkeit. Und
nimmer in Raum und Zeit läßt sich Notwendigkeit ganz in Freiheit
auflösen. Irgend ein Rest deines Wolkenerbes bleibt bestehen. Keine
Verbindung mit dem Menschen kann dir das letzte geben, was Weib und
Mann im Zellenreiche bindet, denn du kannst die Entwicklung des
Lebendigen nicht beginnen vom Anfang der Erde an. Alles vermagst du
nachahmend zu erringen, nur nicht das Gefühl selbst. Der Menschen
Rechtsbund mag dir äußerlich alles ebnen und die Umgebung täuschen,
in dir und in ihm schafft er nicht, was du suchst. Dies sollst du
wissen, damit du nicht vergeblich hoffst!“

„Du schmetterst mich nieder, du raubst mir die letzte Hoffnung.
Warum verließ ich dann mein Reich, wo ich frei war, um eine
Freiheit einzutauschen, die nichts ist als Unvollkommenheit?“

„Und doch ist diese Freiheit die einzige Macht, die auch
Vollkommenheit gewährt. Nur nicht als Besitz eines Teiles der Welt,
denn das ist ein Widerspruch, aber im Wollen, das sich auf das
Ganze richtet. Denn nur eines gibt es im All, das vollkommen ist,
das ist ein reiner Wille. In deiner Freiheit kannst du ihn dir
schaffen. Erhöhe dein Streben zum Wollen des Gesetzes! Was du
unbewußt tatest als Wolke, tu es bewußt im Reiche der Freiheit. Das
ist die ewig neue Schöpfung neuer Welten!“

„Mir aber scheint es, diese Freiheit bedeutet nur – Entsagung.“

„Entsagung ist nur Verzicht auf das Unerreichbare. Sie ist die
Freiheit des endlichen Geistes, eins zu werden mit dem Unendlichen.
Das ist die Befreiung vom Zwange des einzelnen, das ist die
Schöpfung des Ganzen.“

„O, daß ich keine ganze Welt sein kann!“

„Du bist es! Im Reiche der Freiheit ist jeder eine ganze Welt, weil
er das Gesetz des Ganzen sich selbst gibt. Blicke um dich! Diese
zahllosen Sonnen, die Planeten, die sie umkreisen, sind lebende
Wesen, auf ihnen wohnen lebende Wesen, sind Arten des Gefühls und
des Bewußtseins, von denen kein Mensch sich eine Vorstellung machen
kann. Aber soweit sie dem Reiche der Freiheit angehören, sind sie
alle Welten für sich und bestimmen sich nach ihrer Verantwortung.
Sie alle sind begriffen, sich ineinander zu spiegeln und damit
Welten zu formen zur eigenen Welt. Und du magst auf den Strahlen
des Lichtes reisen durch Millionen und Abermillionen von Jahren,
immer wieder wirst du auf Gruppen leuchtender Milchstraßen treffen,
wie du sie hier schimmern siehst. Und doch ist das nicht das Ganze.
Und dennoch ist das Ganze überall in jedem. Du staunst? Ich habe
dir die Legende gekündet. Nun magst du das Geheimnis erblicken im
Schema.

Du weißt, das Menschenhirn setzt sich zusammen aus zahllosen
Zellen, in denen Molekeln sich aufbauen und umschwingen und
zerfallen. Und dieser Prozeß erlebt sich als ein Bewußtsein. Du
weißt, dort in den Räumen, die dein Auge nicht mehr durchdringt,
bauen Sternsysteme sich auf und schwingen um und zerfallen. Auch
dieser Prozeß erlebt sich als ein Bewußtsein. Was sind Sonnen, was
sind Atome? Mittel der Freiheit!

Sieh her! Ich will für deinen Blick diese Weltsysteme und
Milchstraßen zusammenziehen in viel trillionenfacher Verkleinerung,
daß du sie hier von außen anschaust, so wie das System der
schwingenden Atome einem Menschenauge erscheint. Was erblickst
du?“

„O Hoher, ich schaudere. Das ist ein lebendes, zuckendes Organ, das
ist ein Gehirn!“

„Und wieder will ich deinen Blick schärfen und will dich versetzen
in das Gehirn des schlummernden Mädchens dort in der
Gletschergruft, daß du es von innen erschaust in viel
trillionenfacher Vergrößerung. Was erblickst du?“

„Gewaltiger, wie wag' ich es zu sagen? Und sehe den Sternenhimmel
über mir, ich sehe Sonnen kreisen und Milchstraßen schimmern.“

„Und doch ist es nur ein Menschenhirn. Und jene Sternmassen, die du
im Raume leuchten siehst, sind selbst nichts anderes als ein Organ,
das Hirn eines höheren Wesens, deren es unzählige gibt. Überall
formen sich die Gebilde, in denen Vernunft sich ihren Leib und ihre
Mittel schafft. Überall sind Welten, alle dienen einander. Willst
du noch klagen, daß es Grenzen deines Besitzes gibt?“

„Was mögen jene Überwelten sinnen? Was bin ich gegen sie? O~du
machst mich so klein, so klein~–“

„Nein Aspira. Ich mache dich so groß, so groß! Auch dein Hirn ist
ein Weltsystem. Gleich stehst du den Wirbeln der Sonnenmächte. Jene
Sonnenwesen sind nur reicher, nicht freier. Die Freiheit, die im
reinen Willen sich offenbart, hat keine Steigerung, keine
Einschränkung. Sie ist das Vollkommene und sie ist dein. Wandelbar
und begrenzt ist nur das Glück.

Und nun lebe wohl, Aspira. Ich wende deinen Flug wieder sonnenwärts
und rufe den Vater, daß er dich zurücknehme in das Bereich der
Erde.

Du aber wähle! Prüfe dich, was du verantworten kannst. Danach
bleibe ein Mensch, oder kehre zurück ins Spiel der Elemente!

Lebe wohl.“

\section{Abgeschnitten}

Durch den Raum flog Aspira auf Schwingen der Weltenstrahlung, und
als sie in den Schatten der Erde gelangte, fing sie der Vater auf
in seinem Arm und zog sie zurück in das Blau seines Erdenhimmels.

Auf der langen, langen Fahrt aber, allein ihren Gedanken
überlassen, kämpfte sie in ihrer Seele den harten Streit des
Zweifels.

Darfst du es wollen, ein Mensch zu bleiben?

Die Elemente kann man nicht bereden, die Menschen kannst du nicht
überzeugen, und deiner Hilfe bedürfen sie nicht in ihrer Klugheit.
Aber mußt du denn gerade dieses wollen? Blüht dir nicht anderes an
fruchtender Arbeit? Unendlich sind die Aufgaben der strebenden
Menschen.

Köstlich ist das Bewußtsein der Freiheit, herrlich ist die Macht
der Erkenntnis. Wohlbekannt ist dir des Menschenwissens reiches
Gebiet und gegeben sind die dir Mittel, es zu erweitern in ernster
Arbeit. Und wenn du die Elemente nicht bestimmen darfst als
dienende Geister, so kennst du doch der Tiefen Geheimnis und manche
Lagerstätte unerschöpflicher Schätze, um den Reichtum zu gewinnen,
den sonst kein Sterblicher besitzt. Und du stillst die Träne des
Elends und du schaffst gewaltiges Werk, wie es noch keiner
vermochte –~– Lockt dich nicht die Fülle der Macht?

Aber darf ich das? Für Menschenwerk die Mittel entnehmen aus dem
Reiche der Natur, die nicht gewonnen sind durch die sinnvolle
Arbeit, die ich nur kenne aus meiner Wolkenseele? Bringe ich nicht
auch so in das Menschenwerk eine unlautere Gabe, die der
Geisterwelt entstammt? Wird nicht auch sie zerrinnen wie
Spielgewinst, da sie vom Reiche des Spiels entwendet ist? Wer sagt
mir, ob nicht gerade solches Tun den Weg stört, der den Menschen
doch nur durch ihres Denkens Macht auf ihre Weise gestattet ist?
Würde ich solchen Erfolges mich rühmen können, würde ich ein Mensch
sein, der auf seiner Arbeit steht? Und wieder nur ein Mensch sein,
wie die ehrlichen Forscher alle, könnte mich das befriedigen? Wozu
noch der eine zu den vielen? Soll ich darum das Wort der Freiheit
erlernen, das quälend erlösende: Lerne entsagen?

Warum nicht lieber zurückfliehen in das ungetrübte Reich der
Wolken, wo keine Verantwortung mich ruhelos umhertreibt? Wo das
sorglose Spiel des Elements mich in ewigem Wandel durch Höhen und
Tiefen holde Freuden genießen läßt? Leuchtet mir nicht auch dort im
Herzen die wärmende Sonne der Ehrfurcht vor den hohen Gewalten der
lebendigen Natur? Süßes Vergessen aller erlebten Not, ich grüße
dich, meine befreiende Hoffnung!

Und wenn ich trotzdem das Leid auf mich nähme? Wäre das nicht
größer? Wäre das nicht stolzer, Aspira? Ein Mensch sein in seiner
freien Würde, unbekümmert um des Glückes Blütenkranz, der meine
Denkerstirne flieht? Lerne entsagen!

Doch nein, nein! Ich darf es nicht! \emph{Meinem} Glücke entsagen,
ja! Mein Leid wollt' ich ertragen! Aber ich bin ja nicht allein! Es
ist nicht nur mein Leid. Das Glück rauben, das dem andern gehört –
ein andres Menschenleben zerstören, um das meine dafür zu setzen –
darf ich das? Da liegt mein Verhängnis! Liebe zu nehmen und Liebe
zu geben nach Menschenart, das ist mir versagt! Das kündete mir der
Hohe. Und damit vernichte ich ihn, den ich lieben soll! Darf ich
das wollen? Ist das die Aufnahme des Gesetzes in meinen Willen?
Mich gegen ihn! O, das wäre nicht das Vollkommene, das einzige, was
es gibt, der reine Wille. Das wäre nicht gut, das kann ich nicht
wollen – nein! Nein!

So strömte sie zurück zum Vater und ruhte aufgelöst am klaren Azur
des Erdenhimmels.

„Nimm mich zurück, Vater, nimm mich zurück in das weite Reich
deines leuchtenden Gewölbes! Ich will bei dir bleiben, Aspira, die
Wolke!“

„O meine Tochter, so konnte der Hohe dein Leid nicht stillen? Zu
schwer ist es, was das Menschenherz dir kränkt!“

„Nicht so, mein Vater. Nicht was ich leiden würde, ist mir zu
schwer, aber was andre leiden würden durch mich, das kann ich nicht
wollen. Der Hohe belehrte mich, daß jene Hoffnung vergeblich ist,
die ich hegte, einem Menschen das wieder zu geben, was ich ihm
raubte, als Weras Menschenleib mein eigen wurde. Darum will ich den
Menschenleib zurückgeben. Behalte mich bei dir!“

„Besinne dich!“

„Frei will ich sein im Spiele des Traumes, nicht im Ernste der
Würde. Als Mensch bin ich zu schwach zur Vollkommenheit des Guten,
aber als Wolke gut zu sein, genügt, daß ich bin. Behalte mich bei
dir!“

„Du bist willkommen, die Rückkehr steht dir offen. Aber in deinem
Wolkenherzen, das sich als verbindende Einheit durch deinen
Nebelkörper verbreitet, ist ja der Menschenseele Grundkraft mit
verhaftet. Die mußt du zurückgeben, willst du frei werden im
Wolkenreich. Fließe hinab in die Gletscherspalte, ziehe ein in den
Menschenleib, damit alles sich binde und löse nach den Gesetzen der
Natur. Dann ströme wieder heraus mit dem Wolkenherzen allein. So
wird der Mensch aus der Erstarrung aufstehen und wandeln wie zuvor,
ehe du in ihn einzogst. Du aber wirst rein sein von allem, was
Menschensinn umnachtet.“

„Ich will es tun.“

„Aber hüte dich, Aspira. Ganz und ungeteilt mußt du in den
Menschenleib zurückführen, was du ihm einst entzogen hast. Nichts
darf draußen bleiben von dem Wolkenorgan, das jetzt deines
Menschenbewußtseins Träger ist. Noch bist du Mensch durch diese
Seelenmischung, wie du des Denkens und Redens der Menschen kundig
bist, so bist du auch menschlichen Neigungen und Lockungen
unterworfen. Sorge, daß du alles zurückzugeben vermagst, sonst
bleibt der Mensch erstarrt liegen für immer in der Eisgruft, dir
aber, der Wolke, bleibt Kummer und Leid trotz deines Nebelleibes,
und unselig bist du im sonnigen Reich der Lüfte.“

„Ich höre, Vater, und schwebe hinab.“

„Und weißt du wohl, wie lange du fern warst droben beim Hohen im
weiten Äther?“

„Ich weiß es nicht, mir war keine Zeit bestellt.“

„Zum zweiten Tage, seit die Schiefklippe stürzte, stieg die Sonne
empor. – So schwebe hinab, gedenke der Warnung, und sei dann wieder
willkommen im Reiche der Luft!“

Aspira senkte sich hinab zu den Bergen. Unruhiger ward die Luft. Im
Wirbel vorüber stürmte die Bö, und die Geschwister warfen
Schneekristalle hinab aufs verdeckte Land. Dann wieder kam auf
kurze Zeit leuchtend die Sonne hervor und glänzte auf den
blitzenden Schneehäuptern und grünenden Matten. Dem weißen
Blankhorn winkte Aspira einen freudigen Gruß:

„Bald komme und zu dir, bald schmieg' ich mich um deine treuen
Schultern. Warte nur, warte!“

Und sie freute sich des freien Schwebens und fühlte sich erlöst in
ihrem Entschluß. Übermütig stürzte sie sich in das Treiben der
Wolken und blickte flüchtig hinab auf Wald und Flur. Doch da
leuchteten die Häuser von Schmalbrück herüber, da streckte sich die
Spur der gestürzten Schiefklippe, da dampfte drüben von
St.~Florentin ein Zug heran – da waren Menschen –~– Und sie? Zwei
Tage war sie fort vom Menschenreiche, was mochte inzwischen
geschehen sein? Da lag der Dekan Strümpler – nein, der Langberg,
der den Tunnel zerdrücken wollte – doch Martin war stärker gewesen
– und sie konnte sich noch freuen? Sie möchte ihn noch einmal
wiedersehen. Und Sohm, wo war Paul? Seit vier Tagen hatte er keine
Nachricht von ihr, wie mochte er sich um sie Sorge machen!

Drum schnell, schnell! Fort mit Menschenwerk und Menschenhoffnung!
Sie wollte scheiden von ihnen. Statt Aspira wollte sie Wera
schicken zu Paul. Emporsteigen sollst du, Wera Lentius, mit dem
warmen Menschenherzen aus deiner Eiskluft, sollst den Geliebten
umarmen, der sich nach dir sehnt und um dich ängstigt. O~wie selig
wird er sein, daß er dich wiederfindet, wie du vordem warst, wie
anders werden deine Küsse glühen als der Wolkenlippen kühle
Berührung, und wie ein unverständlicher Traum wird dir's manchmal
in der Seele klingen, daß du glaubtest, eine Wolke zu sein~–~–

Da durchschauerte sie ein Schreck. Wenn Weras Leib entdeckt wäre in
der Gletscherspalte, wenn ein umherschweifender Hirtenbub, ein
verirrter Tourist zufällig – o~Gott! Wenn Paul hörte, glaubte, daß
sie tot läge in der Spalte – gewiß hat er nach ihr suchen lassen –
Was die Menschen für Rettung hielten, wäre das Verderben!

Schneller, schneller hinab! Da ist der Gletscher. Und dort, nahe am
Firnfeld, am obern Rande des Gletschers, wo der schwarze Firnblock
hervorragt, dort sind zwei Männer. Ein bißchen Platz, ihr Nebel,
hellt euch auf, daß ich sie sehen kann! Näher heran! Der eine ist
ein Träger, er hat einen Apparat ausgepackt und stellt ihn auf, der
andre hilft ihm – jetzt richtet er sich auf – es ist Sohm!

„Paul!“ möchte sie rufen. Ach, es ist noch soviel Menschliches in
ihr.

Was tut er da? Nun umhüllt sie ihn ganz in leichtem, fast
durchsichtigem Nebel. Diese Glasflaschen mit den Hähnen und
Schläuchen, diese Form der Pumpe hat sie ja noch gar nicht bei ihm
gesehen. Was hat das mit dem geologischen Gutachten zu tun? Sie
kann es sich nicht erklären – aber das muß sie doch beobachten.
Jetzt ist die Eile wohl nicht so dringend, er lebt ja, er arbeitet
ruhig, wie immer, vorsichtig, sorgfältig. Ihr ganzes Wolkenherz,
jetzt das Organ ihrer menschlichen Wahrnehmungsfähigkeit, zieht
sich um den Beobachtenden und seinen Apparat zusammen. Sollte das
die neue Saugpumpe sein, von der er sprach? Jetzt entfernt sich
Sohm vom Apparat~–

Was ist das? Woher diese plötzliche Strömung, die ihr fremd ist?
Wohin wird sie gerissen –~– hier ist kein Ausgang –~– was soll's?

„Aspira ist wieder da,“ sauste ein Windstoß am Blankhorn. „Habt
ihr's schon gehört? Aspira kommt zurück von den Menschen und will
bei uns bleiben.“

„Aspira ist wieder da?“ rief das Rinnsal, das am Felsen munter
hinabschoß. Der Neuschnee hatte es frisch gekräftigt. „O, nun werde
ich sie auch kennen lernen.“

„Das ist doch eine alte Geschichte,“ brummte der herabgleitende
Nebel. „Habe sie schon vorhin am Firnfeld gesehen.“

„Was gibt's schon wieder?“ murrte der Felsen, der ein wenig
geschlummert hatte. „Was ist los?“

„Aspira ist wieder da!“

„Wo, wo? Ich sehe sie nicht. Es ist zuviel Wolkenzeug hier herum.“

„Sie will bald zum Blankhorn kommen,“ sprach der Nebel. „Im
Vorbeiwehen hat sie's schon gegrüßt. Sie bleibt jetzt hier und will
nicht wieder Mensch werden.“

„Aha!“ sagten die Flechten befriedigt. „Aspira ist wieder da! Wir
haben es ja gleich gesagt, daß es Unsinn war, was sie wollte. Sie
paßt nicht zu den Menschen. Das sind Zellenwesen. So was werden sie
hier oben nicht richtig verstehen lernen.“

„Ruhe bei Ihnen!“ knurrte der Fels. „Was wissen Sie von ihren
Angelegenheiten? Die Menschen sollten ihr auch gar nicht gefallen.
Das ist uns gerade recht. Nun werden wir bald mit den Zweibeinern
aufräumen.“

„Wie werden wir denn das machen?“ fragte das Rinnsal.

„Nun,“ sauste der Wind, „das könnt ihr ja so machen, wie drüben der
Langberg, der ihnen einfach die Schiefklippe auf die Köpfe gestürzt
hat.“

„Wer hat dir das vorgeredet?“ rief der Nebel. „Ich hab's selbst
gesehen, daß nicht einmal ein Eichhörnchen dabei verunglückt ist.“

„Na ja, der Langberg!“ rief der Fels. „Der ist auch dumm genug.
Würde mir gar nicht einfallen, mich hier hinabrollen zu lassen.
Aspira wird uns schon was Besseres sagen.“

„Glaubt's ja nicht, was sie euch sagt. Tut's ja nicht, was sie euch
rät!“ klang es von unten her.

„Was ist denn das? Wer mischt sich hier in unser Gespräch? Das paßt
uns nicht!“ riefen die Flechten.

„Pfui!“brummte der Fels. „Diesmal haben die Flechten recht. Das
paßt uns nicht. Das kommt drüben vom Langberg. Das ist die Luft,
die so fatal riecht.“

„Pfui!“ sagte das Rinnsal. „Blast sie doch weg!“

„Haha, sie ist schon fort!“ dröhnte der Fels. „Wie sie Angst hat,
die freche Langbergluft!“

„Seht, seht, wer da kommt! Mach besser Platz, Nebel!“

„Aspira! Aspira! Willkommen Aspira!“ hallte es von den Felswänden
und den Wasserstürzen und dem Neuschnee rings umher.

Und von den Eishöhen des Blankhorn rief es vernehmlich herab:

„Willkommen, Aspira! Nun, nun! Glimmer und Schwefelkies! Was haben
die dir Zweibeiner getan, daß du schon wieder hier bist? Komm
herauf und erzähle mir. Du siehst, wie wir alle uns freuen!“

Vom Firnfeld her hatte sich's zusammengeballt und wuchs herauf,
eine wallende Wolke. Es legte sich um die Abstürze des Blankhorns
und zog höher und höher am Berge hinauf. Es war Aspira.

Und sie schmiegte sich an den alten Freund, den gewaltigen Riesen
der Berge.

All die kleinen Geister aber lauschten aufmerksam, ob sie nichts
vernehmen könnten; denn sie meinten, Aspira werde nun dem Blankhorn
von den Menschen erzählen.

„Nun, nun?“ sagte das Blankhorn. „Was ist denn das, Aspira? Siehst
ja auf einmal so bestürzt aus? Und warst doch vorhin so fröhlich,
daß du wieder bei uns bist? Denke dir, den ganzen Sommer haben sie
wieder auf mir herumgehackt. Nun werden wir aber von dir hören, wie
wir die Nagelfüßler herunterbringen. Werdens gescheiter machen als
der Langberg. Nicht wahr? Das ist recht, daß du wieder bei uns
bleiben willst.“

„Ja,“ seufzte Aspira. „Vorhin glaubte ich, daß ich sogleich zu dir
zurückkommen würde als freie Wolke. O, ich war glücklich! Alles
Leides hatt' ich vergessen, das drunten bei den Menschen wohnt,
wiedergeben wollt' ich ihnen ihre Herrscherseele und all ihre Macht
und ihr Wissen, und wiedergewinnen wollt' ich mir der Höhen
ungetrübtes Spiel. O~dieses wonnige Dehnen und Schweben, das
unbegrenzte Gestalten und Überallsein –~– Und nun? Mein hoher
Freund! Es ist verloren! Es ist nicht möglich, jetzt nicht
möglich!“

„Nun, nun! Bei allen Steinrutschen! Du erschreckst mich. Was ist
denn geschehen?“

„Sie haben ein Stück von mir gefangen, ein Stück von dem
Menschenherzen, das noch in mir war!“

„Wer? Was? das versteh' ich nicht.“

„Du weißt doch, daß ich mein Wolkenherz mit einem Menschenherzen
mischte. Als ich nun wieder Wolke ward, hatte ich die Absicht,
später nochmals in den Menschenleib zurückzukehren, der inzwischen
drunten erstarrt in der Gletscherspalte ruht. Darum mußte ich einen
Teil seines Menschenwesens in meinem Wolkenherzen mitnehmen. Ich
sprach mit dem Vater und mit dem Hohen, ich sann nach und entschloß
mich, dem Menschen wiederzugeben, was sein ist, und frei vom
Menschenwesen dauernd eine Wolke zu werden. Und nun~–“

„Nun was?“

„Du sahst vorhin drunten die Männer über dem Gletscher. Als ich
hinabschwebte, dem Menschen in der Spalte seine Seele
zurückzugeben, bemerkte ich sie. Ich kenne den einen. Ich wollte
sehen, was er treibe. Er sammelte Luft in Gefäße, die er fest
verschloß. Ich kam zu nahe. Unvermutet schloß sich das Gefäß, als
ich dort mein Herz in der Nähe verdichtet hatte, und ein Teil
meines Herzens wurde gefangen – ich habe es nicht mehr.“

„Haha, Aspira, das tut doch einer Wolke nichts. Laß es ruhig dort,
bist auch so noch mächtig genug.“

„Wenn es nur mein Herz wäre! Wir können uns teilen und erneuern.
Des Wolkenherzens verlorenen Teil kann ich freilich entbehren, ja
durch ihn bin ich sogar noch mit dem geraubten Teile verbunden.
Aber es ist ja vom Menschenwesen ein Teil dabei. Und der Mensch,
der festgefügte Mensch mit seinem ins Feinste gegliederten
Nervenleib – o, du weißt nicht, wie wunderbar die Menschen gebaut
sind – er muß alles, alles wieder haben, was ich von ihm empfing,
nichts darf fehlen, nichts! Jetzt darf ich auch jenen kleinen Teil
des Herzens nicht missen, sonst bleibt der Mensch starr und tot
liegen in der Eisgruft.“

„Laß ihn liegen! Es liegen noch mehr Menschen in unsern
Schluchten.“

„Das darf ich nicht. Der Vater verbietet's und der Hohe, und~–~–“

„Und?“

„Etwas, das du nicht verstehen wirst.“

„Nun, nun! das ist wohl gar so was Menschliches?“

„Ja, das Beste am Menschen – das Gewissen.“

„Gewissen? Wo sitzt das? Aber als Wolke braucht dich doch nicht zu
kümmern, was die Menschen haben. Du bist ja hier, du ballst dich
und schwebst und verdunstest, wie dir's paßt, was geht dich als der
Menschen Werkzeug an?“

„Du vergißt, daß mein Wolkenherz noch ganz mit dem Menschenwesen
gemischt ist. Ich bin eine Menschenwolke. Hätte ich vorhin, wie ich
es wollte, dem Menschen drunten all sein Eigentum wieder zustellen
können nach den Vorschriften des Vaters, so wäre ich jetzt bei dir
als freie Königswolke. Solange ich aber den Menschen nicht wieder
zum vollen Menschen machen kann, behalte ich auch seine Seele in
mir mit allen Menschensorgen und Menschengeboten – und ich bin eine
denkende Wolke! O~ich arme!“

Und Aspira schmiegte sich enger an das Blankhorn und weinte,
weinte.

„Nun, nun,“ brummte der Alte. „Nur keine Aufregung! Was ist denn da
so Schlimmes dabei? Du wolltest ja doch Mensch werden.“

„Ich will es aber nicht bleiben, denn ich weiß jetzt, daß ich es
nie in voller Echtheit sein kann. Ein Mensch mit einem Stück
Wolkenherzen, das wollte ich nicht. Aber nun – eine Wolke mit einem
Stück Menschenherzen, – das ist schrecklich!“

„Na ja, armes Ding, das mag ja wohl schäbig sein.“

„O du weißt nicht, was es bedeutet. Der Leib des Riesen mit der
Kraft des Zwerges! Frei sich bewegen, und erkennen, daß alles Zwang
ist. Wissen um die Macht und ohnmächtig sein im Handeln! Schweben
in den Höhen der schönen Welt mit dem Leide um das Unerreichliche!
Willenlos wollen, befohlen spielen, namenlos-elend unsterblich
sein!“

Das Blankhorn erzitterte von einem geheimnisvollen Schauer
ergriffen. Lawinen donnerten von seinen Wänden. Bäche stauten sich
und stürzten mit neuer Gewalt zu Tale. Wirbelwinde brachen aus den
Schluchten. Erschrocken lauschten die Geister des Bezirks. Von
allen Seiten stürmten Winde und Wolken heran, dichter Schnee raste
über die Täler.

„Was willst du?“ rief Aspira. Sie bebte um Sohms Schicksal, den sie
noch auf dem Gletscher wußte. „Schnell stille die Geister! Jetzt
können wir den Aufruhr nicht brauchen. Bitte, bitte, errege dich
nicht!“

„Zerschmettern will ich den Menschenwurm mit seinem Gefäße und
befreien dein gefangenes Herz.“

„Nein, nein, – nicht so! Er darf nicht zugrunde gehen, das will ich
nicht! Zerstreut euch Wolken! Ruhe, Ruhe ihr Wetter. Aspira
gebietet – hellt euch auf!“

Und schnell, wie sie gekommen waren, verzogen sich die Wetter. Es
ward heller über dem Tale.

„Nun, nun!“ brummte das Blankhorn. „Man kann dir doch nichts recht
machen. Muß schon Nachsicht haben mit meiner kranken Aspira. Ha!
Kranke Wolke! Gar nicht zum Lachen! Aber was willst du denn nun?“

„Ich sann schon lange nach. Dem Menschen darfst du nichts tun. Aber
wenn er seine Flaschen im Hause geborgen hat, dann will ich es
versuchen mit Menschenlist, ob ich ihn nicht bewege, sie zu öffnen,
daß ich frei werde.“

„Kannst du denn zu den Menschen reden?“

„Verstehen kann ich alles, was sie sagen und tun, aber zu ihnen
reden, das kann ich nur in der Stille der Nacht, wenn sie
schlummern. Ach, wenn ich im Tunnel hätte zu ihnen sprechen können!
Doch das vermochte ich nicht. Wenn aber ihr Gehirn ruht, so vermag
ich mit dem Wolkenherzen, das mit dem Menschenherzen gemischt ist,
auf sie zu wirken, daß sie mich hören und verstehen. Und das will
ich versuchen.“

„Und wenn es dir nicht gelingt?“

„Wehe mir! Dann bin ich verloren! Laß mich jetzt ziehen, Blankhorn,
den Abend will ich erwarten drunten in einsamer Schlucht, bis die
Menschen schlummern~–~–“

„Geh, mein Liebling. Aber wenn – nun, nun! das Blankhorn ist auch
noch da!“

\section{Die gefangene Wolke}

Nachdem die Tunnelfrage in befriedigender Weise gelöst und der
Fortgang der Arbeiten gesichert war, hatte Sohm sich in Schmalbrück
im Hotel Leberecht eingemietet. Hier hoffte er, die sehnlichst
erwartete Nachricht von Wera baldigst zu erhalten. Die größere Nähe
des Ortes am Gletscher war ihm der Versuche wegen erwünscht, die er
mit seinem neuen Apparat anstellte. Dieser hatte sich vortrefflich
bewährt. Zwar das Wetter war fortwährend ungünstig, kalt und
windig. Doch gestattete es immerhin das Herumsteigen auf den
Bergen.

Heute hatte er bestimmt auf eine Botschaft von Wera gerechnet, aber
wieder war er enttäuscht worden. Er begann, sich zu beruhigen, doch
er tröstete sich mit Weras Telegramm: „Ängstige dich nicht um
mich.“ Er nahm an, daß sie sich in einen jener einsamen Orte,
vielleicht auf eine der Klubhütten, zurückgezogen hätte, von denen
die Verbindung mit der Außenwelt nur durch Boten möglich ist. Aber
jedesmal, wenn der Sturm einhersauste, der jetzt mehrfach
Schneefälle mit sich brachte, erwachte in ihm der Gedanke, daß Wera
ein Unglück zustoßen könne. So heute Abend, als nach leidlichem
Tage plötzlich ein Unwetter über Schmalbrück hernieder gebraust
war, das die Fremden, deren Zahl sich schon gelichtet hatte, in
Ärger und Schrecken versetzte.

Von der Hotelgesellschaft hatte sich Sohm ganz ferngehalten. Auch
jetzt, ermüdet von seinem Umhersteigen auf dem Gletscher, war er
zeitig auf sein Zimmer gegangen, um die Ruhe zu suchen.

Der Sturm hatte sich gelegt. Dennoch fand Sohm keinen erquickenden
Schlaf. Er dachte an Wera und fuhr mitunter erschrocken empor, wenn
er im Halbschlummer ihr Bild vor sich sah, als riefe sie ihn – Wo
mochte sie sein? War er ungerecht gegen sie gewesen? Doch nein,
jetzt nicht grübeln! Schlafen, schlafen!

Um seine Gedanken abzulenken, richtete er sie auf den Erfolg seiner
heutigen Tätigkeit. Er konnte zufrieden sein. Eine stattliche
Anzahl Flaschen, sorgfältig etikettiert und in Watte verpackt,
lagen in den Fächern seines Kastens verwahrt. Das gab eine
interessante Analyse für Wera~–~–

Wieder Wera! Schlafen, schlafen!

Auf einmal war es ihm, als klänge es ganz leise aus seinem Kasten
wie ein Seufzer. Noch einmal! Und dann glaubte er eine feine Stimme
zu vernehmen – deutliche Worte:

„Laß mich heraus! Bitte, laß mich heraus!“

„Spricht da jemand?“ stieß Sohm hervor. „Wer denn nur? Ist jemand
hier?“

„Ich bin's, die Wolke.“

„Wolke? Kenn' ich nicht. Aber wenn sich jemand hier einen
schlechten Witz machen will, so ist jetzt nicht die Zeit. Wo steckt
denn der Störenfried?“

„Hier im Glase, im Kasten.“ Ganz deutlich klang es jetzt aus der
Richtung, wo die Kiste stand. „Eine Wolke bin ich, König Migros
Tochter.“

„Verrückter Traum! Muß ich doch einen Moment eingenickt sein“,
brummte Sohm und drehte sich auf die andere Seite.

Aber aus dem Kasten klang es weiter, leise, doch vernehmbar:

„Eine Wolke bin ich. Gefangen ward ich durch meine Neugier. Ganz
aufgelöst zog ich oben am Firnfeld, unsichtbar ausgebreitet zum
leichtesten Hauch. Da erblickte ich dich und den andern und das
glitzernde Rohr. Und wissen wollt ich, was der fürwitzige Mensch
anhebt in unserm Reiche. Ich schwebte näher an die Dinge, die ihr
aufgestellt hattet, und da ihr nicht nahe dabei standet, glitt ich
mit der Luft hindurch und spähte durch die lichte Wandung. Und auf
einmal war die enge Tür geschlossen, ich konnte nicht fort. Du aber
packtest mich ein, und finster ist es seitdem – ich will hinaus aus
dem Dunkel!“

„Eine dunkle Sache ist das freilich“, sagte Sohm und richtete sich
auf. „Aber man soll alles objektiv betrachten. Dann muß sich's ja
zeigen, daß ich bloß subjektiv träume. Also eine Wolke bist du?“

„Ja.“

„Du wirst mir doch nicht einreden wollen, daß eine Wolke in einem
Glase von 300~Kubikzentimeter Inhalt Platz habe, bei dem mäßigen
Druck, den das Glas aushält?“

„Ich bin ja auch nicht die ganze Wolke, die du gefangen hast. Noch
schweb' ich droben um den Fuß des Blankhorns frei in der Weite,
noch kann ich als Nebel durch die Ritzen des Hauses dringen, aber
nicht durch das Glas. Zum Unglück war es gerade ein Teil meines
Herzens, den ich in der Flasche gesammelt hatte. Und den will ich
wieder haben, denn ich brauche ihn.“

„So? Also Wolken haben Herzen. Ist mir als Meteorologe sehr
interessant. Aber nun ist's genug! Wolken sind überhaupt nur
Ansammlungen von kleinen kondensierten Wassertröpfchen, ihre
Gestalt ist ganz abhängig von den äußeren Wirkungen der Umgebung,
sie haben keine individuelle Einheit; sie sind selbstverständlich
keine organisierten Wesen, haben also auch kein Herz, noch viel
weniger, wie du es zu verstehen scheinst, Leben und Seele; sie
haben kein Bewußtsein, keine Freiheit, keine Könige und
Prinzessinnen, und du bist weiter nichts als ein Spiel meiner
Einbildung. Und somit laß mich jetzt in Ruhe.“

„Ich denke,“ erwiderte Aspira, „man muß alles objektiv betrachten.
Das ist nun zwar sonst nicht gerade Wolkensache, aber mit mir ist
es etwas Besonderes. Mein herz nämlich kann mitschwingen mit allen
Regungen deines Nervensystems, und so kann ich deine Gedanken
verstehen und kann in deiner Sprache mit dir reden.“

„Unsinn!“

„Etwas höflicher könntest du schon mit einer Wolkenprinzessin
reden.“

Ein gewaltiger Windstoß erschütterte das Fenster. Sohm fuhr auf.

„Laßt es gut sein draußen!“ ertönte Aspiras Stimme. „Mit Gewalt ist
nichts auszurichten, das Haus ist zu fest. Ich muß mit dem Menschen
gütlich verhandeln.“

Sogleich schwieg der Sturm. Sohm lauschte eine Weile, und da sich
nichts regte, streckte er sich wieder hin und dachte: „Das kommt
nun davon, du böse, geliebte Wera, von deiner Geistertheorie! Nun
träume ich auch schon von Wolkenseelen!“

„Willst du mich nicht herauslassen?“ klang es wieder aus dem
Kasten.

„Ich denke gar nicht dran. Aber wenn ich einmal den Traum nicht los
werden kann, so rede meinetwegen weiter. Ich schlafe ja.“

„Wenn ich dir aber beweise, daß ich eine Wolke bin?“

„Beweise muß man achten. Immer zu!“

„Du behauptest, Wolken seinen nur Anhäufungen von Tröpfchen und
keine lebenden Wesen? Was bist du denn? Eine Anhäufung von Teilchen
derselben Elemente, nur ein bißchen mühsamer zusammengesucht. Da
müssen sich erst die komplizierten Eiweißklümpchen balanzieren und
Zellen bilden und sich teilen und differenzieren – bei uns ist das
alles viel einfacher. Und du sagst, wir hätten keine innere
Einheit? Woher willst du das wissen? Zerfließt ihr nicht auch und
fließt wieder zusammen aus den Elementen? Ihr atmet und eßt und
trinkt und erneuert euern Leib in jedem Augenblick; bei uns geht
das nur alles viel schneller. Deswegen bestehen wir doch im
Wechsel. Wer gab dir denn deine Gestalt und deine Organe, Haut und
Nerven, Blut, Muskeln und Knochen? Wir, die Luft, das Wasser und
die Erde und die strahlende Sonnenmutter. Wer formte sie? Niemand
anders als das Zusammenwirken der Gesetze in meines Vaters Reiche,
die diese Elemente verbanden und mit der Umgebung in einen
Austausch der Kräfte setzten. Dadurch mußte in ungezählten
Generationen sich diese Gestaltung erzeugen, die ihr einen
Menschenleib nennt, und erzeugt sich immer wieder nach gleichen
Gesetzen, wenn auch immer wieder ein wenig anders unter dem
wechselnden Einfluß der Umgebung. Denn ihr seid eben nur
Versuchsobjekte der Natur, an denen sie herummodelt, weil sie nie
mit euch zufrieden ist – höflicher gesagt, weil sie aus euch noch
etwas ganz Großes machen will. Auch dein Leben ist abhängig von der
Umgebung. Steige doch mit mir in die Höhe des Äthers, wir wollen
sehen, wer besser lebt. Auch wir bilden uns aus dem Zusammenwirken
des eignen Gesetzes mit der Umwelt, genau wie ihr. Nur weil wir
einfacher sind, sind wir de Natur bald gelungen. Wir zerfließen
schneller und bilden uns schneller, aber die Einheit des Gesetzes
halten wir fest. Unter gleichen Bedingungen sind wir wieder
dieselbe Wolke, wenn auch unabhängig von unsrer Gestalt. Ja wir
können uns teilen, und verlieren doch unsre Einheit nicht. Unsre
Einheit ist anders als die eure, aber warum soll es keine Einheit
sein? Hast du doch dafür ein Wort und einen Begriff, warum also
sollen wir darin nicht eine Seele haben?“

„Allen Respekt, Hoheit Wolke, vor deinen Deduktionen! Wenn das so
ist, dann kann dir's ja auch ganz gleich sein, ob ich hier ein paar
Milligramm Wasserdampf in meiner Röhre habe, oder nicht.“

„Ich sage dir ja, das gehört gerade zu meinem Zentralorgan, das wir
das Wolkenherz nennen.“

„Siehst du, da hab ich dich! Eine solche Zentralisation bei euch
ist gar nicht denkbar.“

„Das ist eben unsre Streitfrage. Weil die Meteorologie sie bis
heute nicht entdeckt hat, soll sie nicht denkbar sein! Ich sage
dir, es gibt in allen atmosphärischen Bildungen gewisse
formsetzende Gruppen, gasförmige Kristalle, Spannungssysteme des
Äthers, die unsre Einheit bedingen. In ihnen treffen sich und
regulieren sich die Spannungen, die bei allen Kondensationen und
Verdunstungen unsrer Teilchen maßgebend sind. Es ist schließlich
nichts andres, als was auch in euerm Nervenapparat vor sich geht,
nur ungleich einfacher, weiter und schneller wirkend. Das hindert
nicht, daß dieses Organ teilbar ist wie wir selbst, wir sind eben
keine Zellenwesen. Es gibt viel mehr Einheiten des Gesetzes, als
ihr bisher habt entdecken können.“

„Das Letztere gebe ich ohne weiteres zu. So wäret ihr in der Tat in
gewissem Sinne organisiert? Und dann könnte man euch auch eine Art
Bewußtsein nicht absprechen. Nur wissenschaftlich beweisen läßt es
sich niemals. Und deswegen kann ich darauf keine Rücksicht nehmen.
Überhaupt – du bist gar nicht originell, das träume ich alles bloß,
weil~–~–“

„So kann dir's ja auch gleichgültig sein, ob du mein Herz in deinem
Glase hast, oder ein andres Quantum Luft.“

„Aber dir kann's auch gleichgültig sein. Es scheint doch, daß dein
Zentralorgan auch noch durch das Glas und die Watte und Kiste
hindurch dein Wolkenleben regieren kann.“

„Bis zu einem gewissen Grade, ja. Aber ich brauche dieses
Stoffliche selbst zu einem ganz besondern Zwecke, den ich dir nicht
erklären kann. Außerdem – wenn auch ein Wolkenherz in der Natur
nahezu unsterblich ist, könnte es doch durch eure systematisch
angewendeten technischen Mittel zerstört werden. Du willst mich
doch mit fortnehmen zu irgend welchen Versuchen~–“

„Allerdings. Meine Assistentin wird dich austrocknen, durch
Kalilauge saugen, um die Kohlensäure zu entziehen, dann~–“

„Höre auf! Was hast du davon? Gib mich frei! Verlange von mir, was
du willst. Wir sind mächtig. Wir wissen viele Geheimnisse der Natur
und der Menschen, oder können sie erfahren. Wir kennen in den
Klüften der Berge seltene Gesteine und Metalle. Ich will sie dich
finden lassen und dich zum reichsten Manne der Erde machen. Aber
laß mich frei!“

„Wirklich? Ach sieh doch! Da hätte ich dich also, wie der Fischer
in ›Tausendundeine Nacht‹ den Geist in der Flasche mit dem Siegel
Salomonis. Da könnte ich vielleicht ein mächtiger Zauberer werden?
Aber, liebe Wolke, solche Märchen glaube ich nicht. Und wenn ich
einen Augenblick daran glauben könnte, so dürfte ich dich doch erst
recht nicht frei lassen, denn dann hätte mein glücklicher Fund ja
das höchste wissenschaftliche Interesse. Indessen~–“

Es fiel ihm ein, daß dies doch nicht so fortgehen könne. Er mußte
wenigstens morgen früh konstatieren können, daß er diesen ganzen
logischen Gedankenbau nur geträumt habe. Er dachte wieder an Weras
Phantasien. Gewiß, wenn er träumte, so würde das alles, was ihm
jetzt so klar durch den Kopf ging, morgen als schönster Unsinn
erkannt werden. Wenn er jetzt aufstände und sich ankleidete, dann
müßte er doch morgen merken, ob er das nur geträumt habe.

„Warum sprichst du nicht weiter?“ fragte Aspira.

„Ich habe mir überlegt,“ sagte Sohm, „wenn du wirklich eine
Wolkenprinzessin bist, so erfordert die Höflichkeit, daß ich deiner
Flasche einen bevorzugten Platz anweise, und dazu muß ich doch erst
etwas Toilette machen.“

Er drehte das Licht an und tat wie gesagt. Dazwischen fragte er:

„Wo war es eigentlich, wo ich dich einfing?“

Er glaubte nun, da er das Gefühl hatte, ganz munter zu sein, es
werde keine Antwort mehr erfolgen. Aber es klang ganz deutlich aus
dem Kasten:

„Am obern Rande des Gletschers, wo der schwarze Felsblock
hervorragt.“

Sohm wußte nicht mehr, was er denken sollte. Doch er sammelte sich.
Dort hatte er nur eine Probe entnommen. Es war sein höchster Punkt
gestern, 3024~Meter. Die Flasche mußte leicht zu finden sein. Er
öffnete die Kiste und erkannte sogleich die mit der Meereshöhe
bezeichnete Flasche.

Er hielt sie gegen das Licht, natürlich ohne irgend etwas
Besonderes sehen zu können. Aber fast wäre sie ihm aus der Hand
geglitten, so schrak er zusammen, als deutlich, obwohl leise, dicht
vor seinem Gesicht die Stimme erklang:

„Ja, ich bin es, laß mich heraus!“

Er schloß die Flasche in das Schreibpult, setzte sich an den Tisch
und stützte den Kopf in die Hände.

„Du machst mich noch wahnsinnig!“ rief er verzweifelt. „Es ist doch
unmöglich! Wie kann eine Wolke solche verteufelt schlau geordnete
Vorstellungen haben? Wie kannst du überhaupt reden!“

„Weißt du etwa,“ tönte es aus dem Schreibtisch, „wie dein
Bewußtsein mit deinem Körper verknüpft ist? Wie soll ich es wissen?
Und das sagte ich dir doch, das Organ, wodurch ich mich dir
verständlich mache, ist dein eignes Gehirn. Vermöge meines
Zentralorgans löse ich in dem Deinen zentrale Reize aus. Du glaubst
zu hören, wie du im Traume den redenden Freund hörst~–“

„Also endlich! Gott sei Dank! So träume ich doch!“

„Nein, nein! Der Anlaß des scheinbaren Traums ist objektiv. Ich bin
hier, das Wolkenherz! Gedenke der Schätze!“

„Still! Still! Ich habe wirklich andere Sorgen – ich muß Ruhe
haben.“

Er legte sich wieder zu Bett. Aber Aspira ruhte nicht. Sie begann
aufs neue:

„Wenn du keine Schätze willst – ich kann mehr geben. Erinnerst du
dich, was du mit Wera Lentius von den Elementargeistern gesprochen
hast?“

„Wera? Was? Wie kommst du dazu? Ach – das kannst du nicht wissen.
Also habe ich jetzt den strengen Beweis, daß ich nur träume. Das
stammt alles aus mir selbst.“

„Weißt du, wo Wera ist?“

„Nein!“

„Aber ich weiß es. Willst du mich frei lassen, wenn ich dir
verspreche, sie morgen herzuführen? Morgen früh sollst du sie
wieder haben.“

„Laß mich! Laß mich! Ich will wach sein.“

„Und ich muß hinaus. Wenn es nicht anders geht, so mußt du das
Schreckliche vernehmen! Weißt du wo Wera ist?“

„Um Gotteswillen! Das wird ein Angsttraum!“

„Wenn du mich nicht frei läßt, ist sie verloren! Wera liegt draußen
in der Gletscherspalte. Nur ich weiß den Ort. Nur ich kann sie
retten! Aber nur, wenn ich mein ganzes Herz wieder habe!“

Sohm stieß einen Schrei aus und schlug mit den Armen um sich.

„Eile, eile, Paul Sohm! Weras Leben ruht in dieser Flasche!“

„Wahnsinn!“ schrie er. Er raffte sich auf und nahm aus seiner
Reiseapotheke ein Pulver. Er tat es in Wasser und trank es aus.
Noch einmal hörte er Aspira rufen. Dann wirkte das Pulver. Die
Stimme verschwamm undeutlich. Er hörte nichts mehr. Er schlief
wirklich.

\section{Im Eise}

Nebelig und windig war der Morgen in Schmalbrück. Seufzend blickten
die Fremden in de übelwollende Wetter und fragten sich, ob sie wohl
besser täten abzureisen~–

Unruhig umschwebte Aspira das Haus, wo alle ihre Bemühungen
vergeblich gewesen waren, sich des eingeschlossenen Teils ihres
Zentralorgans wieder zu bemächtigen. Und verzweifelt bebte das
gefangene Wolkenherz mit im Dunkel des Schreibpults. Vergeblich
hatte das Blankhorn seine Stürme gegen die festen Mauern gesandt,
vergeblich war die lockende Rede Aspiras gewesen, vergeblich sogar
die Drohung mit Weras Vernichtung. Sie konnte es ja freilich nicht
anders erwarten. Ein Mensch von der festen wissenschaftlichen
Überzeugung Sohms konnte diese Erscheinung nicht anders deuten als
subjektiv, allein dem eignen Gehirn entsprungen. Und doch war die
Drohung keine leere gewesen, nur zu entsetzlich nahte die Wahrheit.
Konnte Aspira nicht ganz sich an Wera zurückgeben, so waren sie
beide dem Verderben verfallen – Wera dem Erstarrungstod des
Menschen in der Eisspalte, und Aspira dem erstarrten Leben in den
Höhen der Luft als unseliger Geist~–~–

Was tun? Was tun? Nun hatte sich Paul ihrem Einflusse durch ein
Mittel nach Menschenart entzogen, ein Schlafmittel, das sein Gehirn
lähmte. Und wenn er wieder erwachte, dann war es Tag – würde sie
dann noch einmal mit ihm sprechen können? Versuchen mußte sie es.
Wenn er allein war, konnte sie sich doch vielleicht verständlich
machen, sein Geist mochte dann noch unter der Nachwirkung des
vermeintlichen Traumes stehen und sich ihrem Vorstellungskreise
wieder anpassen. Aber würde er ihr mehr glauben als in der Nacht?
Verloren, verloren!

Es war schon spät am Morgen, als Sohm mit wirrem Kopfe sich erhob.
Während er sich ankleidete, besann er sich auf seinen Traum. Es war
doch zu toll, zu deutlich gewesen!

„Nun muß ich doch nachsehen,“ sagte er sich, „ob ich wirklich in
der Nacht~–“

Er schloß den Schreibtisch auf und fuhr zurück – da stand die
Flasche.

„So bin ich tatsächlich heute nachtgewandelt und habe die Flasche
hierher gestellt. So ein Blödsinn! Das ist mir noch nie passiert.
Nun mag sie schon hier stehen bleiben. O, wenn ich nur erst wüßte,
wo du bist, du geliebte, böse Wera. Dann würden alle diese
törichten Angstphantasien auf einmal verschwinden. Nun, vielleicht
liegt schon ein Brief für mich unten!“

Er wollte das Schreibpult abschließen. Da klang es leise:

„Paul Sohm!“

Er erschrak. Was war das? Doch wohl nur die Feder des Schlosses,
die leise schwirrte? Oder ein Ton von draußen?

„Paul Sohm, ich muß mit dir reden.“

War das wieder die Flasche? Bin ich krank? Träume ich denn noch
immer? dachte er.

Er trank ein Glas Wasser. Er öffnete das Fenster und sog die rauhe
Morgenluft ein. Dann setzte er sich vor das Pult und sagte zu sich:
„Ruhig Blut jetzt! Man muß alles prüfen. Nun rede, Flasche, denn du
noch kannst.“

Da klang es deutlich:

„Höre mich! Wenn Wera nicht verderben soll~–“

„Nein! Es ist zum verrückt werden! Oder bin ich es schon?“ Er
wollte aufspringen, aber er zwang sich, sitzen zu bleiben und faßte
krampfhaft seinen Kopf mit den Händen. „Redet denn wirklich die
Flasche? Also Ruhe! Das müssen wir ergründen.“

Sohm öffnete den Kasten. Er nahm noch andere Flaschen heraus. Er
trat ans offene Fenster und betrachtete sie genau. Eine war wie die
andere. Nun stellte er sie direkt vor sich hin, er redete sie an –
eine wie die andere schwieg.

„Also ihr könnt nicht reden“, sagte er. „So kapriziert sich meine
Einbildung auf die eine Flasche. Oder – vielleicht bin ich nun
glücklich den ganzen Schwindel los.“

Was tönt da wieder? Von den Flaschen? Nein, nein, das war Geräusch
von der Straße, eilige Tritte, Stimmen im Hotel und vereinzelte
Rufe: „Wo? Wo?“

Sohm achtete jetzt nicht darauf. „Nun erst die Gegenprobe“, sagte
er aufgeregt. Er ergriff die Flasche mit Aspiras Herzen. Er stand
am Fenster und hielt sie vor sich hin.

„Nun schweige, schweige auch du!“ murmelte er.

„Öffne die Flasche“ rief Aspira jetzt laut wie mit einem
Verzweiflungsschrei. Sohm begann zu zittern.

„Sofort! Sonst stirbt Wera in der Gletscherspalte!“

Zugleich erscholl von draußen die Stimme des Wirts:

„Herr Professor, Herr Professor!“

„Hier! Was gibt's?“

„Es ist ein Unglück geschehen – am Gletscher – Fräulein Lentius~–“

Sohm stellte die Flasche wortlos aus der Hand, wo er stand, aufs
Fensterbrett, und sprang nach der Tür. Er riß sie auf.

Ein Windstoß sauste durchs Haus und schlug das Fenster zu. Die
Flasche wurde hinausgeschleudert. Sie stürzte auf die Steine des
Hofes und zersplitterte.

Leute sammelten sich.

„Was ist mit meiner Braut? Schnell, Herr Leberecht.“

„Es ist schon Hilfe hin, Herr Professor. Der Dumme Peter, der
Hirtenbub, brachte die Nachricht, es läge eine Dame am Gletscher.
Der Herr Oberingenieur Martin schickt ihn, er meint, es sei – hier
ist die Karte.“

„Wera? Wo? Wo? Ich will hin.“

„Ich gehe mit“, sagte der Wirt.

Aber allen voran mit des Sturmes Eile floh Aspira, die Befreite –~–
Jetzt mit ganzem, ungeteiltem Herzen, ein Mensch in Wolkengestalt,
bereit, Leben zu geben, Leben zu gewinnen. Hinaus, hinauf in die
Freiheit!

Martin war, wie immer, schon sehr früh im Tunnel gewesen und hatte
den Fortgang der Arbeiten in bester Ordnung gefunden. Er konnte die
Arbeitsstelle mit der zufriedenen Überzeugung verlassen, daß noch
vor Ablauf des September der Durchschlag des Tunnels ohne Störung
erfolgen werde.

Jetzt stieg er an der Lehne des Langbergs nach dem ehemaligen
Steinbruch in die Höhe, um die Stelle zu besichtigen, wo die
Schiefklippe sich befunden hatte, und die dortigen
Sicherheitsarbeiten zu kontrollieren. Er wanderte dann noch ein
Stückchen weiter nach einem Punkte, wo man über den Gletscher nach
den Schneebergen hinübersehen konnte, die freilich jetzt in Wolken
gehüllt lagen.

Sinnend blickte er hinaus. Auch ihn quälte die Frage, was aus Wera
geworden sei. Ob sie vielleicht schon in Schmalbrück ist? Ach, was
ging es ihn an? Er wollte sie nicht wiedersehen~–

Eben wollte er umkehren, als er in der Ferne, am Rande des Berges,
wo der Weg nach dem Gletscher umbog, eine seltsame Gestalt
auftauchen sah. Sie warf die Arme unregelmäßig in die Luft, und
wollte offenbar Zeichen geben. Jetzt vernahm er auch Ausrufe, ohne
sie verstehen zu können. In der Annahme, daß es sich um einen
Notruf handle, lief er auf dem Pfade dem Kommenden entgegen und
erkannte nun bald den Ziegenhirt Peter, der dumme Peter genannt,
seines trottelhaften aber gutmütigen Wesens wegen.

„Was gibt's, Peter?“ rief er.

Der Atemlose kam immer näher, brachte aber nur zusammenhangloses
Zeug vor. Zuerst von einer Ziege und vom Sturme. Dann von der
Spalte im Gletscher und dem Schnee. Und das wäre die Ziege nicht.
Es wäre eine Dame, eine Fremde.

Martin erschrak. Er beruhigte den Burschen und gewann endlich ein
Bild der Sachlage. Peter war auf der Suche nach einer verlorenen
Ziege auf den Gletscher gekommen, der wegen des Neuschnees schlecht
zu passieren war. Dort hatte er in einer Spalte etwas liegen sehen
und sei nahe herangekrochen. Da war er offenbar enttäuscht, daß es
nicht die Ziege war, sondern ein Mensch. In fliegender Eile
forschte Martin nach Merkmalen, denn sein erste Gedanke war an
Wera. Ja, eine fremde Dame mit dunklen Haaren, die im Sommer hier
gewesen war mit der Mappe. Sie hatte ihm einmal Geld geschenkt~–~–

Martin zweifelt nicht mehr. Er schickte Peter nach Schmalbrück ins
Hotel Leberecht und mahnte ihn zur größten Eile. Zur Sicherheit
schrieb er ein paar Zeilen auf seine Karte und gab sie Peter mit.
Ein großes Silberstück beschleunigte seine Füße. Er flog den Pfad
hinab.

Martin selbst aber eilte pochenden Herzens nach dem Gletscher. Die
Anstrengung war gewaltig. Gegen den scharfen Wind mußte er den
steinigen Pfad hinauf und über die Moräne. Er wußte nicht, wo sich
die Spalte befand, aber überall lag hier noch Neuschnee, und die
Spuren Peters leiteten ihn. Da hörten sie auf. Hier war die Spalte.
Er legte sich nieder und spähte hinab. Dort, weiter hinten, in der
Ecke, wo eine Nebenspalte endete, lag etwas. Wie kam er hinab? Er
bewegte sich vorsichtig auf dem Eise entlang, und merkwürdig – hier
verflachte sich die Spalte auf einmal wegartig, ohne Schwierigkeit
konnte man hineintreten~–~–

Und da, halb im Schnee, nur das bleiche Gesicht völlig frei, lag
eine weibliche Gestalt, ruhig, wie sorgsam gebettet und eingehüllt
– Wera!

Einen Augenblick stand er starr, Atem schöpfend, dann sprang er
vor, in der Aufregung glitt sein Fuß aus, er taumelte und fühlte
einen stechenden Schmerz in der Brust, aber nun war er bei ihr. Er
warf sich auf die Kniee und hob sanft ihren Oberkörper. Kalt,
eisig, ohne Atemzug, ohne Pulsschlag – starr~–

Was sollte er tun? Der Bote mußte jetzt in Schmalbrück sein. Aber
ehe die Hilfe kam –~– Und wie lange lag sie hier? Er hielt sie für
unrettbar. Jammer zerriß sein Herz.

Was war das für ein Sausen in der Luft? Schnee wirbelte auf. Feine
Flocken stürmten von oben herab und fielen auf die bleichen
Lippen.

Die namenlos Geliebte hielt er umfaßt – nun gehörte sie niemand
mehr – nun küßte sie der kalte Schneesturm –~– Und weinend beugte
er sich herab und preßte seine Lippen auf die ihren:

„Wera, geliebte Wera, lebe wohl!“

Aber da – gütiger Himmel! Ist es möglich? Diese Lippen wurden warm,
sie erwiderten seinen Kuß, diese Arme hoben sich und schlangen sich
um seinen Nacken, diese Augen öffneten sich und glänzten ihm
leuchtend entgegen~–~–

Aspira war's, die im Schneesturm hergerast war – sie hatte sich in
Weras starren Leib ergossen – sie hatte ihr Herz mit dem ihren
gemischt und zum ersten, zum einzigen Male hatten sich die Seelen
ganz durchdrungen – Aspira gab Wera das Leben wieder, und für
diesen einen Augenblick gewann sie das volle, innige Menschendasein
– in Wonne bebend hielt sie den geliebten Mann umschlungen und in
ihrem Innern jubelte Menschenglück – einen Augenblick~–

Jetzt konnte sie Mensch bleiben, jetzt konnte sie ihm gehören –
wem? Martin? Nein! Nein! Das durfte nicht sein! Bald würden die
Retter kommen und sie mußten Wera finden – ohne Aspira –~– die
treue, ursprüngliche Wera~–

Noch einen glühenden Kuß – und nun rangen sich die ersten Worte
über ihre Lippen:

„Fliehe! Fliehe, Geliebter! Mein Atem bringt den Tod. In mir muß
sich Seele von Seele trennen und die entweichende würde dich
vernichten. Fliehe!“

„Von dir fliehen? Jetzt , wo ich dich gefunden?“ Er versuchte sich
aufzurichten, aber mit einem unterdrückten Schmerzensrufe brach er
zusammen.

„Ich weiß nicht, was mir ist“, sagte er. „Doch es wird gleich Hilfe
dasein~–“

Und er schlang wieder den Arm um ihren Hals und suchte ihre Lippen.
Noch einmal – dann schlossen sich seine Augen, der Atem stockte –
er hatte das Bewußtsein verloren.

Auch der Körper des Mädchens sank zurück. Eisiger Atem entströmte
ihrem Munde. Aspiras Wolkenherz trennte sich von dem warmen
Menschenleibe und floh hinaus in die Luft. Der Schneesturm
wirbelte, kalte Flocken legten sich auf Martins Gestalt – dichter
und dichter häufte sich der Schnee und verbarg den erstarrten
Körper des Mannes~–~–

Aber kräftiger und wärmer atmete das Mädchen –~– Alles, was Aspira
gehörte, war entflohen, stäubte im wilden Treiben des Schneesturms
– Wera richtete sich auf und blickte verwundert um sich.

„Wo bin ich? Warum bettete ich mich in die Gletschergruft? O~–
welch ein Traum! Glaubt' ich doch eine Wolke zu sein – ich kann
mich nicht erinnern – wie lange lag ich hier? Hinaus, hinaus!“

Sie erhob sich. Wo geht's hinaus? Sie bemerkte nichts von dem
Toten, der hinter ihr unter dem Schnee lag – sie sah, daß der Weg
nicht schwierig war, als ob der Sturm ihn gefegt hätte. Und während
die Wolken sich dichter um die Eiswände drängten, tastete Wera sich
hinaus. Denn körperlich fühlte sie sich stark und kräftig – und es
war, als ob Wind und Wolken sie trügen, so schritt sie über das
Eis, den nahen Steintrümmern zu und noch ein Stück weiter auf dem
Wege~–~–

Da donnerte es vom Firnfeld des Blankhorns – eine Lawine stäubte
herab und bedeckte die Spalte~–

Und nun wurden die Wolken lichter – Stimmen erklangen und Rufen –
und Männer nahten mit Stricken und Beilen – sie aber rief wieder
und lief ihnen entgegen, und einer stürmte voran allen andern.

„Wera! Wera!“

„Ich bin's, Paul! Mein Paul!“

Und sie lag schluchzend an seinem Halse.

\section{Schluß}

Ein Jahr war vergangen.

Am Abhang des Langbergs, wo die Schiefklippe herabgestürzt war,
saßen Paul Sohm und seine junge Frau. Wera ordnete einen Strauß
später Alpenblumen, die sie gepflückt hatte.

„Ist es nicht merkwürdig,“ sagte sie nachdenklich, „daß ich mich
noch dunkel erinnere, wie ich in dieser Gegend das letzte Mal
Blumen zusammenband, daß es mir aber ganz unmöglich ist mich zu
besinnen, was dann weiter erfolgte?“

„Liebste, du warst eben damals leidend. Ich bin noch immer der
Meinung, daß zu deiner nervösen Abspannung eine Vergiftung durch
irgend eine Exhalation in der Höhle getreten ist.“

„Dort oben am Gletscher – freilich, da begann es so plötzlich. Und
ebenso plötzlich, bei meiner Rettung aus der Spalte, war meine
unselige Einbildung verschwunden! Jetzt sehe ich ja ein, daß ich
nervenkrank gewesen sein muß, aber damals – es war schrecklich, ich
war überzeugt, ganz gesund und vernünftig zu denken, und bildete
mir doch ein – nein, ich bringe keinen Zusammenhang heraus! Und es
schmerzt mich, daß ich auch von Martin gar kein deutliches Bild
mehr habe, dessen Botschaft ich doch meine Rettung verdanke. Ich
weiß wohl, was ich mit dir gesprochen habe und was ich hier plante,
aber das eigentlich Motiv, dieser Glaube an mein Verständnis mit
den Elementargeistern, was ich dir nicht zu sagen wagte, das ist
mir alles so nebelhaft, so traumartig geworden, ich kann mich auf
nichts einzelnes mehr erinnern.“

„Und es ist dir auch nie wieder eingefallen, wie du in die
Gletscherspalte geraten bist?“

„Nein, Liebster. Das Letzte, was ich weiß, ist, daß ich, nach der
Unterhaltung mit dem liebenswürdigen Brautpaar im Wagen, durch
Nacht und Sturm in wahnsinniger Hast den Berg hinauf eilte. Alles
andere ist fort – gerade wie mein Bündel mit dem Proviant! Ich
denke mir, daß meine nervöse Überreizung zu einer Krisis kam –
wahrscheinlich habe ich in der Unterkunftshütte am Schmalstein
bewußtlos gelegen und bin dann zuletzt auf den Gletscher geraten –
aber nun, es ist ja vorbei~–“

Er hielt sie im Arm und sagte weich:

„Du geliebtes, armes Geschöpf. Wer weiß, was du ausstehen mußtest!
Es ist gut, daß du gar nichts mehr davon behalten hast. Auch nicht
von deinem Erwachen, mitten im Eise?“

In der Erinnerung suchend hielt sie die Augen geschlossen, den Kopf
an seine Brust gelehnt, dann blickte sie voll zu ihm auf:

„Ich begriff mich nicht. Ich fühlte nur die rauhe Luft und daß ich
hier fortmüsse, und als ich aufstand, war ich ganz frisch und
gesund, und auf einmal hörte ich Stimmen – und dann kamst du~–“

Sie schmiegte sich an ihn.

„Laß uns gehen,“ sagte er zärtlich. „Es wird kühl, und dort steigt
schon die abendliche Wolke aus der Festinaschlucht.“

„Seltsam,“„Ich begriff mich nicht. Ich fühlte nur die rauhe Luft
und daß ich hier fortmüsse, und als ich aufstand, war ich ganz
frisch und gesund, und auf einmal hörte ich Stimmen – und dann
kamst du~–“

Sie schmiegte sich an ihn.

„Laß uns gehen,“ sagte er zärtlich. „Es wird kühl, und dort steigt
schon die abendliche Wolke aus der Festinaschlucht.“

„Seltsam,“ fuhr er fort, indem sie sich bergabwärts wandten, „wie
schnell sich eine Sage bildet. Der Bahnwärter erzählte mir, daß
diese Wolke jeden Abend hier heraufstiege und sich vor den Tunnel
lagere, und zwar seit jenem Tage, der – der mir dich
wiederbrachte~–~–“

Sie waren den wenig begangenen Fußpfad hinabgestiegen, der am
Wärterhaus vor dem Tunnel vorbeiführt. Vor dem Eingang, durch den
die Bahnzüge jetzt gefahrlos rollten, blieben sie stehen.

Hier ist eine Erzplatte in den Fels eingelassen. Sie ist gewidmet
dem Andenken des Erbauers des Tunnels, dem Oberingenieur Theodor
Martin, den eine Lawine verschüttete, als er zur Rettung eines
Menschenlebens im Schneesturm auf den Blankhorngletscher eilte. Der
Tote selbst war nicht aufgefunden worden. Die Gewalten des Berges
hatten ihn bestattet.

Wera Sohm legte den Blumenstrauß auf einem Vorsprung des Felsens
nieder. Paul faßte ihre Hand. Ein leichter Nebel umhüllte die
heimwärts Wandelnden.

Zwischen Tal und Höhen steigend und fließend in freiem Spiel zieht
Aspira dahin. Vergessen unter ihr hastet der Menschen unruhiges
Geschlecht. Nichts stört den heitern Flug der Königswolke. Gelöst
ist die von der Sorge um des Schöpfers Leid und um die dunkeln
Fragen der Erkenntnis.

Nur eine Erinnerung an ihr Menschentum ist ihr geblieben, eine
einzige – die letzte – da sie auftauchte aus der innersten Tiefe
des Menschenherzens zurück zur Höhe der Heimat –~– Ein unendliches
Glück eines wonnigen Augenblicks, verklärend mit geheimnisvollem
Schimmer den ewigen Wandeln in Werden und Vergehen.

\end{document}
