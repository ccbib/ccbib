\usepackage[german,ngerman]{babel}
\usepackage[T1]{fontenc}
\hyphenation{wa-rum}


%\setlength{\emergencystretch}{1ex}

%\renewcommand*{\tb}[1]{\begin{center}#1\end{center}}

\begin{document}
\raggedbottom

\author{Kurd Laßwitz}
\title{Auf zwei Planeten}
\subtitle{Roman in zwei Büchern}
\date{Die Erstausgabe erschien 1897}
\maketitle

\part{Erstes Buch}

\section{1 - Am Nordpol}

Eine Schlange jagt über das Eis. In riesiger Länge ausgestreckt
schleppt sie ihren dünnen Leib wie rasend dahin. Mit
Schnellzugsgeschwindigkeit springt sie von Scholle zu Scholle, die
gähnende Spalte hält sie nicht auf, jetzt schwimmt sie über das
offene Wasser eines Meeresarms und schlüpft gewandt über die hier
und da sich schaukelnden Eisberge. Sie gleitet auf das Ufer,
unaufhaltsam in gerader Richtung, direkt nach Norden, dem Gebirge
entgegen, das am Horizont sich hebt. Es geht über die Gletscher hin
nach dem dunklen Felsgestein, das mit weiten Flecken bräunlicher
Flechten bedeckt mitten unter den Eismassen sich emporbäumt. Wieder
schießt die Schlange in ein Tal hinab. Zwischen den Felsbrocken
sproßt es grün und gelblich, Sauerampfer und Saxifragen schmücken
den Boden, die spärlichen Blätter eines Weidenbuschs zerstieben
unter dem Schlag des mit rasender Geschwindigkeit hindurchfahrenden
Schlangenleibes. Eilend entflieht eine einsame Schneeammer,
erschrocken und brummend erhebt sich aus seinem Schlummer der
Eisbär, dem soeben die Schlange das zottige Fell gestreift hat.

Die Schlange kümmert sich nicht darum; während ihr Schweif über die
nordische Sommerlandschaft hinjagt, hebt sie ihr Haupt hoch empor
in die Luft, der Sonne entgegen. Es ist kurz nach Mitternacht, eben
hat der neunzehnte August begonnen.

Schräg fallen die Strahlen des Sonnenballs auf die Abhänge des
Gebirges, das unter der Einwirkung des schon monatelang dauernden
Tages sich mit reichlichem Pflanzenwuchs bedeckt hat. Hinter jenen
Höhen liegt der Nordpol des Erdballs. Ihm entgegen stürmt die
Schlange. Wo aber ist der Kopf des eilenden Ungetüms? Man sieht ihn
nicht. Ihr dünner Leib verfließt in der Luft, die klar und
durchsichtig über der Polarlandschaft liegt. Doch welch seltsame
Erscheinung? Der Schlange stets voran schwebt, von der Sonne
vergoldet, ein rundlicher Körper. Es ist ein großer Ballon. Straff
schwillt die feine Seide unter dem Druck des Wasserstoffgases, das
sie erfüllt. In der Höhe von dreihundert Meter über dem Boden
treibt ein starker, gleichmäßig wehender Südwind den Ballon dem
Norden zu. Die Schlange aber ist das Schlepptau dieses Luftballons,
der in günstiger Fahrt dem langersehnten Ziel menschlicher
Wißbegier sich nähert, dem Nordpol der Erde. Auf dem Boden
nachschleppend reguliert es den Flug des Ballons. Wenn er höher
steigt, hemmt es ihn durch sein Gewicht, das er mit aufheben muß;
wenn er sinkt, erleichtert es ihn, indem es in größerer Länge auf
der Erde sich ausstreckt. Seine Reibung auf dem Boden bietet einen
Widerstand und ermöglicht es damit den Luftschiffern, durch
Stellung eines Segels bis zu einem gewissen Grade von der
Windrichtung abzuweichen.

Aber das Segel ist jetzt eingezogen. Der Wind weht so günstig
unmittelbar von Süden her, wie es die kühnen Nordpolfahrer nur
wünschen können. Lange hatten sie an der Nordküste von Spitzbergen
auf das Eintreten des Südwinds gewartet. Schon neigte sich der
Polarsommer seinem Ende zu, und sie fürchteten unverrichteter Sache
umkehren zu müssen, wie der kühne Schwede Andrée bei seinem ersten
Versuche. Da endlich, am 17. August, setzte der Südwind ein. Der
gefüllte Ballon erhob sich in die Lüfte; binnen zwei Tagen hatten
sie tausend Kilometer in direkt nördlicher Richtung zurückgelegt.
Der von Nansen entdeckte nordische Ozean war überflogen und neues
Land erreicht, das sich ganz gegen Erwartung der Geographen hier
vorfand. Schon entschwand das Supan-Kap auf Andrée-Land im Süden
ihren Blicken. Bald mußte es sich entscheiden, ob die beiden
Expeditionen, die eine im Ballon, die andere mit Schlitten
unternommen, wirklich, wie ihre Führer meinten, den Pol selbst
erreicht hätten. Bei der Unsicherheit der Ortsbestimmung in diesen
Breiten waren Zweifel darüber entstanden, die Aussicht vom Ballon
war durch Nebel getrübt gewesen, der Schlittenexpedition fehlte ein
weiterer Überblick. Jetzt war durch die Mittel eines reichen
Privatmanns, des Astronomen Friedrich Ell, eine deutsche Expedition
ausgerüstet worden, die noch einmal mittels des Ballons den Pol
untersuchen sollte.

Natürlich hatte man sich die Erfahrungen der früheren Expeditionen
zunutze gemacht. Durch die internationale Vereinigung für
Polarforschung war eine eigene Abteilung für wissenschaftliche
Luftschiffahrt ins Leben gerufen worden. Namentlich hatte man die
Benutzung des Schleppseils ausgebildet und damit für die Leitung
des Ballons wenigstens annähernd ein Mittel zur Lenkung gefunden,
wie es das Segelschiff im Widerstand des Wassers besitzt. Man hatte
Metallzylinder konstruiert, in denen man bis auf 250 Atmosphären
Druck zusammengepreßten Wasserstoff mit sich führte, um bei
Dauerfahrten einen eingetretenen Gasverlust zu ersetzen. Man hatte
dem Korb eine Form gegeben, die es gestattete, ihn nach Bedarf
gegen die äußere Luft abzuschließen. Der neue Ballon ›Pol‹ war mit
allen diesen fortgeschrittenen Einrichtungen ausgerüstet. Außerdem
hing unterhalb des Korbes zur Rettung im äußersten Notfall ein
großer Fallschirm. Unter einer Art Sattel, der einen sicheren Sitz
gewährte, war an demselben für alle Fälle ein Proviantkorb
befestigt.

Der Direktor der Abteilung für wissenschaftliche Luftschiffahrt,
Hugo Torm, hatte selbst die Leitung der Expedition unternommen. Ihn
begleiteten der Astronom Grunthe und der Naturforscher Josef
Saltner. Saltner warf einen Blick auf Uhr und Barometer, drückte
auf den Momentverschluß des photographischen Apparats und notierte
die Zeit und den Luftdruck.

„Diese Gegend hätten wir glücklich in der Tasche“, murmelte er.
Dann streckte er die in hohen Filzstiefeln steckenden Füße so weit
aus, als es der beschränkte Raum des Korbes zuließ, zwinkerte mit
den lustigen Augen und sagte: „Meine Herren, ich bin schauderhaft
müde. Könnte man nicht jetzt ein kleines Schläfchen machen? Was
meinen Sie, Kapitän?“

„Tun Sie das“, antwortete Torm, „Sie sind an der Reihe. Aber
beeilen Sie sich. Wenn wir diesen Wind noch drei Stunden behalten
–“

Er unterbrach sich, um die nötigen Ablesungen zu machen.

„Wecken Sie mich gefälligst, sobald wir – am Pol sind –“

Saltner sprach mit geschlossenen Augen, und beim letzten Wort war
er schon sanft entschlummert.

„Es ist ein unheimliches Glück, das wir haben“, begann Torm. „Wir
fliegen im wahren Sinn des Worts auf das Ziel zu. Ich habe für die
letzten fünf Minuten wieder 3,9 Kilometer notiert. Könnten Sie eine
genauere Bestimmung versuchen, wo wir sind?“

„Es wird sich machen lassen“, antwortete Grunthe, indem er nach dem
Sextanten griff. „Der Ballon geht sehr ruhig, und wir haben die
Ortszeit ziemlich sicher. Wir hatten den tiefsten Sonnenstand vor
einer Stunde und 26 Minuten.“ Er nahm die Sonnenhöhe mit größter
Sorgfalt. Dann rechnete er einige Zeit lang.

In vollkommener Stille lag die Landschaft, über welche die
Luftschiffer eilten. Ein weites Hochplateau, mit Moos und Flechten
bedeckt, hier und da von Wasserlachen durchsetzt, bildete den Fuß
des Gebirges, dem sich der Ballon schnell näherte. Man hörte nichts
als das Ticken der Uhrwerke, von Zeit zu Zeit das regelmäßige
Abschnurren des Aspirationsthermometers, dazwischen die behaglichen
Atemzüge des schlummernden Saltner. Es war freilich eine
angenehmere Polarfahrt, als mit halbverhungerten Hunden die
langsamen Schlitten über die Eistrümmer zu schleppen. Grunthe sah
von seiner Rechnung auf.

„Welche Breite haben Sie aus der Berechnung des zurückgelegten
Weges?“ fragte er Torm.

„Achtundachtzig Grad fünfzig – einundfünfzig Minuten“, erwiderte
dieser.

„Wir sind weiter.“

Grunthe machte eine Pause, indem er noch einmal kurz die Rechnung
prüfte. Dann sagte er bedachtsam, aber mit derselben
Gleichmäßigkeit der Stimme:

„Neunundachtzig Grad 12 Minuten.“

„Nicht möglich!“

„Ganz sicher“, erwiderte Grunthe ruhig und zog die Lippen ein, so
daß sein Mund unter dem dünnen Schnurrbart wie ein Gedankenstrich
erschien. Das war das Zeichen, daß keine Gewalt mehr imstande sei,
an Grunthes unerschütterlichem Ausspruch etwas zu ändern.

„Dann haben wir keine 90 Kilometer mehr bis zum Pol“, rief Torm
lebhaft.

„Neunundachtzigeinhalb“, sprach Grunthe.

„Dann sind wir in zwei Stunden dort.“

„In einer Stunde und 52 Minuten“, verbesserte Grunthe
unerschütterlich, „wenn nämlich der Wind mit derselben
Geschwindigkeit anhält.“

„Ja – wenn“, so rief Torm lebhaft. „Nur noch zwei Stunden, Gott
gebe es!“

„Sobald wir über jenen Bergrücken sind, werden wir den Pol sehen.“

„Sie haben recht, Doktor! Sehen werden wir den Pol – ob auch
erreichen?“

„Warum nicht?“ fragte Grunthe.

„Hinter den Bergen, der Himmel gefällt mir nicht – auf der
Nordseite liegt jetzt seit Stunden die Sonne, es ist dort ein
aufsteigender Luftstrom vorhanden –“

„Wir müssen abwarten.“

„Da – da – sehen Sie – den herrlichen Absturz des Gletschers “,
rief Torm.

„Wir fliegen gerade auf ihn zu; müssen wir nicht steigen?“ fragte
Grunthe.

„Gewiß, dort müssen wir hinüber. Aufgepaßt! Schneiden Sie ab!“

Zwei Säcke Ballast klappten herab. Der Ballon schoß in die Höhe.

„Wie die Entfernung täuscht“, sagte Torm. „Ich hätte die Wand für
entfernter gehalten – es reicht noch nicht. Wir müssen noch mehr
opfern.“

Er schnitt noch einen Sack ab.

„Wir dürfen nicht in die Schlucht geraten“, erklärte er, „kein
Mensch weiß, in was für Wirbel wir da kommen. Aber was ist das? Der
Ballon steigt nicht? Es hilft nichts – noch mehr hinaus!“

Eine schwarze Felswand, welche den Gletscher in zwei Teile
spaltete, erhob sich unmittelbar vor ihnen. Der Ballon schwebte in
unheimlicher Nähe. Mit ängstlicher Erwartung verfolgten die beiden
Männer den Flug ihres Aërostaten. Der Südwind war jetzt, zu ihrem
Glück, hier in der unmittelbaren Nähe der Berge schwächer, sonst
wären sie schon an die Felsen geschleudert worden. Der Ballon
befand sich nunmehr im Schatten der Berge; das Gas kühlte sich ab.
Die Temperatur sank schnell tief unter den Gefrierpunkt. Torm
überlegte, ob er noch mehr Ballast auswerfen dürfe. Was er jetzt an
Ballast verlor, das mußte er dann an Gas aufopfern, um den Ballon
wieder zum Sinken zu bringen, und das Gas war sein größter Schatz,
das Mittel, das ihn wieder aus dem Bereich des furchtbaren Nordens
bringen sollte. Er wußte ja nicht, was ihn hinter den Bergen
erwarte. Aber der Ballon stieg zu langsam. Da – eine seitliche
Strömung bewegt ihn – die Strahlen der Sonne, welche über den
Sattel des Gletschers herüberlugt, treffen ihn wieder – das Gas
dehnt sich aus, der Ballon steigt – tiefer und tiefer sinken die
Eismassen unter ihm. –

„Hurrah!“ rufen die beiden Luftschiffer wie aus einem Munde.

„Was gibt’s?“ fährt Saltner aus seinem Schlummer empor. „Sind wir
da?“

„Wollen Sie den Nordpol sehen?“

„Wo? Wo?“ Im Augenblick war Saltner in die Höhe gefahren.

„Sakri, das ist kalt“, rief er.

„Wir sind über 500 Meter gestiegen“, antwortete Torm.

Saltner hüllte sich in seinen Pelz, was die andern schon vorher
getan hatten.

„Wir sind jetzt fast in gleicher Höhe mit dem Kamm des Gebirges.
Sobald wir darüber hinwegsehen können, muß vor uns, etwa 50
Kilometer nach Norden, die Stelle liegen –“

„Wo die Erdachse geschmiert wird!“ rief Saltner. „Ich bin
verteufelt neugierig. Na, den Champagner brauchen wir nicht erst
kalt zu stellen.“

Die drei Männer standen, am Tauwerk sich haltend, in der Gondel.
Mit gespannten Blicken schauten sie jeden Augenblick, den ihnen die
Bedienung des Ballons und die Beobachtung der Instrumente freiließ,
durch ihre Feldstecher nach Norden, der Sonne entgegen, die erst
wenig nach Osten hin beiseite getreten war. Allmählich versanken
die Berggipfel unter ihnen – noch ein breiterer Rücken hemmte ihnen
die Aussicht – der Ballon glitt jetzt wieder in der Höhe des Kammes
dahin, das Schlepptau schleifte –, noch eine breite Mulde war zu
überfliegen, dann mußte das ersehnte Ziel vor ihnen liegen. Der
Ballon befand sich etwa in der Mitte der Mulde, höchstens 100 Meter
über ihrem Boden, und die gegenüberliegende Talwand verdeckte noch
die Aussicht. Der Wind war etwas weniger lebhaft, aber immer noch
südlich, und der Ballon stieg an der flachen Erhebung des Eisfeldes
hinan.

Jetzt wurden einzelne weiße Bergkuppen in großer Entfernung hinter
dem nahen Horizont der gegenüberliegenden Eiswand sichtbar, die
Luftschiffer befanden sich in gleicher Höhe mit dem letzten
Hindernis, das ihren Blick beschränkte. Die Gipfel mehrten sich,
sie bildeten eine Bergkette.

„Diese Berge liegen schon hinter dem Pol“, sagte Grunthe, und
diesmal bebte seine Stimme doch ein wenig vor Aufregung. Fest
preßte er seine Lippen zur geraden Linie zusammen.

Weiter stieg der Ballon – dunkel gefärbte Bergzüge erschienen unter
den Schneegipfeln, rötlich und bräunlich schimmernd – jetzt
erreichte der Ballon die Höhe und schwebte über einem tiefen
Abgrund – das Schleppseil schnellte hinab, und der Ballon sank
sofort einige hundert Meter tief – dann pendelte er noch einmal auf
und ab – diese plötzliche Schwankung des Ballons hatte die
Aufmerksamkeit der Luftschiffer voll in Anspruch genommen – sie
sahen unter sich, tief unten ein wildes Gewirr von Klippen,
Felstrümmern und Eisblöcken, hinter sich die steil abgebrochene
Wand, an welcher der verzerrte Schatten des Ballons auf- und
niederschwankte – die Instrumente mußten beobachtet werden, und
erst jetzt konnten sie den Blick nach vorwärts lenken, vorwärts und
nordwärts – oder war es vielleicht schon südwärts?

Saltner war der erste, der nach vorn blickte. Aber er sprach
nichts, in einem langgedehnten Pfiff blies er den Atem aus seinen
gespitzten Lippen.

„Das Meer!“ rief Torm.

„Grüß Gott!“ sagte jetzt Saltner. „Da hat halt der alte Petermann
doch recht behalten, aber bloß ein bissel. Ein offenes Polarmeer
ist es schon, man muß sich nur nicht zuviel drauf einbilden.“

„Ein Binnenmeer, ein Bassin, immerhin, gegen tausend
Quadratkilometer schätze ich“, sagte Grunthe. „Etwa so groß wie der
Bodensee. Aber wer kann wissen, was sich dort hinten noch an Fjords
und Kanälen abzweigt. Und auch das Bassin selbst ist durch
verschiedene Inseln in Arme geteilt.“

„Wer da unten zu Fuß oder zu Schiff ankommt, muß Mühe haben zu
entscheiden, ob das Meer im Land liegt oder das Land im Meer“,
sagte Saltner. „Gut, daß wir’s bequemer haben.“

„Gewiß“, meinte Torm, „es ist möglich, daß wir ein Stück des
offenen Meeres vor uns haben, obwohl es von hier den Anschein hat,
als schlössen die Berge das Wasser von allen Seiten ein. Wir werden
ja sehen. Aber vor allen Dingen, was sollen wir tun? Wir haben
wider Erwarten so hoch steigen müssen, daß wir jetzt sehr viel Gas
verlieren würden, wenn wir hinabwollten, und andrerseits werden wir
wieder drüben über die Berge hinaufmüssen. Es ist eine schwierige
Frage. Aber wir haben noch Zeit, darüber nachzudenken, denn der
Ballon bewegt sich jetzt nur langsam.“

„Und diese Gelegenheit wollen wir benutzen, um dem Nordpol unsern
wohlverdienten Gruß zu bringen“, rief Saltner. Mit diesen Worten
zog er ein Futteral hervor, aus welchem drei Flaschen Champagner
ihre silbernen Hälse einladend hervorstreckten.

„Davon weiß ich ja gar nichts“, sagte Torm fragend.

„Das ist eine Stiftung von Frau Isma. Sehen Sie, es steht darauf:
›Am Pol zu öffnen. Gewicht vier Kilogramm.‹“

Torm lachte. „Dachte ich mir doch“, sagte er, „daß meine Frau
irgend etwas einschmuggeln würde, was das Expeditionsreglement
durchbricht.“

„Es ist doch aber auch ein herrlicher Gedanke von Ihrer Frau, sich
am Nordpol in Champagner hochleben zu lassen“, erwiderte Saltner.
„Erstens für sich selbst, denn das ist etwas, was noch nicht
dagewesen ist; das müssen Sie zugeben, Damen sind hier noch niemals
leben gelassen worden. Und zweitens für uns, das müssen Sie auch
zugeben; es ist sehr wonnig, in dieser Kälte den Schaumwein zu
trinken auf das Wohl unserer Kommandeuse. Und drittens, ist es
nicht einfach bejauchzbar, das tragische Antlitz unseres Astronomen
zu sehen? Denn Champagner trinkt er prinzipiell nicht, und auf
weibliche Wesen stößt er prinzipiell nicht an; da er aber auf dem
Nordpol prinzipiell in ein Hoch einstimmen muß und will, so findet
er sich in einem Widerstreit der Prinzipien, aus dem herauszukommen
ihm verteufelt schwerfallen wird.“

„Darauf könnte ich sehr viel erwidern“, sagte Grunthe. „Zum
Beispiel, daß wir noch gar nicht wissen, wo der Nordpol eigentlich
liegt.“

„Schon wahr“, unterbrach ihn Torm, „aber eben darum müssen wir den
Moment feiern, in welchem wir sicher sind, ihn zum erstenmal in
unserm Gesichtsfeld zu haben. Das werden Sie zugeben?“

„Hm, ja“, sagte Grunthe, und ein leichtes Schmunzeln glitt über
seine Züge. „Ich nehme an, wir wären am Pol. So kann ich mit Ihnen
anstoßen, oder auch nicht, ganz wie ich will, ohne mit
irgendwelchen Prinzipien in Widerspruch zu geraten.“

„Wieso?“ fragte Saltner.

„Der Pol ist ein Unstetigkeitspunkt. Prinzipien sind Grundsätze,
die unter der Voraussetzung gelten, daß die Bedingungen bestehen,
für welche sie aufgestellt sind, vor allem die Stetigkeit der Raum-
und Zeitbestimmungen. Am Pol sind alle Bedingungen aufgehoben. Hier
gibt es keine Himmelsrichtungen mehr, jede Richtung kann als Nord,
Süd, Ost oder West bezeichnet werden. Hier gibt es auch keine
Tageszeit; alle Zeiten, Nacht, Morgen, Mittag und Abend, sind
gleichzeitig vorhanden. Hier gelten also auch alle Grundsätze
zusammen oder gar keine. Es ist der vollständige Indifferenzpunkt
aller Bestimmungen erreicht, das Ideal der Parteilosigkeit.“

„Bravo“, rief Saltner, der inzwischen die Trinkbecher von Aluminium
mit dem perlenden Wein gefüllt hatte. „Es lebe Frau Isma Torm,
unsere gnädige Spenderin!“

Saltner und Torm erhoben ihre Becher. Grunthe kniff die Lippen
zusammen und hielt, geradeaus starrend, sein Trinkgefäß unbeweglich
vor sich hin, indem er es passiv geschehen ließ, daß die andern mit
ihren Bechern daran stießen. Nun rief Torm:

„Es lebe der Nordpol!“

Da stieß auch Grunthe seinen Becher lebhaft mit den andern zusammen
und setzte hinzu:

„Es lebe die Menschheit!“

Sie tranken und Saltner rief:

„Grunthes Toast ist so allgemein, daß ein Becher nicht reichen
kann.“ Und er schenkte noch einmal ein.

Inzwischen war der Ballon langsam dem Binnenmeer entgegengetrieben,
das sich nun immer deutlicher den staunenden Blicken der Reisenden
enthüllte. Vom Fuß der steil abfallenden Felsenwand des Gebirges ab
senkte sich das Gelände allmählich, wohl noch eine Strecke von
einigen zwanzig Kilometern weit, nach dem Ufer hin. Aber die
Landschaft zeigte jetzt ein vollständig anderes Gepräge. Die wilde
Gletschernatur war verschwunden, grüne Matten zogen sich, nur noch
mit einzelnen Gesteinstrümmern hier und da bedeckt, in sanfter
Senkung dem Wasser zu. Man glaubte in ein herrliches Alpental zu
schauen, in dessen Mitte ein blauer Bergsee sich ausbreitete. An
dem jenseitigen, entfernten Ufer, das freilich in undeutlichem
Dämmer verschwamm, schien dagegen wieder ein Steilabfall von Fels
und Eis zu herrschen, doch zog sich über den Bergen dort eine
Wolkenwand empor. Das Auffallendste in der ganzen Szenerie aber bot
der Anblick einer der Inseln, die zahlreich und in unregelmäßiger
Gestaltung in dem Bassin lagen, bis an dessen Ufer der Ballon jetzt
herangeschwebt war. Sie war kleiner als die Mehrzahl der übrigen
Inseln. Aber ihre Formen waren so vollkommen regelmäßig, daß es
zweifelhaft schien, ob man eine Gestaltung der Natur vor sich habe.
Die mit Flechten bekleideten Felstrümmer, welche die andern Inseln
bedeckten, fehlten hier vollständig.

Die Forscher mochten sich etwa noch zwölf Kilometer von der
rätselhaften Insel entfernt befinden, die sie mit ihren Ferngläsern
musterten, als Torm sich an Grunthe wandte.

„Sagen Sie uns, bitte, Ihre Meinung. Können wir eigentlich
bestimmen, wo wir uns befinden? Ich muß gestehen, daß ich beim
Überschreiten des Gebirges und dem raschen Höhenwechsel nicht mehr
imstande war, die einzelnen Landmarken zu verfolgen.“

„Ich habe“, erwiderte Grunthe, „einige Peilungen gemacht, aber zu
einer sicheren Bestimmung reichen sie nicht mehr aus. Auch die
Methode aus der Messung der Sonnenhöhe ist jetzt nicht anwendbar,
da wir nicht mehr imstande sind, die Tageszeit auch nur mit einiger
Sicherheit anzugeben. Wir haben die Himmelsrichtung vollständig
verloren. Der Kompaß ist ja hier im Norden sehr unzuverlässig. Auf
alle Fälle sind wir ganz nahe am Pol, wo alle Meridiane so nah
zusammenlaufen, daß eine Abweichung von einem Kilometer nach rechts
oder links einen Zeitunterschied von einer Stunde oder mehr
ausmacht. Wenn unser Ballon aus der Nord-Süd-Richtung vielleicht
seit der Überschreitung des Gebirges um fünf oder sechs Kilometer
abgewichen ist, was sehr leicht sein kann, so haben wir jetzt
nicht, wie wir vermuten, drei Uhr morgens am 19. August, sondern
vielleicht schon Mittag, oder, wenn wir nach Westen abgewichen
sind, so sind wir sogar in den gestrigen Tag zurückgeraten und
haben vielleicht erst den 18. August abends.“

„Das wäre der Teufel“, rief Saltner. „Das kommt von diesem ewigen
Sonnenschein am Pol! Nun kann ich an meinem Abreißkalender das
Blatt von gestern wieder ankleben.“

„Schon möglich!“ lächelte Grunthe. „Nehmen Sie an, Sie machen einen
Spaziergang um den Nordpol in der Entfernung von hundert Metern vom
Pol, so sind Sie in fünf Minuten bequem um den Pol herumgegangen
und haben sämtliche 360 Meridiane überschritten; Sie haben also in
fünf Minuten alle Tageszeiten abgelaufen. Gehen Sie nach Westen
herum, und wollen Sie die richtige Zeit jedes Meridians haben, so
müßten Sie auf jedem Meridian Ihre Uhr um 4 Minuten zurückstellen,
so daß Sie nach besagten fünf Minuten um einen vollen Tag zurück
sind, und wenn Sie in dieser Art eine Stunde lang um den Pol
herumgegangen sind, so muß Ihre Uhr, wenn sie einen Datumzeiger
besitzt, den 7. August anzeigen.“

„Da muß ich mir halt einen Anklebekalender anschaffen“, meinte
Saltner.

„Ja, aber wenn Sie nach Osten herumgehen, kommen Sie um ebensoviel
in der Zeit voran, Sie hätten dann nach zwölfmaligem Spaziergang um
den Pol den 31. August erreicht, wenn Sie bei jedem Umgehen des
Pols ein Blatt in ihrem Kalender abrissen. In beiden Fällen würden
sie sich indessen tatsächlich noch am 19. August befinden. Sie
müßten also, wie die Seefahrer beim Überschreiten des 180.
Meridians, ihren Datumzeiger entsprechend regulieren.“

„Und wenn wir nun gerade über den Pol wegfliegen?“

„Dann springen wir in einem Moment um zwölf Stunden in der Zeit.
Der Pol ist eben ein Unstetigkeitspunkt.“

„Sackerment, da weiß man ja gar nicht, wo man ist.“

„Ja“, sagte Torm, „das ist eben das Fatale. Wir haben uns von
Anfang an darauf verlassen müssen, daß wir unsere Lage aus dem
zurückgelegten Wege bestimmen. Läßt sich denn gar nichts tun?“

„Nur wenn wir landen und unsere Instrumente so fest aufstellen, daß
wir einige Sterne anvisieren können.“

„Daran können wir auf keinen Fall eher denken, bis wir den See
überflogen haben und das jenseitige Gebirge überschauen. Hier
zwischen den Inseln dürfen wir uns nicht hinabwagen. Wir sind also
wirklich nicht besser daran als unsere Vorgänger, und der wahre Pol
bleibt wieder unbestimmt.“

„Zu verflixt“, brummte Saltner, „da sind wir vielleicht gerade am
Nordpol und wissen es nicht.“

\section{2 - Das Geheimnis des Pols}

Langsam zog der Ballon weiter, doch bewegte er sich nicht direkt
auf die auffallende kleine Insel zu, sondern sie blieb rechts von
seiner Fahrtrichtung liegen.

Während Grunthe die Landmarken aufnahm und Torm die Instrumente
ablas, suchte Saltner, dem die photographische Festhaltung des
Terrains oblag, die Gegend mit seinem vorzüglichen Abbéschen
Relieffernrohr ab. Dasselbe gab eine sechzehnfache Vergrößerung und
ließ, da es die Augendistanz verzehnfachte, die Gegenstände in
stereoskopischer Körperlichkeit erscheinen. Sie hatten sich jetzt
der Insel soweit genähert, daß es möglich gewesen wäre, Menschen,
falls sich solche dort hätten befinden können, mit Hilfe des
Fernrohrs wahrzunehmen.

Saltner schüttelte den Kopf, sah wieder durch das Fernrohr, setzte
es ab und schüttelte wieder den Kopf.

„Meine Herren“, sagte er jetzt, „entweder ist mir der Champagner in
den Kopf gestiegen –“

„Die zwei Glas, Ihnen?“ fragte Torm lächelnd.

„Ich glaub es auch nicht, also – oder –“

„Oder? Was sehen Sie denn?“

„Es sind schon andere vor uns hier gewesen.“

„Unmöglich!“ riefen Torm und Grunthe wie aus einem Munde.

„Die bisherigen Berichte wissen nichts von einer derartigen Insel –
unsere Vorgänger sind offenbar gar nicht über das Gebirge
gekommen“, fügte Torm hinzu.

„Sehen Sie selbst“, sagte Saltner und gab das Fernrohr an Torm. Er
selbst und Grunthe benutzten ihre kleineren Feldstecher. Torm
blickte gespannt nach der Insel, dann wollte er etwas sagen, zuckte
aber nur mit den Lippen und blieb völlig stumm.

Saltner begann wieder: „Die Insel ist genau kreisförmig – das haben
wir schon bemerkt. Aber jetzt sehen Sie, daß gerade im Zentrum sich
wieder ein dunkler Kreis von – sagen wir – vielleicht hundert
Metern Durchmesser befindet.“

„Allerdings“, sagte Grunthe, „aber es ist nicht nur ein Kreis,
sondern eine zylindrische Öffnung, wie man jetzt deutlich sehen
kann. Und um den Rand derselben führt eine Art Brüstung.“

„Und nun suchen Sie einmal den Rand der Insel ab. Was sehen Sie?“

„Mein Glas ist zu schwach, um Einzelheiten zu erkennen.“

„Ich habe gesehen, was Sie wahrscheinlich meinen“, sagte Torm.

„Aber was ist das“, unterbrach er sich, „der Ballon ändert seine
Richtung?“

Er gab das Glas an Grunthe und wandte seine Aufmerksamkeit dem
Ballon zu. Dieser wich nach rechts von seinem bisherigen Kurse ab.
Er bewegte sich parallel mit dem Ufer der Insel, diese in sich
gleichbleibender Entfernung umkreisend.

„Wir wollen uns überzeugen, daß wir dasselbe meinen“, sagte
Grunthe. „Rings um die Insel zieht sich ein Kreis von pfeiler- oder
säulenartigen Erhöhungen in gleichen Abständen.“

„Es stimmt“, sagten die andern.

„Ich habe sie gezählt“, bemerkte Torm, „es sind zwölf große,
dazwischen je elf kleinere, im ganzen hundertvierundvierzig.“

„Und der seltsame Reflex über der ganzen Insel?“

„Wissen Sie, es sieht aus, als wäre die ganze Insel mit einem Netz
von spiegelnden metallischen Drähten oder Schienen überzogen, die
wie die Speichen eines Rades vom Zentrum nach der Peripherie
laufen.“

„Ja“, sagte Torm, indem er sich einen Augenblick erschöpft
niedersetzte, „und Sie werden gleich noch mehr sehen, wenn Sie
länger hinschauen. Ich will es Ihnen sagen.“ Seine Stimme klang
rauh und heiser. „Was Sie dort sehen, ist der Nordpol der Erde –
aber, wir haben ihn nicht entdeckt.“

„Das fehlte gerade“, fuhr Saltner auf. „Dafür sollten wir uns in
diesen pendelnden Frierkasten gesetzt haben? Nein, Kapitän,
entdeckt haben wir ihn, und was wir da sehen, ist kein
Menschenwerk. So verrückt wäre doch kein Mensch, hier Drähte zu
spannen! Eher will ich glauben, daß die Erdachse in ein großes
Velozipedrad ausläuft, und daß wir wahrhaftig berufen sind, sie zu
schmieren! Nur nicht den Mut verlieren!“

„Wenn es nicht Menschen sind“, sagte Torm tonlos, „und ich weiß
auch nicht, wie Menschen dergleichen machen sollten, und warum, und
wo sie herkämen – das hätte man doch erfahren – so – eine Täuschung
ist es doch nicht – so steht mir der Verstand still.“

„Na“, sagte Saltner, „Eisbären werden’s nicht gemacht haben,
obgleich ich mich jetzt über nichts mehr wundern würde, und wenn
gleich ein geflügelter Seehund käme und ›Station Nordpol‹ ausriefe.
Aber es könnte doch vielleicht eine Naturerscheinung sein, ein
merkwürdiger Kristallisationsprozeß – – Sakri! Jetzt hab ich’s. Das
ist ein Geysir! Ein riesiger Geysir!“

„Nein, Saltner“, erwiderte Torm, „das habe ich auch schon gedacht –
ein Schlammvulkan könnte etwa eine ähnliche Bildung zeigen. Aber –
Sie haben wohl das Eigentliche, die Hauptsache, das – Unerklärliche
noch nicht gesehen –“

„Was meinen Sie?“

„Ich hab’ es gesehen“, sagte jetzt Grunthe. Er setzte das Fernrohr
ab. Dann lehnte er sich zurück und runzelte die Stirn. Auch um die
fest zusammengezogenen Lippen bildeten sich Falten, daß sein Mund
aussah wie ein in Klammern gesetztes Minuszeichen. Er versank in
tiefes, sorgenvolles Nachdenken.

Saltner ergriff das Glas.

„Achten Sie auf die Färbungen am Boden der ganzen Insel!“ sagte
Torm zu ihm.

„Es sind Figuren!“ rief Saltner.

„Ja“, sagte Torm. „Und diese Figuren stellen nichts anderes dar als
ein genaues Kartenbild eines großen Teils der nördlichen
Halbkugel der Erde in perspektivischer Polarprojektion. Sie sehen
deutlich den Verlauf der grönländischen Küste, Nordamerika, die
Beringstraße, Sibirien, ganz Europa – mit seinen unverkennbaren
Inseln und Halbinseln, das Mittelmeer bis zum Nordrand von Afrika,
wenn auch stark verkürzt.“

„Es ist kein Zweifel“, sagte Saltner. „Die ganze Umgebung des Pols
ist in einem deutlichen Kartenbild in kolossalem Maßstab hier
abgezeichnet, und zwar bis gegen den 30. Breitengrad.“

„Und wie ist das möglich?“

Die Frage fand keine Antwort. Alle schwiegen.

Inzwischen hatte der Ballon eine fast vollständige Umkreisung der
Insel vollzogen. Aber er hatte sich derselben auch noch um ein
Stück genähert. Es war klar, daß er durch eine unbekannte Kraft,
wohl durch eine wirbelförmige Bewegung der Luft, um die Insel
herumgeführt und zugleich nach der Achse des Wirbels, die von der
Mitte der Insel ausgehen mochte, zu ihr hingezogen wurde.

Torm unterbrach das Schweigen. „Wir müssen einen Entschluß fassen“,
sagte er. „Wollen die Herren sich äußern.“

„Ich will zunächst einmal“, begann Saltner, „diese merkwürdige
Erdkarte photographieren. Sie scheint ziemlich richtig selbst in
Details zu sein. Daß sie nicht von Menschenhand herrühren kann,
sehen wir daraus daß auch die noch unbekannten Gegenden des
Polargebietes dargestellt sind. Die innere Öffnung, bei welcher die
Karte abbricht, entspricht in ihrem Umfange etwa dem 86.
Breitengrade; es fehlen also – für uns leider – die nächsten vier
Grade um den Pol herum.“

„Selbstverständlich“, sagte Torm, „müssen Sie die Karte
photographieren. Wir dürfen nicht mehr zweifeln, ein Werk
intelligenter Wesen vor uns zu haben, wenn ich mir auch nicht
erklären kann, wer diese sein mögen. Aber wenn das richtig ist, was
wir kontrollieren können, so müssen wir schließen, daß auch die
Teile des Polargebietes nach den Nordküsten von Amerika und
Sibirien hin zuverlässig dargestellt sind. Und dann hätten wir mit
einem Schlage eine vollständige Karte dieses bisher unerforschten
Polargebietes.“

„Nun, ich denke, wir können mit diesem Erfolg schon zufrieden sein.
Und bedenken Sie, wie nützlich die Karte für unsere Rückkehr werden
kann. So –“, damit brachte Saltner die photographische Kammer
wieder an ihren Platz, „ich habe drei sichere Aufnahmen. Aber der
Ballon bewegt sich ja schneller?“

„Ich glaube auch“, sagte Torm. „Ich bitte nun um die Meinung der
Herren, sollen wir eine Landung auf der Insel wagen, um dieses
Geheimnis zu erforschen?“

„Ich meine“, äußerte sich Saltner, „wir müssen es versuchen. Wir
müssen zusehen, mit wem wir es hier zu tun haben.“

„Gewiß“, sagte Torm, „die Aufgabe ist verlockend. Aber es ist zu
befürchten, daß wir zuviel Gas verlieren, daß wir vielleicht die
Möglichkeit aufgeben, den Ballon weiter zu benutzen. Was meinen
Sie, Dr. Grunthe?“

Grunthe richtete sich aus seinem Nachsinnen auf. Er sprach sehr
ernst: „Unter keinen Umständen dürfen wir landen. Ich bin sogar der
Ansicht, daß wir alle Anstrengungen machen müssen, um uns so
schnell wie möglich von diesem gefährlichen Punkt zu entfernen.“

„Worin sehen Sie die Gefahr?“

„Nachdem wir die eigentümliche Ausrüstung des Pols und die
Abbildung der Erdoberfläche gesehen haben, ist doch kein Zweifel,
daß wir einer gänzlich unbekannten Macht gegenüberstehen. Wir
müssen annehmen, daß wir es mit Wesen zu tun bekommen, deren
Fähigkeiten und Kräften wir nicht gewachsen sind. Wer diesen
Riesenapparat hier in der unzugänglichen Eiswüste des Polargebiets
aufstellen konnte, der würde ohne Zweifel über uns nach Gutdünken
verfügen können.“

„Nun, nun“, sagte Torm, „wir wollen uns darum nicht fürchten.“

„Das nicht“, erwiderte Grunthe, „aber wir dürfen den Erfolg unserer
Expedition nicht aufs Spiel setzen. Vielleicht liegt es im
Interesse dieser Polbewohner, den Kulturländern keine Nachricht von
ihrer Existenz zukommen zu lassen. Wir würden dann ohne Zweifel
unsere Freiheit verlieren. Ich meine, wir müssen alles daransetzen,
das, was wir beobachtet haben, der Wissenschaft zu übermitteln und
es dann späteren Erwägungen überlassen, ob es geraten scheint und
mit welchen Mitteln es möglich sei, das unerwartete Geheimnis des
Pols aufzulösen. Wir dürfen uns nicht als Eroberer betrachten,
sondern nur als Kundschafter.“

Die andern schwiegen nachdenklich. Dann sagte Torm:

„Ich muß Ihnen recht geben. Unsere Instruktion lautet allerdings
dahin, eine Landung nach Möglichkeit zu vermeiden. Wir sollen mit
möglichstes Eile in bewohnte Gegenden zu gelangen suchen, nachdem
wir uns dem Pol soweit wie angänglich genähert und seine Lage
festgestellt haben, und wir sollen versuchen, einen Überblick über
die Verteilung von Land und Wasser vom Ballon aus zu gewinnen.
Dieser Gesichtspunkt muß entscheidend sein. Wir wollen also
versuchen, von hier fortzukommen.“

„Aber nach welcher Richtung?“ fragte Saltner. „Darüber könnte uns
die Polarkarte der Insel Auskunft geben.“

„Ich fürchte“, entgegnete Torm, „von unserm guten Willen wird dabei
sehr wenig abhängen. Wir müssen abwarten, was der Wind über uns
beschließen wird. Zunächst lassen Sie uns versuchen, diesem Wirbel
zu entfliehen.“

Inzwischen hatte sich der Ballon noch mehr der Insel genähert, und
seine Geschwindigkeit begann zu wachsen. Zugleich aber erhob er
sich weiter über den Erdboden.

Die Luftschiffer spannten nun das Segel auf und gaben ihm eine
solche Stellung, daß der Widerstand der Luft sie nach der
Peripherie des Wirbels treiben mußte. Da aber der Ballon viel zu
hoch schwebte, als daß das Schleppseil seine hemmende Wirkung hätte
ausüben können, so mußte das Manöver zuerst versagen. In immer
engeren Spirallinien aufsteigend näherte sich der Ballon dem
Zentrum des Wirbels und vermehrte seine Geschwindigkeit. In großer
Besorgnis verfolgten die Luftschiffer den Vorgang. Sie beeilten
sich, die Länge des Schlepptaus zu vergrößern. Ihre vorzügliche
Ausrüstung gestattete ihnen, ein Schlepptau von tausend Metern
Länge zu verwenden, an welches noch ein hundertundfünfzig Meter
langer Schleppgurt mit Schwimmern kam. Aber auch diese stattliche
Ausdehnung des Seiles reichte nicht bis auf die Oberfläche des
Wassers.

„Es bleibt nichts übrig“, rief Torm endlich, „wir müssen weiter
niedersteigen.“

Er öffnete das Manöverventil. Das Gas strömte aus. Der Ballon
begann zu sinken.

„Wir wollen aber“, sagte Torm, „da wir nicht wissen, wie wir hier
davonkommen, doch versuchen, eine Nachricht nach Hause zu geben.
Lassen Sie uns einige unserer Brieftauben absenden. Jetzt ist der
geeignete Moment. Was wir gesehen haben, muß man in Europa
erfahren.“

Eilends schrieb er die nötigen Notizen auf den schmalen Streifen
Papier, den er zusammenrollte und in der Federpose versiegelte,
welche den Brieftauben angeheftet wurde.

Saltner gab den Tierchen die Freiheit. Sie umkreisten wiederholt
den Ballon und entfernten sich dann in einer Richtung, die von der
Insel fortführte.

Torm schloß das Ventil wieder. Sie mußten jetzt jeden Augenblick
erwarten, daß das Ende des Schlepptaus die Oberfläche des Wassers
berühre. Der Ballon näherte sich seiner Gleichgewichtslage.

Grunthe blickte durch das Relieffernrohr direkt nach unten, da es
durch dieses Instrument möglich war, den breiten Sackanker am Ende
des Schleppgurts zu sehen und den Abstand desselben vom Boden zu
schätzen. Plötzlich griff er mit größter Hast zur Seite, erfaßte
den nächsten Gegenstand, der ihm zur Hand war – es war das Futteral
mit den beiden noch gefüllten Champagnerflaschen – und schleuderte
es in großem Bogen zum Korbe hinaus.

„Sakri, was fällt ihnen ein“, rief Saltner entrüstet, „werfen da
unsern saubern Wein ins Wasser.“

„Entschuldigen Sie“, sagte Grunthe, indem er sich aus seiner
gebückten Stellung aufrichtete, da er an der Bewegung der Wimpel
bemerkte, daß der Ballon wieder im Steigen begriffen war.
„Entschuldigen Sie, aber das Fernrohr konnte ich doch nicht
hinauswerfen, und es war keine halbe Sekunde zu verlieren – wir
wären wahrscheinlich verloren gewesen.“

„Was gab es denn?“ fragte Torm besorgt.

„Wir sind nicht mehr über dem Wasser, sondern bereits am Rande der
Insel. Das Ende des Seils war wohl kaum weiter als zehn Meter von
der Oberfläche der Insel entfernt. Wir hätten sie berührt, wenn
nicht das Sinken des Ballons momentan aufgehört hätte.
Glücklicherweise genügten die Flaschen, unsern Fall aufzuhalten.“

„Und glauben Sie denn, daß wir die Insel nicht berühren dürfen?“

„Ich glaube es nicht, ich weiß es.“

„Wieso?“

„Wir wären hinabgezogen worden.“

„Ich kann noch nicht einsehen, woraus Sie das schließen.“

„Sie haben mir doch beigestimmt“, sagte Grunthe, „daß wir es nicht
darauf ankommen lassen dürfen, in die Macht der unbekannten Wesen –
sie mögen nun sein, wer sie wollen – zu geraten, welche diesen
unerklärlichen Apparat und diese Kolossalkarte am Nordpol
hergestellt haben. Es ist aber wohl keine Frage, daß dieser
Apparat, an den wir mehr und mehr herangezogen werden, nicht sich
selbst überlassen hier stehen wird. Sicherlich ist die Insel
bewohnt, es befinden sich die geheimnisvollen Erbauer
wahrscheinlich in oder unter jenen Dächern und Pfeilern, die wir
mit unsern Fernrohren nicht durchdringen können. Es ist anzunehmen,
daß sie unsern Ballon längst bemerkt haben, und so schließe ich
denn, daß sie denselben sofort zu sich hinabziehen würden, sobald
unser Schleppseil in das Bereich ihrer Arme gelangt.“

„Gott sei Dank“, rief Saltner, „daß Sie den dunkeln Polgästen
wenigstens Arme zusprechen; es ist doch schon ein menschlicher
Gedanke, daß man ihnen zur Not in die Arme fallen kann.“

Torm unterbrach ihn. „Ich kann mich immer noch nicht recht dazu
verstehen“, sagte er, „an eine solche überlegene Macht zu glauben.
Das widerspräche ja doch allem, was bisher in der Geschichte der
Polarforschung, ja der Entdeckungsreisen überhaupt vorgekommen ist.
Freilich die Karte –, aber was denken Sie überhaupt über diese
Insel? Sie sprachen von einem Apparat, so ein Apparat müßte doch
einen Zweck haben–“

„Den wird er ohne Zweifel haben, wir sind nur nicht in der Lage,
ihn zu kennen oder zu begreifen. Denken Sie, daß Sie einen Eskimo
vor die Dynamomaschine eines Elektrizitätswerks stellen; daß das
Ding einen Zweck hat, wird er sich sagen, aber was für einen, das
wird er nie erraten. Wie soll er begreifen, daß die Drähte, die von
hier ausgehen, ungeheure Energiemengen auf weite Strecken
verteilen, daß sie dort Tageshelle erzeugen, dort schwere Wagen mit
Hunderten von Menschen mit Leichtigkeit hingleiten lassen? Wenn der
Eskimo sich über die Dynamomaschine äußert, so wird es jedenfalls
eine so kindische Ansicht sein, daß wir sie belächeln. Und um nicht
diesem unbekannten Apparat gegenüber die Rolle des Eskimo zu
spielen, will ich mich lieber gar nicht äußern.“

Torm schwieg nachdenklich. Dann sagte er:

„Was mich am meisten beunruhigt, ist diese unerklärliche
Anziehungskraft, die die Achse der Insel auf unsern Ballon ausübt.
Und sehen Sie, seitdem wir kein Gas mehr ausströmen lassen, beginnt
der Ballon wieder rapid zu steigen. Dabei wird er fortwährend um
das Zentrum der Insel herumgetrieben.“

„Und wer sagt Ihnen, was geschieht, wenn wir in die Achse selbst
geraten? Ich halte unsere Situation für geradezu verzweifelt, aus
dem Wirbel können wir nur heraus, wenn wir uns sinken lassen. Dann
aber geraten wir in die Macht der unbekannten Insulaner.“

„Und dennoch“, sagte Torm, „werden wir uns entschließen müssen.“

Alle drei schwiegen. Mit düsteren Blicken beobachteten Torm und
Grunthe die Bewegungen des Ballons, während Saltner die Insel mit
dem Fernrohr untersuchte. Mehr und mehr verschwanden die Details,
die vorher deutlich sichtbar waren, ein Zeichen, daß der Ballon mit
großer Geschwindigkeit stieg, auch wenn die Instrumente, ja selbst
die zunehmende Kälte, dies nicht angezeigt hätten.

Da – was war das? – auf der Insel zeigte sich eine Bewegung, ein
eigentümliches Leuchten. Saltner rief die Gefährten an. Sie
blickten hinab, konnten aber mit ihren schwächeren Instrumenten nur
bemerken, daß sich helle Punkte vom Zentrum nach der Peripherie hin
bewegten. Saltner schien es durch sein starkes Glas, als wenn eine
Reihe von Gestalten mit weißen Tüchern winkende Bewegungen
ausführte, die alle vom Innern der Insel nach außen hin wiesen.

„Man gibt uns Zeichen“, sagte er. „Sehen Sie hier durch das starke
Glas!“

„Das kann nichts anderes bedeuten“, rief Torm, „als daß wir uns von
der Achse entfernen sollen. Aber so klug sind wir selbst – wir
wissen nur nicht wie.“

„Wir müssen das Entleerungs-Ventil öffnen“, sagte Saltner.

„Dann ergeben wir uns auf Gnade und Ungnade“, rief Grunthe.

„Und doch wird uns nichts übrig bleiben“, bemerkte Torm.

„Und was schadet es?“ fragte Saltner. „Vielleicht wollen jene Wesen
nur unser Bestes. Würden sie uns sonst warnen?“

„Wie dem auch sei – wir dürfen nicht höher steigen“, sagte Torm.
„Wir werden ja geradezu in die Höhe gerissen.“

Schon hatten sich alle dicht in ihre Pelze gewickelt.

„Warten wir noch“, sagte Grunthe, „wir sind immer noch gegen
hundert Meter von der Achse der Insel entfernt. Die Trübung hat
sich genähert, wir kommen in eine Wolkenschicht. Vielleicht gelangt
doch der Ballon endlich ins Gleichgewicht.“

„Unmöglich“, entgegnete Torm. „Wir haben bereits gegen 4.000 Meter
erreicht. Der Ballon war im Gleichgewicht, als das Gewicht des
Futterals mit den Champagnerflaschen seine Bewegung zu ändern
vermochte. Wenn er jetzt mit solcher Geschwindigkeit steigt, so ist
das ein Zeichen, daß uns eine äußere Kraft in die Höhe führt, die
um so stärker wird, je mehr wir uns dem Zentrum nähern.“

„Ich muß es zugeben“, sagte Grunthe. „Es ist gerade, als wenn wir
uns in einem Kraftfeld befänden, das uns direkt von der Erde
abstößt. Sollen wir einen Versuchsballon ablassen?“

„Kann uns nichts Neues mehr sagen – es ist zu spät. Da – wir sind
in den Wolken.“

„Also hinunter!“ rief Saltner.

Torm riß das Landungsventil auf.

Der Ballon mäßigte seine aufsteigende Bewegung, aber zu sinken
begann er nicht.

Die Blicke der Luftschiffer hingen an den Instrumenten. Wenige
Minuten mußten ihr Schicksal entscheiden. Das Gas strömte in die
verdünnte Luft mit großer Gewalt aus. Brachte dies den Ballon nicht
bald zum Sinken, so war es klar, daß sie die Herrschaft über das
Luftmeer verloren hatten. Sie befanden sich dann einer Gewalt
gegenüber, die sie, unabhängig von dem Gleichgewicht ihres Ballons
in der Atmosphäre, von der Erde forttrieb.

Und der Ballon sank nicht. Eine Zeitlang schien es, als wollte er
sich auf gleicher Höhe halten, aber die wirbelnde Bewegung hörte
nicht auf, die ihn der Achse der Insel entgegentrieb. Diese Achse,
daran war ja kein Zweifel, war nichts anderes als die Erdachse
selbst, jene mathematische Linie, um welche die Rotation der Erde
erfolgt. Immer stärker wurden sie zu ihr hingezogen. Aber je näher
sie ihr kamen, um so heftiger wurde der Ballon noch oben gedrängt.
Schon begannen sich die körperlichen Beschwerden einzustellen,
welche die Erhebung in die verdünnten Luftschichten begleiten. Alle
klagten über Herzklopfen. Saltner mußte das Fernrohr hinlegen, vor
seinen Augen verschwammen die Gegenstände. Atemnot stellte sich
ein.

„Es bleibt nichts andres übrig“, rief Torm. „Die Reißleine!“

Grunthe ergriff die Reißleine. Die Zerreißvorrichtung dient dazu,
einen Streifen der Ballonhülle in der Länge des sechsten Teils des
Ballonumfangs aufzureißen, um den Ballon im Notfall binnen wenigen
Minuten des Gases zu entleeren. Aber – die Vorrichtung versagte! Er
zerrte an der Leine – sie gab nicht nach. Sie mußte sich am
Netzwerk des Ballons verfangen haben. Es war jetzt unmöglich, den
Schaden zu reparieren. Der Ballon stieg weiter. Von der Erde war
nichts mehr zu sehen, man blickte auf Wolken.

„Die Sauerstoffapparate!“ kommandierte Torm.

Obwohl man die Absicht hatte, sich stets in geringer Höhe zu
halten, konnte man doch nicht wissen, ob nicht die Umstände ein
Aufsteigen in die höchsten Regionen mit sich bringen wurden. Für
diesen Fall hatte man sich mit komprimiertem Sauerstoff zur Atmung
versehen. Es war jetzt notwendig, die künstliche Atmung
anzuwenden.

Die Forscher fühlten sich neu gestärkt; aber immer furchtbarer
wurde die Kälte. Sie merkten, wie ihre Gliedmaßen zu erstarren
drohten. Die Nase, die Finger wurden gefühllos, sie versuchten
ihnen durch Reiben den Blutzufluß wieder zuzuführen. Der Ballon
stieg rettungslos weiter, und zwar immer schneller, je mehr er sich
dem Zentrum näherte. Siebentausend – achttausend – neuntausend
Meter zeigte das Barometer im Verlauf einer Viertelstunde an. Die
größte Höhe, welche je von Menschen erreicht worden war, wurde nun
überschritten.

Untätig saßen die Männer zusammengedrängt – sie hatten den
künstlichen Verschluß der Gondel hergestellt, da sie nichts mehr am
Ballon ändern konnten. Sie vermochten nichts zu tun, als sich gegen
die Kälte zu schützen. Kein Mittel der Rettung zeigte sich – ihre
Tatkraft begann unter dem Einfluß der vernichtenden Kälte zu
erlahmen. Der Flug in die Höhe war unhemmbar – nichts mehr konnte
sie retten vor dem Erfrieren – oder vor dem Ersticken. – Was würde
geschehen? Es war ja gleichgültig.

Und doch, immer wieder raffte sich der eine oder andere mit
Anstrengung aller Willenskräfte auf – noch ein Blick auf die
Instrumente – die Thermometer waren längst eingefroren – und – kaum
glaublich – das Barometer zeigte einen Druck von nur noch 50
Millimeter, das heißt, sie befanden sich zwanzig Kilometer über der
Erdoberfläche. Und jetzt – schien es nicht, als käme der Ballon zu
ihnen herab? Die entleerte Seidenhülle senkte sich über die Gondel
– die Gondel flog schneller als der Ballon – wie aus einer Kanone
geschossen fuhr sie in die Seide des Ballons hinein, die Insassen
der Gondel waren verstrickt in das Gewirr von Stoff und Seilen –
halb schon bewußtlos bemerkten sie kaum noch den Stoß der sie traf
– sie waren in die Achse des von der Insel ausgehenden Wirbels
geraten. – –

Sie befanden sich senkrecht über dem Pol der Erde – das Ziel war
erreicht, dem sie so hoffnungsfroh entgegengestrebt hatten. Weit
unter ihnen im hellen Sonnenscheine lagen die glänzenden
Wolkenstreifen und fern im Süden das grünlich schimmernde Land
ausgebreitet, die kühnen Forscher aber sahen nichts mehr davon.
Ohnmächtig, erstickt – erdrückt von der Last des Ballons, flogen
sie, eine formlose Masse bildend, in der Richtung der Erdachse den
Grenzen der Atmosphäre entgegen.

\section{3 - Die Bewohner des Mars}

Unter dem Einfluß der geheimnisvollen Kraft, welche die Trümmer der
verunglückten Expedition in der Richtung der Erdachse vom Nordpol
forttrieb, hatten sie eine ungeheure Beschleunigung erlangt. Der in
die Falten des Ballons hineingetriebene Korb bewegte sich jetzt mit
rasender Geschwindigkeit nach oben. Wenige Minuten mußten genügen,
den Tod der Insassen zu bewirken, da der Verschluß der Gondel sie
nicht hinreichend zu schützen vermochte.

Nicht mehr von der Erde aus erkennbar schien das seltsame Geschoß
einsam und verlassen den Weltraum zu durcheilen, jeder menschlichen
Macht entrückt, ein Spielball kosmischer Kräfte – –

Und dennoch war der Ballon der Gegenstand gespanntester
Aufmerksamkeit.

Die Beobachter desselben befanden sich auf einer Stelle, wo kein
Mensch lebende Wesen vermutet, ja nur eine solche Möglichkeit hätte
verstehen können. Daß der Nordpol von unbekannten Bewohnern besetzt
sei, war ja äußerst seltsam und überraschend; aber er war doch ein
Punkt der Erde, auf welchem lebende Wesen sich aufzuhalten und zu
atmen vermochten. Der Ort dagegen, von welchem aus man jetzt auf
den verunglückten Ballon aufmerksam wurde, befand sich bereits
außerhalb der Erdatmosphäre. Genau in der Richtung der Erdachse und
auf dieser genau so weit von der Oberfläche der Erde entfernt wie
der Mittelpunkt der Erde unterhalb, also in einer Höhe von 6.356
Kilometer, befand sich frei im Raume schwebend ein merkwürdiges
Kunstwerk, ein ringförmiger Körper, etwa von der Gestalt eines
riesigen Rades, dessen Ebene parallel dem Horizont des Poles lag.

Dieser Ring besaß eine Breite von etwa fünfzig Metern und einen
inneren Durchmesser von zwanzig, im ganzen also einen Durchmesser
von 120 Metern. Rings um denselben erstreckten sich außerdem,
ähnlich wie die Ringe um den Saturn, dünne, aber sehr breite
Scheiben, deren Durchmesser bis auf weitere zweihundert Meter
anstieg. Sie bildeten ein System von Schwungrädern, das ohne
Reibung mit großer Geschwindigkeit um den inneren Ring herumlief
und denselben in seiner Ebene stets senkrecht zur Erdachse hielt.
Der innere Ring glich einer großen kreisförmigen Halle, die sich in
drei Stockwerken von zusammen etwa fünfzehn Metern Höhe aufbaute.
Das gesamte Material dieses Gebäudes wie das der Schwungräder
bestand aus einem völlig durchsichtigen Stoffe. Dieser war jedoch
von außerordentlicher Festigkeit und schloß das Innere der Halle
vollständig luft- und wärmedicht gegen den leeren Weltraum ab.
Obwohl die Temperatur im Weltraum rings um den Ring fast
zweihundert Grad unter dem Gefrierpunkt des Wassers lag, herrschte
innerhalb der ringförmigen Halle eine angenehme Wärme und eine zwar
etwas stark verdünnte, aber doch atembare Luft. In dem mittleren
Stockwerk, durch welches sich ein Gewirr von Drähten, Gittern und
vibrierenden Spiegeln zog, hielten sich auf der inneren Seite des
Rings zwei Personen auf, die sich damit beschäftigten, eine Reihe
von Apparaten zu beobachten und zu kontrollieren.

Wie aber war es möglich, daß dieser Ring in der Höhe von 6.356
Kilometern sich freischwebend über der Erde erhielt?

Eine tiefreichende Erkenntnis der Natur und eine äußerst
scharfsinnige Ausbildung der Technik hatten es verstanden, dieses
Wunderwerk herzustellen.

Der Ring unterlag natürlich der Anziehungskraft der Erde und wäre,
sich selbst überlassen, auf die Insel am Pol gestürzt. Gerade von
dieser Insel aus aber wirkte auf ihn eine abstoßende Kraft, welche
ihn in der Entfernung im Gleichgewicht hielt, die genau dem
Halbmesser der Erde gleichkam. Diese Kraft hatte ihre Quelle in
nichts anderem als in der Sonne selbst, und die Kraft der
Sonnenstrahlung so umzuformen, daß sie jenen Ring der Erde
gegenüber in Gleichgewichtslage hielt, das eben hatte die Kunst
einer glänzend vorgeschrittenen Wissenschaft und Technik zustande
gebracht.

In jener Höhe, einen Erdhalbmesser über dem Pol, war der Ring ohne
Unterbrechung der Sonnenstrahlung ausgesetzt. Die von der Sonne
ausgestrahlte Energie wurde nun von einer ungeheuren Anzahl von
Flächenelementen, die sich in dem Ringe und auf der Oberfläche der
Schwungräder befanden, aufgenommen und gesammelt. Die Menschen
verwenden auf der Erdoberfläche von der Sonnenenergie hauptsächlich
nur Wärme und Licht. Hier im leeren Weltraum aber zeigte sich, daß
die Sonne noch ungleich größere Energiemengen aussendet,
insbesondere Strahlen von sehr großer Wellenlänge, wie die
elektrischen, als auch solche von noch viel kleinerer als die der
Lichtwellen. Wir merken nichts davon, weil sie zum größten Teile
schon von den äußersten Schichten der Atmosphäre absorbiert oder
wieder in den Weltraum ausgestrahlt werden. Hier aber wurden alle
diese sonst verlorenen Energiemengen gesammelt, transformiert und
in geeigneter Gestalt nach der Insel am Nordpol reflektiert. Auf
der Insel wurden sie, in Verbindung mit der von der Insel direkt
aufgenommenen Strahlung, zu einer Reihe großartiger Leistungen
verwendet; denn man hatte auf diese Weise eine ganz enorme
Energiemenge zur Verfügung.

Ein Teil dieser Arbeitskraft wurde nun zunächst dazu gebraucht, ein
elektromagnetisches Feld von gewaltigster Stärke und Ausdehnung zu
erzeugen. Die ganze Insel mit ihren hundertvierundvierzig
Rundbastionen stellte gewissermaßen einen riesigen Elektromagneten
vor, der von der Sonnenenergie selbst gespeist wurde. Die
Konstruktion war so angelegt, daß die Kraftlinien sich um den Ring
konzentrierten und dieser, der Schwerkraft entgegen schwebend
gehalten wurde. Daß dies genau in der Entfernung des Erdhalbmessers
vom Pole geschah, hing mit einer Beziehung zwischen
Elektromagnetismus und Schwere zusammen, infolge deren sich gerade
an dieser Stelle eine Art Knotenpunkt für die Wellenbewegung beider
Kräfte zu bilden vermochte und das Gleichgewicht ermöglichte.

Allerdings wurde durch eine Reihe komplizierter und höchst
scharfsinnig ausgedachter Kontrollapparate dafür gesorgt, daß alle
Schwankungen der Energiemengen zur rechten Zeit ausgeglichen
wurden. Einen solchen Apparat aufzustellen wäre indessen an keinem
anderen Punkte der Erde möglich gewesen als in der Verlängerung
ihrer Rotationsachse, also über dem Nordpol oder über dem Südpol.
Denn an jeder andern Stelle hätte, abgesehen von tieferliegenden
Schwierigkeiten, die Verschiebung der Erdoberfläche infolge der
täglichen Umdrehung der Erde unüberwindbare Hindernisse für die
Herstellung des Gleichgewichts zwischen der Schwerkraft und dem
Elektromagneten geboten; auch hätte die gleichmäßige
Sonnenstrahlung gefehlt. Der Pol bietet aber in jeder Hinsicht die
einfachsten Verhältnisse wenn es gelingt, bis zu ihm zu gelangen.

Nun, die unübertroffenen Ingenieure der Insel und des Ringes waren
einmal da. Aber wo kamen sie her? Wie waren sie dorthin gelangt,
ohne daß die internationale Kommission für Polarforschung die
geringste Ahnung davon hatte? Und vor allem – wenn sie einmal da
waren –, was hatte es für einen Zweck, jenen freischwebenden Ring
über dem Pol zu balancieren? Und wenn einmal jener Ring da war, wie
konnte man hinauf- und hinabkommen?

Jener Ring war überhaupt nur ein Mittel, um einen ganz andern Zweck
zu erreichen. Er diente dazu, einen Standpunkt außerhalb der
Atmosphäre der Erde zu gewinnen, eine Station, um zwischen dieser
und der Erde nichts Geringeres auszuführen, als – eine zeitweilige
Aufhebung der Schwerkraft. Der Raum zwischen der inneren Öffnung
des Ringes von zwanzig Metern Durchmesser und der auf der Insel
sich befindenden Vertiefung, also ein Zylinder, dessen Achse genau
mit der Erdachse zusammenfiel, war ein ›abarisches Feld‹. Dies
bedeutet, ein Gebiet ohne Schwere. Körper, welche in diesen
zylindrischen Raum gerieten, wurden von der Erde nicht mehr
angezogen. Dieses abarische Feld bewirkte, daß in der ganzen
Umgebung des Feldes Spannungen im Raum vorhanden waren, wodurch
etwa sich nähernde Körper in das Feld getrieben wurden. Daher war
es gekommen, daß der Ballon der Luftschiffer allmählich der Insel
und damit dem abarischen Felde unentrinnbar zugeführt worden war.

Die Erzeugung jenes Feldes, in welchem die Schwerkraft aufgehoben
war für den inneren Raum zwischen Insel und Ring, war dadurch
möglich gemacht worden, daß man eine der Erdschwere entgegengesetzt
gerichtete Gravitationskraft herstellte. Es war jenen Polbewohnern
bekannt, wie man diejenigen Strahlen, welche hauptsächlich
chemische Wirkung, Wärme und Licht liefern, in Gravitation
überführen kann. Sie wurden zu diesem Zweck bis in den inneren Teil
des Ringes geleitet und traten hier in den ›Gravitationsgenerator‹.
Dies war ein Apparat, durch welchen man Wärme in Gravitation
umwandelte. Ein zweiter, ebenso eingerichteter Gravitationserzeuger
befand sich in der zentralen Vertiefung im Inneren der Insel. Beide
Apparate wirkten derartig zusammen, daß die Beschleunigung der
Schwerkraft im Inneren zwischen Insel und Ring beliebig reguliert
werden konnte. Man konnte sie entweder nur verringern, oder ganz
aufheben – dann war das abarische Feld im eigentlichen Sinne
hergestellt –, oder man konnte die Gegenschwerkraft so verstärken,
daß die Körper innerhalb des abarischen Feldes ›nach oben fielen‹,
das heißt, eine beliebig starke Beschleunigung entgegengesetzt der
Erdschwere, also von der Erde fort, erhielten. Auf diese Weise war
es möglich, mit jeder gewünschten Geschwindigkeit Körper zwischen
der Insel und dem Ringe sowohl von unten nach oben als von oben
nach unten in Bewegung zu setzen, indem man sie in einen zu diesem
Zweck konstruierten Flugwagen einschloß.

Es war nun die schwierige Aufgabe der Ingenieure an den beiden
Endstationen, den Betrieb so zu regulieren, daß jedesmal das
abarische Feld die nötige Stärke besaß, um den Wagen nach oben zu
treiben oder in seiner Bewegung aufzuhalten.

Als der Ballon der Polarforscher in das abarische Feld geriet, war
dasselbe gerade auf ›Gegenschwere‹ gestellt, weil sich ein
Flugwagen auf dem Wege von der Insel nach dem Ringe befand.
Infolgedessen wurde der nach dem abarischen Felde hingedrängte
Ballon, sobald er in die Achse desselben geraten war, mit großer
Geschwindigkeit in die Höhe gerissen.

Äußerlich unterschied sich das Feld von der umgebenden Luft in gar
nichts. Nur ein starker aufsteigender Luftstrom und infolgedessen
ein seitliches Zuströmen der Luft war natürlich vorhanden. Aber bei
dem geringen Durchmesser des Feldes von zwanzig Metern war die in
die Höhe getriebene Luftmasse so gering, daß es dadurch nicht zu
einer merklichen Nebel- oder Wolkenbildung kam, zumal vom Ringe wie
von der Insel aus eine so starke Bestrahlung stattfand, daß der
sich etwa kondensierende Wasserdampf sofort wieder in Gasform
aufgelöst wurde.

Solange der Ballon sich noch in den Luftschichten bis ein oder zwei
Kilometer befand, konnte das Ausströmen des Gases sein Aufsteigen
einigermaßen verzögern. Dann aber wurde die Beschleunigung zu groß.
Die Gondel, welche sich im Zentrum des Feldes befand, erfuhr dabei
eine größere Beschleunigung nach oben als der an Masse zwar
geringere, an Ausdehnung aber soviel größere Ballon. Denn da der
Durchmesser des Ballons zwanzig Meter übertraf, so ragte er zum
Teil über das abarische Feld hinaus. Erst als er durch den Verlust
an Gas zusammengesunken war, geriet er ganz in das abarische Feld,
und nun begann jener kolossal beschleunigte ›Fall nach oben‹, der
den Ballon binnen einer Viertelstunde auf tausend Kilometer
emporgerissen hätte, wenn er nicht zum Glück nach kaum einer Minute
aufgehalten worden wäre.

Als die Ingenieure der Insel den Ballon bemerkten, hatten sie
zunächst versucht, ihn durch Ergreifung des Schleppgurts
festzuhalten. Dies hatte Grunthe durch das Hinauswerfen der
Champagnerflaschen verhindert, da er jede Berührung der Insel
vermeiden wollte. So war der Ballon so weit gestiegen, daß er nicht
mehr ergriffen werden konnte, aber er war dadurch dem abarischen
Felde unrettbar überliefert. Hier hätten ihn nun die Bewohner der
Insel freilich sogleich aufhalten und zurückführen können, wenn sie
die ›Gegenschwere‹ im Felde abgestellt hätten. Dies war ihnen
jedoch darum nicht möglich, weil sich oberhalb des Ballons, längst
nicht mehr sichtbar, ihr eigener Flugwagen befand. Sie konnten also
nicht eher eine Veränderung am Feld vornehmen, als bis ihr Wagen an
der Station des Ringes angekommen war. Zum Glück für die Insassen
des Ballons mußte dies in kürzester Zeit geschehen.

Inzwischen hatten aber auch die Ingenieure auf dem Ring, obwohl sie
den Ballon nicht sehen konnten, doch an ihren Gravitationsmessern
eine Störung im abarischen Felde wahrgenommen. Sie sandten daher
vom Ring eine Depesche nach der Insel.

Diese Übermittlung bot keine Schwierigkeit, denn sie verstanden es,
die Lichtstrahlen selbst als Träger für ihre Depeschen zu benutzen.
Der Raum zwischen Ring und Insel gestattete dies infolge der
intensiven Bestrahlung auch beim feuchtesten Wetter.

Sie telegraphierten nicht nur, sie telefonierten vermöge des
Lichtstrahls. Die elektromagnetischen Schwingungen des Telephons
setzten sich in photochemische um und wurden auf der andern Station
sofort am Apparat abgelesen. Während die unglücklichen
Luftschiffer, von der Seide des Ballons eingehüllt, ihre
blitzschnelle Fahrt auf der Erdachse vollführten, ging an ihnen
eine Depesche vom Ring nach der Insel vorüber, welche lautete:

„E najoh. Ke.“

Und von der Insel wurde zurückdepeschiert:

„Bate li war. Tak a fil.“

Man hätte freilich alle bekannten Sprachen der Erde durchgehen
können, ohne in irgendeiner diese Sätze zu finden. Sie bedeuten:

„Achtung! Störung! Was ist vorgefallen?“

Und die Antwort lautete:

„Menschen im abarischen Feld. Abstellen sobald als möglich.“

Der Empfänger dieser Depesche stand in der Beobachtungsabteilung
des schwebenden Ringes und kontrollierte die Apparate, welche
daselbst an einem großen Schaltbrett angebracht waren. Der Zeiger
am Differenzialbaroskop wies ihm genau die Stelle, wo sich der
Flugwagen im Augenblick befand. Schon war dieser nahe
herangekommen. Einige Handgriffe des Beamten regulierten die
Geschwindigkeit des Wagens, der nach wenigen Minuten auf der
Endstation erschien. Das vorspringende Fangnetz hielt ihn auf, er
ruhte an seinem Ziel.

Der Beamte – es war der Vorsteher der Außenstation selbst – namens
Fru, hatte bis jetzt keinen Blick von den Apparaten verwandt. Man
hätte ihn für einen alten Mann halten mögen, denn langes, fast
weißes Haar flatterte um seine Schläfe. Eine ungewöhnlich hohe
Stirn wölbte sich über den großen Augen, deren Pupillen einen
tiefen Glanz zeigten. Die Haltung des Körpers aber war frei und
leicht. Gewandt bewegte er sich an dem langen Schaltbrett entlang
von einem Apparat zum andern, seine Schritte glichen fast einem
Gleiten über den Boden. Er war offenbar daran gewöhnt, daß die
Schwerkraft eine viel geringere war als auf der Erde. Denn hier, in
der doppelten Entfernung vom Mittelpunkt der Erde als deren
Oberfläche, betrug die Schwere nur ein Viertel von der uns
gewohnten.

Die Tür des Flugwagens wurde jetzt geöffnet. Der Vorsteher der
Ringstation warf nur einen flüchtigen Blick dorthin und wandte sich
dann wieder den Apparaten zu, um nach dem Pol zu telegraphieren,
daß das abarische Feld frei sei.

Die Fahrgäste verließen den Wagen und betraten die Galerie. Es
mochten achtzehn Personen sein, in seltsamer Tracht, mit eng
anliegenden Kleidern. Ihre bedeutenden Köpfe zeichneten sich meist
durch sehr helles, fast weißes Haar und glänzende, durchdringende
Augen aus, die aber jetzt durch dunkle Brillen geschützt waren. Sie
durchschritten die Galerie, deren Überschrift in jener fremden
Sprache besagte, daß man sich auf der ›Außenstation der Erde‹
befinde, und wandten sich über eine Treppe der Ausgangstür nach der
oberen Galerie zu. Über der Tür stand in großen Buchstaben: ›Vel lo
nu‹, das bedeutet: ›Zum Raumschiff nach dem Mars.‹

Jener schwebende Ring war nichts anderes als der Marsbahnhof der
Erde. Er war eine Station in der Nähe der Erde, durch deren
Erbauung es den Bewohnern des Planeten Mars möglich geworden war,
zwischen ihrem Planeten und der Erde eine regelmäßige Verbindung
herzustellen. Die Fahrgäste des Flugwagens waren Martier, die nach
ihrer Heimat zurückkehren wollten.

\section{4 - Der Sturz des Ballons}

Die Regulierung des abarischen Feldes hatte von der Ringstation aus
stattgefunden, um den emporsteigenden Flugwagen mit der nötigen
Geschwindigkeit zu leiten. Der Wechsel von Gegenschwere und
Erdschwere erstreckte sich aber auf das ganze Feld und hatte
natürlich zur Folge, daß auch der verunglückte Ballon den
Schwankungen der Schwere unterlag. So wurde er zuerst in seinem
Fluge nach oben gemäßigt, durchlief dann eine kurze Strecke mit
unveränderter Geschwindigkeit, und von dem Augenblicke an, in
welchem der Flugwagen den Ring erreicht hatte, begann der Ballon
wieder mit immer zunehmender Geschwindigkeit zu fallen. Da in
diesen Höhen von einem Widerstand der Luft nicht die Rede war, so
fielen auch jetzt Ballon und Gondel mit gleicher Geschwindigkeit.
Der stark zusammengesunkene Ballon, der einen großen Teil seiner
Gasmenge verloren hatte, bedeckte in dichten Falten den Korb.

Dieser Umstand hatte die Luftschiffer vor einem sofortigen Tod
bewahrt. Zunächst schützte sie die Einhüllung in den Ballon vor dem
Erfrieren; ja merkwürdigerweise stieg die Temperatur im Inneren des
Korbes wieder, als die Atmosphäre der Erde durchflogen war. Dies
rührte von der Sonnenstrahlung her, welche jetzt in voller Stärke,
durch die Luft nicht mehr aufgehalten, den Ballon traf. Sie wurde
durch die Hülle des Ballons absorbiert und erwärmte alles, was sich
in derselben befand.

Ein glücklicher Zufall hatte es aber auch so gefügt, daß sich noch
ein Teil des Gases im Ballon hielt, dessen Stoff von so
vorzüglicher Beschaffenheit war, daß er die Diffusion des
Wasserstoffs selbst gegenüber dem leeren Raume fast völlig aufhob.
Das Gas konnte nur durch das Landungsventil entströmen. Das
Versagen der Zerreißvorrichtung, das ihr Verderben schien, wurde
jetzt die Rettung der Luftschiffer.

Durch die Einstülpung, welche der Ballon im abarischen Felde
erfahren hatte, war der untere Teil des Ballons so in den oberen
hineingetrieben worden, daß das Ventil zwischen den Falten
zusammengepreßt lag und ein weiteres Ausströmen des Gases
verhindert wurde. Freilich hätte auch dies nicht lange vorgehalten,
aber der ganze Vorgang, von dem Augenblick, in welchem Grunthe die
Reißleine ergriff, bis zum Zusammenklappen des Ballons und dann zum
Abstellen des abarischen Feldes durch die Martier hatte nur wenige
Minuten betragen.

Da es sich bei dem Niedergang des Ballons im abarischen Feld um
einen herabsteigenden Körper handelte, hatten die Ingenieure der
Insel die Regulierung der Bewegung zu besorgen. Sie konnten
denselben zwar der eingetretenen Bewölkung wegen nicht sehen, aber
ihre Instrumente zeigten ihnen genau die Stelle, an welcher er sich
befand, und die Geschwindigkeit, mit welcher er fiel. Sie gaben nun
im geeigneten Moment dem Feld eine so starke Gegenbeschleunigung,
daß der Ballon in der Höhe von etwa dreitausend Meter über der Erde
zur Ruhe kam, gerade in dem Augenblick, in welchem er die
Wolkendecke durchbrochen hatte und der Beobachtung durch das
Fernrohr zugänglich geworden war. Der Ballon war jetzt den
gewöhnlichen Verhältnissen der Atmosphäre überlassen. Das abarische
Feld wurde nun gänzlich abgestellt, so daß es sich in nichts von
den übrigen Teilen der Atmosphäre unterschied. Allerdings hatte der
Ballon so viel Gas verloren, daß er sich nicht in der Höhe halten
konnte. Aber wenn die Luftschiffer noch am Leben waren, durften die
Martier annehmen, daß sie durch Auswerfen von Ballast ihren Abstieg
nunmehr verlangsamen und selbständig regulieren konnten.

Doch was sahen die Martier der Insel durch ihre Fernrohre? Der
Ballon hatte sich allerdings über dem Korb wieder erhoben. Dieser
selbst aber war gegen den Ring gepreßt und in das Gewirr der ihn
tragenden Seile geraten und lag nun vollständig schief zur Seite.
Das Schleppseil hing nicht herab, sondern hatte sich um den Ballon
herumgeschlungen. Der Verschluß des Korbes war geöffnet. Ein großer
Teil des Inhalts der Gondel schien dabei herausgestürzt. Die Last
des Ballons war dadurch so stark erleichtert worden, daß die
übriggebliebene Gasmenge, so gering sie auch war, sie doch noch zu
tragen vermochte. Der Ballon sank nur ganz allmählich und wurde, da
das abarische Feld außer Tätigkeit gesetzt war, vom Wind ergriffen.
So trieb der Ballon von der Insel fort über das Binnenmeer hin,
nahezu in der entgegengesetzten Richtung als in derjenigen, aus
welcher die Luftfahrer gekommen waren.

Die Martier erkannten nun wohl, daß die Insassen des Ballons die
Kontrolle über ihn verloren hatten. Was konnten sie aber zu ihrer
Rettung tun? Sie hätten zwar durch Herstellung des abarischen
Feldes bewirken können, daß sich der Ballon dem Zentrum wieder
nähern mußte, doch sie wollten ihn ja gerade von der Insel
entfernen. Denn sie durften durch diesen fremden Körper nicht
länger ihren Verkehr mit der Ringstation stören lassen.

Während die Martier sich berieten, hatte der Ballon bereits die
Insel überflogen und befand sich über dem Meer. Zugleich war er auf
etwa zweitausend Meter gesunken. Würde er das gegenüberliegende
Ufer erreichen? Würde er in das Meer stürzen? Oder würde er an der
Felswand des steil abfallenden Ufers zerschellen? Das letzte schien
das Wahrscheinlichste, wenn es nicht gelang, den Ballon entweder zu
heben oder zu schnellem Sinken zu bringen.

In der halb umgestürzten Gondel des Ballons sah es wüst aus. Die
Instrumente zum Teil zertrümmert, die Körbe und Kisten zerbrochen,
Vorräte und Menschen in einem wirren Knäuel, nur durch das Netz der
vielfach verschlungenen Stricke am Herausstürzen verhindert.

Von einem stechenden Schmerz im rechten Fuß erweckt, öffnete
Grunthe die Augen. Er sah sich zu seinem Erstaunen am Rande des
Korbes, der sich auf der einen Seite mit dem Ringe verfangen hatte,
zwischen dem Geflecht desselben und einem der Anker des Ballons
eingeklemmt. Dieser hatte ihn am Fuß verletzt. Schnell kam Grunthe
wieder zu vollem Bewußtsein. Er konnte nur seinen Oberkörper und
die Arme bewegen. Ein Blick auf den Zustand des Ballons ließ ihn
befürchten, daß es unmöglich sein würde, die Höhe des Gebirges
jenseits des Sees zu gewinnen. Unter ihm aber lag die Fläche des
Meeres. Besorgt blickte er sich nach seinen Gefährten um. Torm
vermochte er nirgends zu entdecken. Aber nun sah er, wie unter
einem zerbrochenen Korb und einem Haufen von Decken sich etwas
bewegte und ein Kopf mit dunkelbraunem, lockigem Haar zum Vorschein
kam. Es war Saltner, der eben falls aus seiner Ohnmacht erwachte.
Saltner, der keine Ahnung von dem Zustand des Ballons hatte, suchte
sich aus seiner unbequemen Lage zu befreien. Grunthe aber erkannte,
in welcher Gefahr der Reisegenosse schwebte. Jede weitere Bewegung
konnte ihn aus dem Korbe herausschleudern und hinabstürzen lassen.

„Liegen Sie still“, rief er ihm zu, „verhalten Sie sich ganz ruhig
– der Korb ist gekentert – halten Sie sich fest!“

„Sackerment“, brummte Saltner unter der Decke, „liegen Sie doch
einmal still, wenn Sie auf einer zerbrochenen Champagnerflasche
sitzen! Hätten wir nur das ganze Zeug gleich ausgetrunken und die
leere Flasche hinausgeworfen!“

Dabei warf er sich mit einem Gewaltruck zur Seite, zugleich aber
geriet er ins Rollen –

Grunthe stieß einen Schrei des Schreckens aus. Er sah den Gefährten
am äußersten Rande der Gondel schweben – Saltner fuhr mit den Armen
in die Luft, jedoch er fand keinen Halt – der Körper stürzte hinaus
– seine Knie hingen in der Schlinge eines Seiles – in dieser
furchtbaren Lage, den Kopf nach unten, schwebte Saltner mehr als
tausend Meter über dem Spiegel des Polarmeeres.

In der Aufregung des Augenblicks wandte Grunthe, mit beiden Händen
sich festhaltend, seinen Körper so gewaltsam, daß es ihm gelang,
den Fuß unter dem Anker herauszureißen. Er achtete den Schmerz
nicht; so schnell wie möglich, obwohl mit großer Vorsicht,
kletterte er an den Tauen des Korbes nach Saltner hin. Er suchte
nach einem Seil, das er ihm zuwerfen konnte, um ihn wieder in die
Gondel zu ziehen. Aber wo war in diesem Gewirr von Stricken
sogleich ein passendes Tau zu finden? Hier hing eine weite Schlinge
herab. Er versetzte sie in Schwingungen, er zerrte daran, und jetzt
gelang es ihm, das Tau bis in Saltners Nähe zu bringen.

Zum Glück hatte dieser keinen Augenblick seine Geistesgegenwart
verloren. Als er das Tau im Bereich seiner Hände sah, griff er
danach. Es gelang ihm sich festzuhalten, und nun versuchte er an
dem Tau sich wieder zur Gondel emporzuarbeiten. Schon befand er
sich wieder in aufrechter Stellung. Mit den Händen am Seil höher
greifend, zog er seine Füße aus der Schlinge, in welcher er
hängengeblieben war, und setzte sie auf den Rand des Korbes.
Plötzlich entstand über ihm ein Rauschen und Krachen. Das Seil, an
welchem er sich hielt, war ein Teil des über den Ballon gefallenen
Schlepptaus. Es löste sich jetzt mit seinem freien Ende vom Ballon
und glitt abwärts. Kaum hatte Saltner noch Zeit, sich an der Gondel
festzuklammern, als das Seil in seiner ganzen Länge hinabsauste.
Aber indem es über den Ballon hinwegglitt, verfing es sich mit der
Reißleine und zog dieselbe mit voller Gewalt mit sich. Jetzt trat
die Zerreißvorrichtung in Funktion. Die Ballonhülle klaffte
auseinander. Das Gas strömte mit Zischen aus. Der Ballon drehte
sich um seine Achse und begann mit rasender Geschwindigkeit zu
fallen.

„Hinauf in den Ring“, rief Grunthe. „Wir müssen sehen, die Gondel
abzuschneiden.“

„Aber wo ist Torm?“ rief Saltner.

Sie riefen, sie schrieen, sie suchten – Torm war nicht zu finden.
Dennoch war es möglich, daß er sich noch im Korb befand – sie
durften diesen also nicht vom Ballon trennen, sie konnten ihn auch
nicht länger durchsuchen –

„Den Fallschirm, den Fallschirm!“ rief Grunthe wieder.

„Er ist fort!“

Der Ballon wirbelte abwärts –

Ein Schlag, ein Schäumen und Aufspritzen – das Meer schlug über der
Gondel und ihren Insassen zusammen – –

Wie eine riesige Schildkröte schwamm die Hülle des Ballons, sich
aufblähend, auf dem Wasser, die Expedition unter sich begrabend.

\section{5 - Auf der künstlichen Insel}

Das milde Licht des Polartages schien durch die breiten Fenster
eines hohen Gemaches, das im Stile der Marsbewohner ausgestattet
war. An der Decke zogen sich eine große Anzahl metallischer
Streifen entlang, die in ihrer Gesamtheit ein geschmackvolles
Muster darstellten. In der Mitte schlossen sie sich zu einer
Rosette zusammen, von welcher zahlreiche Drähte herabführten und in
einem schrankartigen Aufsatz endigten. Dieser Aufsatz befand sich
auf einem großen runden Tisch und trug an seiner Außenseite ringsum
eine Reihe von Wirbeln oder Handgriffen; Aufschriften über ihnen
bezeichneten ihre Bestimmung. Die den Fenstern gegenüberliegende
Wand war zu beiden Seiten der breiten Mitteltür von geschnitzten
Regalen bedeckt, die zur Aufbewahrung einer reichhaltigen
Bibliothek dienten. Den darüber freibleibenden Raum schmückten
Gemälde; sie stellten Ansichten vom Mars dar. Doch hätte man
glauben mögen, durch eine Reihe von Öffnungen plastische
Darstellungen, oder vielmehr die Natur selbst zu sehen. Denn die
Abstufungen der Farben waren so intensiv, daß sie den Eindruck
vollständiger Wirklichkeit machten. Da sah man in einer Landschaft
die Reflexe der Sonnenstrahlen auf dem sumpfigen Boden wie
leuchtende Sterne, und dennoch vermochte man in dem tiefen Schatten
der riesigen Bäume die feinsten Nuancen deutlich zu unterscheiden.
Über der Tür leuchtete die lebensgroße Büste Imms, des
unsterblichen Philosophen der Martier, der ihnen die Lehre von der
Numenheit enthüllt hatte.

Auf der Fensterseite blühten in Näpfen seltsame Gewächse. Am
merkwürdigsten war darunter die tanzende Blüte ›Ro-Wa‹, eine
lilienartige Pflanze, deren lange Blütenstengel sich
schlangengleich hin- und herbewegten und mit ihren zierlichen
Knospen fortwährend anmutige Bewegungen ausführten, indem sie
zugleich ein leises Zwitschern wie von Vogelstimmen hören ließen.
Zwischen den Blumentischen stand auf der einen Seite eine
Schreibmaschine, auf der andern ein Apparat, der nichts anderes
vorstellte als eine Maschine zur Ausführung schwieriger
mathematischer Rechnungen.

Die Fenster reichten bis zum Boden des Zimmers. Dennoch schien es,
als liefe an denselben etwa bis zur Höhe von einem Meter eine
Bekleidung entlang. Aber seltsam, diese Bekleidung schimmerte in
einem dunkeln Grün und wogte leise auf und ab; und mitunter
leuchteten kleinere und größere Fische darin auf und stießen ihre
Köpfe an die Scheiben. Es war das Meer, das bis zu Meterhöhe über
den Boden des Zimmers hereinblickte. Denn jenes Zimmer befand sich
auf der Außenseite der Insel, welche Torms verunglückte Expedition
am Nordpol der Erde gesehen hatte.

Eine natürliche Insel war jedoch diese Anlage der Martier nicht.
Sie hatten vielmehr in den Binnensee, der am Nordpol sich vorfand,
eine künstliche Insel, richtiger ein schwimmendes Floß von großer
Ausdehnung, hineingebaut, das ihr Feld von riesigen Elektromagneten
zu tragen hatte. Denn diese Elektromagnete brauchten sie zur
Balancierung ihrer Außenstation und dadurch zur Errichtung des
abarischen Feldes. Auf der inneren Seite des ringförmig erbauten
Riesenfloßes befanden sich die Arbeitsmaschinen und Apparate,
während die Außenseite zu Wohnräumen diente sowie zum Stapelplatz
aller der Vorräte und Werkzeuge, welche die Martier hier allmählich
ansammelten, um die Eroberung der Erde vom Nordpol aus
vorzubereiten.

Über die Treppe, die von dem Dach der Insel nach dem Korridor und
den angrenzenden Wohnzimmern führte, stieg eine weibliche Gestalt
herab. Auf das Geländer gestützt bewegte sie sich mühsam, wie durch
eine schwere Last niedergebeugt. Sie zuckte schmerzlich zusammen,
sooft ihr Fuß mit einem krampfhaften Aufschlag die nächst niedere
Stufe berührte. Darauf durchschritt sie ebenso schwer und mühevoll
den Korridor, indem sie sich gleichfalls mit den Händen an einem
der Geländer unterstützte, die sich den Korridor entlangzogen.
Jetzt berührte sie die Tür des Zimmers, die sich geräuschlos in
sich selbst zusammenrollte, und trat ein. Die Tür schloß sich
hinter ihr von selbst.

Mit einem Schlag war die Haltung der Gestalt verändert. Leicht und
kräftig richtete sie sich empor. In einer anmutigen Bewegung warf
sie den Kopf zurück und atmete einige Male tief auf. Sie glitt
einige Schritte durch das Zimmer; nicht mehr gebeugt und mühsam,
sondern wie schwebend durchmaß sie in graziöser Haltung den Raum
und blickte auf dem Tisch nach dem Zifferblatt, das den Stand des
Schweredrucks im Zimmer angab. Ein helles Aufleuchten ihrer großen,
glänzenden Augen mochte ihre Zufriedenheit andeuten, denn sie
korrigierte kaum merklich die Stellung des Handgriffs, durch den
sie die im Zimmer herrschende Schwerkraft regulieren konnte. Eine
Abzweigung des abarischen Feldes gestattete den Bewohnern der
Insel, ihre Wohnräume den Schwereverhältnissen anzupassen, welche
ihre Konstitution erforderte. Denn die Schwerkraft auf dem Mars
beträgt nur ein Drittel von derjenigen auf der Erde.

Jetzt streifte sie mit einer leichten Bewegung die warme Hülle ab,
die ihre Schultern bedeckte, und ohne sich umzublicken warf sie
dieselbe, wo sie gerade stand, achtlos in die Höhe. Von ihrem Kopf
löste sie die Kapotte, die sie draußen getragen hatte, und stieß
sie ebenfalls ziellos in die Luft. An ihren Handschuhen drückte sie
auf ein Knöpfchen und streckte dann ihre Hände mit gespreizten
Fingern leicht in die Höhe, worauf sich die Handschuhe von selbst
abstreiften und emporstiegen. Alle die nach oben geworfenen
Gegenstände flogen von selbst einer Ecke des Zimmers zu, schlugen
eine dort befindliche Klappe zurück und glitten hinter der Wand auf
die ihnen bestimmten Plätze, während die Klappe sich wieder schloß.
Sie waren sämtlich mit einem von den Martiern entdeckten Stoff
gefüttert, der sich nach Art der Pflanzenfaser behandeln ließ, aber
in äußerst kräftiger Weise, so wie das Eisen vom Magnet, von einem
dazu eingerichteten Apparat angezogen wurde. Die anziehende Kraft
trat in Tätigkeit, sobald der Schluß gelöst wurde, der die
Gegenstände am Körper befestigte. Bei der im Zimmer herrschenden
geringen Schwere genügte es, die Sachen einfach mit einem leichten
Ruck nach oben zu werfen; die selbsttätige Garderobe besorgte das
übrige. So war es den Martiern sehr leicht gemacht, ihre Sachen in
Ordnung zu halten. Denn durch die Konstruktion der verschiedenen
Öffnungen, welche die Garderobenstücke zu passieren hatten, während
sie im Inneren des Garderobenschranks wieder herabfielen, wurden
sie automatisch sortiert, gereinigt und in die ihnen bestimmten
Fächer eingefügt, so daß sie sofort wieder zu bequemem Gebrauch bei
der Hand waren.

Ohne sich um die abgelegten Kleidungsstücke weiter zu kümmern,
näherte sich die Dame dem Bücherregal und zog eines der dort
stehenden Bücher hervor, indem sie es an einem daran befindlichen
Handgriff erfaßte. Sie begab sich damit nach dem Sofa und streckte
sich in bequemer Lage hin.

La war die Tochter des Ingenieurs Fru, des Vorstehers der
Außenstation. Hätte sie auf der Erde gelebt, so wäre ihre
Lebenszeit auf mehr als vierzig Jahre zu berechnen gewesen. Als
Bewohnerin des Mars aber, dessen Jahre doppelt so lang sind wie die
der Erde, zählte sie erst einige zwanzig Sommer und stand in der
Blüte ihrer Jugend. Ihr volles Haar, das sie in einen Knoten
geschlungen trug, hatte eine auf Erden wohl nicht leicht zu
findende Farbe, ein helles, etwas ins Rötliche schimmerndes Blond,
einigermaßen der Teerose vergleichbar; in bezaubernder Zartheit
erhob es sich wie eine Krone über dem weißen, reinen Teint ihres
feingebildeten Antlitzes. Die großen Augen, die allen Martiern
eigentümlich sind, wechselten je nach der Beleuchtung von einem
lichten Braun bis zum tiefsten Schwarz. Denn entsprechend den
starken Helligkeitsunterschieden, welche auf dem Mars herrschen,
besitzen die Bewohner desselben ein sehr weitreichendes
Akkomodationsvermögen, und bei schwachem Licht erweitern sich ihre
dunklen Pupillen bis an den Rand der Augenlider. Das Mienenspiel
gewinnt dadurch eine überraschende Lebhaftigkeit, und nichts
pflegte die Menschen mehr an den Marsbewohnern, nachdem sie sie
kennengelernt hatten, zu fesseln als der ausdrucksvolle Blick ihrer
mächtigen Augen. In ihnen zeigte sich die gewaltige Überlegenheit
des Geistes dieser einer höheren Kultur sich erfreuenden Wesen.

Wie eine leichte Wolke umhüllte ein faltenreicher weißer Schleier
die ganze Gestalt und ließ nur den edel geformten Hals und den
unteren Teil der Arme unbedeckt. Darunter aber schimmerten die
Formen des Körpers wie in einen glänzenden Harnisch gekleidet; denn
in der Tat bestand das eng anschließende Kleid aus einem
metallischen Gewebe, das, obgleich es sich jeder Bewegung auf das
bequemste anpaßte und dem leichtesten Drucke nachgab, doch einen
Panzer von größter Widerstandsfähigkeit bildete.

Das Buch, welches La der Bibliothek entnommen hatte, besaß wie alle
Bücher der Martier die Form einer großen Schiefertafel und wurde an
einem Handgriff ähnlich wie ein Fächer gehalten, so daß die längere
Seite der Tafel nach unten lag. Ein Druck mit dem Finger auf diesen
Griff bewirkte, daß das Buch nach oben aufklappte, und auf jeden
weiteren Druck legte sich Seite auf Seite von unten nach oben um.
Man bedurfte auf diese Weise nur einer Hand, um das Buch zu halten,
umzublättern und jede beliebige Seite festzulegen.

La schien es mit ihrem Studium nicht eilig zu haben. Sie hielt das
Buch geschlossen in der nachlässig herabhängenden Hand und gab sich
ihren Gedanken hin. Nach einiger Zeit begann sie die Lippen zu
bewegen und Laute vor sich hin zu sagen, die ihr offenbar nicht
geringe Mühe machten. Mitunter lachte sie leise vor sich hin, wenn
ihr eines der ungewohnten Worte nicht über die Lippen wollte, oder
es lief momentan ein Ausdruck der Ungeduld über ihre Züge. Sie
repetierte ein Pensum, das sie für sich erlernt hatte. Aber nun
blieb sie ganz stecken und sann eine Weile nach. Dann sagte sie für
sich:

„Es ist doch ein närrisches Kauderwelsch, das diese Kalaleks
sprechen!“

Jetzt erst erhob sie das Buch und ließ die Blätter mit großer
Geschwindigkeit sich herumschlagen, bis sie die gewünschte Stelle
gefunden hatte.

Das Buch enthielt eine Zusammenstellung alles dessen, was die
Martier bisher über die Lebensweise und Sprache der Eskimos hatten
in Erfahrung bringen können. Durch die Eskimofamilie, welche sie
aufgefunden hatten und auf ihrer Station ernährten, war es ihnen
gelungen, die Sprache der Eskimos zu erforschen. Ja sie kannten
sogar von einer Anzahl Worte ihre Darstellung in lateinischer
Druckschrift; denn der jüngere der beiden Eskimos hatte sich eine
Zeitlang auf einer Missionsstation in Grönland aufgehalten und war
im Besitz einer grönländischen Übersetzung des Neuen Testaments, in
welcher er zu buchstabieren vermochte. La studierte Grammatik und
Wörterbuch der Eskimos oder ›Kalalek‹.

Nachdem sie wieder eine Reihe von Worten und Redensarten vor sich
hingesagt hatte, fiel ihr ein, ob sie wohl auch die richtige
Aussprache getroffen habe. Die Prüfung war leicht; sie brauchte nur
die Empfangsplatte des Grammophons auf die betreffende Stelle des
Buches zu legen, um den Laut selbst zu hören; denn das Buch
enthielt auch die Phonogramme der direkt vom Mund der Eskimos
aufgenommenen Worte. Aber das Grammophon, welches die Phonogramme
hörbar machte, befand sich in dem Schrankaufsatz des Tisches, und
sie hätte sich zu diesem Zweck vom Sofa erheben müssen; das war ihr
zu unbequem.

Ach, dachte sie, es ist doch eine zu ungeschickt eingerichtete
Welt! Daß man noch nicht einmal so weit ist, daß der Selbstsprecher
zu einem hergelaufen kommt!

Das Grammophon kam aber nicht. La blieb also liegen und begnügte
sich, das Buch neben sich auf einem Tischchen zu deponieren.

Es ist wirklich recht überflüssig, spann sie ihren Gedankengang
weiter, sich mit der Eskimosprache soviel Mühe zu geben. Diese
Eskimos sind doch eine traurige Gesellschaft, und der Trangeruch
ist unerträglich. Sicher ist die große Erde auch von Wesen feinerer
Art bewohnt, die vermutlich eine ganz andere Sprache reden. Weiß
doch sogar unser junger Kalalek mit Erstaunen von der Weisheit
seiner frommen Väter zu erzählen, die ihm das Buch in der seltsamen
Schrift gegeben haben. Wenn wir erst einmal Gelegenheit fänden, mit
solchen Leuten zu verkehren, das möchte sich vielleicht eher
lohnen. Was mag das für ein Luftballon gewesen sein, der heute über
die Insel hinzog und dann in der Höhe verschwand? Da waren doch
gewiß keine Eskimos darin. Was mag aus den Luftschiffern geworden
sein?

La blickte empor. An der Wand war die Klappe des Fernsprechers mit
leichtem Schlag niedergefallen.

„La, bist du da?“ fragte eine weibliche Stimme in dem halblauten
Ton der Martier.

„Hier bin ich“, antwortete La in ihrer tiefen, langsamen
Sprechweise. „Bist du es, Se?“

„Ja, ich bin es. Hil läßt dich bitten, sogleich hinüber in das
Gastzimmer Nummer 20 zu kommen.“

„Schon wieder hinaus in die Schwere. Was gibt es denn?“

„Etwas ganz Besonderes, du wirst es gleich sehen.“

„Müssen wir ins Freie?“

„Nein, du brauchst keinen Pelz. Aber komm gleich.“

„Nun gut denn, ich komme.“

Die Klappe des Fernsprechers schloß sich.

La erhob sich und glitt in ihrem schwebenden Gang der Tür zu. Sie
öffnete sie mit einem leisen Seufzer, denn sie ging nicht gern über
die Korridore, auf denen die Erdschwere herrschte, so daß sie nur
gebückt einherschleichen konnte.

Aber sie war doch neugierig, was auf der Insel Besonderes passiert
sein sollte. Waren neue Gäste vom Mars gekommen? Oder hatte sich
der Ballon wieder gezeigt?

\tb

Als der zertrümmerte Ballon ins Meer stürzte, hatten die Martier
der Insel bereits ihr Jagdboot bemannt, auf welchem sie das
Polarbinnenmeer zu durchforschen pflegten. Eine von Akkumulatoren
getriebene Schraube erteilte ihm eine außerordentliche
Geschwindigkeit. Sechs Martier unter Führung des Ingenieurs Jo
hatten in demselben Platz genommen; auch der Arzt der Station, Hil,
befand sich dabei. Alle trugen die Köpfe in einer helmartigen
Bedeckung, die ihnen sowohl ihre Bewegungen in der Luft
erleichterte, als auch zugleich als Taucherhelm im Wasser diente.
Die Helme waren nämlich aus einem diabarischen, das ist
schwerelosen Stoff und hatten daher für ihre Träger kein Gewicht.
Zugleich enthielten sie in ihrer Kuppel einen ziemlich bedeutenden
luftleeren Raum, so daß sie eine, freilich nur geringe Zugkraft
nach oben hin ausübten. Dennoch genügte dieselbe, wenigstens das
Gewicht des Kopfes soweit zu mindern, daß die Muskeln des Nackens
entlastet wurden und die Martier ihren Kopf fast ebenso frei wie
auf dem Mars zu bewegen vermochten, wenn sie auch sonst von dem
ihnen ungewohnten Körpergewicht bedrückt wurden. Eben deshalb
trugen sie Taucheranzüge, um schwere Arbeiten möglichst in das
Wasser zu verlegen. Denn hier nahm ihnen natürlich der Auftrieb des
Wassers die Last ihres Körpergewichts ab.

Schnell näherte sich das Jagdboot dem Ballon, der von den Spuren
des in ihm noch enthaltenen Wasserstoffes und der Luft, die sich
unter ihm verfangen hatte, auf dem Wasser schwimmend erhalten
wurde. Um zu dem von der Seide des Ballons bedeckten Korb zu
gelangen, tauchten die Martier unter und drangen vom Wasser aus
unter den Ballon. Sie fanden sogleich die beiden verunglückten
Menschen und schafften sie eiligst in ihr Boot. Sodann lösten sie
die Gondel von ihren Verbindungen und bargen ihren gesamten Inhalt
ebenfalls an Bord. Alles übrige ließen sie vorläufig treiben, da es
ihnen zunächst darauf ankam, die aufgefundenen Menschen in ihre
Behausung zu bringen.

Saltner und Grunthe hatten außer der Verletzung, die sich letzterer
bereits vor dem Absturz am Fuß zugezogen hatte, weiter keine
Beschädigungen durch den Fall erlitten. Aber sie hatten sich nicht
aus dem Wasser herausarbeiten können. Keiner gab ein Lebenszeichen
von sich. Indessen begannen die Martier unter Leitung des Arztes
sofort die eifrigsten Wiederbelebungsversuche, wie es schien ohne
Erfolg.

„Da hätten wir nun“, sagte Jo, „endlich einmal ein paar wirkliche
Bate, die keine Kalalek sind, ein paar zivilisierte Erdbewohner,
und nun müssen die armen Kerle tot sein.“

„Wir wollen noch hoffen“, erwiderte einer der Martier. „Der Körper
ist noch warm. Vielleicht haben die Bate ein zähes Leben.“

„Es wäre ein großes Glück“, begann Jo wieder, „wenn wir sie retten
könnten. Es sind nicht bloß kühne Leute, es sind offenbar besonders
hervorragende Männer ihres Volkes, sonst würden sie nicht zu diesem
wunderbaren Unternehmen ausgewählt sein.“

„Ich wußte gar nicht“, sagte ein andrer, „daß die Bate Luftschiffe
haben.“

„Derartige Ballons sind schon mehrfach beobachtet worden“,
erwiderte Jo, „aber man wußte nicht sicher, wozu sie dienen,
wenigstens nicht, daß sich die Bate damit selbst in die Luft
erheben. Ich habe immer geglaubt, sie ließen dadurch nur
irgendwelche Lasten über die Erde heben oder ziehen. Gleichviel,
für uns kommt alles darauf an, daß wir durch die Leute nähere
Nachrichten von den kultivierten Gegenden der Erde erhalten. Alle
unsere Pläne würden alsdann wesentlich gefördert werden. Hil,
versuchen Sie Ihre ganze Kunst.“

Der Arzt antwortete nicht. Seine Aufmerksamkeit konzentrierte sich
auf die Bemühungen, die Atmung der Ertrunkenen wieder in Tätigkeit
zu setzen.

Endlich richtete er sich auf.

„Geben Sie vollen Strom!“ rief er Jo zu. „Es ist eine leise
Hoffnung da, aber hier im Freien bringen wir sie nicht durch. Wir
müssen in einer Minute im Laboratorium sein.“

Das Boot sauste durch die Flut. In zehn Sekunden war die Insel
erreicht. Es schoß durch die Einfahrt bis in den inneren Hafen. Im
Augenblick darauf waren die Verunglückten aufgehoben und in die
Krankenabteilung gebracht. Es war keine leichte Arbeit, denn jeder
der beiden Männer hatte für die Martier, in Rücksicht auf ihre
Fähigkeit, Lasten zu heben, ein Gewicht, das für uns einem solchen
von fünf Zentnern entspricht. Sie hätten zwar ihre Kräne benutzen
können, aber dies hätte zu lange gedauert. Und es kam auch nur
darauf an, die Verunglückten bis über die Schwelle der Tür zu
heben. Dann trat die Wirkung des abarischen Feldes in Kraft, und
der Transport hatte keine Schwierigkeiten mehr.

Hil begann sofort die Behandlung mit allen Hilfsmitteln der
martischen Heilkunst. Er hatte bereits einige Erfahrung aus dem
Studium der Eskimos gewonnen und daraus die Unterschiede in der
Funktion der Organe bei Menschen und bei Marsbewohnern
kennengelernt, die übrigens keineswegs so bedeutend sind, wie man
meinen mochte. Dem durchdringenden Scharfblick des Martiers
genügten die Schlüsse, die er aus der gewonnenen Erfahrung ziehen
konnte, um das Richtige zu treffen.

Die Bewohner der Insel, soweit sie nicht gerade mit einer
dringenden Arbeit beschäftigt waren, hatten sich inzwischen aufs
Lebhafteste für die aufgefundenen Menschen interessiert. Im Vorraum
des Krankenzimmers war ein fortwährendes Kommen, Gehen und Fragen,
die Klappen der Fernsprechverbindungen hoben und senkten sich, aber
noch immer konnte man nichts Bestimmtes erfahren.

Endlich, nach einer halben Stunde angestrengter Tätigkeit, brach
Hil sein Schweigen. Er wandte sich zu dem Direktor der Station, Ra,
der neben ihm stehend aufmerksam die merkwürdigen, wie tot
daliegenden Wesen betrachtete, und sagte:

„Sie werden leben.“

„Ah!“

„Aber es ist fraglich, ob wir sie hier zum Bewußtsein bringen. Wir
müssen sie in Verhältnisse schaffen, die ihren Lebensgewohnheiten
entsprechen. Vor allem dürfen wir ihnen die Schwere nicht
entziehen, und ich glaube, auch die Temperatur des Zimmers muß
höher sein.“

„Gut“, antwortete Ra, „wir haben ja Gastzimmer genug, wir können
sie an der Außenseite, bei unseren Wohnungen unterbringen. Ich
werde sofort das Nötige anordnen.“

Sobald Ra in den Vorraum trat und den hoffnungsvollen Ausspruch des
Arztes mitteilte, pflanzte sich die Nachricht durch die ganze Insel
hin fort. Die Bate, die keine Eskimos sind, waren der Mittelpunkt
aller Gespräche, obgleich erst die wenigsten Martier sie überhaupt
gesehen hatten. Daß übrigens jemand, der bei der Pflege nichts zu
tun hatte, neugierig hätte eindringen wollen, konnte bei dem feinen
Taktgefühl der Martier selbstverständlich nicht vorkommen.

Die beiden Geretteten wurden getrennt in geeigneten Räumen
untergebracht und vollständiger Ruhe überlassen.

Stundenlang lagen sie in tiefem Schlaf.

\section{6 - In der Pflege der Fee}

Saltner schlug die Augen auf.

Was er da über sich sah, war es das Netzwerk des Ballons? Diese
regelmäßigen, goldglänzenden Arabesken auf dem lichtblauen Grund?
Nein, der Ballon war es nicht – der Himmel sieht auch nicht so aus
– doch – was war denn geschehen? Er war ja ins Wasser gestürzt.
Sieht es unten auf dem Meer so aus? Aber im Wasser ist man tot oder
– er wendete den Kopf, doch die Augen fielen ihm wieder zu. Er
wollte nachdenken, doch die Fragen waren ihm zu schwer, er fühlte
sich so matt – – jetzt bemerkte er, daß er einen Gegenstand
zwischen den Lippen hielt, ein Röhrchen. – War es noch immer das
Mundstück des Sauerstoffapparats? Nein. – Ein seltsamer Duft
umwehte ihn – instinktiv sog er an dem Rohr, denn er empfand einen
brennenden Durst. Ach, wie das wohltat! Ein kühler erquickender
Trank! Wein war es nicht – Milch auch nicht –, gleichviel, es
mundete – war es vielleicht Nektar? Seine Sinne verwirrten sich
wieder. Aber der Trank wirkte wunderbar. Neues Leben rann durch
seine Adern. Er konnte die Augen wieder öffnen. Aber was erblickte
er? Also war er doch im Wasser?

Über ihm, höher als sein Kopf, rauschten die Wogen des Meeres. Aber
sie drangen nicht bis zu ihm heran. Eine durchsichtige Wand trennte
sie von ihm, hielt sie zurück. Der Schaum spritzte an ihr empor,
das Licht brach sich in den Wellen. Dennoch konnte er den Himmel
nicht sehen, ein Sonnendach mochte ihn abblenden. Hin und wieder
stieß ein Fisch dumpf gegen die Scheiben. Vergeblich versuchte sich
Saltner seine Lage zu erklären. Er glaubte zunächst, sich auf einem
Schiff zu befinden, obwohl es ihn wunderte, daß sich im Zimmer
nicht die geringste Bewegung spüren ließ. Aber nun blickte er etwas
mehr zur Seite. War es denn nicht mehr Tag? Das Zimmer war doch von
Tageslicht erhellt, aber dort links sah er direkt in dunkle Nacht.
Ein ihm unbekanntes Bauwerk in einem nie gesehenen Stil lag im
Mondschein vor ihm. Er blickte auf das Dach desselben, das von den
Wipfeln seltsamer Bäume begrenzt wurde. Und wie merkwürdig die
Schatten waren –! Saltner versuchte sich vorzubeugen, den Kopf zu
heben. Da standen wirklich zwei Monde am Himmel, deren Strahlen
sich kreuzten. Auf der Erde gab es etwas Derartiges nicht. Ein
Gemälde konnte doch aber nicht so starke Lichtunterschiede zeigen –
es müßte denn ein transparentes Bild sein –

Auf das leise Geräusch, welches seine Bewegung verursachte, schob
sich auf einmal die Landschaft zur Seite. Eine Gestalt lehnte in
einem Sessel und sah Saltner mit großen, leuchtenden Augen an.
Einen Augenblick starrte er verwirrt auf diese neue Erscheinung.
Noch nie glaubte er ein so herrliches Frauenantlitz gesehen zu
haben. Schnell wollte er sich erheben, und nun erst warf er einen
Blick auf seinen eignen Körper. Man hatte ihn während seiner
Bewußtlosigkeit offenbar gebadet und mit frischer Leibwäsche
versehen. Er fand sich in einen weiten Schlafrock von einem ihm
unbekannten Stoff gehüllt.

Jetzt streckte die Gestalt eine Hand aus und drehte an einem der
Knöpfe, die sich neben ihr auf einem Tisch befanden. in demselben
Augenblick durchlief Saltner ein Gefühl, als wollte man ihn
plötzlich in die Höhe heben. Die Hand, deren Stellung er verändern
wollte, fuhr ein ganzes Stück höher, als er sie zu heben
beabsichtigte. Mit Leichtigkeit richtete er seinen Oberkörper
empor, aber bei dem Ruck flogen auch seine Beine in die Luft, und
mit einer überraschenden Geschwindigkeit führte er einige
unbeabsichtigte turnerische Übungen aus, bis es ihm gelang, sich in
sitzender Stellung auf seinem Lager zu balancieren.

Zugleich hatte sich auch die weibliche Gestalt erhoben und schwebte
auf ihn zu. Ein herzgewinnendes Lächeln lag auf ihren Zügen, und
aus den wunderbaren Augen sprach die innigste Teilnahme.

Saltner wollte aufstehen, bemerkte aber schon beim ersten Anziehen
seines Fußes, daß er Gefahr lief, in eine unbestimmte Höhe zu
schnellen. Eine leichte Handbewegung der vor ihm stehenden Gestalt
bedeutete ihm, seinen Sitz wieder einzunehmen. Nun endlich fand er
die Sprache wieder in gewohnter Lebhaftigkeit.

„Wie Sie befehlen“, sagte er. „Es wäre mir eine große Ehre, wenn
Sie ebenfalls Platz nehmen wollten und mir gütigst andeuten, wo ich
mich eigentlich befinde.“

Bei seinen Worten ließ die Gestalt ein leises, silbernes Lachen
vernehmen.

„Er spricht, er spricht!“ rief sie in der Sprache der Martier. „Es
ist zu lustig!“

„Fafagolik?“ versuchte Saltner die fremden Laute wiederzugeben.
„Was ist das für eine Sprache oder was für eine Gegend?“

Die Martierin lachte wieder und betrachtete ihn dabei vergnüglich,
wie man ein merkwürdiges Tier abwartend anschaut.

Saltner wiederholte seine Frage französisch, englisch, italienisch
und sogar lateinisch. Damit war sein Sprachschatz erschöpft. Da ihn
die Fremde offenbar nicht verstand und er noch immer keine Antwort
erhielt, sagte er wieder auf deutsch:

„Die Gnädige scheint mich nicht zu verstehen, aber ich will mich
doch wenigstens vorstellen. Mein Name ist Saltner, Josef Saltner,
Naturforscher, Maler, Photograph und Mitglied der Tormschen
Polarexpedition, augenblicklich verunglückt und, wie mir scheint,
mehr oder weniger gerettet. Eigentlich ist dabei gar nichts zu
lachen, meine Gnädige, oder was Sie sonst sind.“

Darauf zeigte er mehrere Male mit dem Finger auf sich selbst und
sagte deutlich: „Saltner! Saltner!“ Sodann zeigte er mit der Hand
rings auf seine Umgebung und zuletzt auf die schöne Martierin.

Diese ging sogleich auf seine Gebärdensprache ein. Sie bewegte die
Hand langsam auf sich zu und sagte ihren Namen: „Se.“

Darauf deutete sie auf Saltner und wiederholte deutlich seinen
Namen. Und noch einmal wiederholte sie mit den entsprechenden
Gesten:

„Se! Saltner!“

„Se, Se?“ sagte Saltner fragend. „Das ist also Ihr werter Name.
Oder meinen Sie vielleicht, da draußen sei die See? Verstehen Sie
vielleicht doch ein wenig Deutsch? Wo befinden wir uns denn hier?“

Auf seine fragende Handbewegung zeigte Se nach dem Meer, das vor
den bis zum Fußboden reichenden Fenstern wogte, und nannte das
Wort, das in der Sprache der Martier Meer bedeutet. Darauf zog sie
an einem Handgriff, und anstelle des Meeres erschien die
Landschaft, welche Saltner bewundert hatte. Er sah jetzt, daß
dieselbe auf einen Wandschirm gemalt war, den Se soeben vor das
Fenster geschoben hatte. Sie zeigte auf die Landschaft und sagte
„Nu.“ Das bedeutet ›Mars‹, aber Saltner war freilich mit diesem
Wort nicht gedient.

Se ging nun weiter in das Zimmer zurück, das der Wandschirm bisher
seinen Blicken verhüllt hatte, und suchte nach einem Gegenstand,
den sie nicht sogleich zu finden schien. Saltner folgte ihr mit den
Augen. Er glaubte noch nie etwas Anmutigeres gesehen zu haben,
etwas Wunderbareres jedenfalls noch nicht.

Ein rosiger Schleier umhüllte den größten Teil der Gestalt, ließ
jedoch hier und da den metallischen Schimmer des Unterkleides
durchblicken. Die Haare kräuselten sich über dem Nacken in
beweglichen Löckchen, die als Grundfarbe ein lichtes Braun zeigten,
aber bei jeder Bewegung irisierten wie das Farbenspiel auf einer
Seifenblase. Alle Bewegungen ihres Körpers glichen dem leichten
Schweben eines Engels, der von der Schwere des Stoffes unabhängig
ist. Und sobald der Kopf an eine dunklere Stelle des Zimmers
geriet, leuchtete das Haar phosphoreszierend und umgab das Gesicht
wie ein Heiligenschein.

Plötzlich unterbrach sie ihr Suchen und rief:

„Wie bin ich doch zerstreut! Das hat ja alles noch Zeit. Der arme
Bat hat gewiß Hunger, daran hätte ich zunächst denken sollen. Wart,
mein armer Bat, ich will dir gleich etwas braten.“

Sie trat an den Tisch im Hintergrund des Zimmers und machte sich an
dem Schrankaufsatz und verschiedenen Handgriffen zu schaffen. Dann
war sie wieder neben ihm und sagte mit einem unnachahmlichen Ton,
der ihn entzückte: „Saltner“, indem sie die nicht mißzuverstehenden
Pantomimen des Essens machte.

„Glänzender Gedanke, holdselige Se“, rief Saltner, indem er die
Pantomime wiederholte.

Auf einen Handgriff Ses, Saltner wußte nicht wie, stand auf einmal
ein Tischchen vor seinem Lager, und Se setzte ihm eine Speise vor,
die sie soeben bereitet hatte. Er untersuchte nicht lange, was es
sei, zerbrach sich nicht den Kopf über die merkwürdigen Formen der
ihm gereichten Instrumente, sondern gebrauchte sie, unbekümmert um
Ses Lächeln, als Löffel, und tat dann einen langen Zug aus dem
Mundstück eines mit Flüssigkeit gefällten Gefäßes. Sein Hunger war,
wie er jetzt erst merkte, so groß, daß er selbst Ses Anwesenheit
und seine ganze Umgebung momentan vergessen hatte. Erst nachdem der
erste Reiz gestillt war, hörte er wieder aufmerksam auf Ses
Erklärungen, die ihm die einzelnen Gegenstände in ihrer Sprache
benannte, und es gelang ihm bald, einige Worte zu behalten.

Als er sein Mahl beendet hatte, betrachtete ihn Se wieder mit
zufriedener Miene. Wie man ein Schoßhündchen streichelt, glitt sie
mit der Hand über sein Haar und sagte:

„Der arme Bat war hungrig, nun wird er wieder gesund werden. War es
gut, Saltner?“

Saltner verstand freilich ihre Worte nicht, aber den Sinn fühlte er
deutlich heraus. Er kam sich auch etwas gedemütigt vor, denn er
merkte wohl, daß ihn Se nicht als ein gleichberechtigtes Wesen
behandelte. Aber wie sie seinen Namen aussprach, wie sie ihn mit
den Augen ansah, die bis ins Innerste der Seele hineinzuleuchten
schienen, konnte er nicht anders, als ihr mit den herzlichsten
Worten danken. Und auch Se verstand den Dank, ohne die Worte zu
kennen, die er sprach. Lächelnd sagte sie in ihrer Sprache:

„Saltner gefällt mir, er ist nicht wie ein Kalalek.“

Saltner hatte das Wort Kalalek verstanden, das die Eskimos den
Martiern als die Bezeichnung ihres Stammes genannt hatten.

„Nein“, rief er entschieden, „meine schöne Se, ein Eskimo bin ich
nicht, ich bin ein Deutscher, kein Eskimo – Deutscher!“

Und er begleitete die Worte mit so entschiedenen Gesten, daß Se
ihren Sinn sofort verstand.

Sie eilte zu dem Bücherregal an der Zimmerwand – denn Bücher
gehören bei den Martiern zur unentbehrlichen Aus stattung jedes
Zimmers, eher würde man die Fenster entbehren als die Bibliothek –
und holte einen Atlas herbei.

Inzwischen bestürmte Saltner seine Pflegerin mit Fragen nach dem
Schicksal seiner Gefährten, ohne sich genügend verständlich machen
zu können. Se kümmerte sich zunächst nicht um seine Worte und
Gebärden, sondern hielt den Atlas an seinem Griff Saltner vor die
Augen und ließ die Blätter desselben sich rasch umschlagen. Sein
Erstaunen über diese Mechanik wurde aber übertroffen, als sie in
ihrem Umblättern stillhielt und den Griff des Buches in einem
Gestell auf dem Tischchen vor ihm befestigte. Er erkannte sofort
die Karte der Gegenden um den Nordpol der Erde wieder, die er in
dem Riesenmaßstab der Insel vom Ballon aus bewundert hatte.

Se zeigte mit ihrem schlanken, zierlichen Finger, an dem ihm die
große Beweglichkeit der einzelnen Glieder auffiel, auf Grönland und
die nächsten Landmassen um den Pol; dazu sagte sie wiederholt:
„Kalalek, Bat Kalalek.“ Dann zeigte sie auf Saltner, ergriff seine
Hand und führte sie über die andern Teile des Kartenbildes, indem
sie dabei fragte: „Bat Saltner?“

Saltner suchte auf der Karte die Gegend von Deutschland, die
allerdings perspektivisch schon stark verkürzt erschien, und machte
ihr durch Zeichen begreiflich, daß hier seine Heimat sei. Da er aus
dem öfter gehörten Wort ›Bat‹ schloß, daß dies wohl soviel wie
Mensch oder Volksstamm bedeute, so zeigte er auf den Pol und fragte
dazu:

„Bat Se?“

Se antwortete mit einer lebhaft abwehrenden Bewegung. Sie legte die
ganze Hand auf die Karte und sagte: „Bat.“ Dann zeigte sie auf sich
selbst und sprach mit Selbstbewußtsein: „Se, Nume.“

Und als Saltner sie fragend anblickte, wies sie mit ausgestrecktem
Arm nach einer bestimmten Stelle des Bodens und wiederholte noch
einmal: „Nume.“ Wie sie so dastand, leuchteten ihre Augen in
verklärtem Glanz, und Saltner konnte nicht zweifeln, daß er ein
höheres Wesen vor sich habe. Aber sogleich neigte sie sich wieder
mit liebenswürdigem Lächeln zu ihm und ließ einige Blätter des
Atlas zurückschlagen. Es zeigte sich eine Gruppe geometrischer
Figuren, in denen Saltner ohne Schwierigkeit einen Aufriß der
Planetenbahnen im Sonnensystem erkannte. Se wies auf den
Mittelpunkt und sagte: „O.“

„Sonne“, antwortete Saltner, indem er zugleich nach der Richtung
hinzeigte, in welcher die Sonnenstrahlen auf der Oberfläche des
Meeres spielten.

Se nickte befriedigt, beschrieb dann mit ihrem Finger auf der Karte
die Erdbahn und wiederholte den Namen der Erde: „Ba“, und, auf
Saltner weisend: „Bat!“ Dann aber wieder mit dem ganzen Stolz der
Martier den Namen ›Nume‹ aussprechend, bezeichnete sie auf der
Karte die Bahn des Mars und sagte mit einem hoheitsvollen Blick auf
Saltner: „Nu.“

„Der Mars!“ Es kam fast tonlos von Saltners Lippen. Er merkte, wie
sich alle seine Begriffe zu verwirren drohten. Hilflos sah er zu Se
empor, die kaum seine Aufregung bemerkt hatte, als sie ihm schon
bedeutete, sich niederzulegen. Zwar wollte er trotz der Mattigkeit,
die er jetzt an sich spürte, aufspringen, um seine Wißbegierde
weiter zu befriedigen, aber ein Blick, der keinen Widerstand
zuließ, bannte ihn auf sein Lager.

In diesem Augenblick öffnete sich die Tür des Zimmers, und in
derselben erschien zusammengebeugt und schleppend, auf zwei Stäbe
gestützt, die Gestalt des Arztes Hil. Kaum aber hatte Hil das
Zimmer betreten, als er sich in voller Höhe aufrichtete, die Stäbe
fortwarf und schnell auf das Lager zuschritt. Er ergriff sofort
Saltners Hand, und während er den Puls beobachtete, sagte er mit
leichtem Vorwurf:

„Aber Se Se, was machen Sie mir für Geschichten. Stellen Sie nur
gleich die Abarie ab. Unser Bat muß seine richtige Erdschwere
haben, sonst geht er uns ein, ehe wir ihn wieder kräftig sehn.“

„Seien Sie nur nicht böse, Hil Hil“, lachte Se, „ich habe ihn ja so
schön gepflegt und gefüttert – sehen Sie die Schüssel – 150 Gramm
Eiweiß, 240 Gramm Fett und –“

Hil sah nach der Federwaage, die sich unter jedem Speisegerät der
Martier befand und sofort konstatierte, wieviel Nahrungsstoffe man
auf dieselbe gelegt oder dem Körper zugeführt hatte.

„Aber Sie haben die Schwere abgestellt, davon stand nichts in Ihrer
Instruktion.“

„Ja, Hil Hil, Sie können doch nicht verlangen, daß ich im Zimmer
herumkriechen soll, wenn er wach ist.“

„Ach so, die liebe Eitelkeit!“

„Oh, vor dem Bat! Aber als er aufwachte, mußte ich doch schnell
hin, und dann mußte ich die Pastete backen, und – ja, wenn Sie
wüßten: Er heißt Saltner und ist kein Kalalek, sondern ein – ja,
das Wort habe ich vergessen, doch ich zeige Ihnen auf der Karte die
Gegend.“

„Erst lassen Sie es schwer werden – aber halt, noch einen
Augenblick, ich will mir zuvor einen Stuhl holen – so –“

„Und ich will mich auch erst setzen“, sagte Se.

Als beide Platz genommen hatten, griff Se an einen Wirbel, und
Saltner sah, wie Se und Hil sichtlich in ihren Sesseln
zusammensanken und ihre gelegentlichen Bewegungen mühsam und
schwerfällig wurden. Er aber merkte, wie das eigentümliche Gefühl
des Schwindels, das ihn beherrscht hatte, verschwand, seine
Gliedmaßen konnte er wieder normal dirigieren, und er legte sich
behaglich auf sein Lager zurück.

Der Arzt sah ihn mit seinen großen, sprechenden Augen wohlwollend
an.

„Also man ist wieder lebendig?“ sagte er, was Saltner freilich
nicht verstand. Dann fügte er in der Sprache der Eskimos hinzu:
„Versteht ihr vielleicht diese Sprache?“

Saltner erriet die Frage und schüttelte den Kopf. Dagegen sagte er
nunmehr selbst in der Sprache der Martier, was er von Se gelernt
hatte:

„Trinken – Wein – Bat gut Wein trinken –“

Se brach in ihr feines, silbernes Lachen aus, und Hil sagte
belustigt: „Sie haben ja ausgezeichnete Fortschritte gemacht – nun
werden wir uns wohl bald unterhalten können.“

Dabei wies er auf das neben Saltner stehende Trinkgefäß hin, und
dieser bediente sich desselben mit Erfolg zu neuer Stärkung.

Das Schicksal seiner Gefährten lag ihm am schwersten auf der Seele.
Er versuchte noch einmal, darüber Erkundigungen einzuziehen, indem
er einen Finger aufhob und dazu sagte: „Bat Saltner.“ Dann erhob er
drei Finger und suchte durch weitere Zeichen verständlich zu
machen, daß drei ›Bate‹ mit dem Ballon angekommen und herabgestürzt
seien.

Hil, der zum ersten Mal einen Europäer sah, hatte seine
Aufmerksamkeit mehr auf den ganzen Menschen als auf sein Anliegen
gerichtet, und blickte jetzt fragend zu Se hinüber, als sich
Saltner mit der von Se gehörten Anrede ›Hil Hil‹ direkt an ihn
wendete.

Se erklärte:

„Er meint, daß drei Bate angekommen und in das Meer gestürzt sind.
Wir haben aber doch nur zwei gefunden?“

„Allerdings“, sagte Hil, „und dem andern geht es auch besser. Der
Fuß ist nicht schlimm verletzt und wird in einigen Tagen geheilt
sein. Ich habe mich durch La ablösen lassen, um einmal hier nach
dem Rechten zu sehen. Ich glaube übrigens, daß er bei Bewußtsein
ist, er hat wiederholt die Augen geöffnet, doch ohne zu sprechen.
Hoffentlich hat er keine schwere Erschütterung davongetragen. Wir
wollen ihn nicht anreden, um ihn nicht vorzeitig aufzuregen. Wollen
Sie nicht einmal hinübergehen?“

„Recht gern, aber wer bleibt bei Saltner?“

„Der muß jetzt schlafen. Und dann müssen wir überhaupt eine andere
Einrichtung treffen. Wir bringen sie beide zusammen in ein Zimmer,
und zwar in das große. Aus der einen Seite lasse ich die abarische
Verbindung entfernen, desgleichen in den beiden Nebenräumen. Dort
werden ihre Betten und alle ihre Geräte hingebracht, so daß sie in
ihren gewohnten Verhältnissen leben können. Und wir können uns dann
bei ihnen aufhalten und sie studieren, ohne fortwährend unter
diesem Druck umherkriechen zu müssen, indem wir uns in dem andern
Teil des Zimmers die Schwere erleichtern.“

„Schön“, sagte Se, „aber ehe Sie meinen armen Bat einschläfern,
will ich noch einmal mit ihm verhandeln.“

Sie wandte sich zu Saltner und machte ihm so gut wie möglich
begreiflich, daß noch einer seiner Gefährten gerettet sei und daß
er ihn bald sehen solle. Dann brachte sie auf geschickte Weise in
Erfahrung, wie jener heiße, und ließ sich einige deutsche Worte so
lange vorsagen, bis sie sich dieselben eingeprägt hatte. Während
sie Saltner aus ihren großen Augen lächelnd ansah, streckte Hil die
Hand gegen sein Gesicht aus und bewegte sie einige Male hin und
her. Saltner fielen die Augen zu. Noch war es ihm, als wenn zwei
strahlende Sonnen vor ihm leuchteten, dann wußte er nicht mehr, ob
dies zwei Augen seien oder die Monde des Mars, und bald lag er in
traumlosem Schlaf.

\section{7 - Neue Rätsel}

Grunthe erwachte aus seiner Bewußtlosigkeit in einem Zimmer, das
ganz ähnlich eingerichtet war wie dasjenige, in welches man Saltner
gebracht hatte. Denn es gehörte zu derselben Reihe von Gastzimmern,
die für den vorübergehenden Aufenthalt von Martiern auf der
Erdstation bestimmt waren. Auch er konnte von seinem Lager aus
nichts erblicken als die großen Fensterscheiben, hinter denen das
Meer wogte, und den Wandschirm, der den übrigen Teil des Zimmers
verbarg. Dieser Schirm war ebenfalls mit einer Nachtlandschaft des
Mars verziert, welche beide Monde des Mars zeigte – ein bei den
Malern des Mars sehr beliebter Lichteffekt. Hier aber befanden sich
außerdem im Vordergrund zwei Figuren, von denen die eine nach einem
besonders hell leuchtenden Stern hinwies, während eine zweite das
stark vergrößerte Bild jenes Sternes beobachtete, wie es von einem
Projektionsapparat auf einer Tafel entworfen erschien.

Grunthe suchte seine Gedanken zu sammeln. Er lag sorgfältig
gebettet und in einem Schlafgewand, das nicht das seinige war, in
einem erwärmten Zimmer. Seinen Fuß, der ihn übrigens nicht
schmerzte, konnte er nicht bewegen; dieser befand sich in einem
festen Verband. Er fühlte sich matt, aber vollständig bei Sinnen
und ohne merkliche Beschwerden. Kopf und Arme, bis zu einem
gewissen Grad auch den Oberkörper, konnte er willkürlich bewegen.
Er war also nach seinem Sturz ins Wasser gerettet worden. Wo aber
befand er sich, und wer waren die Retter?

Die anfängliche Täuschung, daß er an der Stelle, wo der Schirm
stand, in eine wirkliche Nachtlandschaft sehe, konnte bei ihm nicht
lange anhalten, da diese Figuren enthielt, welche sich nicht
bewegten. Er hatte also ein Bild vor sich. Demnach war das Meer,
wie er auch aus der Farbe und Art der Beleuchtung schloß, wohl
nichts anderes als das Polarmeer, in welches der Ballon gestürzt
war, er befand sich auf der Insel, und seine Retter waren die
Bewohner dieser Insel. Wer waren sie, und was hatte er von ihnen zu
erwarten? Darauf konzentrierten sich alle seine Gedanken.

Er bewegte seine Arme, er beobachtete seine Atmung, seinen Puls, er
hörte das Rauschen des Meeres – alle Erscheinungen der Natur waren
unverändert, er war auf der Erde, und doch konnten die Wesen, die
hier wohnten, keine Menschen sein. Der Stoff seines Gewandes,
seiner Decke, seines Lagers war ihm vollständig unbekannt, daraus
konnte er keinen Schluß ziehen. Aber das Bild! Was stellte das Bild
vor? War es nicht möglich, daraus zu erkennen, in wessen Gewalt er
sich befand?

Die beiden Gestalten auf dem Bild waren, wie es schien,
menschlicher Art. Die stehende Figur, welche nach dem Stern
hinwies, sah nicht anders aus als eine ideale Frauengestalt mit
auffallend großen Augen; um ihren Kopf spielte ein seltsamer
Lichtschimmer – sollte dies eine symbolische Figur mit einem
Heiligenschein sein? Die Gewandung – soweit überhaupt von solcher
die Rede war – ließ keine Schlüsse zu, sie konnte ja einer Laune
des Künstlers entsprungen sein. Die sitzende Gestalt, welche das
Bild des Sternes beobachtete und dem Beschauer den Rücken zuwandte,
schien einen enganliegenden metallenen Panzer zu tragen; in der
Hand hielt sie einen Grunthe unbekannten Gegenstand. Sollten diese
beiden Figuren Vertreter der Bewohner der Polinsel sein? Aber die
Landschaft selbst war keine Landschaft der Erde. Also wohl eine
Erinnerung an die Heimat, aus welcher die Polbewohner stammten? Und
wenn es so war – diese beiden Monde –, sie konnten keiner andern
Welt angehören als dem Mars.

Bewohner des Mars haben den Pol besiedelt! Der Gedanke war Grunthe
schon einmal aufgestiegen, als er zuerst vom Ballon aus die Insel
mit ihren Vorrichtungen und dem merkwürdigen Kartenbild der Erde
betrachtet hatte. Er hatte ihn als zu phantastisch zurückgedrängt,
er wollte nichts mit so unwahrscheinlichen Hypothesen zu tun haben,
so lange er noch auf eine andere Erklärung hoffen konnte. Doch als
der Ballon von jener unerklärlichen Kraft in die Höhe gerissen
wurde, mußte er wieder an diese Hypothese denken. Und jetzt, die
merkwürdige Rettung, die seltsamen Stoffe, das Bild! Was war das
für ein Stern, der auf diesem Bild beobachtet wurde? Er strengte
seine scharfen Augen an, um die Abbildung auf der Tafel zu
erkennen. Eine hell beleuchtete schmale Sichel, der übrige Teil der
Scheibe in einem matten Schimmer – und diese dunklen Flecke, die
weißen Kappen an den Polen – kein Zweifel, das war die Erde, wie
sie vom Mars aus bei starker Vergrößerung erschien, die schmale
Sichel im Sonnenschein, das übrige schwach vom Mondlicht erhellt. –
Grunthe konnte sich nicht länger der Ansicht verschließen, daß er
bei den Marsbewohnern sich befinde – ein Gast, ein Gefangener – wer
konnte es wissen?

Wie konnten Marsbewohner auf die Erde kommen? Grunthe wußte die
Frage nicht zu beantworten. Nahm man aber die Tatsache einmal als
gegeben, so erklärten sich die andern Erscheinungen sehr leicht,
der Wunderbau der Insel, die Beeinflussung des Ballons, die
Rettung, die Einrichtung des Zimmers – die Hypothese der
Marsbewohner war doch wohl die einfachste –

Und auf einmal zuckte Grunthe zusammen – eine Erinnerung wurde ihm
plötzlich lebendig –, seine Lippen schlossen sich fest aufeinander,
und zwischen seinen Augenbrauen bildete sich eine tiefe senkrechte
Falte – Er spannte sein Gedächtnis aufs äußerste an –

Ell, Ell!, sagte er bei sich. – Was war es doch, was ihm Ell gesagt
hatte, ehe er die Reise antrat? Friedrich Ell, der Freund Torms,
lebte als Privatgelehrter in Friedau seinen Studien, aber er war
der eigentliche geistige und der pekuniäre Urheber der Expedition,
die Seele der internationalen Vereinigung für Polarforschung. Mit
ihm hatte er oft über die Möglichkeit disputiert, wie die Bewohner
des Mars mit der Erde in Verbindung treten könnten. Und Ell hatte
immer gesagt: Wenn sie kommen, so haben wir sie am Nordpol oder am
Südpol zu erwarten. Man springt auf einen Eisenbahnzug nicht, wo er
in Fahrt ist, sondern wo er steht. Wer weiß, was Sie am Pol finden!
Grüßen Sie mir die – – ja, das Wort hatte er vergessen. Er hatte
kein Gewicht darauf gelegt. Man wußte nicht immer bei Ell, ob er
scherze oder im Ernst spräche. „Grüßen Sie mir die –“ Es fiel ihm
nicht ein. Aber wohl erinnerte er sich, wie Ell eines Abends sehr
erregt geworden war, als man von den Bewohnern des Mars wie von
Fabelwesen gesprochen hatte. Er hatte dann das Gespräch plötzlich
abgebrochen.

Grunthe wurde aus seinem Nachsinnen gerissen. Hinter dem Bild der
Marslandschaft wurden Stimmen laut. Was war das für eine Sprache?
Grunthe kannte sie nicht, er verstand kein Wort.

Hinter dem Schirm hatte, von Grunthe unbemerkt, La gesessen. Es war
ihr sehr unbequem, unter dem Druck der irdischen Schwerkraft
auszuhalten, und sie hatte sich deshalb unbeweglich auf ihr Sofa
gestreckt. Jetzt kam Se schwerfällig herbei und ließ sich ebenfalls
nieder.

„Wie geht’s dem Bat?“ fragte sie.

„Ich weiß es wirklich nicht“, sagte La, „ich habe noch nicht
gehört, daß er sich bemerklich gemacht hätte, und unter diesem
Druck kannst du nicht verlangen, daß ich zu ihm hingehe.“

„So machen wir es leicht!“ rief Se und streckte die Hand nach dem
Griff des abarischen Apparates aus.

„Aber Hil hat es verboten“, erwiderte La. „Es könnte schädlich
wirken.“

„Ach, ich habe es drüben auch so gemacht, auf kurze Zeit tut es dem
Bat nichts. Hast du ihn denn schon gefüttert?“

„Nein, wie konnte ich?“

„Und doch ist es nötig, meint auch Hil. Und so lange müssen wir
mindestens uns frei bewegen können. Also, auf meine
Verantwortung.“

Se stellte den Apparat auf die normale Marsschwere ein. Die beiden
Damen erhoben sich und atmeten erleichtert auf.

In demselben Augenblick wollte Grunthe eine Bewegung ausführen,
aber sein Arm fuhr plötzlich viel höher, als er beabsichtigt hatte.
Sogleich probierte er die Bewegung noch einmal und konstatierte,
daß alle seine Gliedmaßen sowie die Decke seines Bettes viel
leichter geworden waren. Er suchte nach einem Gegenstand, den er in
die Höhe werfen wollte, um das wunderbare Phänomen zu studieren. Da
er jetzt trotz des Verbandes an seinem Fuß den Oberkörper leicht
aufrichten konnte, erblickte er auf einem Wandbrett über seinem
Lager einige Gegenstände, die ihm gehörten; man hatte sie offenbar
in seinen Taschen gefunden. Er ergriff sein Taschenmesser, hielt es
so hoch wie möglich über den Boden und ließ es fallen. Er konnte
den Fall bequem mit den Augen verfolgen; es dauerte eine Sekunde,
ehe das Messer den Boden erreichte. Grunthe schätzte die Höhe und
sagte sich: Die Schwerkraft ist geringer geworden, und zwar beträgt
sie nur etwa ein Drittel soviel wie gewöhnlich. Das ist die Schwere
auf dem Mars. Und wieder mußte er an Ell denken, der so oft gesagt
hatte: „Von der Schwere frei werden, heißt das Weltall
beherrschen.“

Auf das leichte Geräusch, welches das Auffallen des Messers
erzeugte, hatte Se den Wandschirm beiseite geschoben und war mit La
auf Grunthe zugetreten. Dieser hatte seine Aufmerksamkeit nicht
mehr auf den Schirm gerichtet und schrak daher mit einer Bewegung
der Überraschung zusammen, als er plötzlich die beiden schönen
Martierinnen vor sich sah. Kaum hatte er erkannt, daß sich zwei
lebendige weibliche Gestalten ihm näherten, so legte er sich mit
eisiger Miene zurück und heftete die Augen starr an die Decke. Da
er La und Se nicht anzusehen wagte, konnte er nicht bemerken, mit
welch freundlichen und teilnahmsvollen Blicken sie ihn
betrachteten. Nur an dem Ton der Stimmen, mit welchem sie in ihrer
Sprache einige Worte an ihn richteten, erkannte er die gute
Gesinnung. La zupfte ihm die Decke zurecht, Se aber beugte sich
über ihn und sah mit ihrem leuchtenden Blick tief in seine Augen.
Diese Damengesellschaft war ihm schrecklich; lieber hätte er sich
von feindlichen Wilden umgeben gesehen. Ach, und nun fühlte er eine
weiche Hand auf seinem Kopf, Se streichelte sein Haar – unwillig
stieß er die Hand zurück.

„Armer Mensch“, sagte Se, „er scheint noch ganz verwirrt. Wir
müssen ihm vor allen Dingen zu trinken geben.“ Sie legte die Hand
wieder auf seine Stirn und sagte: „Fürchte dich nicht, wir tun dir
nichts, armer Mensch.“

„Ko bat“, so lautete das letzte Wort Ses in ihrer Sprache, „Ko bat“
– es wirkte überraschend auf Grunthe –, das war einer der seltsamen
Ausdrücke Friedrich Ells. So pflegte Ell zu sagen, wenn er mit
einer seiner wunderlichen Ansichten nicht durchdringen konnte, wenn
er sein Mitleid mit dem Mangel an Verständnis bei den Menschen
bezeichnen wollte. Oft hatte ihn Grunthe gefragt, wo diese
Redensart herstammen wie er dazu käme. Dann hatte Ell immer nur
still gelächelt und wiederholt: „Ko bate, das versteht ihr nicht,
arme Menschen!“ Diese Erinnerungen waren mit dem Wort in Grunthe
wieder aufgetaucht. Er verhielt sich jetzt ganz ruhig.

Inzwischen hatte La ein Trinkgefäß herbeigeholt, mit dem
wunderbaren Nektar der Martier gefüllt. Die Martier tranken stets
durch einen mit Mundstück versehenen Schlauch, der in dem Gefäß
befestigt war, und dieses Mundstück versuchte La jetzt Grunthe
zwischen die Lippen zu schieben. Aber das war vergebliches Bemühen,
Grunthe hielt sie fest geschlossen und wandte sein Gesicht zur
Seite.

„Die Bate sind aber unliebenswürdige Geschöpfe“, sagte La lachend.

„Oh“, entgegnete Se, „Saltner war ganz anders, der redete gleich!“
Grunthe hatte den Namen erfaßt, jetzt öffnete er zum ersten Mal die
Lippen.

„Saltner?“ fragte er, ohne jedoch Se anzublicken.

„Ach“, sagte Se, „siehst du, er kann hören und sprechen. Nun paß
auf, nun werde ich einmal mit ihm reden.“

Sie schlug den Arm freundschaftlich um Las Schulter und stellte
sich nahe an das Lager. Dann sagte sie mit großer Anstrengung ihres
Sprachorgans die von Saltner gelernten deutschen Worte: „Saltner
deutsch Freund trinken Wein, Grunthe trinken Wein, deutsch
Freund.“

Grunthe warf jetzt einen erstaunten Blick auf die deutsch redende
Martierin, während La über die Worte, die ihr furchtbar komisch
vorkamen, kaum ihr Lachen verbergen konnte. Auch Grunthe war im
Begriff zu lächeln, als er aber die beiden verführerischen
Gestalten so nahe vor sich erblickte, starrte er sofort wieder an
die Decke, antwortete jedoch in höflichem Ton:

„Wenn ich recht verstehe, so ist auch mein Freund Saltner gerettet.
Sagen Sie, bitte, wo ich hier bin.“

„Trinken Wein, Grunthe“, wiederholte Se dringend, und La hielt ihm
das Mundstück vor das Gesicht.

Grunthe nahm jetzt die dargebotene Erfrischung. Und bald fühlte er
sich durch das Getränk aufs angenehmste erquickt und belebt. Er
bedankte sich und richtete noch einige Fragen an Se, aber ihre
Sprachkenntnisse waren nunmehr erschöpft. Grunthe sah ein, daß er
die Gebärdensprache zu Hilfe nehmen müsse, und so mußte er sich
wohl oder übel entschließen, die beiden Martierinnen anzusehen. Er
deutete auf sie hin, dann auf das Bild, und sagte auf gut Glück:
„Mars? Mars?“

Das Wort hatte Se behalten. Sie wiederholte es bejahend: „Mars,
Nu!“ Und auf La und sich hinweisend sagte sie, hochaufgerichtet:
„La, Se, Nume!“

Nume! Das war’s! Das war das Wort, das Ell ihm gesagt hatte: Grüßen
Sie die Nume!

Nume also nannten sich die Martier, und Ell hatte das gewußt – –
das gab Grunthe soviel zu denken, daß er momentan seine Umgebung
vergaß. Um in seinem Nachdenken nicht gestört zu werden, schloß er
die Augen, und sein Gesicht nahm wieder den starren Ausdruck an.

Se wollte ihn fragen, ob er essen wolle, aber das deutsche Wort,
das sie sich von Saltner hatte sagen lassen, war ihr entfallen. Da
Grunthe hartnäckig die Augen geschlossen hielt, begab sie sich in
den Hintergrund des Zimmers, um eine Mahlzeit zu bereiten. Denn
dies geschah in jedem Zimmer sehr einfach und schnell durch die
elektrische Küche. La zog es indessen vor, sich einen Sessel
herbeizuziehen und neben dem Lager Platz zu nehmen. Sie musterte
aufmerksam die Gegenstände, welche man bei Grunthe gefunden hatte,
und spielte mit einem kleinen Buch, das sich unter denselben
befand. Es war eine Anweisung zum Verkehr in der Eskimosprache und
zeigte als Titelvignette einen fein ausgeführten Holzschnitt, einen
Eskimo in seinem Kajak auf wildbewegtem Meer dem Fischfang
abliegend.

„Oh, sieh!“ rief sie Se zu, „hier ist ein Kalalek in seinem Kajak.“
Die beiden Eskimoworte schlugen verständlich an Grunthes Ohr und
weckten ihn aus seinem verworrenen Hinbrüten. Sollten die Martier
vielleicht das Grönländische erlernt haben? Er sagte sich, daß dies
ja leicht möglich sei. Er selbst hatte sich zum Zweck der Reise das
Notwendigste für den Verkehr angeeignet und fragte daher:

„Ich spreche einige Worte der Eskimos. Verstehen Sie mich?“

Se wußte nicht, was er meinte. La aber hatte schon seit einiger
Zeit ihre Studien auf die Sprache der einzigen Menschen gerichtet,
die den Martiern bisher begegnet waren. Sie verstand ihn und
antwortete:

„Ich verstehe ein wenig.“

„Wo sind meine Freunde?“ fragte Grunthe.

„Es ist nur einer da. Er ist in einem andern Zimmer.“

„Kann ich zu ihm?“

„Er schläft, aber man wird Sie dann zusammenbringen.“

„Wie kommen Sie vom Nu auf die Erde?“

La wußte die Antwort nicht auszudrücken. Sie fragte dagegen:

„Was wollt ihr hier?“

Grunthe wußte nun seinerseits nicht, wie er den Begriff ›Nordpol‹
im Grönländischen wiedergeben sollte. Er wollte sich in dem
Büchlein Rats erholen und sagte:

„Geben Sie mir das Buch.“

„Was?“ fragte La.

Grunthe richtete sich auf, um das Buch, das sie in der Hand hielt,
zu erfassen. Aber er hatte im Augenblick nicht an die veränderten
Schwereverhältnisse gedacht, und so kam es, daß sein ganzer
Oberkörper bis zu Las Platz hinüberschnellte. Er wäre aus dem Bett
gestürzt, wenn nicht La ihn rasch am Arm gefaßt und gehalten hätte.
Diese Berührung war nun für Grunthe höchst peinlich, er wollte sich
ihr entziehen, aber da er noch gar nicht verstand, seine Glieder
unter den veränderten Umständen zu gebrauchen, und außerdem durch
seinen Fuß behindert war, geschah es, daß er nach der andern Seite
zu fallen drohte und ihn La ganz mit ihren Armen umfassen mußte.
Sie legte ihn sanft auf das Lager zurück, während er ihr
teerosenfarbiges Haar dicht vor seinen Augen flimmern sah. Ein
Schwindel drohte ihn zu erfassen.

Se kam mit dem Speisegerät herangeschwebt.

„Was macht ihr denn?“ rief sie lachend. „Ich höre, daß ihr euch so
eifrig unterhaltet, ohne daß ich verstehen kann, was ihr redet, und
auf einmal –“

„Ja“, lachte La ebenfalls, „es war zu komisch, was der arme Bat für
Sprünge machte!“

„Ach“, sagte Se, „das scheint mir eine gefährliche Geschichte! Erst
wagt er dich nicht anzusehen, und wie ich den Rücken wende, macht
er dir auf grönländisch eine Liebeserklärung.“

„Daran bist du schuld, du hast die Schwere abgestellt. Wenn ihm der
Schreck nur nicht schlecht bekommt – was wird Hil sagen!“

Grunthe lag wieder regungslos. Sein allerdings ganz unverschuldetes
Ungeschick ärgerte ihn schwer, die Berührung mit der schönen La war
ihm entsetzlich, zudem bewirkte die Abnahme der Schwere jetzt ein
körperliches Übelbefinden.

Aber La bat ihn so freundlich in der Eskimosprache, doch zu essen,
und Se bot ihm die einen verlockenden Duft entwickelnden Speisen
mit einem so gebietenden Blick, daß er seinen Hunger zu stillen
wagte.

Mit größter Vorsicht richtete er sich auf und genoß einiges von den
Speisen, von denen er keine Ahnung hatte, was sie vorstellten.
Sobald er sich dankend zurücklehnte, schob Se das Speisetischchen
zur Seite, und La sagte:

„Leben Sie wohl, Grunthe, schlafen Sie.“

Sie schwebte zum Zimmer hinaus, und Se zog den Wandschirm vor, so
daß Grunthe wieder sich selbst überlassen blieb. Der Kopf schwirrte
ihm von allem, was um ihn herum vorgegangen war, neue Fragen
drängten sich auf, und er wollte sich eben noch einmal aufrichten;
da ergriff ihn plötzlich ein Gefühl, als würde er mit Gewalt gegen
sein Lager gedrückt. Se hatte die Erdschwere wieder hergestellt.
Jetzt schoben sich dichte Vorhänge vor die Fenster, und Grunthe lag
im Dunkeln. Eine leise Musik wie aus weiter Ferne ließ sich hören
und mischte sich melodisch mit dem Rauschen des Meeres.

Grunthe versuchte vergebens, seine Gedanken zu konzentrieren. Sie
bewegten sich um die Frage, wie es lebenden Wesen möglich sein
könne, den luftleeren Weltraum zu durchfliegen. Wie hatten Martier
auf die Erde kommen können? Aber nur in unbestimmten Vermutungen
irrten seine Vorstellungen umher, und ehe er zu irgendeiner
Klarheit in dieser Frage gelangte, lösten sich seine Gedanken in
undeutliche Traumbilder auf. Grunthe entschlummerte.

\section{8 - Die Herren des Weltraums}

„Dreifach panzerten Mut und Kraft

Dem das eiserne Herz, der sich zuerst gewagt

Im gebrechlichen Boot hinaus

Auf das tückische Meer. dots{}“

So pries einst Horaz die Kühnheit des Seefahrers, der dem fremden
Element sein unsicheres Fahrzeug anvertraute. dots{} Aber unbedenklich
besteigt der Tourist den luxuriösen Bau des Riesendampfers, um in
wenigen Tagen die wohlbekannte Ozeanstraße zu durchmessen.

Ähnlich rühmte ein Dichter des Mars den Mut und den Scharfsinn
jenes Martiers Ar, der es einst gewagt, auf den Wegen des Lichts
und der kosmischen Schwere in die Leere des Raumes seinen
unvollkommenen Apparat zu werfen, um zum ersten Male den Flug zu
versuchen durch den Weltäther nach dem leuchtenden Nachbarstern,
der strahlenden ›Ba‹, dem Schmuck der Marsnächte, der
jahrtausendlangen Sehnsucht aller ›Nume‹. Jetzt aber kannte man auf
dem Mars genau die Mittel, welche die Marsbewohner, die sich selbst
›Nume‹ nannten, anwenden mußten, die einzelnen Umstände, auf die
sie zu achten hatten, um je nach der Stellung der Planeten die
strahlende Ba, das ist die Erde, zu erreichen. Wohl war eine Reise
zwischen Mars und Erde noch immer ein zeitraubendes und
kostspieliges Unternehmen, aber es hatte seinen ebenso sicheren und
bequemen Gang wie etwa heutzutage für einen Menschen eine Reise um
die Erde.

Die Erforschung der Erde, die Entdeckung des intraplanetaren Weges
nach derselben und die endliche Besitzergreifung vom Nordpol bildet
ein umfangreiches und wichtiges Kapitel in der Kulturgeschichte der
Martier.

Die Durchsichtigkeit der Atmosphäre auf dem Mars hatte seine
Bewohner frühzeitig zu vorzüglichen Astronomen gemacht. Mathematik
und Naturwissenschaft waren zu einer Höhe der Entwicklung gelangt,
die uns Menschen als ein fernes Ideal vorschwebt. Je mehr der
alternde Mars durch seinen verhältnismäßig geringen Wasservorrat
die Existenzbedingungen der Martier erschwerte, um so großartiger
waren die Anstrengungen gewesen, durch welche die Martier die
Technik der Naturbeherrschung ausbildeten. Immer neue Kräfte und
Hilfsmittel wußten sie ihrem Planeten zu entlocken, der sich
freilich durch die Eigentümlichkeit seines Baues in noch viel
höherem Maß zur Erziehung eines Kulturvolkes eignete als die Erde.

Der Tag auf dem Mars hat fast dieselbe Dauer wie auf der Erde, er
ist nur vierzig Minuten länger. Das Jahr des Mars dagegen umfaßt
670 Mars-, das sind 687 Erdentage, ist also fast doppelt so lang
als ein Erdenjahr. Die gesamte Oberfläche des Mars beträgt etwa nur
ein Viertel von derjenigen der Erde. Die südliche Halbkugel des
Mars ist die wasserreichere und daher am stärksten bevölkert; sie
enthält auch die beiden einzigen Meere, welche der Mars besitzt,
wenn man darunter diejenigen Becken versteht, welche das ganze Jahr
hindurch mit Wasser erfüllt sind. Die nördliche Halbkugel besteht
zum größten Teil aus unfruchtbaren Wüsten. Aber die Bevölkerung des
Mars, der die von der Natur genügend bewässerte Region ihres
Planeten längst zu klein geworden, wußte der kargen Natur neue
Gebiete des Anbaus abzugewinnen. Sie durchzog das gesamte
Wüstengebiet mit einem vielverzweigten Netz geradliniger breiter
Kanäle und verteilte auf diese Weise zur Zeit der Schneeschmelze,
im Beginn des Sommers einer je hatten. Die einzelnen Völkerschaften
bildeten einen großen Staatenbund. Wie auf der Erde der Weltverkehr
sich durch der einzelnen politischen Verbände darunter litt, so
hatte die vorgeschrittenere Zivilisation der Martier in ihrer
internationalen Vereinigung ein Zentralorgan, das unbeschadet der
Freiheit der Einzelgemeinden alle Angelegenheiten regulierte,
welche für die Bewohner des ganzen Planeten ein gemeinsames den
Halbkugel, das Wasser, welches sich in Gestalt von Schnee an den
Polen angehäuft hatte, über den ganzen Planeten. Wie die Ägypter
das Anwachsen des Nils benutzten, um der Wüste den fruchtbaren
Boden des Niltals abzugewinnen, so tränkten die Marsbewohner durch
ihre Kanäle beide Ufer derselben. Schnell schoß hier eine üppige
Vegetation auf, und so wurde durch das Kanalnetz das ganze
Wüstengebiet mit fruchtbaren, an hundert Kilometer breiten
Vegetationsstreifen durchzogen, die eine ununterbrochene Kette
blühender Ansiedlungen der Martier enthielten. Wenn hier die
dunkelgrünen Blätter der Pflanzen mit einem Schlag hervorsproßten,
dann hoben sich diese Streifen dunkel von dem rötlichen Wüstenboden
ab, und die Astronomen der Erde wunderten sich, woher dieses
regelmäßige Netz von Streifen auf dem Mars wohl stammen möchte. Die
Riesenarbeit der Bewässerung des Planeten war eine Notwendigkeit
für die Martier geworden, nachdem die in der Kultur
vorgeschritteneren Bewohner der Südhalbkugel allmählich den ganzen
Planeten ihrer Herrschaft unterworfen internationale Verträge
regelte, ohne daß die Selbständigkeit Interesse besaßen.

Nachdem die Oberfläche des Planeten vollständig erforscht und
besiedelt war, richtete sich die Aufmerksamkeit der Martier
naturgemäß stärker wie je über die Grenzen ihres Wohnplatzes hinaus
auf ihre Nachbarn im Sonnensystem. Und was konnte sie hier
mächtiger fesseln als die strahlende Ba, die sagenumwobene Erde,
die bald als Morgen-, bald als Abendstern alle andern Sterne ihres
dunklen Himmels überstrahlt?

Die Ruhe und Durchsichtigkeit der Atmosphäre gestattete ihnen, bei
ihren Fernrohren Vergrößerungen zu benutzen, wie sie auf der Erde
unmöglich waren. Denn auf der Erde vereitelt die stets
ungleichmäßig bewegte Luft, daß wir Instrumente von so starken
Vergrößerungen praktisch anzuwenden vermöchten, als wir sie wohl
theoretisch und technisch konstruieren könnten. Der Druck der
Atmosphäre auf dem Mars ist aber so gering, wie wir ihn nur auf den
allerhöchsten Berggipfeln der Erde besitzen, und die über der
Marsoberfläche lastende Luftschicht ist dementsprechend dünner und
durchsichtiger. Die Astronomen des Mars konnten daher, bei
günstiger Stellung der Planeten gegeneinander, die Erde ihrem Auge
so nahe bringen, als wäre sie nur gegen zehntausend Kilometer weit
entfernt, und vermochten somit noch Gegenstände von zwei bis drei
Kilometer Ausdehnung zu erkennen. Unter diesen Umständen hatten sie
natürlich bemerkt, daß sich auf der Erde Einrichtungen finden, die
nur als das Werk intelligenter Wesen zu erklären seien. Auch
durchschauten sie viel zu klar den Bau und die Natur der Erde sowie
die Analogien im gesamten Sonnensystem, als daß sie nicht die
Überzeugung von der Bewohnbarkeit der Erde und einer gewissen
Kultur der Erdbewohner gehabt hätten. Die Karte der Erde selbst war
ihnen in umfassenderer Weise bekannt als uns Menschen; denn von
ihrem Standpunkt aus konnten sie nach und nach alle jene Gebiete
der Erdkugel, insbesondere die Polargegenden, durchmustern, die
bisher unseren irdischen Forschungen verschlossen geblieben sind.

Es hatte nicht an Versuchen der Martier gefehlt, sich mit den von
ihnen vermuteten Erdbewohnern in Verbindung zu setzen. Aber die
gegebenen Zeichen waren wohl nicht bemerkt oder nicht verstanden
worden. Jedenfalls mochten die Erdbewohner nicht in der Lage sein,
darauf zu antworten. Die Erde war ein sehr viel jüngerer Planet und
in ihrer ganzen Entwicklung auf einer Stufe, wie sie der Mars schon
vor Millionen Jahren durchlaufen hatte. Da sagten sich die
Marsbewohner selbstverständlich, daß die Bate, wie sie die
hypothetischen Bewohner der Erde nannten, jedenfalls auf einem viel
niedrigeren Standpunkt der Kultur ständen als sie, die Nume; ja wer
weiß, ob sie sich überhaupt schon bis zur Höhe der ›Numenheit‹, zur
Vernunftidee der Martier, erhoben haben!

Um jene Zeit, als man auf der Erde von einem Jahrhundert der
Naturwissenschaft zu sprechen anfing, blickten die Martier längst
nicht nur auf das Zeitalter des Dampfes, sondern auch auf das
Zeitalter der Elektrizität wie auf ein altes Kulturerbe zurück.
Damals vollendete sich bei ihnen eine wissenschaftliche Entdeckung,
die eine Umgestaltung aller Verhältnisse nach sich zu ziehen
geeignet war. Die Enthüllung des Geheimnisses der Gravitation war
es, die einen ungeahnten Umschwung der Technik herbeiführte und die
Martier zu Herren des Sonnensystems machte.

Die Gravitation ist jene Kraft, welche die Bewegungen der Gestirne
im Weltraum beherrscht. Sie verbindet die Sonne mit den Planeten,
die Planeten mit ihren Monden, sie hält die Gegenstände an der
Oberfläche der Weltkörper fest und bewirkt, daß diese als dauernde
einheitliche Gruppen im Universum bestehen; sie läßt den geworfenen
Stein wieder zur Erde fallen und die Gewässer nach dem Meer hin
sich sammeln. Sie ist eine allgemeine Eigenschaft der Körper,
welche von ihrer gegenseitigen Lage im Raum abhängt; die Arbeit,
welche ein Körper infolge der Gravitation zu leisten vermag, nennt
man daher Raumenergie.

Wenn es gelänge, einem Körper diese eigentümliche Form der Energie
zu entziehen, die er infolge seiner Lage zu den übrigen Körpern,
insbesondere zu den Planeten und der Sonne besitzt, wenn es
gelänge, seine Gravitation in eine andere Energieform überzuführen,
so würde man diesen Körper dadurch unabhängig von der Schwerkraft
machen; die Schwerkraft würde durch ihn hindurch oder um ihn
herumgehen, ohne ihn zu beeinflussen; er würde ›diabarisch‹ werden.
Er würde ebensowenig von der Sonne angezogen werden wie ein Stück
Holz vom Magneten. Dann aber müßte es ja auch gelingen, den Körper
dem Einfluß der Planeten und der Sonne soweit zu entziehen, daß man
ihn im Weltraum frei bewegen könnte; dann also müßte es gelingen,
den Weg von einem Planeten zum andern, von dem Mars zur Erde zu
finden.

Dies war den Martiern gelungen. Sie vermochten Körper von gewisser
Zusammensetzung herzustellen, so daß jede auf sie treffende
Schwerewirkung spurlos an ihnen und an den von ihnen umschlossenen
Körpern vorüberging – das heißt spurlos als Schwere. Die
Gravitationsenergie wurde in andere Energieformen umgewandelt.
Solche Körper können wir ›diabarisch‹ nennen.

Zwei Umstände hatten es den Martiern erleichtert, dem Geheimnis der
Gravitation auf die Spur zu kommen. Der eine lag darin, daß die
Schwerkraft auf ihrem Planeten nur ein Drittel von demjenigen Werte
beträgt, den sie auf der Erde besitzt. Eine Last, die auf der Erde
tausend Kilogramm wiegt, hat, auf den Mars gebracht, nur ein
Gewicht von 376 Kilogramm; ein freifallender Körper, der bei uns in
der ersten Sekunde 5 Meter herabfällt, fällt auf dem Mars in dieser
Zeit nur um 1,8 Meter und kommt mit der sanften Geschwindigkeit von
3,6, statt bei uns mit fast 10 Meter, an. Infolgedessen war es den
Martiern erleichtert, alle Eigentümlichkeiten der Schwere bequemer
und genauer zu studieren.

Der zweite Umstand war ein geographischer, oder, wie wir beim Mars
sagen müßten, ein areographischer, nämlich die Zugänglichkeit der
Pole des Mars. Während auf der Erde die Pole mit ihrer ewigen
Eisdecke des Besuches sich erwehren, sind die Marspole nicht
vergletschert. Zwar bedecken sie sich im Winter mit einer dichten
Schneehülle, die aber doch viel geringer ist als auf der Erde, weil
die Atmosphäre des Mars viel weniger Feuchtigkeit enthält. Außerdem
dauert der Sommer fast ein volles Erdenjahr, währenddessen der Pol
in fortwährendem Sonnenschein liegt, so daß der Schnee zum größten
Teil fortschmilzt. Die Pole des Mars sind daher den Marsbewohnern
nicht nur bekannt, sondern sie haben gerade auf ihnen ihre
bedeutendsten wissenschaftlichen Stationen angelegt. Denn die Pole
eines Planeten sind ausgezeichnete Punkte, sie unterliegen nicht
der Umdrehung um die Achse im Verlauf eines Tages, und sie bieten
dadurch Gelegenheit zu Beobachtungen, die sich an keiner anderen
Stelle so einfach anstellen lassen.

Gerade nun für die Untersuchung der Schwerkraft zeigte sich dies
von größter Wichtigkeit. Ihre Wirkungen im Kosmos zu studieren, das
heißt ihre Wechselwirkung mit andern kosmischen Kräften, mußte man
sich von der Rotation des Planeten um seine Achse und allen dadurch
entstehenden Komplikationen unabhängig machen. Dies konnte nur am
Pol geschehen. Vom Pol gingen denn auch die Untersuchungen der
Martier aus.

Die Martier hatten entdeckt, daß die Gravitation, ebenso wie das
Licht, die Wärme, die Elektrizität, sich in Form einer
Wellenbewegung durch den Weltraum und die Körper fortpflanzt.
Während aber die Geschwindigkeit der strahlenden Energie, die wir
als Licht, Wärme und Elektrizität beobachten, 300.000 Kilometer in
der Sekunde beträgt, ist diejenige der Gravitation eine
millionenmal größere. Nach den Berechnungen der Martier durchläuft
die Gravitation den Raum mit einer Geschwindigkeit von 300.000
Millionen Kilometern pro Sekunde, sie verhält sich also zu
derjenigen des Lichts etwa so wie die des Lichts zur
Geschwindigkeit des Schalls. Den Weg von der Sonne bis zur Erde
legt somit die Wirkung der Schwere in einem halben Tausendteil
einer Sekunde zurück; kein Wunder, daß es den Astronomen der Erde
nicht gelungen war, die von ihnen allerdings vermutete endliche
Geschwindigkeit der Gravitation zu konstatieren.

Einen Körper, der die Lichtwellen nicht durch sich hindurchgehen
läßt, nennen wir undurchsichtig; ließe er sie vollständig
hindurchgehen, so würde er absolut durchsichtig sein, wir würden
ihn so wenig sehen wie die Luft. Ein Körper, der die Wärmewellen
durch sich hindurchgehen läßt, bleibt kalt; er muß sie in sich
aufnehmen, sie absorbieren, um sich zu erwärmen. So ist es nun, wie
die Martier entdeckten, auch mit der Gravitation. Die Körper sind
darum schwer, weil sie die Gravitationswellen absorbieren. Körper
ziehen sich nur dann gegenseitig an, wenn sie die von ihnen
wechselseitig ausgehenden Gravitationswellen nicht durch sich
hindurchtreten lassen. Sobald aber ein Körper so beschaffen ist,
daß er die Gravitationswellen eines Planeten oder der Sonne nicht
aufnimmt, sondern frei durchläßt, so wird er nicht angezogen, er
hat keine Schwere, er ist diabar, schweredurchlässig, und dadurch
schwerelos.

Die Martier hatte gefunden, daß das Stellit, ein auf ihrem Planeten
vorkommender Körper, sich so verändern läßt, daß die Schwerewellen
hindurchtreten können. Und mit diesem Augenblick wurde dieser
Körper vom Mars wie von der Sonne nicht mehr angezogen. Allerdings
ließ es sich nicht erreichen, absolut schwerelose Körper
herzustellen, wie es ja auch keine absolut durchsichtigen Körper
gibt; wohl aber ließ sich die Schwere so vermindern, daß sie nur
kaum merklich auf den diabaren Körper wirkt. Indem man die
Schwerelosigkeit verstärkte oder verminderte, konnte man nun, wenn
einmal der Körper eine bestimmte Geschwindigkeit besaß, durch
passende Benutzung der Anziehung der Planeten und der Sonne die
Bahn des Körpers im Weltraum regulieren – vorausgesetzt, daß man
sich in einem solchen diabaren Körper befand, in einer Kugel aus
Stellit.

Dieses Wagestück, einen Apparat herzustellen, in welchem ein Mensch
sich in den Weltraum schleudern lassen konnte, um dann durch
Regelung der Anziehung, welche die Weltkörper auf ihn ausübten,
seinen Weg zu lenken, das hatte zuerst der Martier Ar unternommen.
Aber man hatte ihn nie wiedergesehen. War er in die Fixsternwelt
jenseits des Sonnensystems hinausgeflogen? War er in die Sonne
gestürzt? Umkreiste sein Raumschiff die Sonne oder irgendeinen
Planeten als ein neuer Trabant? Niemand wußte es.

Aber andere kühne Forscher ließen sich nicht zurückschrecken. Sie
hatten jetzt die theoretische Möglichkeit des interplanetaren
Verkehrs eingesehen, es war jetzt keine Tollkühnheit mehr, sich dem
Raum anzuvertrauen, sondern eine dringende Aufgabe der Kultur und
somit eine sittliche Forderung, eine Pflicht der ›Numenheit‹. Die
größte Schwierigkeit lag nur darin, die Geschwindigkeit unschädlich
zu machen, welche der Planet in seiner eigenen Bahn besaß und die
sich natürlich auf das schwerelose Raumschiff übertrug, sobald es
den Mars verließ. Man reiste von einem der Pole ab, um von der
Rotation des Planeten unabhängig zu sein, aber die Geschwindigkeit
des Mars in seiner Bahn beträgt 24 Kilometer in der Sekunde, und
mit dieser flog man hinaus in den Raum, fort von der Sonne in der
Richtung der Tangente der Marsbahn. Es kam dann darauf an, sich der
Sonnenanziehung in dem richtigen Augenblick zu überlassen, um durch
die Flugbahn des Raumschiffs in den Anziehungsbereich der Erde zu
gelangen. Man war somit ganz auf die vorhandenen Gravitationskräfte
angewiesen, wie ein Schiff auf dem Meer auf die Richtung der
Wasser- und Luftströmungen; und auf einen weiteren Erfolg konnte
man erst hoffen, wenn es auch noch gelang, Mittel zu finden, die
Richtung der erhaltenen Geschwindigkeit willkürlich abzulenken.

Aber auch dieses Problem war allmählich gelöst worden. Die
Geschichte der menschlichen Entdeckungen auf der Erdoberfläche war
nicht weniger reich an Opfern als diejenige der Versuche der
Martier, den Weltenraum zu durchsegeln. Endlich aber war einmal
nach jahrelangem Ausbleiben ein Raumschiff zurückgekehrt, das die
Erde dreimal in großer Nähe umflogen hatte. Ein anderes war auf dem
Mond der Erde gelandet. Zuletzt war es dem rastlosen Erdforscher
Col auf seiner dritten Raumreise gelungen, den Nordpol der Erde zu
erreichen. Der Südpol wurde zuerst vom Kapitän All betreten. Von
jetzt ab verkürzte sich immer mehr die Reisezeit nach der Erde
durch die vervollkommnete Technik der Raumfahrt, anstelle der
vereinzelten Entdeckungsreisen trat eine planmäßige Besetzung des
Nordpols. Und nachdem durch Konstruktion der Außenstation und die
Errichtung des abarischen Feldes die Landung auf der Erde ebenso
gesichert war wie die eines Dampfschiffes im Schutz eines
trefflichen Hafens, waren die Martier an dem ersehnten Ziel
angelangt, die Erde nach Belieben besuchen zu können.

Nur freilich, die beiden Pole waren bis jetzt die einzigen Punkte,
welche sie zu erreichen vermochten. Am Südpol hatten sie eine
ähnliche, wenn auch kleinere und weniger benutzte Station angelegt
wie am Nordpol. Denn nur während des Sommers der Nordhalbkugel
konnten sie die Nordstation unterhalten. Im Winter verlegten sie
das abarische Feld auf

den Südpol, der zu dieser Zeit Sommer hatte. Dagegen war es ihnen
noch nicht gelungen, zu den bewohnten Teilen der Erde vorzudringen.
Noch niemals hatten sie einen zivilisierten Menschen kennengelernt.
Einige Eskimos waren die einzigen Vertreter, nach denen sie die
Eigentümlichkeiten der Erdbewohner zu beurteilen vermochten. Aber
bei ihren Umkreisungen der Erde in der Entfernung von einigen
tausend Kilometern zeigten ihnen ihre vorzüglichen Instrumente
natürlich die Einrichtungen der Kultur in solcher Deutlichkeit, daß
sie sehr wohl wußten, die Hervorbringer dieser Werke seien keine
Eskimos. Doch an andern Stellen als an den Polen zu landen, war
ihnen bisher nicht gelungen. Durch die Rotation der Erde wurden die
Verhältnisse dort so kompliziert, daß die technischen
Schwierigkeiten nicht überwunden werden konnten. Diese ergaben sich
aus der besonderen Natur der Gravitation und dem dadurch bedingten
Bau der Raumschiffe, welche dem Druck der Luft und ihren Stürmen
nicht widerstehen konnten. Auch am Pol war ja die Landung erst mit
Sicherheit durchzuführen, seitdem es nach vielen Opfern und
Verlusten gelungen war, die Außenstation zu errichten und so die
Raumschiffe außerhalb der Atmosphäre zu halten. Wie die Brandung
einer Insel gegen die Überrumpelung durch landende Feinde schützt,
so deckte die Umdrehung um ihre Achse und die Dichtheit ihrer
Atmosphäre die Erde gegen einen plötzlichen Einfall der
Marsbewohner von der Luftseite her. Nur am Pol konnten sie sich
festsetzen. Und wenn sie nun auf der Erde vordringen wollten, so
mußte dies über die Gletscher und Eisschollen der Polargegenden
geschehen.

Mit diesem Plan trugen sich nun freilich die Marsbewohner. Aber die
Überwindung dieser Eiszonen bot ihnen ebensoviel Schwierigkeiten,
als wenn Europäer in das vernichtende Sumpfklima eines tropischen
Urwaldes oder über die wasserlose Wüste vordringen wollten. Unsere
Schiffe tragen uns wohl ans Ufer unbekannter Länder, aber in das
Innere vermögen wir erst später und unter den größten
Schwierigkeiten einen Einblick zu gewinnen. Die Martier hatten auf
der Erde vor allem mit zwei gewaltigen Hindernissen zu kämpfen:
Luft und Schwere. Die Dichtigkeit der Luft, ihre Feuchtigkeit und
die Größe des Luftdrucks waren für die Konstitution ihres Körpers
verderblich; sie konnten das Klima der Erde nur kurze Zeit
ertragen. Und die Stärke der Schwerkraft, dreimal so groß wie auf
dem Mars, hinderte ihre Bewegungen und drückte jeder ihrer
mechanischen Arbeiten eine dreifache Last auf. Sie hätten dieselbe
überhaupt nicht tragen können, wenn sie nicht für die Verhältnisse
ihres Planeten eine sehr bedeutende Muskelkraft besessen hätten.
Gerade jetzt, als die Nordpolexpedition Torms in ihrem abarischen
Feld scheiterte, waren sie mit den ernstesten Vorbereitungen
beschäftigt, einen Vorstoß nach Süden zu unternehmen. Denn auf dem
Mars waren die Versuche gelungen, einen Stoff herzustellen, der
sich wie das Stellit schwerelos machen ließ, aber dabei genügende
Festigkeit besaß, der Wärme und Feuchtigkeit der Luft zu
widerstehen. Von ihm erhofften die Martier, daß er ihnen die Wege
durch die Erdenluft bahnen werde.

\section{9 - Die Gäste der Marsbewohner}

Als Saltner zum zweiten Mal auf der Insel erwachte, war er nicht
wenig erstaunt, sich wieder in einer völlig veränderten Situation
zu finden. Das Zimmer war verdunkelt, doch schimmerte die Decke
desselben in einem matten Grau, so daß er einigermaßen seine
Umgebung erkennen konnte. Er sah sofort, daß er in einen anderen
Raum gebracht worden war. Fenster waren nicht vorhanden, und das
Rauschen des Meeres vermochte er nicht zu hören. Dagegen sah er in
der Nähe seines Bettes mehrere Körbe und Pakete aufgestapelt, in
denen er einen Teil aus dem Inhalt der Gondel des untergegangenen
Ballons zu erkennen glaubte. Wenn es nur etwas heller gewesen wäre!
Aber wie konnte man Licht erhalten?

Er erhob erst vorsichtig seinen Arm, um nicht etwa wieder einen
unfreiwilligen Luftsprung zu machen, und als er merkte, daß er sich
unter den gewöhnlichen Umständen der Erdschwere befand, setzte er
sich mit einem lebhaften Schwung auf den Rand seines Lagers. Und
siehe da, in dem Augenblick, in welchem seine Füße den Boden
berührten, wurde es hell im Zimmer. An der Decke hatte sich eine
weite Oberlichtöffnung gebildet, und die Sonne, nur durch einen
leichten Schirm gedämpft, schien fröhlich herein. Er erkannte nun,
daß er in der Tat das Eigentum der Expedition vor sich hatte, auch
seine sorgfältig gereinigte und getrocknete Kleidung fand er dabei.
Und am Boden lag sogar sein Eispickel, den er zu etwaigen
Gletscherbesteigungen am Nordpol mitgenommen hatte. Schnell erhob
er sich, und schon bei den ersten Schritten, die er auf dem weichen
Teppich des Zimmers probierte, fühlte er, daß er sich wieder völlig
wohl und bei Kräften befand. Er schob einen Vorhang zu seiner
Linken beiseite und sah dahinter verschiedene Geräte, die ihm
höchst fremdartig vorkamen, aber doch soviel erraten ließen, daß
sie einen Bade-Apparat vorstellten und was sonst zur Toilette eines
Martiers gehören mochte. Ehe er sich jedoch getraute, von diesen
ihm unbekannten Gegenständen Gebrauch zu machen, untersuchte er
erst die übrigen Teile des geräumigen Schlafgemachs. In der Mitte
der dem Bett gegenüberliegenden Wand befand sich eine große Tür,
die er indessen vorläufig nicht zu öffnen wagte. Er wandte sich
Freude – Grunthe, der in ruhigem Schlummer lag. nun nach rechts und
bemerkte, daß die Täfelung dieser Seitenwand ebenfalls eine Tür
enthielt, die aber nicht ganz geschlossen war. Sie führte in ein
verdunkeltes Gemach. Als Saltner an dem ihm unbekannten Mechanismus
herumtastete, rollte sich die Tür auf und ließ dadurch mehr Licht
in das Zimmer. Da erblickte er an der gegenüberliegenden Wand ein
Bett genau wie das seinige und erkannte zu seiner
unaussprechlichen

„Guten Morgen, Doktor“, rief Saltner ohne weiteres. „Wie geht’s?“

Grunthe schlug verwundert die Augen auf.

„Saltner?“ sagte er.

„Hier sind wir, munter und gesund, wer hätte das gedacht! Aber der
arme Torm – niemand weiß etwas von ihm!“

„Und wissen Sie denn“, fragte Grunthe, sogleich ermuntert, „wo wir
uns befinden?“

„Ich weiß es, aber Sie werden’s freilich nicht glauben wollen. Oder
haben Sie etwa schon mit dem biedern Hil oder der schönen Se
gesprochen?“

„Wir sind in der Gewalt der Nume“, antwortete Grunthe finster.
„Sind wir allein?“

„Soviel ich weiß, aber der Teufel traue diesen Maschinerien – wer
kann wissen, ob man nicht von irgendwo alles hört und sieht, was
hier vorgeht, oder ob nicht irgendein geheimer Phonograph jedes
Wort protokolliert. Na, deutsch verstehen sie vorläufig noch
nicht.“

„Welche Zeit haben wir? Wie lange war ich bewußtlos?“

„Ja, wenn Sie das nicht wissen! Ich denke, hier gibt es überhaupt
keine Zeit.“

„Nun, das wird sich alles bestimmen lassen, wenn wir erst einmal
den freien Himmel wiedersehen“, sagte Grunthe. „Aber wie kann man
hier Licht machen?“

„Treten Sie gefälligst mit Ihren Füßen auf den Boden vor Ihrem
Bett, dann wird es Tag. Wir sind hier im Lande der automatischen
Bedienung.“

„Das kann ich nicht, bester Saltner, mein Fuß ist verwundet–“

„I – das wäre – lassen Sie sehen –“

„Es ist nichts, ich bin schon verbunden, aber ich muß vorläufig
noch liegen bleiben.“

Saltner war inzwischen an Grunthes Bett geeilt, und in dem Moment,
in welchem er den Teppich vor demselben betrat, öffnete sich das
Oberlicht.

„Sehen Sie“, rief Saltner, „allmählich lernt man diese Marskniffe.
Ich kann übrigens schon etwas martisch und werde Ihnen gleich ein
Frühstück bestellen. Erlauben Sie nur, daß ich vorher ein wenig
Toilette mache.“

Er eilte nach dem Alkoven, der offenbar als Toilettenzimmer dienen
sollte, und stellte sich überlegend vor die Apparate.

„Das da scheint mir eine Badewanne“, sagte er, während Grunthe
durch die Tür sein halblautes Selbstgespräch vernahm, „aber Wasser
ist nicht darin. Und dies dürfte wohl einen Waschtisch vorstellen.
Aber hier sind drei verschiedene Griffe, und jeder hat eine
Aufschrift – nur daß ich sie nicht lesen kann. Ich kenn mich halt
nicht aus. Na, ich werde mal ein bissel drehen. Vielleicht kommt
ein Wasser heraus.“

Er drehte vorsichtig an dem einen Wirbel, in der Meinung, das
darunter befindliche flache Becken werde sich auf irgendeine Weise
mit Wasser füllen. Aber ehe er sich’s versah, sprang das Becken,
sich fächerförmig zu einem Tisch ausbreitend, hervor und versetzte
ihm einen unhöflichen Schlag gegen den Magen. Mit Hallo sprang er
zurück, fand sich aber sofort wieder stolpernd nach vorn
geschnellt, denn gleichzeitig hatte sich in seinem Rücken ein
Sessel aus dem Fußboden erhoben. Nachdem er sich von seinem ersten
Schreck erholt, betrachtete er sich die Sache eingehend, probierte
an dem Tisch und Sessel, und da er sie standfest fand, ließ er sich
gemütlich auf dem Sessel nieder.

„Was gibt’s denn?“ fragte Grunthe von seinem Bett aus.

„Ein Wasser war’s nicht“, sagte Saltner, „aber es sitzt sich ganz
gut hier. Nun wollen wir einmal den zweiten Wirbel probieren.“ Doch
schnell sprang er wieder auf, er dachte, der zweite Handgriff könne
vielleicht dazu dienen Tisch und Sessel wieder verschwinden zu
lassen, und bei dieser Gelegenheit wollte er sich erst in
Sicherheit bringen. Aber es kam anders. Er erhielt nur von einer
aufspringenden Schublade einen Stoß an die Hand. Die Schublade
enthielt eine Anzahl jener Mundstücke, deren sich die Martier, wie
Saltner wußte, zum Trinken bedienten, und nun bemerkte er auch, daß
oberhalb des Tisches drei Öffnungen freigeworden waren, in welche
die Mundstücke hineinpaßten.

„Halt“, sagte Saltner, „hier gibt’s was zu trinken. Aber damit
wollen wir doch noch warten.“

Er drehte an dem dritten Griff. Eine muldenförmige Schale wurde
sichtbar, und in dieselbe fielen aus einer darüber befindlichen
Öffnung fingerdicke, hellbraune Gegenstände, welche etwa die
Gestalt von kleinen Würsten hatten.

„Das ist also ein Frühstück und keine Toilette“, rief Saltner
lachend und probierte die sehr einladend aussehenden und würzig
duftenden Stangen. Sie schmeckten vorzüglich und erwiesen sich als
ein knuspriges Gebäck, das mit einer kräftigen Fleischfarce gefüllt
war. Wenigstens hielt Saltner sie dafür. Aber während er die erste
Stange verzehrte, setzte der Apparat seine Tätigkeit fort, und
Gebäck auf Gebäck fiel in die Schale, die bald bis zum Rand gefüllt
war. Das ist zuviel des Segens, dachte Saltner und suchte umher,
wie sich wohl der geheimnisvolle Speisequell abstellen ließe. Doch
vergebens, der Wirbel selbst ließ sich nicht zurückdrehen – Saltner
wußte nicht, daß man zu diesem Zweck erst durch Drehen der Schale
den automatischen Spender des Gebäcks abstellen mußte. Einen
weiteren Handgriff aber verstand er nicht zu finden, und so quoll
ein unstillbarer Strom von Fleischrollen auf die Schale, fiel von
dort auf Tisch und Fußboden und begann sich zu einem stattlichen
Haufen aufzutürmen. Saltner lief in Verzweiflung hin und her, aber
er fand kein Mittel – er wollte die Öffnung nicht mit Gewalt
verstopfen. – Schließlich dachte er, der Vorrat muß ja doch einmal
ein Ende nehmen, und wollte der Sache ihren Lauf lassen. Er wollte
nun die auf der Schale liegenden Stücke fortnehmen, um Grunthe eine
Portion zu bringen, dabei merkte er, daß die Schale sich drehen
ließ, und auf einmal hörte die weitere Spedition des Gebäcks auf.

Er sammelte die umherliegenden Massen der Delikatesse bis auf einen
kleinen Rest und trug sie in Grunthes Zimmer, der bei diesem
Anblick und Saltners tragikomischer Miene sich eines Lächelns nicht
erwehren konnte. Dort verbarg er sie in einem der leeren Körbe der
Expedition, denn auch in Grunthes Gemach hatte man einen Teil der
aus der Gondel geretteten Gegenstände geschafft.

„Warum lassen Sie das Zeug nicht einfach liegen?“ fragte Grunthe.

„Das geht nicht, ich bin ja sonst unsterblich vor den Damen als
dummer Bat blamiert. Übrigens sehne ich mich nach dem Frühstück;
aber erst muß ich doch sehen, wo ich ein Waschwasser finde.“

Er drehte der Reihe nach an verschiedenen Griffen, ohne daß er das
Gewünschte antraf. Bald sprang ein Schrank auf, der ihm
unverständliche Geräte enthielt, bald entzündeten sich Lampen an
verschiedenen Stellen des Zimmers. Dann zeigte sich eine Schüssel,
und schon hoffte er am Ziel zu sein, aber erschrocken fuhr er
zurück, denn die Schüssel begann sich zu erhitzen. Endlich
erweiterte sich in der Ecke des Zimmers der Fußboden zu einem
flachen Bassin, und ein Springbrunnen sprühte einen Strahl hervor.
Vorsichtig überzeugte sich Saltner, ob er es auch wirklich mit
Wasser zu tun habe, und war sehr erfreut, als sich seine Vermutung
bestätigte. Nun vervollständigte er mit Hilfe seiner
wiedergefundenen Reiseeffekten seine Toilette und setzte sich mit
Behagen an den Frühstückstisch.

Es war ihm ungewohnt und seltsam, daß das Tischchen so leer war und
weder Gläser noch Tassen oder Löffel und Messer enthielt. Das
Gebäck wenigstens wollte er auf einen Teller legen und sah sich
deshalb nochmals im Zimmer um. Er bemerkte jetzt, daß sich auch ein
großer Spiegel im Zimmer befand, neben welchem ein Gestell mit
mehreren glänzenden runden Scheiben stand, die er für silberne
Teller hielt. Er holte sich einen solchen Teller und legte sein
Frühstücksgebäck darauf. Dann ließ er sich das Getränk munden, das
die Öffnungen über dem Tischchen bereitwillig spendeten, nachdem er
die Mundstücke daran befestigt hatte. Es war eine warme und zwei
kalte Flüssigkeiten, die er erhielt und als Schokolade, Wein und
Selterswasser bezeichnete, da sie mit diesen Getränken am meisten
Ähnlichkeit hatten, obwohl er sich sagte, daß sie sich doch in
vieler Hinsicht von den auf der Erde üblichen Genüssen dieser Art
unterschieden. Insbesondere die Schokolade war sehr fettreich.

Neu gestärkt trat er in seinem kleidsamen Reiseanzug zu Grunthe ins
Zimmer und sagte:

„Ich bin nun bereit, unsere Polarforschung fortzusetzen.
Hoffentlich können Sie auch bald mitkommen. Aber ehe wir uns
beraten, was wir zu tun haben, will ich doch sehen, ob ich Ihnen
nicht ein Getränk verschaffen kann. Sie müssen ja einen grausigen
Durst haben.“

„Danke schön“, erwiderte Grunthe lachend, „sehen Sie, was ich
habe.“ Und er wies auf das Mundstück eines Schlauches hin, das
neben seinem Kopf über dem Bett herabhing. „Und hier“, fuhr er
fort, „kosten Sie einmal diese Pastete oder was es sonst ist. Ich
habe zwar keine Ahnung, wie es eigentlich schmeckt, aber ich fühle
mich dadurch wunderbar gestärkt. Wenn mich mein Fuß nicht hinderte,
stünde ich sogleich auf.“

„Sakra auch, das lasse ich mir gefallen! Wie haben Sie das
entdeckt? Ich habe mich inzwischen abgeschunden, verschiedene Stöße
bekommen und das Zimmer in ziemliche Unordnung gebracht. Wie fanden
Sie das, es war doch vorhin nicht hier?“

„Einfach durch Nachdenken. Ich sagte mir, die Martier sind viel
klüger als wir und jedenfalls viel umsichtiger. Wenn wir nun einen
Patienten haben, der nicht gehen kann, so werden wir ihm doch ein
Frühstück ans Bett bringen, und wenn wir selbst aus irgendeinem
Grund nicht kommen wollen, so werden wir es ihm hinstellen. Ich sah
mich also um. Nun betrachten Sie einmal diese beiden kleinen Zettel
an diesen Ringen.“

„Das sind ja lateinische Buchstaben!“

„Allerdings. Es sind zwei Wörter der Eskimosprache. ›Misalukpok‹
und ›Imerpok‹. Das eine bezeichnet ›Essen‹ und das andere
›Trinken‹.“

„Warum hat man mir aber nicht auch solche Aufschrift angeklebt? Bei
mir sind alle Schilder in einer Zeichenschrift, die jedenfalls
martisch ist.“

„Sie verstehen ja nicht Grönländisch.“

„Woher wissen aber die Nume, daß Sie es verstehen?“

„Weil ich mich gestern mit einer – mit jemand darin unterhalten
habe.“

„Potztausend, Grunthe, Sie sind mir über! Aber eins begreif ich
nicht, wie können die Leute, die Herren Martier, wissen, wie man
diese Worte in unsern Buchstaben schreibt?“

„Darüber bin ich mir auch noch nicht klar. Sie sehen, es ist
Antiqua, der lateinischen Druckschrift genau nachgemalt. Und mein
kleines Wörterbuch ist nicht mehr da, daraus haben sie die Zeichen
entnommen. Aber wie sie die richtigen Worte in dem Buch aufgefunden
haben, das ist mir ein völliges Rätsel. Denn sie kennen doch nur
den Laut der Eskimoworte, aber nicht die gedruckten Zeichen.“

„Es ist eine unheimliche Geschichte“, sagte Saltner. „Aber ein
gutes Weiberl ist sie doch, die Se, ich bin halt ganz hin! Wenn ich
nur wüßt, warum sich kein Mensch bei uns sehen läßt, kein Nume,
wollt ich sagen, denn darauf scheinen sie sich was Großes
einzubilden, daß sie keine Menschen sind.“

„Das kann ich Ihnen auch sagen, Saltner. Würden Sie ihren Gästen
nachts zwischen drei und vier Uhr einen Besuch machen?“

„Ist das die Uhr? Aber vorhin wußten Sie’s ja nicht, und ich denke,
am Pol gibt’s überhaupt keine Zeit.“

„Eine konventionelle Zeit muß es doch geben. Die Leute müssen doch
festsetzen, wann sie schlafen und wann sie zu Mittag essen sollen.
Wir also haben zum Beispiel unsere mitteleuropäische Einheitszeit
auf unseren Taschenuhren mitgebracht, und danach hätten wir jetzt
neun Uhr 55 Minuten vormittags. Als der Ballon scheiterte, war es
nach mitteleuropäischer Zeit gegen sechs Uhr abends. Nun weiß ich
bloß nicht, ob seitdem ein oder zwei Nächte vergangen sind, denn
das hängt von der Länge unserer Ohnmacht und unseres Schlafes ab.“

„Das weiß ich allerdings auch nicht. Ich weiß auch nicht, wann
unser erstes Erwachen stattgefunden hat; das Ihrige vermutlich bald
nach dem meinigen.“

„Nun, das läßt sich nachher aus der Deklination der Sonne
feststellen, welches Datum wir haben. Ich habe meine Uhr auch jetzt
erst wieder entdeckt – beide Uhren, und da sie übereinstimmen, sind
sie auch nicht stehengeblieben –“

„Nein, ich habe dieselbe Zeit –“

„Ja, aber welche Zeit rechnen die Martier hier? Sehen Sie, das
haben sie mir auch mitgeteilt, und daher weiß ich, daß es für sie
jetzt Schlafenszeit ist und daß sie erst in vielleicht zwei Stunden
aufstehen werden. Deswegen sagte ich, es sei zwischen drei und vier
bei unsern Wirten; wie sie die Stunden zählen und benennen, weiß
ich allerdings auch nicht.“

„Aber Doktor, woher wissen Sie denn, was bei den Martiern für eine
Tageseinteilung Mode ist und was die Glocke bei ihnen geschlagen
hat?“

„Glauben Sie wohl, Saltner, in einem Schlafzimmer, das mit allem
Komfort der Martier ausgestattet ist, werde eine Uhr fehlen?“

„Ich habe keine gesehen und Sie vorhin auch nicht.“

„Seitdem aber habe ich sie entdeckt. Sehen Sie die Malerei, welche
die kreisförmige Öffnung des Oberlichts einschließt? Sie ist in
zwölf mal zwölf gleiche Abschnitte geteilt. Und jene schmalen
hellen Streifen, die Sie dazwischen sehen, liegen nicht fest,
sondern bewegen sich auf dem Ring. Das ist mir erst allmählich klar
geworden, als ich während ihrer Toilette hier ruhig lag und in die
Höhe starrte. Hier haben Sie die Uhr der Martier.“

„Ich schau sie wohl an, aber klug werd ich nimmer draus.“

„Entziffern kann ich sie auch nicht. Aber sehen Sie, es sind zwei
Zettel angesteckt, die offenbar nicht zur Uhr gehören, sondern nur
für heute, für uns, eine Nachricht geben. Der eine zeigt ein
geschlossenes, der andere ein offenes Auge. Die Deutung ist klar:
Schlafen und Wachen.“

„Es ist richtig, und dieser helle Strich –“

„Das ist der Stundenzeiger –“

„Dachte ich mir. Er steht noch ungefähr um ein Zwölftel des ganzen
Kreises von dem geöffneten Auge ab.“

„Daher eben schließe ich, daß noch zwei Stunden zirka bis zum
Beginn des Erwachens der Martier sind.“

„Aber finden Sie es nicht seltsam, daß die Martier den Tag
ebenfalls in zwölf Stunden teilen?“

„Ebenfalls? Wir teilen ihn ja in vierundzwanzig –“

„Nun, das sind zweimal zwölf.“

„Daß die Zwölf wiederkehrt, wundere ich mich gar nicht – ich würde
mich wundern, wenn es anders wäre. Es liegt das im Wesen der Zahl,
das heißt im Wesen des Bewußtseins überhaupt. Die Gesetze der
Mathematik sind die Gesetze der Welt. 12 ist 3 mal 4, die kleinste
aller Zahlen, welche die drei ersten Zahlen 2, 3 und 4 zu Teilern
besitzt. Alle intelligenten Wesen, welche Mathematik treiben,
werden die 12, nächstdem die 60 zur Grundlage ihrer Einteilungen
machen.“

„Aber wir haben ja doch die Zehn –“

„Die alte Astronomie wählte die Zwölf – zwölf Zeichen bilden den
Tierkreis – die Zehn ist nur ein unwissenschaftlicher Rückfall in
die sinnliche Anschauung der zehn Finger – Krämerpolitik –, doch
lassen wir das.“

„Meinetwegen“, sagte Saltner. „Aber was tun wir nun? Erst müssen
Sie natürlich Ihren Fuß auskurieren.“

„Ich fürchte“, erwiderte Grunthe, „wir werden auch dann nichts
anderes tun können, als was die Martier über uns beschließen. Mit
der Expedition wird es wohl so ziemlich aus sein. Suchen wir uns
inzwischen möglichst mit den Verhältnissen vertraut zu machen.
Rekognoszieren Sie ein wenig!“

„Im Zimmer habe ich mich schon umgesehen, und ich möchte nicht noch
mehr von den rätselhaften Instrumenten probieren – man kann sich zu
leicht blamieren. Ich komme mir vor wie ein Wilder in einem
physikalischen Institut, bloß daß unsereiner nicht die nötige
Naivität besitzt.“

„Was haben wir denn für Ausgänge?“

„Nur einen aus jedem unserer Zimmer. Ich weiß die Tür nicht zu
öffnen. ich glaube, es ist auch schicklicher, wir warten hier, bis
man uns aufsucht, als daß ich aufs Ungewisse herumstöbere.“

„Sie haben recht! Vielleicht haben Sie die Güte, unsere Sachen ein
wenig zu ordnen, und wenn Sie mein Tagebuch finden, so bitte ich
Sie darum. Zunächst müssen wir sehen, daß wir sowohl Torms Eigentum
als die offiziellen Aktenstücke der Expedition in Sicherheit
bringen.“

„Ich habe schon einiges hier beiseite gelegt“, sagte Saltner, indem
er unter den Gegenständen aufräumte, welche die Martier aus der
Gondel gerettet hatten. Sie waren zum Teil durch den Sturz und das
Meerwasser beschädigt.

„Es wäre mir übrigens gar nicht unangenehm“, fuhr Saltner fort,
„wenn noch einiges von unserm Proviant brauchbar wäre. Denn ich
traue nicht recht, wie einem dieser Würstchenautomat hier
bekommen wird. Sehen Sie einmal, was die Herrn Nume alles
aufgehoben haben! Da haben sie uns ja das Futteral mit den beiden
Flaschen Champagner hergelegt, das Sie in der Not als Ballast auf
die Insel warfen. ich hab halt gedacht, das würde ihnen die Köpfe
zerschlagen und dabei in tausend Trümmer gehen. Aber es scheint
ganz unversehrt. Nun, ich will die beiden Monopol nur aus dem
Kasten nehmen. Die können wir doch nimmer mit Freude ansehn. Arme
Frau Isma!“ Er nahm die Flaschen heraus.

„Halt“, sagte er, „da in dem Futter steckt noch ein Paketchen. –
Was haben wir denn da?“

Der Verschluß hatte sich gelöst. Ein Buch in der Größe eines
Notizkalenders kam zum Vorschein.

„Na“, sagte Saltner, „Frau Isma wird uns doch nicht noch ein Album
mitgegeben haben. Sehen Sie doch einmal, Grunthe, was das ist.“

„Was geht das mich an?“ sagte Grunthe unwirsch.

Saltner schlug das Buch auf. Er stutzte sichtlich, blätterte darin
und sah lange hinein.

„Das ist –“, sagte er dann kopfschüttelnd, „das ist ja – Aber wie
ist das möglich?“

Das kleine Buch enthielt ein Wörterverzeichnis der Sprache der
Martier; die Worte waren mit Hilfe der Lautzeichen des lateinischen
Alphabets transkribiert, daneben befand sich eine deutsche
Übersetzung und zugleich das Zeichen des Wortes in der
stenographischen Schrift der Martier. Saltner hatte an den wenigen
ihm bekannten Worten die Bedeutung des Inhalts erkannt.

„Sagen Sie mir das eine“, fuhr er fort, „mir steht der Verstand
still – wie kann ein deutsch-martisches Wörterbuch hierherkommen –
wie kann es überhaupt existieren?“

Grunthe streckte sprachlos die Hand aus und ergriff das Buch.

Er warf nur einen Blick hinein. Dann sagte er leise: „Das ist die
Handschrift von Ell.“

Grübelnd schloß er die Augen. Das unlösbare Rätsel trat ihm wieder
entgegen – wie kam Ell zur Kenntnis der Sprache der Marsbewohner?
Und wenn er sie kannte, warum hatte er sich nicht offen
ausgesprochen? Warum hatte er nicht ihm oder Torm die
Sprachanleitung mitgegeben? Wie kam sie versteckt in das Futteral,
unter die Flaschen?

Er wußte keine Antwort.

Saltner hatte inzwischen das Buch ergriffen und suchte sich daraus
einige Worte zusammen.

Da hörte er im Nebenzimmer leises Lachen und Stimmen der Martier.
Der Arzt Hil war in Saltners Zimmer eingetreten. Se hatte ihn bis
an die Tür begleitet und amüsierte sich köstlich über die
Unordnung, welche Saltner angestiftet hatte, am meisten aber
darüber, daß er bei seinem Frühstück als Teller die – Kämme benutzt
hatte. Die flachen Scheiben, welche Saltner für Teller gehalten
hatten, dienten den Martiern dazu, das Haar zu ordnen; sie wurden
elektrisch geladen und streckten dann die Haare geradlinig vom Kopf
ab. „Es ist zu lustig“, lachte Se. „Aber wir wollen ihm jetzt
nichts sagen, dem armen ›deutsch Saltner‹.“ Darauf zog sie sich
wieder zurück. Denn es war ihr zu ›schwer‹ in den Zimmern der
Bate.

Hil trat bei Grunthe und Saltner ein.

\section{10 - La und Saltner}

Hil war mit dem Zustand seiner Patienten sehr zufrieden. Mit großem
Interesse betrachtete er ihre Effekten. Sichtliches Erstaunen aber
malte sich auf seinen Zügen, als ihm Grunthe den kleinen
deutsch-martischen Sprachführer überreichte. Er blätterte eifrig
darin, und indem er auf einzelne Zeichen der martischen Schrift
zeigte und sich das danebenstehende deutsche Wort nennen ließ,
gelang es ihm bald, einige Fragen zu stellen, die Grunthe durch das
umgekehrte Verfahren beantwortete. Da es ihm selbst an Zeit
gebrach, den gegenseitigem Sprachunterricht sofort eingehend
aufzunehmen, fragte er Grunthe mit Hilfe des Grönländischen, ob er
nicht mit La, die sich gern mit Sprachstudien beschäftigte,
martisch sprechen wolle, um recht bald zu einem gegenseitigen
Verständnis zu kommen. Grunthe war dies sehr unangenehm. Er war
recht froh, daß sich keine von seinen Pflegerinnen hier bei ihm
sehen ließ, und er wandte sich daher an Saltner mit dem Vorschlag,
ihn in dieser Hinsicht zu vertreten. Obgleich dieser die Sprache
der Eskimos nicht als verbindendes Hilfsmittel benutzen konnte,
glaubte er doch, mit Hilfe des Ellschen Sprachführers auszukommen
und erklärte sich gern zu allen Diensten bereit.

Hil nahm den Sprachführer mit sich und geleitete Saltner in den
anstoßenden großen Salon der Martier. Hier stellte er ihn einer
Anzahl der dort versammelten Martier vor, unter denen sich der
Leiter der Station Ra mit seiner Frau sowie neben einigen andern
Martierinnen auch Se und La befanden.

Saltner wußte nicht, wo er seine Augen zuerst hinwenden sollte.
Fast alles, was er sah, war ihm fremd, am meisten aber überraschten
ihn die Gestalten der Martier selbst. Es war ihm nur lieb, daß er
sich aus Mangel an Sprachkenntnissen in Schweigen hüllen und sich
mit dem Sehen begnügen konnte. Hil nannte ihm die Namen der
einzelnen, die ihn mit ihren martischen Handbewegungen begrüßten,
was Saltner mit europäischen Verbeugungen erwiderte. Nur fielen
dieselben leider etwas steif aus, da er infolge der verminderten
Schwere sehr vorsichtig sein mußte. Er sah wohl an den Gesichtern
derjenigen Martier, welche in ihm zum ersten Mal einen Europäer
erblickten, wie sie sich Mühe gaben, ihre Belustigung über seine
Ungeschicklichkeit zu verbergen. Es war ihm daher sehr angenehm,
als sich die Mehrzahl der Anwesenden zurückzog.

Gleich bei seinem Eintritt war ihm neben der reizenden Se die
Gestalt Las aufgefallen, und als er bei der Nennung der Namen
erkannte, daß dieses wunderbare Wesen seine Sprachlehrerin sein
sollte, heftete er seine Blicke erwartungsvoll auf ihre Züge. Aber
in ihren großen Augen war keine Spur von Spott zu bemerken, sie
begrüßte ihn mit ruhiger Liebenswürdigkeit, und ein Lächeln, das
sie mit Se tauschte, sagte dieser, daß ihr dieser Bat besser gefiel
als der andere. Saltner war überzeugt, daß er riesenschnelle
Fortschritte im Martischen machen würde, wenn ihm die Anerkennung
aus solchen Augen als Lohn winke. Er wußte nur nicht recht, wie die
Sache zu beginnen sei, da keines der beiden die Sprache des andern
kannte. La holte einige Bücher aus der Bibliothek, darunter den ihm
schon bekannten Atlas, der ihm zur ersten Verständigung mit Se
gedient hatte. Sie streckte sich dann in ihrer Lieblingsstellung
auf den Diwan und winkte Saltner, sich dicht an ihrer Seite
niederzulassen. Sie begann zunächst einige Gegenstände zu
bezeichnen, die sich unmittelbar der Anschauung darboten, und sich
die Benennung martisch und deutsch wiederholen zu lassen; dann
verfuhr sie ebenso mit verschiedenen Abbildungen in den Büchern.
Aber so ging die Sache zu langsam. Sie griff zu dem Sprachführer,
den Se in der Hand hielt. Se hatte bis jetzt in dem Büchlein
geblättert und eine Anzahl von deutschen Worten auf einem Streifen
durchsichtigen Papiers einfach dadurch nachgebildet, daß sie das
Papier einen Augenblick auf das betreffende gedruckte Wort legte
und andrückte. Das Papier war lichtempfindlich und gehörte zu einem
kleinen Taschenschnellphotograph, den man als Notizbuch bei sich zu
führen pflegte. Saltner las: „Schüler fleißig. Lehrer streng.
Fernhörer. Alles hören.“

Als er wieder aufblickte, sah er, daß Se schelmisch lachte. Sie
machte sich dann noch an dem Apparatentisch zu schaffen und
entfernte sich mit freundlichen Winken: „Das ist recht“, sagte La,
„sie hat den Phonograph aufgezogen. Danach können wir dann unser
Pensum gut repetieren.“

Darauf nahm La den Sprachführer vor und ging mit Saltner die
Redensarten und kleinen Gespräche durch, welche dort in beiden
Sprachen angegeben waren. Er las sie deutsch, sie martisch, und
beide lachten dazwischen herzlich, wenn sie ihre Aussprache zu
verbessern suchten oder komische Mißverständnisse zutage kamen.
Saltner mußte La dicht über die Schulter blicken, um im Buch zu
lesen. Es ließ sich nicht vermeiden, daß sein Blick nach der
wunderbaren Farbe ihres Haares und den weichen Formen des Nackens
abirrte und die Worte manchmal zerstreut herauskamen. Ein seltsamer
Wärmestrom ging von ihrem Körper aus, und dies war nicht bloß ein
Spiel seiner Phantasie; er erfuhr später, daß die Martier in der
Tat eine höhere Blutwärme besitzen als die Menschen. Er merkte, daß
sich seine Sinne verwirrten. Und auch dies hatte seinen Grund nicht
nur in seinen Gefühlen, sondern war eine Wirkung der geringen
Schwere, an die seine Konstitution noch nicht gewöhnt war. Das Blut
wurde ihm stärker zu Kopfe getrieben.

La erkannte dies bald. Sie gab ihm das Buch zu halten, lehnte sich
zurück und stellte das abarische Feld ab. Alsbald fühlte sich
Saltner wieder wohler, und die Studien nahmen mit erneuter Kraft
ihren Fortgang. So vergingen schnell einige Stunden. Und auf einmal
stellte sich heraus, daß die Lehrerin viel mehr deutsch gelernt
hatte als der Schüler martisch. Nicht weniger als Saltner hatte
Grunthe dabei gelernt, der den Sprechübungen durch den Fernhörer
zugehört hatte. Er fragte an, ob er jetzt vielleicht das Buch auf
einige Zeit erhalten könnte.

La stellte die Schwere ab, um sich wieder frei bewegen zu können.

„Oh, wie zerstreut bin ich doch!“ rief sie aus. „Wir brauchten uns
doch nicht mit dem einen Exemplare zu quälen! Wenn Sie mir das Buch
noch eine halbe Stunde erlauben“ – wandte sie sich durch den
Fernsprecher an Grunthe –, „so werde ich es sofort vervielfältigen
lassen.“

Sie schrieb einige Worte auf ein Stückchen Papier, legte dies in
das Buch und packte das Ganze in einen Umschlag. Dann warf sie das
kleine Paket in einen an der Wand befindlichen Kasten.

Saltner sah ihr verwundert zu.

„Das ist die pneumatische Post nach der Werkstatt“, sagte La
erklärend. „Es wird nicht lange dauern, so bekommen wir die Kopien
des Buches, aber nicht in Ihrem ungeschickten Format, sondern in
unserer hübschen Tafelform.“ Sie erläuterte das Gesagte durch
verschiedene Handbewegungen.

„Und wer besorgt denn dies?“ fragte Saltner.

„Wer von den Technikern gerade an der Reihe für den Tag ist. Die
Arbeitszeit wechselt in geregelter Ablösung. Jeder hat seinen
besonderen Tätigkeitskreis. Ich zum Beispiel muß mich mit der
Erlernung der schrecklichen Menschensprachen quälen. Haben Sie mich
verstanden?“

Da Saltner noch ein ziemlich fragendes Gesicht machte, wiederholte
sie die Antwort noch einmal, zu seiner Verwunderung in zwar etwas
seltsamem, aber doch verstehbarem Deutsch.

„Sie sprechen ja deutsch, La La!“ rief er aus.

„Sie haben nicht aufgepaßt“, sagte sie lachend. „Die Worte sind ja
alle heute in unserm Pensum vorgekommen. Wir wollen es repetieren.“
Sie ging an den Tisch und drückte auf den Knopf des Grammophons.

Man hörte sogleich die Worte wieder, die La zu Se bei ihrer
Verabschiedung gesprochen hatte. La zog sich nun auf ihren Diwan
zurück, stellte die Abarie ab und winkte Saltner, sich zu setzen.

Es war ihm ganz seltsam zumute, als er so seine eigene Stimme,
jedes Wort mit der eigenen Betonung, jeden Sprachfehler –
dazwischen das tiefe, halblaute Organ Las und ihr leises Lachen –
wieder vernahm. Die schräg einfallenden Sonnenstrahlen rückten bis
an Las Ruhestätte und entzündeten ein seltsames Farbenspiel
zwischen den losen Wellen ihres Haares, sie spielten als ein Meer
von Funken auf den glitzernden Fäden ihres Schleiers, die sich bei
ihren Atemzügen leise hoben und senkten. War er noch er selbst,
oder war er in ein fernes Geisterreich entrückt und mußte er nun
sein eigenes Leben an sich vorüberziehen lassen?

„Nicht träumen“, sagte La halblaut, „aufpassen.“

Nun hörte er wieder auf die Worte ihrem Sprachsinn nach, er
repetierte.

Da klapperte es an dem Postkasten.

„Da sind unsere Bücher“, sagte La. „Stellen Sie, bitte, das
Grammophon ab, und öffnen Sie den Kasten.“

Saltner vollzog den Auftrag. Er enthob dem Kasten ein Paket, das
die Kopien des Sprachführers enthielt. La nahm das Original heraus
und gab es Saltner.

„Hier“, sagte sie, „bringen Sie dies Ihrem Freund zurück, mit
bestem Dank. Und wenn es Ihnen recht ist, arbeiten wir am
Nachmittag noch einmal.“

„Verfügen Sie vollständig über mich“, sagte Saltner mit einem
bewundernden Blick. Eine vornehme Handbewegung verabschiedete ihn.

\tb

Die Sprachstudien fanden am Nachmittag eine unerwartete
Unterbrechung.

Eben wollte Saltner, der mit Grunthe zusammen gespeist hatte, sich
wieder in den Salon begeben, als Ra bei ihnen eintrat, um ihnen
eine Mitteilung zu machen, die beide Forscher aufs Lebhafteste
erregte.

Die Martier hatten auf ihren Jagdbooten das Binnenmeer und seine
Ufer noch weiter nach Spuren der Expedition abgesucht. In einem der
Fjorde, welche sich ungefähr in der Richtung des 70. Meridians
westlicher Länge von Greenwich verzweigten, am Fuß eines
unmittelbar in das Wasser abfallenden Gletschers, hatte man den
bisher vermißten Fallschirm der Expedition gefunden, zwischen
losgebrochenen Eisschollen treibend. Derselbe mußte so nahe am Ufer
niedergefallen sein, daß es wohl denkbar war, ein an demselben
hängender Mensch hätte sich auf den Gletscher retten können. Die
Martier hatten das Land selbst nicht betreten können; ohne
besondere maschinelle Vorrichtungen war ihnen dies überhaupt nicht
möglich.

Saltner sprang auf und bat dringend, ihn sofort an Ort und Stelle
zu bringen. Hier war eine Möglichkeit gegeben, daß Torm doch noch
am Leben und zu retten sei. Daß der Fallschirm in so weiter
Entfernung vom Ballon gefunden war, und zwar an einer Stelle, über
die der Ballon nicht geflogen sein konnte, ließ sich nur dadurch
erklären, daß Torm den Schirm vom Ballon getrennt hatte. Dann
konnte die in den unteren Luftschichten herrschende Windströmung
den langsam fallenden Schirm sehr wohl bis dorthin getrieben haben.
Aber ob sich Torm an dem Schirm befunden hatte? Vermutlich hatte er
sich mit demselben niedergelassen; aus welchen Gründen ließ sich
nur unsicher vermuten. Vielleicht hatte er den Ballon dadurch zu
retten gedacht, daß er ihn um sich selbst erleichterte; vielleicht
auch hatte er die Gefährten für erstickt gehalten und für sich
selbst ein letztes Rettungsmittel versucht, ehe der Ballon wieder
über das Meer hinaustrieb. Jedenfalls mußte man alles daransetzen,
etwaige Spuren von Torm aufzufinden.

Ra stellte Saltner bereitwillig ein Boot und Mannschaft zur
Verfügung, sagte aber sogleich, daß die Martier zu einer
Untersuchung des Gletschers selbst sehr wenig geeignet seien. Sie
würden jedoch für einige Apparate sorgen, die zum Transport
etwaiger Lasten oder auch von Personen mit Vorteil benutzt werden
könnten. Insbesondere aber schlüge er ihm vor, die beiden Eskimos,
welche sich auf der Station aufhielten, Vater und Sohn,
mitzunehmen. Sie leisteten den Martiern gute Dienste bei Arbeiten
und Transporten im Freien, bei denen es menschlicher Muskelkraft
bedürfe, und könnten ihn gewiß bei einer etwaigen Besteigung des
Gletschers unterstützen.

Nach einer halben Stunde war das Boot bereit. Da Saltner sich nicht
auf die ihm unbekannten Apparate der Martier verlassen wollte,
hatte er sich mit seinem eigenen Seil und seinem getreuen,
glücklich geretteten Eispickel versehen, die ihn schon bei so
mancher schwierigen Klettertour in den Gebirgen seiner Heimat
begleitet hatten.

Saltner war nicht wenig erstaunt, als er in dem langen, elegant
gebauten Boot neun riesige Kugeln von etwa einem Meter Durchmesser
erblickte, die Kopfhüllen der Martier, die ihnen direkt auf den
Schultern saßen. Sie sahen dadurch wie seltsame Karikaturen aus.
Der Führer des Bootes stand am Land und begrüßte Saltner, worauf er
sich mühsam an Bord begab und nun ebenfalls seinen Kugelhelm
aufsetzte. Die beiden Eskimos befanden sich schon im Boot und
lösten das Seil, sobald der Führer eingestiegen war. Sie verstanden
nicht recht seine Handbewegung, und das Boot begann von der
Landestelle abzutreiben, gerade als Saltner seinen Fuß auf den Rand
desselben setzte. Die Martier, welche glaubten, er müsse unfehlbar
ins Wasser stürzen, winkten lebhaft mit ihren Armen, während er
selbst sich mit einem leichten Schwunge vom Ufer abstieß und
gewandt in das Boot sprang. Für einen geschickten Turner war dies
eine Kleinigkeit, erregte aber bei den Martiern offenbare
Anerkennung. Unter dem Einfluß der Erdschwere wäre diese Leistung
keinem von ihnen möglich gewesen.

Kaum hatte Saltner einige Schritte getan, indem er sich nach einem
passenden Platz umsah, als einer der Martier seine große Kugel von
der Schulter nahm und an ihrer Stelle der anmutige Kopf Las zum
Vorschein kam. Sie sah ihn mit ihren großen Augen heiter an und
nickte ihm freundlich zu.

„Wie kommt es, daß Sie hier sind, La La?“ sagte Saltner, in seiner
Überraschung deutsch sprechend. „Sie scheuen doch die Schwere
draußen. Diese Fahrt ist gewiß sehr anstrengend für Sie?“

„Ganz richtig“, antwortete La ebenfalls deutsch, „ich tue es nicht
zum Vergnügen. Ich bin im Dienst. Wie wollen Sie verstehen diese
Nume? Wie wollen Sie verstehen diese Kalalek? Ich bin als
Dolmetscher hier“, fügte sie auf martisch hinzu.

„Das ist wahr, an diese Schwierigkeit habe ich gar nicht gedacht.
Aber wie leid tut es mir, daß Sie sich so bemühen müssen. Freilich,
was könnte ich mir Besseres wünschen – doch wollen Sie nicht Ihren
Helm wieder aufsetzen?“ La schüttelte den Kopf. Aber sie schlug
hinter ihrem Platz eine Lehne mit weichem Polster in die Höhe und
stützte dort ihr Haupt auf. So lehnte sie sich zurück und ließ ihre
Augen prüfend über das Boot und die ganze Umgebung wandern.

Mit großer Geschwindigkeit durchschnitt das Boot die leise bewegten
Wellen der Bucht und hatte in etwa zehn Minuten die Stelle
erreicht, an welcher sich mehrere Kanäle von verschiedener Breite
verzweigten. Jetzt mußte langsam und vorsichtig gefahren werden,
denn ein Gewirr von Felsblöcken und Eisbergen oder Schollen
erstreckte sich am Stirnende des Gletschers entlang und verengte
das Fahrwasser. Die Martier hatten den Platz bezeichnet, an welchem
sie den Fallschirm gefunden hatten, und Saltner spähte nach einer
geeigneten Stelle aus, wo man den Gletscher erklimmen könnte. Er
schlug seinen Eispickel in eine Scholle und sprang auf dieselbe
hinüber, kam wieder zurück und ließ das Boot weiterfahren. Es
schien sich von selbst zu verstehen, daß er hier kommandierte.

La ließ ihre Augen mit Wohlgefallen auf seinen entschiedenen
Bewegungen ruhen. Dieser Bat, über den sie als Martierin sich so
weit erhaben fühlte, war ihr bisher nur seltsam vorgekommen. Aber
hier, in seinem Element als gewandter Kletterer, machte er ihr doch
einen viel vorteilhafteren Eindruck. Gegenüber den unbeweglichen
Kugeln, die ihre Landsleute auf den Schultern trugen, gegenüber den
grauen, stumpfen Gesichtern der Eskimos mit ihren vorstehenden
Backenknochen bot sein ausdrucksvoller Kopf, seine freie Haltung
und kräftige Kühnheit ein Bild, das sie gern betrachtete.

Der Gletscher fiel an den meisten Stellen mit einem senkrechten
Abbruch von zehn bis fünfzehn Metern Höhe in die See ab. Endlich
hatte Saltner eine Stelle gefunden, an welcher ihm der Aufstieg
möglich schien. Gewandt schlug er Stufe auf Stufe in das ziemlich
weiche Eis und kletterte, von den Augen der Martier unter Spannung
verfolgt, die Eiswand hinauf. Dann warf er das Seil hinab, und die
beiden Eskimos folgten ihm an demselben. Bald waren die drei für
die Insassen des Bootes hinter dem Rand des Eises verschwunden.

Längere Zeit war nichts von den Kletterern zu vernehmen, und La
begann schon ungeduldig nach der Höhe zu blicken. Da erschien
Saltner etwa hundert Meter weiter am Rand des Absturzes und winkte
dem Boot, sich dorthin zu begeben. Als dies geschehen war, rief er
hinunter:

„Ich habe Spuren gefunden. Wird es möglich sein, einige Leute hier
heraufzubringen?“

La übersetzte, und der Führer des Bootes ließ antworten, daß dies
sehr leicht sei, wenn es Saltner gelänge, mit seinem Seil die Rolle
des Aufzuges, den die Martier mit sich führten, hinaufzuziehen und
oben zu befestigen.

Dies geschah nach Wunsch. Alsbald hatten die Martier einen bequemen
Aufzug eingerichtet, den sie mit den Akkumulatoren ihres Bootes
betrieben.

Nicht weit von der Stelle, an welcher die Martier ihren Aufzug
angebracht hatten, stieß eine Seitenschlucht in das Haupttal, und
hier zog sich ein Streifen von Felstrümmern und Moränenschutt, von
Flechten überkleidet, auf dem ganz allmählich ansteigenden
Gletscher in die Höhe. Auf diesem Streifen konnte man, ohne sich
der unsicheren Oberfläche des Gletschers anzuvertrauen, gut ins
Innere des festen Landes gelangen. Saltner hatte nun in dieser
Richtung einen Gegenstand zwischen dem Geröll bemerkt, der zwar der
weiten Entfernung wegen nicht deutlich erkennbar war, aber
jedenfalls untersucht werden mußte, da er von Menschen herzurühren
schien, wenn es nicht gar der nur zum Teil sichtbare Körper eines
Menschen war. Um jedoch das Gestein zu erreichen, mußte man
zunächst eine tiefe und breite Spalte passieren; diese Spalte war
an einer Stelle durch eine Schneebrücke überspannt gewesen, die
offenbar erst vor kurzem zusammengebrochen war. Gegenwärtig war es
unmöglich, dieselbe ohne künstliche Hilfsmittel zu überschreiten,
und deshalb hatte Saltner die Martier heraufgerufen. Er sagte sich,
es sei sehr wohl denkbar, daß Torm mit Hilfe des vom Fallschirm
abgelösten Seiles auf den Gletscher und von dort auf den
Moränenstreifen gelangt sei. Mit großer Aufregung hatte er daher
jenen dunklen Gegenstand in der Ferne betrachtet.

Die Martier wanden nun aus ihrem Boot genügend lange Stangen empor,
um die Spalten überbrücken zu können, und Saltner verrichtete mit
den Eskimos die übrige Arbeit. Alsdann wanderte er über die
Felstrümmer weiter, eine Kletterpartie, die übrigens schwieriger
war und langsamer vor sich ging, als er ursprünglich erwartet
hatte.

La hatte sich ebenfalls emporziehen lassen. Auf ausgebreiteten
Fellen ruhend sah sie den Arbeiten der Menschen zu. Sie hatte noch
niemals einen Gletscher in der Nähe gesehen, geschweige denn
betreten. Auf dem Mars gab es solche Gebilde nicht; die Atmosphäre
war viel zu trocken, um dieselben zu unterhalten. Mit Bewunderung
blickte sie in das Gewirr von Spalten, Trümmern und Zacken, die mit
ihren grünlichen Schatten sich von den rötlich im Sonnenschein
schimmernden Schneeflächen abhoben. Gar zu gern hätte sie einen
Blick in die unergründliche Eisschlucht hineingetan, welche die
Menschen überbrückten, aber sie scheute sich, den mühsamen Gang zu
zeigen, mit dem sie sich hätte hinschleppen müssen.

Jetzt waren die Menschen fortgegangen; sie konnten die seltsame
Figur nicht mehr beobachten, die sich langsam von den Fellen erhob,
ihren Kugelhelm aufsetzte und auf zwei Stöcke gestützt der Spalte
zuschlich. Der Weg war gar nicht so anstrengend, wie La glaubte;
sie hatte sich doch schon einigermaßen geübt, ihre Glieder unter
dem Einfluß der Erdschwere zu bewegen. So gelangte sie an den
angebrachten Steg und ließ sich am Rand der Gletscherspalte
nieder.

An die eine der hinübergelegten Stangen sich haltend, beugte sie
vorsichtig den Kopf über den Abgrund. Dunkelgrün dämmerte die
Tiefe, aus der das Rauschen des Schmelzwassers dumpf herauftönte.
Genau unter ihr streckte sich ein zackiger Grat ihr entgegen, der
die Schlucht der Länge nach durchzog. Ein großer Felsblock war
hinabgestürzt und auf dem Grat festgehalten worden. Er bildete eine
Art Brücke da unten in der Tiefe. Daneben zeigte ein frischer Spalt
seine kristallklaren Eiswände. La konnte sich an dem ungewohnten
Schauspiel nicht sattsehen. Schwindel kannte sie nicht. Sie war
gewohnt, den Weltraum in seiner Unendlichkeit unter ihren Füßen zu
erblicken, wenn das Raumschiff die Leere des Sternenhimmels
durcheilte. Aber sie kannte auch nicht die Gefahren dieses mürben,
abbröckelnden Elements, auf dessen überhängender Kante sie ruhte.
Um besser hinabzublicken, zog sie sich an der Stange weiter und
stemmte ihre Füße gegen einen Vorsprung des Randes. Der Vorsprung
brach. Zerstiebend stürzte er in die Tiefe. Ihr Fuß verlor die
Stütze. Sie wollte sich wieder hinaufschwingen, aber die Last war
zu schwer für ihre Kräfte. Der unförmliche Helm hinderte sie, ihren
Oberkörper frei an dem Steg zu bewegen, an welchen sie sich
geklammert hielt. Sie rief um Hilfe, doch die Stimme drang nur
schwach unter dem Helm hervor. Eine erneute Anstrengung brachte
ihren Körper höher, aber nun glitt die Stange aus ihrer Lage, ihre
Hände verloren den Halt – – La stürzte in den Abgrund.

Ihr Angstschrei verhallte zwischen den Eiswänden der Spalte. Aber
der Helm, der ihren Absturz verschuldete, wurde vorläufig zu ihrem
Retter. Sie fiel auf die Stelle, welche der auf dem Grat ruhende
Felsblock verengte, und der elastische Helm hemmte den Sturz. Er
war zertrümmert, aber sie selbst fühlte sich unverletzt, sie hatte
das Bewußtsein nicht verloren. Mit den Armen sich festklammernd,
lag sie auf dem Felsen, unter sich die finstere Tiefe, über sich
den schmalen Lichtstreifen des Himmels, unfähig, sich zu bewegen.
Sie vermochte nichts zu ihrer Rettung zu tun, Minute auf Minute
verrann. Wann würde man sie bemerken? Konnte sie gerettet werden?
Sie war vollkommen ruhig. Das Bild der fernen Heimat stieg vor ihr
auf. „Noch einmal möcht ich ihn sehen, meinen schönen Nu“, so klang
es in ihr, „aber wenn es nicht sein soll, so füg ich mich Deinem
Willen im Weltenplan.“

Da vernahm sie Rufe über sich. Ein Kugelhelm wurde sichtbar. Die
Martier hatten ihr Verschwinden bemerkt, sie ward gesehen. Man rief
ihr zu, sie möge Mut fassen, man werde den Aufzug herbeischaffen.
Sie wußte, daß darüber lange Zeit vergehen müsse; die Martier
konnten nur langsam arbeiten. Und sie fühlte, wie die Kälte der
Schlucht ihre Glieder erstarren ließ.

Plötzlich hörte sie oben erneute Rufe und schnelle Schritte.
Eilende Gestalten schwangen sich über den Steg. La wußte, wer es
war. Saltner war mit den beiden Eskimos zurückgekehrt. Kaum hatte
er erkannt, was geschehen war, als er sich auch sofort anseilte und
von seinen beiden Begleitern in den Spalt hinabsenken ließ. La sah,
wie seine Gestalt näher und näher kam. Mit der einen Hand hielt er
sich von der Wand der Spalte ab. Und nun kniete er neben ihr auf
dem Felsblock. Er löste den Rest ihres Helmes geschickt von ihren
Schultern; dumpf donnerte er in den Abgrund. Dann hob er sie empor
und sagte besorgte Worte, die sie nur halb verstand. Jetzt erst
erfaßte sie der Schwindel, und das Bewußtsein drohte sie zu
verlassen. Aber sie fühlte, daß Saltner sie fest umschlang, und in
diesem Augenblick wußte sie sich geborgen. Jetzt rief er mit lauter
Stimme in seiner Muttersprache nach oben: „Ein zweites Seil!“

La lächelte, ihn dankbar ansehend, und sagte leise: „Kalalek nicht
verstehen.“

„Doch, doch!“ erwiderte Saltner. Und wirklich, der jüngere der
beiden Eskimos rief die deutschen Worte hinab:

„Nicht hier. Warten. Nume kommen.“

La blickte ihn fragend an. Aber er antwortete nicht, er sah, daß
sie fror.

„Werft die Decke herab!“ rief er.

Man schien ihn jetzt nicht zu verstehen. „Was heißt auf
grönländisch Decke?“ fragte er.

„Kepik.“

„Kepik!“ rief er hinauf.

Eine wollene Decke wurde hinabgeworfen. Saltner schlug den Pickel
fest in die Wand und beugte sich weit vor, um sie aufzufangen. Es
gelang. Er hüllte La hinein. Er zog seine Feldflasche heraus, die
er vorsorglich aus den geretteten Reisevorräten mit Kognak gefüllt
hatte. La wußte zwar damit nicht Bescheid, aber er flößte ihr etwas
von dem feurigen Getränk ein, das ihr sehr wohltat.

Er berichtete kurz. Sein Ausflug war ohne entscheidenden Erfolg
geblieben. Der fragliche Gegenstand war eine der im Ballon
befindlichen Decken gewesen, dieselbe, die La jetzt einhüllte. Aber
ob sie von Torm mitgenommen und dort zurückgelassen war oder ob sie
aus dem Ballon bei seinem Flug verloren und vom Wind hingetrieben
worden war, ließ sich nicht feststellen; das letztere war sogar das
wahrscheinlichere. Dabei hatte sich überraschenderweise
herausgestellt, daß der Sohn des Eskimos einige Worte deutsch
verstand. Er war ein Jahr in Diensten deutscher Missionare auf
Grönland gewesen und hatte einzelne Worte aufgefaßt, als Saltner
mit La deutsch sprach. Nur hatte er in Gegenwart der Martier nicht
gewagt, dies zu erkennen zu geben.

Endlich erschienen die Martier wieder am Rand der Spalte. Ein
zweites Seil wurde herabgelassen. Saltner machte einen erträglichen
Sitz zurecht, und indem er La stützte und mit dem Eispickel beide
von der Wand fernhielt, wurden sie glücklich an die Oberfläche
befördert.

„Ich weiß, was ich Ihnen verdanke“, sagte La.

Eine tiefe Erschöpfung ergriff sie, und sie mußte bis an das Boot
getragen werden.

Man trat sofort die Heimfahrt an.

\section{11 - Martier und Menschen}

Der September hatte begonnen. Noch immer beschrieb die Sonne ohne
unterzugehen den vollen Kreis des Himmels, aber sie stand nur noch
wenige Grad über dem Horizont. Schon streifte sie nahe an die
höchsten Gipfel der Berge, welche an einzelnen Stellen die
Steilufer des Polarbassins überragten. Der lange Polartag neigte
sich seinem Ende zu. Wie in einem ewigen Untergang wanderte der
riesige Glutball der Sonne rings um die Insel, meist drang sie nur
strahlenlos wie eine rote Scheibe durch die Nebel, und ein breites,
rosig glühendes Band zog sich durch die leise wogenden Fluten ihr
entgegen und folgte ihrem Lauf als ein natürlicher Stundenzeiger um
den Pol.

Die beiden deutschen Nordpolfahrer verbrachten ihre Tage wie in
einem köstlichen Märchen. Hätte nicht der Gedanke an den verlorenen
Gefährten auf ihre Stimmung niederdrückend gewirkt und den Genuß
der Gegenwart gedämpft, nichts Freudigeres und Erhebenderes wäre
denkbar gewesen als der beglückende Verkehr mit den Bewohnern der
Polinsel, die, wie sie jetzt erfuhren, den Namen Ara führte, zu
Ehren des ersten Weltraumschiffers Ar.

Die Martier behandelten die beiden Erdbewohner als ihre Gäste,
denen jede Freiheit gestattet war. Gegenüber den kleinen,
unansehnlichen, schmutzigen und tranduftenden Eskimos erschienen
ihnen die stattlichen Figuren der Europäer in ihrer reinlichen
Tracht schon äußerlich als Wesen verwandter Art. Nicht wenig trug
dazu die körperliche Überlegenheit bei, welche die Martier, sobald
sie sich nicht im Schutz des abarischen Feldes befanden, an den
Menschen anerkennen mußten. Aufrecht und leicht schritten diese
einher und verrichteten spielend Arbeiten, denen die unter dem
Druck der Erdschwere gebeugt einherschleichenden Martier nicht
gewachsen waren. Denn auch Grunthe war nach wenigen Tagen wieder in
seiner Gesundheit völlig hergestellt und spürte keinerlei üble
Folgen seiner Fußverletzung. Saltner aber hatte sich durch die
entschlossene und geschickte Rettung Las die Achtung der Martier
erworben.

Überraschend schnell hatte sich das gegenseitige Verständnis durch
die Sprache angebahnt. Dies war natürlich hauptsächlich durch die
glückliche Auffindung der kleinen deutsch--martischen
Sprachanweisung gelungen. Es zeigte sich, daß diese von ihrem
Verfasser Ell ganz speziell für diejenigen Bedürfnisse
ausgearbeitet war, die sich bei einem ersten Zusammentreffen der
Menschen mit den Martiern für beide Teile herausstellen würden.
Denn es waren darin weniger die alltäglichen Gebrauchsgegenstände
und Beobachtungen berücksichtigt, über welche man sich ja leicht
durch die Anschauung direkt verständigen kann, wie Speise und
Trank, Wohnung, Kleidung, Gerätschaften, die sichtbaren
Naturerscheinungen und so weiter; vielmehr fanden sich gerade die
Ausdrücke für abstraktere Begriffe, für kulturgeschichtliche und
technische Dinge darin verzeichnet, so daß es Grunthe und Saltner
möglich wurde, sich über diese Gedankenkreise mit den Martiern zu
besprechen. Ell hatte offenbar vorausgesehen, daß, wenn
wissenschaftlich gebildete Europäer mit den in der Kultur ihnen
überlegenen Martiern zusammenkämen, das Hauptinteresse darin
bestehen müßte, sich gegenseitig über die allgemeinen Bedingungen
ihres Lebens zu unterrichten.

Es erregte übrigens bei den Martiern keine geringere Verwunderung
wie bei den beiden Forschern, daß auf Erden ein Mensch existiere,
der sowohl die Sprache und Schrift der Martier beherrschte als auch
eine ziemlich zutreffende Kenntnis der Verhältnisse auf dem Mars
besaß. Aus gewissen Einzelheiten schlossen sie allerdings, daß
diese Kenntnis sich nur auf weiter zurückliegende Ereignisse bezog,
daß insbesondere die Tatsache der Marskolonie am Pol der Erde dem
Verfasser des Sprachführers nicht bekannt war, wohl aber das
Projekt der Martier, die Erde an einem ihrer Pole zu erreichen. Der
Name Ell war in einigen Landschaften des Mars nicht selten. Die
gegenwärtigen Polbewohner erinnerten sich der Berichte, daß bei den
ersten Entdeckungsfahrten nach der Erde mehrfach Fahrzeuge
verschollen waren, ohne daß man jemals etwas über das Schicksal der
kühnen Pioniere des Weltraums hatte erfahren können. Von einem
berühmten Raumfahrer, dem Kapitän All, wußte man sogar gewiß, daß
er mit mehreren Gefährten infolge eines unglücklichen Zufalls auf
der Erde zurückgelassen worden war, allerdings unter Umständen,
welche allgemein an seinen baldigen Untergang glauben ließen.
Immerhin war es wohl denkbar, daß einer oder der andere dieser
Martier zu Menschen sich gerettet und die Kunde vom Mars dahin
gebracht hätte. Diese Ereignisse aber lagen dreißig bis vierzig
Erdenjahre zurück, und jene Männer selbst waren alle in
vorgeschrittenerem Alter gewesen, da eine Beteiligung jüngerer
Leute an jenen ersten, unsicheren Fahrten nicht bekannt war. Ell
selbst, der etwa mit Grunthe gleichaltrig oder nur ein wenig älter
war, konnte also nicht zu ihnen gehören. Und Grunthe wie Saltner
konnten versichern, daß von einem Auftauchen eines Marsbewohners,
ja überhaupt von der Existenz solcher Wesen, auf der Erde nichts
bekannt sei. Ell war der einzige, der ein solches Wissen besaß,
dies aber bis auf jene beiläufigen Redensarten, die Grunthe nicht
ernsthaft genommen, durchaus verborgen gehalten hatte. Wie er
selbst dazu gekommen war, blieb ebenso unaufgeklärt wie die
Umstände, durch welche jene Sprachanweisung in das Flaschenfutteral
gelangt sein konnte, das Frau Isma Torm der Expedition als eine
scherzhafte Überraschung am Nordpol mitgegeben hatte.

Den Bemühungen der Deutschen, sich die Sprache der Marsbewohner
anzueignen, kamen diese bereitwillig entgegen, so daß Saltner und
insbesondere Grunthe sehr bald ein Gespräch auf martisch führen
konnten; gleichzeitig fand es sich, daß auch die Martier, welche
den täglichen Umgang der beiden bildeten, das Deutsche
beherrschten. Ersteres wurde dadurch möglich, daß die
Verkehrssprache der Martier außerordentlich leicht zu erlernen und
glücklicherweise für eine deutsche Zunge auch leicht auszusprechen
war. Sie war ursprünglich die Sprache derjenigen Marsbewohner
gewesen, die auf der Südhalbkugel des Planeten in der Gegend jener
Niederungen wohnten, welche von den Astronomen der Erde als
Lockyer-Land bezeichnet werden. Von hier war die Vereinigung der
verschiedenen Stämme und Rassen der Martier zu einem großen
Staatenbund ausgegangen, und die Sprache jener Zivilisatoren des
Mars war die allgemeine Weltverkehrssprache geworden. Durch einen
Hunderttausende von Jahren dauernden Gebrauch hatte sie sich so
abgeschliffen und vereinfacht, daß sie der denkbar glücklichste und
geeignetste Ausdruck der Gedanken geworden war; alles Entbehrliche,
alles, was Schwierigkeiten verursachte, war abgeworfen worden.
Deswegen konnte man sie sich sehr schnell soweit aneignen, daß man
sich gegenseitig zu verstehen vermochte, wenn es auch
außerordentlich schwierig war, in die Feinheiten einzudringen, die
mit der ästhetischen Anwendung der Sprache verbunden waren.

Übrigens war dies nur die Sprache, die jeder Martier beherrschte.
Neben derselben aber gab es zahllose, sehr verschiedene und in
steter Umwandlung begriffene Dialekte, die bloß in verhältnismäßig
kleinen Gebieten gesprochen wurden, endlich sogar Idiome, die
allein im Kreis einzelner Familiengruppen verstanden wurden. Denn
es zeigte sich als eine Eigentümlichkeit der Kultur der Martier,
daß der allgemeinen Gleichheit und Nivellierung in allem, was ihre
soziale Zusammengehörigkeit als Bewohner desselben Planeten
anbetraf, eine ebenso große Mannigfaltigkeit und Freiheit des
individuellen Lebens entsprach. Wenn so die schnelle Erfaßbarkeit
des Martischen den Deutschen zugute kam, so brachte die
erstaunliche Begabung der Martier andererseits zuwege, daß sie sich
wie spielend das Deutsche aneigneten. Gegenüber dem verwirrenden
Formenreichtum des Grönländischen erschien ihnen das Deutsche
wesentlich leichter. Was aber die schnellere Erlernung desselben
hauptsächlich bewirkte, war der Umstand, daß das Deutsche als
Sprache eines hochentwickelten Kulturvolkes dem geistigen Niveau
der Martier soviel näherstand. Was der Grönländer in seiner Sprache
auszudrücken wußte, die konkrete Art, wie er es nur ausdrücken
konnte, der enge Interessenkreis, auf den sich das Leben des Eskimo
beschränkte, das alles war dem Martier sehr gleichgültig, und er
beschäftigte sich damit nur, weil er bisher kein anderes Mittel
besaß, mit Bewohnern der Erde in Verkehr zu treten. Ganz anders
aber wurde das Interesse der Martier erregt, als sie mit Grunthe
und Saltner Gesprächsthemata berühren konnten, die ihrem eigenen
gewohnten Gedankenkreis näherlagen. Im Deutschen fanden sie eine
Sprache, reich an Ausdrücken für abstrakte Begriffe, und dadurch
verwandt und angemessen ihrer eigenen Art zu denken. Die
Überlegenheit, mit welcher die Martier die kompliziertesten
Gedankengänge behandelten und in einem allgemeinen Begriff jede
einzelne seiner Anwendungen mit einemmal überblickten, diese
bewundernswerte Feinheit der Organisation des Martiergehirns kam
den Deutschen zum erstenmal zum vollen Bewußtsein, als sie die
Gewandtheit bemerkten, mit welcher die Martier das Deutsche nicht
nur erfaßten und gebrauchten, sondern gewissermaßen aus dem einmal
begriffenen Grundcharakter die Sprache mit genialer Kraft
nachschufen.

Grunthe und Saltner wurde es sehr bald klar, daß die Martier
geistig in ganz unvergleichlicher Weise höher standen als das
zivilisierteste Volk der Erde, wenn sie auch noch nicht zu
übersehen vermochten, wie weit diese höhere Kultur reiche und was
sie bedeute. Ein Gefühl der Demütigung, das ja nur zu natürlich
war, wenn der Stolz des deutschen Gelehrten einer höheren
Intelligenz sich beugen mußte, wollte im Anfang die Gemüter
verstockt machen. Aber es konnte nicht lange vor der übermächtigen
Natur der Martier bestehen. Es wich widerstandslos der ungeteilten
Bewunderung dieser höheren Wesen. Neid oder Ehrgeiz, es ihnen
gleichzutun, konnten bei den Menschen gar nicht aufkommen, weil sie
sich nicht einfallen lassen durften, sich mit den Martiern auf
dieselbe Stufe der Einsicht stellen zu wollen.

Freilich wurden sie von den Martiern wie Kinder behandelt, denen
man ihre Torheit liebevoll nachsieht, während man sie zu besserem
Verständnis erzieht. Aber davon merkten Grunthe und Saltner nichts.
Denn die Martier, wenigstens diejenigen der Insel, waren viel zu
klug und taktvoll, als daß sie je ihre Überlegenheit in direkter
Weise geltend gemacht hätten. Sie wußten es so einzurichten, daß
den Menschen die Berichtigung ihrer Irrtümer als Resultat der
eigenen Arbeit erschien, und ihre unvermeidlichen Mißgriffe
korrigierten sie mit entschuldigender Liebenswürdigkeit.

Die Wunder der Technik, welche die Forscher bei jedem Schritt auf
der Insel umgaben, versetzten sie in eine neue Welt. Sie fühlten
sich in der beneidenswerten Lage von Menschen, die ein mächtiger
Zauberer der Gegenwart entrückt und in eine ferne Zukunft geführt
hat, in welcher die Menschheit eine höhere Kulturstufe erklommen
hat. Die kühnsten Träume, die ihre Phantasie von der Wissenschaft
und Technik der Zukunft ihnen je vorgespiegelt hatte, sahen sie
übertroffen. Von den tausend kleinen automatischen Bequemlichkeiten
des täglichen Lebens, die den Martiern jede persönliche
Dienerschaft ersetzten, bis zu den Riesenmaschinen, die, von der
Sonnenenergie getrieben, den Marsbahnhof in sechstausend Kilometer
Höhe schwebend erhielten, gab es eine unerschöpfliche Fülle neuer
Tatsachen, die zu immer neuen Fragen drängten. Bereitwillig gaben
die Wirte ihren Gästen Auskunft, aber in den meisten Fällen war es
gar nicht möglich, ihnen den Zusammenhang zu erklären, weil ihnen
die Vorkenntnisse fehlten. Grunthe war in dieser Hinsicht so
vorsichtig, nicht viel zu fragen; er suchte sich auf seine eigne
Weise zurechtzufinden, sobald er sah, daß die Erklärung der Martier
über seinen Horizont ging. Saltner machte sich weniger Skrupel
darüber. „Das hilft nun nichts“, pflegte er zu sagen, „wir spielen
einmal hier die wilden Indianer, und was wir nicht begreifen, ist
Medizin.“

Als ihnen Hil zum erstenmal die Einrichtung erklärt hatte, wodurch
sich die Martier in ihren Zimmern den Druck der Erdschwere
erleichterten, und Grunthe mit zusammengekniffenen Lippen in tiefes
Nachdenken verfiel, sagte Saltner einfach: „Medizin“ und hob
Grunthe samt dem Stuhl, auf welchem er saß, mit ausgestreckten
Armen über seinen Kopf. Diese Kraftleistung war zwar für ihn bei
der auf ein Drittel verringerten Erdschwere durchaus nichts
Besonderes, ließ ihn aber doch den Martiern als einen Riesen an
Stärke erscheinen.

Das Zimmer, welches an die beiden Schlafzimmer von Grunthe und
Saltner stieß, war für den bequemen Verkehr der Martier mit den
Menschen in eigentümlicher Weise eingerichtet worden. Da nämlich
die Verringerung der Erdschwere, deren die Martier für die
Leichtigkeit ihrer Bewegungen bedurften, von Grunthe und Saltner
nicht gut vertragen wurde, so hatte man es durch eine am Boden
markierte Linie – Saltner nannte sie den ›Strich‹ – in zwei Teile
zerlegt. Der abarische Apparat konnte für die Hälfte des Zimmers,
welche an die Wohnräume der Menschen grenzte, ausgeschaltet werden,
während in dem übrigen Teil die Gegenschwere auf das den Martiern
gewohnte Maß eingestellt wurde. Hier hielten sich die Martier auf,
wenn sie bei den Deutschen ihre Besuche machten, während diese sich
nach ihren Wünschen eingerichtet hatten, soweit es mit den von den
Martiern bereitwillig hergegebenen Möbeln und den wenigen von ihnen
selbst mitgebrachten Gegenständen geschehen konnte. Freilich
beschränkte sich diese Einrichtung nur auf die Aufstellung eines
Arbeitstisches, einiger Bücher, Schreibmaterialien und Instrumente;
denn in dieser Hinsicht wußten die Forscher nur in der ihnen
gewohnten Weise auszukommen. Was im übrigen die Bequemlichkeiten
des täglichen Lebens anbetraf, so waren sie nicht nur auf die
Apparate und Gewohnheiten der Martier angewiesen, sondern fanden
dieselben auch bald um so viel vorteilhafter und angenehmer, daß
sie gern darüber nachdachten, wie sie dergleichen in ihre Heimat
verpflanzen könnten.

Saltner, der seinen photographischen Apparat unter den geretteten
Gegenständen wiedergefunden hatte, konnte kaum Zeit genug gewinnen,
alle die Ausstattungsstücke der Martier aufzunehmen und die
gänzlich neuen Formen der Verzierungen, die Gemälde, Kunstwerke und
Zimmerpflanzen abzubilden. Ein besonderes Studium machte er aus den
Automaten, deren Mechanismus er zu ergründen suchte und sich immer
wieder aufs neue erklären ließ.

Seine Beraterin in diesen Dingen war in der Regel die immer heitere
Se, seine liebenswürdige Pflegerin beim ersten Erwachen. Sie hielt
sich täglich einen großen Teil ihrer Zeit über in dem
gemeinschaftlichen Gesellschaftszimmer auf und machte den Gästen
gewissermaßen die Honneurs des Hauses. Dagegen bekam Saltner La nur
selten zu sehen, gewöhnlich nur des Abends, wenn sich die Martier
in größerer Anzahl einzustellen pflegten. Und dann hielt sie sich
gern zurück, obwohl er oft fühlte, daß ihre großen Augen mit einem
sinnenden Ausdruck auf ihm ruhten. Sein lebhaftes Gespräch mit Se
aber unterbrach sie häufig durch eine Neckerei. Da man sich meist
bei geöffneten Fernhörklappen unterhielt, so konnte man, sobald man
wollte, einem Gespräch in einem andern Zimmer zuhören und sich
hineinmischen; so war es nichts Ungewöhnliches, daß man von einem
Zwischenruf eines ungeahnten Zuhörers unterbrochen wurde.
Ebensowenig aber nahm es jemand übel, wenn man einfach seine Klappe
abschloß.

Die Sprachstudien waren speziell zwischen La und Saltner nicht
wieder aufgenommen worden. Denn La hatte noch mehrere Tage nach
ihrem Unfall sich vollkommener Ruhe hingeben müssen, und als sie
wieder gesundet war, fand sie das gegenseitige Verständnis zwischen
Menschen und Martiern schon ziemlich weit vorgeschritten. Aber auch
sie hatte ihre unfreiwillige Muße benutzt und nicht nur den
Ellschen Sprachführer, sondern auch die wenigen Nachschlagwerke,
welche die Luftschiffer mit sich hatten, durchstudiert.

Trotz des Eindrucks, den die reizende Se auf Saltners empfängliches
Gemüt machte, flogen seine Gedanken immer zu der stilleren, milden
La zurück, und es war ihm stets wie eine leichte Enttäuschung, wenn
er sie im Zimmer nicht vorfand. Gerade daß er öfter ihre tiefe
Stimme vernahm, ließ ihn ihren Anblick um so mehr vermissen.

Las Zurückhaltung war nicht absichtslos. Daß sowohl sie wie Se eine
unentrinnbare Gefahr für Saltners Herz waren, lag ja für beide auf
der Hand, nachdem sie sich überhaupt erst an den Gedanken gewöhnt
hatten, daß ein Mensch sich verlieben könne. Was aber Se höchst
komisch vorkam und als äußerst spaßhaft erschien, das vermochte La
so harmlos nicht anzusehen. Der ›arme Mensch‹, mit dem Se sich so
lustig unterhielt, war ihr doch in einem andern Licht erschienen,
damals, als er, in seinem eignen Element tätig, Leistungen
verrichtete, die über das Vermögen der Nume hinausgingen.

Sie konnte den Moment nicht vergessen, in welchem sie sich in
seinen starken Armen vom vernichtenden Abgrund zurückgerissen
fühlte. Und so blieb es ihr immer gegenwärtig, daß dieses Spielzeug
der erhabenen Nume, wenn auch nur ein Mensch, doch ein freies
Lebewesen sei, kein ebenbürtiger Geist, aber vielleicht ein
ebenbürtiges Herz. Ein doppeltes Mitleid stritt mit sich selbst in
ihrer Seele, sie vermochte ihn nicht zu kränken durch Kälte und
Zurückweisung, und sie wollte nicht Gefühle erwecken, die ihm doch
nur zu größerem Leid werden konnten. Wer kann wissen, wie
Menschenherzen fühlen mögen? Vielleicht waren die Menschen viel
stärker in ihren Gefühlen als in ihrem Verstand. Und sie war
Saltner zu dankbar, um nicht für ihn zu denken, was er wohl nicht
verstand. – Aber was tun?

Wäre Saltner ein Martier gewesen, so hätte es keiner Vorsicht für
La bedurft. Er hätte dann gewußt, daß ihre Freundlichkeit und
selbst ihre Zärtlichkeit nichts bedeuteten als das ästhetische
Spiel bewegter Gemüter, das die Freiheit der Person nicht
beschränken kann. Wie jedoch mochten Menschen in diesem Fall
denken? Durfte sie hierin ohne weiteres gleiche Sitten
voraussetzen? Und würde er wohl verstehen, was von vornherein und
immer den Menschen, den wilden Erdbewohner, von der heiteren
Freiheit des erhabenen Numen trennte? Und lief er nicht Gefahr, bei
Se demselben Schicksal zu verfallen, vor dem sie ihn selbst zu
behüten suchte?

Wenn sie Se ihre Bedenken andeutete, so lachte diese nur.

„Aber La“, sagte sie, „du bist auch gar zu bedächtig! Ich bitte
dich, er ist ja bloß ein Mensch! Es ist doch furchtbar komisch,
wenn der sich Mühe gibt, so recht liebenswürdig zu sein.“

„Du kannst aber nicht wissen“, antwortete La, „ob ihm auch so
furchtbar komisch zumute ist. Ein Tier, das wir necken, scheint uns
oft äußerst lächerlich, und ich muß dann doch immer denken, daß es
vielleicht bitter dabei leidet. Und ein Mensch ist doch nicht bloß
komisch –“

„Ich habe freilich noch keinen in einer Eisgrube gesehen“, sagte
Se, „doch ich glaube, du brauchst dir um den keine Sorge zu machen.
Wenn es dich aber beruhigt, so kann man ihn ja leicht merken
lassen, wie’s gemeint ist –“

„Ich will ihn aber nicht kränken.“

„Im Gegenteil, wir machen gemeinsame Sache. Wir binden ihn beide.“

„Meinst du, daß ein Mensch das Spiel versteht?“

„Na, wenn er so dumm ist –“

„Wir wissen doch gar nichts von den Anschauungen –“

„So werden wir uns eben alle drei belehren. Schade, daß der steife
Grunthe nicht mitspielen kann. Willst du?“

„Ich werde mir’s überlegen.“

La zog sich zu ihren Studien zurück. Se begab sich in das
Gesellschaftszimmer, wo sie Saltner wieder mit Zeichnen beschäftigt
fand.

„Wenn ich mit meinen Mustern glücklich nach Deutschland
zurückkomme“, rief er vergnügt, „so bin ich ein gemachter Mann.
›Martisch‹ muß Mode werden. Ich gründe einen Bazar für Marswaren.
Schade nur, daß wir die Rohstoffe nicht haben werden. Was ist das
zum Beispiel für ein wunderbares Gewebe, aus dem Ihr Schleier
besteht? Die Stickerei darin bildet lauter funkelnde Sterne, die
sich nirgends untereinander berühren; nirgends ist ein Grund
sichtbar, der sie zusammenhält. Es scheint, als schwebe eine Wolke
von Funken um Sie her.“

„Das tut sie auch“, sagte Se lachend, „aber sie brennt nicht,
fühlen Sie getrost! Kommen Sie gefälligst hierher, denn über den
Strich gehe ich nicht.“

Se hatte sich, mit einer chemischen Handarbeit beschäftigt, auf
einem der niedrigen Diwane, wie die Martier sie lieben,
niedergelassen, während Saltner an seinem eigenhändig
hergerichteten Pult sich befand. Er legte den Zeichenstift fort und
trat an Se heran, die sich mit ihrem Diwan bis dicht an die
Schwerkraftgrenze gerückt hatte.

„Geben Sie Ihre Hände her“, sagte Se.

Sie nahm ein Ende des langen Schleiers und band damit Saltners
Hände zusammen. Man konnte keinerlei Stoff erkennen. Es sah auch
jetzt aus, als wenn ein Strom vom lichten Funken um seine Hände
stöbe.

„Fühlen Sie etwas?“ fragte Se.

„Jetzt, nachdem Sie Ihre Finger fortgenommen haben, nichts. Kann
man denn den Stoff überhaupt nicht fühlen?“

„Wenigstens nicht mit der groben Haut von euch Menschen.“

Saltner führte die zusammengebundenen Hände mit dem Schleier an
seine Lippen.

„Doch“, sagte er, „mit den Lippen fühle ich, daß etwas zwischen
meiner Hand und meinem Mund ist.“

„Nun strengen Sie einmal Ihre Riesenkräfte an, und reißen Sie die
Hände voneinander.“

„Oh, das wäre schade um den Funkenschleier.“

„Versuchen Sie es nur.“

Saltner zerrte seine Hände auseinander, aber je heftiger er zog, um
so enger schloß sich der Knoten, und er merkte jetzt, wie sich die
kleinen Sternchen in seine Haut eingruben.

„Ja“, sagte Se, „der Stoff ist unzerreißbar, wenigstens kann er
kolossale Lasten halten. Diese unsichtbar feinen Fäden, von denen
jeder wohl einen Zentner tragen kann, sind für viele unserer
Apparate ein unentbehrlicher Bestandteil. Jetzt sind Sie also
gefesselt und können ohne meine Erlaubnis nicht mehr fort.“

„Um die bitte ich auch gar nicht, ich finde es reizend hier“, sagte
Saltner und beugte sich über die Lehne des Diwans, auf welche er
die gebundenen Hände stützte.

Se faßte seinen Kopf zwischen ihre Hände und bog ihn zu sich
nieder, während sie ihm in die Augen sah, als wollte sie seine
Gedanken ergründen.

„Seid ihr eigentlich dumm, ihr Menschen?“ fragte sie plötzlich.

„Nicht so ganz“, sagte Saltner, indem er sich noch tiefer
herabbeugte.

„Der Strich!“ rief Se lachend und schob seinen Kopf leicht zurück.
„Geben Sie die Hände her.“

Sie löste im Augenblick den Knoten und ergriff wieder die gläsernen
Stäbchen, mit denen sie in einem Gefäß auf besondere Weise
hantierte.

„Sie haben mir noch immer nicht gesagt“, sprach Saltner, nach
seinem Pult zurückgehend, „was für ein Stoff das ist, auf dem die
Stickerei sitzt.“

„Eine Stickerei ist es überhaupt nicht, sondern es sind Dela – wie
heißt das? Aus Muscheln, kleine Kristalle, die sich darin bilden.“

„Also etwas Ähnliches wie unsere Perlen –“

„Aber sie leuchten von selbst. Und der Stoff ist Lis.“

„Lis? Da bin ich ebenso klug.“

„Lis ist eine Spinne, sie webt ein fast unsichtbares Netz.“

„Und wie findet man das auf? Wie webt man die Fäden?“

„Im polarisierten Licht, sehr einfach, und mit besonderen
Maschinen. Und die Dela sind nicht daraufgesetzt, sondern sie
liegen in Schlingen zwischen dünnen Schichten des Gewebes.“

„Sie nannten die Dela Kristalle – wie ist es denn möglich, daß sie
dieses Eigenlicht dauernd aussenden, ähnlich wie unsre
Glühwürmchen?“

„Sie müssen natürlich von Zeit zu Zeit ins Strahlbad, dann leuchten
sie wieder ein paar Tage.“

„Ins Strahlbad?“

„Nun ja, sie werden einer starken, künstlichen Bestrahlung
ausgesetzt. Das Licht trennt einen Teil der chemischen Stoffe der
Kristalle voneinander, und indem diese sich nachher langsam wieder
vereinigen, entsteht das Selbstleuchten.“

„Also was wir Phosphoreszenz nennen. Und was haben Sie dort für
eine Handarbeit?“

Se antwortete nicht sogleich. Sie stellte gerade eine Kopfrechnung
an, die sich auf ihre Arbeit bezog, und betrachtete dabei den
Sekundenzeiger der Zimmeruhr.

Da klang die Klappe des Fernsprechers, und gleich darauf vernahm
man die Stimme von La. Sie fragte an, ob die ›Menschen‹ für einige
Herren der Insel zu sprechen seien.

„Es wird mir sehr angenehm sein, die Herren zu sehen“, sagte
Saltner. „Mein Freund ist augenblicklich nicht anwesend, aber ich
werde ihn sogleich rufen.“ –

\section{12 - Die Raumschiffer}

Grunthe beschäftigte sich auf der Oberfläche der Insel mit
Messungen. Was ihn sowohl wie Saltner besonders wunderte, war der
Umstand, daß die vom Ballon aus beobachtete Erdkarte auf dem Dach
der Insel selbst durchaus nicht sichtbar war. Wie kamen die Martier
überhaupt auf die Idee, eine solche Riesenkarte anzubringen, und
auf welche erstaunliche Weise war sie hergestellt? Aber gerade
darüber konnten die Forscher auf ihre Fragen keine Auskunft
erhalten.

Grunthe liebte es, sich soviel als möglich im Freien aufzuhalten,
um sowohl die technischen Einrichtungen der Insel als auch die
Erscheinungen der Natur am Nordpol zu studieren, ja er hatte schon
mit Unterstützung einiger Martier Bootfahrten auf dem Binnenmeer
und ebenfalls bis zum gegenüberliegenden Ufer vorgenommen, ohne
jedoch auf weitere Spuren von Torm zu treffen. Er hatte dabei
bemerkt, daß die Polinsel infolge ihrer versteckten Lage zwischen
den übrigen höheren Inseln von den Ufern des Bassins aus überhaupt
nicht wahrnehmbar und somit gegen zufällige Entdeckung geschützt
war. So ernsthaft ihn diese Studien beschäftigten, war es ihm doch
nebenbei sehr angenehm, mit einem triftigen Vorwand sich von dem
Konversationszimmer fernzuhalten. Denn hier waren einen großen Teil
des Tages über Se oder La, manchmal auch eine oder die andre der
übrigen auf der Insel wohnenden Frauen anwesend, und die Aufgabe
der Höflichkeit, sich mit diesen zu unterhalten, überließ er gern
Saltner, der sich derselben mit Vorliebe unterzog. Im Freien
dagegen war er ziemlich sicher, keiner von den Damen zu begegnen.
Außerhalb der Schutzvorrichtungen, die sie von einem Teil der
Erdschwere befreiten, war ihnen der Aufenthalt zu lästig; und sie
wußten wohl, daß der schwerfällige Schritt und die gebeugte
Haltung, die ihnen dort die eigene Körperlast auferlegte, ihre
Anmut keineswegs erhöhten. Insbesondere den Menschen gegenüber, die
sich hier ungezwungen in ihrem Element fühlten, zeigten sie sich
nicht gern in dem Zustand physischer Unfreiheit.

Da Saltner wußte, daß sich Grunthe in der Nähe aufhielt, konnte er
ihn leicht benachrichtigen.

Die Zahl der auf der Insel befindlichen Martier war nicht
unbedeutend, sie mochte gegen dreihundert Personen betragen,
worunter sich ungefähr fünfundzwanzig Frauen, aber keine Kinder
befanden. Die Lebensweise dieser Kolonie entsprach nicht den
Gewohnheiten der Martier auf ihrem eigenen Planeten; es waren nicht
Familien, die sich hier angesiedelt hatten, sondern die Kolonisten
bildeten eine ausgewählte Truppe mit militärischer Organisation,
wie sie von den Martiern zur Vornahme wichtiger öffentlicher
Arbeiten ausgerüstet wurde. Aber auch hier war dem Bedürfnis der
Nume nach möglichst großer individueller Unabhängigkeit Rechnung
getragen. Die einzelnen hatten sich je nach ihrer persönlichen
Neigung zu Gruppen zusammengefunden und danach ihre Wohnung auf der
Insel gewählt. Jede dieser Gruppen wurde durch einen der älteren
Beamten geleitet, der die Ordnung der Arbeiten verteilte. Ihm stand
eine der Damen zur Seite, welche gewissermaßen die häusliche
Wirtschaft der Gruppe führte, die Verteilung der Nahrungsmittel
beaufsichtigte und die regelmäßige Funktion der Automaten
kontrollierte, während jedes Mitglied einer Gruppe eine bestimmte
Zeit der Bedienung dieser Automaten widmete.

Die Pflege der beiden Gäste hatten die Gruppen des Ingenieurs Fru
und des Arztes Hil übernommen, denen als weibliche Assistenten La
und Se angehörten. Es war natürlich, daß Saltner und Grunthe
hauptsächlich mit den Mitgliedern dieser Gruppe verkehrten, wozu
sich noch als täglicher Gast der Direktor der Kolonie, Ra,
gesellte. Mit den übrigen Gruppen waren sie bisher nur gelegentlich
in Berührung gekommen.

Die Martier, welche im Begriff standen, ihren Besuch bei den Gästen
zu machen, gehörten der Gruppe des Ingenieurs Jo an, dessen
Tätigkeit Grunthe und Saltner hauptsächlich ihre Rettung
verdankten. Selbstverständlich hatten sie nicht versäumt, ihm
alsbald nach ihrer Wiederherstellung ihren herzlichsten Dank
abzustatten.

Mit ihnen zusammen erschien La. Sie trat zuerst Saltner entgegen
und bot ihm mit einem reizenden Lächeln über den ›Strich‹ hinüber
ihre Hand. Aber ehe noch Saltner in ein Gespräch mit ihr kam, wußte
Se sie beiseite zu ziehen. Während Jo mit Saltner sprach,
unterhielten sich die beiden Damen eifrig und leise, worauf Se das
Zimmer verließ. Jo begrüßte Saltner in seiner offenen, nach
martischen Begriffen etwas derben Weise und nannte die Namen seiner
Begleiter. Jeder von ihnen grüßte nach martischer Sitte, indem er
die linke Hand ein wenig erhob und die Finger derselben leicht
öffnete und schloß. Saltner bewies die Fortschritte in seiner
Bildung dadurch, daß er den Gruß in derselben Weise erwiderte. Die
Martier wollten ihm jedoch an Höflichkeit nicht nachstehen und
schüttelten ihm der Reihe nach auf deutsche Weise die rechte Hand,
ohne sich merken zu lassen, wie sehr diese barbarische Zeremonie
sie innerlich belustigte. Sie hüteten sich dabei sorglich, den
Strich zu überschreiten, jenseits dessen die Erdschwere begann.

Auf Saltners Einladung nahmen sie an der breiten Tafel in der Mitte
des Zimmers Platz. Man hatte dieses Zimmer in Rücksicht auf
zahlreiche Versammlungen so eingerichtet, daß ein großer Tisch die
Länge desselben erfüllte und mit dem einen Ende über den ›Strich‹
hinüberragte. Hier befanden sich die Plätze für die beiden
Deutschen. In den Besuchsstunden, besonders aber am Abend, wenn die
Arbeiten des Tages beendet waren, pflegte sich hier stets eine
größere Gesellschaft zusammenzufinden. Dann wurde auch bei
gemeinschaftlichen Gesprächen eine leichte Erfrischung in Form von
Getränken eingenommen. Die Einhaltung dieser Plauderstunden war
eine feststehende Sitte der Martier. Die Mahlzeiten dagegen, welche
wirklich zur Sättigung dienten, fanden niemals gemeinschaftlich
statt; dies galt bei den Martiern als unpassend. Beim Essen schloß
sich ein jeder ab, und schon daß Saltner und Grunthe
gemeinschaftlich zu speisen pflegten, erschien den Martiern als ein
Zeichen der stark tierischen Natur der Menschen. Nach ihrer Ansicht
war die Sättigung eine physische Verrichtung, welche nicht in die
Gesellschaft gehörte; in dieser wurden nur ästhetische Genüsse
gestattet. Zu solchen ästhetischen Genüssen gehörten Essen und
Trinken allerdings auch, insofern sie dem reinen Wohlgefallen am
Geschmack entsprachen und sich der Empfindungen der Zunge und des
Gaumens nur zum freien Spiel bedienten, nicht aber insofern sie den
Zweck der Ernährung und die Stillung des körperlichen Bedürfnisses
zu erfüllen bestimmt waren.

Auf Las Aufforderung, welche jetzt die Stelle der Wirtin vertrat,
öffneten die Martier die auf dem Tisch stehenden Kästchen und
bedienten sich der darin befindlichen Piks.

Der Gebrauch dieser Piks ersetzte den Martiern in vollkommener
Weise den Genuß, welchen die Menschen durch das Rauchen erreichen,
ein leichtes, die Sinne mäßig beschäftigendes und die Nerven
beruhigendes, damit den ganzen Gemütszustand behaglich hebendes
Spiel, das aber dem Rauchen gegenüber den Vorteil hatte, daß es die
Luft nicht verdarb und die übrigen Anwesenden nicht belästigte. Die
Piks bestanden in Kapseln, etwa in der Größe und Gestalt einer
kleinen Taschenuhr, die an leichte Aluminiumstäbe gesteckt und
dadurch bequem hin und her bewegt wurden. Brachte man diese Kapsel,
während man den Stiel in der Hand hielt, an die Stirn, so ging ein
schwacher, angenehm erregender Wechselstrom durch den Körper,
wodurch man sich wohltuend erfrischt fühlte. Die Bewegung der Hand
und das Streichen der Stirn und Schläfen war ein sehr anmutiger
Zeitvertreib. Dabei zeigte sich auf der Kapsel ein zartes
Farbenspiel je nach der Größe des Widerstandes, den der Strom fand,
und die Art der Berührung, die Wendungen des Piks boten eine reiche
Abwechslung der Beschäftigung. Der Kenner wußte diese leichten
Reize des Gefühls aufs feinste zu variieren. Wegen der Grazie und
Zierlichkeit der Bewegungen, mit denen Se und La die Piks zu
handhaben pflegten, hatte Saltner diesen Instrumenten den Namen
Nervenfächer beigelegt.

„Freut mich sehr“, sagte Jo, mit seinem Pik an die Stirn klopfend,
„den Herrn Bat wieder wohlzusehen. Hätt’s nicht gedacht, als wir
Sie unter dem Ballon hervorholten. Habe leider wenig Zeit gehabt,
mit Ihnen zu plaudern, hätte gern etwas über Ihre Luftfahrt
gehört.“

„Dazu ist hoffentlich noch Gelegenheit“, sagte Saltner.

„Fürchte nein“, erwiderte Jo. „Kommen nämlich, uns zu
verabschieden. Morgen geht’s heim.“

„Wie?“ fragte Saltner erstaunt.

Jo deutete mit dem Pik nach einer Stelle des Fußbodens und sagte:
„Nu.“

Saltner mußte sich erst besinnen, daß Jo mit seiner Bewegung die
Richtung nach dem Mars bezeichne, denn unwillkürlich stellte er
sich die Fahrt nach dem Mars immer als einen Aufstieg gegen den
Himmel vor. Aber der Mars befand sich gegenwärtig unter dem
Horizont, und dahin deutete Jo.

„Sie sollten mit uns kommen“, sagte Jo lächelnd. „Das ist doch noch
ganz etwas anderes bei uns auf dem Mars wie hier auf der schweren
Erde, wo man sich genieren muß, vor die Tür zu gehen.“

„Ich danke“, erwiderte Saltner, „ich fürchte, auf dem Mars Sprünge
zu machen, die mir nicht gut bekommen würden. Interessant wäre es
ja freilich, Ihre wunderbare Heimat kennenzulernen, aber glauben
Sie denn, daß ein Mensch bei Ihnen existieren kann?“

„Gewiß könnte er das“, sagte einer der anwesenden Martier, „und
zwar viel besser, als wir auf der Erde fortkommen. Ich bin
überzeugt, daß Sie sich an die geringere Schwere bald gewöhnen
würden und ebenso an die dünnere Luft. Beide Umstände kompensieren
sich einigermaßen in der Wirkung auf den Organismus, und Sie müssen
wissen, daß die Luft bei uns relativ reicher an Sauerstoff ist als
hier. Wie wäre es auch sonst möglich, daß die Bewohner beider
Planeten eine so große Ähnlichkeit besitzen?“

„Ich bin Ihnen sehr verbunden für dieses Kompliment“, antwortete
Saltner, „indessen ist unsere Expedition doch nicht auf einen so
weiten Ausflug eingerichtet, und wir müssen zunächst daran denken,
wieder nach Hause zu kommen.“

„Es wird Ihnen wohl etwas einsam hier werden“ mischte sich La in
das Gespräch.

„Wie“, fragte Saltner überrascht, „gehen Sie auch fort?“

„Morgen noch nicht, aber im Verlauf der nächsten – – ja, ich will
es Ihnen lieber in Ihre Zeitrechnung nach Erdtagen übersetzen –,
also in den nächsten vierzehn Tagen ungefähr werden wir fast alle
die Erde verlassen haben.“

„Aber davon höre ich das erste Wort.“

„Weil wir überhaupt noch nicht von der Zukunft gesprochen haben –“

„Es ist wahr, die Gegenwart war zu schön und zu reich –“

„Nun, werden Sie nicht melancholisch! Und dann versteht es sich ja
doch von selbst, daß wir im Winter nicht hierbleiben, ausgenommen
die Wächter.“

„Was für Wächter?“

„Wir erwarten sie mit dem nächsten Fahrzeug vom Nu“, sagte Jo. „Sie
sind unsre Ablösung – nur zwölf Mann, die hier überwintern und die
Insel bewachen. Im Winter können wir unsre Arbeiten nicht
fortsetzen, und die ganze Insel zu heizen, das wäre denn doch zu
kostspielig.“

„Und kommen Sie im Sommer zurück?“

„Wir oder andere.“

„Und ich denke, Sie bringen die Polarnacht nicht hier auf der Insel
zu, sondern bei uns. Dort, wo wir auf dem Mars wohnen, haben wir
dann gerade unsern herrlichen Spätsommer. Und wenn die Sonne hier
am Nordpol wieder aufgeht, reisen Sie vom Mars ab und kommen dann
im Lauf Ihres Mai hier an. Das ist gerade die rechte Zeit für den
Pol – und dann werden Sie, denke ich, Ihre Freunde vom Mars zu
Ihren Landsleuten zu führen wissen. Sie brauchen aber nicht jetzt
schon mit Jo zu reisen, wir verlassen die Erde erst mit dem letzten
Schiff.“

La hatte dies zu Saltner gesagt. Und als sie ihn dabei so
freundlich ansah, schien es ihm, als könne es gar nicht anders
sein, er müsse mit nach dem Mars gehen. Aber was würde Grunthe dazu
sagen?

Allerdings hatten weder Saltner noch Grunthe bisher mit den
Martiern über ihre nächste Zukunft gesprochen. Das hatte
verschiedene Ursachen in zufälligen Umständen. Der Hauptgrund war
jedoch, wohl ohne daß die beiden Deutschen sich darüber klar
wurden, daß die Martier bisher es absichtlich vermieden hatten,
sich über diese Frage zu äußern. Sie hatten selbst noch keinen
Entschluß gefaßt. Auf die erste Lichtdepesche nach dem Mars über
die Auffindung der Menschen hatte die Zentralregierung der
Marsstaaten geantwortet, daß man zunächst die Fremdlinge beobachten
und ausforschen und dann über sie Bericht erstatten solle. Dieser
Bericht war vor kurzem abgegangen, die Antwort jedoch noch nicht
eingetroffen. Deshalb hatten die Martier jede Hindeutung auf das
weitere Schicksal ihrer Gäste vermieden, und sobald Grunthe und
Saltner eine Frage in dieser Hinsicht zu stellen oder einen Wunsch
zu äußern versuchten, waren sie darüber mit einer ausweichenden
Antwort hinweggegangen. Wenn aber die Martier auf irgendeine Frage
nicht eingehen wollten, so war es für die Menschen ganz unmöglich,
sie dahin zu bringen. Die Leichtigkeit, mit welcher sie die
Gedanken lenkten, und die Überlegenheit ihres Willens waren so
groß, daß die Menschen ihnen folgen mußten und dabei kaum merkten,
daß sie geleitet wurden. Aber Grunthe wie Saltner waren in der Tat
noch so erfüllt von den Aufgaben, die ihnen auf der Insel gestellt
waren, daß sie die Pläne über die Fortsetzung ihrer Reise selbst in
ihren Gesprächen untereinander nur vorübergehend berührt hatten.
Sie hatten sich zwar vorgenommen, in den nächsten Tagen einen
definitiven Entschluß zu fassen und zu gelegener Zeit mit den
Martiern darüber zu reden, bis jetzt war es aber noch nicht dazu
gekommen. Grunthe glaubte nämlich, daß sie, falls nur die Erlaubnis
der Martier erlangt war, jederzeit die Insel ohne Schwierigkeit
würden verlassen können, weil er nach einer allerdings nur
vorläufigen Untersuchung sich für überzeugt hielt, daß der Ballon
mit verhältnismäßig geringer Mühe sich wieder herstellen ließe. Mit
dem größten Teil ihrer Ausrüstung waren auch einige Reservebehälter
gerettet worden, die komprimierten Wasserstoff enthielten.
Allerdings konnte derselbe zu einer vollständigen Füllung des
Ballons nicht ausreichen. Doch hoffte Grunthe, von den Martiern die
Mittel zur genügenden Entwicklung des Gases zu erhalten. Er hatte
bei seinen Studien auf der Insel gesehen, daß die Martier über so
gewaltige Mengen elektrischen Stromes verfügten, daß er dadurch den
Wasserstoff leicht aus dem Wasser des Meeres erhalten konnte.
Sollte ihm aber hierzu die Beihilfe verweigert werden, so war er
entschlossen, den Ballon entsprechend zu verkleinern und mit dem
Reservevorrat an Gas und nur dem notwendigsten Gepäck die Heimreise
anzutreten. Er hatte in der Bibliothek der Martier die
Witterungsbeobachtungen gefunden, welche Jahre hindurch von ihnen
am Nordpol ausgeführt waren. Daraus hatte er entnommen, daß während
des Novembers regelmäßig andauernd nach Europa hinwehende Winde
einzutreten pflegten, daß er aber früher keine Aussicht hatte,
günstige Windverhältnisse zu erwarten. Demnach mußte er sich
entscheiden, ob er sich jetzt, kurz vor Beginn der Polarnacht,
unbestimmten atmosphärischen Verhältnissen anvertrauen wollte, oder
ob er mitten in der Polarnacht es wagen wollte, bei günstigem Wind
aufzusteigen. Das letztere schien ihm das Empfehlenswertere zu
sein, da er bei gutem Wind hoffen durfte, in wenigen Tagen bewohnte
Gegenden zu erreichen.

Diese Überlegungen, welche Grunthe für sich angestellt hatte, waren
von ihm zwar Saltner gegenüber beiläufig erwähnt worden, doch hatte
sie dieser, eben weil sie die Zeit zur Ausführung noch nicht für
gekommen hielten, zunächst nicht weiter erwogen. Ihm war vorläufig
die Gegenwart alles, und jetzt erst stellten ihn Las Worte
unmittelbar vor die Frage, was zu tun sei, wenn die Martier fast
sämtlich die Insel verließen. Zugleich aber schien ihm im
Augenblick eine so schnelle Trennung von seinen innig verehrten
Gastfreunden und von La und Se insbesondere als etwas kaum
Mögliches. Indem ihm Grunthes Pläne momentan durch den Kopf
schossen, fühlte er doch, daß er nicht sofort eine Zusage geben
dürfe, und in seiner Verwirrung zögerte er mit der Antwort, während
die Martier mit allerlei verlockenden Schilderungen Las Einladung
unterstützten.

Zum Glück trat Grunthe jetzt ein, und die Zeremonie der Begrüßung
mit den Martiern wiederholte sich. Nur La, an welcher Grunthe nach
Möglichkeit vorbeisah, mußte sich mit einem steif ausfallenden
martischen Fingergruß begnügen. Sie lächelte zu Saltner hinüber,
und ihr Blick schien sagen zu wollen: „Wir werden ihn doch
mitnehmen.“

Grunthe hatte bereits auf dem Weg von Hil gehört, daß morgen ein
Fahrzeug nach dem Mars abgehe.

„Wie viele Nume verlassen uns denn?“ fragte er.

„Dreiundfünfzig, darunter fünf Damen“, antwortete Jo.

„Dann ist es wohl ein bedeutendes Fahrzeug? Wenn ich recht gehört
habe, sind selbst Ihre größten Raumschiffe nicht auf viel mehr
berechnet.“

„Das ist richtig. Auf mehr wie sechzig können wir unsere Schiffe
nicht gut einrichten, das Verhältnis zu den Richt-Bomben wird sonst
zu ungünstig. Aber der ›Komet‹ ist ein vorzügliches Fahrzeug und
trägt gut seine sechzig Personen – Sie haben also noch bequem
Platz, und ich würde mich sehr freuen, Sie mitzunehmen.“

„Sie sind selbst der Kommandant?“ fragte Grunthe.

„Ich habe die Ehre, das Raumschiff ›Komet‹ zu führen, bestimmt nach
der Südstation des Mars. Sie fahren darin sicherer durch den
Weltraum als in Ihrem Ballon durch die Luft der Erde. Also
abgemacht, kommen Sie mit?“

„Daran ist nicht zu denken“, sagte Grunthe lächelnd. „Aber es würde
mich sehr interessieren, der Abfahrt beizuwohnen. Wann findet sie
statt?“

„64,63“, erwiderte Jo.

Grunthe sah ihn fragend an.

„Mittlere Marslänge“, fügte Jo hinzu.

„Sie müssen schon“, begann La, „den Herren alle Maßangaben in ihre
irdische Rechnungsweise umrechnen. In unsre Messungsmethode können
sie sich nicht so schnell hineinfinden. Morgen um 1,6 ist die
Abfahrt, das heißt nach Ihrer Stundenrechnung um drei Uhr. Sehen
Sie sich nur die Sache einmal an, Grunthe, Sie werden Lust
bekommen, bald selbst eine Fahrt mitzumachen. In der nächsten Zeit
geht jeden dritten Tag ein Schiff ab!“

„Der Mars“, fiel Jo ein, „ging sechs Tage vor Ihrer Ankunft durch
sein Perihel – ich meine den Punkt, wo er der Sonne am nächsten
steht –, und da er sich gerade jetzt auch in der Erdnähe befindet –
Sie wissen, daß die Opposition vor wenigen Tagen stattfand –, so
gibt es keine günstigere Reisezeit. Aber ›piken‹ Sie denn nicht?“

„Ich danke, niemals“, sagte Grunthe, die angebotenen Piks
zurückweisend. Dabei starrte er geradeaus und zog seine Lippen
zusammen. Er rechnete in der Eile die augenblickliche Entfernung
von Mars und Erde aus.

„Wie lange Zeit pflegen Sie denn zur Fahrt zu brauchen?“ fragte
Saltner.

„Das kommt ganz auf die Umstände an. Bei günstiger Stellung der
Planeten läßt sich die Reise auf dreißig Ihrer Tage und weniger
reduzieren, ja wenn wir tüchtige Bombenhilfe geben, was freilich
sehr teuer wird, so könnte man bei so großer Planetennähe wie jetzt
sogar auf acht oder neun Tage herabkommen. Aber ich muß freilich
bemerken, daß man eine solche Geschwindigkeit von 90 bis 100
Kilometern in der Sekunde nur unter ganz besonderen Umständen
benutzen würde.“

„Ich begreife überhaupt noch nicht“, sagte Grunthe, sich wieder am
Gespräch beteiligend, „wie sie Ihre Geschwindigkeit und Richtung in
verhältnismäßig so kurzer Zeit verändern können. Ich weiß, daß Sie
Ihr Fahrzeug mehr oder weniger diabarisch machen, daß Sie also die
Anziehung der Sonne schwächer oder auch gar nicht auf dasselbe
einwirken lassen können. Bei der Abfahrt heben Sie die Gravitation
ganz auf, um zunächst genügend weit aus dem Anziehungsbereich der
Erde zu kommen, nicht wahr?“

„Ganz richtig. Aber sprechen Sie, bitte, weiter, damit ich sehe,
wie weit Sie mit den Prinzipien unserer Raumreisen vertraut sind.“

„Wenn Sie abreisen, verlassen Sie also die Erde und die Erdbahn in
der Richtung ihrer Tangente mit einer Geschwindigkeit von etwa 30
Kilometern in der Sekunde, denn das ist die Geschwindigkeit der
Erde in ihrer Bahn, die Sie nach dem Beharrungsgesetz beibehalten.
Sie kommen dadurch in immer größere Entfernung von der Sonne. Wenn
Sie nun die Gravitation wieder wirken lassen, vielleicht nur
schwach, so wird das denselben Erfolg haben, als wenn Sie sich mit
der Geschwindigkeit der Erde in sehr großer Entfernung von der
Sonne, zum Beispiel in der Entfernung des Uranus befänden, und die
Bahn müßte dann eine hyperbolische werden, Sie würden sich auf
einer Hyperbel von der Sonne entfernen.“

Jo machte ein Zeichen der Zustimmung.

„Nun kann ich mir wohl denken“, fuhr Grunthe fort, „daß Sie durch
geschickte Kombinierung solcher Bahnen, indem Sie die Gravitation
schwächen oder verstärken, in das Anziehungsgebiet des Mars
gelangen können. Aber ich verstehe nicht, wie dies in so kurzer
Zeit möglich ist. Sie müssen jedenfalls einen sehr weiten Weg
durchlaufen, und wenn Sie sich von der Sonne entfernen wollen, so
wird doch unter dem Einfluß der Gravitation Ihre Geschwindigkeit
immer kleiner, niemals aber größer.“

„Sie haben darin vollkommen recht“, erwiderte Jo. „Dies war der
einzige Weg, der unsern Raumschiffern in der ersten Zeit unserer
Weltraumfahrten zu Gebote stand. Sie hatten damals nur das Mittel
der Gravitationsänderung, infolgedessen waren die Fahrten sehr
zeitraubend, mühsam und gefährlich. Man konnte unter Umständen
Jahre brauchen, um von der Erde bis in die Nähe des Mars
zurückzugelangen, und ein kleiner Fehler in der Berechnung oder
eine unvorhergesehene Störung konnte weitere Jahre kosten. Ja, wir
haben damals noch manches Schiff verloren, von dem man nie wieder
etwas gehört hat.“

„Und wieso ist das jetzt besser geworden?“ fragte Grunthe.

„Sie scheinen noch nichts von der Speschen Erfindung der
Richtschüsse zu wissen“, bemerkte Jo.

„Was ist das?“

„Das ist alles zugleich, was bei Ihren Schiffen Schraube, Steuer
und Anker sind. Wir können dadurch unsere Geschwindigkeit
vergrößern, verringern, vernichten und umkehren sowie in jede
beliebige Richtung lenken. Da es sich dabei aber um kolossalen
Energieaufwand handelt, wie Sie sich denken können – wir haben es
ja mit Geschwindigkeiten von durchschnittlich 30 Kilometer zu tun,
deren Quadrate hier in Ansatz kommen –, so benutzen wir sie nur mit
Maß. Die Gravitation arbeitet billiger.“

Grunthe schwieg. Es war ihm unheimlich, sich dieser Macht gegenüber
zu fühlen, welche selbst die Herrschaft der Sonne im Weltraum zu
bändigen wußte.

„Wie in aller Welt ist das möglich?“ fragte Saltner. „Sie haben ja
im Raum keinerlei Widerstand, wie unsre Schiffe im Wasser. Können
wir doch nicht einmal unsern Luftballon ohne Schleppseile lenken.“

„Es fehlen Ihnen nur die nötigen Energiequellen und allerdings auch
der nötige Platz zum Losschießen, wie wir ihn im Weltraum zur
Verfügung haben. Sehen Sie, ein solcher Schuß, man nennt ihn einen
›Spe‹, entwickelt eine Energiemenge von ungefähr 500 Billionen
Meterkilogramm, wenn ich richtig umgerechnet habe –“

„Es trifft ziemlich zu“, sagte La, da Jo sie fragend ansah.

„Dadurch können wir also“, fuhr Jo fort, „einem Raumschiff, das
eine Masse von etwa einer Million Kilogramm besitzt, eine
Geschwindigkeit von einem Kilometer in der Sekunde erteilen – wenn
wir somit dreißig Spes anwenden, so ist es möglich, die
Geschwindigkeit, die unser Fahrzeug von der Erde mitnimmt, auf Null
herunterzubringen. So ein Schuß wird ganz allmählich entladen,
sonst könnte ja niemand den Ruck aushalten – immerhin bringen wir
das Schiff binnen drei Stunden zum Stehen. Sie sehen also, daß wir
auf diese Weise an jeder beliebigen Stelle des Weltraums einfach
haltmachen können. Wir heben die Anziehung der Sonne auf und heben
die planetarische Tangentialgeschwindigkeit auf, und damit stehen
wir still, unverändert in unsrer Lage zu allen Körpern unsres
Sonnensystems. Hier können wir warten, so lange wir Lust haben; wir
stellen uns zum Beispiel auf die Marsbahn und lassen den Planeten
einfach herankommen. Aber das würde immer noch viel zu lange
dauern. Wenn wir noch etwas mehr Bomben in passender Richtung
anwenden, so können wir uns sofort direkt auf den Planeten oder
vielmehr auf den Punkt seiner Bahn hinbewegen, an welchem wir ihn
am schnellsten antreffen. Natürlich nehmen wir dabei, so gut es
sich machen läßt, die Gravitation mit in Anspruch,
selbstverständlich immer, wenn wir uns der Sonne zu nähern haben,
also wenn wir vom Mars hierherfahren.“

Grunthe verharrte noch immer in seinem Schweigen. Er rechnete jetzt
aus, welche Geschwindigkeit wohl das Geschoß bekommen müsse, wenn
durch den Rückschlag beim Abfeuern das ganze Raumschiff mit einer
Geschwindigkeit von einem Kilometer pro Sekunde zurückgeschleudert
werden solle. Schon begann das Gespräch der Martier sich anderen
Gegenständen zuzuwenden, als er sagte:

„Ich kann natürlich in Ihre Worte keinen Zweifel setzen. Aber wenn
Sie der Masse des Schiffs von einer Million Kilogramm eine
Geschwindigkeit von 1.000 Metern erteilen, so würde dies ja
voraussetzen, daß das Geschoß selbst eine so ungeheure
Geschwindigkeit erhielte, wie sie auf keine Weise sich erzeugen
läßt.“

„Warum nicht?“ fragte Jo.

„Und wenn auch“, unterbrach Saltner, „was nützt Ihnen denn das
Abschießen? Dadurch kann doch Ihr Schiff nicht bewegt werden?“

„Das schon“, berichtigte ihn Grunthe, „nur der Schwerpunkt des
ganzen Systems kann nicht verrückt werden. Der Schwerpunkt von
Geschoß und Schiff behält seine Geschwindigkeit, aber dort befindet
sich ja niemand, das Raumschiff entfernt sich von diesem
Schwerpunkt infolge des Rückschlags, wie wir hören, um einen
Kilometer in der Sekunde, das heißt, es bewegt sich dann nur noch
mit einer Geschwindigkeit von 29 Kilometern vorwärts. Gleichzeitig
aber muß das Geschoß nach der entgegengesetzten Seite mit einer
solchen Geschwindigkeit fliegen, daß das Produkt aus dieser und der
Masse des Geschosses gleich ist dem Produkt aus Masse und
Geschwindigkeit des Schiffs (in bezug auf den Schwerpunkt), das
gibt in unserm Fall die Zahl von tausend Millionen. Es fragt sich
nun, welche Masse Ihre Geschosse besitzen –“

„Hundert Kilogramm“, sagte Jo.

„Dann würde ja“, sagte Grunthe kopfschüttelnd, „das Geschoß eine
Geschwindigkeit von zehn Millionen Meter, das sind zehntausend
Kilometer in der Sekunde bekommen – das ist mir undenkbar!“

„Und dennoch ist es so“, versicherte Jo. „Ja, es ist dies noch gar
nicht die Grenze des Erreichbaren. Wir haben berechnet, daß sich
die Geschwindigkeit bis über die Lichtgeschwindigkeit hinaus muß
steigern lassen –“

„Sie wollen mich zum besten haben –“

„Nicht im geringsten.“

„Durch die Entwicklung von Explosionsgasen?“

„Wer behauptet das? Das ist natürlich nicht möglich. Aber durch die
Explosion des Weltäthers selbst.“

Grunthe schüttelte nur den Kopf.

„Ich las in Ihren Büchern“, fuhr Jo fort, „daß Sie Ihre Geschosse
durch die Entwicklung der Pulvergase mit Geschwindigkeiten
schleudern, welche größer sind als die Geschwindigkeit, mit der
sich der Schall in der Luft fortpflanzt. Nun – der Vergleich trifft
zwar nicht vollständig zu, aber in der Hauptsache – warum sollen
wir nicht durch Entwicklung großer Äthervolumina Geschwindigkeiten
erzeugen, die größer sind als diejenige, mit welcher sich das Licht
im Äther fortpflanzt? Es kommt nur darauf an, Apparate zu haben,
die das leisten.“

„Und diese haben Sie?“

„Allerdings. Wir können Ätherspannungen erzeugen, die wir plötzlich
entlasten. Der kondensierte Äther heißt ›Repulsit‹. Unsere
Geschütze und Geschosse bestehen aus – ja, wie soll ich Ihnen das
übersetzen? Übrigens kommt die Sache im Grunde darauf hinaus, große
Elektrizitätsmengen unter kolossalen Spannungen zu halten – und die
Entdeckung hängt wieder mit derjenigen der Diabarie zusammen.“

„Das ist uns freilich jetzt nicht möglich, so schnell zu fassen“,
sagte Grunthe. „Und Sie wollen die Geschwindigkeiten noch
steigern?“

„Wir hoffen bis auf fünf mal hunderttausend Kilometer zu kommen.
Wir überholen dann das Licht. Und wer auf einem solchen Geschoß in
den Weltraum reiste, der würde zurückblickend die Zeiten der
Vergangenheit auftauchen sehen, denn er käme zu jenen Lichtwellen,
die vor seiner Abreise den Planeten verlassen haben.“

„Ich danke Ihnen“, sagte Grunthe verstummend.

„Übrigens“, setzte Jo noch hinzu, „ist es für die Richtschüsse
natürlich kein Vorteil, so große Geschwindigkeiten zu wählen, denn
der Energieverbrauch wächst ja mit der Geschwindigkeit im Quadrat.
Wir würden viel besser fortkommen, wenn wir kleinere
Geschwindigkeiten anwendeten, aber dann würden die Massen der
Geschosse so groß werden müssen, daß wir sie nicht mitnehmen
können. Tausend Richtgeschosse zu je hundert Kilo Masse machen
ohnehin schon zehn Prozent unsrer gesamten Schiffsmasse aus.“

Es traten jetzt neue Gäste ein, um sich ebenfalls die Menschen noch
einmal anzusehen, ehe sie nach dem Mars abreisten. Denn sie wollten
doch bei der Heimkehr auch etwas von den Eingeborenen der Erde zu
erzählen haben. Ein Teil der Anwesenden erhob sich und
verabschiedete sich. Auch Jo stand auf.

„Nun“, sagte er, „schade, daß Sie nicht mit mir kommen wollen, doch
wir sehen uns morgen vor der Abreise.“

„Und auf dem Nu treffen wir uns alle bald wieder“, fügte La hinzu.
„Wer weiß“, sprach sie neckend zu Jo, „ob wir Sie im ›Meteor‹ nicht
noch überholen und eher zu Hause sind als Sie. Oß wird
wahrscheinlich den ›Meteor‹ führen.“

„Da kennen Sie den alten Jo schlecht“, erwiderte Jo lachend. „Man
fährt nicht fünfundzwanzig Jahre zwischen Mars und Erde, um sich
von solch jungem Springinsfeld überholen zu lassen.“

„Sie sind eben ein zu guter Lehrer für Oß gewesen, da ist’s kein
Wunder, daß er jetzt auch seine Sache versteht.“

„Das tut er, gewiß, das tut er“, sagte Jo, indem er La
freundschaftlich das Haar streichelte. „Aber was will das jetzt
sagen – das heißt, Oß ist ein tüchtiger Techniker, brillanter
Abariker, weiß es – doch um die Überfahrt zu machen, dazu gehört
heute nicht mehr viel, das kann man lernen. Ja, liebe La, vor –
nun, Sie lebten wohl noch nicht, als ich meine erste Fahrt als
Lehrling machte, da war’s etwas anderes; da gab’s noch keine
Außenstation auf der Erde, von der aus man den Mars jederzeit sehen
und nach ihm telegraphieren konnte. Und wenn so ein Schiff zehn
oder zwanzig Richtschüsse zum Anlegen mithatte, da galt es schon
als besonders fein ausgerüstet. Da haben wir Dinge erlebt, wovon
Ihr junges Volk keine Ahnung habt.“

„Erzählen Sie“, bat La, „bleiben Sie noch, Jo, Sie müssen uns etwas
erzählen. Sie haben es eigentlich längst versprochen. Setzen Sie
sich, die Bate müssen es auch hören.“

\section{13 - Das Abenteuer am Südpol}

Grunthe und Saltner hatten sich inzwischen mit den übrigen Martiern
unterhalten. Diesmal waren sie recht gründlich nach allerlei
Einrichtungen der Menschen ausgefragt worden. Grunthe beschrieb
ihnen auf der Karte die Wohnplätze der verschiedenen Rassen und die
Abgrenzungen der bedeutendsten Staaten. Sie waren sehr erstaunt zu
hören, daß es große Gebiete der Erde gäbe, die man noch gar nicht
oder sehr wenig kenne, und daß ihre Einwohner keinerlei Einfluß auf
die Geschicke der ganzen Menschheit ausübten. Bei den Martiern
bestehe zwar auch ein sehr großer Unterschied zwischen der Bildung
der einzelnen Bewohner und Volksstämme, aber gänzlich
unzivilisierte Landschaften gäbe es überhaupt nicht. Grunthe fragte
nach der Anzahl der Marsbewohner und erfuhr zu seiner Überraschung,
daß sie nicht weniger als dreitausendeinhundert Millionen betrüge,
also das doppelte der Zahl der Menschen, auf einer viermal so
kleinen Oberfläche zusammengedrängt wie die der Erde.

„Da können wir Ihnen einen Teil von uns überlassen“, sagte einer
der Martier scherzend.

„Es würde Ihnen auf der Erde zu schwer werden“, erwiderte Saltner,
dem der Gedanke eines Einfalls der Martier auf die Erde recht
bedenklich erschien. „Lieber kommen wir ein wenig zu Ihnen.“

„Aber erst lernen Sie ordentlich balancieren“, ertönte eine Stimme
aus der Luft. „Ich werde gleich einmal nachsehen.“

Es war Ses Stimme. Sie hatte die Klappe des Fernsprechers geöffnet
und gerade Saltners Worte verstanden.

Gleich darauf erschien sie an der Tür. Um seine Geschicklichkeit zu
erweisen, überschritt Saltner den ›Strich‹ und ging ihr vorsichtig
entgegen. Sie lachte herzlich und rief, ihm die Hand
entgegenstreckend: „Es geht schon ganz gut, Sie haben Fortschritte
gemacht.“

Saltner ergriff die Hand und bückte sich, um sie an seine Lippen zu
führen. Diese Verbeugung ging auch ganz gut vonstatten, aber als er
sich aufrichten wollte, geschah es zu plötzlich, und er lief
Gefahr, nach hinten zu stürzen. Da er sich über sich selbst lustig
machte, so zeigten auch die Martier ihre Heiterkeit über seine
vorsichtigen Bewegungen und baten ihn dann, ihnen doch einige
seiner Kraftproben zu zeigen, von denen sie gehört hatten.

Eben hatte er zwei der Martier mit Leichtigkeit in die Luft
gehoben, als sich La nach ihm umdrehte.

„Was wollen Sie über dem ›Strich‹?“ sagte sie scherzhaft drohend.

Saltner sprang schleunigst einen Schritt zurück, hatte aber die
beiden Herren vom Mars noch nicht niedergesetzt, und in dem
Augenblick, als er den ›Strich‹ passierte, wurden sie ihm zu
schwer, so daß sie ziemlich unsanft zur Erde kamen.

Während er sich entschuldigte, rief La: „Alle an den Tisch! Jo
erzählt von seiner ersten Erdfahrt, bitte, bitte!“

Dem allgemeinen Drängen konnte Jo nicht widerstehen. Auch auf dem
Mars spinnt ein alter Seemann gern ein Garn. Er setzte sich oben an
den Tisch. Se und La saßen dicht am ›Strich‹ neben den beiden
Deutschen.

Jo nahm bedächtig ein Pik, legte es an die Stirn, an das rechte und
an das linke Auge, und sah sich dann noch einmal im Zimmer um.

Se verstand ihn.

„Unter dem Tischrand“, sagte sie. „Greifen die Herren nur zu.“

Schmunzelnd zog Jo ein Mundstück hervor und probierte das Getränk.

„Ein feiner Tropfen“, sagte er.

Ein Teil der Martier und auch Saltner folgten seinem Beispiel. La
lehnte sich bequem zurück, Se nahm ihre chemische Handarbeit auf,
und Grunthe zog sein Notizbuch hervor, um sich einige
stenographische Aufzeichnungen zu machen.

„War damals siebzehn Jahr alt“, begann Jo seine Erzählung.

„Marsjahre“, sagte La leise zur Erklärung.

„– hatte eben meinen technischen Kursus absolviert, als ich mich
beim Kapitän All meldete, der mit der ›Ba‹, vierundzwanzig
Personen, nach der Erde abgehen sollte. Wollte mich eigentlich
nicht mitnehmen, weil ich noch zu jung sei, aber da im letzten
Augenblick einer von der Mannschaft verhindert wurde und kein
andrer sich gemeldet hatte, so kam ich mit. Fünf Monate waren wir
unterwegs und hatten glücklich so manövriert, daß wir der Erde
parallel flogen, genau in der Achse über dem Südpol. Sie hatten
Sommer dort unten, aber um den Pol herum war alles von dichten
Wolken bedeckt. Wir sahen auf der Erde nur ihre weiße, von der
Sonne beglänzte Wolkenoberfläche, und wo sie im Schatten
verschwand, spielten die Südlichter in rötlichen Streifen. Wir
ließen uns sinken und machten uns, als wir tief genug gekommen
waren, so leicht, daß wir als Luftballon in der Atmosphäre
schwammen. Dann ging es durch die Wolken hinab, und wir kamen auch
glücklich, leider aber mit einer Abweichung von ein paar
Kilometern, auf den Pol. Nun, Sie wissen, auf dem Südpol ist’s
nicht so schön wie hier, ’s ist ringsum Festland-Eis, eine
Hochfläche von ein paar tausend Metern, wie Sie’s hier nebenan
haben – in – wie heißt das Ding?“

„Grönland.“

„Gut. Nun mußten wir aber das Schiff nach dem Pol schaffen, denn
wir hatten das schwere Schwungrad für die Station, die wir
vorbereiten sollten, auszuladen. Deshalb war All sehr ungehalten,
daß er von der Erdachse abgekommen war. Aber dieselbe Ursache, die
uns abgetrieben hatte, verhinderte uns, auch jetzt ans Ziel zu
gelangen. Das war der herrschende Wind. Ich sagte schon, daß wir
uns in der Atmosphäre nicht anders wie einer ihrer Luftballons
verhalten können. Wir können uns leichter machen als die Luft, aber
ihren Strömungen unterliegen wir dabei ebenso wie ihrem
Widerstand.“

„Verzeihen Sie“, begann Grunthe, „ich habe mich schon immer
gewundert, gerade weil sich Ihr Raumschiff in der Atmosphäre wie
ein Luftballon handhaben läßt, und zwar mit dem wunderbaren
Vorteil, weder Ballast noch Gas opfern zu müssen, da Sie sich nach
Belieben leicht oder schwer machen können, ich habe mich gewundert,
daß Sie nicht, nachdem Sie einmal am Pol die Erdgeschwindigkeit
gewonnen haben, einfach mit Ihren Raumschiffen nach Europa oder den
Vereinigten Staaten von Nordamerika gekommen sind – kurzum, warum
Sie so ängstlich in der Befahrung unsres Luftmeers sind.“

„Und ich“, erwiderte Jo, „habe mich allerdings auch gewundert, wie
Sie sich diesen gebrechlichen Dingern in einer Atmosphäre
anvertrauen können, die so dicht und schwer ist wie die Ihrige, und
in welcher nach allen Richtungen die tollsten Stürme einherrasen.“

„Ich habe“, bemerkte La, „in einem der Bücher gelesen, die Sie
mitgebracht haben, von den Entdeckungsreisen der Menschen auf der
Erde. Da spricht ein Seefahrer seine Verwunderung darüber aus, daß
die Eingeborenen in irgendeiner Inselgruppe in ihren gebrechlichen
Kähnen weite Fahrten unternehmen, an die er sich in seinem großen
Dampfschiff nicht wagen würde, weil er die Gefahren der Tiefe nicht
zu vermeiden weiß. Ähnlich mag es sich wohl mit unsern Raumschiffen
und Ihren Luftballons verhalten. Bedenken Sie, daß wir Ihre
Atmosphäre noch sehr wenig kennen –“

„Und vor allen Dingen“, fuhr Jo fort, „daß unsre Raumschiffe, die
aus Stellit bestehen, nicht darauf eingerichtet sind, den großen
Druck Ihrer Luft und den Widerstand, wenn wir nicht mit dem Wind
fliegen, zu ertragen. Das Stellit ist sehr fest in der Kälte des
Weltraums, aber in der Wärme und Feuchtigkeit der Luft wird es
schnell angegriffen. Außerdem sind wir luftdicht durch unsre Kugel
von außen abgeschlossen und können uns darum außerhalb derselben an
nichts wagen. Die Technik unserer Luftschiffahrt auf dem Mars läßt
sich auf der Erde aus verschiedenen Gründen nicht anwenden. Sie
dürfen sich also nicht wundern, daß es uns bis jetzt noch nicht
eingefallen ist, unsre Raumschiffe an unbekannte Gefahren zu wagen,
durch die uns möglicherweise die Rückkehr abgeschnitten worden
wäre. Doch sind bereits Versuche geglückt, diabarische Fahrzeuge
mit Öffnungen herzustellen, und das, was uns noch fehlt, ist
eigentlich nur ein genügend widerstandsfähiger Stoff für dieselben.
Aber auch hier steht die Abhilfe bevor, und dann fahren wir zu
Ihnen.“

„Wenn Sie zu uns kommen“, sagte La lächelnd zu Grunthe, „werde ich
Ihnen mit Se eine Privatvorlesung über Raum- und Lufttechnik
halten.“

„Dann fürchte ich leider, darauf verzichten zu müssen, denn ich
gedenke vorläufig hierzubleiben.“

„So werde ich Ihnen einen ausführlichen, schönen, gelehrten Brief
schreiben, verlassen Sie sich darauf!“

Grunthe verbeugte sich mit zusammengepreßten Lippen, und Jo fuhr
fort:

„Nun kurzum, wir hatten keine Wahl, wir mußten jetzt mit dem
Raumschiff nach dem Pol. Da nun aber das Wetter nicht besser wurde
– das heißt, der Himmel war klar, aber die Luft blies vom Pol her
–, so beschloß All, den Versuch zu wagen, uns nach dem Pol
hinzuwinden. Wir hatten große Mengen von mit Lis durchzogenen Tauen
mit. Dieses Tau legten wir vom Schiff bis zum Pol aus, verankerten
es dort gründlich und setzten mit der Winde an. Das Schiff wurde
nur soweit leicht gemacht, daß es sich gerade hob, ohne Gefahr, auf
dem Eis aufzulaufen. Denn es zu schleifen durften wir nicht wagen,
darauf ist unsere Stellitkugel nicht eingerichtet.

Die Arbeit ging natürlich langsam vorwärts, aber wir waren in
vierundzwanzig Stunden doch einen Kilometer vorgerückt. Leider
frischte der Wind immer stärker auf und wurde böig. Bei den Stößen
bog sich die Kugel bedenklich an der Haftstelle des Seiles, und All
hielt es für nötig, die ganze Kugel in ein Netz zu fassen. Es war
eine furchtbare Arbeit, in dieser Luft und Schwere die Seile über
die fünfzehn Meter hohe Kugel zu spannen, und daß keiner von uns
dabei verunglückt ist, bleibt mir heute noch ein Rätsel. Todmüde
ging es am dritten Tag wieder an die Winde. Eine Maschine hatten
wir leider nicht mit, wir mußten mit unsern eignen Kräften
arbeiten. Am fünften Tag waren wir bis auf einen Kilometer heran.
Wir arbeiteten immer vier Mann und wurden alle Stunden abgelöst.
Lieber machten wir den Weg hin und her zum Schiff, als daß wir uns
ohne Erholung dem Druck der Schwere länger ausgesetzt hätten. Zur
Rückfahrt benutzten wir übrigens einen Segelschlitten; das war
unsre größte Freude, so der Ruhe mit Bequemlichkeit
entgegenzufahren. Eben hatte ich mich mit meinen Kameraden
aufgesetzt, und in zwei Minuten waren wir bis auf die Hälfte des
Weges zum Schiff herangekommen, das nicht höher als etwa zehn Meter
über dem Eis schwebte. Die Strickleiter hing aus der Luke bis zum
Boden herab, und in weiteren zwei Minuten hofften wir in unsren
Hängematten zu liegen.

Plötzlich sehen wir von der Seite und halb nach vorn hin etwas
Gelblich-Weißes herantrotten, zwei große vierfüßige Tiere, wie wir
sie noch nie gesehen hatten. Es waren, was Sie Eisbären nennen,
aber damals wußten wir noch nicht, was das heißen will, wenn man
ihnen waffenlos begegnet. Waffen hatten wir überhaupt nicht mit,
nur die langen, mit Eisenspitzen versehenen Stangen, mit denen wir
unsern Schlitten dirigierten und ihm nachhalfen. Noch niemals war
uns auf dieser öden Erdfläche, außer einigen Vögeln, irgendein Tier
begegnet. Von Raubtieren, die dem Numen gefährlich sind, wußten wir
überhaupt nichts als aus den alten Überlieferungen der Vorzeit, da
es solche auf dem Mars noch gegeben haben soll. Aber als diese
Bestien, sobald sie uns erblickten, mit gierigen Augen auf unsern
Schlitten zutrabten, dachten wir uns doch, daß die Sache nicht
geheuer sei. Wir konnten freilich nichts tun, als mit unsern Picken
die Fahrt unsres Schlittens beschleunigen, wobei wir dem Wind das
Beste überlassen mußten. Ließ der Wind einen Augenblick nach, so
mußten uns die Bären den Weg abschneiden. Es war eine fatale
Situation, doch sahen wir dieselbe nicht als besonders bedenklich
an, da wir glaubten, ihnen mit unsern Stöcken gewachsen zu sein.
Wir waren jetzt nur noch hundert Meter von der Strickleiter
entfernt, und man war bereits vom Schiff aus auf uns aufmerksam
geworden. All selbst und zwei Mann, mehr hatten an der Luke nicht
Platz, standen mit Gewehren bereit, denn damit war die Expedition
für alle Fälle versehen. Sie wagten aber nicht zu schießen, weil
das Schiff an dem langen Tau stark hin- und herschwankte und die
Bären jetzt so dicht an dem Schlitten waren, daß wir selbst hätten
getroffen werden können; ein sicheres Zielen war ja nicht möglich.
Zudem hatten wir auch noch keine Erfahrung, wie Luftwiderstand und
Schwere auf der Erde unsere Geschosse ablenken. Das Telelyt war
damals noch nicht für Handwaffen im Gebrauch.

Ich stand vorn am Schlitten. Die Gefährten riefen mir zu, direkt
auf die Strickleiter zu halten und sie sofort zu erfassen. Wir
durften ja die Geschwindigkeit des Schlittens nicht mäßigen. Es
handelte sich noch um Sekunden. Da stößt der Schlitten an irgendein
kleines Hindernis und wird von seinem Weg abgelenkt. Ich fürchte,
daß ich die Strickleiter verfehle, und renne den Stock so stark in
das Eis, daß er mir aus der Hand gerissen wird. Wir sausen an der
Leiter vorbei. Da pfeift es über uns, und der eine Bär wälzt sich
in seinem Blut. Durch die Wendung des Schlittens hatte All zum
Schuß kommen können. Der andere aber ist unmittelbar am Schlitten.
Unglücklicherweise stechen die beiden zuletzt Stehenden mit ihren
Picken nach ihm. Der Bär ist verwundet, aber mit einem Tatzenschlag
hat er den armen Tam vom Schlitten gerissen. Er erfaßt ihn an
seinen Kleidern und trabt mit ihm davon.

Inzwischen war All mit einer Anzahl bewaffneter Leute die Leiter
herabgestiegen, und wir hatten den Schlitten zum Stehen gebracht.
Der Bär aber lief mit seiner Beute so schnell, daß All ihm nicht
folgen konnte; Sie wissen ja, daß wir schwer an uns zu tragen
haben, wenn wir uns auf der Erde bewegen sollen. Zu schießen wagte
All nicht um Tams willen; wenn auch dieser nicht selbst getroffen
wurde, so wäre er doch verloren gewesen, sobald der Bär nicht
sofort auf der Stelle tot war.

Unsre Bestürzung war groß. Wir suchten den Bären durch Schreien
einzuschüchtern, aber er kümmerte sich um nichts. Die Entfernung
zwischen ihm und uns vergrößerte sich schnell.

›Wir können ihn nicht stellen‹, rief All, ›doch folgen müssen wir
ihm. Ich gehe selbst, zwei Leute genügen zur Begleitung. Die andern
zurück aufs Schiff!‹

Jetzt sahen wir, daß der Bär die Richtung auf unsern Arbeitsplatz
am Pol einschlug. Unsre Gefährten an der Winde hatten ebenfalls den
Vorgang bemerkt. Sie stellten die Arbeit ein und beratschlagten
offenbar, ob sie sich dem Schlitten anvertrauen oder auf das Gerüst
flüchten sollten, das über der Winde erbaut war. Da der Bär sich
schnell näherte, so wählten sie das letztere. Auch sie suchten den
Bär durch Lärm zu verscheuchen, aber vergebens.

Als All erkannte, daß der Bär auf die Arbeiter an der Winde zulief,
hieß er jeden seiner Begleiter noch ein Gewehr mitnehmen, um sie
womöglich ihnen zuzustellen. All hatte noch nicht die Hälfte des
Weges zurückgelegt, als der Bär bereits bei der Winde ankam. Wir
waren inzwischen, mit Ausnahme Alls und seiner Begleitung, in das
Schiff zurückgekehrt und beobachteten von dort den Vorgang. Die
Leute auf dem Gerüst ärgerten offenbar den Bären. Er ließ Tam am
Fuß des Gerüstes liegen, setzte sich auf die Hinterbeine und schlug
seine Tatzen in die Winde ein, als wolle er sie umreißen. Kaum
hatte All bemerkt, daß Tam nicht mehr geschleppt wurde, als er auf
etwa fünfhundert Meter auf den Bären anlegte. Einen Augenblick
zögerte er noch, um eine günstigere Stellung abzuwarten. Da schien
es, als wolle der Bär von der Winde ablassen und sich wieder seiner
Beute zuwenden.

All drückte los.

Eine Sekunde später sahen wir den Bären zusammenstürzen. Mehr sahen
wir nicht. im Moment darauf erhielten wir einen Stoß, daß wir alle
übereinander fielen. Als wir uns aufrafften, fanden wir das
Raumschiff um wenigstens fünfzig Meter gehoben und vom Wind mit
großer Geschwindigkeit davongetrieben. Es war nicht anders denkbar,
als daß Alls Kugel das dünne Tau zerschnitten, der Druck des Windes
es vollends zerrissen hatte.

Der erste Steuermann übernahm das Kommando. Aber es war sehr
schwierig, etwas zu tun.

Die Anker heraus und tiefer!

Das Schiff streifte in drohender Nähe des Eises hin. Wenn die Anker
nicht bald faßten, so war keine Aussicht, die Gefährten
wiederzusehen.

Aber die Anker tanzten über die völlig glatte, hart gefrorene
Fläche des Eises hin, ohne zu fassen. Glücklicherweise leistete uns
das lange Seil ausgezeichnete Dienste, an welchem wir das Schiff
nach dem Pol hinbugsiert hatten. Es diente uns jetzt als
Schleppseil, indem wir es in einer Länge von fast tausend Meter
nachzogen. Von Minute zu Minute hofften wir über Spalten zu kommen,
in denen es sich vielleicht verfangen könne. Leider wurde der Wind
immer stärker und steigerte sich zum Sturm. Wir wußten aus der
Karte, daß es nicht mehr lange dauern konnte, bis wir zu der Stelle
gelangten, an der das Eisfeld in steilem Abfall nach dem Meer hin
abstürzt. Vorher freilich mußten große Bruchspalten kommen, und
darauf setzten wir unsre Hoffnung.

Fast eine Stunde mochten wir so dahingerast sein, schon sahen wir
in der Ferne das Meer auftauchen – da kamen auch die Spalten. Würde
das Tau sich verfangen? Die Anker nutzten uns nichts mehr, denn die
Oberfläche des Eises wurde jetzt so unregelmäßig, daß wir uns höher
erheben mußten, um nicht gegen einen Vorsprung geschleudert zu
werden, und die Ankerseile waren nur kurz. Da, endlich gibt es
einen Ruck, daß wir taumeln – doch die Fahrt geht wieder weiter –
aber jetzt, jetzt halten wir an, das Seil hat sich gespannt! Doch
was ist das? Ein furchtbarer Windstoß von oben drückt unser Schiff
nach dem Boden zu; da wir dem Sturm nicht mehr folgen, drängt er
uns hinab, das Schiff prallt gegen den Boden und erhebt sich aufs
neue – noch ein solcher Stoß, und wir sind verloren. Wir müssen
steigen, wir machen uns schwerelos und heben uns in die Höhe. Aber
war die Hebung zu stark oder hat die veränderte Richtung das Seil
aus der Spalte gelöst – kurzum, es gibt nach, wir schnellen in die
Höhe, das Seil hängt frei herab, und wir folgen wieder dem Sturm –
wir schweben über dem Absturz des Gletschers, vor uns das wütende,
mit Eisschollen erfüllte Meer. – Jetzt blieb nichts übrig, als nach
oben zu entfliehen, in höhere Schichten der Atmosphäre. Wir wußten
aus der Karte, daß wir eine breite Meeresbucht zu überfliegen
hatten, jenseits deren sich hohe feuerspeiende Berge erheben. Schon
sahen wir von unsrer Höhe ihre Rauchwolken am Horizont. Wir fliegen
immer direkt nach Norden auf einem Meridian, der in der Richtung
nach der großen Insel hinläuft, die Sie, wie ich aus Ihrer Karte
gesehen habe, Neuseeland nennen. An Landung konnten wir nicht mehr
denken, wir mußten hinauf. Aber dazu mußten wir noch eine schwere
Arbeit vollbringen, an die ich nicht gern denke. Das Netz um unser
Schiff mit dem langen Seil mußte fort. Denn was außerhalb unsrer
Kugel ist, können wir nicht diabarisch machen, es hätte unsre
Bewegung im Raum gehindert. Ich war der Jüngste, ich mußte in der
untern Luke hängend das Seil kappen; dann wurden von oben die
Verbindungen des Netzes gelöst, und ich hatte die Aufgabe, die
Seile nach unten zu ziehen. Dabei herrschte hier oben eine Kälte,
daß das Quecksilber gefror. Glücklicherweise behalten die Lisseile
ihre Geschmeidigkeit, sonst wäre die Arbeit unmöglich gewesen. Ich
wundere mich noch heute, daß ich nicht abgestürzt bin, denn ich
mußte in der Erdschwere arbeiten.

Endlich war auch das geschehen. Die Luken wurden geschlossen, und
wir ließen die Erde hinter uns.“

\section{14 - Zwischen Erde und Mars}

Jo tat einen Zug aus seinem Mundstück und fuhr dann in seiner
Erzählung fort.

„Was war nun zu tun? Nach kurzer Ruhepause versammelte uns der
erste Steuermann, Mitt hieß er, der später die berühmte Umschiffung
des Jupiter ausführte, zu einer Beratung. Sollten wir versuchen,
noch einmal die Erdachse zu gewinnen und nach dem Pol
zurückzukehren? Sollten wir die Unseren ihrem Schicksal überlassen
und die Heimreise nach dem Mars antreten? Wir hatten den vierten
Teil unserer Mannschaft und den Kapitän verloren. Es war natürlich,
daß wir zu ihnen zurückwollten. Aber es war auch nicht leicht. Eine
nochmalige Landung und eine zweite Abfahrt von der Erde verlangten
einen solchen Aufwand von Energie und vor allem von Richtschüssen,
daß die Gefahr vorlag, dadurch unsre Rückkehr nach dem Mars
überhaupt in Frage zu stellen. Trotzdem wurde beschlossen
umzukehren, nachdem Mitt eine Berechnung gemacht und gefunden
hatte, daß wir unter günstigen Umständen gerade auskommen könnten.
Wären wir nämlich nach dem Mars gegangen und wäre von dort sofort
ein neu ausgerüstetes Schiff nach der Erde geschickt worden, so
hätte doch erst im nächsten Frühjahr den Zurückgebliebenen Hilfe
gebracht werden können. Daß sie aber den Polarwinter auf der Erde
nicht überstehen konnten, war gewiß.

Alle diese Überlegungen, insbesondere die genauere Berechnung und
ihre wiederholte Prüfung, hatten längere Zeit in Anspruch
genommen.

Seitdem wir die Atmosphäre der Erde verlassen und in der Richtung
der Tangente der Erdbahn uns bewegten, mochten etwa sechs Stunden
vergangen sein. Obwohl wir in dieser Zeit einen Weg von über
600.000 Kilometern zurückgelegt hatten, waren wir doch von der Erde
selbst, die ja in gleicher Richtung auf ihrer Bahn hinlief, noch
kaum 1.500 Kilometer entfernt. Wenn wir uns jetzt volle Schwere
gaben, konnten wir sie in kurzer Zeit wieder erreichen, und es kam
darauf an, uns durch einen mäßigen Korrekturschuß eine solche
seitliche Geschwindigkeit zu erteilen, daß wir nach dem Pol
gelangten.

Die äußere Kugelhülle unseres Schiffes, in welcher sich die innere
Kugel fast ohne Reibung nach jeder Richtung drehen kann, hatte
natürlich durch die Abenteuer, die wir bei der Abfahrt und in der
Atmosphäre erlebten, eine starke Rotation erhalten. Wir hatten
bereits zu unserm großen Mißbehagen bemerkt, daß der Apparat nicht
richtig funktionierte, welcher die innere Kugel in ihrer
Gleichgewichtslage zu halten hatte, indem wir fortwährend
Schwankungen durch die äußere Kugel erlitten. Bis jetzt war jedoch
noch keine Zeit gewesen, dem Übelstand abzuhelfen. Nun aber kam es
darauf an, die Rotation der äußeren Kugel sowohl wie die
Schwankungen der inneren vollständig zu hemmen. Es war dies
einerseits wünschenswert, um eine genaue Aufnahme unserer Lage zu
machen, obwohl dieser Zweck allenfalls auch durch
Momentphotographie erreicht werden kann; andererseits war es
durchaus notwendig für die genaue Abgabe des Richtschusses, der
durch das Ventil an der Außenseite der äußeren Kugel gelöst wird.
Denn wenn dieser auch nur um geringe Differenzen fehlerhaft wird,
so können daraus Abirrungen vom Weg entstehen, die nur schwer
wieder zu korrigieren sind, für uns aber, die wir keine Kraft zu
verschwenden hatten, verhängnisvoll werden konnten.

Als wir nun das Schiff einer genauen Besichtigung unterwarfen,
stellte sich zu unserm nicht geringen Schrecken heraus, daß der
Winddruck während der Verankerung und das Aufschlagen des Schiffes
Formveränderungen der äußeren Kugel bewirkt hatten, die eine
umständliche Reparatur erforderten. Bevor diese nicht
fertiggestellt war, durften wir keine Schwere geben und überhaupt
kein Manöver ausführen. Und diese Reparatur nahm leider, das war zu
sehen, einige Tage in Anspruch. Während dieser Zeit mußten wir auf
unsrer gradlinigen Bahn verharren, die uns auf Strecken von der
Erde entfernte, welche dem Quadrat der Zeit proportional waren.

Aber es war auf dieser Reise, als wenn uns nichts gelingen sollte.
Ein neuer Mißstand trat auf.

Der Mond der Erde näherte sich der Stellung, in welcher die Erde
Vollmond hat. Unglücklicherweise entfernten wir uns also von der
Erde gerade in der Richtung auf den Mond zu. Dies wäre ja für uns
ziemlich gleichgültig gewesen, wenn wir in der Nähe der Erde,
wenigstens am ersten Tag unsrer Fahrt, unsere Umkehr hätten
bewerkstelligen können. Nach Ablauf des dritten Tages aber mußten
wir, sobald wir uns der Gravitation unterwarfen, in den
Anziehungsbereich des Mondes statt in denjenigen der Erde geraten.
Konnten wir also unsere Reparatur nicht vorher beendigen, so hatten
wir nur die Wahl, unsere Richtschüsse auf gut Glück bloß zur
Verringerung unsrer Geschwindigkeit zu verschwenden oder uns in so
weite Entfernung von der Erde hinaustragen zu lassen, daß sich
unsere Rückkehr auf lange verzögern mußte. Und wer weiß, ob wir
dann unsere Gefährten noch lebend angetroffen hätten?

Wir arbeiteten also in fieberhafter Eile an der Herstellung des
Schiffes, um möglichst bald einen sichern Richtschuß abgeben zu
können. Und wirklich, im Verlauf des dritten Tages war es gelungen,
die Kugeln zeigten keine merkliche Drehung mehr. Es war die höchste
Zeit; noch wenige Stunden, und wir hätten den Einfluß des Mondes
bekämpfen müssen. Jetzt konnten wir es noch wagen, uns
schwerzumachen und der Anziehung der Erde nur durch einen schwachen
Korrekturschuß nachzuhelfen.

Die Diabarität wurde aufgehoben. Mit höchster Spannung warteten wir
die nächste Beobachtung ab. War in der früheren Berechnung
irgendein kleiner Fehler vorgekommen, so konnte es sein, daß wir
nach dem Mond statt nach der Erde fielen. Noch stand er über uns,
mit seiner glänzenden Scheibe einen beträchtlichen Teil des Himmels
verdeckend, denn sein Durchmesser erschien 26mal so groß wie hier
von der Erde aus. Deutlich unterschieden wir jede Einzelheit an
seiner Oberfläche. Die riesigen Ringgebirge lagen wie zum Greifen
vor uns. Die langgestreckten Lavafelder, durch die tiefschwarzen
Schatten breiter Risse unterbrochen, glänzten blendend im
Sonnenlicht. Unter uns, bereits merklich kleiner als der Mond,
schwebte die Erde als matte Scheibe, vom Schimmer des Mondlichts
erleuchtet; nur eine schmale Sichel zeigte sich im Strahl der
Sonne. Wenn wir uns von der Sonne, die nahe neben der Erde stand,
abwendeten, glänzten überall am tiefschwarzen Firmament die Sterne
in leuchtender Pracht. Es war ein herrlicher Anblick, aber wir
achteten nicht darauf. Wir warteten nur, ob unsere Kugel beginnen
würde, sich zu drehen, das heißt, den Boden unter unsern Füßen dem
Mond zuzuwenden; dies wäre das Zeichen gewesen, daß wir dem Mond
und nicht mehr der Erde tributär waren. Noch näherten wir uns dem
Mond, da er noch immer ein wenig vor uns in unserer Richtung stand.
Noch überwog die Anziehung der Erde, doch war sie von der des
Mondes so geschwächt, daß wir kaum einen Zug nach dem Boden
bemerkten; wir mußten uns verhalten wie im schwerelosen Feld. Die
Sorge um unsere Gefährten ließ es uns jeden Augenblick erscheinen,
als begönnen die Gegenstände sich zu erheben, als wollte unsre
innere Kugel sich drehen. Aber noch immer schwebte der Mond über
uns.

Endlich hatte Mitt seine Beobachtung beendet. ›Wir kommen durch‹,
sagte er. ›Wir sinken.‹ Alle atmeten auf.

Noch eine Viertelstunde, und die Erdschwere machte sich wieder
geltend. Die Instrumente ließen deutlich erkennen, daß wir uns der
Erde wieder zu nähern begannen. Nun kam es darauf an, den passenden
Richtschuß zur Korrektur unsres Falls abzugeben. Wir hätten zwar
damit warten können, bis wir der Erde näher waren. Aber je eher wir
es taten, um so weniger Energie brauchten wir aufzuwenden. Denn
wenn erst unsre Fallgeschwindigkeit größer geworden war, so mußte
die Kraft auch um so stärker sein, welche unsre Richtung zu
verändern vermochte.

Mit größter Sorgfalt wurde die Bombe gewählt, die äußere Kugel in
die berechnete Stellung gebracht und die Entladung durch Verbindung
mit dem Chronometer im richtigen Moment bewirkt. Die Reaktion war
schwach, und wir schwankten nur wenig auf unsern Plätzen. In
wenigen Minuten war alles vollbracht, was wir vorläufig tun
konnten. Todmüde suchten wir unsere Lagerstätten auf, denn Ruhe
hatte es bis jetzt für uns nicht gegeben.

Ich hatte einige Stunden fest geschlafen, als ich durch ein
allgemeines Stimmengewirr aufgeweckt wurde. Ich eilte in den
Außenraum, und das erste, was mir in die Augen fiel, war der
veränderte Anblick des Mondes. Er war kleiner geworden, wir
entfernten uns also von ihm; das beruhigte mich. Aber seine
erleuchtete Fläche zeigte eine Abplattung, das heißt, wir sahen auf
ein Stück der nicht erleuchteten Mondkugel, das meiner Ansicht nach
größer war, als es hätte sein dürfen, wenn wir nach der Erde zu
fielen. Schnell begab ich mich nach der unteren Seite, und hier sah
ich, daß auch die Erde entschieden an Größe abgenommen hatte. Wir
entfernten uns also von beiden Himmelskörpern, und zwar, wie sich
sogleich herausstellte, in einer nahezu kreisförmigen Ellipse,
deren Ebene mit der der Erdbahn fast einen rechten Winkel bildete.

Wie dies geschehen konnte, ist bis heute unaufgeklärt geblieben.
Daß es nicht eher bemerkt wurde, daran trug der Mann schuld,
welcher die Wache hatte und aus Übermüdung eingeschlafen war. Sonst
hätte er sehr bald am Richtungszeiger den Fehler bemerken müssen,
und dann hätte noch ein Korrekturschuß angebracht werden können.
Jetzt aber war unsere Entfernung von der Erde bereits so groß
geworden, daß wir unsere Richtung fast hätten umkehren müssen, um
die Erde wieder zu erreichen. Das durften wir bei unserm geringen
Vorrat an starken Richtschüssen nicht tun.

Einige von Ihnen wissen vielleicht, daß Mitt nach unsrer Rückkehr
auf den Mars seines Fehlers wegen zur Verantwortung gezogen wurde.
Es konnte ihm aber kein Versehen nachgewiesen werden, und er wurde
freigesprochen. Die Rechnungen wurden sämtlich aufs genauste
geprüft, und es blieben nur zwei Erklärungen übrig. Es war möglich,
daß nach dem Verlassen der Erdatmosphäre wegen der mangelhaften
Beschaffenheit unsres Schiffes die erste Ortsbestimmung fehlerhaft
gewesen ist und dieser Fehler auf die Beurteilung unsrer Richtung
oder Geschwindigkeit nachgewirkt hat. Infolgedessen wäre der
Korrekturschuß unrichtig abgegeben worden. Es konnte aber auch die
Beobachtung als richtig vorausgesetzt und der Rechnung durch die
Hypothese genügt werden, daß wir, ohne es zu wissen, während des
Schlafs der Wache durch einen unbekannten kosmischen Körper
abgelenkt worden sind, den wir, obgleich er ziemlich groß gewesen
sein muß, nachträglich nicht bemerkten, weil er bereits in den
Erdschatten getreten war.

Nun, wie dem auch sein mochte, wir konnten nicht mehr zur Erde
zurück. Unsre Niedergeschlagenheit können Sie sich denken. Sie
wurde noch größer, als wir erkannten, wie es mit unsrer Rückkehr
zum Mars beschaffen sei.

Gingen wir in unsrer Bahn weiter, so kamen wir nach einem halben
Erdenjahr wieder der Erde so nahe, daß wir sie hätten erreichen
können. Aber dann hatte der Südpol Winter, und wir wären dort
verloren gewesen. Der gewöhnliche Weg nach dem Mars war uns zum
Unglück durch einen großen Kometen versperrt, dessen
Anziehungsbereich wir berücksichtigen mußten. Ein zweiter Weg – Sie
müssen bedenken, daß wir unsre Richtung und Geschwindigkeit nicht
so oft und beliebig ändern konnten wie heutzutage –, ein zweiter
Weg hätte uns bis in die Nähe der Asteroidenbahnen geführt, und das
ist so, als wenn Sie auf dem Meer zwischen unbekannten Klippen
segeln wollten. Denn wenn wir auch damals schon gegen 2.000 dieser
kleinen Planeten kannten, so gibt es doch noch unzählige, die so
klein sind, daß wir sie noch nie gesehen haben, kleiner als unsre
Kugel, aber genügend, um uns in Grund und Boden zu bohren, wenn wir
auf einen treffen. Außerdem hätte auch dieser Weg so lange Zeit in
Anspruch genommen, daß es fraglich wurde, ob unser Proviant dazu
ausreichte. Alle übrigen Wege waren noch weiter und mußten deshalb
verworfen werden. Der Mars stand, wie ich bemerken will, hinter der
Sonne, denn seit unsrer Abreise von ihm war ein halbes Erdenjahr
vergangen.

Mitt hatte uns das Resultat seiner Berechnungen mitgeteilt und sich
dann zu neuen Prüfungen in seine Kajüte zurückgezogen. Wir saßen in
uns gekehrt da, jeder machte sich mit dem Gedanken vertraut, unsren
lieben Nu nicht wieder zu betreten. Einer der Gefährten äußerte
sich endlich dahin, man solle die jetzige Bahn einhalten, nach
einem halben Jahr die Erde zu treffen suchen, diese aber am Nordpol
anlaufen. Da alsdann dort Sommer wäre so würden wir wahrscheinlich
eins unsrer Schiffe antreffen, von dem wir genügende Vorräte
bekommen könnten, um im nächsten Südsommer nach dem Südpol
zurückzukehren. Die Hoffnung freilich, unsre Gefährten noch zu
retten, mußten wir wohl aufgeben, immerhin aber konnten wir auf
diese Weise unsre Rückkehr nach dem Mars sichern, selbst für den
Fall, daß wir kein Schiff daselbst antrafen. Wir konnten ja dann
die günstigste Stellung zur Reise abwarten und fanden auf alle
Fälle einige Vorräte in den Depots. Dieser Plan fand allseitigen
Beifall, und wir schickten uns eben an, den Kapitän zu rufen, um
ihm unsre Vorschläge zu machen, als dieser mit glänzenden Augen
unter uns trat und rief: ›Freunde, wollen wir in sechzig Tagen auf
dem Mars sein?‹

Wir sprangen auf und umringten ihn. Alle wollten wir näheres hören.
Nun –“

Jo unterbrach sich und warf einen Blick auf die Uhr.

„Pik und Spe!“ rief er. „ist das schon spät geworden! Nun, ich will
schnell ein Ende machen!“

„O bitte, bitte, es ist noch Zeit.“

„Kurz und gut! Mitt hatte den kühnen Plan erdacht, in einer
rückläufigen Hyperbel mit kurzer Periheldistanz quer über die
Erdbahn weg auf den Mars zu stoßen. Er setzte uns das kurz
auseinander. Allerdings mußten wir unsre Richtschüsse bis auf einen
letzten, zum Landen bestimmten Notvorrat daran wagen. Nur eine
Gefahr war dabei, und deshalb wollte Mitt nicht ohne unsere
Einwilligung handeln – wir kamen der Sonne in einer Weise nahe, wie
es noch kein Raumschiffer gewagt hatte, und es fragte sich, ob wir
die Strahlung würden aushalten können.

Auch der Plan, auf der Erde am Nordpol anzulegen, schien Mitt sehr
erwägenswert, und lange wurde hin und her überlegt, was zu tun
sei.

Aber Sie wissen ja, in jedem rechten Raumschifferherzen steckt die
Lust, das Ungewohnte zu wagen, wenn es einigermaßen aussichtsvoll
ist. Den Gefährten konnten wir in diesem Südpol-Sommer doch nicht
mehr helfen, und so wurde beschlossen, die kühne Hyperbelfahrt zu
versuchen.

Nun, Gott war gnädig, wir sind heimgekommen. Aber die zwei Tage,
die wir um die Sonnennähe jagten, die möchte ich nicht wieder
erleben. Ich habe manches durchgemacht – solche Glut noch nicht.
Wir konnten unsre äußere Stellitkugel nur dadurch vor dem Schmelzen
bewahren, daß wir sie schnell rotieren ließen; so strahlte sie die
auf der einen Seite empfangene Hitze auf der andern wieder aus –
weiß nicht, bekomme sogleich einen wahren Merkursdurst, wenn ich
daran denke!“

Damit tat Jo einen tiefen Zug aus seinem Mundstück und erhob sich.

„Schade, schade, daß Sie morgen schon fortgehen!“ sagte La zu Jo.
„Von der Sonnennähe müssen Sie uns noch einmal erzählen!“

„Wenn’s einmal recht kalt ist!“

„Und All? Hat man nichts mehr von ihm gehört?“ fragte Grunthe.

„Nichts! Auch bei wiederholten Besuchen des Südpols hat man keine
Spuren mehr gefunden, keine Aufzeichnungen. Und nun, Gott befohlen!
Auf Wiedersehen morgen vormittags!“

Jo schüttelte den Deutschen die Hände, und alle Martier
wiederholten die Begrüßung. Dann zogen sie sich zurück. Nur La und
Se blieben noch einige Minuten und redeten ihren Gästen zu, ihre
Reise nicht im Winter zu wagen, sondern mit ihnen nach dem Mars zu
gehen.

„Lassen Sie sich durch Jos Erzählung nicht bange machen“, sagte La
lächelnd. „Wir nehmen jetzt soviel Richtschüsse mit, daß wir allen
Hindernissen schleunigst ausweichen können. Die Gefahr lag ja
früher darin, daß man auf der Erdoberfläche landen und von dort
abreisen mußte; jetzt aber haben wir auf beiden Planeten Stationen
außerhalb der Atmosphäre.“

„Solche Besorgnisse würden uns nicht abhalten“, sagte Grunthe
ernst. „Wir hoffen ja später mit der Hilfe Ihrer Landsleute auf den
Mars zu reisen.“

„Und was hält Sie denn ab, schon jetzt mit uns zu kommen?“ fragte
Se.

„Die Pflicht“, erwiderte Grunthe.

La und Se schwiegen einen Augenblick. Dann sagte Se mit einem Blick
auf Saltner:

„Es gibt auch eine Pflicht gegen die Freunde.“

„Die Pflicht der Dankbarkeit gegen unsre Retter wird mir stets
heilig bleiben“, sagte Grunthe, „aber im Falle des Widerstreits
entscheidet die ältere –“

„Oder die höhere“, fiel La ein, „und das werden wir schon noch
untersuchen.“

„Das wissen Sie ja“, sagte Saltner herzlich, „daß ich nichts lieber
täte, als mit Ihnen zu gehen, wohin’s auch immer wäre.“

„Mit wem denn?“ scherzte La. „Wir wohnen leider auf dem Mars
dreitausend Kilometer voneinander.“

„Das ist nicht so schlimm“, erwiderte Saltner. „Sie haben dort
gewiß so schnelle Beförderungsmittel, daß man einen Tag hier und
einen da sein kann. Und das hat auch seine guten Seiten.“

„Das ist reizend“, rief Se. „Sie passen ausgezeichnet auf den Mars.
Wenn wir Sie nun beim Wort nehmen?“

Se und La warfen sich einen Blick des Einverständnisses zu. Dann
faßten sie jede einen seiner Finger und sagten gleichzeitig:

„Gebunden.“

Saltner machte ein etwas verdutztes Gesicht, da er nicht recht
wußte, was das bedeuten sollte.

„Wieso?“ fragte er. „Was soll das sein?“

„Ein Spiel!“ rief La, und beide sahen ihn so sonderbar und
freundlich an, daß ihm ganz seltsam ums Herz wurde.

„Gehen’s“, sagte er etwas verlegen, „Sie wollen mich gewiß zum
besten haben. Was muß ich denn jetzt tun?“

„Das wird sich schon finden. Recht liebenswürdig sein müssen Sie!“
sagte Se. „Und jetzt gute Nacht! Sie müssen morgen zeitig
aufstehen, eigentlich schon heute, der Flugwagen nach der
Außenstation geht um ein Uhr.“

„Auf Wiedersehen morgen am abarischen Feld!“ rief La.

Und beide nickten ihm freundlich zu, grüßten Grunthe und schwebten
mit ihrem leichten, gleitenden Schritt nach der Tür. Die Wolke
glühender Funken wogte um Se, und über den schlanken Formen ihres
Halses schimmerte der zarte Regenbogen ihres Haars. Über Las Haupt
glänzte es wie ein Heiligenschein, und aus ihren tiefen Augen fiel
ein langer Blick auf Saltner zurück. Dann schloß sich die Tür. Die
Feen der Insel waren verschwunden.

Saltner stand noch lange stumm und blickte nach der geschlossenen
Tür. Was meinten sie wohl? Wie sollte er sie verstehen? Und welche
von beiden – –

Dann drehte er sich auf dem Absatz herum und pfiff leise vor sich
hin.

„Das ist gescheit“, sagte er, „die scheinen halt nicht
eifersüchtig. Aber – am Ende ist das gar nicht sehr schmeichelhaft
für mich. Wer kann sich auch gleich bei den Feen auskennen? Kommen
Sie, Grunthe, wir wollen soupieren.“

Die beiden Männer zogen sich in ihr Zimmer zurück, aßen zu Abend
und sprachen dabei hin und her über die Frage, ob sie imstande sein
würden, dem Wunsch der Martier zu widerstehen und am Pol
zurückzubleiben.

„Ich ging schon gern hin“, sagte Saltner endlich, „aber von Ihnen
geh ich nicht, alter Freund. Und nun sehen Sie zu, was Sie
durchsetzen.“

\section{15 - 6.356 Kilometer über dem Nordpol}

Grunthe und Saltner ruhten noch in ihren Betten, als bereits im
abarischen Feld ein reges Leben herrschte. Die Martier, welche das
Raumschiff besteigen sollten, begaben sich in Abteilungen von je
vierundzwanzig Personen nach der Außenstation. So viele faßte der
Flugwagen, der den Verkehr von der Insel nach dem Abgangspunkt der
Raumschiffe vermittelte, nach jenem in der Höhe von 6.356
Kilometern über dem Pol schwebenden Ring. Es waren also, um die
Reisenden und diejenigen ihrer Freunde, die sie bis an das Schiff
begleiten wollten, nach dem Ring zu befördern, drei Flugwagen
erforderlich. Der Aufstieg nahm ungefähr eine Stunde in Anspruch,
und da sich niemals mehr als ein Wagen im abarischen Feld befinden
durfte, so verließ der erste Wagen schon am frühsten Morgen,
richtiger noch in der konventionellen Schlafenszeit, denn die Sonne
ging ja nicht auf noch unter, die Inselstation. Dies war nach der
Tageseinteilung, welche die Martier für den Nordpol der Erde
festgesetzt hatten, um 11,6 Uhr, nach mitteleuropäischer Zeit
ungefähr um 11 Uhr vormittags, eine Stunde vor dem Aufstehen, wie
es sonst auf der Insel üblich war.

Diesmal mußten Grunthe und Saltner freilich etwas früher ihre Ruhe
unterbrechen, denn der dritte Flugwagen, der sie nach der
Außenstation bringen sollte, verließ die Insel gegen 0,6 Uhr, um
eine Stunde vor der Abfahrt des Raumschiffes am Ring zu sein.

Die Martier waren schon fast vollständig in der Abfahrtshalle am
abarischen Feld versammelt, als Grunthe und Saltner ankamen. Die
meisten der Anwesenden waren ihnen bereits bekannt, und alle
begrüßten sie aufs liebenswürdigste. Auch Hil, der Arzt, hatte sich
eingefunden. Da die Menschen zum erstenmal eine Fahrt im abarischen
Feld machten – wenn man die unfreiwillige in ihrem Luftballon nicht
mitrechnen wollte –, so war es ihm von größtem wissenschaftlichem
Interesse, ihr Verhalten dabei zu beobachten. Auch konnte man ja
nicht wissen, ob nicht vielleicht unter den ungewohnten
Bedingungen, denen die Menschen hier ausgesetzt waren, seine Hilfe
vonnöten würde. Indessen wußten sich Grunthe und Saltner schon ganz
geschickt zu benehmen, als sie die auf Marsschwere gestellte
Vorhalle betraten. Zu ihrer Verwunderung sahen sie, daß die Martier
die Pelzkragen nicht mehr trugen, in denen sie den Weg über die
Insel zurückgelegt hatten, sondern sich in ihrer gewöhnlichen
Zimmertoilette befanden.

Hil forderte sie auf, ebenfalls ihre Mäntel abzulegen, da sie nun
bis zu ihrer Rückkehr nicht mehr ins Freie kämen. Wagen und
Ringstation seien selbstverständlich künstlich erwärmt.

Vergeblich sah sich Saltner nach La und Se um. Schon ertönte das
Signal zum Einsteigen, als La eilig hereinkam und die Anwesenden
begrüßte. Ihre Blicke flogen alsbald zu Saltner, der sich ihr noch
schnell näherte und ihr die Hand reichen wollte. Sie aber legte
beide Hände auf seine Schultern und sah ihm zärtlich in die Augen.
Die Begrüßung überraschte ihn, er mußte sich einen Augenblick
sammeln, denn er wußte, daß diese Form des Willkomms nur unter ganz
nahestehenden Freunden oder Liebenden üblich war und ungefähr die
Bedeutung eines Kusses unter den Menschen besaß. Aber ihre Blicke
gaben ihm schnell den Mut, sie zu erwidern, und zu seiner großen
Freude glückte es ihm, ihre Schultern mit seinen Händen zu
berühren, ohne zu hoch in die Luft zu greifen, und sie auch wieder
zu entfernen, ohne das Gleichgewicht zu verlieren. Nur das rosig
schimmernde Haar streiften seine Finger, und er fühlte diese
Berührung wie ein leises Überspringen elektrischer Funken.

Schon bestiegen die übrigen den Flugwagen. Hil geleitete Grunthe
hinein. La faßte Saltner an der Hand, um ihn beim Hinaufsteigen der
ungewohnten Stufen ins Innere des Wagens zu unterstützen. Ehe er
dieselben betrat, blickte er noch einmal zurück, um nach Se zu
schauen, ob sie nicht käme.

„Heute nicht“, sagte La, seinen Gedanken erratend, „morgen sehen
Sie sie wieder. Heute müssen Sie mit mir vorliebnehmen.“

Es war keine Zeit zu Erklärungen. Der Wagen wurde geschlossen. Dies
geschah, indem der außenstehende Beamte die Falltür hob, durch
welche die Reisenden in das Innere des Wagens gestiegen waren. Der
Boden bildete jetzt die ebene, mit weichen Teppichen belegte Fläche
eines geräumigen Zimmers. Die Decke war gleichfalls eben, während
der ganze Wagen äußerlich die Gestalt einer vollkommenen Kugel
besaß. In den beiden Segmenten, welche durch Boden und Decke
gebildet waren, befand sich je ein Wagenführer, die beide durch
Signale mit der untern wie mit der oberen Station verkehrten.

Nirgends zeigte sich ein Fenster, von der Außenwelt war nichts zu
sehen. Eine Anzahl von Kugeln, welche an unsichtbaren Lisfäden von
der Decke herabhingen, verbreitete ein angenehmes Licht. Die
Deutschen sahen hier zum erstenmal die künstliche Beleuchtung der
Martier durch fluoreszierende Lampen, die nur aus absolut luftleer
gemachten, durchscheinenden Kugeln bestanden und infolge der
schnellen Wechselströme leuchteten, welche von dem mittleren Teil
der Wagenwand ausgingen. In diesem befand sich auch der
Heizapparat. Das Zimmer hatte im Grundriß die Gestalt eines
Quadrats, so daß zwischen seinen Wänden und der Kugel noch Raum für
einige kleinere Gelasse blieb. Die Ausstattung war die bei den
Martiern übliche mit einem festen Tisch in der Mitte, der zugleich
als Büffet diente. Nur dadurch unterschied sie sich von der eines
gewöhnlichen Gesellschaftszimmers, daß sich ringsum an den Wänden
auffallende Gestelle hinzogen, deren Zweck Grunthe nicht zu erraten
vermochte. Er war geneigt, sie für Turngeräte zu halten, und etwas
Ähnliches waren sie auch. Eigentümlich waren ferner die Stühle,
sämtlich mit Seitenlehnen und Leisten an den Füßen versehen. Diese
Stühle konnte man zwar, infolge einer besonderen Mechanik, nach
Verlangen hin- und herschieben, nicht aber vom Boden aufheben.

Kaum war der Wagen verschlossen, als ein zweites Signal ertönte.
Schnell suchte sich jeder der Martier eines der Gestelle am Rand
des Zimmers und begab sich in dasselbe. Grunthe und Saltner wurden
angewiesen, wie sie sich dabei zu benehmen hätten. Sie steckten die
Füße in schuhartige Vorsprünge am Boden, so daß sie nicht
ausgleiten konnten, stemmten sich mit den Armen fest an den zur
Seite befindlichen Griffen und lehnten sich mit dem Rücken an die
gepolsterte Wand, während der Kopf zwischen weichen Kissen wie in
einer Grube ruhte.

„Nun bin ich nur neugierig, was das soll“, sagte Saltner.
„Hoffentlich brauchen wir nicht zwei Stunden lang hier als Mumien
zu stehen.“

„Es dauert nicht lange“, sagte einer der Martier.

„Halten Sie sich ganz fest“, fügte La hinzu, „von dem Augenblick,
in welchem die tiefe Glocke erklingt und das Licht sich verdunkelt,
bis es wieder hell wird, und rühren Sie sich ja nicht.“

„Ich folge blindlings –“

„Warum –“ Grunthe wollte etwas fragen. Da erscholl das Signal. Das
Licht wurde so schwach, daß man eben nur noch die Stellen sah, wo
die Lampen hingen.

Es erfolgte ein dumpfer Knall. Die Insassen der Kugel erlitten eine
leichte Erschütterung und fühlten sich kräftig gegen den Boden
gedrückt. Unter die Kugel war nämlich ein Behälter mit stark
komprimierter Luft gebracht worden, durch deren Entspannung der
Flugwagen mit einer Geschwindigkeit von 30 Metern pro Sekunde in
dem abarischen Feld aufwärtsgeschleudert wurde. Gleichzeitig wurde
die Schwere im Feld vollständig kompensiert. Während bisher die
Schwerkraft innerhalb der Kugel, der Gewohnheit der Martier
entsprechend, immer noch ein Drittel der Erdschwere betragen hatte,
war sie jetzt gänzlich aufgehoben.

Das Gefühl, welches die Menschen ergriff, war nicht unangenehm und
keineswegs stark, ähnlich wie in einem Bad, nur daß die
Berührungsempfindung des Wassers fehlte. Man gewöhnte sich schnell
daran und gewahrte nur einen schwachen Blutandrang nach dem Kopf.

Die Lampen wurden wieder hell, und ein Teil der Martier kam
vorsichtig aus den Gestellen hervor. Sie machten sich das
Vergnügen, in dem absolut schwerelosen Raum durch einen leichten
Stoß gegen den Boden bis zur Decke in die Höhe zu schwingen und
sich von dort wieder abzustoßen oder eine Zeitlang ohne jede
Unterstützung völlig frei in der Luft zu schweben.

Saltner hätte dies gern auch einmal probiert, aber La riet ihm
dringend, sein Gestell noch nicht zu verlassen, da es längerer
Übung bedürfe, ehe man sich in dem schwerelosen Raum geschickt
bewegen könne. Dagegen forderte sie zwei Damen, welche die Fahrt
mitmachten, zu einem kleinen Tänzchen auf, und die drei graziösen
Figuren schwebten nun, indem sie mit geschickten Bewegungen sich
vom Boden und den Wänden abstießen, Hand in Hand um das Zimmer. In
ihren wehenden Schleiern glichen sie den Elfen des Märchens, die in
der Mondnacht ihren luftigen Reigen führen. Darauf zogen sie sich
wieder auf ihre Plätze zurück.

Grunthe nahm sein Fernrohr aus der Tasche, streckte die Hand aus
und öffnete sie dann. Das Fernrohr blieb frei in der Luft schweben,
ohne zu fallen. Er konnte es sich nicht versagen, selbst einmal zu
versuchen, wie es sich ohne Schwere gehe, und trat aus seinem
Gestell. Sobald er aber dasselbe losgelassen und den Fuß zum ersten
Schritt erhob, verlor er das Gleichgewicht und focht mit Händen und
Füßen in der Luft herum, ohne wieder auf den Boden kommen zu
können. Es sah ungeheuer possierlich aus, wie der ernste Mann hin-
und herstrampelte, und Saltner war sehr froh, daß er Las Rat
gefolgt war, sich nicht von seinem festen Punkt fortzuwagen. Erst
durch Hilfe einiger Martier kam Grunthe wieder auf den Boden zu
stehen und wurde in sein Gestell zurückgeführt.

„Es schadet nichts“, sagte er, „man muß alles versuchen.“

Jetzt erscholl ein neues Signal, worauf alle sich schleunigst in
ihre Gestelle begaben. Gleich darauf wurde es ganz dunkel bis auf
den matten Schimmer einer Lampe, welche genau die Mitte des Zimmers
einnahm. Doch reichte ihr Schein nur aus, ihre Stelle zu
bezeichnen, nicht aber, irgendwelche andere Gegenstände zu
erkennen.

„Was kommt denn nun?“ fragte Saltner.

Hil antwortete ihm. „Bis jetzt“, sagte er, „sind wir ohne Schwere
durch den gegebenen Anstoß mit gleichmäßiger Geschwindigkeit
gestiegen, und zwar sechs Minuten lang. Wir haben dadurch eine Höhe
von ungefähr 10.000 Metern erreicht. Die Luft ist hier dünn genug,
daß wir eine größere Geschwindigkeit annehmen können. Das Feld wird
jetzt überkompensiert, das heißt, die ›Gegenschwere‹ überwiegt nun
die Schwere, und wir ›fallen‹ nach oben, nach dem Ring zu. Sie
werden bald merken, daß unsere Geschwindigkeit stark zunimmt, denn
unser Fall nach dem Ring beschleunigt sich natürlich rasch.“

In der Tat bemerkten Grunthe und Saltner bald dasselbe Gefühl,
welches sie bei sehr beschleunigtem Fallen des Ballons zu haben
pflegten. Es war, als würde ihnen der Boden unter den Füßen
entzogen.

„Was ist denn das?“ rief Saltner. „Wir stürzen ja ab!“

„Freilich fallen wir“, lachte La, „aber nach oben, das heißt, von
der Erde fort.“

„Ich fühle doch, daß der Boden unter den Füßen sich senkt.“

„Ganz richtig, aber wo, glauben Sie, daß die Erde sich befindet?“

„Nun, doch unter uns!“

„Fehlgeschossen! Sie stehen jetzt auf dem Kopf wie ein Antipode.
Die Erde ist über Ihrem Scheitel, unsre Füße sind dem Ring der
Außenstation zugekehrt, wohin jetzt die Richtung der Fallkraft
hinweist.“

„Ach, liebste La, wollen Sie mich denn vollständig verdreht
machen?“

Als Antwort hörte er ihr leises Lachen.

Es wurde wieder hell. Nichts im Zimmer hatte sich verändert.

Die Martier verließen nun ihre Gestelle und bewegten sich wie
gewöhnlich im Zimmer.

Auch Grunthe und Saltner bemerkten, daß sich das eigentümliche
Gefühl des Fallens ziemlich verloren hatte. Doch kam dies nur
daher, daß sie sich daran gewöhnt hatten. Tatsächlich flog die
Kugel mit immer größerer Geschwindigkeit auf ihr Ziel zu, von der
Erde fort, und diese Geschwindigkeit sollte sich allmählich bis auf
die kolossale Zahl von gegen zweitausend Meter in der Sekunde
steigern.

Der untere Teil der Kugel, unter dem Fußboden, war beschwert, so
daß sich die Kugel je nach der Richtung der Fallkraft immer mit dem
Boden des Zimmers nach unten einstellte. Diese Drehung hatte sich
sofort vollzogen, als das Feld überkompensiert wurde und die
Beschleunigung nach oben begann. Aber die Insassen hatten gar
nichts davon bemerkt, da sie fest in ihren Gestellen ruhten und die
Wirkung der Schwere im Anfang so gering war, daß es zu ihrer
Aufhebung keiner merklichen Muskelkraft bedurfte. Sie standen
jetzt, im Vergleich zu ihrem Aufenthalt am Pol, tatsächlich auf dem
Kopf; im Vergleich zu der auf sie wirkenden Anziehungskraft
befanden sie sich jedoch in der normalen Lage; sie standen auf
ihren Füßen. Immerhin mußten sich Grunthe und Saltner vorsichtig
bewegen, da das Feld nur um ein Drittel der Erdschwere
überkompensiert war, das heißt so, daß die Insassen der Kugel unter
einer anziehenden Kraft standen, wie sie sie auf dem Mars gewohnt
waren. Die Menschen zogen es daher vor, sich auf den Sesseln am
Tisch niederzulassen und dort zu bleiben. Es fehlte nicht an
Unterhaltung mit den Martiern, die jetzt zu ihren Piks gegriffen
hatten. Hil hatte sich überzeugt, daß die Menschen die
Schwerelosigkeit leicht ertrugen. Saltner saß Hand in Hand mit La
in vertraulichem Gespräch. Niemand kümmerte sich um sie.

Eine halbe Stunde etwa nach der Abfahrt von der Erde mußten die
Insassen des Wagens auf das gegebene Signal noch einmal ihre Plätze
in den seitlichen Verschlägen einnehmen. Der Wagen hatte jetzt
seine größte Geschwindigkeit erreicht und über die Hälfte seines
Weges zurückgelegt. Es kam nunmehr darauf an, seine Geschwindigkeit
zu vermindern und so zu regulieren, daß er gerade innerhalb des
Ringes zur Ruhe kam. Dies geschah, indem man die Erdschwere wieder
wirken ließ. Diese besaß jedoch in dieser Höhe nicht mehr die volle
Stärke wie am Pol, sondern war nur noch etwa so groß wie auf dem
Mars, ja auf dem Ring selbst betrug sie nur ein Viertel der unten
herrschenden Schwere. Der Wagen glich jetzt einem Körper, den man
mit großer Geschwindigkeit in die Höhe geworfen hat und der sich
nun mit abnehmender Geschwindigkeit dem höchsten Punkt seiner Bahn
nähert. Der Fußboden des Wagens mußte sich demnach wieder der Erde
zuwenden, und diese Drehung wartete man bei verdunkeltem Wagen in
den schützenden Gestellen ab. Den übrigen Teil der Fahrt über
konnte man sich nach Belieben im Wagen bewegen, nur kurz vor der
Ankunft wurden die Gestelle wieder aufgesucht. Denn der letzte Teil
des Weges mußte mit gleichmäßiger, nicht sehr bedeutender
Geschwindigkeit zurückgelegt werden, um das Anhalten des Wagens im
richtigen Zeitpunkt zu regulieren. Dazu aber war es notwendig,
diese Strecke abarisch, ohne jede Schwere zu durchlaufen, bis das
Wiedereinstellen der Schwere in der letzten Sekunde den Wagen
anhielt.

Man bemerkte kaum das Anhalten des Wagens, so allmählich war es
geschehen. Das Fallnetz hatte sich unter ihm geschlossen und war
nach der Befestigung des Wagens wieder entfernt worden. Die Tür im
Boden wurde geöffnet.

Ehe die Reisenden den Wagen verließen, versahen sich alle mit
Schutzbrillen für die Augen, da hier oben das direkte Sonnenlicht
durch keine Atmosphäre gemildert war und alle Gegenstände, auf die
es traf, in blendendem Glanz erscheinen ließ, während sich die
Schatten tiefschwarz abhoben. Nun trat man in die mittlere Galerie
des Ringes.

Die Martier durchschritten dieselbe und begaben sich sogleich durch
die Tür, welche die Überschrift trug ›Vel lo nu‹ – ›Raumschiff nach
dem Mars‹ –, nach der oberen Galerie, über welcher das Raumschiff
ruhte. Grunthe und Saltner dagegen wurden von Hil und La zunächst
durch eine andere Tür nach der unteren Galerie geleitet, und zwar
nach derjenigen, welche den Ring auf seiner äußeren Seite umzog.

Eine zweite solche untere Galerie umgab den Ring auf der inneren
Seite und enthielt die Apparate, durch welche das abarische Feld
kontrolliert wurde. Hier befanden sich auch die Arbeitsräume der
Ingenieure. Um nach der äußeren Galerie durch einen Verbindungsweg
zu gelangen, mußte man zunächst die innere durchschreiten, und La
begrüßte ihren Vater Fru, dem die Leitung der Außenstation oblag.
Die äußere, sechs Meter breite Galerie sprang noch etwa zwei Meter
über die Seitenwand des Ringes vor, so daß man an dieser vorüber in
die Höhe blicken konnte. Sie diente als Aussichtsraum, von welchem
aus der Blick auch nach der inneren Seite des Ringes frei war, so
daß man nach unten den ganzen Horizont beherrschte.

Ihrer vollen Länge nach hatte man nach Art eines Balkons eine
Brüstung angebracht, so daß man glaubte, von diesem erhabenen
Standpunkt aus direkt ins Freie zu sehen. Tatsächlich war man durch
den vollkommen durchsichtigen Stoff der Außenwand vom luftleeren,
eisigen Weltraum geschieden. Aber die in weiten Zwischenräumen sich
folgenden Träger dieser Galerie hinderten ebensowenig die Aussicht
wie der weiter oberhalb sich drehende durchbrochene Schwungring.
Die Stelle, an welcher Grunthe und Saltner mit ihren Begleitern die
Galerie betraten, lag von der Sonne abgewendet, so daß die Strahlen
derselben, trotz ihres niedrigen Standes, durch die ganze Breite
des über der Galerie befindlichen Ringes abgeblendet wurden. Sie
standen in einer geheimnisvollen Dämmerung, die nur durch den
Reflex des Mondlichtes auf dem einen Rand der Galerie und durch
denjenigen des Erdlichtes an der Decke über ihnen erhellt wurde.

Tiefschwarz lag der Himmel ringsum, über ihnen, an den Seiten, zu
ihren Füßen; auf dem schwarzen Grund glänzten die Sterne in nie
geschauter Klarheit, ohne zu funkeln, als tausend ruhig leuchtende
Punkte. Im ersten Augenblick glaubten die Forscher in einen tiefen
See zu blicken, in welchem der Himmel sich spiegele. Dann erst
erkannten sie, daß sie zu ihren Füßen einen großen Teil der
Sternbilder des südlichen Himmels vor sich hatten. Denn ihr Blick
beherrschte den Himmel bis zu sechzig Grad unter den Horizont des
Nordpols.

In der Mitte zu ihren Füßen schwebte die Erde als eine glänzende
Scheibe. Sie hatte die Gestalt des zunehmenden Mondes kurz nach
seinem ersten Viertel, doch erblickte man auch den von der Sonne
nicht beleuchteten Teil, da ihn das Licht des Mondes in einen
schwachen Schimmer hüllte. Die ganze Scheibe der Erde erschien
unter einem Gesichtswinkel von sechzig Grad und erfüllte somit
gerade den dritten Teil des Himmels unterhalb des Horizontes. Die
Schattengrenze schnitt das Eismeer in der Nähe der Jenisseimündung,
so daß der größte Teil Sibiriens und die Westküste Amerikas im
Dunkel lagen. Hell glänzten die Gletscher an der Ostküste Grönlands
im Schein der Mittagssonne, und als ein strahlender weißer Fleck
hob sich Island aus den dunklen Fluten des Atlantischen Meeres. Der
westliche Teil des Ozeans und der amerikanische Kontinent waren
nicht zu erkennen. Über ihnen ruhte eine nur selten unterbrochene
Wolkenschicht, deren obere Seite die Sonnenstrahlen in blendendem
Weiß zurückwarf, so daß ihr Anblick ohne die schützenden
Augengläser unerträglich gewesen wäre. Dagegen lag die Karte von
ganz Europa, wenigstens in seinem nördlichen Teil, in günstigster
Beleuchtung vor den entzückten Blicken. Unter dem Einfluß eines
ausgedehnten Hochdruckgebiets war die Luft dort völlig klar und
rein, so daß man die nördlichen Inseln und Halbinseln und die tief
eingeschnittenen Meeresbuchten deutlich erkannte. Weiterhin
verschwammen die Formen der Ebenen in einem bläulich-grünlichen
Luftton, aber als feine helle Linien blitzten für ein scharfes Auge
die Ketten der Alpen und selbst des Kaukasus auf. In matterem Licht
schimmerte der Rand des beleuchteten Teils der Scheibe, und nur an
der Schattengrenze bezeichneten einige helle Lichtpunkte den
Untergang der Sonne für die Schneegipfel des Tianschan und des
Altai.

In tiefem Schweigen standen die Deutschen, völlig versunken in den
Anblick, der noch keinem Menschenauge bisher vergönnt gewesen war.
Noch niemals war es ihnen so klar zum Bewußtsein gekommen, was es
heißt, im Weltraum auf dem Körnchen hingewirbelt zu werden, das man
Erde nennt; noch niemals hatten sie den Himmel unter sich erblickt.
Die Martier ehrten ihre Stimmung. Auch sie, denen die Wunder des
Weltraums vertraut waren, verstummten vor der Gegenwart des
Unendlichen. Die machtvollen Bewohner des Mars und die schwachen
Geschöpfe der Erde, im Gefühl des Erhabenen beugten sich ihre
Herzen in gleicher Demut der Allmacht, die durch die Himmel waltet.
Aus der Stille des Alls sprach die Stimme des einen Vaters zu
seinen Kindern und füllte ihre Seelen mit andächtigem Vertrauen.

La hatte Saltners Hand ergriffen, sanft lehnte sie sich an seine
Schulter, und mit der Rechten auf den hellsten der Sterne weisend,
der unterhalb des Horizonts des Pols leuchtete sagte sie leise:
„Dort ist meine Heimat.“

Saltner zog sie an sich und sprach: „Und dort meine Erde, ist sie
nicht schön?“

Grunthe holte sein Relieffernrohr hervor und trat dicht an den
inneren Rand der Galerie, welcher den Blick auf den Nordpol
gestattete. Auch ihn hatte die Erinnerung an die so greifbar nahe
vor ihm ausgebreiteten und doch so unerreichbaren fernen Lande
seiner Heimat weichgestimmt. Aber er wollte nichts wissen von dem,
was La und Saltner sich zu sagen hatten. Ihn beschäftigte jetzt,
nachdem das Überwältigende des ersten Eindrucks vorüber war, vor
allem der Gedanke, wie er es ermöglichen könne, die Reise über die
Eisfelder und Meere des Polargebiets zurückzulegen. Und er wollte
die günstige Gelegenheit benutzen, von hier oben den Weg zu
überblicken, den er auf den Karten der Martier schon wiederholt
studiert hatte. Ein kleiner dunkler Fleck direkt unter ihm stellte
das Binnenmeer am Pol vor, und mit seinem Glas konnte er die Insel
in der Mitte desselben erkennen. Er wandte sich mit einer Frage an
Hil, der ihn an eine andere Stelle der Galerie führte.

„Sie können hier“, sagte er, „die Erde bequemer mit einem unsrer
Apparate betrachten, der Ihnen eine hundertfache Vergrößerung gibt.
Später sollen Sie im Laboratorium unser großes Fernrohr mit
tausendfacher Annäherung kennenlernen.“

La blickte lange nach der Erde hinab. Dann sagte sie in ihrer
langsamen, tiefen Sprechweise: „Größer und schöner mag eure Erde
sein, aber ich müßte dort sterben in eurer Schwere. Und schwer wie
die Luft sind eure Herzen. Ich aber bin eine Nume.“

Sie ließ das schützende Augenglas herabfallen und wendete ihm voll
das Gesicht zu. In ihrem Blick flammte wieder jene unbeschreibliche
Überlegenheit, welche den Menschenwillen brach. Aber es war nur ein
Moment. Dann wechselte der Ausdruck ihrer Züge, ihre Wimpern
senkten sich über die Sterne ihrer Augen, und Saltner fühlte, wie
ein Strom von Wärme ihrem Antlitz entstrahlte, das sie nun zur
Seite wandte.

Vom Zauber ihrer Nähe hingerissen, beugte er sich ihr entgegen und
drückte seine Lippen auf ihren Hals.

La zuckte zusammen. Schon fürchtete Saltner, sie beleidigt zu
haben, aber sie wandte sich mit einem glücklichen Lächeln und
duldete seinen Kuß auf ihren Mund.

„Geliebte La“, flüsterte er, „wie glücklich machst du mich! Ist es
denn möglich, du Wunderbare, daß ein armer Mensch eine Nume lieben
darf?“

Sie sah ihn freundlich an und antwortete: „Ich weiß es nicht, was
ihr Liebe nennt und was ein Mensch darf. La aber darf dem Menschen
nicht zürnen, ohne den sie den Nu nicht wiedersehn würde – – doch,
mein Freund –“, und ihr Blick wurde ernst, „– vergiß nicht, daß ich
eine Nume bin.“

„Aber ich liebe dich!“

„Ich will es nicht verbieten, nur vergiß niemals –“

„Das verstehe ich nicht, wenn ich nur dein sein darf –“

„Die Liebe der Nume macht niemals unfrei“, sagte La.

„Und wenn du mich lieb hast –“

„Wie Nume lieb haben. Und du mußt wissen, wenn sie es tun, daß dies
niemand etwas angeht als sie selbst, und daß –. Ich weiß es auf
deutsch nicht recht zu sagen –“

„Auf martisch versteh ich’s ganz gewiß nicht, aber ich weiß –“, und
Saltner zog ihre Hand an seine Lippen, „– ich weiß, daß du –“ Seine
beredten Schmeichelworte wurden durch die Annäherung Hils
unterbrochen.

„Wenn wir vor dem Abgang noch einen Blick in das Schiff werfen
wollen“, sagte er, „so ist es jetzt Zeit.“

„Schon?“ rief La. „Wir haben die Erde noch gar nicht durchs
Fernrohr betrachtet.“

„Das können wir noch vor der Rückfahrt.“

„Aber dann ist es vielleicht in Deutschland schon Abend“, sagte
Saltner, „ich möchte doch gern –“

„Durchaus nicht“, erwiderte Hil. „In einer halben Stunde ist alles
vorüber, und dann haben Sie erst ein Viertel nach drei Uhr. – Aber
lassen Sie uns jetzt eilen!“

\section{16 - Die Aussicht nach der Heimat}

Die vier Besucher des Ringes begaben sich über die mittlere Galerie
nach der Treppe zur oberen. Hier gelangten sie in die weite Halle,
von welcher aus die Abfahrt der Raumschiffe stattfand. Das rege
Leben, das hier geherrscht hatte, begann sich jetzt zu beruhigen.
Denn die Einschiffung der Abreisenden war vollendet, und ihre
Begleiter verließen soeben das Schiff. Die Luke sollte geschlossen
werden.

Hil mit seiner Begleitung hatte sich doch verspätet, und so mußten
Grunthe und Saltner sich diesmal darauf beschränken, das Raumschiff
von außen zu betrachten. Sie trösteten sich damit, daß in drei
Tagen bereits eine neue Abfahrt stattfände; überdies fesselte sie
der Anblick, der sich ihnen darbot, zur Genüge.

Die riesige Halle besaß einen Radius von 60 Meter. An ihrer Decke,
und zwar rings um den Rand herum, befanden sich kreisförmige
Einschnitte. Auf fünf von ihnen ruhte je ein Raumschiff, so daß das
untere Segment desselben in die Halle hineinragte und von hier aus
zugänglich war. Der überwiegende Teil jedes Raumschiffs befand sich
natürlich oberhalb der Decke nach außen, wodurch die Halle, wenn
man sie von oben her hätte betrachten können, wie von fünf
Riesenkuppeln gekrönt erschienen wäre. Bei vollbesetzter Station
hätten sich acht Kuppeln über der Halle erhoben. Die Martier waren
imstande, acht Raumschiffe gleichzeitig auf der Station zu halten.
Die vorhandenen fünf Schiffe sollten in dreitägigen Zwischenräumen
die Station verlassen; sie vermochten sämtliche anwesende Martier
fortzufahren, so daß also der Aufenthalt der Martier auf der Insel
in fünfzehn Tagen beendet sein mußte. Man konnte durch die
vollständig durchsichtige Decke die Außenseite der Schiffe genau
betrachten. Sie stellten vollkommene Kugeln dar, die mit ihrem
größten Umfang noch weit über den Rand der Galerie hinausragten.
Auch nicht der geringste Vorsprung, nicht die kleinste Unebenheit
war an ihnen zu entdecken. Die äußeren Hüllen dieser Kugeln waren
durchsichtig. Man erblickte hinter ihnen die innere Kugel, den
eigentlichen Schiffsraum, von welchem aus eine Reihe von Öffnungen
in den Zwischenraum zwischen beiden Kugeln hineinführte. Dieser
über zwei Meter breite Raum trug in regelmäßiger Anordnung allerlei
Gerüste, die den verschiedenen Zwecken der Raumfahrt dienten. Jetzt
waren sie zum größten Teil von den Martiern besetzt, die mit ihren
Freunden in der Abfahrtshalle noch Abschiedsgrüße austauschten.

An der tiefsten Stelle der Kugel befand sich ein abgegrenzter Raum,
der die Kommandobrücke bildete. Hier erschien jetzt Jo. Er warf
einen Blick auf die Apparate, die rings um seinen Platz angeordnet
waren. Dann grüßte er mit einer Handbewegung in die Halle hinein
und drückte auf einen Knopf. In diesem Augenblick leuchtete zu
seinen Füßen auf der Innenseite der durchsichtigen Kugel das Bild
eines Kometen und der Name des Schiffes, das der ›Komet‹ hieß, in
bläulichem Fluoreszenzlicht auf. Dies war das Zeichen, daß der
›Komet‹ bereit war, seine Reise anzutreten.

Man hatte schon vorher die ganze Galerie, die sich um ihre
vertikale Achse drehen ließ, für die Abfahrt passend eingestellt.
Genau in der Sekunde, in welcher diese stattfinden sollte, mußte
der Punkt der Galerie, wo das Schiff sich befand, von der Sonne
abgewendet stehen. Denn sobald das Schiff bei seiner Abfahrt völlig
schwerelos gemacht wurde, bewegte es sich in der Tangente der
Erdbahn. Da aber die Erde gleichzeitig in ihrer Bahn fortlief, so
hatte dies zur Folge, daß das Schiff in bezug auf die Erde sich auf
einer Linie entfernte, welche genau von der Sonne fortwies. Nach
dieser Richtung hin also mußte die Bahn frei sein. Die Sonne hatte
den niedrigen Stand von gegen sieben Grad über dem Horizont, die
Bewegung wich somit von der horizontalen wenig ab.

Die Martier im Innern der Abfahrtshalle fuhren jetzt auf Schienen
eine eigentümliche Hebemaschine unter das Schiff. Sie bestand in
einem oben offenen, unten geschlossenen Zylinder, welcher dazu
diente, das Schiff aus seinem Lager zu heben und gleichzeitig die
Öffnung der Abfahrtshalle luftdicht zu schließen. Der Zylinder
wurde in die Höhe geschraubt und hob dadurch auf seinem oberen
Rande das fast schon schwerelos gemachte und darum leicht
bewegliche Schiff empor. Als das Schiff so hoch gebracht war, daß
sein tiefster Punkt höher stand als das Dach der Halle, wurde der
Hebungszylinder angehalten. Auf ein gegebenes Zeichen mußte er
herabfallen und damit das Schiff freigeben.

Der entscheidende Augenblick nahte. Die vollkommene Diabarie des
Schiffes mußte genau in dem berechneten Moment eintreten, wenn
nicht die Disposition der ganzen Raumreise dadurch verändert werden
sollte.

Jo hatte seinen Blick auf die Uhr gerichtet, während seine Hand den
Griff des diabarischen Apparats umfaßt hielt. Mit größter
Aufmerksamkeit beobachtete ihn der Ingenieur im Innern der Halle,
um das Zeichen zum Fallen des Stütz-Zylinders zu geben.

Jetzt blickte Jo hinab und drückte auf den Griff. Zugleich sank der
Zylinder nach unten. Die riesige Kugel schwebte, vollständig frei,
dicht über dem Dach der Halle.

Die Martier im Schiff und in der Halle schwenkten grüßend Hände und
Tücher. Mit angehaltenem Atem folgten Grunthe und Saltner dem
wunderbaren Schauspiel, das so gar keine Ähnlichkeit mit dem
Aufstieg eines Luftballons hatte. Es schien den Menschen, als müßte
die freischwebende Riesenmasse sie im nächsten Augenblick
zerschmettern.

In den ersten Sekunden bemerkte man kaum, daß das Raumschiff sich
bewege, denn die Abweichung von der Erdbahn, welche in der ersten
Sekunde nur 3 Millimeter beträgt, steigt nach 10 Sekunden erst auf
30 Zentimeter. Nach einer Minute aber war die Entfernung schon auf
11 Meter gewachsen. Die Kugel passierte jetzt den Rand der Galerie
und schwebte frei über der unendlichen Tiefe, 6.300 Kilometer hoch
über der Erde. Selbst die geübten Luftschiffer Grunthe und Saltner
überkam ein beängstigendes Gefühl, als sie das Schiff so ganz
langsam, ohne jede bemerkbare Triebkraft, über den Abgrund ziehen
sahen. Schon wuchs die Entfernung merklicher. Nach zwei Minuten war
es 44, nach drei Minuten 100 Meter entfernt, und immer mehr
verschwanden die wehenden Tücher. Genau in der Richtung der
Sonnenstrahlen, sanft nach unten geneigt, hart am Rand des –
übrigens im leeren Raum nicht sichtbaren – Schattens des Ringes zog
das Schiff hin. Die Kugel wurde sichtlich kleiner; nach zehn
Minuten hatte sie einen Abstand von 1.100 Metern erreicht.

„Es ist nun hier weiter nichts mehr zu sehen“, sagte Hil zu
Saltner. „Wenn es Ihnen recht ist, werfen wir jetzt einen Blick auf
die Erde durch unsern großen Apparat.“

„Wie lange kann man den ›Komet‹ noch erblicken?“ fragte Grunthe.

„Mit dem Fernrohr“, erwiderte Hil, „können wir ihn so lange sehen,
bis er Richtschüsse gibt und durch den Erdschatten geht. Wie mir Jo
sagte, beabsichtigt er dies zu tun, sobald er 1.000 Kilometer von
hier entfernt ist. Das wird in 5 Stunden der Fall sein. Nachher
entfernt er sich natürlich mit viel größerer Geschwindigkeit, weil
er von der Erdbahn abbiegt.“

„Kann man die Lösung der Richtschüsse von hier beobachten?“

„Davon sehen Sie gar nichts. Ich will Ihnen jetzt etwas
Interessanteres zeigen, und Sie sollen mir mancherlei erklären.“

In der inneren auf der Unterseite des Ringes befindlichen Galerie
traf die kleine Gesellschaft auf Las Vater, der erst jetzt Saltner
und Grunthe freundlich begrüßte, da er bisher zu sehr mit der
Expedition des Schiffes beschäftigt gewesen war. Hil bat um
Erlaubnis, das große Instrument der Station benutzen zu dürfen. Fru
erklärte sich gern bereit, selbst die Einstellung zu übernehmen.

„Aber du mußt die ganz starke Vergrößerung anwenden“, sagte La
schmeichelnd zu ihrem Vater, „der arme Bat hier möchte einmal
sehen, wo er zu Hause ist.“

„Und die neugierige La auch, nicht wahr? Nun, du weißt, es kommt
alles auf die Beleuchtung an.“

Es gesellten sich noch einige andere Martier hinzu, die ebenfalls
die Gelegenheit wahrnehmen wollten, sich die Erde von ihren
Bewohnern erklären zu lassen.

„Ach“, sagte Saltner leise zu La, „das wird eine große
Gesellschaft, da werden wir wohl nicht viel zu sehen bekommen.“

„Warte nur ab“, antwortete sie ebenso, „das wird gerade hübsch. Du
weißt ja gar nicht, wie man bei uns ins Fernrohr sieht.“

Man sammelte sich vor einer geschlossenen Tür.

„Sie denken vielleicht“, sagte La, „daß bei uns jeder für sich
durch ein Rohr guckt. O nein, das ist viel bequemer.“

Fru öffnete die Tür. Man trat in ein vollständig verdunkeltes
Zimmer, das nur künstlich durch eine Lampe beleuchtet war. Die eine
Wand war rein weiß, alle übrigen schwarz angestrichen. Man
gruppierte sich vor der weißen Wand, im Vordergrund La, Saltner und
Grunthe als Gäste neben ihr. Hinter den Zuschauern befand sich ein
Gestell mit verschiedenen Apparaten und Meßinstrumenten, von
welchem aus schwarz angestrichene Rohre nach der Decke liefen. Hier
stellte sich Fru auf. Das Licht verlosch. Nur die Schrauben und
Skalen der Apparate phosphoreszierten in schwachem Eigenlicht.

Als Fru den Verschluß des Suchers öffnete, projizierte sich auf der
Wand ein Teil des südlichen Sternenhimmels, und nach einigen
Verschiebungen erschien das Bild der Erde, nicht vergrößert, aber
sehr scharf in allen Umrissen. Es nahm fast die ganze Fläche der
Wand ein, und man konnte deutlich die Abnahme der Beleuchtung an
der Schattengrenze beobachten, die jetzt schon etwas weiter nach
Westen gerückt war. Zum Glück zeigte sich der Himmel über
Deutschland ganz klar, so daß Fru nicht zweifelte, die stärkste
Vergrößerung anwenden zu können. Fru ersuchte Grunthe, ihm auf dem
Bild an der Wand die Stelle zu bezeichnen, an welcher ungefähr die
Hauptstadt seines Landes zu suchen sei. Grunthe deutete auf einen
Punkt in Norddeutschland und Fru stellte nun den Projektionsapparat
so ein, daß dieser Punkt genau in die Mitte des Bildes kam. Jetzt
wandte er hundertfache Vergrößerung an, um die Stadt Berlin
erkennen zu lassen. Die Entfernung von der Außenstation bis nach
Berlin betrug 8.600 Kilometer; bei der angewandten Vergrößerung
wurden also die Gegenstände bis auf 86 Kilometer nahegerückt, und
es war somit möglich, Ausdehnungen von etwa hundert Meter Länge zu
unterscheiden und bei besonders heller Beleuchtung auch noch
kleinere. Der Kreis an der Wand, der jetzt freilich sehr viel
lichtschwächer erschien, zeigte sich von bräunlichen und grünlichen
Streifen und Vierecken bedeckt, die an zahlreichen Stellen von
dunkleren, unregelmäßigen Flecken unterbrochen waren; jene waren
die bebauten Felder, diese die dazwischen liegenden Wälder und
Seen.

Grunthe hatte richtig geschätzt. An der rechten Seite des Bildes
waren die ausgedehnten Seen der Havel bei Potsdam unverkennbar,
links erschien noch der Lauf der Oder bei Frankfurt auf dem Bild.
Eine verwaschene Stelle nach rechts unten zeigte die von Rauch
erfüllte Atmosphäre der Millionenstadt an. Diese wurde nun in die
Mitte der Projektion gebracht und nochmals um das Zehnfache
vergrößert. Dadurch rückte die Stadt bis auf kaum neun Kilometer an
den Standpunkt des Beschauers heran. Es war, als ob man sie aus
einem dreitausend Meter über dem Nordende der Stadt schwebenden
Luftballon betrachtete, nur freilich bei einer außerordentlich
matten Beleuchtung. Der auf der Wand abgebildete Kreis umfaßte in
Wirklichkeit einen Durchmesser von zehn Kilometern.

Dem Mangel an Licht, welcher eine Folge der Projektion bei starker
Vergrößerung war, konnten die Martier durch eine ihrer genialen
Erfindungen abhelfen; sie schalteten in den Gang der Lichtstrahlen
ein sogenanntes optisches Relais ein. Die Strahlen passierten dabei
eine Vorrichtung, durch welche sie neue Energie aufnahmen, und zwar
jede Farbengattung genau Licht derselben Art und im Verhältnis
ihrer Helligkeit. Dadurch erhielt das ganze Bild, ohne seinen
Charakter zu verändern, die erforderliche Lichtstärke. Eins aber
konnte freilich nicht entfernt werden – der über der ganzen Stadt
lagernde Dunst und Qualm. Die Felder nördlich von der Stadt und ein
Teil der Vororte waren zu erkennen. Man bemerkte die feinen Linien,
von einem Rauchwölkchen gekrönt, welche die der Hauptstadt
zustrebenden Eisenbahnzüge vorstellten. Das Häusermeer selbst aber
verschwamm in einem grauen Nebel, über den nur die Türme und
Kuppeln der Kirchen hervorragten. Deutlich erkannte man den Reflex
der Sonne an dem Dach des Reichstagsgebäudes und an der
Siegessäule.

Grunthe und Saltner hatten natürlich schon öfter Gelegenheit
gehabt, bei ihren Gesprächen mit den Martiern die wichtigsten
geographischen und politischen Aufklärungen über die Menschen zu
geben. Sie würden noch besseres Verständnis dafür gefunden haben,
wenn nicht die Inselbewohner als Techniker hauptsächlich
mathematisch-naturwissenschaftlich gebildet gewesen wären, so daß
ihre historischen Kenntnisse nur der allgemeinen Bildung der
Martier entsprachen. So wußten diese bloß im allgemeinen zu sagen,
daß ihnen die Einrichtungen der Erde auf dem Standpunkt zu stehen
schienen, den man auf dem Mars als Periode der Kohlenenergie
bezeichnete. Sie lag für die Geschichte der Martier um mehrere
hunderttausend Jahre zurück. Rassen, Staaten und Stände in heißem
Konkurrenzkampf um Lebensunterhalt und Genuß, die ethischen und
ästhetischen Ideale noch nicht rein geschieden von den
theoretischen Bestimmungen, der Energieverbrauch ganz auf das
Pflanzenreich angewiesen, ob diese Energie nun von der
Landwirtschaft aus den lebenden oder von der Industrie aus den
begrabenen Pflanzen, den Kohlen, gezogen wurde.

„Woher kommen diese Nebel über Ihren großen Städten?“ fragte einer
der Martier.

„Hauptsächlich von der Verbrennung der Kohle“, erwiderte Grunthe.

„Aber warum nehmen Sie die Energie nicht direkt von der
Sonnenstrahlung? Sie leben ja vom Kapital statt von den Zinsen.“

„Wir wissen leider noch nicht, wie wir das machen sollen. Übrigens
sind die Kohlen doch nur zurückgelegte Zinsen, die unsere geehrten
Vorfahren im Tierreich nicht aufzehren konnten.“

„Die Wolken sind häßlich, man kann ja nichts deutlich sehen“, sagte
La.

„Ich wünschte“, sprach Hil mehr für sich als zu den andern, „wir
hätten bei uns einen Teil Ihrer Wolken. Welch gewaltige
Wasserbecken haben Sie auf der Erde!“

„Es ist aber hier an der Stadt wirklich nichts zu sehen“, bemerkte
Fru. „Die Luft ist zu unruhig in größerer Höhe über der Stadt, wir
bekommen keine klaren Bilder.“

„Lassen Sie uns einmal meine Heimat schauen“, rief Saltner. „Bitt’
schön! Da ist die Luft klar wie auf dem Mars.“

„Das wollen wir sehen“, sagte La. „Aber Heimweh dürfen Sie nicht
bekommen.“

„Ich will Ihnen sagen, wie Sie reisen müssen. Drehen Sie einmal so,
daß wir nach Westen kommen –“

„Wie weit ist es bis nach Ihrer Heimat?“

„Von Berlin? Nun so siebenhundert Kilometer oder etwas mehr
werden’s wohl sein.“

„Nun, da kommen wir doch rascher zum Ziel, wenn wir erst noch
einmal die hundertfache Vergrößerung nehmen und dann einstellen.
So, jetzt dirigieren Sie. Das Bild faßt nunmehr hundert Kilometer
im Durchmesser.“

„Also westlich bitte – aber nicht zu schnell, sonst erkenn ich
nichts. Das ist Potsdam, nun weiter –. Das ist die Elbe – meinen
Sie nicht, Grunthe? Das dort muß Magdeburg sein – halt! Nun immer
direkt südlich.“

Fru ließ die Karte von Deutschland über die Tafel wandern. Der
Harz, die Hügel- und Waldlandschaften Thüringens und des
fränkischen Jura zogen schnell vorüber, die bayerische Hochebene
beherrschte das Bild.

„Das dort muß München sein, da ist’s schön!“ rief Saltner. „Bitte,
machen Sie einmal groß. Und dann erst weiter, dann kommen die
Alpen.“

Fru stellte den Apparat wieder auf tausendfache Vergrößerung und
schaltete das optische Relais ein. Die Hauptstadt Bayerns zeigte
ihre Kuppeln.

„Jetzt dachte ich doch wirklich einen Augenblick“, rief La, „dort
eine Frau zu erkennen. Aber das müßte ja eine seltsame Riesin
sein.“

„Das ist sie auch“, sagte Saltner lachend. „Es ist die Bildsäule
der Bavaria, die Sie sehen.“

„Bavaria? Wodurch hat sich die Frau so verdient gemacht, daß man
ihr Bildsäulen setzt? Hat sie ein Problem gelöst?“

„Die Bierfrage“, sagte Saltner.

„Die Bildsäule stellt die Personifikation eines unsrer Staaten
vor“, erklärte Grunthe.

„Warum nehmen Sie aber dazu nicht einen Mann?“ fragte La wieder.

„Das hätte Grunthe auch sicher getan, wenn er gefragt worden wäre“,
neckte Saltner.

„Ich denke“, sagte Grunthe, „es ist Zeit weiterzureisen.“

„Nun immer weiter nach Süden!“ rief Saltner.

Die Vorberge der Alpen erschienen im klaren Licht der
Nachmittagssonne. Ein dunkler Bergsee erfüllte die Wand, dahinter
erhoben sich die Spitzen der bayerischen Alpen –

„Der Walchensee!“ rief Saltner.

„Das ist schön – so schön gibt es nichts bei uns –“, sagte La.

„Wartens nur“, rief Saltner, der jetzt alles um sich und beinahe
selbst La vergaß. „Es kommt noch schöner. Nun drehens nur
langsam!“

Es war ein wunderbares Wandelpanorama, das sich jetzt entfaltete.
Je höher die Gebirgswelt anstieg, um so klarer und reiner wurde die
Luft und damit die Schärfe der Bilder. Man betrachtete das Gebirge
aus einer Entfernung von neun Kilometern und unter einem
Neigungswinkel von annähernd zwanzig Grad, also wie aus einer Höhe
von dreitausend Metern, doch so, daß man unter dieser Neigung stets
einen Umkreis von zehn Kilometern Durchmesser vor sich hatte,
entsprechend einem Flächenraum von achtzig Quadratkilometern. So
sah man jetzt gerade den Nordabfall der Karwendelwand vor sich,
aber man blickte darüber hinweg auf die dahinterliegenden
Gebirgsketten. Alles dies erschien im höchsten Grade plastisch,
genau wie ein Relief der Gegend; denn das Fernrohr wirkte durch
seine Konstruktion wie ein Stereoskop.

So schob sich die Gegend nach und nach vor den Blicken der
Zuschauer vorüber, als ob dieselben in einem Luftballon schnell
darüber hinschwebten. Der Einschnitt des Inntals wurde passiert,
und nun leuchteten hell im Sonnenstrahl die Ferner der Ötztaler
Alpen. Fru war bei der Drehung des Fernrohrs nach Westen
abgewichen. Wieder erblickte man den schmalen Streifen eines tief
eingeschnittenen Tales, und dahinter erschien eine herrliche
Berggruppe, alle Gipfel mit glänzendem Weiß bedeckt.

„Was ist denn das“, rief Saltner, „da sind wir von der Richtung
abgekommen. Das ist der Ortler! Nun müssen Sie wieder nach Osten
drehen – so – immer weiter! Sehen Sie, immer an diesem Streifen
hin, das ist nämlich das Etschtal, und jetzt können Sie gerad
hineinschauen, hier schwenkt es nach Südost ab. Noch immer weiter,
bis es ganz nach Süden geht – da – da schaun Sie hin – ah, wie
schade, aus dem Tal steigt die Luft so unruhig in die Höhe, aber
die Etsch können Sie durchschimmern sehn. Und jetzt, ganz langsam,
noch ein bißchen, hier, die Berge am linken Ufer, hier ist’s wieder
klar – nun bitte, halt!“

Er beugte sich ganz dicht vor, daß der Schatten seines Kopfes auf
die Wand fiel und die andern nicht mehr gut sehen konnten.

„Da, da ist’s“, rief er jubelnd, „ich kann’s deutlich erkennen. Das
ist die alte Burg, links daneben liegt das Haus, mein Haus – Jesus
Maria – ich kann’s wahrhaftig sehen, wie ein kleines, weißes
Pünktchen! Da wohnt mein Mutterl.“

Jetzt beugte auch La sich vor.

„Wo?“ fragte sie.

Mit der Spitze einer Nadel bezeichnete Saltner den Punkt.

Ihre Köpfe berührten sich. Lange betrachtete La die Gegend, als
wollte sie sich jede Einzelheit einprägen. Saltner trat beiseite.

„Ich hab nun genug geschaut, mir tun die Augen weh“, sagte er und
zog sich auf einen der Stühle zurück. Er bedeckte die Augen mit der
Hand und saß schweigend. La setzte sich neben ihn und drückte leise
seine Linke.

Nach längerer Pause, während deren Fru die Schattengrenze der Erde
betrachten ließ, die jetzt schon bis an den Ural vorgerückt war,
sagte La zu Saltner: „Du möchtest wohl jetzt den Mars nicht mehr
sehen?“

„Warum nicht?“ entgegnete Saltner. „Ich will ihn auch liebgewinnen
– aber du mußt verzeihen! Es ist ein bissen viel auf einmal, was
jetzt durch meinen dummen Menschenverstand geht.“

„Ja, ihr armen Menschen“, sagte La, „es wird wohl noch ein Weilchen
dauern, eh ich recht begreife, wie es in solchem Kopf aussieht. Die
Heimat liebhaben und die Eltern und die Freunde, das ist gut. Und
was gut ist, wie kann das traurig machen?“

„Wenn man es nicht hat –“

„Nicht hat? Wie kann man das nicht haben, was doch nur vom Willen
abhängt? Wer kann dir die Treue nehmen, die du für recht hältst?
Diese Liebe hast du doch, ob hier oder dort, weil du sie selbst
bist.“

„Aber La, kennt ihr Nume die Sehnsucht nicht?“

„Die Sehnsucht? Siehst du, du törichter Lieber, was wirfst du doch
durcheinander! Also bist du gar nicht gut aus reinem Willen,
sondern dich treibt das Verlangen nach dem Besitz. Und aus diesem
Widerstreit bist du traurig. Oh, was seid ihr für Wilde!“

„So würdest du dich nie nach mir sehnen?“

„Nach dir? Das ist doch ganz etwas anderes. Ich hab dich doch nicht
lieb, weil es Pflicht ist, weil es gut ist, sondern lieb hab ich
dich, weil es schön ist zu lieben und geliebt zu werden. Deine Nähe
wünsche ich, wie ich den Ton des Liedes wünsche, um mich an seiner
Schönheit zu erfreuen – aber nein, das ist auch noch nicht richtig,
du könntest denken, das sei nur ein Mittel zur ästhetischen Lust –
nein, so brauch ich deine Liebe und Nähe, wie der Künstler die
eigne Seele braucht, um das Schöne zu schaffen. – Ach, ich komme
mit eurer Sprache nicht zurecht. Ihr sprecht von Liebe in
hundertfachem Sinn. Ihr liebt Gott und das Vaterland und die Eltern
und die Kinder und die Gattin und die Geliebte und den Freund, ihr
liebt das Gute und das Schöne und das Angenehme, ihr liebt euch
selbst, und das sind doch absolut verschiedene Zustände des Gemüts,
und immer habt ihr nur das eine Wort.“

„Ich will dich ja ohne alle Worte lieben, du kluge La –“

Sie blickte tief in seine Augen und sprach: „Wie nennt ihr das, was
niemals wirklich ist, was man nur in der Phantasie sich vorstellt,
und indem man es sich vorstellt, ist das Glück wirklich in uns? Wie
nennt ihr das?“

Saltner zauderte mit der Antwort, und La fuhr fort: „Und das, was
man wollen muß, ob es auch nicht glücklich macht, und was im Wollen
erfreut, wenn es auch nicht wirklich wird, wie nennt ihr das?“

„Ich glaube“, erwiderte Saltner, „das erste nennen wir schön, und
das zweite gut.“

„Und wenn ihr eine Frau liebt, rechnet ihr das zum Schönen oder zum
Guten?“

Es kam zu keiner Antwort.

„Was ist das?“ hörte man plötzlich Fru laut rufen. Eine Bewegung
entstand bei den Martiern. Sie drängten sich nahe an die Wand und
hefteten ihre Augen auf eine bestimmte Stelle des Bildes, das
soeben vom Fernrohr projiziert wurde.

Grunthe hatte Fru gebeten, ihm die Einrichtung des Apparats zu
erklären. Hierbei hatte Fru die Schrauben hin und her gedreht, das
Bild der Erde war nicht mehr im Gesichtsfeld, zahllose Sterne
liefen infolge der Umdrehung der Erde über den projizierten Teil
des Himmels. Jetzt setzte Fru, weiter demonstrierend, das Uhrwerk
in Gang, welches das Fernrohr der Erdbewegung entgegen drehte, so
daß die Sterne auf dem Bild stillstanden. Fru warf einen Blick auf
den Teil des Himmels, der sich zufällig eingestellt hatte. Es war
ein Stückchen der ›südlichen Krone‹, das sich abbildete. Verwundert
blickte er schärfer hin. Er kannte die Stelle zu genau, als daß ihm
nicht ein Stern hätte auffallen sollen, der sich sonst nicht hier
befand. Einer der Asteroiden konnte es nicht sein. Er änderte die
Einstellung ein wenig und erkannte daran, daß der fragliche Körper
sich in verhältnismäßig großer Nähe befinden müsse.

Dies hatte ihn zu dem lauten Ausruf veranlaßt. Aufmerksam prüften
alle den Lichtpunkt, der sich deutlich von den Bildern der
Fixsterne als eine kleine rötliche Scheibe unterschied.

„Es ist ein Schiff!“ rief endlich einer der Martier.

„Der ›Komet‹?“ fragte Grunthe.

„Das ist nicht möglich“, sagte Fru. „Es ist der ›Glo‹! Kein
Zweifel, er ist an seiner roten Farbe kenntlich, es ist das
Staatsschiff.“

„Die Ablösung!“ hieß es in den Reihen der Martier.

„Und Instruktionen von der Regierung“, rief Fru.

„Wie lange Zeit braucht das Schiff noch bis zur Ankunft?“ fragte
Grunthe.

„Darüber können noch Stunden vergehen. Aber trotzdem muß ich leider
um Entschuldigung bitten, daß ich Ihnen heute den Mars nicht mehr
zeigen kann. Ich hoffe, es wird nächstens Gelegenheit dazu sein.
Denn ich muß sofort die Vorbereitungen zur Landung treffen. Und
deshalb, so leid es mir tut, muß ich auch den Flugwagen früher als
beabsichtigt hinabgehen lassen. Sie müssen also die Güte haben,
sich zur Rückfahrt nach der Insel bereitzuhalten.“

Fru verabschiedete sich herzlich von Grunthe, Saltner und La, und
diese wie die übrigen Martier begaben sich nach der Abfahrtsstelle
der Flugwagen, um auf die Insel zurückzukehren.

\section{17 - Pläne und Sorgen}

Als Saltner am folgenden Morgen in Grunthes Zimmer trat, fand er
diesen bereits eifrig mit Schreiben beschäftigt.

„Schon so fleißig?“ fragte Saltner. „Sie haben wohl noch nicht
einmal gefrühstückt?“

„Nein“, sagte Grunthe, „ich warte auf Sie. Ich habe nicht schlafen
können und unsere Lage nach allen Seiten hin erwogen. Wir haben
Wichtiges zu besprechen.“

Beide pflegten, ohne sich um die martische Sitte des Alleinspeisens
zu bekümmern, ihre Mahlzeiten gemeinschaftlich in ihren
Privatzimmern einzunehmen. Hier bot sich ihnen fast die einzige
Gelegenheit, sich völlig ungestört auszusprechen.

„Nun“, sagte Saltner, nachdem sie sich aus den Automaten die Teller
und Becher gefüllt hatten, die zu ihrer Reiseausrüstung gehörten –
denn es war ihnen bequemer, nach europäischer Art zu speisen –,
„nun, schießen Sie los, Grunthe! Ich höre.“

Grunthe sah sich um, ob die Klappen des Fernsprechers geschlossen
seien. Dann sagte er leise:

„Ich habe die Überzeugung, daß sich unser Schicksal heute
entscheiden wird. Und nach allem, was ich aus den Gesprächen der
Martier entnommen habe, insbesondere gestern bei der Rückfahrt,
erwartet man, daß das Staatsschiff den Befehl mitbringen wird, uns
nach dem Mars zu transportieren.“

„Ich glaube, Sie haben recht“, erwiderte Saltner. „Soweit ich mit
La darüber gesprochen habe, sieht sie es als bestimmt an, daß wir
beide mit nach dem Mars gehen, und wir werden wohl schließlich
einfach dazu gezwungen werden.“

Grunthe sah starr geradeaus. Dann sprach er langsam: „Ich gehe nach
Europa zurück.“

Seine Lippen zogen sich zu einer geraden Linie zusammen. Sein
Entschluß war unabänderlich.

Saltner blickte ihn erstaunt an.

„Na“, sagte er, „ich gebe zu, daß wir alle Kräfte daranzusetzen
haben, unsrer Instruktion nachzukommen, das heißt, nach Auffindung
des Nordpols auf dem kürzesten Wege heimzukehren. Und wenn ich auch
eine Reise nach dem Mars in schöner Gesellschaft nicht so übel
fände, so habe ich doch einen gewissen Horror vor Balancierkünsten
und insbesondere vor diesen furchtbar fetten Speisen – ich denke
noch mit Entsetzen an die flüssige Butter oder was es war, das wir
neulich zum Frühstück erhielten – und bei dem Klima bleibt einem ja
nichts übrig, als früh, mittags und abends ein Pfund Fett zu
verschlingen –“

Grunthe runzelte die Stirn.

„Ja, Ihnen tut das nichts, Sie wissen ja nie, was Sie essen –“, er
klopfte ihn auf die Schulter. „Seien Sie nicht böse, ich kann es
nur nicht leiden, wenn Sie dieses fürchterlich ernste Gesicht
machen. Aber ohne Scherz, was ich sagen wollte, ist dies: Wie
stellen Sie sich denn das vor, gegen den Willen der Martier von
hier fort- oder woanders hinzukommen, als wo man Sie freundlichst
hinkomplimentiert?“

„Der Gewalt muß ich weichen“, erwiderte Grunthe. „Aber verstehen
Sie, nur der Gewalt. Ich werde mich ihr indessen zu entziehen
suchen.“

„Denken Sie die Nume zu überlisten?“

„Ich würde selbst das versuchen, wenn sie wirklich Gewalt
brauchten, denn ich würde dann meinen, mich im Zustand der Notwehr
zu befinden. Aber nach alledem, was ich von ihnen weiß, glaube ich
nicht, daß sie so unwürdig und barbarisch handeln. Sie werden nur
keine Rücksicht auf uns nehmen und uns dadurch in die Lage
versetzen, ihnen freiwillig auf den Mars zu folgen.“

„Wie meinen Sie das?“

„Ich habe mir überlegt, sie werden uns nicht mit Gewalt
einschiffen; das wäre ein Bruch des Gastrechts. Aber sie werden uns
nicht erlauben, länger auf der Insel zu bleiben, als bis dieselbe
für die Wintersaison geräumt wird. Und das kann man ihnen nicht
verdenken, wenn sie uns nicht im Winter hierlassen wollen, während
die Wirte selbst bis auf ein paar Wächter das Haus verlassen. Und
somit werden wir vor die Alternative gestellt werden, entweder mit
nach dem Mars zu ziehen oder die Heimreise mit unzulänglichen
Mitteln bei Beginn des Polarwinters und wahrscheinlich bei widrigen
Winden anzutreten. Und das ist es, was ich Ihnen sagen wollte. Wir
müssen auf diesen Fall vorbereitet sein und genau wissen, was wir
wollen; und ich muß wissen, wie Sie darüber denken. Denn ich bin
überzeugt, daß der heutige Tag nicht ohne Ultimatum vorübergeht.“

„Das ist eine kitzlige Sache, liebster Freund. Unter diesen
Umständen könnte es sicherer sein, auf dem kleinen Umweg über den
Mars nach Berlin oder Friedau zurückzukehren. Nehmen Sie an, wir
kommen glücklich über das Eismeer und geraten nicht in einen der
Ozeane, aber wir gelangen nach Labrador oder Alaska oder nach
Sibirien oder sonst einer dieser lieblichen Sommerfrischen – wenn
wir dann überhaupt wieder herauskommen, so ist doch vor dem Sommer
an keine Heimkehr zu denken; und für den Sommer haben uns die
Martier ja sowieso versprochen, uns wieder herzubringen.“

„Die Gefahren kann ich leider nicht leugnen, aber wir müssen sie
auf uns nehmen. Es ist doch immer die Möglichkeit vorhanden, daß
wir nach Hause kommen oder wenigstens bis zu einem Ort, von welchem
aus wir Nachricht geben können. Und das scheint mir das
Entscheidende. Wir dürfen nichts unterlassen, die Kunde von der
Anwesenheit der Martier am Pol den Regierungen der Kulturstaaten zu
übermitteln, ehe jene selbst in unsern Ländern eintreffen. Man muß
in Europa wie in Amerika vorbereitet sein.“

Saltner nickte nachdenklich. „Wenn wir unsre Brieftauben noch
hätten! Aber die armen Dinger sind alle ertrunken.“

„Sehen Sie“, fuhr Grunthe noch leiser fort, „ich fürchte, wir
können die Sachlage nicht ernst genug nehmen. Wir haben eine
wissenschaftliche Pflicht; in dieser Hinsicht könnte man vielleicht
sagen, daß wir ein Recht hätten, die sicherste Heimkehr zu wählen,
auch daß der Besuch des Mars eine so unerhörte Tat wäre, daß sie
die Übertretung unserer Instruktion entschuldigen könnte, obwohl
sie dies für mein Gewissen nicht tut. – Bitte, lassen Sie mich
aussprechen. Wir haben aber nach meiner Überzeugung außerdem eine
politische und kulturgeschichtliche Pflicht, wenn man so sagen
darf, die uns zwingt, alles daranzusetzen, selbst den geringsten
Umstand auszunutzen, der uns eine Chance bietet, der Ankunft der
Martier zuvorzukommen. Wer garantiert Ihnen, was die Vereinigten
Staaten des Mars beschließen, wenn sie erst im vollen Besitz der
Nachrichten über die Erdbewohner sind? Und selbst, wenn sie uns
Wort halten, durch welche unbekannten Einflüsse können sie uns
nicht verhindern, das zu tun, was für die Menschen das Richtige
wäre? Wenn wir erst zugleich mit ihnen in Europa ankommen, wenn die
Regierungen überrascht werden, ist es vielleicht zu spät, die
geeigneten Maßregeln zu treffen.“

„Ich hätte unsre Stellung nicht für so verantwortlich gehalten“,
sagte Saltner.

„Und ich sage Ihnen“, sprach Grunthe weiter, „nach reiflicher
Überlegung – Sie wissen, daß ich keine Phrasen mache – ist es mir
klar geworden, daß, solange die Menschheit existiert, von dem
Entschluß zweier Menschen noch niemals so viel abgehangen hat wie
von dem unsrigen.“

Saltner fuhr in die Höhe. „Das ist ein großes Wort –“

„Ein ganz bescheidenes. Wir sind durch Zufall in die Lage versetzt
worden, einen Funken zu entdecken, der vielleicht einen Weltbrand
entfacht. Unsere Entscheidung gleicht nicht der des Machthabers,
der über Völkerschicksale bestimmt, sondern der des Soldaten, der
sein Leben aufs Spiel zu setzen hat, um eine wichtige Meldung zur
rechten Zeit zu überbringen. Sie werden mir zugeben, daß es noch
niemals für die zivilisierte Menschheit ein bedeutungsvolleres
Ereignis gegeben hat, als es die Berührung mit den Bewohnern des
Mars sein muß. Die Europäer haben so viele Völker niederer
Zivilisation durch ihr Eindringen vernichtet, daß wir wohl wissen
können, was für uns auf dem Spiel steht, wenn die Martier in Europa
Fuß fassen.“

„So wollen Sie überhaupt verhindern, daß die Martier in Europa
aufgenommen werden?“

„Wenn ich es könnte, würde ich es tun. Aber wir sind einfache
Gelehrte, wir haben keine politischen Entscheidungen zu fällen. Und
eben darum dürfen wir unter keinen Umständen auf eigene Faust den
Martiern die Hand bieten, dürfen nicht mit ihnen zugleich nach
Europa gelangen, sondern wir müssen versuchen, den Großmächten die
Nachricht von dem Bevorstehenden so zeitig zu bringen, daß sie sich
über ihr gemeinsames Vorgehen entschließen können, ehe die
Luftschiffe der Martier über Berlin und Petersburg, über London,
Paris und Washington schweben.“

„Um Gottes willen, Sie sehen die Sache zu tragisch an. Die paar
hundert Martier werden uns nicht gleich zugrunde richten; und wenn
sie uns gefährlich werden, ist es immer noch Zeit, sie wieder
hinauszuwerfen. Aber es ist doch viel wahrscheinlicher, daß wir sie
als Freunde aufnehmen und den unermeßlichen Vorteil ihrer
überlegenen Kultur für uns ausbeuten.“

„Die Frage ist zu schwer, um sie jetzt zu diskutieren, und wir eben
müssen dafür sorgen, daß sie an den entscheidenden Stellen zur
rechten Zeit erwogen werden kann. Nur unterschätzen Sie ja nicht
die Macht der Martier. Denken Sie an Cortez, an Pizarro, die mit
einer Handvoll Abenteurer mächtige Staaten zerstörten. Und was will
die Kultur der Spanier gegenüber den Mexikanern oder Peruanern
bedeuten im Vergleich zu dem Fortschritt von Hunderttausenden von
Jahren, durch welchen die Martier uns überlegen sind? Das eben ist
meine größte Sorge, daß man diese Überlegenheit überall
unterschätzen wird, wenn nicht wir, die wir das abarische Feld und
die Raumschiffe gesehen haben, soviel an uns ist, darüber
Aufklärung verbreiten.“

„Sehen Sie nicht zu schwarz, Grunthe?“

„Ich will es von Herzen hoffen. Aber das sage ich Ihnen als meine
Überzeugung: Mit dem Augenblick, in welchem das erste Luftschiff
der Martier über dem Lustgarten erscheint, ist das deutsche Reich
ein Vasall, der von der Gnade der Martier, vielleicht von der Gnade
irgendeines untergeordneten Kapitäns lebt, und so alle übrigen
Staaten der Erde.“

„Daran habe ich noch nicht gedacht.“

„Was wollen Sie gegen diese Nume tun? Ich will gar nicht von ihrer
moralischen Überlegenheit und ihrer höheren Intelligenz reden;
durch diese werden sie wahrscheinlich Mittel finden, uns nach ihrem
Willen zu lenken, ehe wir es merken. Denken Sie allein an ihre
technische Übermacht.“

„Man wird ihnen ihre Luftschiffe, die übrigens noch gar nicht
fertig sind, einfach mit Granaten entzweischießen, oder man wird
sie auf der Erde, wo sie nur kriechen können, gefangennehmen.“

„Das kann vielleicht mit der ersten Abteilung geschehen, die zu uns
kommt; aber der Mars hat doppelt soviel Bewohner als die ganze
Erde. Das zweite Luftschiff würde uns vernichten. Lieber Saltner,
Sie haben vorgestern gehört, was Jo von der Raumschiffahrt
erzählte. Durch ihre Repulsitschüsse erteilen die Martier einer
Masse, die auf der Erde zehn Millionen Kilogramm wiegt,
Geschwindigkeiten von 30, 40, ja bis 100 Kilometern. Wissen Sie,
was das heißt? Leute, die das können, werden aus Entfernungen,
wohin kein irdisches Geschütz trägt, ganz Berlin in wenigen Minuten
in Trümmer legen, falls sie dies wollen. Die Europäer können dann
einmal erleben, was sie sonst an den Wohnstätten armer Wilden getan
haben. Freilich werden die Martier zu edel dazu sein. Sie hätten es
wohl auch nicht nötig. Sie können die Schwerkraft aufheben. Was
nützt uns die größte, tapferste, glänzend geführte Armee, wenn auf
einmal Bataillone, Schwadronen und Batterien zwanzig, dreißig Meter
in die Luft fliegen und dann wieder herunterfallen? Ich weiß, ich
werde die Regierungen nicht überzeugen, aber die Pflicht habe ich,
unsre Erfahrungen mitzuteilen. Schon die Freundschaft der Martier
halte ich für gefährlich, ihre Feindschaft für verderblich. Kommen
sie vor oder mit uns zu den Menschen, so werden sie dieselben so
für sich einnehmen, daß unsere Warnung, unsere Beschreibung ihrer
Macht zu spät kommt. Deshalb ist mir der Entschluß gereift, daß
unsere Abreise so bald wie möglich vor sich geht. Ich werde sofort
zur Instandsetzung des Ballons schreiten.“

„Es versteht sich von selbst, daß ich Ihnen dabei helfe.“

„Das nehme ich natürlich an. Aber es ist eine andere Frage, Saltner
– es ist vielleicht richtiger, daß ich allein zurückgehe, während
Sie die Studien auf dem Mars fortsetzen.“

„Das ist unmöglich, allein können Sie nicht –“

„Doch, ich kann sogar besser allein zurück. Der Ballon ist kaum
noch für zwei Personen tragfähig. Fahre ich allein, so kann ich
mich auf viel längere Zeit verproviantieren, ich gewinne dadurch an
Wahrscheinlichkeit, bis in bewohnte Gegenden zu gelangen.
Beobachtungen will ich jetzt natürlich nicht mehr machen, also
genügt eine Person vollständig zur Leitung des Ballons. Und
andererseits ist es vielleicht von größter Wichtigkeit zu erfahren,
was die Martier inzwischen vorgenommen haben –“

„Nein, Grunthe, ich kann und will mich nicht von Ihnen trennen.“

„Ich sage Ihnen, es wird das beste sein. Überlegen Sie sich die
Sache. Und nun an die Arbeit.“

Sie räumten unter ihrem Gepäck auf.

Die Klappe des Fernsprechers erklang. Saltner wurde in das
Sprechzimmer gerufen.

„Sehen Sie zu“, rief ihm Grunthe nach, „daß Sie unsern Ballon
herausbekommen. Wie ich bemerkt habe, hat man ihn unter Verschluß
gebracht, was auch ganz vernünftig war. Lassen Sie ihn auf das
Inseldach hinaufschaffen.“

Saltner hatte gestern mit La nicht mehr ungestört sprechen können.
Es war den ganzen Abend über viel Besuch im gemeinsamen Zimmer
gewesen, man erwartete eine Nachricht über die Landung des
Staatsschiffes. Doch hatte man sich trennen müssen, ehe eine solche
eingelaufen war. Daß Se nicht mehr zum Vorschein gekommen war,
hatte Saltner kaum bemerkt. Der Gedanke an La erfüllte ihn ganz,
und dennoch sagte er sich selbst, daß er in seinem Liebesglück nur
einen Traum sehen dürfe, dem jeden Augenblick ein unerwartetes
Erwachen folgen könne. Aber warum nicht träumen?

Diesen Feen gegenüber konnte er, der ›arme Bat‹, gewiß kein Unglück
anrichten, sie würden ihn aufwachen lassen, wann sie wollten. Doch
wie hätte er ihnen widerstehen können?

Es war ihm wie eine Enttäuschung, daß er jetzt nicht La, sondern Se
im Sprechzimmer vorfand. Sie begrüßte ihn mit derselben
Liebenswürdigkeit und Vertraulichkeit wie gestern La, doch aber
wieder anders, ihrem lebhafteren Wesen entsprechend. Und als er
nach den ersten Minuten der Unterhaltung neben ihr saß, zog es ihn
mit so unwiderstehlicher Macht zu ihr hin, daß er sein Gefühl gegen
La gar nicht von dem gegen Se zu unterscheiden wußte. Nur einen
neuen, eigentümlichen Reiz hatte es durch die Veränderung der
Persönlichkeit gewonnen.

Wundersamerweise war es ihm nun gar nicht möglich, nach La zu
fragen, und Se erwähnte ihrer mit keinem Wort. Aber er konnte es
nicht unterlassen, ihr zu sagen, wie glücklich es ihn mache, neben
ihr zu weilen, ihr ins Auge zu sehen und ihre Stimme hören zu
dürfen.

Sie ließ ihn ausreden und antwortete dann mit einem hellen Lachen,
das aber durchaus nichts Beleidigendes für ihn hatte.

„Das freut mich ja sehr“, sagte sie, „daß wir nun so gute Freunde
geworden sind. Sie haben mir gleich von Anfang an gut gefallen. Es
ist merkwürdig, ihr Menschen seid so ganz anders, und doch – oder
vielleicht darum habt ihr etwas, wodurch man sich zu euch
hingezogen fühlt.“

Saltner ergriff ihre Hand.

„Freilich kennt man euch auch noch zu wenig. Vielleicht verdient
ihr gar nicht –“

„Ich hoffe, liebste Freundin, mich werden Sie immer bereit finden,
ihnen zu dienen.“

„Daran zweifle ich gar nicht“, lachte Se, „man weiß nur nicht, ob
Sie nicht einmal vergessen, daß wir Nume doch in vielem anders
denken –“

„Es ist nicht schön, mich sogleich daran zu erinnern, daß ich armer
Mensch es gewagt habe –“

„Sie verstehen mich nicht, Sal, wie könnt’ ich mich überheben
wollen? Nur – doch das führt zu nichts, jetzt auseinanderzusetzen,
was erst erfahren sein will. Ich bin ja auch zu ganz anderem Zweck
hierhergekommen. Obwohl aus wirklicher Freundschaft“, setzte sie
hinzu.

Jetzt erst fiel es Saltner wieder aufs Herz, vor welch wichtiger
Entscheidung er stünde. Er wurde sehr ernst. Er wußte nicht, was er
zuerst sagen sollte.

Se kam ihm zuvor.

„Sie wissen, daß der ›Glo‹ angekommen ist?“ fragte sie.

„Ist er schon gelandet?“

„Diese Nacht. Er bringt wichtige Nachrichten für Sie mit. Und
deshalb bin ich hierhergekommen.“

„Sie wollen mir einen Rat geben, liebe Se? Und Sie werden uns Ihre
Hilfe nicht versagen?“

„Soweit ich darf. Amtlich habe ich nichts erfahren, sonst wäre ich
nicht hier. Aber was jedermann bei uns weiß, darf ich auch Ihnen
sagen. Machen Sie sich darauf gefaßt, daß Sie mit uns nach dem Nu
reisen.“

Saltner schwieg nachdenklich.

„Ich habe so etwas erwartet“, sagte er dann. „Ich bin in einer
fatalen Lage.“

„Sie machen ein erschrecklich böses Gesicht“, sagte Se, indem sie
ihm mit ihrer Hand freundlich über die Stirn strich. „Ich weiß ja
schon, daß Sie sehr gern mit uns kämen und doch Ihren Freund nicht
verlassen wollen. Aber er wird auch mit uns kommen.“

„Das wird er nicht“, platzte Saltner heraus. „Das heißt“, fuhr er
fort, „wenn Sie uns mit Gewalt zwingen –“

„Zwingen? Wie meinen Sie das?“

„Nun, Sie sind die Stärkeren. Sie können uns einfach als Gefangene
auf Ihr Schiff bringen.“

„Können? Ich weiß nicht, ich verstehe Sie nicht recht, liebster
Freund. Man kann doch immer nur das, was nicht Unrecht ist. Ihre
Sprache ist so unklar. Sehen Sie diesen Griff? Sie sagen, ich kann
ihn drehen, und meinen, ich habe die physische Möglichkeit dazu.
Wenn ich aber drehe, so versinkt der Sessel unter Ihnen, und so
kann ich ihn nicht drehen, das heißt, ich kann es nicht wollen.
Diese moralische Möglichkeit oder Unmöglichkeit können Sie auch
nicht anders ausdrücken. Könnte es denn bei Ihnen vorkommen, daß
Sie Menschen aus dem Wasser erretten und ihnen dann das Leben
nehmen? Und die Freiheit, ist das nicht noch schlimmer?“

„Ich weiß nicht“, sagte Saltner, „wie man bei uns verfahren würde,
wenn Europäer auf einer Insel in einem fremden Weltteil, wo noch
keine zivilisierte Macht Fuß gefaßt hat, ein reiches Goldlager
entdeckten und, um dasselbe zu sichern, eine Befestigung anlegten;
wenn dann Kundschafter der Eingeborenen in diese Befestigung
gerieten – ich weiß nicht, ob wir uns nicht das Recht zuschreiben
würden, diese Wilden um unserer eigenen Sicherheit willen an der
Rückkehr zu verhindern. Das scheint mir ungefähr die Lage zwischen
Ihnen und uns. Vielleicht würden wir auch sagen, wir schicken diese
Leute wieder zurück, damit sie uns als Boten und Vermittler dienen;
aber erst führen wir sie nach Europa, damit sie unsere ganze
Machtfülle kennenlernen und ihren heimatlichen Häuptlingen sagen,
daß diese unsern Kanonen nicht würden widerstehen können; und wir
entlassen sie erst, wenn unsre Befestigungen soweit fertig sind,
daß wir von dort aus die ganze Insel beherrschen und wir Herren der
Lage sind.“

Se nickte ernsthaft. „Sie erkennen die Sachlage ganz richtig“,
sagte sie. „Ich glaube, daß wir unser Verhältnis zu Ihnen in der
Tat so auffassen, nur mit dem Unterschied, daß wir diese
Kundschafter nicht gegen ihren Willen festhalten können.“

„Dann ist doch die Sache sehr einfach – wir reisen eben ab.“

„Nein, nein – so einfach ist das nicht. Ich weiß nur nicht, wie ich
es Ihnen klarmachen soll. Sie verstehen unter ›Willen‹ allerlei
Gemütskräfte, die bloß individuelle Triebe sind; diese können wir
bezwingen, gegen diesen Willen können wir Sie festhalten. Zum
Beispiel, ich binde Ihnen mit diesem Schleier wieder die Hände. Nun
wollen Sie fort, weil Sie gern etwas Interessanteres tun möchten,
als hier zu sitzen. Daran kann ich Sie verhindern.“

„Dazu brauchten Sie mich gar nicht zu binden.“

„Oder es entstände draußen ein Lärm, Sie erschrecken plötzlich,
Ihre Sinne verwirren sich, und Sie wollen deshalb fort – daran
hindert Sie dieser Knoten. Nun, wenn Sie in dieser Weise fort
wollen, nur weil es Ihnen lieber ist, heimzukehren als auf den Mars
zu gehen, dann wird man Sie hindern. Wenn aber nicht Ihr
individueller Wille, sondern Ihr sittlicher Wille im Spiel ist,
Ihre freie Selbstbestimmung als Persönlichkeit, oder wie Sie das
nennen, was wir als Numenheit bezeichnen – dann gibt es keine
Macht, die Sie hindern kann.

Sehen Sie, liebster Freund“, fuhr sie fort und löste den Knoten,
den sie im Spiel geschlungen, „das wollte ich Ihnen sagen. Ihr
Wille ist nichts gegen den unsern, nur das Motiv des Willens gilt.
Gibt es eine gemeinsame Bestimmung der sittlichen Würde zwischen
Numen und Menschen, so werden Sie Freiheit haben; gibt es für
Menschen nur Motive der Lust, so werden Sie uns nie widerstehen.
Ich weiß ja nicht, wie Ihr Bate im Grunde seid. Und noch dies.
Glauben Sie niemals, Sal, daß ich an Ihrer Neigung zweifle, aber
vergessen Sie nicht, daß ich eine Nume bin; Liebe darf niemals
unfrei machen. Und daran denken Sie!“

„Ich will“, sagte Saltner. „Aber sehen Sie, das eben ist für uns
Menschen das Schwere und dem einzelnen oft unmöglich, diese
Trennung zu vollziehen, die Ihnen selbstverständlich ist. Unser
Denken vermag nicht immer Neigung und Pflicht auseinanderzuhalten,
oft erscheint die eine im Gewand der andern. Was darf ich um
Ihretwillen tun, was bin ich Ihnen schuldig und was darf ich nicht
mehr tun? Sie Glücklichen haben gelernt, wie Götter ins eigene Herz
zu schauen, wir armen Menschen aber wenden uns in solchen Fällen an
unser Gefühl. Wir nennen es zwar Gewissen, sittliches Gefühl, weil
es das umfaßt, was uns allen als Menschen gemeinsam sein soll. Aber
als Gefühl bleibt es doch immer so eng verwachsen mit dem
Einzelgefühl, daß wir nur zu leicht für Pflicht halten, was im
Grunde Neigung ist; und wenn nicht unsre Neigung, vielleicht die
Neigung, die Gewohnheit unsres Stammes, unsrer Zeitgenossen. Und
wir tun aus bester Absicht das Unrechte. Auch der Indianer folgt
seinem Gewissen, wenn er den Feind skalpiert. Wir irren, weil wir
blind sind.“

„Sie mischen schon wieder einen anderen Irrtum dazwischen, Sal.
Nicht darauf kommt es an, ob wir das Richtige treffen, sondern
darauf, ob wir aus den richtigen Motiven wollen. Wer das kann,
besitzt Numenheit. Wenn der Indianer den Feind skalpiert, so wird
er von der höheren Gesittung eines Besseren belehrt oder
vernichtet. Aber dies trifft nur seinen Irrtum, nämlich die Folgen,
die daraus in der Welt entstehen. Doch die Heiligkeit seines
Willens bleibt unberührt, wenn er lieber zugrunde geht, als das
aufgibt, was er für sittliche Pflicht hält. Sie brauchen also nicht
darum zu sorgen, ob Sie bei Ihrer Entscheidung das Richtige treffen
in dem, was Sie tun, sondern nur, ob Ihr Motiv rein ist in dem, was
Sie wollen.“

„Das meinte ich ja; eben auch darin können wir uns täuschen. Se,
ich muß Ihnen gegenüber ganz offen sein. Wir wollen, daß unsere
Mitmenschen von dem Besuch der Martier nicht überrascht werden;
diese Überraschung zu verhüten, halten wir für unsere Pflicht. Wir
irren vielleicht darin, daß wir den Menschen damit zu nützen
glauben; aber unser Motiv ist rein. Meinen Sie es nicht auch so?“

„Ganz richtig.“

„Aber damit ist es nicht entschieden, wie ich zu handeln habe. Und
hier spielt unsere theoretische Unwissenheit in die ethische Frage
hinein. Wenn nun zum Beispiel einer von uns allein den Erfolg
leichter erreichte, hätten wir nicht die Pflicht uns zu trennen?
Und wenn nicht, ist es nicht Pflicht, daß wir zusammenhalten auf
alle Fälle? Wie also soll ich hier entscheiden, was meine Pflicht
erfordert?“

„Aber Sal! Ich hatte mich schon gefreut, daß Sie auch so vernünftig
reden können, und nun urteilen Sie wieder wie ein Wilder!“

„Sie sind grausam, Se!“

„Was reden Sie denn da von Pflicht? Das ist doch einzig eine Frage
der Klugheit. Was Ihre Klugheit erfordert, das können Sie fragen.
Die Pflichtfrage ist schon längst mit dem Willen entschieden, nur
das Klügste hier zu tun. Die dürfen Sie gar nicht mehr in Betracht
ziehen.“

„Wenn ich mit Ihnen nach dem Mars ginge und mein Freund allein nach
Europa, und er verunglückte unterwegs, würde ich mir nicht immer
Vorwürfe machen, daß ich nicht mit ihm gegangen bin? Würde man mich
nicht pflichtvergessen nennen?“

„Was die Menschen tun würden, weiß ich nicht und geht mich auch
nichts an. Sie aber können sich höchstens den Vorwurf machen,
unklug gehandelt zu haben.“

„Also meinen Sie, ich müßte ihn begleiten?“

„Das habe ich nicht gesagt. Ich habe nur unter Ihrer Voraussetzung
gesprochen, daß er mit Ihnen sicherer reise. Das ist aber doch erst
zu untersuchen.“

„Was raten Sie mir?“

„Zunächst die Entscheidung der Martier abzuwarten. Sie wissen ja
noch gar nicht, ob Ihnen die Mittel zur Abreise gewährt werden
können. Erst wenn Sie diese Mittel kennen, vermögen Sie zu
entscheiden, ob Ihre Begleitung entbehrlich ist. Und wenn sie
entbehrlich ist, so würde ich mich sehr freuen, Sie mit zu uns zu
nehmen.“

„Ich rechne auf Ihre Hilfe. Lassen sie unsern Ballon auf das innere
Inseldach schaffen!“

„Das geht nicht, bevor Sie die Erlaubnis der Regierung haben –“

„Und die Ihrige würde ich erhalten? Ich meine, Sie würden mich
nicht für unwürdig Ihrer Freundschaft halten, wenn ich Ihrem Wunsch
nicht entspräche, nach dem Mars –“

„Was habe ich Ihnen gesagt, Saltner? Das wäre keine Liebe, die
unfrei machte.“

„Se, wie glücklich machen Sie mich!“ Saltner ergriff zärtlich ihre
Hände.

„Jetzt sind Sie wieder der alte Saltner! Kaum ist die Angst von ihm
genommen, ich könnte ihm böse werden, wenn er etwas Vernünftiges
tut, so ist er wieder seelenvergnügt. Und ich habe wirklich
geglaubt, Sie wären so ernsthaft, weil es sich um Ihre Pflicht
handelt –“

„Das ist nicht Ihr Ernst, Se, Sie kennen mich besser!“

„Gar nicht kennt man euch Menschen! Wozu denn überhaupt erst
traurig? Was wollen Sie übrigens über dem Strich?“

„Sehen Sie, Se, Sie sind auch nicht vollkommen – ich meine, nicht
so absolut vollkommen –“

„Ich begreife!“

„Sie haben gar nicht gemerkt, daß ich schon eine Viertelstunde lang
neben Ihnen sitze – ich habe gestern das Balancieren gründlich
gelernt.“

„Ach, gestern! Bei La?“

„Ja, sagen Sie, was ist das? Wo ist sie heute? Wo waren Sie
gestern? Was ist das mit dem Spiel, von dem Sie sprachen? Ich bitte
Sie, Se –“

Aber seine weiteren Fragen wurden abgeschnitten. Ra, der Leiter der
Station, trat in das Zimmer. Er hatte eine amtliche Mitteilung zu
machen. Der Regierungskommissar, welcher mit dem ›Glo‹ angekommen
war, ließ Grunthe und Saltner zu einer offiziellen Konferenz
bitten, um drei Uhr. Er würde sich vorher beehren, den Herrn seine
private Aufwartung zu machen.

Saltner erklärte sich natürlich bereit. Er werde sofort seinen
Freund benachrichtigen. Schnell verabschiedete er sich von Ra und
Se.

„Ein ganz ehrliches Spiel!“ flüsterte Se ihm zu, als sie ihm die
Hand zum Abschied reichte. „Und nun Kopf oben! Einschüchtern
brauchen Sie sich nicht zu lassen!“

Eilig teilte Saltner das Wesentlichste aus seiner Unterredung mit
Se Grunthe mit und benachrichtigte ihn von dem bevorstehenden
Besuch.

Kaum hatte Grunthe Zeit gefunden, seine Toilette einigermaßen in
Ordnung zu bringen, als auch die Deutschen schon gebeten wurden,
sich im Empfangszimmer einzufinden. Fast gleichzeitig mit ihnen
trat der Kommissar, von Ra geleitet, ein.

Seine Persönlichkeit machte auf Grunthe und Saltner einen tiefen
Eindruck. Er war größer als alle Martier, die sie bisher gesehen
hatten, und überragte sogar um ein weniges noch die lange Gestalt
Grunthes. Ein stattlicher weißer Bart gab ihm ein ehrwürdiges
Aussehen. Seiner Haltung und seinem Blick war zu entnehmen, daß man
es mit einem vornehmen Mann zu tun hatte, der gewohnt war, sowohl
zu repräsentieren als zu dirigieren. Aber aus seinen großen dunklen
Augen sprach ein Vertrauen erweckendes Wohlwollen, man war
überzeugt, daß dieser Mann bei seinen Anordnungen niemals an sich
selbst dachte, sondern nur an das Wohl derer, die er zu vertreten
hatte.

Ill, dies war sein Name, zeigte sich bis in alle Einzelheiten über
die bisherigen Vorgänge auf der Insel unterrichtet. Er bat um
Entschuldigung, daß er sich seiner Muttersprache bedienen müsse und
erkundigte sich in der liebenswürdigsten Weise nach dem
persönlichen Wohlergehen der Gäste. Insbesondere sprach er in
warmen Worten sein Bedauern über das Verschwinden des Leiters der
Expedition aus. Es schien ihm unbegreiflich, daß man keine weiteren
Spuren von Torm gefunden habe, und er meinte, daß das Binnenmeer
und womöglich seine Umgebung noch einmal genauer durchsucht werden
müsse. Er kam dann auf die Methode zu sprechen, wie sich die
Deutschen das Martische angeeignet hätten, und nun flocht er einige
sehr interessierte Fragen nach Ell ein, wie alt er sei, woher er
stamme, wie Grunthe ihn kennengelernt habe, wo er jetzt lebe.

Grunthe antwortete ausführlich, soweit er vermochte. Ell mochte
etwa gleichaltrig mit ihm sein, einige dreißig Jahre. Er sei in
Südaustralien geboren, wo Ells Vater große Besitzungen gehabt habe.
Seine Mutter sei eine in Australien eingewanderte Deutsche gewesen.
Nach dem Tod der Eltern habe sich Ell nach Deutschland begeben, um
seine Studien, die sich hauptsächlich auf Astronomie und technische
Fächer bezogen, fortzusetzen. Damals, vor etwa zehn Jahren, habe
ihn Grunthe in Berlin kennengelernt und viel mit ihm verkehrt,
obwohl Ell stets ein fremdartiges und zurückhaltendes Wesen eigen
war. Kurze Zeit darauf war Ell plötzlich verschwunden, man hörte
nichts von ihm und nahm an, er sei in seine australische Heimat
zurückgekehrt. So verhielt es sich auch. Seit etwa vier Jahren war
Ell wieder in Deutschland erschienen. Er hatte sein jedenfalls
bedeutendes Vermögen flüssig gemacht und sich in Mitteldeutschland
eine Privatsternwarte erbaut, auf der er sich mit Vorliebe
Marsbeobachtungen widmete. Hier hatte Grunthe eine Zeitlang bei ihm
gearbeitet und bei dieser Gelegenheit Torm kennengelernt. Ell war
es gewesen, der durch eine großartige Geldspende die Errichtung der
Abteilung für wissenschaftliche Luftschiffahrt ermöglicht und Torm
an ihre Spitze gezogen hatte. Der Sitz derselben war Friedau, eine
mitteldeutsche Residenz, die durch ihre wissenschaftlichen
Institute berühmt ist.

Nachdem sich Ill noch die Lage von Friedau und die der
Privatsternwarte Ells genau hatte beschreiben lassen, brach er das
Gespräch ab. irgendwelche Fragen nach den bevorstehenden
Ereignissen wurden nicht berührt, und Ill verabschiedete sich bald
mit dem Wunsch, daß die Verhandlungen, zu denen er die Herren
erwartete, zur beiderseitigen Befriedigung verlaufen möchten.

Nach dem Fortgang der Martier zogen sich Grunthe und Saltner in
ihre Zimmer zurück und besprachen noch einmal die Sachlage; Grunthe
brachte ihre Ansichten zu Papier. Beide aber sahen jetzt der
Verhandlung mit besserer Zuversicht entgegen.

\section{18 - Die Botschaft der Marsstaaten}

Punkt drei Uhr öffnete sich die Tür, die das Zimmer der Gäste mit
dem Konferenzsaal verband, und der Vorsteher Ra lud Grunthe und
Saltner mit einer höflichen Handbewegung zum Eintreten ein. Sie
stutzten beim ersten Anblick des Saales, denn derselbe erschien
vollständig verändert. Um Platz zu gewinnen, hatte man die Grenze
der Schwere bis dicht an die Tür gerückt, durch welche die Menschen
den Saal betraten, und die Tafel in der Mitte entsprechend
verlängert, so daß nur die beiden Plätze am untern Ende des
Tisches, die sich aber jetzt nahe der Tür befanden, noch innerhalb
des Gebietes der Erdschwere lagen. Der ganze übrige Teil des Raumes
war von festlich gekleideten Martiern erfüllt, die sich beim
Eintritt der Gäste erhoben. Nachdem Ra an seinen Sessel am oberen
Ende der Tafel neben dem Präsidenten Ill gelangt war, gab dieser
ein Zeichen mit der Hand, und alle nahmen wieder schweigend Platz.
Grunthe und Saltner folgten ihrem Beispiel.

Durch die geöffneten Fernsprechklappen des Saales ertönte eine
leise Musik, wie sie die Menschen noch nie vernommen hatten. Sie
bewirkte eine feierliche, aber zugleich freudig erhebende Stimmung.
Es herrschte vollständige Ruhe, während deren Grunthe und Saltner
die Versammlung erwartungsvoll musterten.

Das Tageslicht war durch dichte Vorhänge abgeschlossen. Die sehr
helle, aber für menschliche Augen zu stark ins Bläuliche
schimmernde Beleuchtung ging von der Decke aus, deren Arabesken in
fluoreszierendem Schein glühten. Am Ende des Zimmers war das große
Banner des Mars in selbstleuchtenden Farben entfaltet. Es zeigte
auf schwarzem Grund den Planeten als eine weiße Scheibe, die in der
Mitte einen Kranz trug; bei näherer Betrachtung konnte man darin
die Symbole der 154 Staaten des Mars unterscheiden. Vor dem Banner,
an der Spitze der Tafel saß zwischen den beiden ersten Beamten Ra
und Fru der Kommissar der Marsstaaten Ill, an den Seiten reihten
sich die Vorsteher der einzelnen Abteilungen der Station an.
Seitlich von der Haupttafel, in der Mitte des Zimmers, war ein
phonographischer Apparat aufgestellt, der von einer Dame bedient
wurde. Auf der andern Seite saßen La und eine zweite Martierin vor
ihren Schreibmaschinen als Schriftführerinnen. Der übrige Raum des
Zimmers war dicht von Martiern und Martierinnen erfüllt, die der
öffentlichen Verhandlung beiwohnen wollten. Auch Se befand sich
unter ihnen und hatte sich in der Nähe Saltners niedergelassen, der
ihr einen dankbaren Blick zuwarf. Das Lächeln, mit welchem Saltner
anfänglich die Versammlung überflog, verschwand bald unter dem
Eindruck der Musik und der Haltung der schweigenden Martier. Alle
trugen heute über ihrer anschließenden metallisch glänzenden
Rüstung einen leichten, in malerischen Falten geworfenen Mantel.
Ihre Blicke waren ruhig und ernst, aber erfüllt von einem freudigen
Stolz; sie fühlten sich als die freien Mitglieder ihrer großen und
mächtigen Gemeinschaft, die sie zum ersten Mal den Menschen in
ihrem festlichen Glanz zeigten. Sie wußten, daß sie heute nicht nur
als Wirte ihren Gästen, sondern als Vertreter der Numenheit den
Männern gegenüberstanden, die für sie die Vertreter der Menschheit
waren. Und dieses Bewußtsein, das den ganzen Charakter der
Versammlung beherrschte, wirkte sehr bald auf Grunthe und Saltner
zurück; sie fühlten, wie sie der übermächtigen Gegenwart der
Martier in ihrem Willen zu erliegen drohten. Grunthe preßte die
Lippen zusammen und starrte auf sein Notizbuch, das er krampfhaft
in der Hand hielt, um sich dem Einfluß zu entziehen, den das
Äußerliche der Versammlung auf ihn machte.

Nur wenige Minuten hatte die musikalische Einleitung gedauert.
Jetzt erhob sich Ill. Absolute Stille herrschte im Saal, als er
seine großen, strahlenden Augen auf die Versammlung richtete und
dann wie in weite Ferne blickte. Darauf sprach er klangvoll die
einfachen Worte:

„Den wir im Herzen tragen, Herr des Gesetzes, gib uns deine
Freiheit.“

Wieder erfolgte eine Pause, in welcher jeder mit sich selbst
beschäftigt war.

Jetzt ließ sich Ill auf seinem Stuhl nieder und begann:

„Gesandt bin ich, Grüße zu bringen den Numen von der Heimat, Grüße
vom Nu und seinem Bund!“

„Sila Nu!“ hallte der gedämpfte Gegengruß der Martier durch den
Saal.

„Grüße vom Nu auch den Bewohnern der leuchtenden Ba, des
benachbarten Planeten, den Menschen, die wir zum ersten Mal heute
in der Festversammlung zu sehen uns freuen. Eine alte Sehnsucht zog
uns Nume durch den Weltraum hinüber zum lichten Abendstern, und es
gelang uns Fuß zu fassen auf der Erde. Aber noch immer war es uns
versagt, diejenigen kennenzulernen, die diesen mächtigen Planeten
beherrschen als vernünftige Wesen. Da kam zu uns vor wenigen Wochen
die erste frohe Kunde, daß zwei willkommene Gäste unserer Station
am Pol genaht, daß die ersten zivilisierten Bewohner der Erde
entdeckt seien. Ausführliche Lichtdepeschen meldeten uns bald, was
wir bisher wohl vermutet, aber doch aus direkter Anschauung nicht
gekannt hatten, daß unser Nachbarstern bewohnt ist von
hochgebildeten Völkern, mit denen wir uns verständigen können in
den Aufgaben der Kultur. Eine unbeschreibliche Aufregung ging auf
diese Nachricht durch die verbündeten Staaten des Mars. Die
öffentliche Meinung drang darauf, keine Zeit zu verlieren, unsern
Brüdern auf der Erde die Hand zu reichen. Und da der Winter auf
diesem Nordpol bevorsteht, der unsre Verbindung unterbricht, so
beschloß der Zentralrat des Nu, ohne die Ankunft der Raumschiffe
abzuwarten, sich in direkten Verkehr mit den Bürgern der Erde zu
setzen. Wir schätzen es von unermeßlicher Wichtigkeit für die
beiden Planeten, welche allein im ganzen Sonnensystem in der Art
und der Kultur ihrer Bewohner sich berühren, daß diese in
gemeinsamem Einverständnis ihre Interessen fördern. Das erste
Zusammentreffen mit den hier anwesenden Vertretern der Menschheit
halten wir daher für einen Akt von höchster kulturgeschichtlicher
Bedeutung. Wir sehen darin den ersten Schritt zum unmittelbaren
Verkehr mit den Regierungen der

keiten trennen, die wir indessen bald zu überwinden hoffen. Erde,
von denen uns gegenwärtig noch technische Schwierig In gerechter
Würdigung der Wichtigkeit dieser ersten Begegnung und um bei dieser
Gelegenheit zugleich zu zeigen, welch hohen Wert die Marsstaaten
auf die freundschaftlichen Beziehungen mit den Staaten der Erde
legen, endlich um von seiten der Nume in feierlicher Handlung die
ganze Menschheit bei der ersten Begrüßung zu ehren, hat der
Zentralrat beschlossen, eines seiner Mitglieder in eigener Person
auf die Erde zu senden.“

Eine allgemeine Bewegung gab sich bei diesen Worten unter den
Zuhörern zu erkennen. Man sah sich erwartungsvoll an, leise Fragen
flogen herüber und hinüber. Grunthe warf Saltner einen Blick zu,
und dieser flüsterte: „Sie behalten recht.“ Er blickte nach Se
hinüber, aber ihre Augen waren auf Ill gerichtet. Dieser erhob
langsam und feierlich die rechte Hand und sprach:

„Kraft des Amtes, das der Wille der Nume mir übertragen hat,
enthülle ich das heilige Symbol der Numenheit als das Zeichen des
Gesetzes in Vernunft und Arbeit, dem wir gehorchen.“

Die Martier erhoben ihre Augen in andächtigem Aufblick nach einem
Punkt, den Ills Hand ihnen zu weisen schien. Vergebens strengten
Grunthe und Saltner sich an, das zu erblicken, was alle andern
ehrfurchtsvoll erschauten. Sie vermochten nichts wahrzunehmen, wo
die Wissenden in würdevollem Schweigen einer geheimnisvollen
Erscheinung huldigten, die ihnen den Gedanken ihres Weltbürgertums
repräsentierte.

Der Schauer des Unbegreiflichen erfaßte das Gemüt der Menschen.
Grunthe starrte auf die ehrwürdige Gestalt, und wieder kam die
Erinnerung an Ell über ihn. Saltner fühlte sich von dem Eindruck
der ganzen Szene wie berauscht, er merkte, daß er die Gewalt über
seine Entschlüsse verlieren würde, und richtete einen
hilfesuchenden Blick auf Se.

Da ließ Ill seine Hand sinken, und die Martier begannen wieder sich
zu bewegen. Nach kurzer Pause hob Ill ein Schriftstück in die Höhe
und begann:

„Vernehmen Sie, Nume und Menschen, den Beschluß des Zentralrats.“

Jetzt blitzte Ses Auge zu Saltner hinüber. Instinktiv verstand er
die Mahnung. Er stieß Grunthe an und flüsterte: „Reden Sie, ehe er
liest!“

Aber auch dieser hatte schon begriffen, daß er sofort handeln
müsse, und war bereits aufgesprungen. Alles dies vollzog sich
momentan in der kurzen Pause, während deren Ill das Schriftstück
entfaltete, und ehe er zu lesen begann, rief Grunthe: „Ich bitte
ums Wort!“

Er hatte in der Erregung deutsch gesprochen. Seine laute Stimme
tönte grell über den Saal, im Gegensatz zu dem auch in der
feierlichen Rede halblauten Organ der Martier. Die ganze
Versammlung wandte sich unwillig nach Grunthe um, und Ill warf
einen erstaunten Blick auf ihn.

„Ich bitte ums Wort“, wiederholte Grunthe jetzt in der Sprache der
Martier. „Ich bitte um Verzeihung, wenn ich Sie ersuche, mich vor
der Verlesung des Beschlusses eines hohen Zentralrats der
Marsstaaten zu hören, und ich bitte im voraus um Verzeihung, wenn
ich aus Unkenntnis der Sprache mich vielleicht nicht völlig
angemessen auszudrücken vermag.“

Ill nickte langsam mit dem Haupt. „Es liegt kein Grund vor“, sagte
er, „unsern Gästen das Wort zu verweigern, wenn ich auch Ihre
Antwort erst nach der Verlesung erwartet habe.“

„Ich aber und mein Freund“, fiel Grunthe schnell ein, „wir
beantragen, die Verlesung zu unterlassen; wir protestieren gegen
die Verlesung; wir fühlen uns nicht als kompetent, Beschlüsse des
Zentralrats der Marsstaaten entgegenzunehmen.“

Auf den Gesichtern der Martier malte sich deutlich das Erstaunen
über diese unerwartete Erklärung. Es herrschte ein bedeutsames
Schweigen. Keinerlei Urteil machte sich geltend. Die Mißbilligung
des kühnen Eingriffs, welchen ein armseliger Bat sich gegen die
Beschlüsse der höchsten Behörde des Mars erlaubte, stritt bei den
Martiern mit der Achtung vor der Entschiedenheit dieses offenen
Bekenntnisses, doch überwog bei den meisten ein Gefühl des
Mitleids. Diese armen Menschen wußten offenbar nicht, was sie sich
erlaubten; man konnte sie wohl nicht ernst nehmen. Nur die nächsten
Freunde der Deutschen ermutigten sie durch ihre beipflichtenden
Blicke.

Ill richtete sein ruhiges Auge auf Grunthe und Saltner, der sich
ebenfalls erhoben hatte, und fragte:

„Wollen die Menschen ihren Protest begründen?“

„Ich will es“, sagte Grunthe sofort. „Ich fühle tief die große
Ehre, welche die Vertreter des Mars durch ihr freundliches
Entgegenkommen den Bewohnern der Erde erweisen. Auch ich bin
überzeugt, daß die Berührung der Bewohner dieser beiden großen
Kulturplaneten ein weltgeschichtliches Ereignis ersten Ranges sein
wird. Und mein Freund und ich sind allen Numen, denen wir bisher zu
begegnen das Glück hatten, den herzlichsten Dank schuldig für die
Rettung vom Untergang und für die gastfreundliche Aufnahme in ihrer
Kolonie. Wir werden das nie vergessen.“

„Niemals“, sagte hier Saltner dazwischen.

Bei diesen warm gesprochenen Worten wurden die Blicke der Martier
freundlicher. Grunthe fuhr sogleich fort:

„Als Menschen sprechen wir auch unsern ehrerbietigen Dank der
Regierung der Vereinigten Staaten des Mars aus für die Beachtung,
welche sie den Mitgliedern der Tormschen Polarexpedition zuteil
werden läßt, indem sie durch ihren Repräsentanten in eigener Person
uns eine Botschaft entbieten will. Aber diese Ehre müssen wir
ablehnen.

Wir sind nicht Vertreter irgendeiner Regierung. Wir haben kein
Recht, diplomatische Erklärungen entgegenzunehmen oder abzugeben.
Wir sind einfache Privatleute, die in ihrer Heimat keine andere
Geltung haben, als ihr Ruf als Gelehrter ihnen verschafft, und
diese ist nach den Sitten unsrer Heimat in politischer Hinsicht
verschwindend. Und selbst wenn wir uns als Boten betrachten
wollten, die ihrer Regierung eine Mitteilung zu überbringen hätten,
so habe ich zu betonen, daß, wie dem Herrn Repräsentanten bekannt
sein wird, außer dem Deutschen Reich noch fünf andre europäische
Großmächte, außerdem die Vereinigten Staaten von Nordamerika die
politische Macht über die Erde in Händen haben, daß wir demnach
nicht in der Lage sind, für die Staaten der Erde Aufträge zu
übernehmen.“

Hierauf sprach Ill, da Grunthe eine kleine Pause machte, mit
unveränderter Höflichkeit, aber sehr überlegen:

„Die Worte unseres werten Gastes sagen uns nichts Neues. Sie haben
keinen Einfluß auf die mitzuteilende Botschaft, und es wäre daher
einfacher gewesen, dieselbe erst anzuhören, da sie sich allein auf
die beiden hier anwesenden Personen unserer Gäste bezieht.“

Grunthe biß die Lippen aufeinander. Er ärgerte sich über die
Zurechtweisung, zumal er auf den Gesichtern der Martier wieder das
mitleidige Lächeln erscheinen sah. Er rief daher etwas erregter:

„Wir müssen es aber auch für unsre Personen ablehnen, irgendwelche
Bestimmungen seitens der Regierung des Mars entgegenzunehmen, und
zwar aus formellen Gründen. Wir dürfen es prinzipiell nicht
geschehen lassen, daß die Regierung des Mars hier irgendwelche
offizielle Anordnungen treffe über die Bürger eines Staates der
Erde. Über unser Tun und Lassen kann nur diejenige Regierung
Verordnungen geben, auf deren Gebiet wir uns befinden. Wir stehen
aber hier auf der Erde, nicht auf dem Mars. Und wenn Sie hier die
Flagge der Marsstaaten entfaltet haben, so können wir derselben
doch nur eine dekorative, aber keine staatsrechtliche Bedeutung
zusprechen. Mit welchem Recht Sie hier eine Niederlassung begründet
haben, darüber mögen die Regierungen der Erde bestimmen, es ist
nicht unseres Amtes; aber unseres Amtes ist es, dagegen zu
protestieren, daß auf Grund dieser noch nicht anerkannten
Niederlassung Rechte über uns ausgeübt werden.“

„Kann mir der Herr Redner vielleicht sagen“, fiel Ill ein, „auf dem
Gebiet welches Erdenstaates wir uns seiner Ansicht nach hier
befinden?“

Das war eine heikle Frage. War der Nordpol schon von einer
zivilisierten Macht in Besitz genommen? Grunthe wich der Frage aus,
er sagte schnell:

„Jedenfalls nicht im Gebiet der Marsstaaten. Auf der Erde gibt es
bis jetzt keine völkerrechtlich anerkannte Ansiedlung der
Martier.“

Die Blicke der Martier waren drohend geworden. Ill richtete sich
hoch auf und sprach mit leuchtenden Augen und erhobener Stimme:

„Meines Wissens gibt es keine Organisation der Staaten der Erde,
mit welcher wir über den Besitz des Nordpols verhandeln könnten,
oder wenigstens war eine solche Verhandlung bisher nicht möglich.
Wir sind an dieser Stelle des Sonnensystems die ersten Ankömmlinge
gewesen, wir also bestimmen über dieselbe. Es gibt kein
interplanetarisches Recht, wonach die Besitzergreifung von Gebieten
sich auf einen einzelnen Planeten beschränken müsse. Die Nume sind
die einzigen Wesen, welche zwischen den Planeten verkehren; sie
schaffen damit das Recht dieses Verkehrs. Kraft dieses Rechtes hat
die Regierung der Marsstaaten Besitz von diesem Teil der Erde
ergriffen. Kraft dessen gilt hier das Gesetz des Mars. Und kraft
dieses Gesetzes und des Beschlusses des Zentralrats vom 603. Tag
des Jahres 311770 werde ich hiermit den Beschluß vom gleichen Tag
verkünden.“

Grunthe fühlte, wie ihm das Herz pochte. Er vermochte nichts zu
erwidern. Die Menschen waren geschlagen, ihr erster Versuch der
Opposition gegen die Übermacht der Martier war gescheitert. Sie
mußten die Befehle der Regierung des Mars anhören, auf ihrem
eigenen Planeten, an der Stelle, welche sie zuerst von den Menschen
erreicht hatten. Und das Schlimmste war, daß beide, Grunthe wie
Saltner, ihre Widerstandskraft erlahmen fühlten. Gegen diesen
Willen, der aus den großen Augensternen des Repräsentanten
leuchtete, der sich in den Blicken der ganzen Versammlung
widerspiegelte, vermochten sie nicht aufzukommen.

Und schon begann Ill, die kurzen Worte vorzulesen, welche über ihr
Schicksal bestimmen sollten. Er las:

„Der Zentralrat des Nu, im Namen der Vereinigten Staaten des Mars,
hat beschlossen, wie folgt: Die beiden an der Station des Mars auf
dem Nordpol der Erde angelangten Menschen, namens Grunthe und
Saltner, stehen unter dem Schutz der Marsstaaten. Die Freiheit
ihrer Person, ihres Verkehrs und Eigentums wird ihnen gewährleistet
im gesamten Gebiet des Mars. Sie werden eingeladen, innerhalb sechs
Tagen nach Verlesung dieser Botschaft auf einem der Raumschiffe der
Erdstation sich nach dem Mars zu begeben. Sie sind Gäste der
Marsstaaten, denen jede Förderung zuteil werden soll, Einrichtungen
und Gesinnungen der Nume zu studieren. Sie werden ersucht, im
Frühjahr der Nordhalbkugel der Erde nach derselben zurückzukehren,
um alsdann eine nach den Hauptstädten der Erde aufbrechende
Expedition zu begleiten. Der Repräsentant Ill wird mit der
Überbringung dieser Botschaft nach der Erde beauftragt.

Gezeichnet Del. Em. An.“

Die Martier ließen sich auf ihren Sitzen nieder, auch Grunthe und
Saltner sanken in ihre Sessel.

\section{19 - Die Freiheit des Willens}

Nach der Verlesung der Botschaft faltete Ill das Dokument zusammen
und sprach mit liebenswürdigster Miene:

„Nachdem die Menschen den Willen des Zentralrats vernommen haben,
darf ich annehmen, daß sie der Einladung und dem Ersuchen der
Martier Folge leisten werden. Ich bitte Sie daher, Ihre
Vorbereitungen so treffen zu wollen, daß Sie mit dem am fünften Tag
von heute abgehenden Schiff Ihre Reise antreten können.“

Da weder Grunthe noch Saltner sogleich antwortete, erhob sich Ra
und hielt eine versöhnliche Rede. Aus dem Inhalt der Botschaft,
führte er aus, würden sich die Gäste gewiß überzeugt haben, daß sie
gar keinen Grund hätten, gegen die Verlesung zu protestieren. Er
wüßte wohl, daß man ihnen mit der Reise nach dem Mars ein
ungewöhnliches und anstrengendes Unternehmen zumute. Er verstünde,
daß sie es vorziehen würden, alsbald in ihre Heimat zurückzukehren.
Dies – und damit deckte er offen ihre Motive auf – wäre wohl auch
der eigentliche Grund des Protestes gewesen, da die Menschen die
Einladung nach dem Mars erwartet und sich der Verlegenheit hätten
entziehen wollen, sie abzulehnen. Und dann stellte er ihnen die
Reise und den Aufenthalt auf dem Mars in verlockenden Farben vor.

Grunthe und Saltner wußten nicht recht, ob sie diese Rede zu ihren
Gunsten deuten dürften, da sie die Schwäche ihres Protestes
enthüllte und ganz geeignet schien, ihnen die Ablehnung zu
erschweren. Aber Saltner erkannte an dem stillen Lächeln in Ses
Zügen, daß Ra ihnen tatsächlich zu Hilfe kommen wollte, daß er sie
wohl nur warnen wollte, neue Fehler zu begehen. In der Tat schloß
er mit den Worten:

„Der Zentralrat garantiert Ihnen volle Freiheit. Er kommandiert Sie
nicht nach dem Mars, er lädt Sie ein; er befiehlt nicht, daß Sie
uns nach Europa geleiten sollen, er ersucht Sie darum. Er setzt
dabei voraus, daß es keine berechtigten ethischen Motive gibt,
weshalb Sie diesen Wünschen nicht nachkommen sollten, und er
erwartet daher, daß Sie ihnen Folge leisten.“

Während Grunthe finster vor sich hinblickte und darüber nachsann,
in welche Form er seine Weigerung kleiden sollte, erhob sich
Saltner. Obwohl er sich sagte, daß er mit seinen Worten den
Entschluß der Martier nicht würde ändern können, wollte er doch
versuchen, etwas Näheres über ihre Pläne zu hören, und die
Ablehnung der Einladung aus Zweckmäßigkeitsgründen motivieren. Er
legte dar, daß der Besuch auf dem Mars gegenwärtig für beide Teile
keine besonderen Vorteile biete. Sein Freund und er hätten bereits
vollständig die Überzeugung von der Macht und Leistungsfähigkeit
der Martier gewonnen. Was sie vom Mars wüßten, wäre schon so viel,
daß sie Mühe haben würden, es ihren Mitbürgern begreiflich zu
machen. Es wäre daher sicherlich das beste, wenn sie sogleich in
ihre Heimat zurückkehrten, um den Erdbewohnern ihre Erfahrungen
mitzuteilen und sie durch die Presse allmählich auf das Erscheinen
der Martier vorzubereiten. Das gegenseitige Verständnis zwischen
Mars und Erde würde auf diese Weise am sichersten gefördert; die
Überraschung durch die Bewohner des Mars könnte die Erdbewohner,
bei ihrer mangelhaften Kenntnis der Verhältnisse auf dem Mars,
vielleicht zu falschen Maßregeln verleiten, unter denen alsdann
beide Teile zu leiden hätten. Deswegen müßten sie darauf dringen,
nach Europa zurückzukehren, ehe die Martier dahin kämen. Sie zu
begleiten, könnte für die Martier jedenfalls von viel geringerem
Nutzen sein. Im übrigen wäre es ihnen, den Menschen, vom größten
Interesse, zu erfahren, welche Vorteile eigentlich die Martier sich
vom Verkehr mit der Erde versprächen und was sie etwa von den
Menschen zu erlangen wünschten.

Die Martier hatten unter wachsender Aufmerksamkeit zugehört. Ills
Antlitz war wieder ernster geworden. Nachdem er die Mitteilung des
Zentralratsbeschlusses durchgesetzt, hatte er geglaubt, daß die
Menschen nicht länger wagen würden, sich zu weigern. Aus Saltners
Worten erkannte er jedoch, daß es keinen Sinn mehr hätte, den
eigentlichen Kernpunkt der Frage zu verschleiern. Die Deutschen
hatten offenbar die Absicht der Martier durchschaut, eine Warnung
der Großmächte zu verhindern. Der Hilfe der Menschen bedurften die
Martier nicht; aber sie wollten bei dem ersten Besuch in den
zivilisierten Staaten der Erde sogleich in einer Weise auftreten,
die sie zum unbedingten Herren der Situation machte. Die
Vorbereitungen dazu waren schon in viel höherem Maß getroffen, als
Grunthe und Saltner wußten. Ihre Landung am Nordpol und die
Kenntnis, welche die Martier dadurch von den zivilisierten Staaten
der Erde erhielten, hatte den Zentralrat nur in der Ansicht
bestärkt, daß man mit den Bewohnern der Erde in sehr ernsthafter
Weise zu rechnen haben würde und daß alles darauf ankäme, sich bei
der ersten Begegnung keine Blöße zu geben. Dies wäre aber sehr
leicht möglich gewesen, wenn die Erdbewohner zu früh erfuhren, mit
welchen Schwierigkeiten die Martier auf der Erde zu kämpfen hatten.
Diese zu heben war daher ihr Hauptaugenmerk bei den Vorbereitungen
zur Expedition und zugleich der Grund ihrer langen Verzögerung
gewesen. Nun hatte der Zentralrat beschlossen, die Vorbereitungen
aufs äußerste zu beschleunigen, ehe die Besitznahme des Nordpols
auf der Erde bekannt wurde, und vorläufig die Rückkehr der Menschen
zu verhindern. Doch konnte er sich dazu nach der sittlichen
Weltanschauung der Martier keiner Mittel bedienen, die das Recht
der Persönlichkeit der Menschen verletzt hätten.

Es wäre unter der Würde der Martier gewesen, wenn sie sich hinter
Vorwänden hätten verstecken wollen, nachdem der Versuch, die
Menschen durch bloße Autorität zu leiten, gescheitert war. Ill
sagte daher:

„Es ist allerdings unsre Absicht, den Erdstaaten unsre Ankunft
nicht eher bekanntwerden zu lassen, als bis dieselbe wirklich
erfolgt. Und zwar aus demselben Grund, welcher unsere Gäste
wünschen läßt, das Entgegengesetzte herbeizuführen und die
Erdstaaten vorzubereiten. Wir fürchten, daß gerade die lückenhaften
Nachrichten, welche sie durch die hier anwesenden Menschen erhalten
würden, sie dazu veranlassen könnten, falsche Maßregeln zu
ergreifen und unser gegenseitiges Verständnis zu erschweren. Denn
wenn Sie auch, meine Herren Gäste, mancherlei von unserer äußeren
Macht kennengelernt haben, so kennen Sie doch noch zu wenig die
Grundsätze unsres Handelns, um Ihre Freunde belehren zu können, wie
sie sich gegen uns zu verhalten haben. Die traurigsten
Mißverständnisse sind leicht möglich. So müssen wir denn darauf
bestehen, daß Sie uns zuerst nach dem Mars begleiten, da wir,
unmittelbar vor Beginn des Polarwinters, noch nicht in der Lage
sind, mit Ihnen zusammen nach Europa aufzubrechen.“

„Ich bin dem Herrn Repräsentanten sehr dankbar“, erwiderte Saltner,
„daß er uns so offen die Gründe des hohen Zentralrats für seine
Botschaft dargelegt hat. Sie konnten uns aber nicht überzeugen, um
so weniger, da wir über die eigentlichen Absichten der Martier
gegen die Erdbewohner nicht näher unterrichtet wurden. Wir müssen
daher darauf bestehen, nach der Heimat zurückzukehren, um den
Unsrigen Gelegenheit zu geben, sich ihrerseits schlüssig zu machen,
wie sie den Martiern zu begegnen haben.“

Ill entgegnete ziemlich scharf.

„Nach dem, was wir soeben gehört haben“, sagte er, „scheinen uns
die anwesenden Menschen wenig geeignet, ihren Landsleuten als
Berater zu dienen, wie sich letztere gegen uns verhalten sollen.
Wenn Sie ihnen vielleicht zu raten gedenken, unserm Aufenthalt auf
der Erde Schwierigkeiten entgegenzusetzen, so würden Sie eben das
erreichen, was wir zu vermeiden hoffen, Mißtrauen und Spannungen
zwischen den Bewohnern beider Planeten, während wir ein friedliches
Verhältnis zu gemeinsamer Arbeit anstreben. Die Menschen haben von
uns nichts zu befürchten, sobald sie gelernt haben werden, uns zu
verstehen. Wir bedürfen der Erdbewohner nicht; wir kommen zu ihnen,
um ihnen die Segnungen unsrer Kultur zu bringen. Ich bin überzeugt,
daß auch wir im Eintausch der Produkte der Erde viel Neues und
Nützliches gewinnen werden. Aber das wirtschaftliche Bedürfnis
welches uns außer dem allgemeinen wissenschaftlichen Interesse nach
der Erde trieb, erfordert nicht die Beteiligung der Menschen. Wir
können es vollauf hier am Nordpol befriedigen, und ich stehe nicht
an, es Ihnen zu sagen, was wir von der Erde holen wollen, damit Sie
Ihre Mitbürger und Regierungen über unsre Absichten beruhigen. Wir
wollen nichts anderes als Luft und Sonne, atmosphärische Luft und
Strahlung, die Sie ja in ausreichendem Maß besitzen und die niemand
gehört. Wir haben sie bereits reichlich exportiert und werden sie
weiter exportieren.

Was uns aber nun veranlaßt, die Menschen selbst aufzusuchen, das
sind Beweggründe rein idealen Charakters. Es ist nicht möglich, sie
Ihnen, als Menschen, hier in Kürze zum Verständnis zu bringen. Wir
sind Nume. Wir sind die Träger der Kultur des Sonnensystems. Es ist
uns eine heilige Pflicht, das Resultat unsrer
hunderttausendjährigen Kulturarbeit, den Segen der Numenheit, auch
den Menschen zugänglich zu machen.“

Grunthe machte eine ungeduldige Bewegung. Er wollte sprechen, aber
Ill fuhr fort:

„Fürchten Sie nichts für Ihre Überzeugung und ihre Freiheit. Ihre
Freiheit werden wir achten, denn sie ist die Grundbedingung zur
Numenheit. Die Kultur kann nicht aufgedrängt und nicht geschenkt
werden, denn sie will erarbeitet sein. Aber zu dieser Arbeit kann
man erzogen werden. So war es auch auf Ihrem Planeten; die
vorgeschrittenen Nationen haben die barbarischen zur Kulturarbeit
erzogen. Dazu bieten wir nun vermöge unsrer so viel älteren
Erfahrung uns Ihnen als Lehrer an. Weisen Sie uns nicht in falschem
Stolz zurück. Nachdem einmal die Erde von uns betreten ist, läßt
sich die Berührung der beiden Planetengeschlechter nicht vermeiden.
Sie ist eine Notwendigkeit. Erwecken Sie also nicht erst die
Täuschung, als könnte die Menschheit unsrem Einfluß sich entziehen.
Vertrauen Sie unsern Maßregeln und bewahren Sie die Menschen vor
dem Fehler, uns aufgrund kurzsichtiger menschlicher Überlegungen
Schwierigkeiten zu bereiten, die nur zum Nachteil für sie
ausschlagen könnten. Erfahren die Menschen von unserer Ankunft,
ohne zugleich dem vollen Gewicht unsres unmittelbaren Einflusses
ausgesetzt zu sein, so begehen sie sicherlich eine Torheit. Auch
Ihr Rat, meine Herren Gäste, würde sie nicht davor bewahren, zumal
Sie uns selbst Ihre Einflußlosigkeit eingestanden. Überlassen Sie
uns also ganz allein die Verantwortung für die Gestaltung der
Verhältnisse, indem Sie sich dem entschieden ausgesprochenen Wunsch
des Zentralrats fügen.“

Grunthe fühlte aufs neue, daß er der Macht dieser Gründe zu
unterliegen drohte. Hatte er sich zunächst aufgebäumt gegen die
stolze Sprache des Martiers, so mußte er sich jetzt doch fragen, ob
er nicht durch eine Warnung das Schicksal der Menschen nur
verschlimmern würde. Was konnten sie gegen die Martier tun? Ihnen
feindlich begegnen? Es wäre ja wohl das Klügste gewesen, sich der
Verantwortung zu entziehen und den Martiern zu folgen. Aber nein!
Das Klügste hatte er nicht zu tun, sondern seine Pflicht. Und es
war ihm kein Zweifel, daß er die Verantwortung nicht übernehmen
durfte, sein Vaterland ohne Nachricht zu lassen.

Er erhob sich in tiefem Ernst. Er sah weder Ill noch die Martier
an, sondern heftete sein Auge vor sich auf den Tisch. Seine Lippen
zogen sich fest zusammen. Dann öffnete er sie mit einem festen
Entschluß. Er warf einen Blick auf Saltner. Auch dieser hatte in
sich verloren mit ähnlichen Gedanken gesessen. Als Grunthe ihn
ansah, sagte er leise: „Ablehnen.“

Grunthe begann. Erst stockend und leise. Allmählich hob sich seine
Stimme.

„Wir sind als Menschen nicht so eingebildet“, sagte er, „daß wir
glauben, von einer älteren Kultur nicht lernen zu können. Es kann
ein hohes Glück sein, den Martiern zu folgen. Es kann auch unser
Unglück sein. Ich wage darüber nicht zu entscheiden. Und eben
darum, weil ich nicht darüber entscheiden kann, darf ich, soviel an
mir liegt, nicht zugeben, daß mein Verhalten einer Entscheidung
gleichkommt; die Menschen, die Erdbewohner, müssen sich eine
Meinung bilden können. Dies zu ermöglichen, ist meine Pflicht.
Dadurch ist meinem Freund und mir unsere Handlungsweise klar und
deutlich vorgeschrieben. Unsre Instruktion lautet dahin, nach
Erreichung des Nordpols so schnell als möglich nach Hause
zurückzukehren. Schon dies verbietet uns, auf Ihre Aufforderung
einzugehen. Doch es könnten Zweifel entstehen, ob nicht unser
kürzester Weg über den Mars führe. Diese Zweifel erledigen sich nun
durch unsere gegenseitige Aussprache. Sie wollen uns nicht vor
Ihrer eigenen Ankunft bei den Unseren heimkehren lassen. Das müssen
wir verhüten. Es ist keine Frage der Klugheit, es ist eine Frage
des Gewissens. Mag daraus entstehen, was da wolle, wir müssen unsre
ganze Kraft und unser Leben einsetzen, um die Nachricht von der
Ankunft der Martier auf der Erde sofort in die Heimat zu bringen.
Dies erfordert die Pflicht gegen das Vaterland und gegen die
Menschheit. Jedes weitere Wort ist überflüssig. Mein Freund und ich
werden mit Hilfe unsres von Ihnen geborgenen Ballons sobald als
möglich abreisen. Wenn Sie wirklich jene erhabene Gesinnung der
Nume besitzen, nach der die Freiheit der Persönlichkeit unbedingte
Achtung erfordert, so erwarte ich von Ihnen, daß Sie uns Ihre
Beihilfe zu unsrer Abreise nicht versagen. Wir bitten, uns zu
entlassen.“

Grunthe und Saltner, der sich ebenfalls erhoben hatte, verließen
ihre Plätze und wandten sich nach der Tür.

Tiefes Schweigen herrschte in der Versammlung der Martier. Die
meisten blickten finster vor sich hin, nur die näheren Freunde der
Menschen zeigten ihnen durch ihre Mienen, daß sie ihr Verhalten
billigten. Saltner sah im Fortgehen, daß ihm Se freundlich mit den
Augen folgte, während er von La vergeblich noch einen Blick zu
erhaschen suchte. Schon hatte Grunthe die Tür geöffnet. Niemand
hielt die beiden auf. Sie verließen den Saal.

Die Martier setzten ihre Beratung fort. Sie waren in ihrer
Majorität sichtlich durch den Mißerfolg verstimmt, ja es wurden
Stimmen laut, ob man die Menschen nicht auch gegen ihren Willen zur
Reise nach dem Mars zwingen könne. Der junge Kapitän Oß warf die
Frage auf, ob nicht den Menschen das Recht der Persönlichkeit
abzusprechen sei, da sie nicht das genügende Verständnis für das
Wesen der Numenheit gezeigt hätten. La blickte ihn sehr erstaunt
an, und Fru erhob sich darauf, um diesen Vorwurf zurückzuweisen.
Daß sie die Fähigkeit gehabt hatten, ihren Willen gegen den der
Martier zu behaupten, sei der genügende und allerdings einzig
mögliche Beweis dafür, daß ihnen die Selbstbestimmung der
sittlichen Person zukomme. Man könne sie also nicht zur Mitreise
zwingen, ja man müsse sogar ihrer Abreise jetzt jede Unterstützung
angedeihen lassen.

Ill entschied dahin, daß die Frage nach dem Recht der Menschen auf
freie Entschließung nicht mehr zur Diskussion stehen könne, da der
Zentralrat ihnen dasselbe bereits zugesichert habe. Dagegen brauche
man nicht soweit zu gehen, ihre Rückreise geradezu zu fördern, wenn
man sie auch nicht verhindern könne. Man müsse aber wohl oder übel
sich damit abfinden, daß die Menschen von der Anwesenheit der
Martier früher erführen, als die ursprüngliche Absicht war.
Andrerseits jedoch läge ihm auch sehr viel daran, wenigstens einen
der Menschen nach dem Mars mitzunehmen, damit dieser den Martiern
später als Augenzeuge dienen könne. Dies könne indessen nur mit
seiner freien Einwilligung geschehen. Dazu bemerkte Ra, vielleicht
würde sich Saltner zur Mitreise bereit erklären, wenn man dafür
Grunthe die vollständige Sicherheit der Heimkehr gewährleisten
könne. Aber eine solche Garantie könne man doch wohl nicht
übernehmen.

Ill sagte darauf nach kurzem Besinnen:

„Ich glaube die Gewähr übernehmen zu können, Grunthe nach Europa zu
bringen, und zwar, wenn es sein müßte, binnen vierundzwanzig
Stunden.“

Bei der Mehrzahl der Martier erweckte diese Äußerung das
lebhafteste Erstaunen. Wie konnte man Grunthe die Rückkehr
garantieren? Hätte man dann nicht selbst sogleich nach Europa
aufbrechen können?

Ill ließ sich zunächst überzeugen, daß Grunthe und Saltner von der
weiteren Verhandlung nichts vernehmen könnten. Sie hatten sich
bereits an die Arbeit an ihrem Ballon gemacht und befanden sich auf
dem Dach der Insel, wo sie genügenden Raum hatten, um den Ballon
einer Untersuchung zu unterziehen. Man hatte ihnen denselben ohne
weiteres zur Verfügung gestellt, da auch die im Dienst befindlichen
Beamten durch den Fernhörer von dem Resultat der Versammlung
bereits unterrichtet waren.

Ill ließ nun die Klappen der Fernsprecher schließen und den
phonographischen Apparat außer Tätigkeit setzen. Gespannt lauschten
die Martier den näheren Mitteilungen, welche ihnen Ill jetzt über
die Fortschritte machte, die in der Vorbereitung der Expedition
nach Europa geglückt waren. Sie hatten bisher von den mit Ill auf
dem ›Glo‹ angekommenen Martiern nur im allgemeinen gehört, daß auf
dem Mars neue wichtige Entdeckungen in bezug auf die Luftschiffahrt
gelungen seien. Die schleunige Absendung des ›Glo‹ hatte
vornehmlich den Zweck, diese neuen Entdeckungen und Apparate in der
Atmosphäre der Erde, für welche sie berechnet und konstruiert
waren, praktisch zu erproben, um alsdann bis zum Frühjahr den Bau
zahlreicher Luftschiffe für die Erde auszuführen. Die Überbringung
der Botschaft des Zentralrats war mit dem Transport dieser neuen
Apparate verbunden worden. Andernfalls hätte man sich
wahrscheinlich damit begnügt, sie durch den Lichttelegraphen zu
übermitteln oder die Ankunft des nächsten Raumschiffs von der Erde
vor der Absendung abzuwarten. Aber die letzten Tage des
Sonnenscheins am Nordpol mußten ausgenutzt werden, um Erfahrungen
über die Brauchbarkeit der neuen Erfindung zu machen. Ill gab nun
Aufklärungen über seine weiteren Absichten. Daran schloß sich eine
längere Beratung der Martier, so daß die Feierstunde herangekommen
war, als die Martier auseinandergingen.

\tb

Grunthe und Saltner kehrten sehr entmutigt von ihrer Tagesarbeit
zurück. Die Untersuchung des Ballons hatte ergeben, daß er in
seiner ursprünglichen Gestalt nicht wieder herstellbar sei.
Glücklicherweise waren die Ventile und das Netzwerk unverletzt. Vom
Stoff des Ballons war jedoch ein großer Teil unbrauchbar geworden.
Der Rest konnte indessen ausreichen, einen kleineren Ballon
zusammenzunähen, vorausgesetzt, daß die Martier bei dieser Arbeit
ihre Hilfe leisten wollten, denn die beiden Gelehrten allein hätten
damit nicht zustande kommen können. Aber die Tragkraft dieses
Ballons, bei dem man Proviant und Ballast sehr reichlich mitnehmen
mußte, um auf eine lange Fahrt gerüstet zu sein, hätte dann nicht
ausgereicht, um beide Forscher aufzunehmen. Grunthe kam deshalb
wieder auf seinen Plan zurück, allein abzureisen und Saltner die
Fahrt nach dem Mars mitmachen zu lassen. Vielleicht, so meinte er,
würden die Martier ihnen ihre Hilfe bei der Herstellung des Ballons
nicht versagen, wenn sie ihnen insoweit entgegenkämen, daß
wenigstens einer von ihnen ihre Einladung nach dem Mars
nachträglich annähme. Endlich dürfe man die Chance nicht aus der
Hand geben, daß, wenn der Ballon verunglücke, wenigstens Saltner
seine Erfahrungen auf dem Umweg über den Mars nach Europa bringe,
wiewohl dies dann freilich nicht vor Ankunft der Martier geschehen
könne.

Saltner überzeugte sich schließlich, daß dieser Ausweg in der Tat
der vorteilhafteste sei, da unter den gegebenen Verhältnissen ein
Luftschiffer die Fahrt sicherer zurücklegen könne als zwei.
Persönlich war er ja überhaupt nicht abgeneigt, die Martier zu
begleiten. Auch Grunthe wäre, was seinen Forschereifer anbetraf,
gern nach dem Mars gegangen, aber einer von ihnen mußte notwendig
als Bote nach Europa. Freilich hatte sich auch Saltner zu dieser
gefährlichen Fahrt erboten, aber es verstand sich von selbst, daß
Grunthe, als der erfahrenere Luftschiffer, die Fahrt unternahm. So
beschlossen denn beide, am nächsten Morgen mit den Martiern in
diesem Sinn zu verhandeln. Für heute war die Verkehrsstunde schon
vorüber.

\section{20 - Das neue Luftschiff}

Grunthe erwachte aus einem unruhigen Schlummer und sah nach der
Uhr. Sie zeigte auf 9,6, das entsprach nach mitteleuropäischer Zeit
ein Uhr früh; es war also noch mitten in der konventionellen
Nacht.

Er legte sich daher wieder auf sein Lager zurück. Während er sich
seinen Gedanken hingab, vernahm er ein eigentümliches Zischen. Es
unterschied sich deutlich von dem leichten, gleichmäßigen Rauschen
des Meeres, das in den Schlafräumen nur schwach durch die Stille
der Nacht hörbar war. Auch schien es aus der Luft herzukommen, nahm
erst zu, um dann allmählich schwächer zu werden und schließlich zu
verschwinden. Nach einiger Zeit begann das Zischen wieder, kam aber
deutlich von einer andern Seite her. Sollten es Windstöße sein, die
sich um die Insel erhoben? Aber auf diese Weise hätten sie sich
wohl nicht geäußert. Als sich das Geräusch mehrfach wiederholte,
stand Grunthe auf, und die Läden der Decke schoben sich, als sein
Fuß den Boden berührte, an einer Stelle automatisch beiseite. Ein
schräger rötlicher Sonnenstrahl schlich sich in das Zimmer, und ein
Streifen des Himmels wurde sichtbar. Es war also noch immer klares
Wetter, nur stand die Sonne bereits so tief, daß sie nur schwach
durch die Atmosphäre hindurchdrang. Plötzlich verdunkelte sich der
sichtbare Streifen des Himmels auf einen Moment, es war, als ob ein
großer Gegenstand mit namhafter Geschwindigkeit über die Insel
fortgeflogen wäre. Zugleich war das Zischen besonders laut
geworden.

Da das Zimmer keine seitlichen Fenster hatte, konnte Grunthe keinen
Rundblick gewinnen. Er wußte aber, daß man an einigen Stellen die
Hartglasbedachung der Decke öffnen konnte. Nur mußte man dazu die
genügende Höhe erreichen, um bis zur Decke zu gelangen. Eine Leiter
hatte er nicht zur Verfügung, er wollte deshalb zunächst versuchen,
ob er nicht durch die Fenster des Sprechzimmers eine genügende
Aussicht finden könne. Zu seiner Überraschung fand er die
Verbindungstür von außen gesperrt. Dies ließ darauf schließen, daß
bei den Martiern etwas im Werk sei, wobei sie von den Menschen
nicht beobachtet zu werden wünschten. Um so mehr steigerte sich bei
Grunthe das Verlangen, seine Wißbegier zu befriedigen.

Er betrachtete sorgfältig die Decke in der Nähe der Luken und
erkannte, daß sich dort verschiedene, zu den Apparaten der Martier
gehörige Haken befanden, an denen man sehr gut Stricke befestigen
konnte. Solche waren zur Genüge an den Körben vorhanden, die zur
Ausrüstung des Ballons gedient hatten und in seinem Zimmer
lagerten. Aus einem der leeren Körbe und zwei Seilen ließ sich eine
Art schwebendes Trapez herstellen, das, an der Decke angehängt,
gestatten mußte, den Kopf bis über das Dach zu erheben. Aber wie
hinaufkommen? Er entschloß sich, Saltner zu wecken. Der räsonnierte
eben ein wenig über die nächtliche Störung, als sich das Zischen in
der Entfernung wieder hören ließ. Nun sprang er mit einem Satz in
die Höhe und fuhr in seine Kleider. Auf Grunthes Schultern stehend,
gelang es ihm, zwei Seile an der Decke zu befestigen, und nun war
es nicht mehr schwer, einen Beobachtungsposten einzurichten.

Vorsichtig steckten die beiden indiskreten Beobachter ihre Köpfe
aus der Luke und wandten sich nach der Richtung, in welcher sich
jetzt deutlich, aber in der Ferne, ein gleichmäßiges leises Sausen
vernehmen ließ. Zu ihrem grenzenlosen Erstaunen sahen sie, daß
dieses Geräusch von einem riesigen Vogel herzurühren schien, der
mit ausgebreiteten Schwingen in ruhigem Segelflug durch die Luft
glitt und in geringer Höhe über dem Wasser rings um die Insel
schwebte. Jetzt näherte er sich derselben und schoß mit rasender
Geschwindigkeit vielleicht zwanzig Meter über dem Dach der Insel
hinweg. Trotz der kurzen Zeit, in welcher die beiden Männer den
seltsamen Vogel beobachten konnten, sahen sie doch, daß er weder
Kopf noch Füße besaß. Sein langgestreckter Körper hatte die Gestalt
einer nach vorn und hinten konisch zulaufenden Zigarre, am hinteren
Ende befand sich ein langer, flacher Schwanz als Steuerruder.
Natürlich war den Beobachtern sofort klar, daß sie eine neue
Erfindung der Martier vor sich hatten, ein den Verhältnissen der
Erde angepaßtes Luftschiff. Die Martier stellten damit Übungen und
Versuche an, wobei sie von den Menschen nicht beobachtet sein
wollten.

Das Luftschiff entfernte sich, kehrte dann in einem kurzen,
eleganten Bogen um, brauste zurück und hielt plötzlich direkt über
der Insel an. Man konnte beobachten, wie der ganze Schiffskörper in
Schwingungen geriet, als der äußerst schnelle Flug binnen drei
Sekunden zum Stillstand kam. Und nun geschah etwas noch
Merkwürdigeres. Die Flügel und das Steuerruder waren plötzlich
verschwunden. Etwa zehn Meter über dem Dach der Insel, aber so weit
vom Standpunkt der beiden Deutschen entfernt, daß ihre eben nur aus
der Luke hervorblickenden Köpfe kaum bemerkt werden konnten,
schwebte der Schiffskörper frei in der Luft. Seine Länge mochte
etwa zehn, sein Durchmesser gegen vier Meter betragen. Das Material
zeigte dasselbe glasartige Aussehen wie die Raumschiffe der
Martier, gestattete aber keine Durchsicht. Den Boden wie das
Verdeck bildeten zwei glatte, nach oben und unten gewölbte Schalen,
zwischen denen ein nur vorn und hinten geschlossener, etwa
meterhoher Streifen freiblieb. Durch denselben konnte man
beobachten, daß das Luftboot von zwölf Martiern bemannt war.

Jetzt senkte sich das Boot auf das Dach der Insel langsam herab, wo
es ohne Verankerung liegenblieb. Die Besatzung stieg aus, und
andere Martier traten an ihre Stelle. Nur die beiden Männer, die an
den beiden Enden des Bootes sich befunden hatten, nahmen ihre
Plätze wieder ein. Sie waren Grunthe und Saltner unbekannt, und
diese schlossen daher, daß es die mit dem ›Glo‹ angekommenen
Konstrukteure des neuen Luftschiffes seien, die hier die Martier
mit der Behandlung des Bootes bekannt machten. Die Galerien der
Insel waren von zuschauenden Martiern besetzt, doch konnte man
diese von dem tiefen Standpunkt Grunthes und Saltners aus nicht
erblicken; auch der untere Teil des Luftschiffes blieb ihnen
verborgen, und sie konnten die Bemannung nur in dem Augenblick
sehen, in welchem sie das Schiff verließ oder betrat, was durch das
Verdeck desselben zu geschehen schien.

Ein neues Manöver begann. Ohne Flügel und Steuer, horizontal
liegend, stieg das Boot mit zunehmender Geschwindigkeit senkrecht
in die Höhe. Da war kein Luftballon sichtbar, keine Schraube, kein
Flügelschlag hob es. In wenigen Minuten war es so hoch gestiegen,
daß es dem bloßen Auge nur als ein Pünktchen mit Mühe wahrnehmbar
erschien. Plötzlich vergrößerte sich der Punkt schnell. Das Schiff
stürzte herab. Aber jetzt entfaltete es seine Flügel und sein
Steuer, und wie ein riesiger Raubvogel sauste es in weitem Kreis um
die Insel, streifte fast an der Meeresoberfläche hin und erhob sich
dann wieder in einer Spirale. Dabei wurden offenbar Signale mit den
Martiern der Insel gewechselt, die aber für Grunthe nicht
verständlich waren. Man sah nun, daß das Schiff seine Flügel
verkürzte oder zurücklegte, das Steuer stellte sich gerade, eine
weiße Dampfwolke brach aus dem Hinterteil des Schiffes hervor, der
ein kanonenschußartiger Knall und ein gewaltiges Brausen folgte –
das Boot schoß schräg aufwärts steigend wie aus einem Geschütz
geschleudert in die Ferne und war nach weniger als einer Minute in
der Richtung des zehnten Meridians dem Auge entschwunden.

Aus der Bewegung, welche sich jetzt auf der Insel bemerkbar machte,
schlossen Grunthe und Saltner, daß das Schiff eine Fernfahrt
angetreten habe und fürs nächste nicht wieder zu erwarten sei. Sie
verließen daher ihren unbequemen Posten und zogen sich in ihr
Zimmer zurück, jedoch entschlossen, die Rückkehr des Schiffes zu
erwarten. Zu diesem Zweck wollten sie sich im Wachen ablösen.

Diese Mühe hätten sie sich freilich sparen können, wenn sie gewußt
hätten, wie weit das Schiff seine Aufklärungsfahrt ausdehnen
sollte. Es wurde erst in der folgenden Nacht von den Martiern
zurückerwartet.

Die Versuche der Martier waren vollständig gelungen. Ihre auf dem
Mars in Berücksichtigung der terrestrischen Verhältnisse
ausgeführten Konstruktionen bewährten sich in überraschender Weise.
Sie waren nun im Besitz eines Luftschiffs, welches sie nach
Belieben in der Erdatmosphäre lenken konnten und mit welchem sie
selbst einem Sturm zu widerstehen vermochten. Was die Menschen so
lange vergeblich angestrebt hatten, die Techniker des Mars hatten
es in verhältnismäßig kurzer Zeit erreicht.

Allerdings besaßen ja die Martier vor allem ein Mittel, sich in die
Luft zu erheben, das den Menschen fehlt, die Anwendung der
Diabarie. Der Fortschritt der Luftschifffahrt bei den Menschen war
früher immer daran gescheitert, daß man das aerostatische und das
dynamische Luftschiff nicht in geeigneter Weise verbinden konnte.
Wandte man den Luftballon an, um Lasten in die Höhe zu heben, so
mußte der Apparat riesige Dimensionen annehmen, und es war dann
unmöglich, ihn gegen die Windrichtung zu bewegen, weil er dem Wind
eine zu große Angriffsfläche bot oder nicht genügend
widerstandsfähig gegen seinen Druck gemacht werden konnte. Wählte
man aber die dynamische Form des Luftschiffs, wobei durch Schrauben
oder Flügel die Erhebung bewerkstelligt wurde, so fehlte es an
Maschinen, um die erforderliche große Kraft zu entwickeln; denn um
dies zu leisten, mußten die Maschinen selbst zu schwer werden.

Diesen Schwierigkeiten waren nun die Martier dadurch enthoben, daß
sie diabarische Fahrzeuge zu bauen vermochten, das heißt Fahrzeuge,
für welche die Anziehungskraft der Erde nahezu ganz aufgehoben
werden konnte. Bei der Luftschiffahrt geschah diese Aufhebung
natürlich nicht so vollständig wie bei der Raumschiffahrt, sondern
nur soweit, daß das Gewicht des Schiffes samt seinem Inhalt
geringer wurde als das Gewicht der von ihm verdrängten Luft. Nach
dem archimedischen Gesetz mußte es dann in der Luft in die Höhe
steigen, und, je nachdem man seine Schwere vergrößerte oder
verkleinerte, konnte man es senken oder heben. Man bedurfte dazu
keiner Riesenballons und keiner Ballastmassen. Das Probeschiff der
Martier wog mit seinem ganzen Inhalt etwa fünfzig Zentner und besaß
eine Luftverdrängung von über hundert Kubikmeter. Es genügte also
eine Erniedrigung des Gewichts bis auf fünf Prozent des
eigentlichen Betrages, das heißt bis auf 125 Kilogramm, um zu
bewirken, daß das Schiff in der Nähe der Erdoberfläche schwebte,
denn soviel beträgt hier ungefähr das Gewicht der verdrängten
hundert Kubikmeter Luft. Was aber die Martier bisher verhindert
hatte, sich mit ihren Raumschiffen in die Atmosphäre zu wagen, war
die mangelhafte Widerstandsfähigkeit des Stellits. Es galt somit
für die Martier vor allem, einen Stoff zu finden, der sich
diabarisch machen ließ und dabei doch die genügende Festigkeit
besaß, um eventuell nicht nur den gewaltigen Druck eines
Sturmwindes auszuhalten, sondern auch mit großer Geschwindigkeit
gegen die Luft anzufliegen. Das war jetzt gelungen. Das neue
Luftschiff vermochte einer mit 400 Metern Geschwindigkeit gegen
dasselbe bewegten Luftmasse Widerstand zu leisten, ohne eine
schädliche Verbiegung seiner Umhüllung zu erleiden. Dieser Stoff
führte den Namen Rob.

Diabarie und Rob fanden nun ihren dritten Verbündeten zur
Vollendung der Aerotechnik in einer Modifikation des Repulsit. Man
konnte natürlich in der Luft der Erde nicht wie im leeren Raum
Repulsitbomben schleudern. Aber man hatte dafür eine Vorrichtung
ersonnen, den kondensierten Äther des Repulsits so allmählich zu
entspannen, daß man den unmittelbaren Rückstoß zur Fortbewegung
benutzen konnte. So bedurfte es keiner Schrauben oder Flügel, die
nicht nur viel Raum einnahmen, sondern auch leicht der Havarie
ausgesetzt waren; man schoß sich direkt durch Reaktion, wie eine
Rakete, durch die Luft. Die beiden großen Flügel und das Steuer,
welche das neue Luftschiff trug, konnten unter Umständen gänzlich
zusammengeschoben und eingezogen werden; sie dienten nur dazu, um
das Gleichgewicht bei plötzlicher Änderung der Richtung zu bewahren
und um nicht die großen Vorteile zu verlieren, welche der Segelflug
bei günstigem Wind darbietet.

Ill hatte das Schiff vom Mars mitgebracht und sich jetzt von seiner
Tauglichkeit überzeugt. Die Versuche geschahen in der Nacht, das
heißt während der Schlafenszeit, weniger, weil man die neuen
Erfolge vor den Menschen verbergen wollte, als weil man bei einem
etwaigen Mißerfolg keinerlei Zeugen zu haben wünschte. Immerhin
beabsichtigte Ill nicht, die Menschen in die Fortschritte
einzuweihen, welche die Martier gemacht hatten; da aber noch
weitere Übungen angestellt werden sollten und man höchstens noch
auf zwei Wochen Tageslicht rechnen konnte, so lag ihm selbst daran,
Grunthe, wenn dieser auf seiner Weigerung beharren sollte,
möglichst schnell von der Insel zu entfernen. Die Aufklärungsfahrt
des Luftschiffs in der Richtung nach Europa hing mit dieser Absicht
zusammen. Es stellte sich heraus, daß mit Anwendung des Repulsits
Geschwindigkeiten von 200 Metern in ruhiger Luft mit Leichtigkeit
erreicht werden konnten. Man bewegte sich dabei in Höhen von
ungefähr zehn Kilometern, bei einer Luftverdünnung, welche
allerdings von Menschen nur bei künstlicher Sauerstoffatmung
ertragen werden konnte, den Martiern aber, wenn sie nur von Zeit zu
Zeit etwas Sauerstoffzuschuß erhielten, keine besonderen
Beschwerden verursachte. Heftige Luftströmungen konnten hier die
Geschwindigkeit des Luftschiffs wohl zeitweise um die Hälfte
steigern oder mindern, im Mittel jedoch vermochte man in der Stunde
siebenhundert Kilometer zurückzulegen. Auf diese Weise konnte man
vom Nordpol nach Berlin in sechs Stunden gelangen.

Als die Lichtdepesche über das Gelingen dieser Probefahrten nach
dem Mars gelangte, bewilligte der Zentralrat die Mittel zum Bau von
hundertvierundvierzig Erd-Luftschiffen, welche bis zum nächsten
Erd-Nordfrühjahr fertigzustellen seien – – – –

Es war noch früh am Tage, und Saltner wollte sich eben von seinem
Posten, auf dem er vergeblich nach der Rückkehr des Luftschiffes
ausgeschaut hatte, nach dem Dach der Insel begeben, um Grunthe bei
der Arbeit am Ballon zu helfen, als er in das Sprechzimmer gerufen
wurde.

Dort erwartete ihn La. Der kühle Ernst, welchen sie gestern gezeigt
hatte, die fremde Haltung war verschwunden. Mit einem Lächeln auf
den Lippen, in der ganzen hinreißenden Anmut ihres Wesens schwebte
sie ihm entgegen und begrüßte ihn mit einer Zärtlichkeit, die ihn
wehrlos machte. Sie zog ihn neben sich auf einen Sitz und sagte,
seine Hand haltend:

„Sei nur nicht gar so verwundert, Sal, heute ist wieder mein Tag,
und was gestern war, geht uns nichts an. Oder hast du schon
vergessen –?“

„Wie könnte ich! Aber ich begreife nur nicht –“

„Aber liebster Freund, das ist doch ganz einfach! Haben wir uns
lieb?“

„La!“

„Und hab ich dir nicht schon gesagt, Liebe darf nicht unfrei
machen? Und hast du nicht auch Se lieb?“

„Ich bitte dich.“

„Ich weiß es, und es ist ein Glück, sonst dürften wir uns so nicht
sehen.“

„Was ihr für seltsame Sitten habt!“

„Können wir uns gehören für immer? Kannst du dauernd auf dem Mars
leben oder ich auf der Erde? Oder irgendwo zwischen den Planeten?
Und was hat das überhaupt mit der Liebe zu tun? Das sind ganz
andere Fragen. Wir aber wollen uns der Schönheit freuen und des
Glücks, das wir im freien Spiel des Gefühles genießen. Liebtest du
mich allein, du wärest bald unfrei, über dich herrschte die
Leidenschaft, der das Herzeleid folgt, und ich müßte mich dir
entziehen. Wohl gibt es ein Glück zwischen Mann und Frau, das kein
Spiel ist, sondern Ernst; doch davor stehen viele Prüfungen, und ob
es möglich ist zwischen Nume und Mensch, das weiß noch niemand. Und
damit wir nicht vergessen, daß Liebe ein Spiel ist, dürfen wir
nicht ganz allein es führen, und doch allein, wann wir wollen. Und
nun zerbrich dir nicht den törichten Kopf! Ich habe dir etwas
Ernstes zu sagen.“

„Noch etwas Ernsteres? Ich werde Mühe haben, mich in das eine zu
finden. Aber es ist wahr, allgemeineres dürfen wir nicht über
unserem – Spiel vergessen.“

„Ich glaube, du verstehst mich noch immer nicht – Spiel heißt doch
Kunstwerk, ein Trauerspiel ist auch ein Spiel, nur daß man nicht
selbst dabei umkommt, sondern der Held, mit dem man fühlt. Und den
Wert unseres Gefühls setzen wir nicht herab, nein, wir machen ihn
reiner und höher, wenn wir ihn in die Freiheit des Spiels, in das
Reich des selbstgeschaffenen schönen Scheines erheben. – Du Tor!
Ist dieser Kuß ein Schein? – Nein, Schein ist nur, daß ich damit
die Freiheit meines Selbst verliere. Und nun höre! Du kommst mit
uns auf den Mars, damit du endlich einmal verständig wirst.“

„Sprichst du so als meine geliebte La? Dann muß ich dir zeigen, daß
ich dein gelehriger Schüler bin, indem ich meine Freiheit bewahre.
Du weißt, warum ich nicht mit euch kommen kann.“

„Ich weiß es, und du bist brav, und ich hab dich darum nur lieber.
Ihr wart gestern Männer. Aber wenn wir nun die Bedingung erfüllen,
daß ihr eure Nachrichten überbringen könnt, wenn wir einem von euch
die Mittel zur Heimkehr verschaffen, will dann nicht der andere mit
uns kommen?“

„Und wer soll der andere sein?“

„Das wird sich ja finden. Doch im Ernst, ich bin beauftragt, bei
euch anzufragen, ob ihr darauf eingehen wollt. Sobald sich einer
von euch beiden bereiterklärt, nach dem Mars mitzugehen, schaffen
wir den andern sofort in seine Heimat.“

„Merkwürdig! Und ich wollte euch heute denselben Vorschlag machen.
Es hat sich gezeigt, daß der Ballon nur eine Person wird tragen
können, das muß natürlich Grunthe sein. Wollt ihr uns eure Hilfe
leihen, den Ballon herzustellen, so daß Grunthe abreisen kann, so
bin ich bereit, mit euch nach dem Mars zu gehen.“

„Das ist herrlich, liebster Freund, dafür muß ich dir danken. Und
wegen des Ballons mache dir keine Sorge – wir haben einen
sichereren Weg nach Deutschland –“

„Das Luftschiff?“

„Ihr habt gelauscht?“

„Gesehen. Und damit wollt Ihr uns – aber dann könnte ich ja auch
mit zurück?“

„Nein, das ist Bedingung. Du mußt mit uns kommen –“

„Ach, La, ich sträube mich ja nicht.“

„So komm, wir wollen mit Ra und deinem Freund sprechen.“

„Aber zuvor dürfen wir wohl noch ein wenig hier plaudern?“

\tb

Es war sieben Uhr, zwei Stunden nach Feierabend, als das Luftschiff
von seiner Fahrt zurückkehrte. Nachdem Ill den erstatteten Bericht
mit großer Zufriedenheit entgegengenommen hatte, wurde das Schiff
sofort zu einer neuen Fernfahrt in Bereitschaft gesetzt.

Grunthe und Saltner hatten sich bereits in ihre Zimmer
zurückgezogen, als das Schiff ankam, und daher nichts mehr von
demselben bemerkt. In einer Unterredung mit Ill und Ra hatte
Grunthe eingewilligt, die Fahrt auf dem Luftschiff der Martier
anzutreten. Er bereitete sich darauf vor, indem er alle Gegenstände
zusammenpackte, die er mitzunehmen wünschte. Man hatte ihm Gepäck
im Gewicht von einem Zentner bewilligt, und außer seinen Büchern
und Instrumenten packte er noch eine Anzahl Kleinigkeiten ein,
welche seinen Landsleuten die Industrie der Martier verdeutlichen
sollten. Darauf legte er sich zur Ruhe.

Am folgenden Morgen, am zweiten Tag nach der Beratung mit den
Martiern, hatten Grunthe und Saltner eben ihr Frühstück beendet,
und Saltner hatte sich nach dem Sprechzimmer begeben, als Hil bei
Grunthe eintrat. Dieser war damit beschäftigt, seine Effekten auf
einen Platz zusammenzustellen.

„Das ist Ihr Gepäck?“ fragte Hil. „Wünschen Sie sonst noch etwas
mitzunehmen?“

„Nichts weiter – es ist alles vollständig und wird das Gewicht von
einem Zentner nicht überschreiten.“

„So sind Sie also reisefertig?“

„Ganz und gar – Sie sehen, ich bin sogar schon in meinem
Reiseanzug, und da liegt mein Pelz. Wann soll die Fahrt beginnen?“

„Sehr bald, vielleicht schon in dieser Stunde. Haben Sie Ihrem
Freund noch etwas mitzuteilen?“

„Nein, wir haben uns hinreichend ausgesprochen, hier sind seine
Briefe und Tagebücher für die Heimat.“

„Sie wären also bereit, sogleich aufzubrechen?“

„Ich bin bereit.“

Hil trat dicht an ihn heran und faßte seine Hände, als wollte er
sich verabschieden. Dabei sah er ihm fest in die Augen. Grunthe
fühlte sich von diesem Blick eigentümlich betroffen. Er konnte die
Augen nicht fortwenden, und doch begann die Umgebung vor seinen
Blicken zu verschwimmen. Er sah nur noch die großen, glänzenden
Pupillen des Arztes.

Dieser legte ihm jetzt langsam die Hände auf die Stirn und sagte
bedeutsam:

„Sie schlafen!“

Grunthe stand starr, bewußtlos, mit offenen Augen. Hil drückte
leise seine Augenlider herab und winkte mit dem Kopf rückwärts.
Zehn Martier, die sich bereitgehalten hatten, traten ein. Sechs von
ihnen nahmen Grunthe behutsam in die Arme, legten ihn auf ein
Tragbett und schafften ihn aus dem Zimmer. Die vier andern folgten
mit dem Gepäck. Grunthe wurde in das Luftschiff gebracht und
sorgfältig in seinen Pelz gehüllt. Das Rohr des Sauerstoffbehälters
wurde in seinen Mund geführt.

Wenige Minuten darauf erhob sich das Luftschiff senkrecht in die
Höhe. Nachdem es tausend Meter gestiegen war, schlossen sich die
seitlichen Öffnungen. Der Reaktionsapparat spielte. Schräg aufwärts
schoß es in der Richtung des zehnten Meridians nach Süden.

Ra begab sich zu Saltner in das Sprechzimmer und sagte:

„Wundern Sie sich nicht, daß Sie Ihren Freund nicht mehr vorfinden
werden. Ich hoffe, daß wir Ihnen bald die Nachricht seiner
glücklichen Ankunft in der Heimat melden können. Wir hielten es für
notwendig, die Abreise zu beschleunigen.“

Saltner sprang an das Fenster. Fern am Horizont leuchtete ein
schwaches Dampfwölkchen auf, um alsbald zu verschwinden.

Er war jetzt der einzige Europäer am Nordpol.

Se trat zu ihm.

„Seien Sie guten Muts, lieber Freund“, sagte sie. „Morgen geht
unser Raumschiff nach dem Nu!“

\section{21 - Der Sohn des Martiers}

Auf der Nordseite der Stadt Friedau dehnt sich ein langgestreckter
Hügelrücken. Sorgsam gepflegte Gärten ziehen sich an seinen
Abhängen in die Höhe, aus deren Grün schmucke Villen hervorlugen.
Vom Gipfel hernieder glänzt über den Baumkronen eines parkartigen
Gartens ein weißes Landhaus, das ein erhöhter Kuppelbau auf den
ersten Blick als eine Sternwarte erkennen läßt.

Der wunderbar klare Septembertag, an dem die Besucher jenes über
dem Nordpol schwebenden Ringes mit ihrem tausendmal vergrößernden
Projektionsfernrohr die Karte von Deutschland durchmusterten,
neigte sich seinem Ende zu. Sein mildes Licht lag über den
zierlichen Gärten Friedaus, in denen großblumige Georginen den
Rosenflor verdrängten, über den alten Bäumen des weiten fürstlichen
Parks, der vom Fuß des Hügels beginnend fast die ganze Stadt umzog,
und spiegelte sich dort im ruhigen Wasser des Teiches.

Den breiten Kiesweg, welcher vom Hügel herab zwischen den Vorgärten
der Villen nach dem Eingang des Parkes führte, schritt in Gedanken
verloren der Besitzer jener Privatsternwarte. Im Schatten der Bäume
angelangt, nahm er den weichen hellfarbigen Filzhut ab, und man
sah, daß volles graues Haar seinen Kopf bedeckte. Aber es war nicht
ergraut von der Last des Alters, es hatte stets diese Farbe gehabt.
Unter der hohen Stirn leuchteten zwei mächtige tiefdunkle Augen.
Sie waren jetzt nicht mehr sinnend zur Erde gerichtet, sondern
spähten erwartungsvoll durch die Gänge des Parkes.

Zwischen den Büschen am Ufer des Teiches schimmerte ein heller
Sonnenschirm. Beim Geräusch der nahenden Schritte erhob sich von
einer Bank unter dem Schatten einer breitästigen Linde eine
anmutige Frauengestalt in eleganter Sommerkleidung. Der
nachdenkliche Ernst, der über ihren feinen Zügen gelegen hatte,
wich einem freundlichen Lächeln, als sie jetzt Ell entgegentrat,
und in ihren dunkelblauen Augen blitzte es auf wie von einem
stillen Glück, als sie ihm die Hand reichte.

„Verzeihen Sie“, sagte Ell, indem er an ihrer Seite den Parkweg am
Ufer des Teiches entlangwandelte, „ich habe mich verspätet,
natürlich ohne meine Schuld.“

„Auch ich bin eben erst gekommen“, erwiderte Isma Torm. „Ich habe
Besuch gehabt. Frau Anton hat mir sehr weise Reden gehalten. Sie
konnte gar kein Ende finden.“

„Ich kann es mir denken, aber machen Sie sich nichts daraus. Sie
können tun, was Sie wollen, den Menschen werden Sie es doch nicht
recht machen.“

Isma seufzte leise. „Sie sehen, ich bin doch gekommen!“

Ell dankte ihr durch einen Blick. „Es ist die einzige Stunde am
Tag, Isma, in der einmal der Weltärger verschwindet und ich frei
und glücklich bin.“

„Und Ihre Arbeit?“

„Selbst diese ist nicht frei von Enttäuschung. Beschränktheit und
Engherzigkeit, wohin Sie sehen. Sie wissen, daß ich mich über Kampf
und Streit nicht beklage, denn das ist die Form, wodurch wir
weiterkommen. Aber diese Unfähigkeit, das Ziel zu sehen, dieser
Eigensinn, daß die Dinge nicht auch anders gingen!“

„Was hat Sie denn heute geärgert, Ell? Schütten Sie nur das Herz
aus.“

„Es ist ja nichts Neues. Sie wissen, daß ich mich vor Jahresfrist
entschlossen habe, meine Theorie der Gravitation zu
veröffentlichen. Grunthe redete mir zu, obwohl er sagte, es wird
niemand begreifen.“

„Ich erinnere mich sehr gut. Es war –“

„Ja damals –“

„Und damals sagten Sie, das wäre Ihnen ganz gleichgültig.“

„Das ist auch wahr. Was meine Person angeht, meinen Ruhm oder wie
Sie es nennen wollen, das ist mir auch ganz gleichgültig. Aber um
der Sache willen tut es mir leid. Was die Menschheit dadurch
verliert, das schmerzt mich, und ich sehe, daß ihr so nicht zu
helfen ist. Erst wird das Buch totgeschwiegen, die Gelehrten wissen
nicht, was Sie damit anfangen sollen, dann kommt einer und
behauptet, das wäre eine phantastische Hypothese, durch nichts
bewiesen. Dabei habe ich aufgrund meiner Theorie das sogenannte
Drei-Körper-Problem gelöst und die Richtigkeit bis auf die
Hundertstelsekunden an der Störung der Marsmonde nachgewiesen. Aber
glauben Sie, daß ein einziger Astronom meine Methode der Rechnung
verstanden hat?“

„Ko Bate“, sagte Isma lächelnd. „Das wollten Sie doch wohl sagen?
Wahrscheinlich haben Sie sich nicht klar genug ausgedrückt.“

„Allerdings, ich hätte darüber ein besonderes Buch schreiben müssen
– ich glaubte nicht, daß man so schwerfällig sein würde. Ich habe
die Methode gar nicht selbst erfunden, sondern schon in meinem
achtzehnten Jahr von meinem Vater erlernt –“

„Aber warum haben Sie das alles so lange geheimgehalten?“

„Sie sehen ja, daß es noch immer zu früh ist. Könnten die andern
mit mir in der gleichen Richtung weiterarbeiten, man würde auch
technisch zu Resultaten kommen, die eine ganz neue Welt eröffnen
müßten. Ach, dann würden wir vielleicht einmal frei von dieser
schweren Erde.“

„Immer wieder dieselbe Sehnsucht. Es ist ja doch hier ganz
leidlich. Sie müssen Geduld haben. Und dies hat Sie heute verstimmt
und aufgehalten?“

„Dies weniger. Heute waren es praktische Sachen, Ärger mit den
Behörden. Das ist eine Schwerfälligkeit – vornehmlich drüben im
Nachbarstaat –, ein Reglementieren – alles muß in eine Schablone
gepreßt werden. Und das hat mich mißmutig gemacht, ganz besonders,
weil es auch Sie angeht.“

„Mich? Ist etwas vorgefallen?“ fragte Isma ängstlich.

„Nein, ich meine unsere Luftschifferstation. Man will sie
verstaatlichen, neben die militärische unter das Kriegsministerium
stellen, wahrscheinlich dann auch von hier fort verlegen.
Jedenfalls verlangt man eine Staatsaufsicht – obwohl der Staat noch
nicht einen Pfennig dazu gegeben hat.“

„Aber warum denn?“

„Ich glaube, man traut mir nicht. Im Falle eines Krieges will man
wohl Sicherheiten haben. Sie wissen, die Abteilung ist eine
internationale Gründung. Ich selbst habe meine besonderen Ansichten
über Patriotismus.“

„Ich bitte Sie, Ell, Sie sind doch ein Deutscher. Im Kriegsfall
müssen wir uns selbstverständlich zur Verfügung stellen – aber, wer
wird denn an Krieg denken. Ach, machen Sie mir nicht noch mehr
Sorge!“

„Ich bin ein Deutscher mit meinen Sympathien, staatsrechtlich bin
ich es nicht, man kann mich also im Notfall ausweisen. Die Sache
ist doch so – Deutschland oder Frankreich oder England, irgendeine
Nation oder ein Staat ist ja kein Selbstzweck; Selbstzweck kann nur
die Menschheit als Ganzes sein. Die einzelnen Völker und Staaten
sind Mittel, im gegenseitigen Wettbewerb die Idee der Menschheit zu
erfüllen. Wenn nun einmal der Staat, dem ich angehöre, durch seinen
Erfolg nicht das zweckentsprechende Mittel wäre in Rücksicht auf
die Idee der Menschheit, so wäre es unmoralisch, wenn ich als freie
Persönlichkeit mich nur darum für ihn entschiede, weil ich ihm viel
verdanke. Die ethische Forderung ist eine andere. Aber bei den
Menschen wird immer nach dem unmittelbaren Gefühl entschieden, und
das nennt man dann Patriotismus und hält für Pflicht, was doch bloß
Neigung ist.“

Isma blieb stehen. „Aber dann“, sagte sie langsam, „mit welchem
Recht gehen wir hier spazieren? Ist das auch Pflicht?“

„Gewiß, wenn sie auch mit der Neigung zusammenfällt. Sie werden
sich selbst doch nicht danach beurteilen, was die Friedauer für
richtig halten?“

„Nein“, sagte Isma, indem sie lächelnd zu ihm aufblickte, „kommen
Sie ruhig mit durch die Stadt. Glauben Sie nicht, daß wir bald eine
Nachricht erwarten können?“

„Die Depesche von Spitzbergen sagt uns, daß die Fahrt am 17. August
angetreten ist. Es ist wohl möglich, daß in den nächsten Tagen eine
Nachricht eintrifft.“

„Sie sind noch immer guten Muts?“

„Ich hoffe zuversichtlich. Glauben Sie mir, ich hätte Ihrem Mann
nicht so aufrichtig zugeredet, wenn ich nicht überzeugt wäre, daß
ihm die Expedition in besonderer Weise glücken wird.“

„Ell, Sie denken noch an irgend etwas Unerwartetes; ich bitte Sie,
seien Sie offen, fürchten Sie eine bestimmte Gefahr?“

„Nichts, was zu fürchten ist, ich versichere Sie, Isma! Etwas
Unerwartetes vielleicht, aber nichts zu fürchten!“

„O bitte, was denken Sie? Ich habe schon oft bemerkt, daß Sie mir
noch etwas verschweigen.“

„Wahrhaftig, Isma, ich verschweige Ihnen nichts, was ich weiß, aber
verlangen Sie nicht, daß ich Vermutungen Ausdruck gebe, die
vielleicht völlig nichtig sind. Ich setze eine große Hoffnung auf
die glückliche Wiederkehr der Expedition, und ich rechne mit
Sicherheit darauf. So sicher, daß ich mir größte Mühe gebe, eine
Stellung für Saltner ausfindig zu machen. Denn was soll er dann
tun, wenn er zurückkehrt? Und sehen Sie, das hat mich auch heute
gekränkt – glauben Sie, daß die Regierung den Mann anstellt, der
eine so ruhmvolle Expedition mitmacht? Er ist ja ein Ausländer und
hat seine Prüfungen nicht bei uns abgelegt!“

„Lassen Sie ihn nur erst zurück sein. Mich beunruhigt dieses
Unerwartete, wie Sie es nennen.“

„Wirklich, es ist nur eine Art Ahnung, daß uns mit der Auffindung
des Nordpols mehr gegeben werden wird als eine geographische
Entdeckung.“

„Das müssen Sie mir noch erklären.“

„Vielleicht bald. Aber heute haben wir noch nicht einmal von Ihnen
gesprochen. Was haben Sie getan, gelesen, erfahren?“

„Herzlich wenig. Die Polarkarte habe ich wieder einmal studiert.“

Im lebhaften Gespräch durchschritten sie die belebteren Teile der
Anlagen. Hinter den Bäumen sank die Sonne, rot und golden leuchtete
der Abendhimmel. Öfter begegneten sie jetzt Spaziergängern. Den
meisten waren sie bekannt, man grüßte die beiden höflich, aber
hinterher drehte man sich um und sah ihnen nach. Man warf sich
Blicke zu oder zischelte eine Bemerkung.

„Sie haben gut spazierengehen“, näselte ein kleiner Herr mit
breitem, schnüffligem Gesicht seinem Begleiter zu, „er hat den Mann
nach dem Nordpol geschickt.“

„Das ist die Torm“, sagte ein junges Mädchen. „Jeden Tag geht sie
mit dem Doktor Ell hier vorüber.“

Die Friedauer waren sehr stolz darauf, daß alle Zeitungen von ihrer
Nordpolexpedition erfüllt und die Lebensbeschreibungen ihrer
Mitbürger überall zu lesen waren. Darum waren sie glücklich, auch
über sie reden zu können. Sie taten es nach Herzenslust in ihrer
menschenfreundlichen und liebevollen Weise und um so mehr, je
weniger sie von ihnen wußten.

Ell und Isma hatten die Anlagen verlassen und waren in eine der
breiten mit Vorgärten vor den Häusern versehenen Alleen
hineingeschritten. Sie standen vor der Tormschen Wohnung. Ell hatte
schon Isma die Hand zum Abschied gereicht, und beide zögerten nur
noch einen Augenblick, sich zu trennen. Da öffnete sich die Haustür
und ein Telegraphenbote kam ihnen entgegen.

„Guten Abend, Frau Doktor“, sagte er. „Da treff ich Sie ja noch. Es
war oben niemand zu Hause.“

Isma griff nach dem Telegramm. Sie riß es auf.

„Von ihm! Aus Hammerfest!“ rief sie fieberhaft.

„Das ist die Brieftaubenstation“, sagte Ell.

Es dunkelte schon. Sie konnte die Buchstaben nicht mehr recht
erkennen. Die Leute sahen ihr von den Fenstern aus zu.

„Kommen Sie mit hinauf, Ell“, sagte sie. „Die Sache ist nicht so
kurz. Das ist eine Ausnahme, heute dürfen Sie kommen!“

Isma eilte voran. Als Ell in das Wohnzimmer trat, stand sie schon
unter der elektrischen Lampe und las das Telegramm. Ihren Hut, der
ihr das Licht nahm, hatte sie herabgerissen.

„Da“, sagte sie, Ell das Papier reichend. „Er lebt! Er ist gesund!
Lesen Sie, lesen Sie vor. Ich werde nicht daraus klug.“ Sie ließ
sich in einen Sessel sinken und begann ihre Handschuhe
abzustreiten.

Ell warf einen Blick auf das Telegramm. Seine Hände bebten
sichtlich. Er setzte sich.

„Um Gottes willen, Ell, was ist – Sie zittern –“

„Nicht aus Sorge, nein, nein – es war nur ein Augenblick der
Überraschung. Hören Sie, Isma.“

Er las:

„Hammerfest, 5. September, 3 Uhr 8 Minuten. Soeben Brieftaube mit
dem Stempel ›Ballon Pol‹ zurückgekehrt, brachte folgende
Nachricht:

Frau Isma Torm, Friedau, Deutschland.

19. August, 5 Uhr 34 Minuten M.E.Z., nachmittags. Alle gesund. Nach
dreißigstündiger direkt nördlicher, günstiger Fahrt schweben wir
über dem Pol. Gewirr von Inseln in meist eisfreiem, nicht sehr
ausgedehntem Bassin. Kleine, kreisrunde Insel, etwa fünfhundert
Meter Durchmesser, von unbekannten Bewohnern als Pol markiert,
trägt unerklärliche Apparate. Ihre Oberfläche enthält im größten
Maßstab stereographische Polarprojektion der Nordhalbkugel bis
gegen den dreißigsten Breitengrad. Bewohner nicht sichtbar. Da
Landung nicht ratsam, setzen wir Reise fort. Innigsten Gruß.

Torm.“

Ell las die Depesche noch einmal sorgfältig durch, während Isma ihn
erwartungsvoll ansah. Dann sprang er auf und machte einige Schritte
durch das Zimmer. Auch Isma hatte sich erhoben.

„Wir setzen die Reise fort! Das heißt, wir kommen wieder – nicht
wahr, Ell, das heißt es doch? Es ist gelungen? O Gott sei Dank!“

„Ja, es ist gelungen“, sagte Ell bedeutungsvoll.

Isma trat auf ihn zu und ergriff seine beiden Hände.

„Ich danke Ihnen, lieber Freund“, sagte sie, ihre tränenfeuchten
Augen zu ihm aufschlagend, „ich danke Ihnen, es ist Ihr Werk!“

Er zog sie sanft an sich, sie lehnte weltvergessen ihren Kopf an
seine Schulter.

„Isma!“ sagte er. Seine Lippen berührten ihre Stirn.

Sie schüttelte leise den Kopf und trat zurück. „Setzen Sie sich“,
sagte sie. „Und nun sprechen Sie, erklären Sie mir – das
Rätselhafte, das Unerwartete –“

„Es ist da.“

„Aber was bedeutet es – ich verstehe nicht, ich bin ganz verwirrt.
Ist es eine Gefahr?“

„Es bedeutet – Isma, Sie werden es nicht glauben wollen, was es
bedeutet – für uns alle. Wie soll ich es Ihnen sagen?“

Er zog seinen Sessel an den ihrigen und ergriff ihre Hand.

„Was ist Ihnen?“ fragte sie, ihn ängstlich anblickend.

„Es bedeutet, daß die Bewohner des Planeten Mars auf dem Nordpol
der Erde gelandet sind. Es, bedeutet, daß sie mit ihren Apparaten
und Maschinen festen Fuß auf der Erde gefaßt haben. Es bedeutet,
daß die Erde, die Menschheit binnen kurzem unter ihrer Leitung
stehen wird – daß ein goldenes Zeitalter des Glückes und des
Friedens die Not der Menschheit ablösen soll – und daß wir es
erleben!“

Seine Stimme hatte sich gehoben, er hatte mit Begeisterung
gesprochen, seine Augen flammten tief, groß, dunkel und hafteten
wie in weiter Ferne.

Isma wußte nicht, was sie denken sollte.

„Ell“, sagte sie schüchtern, „ich bitte Sie, Sie können in dieser
Stunde nicht scherzen – wie soll ich das verstehen?“

„Es ist die Wahrheit.“

Es war mit einem Ausdruck gesprochen, daß ein Zweifel nicht möglich
war.

Isma schwieg. Sie lehnte sich zurück und strich das lichtbraune
Haar aus der schmalen Stirn. Dann faltete sie ihre Hände und sah
ihn bittend an.

„Hören Sie, Isma, geliebte Freundin“, sprach Ell langsam, „hören
Sie, was noch niemand weiß, noch niemand wissen durfte, und was
ihnen manches erklären wird, das Ihnen an mir rätselhaft war. Es
ist eine lange Geschichte.“

Er verfiel in Schweigen.

„Erzählen Sie“, bat sie innig. „Sie bleiben über Abend – ich kann
heute nicht allein sein, und andere mag ich heute nicht sehen – ich
muß alles wissen.“

Ell erzählte. Er sprach vom Mars, von seinen Bewohnern, von ihrer
Kultur, ihrer Güte, ihrer Macht. Er erklärte, wie sie zur Erde zu
gelangen hofften, um die Menschheit ihrer Kultur, der Numenheit,
entgegenzuführen, wie er sein Leben lang auf die Nachricht gehofft
habe, daß der Pol im Besitz der Martier sei, wie er hauptsächlich
darum die Polarforschung und Ausrüstung der Expedition betrieben
habe. Und nun habe er keinen Zweifel mehr.

Isma hatte ihm schweigend zugehört. Ihre Fassungskraft schien zu
Ende.

Als er schwieg, sagte sie:

„Sie erzählen ein Märchen, ein schönes Märchen. Ich würde das alles
für ein Märchen halten, wäre nicht die Depesche, und wären Sie
nicht mein lieber, treuer Freund. So muß ich Ihnen glauben, obwohl
ich nicht begreife, woher Sie das alles wissen und warum Sie
niemals davon gesprochen haben. Wenn Sie es wußten, was am Pol zu
erwarten war, so mußten Sie doch meinen Mann darauf vorbereiten.“

Ell lächelte jetzt. „Das hab ich auch“, sagte er, „soweit ich
durfte. Ich wußte ja nicht, ob meine Vermutung eintreffen würde,
also durfte ich auch nicht davon sprechen. Denn eben haben Sie
selbst gesagt, daß Sie mir ohne die Depesche nicht geglaubt hätten.
Man hätte mir nicht geglaubt, man hätte mich für einen Narren
gehalten, und ich hätte meine ganze Tätigkeit diskreditiert. Aber
ich habe für alle Fälle gesorgt. Erinnern Sie sich der drei
Flaschen Champagner, die Sie durch Saltner in den Korb schmuggeln
ließen? Sie gingen durch meine Hände. Unter denselben befindet sich
ein von mir entworfener Sprachführer – deutsch und martisch –, der
beim Zusammentreffen mit den Marsbewohnern am Pol, auf das ich
hoffte, gefunden werden mußte.“

Isma reichte ihm lächelnd die Hand und sagte kopfschüttelnd: „Und
nun sagen Sie mir das eine und Hauptsächlichste. Woher konnten Sie
alles das wissen – wenn es wirklich wahr ist?“

„Sie sollen auch dies wissen. Mein Vater war ein Nume. Er war kein
Engländer, wie es hieß, kein auf der Erde Geborener. Ich stamme
väterlicherseits von den Bewohnern des Mars.“

Isma sah ihn sprachlos an. Sie konnte nicht zweifeln. Das
Fremdartige seines Wesens, selbst seiner Erscheinung, das sie
anfänglich abgestoßen, später so viel stärker gefesselt hatte, als
sie sich selbst gestehen mochte – alles wurde ihr auf einmal
erklärlich.

Das Mädchen erschien an der Tür.

„Kommen Sie“, sagte sie. „Wir wollen uns wenigstens zu Tisch
setzen, es ist Zeit. Ich muß aber noch mehr hören, viel mehr.“

„Wie oft haben wir Sie geneckt“, sagte Isma bei Tisch, „wenn Sie
hier bei uns saßen und von den Marsbewohnern phantasierten. Es ist
mir nie der Gedanke gekommen, daß Sie Ihre Erzählungen ernst meinen
könnten.“

„Ich habe mich auch gehütet, es so erscheinen zu lassen. Dann säße
ich wohl im Irrenhaus. Und doch ist es so. Ich werde Ihnen die
Aufzeichnungen meines Vaters zeigen, wenn Sie wieder einmal auf
meinen Berg steigen. Und das meiste weiß ich aus seinem eigenen
Mund. Sie sehen mich ungläubig an?“

„Seien Sie nicht böse – ich glaube Ihnen, aber es will mir noch
nicht in den Kopf, das Unerhörteste, was je geschehen ist – und
mir, mir soll es begegnen –“

„Zwischen uns soll sich nichts ändern, Isma! Aber ich hoffe, Ihnen
jetzt erst ganz zeigen zu können, wie lieb ich Sie habe. Meine
Pläne sind groß.“

„Lassen Sie mich nur erst das Vergangene verstehen. Ihr Vater –“

„Mein Vater hieß All. Er war Kapitän des Raumschiffes ›Ba‹, das
heißt ›Erde‹, mit dem er bereits mehrere Fahrten nach dem Nordpol
wie nach dem Südpol der Erde gemacht hatte, als er infolge eines
Unglücksfalls mit sechs Gefährten auf dem Südpol zurückgelassen
wurde. Als das Schiff in den nächsten Tagen nicht zurückkehrte,
wußten sie, daß sie vor dem nächsten Frühjahr keine Hilfe zu
erwarten hatten. Den Polarwinter am Südpol zu durchleben, war
unmöglich. Unter unsäglichen Strapazen schleppten sie sich nach
Norden bis an das Meeresufer. Mein Vater allein gelangte dort an,
die übrigen waren den Anstrengungen erlegen. Es glückte ihm, von
einem verspäteten Walfischjäger aufgenommen zu werden. Man hielt
ihn für einen Schiffbrüchigen, der den Verstand verloren hatte. Er
aber benutzte die Zeit der Überfahrt nach Australien, um die
Sprache zu erlernen, ohne daß die Seeleute es wußten. Man brachte
ihn in ein Hospital. Durch unerschütterliche Energie gewöhnte er
sich an die Erdschwere und machte sich mit menschlichen
Verhältnissen vertraut. Dann gewann er Freunde, die ihm die Mittel
gaben, seine technischen Kenntnisse zu verwerten. Einige
Erfindungen, die auf dem Mars längst bekannt waren, machten
ungeheures Aufsehen. Es dauerte nicht lange, so war mein Vater ein
reicher Mann. Er lernte meine Mutter kennen, die als deutsche
Erzieherin in einem englischen Haus lebte. So wurde ich in
deutscher Bildung aufgezogen. Außer meiner Mutter und mir erfuhr
niemand das Geheimnis der Herkunft meines Vaters. Aber in mir
pflegte er den Stolz, als Sohn eines Martiers teilzuhaben an der
Numenheit. Immer habe ich den roten Planeten als meine eigentliche
Heimat betrachtet, und einmal auf ihn zu gelangen, war mein
Jugendtraum. Aber mein Vater starb, ehe ich das zweiundzwanzigste
Jahr erreichte, ohne daß den Menschen eine Nachricht vom Mars
gekommen war. Und das Vermächtnis meines Vaters – meine Mutter war
noch vor ihm dahingegangen – stellte mir eine größere Aufgabe: die
Erde den Martiern zu erschließen, die Menschheit teilnehmen zu
lassen am Segen der martischen Heimat.

Ich ging nach Deutschland, ich studierte und lernte den ganzen
Jammer dieses wilden Geschlechtes kennen an der Stelle, wo die
höchste Zivilisation des Planeten sich zeigen soll. Auch ein
großes, herrliches Glück trat mir entgegen, aber es sollte mir
nicht beschieden sein. Ich lernte Isma Hilgen kennen –“

„Sie wissen –“

„Ja, ja, Isma, Sie haben recht gehabt damals. Sie wären unglücklich
geworden, wie ich es war. Ich ging nach Australien zurück. Aber
meine Pläne, die Martier am Nordpol aufsuchen zu lassen, konnte ich
nur von Europa aus verfolgen. Ich kaufte mich hier an – das andere
wissen Sie.“

Sie reichte ihm die Hand über den Tisch hinüber.

„Ich nehme Sie bei Ihrem Wort“, sagte sie herzlich, „zwischen uns
soll sich nichts ändern. Nein, ich fange an, vieles zu verstehen,
was mich manchmal von Ihnen zurückschreckte. Wie konnte ich mir
anmaßen, Ihnen das sein zu können, was Sie bei den Menschen
suchten?“

„Ich habe Sie niemals mehr geliebt, als wenn Sie mich für wandelbar
hielten.“

„Lassen Sie – wir dürfen jetzt nicht von uns sprechen. Was werden
Sie zunächst tun?“

„Das Telegramm muß natürlich veröffentlicht werden. Ich nehme es
gleich mit. Aber die Aufklärung, welche ich Ihnen gegeben habe,
bleibt vorläufig unter uns. Die Presse wird sogleich ihre Zweifel,
Vermutungen und weisen Bemerkungen laut werden lassen. Dann gebe
ich den Hinweis auf die Martier als eine Hypothese, ganz
vorsichtig, nur um vorzubereiten.“

„Aber sind Sie denn auch Ihrer Sache ganz sicher? Ich meine, daß es
wirklich Ihre Landsleute sind, die sich am Pol befinden?“

„Ich habe keinen Zweifel. Ich kann Ihnen noch etwas sagen, was ich
selbst erst seit einigen Tagen weiß. Es wird sicherlich ebenfalls
öffentlich zur Sprache kommen, sobald die Nachricht von der
Expedition bekannt wird. Sie müssen wissen – mein Vater hat es mir
erklärt –, daß die Martier nur am Nordpol oder am Südpol auf der
Erde landen können. Ihre Raumschiffe suchen, sobald sie der Grenze
der Atmosphäre sich nähern, genau in der Richtung der Erdachse
heranzukommen. Es ist aber für sie gefährlich, in die Atmosphäre
einzudringen. Deswegen ging man auf Anregung meines Vaters mit dem
Plan um, in der Verlängerung der Erdachse außerhalb der Atmosphäre
eine Station zu errichten, auf welcher die Schiffe bleiben und von
der aus man dann auf andere Weise nach unten gelangt – ich erkläre
Ihnen das ein andermal genauer, auch weiß ich ja nicht, ob die
Pläne so ausgeführt worden sind, wie sie damals, vor mehr als
vierzig Erdenjahren, bestanden. Sicherlich aber haben die Martier
in irgendeiner Weise ihre Absicht durchgesetzt und eine
Außenstation gegründet. Danach habe ich mit meinem Instrument
gesucht, aber nur einmal einen Lichtpunkt bemerkt, den ich für die
Station halten konnte, da er sich nicht mit den übrigen Sternen um
die Weltachse drehte. Ich habe ihn seitdem nicht wieder finden
können, obgleich ich die Stelle genau gemessen hatte; aber das
wundert mich auch nicht, denn die Martier werden schon dafür
sorgen, daß die Station möglichst wenig Licht ausstrahlt, und es
sind gewiß nur vereinzelte Stunden, in denen die Station einmal auf
so große Entfernung – ich berechne sie auf gegen 9.000 Kilometer –
sichtbar wird. Nun wurde vor einigen Tagen von der Zentralstation
für Kometen in Kiel ein Telegramm versendet, daß in Helsingfors ein
Stern entdeckt wurde, der kein Stern sein kann, weil er am Umlauf
des Himmels nicht teilnimmt und doch nicht im Pol steht, dagegen
genau im Meridian in 36 Grad Höhe. Daraus läßt sich leicht
berechnen, daß sich auf der Erdachse, genau in der Entfernung des
Erdradius über dem Pol, ein leuchtender Körper befinden muß.
Allerdings konnte dieser wegen leichten Nebels, vielleicht auch,
weil er schwächer leuchtend wurde, bisher nicht wiedergefunden
werden, aber die Angabe stimmt genau mit meiner früheren
Beobachtung. Ein Körper, der an dieser Stelle über dem Nordpol
stillsteht, kann gar nichts anderes sein als die geplante Station
der Marsbewohner; eine andere Erklärung ist undenkbar. Diese
Entdeckung wird meine Hypothese bestätigen, sobald sie bekannt
werden wird. Man hat sie nur von Helsingfors aus mit so großer
Vorsicht weitergegeben, weil man keine Erklärung dafür weiß und
daher an eine Täuschung denken mußte. Wir werden also vorbereitet
sein, wenn die Expedition zurückkommt –“

„Wann, wann glauben Sie, daß dies möglich ist?“

„Jeden Tag, jede Stunde kann die Nachricht eintreffen, daß sie
bewohnte Gegenden erreicht haben, ja –“

Ell unterbrach sich und sann nach.

„Sie wollten noch etwas sagen, Ell! Sie wollten sagen, es müßte
schon Nachricht da sein, wenn alles gut gegangen? Nicht wahr?“

„Allerdings, es könnte schon Nachricht da sein, aber es ist auch
durchaus kein Grund zur Beunruhigung, daß sie noch nicht da ist.
Bedenken Sie – wir haben heute den fünften – also siebzehn Tage,
nachdem die Expedition den Pol verlassen hat – sie können in
Gegenden gelandet sein, von denen aus ein Bote Wochen braucht, um
die nächste Telegraphenstation zu erreichen.“

Isma preßte die Hände an ihre Stirn.

„Es ist so seltsam“, sagte sie nachdenklich, „wie sehnte ich mich
nach einer Nachricht, alle Gedanken gingen um die Expedition – und
nun, nachdem Sie mir dies gesagt haben, dies Ungeheuerliche, das
uns bevorsteht – wie schrumpft das alles zusammen, was Menschen
tun. Ach, Ell, es ist eigentlich Unrecht –“

„Durfte ich länger schweigen?“

„Nein, mein Freund, ich danke Ihnen ja doch – aber – Sie müssen mir
noch mehr sagen, vom Mars –. Sie müssen mich lehren –“

„Was Sie wollen, Isma.“

„Doch nicht heute – es ist schon spät.“

„Wirklich, in der zehnten Stunde. Ich muß Sie verlassen. Aber auf
Wiedersehen! Morgen wie gewöhnlich?“

„Wie gewöhnlich – wenn nicht – – Nein doch, wir haben zu viel zu
sprechen – kommen Sie hierher –“

„Ich gehe jetzt auf die Redaktion und zur Post, das Telegramm steht
morgen in allen Zeitungen, Sie werden den ganzen Tag über von
Besuchen belagert sein.“

„Dann flüchte ich lieber –. Ich komme hinaus zu Ihnen, bald nach
Tisch. Ich will martisch lernen“, setzte sie mit einem halb
komischen Seufzer hinzu. „Ach, Ell, was werden die nächsten Zeiten
bringen?“

„Großes für die Menschen!“ war seine ernste Antwort.

Ell ging.

\section{22 - Schnelle Fahrt}

Auf die Veröffentlichung der Depesche Torms folgten heiße Tage für
Isma. Glückwünsche, Anfragen und Besuche, teilnahmsvolle und
neugierige, drängten sich. Einige Zeitungen schickten ihre
Reporter, um ihren Lesern möglichst genau die Ansicht von Frau Torm
über die Zustände auf dem Nordpol auseinanderzusetzen. Soweit Isma
die Besuche nicht ablehnen konnte, beschränkte sie sich darauf zu
sagen, sie teile die Vermutungen, welche Friedrich Ell sogleich am
Tag nach dem Erscheinen des Telegramms in der Vossischen Zeitung
ausgesprochen habe.

Über die Möglichkeit einer Besiedelung des Pols durch die
Marsbewohner erhob sich ein heftiger Streit in den Tagesblättern.
Ein großer Teil des Publikums fand die Aussicht höchst interessant,
welche sich für einen Verkehr mit den Martiern eröffnete. Andere
hätten am liebsten die ganze Depesche für Schwindel erklärt; da
dies aber nicht anging, behaupteten sie, Torm müsse sich jedenfalls
getäuscht haben. Es wäre ja möglich, daß es Bewohner des Mars gebe,
sie könnten aber nicht auf die Erde gelangen. Und selbst wenn sie
das könnten, so wäre nicht einzusehen, warum sie nicht nach Berlin
oder Paris kämen, sondern sich das Vergnügen machten, eine
Riesenerdkarte am Nordpol zu konstruieren. Ein berühmter Physiker
erklärte es als absolut unmöglich, daß menschenähnliche Wesen
jemals von einem Planeten nach dem andern durch den Weltraum
hindurchdringen könnten. Darauf stellte ein Geologe eine höchst
geistreiche Hypothese auf, derzufolge sich notwendigerweise am Pol
ein Vulkan bilden müsse, aus welchem von Zeit zu Zeit ein Teil des
Erdinnern herausquelle. Die Lavaablagerungen seien infolge einer
zufälligen Ähnlichkeit von Torm für eine Karte gehalten worden.
Endlich erklärte sich der Redakteur der ›Geographischen
Mitteilungen‹ dahin, daß es keinen Zweck habe, Vermutungen
aufzustellen, weil man überhaupt erst weitere Nachrichten abwarten
müsse. Der Mann hatte recht, fand aber am wenigsten Beifall.

Die Friedauer fühlten sich mehr wie je befriedigt. Die Beachtung,
welche ihre Stadt in der ganzen Welt fand, gab eine erhabene
Veranlassung, um Glossen über Frau Torm daran zu knüpfen, wenn sie
ihr in der Nähe der Ellschen Besitzung begegneten oder Ell an ihrer
Seite durch die Gänge des Parkes wandelte; das taten sie zwar schon
seit Jahren, aber jetzt war es doppelt schön, noch dieses
Privatwissen über das allgemeine hinaus zu haben. Isma selbst
kümmerte sich darum nicht. Mehr wie je war ihr das Urteil der
Menschen gleichgültig geworden, während ihr der tägliche Verkehr
mit Ell allein einigermaßen Beruhigung gewähren konnte. Ell hatte
sie schon geliebt und um sie geworben, als sie noch als Isma Hilgen
bei ihrer früh verwitweten Mutter in Berlin lebte. Damals hatte sie
seine Bewerbung zurückgewiesen. Die Neigung des seltsamen Mannes
konnte sie zwar nicht unberührt lassen, aber von der Fremdartigkeit
seines Wesens war sie immer wieder abgestoßen worden. Als sie mit
Torm sich verlobte, war Ell in die Fremde gegangen. Nach seiner
Rückkehr hatte er sich ihr in uneigennützigster Freundschaft
genähert. Sie wußte, daß er sie liebte, und sie ahnte die Kämpfe,
die er im stillen mit seiner Leidenschaft führte. Aber sie hing an
ihrem Mann mit inniger Zuneigung, und sie hatte Ell bald im Anfang
gesagt, daß daran eine Änderung niemals eintreten würde. Damals gab
er ihr das Versprechen, daß sie niemals durch ihn eine Störung
ihres Glückes, ja nur eine trübe Stunde erfahren solle. Und dies
Versprechen hatte er die Jahre hindurch gehalten. Wohl hatte manche
andere sein Interesse gewonnen, und obwohl Isma sein gutes Recht
dazu anerkannte, hatte sie sich dann doch eines schmerzlichen
Gefühls nicht erwehren können. Aber sie wollte sich über ihr Gefühl
keine Rechenschaft geben. Sie wußte, daß er ihrer Nähe, ihrer
Freundschaft und ihres Glückes bedurfte, und jene seltsame
Abstraktionsgabe, das Erbteil der Martier, in seiner Vorstellung
sein Gefühl zu trennen von den harten Pflichten der Wirklichkeit,
ermöglichten es Ell, als ein treuer und aufopfernder Freund ihr zu
dienen. So herrschte zwischen beiden ein unbedingtes Vertrauen, das
Isma die volle Sicherheit gab, auch sein Freundschaftsverhältnis
mit Torm könne unter ihrem Verkehr nicht leiden. Zum Glück waren
alle in der Lage, über das Gerede derer, die sie nicht kannten, die
Achseln zucken zu können.

Es war am achten September, am dritten Tag nach der Ankunft des
Tormschen Telegramms. Gegen Abend hatte Ell seinen gewohnten
Spaziergang mit Isma gemacht, die über das Ausbleiben jeder
weiteren Nachricht lebhaft beunruhigt war. Auch Ell war es schwer
geworden, ihr Mut zuzusprechen. Denn er sagte sich, daß man
allerdings eine Nachricht hätte erwarten dürfen. Die Expedition
hatte eine Anzahl Brieftauben mit, und man mußte annehmen, daß sie
alsbald über die weitere Richtung ihrer Reise eine Depesche
absenden würde. Doch die geflügelten Boten konnten auf dem weiten
Wege leicht verunglücken. Es ließ sich zunächst gar nichts tun als
geduldig warten.

Eine milde Spätsommernacht lag über der Stadt, alles in tiefe
Dunkelheit begrabend. Der Mond war noch nicht aufgegangen, ein
leichter Wolkenschleier verhüllte das Sternenlicht. Regungslos
streckten die hohen Bäume ihre dichtbelaubten Zweige aus und
deckten mit undurchdringlicher Finsternis die Rasenplätze, die sich
zwischen ihnen auf dem Hügel hinbreiteten, wo Ell seine Warte
erbaut hatte. Es war schon spät, und nur aus der hohen geöffneten
Tür, die von Ells Arbeitszimmer nach der weinumlaubten Veranda
führte, schimmerte noch Licht. Von dort ging eine Freitreppe in den
Garten. Ell war an seinem Schreibtisch mit einer Arbeit
beschäftigt, die er schon seit Jahren betrieb, einer Darstellung
der Verhältnisse der Marsbewohner und einer Anleitung, ihre Sprache
zu erlernen. Er wollte diese Bücher in dem Augenblick
veröffentlichen, in welchem die ersten Martier mit den Menschen
zusammenträfen.

In seine Arbeit vertieft, vernahm er nicht, daß langsame Schritte
über den Kiesweg des Gartens sich nahten, daß jemand die Treppe der
Veranda erstieg. Erst als der Tritt auf der Veranda selbst erklang,
drehte er sich um. In der Tür stand die Gestalt eines Mannes.

„Wie kommen Sie in den verschlossenen Garten?“ fuhr Ell auf, indem
er nach der Waffe auf seinem Schreibtisch griff.

Seine vom Licht des Arbeitstisches geblendeten Augen konnten nicht
sogleich erkennen, wen er vor sich habe.

„Ich bin es!“ sagte eine ihm wohlbekannte Stimme.

Ell zuckte zusammen und sprang empor. Er faßte mit den Händen nach
seinem Kopf.

„Eine Halluzination“, war sein Gedanke.

Die Gestalt trat näher. Ell wich zurück.

„Ich bin es wirklich, Herr Doktor, es ist Karl Grunthe.“

„Grunthe!“ rief Ell. „Ist es möglich? Wo kommen Sie her?“

„Direkt vom Nordpol, den ich heute gegen Mittag verließ.“

Ell hatte ihm die Hände entgegengestreckt. Bei diesen Worten trat
er wieder zurück.

„Ich will Ihnen etwas sagen, Grunthe“, begann er. „Ich bin bei der
Arbeit eingeschlafen, ich träume – Sie können es ja nicht sein. Das
sehen Sie doch ein. Das Tor ist ja auch verschlossen, Sie können
nicht über die Mauer klettern.“

Grunthe trat jetzt auf ihn zu. Er schüttelte ihm die Hände.
„Glauben Sie’s!“ sagte er. „Sie träumen nicht, Sie wachen. Es ist,
wie ich sage. Erlauben Sie mir ein Glas Wasser, richtiges, frisches
Quellwasser, das habe ich vermißt. Hier, trinken Sie auch. Kommen
Sie, setzen Sie sich. Ich will Ihnen alles erklären. Aber so
schnell geht das nicht.“

Ell faßte Grunthe an den Schultern und schüttelte ihn. Er lachte.
Dann setzte er sich und starrte Grunthe noch einmal an.

Grunthe zog seine Uhr und verglich sie mit dem Chronometer in Ells
Zimmer.

„Keine Abweichung“, sagte er.

„Sie sind es doch, Grunthe!“ rief Ell. „Jetzt glaube ich es.
Verzeihen Sie, aber nun bin ich wieder klar. Um Gottes willen,
sprechen Sie, schnell! Wo ist Torm?“

„Torm ist nicht zurückgekehrt“, sagte Grunthe langsam, indem sich
die Falte zwischen seinen Augen vertiefte.

Ell sprang wieder auf.

„Er ist verunglückt?“

„Ja.“

„Tot?“

„Wahrscheinlich. Der Ballon wurde in die Höhe gerissen. Wir
verloren das Bewußtsein. Als wir wieder zu uns kamen, war Torm
verschwunden. Er ist bis jetzt nicht wiedergefunden worden.“

„Bis jetzt? Das heißt, Sie haben noch eine Hoffnung?“

„Auch der Fallschirm fehlte, es ist möglich, daß er sich damit
gerettet hat – aber sehr unwahrscheinlich. Wohin sollte er gekommen
sein?“

Ell trat an die Tür und starrte in die Nacht, wortlos – dann drehte
er sich plötzlich um.

„Und Sie, Grunthe?“ rief er. „Und Saltner?“

„Wir wurden von den Bewohnern der Polinsel gerettet. Mich brachten
sie hierher in einem Luftschiff. Saltner ist noch am Pol, er reist
morgen auf den Mars. Da sind seine Briefe, da sein Tagebuch.“ Er
legte zwei Päckchen auf den Tisch.

„So sind sie da?“ fragte Ell fast jubelnd.

„Sie sind da. Wir haben Ihren Sprachführer gefunden. Und wenn Sie
sich gefaßt haben, so kommen Sie mit mir. Ich bin nicht allein,
meine Begleiter sind hier.“

„Wo? Wo?“

„Auf dem mittleren Rasenplatz neben dem Sommerhäuschen liegt das
Luftschiff. Man erwartet Sie!“

Ell wollte hinausstürzen. Die Füße versagten ihm. Er setzte sich
wieder.

„Ich kann noch nicht. Bitte, erzählen Sie mir erst noch etwas. Dort
steht Wein, geben Sie mir ein Glas!“

Grunthe holte den Wein. Dann schilderte er kurz ihr Schicksal am
Pol, die Aufnahme bei den Martiern, die Station des Ringes.
Allmählich wurde Ell ruhiger.

Er holte eine Laterne.

„Gehen wir!“ sagte er.

Grunthe nahm die Laterne. Sie durchschnitten die dunkeln Gänge des
Gartens. An dem bezeichneten Rasenplatz angekommen, blieb Grunthe
stehen und erhob die Laterne. Ein dunkler Körper zeigte sich
undeutlich in der Mitte des Platzes. Grunthe gab die Losung: „Bate.
Grunthe it Ell.“

Darauf setzte er in der Sprache der Martier hinzu: „Wir sind
vollständig ungestört und sicher. Sie können Licht machen.“

Seit dem Tod seines Vaters hatte Ell kein martisches Wort mehr
vernommen. Die Laute berührten ihn überwältigend. Jetzt sollte er
den Numen, den Stammesgenossen des Vaters entgegentreten.

Ein mattes Licht durchglänzte den Bau des Luftschiffs und ließ eine
Falltreppe erkennen, welche auf das Verdeck führte. Ell folgte dem
vorankletternden Grunthe. Oben erwartete sie der wachhabende
Steuermann und geleitete sie in das Innere des Schiffes hinab.
„Warnen Sie den Herrn“, sagte er zu Grunthe, „wir haben
Marsschwere.“

„Ich danke“, versetzte Ell, „ich passe auf.“

Der Steuermann sah den martisch redenden Menschen verwundert an,
ging aber schweigend voran. Sie durchschnitten einen schmalen Gang,
zu dessen beiden Seiten die Mannschaften in Hängematten nach ihrer
anstrengenden Fahrt ausruhten, und befanden sich vor der Tür der
Kajüte. Sie öffnete sich. Der Steuermann trat zurück; Grunthe und
Ell standen in dem hellerleuchteten Raum.

Ell schrak zusammen und drohte das Gleichgewicht zu verlieren, da
er seine Bewegungen der geringen Schwere noch nicht anzupassen
vermochte. Von Grunthe gestützt, starrte er sprachlos mit
weitgeöffneten Augen auf die hohe Gestalt, die ihm gegenüberstand.

„Vater“, wollte es sich auf seine Lippen drängen – –

„Mein Freund, Dr. Friedrich Ell“, sagte Grunthe vorstellend. „Der
Herr Repräsentant der Marsstaaten, Ill.“

„Ill re Ktohr, am gel Schick – Ill, Familie Ktohr aus dem
Geschlechte Schick“, sagte Ill mit Betonung, indem er Ell scharf
beobachtete. Auch ihm klopfte das Herz, er sah seine Vermutung
bestätigt. „Ich bin“, setzte er hinzu, „der jüngste Bruder des
Kapitän All, der im Jahre –“

„Mein Vater!“ rief Ell. „Er war mein Vater! Und so sah er aus, nur
gebeugter vom Druck –“

Ill schloß seinen Neffen in die Arme und ließ ihn dann sanft auf
den Diwan gleiten.

„Ich dachte es mir“, sagte er, „als die erste Nachricht zu uns kam,
daß ein Ell auf der Erde unsre Sprache kenne. Darum erbot ich mich
freiwillig hierherzugehen, als einer von uns den Auftrag übernehmen
sollte. Laß dich noch einmal ansehen! Welch ein Glück, dich zu
finden! Und nicht bloß für uns. Nun habe ich die Hoffnung, daß sich
die Planeten verstehen werden.“

\tb

Stunden vergingen, und noch immer saßen der Oheim und sein Neffe in
der Kajüte des Raumschiffes in eifrige Besprechungen vertieft.
Grunthe hatte sich sogleich nach der Erkennungsszene zurückgezogen.
Er war nach Torms Arbeitszimmer gegangen. Das Bedürfnis nach Schlaf
fühlte er nicht, denn fast während der ganzen Fahrt hatte er in
Schlummer gelegen. Erst in der Abenddämmerung hatte man ihn
geweckt. Er sah unter sich das Häusermeer von Berlin, welches das
Luftschiff in weitem Bogen umkreiste. Man ließ sich jetzt
Erklärungen von ihm über die Bedeutung der hervorragenden Gebäude
geben und dann den Weg nach Friedau zeigen, das man von Berlin aus
mit dem Luftschiff in 25 Minuten erreichen konnte. Man hatte jedoch
im Dunkeln zu der Fahrt absichtlich eine Stunde gebraucht. Längere
Zeit nahm dann die Landung in Anspruch, weil diese ganz langsam und
geräuschlos vor sich gehen sollte. Die Martier wollten dabei nicht
bemerkt werden, um nicht während ihrer Anwesenheit im Land
irgendwie die Aufmerksamkeit auf sich zu ziehen.

Sie wußten ja nicht, ob man sie nicht bei der Abfahrt stören
könnte, und wollten auf alle Fälle jeden Konflikt vermeiden. Ob sie
dagegen bei ihrer freien Fahrt in der Luft zufällig einmal gesehen
wurden, darauf kam es ihnen jetzt nicht mehr an. Nachdem sie
Grunthe zurückgebracht, mußte es ja doch bekannt werden, daß sie da
waren und mit ihren Luftschiffen über die Erde fuhren. Nur ihre
volle Freiheit wollten sie nicht aufs Spiel setzen.

Grunthe hatte sich in Torms Zimmer die Zeitungen der letzten Wochen
zusammengesucht. Es war ihm ein Bedürfnis, sich über die Vorgänge
bei den Menschen während der Zeit seiner Abwesenheit zu
unterrichten. Aber wie engherzig und beschränkt kam ihm jetzt alles
vor! Und dennoch, er war entschlossen, das Mögliche zu tun, um den
Einfluß der Martier abzuwehren.

Die ersten Spuren der Dämmerung zeigten sich im Osten, als Ell mit
fieberhaft leuchtenden Augen wieder eintrat.

„Sind sie schon fort?“ fragte Grunthe, sich erhebend.

„Noch nicht.“

„Aber es wird bald hell.“

„Ill bleibt noch bis zur Nacht. Ich soll ihn begleiten, er will
über die Hauptstädte Europas einen Überblick gewinnen. Aber ich
kann heute früh noch nicht fort. In der Sache ist es eigentlich
nicht recht zu zögern, aber ich kann nicht.“

„Sie dürfen freilich jetzt nicht fort. Wir müssen die Resultate der
Expedition bekanntmachen. Sie sind dabei unentbehrlich.“

„Wir haben uns schon geeinigt. Ich will nur eben Anordnung treffen,
daß heute niemand im Garten zugelassen wird. Auf den alten Schmidt
können wir vertrauen, er wird die Tür geschlossen halten und wie
ein Cerberus wachen. Mein Oheim ist mit dem Ruhetag einverstanden,
den die Mannschaft wie er selbst nötig hat. Jetzt will er mir nur
einmal die Leistungsfähigkeit des Luftschiffs bei größter
Geschwindigkeit zeigen. Die Luft ist ganz still. Wir wollen uns
Wien betrachten. In einer Stunde, noch vor Sonnenaufgang, sind wir
zurück. Wir fahren jetzt nach Osten, über Wien wird es schon hell
genug sein. Kommen Sie mit, wir können die Zeit zum Erzählen
benutzen. Nachher frühstücken wir zusammen.“

Er sprach in großer Aufregung und suchte dabei nach seinem Mantel.

„Sie brauchen weiter nichts mitzunehmen“, sagte Grunthe. „Pelze
sind im Schiff. Instruieren Sie nur Schmidt, daß er niemand
einläßt. Ich aber will lieber hierbleiben.“

Ell weckte den Kastellan. Es dürfe niemand in den Garten. Auch die
Sternwarte bleibe heute geschlossen. In besonderen Fällen oder wenn
Bekannte kämen, solle man ihn selbst rufen. Er verlasse sich auf
sein unbedingtes Schweigen über alles, was er etwa
Außergewöhnliches sehe.

Der alte Mann, der schon seinem Vater gedient hatte und mit Ell
nach Deutschland gekommen war, versprach sein Bestes. Seine Frau,
welche auch die häusliche Bedienung für Ell führte, kam niemals
über ihr eigenes kleines Gemüsegärtchen, das außerhalb der
Gartenmauer lag, hinaus. Von ihr war keine Störung zu befürchten.

Ell begab sich nach dem Rasenplatz. Das Luftschiff war zur Abfahrt
bereit. Die Lichter wurden gelöscht. Geräuschlos hob es sich
senkrecht in die Höhe. Die Stadt lag im Schlummer, niemand bemerkte
den dunkeln Körper, der in wenigen Augenblicken in der Dämmerung
entschwunden war.

Ell saß stumm in seinen Pelz gehüllt und blickte durch die
Robscheiben dem schnell emporsteigenden Frührot entgegen.

„Ein neuer Tag“, sagte er leise, „wirklich ein Tag! Ich fliege! O
heiliger Nu!“

Aber sie, Isma, was würde sie sagen? Er vergaß seine Umgebung. Das
Herz krampfte sich ihm schmerzhaft zusammen. Wie sollte er ihr das
Schreckliche mitteilen? Da ihm alles geglückt, da seine höchste
Sehnsucht erfüllt, seine Heimat wiedergefunden war, da sollte ihr
das Lebensglück entrissen werden? Er suchte sich in ihre Seele zu
versetzen und vermochte es nicht. Er trauerte um den Freund, und
inniges Mitgefühl mit der Freundin drängte die Tränen in sein Auge.
Er sah sie die schmalen Hände ringen, er sah, wie ihre großen
dunkelblauen Augen starr wurden. Er hätte sein Leben dafür gegeben,
diesen Schmerz ihr abzunehmen, ihr den Verlorenen zu retten und
wiederzubringen. Es war aussichtslos. Was vermochte er für sie zu
tun? Und in allem Schmerz konnte er es nicht hindern, daß es wie
eine leise Hoffnung ihn durchzog, ob es ihm nicht möglich sei, ihr
das entschwundene Glück zu ersetzen. Er drängte den Gedanken
zurück. Er dachte an seine nahen, großen Aufgaben. Aber die nächste
war ja doch – sie zu benachrichtigen.

Eine Frage Ills riß ihn aus seinen Grübeleien.

Zur Rechten erglänzte die Kette der Alpen im Licht der aufgehenden
Sonne. Das Luftschiff breitete seine Schwingen aus und umkreiste,
sich tiefer senkend, in weitem Bogen die Kaiserstadt an der Donau.
Drei-, viermal strich es bis dicht über den Spitzen der Türme hin,
dann erhob es sich wieder und floh vor den Strahlen der Morgensonne
nach Nordwesten. Es erreichte Friedau, noch bevor der erste
Sonnenstrahl die Kuppel der Sternwarte, des höchsten Punktes der
Umgebung, vergoldete, und ließ sich langsam auf den Rasenplatz
nieder.

Einige Arbeiter, die aufs Feld gingen, liefen herzu, aber da sie
das Schiff hinter den Bäumen des Gartens verschwinden sahen,
setzten sie ihren Weg wieder fort. Sie waren gewohnt, die
Übungsballons der Luftschiffer bei Friedau aufsteigen zu sehen, und
wunderten sich daher nicht weiter, daß einmal ein so sonderbarer
Ballon hier niederging.

\section{23 - Ismas Entschluß}

Um dieselbe Zeit wurde Frau Isma Torm durch heftiges Läuten aus dem
Schlummer geweckt.

Man brachte ihr ein Telegramm. Mit klopfendem Herzen las sie:

„Hammerfest, den 9. September.

Brieftaube ›Ballon Pol‹ brachte folgende Nachricht:

Frau Isma Torm. Friedau. Deutschland, 21. August, 2 U. 30 Min.
nachm. M.E.Z.

Ballon durch unbekannte Kraft in die Höhe gerissen. Ich verlor das
Bewußtsein. Erwachte, als der Ballon auf dichte Wolkendecke schnell
abstürzte. Korb gekentert. Ballon nur durch stärkste Erleichterung
zu retten. Grunthe und Saltner bewußtlos, nicht transportierbar.
Ich verließ den Ballon mit dem Fallschirm, konnte Brieftauben
mitnehmen. Ich fiel langsam durch

Wolken, trieb vom Pol in unbekannter Richtung ab, konnte mich auf
Festland retten. Entdeckte Spuren von wandernden Eskimos und fand
ihr Lager. Ziehe mit ihnen nach Süden, habe noch zwei Tauben. Hoffe
auf glückliche Heimkehr. Sei unbesorgt. Ich bin unverletzt und bei
Kräften. Torm.“

Sie klammerte sich an die letzten Worte. „Hoffe auf glückliche
Heimkehr. Sei unbesorgt. Ich bin unverletzt und bei Kräften.“ Aber
wo? Wo? Jenseits unzugänglicher Meere und Eiswüsten, kurz vor
Beginn der ewigen Nacht, angewiesen auf das Mitleid einiger
armseliger Eskimos! Der Ballon gescheitert – die gehofften, stolzen
Resultate verloren! Wie konnte er heimkehren – und wann?

Und sie – sie hatte ihn ermutigt, ihm zugeredet, als er darum
sorgte, sie allein zurückzulassen. War sie nicht mitschuldig an
seinem Unglück? Hatte sie nicht zu sehr dem Freund vertraut, der
des Gelingens so sicher schien? Eine furchtbare Angst erfaßte sie.
Hätte sie ihn nicht beschwören müssen, das gefährliche Unternehmen
um ihretwillen zu unterlassen? Sie hatte sich eingebildet, der
großen Sache, der Wissenschaft mutig das Opfer ihres häuslichen
Glückes zu bringen, aber nun kam es über sie wie eine schreckliche
Anklage – hätte sie den Mut auch gehabt, wenn nicht Ell sie gebeten
hätte? Wenn sie nicht dem Freund zuliebe, dem sie das eine
Lebensglück versagt, nun zur Erreichung seines innigsten Wunsches
ein Opfer hätte bringen wollen? Und wenn das Opfer angenommen war?
Sie schauderte zusammen. Nein, nein, sie wollte nicht mutlos sein.
Das durfte sie sich ja sagen, sie hatte sich nie verhehlt, daß sie
jeden Augenblick auf das Schlimmste gefaßt sein mußte. Aber was sie
dann tun würde? Das hatte sie niemals sich zur vollen Klarheit
gebracht. Jetzt mußte es sein. Sie wollte handeln. Wenn Hilfe
möglich war – es gab von den Menschen nur einen, der hier helfen
konnte. O, er würde ihr helfen! Sie glaubte an ihn.

Eine Stunde später zog sie die Klingel an dem großen eisernen
Gitter, das den Vorgarten des Wohngebäudes neben der Sternwarte von
der Straße abschloß.

„Ist der Herr Doktor schon zu sprechen?“ fragte sie den öffnenden
Kastellan.

Der Alte nahm sein Käppchen ab und kratzte sich verlegen hinter dem
Ohr.

„Ei, ei, die Frau Doktor sind es? Hm! Hm! Na, ich will gleich
einmal fragen. Kommen Sie nur inzwischen herein. Es ist freilich –
Hm! –“

„Sagen Sie, ich müßte den Herrn Doktor sofort sprechen, es sind
wichtige Nachrichten angekommen.“

Der Alte schlurfte ins Haus.

Ell beriet mit Grunthe die Form, welche den ersten Mitteilungen zu
geben sei, als ihm Frau Torm gemeldet wurde.

Er sprang auf und warf die Feder weg.

„Führen Sie die gnädige Frau sogleich in die Bibliothek.“

„Es sind wichtige Nachrichten da, sagte die Frau Doktor.“ Mit
diesen Worten ging der Kastellan ab.

„Sie hat Nachrichten!“ rief Ell erbleichend. „Und sie kommt selbst,
um diese Zeit! Woher kann sie es wissen?“

Er stürzte hinaus. Vor der Tür des Bibliothekzimmers hielt er an.
Er mußte sich erst sammeln. Dann trat er ein, ruhig, gefaßt. Aber
das Herz schlug ihm. Sein Gesicht war bleich und übernächtig.

Isma stand mitten im Zimmer und stützte ihre Hand auf den großen
Tisch, der mit aufgeschlagenen Kartenwerken und Tabellen bedeckt
war. Sie fand keine Worte.

„Isma“, sagte er, „Sie haben – was wissen Sie?“ Sie brach in
Schluchzen aus. Er eilte an ihre Seite. Wieder lehnte sie an seiner
Schulter. Er führte sie an das Sofa.

„Fassen Sie sich, liebste Freundin, fassen Sie sich!“

„Ich weiß nicht, was ich tun soll“, sagte sie unter Tränen. Sie zog
die Depesche aus ihrer Tasche und reichte ihm das zerknitterte
Papier.

Ell las.

Er atmete tief auf.

„Gott sei gedankt!“ rief er aus tiefstem Herzen.

Isma sprang auf und wich zurück. Ihr Blick fiel feindlich auf ihn.
ihre Augen wurden starr. Sie drohte zusammenzubrechen.

„Was ist Ihnen, Isma?“

„Ich – ich –“, sagte sie, die Hand auf das Herz pressend, „ich habe
wohl nicht recht verstanden – oder – oder – sagten Sie nicht –?“

„Gott sei Dank, sagte ich, denn Ihr Mann ist gerettet.“

„Gerettet?“

„Ja, hier steht es ja.“

„Gerettet?“

„Ihre Nachricht ist jünger als die meinige, ist von ihm selbst“,
fuhr Ell fort. „Ich aber empfing diese Nacht durch Grunthe die
Nachricht, daß der Ballon abgestürzt und Ihr Mann verschwunden sei.
Ich glaubte ihn tot und wußte nicht, wie ich Ihnen, Isma – aber was
ist Ihnen?“

Isma ergriff seine Hände. „O, Ell, Ell, verzeihen Sie mir!“

Er sah sie erstaunt an.

„Sie halten ihn für gerettet?“ rief sie, indem ihr das Blut in die
Wangen stieg. „Im ewigen Eis, in der Polarnacht? Wie soll er
gerettet werden?“

„Da er glücklich aus dem Ballon auf die Erde gelangt ist und im
Schutz der Eskimos steht, so droht ihm unmittelbar keine Gefahr.“

„Aber der Winter?“

„Wo die Eskimos überwintern, wird es Ihrem Mann auch gelingen. Es
ist gewiß keine angenehme Aussicht, aber wie viele Forscher haben
schon einen Winter in den Schneehütten der Eskimos zugebracht. Und
darauf war er, mußten wir alle gefaßt sein, daß ein solcher Unfall
eintrat. Nein, Isma, liebste Freundin, ängstigen Sie sich nicht.
Wir werden dafür sorgen, daß im Frühjahr auf allen Seiten des Pols
nach ihm gesucht wird. Vielleicht erhalten wir noch eine Nachricht.
Er hat ja noch Tauben. Sehen Sie –“, er streichelte ihre Hand und
versuchte zu lächeln, „verzeihen Sie mir, aber die Depesche, die
Ihnen nur Trauriges meldete, für mich war sie eine Erlösung. Alles,
was Grunthe und Saltner von Ihrem Mann wußten, bestand darin, daß
er aus dem Ballon verschwunden war, als sie von ihrer Ohnmacht
erwachten. Der Fallschirm wurde im Meer gefunden, von Torm keine
Spur. Sie können sich denken, Isma, was ich in Ihrer Seele fühlte,
wie mir zumute war, als ich Sie jetzt vor mir sah. Da atmete ich
auf, als ich Ihre Depesche las. Nach dem, was ich wußte, ist es
vielleicht die beste Nachricht, die sich überhaupt erhoffen ließ.
Ich brauche nicht zu sagen, wie sehr ich den Unfall Ihres Mannes
bedauere; Sie aber dürfen stolz sein. Er hat sich selbst geopfert
und die Gefährten dadurch gerettet. Alle Resultate der Expedition
sind geborgen, selbst meine kühnsten Hoffnungen erfüllt.“

Isma starrte in die Ferne. Das Schicksal Torms nahm noch alle ihre
Gedanken in Anspruch.

„Und ist Ihnen denn dies alles gleichgültig geworden?“ fragte Ell.
„Sie fragen nicht einmal, woher ich meine Nachricht habe?“

„Wie können wir uns des Erreichten freuen, und er, dem wir es
verdanken, hat nichts von alledem? Den langen Winter – ach, wohl
noch ein Jahr. – Ist es denn nicht möglich, noch jetzt, gleich,
etwas für ihn zu tun?“

Ell sah sie schmerzlich enttäuscht an und schüttelte nur den Kopf.

Sie verstand seinen vorwurfsvollen Blick. Eine feine Röte überzog
ihr Gesicht, und sie schlug ihre großen, sanften Augen wie bittend
zu ihm auf. Sie sah entzückend aus. Ell wendete sich ab, er konnte
den Anblick nicht länger ertragen.

Isma legte ihre Hand auf seinen Arm.

„Verzeihen Sie mir, mein lieber Freund“, sagte sie herzlich.
„Erzählen Sie mir! Ich sehe ja selbst ein, daß ich mich in Geduld
fassen muß. Aber es hätte mich so glücklich gemacht, sogleich etwas
tun zu können.“

Ell schwieg noch immer. Er stützte den Kopf in seine Hand.

„Ich hab Sie darum nicht weniger lieb“, sagte Isma einfach.

Beide sahen sich tief in die Augen.

Ell sprang auf und machte einige Schritte durch das Zimmer. Dann
blieb er vor Isma stehen.

„Ich dachte einen Augenblick – eine Möglichkeit, aber nein, es geht
nicht. Es geht nicht.“

Er setzte sich ihr gegenüber.

„Hören Sie zu“, sagte er. „Was ich Ihnen jetzt sage, wird Ihnen
unglaublich erscheinen. Aber die Beweise sollen Sie selbst sehen.
Grunthe ist hier. Und Saltner ist auf der Reise nach dem Mars. Oben
in meinem Garten liegt ein Luftschiff der Martier. Mein Oheim Ill,
der Bruder meines Vaters, hat Grunthe darin hierhergebracht. Die
Fahrt nach dem Pol dauert sechs Stunden –“

„Um Gottes willen, Ell, hören Sie auf!“ rief Isma zurückweichend,
die gefalteten Hände nach ihm ausstreckend. In ihren Augen malte
sich Angst. Sie fürchtete für seinen Verstand. War das seine fixe
Idee, die jetzt mit ihren Wahnvorstellungen zum Ausbruch kam?

Er stand auf und ging zur Tür. Isma blieb ratlos sitzen. Nur wenige
Augenblicke, dann sprang sie auf.

Grunthe trat in das Zimmer. Er machte seine steife Verbeugung.

Isma starrte auf ihn wie auf eine Erscheinung.

„Lesen Sie diese Depesche“, sagte Ell zu Grunthe. „Frau Torm hat
sie heute früh empfangen.“

Grunthe las, sah noch einmal nach dem Datum, und sagte dann: „Das
ist eine sehr günstige Nachricht, unter den einmal vorhandenen
Umständen.“

„Und nun bitte, Grunthe“, rief Ell, „tun Sie mir den Gefallen und
geben Sie Frau Torm einen kurzen Bericht über Ihre Erlebnisse.
Kommen Sie, setzen wir uns.“

Grunthe sprach in seiner knappen, fast trockenen Weise. Da war
nichts übertrieben, keine Vermutungen, kein subjektives Urteil,
alles klar wie ein mathematischer Beweis.

Isma saß regungslos. Ihre weitgeöffneten Augen hingen an Ell. Es
überkam sie wie ein Gefühl der Ehrfurcht.

„Und nun ich hier bin“, schloß Grunthe, „darf ich keine Minute
versäumen, den Bericht fertigzustellen. Wir haben alle unsre Kräfte
anzustrengen, das zu beweisen, was uns niemand wird glauben wollen.
Ich darf daher wohl auf Entschuldigung rechnen, wenn ich mich jetzt
wieder zurückziehe. Würden Sie mir noch einen Augenblick schenken?“
setzte er zu Ell gewendet hinzu.

Er verbeugte sich gegen Isma und wollte gehen.

Da sprang Isma auf und trat dicht vor Grunthe, der mit
zusammengekniffenen Lippen stehenblieb.

„Ist es wahr“, fragte sie, „das Luftschiff liegt noch draußen?“

„Gewiß.“

„Und in sechs Stunden kann man zum Nordpol gelangen?“

Grunthe nickte bestätigend.

„Ich bin heute früh selbst in einer Stunde nach Wien und wieder
zurückgefahren“, setzte Ell hinzu.

„Ich danke Ihnen“, sagte Isma zurücktretend.

„Entschuldigen Sie mich auf einen Augenblick – ich bin sogleich
wieder hier“, sagte Ell zu Isma, indem er mit Grunthe das Zimmer
verließ.

Sie nickte schweigend. Ihre Gedanken waren bei dem Luftschiff. In
sechs Stunden konnte man am Nordpol sein – nur sechs Stunden! So
lange braucht der Schnellzug nach Berlin. Das ist eine
Spazierfahrt. Sechs Stunden nur trennten sie von Hugo. – – Wenn das
Glück günstig war, wenn das Schiff die richtige Bahn beschrieb, so
mußte man ihn bemerken, so konnte man ihn aufnehmen und
zurückbringen – noch heute konnte er in Friedau sein –

Ach, aber ihn scheiden die Wüsten des Eises, die unzugänglichen
Meere, die noch kein Forscher zu durchqueren vermochte – dort sitzt
er in der kläglichsten Schneehütte, Monat auf Monat, ohne Licht,
ohne Tat – in ewiger Nacht trauernd und sich sehnend nach der
Heimat, umgeben von den Gefahren des furchtbaren Winters. – – Und
hier daheim, hier reifen die Früchte seiner kühnen Fahrt, hier
drängt sich von Stunde zu Stunde neues, lebendiges Schaffen, hier
vollzieht sich das Unerhörte, noch nie Gewesene – von den Sternen
steigen die Götter herab, um die Menschen zu laden zu ihrem seligen
Wandel – hier, in dieser Stadt, in diesem Hause wird ein neues
Zeitalter geboren, und er weiß nichts davon, kann nicht teilnehmen
an dem Großen, was die ganze Erde erfüllt, an dem Höchsten, was
erlebt wurde und was ihr Herz so erwartungsvoll schlagen macht –
und sie muß es allein erleben –

Und vielleicht nur sechs Stunden –

Allein – den ganzen Winter allein in solcher Zeit, wo Seele zu
Seele gehört – allein? Ja, wenn sie allein wäre! Aber der Freund?
Wo bleibt er? Er ist länger draußen aufgehalten, aber er wird
kommen – er wird kommen so wie heute, dann jeden Tag, der einzige
Vertraute, mit dem sie alles teilen muß, was das Herz bewegt – mit
ihm wird sie allein sein, der ihr so wert ist, so lieb, und nun vor
ihr steht in einem neuen, geheimnisvollen Licht, der Sohn einer
höheren Welt, zu dem sie aufblickt – –

Nein, nein! Sie will nicht allein sein, und nicht allein mit ihm –

Sie ringt die Hände und geht auf und ab im Zimmer. Sie blickt nach
der geschlossenen Tür und glaubt seine Stimme zu hören. Sie blickt
nach der Uhr – und der Gedanke läßt sie nicht los: Nur sechs
Stunden! In sechs Stunden kann alles entschieden sein –

Ja, wenn sie mitfahren könnte, durch die Lüfte reisen nach dem
Reich des Eises, wo er weilt – sie würde ihn finden, sie würde ihn
ausspähen, wo er sich auch bärge, im Boot von Seehundsfell, in der
Hütte von Schnee – bis in die Gletscherspalte würde ihr Auge
dringen – sie schauerte zusammen. Vielleicht schon lag er – sie
mochte das Schreckliche nicht denken. Diese furchtbare Ungewißheit
– nein, das konnte, das wollte sie nicht ertragen. Und die Fragen,
die ewigen, und das Mitleid – und das höhnische Zischeln, ob sie
sich wohl tröstet – – oh!

Sie stampfte mit dem Fuß auf und preßte die Hände krampfhaft
zusammen. Dann stand sie still wie ein Bild aus Stein. Und nun
wußte sie es. Sie atmete tief auf. Die Starrheit löste sich. Ihr
Entschluß war gefaßt.

Nur sechs Stunden!

Das Luftschiff zog sie mit magischer Gewalt an. Sie wollte fort,
sie wollte an den Pol, sie würde ihn finden, den Verlorenen, sie,
Isma Torm. Wenn es ein Unrecht war, daß sie um des Wunsches des
Freundes willen den Mann nicht zurückhielt, so mochte dies ihre
Buße sein, und die seinige!

Sie setzte sich und überdachte alles noch einmal in voller Ruhe.

Es war das Richtige, es mußte so sein.

Isma erhob sich und schritt auf die Tür zu, als ihr Ell aus
derselben entgegentrat.

Er stutzte bei ihrem Anblick. Die Trauer und Angst aus ihren Zügen
war verschwunden. Sie stand aufgerichtet vor ihm. Aus ihren
tiefblauen Augen sprach jene Innigkeit des Gefühls, die ihn immer
hingerissen hatte. Auf ihren Lippen lag es wie ein leises Lächeln.

„Ell“, sagte sie – sie stockte einen Augenblick wie verlegen „bei
Ihrer Freundschaft, wenn Sie mich liebhaben –“

„Isma!“

„Wollen Sie mir eine Bitte erfüllen?“

„Was Sie wollen!“

„Sprechen Sie bei Ihrem Oheim für mich, daß er mich in seinem
Luftschiff mit nach dem Pol nimmt und mich wieder hierherbringt,
wenn wir Hugo gefunden haben – ja, ja – ich werde ihn finden, wenn
ich mit dem Luftschiff ihn suchen darf – o, weigern Sie sich nicht
–“

Sie faßte seine Hände und sah ihn flehend an. Zwei Tränen traten in
ihre Augen.

„Und – kommen Sie selbst mit!“ setzte sie hinzu.

Ell fand nicht sogleich Worte. Das hatte er nicht erwartet.

„O Isma, Isma“, rief er endlich. „Was verlangen Sie? Diese Reise
ist nichts für Sie. Die Nume werden selbst suchen, sie suchen
schon, und was die nicht finden, werden auch Sie nicht finden.“

„Ich werde es. Was sind fremde Augen gegen die der Frau? Ich werde
sehen, wo andere nicht hinblicken. Es sind nur sechs Stunden – so
nahe –, und ich soll hier müßig sitzen – den Gedanken ertrage ich
nicht –“

„Ich bitte Sie, Isma, bedenken Sie meine Lage. Jetzt darf ich, kann
ich nicht von hier fortgehen. Jetzt gilt es, die Menschheit auf den
Besuch der Martier vorzubereiten. Was ich seit Jahren erwartet, ich
muß nun die Konsequenzen ziehen –“

„Es handelt sich vielleicht nur um wenige Tage.“

„Die habe ich meinem Oheim zu andern Zwecken versprochen. Und dann
muß ich wahrscheinlich nach Berlin.“

„Dann bin ich also ganz allein“, sagte Isma leise.

„Nein, nein – ich komme bald wieder.“

Isma wandte sich schweigend ab. Dann kehrte sie plötzlich zurück
und sagte fast hart:

„Führen Sie mich zu Ihrem Oheim, ich will ihn bitten. Und wenn Sie
nicht fortkönnen, lassen Sie mich allein mitgehen. Lassen Sie mich
hingehen, Ell!“

Ell kämpfte mit sich. Mit düstern Blicken starrte er durchs
Fenster.

„Wo ist das Schiff?“ fragte Isma. „Ich will die Nume bitten, sie
werden einer verlassenen Frau nicht abschlagen, was der einzige
Freund ihr nicht gewähren will.“

„Isma, seien Sie vernünftig!“

„Das Vernünftige ist die Pflicht. Und dies ist der einzige Weg, sie
zu erfüllen.“

„Und meine Pflicht ist die Versöhnung der Planeten. Dagegen muß das
Geschick des einzelnen zurücktreten.“

„Darum eben gehe ich allein.“

„Das werde ich nie zugeben.“

„Ich will“, sagte Isma finster. „Ich will zu meinem Mann.“

Ell stöhnte. Er sah, wie sie entschlossen der Tür zuschritt. Sie
drehte sich noch einmal um, mit tiefer Trauer im Antlitz.

„Bleiben Sie, Isma“, rief er. „Ich bringe Ihnen Hugo, wenn es in
der Macht der Menschen steht und der Nume!“

„Nehmen Sie mich mit!“

„Kommen Sie zu Ill. Alles hängt von seiner Entscheidung ab.“

Ell brachte Isma zu seinem Oheim. Es hätte ihr wenig genutzt, ihre
Sache bei Ill zu vertreten, wenn nicht Ell sie zu der seinigen
gemacht hätte. Denn Ill verstand nicht deutsch, Ell mußte daher die
Verhandlungen führen. Ill, der Isma mit herzlichster Teilnahme
begegnete, versprach sofort, daß nach seiner Rückkehr mit Hilfe des
Luftschiffes die sorgfältigste Durchforschung des arktischen
Gebietes vorgenommen werden solle, so lange die Martier dazu noch
Zeit hätten. Dazu wäre er ohnehin entschlossen gewesen, und nur die
Zurückführung Grunthes und die Aufsuchung Ells hätten zuvor
erledigt werden müssen. Übrigens würde schon jetzt nach Torm
gesucht, da noch ein kleineres Luftboot, freilich zu weiteren
Reisen nicht verwendbar, in Dienst gestellt werde. Er sähe daher
nicht ein, wozu es notwendig sei, daß Ell oder gar Isma zu diesem
Zweck ihm an den Pol folgen sollten. Ersterer wäre jetzt in
Deutschland nicht zu entbehren, um Grunthe in der Darstellung der
Resultate der Expedition zu unterstützen. Man würde ihn auch
jedenfalls seitens der Regierung zu Rate ziehen.

Ell gab dies gern zu; es war ja vollständig seine Ansicht. Er
sagte, daß er nur den innigsten Wunsch von Frau Torm vertrete. Isma
brachte nun selbst ihre Bitte vor, mit rührendem Ton, in Ills
Gegenwart. Ell, der jetzt erst hörte und im übrigen erriet, was
Isma zur Reise antrieb, fühlte seinen Widerstand gebrochen. Er
unterstützte nunmehr ihre Bitten und wollte sie unter keinen
Umständen verlassen. Er stellte daher Ill vor, daß sich seine Reise
wohl mit seinen Pflichten gegen die Martier vereinen lasse, da sie
doch nicht länger als acht bis zehn Tage dauern würde. Denn
gleichviel, ob Torm gefunden werde oder nicht, vor ihrer Abreise
nach dem Mars würden ja die Martier ihn und Isma zurückbringen. In
dieser Zeit aber sei er um so eher entbehrlich, als sich die erste
Aufregung über das Erscheinen der Martier erst einigermaßen legen
müsse, ehe es zu ernsthaften Entschlüssen der Regierungen kommen
könne. Bis dahin sei er wieder zu Hause; inzwischen reiche Grunthe
vollständig aus, die erforderliche Auskunft zu geben. Es stehe also
dabei eigentlich weiter nichts in Frage, als daß die Martier sich
der Mühe unterzögen, noch einmal eine Fahrt vom Pol nach Friedau
und zurück zu machen. Das aber sei doch in zwölf Stunden erledigt.

Ell führte dies, hin und wieder von seinem Oheim unterbrochen, in
eifriger Rede aus. Isma hörte dem Gespräch, von dem sie kein Wort
verstand, geduldig zu. Sie erschrak, wenn sie aus Ills Augen auf
eine ablehnende Antwort schließen zu müssen glaubte. Jetzt aber
lächelte Ill und sagte:

„Die Transportfrage, euch beide mitzunehmen und wieder
herzubringen, ist für uns kein Hindernis. Persönlich würde es mich
sehr freuen, dich bei mir zu haben, und sogar sachlich könnte es
von Vorteil sein, da Fälle denkbar sind, in denen wir unser Schiff
verlassen müssen, um das Land zu betreten; und dann würdest du mit
den Eskimos, die wir mitnehmen werden, mehr leisten können als wir.
Ich wundere mich aber, warum du für den Wunsch der Frau Torm so
eifrig eintrittst, der eigentlich nur einer Stimmung, man möchte
fast sagen, einer Einbildung entspringt.“

„Sie hegt nun einmal den Wunsch“, erwiderte Ell etwas verlegen,
„sie hält die Reise für ihre Pflicht, und es ist der einzige Trost,
den ich ihr gegenwärtig geben kann, wenn ich ihren Wunsch zu
erfüllen suche.“

Ill blickte seinem Neffen mit Herzlichkeit ins Auge. „Du liebst
diese Frau.“

Ell schwieg.

„Und du willst sie mitnehmen und begleiten, um ihr den Gatten
wiederzugeben?“

„Ja.“

„So machst du ihren Wunsch zu dem deinen?“

„Vollständig.“

„Ich möchte dir deine erste Bitte nicht abschlagen. Aber es ist
noch ein prinzipielles Bedenken. Zugegeben, deine Abwesenheit von
hier für kurze Zeit wäre allenfalls belanglos. Es könnte aber ein
unglücklicher Zufall eintreten, der uns verhindert, hierher
zurückzukehren. Deine Abwesenheit könnte sich auf den ganzen Winter
ausdehnen. Dann übernehmen wir eine furchtbare Verantwortung. Das
Verständnis zwischen den Planeten steht auf dem Spiel.“

„Ich weiß es. Es ist der Gedanke, der mich zuerst der Bitte von
Frau Torm widerstehen ließ, der mich in Konflikt mit mir selbst
brachte. Aber gerade, weil wir nicht allwissend sind, dürfen wir
einen solchen Umstand nicht in die Berechnung ziehen; er ist nur
als Zufall zu behandeln; ich kann morgen tot sein, auch wenn ich
nicht aus meinem Zimmer gehe. Ich habe mich nun einmal um Ismas
willen entschlossen; was daraus wird, muß ich mit meinem Gewissen
abmachen. Daß ich nicht eigennützig handle, weißt du.“

„Sonst hätte dein Wunsch für uns nicht existiert.“

„So aber, da es sich nur um Chancen des Gelingens oder Mißlingens
handelt, dürfen wir auch nicht vergessen, daß mit der größeren
Wahrscheinlichkeit unsre Reise das Verständnis zwischen den
Planeten fördern wird. Wenn es uns gelingt, Torm zu retten, wenn er
durch die Nume hierhergebracht wird, so haben wir das Zutrauen der
Menschen und ihren Glauben an uns in viel höherem Grad gewonnen,
als sie selbst durch mein Fernsein verloren werden könnten. Ich
glaube also, daß wir im Interesse der Planeten selbst wirken, wenn
wir Torm suchen. Dieser Grund ist mir allerdings erst jetzt
eingefallen.“

Ill lächelte wieder. „Er würde auch gelten, wenn Frau Torm uns
nicht begleitete. Wir gewinnen aber durch sie eine Zeugin, die uns
von Nutzen sein kann. Doch gleichviel. So will ich denn einen
Vorschlag machen, das Äußerste, was ich zugeben kann. Ich beurlaube
dich von der Begleitung nach Rom, Paris und London. Dagegen kürze
ich unsern Aufenthalt in Europa ab und komme von Petersburg aus
nicht erst hierher zurück, sondern gehe sogleich von dort nach
Norden. Wollt Ihr also mit, so müßt ihr – wir haben heute, nach
eurer Zeitrechnung –?“

„Den 9. September.“

„Nun gut. So haltet euch bereit, im Laufe des 11. Septembers mit
uns aufzubrechen.“

Ell sprang in die Höhe. Er dankte Ill und sagte freudig zu Isma:
„Wir dürfen mit. Aber wir müssen übermorgen reisefertig sein.“ Und
mit ernsterem Ausdruck setzte er hinzu: „Wollen Sie nicht lieber
von Ihrem Vorhaben abstehen? Sie können gewiß sein, daß die Nume
alles tun werden, um Hugo aufzufinden. Bleiben Sie hier, Isma!“

Isma stand einen Augenblick unschlüssig. Sie sah sich in der Kajüte
des Luftschiffes um, in welcher sie saßen.

Ill drückte auf einen Griff. Auf beiden Seiten der Kajüte öffnete
sich je eine Tür.

„Hier sind noch zwei Kabinen, je für einen Gast“, sagte er. „Sie
werden es etwas eng, aber sonst ganz bequem haben. Es versteht sich
von selbst, daß ihr meine Gäste seid“, setzte er zu Ell gewendet
hinzu.

Isma verstand nicht seine Worte, aber seine Handbewegung. Sie
streckte Ill schüchtern ihre Hand entgegen, die er zwischen die
seinigen nahm.

„Ich danke Ihnen“, sagte sie, „von ganzem Herzen.“ Dann wandte sie
sich zu Ell. Sie sah ihn mit einem Blick an, dem er nicht
widerstehen konnte.

„O zürnen Sie mir nicht, mein lieber, treuer Freund. Ich werde es
Ihnen nie vergessen, was Sie heute für mich taten. Ich kann nicht
hierbleiben, ich will hinaus. Und wenn Sie mitgehen, so danke ich
ihnen, denn unter diesen Fremden allein – es ist mir alles so
beängstigend – und keiner versteht mich – aber mit Ihnen – o Ell,
ich weiß, welches Opfer Sie mir bringen, und ich habe es nicht um
Sie verdient.“

Mit Tränen in den Augen reichte sie ihm die Hände.

„Also übermorgen.“

„Noch eins“, sagte Ill, „eine Bedingung, die ich machen muß. Unsere
Nachforschungen werden am 12. September beginnen. Sie müssen aber
am 20. unter allen Umständen aufhören. Sind wir bis dahin nicht
glücklich gewesen, so müssen Sie es tragen. Am Morgen des 21.
September setzt Sie dieses Schiff wieder hier ab. Und so Gott will,
schon früher und – zu dreien.“

Ell übersetzte Isma die Worte.

„Gott sei uns gnädig!“ sagte sie leise.

„Und wie ist es mit der Reise nach den Hauptstädten?“ fragte Ell.

„Die mache ich morgen. Ich habe es mir nach deinen Karten und
Angaben schon berechnet. Die ganze Fahrt von hier nach Rom, über
Paris nach London und von dort zurück könnten wir in kaum fünf
Stunden zurücklegen. Wir werden uns aber viel mehr Zeit nehmen. Nur
hier breche ich ungesehen auf, vor Sonnenaufgang. Denn da wir
wieder hierher zurückkommen, würde ich dir und uns die ganze
Bevölkerung auf den Hals ziehen und vielleicht ernstliche
Schwierigkeiten haben, wenn man von unserm Hiersein wüßte. Dagegen
werden wir unsere Fahrt, wenn wir erst jenseits der Alpen sind, und
dann in Frankreich und England, zum Teil absichtlich langsam und
möglichst vor aller Augen ausführen. Die Menschen sollen sehen, was
wir können, sie werden dann Grunthe eher glauben. Auf irgendeinem
unzugänglichen Alpengipfel werden wir einige Stunden ungestört
Mittagsruhe halten. Paris, London, Amsterdam, Brüssel besuchen wir
im Lauf des Nachmittags und Abends. Sobald es dunkel genug ist,
landen wir wieder hier. Und nun besorge deine Geschäfte und bereite
alles vor.“

Ell führte Isma aus dem Schiff. Sie zitterte an seinem Arm.

„Sie muten sich zuviel zu, liebste Freundin.“

„Nein, nein“, sagte sie. „Ich weiß, was ich kann. Es ist nur die
ungewohnte geringe Schwere in dem Schiff – aber ich werde mich
daran gewöhnen. Es ist schon wieder besser in der freien Luft.“

„Ill wird es gewiß arrangieren können, daß Sie nicht immer in der
Marsschwere zu sein brauchen.“

„Das ist ja alles gleichgültig. Nun will ich nur schnell nach
Hause. Sie können sich denken, daß ich viel zu tun habe“, sagte sie
mit schwachem Lächeln.

„Warten Sie, ich will einen Wagen holen lassen.“

„Das dauert zu lange. Können Sie mich nicht hier aus dem
Parkpförtchen lassen? Dann spare ich Weg.“

„Gewiß, ich habe den Schlüssel hier.“

Ell öffnete die kleine Tür in der Mauer. Sie führte auf einen
Promenadenweg, der von den Friedauern vielfach benutzt wurde, da er
zu einem beliebten Spazierort führte. Es war inzwischen neun Uhr
geworden.

Isma zog den Schleier vor das Gesicht. Noch ein herzlicher
Händedruck, und sie schritt schnell den Weg nach der Stadt hinab.

Zwei Herren begegneten ihr, die sie scharf ansahen und sich dann
etwas zuflüsterten.

Ell war noch einen Augenblick stehen geblieben und hatte ihr
nachgeblickt. Als er in die Tür zurücktreten wollte, waren die
beiden Spaziergänger herangekommen.

„Ach, guten Morgen, Herr Doktor“, sagte der eine mit näselnder
Stimme. „Was macht der Nordpol?“

„Schon so früh interessanten Besuch gehabt? Wie?“ sagte der andere.
„Wohl sehr besorgt um den Herrn Gemahl?“

Ell sah den Sprecher von oben bis unten an und drehte ihm, ohne ein
Wort zu sagen, den Rücken. Vor dem Blick Ells wich er erschrocken
zurück, und aus Ärger über seine eigene Verlegenheit rief er Ell
protzig nach:

„Na, na, man wird doch wohl fragen dürfen?“

Ell drehte sich um. „Nein, Herr von Schnabel, was einen nichts
angeht, wird man nicht fragen dürfen. Adieu.“

„Ich bitte doch, soll das vielleicht eine Zurechtweisung sein? Dann
möchte ich allerdings noch um eine Aufklärung bitten.“

„Tun Sie, was Sie wollen“, sagte Ell. „Ich habe keine Zeit.“ Er
schloß die Tür hinter sich und ging zu Grunthe zurück.

\section{24 - Die Lichtdepesche}

Sobald die Redaktion der ersten Berichte beendet war, begab sich
Grunthe nach dem Ministerium, um seine Anwesenheit in Friedau und
die vorgelegten Dokumente beglaubigen zu lassen. Von dort trug er
die Depeschen sogleich nach dem Telegraphenamt. Die Beamten hatten
ihn verwundert angestarrt. Einige Friedauer erkannten ihn unterwegs
und versuchten, ihn auszuforschen. Aber auf alle Fragen hüllte er
sich in Schweigen, und so gelang es ihm, noch ziemlich ohne
Aufsehen nach der Sternwarte zurückzugelangen, während sich in der
Stadt bereits das Gerücht von der Rückkehr der Expedition und
wunderbare Fabeln von den Bewohnern des Mars verbreiteten.

Noch ehe Grunthe zurückkehrte, erhielt Ell den Besuch eines ihm
befreundeten Oberlehrers des Friedauer Gymnasiums, Dr. Wagner. Der
elegant gekleidete Herr trat mit einem etwas gezwungenen Lächeln
ein und sagte, nach der ersten Begrüßung verlegen sein
Schnurrbärtchen drehend: „Ich habe da einen etwas fatalen Auftrag,
den ich aber nicht ablehnen konnte. Weil wir uns ja kennen, dachte
ich, ich könnte die Sache am besten beilegen. Weißt du, du hast da
heute früh mit dem Herrn von Schnabel –“

Ell machte eine abwehrende Bewegung.

„Na ja“, sagte Wagner, „es ist ein nicht sehr angenehmer Herr, hä,
außerdem so etwas“ – er klopfte mit dem Finger an die Stirn –
„seinerseits taktlos und dabei furchtbar empfindlich. Du hast ihn
ja wahrscheinlich ganz mit Recht abfallen lassen, aber er fühlt
sich von dir brüskiert, und ich soll da eine Art von Erklärung
fordern.“

„Mit dem größten Vergnügen“, erwiderte Ell lächelnd, „ich habe ihm
verwiesen, naseweise Bemerkungen zu machen über Dinge, die ihn
nichts angehen. Ich habe ihn vielleicht etwas schroff behandelt,
aber einerseits hat er es verdient, andrerseits hatte ich den Kopf
wirklich mit wichtigeren Dingen voll, als sie die Neugier von Herrn
Schnabel erregen. Wenn es ihn tröstet, so sage ihm, daß mir nichts
ferner gelegen hat als ihn beleidigen zu wollen.“

„Hm – ich weiß nicht, ob ihm das genügen wird, er verlangt, daß du
deine Äußerungen formell zurücknimmst.“

„Ich habe nichts zurückzunehmen, da ich nur die Wahrheit gesagt
habe, er muß sich also schon an der Erklärung genügen lassen, daß
ich ihn nicht beleidigen wollte. Eine Unhöflichkeit ist noch keine
Beleidigung. Wenn er sich aber seiner Fragen wegen entschuldigen
will, so bin ich auch bereit, wegen der unhöflichen Form meiner
Antwort um Entschuldigung zu bitten. Ich dächte, die Angelegenheit
wäre erledigt.“

„Ich fürchte“, sagte Wagner verlegen, indem er aufstand, „es werden
sich da wohl noch weitere Folgen daran knüpfen. Ich kenne ja deine
Ansichten über dergleichen Affären, ich bin auch ganz deiner
Meinung, aber, hä, in meiner Stellung, ich muß da Rücksichten
nehmen, weißt du, du wirst mir’s also zugutehalten – ich wollte nur
vermitteln und werde ihm zureden. Wenn es nur nützt! Er wird dir
wohl da noch einen Kartellträger schicken.“

„Er soll sich nur die Mühe sparen, ich würde den Herrn an die Luft
setzen. Aber ich danke dir für deine Bemühung. Also, wie gesagt,
erkläre ihm in aller Form, daß mir jede Absicht einer Beleidigung
ferngelegen hat, daß ich mir aber das Recht vorbehalten müßte, mir
unberufene Fragen zu verbitten, und er sich in bezug hierauf
zunächst selbst zu entschuldigen hätte. Und nun entschuldige du
auch mich, alter Freund, du wirst heute noch merkwürdige Dinge von
mir hören.“

Wagner wollte weiter fragen, aber Ell verabschiedete sich
freundschaftlich, und Wagner ging kopfschüttelnd ab.

Schon eine Stunde später – Grunthe war eben zurückgekommen, und Ell
wollte sich mit ihm zu Tisch setzen – Ill hatte die Einladung
abgelehnt, er wollte ruhen –, erschien der Kartellträger des Herrn
von Schnabel und überbrachte Ell eine Forderung.

Der Herr, ein junger Assessor, hatte sich seines Auftrages kaum in
feierlichster Weise erledigt, als Ell ihm mit blitzenden Augen
entgegentrat und ihn anfuhr:

„Wie können Sie sich unterstehen“, rief er, „mich durch eine
derartige Zumutung zu beleidigen? Wofür halten Sie mich? Bin ich
ein rauflustiger Bruder Studio oder ein pflichtvergessener Narr?
Ich bin ein Mann, der seine Arbeitskraft ernsten Dingen schuldet.
Übrigens bedauere ich Sie“, sagte er milder, „Sie haben sich
jedenfalls nicht klargemacht, was Sie tun. Ich wünsche von der
Sache nichts mehr zu hören.“

Der Assessor wollte auffahren, aber auf eine Handbewegung Ells
machte er kehrt und verließ das Zimmer.

Ell setzte sich mit Grunthe zu Tisch.

„Das wird auch Zeit“, sagte er, noch etwas erregt von dem letzten
Auftritt, während er seine Serviette entfaltete, „daß mit diesem
Unfug einmal aufgeräumt wird. Das ist so einer von den Punkten, in
denen die Martier keinen Spaß verstehen. Ich will hoffen, daß es
nicht zu Konflikten kommt.“

\tb

Im Lauf des Nachmittags wurden von allen Zeitungen, nicht bloß in
Deutschland, sondern in ganz Europa, Extrablätter ausgegeben.

„Neues vom Nordpol!“ – „Die Bewohner des Mars auf der Erde!“ – „In
sechs Stunden vom Nordpol!“

So und ähnlich lauteten die Ausrufe auf den Straßen. Man riß sich
die Blätter aus der Hand. Vom Erlös für dieselben hätte man allein
eine neue Nordpolexpedition ausrüsten können.

Die Blätter enthielten zuerst die Depesche Torms an Isma. Sodann
folgten ein knapper Bericht Grunthes über die weiteren Erlebnisse
der Expedition und kurze Angaben über die Martier und seine
Heimkehr. Endlich eine Bestätigung der letzteren durch Ell und die
Beglaubigung seitens des fürstlichen Staatsministeriums in
Friedau, daß Grunthe die im Bericht erwähnten Dokumente und
Effekten persönlich vorgelegt habe.

Nur eines war mit Stillschweigen übergangen, nämlich daß sich das
Luftschiff noch in Friedau befinde. Dagegen war die Abstammung Ells
kurz erwähnt worden, weil sie dazu dienen konnte, das
Unbegreifliche einigermaßen der menschlichen Vorstellungskraft
näherzurücken.

Ein ausführlicher schriftlicher Bericht war noch vormittags an den
Reichskanzler abgegangen. Am Abend schon traf eine telegraphische
Depesche ein, durch welche Grunthe und Ell ersucht wurden, sich
sobald als möglich mit allen Beweisstücken persönlich in Berlin
einzustellen. Se. Majestät habe sofortigen Bericht eingefordert.
Eine Stunde später erhielt Grunthe ein Glückwunsch-Telegramm des
Kaisers, ebenso Frau Torm eine in sehr liebenswürdiger Form
ausgesprochene Beileidsbezeugung, in welcher das Vertrauen auf die
glückliche Heimkehr ihres Gatten ausgedrückt war.

Von dem Augenblick an, in welchem die Extrablätter ausgegeben
wurden, war die Sternwarte Ells von Besuchern bestürmt. Das
Läutwerk des Telephons kam so wenig zur Ruhe wie die Türklingel,
und bald häuften sich Telegramm auf Telegramm, Glückwünsche und
Anfragen. Da dies vorauszusehen war, hatte Ell einige seiner
persönlichen Freunde in Friedau gebeten, ihn zu unterstützen. Sie
ordneten die Eingänge der Depeschen und empfingen die Besuche. Ell
und Grunthe ließen sich nicht sehen. Beide trafen die
Vorbereitungen zu ihren Reisen. Grunthe mußte allein nach Berlin
gehen, was ihm nicht sehr angenehm war. Ell gab ihm die
fertiggestellten Manuskripte mit. Ein Berliner Verleger hatte ihm
bereits telegraphisch einen hohen Preis geboten für alles, was er
über die Martier schreiben wolle. Ell verlangte das Zehnfache und
erhielt es sofort zugestanden, da der Verleger wußte, daß man von
London aus das Zwanzigfache geben würde. Ell bestimmte das Honorar
für die Teilnehmer der Expedition.

Isma hatte auf Ells Rat ihre Besorgungen sogleich am Vormittag
gemacht, soweit sie dazu in die Stadt gehen mußte. Denn es ließ
sich erwarten, daß sie keine Ruhe mehr finden würde, sobald die
Nachricht bekannt geworden sei. Sie fühlte sich zu angegriffen, um
die sich drängenden Besuche anzunehmen, fand aber ebenfalls einige
Freundinnen, die ihr diese Mühe abnahmen und sich ein Vergnügen
daraus machten, ihr spezielles Wissen immer wieder aufs neue
mitzuteilen. Von ihrer Absicht, zu verreisen, sagte sie nichts. Nur
ihrem Mädchen teilte sie mit, daß sie in den nächsten Tagen auf
etwa eine Woche von Friedau fortgehen würde; sie konnte ihr
vertrauensvoll die Wohnung überlassen.

Am folgenden Tag reiste Grunthe frühzeitig, bald nachdem sich das
Luftschiff der Martier unbemerkt entfernt hatte, nach Berlin ab.
Die Flut der Anfragen bei Ell nahm noch zu. Es kamen jetzt auch
auswärtige Besucher, und nicht alle durfte er abweisen. Vor dem
Gittertor der Sternwarte stand den ganzen Tag über eine Menge
Neugieriger und guckte in den Hof, als ob dort etwas zu sehen wäre.
Gegen Abend verließ Ell durch die Parkpforte sein Grundstück und
begab sich zu Isma, um sie zu fragen, ob er ihr noch irgendwie
behilflich sein könne. Isma dankte.

„Es ist ja nur eine kurze Reise“, sagte sie wehmütig lächelnd.

Man verabredete, daß sie am andern Morgen frühzeitig an der
Parkpforte sein solle. Ihren kleinen Handkoffer konnte das
Dienstmädchen tragen.

Auf dem Rückweg besorgte Ell noch einigen Proviant, den er auf
Grunthes Rat mitnehmen wollte, weil die Lebensmittel der Martier
für den Anfang vielleicht Isma und ihm nicht zusagen würden. Er
nahm daher seinen Weg durch die Stadt. Hier aber heftete sich bald
die Straßenjugend neugierig an seine Fersen und folgte ihm auf
jedem Schritt. Anfänglich hielten die Kinder sich scheu zurück,
dann brachte ein Witzbold das Wort auf: „Das ist der vom Monde, der
Mann vom Monde! Guck här, ’s kummt eener vom Monde!“ Ell beeilte
sich, nach Hause zu gelangen. Er nahm sich nicht Zeit, eines der
Extrablätter zu kaufen, zu denen sich das ›Friedauer
Intelligenzblatt‹ in Ermangelung einer Abendausgabe aufgerafft
hatte.

Das Extrablatt brachte bereits einen Bericht über den Empfang
Grunthes beim Reichskanzler, der indessen offenbar der Phantasie
eines Berliner Korrespondenten entsprungen war. Dann aber enthielt
es Depeschen aus Rom, Florenz, von der meteorologischen Station des
Montblanc, aus Paris und London über die Beobachtung eines
Luftschiffs. Das Luftschiff war zuerst in Rom wahrgenommen worden,
wo es am Morgen schon um sieben Uhr auftauchte, die Stadt umkreiste
und nach allen Richtungen hin überflog. Es entfernte sich nach
einer Stunde, wurde im Laufe des Vormittags noch in verschiedenen
italienischen Städten gesehen, um 11 Uhr umflog es in unmittelbarer
Nähe die Spitze des Montblanc, so daß die anwesenden Touristen die
Bemannung des Fahrzeugs erkennen konnten. In Paris und London waren
diese Nachrichten schon durch Extrablätter bekanntgegeben, man
achtete also am Nachmittag gespannt darauf, ob sich das Schiff
zeigen würde. Alsbald verbreitete sich in Paris das Gerücht, das
Luftschiff sei eine Erfindung der Preußen und speziell dazu
bestimmt, die Befestigungen von Paris auszukundschaften. In der Tat
erschien das Luftschiff um 3 Uhr nachmittags am Horizont und
umkreiste in langsamem Segelflug die Forts im Südosten der Stadt.
Man wurde unruhig und löste einen Warnungsschuß. Darauf stieg das
Schiff etwas höher und umflog nun den ganzen Kreis von
Befestigungen, aber auf der inneren Seite nach der Stadt zu, so daß
man ihm nichts anhaben konnte, ohne die Stadt selbst zu gefährden.
Um fünf Uhr schoß es in die Höhe und erschien eine halbe Stunde
später in London. Es überschritt die Themse bei Greenwich, zog dann
in einem weiten Halbkreis nördlich um die Stadt, wandte sich am
Hyde Park wieder nach Osten und kreuzte über dem Häusermeer. Auf
allen freien Plätzen standen dichtgedrängte Volksmassen, welche mit
Tüchern winkten und Hurra schrien. Böllerschüsse wurden gelöst, und
die Schiffe auf dem Fluß hißten ihre Flaggen. Das Luftschiff aber
kümmerte sich um nichts. Sobald die Sonne sich zum Untergang
neigte, zog es die Flügel ein und stieg senkrecht so hoch in die
Lüfte, daß es den Blicken entschwand, und man nicht angeben konnte,
wohin es sich gewendet hatte.

Um zehn Uhr abends senkte sich eine dunkle Masse langsam auf den
Garten der Sternwarte von Friedau.

Es war zwischen zwei und drei Uhr nachts, als Ell davon erwachte,
daß die Sonne hell in sein nach Norden gelegenes Schlafzimmer
hineinschien. Verwirrt richtete er sich auf, aber ehe er bis an das
Fenster gelangte, war die Erscheinung verschwunden. Die Nacht war
nur vom matten Schimmer des aufgehenden Mondes erhellt. Plötzlich
aber leuchtete ein beschränkter Bezirk der Landschaft wieder im
Sonnenlicht, und diese erhellte Stelle veränderte ihren Ort, in
gerader Linie von Norden nach Süden laufend, bis sie den Garten der
Sternwarte, jetzt etwas westlich vom Haus, wieder erreichte. Da die
Richtung des in der Luft deutlich erkennbaren Lichtstreifens unter
einer Neigung von etwa 24 Grad direkt nach Norden lief, so war es
Ell sofort klar, daß man die Gegend von der Ringstation der Martier
aus mit einem riesigen Reflektor systematisch absuchte. Denn dieser
Punkt lag für die Friedauer Warte in einer Höhe von 23 Grad 56
Minuten. Ell kleidete sich daher schleunigst an und begab sich nach
dem Garten, wo das Luftschiff lag.

Er bemerkte, daß das Schiff seine Lage verändert hatte. Es befand
sich jetzt auf der Südseite des geräumigen Rasenplatzes, so daß der
Blick nach Norden über die Bäume freier wurde und die Spitzen
derselben tiefer als 24 Grad lagen. Als er auf den Platz trat, war
das Schiff und die südliche Baumwand so stark von der Sonne
beleuchtet, daß er geblendet wurde. Aber noch hatte er das Schiff
nicht erreicht, als das Licht verschwand. Sein Weg wurde jetzt nur
durch den schwachen Schein einer Lampe aus dem Innern des Fahrzeugs
erhellt.

Ill war damit beschäftigt, einen Ell unbekannten Apparat
einzustellen. Ein Offizier des Schiffes war ihm dabei behilflich.

„Entschuldige, wenn ich störe“, sagte Ell, „aber ich glaubte
bemerkt zu haben, daß man Zeichen von der Außenstation gibt.“

„Es ist so“, sagte Ill, „und sie haben uns jetzt gefunden. Es muß
etwas Wichtiges passiert sein. Nimm Platz und gedulde dich ein
wenig. Wir werden sogleich die Unterhaltung beginnen können. Die
Verbindung ist bereits optisch hergestellt, wir müssen jetzt
langwellige unsichtbare Strahlen anwenden, um telephonieren zu
können.“

Ell fragte erstaunt: „Telephonieren? Du willst mit der Station
sprechen?“

„Ja“, sagte Ill, „vermittels der Strahlen. Aber es muß nun
vollständige Ruhe herrschen.“

Ell setzte sich still in den Hintergrund. Eine Hoffnung stieg in
ihm auf. Sollte man vielleicht Torm gefunden haben?

Ill brachte sein Ohr an den Apparat. Ell vermochte nichts zu hören,
auch was Ill sprach, konnte er nicht vernehmen, da es ganz leise in
den telephonischen Apparat gesprochen wurde.

Etwa eine halbe Stunde mochte so vergangen sein. Dann wendete sich
Ill zu seinem Neffen.

„Wir müssen unsern Aufbruch aufs möglichste beschleunigen“, sagte
er. „Meine Anwesenheit auf der Insel ist dringend erforderlich,
voraussichtlich unsere Hilfe.“

„Was ist geschehen? Keine Nachricht von Torm?“

„Bis jetzt nicht. Ich sagte dir bereits, daß wir noch ein kleineres
Luftboot in Betrieb setzen wollten. Das ist geschehen. Es bedarf
nur vier Mann zur Besatzung, kann aber auch nur die halbe
Geschwindigkeit im Mittel erreichen wie hier unser Luftschiff. Für
die Fahrten im Polargebiet hat es sich jedoch, wie ich eben
erfahre, als sehr geeignet erwiesen. Die Unsern sind damit in drei
Stunden bis zum 80. Breitengrad nach Süden gelangt. Mit diesem Boot
sind die Nachforschungen nach Torm aufgenommen worden. Und bei
dieser Gelegenheit ist es zu dem unangenehmen Zwischenfall
gekommen, der meine sofortige Rückkehr erfordert.“

„Ein Unglücksfall?“

„Ein Konflikt mit einem europäischen Kriegsschiff.“

„Nicht möglich! Wo?“

„Auf 81 Grad Breite, 294 Grad Länge ungefähr. Infolge eines
Mißverständnisses jedenfalls – ich sehe darin noch nicht ganz klar
– sind unsre Leute am festen Land, während sie verunglückten
Matrosen des Kriegsschiffs Hilfe zu bringen versuchten, von anderen
überfallen worden. Zwei gerieten in Gefangenschaft der Menschen,
die beiden anderen konnten auf dem Luftboot entfliehen. Das Boot
selbst ist beschossen worden und scheint dabei gelitten zu haben.
Ich muß also mit unserm Schiff hin, um auf jeden Fall die beiden
Leute zurückzuholen. Und so bleibt gar nichts übrig, du mußt dich
sogleich aufmachen und versuchen, Frau Torm zu wecken und
hierherzubringen, wenn sie dabei beharrt, uns zu begleiten. Größte
Eile tut not. Wir machen inzwischen unser Schiff klar.“

Es war für Ell eine recht peinliche Aufgabe, mitten in der Nacht
und möglichst ohne Aufsehen zu erregen Isma zur Reise nach dem
Nordpol abzuholen. Doch es mußte geschehen. Schließlich kam es
jetzt schon nicht mehr darauf an, ob sich die bösen Zungen von
Friedau noch etwas mehr aufregten.

Isma, die in dieser Zeit stets gefaßt war, durch eine Nachricht aus
dem Schlaf geweckt zu werden, eilte ans Fenster, als Ell die
Hausklingel ertönen ließ. Sie erkannte Ell. Wenige Worte genügten
zur Verständigung. Eine halbe Stunde später verließ sie das Haus,
ohne daß ihr Mädchen, das auf der andern Seite der Wohnung schlief,
erwacht wäre. Ein paar Worte, die Isma auf einem Zettel zurückließ,
besagten nur, daß sie ihre Reise unerwartet schnell hätte antreten
müssen. Aus der Dunkelheit tauchte Ell neben ihr auf und nahm ihr
den Handkoffer ab. Ein verschlafener Nachtwächter sah ihnen
verwundert nach.

In tiefer Ruhe, wie ausgestorben lag die Stadt Friedau, als im
ersten Grauen der Morgendämmerung das Luftschiff der Martier sich
erhob, um alsbald mit der größten Anspannung seiner Maschine sich
durch die Höhen des Luftmeers nach Norden zu schnellen.

\section{25 - Engländer und Martier}

Das englische Kanonenboot ›Prevention‹ hatte den Auftrag, die im
Interesse der Polarforschung angelegten Depots im Smith-Sund und
weiter nach Norden, soweit es die Eisverhältnisse ohne Gefährdung
des Schiffes gestatteten, zu revidieren und zu vermehren. Kapitän
Keswick traf die Lage sehr günstig. Die Kane-Bai war in ihrer Mitte
völlig eisfrei, sie wurde in rascher Fahrt passiert, die
›Prevention‹ dampfte in den Kennedy-Kanal hinein und drang ohne
Schwierigkeiten bis über 80,7 Grad Breite vor; hier legte sie sich
an einer günstigen Stelle vor Anker und schickte ein Boot zur
Aufsuchung eines passenden Ortes aus, um an dem felsigen Ufer eine
Niederlage von 3.600 Rationen zu errichten. Man fand in einer
kleinen Bucht eine natürliche Felsenhöhle, in welcher die Vorräte
sicher geborgen werden konnten. Während die Bemannung des Bootes
zum größten Teile mit dieser Arbeit beschäftigt war, erstieg
Leutnant Prim mit zwei Matrosen den Hügel über der Höhle, um dort
als Signal einen Cairn zu errichten. Die Spitze des Hügels sah auf
eine breite, teilweise mit Eis bedeckte Ebene, so daß der Cairn auf
weithin, sowohl vom Land als vom Wasser aus, zu sehen sein mußte.
Denn dieser zu errichtende ›Steinmann‹ sollte dazu dienen, in
seinem Innern die Dokumente aufzunehmen, welche die Lage der in der
Umgegend niedergelegten Depots bezeichneten, er mußte daher einen
Platz erhalten, wo er für etwa hierher vordringende Reisende auf
weithin wahrgenommen werden konnte. Der Steinmann war soweit
fertig, daß der Offizier die Blechbüchse mit den Papieren darin
deponieren konnte, und die Matrosen waren damit beschäftigt, den
Bau zu schließen und noch mehr zu erhöhen. Als Leutnant Prim
inzwischen auf dem Hügel herumkletterte, bemerkte er in der Ferne
einige dunkle Punkte, die er alsbald als weidende Moschusochsen
erkannte. Sie zogen nach Süden und näherten sich langsam seinem
Standpunkt. Alsbald war die Jagdlust in ihm erwacht, er ergriff
eines der mitgebrachten Gewehre und bedeutete seine Leute, ihre
Arbeit zu vollenden und ihm dann nachzukommen. Er hoffte rasch
einen guten Schuß tun zu können. Bald war er hinter einigen
Felsvorsprüngen verschwunden.

Die Matrosen schlenderten ebenfalls in der Umgebung umher, um noch
einige große Steine aufzusuchen, als sie im Norden, rechts von der
Seite, wohin der Offizier, nur die Moschusochsen im Auge haltend,
gegangen war, einen dunklen Punkt über dem Horizont auftauchen
sahen. Derselbe nahm schnell an Größe zu und erwies sich zu ihrem
nicht geringen Erstaunen als ein riesiger Vogel, der seinen Flug
mit großer Geschwindigkeit direkt auf sie zu nahm. Eine Weile
standen sie still und starrten auf die merkwürdige Erscheinung.
Dann liefen sie nach dem Cairn zurück, um ihre Gewehre zu holen. Da
sich das rätselhafte Tier bereits stark genähert hatte, ergriff sie
Furcht, und sie zogen es vor, so schnell wie möglich den Hügel
hinabzulaufen, um Zuflucht bei ihren Gefährten zu finden. Zwischen
den Felstrümmern, von Zeit zu Zeit nach dem Ungeheuer sich
umblickend, das sich jetzt in weitem Bogen nach dem Steinmann hin
zu senken schien, verfehlten sie jedoch die Richtung und kamen an
eine mit Eis gefüllte, steil abfallende Schlucht. Plötzlich stieß
der Vorangehende einen Schrei aus. Er hatte auf dem steilen Abhang
einen Fehltritt getan und stürzte, auf die Felsvorsprünge
aufschlagend, in die Schlucht. Sein Gefährte blickte ihm mit
Entsetzen nach und wollte den Versuch machen, zu ihm
hinabzuklettern. Mit den Händen sich anklammernd, ließ er sich eben
auf einen tiefer liegenden Felsen nieder, als plötzlich über ihm
der glänzende Leib des Riesenvogels mit eingezogenen Flügeln
erschien. Er bebte in abergläubischer Furcht, seine Glieder
zitterten, er vermochte sich nicht länger zu halten und stürzte
ebenfalls in die Tiefe.

Kaum hatten die vier Martier in dem vom Pol herkommenden Luftboot,
das die Matrosen für ein Luftungeheuer gehalten hatten, das Unglück
erkannt, das sie durch ihr Erscheinen unschuldigerweise angerichtet
hatten, als sie das Luftboot langsam und vorsichtig sich in die
Schlucht hinabsenken ließen. Bald hatten sie die Körper der
Unglücklichen erreicht. Blutüberströmt lagen sie vor ihnen.
Obgleich keine Hoffnung war, die Menschen ins Leben zurückzurufen,
wollten sie doch ihre Leichen nicht in der Schlucht liegen lassen.
Da es unzweckmäßig war, sie in das Boot hineinzunehmen, legten sie
die Verunglückten in das Netz, das sich unter ihrem Boot ausspannen
ließ. Dann erhoben sie sich mit ihnen und dirigierten das Boot nach
der Spitze des Hügels. Sie überzeugten sich hier, daß beide
Menschen tot seien. Sie legten sie am Fuße des Cairn nieder und
brachten dann ihr Luftboot in eine gesicherte Lage in der Nähe.
Zwei von ihnen blieben im Boot zurück, während die beiden andern
noch einmal nach dem Steinmann zurückgingen, um ihn näher zu
untersuchen. Die Öffnung war noch nicht vermauert, und sie
entdeckten bald die Büchse mit den Dokumenten. Sie öffneten diese
und musterten den ihnen unverständlichen Inhalt. Während sie
hiermit beschäftigt waren, kehrte Leutnant Prim zurück. Das Boot
der Martier konnte er von seinem Standpunkt aus nicht sehen, auch
hatte er es vorher, nur auf das Wild und seinen Weg achtend, nicht
wahr genommen. Jetzt erblickte er zwei fremde, seltsam gekleidete
Männer, die sich seiner Papiere bemächtigt hatten. Und neben ihnen
– entsetzt wich er zurück – lagen die beiden Matrosen, entseelt,
mit blutigen, zerschmetterten Stirnen. Er konnte nicht anders
glauben, als daß er ihre Mörder vor sich habe. Er riß das Gewehr in
die Höhe und rief sie an.

Die Martier blickten erstaunt empor. Sie deuteten auf die
verunglückten Matrosen und riefen Prim zu, daß sie sie aus der
Schlucht herausgebracht hätten. Er dagegen befahl ihnen, die
Papiere hinzulegen und sich zu ergeben. Natürlich verstanden sie
sich gegenseitig nicht. Noch einige Rufe hin und her, ohne daß die
Martier Miene machten, sich zurückzuziehen, wie es Prim verlangte,
da knallte sein Gewehr, und die Kugel durchbohrte die blecherne
Büchse, welche der eine der Martier in der Hand hielt. Ein zweiter
Schuß aus dem Repetiergewehr folgte sofort, aber der Martier hatte
sich bereits beiseite geworfen, die Kugel ging fehl. Im nächsten
Augenblick ließ Prim das Gewehr machtlos aus der Hand fallen. Er
war nicht verwundet, aber die Hand war gelähmt, er konnte sie nicht
bewegen. Der andere Martier hatte mit seinem Telelyt-Revolver die
motorischen Nerven der Hand gelähmt.

Inzwischen hatten die mit der Hinterlegung des Depots beschäftigten
Mannschaften ihre Arbeit beendet. Die im Boot zurückgelassene Wache
war auf das Erscheinen des Luftboots, das jedoch bald wieder durch
die Felshöhe über ihnen verdeckt wurde, aufmerksam geworden und
hatte die übrigen Seeleute verständigt. Diese machten sich sofort
unter Führung eines Unteroffiziers daran, den Hügel zu ersteigen.
Da ertönten die beiden Schüsse, welche ihre Schritte
beschleunigten. Im Augenblick darauf rannten sie mit Geschrei auf
den Gipfel des Hügels zu. Prim, der sich von seiner
augenblicklichen Verwirrung erholt hatte, riß mit der linken Hand
seinen Revolver aus dem Gürtel und stürzte auf die Martier zu,
indem er rief: „Hierher, Leute, hier sind die Mörder! Faßt sie!“

Der Martier erhob aufs neue seine Waffe – sein Begleiter war
unbewaffnet –, und auch der Revolver entfiel dem Offizier – er
konnte seine linke Hand ebenfalls nicht mehr bewegen. Gleichzeitig
aber wurde der Martier durch einen Stoß in den Rücken
niedergeworfen. Die Matrosen waren im Sturmlauf herangekommen. Im
Handgemenge waren die Martier ohnmächtig. Sie wußten dies und
machten daher auch keinen weiteren Versuch, sich zu wehren. Auf den
Befehl des wütend gewordenen Offiziers wurden sie gefesselt, und
die Matrosen trieben sie mit Fauststößen vor sich her, um sie in
das Boot zu bringen.

Die Schüsse und das nachfolgende Geschrei hatten die beiden im Boot
zurückgebliebenen Martier aufmerksam gemacht; da sie aber nicht
schnell genug über die Felsen hätten klettern können, die sie vom
Schauplatz des Kampfes trennten, ließen sie das Luftboot so weit
aufsteigen, daß sie beobachten konnten, was geschehen. Sobald sie
ihre Kameraden gefangen sahen, versuchten sie, ihnen mit dem
Luftboot zu Hilfe zu kommen. Aber kaum näherte sich dieses, als die
Engländer ein Schnellfeuer eröffneten. Die Geschosse drangen in die
Rob-Wände des Bootes ein, und wenn sie dieselben auch nicht
durchschlugen, so lag doch die Gefahr nahe, daß sie Stellen trafen,
an denen der feine Mechanismus des Steuerapparates beschädigt
werden konnte. Die Martier stiegen daher mit ihrem Boot schleunigst
so hoch, daß sie von den Kugeln nicht mehr gefährdet waren, und
überlegten, was zu tun sei. Sie besaßen zwei Telelytgewehre, mit
denen sie imstande gewesen wären, aus sicherer Entfernung die ganze
Mannschaft zu vernichten oder wehrlos zu machen, um dann ihre
Kameraden zu befreien. Aber da sie sowohl selbst, der Luftströmung
wegen, nicht völlig ruhig liegen konnten, und auch die Gefangenen
mitten zwischen den Matrosen in Bewegung waren, konnten sie aus so
großer Entfernung nicht auf ein sicheres Zielen und genau
berechenbare Wirkung vertrauen. Während sie zögerten, wurden ihre
Kameraden in das Boot gebracht, das sich mit schnellen
Ruderschlägen vom Ufer entfernte. Sie folgten ihm in der Höhe und
sahen bald das Kriegsschiff in der Ferne. Als sie dieses nun in
schnellem Flug erreichen und umkreisen wollten, bemerkten sie zu
ihrem Schrecken, daß der Mechanismus des Steuerruders nicht mehr
völlig funktionierte. Sie konnten ihr Boot nur langsam und in
beschränkter Weise lenken. Unter diesen Umständen beschlossen sie,
so schnell wie möglich nach der Insel am Pol zurückzukehren. Sie
brauchten dazu die doppelte Zeit wie gewöhnlich. Von hier aus wurde
nach der Außenstation gesprochen, von der aus es möglich war, Ill
mit seinem größeren Luftschiff, das zur Verteidigung wie zum
Angriff mit Repulsitgeschützen ausgerüstet war, zur Hilfe
herbeizurufen.

Kapitän Keswick scheitelte bedenklich den Kopf zum Bericht des
Leutnants Prim, der es übrigens nicht für nötig hielt, sich über
seinen mißglückten Jagdversuch näher auszulassen. Keswick konnte
sich nicht recht vorstellen, wie diese beiden Männer, die sich
offenbar nur mit Mühe aufrecht zu erhalten vermochten, ohne Waffen
die harten Köpfe seiner Matrosen hätten zerschlagen können. Noch
mehr freilich wunderte ihn die Lähmung der Hände seines Leutnants.
Eine nähere Untersuchung erforderte aber vor allem, daß mit den
beiden Fremdlingen ein Verhör angestellt wurde. Diese indessen
sprachen kein Wort.

Keswick trat zu ihnen und betrachtete sie näher. Er redete sie auf
englisch und französisch an und auch in der einzigen Sprache, von
der er noch etwas wußte, auf chinesisch. Sie verstanden ihn
offenbar nicht. Aber sie öffneten jetzt zum erstenmal ihre bisher
halb geschlossen gehaltenen Augen. Finster blickten sie auf ihre
Fesseln und richteten dann ihre Augen voll auf den Kapitän. Es lag
nichts Feindseliges in diesem Blick, aber ein tiefer Vorwurf und
zugleich ein mächtiger Stolz. Unwillkürlich wich Keswick zurück.
Auch die herumstehenden Offiziere und Matrosen fühlten sich seltsam
betroffen.

„Nehmen Sie den Leuten die Fesseln ab“, sagte der Kapitän. „Das ist
hier nicht nötig. Und behandeln Sie sie anständig.“

Sobald die Stricke entfernt waren, begann der ältere der Martier zu
sprechen. Obgleich der Kapitän kein Wort verstand, machte die Rede
doch den Eindruck, daß er hier etwas noch nie Vorgekommenes und
Unerklärliches erfahre. Er wußte nichts zu tun, als die Achseln zu
zucken.

„In dieser Sache entscheide ich nicht allein“, sagte er dann zu
seinem ersten Offizier. „Die Geschichte mit dem Luftschiff ist zu
rätselhaft. Hätten wir nicht selbst in der Ferne so ein Ding
gesehen, ich würde nichts glauben. Die Leute sehen nicht aus, als
ob sie von der Erde stammten. Und verstehen kann man sie nicht. Ich
nehme sie mit nach England. Wir sind überdies hier mit unserer
Aufgabe fertig.“

Die ›Prevention‹ machte Dampf auf und steuerte nach Süden.

\tb

Mit rasender Geschwindigkeit jagte Ills Luftschiff in einer Höhe
von zwölf Kilometern über das europäische Nordmeer, der Küste
Grönlands entgegen. Im Osten glänzten schillernde Nebensonnen,
während das Tagesgestirn selbst unterm Horizont blieb. Denn die
Fahrt war nach Nordwesten gerichtet, und die aufgehende Sonne
konnte das Luftschiff nicht einholen. Ein ewiger Dämmerschein
erleuchtete die unter leichtem Cirrusgewölk lagernde Meeresflut,
daß sie wie eine ungeheure Schale von dunklem, mit lichten Streifen
durchzogenem Marmor schimmerte. Still war’s ringsum. Nur das
gleichmäßige Zischen des Reaktionsapparats und das Pfeifen der
durchschnittenen Luft um den zusammengepreßten Robpanzer des
Schiffes ließ seine eintönige Weise vernehmen.

„Luftdruck 170 Millimeter.“ Ell las die Angabe an seinem eigenen
Barometer ab. Er warf einen nachdenklichen Blick auf die Wand,
hinter welcher Isma schlummerte. Ill hatte dort selbst aufs
umsichtigste für ihr Wohlbefinden gesorgt.

„Schlafen Sie“, hatte er gesagt. „Sie müssen jetzt Ruhe haben. Wenn
wir in die hohen Breiten gekommen sind, werden wir unseren Flug
mäßigen und in die Nähe der Erdoberfläche hinabsteigen. Dann wollen
wir Sie wecken.“

In einen warmen Pelz gehüllt ruhte Isma in ihrer Hängematte. Über
Mund und Nase schloß sich die weiche Maske, die mit dem Ventil des
Sauerstoffapparats verbunden war, um ihr Handgelenk war ein
elastischer Ring gelegt, der ihren Pulsschlag auf ein Meßinstrument
übertrug. An der Außenwand ihrer Kabine, die Ell jetzt beobachtete,
zeigten zwei Zifferblätter den Gang, die Frequenz und die Stärke
der Atmung und des Pulses. „Vollständig normal“, sagte Ill
lächelnd, der Ells Augen gefolgt war. Dann blickte er wieder auf
die Orientierungsscheibe. Der Projektionsapparat, welcher auf der
Unterseite des Schiffes angebracht war, bildete auf der Scheibe die
überflogene Gegend ab.

„Im Nordwesten taucht die Küste auf“, begann Ill wieder. „Es ist
die Gegend, die auf euren Karten als ›König-Wilhelms-Land‹
bezeichnet ist. Noch eine Stunde, bis das Festlandeis überflogen
ist, dann wollen wir hinabsteigen. So lange laß sie nur
schlummern.“

„Ich denke“, sagte Ell, „daß wir das Schiff im Kennedy-Kanal oder
in der Kane-Bai treffen. Ich bin nur neugierig, was es für ein
Landsmann ist.“

„Unser Feind, leider“, sagte Ill ernst, „wer es auch sei.“

Ill war längere Zeit schwankend gewesen, ob er zuerst nach dem Pol
fahren solle, um noch nähere Erkundigungen einzuziehen, oder ob er
besser täte, direkt das Kriegsschiff aufzusuchen. Er entschloß sich
für das Letztere. Denn jede Minute konnte kostbar sein, jede mußte
die Leiden der Nume verlängern, jede konnte ihr Leben gefährden.
Dazu stand die Wichtigkeit dessen, was er am Pol erfahren konnte,
in keinem Verhältnis, selbst eine genauere Ortsangabe für den
Schauplatz des Ereignisses hätte nichts ihm genützt. Es waren
seitdem über zwölf Stunden vergangen, und das Schiff konnte
inzwischen seinen Ort um hundert und mehr Kilometer verändert
haben. Er durfte darauf rechnen, von seinem Luftschiff aus die
Fahrstraße in jenen Gegenden verhältnismäßig schnell zu
durchforschen. Schwere Bedenken erregte ihm die Frage, wie er
verfahren solle, wenn man ihm die friedliche Herausgabe der Martier
verweigere. Zwar besaß er die Mittel, selbst ein mächtiges
Kriegsschiff zu vernichten. Aber dazu hätte er sich nie
entschließen können, es sei denn, wenn er die eigene Existenz nicht
anders retten konnte. Mußte er Gewalt anwenden, so sollte es nur so
geschehen, daß die Menschen doch nachträglich imstande waren, mit
ihrem Schiff in ihre Heimat zurückzukehren. Ob es aber möglich sein
würde, bei den Menschen etwas durchzusetzen, ohne sie zuvor schwer
zu schädigen, das war die Sorge, die Ill beschäftigte. Er mußte die
schließliche Entscheidung den Verhältnissen überlassen, wie der
Augenblick sie bieten würde.

Nach einer Stunde war das ewige Eis des grönländischen Festlands
überflogen. Die weiten Felder des Humboldtgletschers senkten sich
zum Meer hinab. Das Luftschiff mäßigte seinen Flug und stieg
abwärts, so schnell es die Rücksicht auf die Insassen gestattete,
die sich an den höheren Luftdruck erst gewöhnen mußten. Jetzt war
die Höhe von 1.500 Metern erreicht.

Ill schob leise die Tür zu Ismas Schlafraum beiseite und entfernte
die Maske von ihrem Gesicht. Sie erwachte und schaute sich erstaunt
um. Er löste den Ring von ihrem Handgelenk und sagte ihr, daß sie
jetzt, falls sie es wünsche, sich erheben könne. Darauf entfernte
er sich und zog die Tür wieder zu.

Wenige Minuten darauf trat Isma in die Kajüte. Ihre Wangen waren
gerötet. Verlegen blickte sie umher.

„Wo sind wir?“ fragte sie.

„An der Westküste von Grönland, auf dem 80. Grad nördlicher
Breite“, sagte Ell, ihr die Hand reichend. Sie ließ sich auf einen
Sessel fallen und bedeckte die Augen mit den Händen. Sie schwieg
lange.

„Lassen Sie mich sehen“, sagte sie dann.

Man trat aus der Kajüte in das Schiff. Die seitlichen Fenster waren
jetzt teilweise geöffnet. Man konnte hinausblicken.

Ein farbenprächtiges Nordlicht entsandte seine zuckenden Strahlen
über das Firmament, während im Nordosten die Morgendämmerung ihren
bleichen Schein entfaltete. Tief unten, in undeutlichen Reflexen
schimmernd, erstreckten sich die zerrissenen Eismassen des
Humboldtgletschers, der als eine Riesenmauer von Eis über dem Meer
abbrach. Am westlichen Horizont erhob sich wie eine dunkle Wand der
eisfreie Meeresspiegel der Kane-Bai.

Isma stand lange in den überwältigenden Anblick versunken.

„Es ist ja noch Nacht?“ sagte sie dann fragend. „Wie spät ist es
denn?“

„Es ist sogar, nach Ortszeit, noch eine Stunde früher, als bei
unsrer Abfahrt in Friedau“, antwortete Ell, „weil wir nach Westen
gefahren sind. Trotzdem sind wir vier Stunden unterwegs. In Friedau
ist es jetzt etwa acht Uhr morgens.“

„In Friedau!“ Isma zog den Pelz dichter um ihre Schultern. Und
unter ihr die Gletscher Grönlands!

Ein Schwindel drohte sie zu erfassen.

„Kommen Sie in die Kajüte“, sagte Ill. „Es ist jetzt erst wenig da
unten zu erkennen, aber wir steigen noch tiefer und reisen nicht
weiter nach Westen. Nun wird die Sonne bald aufgehen, es wird
heller und wärmer werden. Inzwischen lassen Sie uns für ihre
Kräftigung sorgen. Auch in den ungewohntesten Situationen ist
Frühstücken eine empfehlenswerte Handlung. Ell hat daran gedacht,
daß Sie ihren Friedauer Morgenkaffee nicht zu entbehren brauchen.“

Ell übersetzte getreulich die Worte des Oheims.

Ein Lächeln glitt über Ismas Züge. „Sie denken an alles“, sagte
sie, Ell anblickend, „und ich – was werde ich nicht alles vergessen
haben! Hoffentlich hat Luise meinen Zettel gefunden.“

„Etwas habe ich doch vergessen“, sagte Ell zu Ill, „nämlich ein
Signalbuch, für den Fall, daß uns das Schiff Signale macht.
Übrigens würden wir sie doch nicht beantworten können.“

„Richtig, es ist schade“, antwortete Ill, „dafür besitzen wir ein
vorzügliches Sprachrohr, mit dem wir uns verständlich machen
können.“

Sie begaben sich in die Kajüte, und ausnahmsweise, um Isma zu
ehren, wohnte Ill dem gemeinschaftlichen Frühstück bei, obwohl er
sich auf einige Züge aus einem martischen Mundstück beschränkte. Er
verfolgte inzwischen den Gang des Schiffes auf der
Projektionsscheibe.

Als Ell und Isma wieder den offenen Schiffsraum betraten, war es
Tag geworden. Das Schiff strich in mäßiger Bewegung – immerhin noch
mit Schnellzugsgeschwindigkeit – mit weit ausgebreiteten Flügeln in
etwa dreihundert Meter Höhe über die Meeresoberfläche hin. Es hatte
sich der Ostküste von Grinnell-Land genähert und folgte nun dem
offenen Fahrwasser in ihrer Nähe nach Norden. Isma spähte mit Ells
Relieffernrohr eifrig nach der Küste hinüber. Auf den Uferschollen
sonnten sich Seehunde, zahllose Vögel saßen auf den Klippen, selbst
einige Moschusochsen konnte sie auf einer entfernten Ebene mit
Hilfe des vorzüglichen Glases erkennen. Überall glaubte sie
Menschen oder Hütten von Eskimos zu sehen, es war ihr, als müßte
sie jeden Augenblick auf Torms Spuren stoßen, und erst allmählich
begann sie ruhiger zu werden. So also sah die Gegend aus, die er im
Geleit der tranduftenden Gastfreunde durchzog! Ob es wohl glücken
würde?

Der Anruf des Martiers, der den Ausguck im Vorderteil des Schiffes
hielt, unterbrach ihr Sinnen.

\section{26 - Der Kampf mit dem Luftschiff}

Am Horizont zeigte sich eine Rauchwolke, die sich vergrößerte. Das
Dampfschiff, nach Süden steuernd, und das nach Norden fliegende
Luftschiff, das seine Geschwindigkeit sogleich steigerte und die
Flügel verkürzte, näherten sich rasch. Bald konnte man die Formen
des Schiffes durch das Glas unterscheiden. Der Wimpel am Großtopp
ließ es als Kriegsschiff erkennen. Jetzt hatte man auch an Bord der
›Prevention‹ das Luftschiff gesehen. Dieses senkte sich bis auf
hundert Meter über die Oberfläche des Meeres und schoß direkt auf
das Kanonenboot zu. Dort stieg eine weiße Dampfwolke in die Höhe,
und ein Kanonenschuß donnerte über die Flut. Man konnte jetzt die
Flagge erkennen.

„Es ist ein Engländer“, sagte Ell. „Er fordert uns auf, unsere
Flagge zu zeigen.“

Eine Flagge führte zwar das Luftschiff nicht, man hatte aber diesen
Fall vorgesehen und, um keine besonderen Verwicklungen
hervorzurufen, eine Flagge improvisiert, die dem Banner der
vereinigten Marsstaaten nachgebildet war. Sie bestand einfach in
einem schwarzen Tuch von dreieckiger Gestalt, das in der Mitte
einen großen orangenfarbigen Kreis trug.

Die Flagge wurde jetzt gehißt, das Luftschiff setzte aber seinen
Lauf fort. Ill wollte denselben erst in unmittelbarer Nähe des
Schiffes anhalten. Vorsichtshalber stieg er jedoch schnell in
größere Höhe.

Ell beobachtete mit dem Glas die Vorgänge an Deck des Schiffes.

„Die gefangenen Martier sind jedenfalls unter Deck“, sagte er. „Das
Schiff ist klar zum Gefecht – ich glaube, man will auf uns
schießen. Willst du nicht lieber anhalten?“

„Wie ist das Schiff bewaffnet?“ fragte Ill.

„Es ist, soviel ich davon verstehe, ein sogenannter Torpedo-
Rammkreuzer. Den Rammsteven und die Torpedos haben wir freilich
nicht zu fürchten, aber das 25-Zentimeter-Geschütz auf dem Deck ist
eine furchtbare Waffe. Es schleudert mit einer Geschwindigkeit von
über 600 Metern Granaten, die vielleicht den dritten Teil des
Gewichts unseres ganzen Schiffes haben. Ein einziger Schuß
zerschmettert uns in Atome.“

„Wenn er uns trifft. Aber wie du siehst, sind wir bereits wieder
auf achthundert Meter gestiegen und dem Schiff so nahe, daß sie dem
Geschütz nicht die genügende Erhebung geben können.“

Ein gewaltiger Knall unterbrach ihn. Kapitän Keswick hatte sein
Riesengeschütz sprechen lassen. Aber das Geschoß flog, bedeutend
tiefer als das Luftschiff, unter ihm hin, ohne Schaden zu tun.

„Die Sache ist nicht so gefährlich“, sagte Ill, „selbst wenn wir in
der Schußlinie wären, könnten wir den Schuß aufnehmen – da wir
dreimal so viel Masse haben als das Geschoß, würde es uns nur eine
Geschwindigkeit von höchstens zweihundert Metern geben, und das ist
für uns das Gewöhnliche.“

Ell sah ihn erstaunt an.

„Ich meine, wenn wir den Stoß auffangen.“

„Aber wir werden doch zerschmettert.“

„Keine Sorge! Wir müssen nur aufpassen. Jetzt aber wollen wir
verhandeln.“

„Wollen Sie sich nicht lieber in die Kajüte begeben?“

Diese Frage richtete Ill an Isma, die den Vorgängen mit Herzklopfen
gefolgt war. „Diese Herren sehen mir gerade so aus, als wollten sie
uns mit ihren Flintenschüssen begrüßen.“

„O lassen Sie mich hier“, bat Isma. „Könnte nicht vielleicht – mein
Mann – auf dem Schiff sein?“

„Das werden wir alles erfahren. Ell soll durch das Sprachrohr die
Verhandlung als Dolmetscher führen.“

Wirklich beschoß man das Luftschiff jetzt aus den Gewehren. Es
schwebte aber bereits so hoch und so nahe senkrecht über dem
englischen Kanonenboot, daß die Kugeln ihm keinen Schaden tun
konnten, obwohl sich die Engländer zum Zielen auf den Rücken
legten. Jetzt fiel eines der abgeschossenen Langbleie auf das
Verdeck des Schiffes selbst zurück und durchschlug seine Planken.
Das Feuer mußte eingestellt werden, da die Kugeln die Schützen
selbst zu treffen drohten.

Die Martier entfalteten nunmehr eine große, weiße Fahne als Zeichen
der Freundschaft und des Friedens. Alsdann senkte sich das
Luftschiff, immer mit gleicher Geschwindigkeit senkrecht über dem
Kriegsschiff bleibend, zu diesem herab, erst schnell, dann
langsamer, bis es sich in einer Höhe von etwa fünfzig Metern über
den Spitzen der Masten hielt.

Die Besatzung des Schiffes bestand aus tapferen Männern. Aber bei
diesem Anblick pochte allen das Herz in der Brust. Wenn die Fremden
Verräter waren? Wenn sie jetzt eine Dynamitbombe herabfallen ließen
jeder sagte sich, daß das Schiff dann verloren war. Und sie waren
wehrlos. Aber hätte das Luftschiff feindlich vorgehen wollen, so
hätte es dies sicherer aus der früheren Höhe tun können.

Der Kapitän stand mit finsteren Blicken auf der Kommandobrücke.

Jetzt zuckte er zusammen. Aus der Höhe kam ein Anruf in englischer
Sprache.

„Wer seid Ihr?“ fragte er durch das Sprachrohr entgegen.

Ell versuchte eine Erklärung zu geben. Das Luftschiff habe keine
feindlichen Absichten. Es gehöre demselben Staat an wie die beiden
Gefangenen, die sich auf dem englischen Schiff befänden. Sie seien
Bewohner des Planeten Mars, die auf dem Nordpol der Erde eine
Kolonie angelegt hätten. Die beiden würden zu Unrecht
gefangengehalten, sie hätten sich an den Engländern nicht
vergriffen, vielmehr die in den Abgrund gestürzten heraufbefördert.
Das Luftschiff wolle nichts als die beiden Gefangenen zurückhaben.
Man möge sie in der Nähe ans Land setzen, wo das Luftschiff sie
abholen werde. Außerdem wolle man wissen, ob das Schiff Nachricht
von der deutschen Nordpolexpedition Torm habe.

Kapitän Keswick erwiderte, von der Tormschen Expedition habe er bis
jetzt keinerlei Spuren gefunden. Was die andere Frage beträfe, so
verböte es ihm seine Ehre, mit dem Luftschiff zu verhandeln, so
lange es sich über seinem eigenen Schiff in bedrohender Stellung
befände. Der Kommandant möge zu ihm an Bord kommen; er garantiere
ihm unbehinderte Rückkehr.

Es trat eine Pause ein. Auf beiden Schiffen wurde Kriegsrat
gehalten.

Ill wollte ohne weiteres dem Wunsch des Kapitäns nachgeben und ihn
besuchen, aber Ell riet ihm dringend davon ab.

„Traust du ihm nicht?“ fragte Ill.

„Das nicht“, sagte Ell, „sein Wort wird er halten. Aber nach den
Anschauungen der Menschen würden wir damit anerkennen, daß wir uns
den Bestimmungen des englischen Kriegsschiffs unterordnen. Der
Hochmut der Engländer würde dadurch nur wachsen und die
Verhandlungen erschweren. Wir nehmen für uns selbst den Charakter
eines Kriegsschiffs in Anspruch.“

„Es mag sein, doch liegt kein Grund vor, unsre Stellung über dem
Schiff beizuhalten, wenn sie den Kapitän beunruhigt. Ich habe mich
nur hierhergelegt, um überhaupt zu Wort zu kommen. Wir können ja
auch jeden Augenblick hierher zurückkehren, wenn wir wollen; nur
nützt es uns wenig. Mit einer Vernichtung des Schiffes zu drohen,
geht nicht an, da ich sie doch nicht ausführen würde und auch die
Leute sich sagen dürften, daß wir das Schiff nicht in Grund bohren
werden, so lange unsere Kameraden sich darauf befinden.“

Ell rief nun durch das Sprachrohr hinab, daß sich das Luftschiff in
einiger Entfernung niederlassen werde. Auf demselben befinde sich
einer der höchsten Beamten des Mars, der nicht daran denke, sich
zuerst dem Kapitän vorzustellen. Der Kapitän möge daher entweder zu
ihm an Bord kommen oder eine Stelle am Ufer zur Zusammenkunft
bestimmen. Im übrigen genüge es, wenn der Kapitän die beiden
Martier ans Land sende. Das Luftschiff werde sich dann sogleich
entfernen, sobald es die beiden aufgenommen hätte.

Ohne eine Antwort abzuwarten, ließ Ill das Luftschiff nach dem Land
zu lenken.

Der Engländer hatte inzwischen seinen Lauf angehalten und lag jetzt
still. Ihm gegenüber, etwas über einen Kilometer entfernt, in
geringer Höhe über dem Ufer, schwebte das Luftschiff der Martier in
vollkommener Ruhe. Flügel und Steuer waren eingezogen. Der
Hinterteil des Fahrzeugs war gegen das Kriegsschiff gewendet und
zeigte die Öffnung eines bis dahin nicht sichtbar gewesenen Rohres.
Kapitän Keswick hatte seinen Zweck erreicht, Zeit zu gewinnen und
das unheimliche Fahrzeug über seinem Kopf zu entfernen. Er fühlte
sich wieder sehr erhaben. Er dachte nun erst recht nicht daran, die
Gefangenen auszuliefern. Verhielt es sich wirklich so, daß sie
Marsbewohner waren – und eine bessere Erklärung angesichts des
Luftschiffes wußte keiner seiner Offiziere –, so wollte er sich den
Triumph nicht nehmen lassen, diese seltsamen Geschöpfe nach London
zu bringen. Daß man auf dem Mars auch englisch verstand und sich
nach der deutschen Nordpolexpedition erkundigte, war schließlich
nicht wunderbarer als die Existenz des Luftschiffes überhaupt. Die
Zumutung, einem englischen Kriegsschiff Bedingungen zu stellen,
hielt Kapitän Keswick für eine Frechheit. Seiner Ansicht nach hatte
das fremde Schiff einfach zu gehorchen.

Er signalisierte daher jetzt, das Schiff möge sofort die Flagge
streichen und sich ergeben. Da er sich aber allerdings selbst
sagte, daß man drüben die Signale nicht verstehen würde, so
schickte er einen Offizier in der Jolle soweit vor, bis er durchs
Sprachrohr mit dem Luftschiff reden konnte, und ließ durch ihn
seinen Befehl ausrichten. Das Luftschiff solle landen und die
Besatzung sich von demselben ohne Waffen auf tausend Schritt
zurückziehen. Geschähe das nicht, bis das Boot wieder an Bord sei,
so würde er Gewalt anwenden.

Ill ließ antworten, es würde ihm sehr leid tun, wenn er seinerseits
Gewalt anwenden müßte, um seine Genossen wieder zu erhalten. Bei
der geringsten Feindseligkeit seitens der Engländer würde er sich
jedoch gezwungen sehen, ihr Schiff kampfunfähig zu machen. Sollte
einem der Martier Leides geschehen, so hafteten Kapitän, Offiziere
und Mannschaft mit ihrem Leben.

Der Offizier brachte diese Antwort zurück.

„Wir werden mit den Leuten deutlicher reden“, sagte Keswick.

Leutnant Prim hätte sich gern aus Vergnügen die Hände gerieben,
aber sie waren immer noch steif. Er konnte nicht einmal seinen
Feldstecher halten. Das Luftschiff lag vollkommen ruhig, es konnte
gar kein besseres Ziel für das 25-Zentimeter-Geschütz geben, es
war nicht zu verfehlen.

Ell beobachtete, daß das Boot kaum beim Schiff angekommen war, als
man das Geschütz richtete.

„Wir sind verloren“, rief er Ill zu.

Dieser hatte schon seine Vorkehrungen getroffen. Er sah scharf auf
die Mündung des Geschützes.

„Halte dich fest und befürchte nichts“, sagte er zu Ell gewendet.
Seine Hand lag am Griff des Repulsitapparates. Von dem Moment, in
welchem der Schuß an Bord des Kriegsschiffs gelöst wurde, bis zu
demjenigen, in welchem das Geschoß das Luftschiff treffen konnte,
mußten fast zwei Sekunden vergehen. Das genügte ihm.

Jetzt blitzte drüben der Schuß auf. Das vernichtende Geschoß war
entsandt. Ell fühlte, wie sich ihm die Kehle zusammenschnürte, aber
er vertraute auf die Kraft der Nume. Isma hatte sich auf seine
Bitte schon vorher zurückgezogen und war sich der unmittelbaren
Gefahr glücklicherweise nicht bewußt.

Ill hatte gleichzeitig den Griff des Repulsitgeschützes gedreht.
Das Luftschiff erhielt einen Stoß und sauste durch die Luft. Hinter
ihm, etwa in der Mitte zwischen dem englischen Schiff und dem
martischen, gab es einen ohrenbetäubenden Krach. Die Granate
zersprang in der Luft, als sei sie an eine feste, unsichtbare Mauer
gestoßen. Die Bruchstücke flogen nicht weiter, sie fielen direkt
nach unten und ließen das Meer unter sich aufschäumen.

Im Moment aber spannte das Luftschiff seine Flügel aus, in engem
Kreis kehrte es zurück, binnen zehn Sekunden war es wieder bei der
›Prevention‹ angelangt, hinter dem Kanonenboot sank es bis zur
halben Höhe seiner Masten. Ein zweiter Repulsitschuß knickte die
eisernen Masten wie Strohhalme, die mit einer scharfen Sense
abgeschnitten werden. Zugleich aber wurden sie wie von einem
Sturmwind fortgetragen, der sie über das Schiff hinwegfegte und
gegen hundert Meter weiter ins Meer fallen ließ. Auf dem Verdeck
selbst wurde nichts direkt von dem Schuß betroffen; nur die
entstehende gewaltige Luftwelle warf die gesamte Mannschaft über
den Haufen und setzte das ganze Schiff in schwankende Bewegung. Ehe
sich die Engländer wieder auf ihre Füße gefunden hatten, war das
Luftschiff, in kurzer Wendung aufsteigend, umgekehrt und ruhte in
etwa tausend Meter Höhe senkrecht über dem Kanonenboot.

Ill hatte nur die Wirkung seiner Waffen zeigen wollen. Der im
Repulsitgeschütz sich entspannende Äther entwich mit einer
Geschwindigkeit, welche der des Lichtes vergleichbar war, und riß
die Luft und alles, was in seinem Weg lag, mit sich fort, obgleich
seine Masse nur wenige Gramm betrug. Er breitete sich kegelförmig
aus und mußte daher das ihm entgegen fliegende Sprenggeschoß
auffangen und zur Ruhe bringen. Ill wollte jetzt das Luftschiff
wieder sich herabsenken lassen, um neue Verhandlungen zu beginnen,
aber die zur Wut gereizten Feinde beschossen es aus ihren Gewehren
ohne Rücksicht auf die Gefahr, von ihren eigenen Kugeln getroffen
zu werden. Wie sollte er nun, ohne Menschenleben zu vernichten und
das Schiff selbst unbrauchbar zu machen, die Herausgabe der
Gefangenen erzwingen?

Ill hätte durch den Telelyten das Geschütz demontieren oder das
Schiff leck machen können. Der Telelyt ist ein Apparat, durch
welchen chemische Wirkung in jeder beliebigen Form erzeugt werden
kann, soweit nur die direkte Bestrahlung des Gegenstandes vom
Apparat aus möglich ist. Wenn man zum Beispiel glühenden Sauerstoff
durch den Telelyten treten ließ, so wurde die chemische Energie
durch Strahlung fortgepflanzt und kam auf dem bestrahlten Körper,
etwa dem Gußstahl des Geschützes, wieder als chemische Energie zum
Vorschein, so daß der Stahl einfach verbrannt wurde.

Ill hätte auch sein Repulsitgebläse auf das Schiff richten und
dieses an beliebiger Stelle auf den Strand treiben können.

Aber er wollte sich nicht dazu entschließen. Das Geschütz konnte
ihm nicht schaden, wenn er sich über dem Schiff hielt, und auch
sonst nicht, wenn er die Abgabe des Schusses rechtzeitig bemerkte.
Und das Schiff selbst wollte er nicht untauglich zur Fortsetzung
der Reise machen. Er versuchte daher nochmals zu verhandeln und
ließ zu diesem Zweck wieder die weiße Fahne aufziehen, obwohl Ell
meinte, daß dieses Entgegenkommen falsch verstanden werden würde.

„Was wollen die Schufte?“ rief der Kapitän wütend, ließ aber das
Feuer einstellen. Das Luftschiff senkte sich. Als es so nahe
gekommen war, daß man sich durchs Sprachrohr verständigen konnte,
fragte Ell, ob man jetzt bereit sei zu kapitulieren.

„Mit euch Freibeutern gibt es keine Verhandlungen“, schrie Keswick
zurück. „Ehe ich meine Flagge streiche, sprenge ich das ganze
Schiff samt euren sauberen Brüdern in die Luft.“

„Wir verlangen nicht, daß ihr die Flagge streicht“, lautete die
Antwort. „Es genügt, wenn ihr die Gefangenen ans Land setzt. Aber
unsere Geduld ist jetzt zu Ende. Stößt das Boot mit unseren
Landsleuten nicht binnen zehn Minuten vom Schiffe ab, so macht euch
auf das Schlimmste gefaßt. Bis jetzt haben wir euch nur eine Probe
gegeben.“

„Der Teufel soll euch holen. Feuer auf die Hunde!“ schrie Keswick
wütend.

Aber schon hatte sich das Luftschiff fortgeschnellt. Nach wenigen
Sekunden war es bereits wieder über einen Kilometer vom Schiff
entfernt, das jetzt mit voller Dampfkraft nach Süden strebte.

Da Ill keine Zeit dadurch verlieren wollte, daß sich die Entfernung
des Schiffes von der Küste vergrößerte, beschloß er zunächst, den
Dampfer aufzuhalten. Er erhob sich so hoch, daß er nicht beschossen
werden konnte, und richtete dann einen Repulsitstrom gegen die
Meeresoberfläche in einiger Entfernung vor dem Schiff. Das Meer
kochte auf, als hätte man einen Berg hineingestürzt. Ein haushoher
Wogenwall wälzte sich von der getroffenen Stelle im Kreise nach
außen und zwang das englische Schiff, seinen Kurs zu ändern.
Alsbald erregte das Luftschiff durch einen zweiten Repulsitschuß an
geeigneter Stelle einen neuen Wirbel, und so zwangen die Martier
ihren Gegner, sich dahin zu wenden, wohin sie ihn haben wollten.
Bald aber war die ganze Umgebung wie von einem Sturm aufgewühlt,
und die ›Prevention‹ hatte die größte Mühe, sich in dem tollen
Wogengang zu halten. Von einem Gebrauch des Geschützes konnte beim
Schwanken des Schiffes jetzt nicht die Rede sein. Inzwischen waren
die zehn Minuten Frist längst abgelaufen. Ill ließ dem Schiff noch
Zeit, um einen Felsenvorsprung herum in ruhigeres Wasser zu
gelangen. Hier erwartete er den Engländer.

Der Kapitän sah nun wohl ein, daß er dem Luftschiff nicht entkommen
könne. Aber er war immer noch zu hartnäckig, um nachzugeben. Das
Luftschiff lag wieder vollständig ruhig und ließ das Kanonenboot
herankommen, während die Vorgänge auf demselben aufs genaueste
beobachtet wurden. Ill konnte mit seinem Sprachrohr sich bis auf
tausend Meter verständlich machen. Er rief nochmals hinüber, wenn
man jetzt nicht gehorche, werde er auf das Schiff selbst schießen.

Der Dampfer machte eine Wendung und stoppte. Die Martier glaubten,
es geschehe, um ein Boot auszusetzen; aber das Manöver hatte nur
den Zweck, zum Schuß zu kommen. Ehe die Martier es erwarten
konnten, blitzte der Schuß auf. Die Entfernung war zu kurz, um den
Gegenschuß der Martier genau abzumessen. Er erfolgte sofort, aber
er war zu heftig. Mit rasender Geschwindigkeit schleuderte der
Rückstoß das Luftschiff fort. Die Insassen wurden von ihren Plätzen
geworfen. Isma stieß einen Schrei aus und klammerte sich
schreckensbleich an die Wand. Zum Glück hatte sie keinen Schaden
genommen. Das Luftschiff gehorchte wieder dem Steuer, die Bewegung
wurde gemäßigt, es kehrte in weitem Bogen zurück und lagerte sich
in einer Entfernung von etwa acht Kilometern vom Kriegsschiff auf
der Spitze eines Hügels, von wo aus man mit dem Fernglas die
Vorgänge auf dem Schiff gut beobachten konnte.

Hier sah es schlimm aus. Unter dem Gegenstoß des Repulsits war das
Sprenggeschoß explodiert, aber die Trümmer waren nicht in das Meer
gefallen, sondern, weil die Wirkung zu stark gewesen war, auf das
Schiff zurück. Ein Teil der Mannschaft und der Kapitän selbst waren
verwundet. Der Verschluß des Geschützes war abgeschlagen. Dichter
Qualm drang aus einem der zertrümmerten Schornsteine.

Ill nahm das Glas vom Auge. Ein finsterer Ernst lagerte über seinen
Zügen.

„Es ist schrecklich“, sagte er. „Ich habe das Meinige getan, um
Blutvergießen zu vermeiden. Auch das jetzige Unglück ist gegen
meine Absicht geschehen, wir hatten bei der Plötzlichkeit des
Überfalls nicht länger Zeit, unsern Schuß abzuwägen. Die Menschen
sind wahnsinnig.“

Er sann lange nach.

„Ich erwäge“, sagte er dann, „ob ich es gegen unsere Genossen
verantworten kann, wenn ich jetzt nachgebe und das Schiff entlasse.
Aber ich bin ja nicht einmal sicher, ob man ihr Leben schonen wird,
nachdem dieses Blut geflossen ist. Das also ist unser erstes
Zusammentreffen mit den Menschen, das ist die Verbrüderung der
Planeten! Ich hatte es mir anders gedacht. Ich höre, die Menschen
haben unsern Planeten nach dem Gott des Krieges genannt; wir
wollten den Frieden bringen, aber es scheint, daß die Berührung mit
diesem wilden Geschlecht uns in die Barbarei zurückwirft. Gott
gebe, daß diese Begegnung kein Vorzeichen ist. Indessen – wir
können nicht mehr zurück. Wir wollen aus dem einen Fall noch keine
Schlüsse ziehen.“

Er wandte sich zu Isma und sagte ihr bedauernde Worte, daß ihre
Reise mit so schrecklichen Ereignissen begönne. Ell wollte eben
seine Äußerungen übersetzen, als der wachthabende Martier meldete:

„Das Schiff setzt ein Boot aus.“

Es war so, man sah, daß die beiden Martier in das Boot
hinabgelassen wurden. Dieses ruderte dem Land zu. In einer kleinen
Bucht, deren Ufer mit Eisschollen bedeckt waren, landeten die
Engländer. Sie warfen die Gefangenen rücksichtslos auf eine
Scholle, feuerten ihre Gewehre in die Luft ab, um ein Signal zu
geben, und kehrten dann schleunigst zurück an Bord ihres Schiffes.

Sofort befahl Ill, daß das Luftschiff aufsteigen solle, um die
Genossen abzuholen. Der Weg war nicht weit, doch lag die kleine
Bucht auf der anderen Seite des Kriegsschiffs, das man in einem
Bogen umgehen mußte, um sich nicht etwaigem Gewehrfeuer
auszusetzen. Dann senkte sich das Schiff mit eingezogenen Flügeln
nahe am felsigen Abhang hinab. Hierbei streifte es einmal bis dicht
an einen Felsen und legte sich stärker nach der Seite, als
beabsichtigt war. Der Ingenieur machte ein bedenkliches Gesicht. Es
kam bei diesen langsamen Bewegungen auf und nieder auf die äußerste
Präzision in der Funktion des diabarischen Apparats an, und es
schien ihm, als ob das Schiff auf der linken Seite nicht mit
derselben Geschwindigkeit seine Schwere ändere wie auf der rechten.
Man war jetzt auf der breiten Eisscholle angelangt.

Die gefangenen, nunmehr befreiten Martier befanden sich in üblem
Zustand. Sie waren zwar nicht gefesselt, aber der Druck der
Erdschwere, dem sie seit achtzehn Stunden – denn es war inzwischen
Mittag geworden – ausgesetzt waren, die beim Kampf und zuletzt beim
Transport erlittenen Mißhandlungen und der Mangel an für sie
genießbarer Nahrung hatten sie körperlich schwer mitgenommen. Sie
atmeten beglückt auf, als im Innern des Luftschiffes ihre Leiden
gemildert wurden. Ill wandte sich betrübt ab, als er erfuhr, welche
Behandlung ihnen zuteil geworden war. Die Strafe der Engländer war
hart, dachte er, aber verdient. Und doch, im Grunde waren sie
unschuldig an ihrem Irrtum.

Und nun vorwärts zum Pol! In anderthalb Stunden konnte er erreicht
sein. Das Luftschiff erhob sich langsam, und wieder bemerkte der
Steuermann die Ungleichmäßigkeit der Diabarie auf den beiden Seiten
des Schiffes. Er machte Ill darauf aufmerksam, doch konnte man die
Ursache nicht sogleich auffinden. Inzwischen war die Höhe des
Felsufers überstiegen. Die Flügel wurden nun ausgebreitet, und vom
Reaktionsapparat getrieben glitt das Schiff auf schiefer Ebene
weiter aufwärts und nordwärts.

Plötzlich vernahm man einige scharfe Schläge gegen die Flügel des
Schiffes.

„Höher!“ rief Ill. „Höher und schneller!“

Mit dem Schiff und den geretteten Gefährten beschäftigt, hatte man
kaum noch auf den Engländer geachtet. Auch war man so weit von ihm
entfernt, daß die Martier außer Schußweite zu sein glaubten. Die
Engländer aber hatten, als sie sahen, daß das Luftschiff sich
entfernte, ihm auf gut Glück noch einige Schüsse aus ihren
weittragenden Gewehren nachgesendet, und einige Kugeln hatten es
erreicht.

„Höher“, lautete der Befehl. Aber als der diabarische Apparat
dementsprechend gestellt wurde, legte sich das Schiff auf die
Seite. Infolge der Flügelstellung beschrieb es sofort eine Spirale
nach rückwärts und kam dadurch nochmals in den Bereich der
feindlichen Geschosse. Man mußte die Diabarie der rechten Seite
wieder vermindern, da die linke nicht folgte. Das Schiff schwebte
zwar, aber man konnte es nur langsam und in engen Grenzen heben und
senken. Der Repulsitapparat war dagegen in Ordnung und trieb das
Schiff vorwärts. Es entfernte sich nun vom Schauplatz des Kampfes
nach Norden, in verhältnismäßig geringer Höhe über der Erde. Ein
Gebirge, das noch zu überwinden war, konnte nur durch das
Vorwärtstreiben mit schräggestellten Flügeln genommen werden.
Infolgedessen nahm die Fahrt bis zum Pol die vierfache Zeit wie
gewöhnlich in Anspruch.

Endlich kam die Polinsel Ara zu Gesicht, und das Schiff senkte sich
vorsichtig auf das Dach derselben. Aufs äußerste ermüdet entstiegen
die Martier dem Fahrzeug, von den Bewohnern der Insel freudig
bewillkommt. Isma wurde der Obhut der Gemahlin Ras übergeben und
von ihr aufs freundlichste aufgenommen. Ehe sie die Treppe in die
Wohnung hinabstieg, warf sie noch einen forschenden Blick auf die
Umgebung und suchte in Gedanken die Stelle zu finden, wo der
Fallschirm des Ballons herabgestürzt war. Dann reichte sie Ell die
Hand. Sie wollte zu ihm sprechen, aber sie fand keine Worte. Nur
ihr Blick dankte ihm. „Auf Wiedersehen!“

\tb

Bereits vierundzwanzig Stunden hatte Isma auf der Polinsel
zugebracht, ohne daß die in Aussicht genommenen Entdeckungsfahrten
nach ihrem Mann angetreten wurden. So sehr sie sich danach sehnte,
hatte sie doch keine Zeit, ungeduldig zu werden, denn die Fülle der
neuen Umgebung beschäftigte sie ausreichend. Die Gegenwart Ells gab
ihr die erforderliche Zuversicht in den neuen Verhältnissen.
Saltner mit Se, La und Fru waren bereits nach dem Mars abgegangen,
aber unter den noch anwesenden Martiern befanden sich noch mehrere,
mit denen sie sich deutsch unterhalten konnte, so vor allem der
Vorsteher Ra, dessen Frau und der Arzt Hil. Von ihnen erhielt sie
nicht nur Nachricht über die Verhältnisse des Mars, sondern auch
Einzelheiten über die Schicksale der Gefährten ihres Mannes, die
ihr Gemüt lebhaft bewegten.

Man begab sich eben zu der üblichen Plauderstunde ins
Empfangszimmer, wo Isma und Ell jetzt die Plätze einzunehmen
pflegten, die für Grunthe und Saltner eingerichtet waren, als Ell
mit bekümmertem Antlitz eintrat.

Isma sah ihn erschrocken an.

„Was ist geschehen?“ rief sie.

„Fassen Sie sich, liebste Freundin.“

„Hugo ist –?“

„Nein, nein – wir wissen nichts – aber wir können ihn nicht
suchen.“

„Warum nicht?“

„Das Luftschiff ist unbrauchbar geworden.“

„Um Gottes willen!“

„Der diabarische Apparat hat durch den übermäßigen Luftdruck bei
unserm zweiten Verteidigungsschuß auf das Kanonenboot einen Fehler
erhalten. Außerdem ist eine verirrte Gewehrkugel in denselben
eingedrungen und hat den Differential-Regulator verletzt. Bei der
Untersuchung stellte sich heraus, daß die Reparatur hier nicht
möglich ist. Der auseinandergenommene Apparat läßt sich nur in der
Werkstätte auf dem Mars mit den dortigen Mitteln wieder einsetzen.
Leider ist auch das kleine Luftboot für weitere Fahrten nicht mehr
zu verwenden. Wir müssen die Nachsuchungen aufgeben.“

Isma saß starr. „Mein armer Mann!“ sagte sie tonlos.

„Geben Sie sich um seinetwillen nicht so großer Sorge hin“, suchte
Ell sie zu trösten. „Er wird sicherlich glücklich heimkehren.
Vielleicht früher als wir“, setzte er zögernd hinzu.

Isma sah ihn an. Dann schlug sie die Hände vor das Gesicht und ließ
sie endlich langsam herabsinken.

„Wir können nicht – zurück –?“

„Es ist unmöglich – in diesem Jahr.“

„Und ich – ich glaubte – in acht Tagen – – o ich Törin! Was hab ich
getan! O wäre ich nicht so eigensinnig gewesen.“

„Es ist der Fall, vor dem Ill uns warnte.“

Isma weinte still. Ell saß ratlos neben ihr.

„Was nun?“ fragte sie endlich.

„Es bleibt uns nichts übrig, als mit Ill und Ra nach dem Mars zu
gehen. Im ersten Frühjahr kehren wir mit neuen Luftschiffen zurück.
Bis dahin hilft uns nichts als Fassung.“

„Nach dem Mars!“ flüsterte Isma wie geistesabwesend. Dann stand sie
auf. Sie trat vor Ell. Ihren Schmerz bezwingend, reichte sie ihm
beide Hände.

„Vertrauen Sie mir!“ sagte er.

Sie sahen sich in die Augen.

„Ich werde tun, was Sie verlangen“, erwiderte Isma. „Ich habe das
Geschick herausgefordert. Ich muß es tragen.“

„Ob auf dem Mars oder auf der Erde – wir können dieselben
bleiben.“

\part{Zweites Buch}

\section{27 - Auf dem Mars}

Über dem Südpol des Mars, um den Halbmesser des Planeten von seiner
Oberfläche entfernt, also in einer Höhe von 3.390 Kilometern,
schwebt die ausgedehnte Außenstation für die Raumschiffahrt.

Ungleich gewaltiger ist die Anlage als die am Nordpol der Erde,
denn über siebzig Raumschiffe vermögen gleichzeitig hier Platz zu
finden. Das abarische Feld, das die Außenstation in der Richtung
der Achse mit dem Pol des Planeten verbindet, befördert stündlich
einen geräumigen Flugwagen.

Heute waren die aufsteigenden Wagen bis auf den letzten Platz
besetzt. Nicht nur die Bevölkerung der nächsten Umgebung drängte
sich zu den Flugwagen, selbst aus den entlegeneren Gegenden waren
Neugierige auf den schnellen Bahnwagen herbeigeeilt, um der
Rückkehr des Regierungsschiffes von der Erde beizuwohnen. Denn
heute wurde der ›Glo‹ erwartet. Die Lichtdepesche hatte gemeldet,
daß der Repräsentant Ill auf der Erde den Sohn seines verunglückten
Bruders, des verschollenen Raumfahrers All, aufgefunden habe und
zurückbringe. Man durfte auf merkwürdige Neuigkeiten von der Erde
rechnen. Auch das Raumschiff ›Meteor‹, Kapitän Oß, welches bereits
vor dem ›Glo‹ die Erde verlassen hatte, wurde erwartet. Es sollte
den ersten Menschen von der Erde auf den Mars bringen. Man erzählte
die wunderbarsten Geschichten von seiner furchtbaren Stärke. Zehn
Nume seien notwendig, um ihn in Schranken zu halten.

„Ist es denn wahr“, fragte eine besorgte Mutter, ihr Töchterchen
ängstlich an sich ziehend, „daß die Menschen kleine Kinder
fressen?“

Ihre Nachbarin im Flugwagen antwortete: „Ich weiß es nicht im
allgemeinen, aber der, den wir jetzt erwarten, frißt keine Kinder.
Ich weiß es ganz genau, denn ich erwarte meine Schwester Se, die
ihn kennt; wir haben mit dem ›Kometen‹, Kapitän Jo, Briefe von ihr
bekommen, und sie schreibt, daß er ein ganz netter, beinahe
zivilisierter Mann sei. Sie sehen, ich habe ja auch meinen kleinen
Wast und sogar meine Ern mitgebracht. Haltet euch fest, Kinder, wir
sind gleich da!“

Die weiten Galerien des Ringes der Außenstation waren seit Stunden
dicht mit Zuschauern besetzt, die sich vor den
Projektionsfernrohren drängten und bald die Aussicht auf den Mars
bewunderten, bald den gestirnten Himmel durchmusterten. Mit
besonderer Vorliebe wurde die Erde aufgesucht, doch da sie fast in
derselben Richtung wie die Sonne stand, konnte sie nicht gut
beobachtet werden.

Der ›Glo‹ war bereits nahe herangekommen, sein roter Glanz ließ ihn
im Fernrohr nicht verkennen. Man konnte die Landung in zwei bis
drei Stunden erwarten. Aber auch der ›Meteor‹ war schon
signalisiert. In acht bis zehn Stunden mochte er eintreffen.

Die Reise des ›Glo‹ war so beschleunigt worden, wie man es nie bei
einem Raumschiff gewagt hatte. Die allgemeine Aufregung, die in
allen Marsstaaten aufgrund der neuen Depeschen von der Erde
entstanden war, machte wichtige politische Erwägungen und die
Anwesenheit Ills im Zentralrat notwendig. Ill hatte außerdem das
persönliche Interesse, Isma, der er sehr zugetan war, die
Beschwerden der Reise möglichst abzukürzen. So war, durch die
Stellung der Planeten begünstigt, das Außerordentliche gelungen;
die Reise von der Erde zum Mars, also der Sonnenanziehung entgegen,
war in acht Tagen zurückgelegt worden. Man hatte den ›Meteor‹,
welcher sieben Tage früher von der Erde abgegangen war, überholt.
Freilich durfte er sich nicht die Repulsitverschwendung gestatten
wie das im Auftrag des Zentralrats fliegende Eilraumschiff.

Mit rührender Sorgfalt hatte Ill, den Ratschlägen Ells folgend,
Isma den Aufenthalt im Raumschiff behaglich zu machen gesucht. Die
Raumkrankheit, eine Folge der zeitweiligen Aufhebung der
Gravitation, pflegte selbst erprobten Raumschiffern nicht ganz
fernzubleiben. Auch Isma hatte unter ihr zu leiden. Aber die
Beschwerden, die ihr durch die geringe Schwere innerhalb des
Raumschiffes drohten, waren ihr durch eine sinnreiche Konstruktion
ihres Schlafraumes sehr erleichtert worden. Derselbe stellte zwar
nicht viel mehr als einen durch geeignete Ventile ausreichend
gelüfteten Kasten vor, aber es war darin künstlich Schwere und
Luftdruck der Erde erzeugt. Und so konnte Isma nicht nur während
des Schlafes ganz nach ihrer Gewohnheit ruhen, sondern auch im
Laufe des Tages sich von Zeit zu Zeit zur Erholung dahin
zurückziehen. Sie fühlte sich daher vollkommen wohl, als der ›Glo‹
sich bereits dem Mars näherte.

Wie oft auch ihre Gedanken sehnsüchtig nach der Erde zurückeilten
und sich um das Schicksal ihres Mannes mit Bangen bewegten, so war
doch die Fülle der neuen Eindrücke gewaltig genug, um sie aufs
lebhafteste zu beschäftigen und zu zerstreuen. Die Notwendigkeit,
nun ein halbes Erdenjahr auf dem Mars zuzubringen, ließ sie die
Muße der Reise benutzen, mit Ells Hilfe in die Sprache der Martier
einzudringen, während sich Ill gleichzeitig das Deutsche aneignete.
Auch an weiblicher Gesellschaft während der Überfahrt fehlte es
Isma nicht, da gegen zehn Frauen verschiedenen Lebensalters mit dem
›Glo‹ von der Erde zurückkehrten.

Längst war die schmale Sichel der Erde als ein lichter Stern unter
die übrigen zurückgesunken, und die Verkleinerung des Sonnenballs
infolge der größeren Entfernung von ihm ließ sich, wenn man die
Strahlung durch ein dunkles Glas abblendete, sichtlich bemerken.
Immer mächtiger trat das Ziel der Reise, der Mars, als hell
leuchtende Scheibe hervor. Jetzt hatte man sich über die Marsbahn
erhoben, um, in unmittelbarer Nähe des Planeten, sich in der
Richtung der Achse auf seinen Südpol hinabsinken zu lassen. Nur
noch etwa 13.000 Kilometer trennten das Raumschiff von der
Außenstation. Aber um diese Strecke zu durchfliegen, die man bei
der vollen Fahrtgeschwindigkeit fern vom Planeten in zwei bis drei
Minuten zurücklegte, bedurfte man jetzt ebenso vieler Stunden. Es
galt, die Geschwindigkeit zuletzt durch Repulsitschüsse so zu
vermindern, daß man gerade auf dem Ring der Außenstation zur Ruhe
kam. Die Schwierigkeit der Landung erforderte die volle
Aufmerksamkeit des Kapitäns Fei.

Als bevorzugte Gäste des Zentralrats konnten sich Isma und Ell bei
Ill auf einer kleinen reservierten Tribüne dicht neben der
Kommandobrücke aufhalten. Isma mit bangem Herzen, Ell in freudiger
Aufregung, die nur durch die Teilnahme am Geschick der Freundin
gedämpft war, hefteten ihre Blicke erwartungsvoll auf die neue
Welt, die sich zu ihren Füßen auftat.

\tb{}
Es war Sommer am Südpol des Mars, und so zeigten sich, hier von der
Achse aus gesehen, etwa zwei Drittel von der Scheibe des Planeten
beleuchtet, während ein Drittel in tiefem Dunkel lag. Auf dem
erhellten Teil vermochte man jetzt die Südhalbkugel bis gegen den
zehnten Grad südlicher Marsbreite zu überblicken. Dieser Horizont
verengte sich mehr und mehr beim Herabsinken des Raumschiffes,
während infolge der größeren Annäherung das Bild des Planeten an
Ausdehnung zunahm und die Einzelheiten immer deutlicher
hervortraten. Infolge der dünnen, durchsichtigen, wolkenlosen
Atmosphäre lag die Gestaltung der Oberfläche bis an den Rand der
sichtbaren Fläche klar vor Augen. in der Nähe des Poles und nach
der Schattengrenze hin dehnten sich weite Gebiete von grauer, ins
Blaugrüne spielender Färbung, das Mare australe der Astronomen der
Erde. Der Pol selbst war eisfrei, aber westlich von ihm lagen
zwischen den dunklen Landesteilen noch langgestreckte Schneeflächen
bis zum 80. Breitengrad hinab. Zwei ausgedehnte große Flecken, die
weiter nördlich zwischen dem 60. und 70. Breitengrad hellrot im
Sonnenschein glänzten, bezeichnete Ill als die Wüsten Gol und Sek;
sie werden auf der Erde die beiden Inseln Thyle genannt. Im übrigen
Teil der sichtbaren Scheibe herrschte diese hellrote Farbe vor,
doch an mehreren Stellen von breiten und ausgedehnten grauen
Gebieten unterbrochen. Alle diese dunkeln Stellen waren
untereinander durch dunkle Streifen verbunden, die sich geradlinig
durch die hellen Gebiete hindurchzogen. Die hellen Teile sind teils
sandige, teils felsige Hochplateaus, trockene und fast
vegetationslose Gegenden, in denen sich nur spärliche Ansiedlungen
zur Gewinnung der Mineralschätze des Bodens befinden. Dicht
bevölkert dagegen sind die dunklen Teile, deren Erdreich von
Feuchtigkeit durchdrungen und mit einem üppigen Pflanzenwuchs
bedeckt ist.

\tb{}
Ein seltsames Farbenspiel entwickelte sich an der Schattengrenze,
an welcher die Sonne für die Marsbewohner im Aufgehen und die Nacht
zu entschwinden im Begriff war. Während der Nacht bedeckte sich die
Oberfläche des Planeten infolge der starken Abkühlung weithin mit
einer Nebelschicht. Wo diese dichter war, dauerte es einige Zeit,
ehe sie von den Strahlen der Sonne aufgesogen wurde, und hier
erschienen glänzende Lichter durch den Reflex der Strahlen auf den
Nebeln. Einzelne der Hochplateaus erhoben sich so weit, daß sie mit
Schnee oder Reif bedeckt waren, der aber bald in den Strahlen der
Sonne verschwand.

Ill wies nach einer Stelle nahe am nördlichen Rand des
Vegetationsgebiets, schon an der Grenze des Horizonts, wo der graue
Grund eine Mannigfaltigkeit von teils helleren, teils dunkleren
Konturen aufwies und wohin durch die benachbarten roten Wüsten eine
besonders große Anzahl dunkler Streifen zusammenliefen.

„Dort liegt Kla“, sagte er, „der Sitz des Zentralrats, und dort
werden wir zunächst wohnen. Nur wenn der Sommer noch weiter
fortgeschritten ist, rücken wir weiter nach dem Südpol vor.“

„Es wird mir leicht werden“, bemerkte Isma mit einem wehmütigen
Lächeln, „denn ich werde nicht viel Gepäck haben.“

„Daran wird es Ihnen nicht fehlen, ich werde es mir nicht nehmen
lassen, Ihnen eine vollständig eingerichtete Wohnung zur Verfügung
zu stellen. Sie werden sich dann wohl bequemen, unsere Tracht
anzunehmen, denn es wird Ihnen nicht angenehm sein, aufzufallen.
Übrigens müssen Sie wissen, daß ein Umzug von einem Ort zum andern
kein Einpacken und Umräumen erfordert. Wir ziehen mit unserm ganzen
Haus. Sie bestellen nur beim nächsten Transportbüro, wann und wohin
Sie befördert sein wollen, legen sich ruhig schlafen und sind am
andern Morgen an Ort und Stelle.“

„Es wird nämlich meistens in der Nacht gezogen“, erklärte Ell
weiter. „Die Häuser stehen auf Rollschlitten und werden auf unsern
Gleitbahnen befördert. Größere Lasten lassen sich vorteilhafter in
der Nacht fortbringen, am Tag würden wir bei der herrschenden
Trockenheit stärkeren Wasserverbrauch haben.“

„Hat denn jede Familie ihr eigenes Haus?“

„In den wohlhabenden Staaten gewiß, und wo man es sich gestatten
kann, sogar jede einzelne Person. Die Häuser sind nicht sehr groß,
es werden aber diejenigen einer Familie zu einer zusammenhängenden
Gruppe verbunden. Sie werden es bald sehen, denn wir nähern uns dem
Ziel. Blicken Sie gerade unter uns. Der glänzende Punkt – es ist
schon eine kleine Scheibe – ist der Ring der Außenstation. Von dort
bringt uns der Fallwagen nach Polstadt, wo wir zunächst
übernachten.“

„Das Letztere“, bemerkte Ill, „ist noch nicht gewiß. Vielleicht
müssen wir unsre Reise sogleich fortsetzen. Doch gehen unsre Wagen
so ruhig und sind so bequem eingerichtet, daß Sie keinerlei
Anstrengung zu fürchten haben.“

An der unteren Wölbung des Raumschiffs flammte das Zeichen der
Marsstaaten auf. Der ›Glo‹ hatte sich bis dicht über die Station
gesenkt, deren Raumschiffe wie eine Stadt aus riesigen Kuppeldomen
im Sonnenschein strahlten. Alle diese Schiffe ließen jetzt ihre
Symbole und Flaggenzeichen an ihren Wölbungen zur Begrüßung
aufleuchten. Fast unmerklich langsam glitt das Schiff auf seinen
Platz nieder. Kein Laut unterbrach die Stille, durch die Leere des
Weltraums pflanzte sich kein Schall fort. Aber hinter den
durchsichtigen Wänden der Galerien sah man eine gedrängte Menge,
die dem nahenden Schiff mit Schleiern ihr Willkommen zuwinkte.

Der aufnehmende Zylinder senkte sich in die Empfangshalle, der
›Glo‹ ruhte an seinem Ziel; der Stationsbeamte betrat durch die
Eingangsluke das Schiff. Ill mit seinen Gästen zog sich zunächst in
das Innere des Schiffes zurück. Nach Erfüllung der erforderlichen
Förmlichkeiten wurde das Verlassen des Schiffes gestattet. Zunächst
strömten die von der Erde abgelösten Martier heraus und wurden von
ihren Verwandten und Freunden jubelnd bewillkommnet. Erst nachdem
dieses rege Gewühl sich einigermaßen gelegt hatte, nahte sich eine
Deputation von Mitgliedern des Zentralrats und andern offiziellen
Persönlichkeiten und betrat das Innere des Raumschiffs. Hier
erfolgte die Begrüßung und formelle Vorstellung von Ell und Isma,
indem Ill in Kürze die notwendigsten Erklärungen gab. Ein erster
telephotischer Bericht war bereits von der Erde aus vorangegangen.

Obgleich dieser Empfang im Innern des Schiffes ziemlich lange
währte, hatten die Zuschauer es sich doch nicht nehmen lassen, in
der Empfangshalle zu warten. Absperrungen gab es nicht. Es verstand
sich von selbst, daß die Martier den Ausgang des Schiffes und den
Weg nach der Abfahrtshalle des Fallwagens im abarischen Feld
freiließen.

Endlich erschien die Empfangsdeputation wieder und schritt den Weg
nach dem Fallwagen voran. Hinter ihr kam Ill, der Isma führte,
während Ell an seiner linken Seite ging.

Isma hatte den Schleier dicht vor ihr Gesicht gezogen, sie wagte
nicht, sich umzuschauen. Ill und Ell dankten nach martischer Sitte
für die Willkommrufe, die ihnen entgegenschallten. Erst als Isma
bereits auf der Treppe des Fallwagens stand, schob sie ihren
Schleier zurück und warf einen Blick auf das bunte Bild der
bewegten Menge. Ein enthusiastischer junger Mann, der sich bis
dicht an die Treppe gedrängt hatte, warf ihr einen Gegenstand zu,
den sie nicht kannte; doch ahnte sie wohl, daß dies eine Huldigung
sein sollte. Es war allerdings nicht, wie sie vermutete, ein
Blumenstrauß, sondern ein buntes Spielzeug, wie man sie kleinen
Kindern schenkte. Hier auf der Außenstation, um den Marsdurchmesser
vom Mittelpunkt des Planeten entfernt, herrschte nur der vierte
Teil der Marsschwere, also nur ein Zwölftel der Erdschwere. Der
Gegenstand, etwas höher als Ismas Kopf geworfen, schwebte daher so
langsam herab, daß sie ihn bequem mit der Hand ergreifen konnte.
Sie tat es und verneigte sich in ihrer natürlichen Anmut gegen die
Anwesenden, für welche die Fremdartigkeit ihres Grußes einen
besondern Reiz hatte.

„Sila Ba!“ – „Es lebe die Erde!“ rief der Jüngling, und die
Versammlung stimmte in den Ruf ein. „Sila Ill, Sila Ell, Sila Ba!“

In der Tür des Wagens wandte sich Isma nochmals zurück. Sie faßte
Mut und rief: „Sila Nu!“ Sie erschrak über ihre eigene Stimme. Denn
selbst die Hochrufe der Martier klangen tief und halblaut, sie aber
hatte ihre helle Menschenstimme nicht gedämpft, und so hob sich ihr
Gruß deutlich in dem allgemeinen Geräusch ab. Die Martier waren
entzückt.

-

Der Verkehr auf weite Strecken und mit großer Geschwindigkeit wurde
auf dem Mars durch zwei Arten von Bahnen vermittelt, Gleitbahnen
und Radbahnen. Die Kraftquelle war die Sonnenstrahlung selbst; sie
wurde auf den glühenden, trockenen Hochplateaus in ausgedehnten
Strahlungsflächen gesammelt und den Motoren in Form von
Elektrizität zugeleitet. Bei den Gleitbahnen befand sich zwischen
der Schienenbahn und der Last, die auf Schlittenkufen mit
eingelassenen Kugeln ruhte, eine dünne Wasserschicht, wodurch die
Reibung so vermindert wurde, daß man riesige Massen mit großer
Geschwindigkeit transportieren konnte. Noch viel rascher indessen
fand der Personenverkehr auf den Radbahnen statt. Die zwischen drei
Schienen laufenden Einzelwagen legten in der Stunde 400 Kilometer
zurück. Der Verkehr durch Luftschiffe hatte sich bis jetzt nicht
als vorteilhaft bewährt, doch beabsichtigte man nunmehr nach den
neuen Entdeckungen, zu denen die Fahrten nach der Erde geführt
hatten, den Bau neuer Luftschiffe mit Repulsitmotoren in Angriff zu
nehmen. Ill hatte beim Empfang erfahren, daß er die Reise sogleich
fortsetzen solle. Er bestieg daher mit seinen Gästen den von der
Regierung gestellten Zug, um ohne Aufenthalt nach Kla zu gelangen.
Trotzdem war hierzu eine zwölfstündige Fahrt erforderlich.

Jene Bahnen wurden aber nur dann benutzt, wenn es sich darum
handelte, große Strecken in kürzester Zeit zurückzulegen. Das
Hauptverkehrsmittel war stets der Radschlitten, ein leichter, teils
auf Kufen, teils auf Rädern ruhender Wagen für ein oder zwei
Personen, den ein unter dem Sitz befindlicher kleiner Motor
bewegte. Ferner kamen dazu die Stufenbahnen, die in regelmäßigen
Abständen von etwa zehn Kilometern alle bewohnten Gegenden mit
ihrem dichten Netz überspannten. Diese Stufenbahn war das Ideal
einer Straße, in ihr war jene Phantasie des Märchendichters
realisiert, daß statt des Reisenden die Wege selbst sich bewegten.
Die Breite der eigentlichen Fahrstraße betrug etwa 30 Meter, und
ebenso breit waren die parallellaufenden Zugangsstraßen. Diese
bestanden aus zwanzig eng nebeneinander befindlichen Streifen von
anderthalb Meter Breite, von denen der äußere sich mit einer
Geschwindigkeit von drei Metern in der Sekunde fortschob. Jeder
folgende, nach innen zu, hatte eine um drei Meter größere
Geschwindigkeit, so daß die Bahn in der Mitte, die eigentliche
Fahrstraße, sich mit einer Geschwindigkeit von 60 Metern in der
Sekunde bewegte. Jeder Punkt derselben legte also in der Stunde
über 200 Kilometer zurück. Die Streifen selbst erhielten ihre
Bewegung durch Walzen, über welche sie in der Art von
Transmissionsriemen gezogen waren. Man konnte die Stufenbahn sowohl
zu Fuß als auf dem eigenen Radschlitten benutzen. An jeder Stelle
konnte sie betreten und verlassen werden. Die Geschwindigkeit des
ersten Streifens von drei Metern konnte man auf dem Mars, wo wegen
der geringeren Schwere das Springen eine jedermann geläufige Sache
war, leicht erreichen, noch bequemer mit Hilfe des Radschlittens.
Man sprang oder fuhr also einfach auf diesen Streifen, und da jeder
folgende Streifen zum vorhergehenden dieselbe relative
Geschwindigkeit besaß, so gewann man, von Streifen zu Streifen
schräg vorwärts gehend oder fahrend, die Geschwindigkeit der
Hauptstraße. Diese benutzte man, ebenfalls fahrend oder gehend,
soweit man wollte, um alsdann in derselben Weise sie wieder zu
verlassen. Die linke Seite war zum Aufstieg, die rechte zum Abstieg
bestimmt. Über die Stufenbahn führten alle hundert Meter leichte
Brücken.

Über den Bahnen erhoben sich, die ganze Breite in kühnen Bogen
überspannend, die Riesengebäude des gewerblichen und
Geschäftsverkehrs. Diese stiegen bis zur Höhe von hundert Meter an.
Das leichte, feste Baumaterial gestattete bei der geringen
Marsschwere diese gewaltigen Wölbungen und Säulenmassen. Gleich
Palästen und Domen, in zierlichen Formen und lichten Farben,
stiegen die Gebäude wie spielend in die klare Luft, überall auf
ihren Dächern die Sonnenstrahlen sammelnd, um ihre Kraft zu
verwerten. So zogen diese Hallen ohne Unterbrechung durch das Land,
es in große Abschnitte von durchschnittlich hundert
Quadratkilometer Fläche zerlegend. Eigentliche Städte oder Dörfer
gab es hier nicht, die Orte gingen ineinander über, und nur als
Verwaltungsbezirke schieden sich die Gebäude in zusammengehörige
Gruppen. Diese Bauten überbrückten auch die Kanäle und die Bahnen,
die sich meist in derselben Richtung mit ihnen hinzogen.

Entfernte man sich aber von diesen Industriestraßen nur um einige
hundert Schritte, so befand man sich in einer vollständig anderen
Gegend. Gewaltige Riesenbäume, deren Gipfel zum Teil sogar die
hundert Meter hohen Gebäude noch überragten, verdeckten mit ihren
Zweigen die Nähe der Bauwerke. Es waren teils den Platanen, teils
den Fichten gleichende Pflanzen, mit denen sich kein irdischer
Baum, selbst nicht die berühmten Riesen des Yosemite-Tales,
vergleichen konnte. Erst in einer Höhe von etwa vierzig Metern
begann der Astansatz, und von hier aus bildete das Laubdach eine
natürliche Wölbung, auf den geradlinig aufsteigenden Pfeilern der
Stämme ruhend. Kein direkter Sonnenstrahl vermochte den Boden zu
treffen, aber ein mildes, bläulich-grünes Licht schimmerte von den
Blättern hernieder und verteilte sich gleichmäßig im Raum. Diese
lebendigen Kuppeln ersetzten den Martiern den Schutz einer
dichteren Atmosphäre, sie milderten den Gegensatz der Einstrahlung
am Tag und der Ausstrahlung in der Nacht und schützten den Boden
vor Verdunstung. Der gesamte Raum der von den Industriestraßen
begrenzten Bezirke war eine solche entzückende Waldlandschaft, die
übrigens nach der Mitte der Bezirke zu auch zuweilen von Lichtungen
unterbrochen wurde und eine reiche Abwechslung des Pflanzenwuchses
darbot.

Auf beiden Seiten der Industriestraßen, in einem Streifen von etwa
tausend Metern Breite, erstreckten sich die Privatwohnungen der
Martier. Unter dem Riesendach der Bäume dehnte sich ein reizendes
Gewirr von Garten- und Parkanlagen aus, Blumenbeete und kleine
Teiche wechselten mit Gebüsch und Baumgruppen, deren Höhe das auf
der Erde gewohnte Maß nicht überstieg. Mitten in diesen Gärten, die
bald aufs anmutigste gepflegt, bald als einfache Rasenplätze sich
darstellten, standen die Häuser der Martier, kleine einstöckige
Gebäude, manchmal zu Gruppen zusammengeschlossen, im allgemeinen
aber villenartig durchs Gelände zerstreut. Sie reihten sich, vom
Blaugrün der Sträucher und bunten Blumenbosketts umgeben,
unregelmäßig zu beiden Seiten der Wege, auf deren festem
Moosteppich sich, für das Auge wenig bemerkbar, die Geleise der
Gleitbahn hinzogen. Sämtliche Martier in den kulturell entwickelten
Teilen des Planeten hielten sich in solchen ländlichen Wohnsitzen
auf, sofern sie nicht gerade geschäftlich oder dienstlich in den
Industrieräumen zu tun hatten. Es kamen hier auf einen
Quadratkilometer ungefähr tausend Einwohner, so daß ein solches
Straßenviertel von zehn Kilometern Länge und Breite in dem
Streifen, der es umfaßte, gegen vierzigtausend Einwohner zählte.
Hatte man diese Zone von Wohnstätten durchschritten und drang man
auf einer der schmalen, sauber angelegten Straßen weiter in das
Innere des Bezirks vor, so nahm die Landschaft wieder einen neuen
Charakter an. Die Gärten hörten auf, an ihre Stelle trat die
Wildnis des Waldes. Tiefe Stille herrschte ringsum, nur
unterbrochen durch das leichte Summen kleiner Vogelarten oder das
Zwitschern der singenden Blüten, die sich auf ihren schwanken
Stengeln wiegten. Zahlreiche Wasseradern verzweigten sich unter den
breiten Blättern einer Sumpfpflanze und sammelten sich zu einem
stillen See, dessen dunkle Fläche seine Ufer widerspiegelte. Und
alles dies war überragt und geschützt von dem sanft leuchtenden
Blätterdach der Riesenbäume, das sich wie ein grünes Himmelszelt
über die niedere Waldlandschaft hindehnte. Man war entrückt in die
Einsamkeit ungestörter Natur, und nichts verriet, daß man auf dem
eilenden Radschlitten in wenigen Minuten auf die Weltstraße
gelangen konnte, wo Millionen geschäftiger Bewohner, die Kräfte der
Sonne und des Planeten ausnutzend, arbeiteten. Es war ein Gesetz,
daß in jedem Bezirk drei Fünftel des Flächenraums im Innern als
Naturpark von jeder Ausbeutung und Bewohnung geschützt blieb, was
jedoch eine geregelte Forstkultur darin nicht ausschloß. Je nach
der areographischen Breite wechselte natürlich die Art der
vorherrschenden Pflanzen. Ihr Wuchs wurde üppiger in der Nähe des
Äquators, spärlicher nach den Polen zu. Doch gab es in den
Niederungen nirgends eine eigentliche Waldgrenze, da nach den Polen
hin die Feuchtigkeit das Klima milderte.

Einen starken Gegensatz zu dem reichen Kulturleben und der
Lebensfülle der Niederungen boten die felsigen Hochplateaus, auf
denen es an einigen Stellen sogar beträchtliche Gebirge gab. Im
allgemeinen erhoben sie sich jedoch nicht bedeutend über die
Tiefebenen. Auch durch jene Wüsten zogen sich, uralten Kulturwegen
folgend, die Industriestraßen hin, nur daß sie hier nicht ein
dichtes Netz bildeten, sondern parallel verliefen und dadurch
Streifen von dreißig bis dreihundert Kilometern Breite darstellten,
die mit Bewohnern besetzt waren. Denn jeder solcher Streifen war
von einem Kanal begleitet, der das Wasser von den Polen über den
ganzen Planeten verbreitete. Nicht immer reichte die Wassermenge
aus, alle diese Kanäle zu füllen, so daß die Breite des
Vegetationsstreifens je nach der Stärke der Bewässerung wechselte.
Es schien dann, von der Erde aus gesehen, als ob die dunklen
Streifen, welche die Wüstengebiete auf die Länge von Tausenden von
Kilometern durchsetzen, sich seitlich verschoben, verengten,
verbreiterten oder auch verdoppelten. Sobald der Wasserzufluß hier
aufhörte, verloren die schützenden Bäume ihr Laub und der Boden
verdorrte, wenige Tage aber genügten auch wieder, dem Pflanzenwuchs
seine Frische zurückzugeben.

Die Bevölkerung dieser Weltstraßen stand unter ungünstigeren
Lebensbedingungen als die der immer feuchten Niederungen, aber sie
war doch ungleich besser gestellt als die Bewohner der Wüsten. Hier
hausten in der Kultur zurückgebliebene Gruppen der Bevölkerung des
Planeten, die zum Teil sogar noch Ackerbau trieben, wo geringe
Einsenkungen infolge der nächtlichen Niederschläge den Anbau von
Früchten gestatteten, zum größeren Teil aber im Bergbau und in den
Strahlungs-Sammelstätten tätig waren. Denn jene Wüstengegenden,
einst leer und unbewohnt, waren in der gegenwärtigen Kulturperiode
des Planeten das Hauptreservoir und die Hauptquelle für die Energie
geworden. Aus den Kalkfelsen, dem ausgetrockneten Ton- und
Lehmboden und den darunter befindlichen, von Erzgängen reich
durchsetzten Schichten zog die Bevölkerung des ganzen Planeten ihre
Nahrung und ihre Macht. Aber die klimatischen Verhältnisse
gestatteten nicht, die Verarbeitung an Ort und Stelle vorzunehmen.
Die Gesteinsmassen wurden an den Rändern der Verkehrsstreifen
gebrochen, wodurch diese sich allmählich verbreiterten. Die
Sonnenstrahlung wurde auf der ganzen Hochfläche gesammelt und in
der Form von Elektrizität über den Planeten verteilt. Die Bergleute
an den Rändern der Kulturstreifen gelangten dabei zu Wohlstand,
vermischten sich stetig mit der Bevölkerung der Niederungen und
rekrutierten sich immer aufs neue aus dem Stand der Beds, den
Wüstenbewohnern, die für die Besorgung der Sammelwerke
unentbehrlich waren. Diese abgehärteten Wüstensöhne durchzogen im
Sonnenbrand die weiten Hochflächen, um im Dienst der großen
Strahlensammelkompagnien die Stromleitungen bei Sonnenaufgang in
Tätigkeit zu setzen und bei Sonnenuntergang wieder abzustellen. Sie
erhielten einen reichlichen Lohn, der ihnen wohl gestattet hätte,
nach einer Reihe von Jahren ihren beschwerlichen Beruf aufzugeben,
aber sie liebten ihre Hochflächen, wie ihre Väter sie geliebt
hatten, wo in der Nacht der Himmel mit Millionen Sternen leuchtete,
wo wallende Nebel der Morgensonne vorauszogen und dann das
Glutgestirn den Boden unter den Füßen brennen ließ. Sie liebten die
Wüste und schüttelten die Köpfe, sobald einer der Ihren in die
Schächte am Wüstenrand hinabstieg. Sie betrachteten die Bewohner
der Täler nur als die Lieferanten ihrer Bedürfnisse und fühlten
sich als die eigentlichen Spender der Kraft des Planeten; aber sie
wußten auch, daß sie trotz ihrer Sonne und Sterne verhungern
müßten, wenn nicht die klugen Männer der Tiefe ihnen Steine in Brot
verwandelten.

Steine in Brot! Eiweißstoffe und Kohlenhydrate aus Fels und Boden,
aus Luft und Wasser ohne Vermittlung der Pflanzenzelle! – Das war
die Kunst und Wissenschaft gewesen, wodurch die Martier sich von
dem niedrigen Kulturstandpunkt des Ackerbaues emanzipiert und sich
zu unmittelbaren Söhnen der Sonne gemacht hatten. Die Pflanze
diente dem ästhetischen Genuß und dem Schutz der Feuchtigkeit im
Erdreich, aber man war nicht auf ihre Erträge angewiesen. Zahllose
Kräfte wurden frei für geistige Arbeit und ethische Kultur, das
stolze Bewußtsein der Numenheit hob die Martier über die Natur und
machte sie zu Herren des Sonnensystems.

\section{28 - Sehenswürdigkeiten des Mars}

In einem der großen Bezirke, welche den Sitz der Zentralregierung
des Mars umschlossen und den Gesamtnamen Kla führten, lag die
Wohnung Ills nahe an der Grenze der Waldwildnis. Sie bestand aus
mehreren miteinander verbundenen Einzelhäuschen, so daß das Ganze
eine geräumige Villa darstellte. Die Anlagen, die sich um die
Gebäude erstreckten, zeugten von sorgfältiger Pflege und feinem
Geschmack. Am Eingang des Gartens saßen rechts und links in
anmutiger Haltung zwei Frauengestalten, die sich im Scherz eine
Blumengirlande zu entreißen suchten; sie zogen quer über den Weg an
den entgegengesetzten Enden derselben und versperrten dadurch den
Zutritt.

Auf der schmalen, glatten Straße, die zwischen den Nachbargärten
von dem Hauptweg abzweigend auf diesen Eingang hinführte, näherte
sich rasch ein leichter, zweisitziger Radschlitten. Ein jüngerer
Mann in der anliegenden Sommerkleidung der Martier lenkte
denselben; der Sitz neben ihm war leer. Wer Ell mit dem grauen Haar
und der Falte zwischen den Augen nachdenklich von seiner Sternwarte
in Friedau hatte herabsteigen sehen, hätte ihn in diesem Martier
nicht wiedererkannt. Ell fühlte sich in der Tat wie verjüngt,
gleich als ob seine Erdenjahre ihm nach der Rechnung des Mars, zwei
auf ein Marsjahr, angerechnet werden sollten. Ein unaussprechliches
Glücksgefühl durchzog seine Seele; das Bewußtsein, dem Planeten
zurückgegeben zu sein, den er für seine Heimat hielt, mitzuleben
unter den Numen und ihren Götterwandel zu teilen, erhob ihn
zunächst über alle die Sorgen, die bei dem Gedanken an das Geschick
der Erde und seiner irdischen Freunde sich ihm aufdrängten. Es war
ihm, als müßten alle diese Schwierigkeiten unter den Händen der
Nume von selbst sich lösen, und er genoß in vollen Zügen die
Seligkeit, all das Große und Herrliche zu sehen, von dem sein Vater
mit dem Schmerz des Verbannten in unstillbarer Sehnsucht geredet
hatte.

Der Radschlitten glitt auf den Eingang des Gartens zu, und Ell ließ
den Strahlenkegel einer kleinen, an der Lenkstange des
Radschlittens befestigten Lampe einen Moment auf die Augen der
rechts sitzenden Frau fallen. Sogleich richteten beide Figuren sich
in die Höhe und erhoben wie zum Gruß die Arme, indem sich dabei die
Girlande wie ein Triumphbogen emporschwang und den Eingang freigab.
Der Schlitten glitt hindurch und hielt gleich darauf vor der
Veranda des Hauses.

Die beiden anmutigen Pförtnerinnen waren Automaten. Die Bestrahlung
der Augen der rechtssitzenden löste eine chemische Reaktion aus und
öffnete dadurch die Pforte. Zugleich wurde damit der Eintritt eines
Ankommenden im Innern des Hauses signalisiert.

Ell sprang aus dem Schlitten und eilte die Stufen der Veranda
empor.

Eine schlanke Frauengestalt trat ihm aus dem Haus entgegen.

Ell blieb erstaunt stehen. Er erkannte nicht sogleich, wen er vor
sich hatte. Er hatte Isma noch nicht im Kostüm der Martierinnen
gesehen.

„Isma!“ rief er jetzt, mit bewundernden Blicken sie anstarrend. Er
wollte nach Martiersitte die Hände auf ihre Schultern legen, aber
sie ergriff sie nach alter Gewohnheit mit den ihrigen und drückte
sie freundschaftlich.

„Ich kann nicht dafür“, sagte sie, verlegen errötend, „Frau Ma hat
es nicht anders gewollt.“

„Sie konnte es nicht besser treffen“, sagte Ell heiter, „ich
wünschte, ich könnte so mit Ihnen durch die Straßen von Friedau
gehen. Passen Sie auf, das kommt auch noch.“

Isma schüttelte leise den Kopf. „Lassen Sie uns jetzt nicht an die
Erde denken. Wenn ich allein bin, kommen meine Gedanken nicht fort
davon, immer sehe ich den Zettel auf dem Tisch meines Zimmers, als
ich die Lampe abdrehte, und dann die Gletscher zwischen den Felsen,
wo – –. Nein, Ell, bis wir nicht handeln können und nichts Neues
erfahren, lassen Sie mich in Ihrer Gegenwart versuchen, mit Ihnen
auf dem Mars zu leben. Versuchen – wie ich dies Kleid versuche.“

„Verzeihen Sie mir“, sagte Ell, „ich bin so überrascht von allem
Neuen, daß ich nicht sogleich den richtigen Ton traf. Aber ich
werde es. Und jetzt wollen Sie mir die Freude machen, mich zu
begleiten?“

Sie blickte wieder lächelnd an sich herab und zupfte an den dichten
Falten des Schleiergewandes.

„Ich will nur fragen, was noch zur Straßentoilette gehört“, sagte
sie. „Nehmen Sie Platz.“ Sie schlüpfte in das Zimmer.

Nach wenigen Minuten kehrte sie zurück. Sie trug jetzt den leichten
Kopfputz der Martierinnen, wie er im Sommer üblich war, der nur den
Vorderkopf bedeckte. Ein Kranz sehr feiner und zarter Federn
schützte die Stirn und die Augen, indem er als ein
halbkreisförmiger Schirm vortrat. Die Farbe war genau das tiefe
Blau ihrer Augen, und von derselben Farbe war das den schlanken
Formen sich anschließende weiche Panzerkleid, das, stärker als
Seide, metallisch, wie die Flügeldecken mancher Käfer schimmerte.
Der Schleier, auf beiden Schultern befestigt, wurde von einem
Gürtel zusammengehalten, dessen Grund unsichtbar war, er erschien
nur wie ein Kranz ineinander verschlungener Zweige. Vom Gürtel ab
floß der Schleier, dessen Farbe genau dem Lichtbraun des Haares
angepaßt war, in dichten Falten um die ganze Gestalt bis zu den
Knöcheln, wurde aber von scheinbar vom Gürtel herabhängenden
Blütengewinden durchsetzt. Dunkelblaue Schuhe vollendeten den
Anzug. Es war, als hätte sich der schimmernde Lichtglanz der Augen
und das zarte Gewölk des Haares um den ganzen Körper verbreitet.

Hinter Isma erschien eine ältere, würdige Dame, Frau Ma, die Gattin
Ills.

„Guten Morgen“, rief Ell, ihr freudig entgegentretend. „Darf ich
dir deinen Gast entführen?“

Ma warf mit jugendlicher Frische den Kopf zurück und blinzelte Ell
mit ihren gutmütigen Augen vergnügt an, ihn von oben bis unten
musternd.

„Ganz wie eingeboren!“ sagte sie lachend. „Eigentlich hatte ich
mich auf einen Menschenneffen gefreut, der in Felle gekleidet
umherläuft. So macht man’s wohl? Nicht?“

Dabei streckte sie Ell ihre linke Hand entgegen.

„Die rechte, Tante!“ sagte Ell.

„Na also dann wohl die?“

Ell ergriff die Hand und zog sie an seine Lippen.

„So also wird das gemacht?“

„Herren gegen Damen, wenn sie besonders aufmerksam sein wollen.
Einer Tante darf man sogar um den Hals fallen.“

„Na, ein andermal. Aber nun sag einmal, Neffe, wie gefällt dir das
Kind?“ Dabei faßte sie Isma am Arm und drehte sie ohne weiteres um
sich selbst. „Mir gefällt bloß nicht“, fuhr sie sogleich fort, „daß
sie so traurige Augen macht. Das ist nichts, auf dem Nu muß man
lustig sein. Nun nimm sie einmal mit und zeig ihr die Welt. Du
sollst mir sie ein bißchen munter machen.“

Sie ließ Isma gar nicht zu Wort kommen, sondern schob die beiden,
sie freundlich auf die Schulter klopfend, nach der Treppe. Schon
hatte Isma den Wagen bestiegen, und Ell wollte ihn eben in Bewegung
setzen, als Ma rief:

„Halt, halt! Isma, Frauchen, Sie haben ja Tuch und Schirm
vergessen. Bleiben Sie nur sitzen. Ich hab’s schon drin
zurechtgelegt.“

Im Augenblick erschien sie wieder und warf ein kleines Rohr hinab.
Ell fing es auf.

Isma dankte.

„Wenn Sie auf der einen Seite ziehen, ist’s ein Schirm, und auf der
andern bekommen Sie ein Umschlagetuch. Da, an den Gürtel hängt
man’s – – zeig’s ihr doch, Ell! Fahrt wohl, ihr Kinder.“

Isma betrachtete das zierliche Röhrchen. „Ich denke“, sagte sie,
„hier regnet es nur in der Nacht. Wozu braucht man da einen
Schirm?“

„Es ist auch eigentlich ein Sonnenschirm.“

„Aber hier ist überall der wunderbare Baumschatten, und die Straßen
draußen sind alle überwölbt.“

„Es gibt auch Lichtungen und Übergänge, wo der Schirm unentbehrlich
ist; denn wo die Sonne scheint, brennt sie gewaltig. Obwohl wir
soviel weiter von ihr entfernt sind als auf der Erde, schützt uns
doch nicht die dichte Erdenluft; es ist, als ob wir auf dem
Gaurisankar ständen.“

„Aber diese herrliche Vegetation.“

„Den Verhältnissen angepaßt, und die sind doch wieder ganz andere
als auf einem Gebirge. Hier in den Niederungen halten wir alle
Wärme fest und geben keine wieder heraus. Dafür sorgen die großen
pelzverbrämten Blätter unsrer Riesenbäume. Aber Sie sind an das
Klima nicht gewöhnt, es ist vielleicht besser, wenn Sie während des
Fahrens sich in das Tuch hüllen. Erlauben Sie.“

Ell nahm Isma das Schirmröhrchen aus der Hand und zog an dem Ring,
welcher das eine Ende abschloß. Eine kleine Rolle, nicht größer als
ein Zeigefinger, schob sich heraus, scheinbar schwarz; aber unter
Ismas Händen entfaltete sich das Röllchen zu einer großen Decke, in
die man den ganzen Körper einhüllen konnte. Das Gewebe war ganz
weich, locker und vollständig unsichtbar, die eingewebten dunklen
Fäden dienten nur dazu, überhaupt erkennen zu lassen, wo das Tuch
sich befand und wie weit es reichte. Isma hüllte sich behaglich
hinein, und man bemerkte nicht, daß sie überhaupt ein Tuch
umgeschlagen hatte; ihre Toilette blieb vollständig sichtbar.

„Das ist ja wie das Zelt der Fee Paribanu“, sagte sie lächelnd.
„Aber wie bekommt man denn das Tuch wieder in das Futteral?“

„Man knüllt es einfach in der Hand zusammen und stopft es hinein.
Diesen Lisfäden ist es ganz gleichgültig, wie sie zu liegen kommen,
man kann sie zusammenpressen wie Luft.“

„Jetzt ist es erst behaglich“, sagte Isma. „Und wie still und
schön. Das ist ja wie in unserem Wald, nur Felsen scheint es nicht
zu geben. Aber so viel Wasser! Und ich denke, der Mars ist so
wasserarm?“

„Das ist auch richtig, wir haben kein Meer, wenigstens kein
nennenswertes. Unser ganzer Reichtum ist auf dem Land verteilt, da
nutzen wir ihn aus.“

Es war am frühen Vormittag. Die Wege hier im Waldesdickicht waren
einsam, nur hin und wieder begegnete man einem Gleitwagen oder
einem Spaziergänger. Ell hatte sein Gefährt langsam durch den
Naturpark gelenkt, es näherte sich jetzt der gegenüberliegenden
Grenze des Bezirks, die Wege wurden belebter, und die ersten Häuser
der Wohnungszone erschienen. Ein starkes Geräusch wie das einer
Säge unterbrach die Ruhe der Umgebung. Bei einer Wegbiegung wurde
die Ursache sichtbar. Es war in der Tat eine große Säge, die, von
einem elektrischen Motor getrieben, den sieben Meter im Durchmesser
haltenden Stamm eines der Waldriesen bereits bis auf einen kleinen
Rest durchnagt hatte. Das Alter – er zählte über sechstausend Jahre
– hatte ihn vollständig gehöhlt und der Zusammensturz war zu
befürchten; man mußte ihn beseitigen.

„Mitten zwischen diesen anderen Bäumen“, rief Isma, „wie ist das
möglich? Er muß ja in seinem Fall ringsum alles zermalmen.“

Auch Ell wußte keine Auskunft zu geben. „Vielleicht sehen wir bald,
was geschieht, wenn wir ein wenig warten, die Säge ist ja schon
fast hindurch. Man muß doch keine Gefahr befürchten, denn nur ein
kleiner Kreis ringsum ist abgesperrt.“

Nach wenigen Minuten war die Säge aus der Rinde vollends
herausgedrungen. Die Maschine schob sich beiseite, und die Arbeiter
zogen sich außerhalb des abgesperrten Kreises zurück. Der
Arbeitsleiter sprach in ein Telephon, dessen Drähte sich nach oben
zwischen den Ästen der Bäume verloren. Gleich darauf vernahm man
ein gewaltiges Rauschen zwischen den Blättern, einzelne Zweige
wurden geknickt, und Blätter fielen herab. Der Riesenbaum schwankte
ein wenig und hob sich langsam in die Höhe. Wie er gestanden,
senkrecht, schwebte er aufwärts zwischen seinen gesunden Nachbarn,
von denen nur einzelne Äste und Zweige mitgerissen wurden, die sich
zu eng mit denen des gefällten Baumes verbunden hatten. Ein
Streifen Sonnenlicht durchbrach das blaugrüne Laubdach.

„Ich sehe es jetzt“, rief Ell. „Sie heben den Baum mittels
Luftballons in die Höhe. So wird er sogleich bis zur Fabrik
transportiert werden, wo man das Holz verarbeitet. Und raten Sie,
was in dem hohlen Baum steckt!“

„Nichts, vermutlich.“

„Hier, Ihr Tuch. Vielleicht hunderttausend solcher Tücher. Sehen
Sie, da –“

Eine Anzahl Neugieriger, besonders aber Kinder, hatten sich um den
abgesperrten Kreis gesammelt. Als die Schranken fielen, stürzten
sie mit Jubel auf den Stumpf des Baumes zu und kletterten auf den
Rand. Gleich darauf sah man sie, die Hände fest zusammengedrückt,
davonlaufen.

„Was haben sie da?“ fragte Isma.

„Das Gewebe der Lisspinne, es füllt die Höhlung des Baumes zum
großen Teil aus, und was unten im Stumpf bleibt, gehört dem, der es
nimmt.“

Ein kleiner Junge rannte auf Ells Wagen zu, den er im Eifer so spät
bemerkte, daß er beim Ausweichen hinstürzte. Gleich war er wieder
auf den Beinen, aber jetzt suchte er nach seiner Handvoll Lis, die
ihm entfallen und nun kaum zu sehen war. Isma, die den Wagen
verlassen hatte, sah das Gewebe zufällig am Boden glitzern und hob
es auf. Sie betrachtete es neugierig. Der Knabe bemerkte es. Es war
ein kleines, dickes, pausbäckiges Kerlchen, sehr ärmlich gekleidet.
Er starrte Isma an. Sie hielt ihm das wirre, weiche Fadenknäuel
hin. Seine Augen leuchteten groß auf, als er es wieder erhielt,
aber er blieb wie angenagelt mit gespreizten Beinchen vor Isma
stehen. Seine Blicke gingen jetzt zwischen Isma und seinen Händen
hin und her. Er kämpfte offenbar einen großen Kampf. Dann hielt er
das Päckchen Isma wieder hin und sagte, als wenn er ein Königreich
vergäbe:

„Ich schenke es dir.“

„Warum?“ fragte Isma lächelnd.

„Weil du kleine Augen hast.“

Isma wußte nicht, ob sie recht verstanden habe, und sah Ell
zweifelnd an.

„Weil ich kleine Augen habe?“ wiederholte sie fragend.

„Kleine Augen sind traurig, man schenkt ihnen“, sagte der Knirps.

„Ich will dir –“ Isma unterbrach sich. „Ich will dir auch etwas
schenken, weil du große Augen hast“, wollte sie sagen. Aber es fiel
ihr ein, daß sie nichts zu verschenken habe. Der kleine Nume auf
seinen Wackelbeinchen – was konnte sie ihm als Gegengabe bieten?

Ell verstand sie. Er griff in die Wagentasche, in der sich einige
kleine Erfrischungen befanden, und gab Isma ein Stückchen
Naschwerk.

„Das ist etwas für ihn“, sagte er.

Der Junge lachte über das ganze Gesicht, als ihm Isma den Kuchen
reichte. Diese Sprache verstehen die Kinder aller Planeten. Aber er
biß nicht sogleich hinein.

„Gib ihr auch“, sagte er zu Ell. „Du hast große Augen. Große Augen
dürfen nicht essen, wenn kleine hungern.“

Er beruhigte sich nicht eher, bis Isma einen Kuchen in der Hand
hielt. Dann rannte er spornstreichs davon.

Isma stieg ein. Der Wagen rollte weiter.

„Was meinte er mit den kleinen Augen?“ fragte Isma.

„Das ist eine sprichwörtliche Redensart. ›Kleine Augen‹ nennt man
unglückliche, kranke, armselige Leute. Der Junge hat die Sache
wörtlich genommen.“

Man durchfuhr die Zone der Wohnhäuser, die Bäume hörten auf, der
Wagen glitt unter die Säulenhallen der Industriestraße. Ell
beschleunigte sein Tempo, er fuhr auf den Außenstreifen der
Stufenbahn und war schnell auf der breiten Mittelstraße. In einem
Gewühl von Fahrzeugen legte er hier seinen Weg zurück.

Aus der Ruhe des ländlichen Hauses, in der Isma sich zunächst
einige Tage bei der liebenswürdigen Pflege ihrer Wirte hatte
erholen sollen, und jetzt aus der Einsamkeit des Waldfriedens fand
sich Isma plötzlich in das Gedränge des Weltverkehrs, der Weltstadt
im wörtlichen Sinn, versetzt. Denn diese Palastreihen bildeten in
der Tat den Zusammenhang einer Riesenstadt, die sich über den
größten Teil des Planeten verbreitete, nur mit der glücklichen
Anordnung, daß sie meilenweite Wälder und auch Hunderte von Meilen
ausgedehnte Wüsten zwischen ihren Mauern umschloß. Wenn Isma den
Blick auf die Wagen und Fußgänger richtete, die sich in
ununterbrochener Kolonne in derselben Richtung mit ihr bewegten
oder auf der andern Seite der Straße ihr in rascher Gangart
entgegenkamen, so glaubte sie in einer ungeheuern Völkerwanderung
zu stecken. Dabei war das Geräusch keineswegs betäubend, denn auf
diesem Planeten wickelte sich alles verhältnismäßig leise ab. Auch
die relative Geschwindigkeit der Wagen und Fußgänger gegeneinander
war nicht groß. Nur wenn sie nach den kühn aufstrebenden Säulen
blickte, welche die mächtigen Wölbungen trugen, nach den Treppen
und Aufzügen, die an den Seiten in die oberen Stockwerke führten,
nach den Plakaten und Anschlägen, die sie von hier aus nicht zu
entziffern vermochte, erkannte sie, daß der Weg selbst, auf dem ihr
Radschlitten hinglitt, mit der dreifachen Geschwindigkeit eines
irdischen Schnellzugs sie fortriß.

Mit Erstaunen blickte sie auf ihren Nachbar zur Rechten, der den
Wagen mit einer Sicherheit zwischen den übrigen hinlenkte, als wäre
er seit Jahren an diese Beschäftigung gewöhnt. Allerdings hatte Ell
bereits die wenigen Tage seines Aufenthalts benutzt, um sich
gründlich in der Umgebung umzusehen. Er wohnte nicht weit von der
Illschen Villa in einem eigenen Häuschen, hatte sich aber immer nur
des Abends auf eine Stunde bei seinen Verwandten sehen lassen. Isma
empfand diese Zurückhaltung nicht gerade als Zurücksetzung. Hatten
sich doch beide auch in Friedau stets nur kurze Zeit gesprochen,
und mußte sie sich doch sagen, daß ihn die neue Umgebung voll in
Anspruch nahm. Aber nach dem gemeinsamen Erlebnis der Reise und
hier, in der völligen Fremde, vermißte sie die Nähe des Freundes
stündlich, des einzigen, der sie ganz zu verstehen vermochte.
Gestern abend war dann der heutige Ausflug verabredet worden.

Beide hatten, seitdem sie die Stufenbahn benutzten, kaum
miteinander gesprochen. Ell mußte seine Aufmerksamkeit ganz auf den
Weg richten, und Isma musterte neugierig und überrascht die
Gesichter und Trachten rings um sie her. Offenbar strömten hier
alle Klassen der Bevölkerung durcheinander, das ärmlichste Kleid
erschien neben der elegantesten Toilette, der einfache Arbeitsanzug
herrschte vor. Sie bemerkte bald, daß ihre von Ma ausgewählte
Toilette sich sehen lassen durfte und sie sowohl wie ihr Gefährte
nur durch ihre Züge und ihre bleichere Gesichtsfarbe auffielen.

Nun wendete sich Ell wieder zu ihr. „Wir sind am Ziel“, sagte er.
„jene helle Zahl dort – 608 – zeigt es an; bei 609 müssen wir die
Bahn verlassen.“

Er lenkte das Gefährt nach rechts. Die Bewegung verminderte sich
merklich, Isma mußte sich fest im Wagen zurücklehnen. Jetzt glitt
der Wagen auf die ruhende Straße. Nach wenigen Augenblicken hielt
er unter einem Riesenportal hinter einer langen Reihe ähnlicher
Fahrzeuge.

Ell half Isma aus dem Wagen.

„War es Ihnen unangenehm?“ fragte er, ihre Hand festhaltend. Sie
erwiderte den leisen Druck seiner Finger. Sie freute sich, in
seinen Augen wieder die gewohnte Sorge um sie zu lesen, die sie
daheim so oft im stillen beglückt hatte.

„Zuletzt begann ich etwas schwindlig zu werden“, antwortete sie.
„Ich bin ganz froh, wieder einmal ein Stück zu Fuß gehen zu können.
Wo führen Sie mich denn hin?“

Er sah sie noch immer an. „Ich bin so glücklich, Sie hier zu
haben!“

Sie hob die Augen bittend zu ihm auf.

„Was wollen Sie sehen?“ fragte er in anderem Ton. „Wir sind hier am
Museum der Künste. Eine oder die andre Abteilung wollen wir
betrachten.“

„Was Sie wollen!“ sagte Isma heiter. „Wir ziehen nun einmal auf
Abenteuer aus.“

Ein Beamter befestigte eine Marke an Ells Wagen und reichte ihm die
Gegenmarke. Dann schritten sie beide der Tür eines Aufzugs zu und
ließen sich in das erste Stockwerk heben.

\section{29 - Das heimliche Frühstück}

Isma und Ell standen vor einem prachtvollen Portal, das die
Aufschrift trug: ›Museum der schönen Künste‹. Es führte auf eine
kreisförmige Galerie, die eine mächtige Rotunde umschloß. Der Blick
öffnete sich sowohl nach unten wie nach oben. Man glaubte unten in
das Gewühl des wirklichen Lebens zu blicken, in rascher
Veränderung, von den Seiten immer neu herandrängend, sah man
Gestalten in ihren gewohnten Beschäftigungen, in der Arbeit des
Tages, andere mit dem Ausdruck des Leidens und den Mängeln der
Wirklichkeit. Aber in der Mitte emporwallende Nebel umhüllten diese
Figuren und hoben sie langsam in die Höhe. Je höher sie
emporstiegen, um so mehr verschwand der Nebel und löste sich nach
oben in immer helleres Licht auf. Die Gestalten wechselten ihren
Ausdruck, ihre Blicke wurden frei, ihre Mienen verklärt, sie waren
zu Werken der Kunst, zu reinen Formen geworden. Sie schienen zu
ruhen, und doch stiegen immer neue Gestalten auf, ohne daß jene
Bilderwelt an der Kuppel der Wölbung zunahm oder sich überfüllte.
Es wär nicht möglich zu verfolgen, wie dieser Übergang in die Höhe
sich vollzog, ein lebendiges Abbild des Mysteriums in der Seele des
Künstlers.

„Eine symbolische Darstellung des künstlerischen Schaffens“, sagte
Ell.

„Aber wo kommen diese Gestalten her und wohin gehen sie?“

„Das Ganze beruht auf einer optischen Täuschung, und nach einigen
Stunden würde man bemerken, daß dieselben Gruppen wiederkehren.
Aber die Illusion ist vollständig. Nun suchen Sie sich eine dieser
Überschriften aus.“

Sie umschritten die Galerie. Die äußere Seite war ringsum von
schmalen Türen umgeben, deren Aufschriften die Abteilungen nannten,
zu denen man durch jene gelangte. Aber jede Hauptgruppe hatte
wieder eine große Zahl Unterabteilungen, die historisch geordnet
waren. Da zählte zum Beispiel bei der Malerei die ältere Malerei in
der archaistischen Periode, das heißt vor Erfindung der
selbstleuchtenden Farben, allein 30 Abteilungen, die jede mehrere
Jahrhunderte umfaßte; die agrarische Periode zählte aus der Zeit
der Handarbeit 315, aus der Zeit der Dampfkraft 56, der
Elektrizität 212, der Energiestrahlung 25 Abteilungen. Die neuere
Malerei begann erst seit der Erfindung der künstlichen Darstellung
der Nahrungsmittel. Zwischen beiden lag eine Periode des Verfalls,
die man den dreitausendjährigen sozialen Krieg nannte. Es war dies
eine jetzt etwa 18.000 Jahre zurückliegende Zeit, in welcher ein
allgemeiner Niedergang der Marskultur stattgefunden hatte. Sie war
nämlich ausgefüllt durch furchtbare Kämpfe zwischen der
ackerbautreibenden und der industriellen Bevölkerung. Durch die
Darstellung der Lebensmittel aus den Mineralien ohne Vermittlung
der Pflanzen glaubte sich die agrarische Bevölkerung in ihrer
Existenz bedroht, obwohl sie längst nicht mehr den Bedarf an
Lebensmitteln hatte decken können. Die Besitzer des Grund und
Bodens waren als Herren der Nahrungsmittel zu unumschränkter Macht
gelangt und wollten die Verbilligung der Volksernährung durch die
neuen gewaltigen Fortschritte der Wissenschaft und industriellen
Technik nicht dulden. Dieser Kampf füllte fast drei Jahrtausende in
wechselnden Formen aus und endete erst mit der Vernichtung der
Macht der Ackerbauer und der Begründung der vereinigten
Marsstaaten. Während dieser Zeit hatte die Kunst keinerlei
Förderung empfangen. Sie war erst wieder aufgeblüht, als statt der
nüchternen Getreidefelder die anmutigen Wälder entstanden waren und
der Erwerb von Grund und Boden für den einzelnen auf ein mäßiges
Maximum beschränkt war.

Isma ging ratlos an der Reihe der Überschriften entlang, die ihr
Ell zu entziffern behilflich war. Sie schüttelte mutlos den Kopf.

„Das ist mir zu viel und macht mich nur verwirrt. Suchen wir
zunächst etwas ganz Einfaches, das ich verstehen kann. Was ist denn
hier hinter der Malerei für eine Kunst?“

„Die Tastkunst.“

„Was ist das?“

„Ich muß gestehen, ich weiß es selbst nicht recht.“

„Lassen Sie uns sehen.“

Ell öffnete die Tür. Sie führte in einen kleinen, mit zwei
gepolsterten Bänken ausgestatteten Raum. Ell sah jetzt erst, daß
sich in demselben ein Anschlag befand: ›Abgang alle zehn Minuten‹;
eine Uhr zeigte, daß nur noch eine Minute zur Abgangszeit fehlte.
Es war also nicht ein Zimmer, sondern eine Art Omnibus, worin man
sich befand. Alle die Türen aus der Galerie führten in solche
Coupés, die zu bestimmten Zeiten die Insassen nach den betreffenden
Abteilungen des Museums beförderten. Denn die Anlagen waren zu
ausgedehnt, um sie zu Fuß zu erreichen und sich bis dahin
zurechtzufinden. Die Tür öffnete sich jetzt noch einmal, und zwei
Damen flogen förmlich in den Raum. Gleich darauf setzte sich der
Wagen in Bewegung.

Die ältere der beiden Damen schnappte nach Luft; sie war eine sehr
korpulente Erscheinung und nahm wenigstens zwei Plätze des Sofas
ein.

„Das war gerade die höchste Zeit!“ rief sie erhitzt und atemlos,
indem sie ein feines Tuch hervorzog und fortwährend zwischen ihren
kurzen, dicken Fingern rieb. „Diese Wagen gehen ja nur alle zehn
Minuten. Der Besuch ist so schwach! Ja, es ist nur eine Kunst für
Auserwählte. Sie schwärmen auch dafür?“ wandte sie sich zu Isma.
„Sie sind Spitzistin? Nicht wahr?“ sagte sie, indem sie einen Blick
auf Ismas schlanke und zarte Finger warf. „Ich bin natürlich
Rundistin, aber das tut nichts. Sie wollen gewiß auch das neue
Meisterwerk tasten? Blu hat sich wieder selbst übertroffen! Das ist
das hohe Lied des Widerstandes, die Sphärenmusik des Hautsinns!“
Und sie kniff die Augen schwärmerisch zusammen, daß sie zwischen
den Fettpolstern ihrer Augenlider verschwanden.

„Ich muß gestehen“, sagte Isma schüchtern, „ich bin noch ganz
unerfahren in der Tastkunst. Ich weiß gar nicht –“

„Was? Wie? Sie wissen nicht?“ Sie betrachtete Isma näher. „Sie sind
wohl aus dem Norden von den Streifen, wenn ich fragen darf? Sie
waren noch nie in Kla?“

„Nein, meine Heimat ist fern von hier.“

„Aber Blu sollten Sie doch kennen. Sie ist doch die größte –
neidlos gestehe ich es, obwohl ich selbst Künstlerin bin. Und von
allen Künsten ist wieder die Tastkunst die höchste. Auge, Ohr,
Geruch, selbst Geschmack – was will das alles sagen! Der Tastsinn
ist doch der intimste aller Sinne. Hier berühren wir die Dinge
unmittelbar, sie bleiben uns nicht in der Ferne. Und schmecken ist
ja eigentlich auch ein Tasten, nur ein unreines, gestört durch
Gerüche und durch Salziges, Saueres, Bitteres, Süßes – aber die
Fingerspitzen, die Handflächen, das sind die wahren Schlüssel zur
Schönheit. Und hier im Tasten enthüllt sich die Kunst in ihrer
höchsten Freiheit. Hier überwindet sie am reinsten die Macht des
Wirklichen, das vitale Interesse. Was wir sehen, was wir hören,
bleibt uns immer noch fern. Es ist keine Kunst, das ohne Verlangen
zu betrachten, was wir doch nicht erreichen können. Aber die
Gegenstände in den Händen halten und doch nichts von ihnen zu
wollen als das reine, freie Spiel des Wohlgefallens, das ist echte
Kunst. Spielt nicht ein jeder unwillkürlich mit dem, was er
zwischen den Fingern hält? Dies zur Kunst zu erheben, das ist das
wahrhaft Geniale! Das Rauhe, Glatte, Scharfe, Spitzige, Runde,
Nachgebende, Elastische, Harte, Kratzende, Kribblige – ohne
Gedanken, ohne Wünsche –, das ist das wahrhaft Ästhetische. Eine
Tastsymphonie von Blu ist für mich das Höchste. Kommen Sie nur mit,
ich werde sie Ihnen zeigen.“

Isma blickte zu Ell hinüber.

„Ich fürchte“, sagte er deutsch – es fiel auf dem Mars nicht auf,
wenn man in Sprachen redete, die andere nicht verstanden, da die
meisten Familien eigene Mundarten besaßen –, „ich fürchte, das wird
für uns nichts sein. Wir sind wohl zu wenig auf diesen Kunstgenuß
vorbereitet.“

Die Dicke begann eben einen neuen Redestrom, als der Wagen hielt.
Sie stürzte schleunigst hinaus. Ihre Begleiterin, die stumm
geblieben war, folgte ihr, und Isma und Ell taten das gleiche.

Man befand sich in einem großen Saal, in welchem man nichts
erblickte als zahllose Kästen verschiedener Größe. Aufschriften
gaben Verfasser und Inhalt des Tastkunstwerkes an, das sie
enthielten. Vor einigen saßen Besucher in stiller Andacht und
hielten die Arme bis zum Ellenbogen in zwei Öffnungen der Kästen
versenkt.

Die beleibte Dame suchte nach ihrem Katalog eine bestimmte Nummer.
Vor dem betreffenden Kasten angelangt, streifte sie die Ärmel auf
und steckte die Arme zunächst in ein Becken. Es war nicht mit
Wasser gefüllt, sondern ein Luftstrom führte ein fein verteiltes
ätherisches Öl gegen die Haut und bereitete durch diese Reinigung
auf den nachfolgenden Kunstgenuß vor. Alsdann brachte die
Kunstjüngerin durch einen Handgriff ein Uhrwerk in Gang, setzte
sich auf einen Stuhl vor dem Kasten, steckte ihre Arme in die
Öffnungen und versank in Schwärmerei. Isma und Ell hatten ihr auf
gut Glück an dem ersten besten Kasten, der unbesetzt war, alles
nachgeahmt. Aber nach wenigen Minuten zog Isma ihre Hände zurück.

„Wollen Sie noch bleiben?“ fragte sie Ell.

„Fällt mir nicht ein, wenn Sie nicht Lust haben. Ich wollte Sie nur
nicht stören.“

„Ich verzichte auf den Genuß. Ich kann nichts spüren als ein
abwechselndes Drücken, Ziehen, Prickeln, Reiben – für mich ist das
nur eine Art Massage.“

„Mir ging es auch so. Es ist eine Kunst für Blinde. Wir müssen
nicht tastkünstlerisch veranlagt sein. Wir wollen lieber nur einen
kurzen Gang durch einen der andern Säle machen, und dann will ich
Sie in das technische Museum führen.“

Ohne von der Tast-Enthusiastin bemerkt zu werden, gingen die
untastlichen Erdgeborenen nach dem Coupé zurück, das sie bald
wieder in der Rotunde absetzte. Ein anderer Wagen, dicht von
Besuchern erfüllt, trug sie in eine der Abteilungen für Skulptur.
Hier fand sich Isma leichter zurecht. Es war eine Kunst für
menschliche Sinne, eine Fülle großer Gedanken in wunderbarer
Ausführung, aber doch im Grunde dieselbe unsterbliche Schönheit
aller Vernunftwesen, wie sie auf der Erde auch schon vor
Jahrtausenden ihre Meister fand. Das Neue und Überraschende lag nur
in der Verfeinerung der Technik, in der Zartheit des Materials, in
der spielenden Überwindung der Schwere, wodurch sich ungeahnte
Effekte darboten. Nicht minder bewundernswert erschien die
Architektur dieser Hallen, Wölbungen, Galerien. Oft sprangen die
einzelnen Gemächer aus einem breiten Grundpfeiler in der Form von
Blumenkelchen vor, die auf schlanken Stielen sich zu wiegen
schienen. Diese Stiele enthielten die Treppen verborgen, auf denen
man in die Gemächer gelangte. Isma erhielt den Eindruck, daß das
Eigentümliche der martischen Kunst, das sie von der menschlichen
unterschied, nicht in einer neuen Auffassungsform des Schönen lag;
hier wirkten offenbar zeitlose Gesetze als bestimmende Ideen für
die freie Gestaltung des Schönen bei allen bewußten Wesen. Der
Fortschritt hing vielmehr ab von dem überlegenen Standpunkt der
Technik, wodurch sich das Gebiet für die Anwendung des Ästhetischen
ins Unermeßliche erweiterte. Nur die Intelligenz ist es, welche der
ewigen Idee entgegenwächst. Ell bestätigte diese Bemerkung und
stimmte Isma bei, nun zunächst ein oder das andre der technischen
Wunderwerke aufzusuchen.

„Ich fürchte nur, ich werde nichts davon verstehen“, sagte Isma.

Sie waren inzwischen wieder in der Eingangsrotunde angelangt und
hatten sich nach dem Ausgang hinabsenken lassen, wo ihr Schlitten
bereitstand.

„Was soll ich jetzt sehen?“

„Frau Ma hat mir auf die Seele gebunden, Sie nach dem Retrospektiv
zu führen. Das ist wohl die neueste und großartigste Entdeckung.“

„Ich habe davon gehört und auch zu lesen versucht, aber Sie müssen
mir die Sache noch einmal erklären. Ist es weit bis dorthin?“

„Mit der Stufenbahn wenige Minuten. Aber wir können auch in einer
halben Stunde quer durch den Wald fahren, und das will ich eben
tun.“

Er lenkte den Radschlitten über eine der Brücken, welche, die
Bahnen und Kanäle überschreitend, in die Waldregion führten. Rasch
glitt das Gefährt unter den Schatten der Bäume in die Zone der
Wohnungen. Isma atmete auf.

„Wie schön, daß wir bald wieder in die Waldeinsamkeit kommen!“
sagte sie. „Da denke ich, wir sind daheim unter unsern Tannen, und
Sie erzählen mir wieder von den Märchen des Mars –“

„Und dabei packen wir unsre Butterbrote aus und frühstücken.“

„Ach, Ell, ich wünschte, das ginge hier! Mir armem Menschenkind ist
es schrecklich langweilig, immer so allein bei verschlossenen Türen
essen zu müssen.“

„Hier an der Straße und zwischen den Wohnungen geht es natürlich
nicht. Sehen Sie, da ist die großartige Restauration, aber wenn wir
zu speisen verlangten, würde man uns sofort jedem ein Extrakabinett
anweisen, anders ist es unmöglich. Doch ich habe daran gedacht. Ich
habe aus meinem Reisevorrat ein richtiges Erdenfrühstück
eingesteckt; zwar das Brot ist trotz des luftdichten Verschlusses
etwas altbacken, aber denken Sie, Friedauer Wurst und wirklichen
Rheinwein! Wir suchen uns ein Plätzchen, wo uns niemand sehen kann.
Ich freue mich wie ein Kind! Jedoch die gute Tante darf um Himmels
willen nichts erfahren! Das wäre schlimmer, als wenn ich Ihnen auf
dem Marktplatz von Friedau um den Hals fallen wollte!“

„Stille von Friedau! Aber das Frühstück nehme ich an. Wir wollen
dem Nu ein Schnippchen schlagen.“

Ihre Augen glänzten schelmisch, indem sie zurückblickte, als
fürchtete sie, gehört zu werden.

„Eigentlich darf ich’s ja nicht als Nume. Ich bin da in meine
Menschlichkeit zurückgefallen –“

Isma richtete die Augen auf Ell. Er sprach im Scherz, aber sie
hörte an der Art, wie er den Satz abbrach, daß ein ernstes Bedenken
in ihm aufzutauchen begann.

Ell sah, wie das glückliche Lächeln aus ihren Zügen zu verschwinden
drohte, und er griff schnell nach ihrer Hand.

„Nein, nein“, rief er, „geliebte Freundin, für Sie will ich nichts
sein als der Mensch, der glücklich ist, wenn er Ihnen dienen kann.
Aber ganz leicht ist es nicht. Denn sehen Sie – ein Nume soll ich
nicht sein, damit Sie mich nicht verändert finden; und von der Erde
soll ich nicht reden, damit Sie nicht traurig werden –“

„Sie haben recht, mein treuer Freund – ich weiß ja selbst nicht,
was ich will – ich verdiene gar nicht, daß Sie so gut sind –“

Er ergriff ihre Hand und hielt sie fest. Seine Rechte lenkte den
Radschlitten mühelos auf der glatten Bahn. Die letzten Wohnungen
verschwanden. Dichtes Buschwerk bildete auf dem freien Rasen des
Bodens ein Labyrinth von Plätzen und Gängen. Ein leichter,
erfrischender Luftzug strömte über den Boden, denn die Lichtungen
und die Industriestraßen, auf denen die Sonne brannte, wirkten um
die Mittagszeit wie Schornsteine, welche die Umgebung ventilierten
und die erwärmte Luft in die Höhe führten. Die Straße war einsam.
Die Blumen musizierten leise, und kleine eichkätzchenartige Tiere
spielten an den Stämmen der Bäume.

Ell löste mit einem Druck des Fußes den Mechanismus aus, der die
Kugelkufen emporhob und den Wagen auf zwei hochachsigen Rädern
laufen ließ, so daß er sich auch auf unebenem Weg ohne
Schwierigkeit bewegen konnte. Er verließ die Fahrstraße und fuhr
auf dem Waldrasen zwischen Buschwerk und Bäumen dahin. Ein kleiner
Weiher kam in Sicht, von einem klaren Bächlein genährt. Am Rande
desselben hielt Ell den Wagen an; es war ein reizendes, stilles
Ruheplätzchen. Kein Liebespaar konnte sich besser verstecken.

„Hier können wir es wagen“, sagte Ell.

Sie wollten nur frühstücken.

Isma sprang aus dem Schlitten. Ell reichte ihr die Tasche mit dem
heimlichen Vorrat. Beide sahen sich vorsichtig um und lachten dann
über ihre Furcht. Sie packten ihre Schätze aus und vergaßen in
heiterem Geplauder, daß über den Baumzweigen zu ihren Häuptern
nicht der blaue Himmel der Erde, sondern das Blätterdach des
martischen Riesenwaldes sich wölbte.

„Kann man durch das Retrospektiv alles Vergangene sehen?“ fragte
Isma.

„Nein“, erwiderte Ell, „nur dasjenige, was unter freiem Himmel und
bei genügender Beleuchtung vorgegangen ist. Der Erfolg beruht ja
darauf, daß wir das Licht, welches damals von den Gegenständen
ausgestrahlt wurde, auf seinem Lauf durch den Weltraum wieder
einholen, sammeln und zurückbringen.“

„Und wie ist das möglich?“

„Ich habe Ihnen schon früher gesagt – was mir freilich die andern
Menschen noch nicht glauben wollen –, daß die Gravitationswellen
sich eine Million Mal so schnell fortpflanzen als das Licht. Sie
können also das Licht auf seinem Weg einholen. Wenn zum Beispiel
vor einem Erdenjahr irgend etwas unter freiem Himmel geschehen ist,
so hat sich das von diesem Ereignis ausgesandte Licht jetzt bereits
gegen zehn Billionen Kilometer weit in den Raum verbreitet. Die
Gravitation aber durchläuft diesen Weg in einer halben Minute,
trifft also nach einer genau zu berechnenden Zeit mit den damals
ausgesandten Lichtwellen zusammen. Nun haben die Gelehrten der
Martier ein Verfahren entdeckt, wodurch man bewirken kann, daß die
den Lichtwellen nachgeschickten Gravitationswellen jene selbst in
Gravitationswellen von entgegengesetzter Richtung verwandeln und
somit zu uns zurückwerfen; sie laufen also in der nächsten halben
Minute in der Form von Gravitationswellen den Weg zurück, den sie
als Licht im Laufe eines Jahres durcheilt haben. Hier werden sie im
Retrospektiv – und das ist die Großartigkeit dieser Erfindung – in
Licht zurück verwandelt und durch ein Relais verstärkt, so daß man
auf dem Projektionsapparat genau das Ereignis sich abspielen sieht,
wie es sich vor einem Jahr vollzogen hat. Man kann den Versuch
natürlich auf jeden beliebigen Zeitraum ausdehnen, aber die Bilder
werden immer schwächer, je größer die vergangene Zeit ist, weil das
Licht inzwischen im Weltraum zuviel Störungen erfahren hat. Es
erfordert nun eine sorgfältige Berechnung, wann und wo ein Ereignis
stattgefunden hat, das man zu sehen wünscht. Man kann daher das
Retrospektiv – wenigstens vorläufig – nicht nach Belieben und
schnell wie ein Fernrohr einstellen, sondern es gehört dazu ein
umfangreicher Apparat, ein ganzes Laboratorium.“

„Wir können also nicht zu sehen bekommen, was wir wollen?“

„Nein, wir müssen uns mit dem begnügen, worauf der Apparat
gegenwärtig eingestellt ist. Aber wenn es für einen bestimmten
Zweck gerade notwendig ist, zum Beispiel um eine wichtige
Rechtsfrage oder dergleichen zu entscheiden, so wird für diesen
Zweck eine Berechnung und Einstellung vorgenommen.“

„Kann man damit auch sehen, was zum Beispiel zu einer bestimmten
Zeit auf der Erde vorgegangen ist?“

„Ich zweifle nicht, daß sich das ermöglichen läßt.“

„Und was kostet so eine Beobachtung, wenn man sie für einen
besonderen Zweck machen lassen will?“

„Dazu ist überhaupt die Erlaubnis der Staatsbehörde erforderlich.
Es gibt nämlich, soviel ich weiß, bis jetzt kein Privat-
Retrospektiv.“

Isma schwieg nachdenklich. Dann sagte sie: „Nun weiß ich ja, was es
mit dem Retrospektiv auf sich hat, und gefrühstückt haben wir auch,
so daß wir eigentlich aufbrechen könnten. Aber es ist so schön
hier, und ich bin gar nicht sehr neugierig, den Apparat zu sehen,
denn was man wirklich dabei beobachtet, kann ja nicht viel sein,
wenn man an dem vergangenen Ereignis kein Interesse hat.“

„Das ist schon wahr, indessen Ma würde –“

„Ich will es mir ja auch auf jeden Fall ansehen. Aber wir können
wohl noch hier ein wenig ruhen.“

Sie legte ihr Listuch unter den Kopf und streckte sich behaglich
hin. „Wenn ich noch einen Schluck Wasser bekommen könnte!“ sagte
sie.

Ell nahm den mitgebrachten Becher und füllte ihn am Quell. Isma
trank und gab das Glas dankend halb geleert zurück. Eben setzte es
Ell an seine Lippen, um den Rest selbst zu trinken, als sich in der
Ferne ein dumpfes Brausen erhob. Isma richtete sich erschrocken
auf.

„Was ist das?“ fragte sie. „Kommt jemand?“

Ell hatte das Glas ohne zu trinken abgesetzt. Er lauschte. Das
Brausen nahm zu. Er zog seine Uhr.

„Es ist nichts“, sagte er, „es ist das Mittagszeichen.“ Er verglich
sorgfältig die Uhr. Das Brausen mochte eine Minute gedauert haben,
dann brach es mit einem hellen Schlag plötzlich ab.

„Der Anfangspunkt der Planetenzeitrechnung wird so markiert. Hier
bei uns, nicht weit von der Zentralwarte, fällt er nur kurze Zeit
nach dem wahren Mittag. Aber ich glaube, wir müssen doch
aufbrechen.“

Er hatte nicht getrunken, sondern das Wasser unbemerkt, wie er
glaubte, auf die Erde fließen lassen, und bückte sich jetzt, um
alle Spuren des gemeinsamen Frühstücks zu beseitigen.

Isma stand schweigend auf und begab sich in den Wagen. „Wir sind
auf dem Mars“, seufzte sie leise. Sie lehnte sich zurück und schloß
die Augen.

Bald darauf kam Ell. Er betrachtete sie mit einem innigen Blick.
Der Mittagston hatte ihn wieder auf den Mars zurückgeführt. Ein
tiefes Mitleid mit dem Geschick der Freundin überkam ihn, und die
ganze Fülle seiner Liebe fühlte er in sich aufsteigen. Er hätte
sich zu ihr herabbeugen und ihre Lippen mit Küssen bedecken mögen.
Und doch war etwas Trennendes zwischen sie getreten, dessen er sich
nicht zu erwehren wußte. Er küßte die schmale Hand, die auf der
Seitenlehne des Wagens ruhte.

Isma öffnete die Augen und schüttelte leicht den Kopf.

„Sie sind müde, Isma“, sagte Ell. „Hier, nehmen Sie von diesen
Pillen, und Sie werden sich erquickt fühlen wie nach einem festen
Schlaf.“

„Nein, nein, solche Nervenreize mag ich nicht, das ist eine falsche
Erquickung.“

„Diese nicht. Es ist kein anregendes Nervengift, das den Körper zur
Abgabe seiner letzten Energiereserve veranlaßt wie unsre irdischen
Reizmittel. Es führt dem Blut und damit dem Gehirn wirklich die
verbrauchte Energie wieder zu, und zwar genau in der Form, wie es
durch den Schlaf geschieht. Die Pillen sind ganz unschädlich. In
einer halben Stunde sind Sie wieder frisch wie am Morgen. Sie sind
noch zu wenig an unsere Luft gewöhnt, Sie brauchen eine Hilfe in
diesem Klima.“

Isma nahm die Pillen. Ell schwang sich an ihre Seite, und der Wagen
rollte nach der Straße zu. Der übrige Teil des Waldes und die
Wohnungsräume wurden durchschnitten und die Industriestraße im
Quartier Tru erreicht. Ell hemmte den Wagen vor einem Tor, das er
für den Zugang zum Retrospektiv hielt. Er hatte sich jedoch in der
Richtung getäuscht, in der er durch den Wald gefahren war, und
bemerkte jetzt erst, daß er sich vor dem Erdmuseum befand.

„Corsan ba“, las Isma die Rieseninschrift, „das heißt ja doch wohl
›Sammlungen von der Erde‹?“

„Ja“, antwortete Ell, „ich habe mich geirrt. Wir müssen nach der
anderen Seite – die Stufenbahn bringt uns in einer Minute hin.“

„Ich hätte eigentlich Lust –“, sagte Isma zögernd, „könnten wir
nicht hier einmal uns umsehen?“

„Gewiß, aber Sie wollten ja heute nichts von der Erde wissen.“

„Es ist schon wahr – aber ich bin neugierig, was ihr hier von dem
wilden Planeten gesammelt habt. Und man wird die alte Erde doch
nicht los.“ Sie seufzte. Unentschieden sah sie abwechselnd auf die
Menge, die in den Eingang strömte, und dann auf Ell.

„Es ist heute besonders stark besucht“, sagte dieser, „alles redet
jetzt von den Menschen. Wenn man uns nur nicht erkennt – wir tun
vielleicht besser, eine andere Zeit zum Besuch zu wählen.“

„Sie sehen, man achtet gar nicht auf uns.“

„Weil diese Leute erst hineingehen. Wenn wir am Ausgang ständen,
wäre es vielleicht anders, unsere Gesichter würden auffallen.“

„Ach was“, rief Isma lebhaft. „Nun will ich gerade hinein. Ich habe
meinen dunklen Schneeschleier eingesteckt, durch den man nicht
hindurchsehen kann. Wir sind nun einmal hier – kommen Sie, Ell!“

Ell lächelte. „Das kommt von den Energiepillen“, sagte er. „Jetzt
haben Sie wieder Mut. Nun, man wird uns nichts tun, aber wenn man
Ihnen wieder Spielzeugtüten zuwirft, wie an der Polstation, so
halten Sie sie nicht für Blumensträuße.“

Isma schlug ihn mit ihrem Schirmröhrchen auf die Hand. „Zur Strafe
kommen Sie mit“, sagte sie, „damit Sie meine Trophäen tragen
können. Und nun gehe ich auch ohne Schleier trotz der kleinen
Augen.“

Sie traten in das Gebäude.

\section{30 - Das Erdmuseum}

Die einströmende Menge verteilte sich in den weiten Räumlichkeiten
des Erdmuseums, so daß Isma und Ell zwar nirgends allein, aber doch
nicht gerade beengt waren. Isma wollte gern sehen, was an der Erde
die Aufmerksamkeit der Martier besonders fessele, und wandte sich
daher solchen Gängen und Sälen zu, in denen sich die Hauptmasse der
Besucher zusammendrängte; Ell folgte ihr und musterte wie sie nicht
weniger die Beschauer als die Gegenstände. Ein riesiger Saal
enthielt in historischer Darstellung eine vollständige Entwicklung
der Raumschiffahrt. Ell hätte sich gern hier näher in die
Einzelheiten vertieft, aber Isma interessierte sich wenig dafür und
drängte weiter. Ein Wandelpanorama, das eine Reise nach der Erde
darstellte, ließen sie beiseite liegen und hielten sich nur kurze
Zeit bei der Darstellung des Luftexports von der Erde auf. Die
Maschinen, die den Menschen auf der Polinsel nicht zugänglich
gemacht worden waren, arbeiteten hier vor ihren Augen in gefälligen
Modellen. Sie sahen, wie die Luft in starke Ballons gepumpt und im
leeren Raum zum Erstarren gebracht wurde. Die gefrorenen Luftmassen
hatten das Aussehen von bläulichen Eiskugeln und die Dichtigkeit
des Stahls.

Sehr dürftig war die Sammlung der pflanzlichen und tierischen
Produkte der Erde, da sie nur aus den polaren Regionen stammte. Was
der ›Glo‹ mitgebracht hatte, war noch nicht dem Museum übergeben
worden. Dagegen hatte man schon die Nachrichten, Gegenstände und
Abbildungen verwertet, die Jo im ›Meteor‹ von der Tormschen
Expedition mitgebracht hatte. Hier drängten sich die Zuschauer
dicht zusammen, und Isma und Ell waren gezwungen, ihrem langsamen
Zug zu folgen. Es berührte sie ganz seltsam, als sie hier Grunthe
und Saltner in verschiedenen lebensgroßen Aufnahmen vor sich sahen
und auf dem Tisch eine Reihe von Ausrüstungsstücken, Kleidern und
Kleinigkeiten ausgebreitet bemerkten, die Grunthe den Martiern
überlassen hatte. Isma mußte an sich halten, um sich nicht
einzumischen, als sie die Bemerkungen der Martier und die Scherze
vernahm, die sie über die Menschen und ihre Industrie machten.

Plötzlich faßte sie Ells Arm und drückte ihn, daß es schmerzte.

„Was gibt es?“ fragte er.

„O sehen Sie!“

Eine Gruppe von Herren und Damen musterten eine Photographie.

„Eine weibliche Bat!“ sagten sie. „Sie ist hübsch“, meinten die
einen.

„Viel zu mager“, die andern.

Es war Ismas Bild. Die Photographie hatte sich unter Torms Effekten
gefunden und war mit andern Kleinigkeiten hierhergekommen.

Die neben Isma stehende Dame, die sie eben zu mager gefunden hatte,
warf zufällig einen Blick auf ihr Gesicht. Sie stutzte und stieß
ihre Nachbarin an. Ell sah, daß man auf seine Begleiterin
aufmerksam wurde. Die Umstehenden wurden still.

„Kommen Sie“, sagte er hastig zu Isma. „Man erkennt Sie.“

Er zog sie fort, beide drängten sich durch das Gewühl. Sie wandten
sich nach einer Stelle, wo das Gedränge geringer war, und glaubten
plötzlich auf dem Dach der Polinsel zu stehen. Das Panorama des
Nordpols breitete sich in naturgetreuer Nachahmung vor ihnen aus.
Dicht zu ihren Füßen schien das Meer zu branden. Das Jagdboot der
Martier lag zur Abfahrt bereit – zwei Eskimos lösten das Seil, das
es am Ufer hielt. Im Boot saßen Martier mit ihren Kugelhelmen. Und
dort – auf der andern Seite –, da standen Grunthe und Saltner, wie
sie leibten und lebten. Grunthe, mit zusammengezogenen Lippen,
schrieb eifrig in sein Notizbuch, Saltner sah lächelnd einer
verhüllten Gestalt nach, die auf zwei Krücken dahinschlich und die
Wirkung der Erdschwere auf die Martier veranschaulichen sollte.

„Da sind unsre Freunde!“ rief Ell, wirklich überrascht. Es waren
meisterhaft nachgebildete Figuren.

Isma stand lange still. Die Plattform begann sich mit andern
Besuchern zu füllen. „Wir wollen lieber gehen“, sagte sie. „Hier
unten scheint es leer zu sein, vielleicht kommen wir dort an den
Ausgang.“

Gegenüber dem Haupteingang führte von dem nachgeahmten Teil des
Inseldaches eine schmale Treppe abwärts. Ell blickte hinunter. „Es
scheint niemand da zu sein“, sagte er.

Sie stiegen hinab und befanden sich in einem Gemach, das einem der
Gastzimmer auf der Insel nachgebildet war. Keiner von ihnen hatte
beachtet, daß über der Tür die Inschrift ›Vorsicht‹ stand und vor
derselben eine Anzahl Stöcke zum Gebrauch aufgestellt waren.

„Oh, hier ist es angenehm“, rief Isma, indem sie sich auf einen der
an der Wand stehenden Lehnstühle setzte. „Hier wollen wir uns ein
wenig ausruhen.“ Sie bemerkte, daß irgendeine Veränderung mit ihr
vorging, die ihr wohltat, wußte jedoch nicht, was der Grund sei.

Ell wollte seinen Sessel in ihre Nähe heben, mußte aber dazu eine
ungewohnte Kraft aufwenden. „Sind diese Sessel schwer!“ sagte er.
Im selben Augenblick fiel ihm die Ursache ein.

„Hier herrscht ja Erdschwere“, rief er überrascht. „Das ist also
auch eine Demonstration, und darum ist es so leer hier.“

„Das ist herrlich!“ sagte Isma vergnügt.

Ein Martier trat in die Tür, knickte zusammen und zog sich sogleich
zurück. Isma lachte laut. Sie sprang auf, drehte sich vor Vergnügen
im Kreis und rief:

„Kommt nur herein, meine Herren Nume, hier ist die Erde, hier
zeigt, ob ihr tanzen könnt!“ Sie schlüpfte hierhin und dahin,
rückte an den Stühlen und nahm ihren Hut ab. „Ich bin wie zu
Hause!“ sagte sie. „Jetzt sieht man erst, daß die angebliche
Leichtigkeit dieser Federhaube eigentlich Schwindel ist. Sehen Sie
nur, wie eilig sie es hat, hinabzufallen!“

Ell sah ihr schweigend zu. Er schüttelte leicht den Kopf. „Ein Kind
der Erde“, dachte er bei sich. „Sie würde hier oben niemals
heimisch werden.“

Isma war vor eine Tür getreten. „Ob es dahinten auch noch schwer
ist?“ fragte sie.

Ell zog den Vorhang zurück. Es zeigte sich ein Balkon, von dem aus
man ins Freie unter die Wipfel der Bäume blickte. Die Gestalt eines
Mannes lehnte am Geländer. Er drehte der Tür den Rücken zu und sah,
mit der Hand die Augen schützend, auf die Straße hinab.

Ell und Isma blickten sich an. Dann lachte Isma auf.

„Da haben sie ja den Saltner noch einmal hingestellt“, rief sie.
„Und wie natürlich! Man möchte meinen, er müßte sich umdrehen und
›Grüß Gott‹ sagen.“

Die Gestalt schnellte herum.

„Grüß Gott!“ rief Saltners Stimme. Er sprang auf Ell und Isma zu
und schüttelte ihnen die Hände.

„Das ist gescheit“, rief er, „daß man schon einmal Menschen trifft.
Das ist eine Freud! Aber um alles in der Weit, wie kommen denn Sie
alleweil hierher? Ich bin ja gerad auf dem Weg zu Ihnen. Haben’s
denn meine Depesche nicht erhalten?“

„Wir sind seit heute früh von Hause fort.“

„Ja, da wird sie halt dort liegen. Schauens, ich hab Ihnen heut
früh telegraphiert, als wir von Frus Wohnort weggereist sind, um
Sie zu besuchen. Unterwegs wollten sie mir den Kram hier zeigen,
aber wie ich hier in das schöne schwere Zimmer gekommen bin, hab
ich gesagt, nun lassens mich aus, jetzt bleib ich hier, bis Sie
sich alles angeschaut haben, und dann holens mich wieder ab. Denn
das hatt’ ich satt, daß mir die Herren Nume alle nachschauten und
die Kinder mir nachliefen und meine gute Joppe anfaßten.“

„Aber wie konnten Sie auch in Ihrem Reisekostüm von der Erde sich
hier sehen lassen?“

„Wissen Sie, ich bin halt ein Mensch, und so bleib ich einer. Ich
werd mich doch nicht in eine neue Haut stecken, wo ich nicht einmal
eine richtige Westentasch’ für meinen Zahnstocher hab? Und so gut
wie Ihnen, Gnädige, würd mir’s Marsröckel auch nicht stehn.“

Isma schüttelte ihm nochmals die Hand. „Sie sind der alte
geblieben, Herr Saltner! Nun setzen Sie sich mit her, und lassen
Sie sich erst einmal ordentlich von mir ausfragen!“

Saltner schilderte in seiner anschaulichen und drastischen Weise
auf Ismas Fragen die Einzelheiten der Expedition, über die Grunthe
nur in seiner knappen Formulierung berichtet hatte, und ließ sich
von Isma die Ereignisse aus Deutschland und ihre eigenen Erlebnisse
seit der Ankunft Grunthes in Friedau erzählen. Über die Reise Ills
nach dem Pol, den Kampf der Schiffe und die Fahrt nach dem Mars
hatte er bis jetzt nur die Darstellungen kennengelernt, welche die
kurzen Depeschen gaben, und die Gerüchte und Betrachtungen, welche
die Zeitungen daran knüpften. Letztere gründeten sich auf die
mündlichen Mitteilungen der von der Erde zurückgekehrten Martier.
Der offizielle Bericht sollte erst erscheinen, nachdem er vom
Zentralrat dem Hause der Deputierten vorgelegt worden. Dies mußte
inzwischen geschehen sein, denn heute sollte die betreffende
Sitzung stattfinden. Es war zu vermuten, daß die Beratungen darüber
sich noch einige Tage hinziehen würden. Dann erst, nach Anhörung
der Deputiertenversammlung, konnte der Zentralrat einen definitiven
Beschluß fassen über die der Erde gegenüber zu treffenden
Maßnahmen. Da hierbei alle auf der Erde tätig gewesenen höheren
Beamten als Sachverständige eventuell gebraucht wurden, mußte Fru
seinen Urlaub, auf den er sonst nach der Rückkehr von der Erde
Anspruch hatte, unterbrechen, um sich in Kla aufzuhalten. Saltner,
der als Gast der Marsstaaten selbst die Rechte eines Numen erhalten
hatte, war auf seinen eigenen Wunsch unter die spezielle Fürsorge
Frus gestellt worden und wollte nun auch in Kla in seiner Obhut
bleiben. Der weiten Entfernung wegen, welche den gewöhnlichen
Wohnort Frus von Kla trennte, mußte der Transport der Wohnungen
schon am Tag beginnen, und Fru war mit Frau und Tochter und seinem
Gast Saltner vorangereist. Sie wollten sich das Erdmuseum ansehen,
und hier hatte Saltner seine Freunde von der Erde getroffen.

Ill, von den Verhandlungen im Zentralrat völlig in Anspruch
genommen, hatte sich zu Hause über die zu erwartenden Maßnahmen
nicht geäußert und auch aus Schonung für Isma von den letzten
Ereignissen nicht gesprochen. Ell war ganz in der Begeisterung für
die wiedergefundene Heimat des Vaters aufgegangen. So erfuhr er
sowohl wie Isma zuerst von Saltner, daß, wenigstens in den
südlichen Teilen des Mars, aus denen Saltner kam und wo auch die
Mehrzahl der auf der Erde gewesenen Martier herstammte, die
anfängliche Begeisterung für die Erdbewohner sich stark abzukühlen
begonnen hatte.

Der Umschwung war durch das Verhalten der Engländer gegen das
Luftschiff herbeigeführt worden, und sobald die Zeitungen Berichte
über die Behandlung gebracht hatten, die den beiden gefangenen
Martiern zuteil geworden war, begann in einigen Staaten, deren
Bewohner sich durch lebhaftes Temperament auszeichneten, eine
gereizte Stimmung Platz zu greifen. Man verlangte ein entschiedenes
Vorgehen gegen das Barbarentum der Erdbewohner, und nur der Hinweis
der ruhigeren Elemente darauf, daß man keinerlei Urteile abzugeben
berechtigt sei, bevor nicht der amtliche Bericht vorliege, hielt
die menschenfeindliche Bewegung in mäßigen Grenzen. Fru besorgte
jedoch, wie Saltner mitteilte, daß die öffentliche Meinung nach dem
Bekanntwerden des Berichts stark genug sein würde, um auf die
Entschließungen des Zentralrats einen dem guten Verhältnis zur Erde
ungünstigen Einfluß auszuüben.

Isma fühlte sich beängstigt. Sie fürchtete, wenn es zu
Feindseligkeiten der Martier gegen die Erde käme, daß sich ihrer
Rückkehr Schwierigkeiten in den Weg legen könnten, daß vielleicht
die erneute Aufsuchung Torms im Frühjahr durch Maßregeln vereitelt
werden würde, die den Martiern wichtiger erschienen. Ell suchte sie
zu beruhigen. Er sah die Sachlage in viel günstigerem Licht. Ill
werde seinen Bericht jedenfalls so mild wie möglich gestalten. Aus
der ungerechtfertigten Handlungsweise eines einzelnen Kapitäns
könne man unmöglich ein Zerwürfnis zwischen den Planeten herleiten.
Momentane Stimmungen des Publikums hätten auf dem Mars niemals
einen dauernden politischen Einfluß, da ein jeder der Belehrung des
Besseren zugänglich sei.

„Aber wer weiß“, sagte Isma, „wie man auf der Erde denken mag!“

„Wir hätten uns nicht der Gefahr aussetzen sollen, sie verlassen zu
müssen“, sagte Ell etwas verstimmt.

Isma wandte sich schmerzlich berührt ab, und Ell fuhr sogleich
fort:

„Aber an dem feindlichen Zusammenstoß der Schiffe hätten wir ja
doch nichts geändert, auch wenn wir zu Hause geblieben wären. Ich
wollte Ihnen keinen Vorwurf machen, Frau Torm, ich meine nur, wir
dürfen uns jetzt keinen trübsinnigen Grübeleien hingeben. Da wir
nun einmal hier sind –“

„Da lassen wir ruhig die Nume weitersorgen, das will ich auch
meinen“, sagte Saltner. „Es sind wirklich ganz prächtige Leute
dabei, und wir Menschen müssen halt ein bissel zusammenhalten. Hier
unser Doktor Ell, der wird sich ja wohl auch noch zu uns rechnen.
Oder –“

„Wo bleiben Sie, Sal?“ fragte eine tiefe Frauenstimme zur Tür
herein. „Kommen Sie gefälligst heraus, wir haben auf der Erde
Schwere genug genossen. Es ist übrigens irgend etwas Besonderes zu
sehen, wo wir hingehen müssen.“

„Das ist La“, rief Saltner, eilig aufspringend. „Oh, kommen Sie
mit, ich mache Sie gleich alle bekannt.“ Und sich zu den
Angekommenen wendend, rief er: „Da bringe ich Ihnen neue Menschen!
Nun bin ich doch nicht mehr das einzige Wundertier.“

Fru und die Seinigen begrüßten Ell und Isma sehr freundlich. Isma
fühlte sich trotzdem etwas verlegen; bei aller taktvollen
Zurückhaltung der Martier wußte sie doch, daß sie von ihnen, die
zum ersten Mal ein weibliches Wesen von der Erde sahen, einer
lebhaften Prüfung unterworfen wurde. Aber Las Herzlichkeit half ihr
sogleich über diesen Zustand fort. Sie gab Isma nach Menschenart
die Hand und redete sie deutsch an.

„Ich weiß“, sagte sie, „welch bedauerliche Zufälle Sie zu uns
führten, uns aber müssen wir es zum Glück anrechnen, eine Schwester
von der Erde in Ihnen begrüßen zu dürfen. Unser Freund Saltner hat
schon viel von Ihnen erzählt. Und Sie sind es ja gewesen, der die
Martier die erste Gabe europäischer Arbeit verdanken – den
Flaschenkorb nämlich, den Grunthe den unsrigen beinahe auf den Kopf
geworfen hat. Ohne den Flaschenkorb hätten wir –“, sie wandte sich
zu Ell, „Ihren prächtigen Leitfaden nicht gefunden, und ich könnte
wahrscheinlich jetzt nicht in Ihrer Sprache mit Ihnen reden.“

Sie zog dabei die Reproduktion des Büchleins aus ihrem
Reisetäschchen und zeigte sie Ell, mit dem sie jetzt martisch
weitersprach.

Sie fragte ihn, welchen Eindruck das Denkmal auf ihn gemacht habe,
das die Marsstaaten seinem Vater in der Ruhmesgalerie der
Raumschiffer errichtet hatten. Aber dorthin war Ell noch gar nicht
gekommen. Er wollte sogleich diesen Besuch nachholen, die andern
aber wünschten einer soeben neu eröffneten Schaustellung
beizuwohnen, nach der dichte Scharen von Besuchern hinströmten. Die
Richtungsweiser, denen sie folgten, besagten nur ›Neues von der
Erde‹, ohne nähere Angabe. Auch Isma war daher sehr gespannt,
dieses Neue kennenzulernen, Ell ließ sich jedoch von seinem
Vorhaben nicht abhalten. Er trennte sich am Eingang der Galerie von
den übrigen, und man verabredete nur, sich in einer halben Stunde
in der Lesehalle des Museums zu treffen.

Die Besucher drängten nach dem Theater des Museums, worin von Zeit
zu Zeit Vorträge über die Erde oder die Raumschiffahrt gehalten
wurden. Diese wurden durch bewegliche Lichtbilder illustriert, die
mit aller Kraft martischer Malerei und Technik so plastisch
wirkten, daß sie vollkommen den Eindruck der Wirklichkeit
hervorriefen. Als Frus mit ihrer Begleitung ankamen, war das
Theater, obwohl es Raum für zwanzigtausend Personen bot, schon
überfüllt. Da jedoch Fru bei der Einrichtung des Erdmuseums tätigen
Anteil genommen hatte, wußte er seine Gesellschaft einen von den
weniger ortsbekannten Besuchern meist übersehenen Gang zu führen,
der auf eine Reihe noch freier Plätze auslief. Sie befanden sich in
ziemlich versteckter Lage zwischen den architektonischen
Verzierungen über einem der Eingänge. Sehr bald ertönte ein Signal,
das den Beginn der Vorstellung bezeichnete, und die Riesenhalle
verdunkelte sich. Auf der Bühne, das heißt auf einer Kreisfläche
von etwa dreißig Metern Durchmesser zeigte sich eine vorzüglich
dargestellte Gegend aus dem Polargebiet der Erde, ein Teil des
Kennedy-Kanals, mit felsigen Ufern und Gletscherabstürzen, wie er
aus der Vogelperspektive des Luftboots in einigen hundert Meter
Höhe erschien. Die Polardämmerung lag über der Landschaft, die von
einem strahlenden Nordlicht erhellt wurde. Nun erfolgten die
Lichteffekte des Sonnenaufgangs, und es erschien das kleine
Luftboot der Martier. Im Vordergrund erkannte man den Cairn, an
welchem die Engländer bauten, man sah, wie sie denselben verließen,
in den Abgrund stürzten, von den Martiern herausgeholt und am Fuße
des Steinmannes niedergelegt wurden. Die ganze Szene, von den
Zuschauern mit lebhaftem Beifall begleitet, wurde durch die
künstlich verstärkte Stimme eines gewandten Redners erklärt.

Es erschienen nun, vom Standpunkt der am Cairn befindlichen Martier
aus nicht sichtbar, die englischen Seesoldaten; fratzenhafte
Gestalten, wahre Teufel, in unmöglicher Kleidung, führten sie, ihre
Gewehre schwingend, einen wilden Kriegstanz auf, der durchaus der
Phantasie des martischen Wirklichkeitsdichters entstammte. Isma und
Saltner war es peinlich, den Eindruck zu beobachten, den diese
Szene auf das Publikum ausübte. Es nahm sie in vollem Glauben auf
und wollte sich über die abenteuerlichen Wilden totlachen.

Saltner schüttelte den Kopf. „Ich bin kein Freund der Englishmen“,
sagte er, „aber so sehen sie doch nicht aus, und so benehmen sie
sich auch nicht. Man bringt ja den Martiern ganz falsche Begriffe
von den Menschen bei.“

„Unseren gefangenen Landsleuten, denen so übel mitgespielt wurde,
sind sie jedenfalls so erschienen“, sagte La. „Sie haben ihre
Schilderungen offenbar unter dem Eindruck der erlittenen
Mißhandlungen gemacht.“

„Ich bedauere trotzdem“, bemerkte Fru unwillig, „daß man hier diese
Aufführung veranstaltet, es ist unsrer nicht würdig. Aber seit
jenem Zwischenfall ist leider von einem Teil der Presse die Ansicht
verbreitet worden, daß die Menschen nicht als vernünftige Wesen zu
betrachten und als gleichberechtigt zu behandeln seien. Das ist
nicht gut.“

Die Szene änderte jetzt ihren Charakter aus dem Komischen in das
Schauerliche. Die Engländer stürzten unter wildem Geheul, das
akustisch wiedergegeben wurde, auf die beiden Martier zu und
überfielen sie. Die Martier scheuchten sie majestätisch zurück, und
es entwickelte sich zunächst eine Art Diskussion, die durch das
menschliche Kauderwelsch, welches Englisch vorstellen sollte, einen
Augenblick ins Komische umzuschlagen schien, aber sofort die
Entrüstung der Zuschauer wachrief, als eine neue Schar von Wilden
den Martiern in den Rücken fiel und sie hinterrücks niederriß. Dann
wurden den unglücklichen Opfern die Arme zusammengeschnürt und sie
an langen Stricken fortgeschleppt.

Bei diesem Anblick brach im Theater ein unheimlicher Lärm aus. Wie
ein Wutschrei ging es durch die Masse der Zuschauer. Die Fesselung,
die Beraubung der persönlichen Bewegungsfreiheit, war die größte
Schmach, die einem Numen angetan werden konnte. Die Gesamtheit der
Martier fühlte sich dadurch beleidigt. Und seltsam, während man die
Menschen eben als unvernünftige Wesen belacht hatte, betrachtete
man sie doch jetzt als verantwortlich für ihre Handlungen. Die
Darstellung hatte offenbar die Tendenz, die Menschen als böse zu
zeigen, indem das Folgende ihre Intelligenz zu verdeutlichen
bestimmt war. Das englische Kriegsschiff dampfte herbei. Es schien
ganz im Vordergrund zu liegen, und in einem kaum verfolgbaren
Wechsel des Bildes befand man sich plötzlich an Bord desselben. Die
vorzügliche Einrichtung, die musterhafte Ordnung, die Waffen und
Maschinen bewiesen die hohe technische Kultur der Menschen; dagegen
stach die rohe Behandlung der Gefangenen häßlich ab und empörte die
Zuschauer nur um so heftiger. Mit Jubel wurde daher das Erscheinen
des großen Luftschiffes begrüßt und der Kampf zwischen den Martiern
und Menschen mit Enthusiasmus verfolgt. Die erhabene Friedensliebe
der Nume schien verschwunden, in dieser gereizten Versammlung
wenigstens kam sie nicht zum Ausdruck. Und als in einem ästhetisch
wunderbar gelungenen Schlußtableau auf der Eisscholle am Felsenufer
Ill selbst erschien und den Gefangenen die Fesseln löste, artete
die Vorstellung zu einer eindrucksvollen patriotischen Kundgebung
aus. Die Rufe „Sila Nu“ und „Sila Ill“ brausten durch das Haus.

Isma lehnte sich ängstlich zurück. Sie fürchtete jeden Augenblick,
sich selbst oder wenigstens Ell auf der Bühne erscheinen zu sehen;
aber mit diesen den Martiern befreundeten Menschen wußte die
tendenziöse Dichtung nichts anzufangen, sie waren einfach
fortgelassen. Saltner war wütend. „So was dürfte die Polizei gar
nicht erlauben“, sagte er, „bei uns würde man das gleich verboten
haben.“

„Was wollen Sie“, sagte La, „dies ist eine Privatveranstaltung. Sie
können das Theater mieten und morgen eine Verherrlichung der Erde
aufführen.“

Sie sah ihn lächelnd an, und er schwieg.

„Es muß auch etwas geschehen“, sagte Fru, „um der Verbreitung
dieser Menschenhetze entgegenzuwirken. Lassen Sie uns gehen.“

\section{31 - Mars-Politiker}

Die Entleerung des Theaters geschah trotz der ungeheueren
Zuschauermenge in wenigen Minuten, denn zahlreiche breite Gänge
führten nach allen Seiten auseinander und mündeten nach der Straße
hin. Man hörte überall unter dem Eindruck der Vorstellung
verächtlich über die Menschen sprechen, doch hatte die übertriebene
Darstellung der Menschen als Wilde das Gute, daß niemand auf die
Vermutung kam, in Isma und Saltner solche Erdbewohner vor sich zu
haben, obwohl Saltner in seiner Joppe, die Hände in den Taschen, in
recht auffallender Weise einherschlenderte und den prüfenden
Blicken, die ihn gelegentlich trafen, ungeniert begegnete. Aber da
jetzt alle in gleicher Richtung sich bewegten und noch von den
Eindrücken erfüllt waren, die sie eben erhalten hatten, so achtete
man wenig auf ihn.

Erst als sich Fru mit seiner Begleitung in dem Vorraum der
Lesehalle zwischen dichten Gruppen sich lebhaft unterhaltender
Martier hindurchdrängen mußte, wurde man wieder auf ihn aufmerksam.
Hier begegneten sich Besucher des Theaters und solche, die aus der
Lesehalle kamen und sich soeben mit den neuesten Nachrichten
bekanntgemacht hatten. Es herrschte eine sichtliche Erregung.
Verkäufer riefen die neuen Blätter aus für diejenigen, die sich das
in der Halle Gelesene in eigenen Exemplaren mit nach Hause nehmen
wollten.

„Der Bericht des Zentralrats!“ – „Die Rede des Repräsentanten
Ill!“

„Die Rede des Deputierten Eu!“ – „Der Antrag Ben.“

„Karte der Erde!“ – „Leben und Tod des Kapitäns All.“

„Der Sohn des Numen auf der Erde.“ – „Bild des Baten Saltner!“ –
„Bildnis der Batin Torm.“

Isma und Saltner verstanden das in eigentümlichem Tonfall
herausgestoßene Martisch der Ausrufer nicht. Fru und La suchten
schnell mit ihren Begleitern aus dem Gewühl in die Lesehalle zu
gelangen. Aber Saltner erkannte in der Hand eines Verkäufers sein
wohlgetroffenes Bildnis.

„Was?“ rief er. „Da werd ich wohl gar feilgehalten. Das ist mir
doch noch nicht passiert, das muß ich mir mitnehmen.“

Die um den Verkäufer Herumstehenden hatten ihn nun natürlich
sogleich erkannt. Bald war die Gruppe von Neugierigen umringt, und
es fielen manche nicht sehr schmeichelhafte Äußerungen.

Saltner nahm sein Bild in Empfang und zahlte. Man hatte ihm als
Gast der Regierung einen anständigen Reisefonds übermittelt.

„Da schaut mich an“, sagte er, sich in Positur stellend, „wenn Ihr
noch keinen anständigen Bat gesehen habt.“ Und auf martisch fügte
er hinzu: „Nun, seh ich aus wie ein Engländer?“

La drängte ihn vorwärts. Sie führte Isma am Arm, die ihren Schleier
vorgezogen hatte und ihrer martischen Tracht wegen nicht auffiel.
Die Nahestehenden blickten Saltner nicht gerade wohlwollend an,
belästigten ihn aber in keiner Weise und folgten ihm auch nicht,
als er sich durch sie hindurchdrängte, obwohl ihm jetzt jeder
nachsah. So gelangten alle in das Innere der Lesehalle, die aus
einer Reihe großer Säle bestand.

Die langen Tafeln waren dicht besetzt. Viele der Lesenden benutzten
diese Zeit, um ihrer offiziellen Lesepflicht zu genügen. Denn jeder
Martier war verpflichtet, bei Verlust seines Wahlrechts, aus zwei
Blättern, von denen eines ein oppositionelles sein mußte, täglich
über die wichtigsten politischen und technischen Neuigkeiten sich
zu unterrichten. Die größeren Blätter gaben zu diesem Zweck kurze
Auszüge besonders heraus.

Im Saal herrschte absolute Stille. Hier wurde nicht gesprochen. An
den Wänden befanden sich jedoch kleinere Abteilungen, verschlossene
Logen, in mehreren Stockwerken übereinander, in denen sich Bekannte
zusammensetzen und ihre Meinungen austauschen konnten. In eine
solche Plauderloge begab sich Fru mit seinen Begleitern. Er schloß
die Tür und trat an einen Fernsprecher, der zur Verwaltung führte.
Hier nannte er seinen Namen und die Nummer der Loge. Dann fragte
er, ob Ell re Kthor, am gel Schick, nach ihm gefragt habe. Die
Antwort besagte, ja, er befinde sich in Loge 408. Fru ließ ihm nun
die Nummer seiner Loge sagen und ihn zu sich bitten. Auf demselben
Weg machte er eine Bestellung auf eine Reihe Erfrischungen, die
alsbald auf automatische Weise in dem Schrankaufsatz des Tisches
erschienen, der auch hier die Mitte des Zimmers einnahm.

Es befand sich darunter für jeden Anwesenden eine Schüssel mit
Wasser, das durch eine kleine Flamme in lebhaftem Sieden erhalten
wurde.

„Ach“, rief Saltner, „das sind heiße Boffs, das ist die beste
Frucht auf diesem künstlichen Planeten; das ist wirkliche Natur.“

Isma kannte die Speise noch nicht und fragte danach.

„Um Himmels willen“, sagte Saltner, „nennen Sie die Boffs nicht
eine Speise, sonst dürfen wir sie ja nicht zusammen essen. Das ist
eben das Beste daran, daß sie nicht als Speise, sondern als
Erfrischung gelten, weil sie wirkliche Früchte, in der Natur
gewachsen sind, eine Art Erdgurken, oder wie man sie nennen soll,
und deshalb hier gemeinschaftlich gegessen –“

„O pfui“, sagte La, ihn leicht auf den Arm schlagend. Sie las
bereits eifrig in einer Zeitung, in die sie sich jetzt wieder
vertiefte.

„Ich wollte sagen, genossen, ästhetisch verwendet werden dürfen.
Aber gut schmecken sie doch.“

Er zog die Schüssel an sich heran und griff – zum Erschrecken Ismas
– mit der Hand in das siedende Wasser, eine der rötlichen,
gurkenartigen Früchte hervorziehend.

„Sie brauchen nicht zu fürchten, daß Sie sich verbrennen“, sagte er
lachend zu Isma. „Das Wasser ist gar nicht heiß, es siedet in
unverschlossenen Gefäßen hier schon bei 45 Grad Celsius. Das ist ja
ein Planet ohne Luftdruck.“

„Lassen Sie endlich unsern Nu in Frieden“, sagte La lachend, indem
sie die Zeitung beiseite legte, „sonst werden Sie mit dem nächsten
Schiff nach dem Südpol Ihrer abscheulichen schweren Erde
transportiert. Lesen Sie lieber die neuesten Beschlüsse, sie werden
Sie interessieren. Ich fürchte nur, mit dem Urlaub wird es diesmal
nichts sein. Wer weiß, ob wir nicht wieder fort müssen!“

Isma horchte auf.

„Nach der Erde?“ fragte sie. „Schickt man Schiffe jetzt nach der
Erde?“

In diesem Augenblick trat Ell ein. Er sah erregt aus. In der Hand
hielt er einen Stoß Blätter und Zeitungen, die er teils gekauft,
teils aus dem Saal entnommen hatte.

Ehe er sich in die Lesehalle begab, hatte er lange vor dem Denkmal
seines Vaters gesessen. Es war eine Porträtstatue in Lebensgröße,
die All in seinen jüngeren Jahren darstellte, in der Kleidung des
Raumschiffers. Man glaubte ihn durch die Stellithülle des
Raumschiffs auf der Kommandobrücke stehend zu sehen und mit ihm auf
die unter ihm liegende Erde hinabzublicken. Aus seinen Augen sprach
der feste Entschluß, auf diesen Planeten siegreich seinen Fuß zu
setzen.

„Du hast uns den Weg gezeigt, den wir nun betreten“, so klang es in
Ells Seele. „Dir verdanken wir die Erde, die du mit deinem Leben
uns gewonnen hast.“

Die jugendlichen, kräftigen Züge des Bildes schienen sich zu
verwandeln. Ell sah in ihnen wieder den schwermütigen, ernsten
Mann, wie er ihn gekannt, nur der siegreiche Blick des Auges war
geblieben, der ihm entgegenfunkelte, wenn der Vater dem Jüngling
von der Heimat sprach und von der großen Aufgabe, die Erde zu
gewinnen für die Numenheit. Er gedachte des eigenen Lebens und der
letzten Jahre, die er auf der Erde gearbeitet hatte, erfüllt von
dem Gedanken, daß der Menschheit Glück abhinge von ihrer Befreiung
durch die Kultur der Martier. Und jetzt stand er auf dem Mars, nun
blickte er hinab auf die Erde, und es war ihm, als verlöre sich das
Schicksal der Erdbewohner wie eine Episode in der Geschichte der
Sterne, als lebe er mit den Numen um der Nume willen und sähe in
der Besetzung der Erde nur eine der Stufen, das höchste Leben des
Geistes im Kampf mit den widerstrebenden Kräften der Natur zu
erhalten. Was war ihm nun die Menschheit? Was wär sie ihm gewesen?
Wenn er sie zu lieben glaubte, war es nicht allein die Eine
gewesen, in der er die Menschheit liebte? Was hielt ihn noch an dem
barbarischen Planeten? Das Andenken seiner Mutter? Sie war dahin;
dieses Andenken blieb ihm überall. Und die tiefblauen Augen der
heißgeliebten Frau, deren weltfernes Leuchten durch all die Jahre
hindurch mit unverminderter Kraft in seinem Herzen gewirkt hatte?
Sie wirkten fort und fort mit ihrer zarten Gewalt – die teuren
milden Züge, von denen ein glückliches Lächeln zu gewinnen sein
steter Gedanke, für die er nahe daran gewesen war, seine Numen heit
zu vergessen, um sie zu erobern mit den Mitteln der Menschen für
sich und um ihretwillen ein Mensch zu werden wie die andern! Und
jetzt, jetzt konnte er dies sicherer wie je, nie war er diesem Ziel
näher gewesen. Aber diese Frau war hierhergekommen, weil sie
ausgezogen war, ihren Mann zu suchen, und er hatte sich ihr gelobt,
ihn finden zu helfen. Sie wird ihn finden, und drunten in Friedau
oder in einer andern Stadt, wohin der Ruhm des Polentdeckers sie
führen würde, da wird sie glücklich sein und der Reise nach dem
Mars und des fernen Freundes gedenken wie eines Traumes, der
zerronnen ist. Und er? Sollte er weiter leben dort unten, um eine
flüchtige Stunde ihrer Nähe zu gewinnen, um sich zu versichern, daß
er zu ihr gehöre, wie ein teures Schmuckstück ihres Daseins? Sollte
er wieder zwischen diesen engherzigen Schlauköpfen wandeln, um ihre
ganze, verständnislose Wirtschaft zu verachten?

Nein, nun er die Freiheit der Heimat gekostet hatte, konnte er
nicht dauernd auf die Erde zurückkehren. Was war ihm noch die
Menschheit?

„Ein Vermächtnis hast du uns gelassen, o Vater“, so sagte Ell im
stillen zu sich, „die Erde, auf der du littest, zu gewinnen zu
einem höheren Zweck. Und ich vor allen habe die Pflicht, dies
Vermächtnis anzutreten. In Frieden wollen wir die Menschheit
gewinnen und ihr zum Segen. Und, ein Menschensohn, weiß ich von
ihren Schmerzen zu sagen. Aber wenn unseliger Mißverstand zum
Streit führt, so kann mein Platz nur dort sein, wo du gestanden
hast.“

Er erhob sich. Bald umwogte ihn wieder der Verkehr des Tages. Er
begab sich nach der Lesehalle. Begierig griff er nach den neuen
Depeschen und studierte sie, bis Fru ihn rufen ließ.

„Wissen Sie schon alles?“ war gleich seine erste Frage beim
Eintritt. Er sprach martisch. La antwortete lebhaft. Fru und seine
Gattin mischten sich ein. Die Martier sprachen schnell und eifrig.
Ell hatte offenbar noch etwas Wichtiges erfahren. Isma und Saltner
konnten dem schnellen Gespräch nicht folgen. Die Menschen schienen
einen Augenblick vergessen. Es war nur eine Minute der Erregung.
Dann wandte sich La mit ihrem freundlichen Lächeln zu Isma.

„Haben Sie alles verstehen können?“ fragte sie. „Ihr Freund bringt
uns wichtige Mitteilungen.“

„Ich konnte nicht folgen“, sagte Isma.

Jetzt erst wandte sich Ell zu Isma. Sie sah ihn an. In ihren Augen
lag es wie eine schmerzliche Bitte: Verlaß mich nicht. Ich bin
einsam. Ich weiß nicht, was das alles soll. Sie fragte ihn jetzt:

„Was gibt es denn Neues? Erzählen Sie nun auch mir einmal.“

Und während dieser kurzen Worte wechselte ihr Ausdruck. Der
ängstliche Zug wich einem frohen Vertrauen. Sie fühlte sich wieder
sicher, seitdem er neben ihr weilte.

„Ist es etwas Ungünstiges?“ fragte sie weiter, als Ell zögerte.

La war der Wechsel in Ismas Mienen nicht entgangen. Sie hatte das
Aufblitzen ihrer Augen beobachtet, als Ell eintrat, und jetzt die
Beruhigung ihrer Stimmung. Und ebenso unbemerkt blickte sie auf Ell
und las in seiner Seele. Er wandte sich mit einem fragenden Blick
an sie, aber sie beugte sich schnell zu Saltner hinüber.

„Ich weiß wirklich nicht“, sagte Ell, „was wir von der Sache zu
erwarten haben, aber jedenfalls können wir jetzt auf eine
schnellere Entwicklung gefaßt sein. Es werden Schritte getan, noch
während des Winters Nachrichten von der Erde zu erhalten.“

„Wie ist das möglich?“ fragte Isma.

„Haben Sie die Vorgänge in der Kammer von heute vormittag gelesen?“
Isma schüttelte den Kopf.

„Wir sind eben erst gekommen“, sagte Fru, „und wissen selbst noch
nichts Zusammenhängendes. Wir bemerkten nur, daß die Stimmung gegen
die Erde umgeschlagen zu sein scheint.“

„Ich sah eben hier zufällig“, fügte La hinzu, „daß die Südbezirke
auf eine starke Armierung des Erdsüdpols dringen und ein Vorgehen
von dort aus wünschen, und ich wußte nicht, was das zu bedeuten
hat.“

„Dann erlauben Sie, daß ich in Kürze mitteile, was ich gelesen
habe. Der Bericht der Regierung stellte den Konflikt mit dem
englischen Kriegsschiff und die Gefangennahme und Behandlung
unserer Leute als das dar, was er war, ein unglücklicher Zufall
und die Tat eines untergeordneten Kapitäns, für die man kaum die
englische Regierung, geschweige denn die Bewohner der Erde
verantwortlich machen dürfe. Sie erklärte, daß durch diesen
Zwischenfall an dem ursprünglichen Plan nichts geändert werde. Man
wolle im Beginn des nördlichen Erdfrühjahrs eine starke
Luftschifflotte bereithalten, um, sobald die Nordstation zugänglich
sei, die zu diesem Zweck früher als sonst eröffnet werden sollte,
sofort sämtliche Großmächte der Erde in ihren Hauptstädten
aufzusuchen. Man werde den Regierungen einen Vertrag über den
Verkehr und die Handelsbeziehungen zum Mars vorschlagen und die
Vorkehrungen so treffen, daß sich das Übereinkommen ruhig und
friedlich vollziehe. Nur einen böswilligen Widerstand werde man im
Interesse der Gesamtheit eventuell mit Gewalt niederwerfen, indem
man über den betreffenden Staat das Protektorat der Marsstaaten
aussprechen werde.

Diese Erklärung fand aber lebhaften Widerspruch, sowohl von der
Opposition gegen die Erdkolonisation, die unter der Führung von Eu
schon immer die weitgehenden Pläne der Erdbesiedelung bekämpfte,
als auch von einer erst infolge der letzten Nachrichten
entstandenen Gruppe, denen das Vorgehen gegen die Erde nicht scharf
genug erschien. Und beide standen nun zusammen. Denn Eu vertrat
jetzt den Standpunkt, es wäre von Anfang an das beste gewesen, sich
überhaupt nicht um die Menschen zu kümmern; nachdem aber die
Regierung einmal den Fehler gemacht habe, die Existenz der Nume zu
verraten und durch feindselige Handlungen gegen die Menschen sich
bloßzustellen, verlange es die Pflicht der Numenheit, den
Erdbewohnern auch den richtigen Begriff derselben und Aufklärung
über die Bedeutung und die Absicht der Nume zu geben. Es sei
Menschenblut geflossen und der Numenheit Schmach angetan worden.
Die Sühne könne nur in einer großen Tat friedlicher Kultur
bestehen. Es müsse den Erdbewohnern gezeigt werden, daß wir ihre
Gewohnheit des Kampfes mit den Waffen verabscheuen und als
unsittlich verwerfen. Es sei deswegen über die ganze Erde der
Planetenfrieden zu gebieten und die Entwaffnung sämtlicher Staaten
zu verlangen.“

„Jesus Maria!“ rief Saltner. „Das nenn ich einen Radikalen! Ich hab
schon nichts dagegen, aber, was meinens, was sie bei uns auf dem
Kriegsministerium dazu sagen werden?“

„Das Verlangen der chauvinistischen Gruppe“, fuhr Ell fort, „war
nicht weniger radikal. Sie erklärten, die Menschen hätten durch ihr
Verhalten bewiesen, daß sie dem Begriff der Numenheit noch nicht
zugänglich seien. Sie seien nicht als freie Persönlichkeiten zu
behandeln und nicht würdig des Weltfriedens. Man solle sie im
Gegenteil ruhig untereinander wüten lassen, aber die ganze Erde und
ihre Bewohner als Eigentum der Marsstaaten erklären. Die einzelnen
Gebiete der Erde seien unter die einzelnen Marsstaaten aufzuteilen,
um die Einkünfte derselben zu vermehren. Die Menschen seien
ausdrücklich als unfrei und Nicht-Nume zu bezeichnen und die
Erdstaaten durch vom Zentralrat eingesetzte Gouverneure zu
beaufsichtigen. Im Resultat aber waren beide oppositionelle
Parteien einig, die Unterwerfung der Erde müsse sofort mit allen
Mitteln in Angriff genommen werden.

Die Debatte war sehr heftig, und die Regierung hatte einen schweren
Stand. Noch während der Sitzung schloß sich die chauvinistische
Gruppe zu einer Fraktion der ›Antibaten‹ zusammen, und aus großen
Teilen des Landes trafen bereits zustimmende Erklärungen ein für
die Menschenfeinde.

Allmählich gelang es jedoch der Regierung, den Parteien begreiflich
zu machen, daß man die Erdbewohner aus Unkenntnis unterschätze;
eine derartige Bestimmung über sie werde nicht durchzusetzen sein,
ohne zu den Gewalttätigkeiten zu führen, die man gerade verabscheue
und verhüten wolle. So kam ein Kompromiß zustande, zunächst noch
ein genaueres Studium der Machtverhältnisse der Erdstaaten
abzuwarten. Doch mußte die Regierung ihrerseits zugestehen,
sogleich wenigstens von England eine Bestrafung des Kapitäns der
›Prevention‹ und eine Genugtuung für die Mißhandlung der Gefangenen
zu verlangen.

Dieser Kompromiß zwischen Opposition und Regierung fand endlich im
Antrag Ben seinen Ausdruck. Danach sollte sobald als möglich ein
Raumschiff nach der Südstation der Erde abgehen und drei große
Erdluftschiffe dahin bringen. Vom Südpol aus sollte zunächst mit
der englischen Regierung verhandelt werden, um eine Genugtuung für
die Gefangennahme der beiden Martier und die Beschädigung des
Luftschiffs zu verlangen. Man sollte sich jedoch dabei der größten
Mäßigung befleißigen, einerseits um die wohlmeinende, obwohl ernste
Gesinnung der Martier zu zeigen, andrerseits weil man es vor Beginn
des Frühjahrs der nördlichen Erdhalbkugel nicht zu einer größeren
Aktion kommen lassen durfte. Denn die Station auf dem Südpol bot
weder den Raum noch die Sicherheit der Landung für eine größere
Flotte der Martier; auch wäre es wenig praktisch gewesen – soviel
sah auch die antibatische Opposition ein –, vom Südpol aus mit den
Großmächten der Erde zu verhandeln, da der Weg vom Südpol bis
Berlin oder Petersburg selbst für ein Luftschiff der Martier fast
vierundzwanzig Stunden in Anspruch nahm. Der Zentralrat wurde mit
der sofortigen Ausführung der Maßnahmen beauftragt. Die letzte
Depesche besagte bereits, daß der Zentralrat einen besonderen
Erdausschuß mit einjähriger Amtsdauer und weitreichenden
Vollmachten ernannt und den Repräsentanten Ill zum Leiter desselben
bestimmt habe. Das ist augenblicklich der Stand der Dinge“, schloß
Ell. „Was sagen Sie dazu?“

Er warf die Zeitungen auf den Tisch und ging erregt auf und ab.
Niemand antwortete sogleich. Die Nachrichten waren nicht nur von
weittragender politischer Wichtigkeit, sie mußten zugleich das
private Geschick der hier Versammelten unmittelbar beeinflussen.
Die Mienen waren düster geworden. Nur Isma pochte das Herz freudig.
Sie war sofort entschlossen, alles daranzusetzen, um nach dem
Südpol und von dort nach Hause zurückzukehren. Wollte man sich mit
England in Verbindung setzen, so mußte doch ein Luftschiff nach
bewohnten Gegenden, wenn nicht nach London, so wenigstens nach den
Kolonien, wahrscheinlich nach Australien, abgesandt werden, und mit
diesem hoffte sie reisen zu können. Und wenn dies nicht möglich
war, so konnte sie immerhin auf baldige Nachrichten von der Erde
rechnen. Diese Gedanken und Wünsche gingen durch ihr Gemüt, während
ihre Hände das Flugblatt zerknitterten, das ihr Porträt zeigte; es
hatte sich unter den von Ell mitgebrachten Papieren befunden.

„Wollen Sie nicht Platz nehmen?“ sagte La zu Ell. Er setzte sich
hastig und beschämt. Das Menschenblut in ihm hatte ihn hin- und
hergetrieben. Als Martier schickte sich das ja nicht, da verhielt
man sich ruhig. Er ärgerte sich.

„Das ist fatal, höchst fatal“, begann jetzt Fru. „Ich halte den
Beschluß für einen schlimmen politischen Fehler. Auch Ill wird
dieser Ansicht sein, aber er konnte jedenfalls nicht mehr
durchsetzen. Unsre Politiker kennen die Verhältnisse zu wenig.
Verhandlungen, denen wir nicht die Tat auf dem Fuße folgen lassen
können, müssen unsern Standpunkt erschweren und bei den Regierungen
der Erde nur die Meinung erwecken, daß sie uns nicht ernst zu
nehmen brauchen.“

„Eine sakrische Dummheit ist’s“, platzte Saltner deutsch heraus.

„Sie fürchten“, sagte La, sich zu Ell wendend, „daß wir auf diese
Weise zur Anwendung von Gewalt gedrängt werden?“

„Wohl möglich“, erwiderte Ell. „Doch die Menschen werden bald
begreifen, daß sie sich uns fügen müssen. Wir werden ihnen zeigen,
daß wir nur ihr Bestes wollen.“

„Ich fürchte Schlimmeres“, entgegnete La lebhaft. „Die Verhältnisse
werden sich so entwickeln, daß die antibatische Bewegung immer mehr
Nahrung erhält. Statt des Friedens werden wir den Kampf zwischen
den Planeten bekommen, man wird die Menschen nicht als
gleichberechtigt anerkennen – es wird furchtbar werden.“

„Oh, lassen Sie uns zurückkehren!“ rief Isma. „Bitten Sie Ihren
Oheim, daß uns das erste Schiff nach dem Südpol mitnimmt.“

Ell antwortete nicht. Er blickte finster vor sich hin. Fru stand
auf. „Ich glaube“, sagte er, „es ist das beste, wenn wir unsere
Reise fortsetzen und Ihren Oheim aufsuchen. Bis wir an unser
provisorisches Quartier und an Ihre Wohnung gelangen, ist die
Ruhezeit gekommen. Wir treffen uns dann alle zur Plauderstunde bei
Ill.“

„Wir wollten noch nach dem Retrospektiv“, sagte Ell.

„Dazu ist es ohnehin schon zu spät.“

„Ich habe heute genug gesehen“, fügte Isma hinzu.

Man brach auf. Der Weg bis zu dem Depot der Radschlitten, wo auch
Fru seinen viersitzigen Gleitwagen gelassen hatte, betrug einige
Minuten. La nahm Ismas Arm.

„Am Schlitten bekommen Sie Ihre Damen wieder“, sagte sie zu Ell und
Saltner. Sie schritt mit Isma voran.

„Sie möchten gern nach der Erde zurück, nicht wahr?“ sagte sie zu
Isma. „Ich sah es Ihnen an, und Sie hoffen, nach dem Südpol
mitzugehen. Aber würden Sie auch ohne Ell gehen?“

„Er wird mitgehen, wenn man uns überhaupt mitnimmt. Er muß gehen,
er gehört jetzt auf die Erde.“

La schwieg. Sie streifte Isma mit einem teilnehmenden Blick und
sah, wie eine feine Röte ihre Wangen bedeckte.

„Meine Frage darf Sie nicht verletzen“, sagte sie bittend. „Ich
kann mir wohl denken, daß es für Sie schwer sein muß, die weite
Reise ohne Begleitung eines Menschen zu machen. Aber ich glaube
auch nicht, daß man Sie jetzt mitnehmen wird. Das Schiff wird zu
Ausrüstungszwecken voll in Anspruch genommen sein. Und auf dem
Südpol finden Sie nicht die Bequemlichkeit wie auf dem Nordpol. Ich
wollte Sie nur bitten, sich nicht Hoffnungen zu machen, die
vermutlich enttäuscht werden müssen. Aber jedenfalls dürfen Sie
darauf rechnen, daß Sie nun bald Nachrichten von der Erde bekommen.
Dafür wird Ill Sorge tragen. Und Sie brauchen sich nicht verlassen
unter uns zu fühlen. Ich werde mich herzlich freuen, wenn ich Ihnen
dienen kann.“

Isma dankte, aber sie konnte sich eines bedrückenden Gefühls nicht
erwehren. Warum bedurfte sie dieses Mitleids? Sie fühlte sich
verletzt, ohne La zürnen zu können.

Die Radschlitten erschienen. Man verabschiedete sich. Mit Benutzung
der Stufenbahn konnte man in einer halben Stunde zu Hause sein.

Isma saß stumm an Ells Seite. Sie sah, daß seine Gedanken nicht mit
ihr beschäftigt waren. Sie wollte ihn jetzt nicht fragen, was er zu
tun gedenke. So tauschten sie nur flüchtige Worte, bis der Wagen
vor Ills Haus hielt.

\section{32 - Ideale}

La ließ ihre Hände von der Schreibmaschine herabgleiten und lachte
herzlich, indem sie sich in ihrem Sessel zurücklehnte.

„Nein“, sagte sie, „das ist ja nicht zu glauben! Das ist wirklich
zu komisch. Diese Bate! ich glaube, da muß selbst Schti lachen.“

Ein allerliebstes, schneeweißes Flügelpferdchen, nicht größer wie
ein kleines Kätzchen, flatterte von dem Büchergestell, wo es
gesessen, auf die Lehne von Las Armstuhl und blickte sie mit seinen
klugen Augen ernsthaft an. Das Tierchen sah wirklich aus wie ein
Miniatur-Pegasus, nur hatte es statt der Hufe zierliche Zehen, mit
denen es sich anklammern konnte. Zoologisch betrachtet gehörte es
zu den Insekten und war eine Art Heuschrecke, die aber auf dem Mars
warmes Blut besaßen und die höchstentwickelte Gruppe der Insekten
darstellten. Der Kopf war der eines Pferdes mit fast
menschenähnlichem Ausdruck, die Flügel saßen an den Schultern und
glichen denen einer Libelle.

„Schti muß lachen!“ sagte La.

Das Tierchen stieß einen Laut aus wie eines helles Lachen. „Ko
Bate, Ko Bate“, sprach es dann deutlich.

La streichelte ihm das weiche Fellchen, und es rieb sein Köpfchen
an ihrer Hand.

„Schti muß studieren!“ Das gut abgerichtete Tierchen flog auf das
Bücherbrett und setzte sich gravitätisch hin.

La fuhr in ihrer Schreibarbeit fort. Auf dem Gestell über der
Schreibmaschine stand eines der deutschen Bücher, die Ell
mitgebracht hatte. Es war ein kurzgefaßter Grundriß der
›Weltgeschichte‹, das heißt des wenigen, was man über die
Geschichte der abendländischen Menschheit wußte. La übersetzte das
Buch in Ills Auftrag ins Martische. Während sie ihre Augen langsam
über den Text gleiten ließ, lagen ihre Hände auf der Klaviatur und
ihre Finger schrieben ganz mechanisch in martischen Zeichen den
Sinn der gelesenen Sätze nieder. Die Arbeit nahm ihre
Aufmerksamkeit nicht mehr in Anspruch, als der Strickstrumpf die
einer älteren Kränzchendame, und hinderte sie nicht, sich lebhaft
mit Ell zu unterhalten, der zum Besuch gekommen war.

„Es ist eigentlich mehr traurig als komisch“, sagte Ell. „Denn die
Sache geht nicht immer bloß mit dem Knallen ab. Oft genug kommen
schwere Verwundungen und Todesfälle vor, und das Leben eines
Mannes, der verpflichtet wäre, der Menschheit und den Seinigen sich
zu erhalten, ist einem sinnlosen Vorurteil hingeopfert.“

„Das ist abscheulich. Aber ich denke, Vernunft und Gesetz verbieten
den Zweikampf. Wie ist er denn noch möglich?“

„Durch Unvernunft. Es gibt nämlich Menschen, die sich einbilden
Vernunft und Gesetz seien zwar ganz gut für das Volk, aber dieses
würde den Respekt vor Vernunft und Gesetz verlieren, wenn es nicht
durch eine auserwählte Gruppe von Menschen in Schranken gehalten
würde. Diese Auserwählten könnten sich jedoch nur dadurch als
solche erweisen, daß sie sich einen gewissen Zwang, eine Pönitenz
auferlegten, indem sie selbst zum Teil auf das höchste Gut der
Menschheit, Vernunft und Freiheit, verzichten und sich zum Sklaven
überlebter Formen machen. Sie meinen wohl, durch den Widerspruch,
den ihre Sitten erwecken, in der Allgemeinheit die Herrschaft der
Vernunft um so mehr zu stärken.“

„Welch edle Seelen, so zum Besten der Kultur sich selbst zu opfern!
Ein wahrhaft menschlicher Gedanke, die Kultur durch Unkultur des
eigenen Lebens zu fördern! Es wäre ein bloßer Irrtum, wenn er nicht
leider dadurch unmoralisch würde, daß der egoistische Zweck
unverkennbar ist.“

„Gewiß, sich selbst als Kaste zu unterscheiden. Es will jeder etwas
Besonderes sein.“

„Das soll er ja auch“, sagte La, „etwas Besonderes aber nur durch
seine Freiheit, durch die innere Freiheit, mit der wir die Mittel
bestimmen, in unserm Leben das Vernunftgesetz zu verwirklichen.
Aber diese Leute lassen, nach Ihrer Schilderung, die innere
Freiheit gar nicht gelten, weil sie sie nicht kennen. Sie setzen
ihre Ehre in Äußerlichkeiten. Ich kann mir denken, wie schwer es
ihnen sein mußte, in dieser Gesellschaft zu leben.“

„Ich kann auch nicht in ihr leben“, erwiderte Ell. „Für sie besteht
die Ehre eines Menschen in dem, was andere von ihm halten und
sagen; deswegen glauben sie auch, sie könnte durch Beleidigungen
vernichtet, durch rohe Gewalt wiederhergestellt werden. Als ob mich
der Wille eines andern erniedrigen könnte, als ob es nicht die
größte Selbsterniedrigung wäre, die eigene Vernunftbestimmung der
fremden Meinung unterzuordnen! Und da ich in ihr leben mußte, so
war mein inneres, wahres Leben eine Lüge in ihrem Sinne, eine
Umgehung ihrer konventionellen Sitten. Doch das ist das wenigste,
das ist für mich nur unangenehm, für meine Freunde beschwerlich.
Aber das Unerträgliche, das Schmerzende liegt in dem Gedanken, daß
diese Millionen und Abermillionen vernünftiger Wesen durch ihre
bloße Dummheit, durch die mangelhafte Entwicklung ihres Gehirns,
durch die fehlende Bildung in einem Zustand gehalten werden, der
sie schwach, elend, unglücklich, unzufrieden und ungerecht macht.
Denn sie sind nicht böse. Sie wollen das Gute, sie wollen die
Freiheit. Ihr Gefühl ist lebendig und warm. Darin sind sie uns
gleichstellend; die Idee des Guten, als die Selbstbestimmung, durch
die wir Vernunftwesen sind, ist in ihnen wirksam wie in uns,
insofern sind sie unsere Brüder. Aus der Menschheit erblühten
Religionen tiefster Wahrhaftigkeit und Kraft, die ihnen die
Offenbarung gaben, um die es sich handelt – unser individuelles
Leben in Raum und Zeit, den Inhalt unsres Daseins, den wir Natur
nennen, zu gestalten zu einem Mittel, um als freie Vernunftwesen
über Raum und Zeit das Reich der Ideen zu umfassen. Und Weise sind
ihnen erstanden, die gezeigt haben, wie es zu begreifen sei, daß
das Leben des einzelnen abrollt wie ein Rädchen im Getriebe der
Weltenuhr und dennoch das Ich dessen, der es selbst lebt, das ganze
Uhrwerk erst zu schaffen hat. Aber die wenigsten haben die Weisheit
verstanden. Sie haben das Gesetz, aber sie mißdeuten es und wissen
es nicht anzuwenden; sie verfallen stets in Irrtum. Und deswegen,
weil es Unwissenheit ist und nicht Mangel an Wille und an Gefühl
für das Gute, deswegen glaube ich, daß wir der Menschheit helfen
können. Verständiger müssen wir sie machen – nur nicht verständiger
im Sinne der Menschen, für die verständig nur bedeutet: klug sein
auf Kosten der andern.“

„Möge Sie dieser Glauben nicht täuschen. Ich fürchte, es ist nicht
bloß der Mangel an Verständnis des Zusammenhangs der Dinge, es ist
noch mehr die Unfähigkeit, das wirklich zu wollen, was man als gut
erkannt hat, es ist die Schwäche des Charakters hier, die Stärke
des Egoismus dort, weshalb die Menschen den unvermeidlichen Kampf
ums Dasein in so bedauernswerter Weise führen.“

„Das bestreite ich nicht, daß diese Mängel zur Erniedrigung der
Menschen beitragen, aber doch nur subjektiv, indem sie den
einzelnen unfähig machen, des Glücks der inneren Freiheit sich zu
erfreuen. Aber auch hier kann nur eine Vertiefung der Einsicht
helfen. Die Handlungen sind ja immer bedingt durch diejenigen
Vorstellungen, denen der höchste Gefühlswert zukommt, und diese
Gefühlswerte richtig zu verteilen, ist Sache der Bildung.“

„Wenn aber jemand“, sagte La, „ganz genau weiß, zum Beispiel ein
Schüler, deine Pflicht verlangt jetzt, das und das zu tun, diese
Arbeit zu vollenden, und wenn du es nicht tust, so wirst du nicht
bloß Reue haben, sondern auch sinnlich schwer dafür büßen, und
trotz dieser klaren Einsicht verleitet ihn doch eine momentane
Lust, und sei es bloß das Lustgefühl der Faulheit, die Arbeit nicht
zu tun, so sehen Sie doch, alle Einsicht hilft nichts gegen die
Willensschwäche.“

„Das spricht gerade für mich“, erwiderte Ell lebhaft.
„Willensschwäche ist doch nur falsche Richtung des Willens,
Richtung auf das Unterlassen statt auf das Handeln. Vorstellungen
sind immer dabei entscheidend. Die Einsicht war dann eben
tatsächlich noch nicht vorhanden, nicht umfassend genug. Dem
Schüler in Ihrem Beispiel haben sich etwa die Vorstellungen
eingeschlichen, die an ihn gestellte Forderung sei ein
unberechtigter Zwang oder die gefürchteten Nachteile werden zu
umgehen sein und dergleichen. Der Erwachsene, der den Zusammenhang
klarer durchschaut, wird einfach seine Pflicht tun. In anderen
Fällen wird er sich in der Lage des Schülers befinden, aber diese
Fälle werden immer seltener, je weiter die Einsicht reicht. Wenn
mich der Zorn übermannt, so daß ich den Gegner verletze, so beruht
mein Fehler darauf, daß ich nicht Zeit zur Überlegung hatte. Warum
sind die Nume soviel milder als die Menschen? Weil sie schneller
denken. Im Augenblick des Affekts ist das Bewußtsein des Menschen
ganz vom sinnlichen Reiz erfüllt, er vermag nicht alle die
Gedankenreihen zu durchlaufen, die ihm die Folgen seiner Handlungen
zeigen; er braucht dazu längere Zeit, und dann ist es zu spät. Der
Nume fühlt nicht minder lebhaft den Reiz, vielmehr noch viel
feiner; aber sein Gehirn ist so geübt, daß im Moment der ganze
Zusammenhang der Folgen seines Zustandes ihm ins Bewußtsein tritt
und sein Handeln bestimmt. Das ist es, was man Besonnenheit nennt.
Nicht mit Unrecht hielten sie die Griechen für die höchste der
Tugenden, aber sie wußten sie nicht zu erringen. Lassen Sie uns den
Irrtum verringern, und wir werden die Menschen bessern.“

„Die Leidenschaften werden Sie nicht ausmerzen.“

„Daran denke ich natürlich nicht. In ihnen ruht ja der Wert des
Lebens, und die Nume freuen sich ihrer. Nur die Art ihrer Wirkung
können wir und müssen wir durch den Verstand regulieren. Auch die
Schwächen der Nume – und die werd en Sie nicht leugnen – beruhen
auf demselben Grund wie die der Menschen. Sie sind vom Leben
sinnlicher Wesen untrennbar. Die starken Gefühle sind die großen
Reservoirs der Energie des Gehirns, aus denen sie zur
Wechselwirkung des Lebens herausströmt. Wären sie nicht mehr da, so
hörte das Leben auf, so hörte das Denken auf. Aber auf den Weg
kommt es an, den die Entladung der Gehirnenergie bei der Explosion
des Gefühls nimmt. Es ist damit wie bei unsern Gebirgen auf der
Erde. Sie sind die Sammelbecken der Gewässer, die von ihnen
herabströmend den Völkern ihre segenspendende Kraft verbreiten. Die
Niveauunterschiede müssen überall sein, wo Energieaustausch, wo
Leben und Geschehen sein soll. Aber wie dieses Herabströmen
stattfindet, das macht den Unterschied von Barbarei und Kultur. Der
reißende Wildbach zerstört und verrinnt nutzlos. Bepflanzen wir die
Abhänge, verteilen wir die Wasser, führen wir sie durch Turbinen
und wandeln ihre Arbeit durch Maschinen um, so schaffen sie die
Kultur. Diese Pflanzungen, diese Maschinen sind im Gehirn die
Zellen der Rindensubstanz, in denen der Weltzusammenhang sich
bildet. Die Macht des Gedankens ist es, die den Ausgleich der
Gefühle zur Kultur lenkt. Und diese läßt durch Lehre und Erziehung
sich erweitern. Das zu tun, sind wir den Menschen schuldig, wie
Erwachsene den Kindern. Denn Kinder sind sie.“

„Ja“, sagte La, „Kinder sind sie, das habe ich auch gefunden, und
darum mögen Sie in Ihren Ansichten recht haben, Ell. Wie das Kind
nur die eine Wirklichkeit kennt, wie das Spielzeug, die Mutter und
die Erde am Himmel ihm keine andre Realität besitzen als seine
Hand, und diese keine andere als das Produkt seiner Phantasie, so
können auch die Menschen die Arten der Wirklichkeit nicht
unterscheiden. Selbst ein geistig so hochstehender Mann wie Saltner
vermag es nicht zu begreifen, daß dasselbe lebendige Individuum
gleichzeitig ganz verschiedene Realitäten besitzt, je nach dem
Zusammenhang, in welchem es sich bestimmt. Die Frau an der
Schreibmaschine ist ein Stück Naturmechanismus, die den notwendigen
Zusammenhang zwischen verschiedenen Zeichen für dieselbe
Vorstellung registriert, wenn sie, wie ich hier, eure langweilige
Geschichte übersetzt. Dieselbe Frau, wenn sie den Freund zärtlich
anblickt, ist ein Stück des Phantasiespiels, das unser Leben mit
seinem schönen Schein verklärt. Und wenn sie ein Versprechen
einlöst, ist sie ein Stück der ethischen Gemeinschaft der Nume.
Aber keine dieser Realitäten wirkt auf die andere, kann die andere
verpflichten, außer in der freien Bestimmung der Persönlichkeit
dieser Frau selbst. Das kann unser Freund nicht verstehen. Er denkt
immer, es müsse noch ein anderer Zusammenhang bestehen, notwendig
wie die Natur in Raum und Zeit, zwischen diesen Tätigkeiten –“

„Sehen Sie – dieser Mangel der Einsicht ist es, welcher die
menschliche Gesellschaft beschwert. Stets werfen sie das
Verschiedene zusammen als Eines, indem sie es mit falschen
Gefühlswerten belasten. Da ist der religiöse Glaube; er ist die
Form, wie die Persönlichkeit das Weltgesetz in ihr Gefühl aufnimmt;
die Menschen aber machen daraus ein Bekenntnis, das andre
verpflichten soll und sich damit aufhebt. Da ist das Vaterland, die
nationale Gemeinschaft; sie ist ein Mittel, die Macht des einzelnen
zusammenzufassen, um für die Menschheit zu wirken; die Menschen
umkleiden sie mit einem Gefühl, das sie zum Selbstzweck macht und
infolgedessen Feindschaft der Nationen bewirkt. Da ist der
natürliche, berechtigte Trieb der Selbsterhaltung; die Menschen
machen daraus einen vernichtenden Egoismus, der zum Kampf der
Gesellschaftsklassen führt. Und so mit allem. Hier kann Aufklärung
helfen. Natürlich nicht, um Vollendung zu schaffen, die es
überhaupt nicht gibt, aber eine höhere Stufe der Kultur. Es wäre
nicht das erste Mal, daß Aufklärung die Menschen befreit hat, aber
da mußte sie sich blutig durchkämpfen. Diesmal soll eine überlegene
Macht den Sieg von vornherein gewähren.“

„Aber wie denken Sie sich diese Einwirkung? Ehe Anschauungen und
Gewohnheiten sich ändern, müssen Generationen vergehen – die
Menschheit selbst muß sich ändern –“

„Die Planeten haben Zeit. Aber die Hauptsache wird schnell
geschehen. Die Menschen brauchten Jahrtausende, um den
gegenwärtigen Stand ihres Wissens zu gewinnen; unter der Leitung
geschickter Lehrer eignet sich heute der einzelne dieses Wissen in
wenigen Jahren an. Wir werden die heutigen Menschen nicht zu Numen
machen, aber wir werden sie in diesem Sinn führen. Nur muß unsre
Bevormundung ihre Freiheit nicht beschränken, sondern allein den
richtigen Gebrauch derselben erzielen. Das Niveau der Gesamtbildung
läßt sich binnen kurzem so heben, daß sie eine klare Einsicht in
das gewinnen, was im Leben möglich und erstrebbar ist. Sie werden
erkennen, daß es eine Utopie ist, die Gleichheit der
Lebensbedingungen anzustreben, daß die Gleichheit nur besteht in
der Freiheit der Persönlichkeit, mit der ein jeder sich selbst
bestimmt, und daß diese Freiheit gerade die Ungleichheit der
Individuen in der sozialen Gemeinschaft voraussetzt. Wir haben ja
doch viele Jahrtausende hindurch die sozialen Kämpfe durchgemacht,
bis wir erkannt haben, daß der Kampf selbst unvermeidbar, die
Gehässigkeit aber auszuschließen ist, daß in einem edlen Wettstreit
alle Stufen der Lebensführung nebeneinander bestehen können. Nur
eines ist dazu notwendig: dem einzelnen die Zeit zu geben, sich
selbst zu bilden, zu kultivieren. Die Menschen können sich darum
nicht selbst helfen, wenigstens nicht helfen, ohne den furchtbaren
Kampf von Jahrtausenden, weil sie die Mittel nicht haben, den
Massen die Sicherheit der notwendigsten Lebenshaltung zu geben.
Diese Not der Massen können wir abstellen, ohne jene Utopie der
Nivellierung des Vermögens. Wir können ihnen zeigen, daß das Hin-
und Herschwanken des individuellen Besitzes sich nicht ändern läßt
und auch nicht geändert zu werden braucht, daß aber jedem, der
arbeitet, ein befriedigendes, seinen Fähigkeiten angemessenes
Auskommen gewährleistet werden kann und daß niemand Not zu leiden
braucht. Denn wir können den Menschen die Quelle des Reichtums
erschließen durch unsre Technik, und wir können erzwingen, daß die
damit verbundenen Besitzänderungen sich in Ruhe vollziehen. Den
kleinlichen Eigennutz, den Krämersinn, die Unduldsamkeit, die
Klassenherrschaft bringen wir zum Verschwinden, sobald ein jeder
klar zu durchschauen vermag, welche Stelle im großen Zusammenwirken
der einzelnen er ausfüllt. Der tückische, nagende Neid entflieht
aus der Welt, und Menschenliebe hält den siegreichen Einzug.“

Ell war aufgestanden, seine Augen leuchteten, begeistert sah er in
die Zukunft, die ihm nahe herangekommen schien. La hatte die Hände
von der Schreibmaschine herabsinken lassen. Sie blickte ihn an.

„Halten Sie mich nicht für einen Schwärmer“, fuhr er fort. „Nicht
daß ich meinte, Leid und Schmerz aus der Menschheit verbannen zu
können. Ohne sie stände das Weltgetriebe still. Aber reinigen
können wir dieses Leid, veredeln zu dem heiligen Schmerz, der
untrennbar ist von der Liebe und dem Einblick in uns selbst. Die
fremden Schlacken können wir ausstoßen, die aus der Not, der Roheit
und der Dummheit stammen.“

„Sie glauben an die Menschheit“, sagte La. Auch sie erhob sich und
streckte ihm die Hand entgegen. „Ich begann an ihr zu zweifeln, ich
will es Ihnen gestehen. Ob sich Ihr Traum erfüllen läßt, ich weiß
es nicht, aber ich danke Ihnen, daß Sie ihn träumen, daß Sie ihn
mir erzählten. Sie haben mir neuen Mut gemacht, denn ich fürchtete
manchmal, daß das Zusammentreffen mit den Menschen beiden Teilen
verderblich werden könnte.“

„Fürchten Sie das nicht, La. Die Erde ist reich, viel reicher als
der Mars. Sie empfängt von der Sonne fast das Zehnfache der Energie
wie wir. So lange die alte Sonne strahlt, ist das Leben gesichert.
Was läßt sich unter unseren Händen aus dieser Riesenkraft schaffen!
In einem Jahr wird die Erde bedeckt sein mit Fabriken, in denen wir
mit Hilfe der Sonnenenergie aus den unerschöpflichen Quellen der
Erde von Luft, Wasser und Gesteinen Lebensmittel erzeugen und
verteilen, die nahezu nichts kosten. Die äußere Not ist mit einem
Schlag auch von dem Ärmsten genommen. Die Besitzer des Bodens
können wir ohne Mühe entschädigen. Ich rechne, daß wir für jeden
Menschen in den zivilisierten Staaten – denn diese können
allerdings vorläufig erst in Betracht kommen – im Durchschnitt vier
bis sechs Stunden gewinnen, die er nunmehr allein seiner geistigen
Ausbildung widmen kann. Wir führen unsere Lehrmethoden ein. Die
Menschen sind lernbegierig. Die unmittelbare Zuführung von
Gehirnenergie wird ihnen die neue Anstrengung zur Lust machen. Die
Wahnvorstellungen der Tradition in allen Bevölkerungsklassen werden
verschwinden. Die Rüstungen, die Kriege hören auf. Wir üben in
dieser Hinsicht zunächst einen leichten Zwang aus, bis die bessere
Einsicht durchgedrungen, die bessere Haltung zur Gewohnheit
geworden ist. Denn dies freilich wird notwendig sein; der Mensch
muß zu jeder großen Veränderung erst gezwungen werden, bis er den
Vorteil begreift und das Neue lieben lernt. Ich habe alles schon
mit Ill durchgesprochen.“

„Sie müssen die Menschen besser kennen als ich“, sagte La. „Aber
glauben Sie denn, daß das alles sich ohne Gewalt durchführen
läßt?“

„Ich hoffe es. Wenn aber nicht, so werden wir sie anwenden –“

„O Ell, da sprechen Sie als Mensch – und das ist meine große Sorge
– ihr Menschen werdet uns vergessen machen, daß Gewalt ein Übel
ist, unwürdig –“

Die Klappe des Fernsprechers löste sich.

„Ist La zu Hause?“ fragte Saltners Stimme.

„Ja, ja“, rief La. „Kommen Sie nur. Sie haben sich den ganzen Tag
noch nicht sehen lassen.“

„Ich komme sogleich.“

\section{33 - Fünfhundert Milliarden Steuern}

Eine Minute später trat Saltner ein. Seine Miene verzog sich ein
wenig enttäuscht, als er Ell in lebhaftem Gespräch mit La fand.
Gleich nach der Begrüßung holte er ein Zeitungsblatt hervor.

„Da“, sagte er, „lesen Sie bitte. Wenn die Nume so sind, weiß man
wirklich nicht, ob man lachen soll oder sich entrüsten. Zur
Abwechslung werde ich mich einmal entrüsten. Es ist –“

„Sal, Sal“, rief La lachend, „setzen Sie sich, bitte, ruhig her,
und dann wollen wir sehen, ob wir nicht lieber lachen wollen.“

Sie faßte seine Hand und zog ihn an ihre Seite. „Der Streit der
Planeten soll uns nichts anhaben“, sagte sie leise.

Ell ergriff das Blatt und las:

„Wie wir aus sicherer Quelle erfahren, soll die Ausrüstung des nach
dem Südpol der Erde zu entsendenden Raumschiffs weitere zwanzig bis
dreißig Tage in Anspruch nehmen. Man macht angeblich noch Versuche,
um die Luftschiffe gegen etwaige Angriffe von Menschen
widerstandsfähiger zu machen. Ja, es soll der Bau dieser Schiffe
überhaupt stark im Rückstand sein. Wir finden diese Verzögerung
seitens der Erdkommission unverantwortlich. Die Erregung gegen die
Menschen wächst sichtlich und mit vollem Recht. Man hat aus den
Berichten der Augenzeugen erfahren, daß die Darstellung jenes
Zwischenfalls mit dem englischen Kriegsschiff von der Regierung
viel zu milde gefärbt war. Die den Numen angetane Schmach erfordert
eine schnelle Bestrafung der Schuldigen. Wozu überhaupt diese
Umstände mit dem Erdgesindel?“

„Erdgesindel! Hören Sie!“ rief Saltner, „Da soll doch gleich –“

Ein Händedruck Las hielt ihn auf seinem Platz. „Lesen Sie weiter“,
sagte sie zu Ell.

„Wir haben genaue Informationen über die Verhältnisse auf der Erde
eingezogen. Sie sind geradezu haarsträubend. Von Gerechtigkeit,
Ehrlichkeit, Freiheit haben diese Menschen keine Ahnung. Sie
zerfallen in eine Menge von Einzelstaaten, die untereinander mit
allen Mitteln um die Macht kämpfen. Darunter leidet die
wirtschaftliche Kraft dermaßen, daß viele Millionen im
bedrückendsten Elend leben müssen und die Ruhe nur durch rohe
Gewalt aufrecht erhalten werden kann. Nichtsdestoweniger überbieten
sich die Menschen in Schmeichelei und Unterwürfigkeit gegen die
Machthaber. Jede Bevölkerungsklasse hetzt gegen die andere und
sucht sie zu übervorteilen. Wer sich mit der Wahrheit hervorwagt,
wird von Staats wegen verurteilt oder von seinen Standesgenossen
geächtet. Heuchelei ist überall selbstverständlich. Die Strafen
sind barbarisch, Freiheitsberaubung gilt noch als mild. Morde
kommen alle Tage vor, Diebstähle alle Stunden. Gegen die
sogenannten unzivilisierten Völker scheut man sich nicht, nach
Belieben Massengemetzel in Szene zu setzen. Doch genug hiervon! Und
diese Bande sollen wir als Vernunftwesen anerkennen? Wir meinen, es
ist unsre Pflicht, sie ohne Zaudern zur Raison zu bringen durch die
Mittel, die ihr allein verständlich sind, durch Gewalt. Es sind
wilde Tiere, die wir zu bändigen haben. Denn sie sind um so
gefährlicher, als sie Spuren von Intelligenz besitzen. Leider hat
man sich, wie es scheint, in der Regierung durch einzelne Exemplare
dieser Gesellschaft täuschen lassen, und wir wollen nur hoffen, daß
hierbei bloß ein Irrtum und nicht eine Rücksicht auf gewisse
Beziehungen vorliegt –“

Ell unterbrach sich.

„Das ist denn doch zu arg!“ rief er. „Das sind Verdächtigungen, die
man sich nicht gefallen lassen kann.“

„Meine Befürchtung!“ sagte La. „Die Berührung mit den Menschen
bringt einen Ton in unser Verhalten, wie er sonst im öffentlichen
Leben nicht Sitte war. Nein, Ell, nein, meine lieben Freunde, Sie
sind gewiß nicht daran schuld; es liegt in der Sache selbst – die
antibatische Bewegung setzt eine Verrohung des Gemüts überhaupt
voraus.“

Saltner rieb sich ingrimmig die Hände. „Lesen Sie nur weiter“,
sagte er. „Jetzt haben Sie sich entrüstet, und ich werde wieder
lachen.“

„Wir halten es für sinnlos“, las Ell weiter, „daß zwischen Wilden
wie den Erdbewohnern und zwischen Numen überhaupt eine Verbindung
verwandtschaftlicher Art stattfinden könne. Der Fall Ell bedarf
entschieden einer näheren Untersuchung und Aufklärung. Wir haben
diesen angeblichen Halbnumen noch nicht gesehen. Aber ein richtiges
Exemplar der Menschheit hatten wir zu betrachten das zweifelhafte
Vergnügen. Wer dieses stupide Gesicht mit den blinzelnden Punkten,
die Augen sein sollen, diesen unanständigen, ungefärbten Anzug,
diese rohen Bewegungen einmal gesehen hat, der wird sich sagen,
diese Rasse kann von uns nur als vielleicht nutzbares Haustier
geduldet werden.“

Ell warf das Blatt fort.

La brach in ein herzliches, leises Lachen aus, in das Saltner
einstimmte. Sie trat vor Saltner und nahm seinen Kopf zwischen ihre
Hände. „Ich muß mir doch einmal unser Haustierchen betrachten“,
sagte sie lustig. „Sie sind wirklich ausgezeichnet geschildert.“

Sie sah in seine Augen, ihre Züge wurden ernster, ihr Blick inniger
und tiefer. „Mein lieber, braver Freund“, sagte sie. Sie bog seinen
Kopf zurück und küßte ihn.

Ell lächelte nun auch. „Wenn man so entschädigt wird“, sagte er,
„muß man ja bedauern, nicht auch kräftiger geschildert zu sein.
Aber Sie haben recht, man muß auf dieses dumme Zeug keinen Wert
legen. Trotzdem bin ich froh, daß man Frau Torm wenigstens aus dem
Spiel gelassen hat.“

„Es lohnt sich natürlich nicht, sich darüber zu ärgern“, sagte
Saltner, „nichtsdestoweniger kann das Geschreibe Unheil
anrichten.“

„Dazu ist es doch zu dumm, das nimmt niemand ernsthaft. Man kennt
das Blatt als unzuverlässig.“

„Aber ich habe hier noch etwas anderes, das vielleicht politisch
nicht ohne Einfluß sein dürfte. Ich hörte, daß ähnliche Ansichten
nicht nur in weiten Kreisen geteilt werden, sondern sogar im
Zentralrat Anhänger besitzen. Lesen Sie folgende Vorschläge, die
das neugegründete Blatt, die ›Ba‹, macht.“

Ell nahm das Blatt und las: „Es ist bezeichnend für unsre
Regierung, die sich 144 Luftschiffe für die Erde bewilligen ließ,
daß sie jetzt im entscheidenden Augenblick kein einziges bereit
hat. Aber für die Staaten ist es ein Glück. Die Begeisterung der
Kolonialschwärmer hat Zeit, sich abzukühlen, und diese Abkühlung
schreitet schnell vorwärts. Es wird auffallend still über unsre
Brüder im Sonnensystem, die wir mit der Liebe und Freiheit der Nume
umschließen sollen. Und es ist gut, daß wir zur Besinnung kommen.
Man glaube nur nicht, daß uns die Menschen mit offenen Armen
entgegenkommen werden. Unser Stand wird nicht leicht sein, und
unsre Opfer werden sich höher und höher steigern. Sowohl die
Menschenfreunde als die Antibaten unterschätzen den Widerstand, den
wir zu erwarten haben. Deswegen sollen wir von vornherein klar
sagen, was wir wollen, und dann rücksichtslos handeln, nicht auf
ein Entgegenkommen rechnen, sondern ohne weiteres unsere
Bedingungen mit dem Telelyt und Repulsit diktieren. Es mag sein,
daß die Menschen sich zur Numenheit erziehen lassen, und wir sind
die ersten, welche bereit sind, sie als Brüder anzuerkennen; aber
dies wird uns nur möglich sein, wenn sie sehen, daß jeder
Widerstand aussichtslos ist.“

„Es kommen nun einige Stellen“, sagte Saltner, „die eigentlich
nichts andres verlangen, als was die Regierung selbst wollte,
nämlich warten, bis die Martier überall zugleich losschlagen
können. Aber lesen Sie, bitte, die Vorschläge hier unten.“

„Wir warnen davor, von der Erde zu viel zu erwarten. Wir werden sie
niemals besiedeln können. Die Schwere und die Atmosphäre machen uns
den dauernden Aufenthalt unmöglich. Wir werden immer nur einzelne
Stationen mit wechselnder Besatzung drüben erhalten können. Die
Ausnutzung des Reichtums der Erde muß durch die Menschen für uns
geschehen. Etwa in folgender Weise. Die Gesamtstrahlung der
Sonnenenergie auf die Erde beträgt –“ Ell unterbrach sich.

„Ja“, sagte Saltner, „die Zahlen verstehe ich nicht. Aber es wäre
mir doch ganz interessant zu wissen, wie hoch uns die Herren Nume
eigentlich einschätzen.“

„Ich will sie schnell umrechnen“, rief La. „Es ist ganz leicht. Sie
wissen, unsere Münzeinheit gründet sich auf die Energiemenge, die
von der Sonne während eines Jahres auf die Einheit der Fläche des
Mars ausgestrahlt wird.“

„Gehört hab ich’s schon“, sagte Saltner, „als man mir meinen
›Energieschwamm‹ ausgezahlt hat, aus dem ich alle Tage mein
Taschengeld abzapfe. Aber warum Sie so rechnen, das weiß ich
nicht.“

„Es ist das Einfachste. Einen vergleichbaren Preis mit allen
Kräften der Natur hat doch nur die Arbeit, eine gleichbleibende
Arbeitsmenge können wir leicht mechanisch definieren und
herstellen, und alle Arbeitskraft, die wir zur Verfügung haben,
stammt von der Sonne. Wir fangen die gesamte Sonnenstrahlung auf,
benutzen sie, um eine bestimmte Menge Äther zu kondensieren, und so
besitzen wir eine überall verwertbare Einheit der Arbeit. Die
Sonnenstrahlung haben wir mit der Erde gemeinsam, hier muß sich
also auch eine Vergleichbarkeit unserer Währungen ergeben.“

„Verzeihen Sie“, unterbrach sie Ell, „es besteht dabei noch eine
Schwierigkeit. Ich habe nämlich die Umrechnung schon gemacht, um
ein Urteil über das Budget der Erde aufzustellen. Aber auf der Erde
vermögen wir Menschen nur einen sehr beschränkten Teil der
Sonnenstrahlung, eigentlich nur die Wärme zu verwerten, während Sie
auf dem Mars auch die langwelligen und die kurzwelligen Strahlen,
die gar nicht durch unsere Atmosphäre gehen, in Wärme umwandeln und
daher mitrechnen. Ich muß gestehen, daß ich nicht weiß, wie groß
dieser Betrag ist.“

„Das schlagen wir nach“, sagte La. Sie hatte schon das
physikalische Lexikon ergriffen. „Hier steht es. Wir können
rechnen, daß die Ihnen bekannte Strahlung der Sonne etwa den
zwölften Teil der von uns benutzten beträgt.“

„Dann ist es sehr einfach“, meinte Ell. „Die übrige Umrechnung habe
ich schon früher für mich in Tabellen gebracht. Hier ist sie. Wir
wollen also von den Angaben für den Mars nur ein Zwölftel rechnen.
Dann kommt die Einheit der Sonnenstrahlung auf dem Mars etwa gleich
500.000 Wärmeeinheiten auf der Erde, was ungefähr, soweit sich der
Kohlenpreis fixieren läßt, einem Wert von fünfzig Pfennig
entsprechen dürfte. So – nun will ich Ihnen die Berechnungen gleich
in Mark vorlesen –“

„Hören Sie“, warf Saltner ein, „der Wert einer Wärmeeinheit ist
doch aber sehr schwankend, je nachdem –“

„Ganz gewiß, ich will auch nur zur Bequemlichkeit statt einer
Million Kalorien, was das genaue Maß des Arbeitswertes wäre, der
Anschaulichkeit wegen eine Mark sagen; ein ungefähres Bild der
Größenverhältnisse gibt es doch. Nach meiner Umrechnung also lautet
der Artikel weiter:

›Die Gesamtstrahlung der Sonnenenergie auf die Erde beträgt im
Laufe eines Erdenjahres 3.000 Billionen Mark, wovon aber nur 1.200
bis auf die Erdoberfläche gelangen. Wir können indessen auf der
Erde nur einen relativ viel kleineren Teil mit Strahlungssammlern
besetzen als auf dem Mars, für den Anfang sicher nicht mehr als ein
Prozent. Das gibt eine Billion Mark, die wir durch diese Anlagen
den Menschen jährlich schenken. Allerdings müssen sie dafür
arbeiten, aber die Arbeit wird ihnen reichlich bezahlt, wenn wir
jährlich nur 500.000 Millionen Mark für uns als Steuer
beanspruchen. Sie werden sich immer noch zehnmal besser stehen als
bei ihren bisherigen Hilfsquellen, die ihnen außerdem noch zum
großen Teil bleiben. Außer der Strahlungsenergie können wir uns
noch Luft, Wasser, kohlensauren Kalk und andere Mineralien liefern
lassen. Wir müssen nur die Lieferungen an Arbeit und Stoffen auf
die einzelnen Staaten nach ihrer Bevölkerungszahl verteilen. Es
wird sich empfehlen, dies so zu tun, daß die einzelnen Marsstaaten
sogleich die betreffenden Erdgebiete zugeteilt erhalten, an die sie
sich zu halten haben. Eine Vorschlagsliste gedenken wir demnächst
zu veröffentlichen. Doch müssen wir den Anspruch unseres
Nachbarstaates Berseb, die gesamten Vereinigten Staaten von
Nordamerika für sich zu verlangen, schon heute zurückweisen; wenn
diese große Ländermasse nicht geteilt werden soll, so wäre
jedenfalls unser Hugal als der volkreichste Marsstaat am meisten
berechtigt.‹“

„Sackerment, das nenn ich bescheiden“, sagte Saltner nach einer
Pause. „Fünfhundert Milliarden jährlich, ohne das übrige! Da haben
Sie uns eine schöne Suppe eingebrockt, Meister Ell, mit Ihren
berühmten Numen.“

„Ich bitte Sie, Saltner“, antwortete Ell ärgerlich, „erstens sind
das vage Projekte, auf die nicht viel zu geben ist; und zweitens,
wenn der Mars Revenuen von der Erde zieht, so macht er sich eben
nur für das Kapital und die Arbeit bezahlt, die er für die Kultur
der Erde aufwendet, die Menschheit aber wird davon den größten
Vorteil haben. An dieser meiner Überzeugung können alle die
Auswüchse nichts ändern, die sich natürlich im Anfang einer so
gewaltigen Unternehmung in der Phantasie unserer Landsleute bilden.
Sie müssen sich nicht wundern, daß selbst den Numen der Gedanke zu
Kopf steigt, durch die Erde auf einmal das Zehnfache derjenigen
Energie zur Verfügung zu haben, welche die Sonne unserm Planeten
allein spendet. Denn daß die Martier über die Erde verfügen können,
ist doch nun nicht mehr zu leugnen.“

„Na, darüber ließe sich doch noch Verschiedenes sagen. Ich würde
den ersten martischen Satrapen, der mir meine Million Kalorien
abknöpfen wollte, mir doch erst ein wenig mit meinen Fäusten
betrachten. Darin sind wir halt eigen.“

Ell zuckte die Achseln. „Es wird Ihnen wenig nützen“, sagte er.

„Vielleicht doch“, entgegnete Saltner trocken, „wenn alle so
dächten, oder wenigstens viele. Es könnte nützen. Zunächst denen,
die etwa Lust hätten, sich auf die Seite der Martier zu stellen;
die könnte es zur Besinnung bringen, wenn sie sehen, wie ehrliche
Menschen über die Treue zum Vaterland denken. Und im Notfall mir
selbst. Denn besser ist es, mit ein bissel Repulsit ausgelöscht zu
werden, als unter die Fremdherrschaft sich beugen, und wenn sie
sich noch so sehr mit dem Namen der Freiheit ausstaffiert. Aber wir
wollen uns nicht erhitzen. Darf ich mir ein Pik nehmen?“ sagte er
zu La.

„Wir wollen uns allerdings nicht erhitzen“, erwiderte Ell mit
eisiger Miene. „Darum sollten Sie sich selbst etwas vorsichtiger
ausdrücken. Man könnte auch auf dem Mars fragen, was ein jeder, der
auf seiner Oberfläche wandelt, der Sache der Nume schuldig ist. Und
was den Begriff der Fremdherrschaft anbetrifft, so kommt es doch
ganz darauf an, was man als fremd ansieht. Die Staatsangehörigkeit
jedes einzelnen würde unangetastet bleiben; wenn aber der Staat
selber der Leitung einer höheren Vernunft unterliegt, so würde das
für jeden Bürger nur eine größere bürgerliche Freiheit, einen
weiteren Schritt zur Selbstregierung bedeuten.“

„Die sich in der Freiheit äußern würde, mehr Steuern zu zahlen.
Oder meinen Sie vielleicht, man würde uns das Wahlrecht in den
Marsstaaten oder einen Sitz im Zentralrat gewähren? Man wird uns
immer nur als die Handlanger betrachten, die man vielleicht
anständig füttert und im übrigen nach Belieben gängelt. Aber ein
Haustier bin ich nit und werd ich nit. Ich nit!“

„O ihr Blinden!“ rief Ell. „Seht ihr denn nicht, daß ihr nichts
anderes seid als Sklaven, Sklaven der Natur, der Überlieferung, der
Selbstsucht und eurer eigenen Gesetze, und daß wir kommen, euch zu
befreien, daß ihr nur frei werden könnt durch uns?“

„Ich glaub nicht an die Freiheit, die nicht aus eigner Kraft
kommt.“

„Wir wollen ja nur diese eigne Kraft stärken. Und nun weigert ihr
euch wie ein Kind, das Arznei nehmen soll.“

La hatte schweigend zugehört. Ell hatte sie wiederholt angeblickt,
als wollte er sich ihrer Zustimmung versichern, aber ihre Augen
ruhten auf Saltner. Was er sagte, war ihr aus dem Herzen
gesprochen, sie freute sich des kräftigen Ausdrucks seiner
einfachen, natürlichen Gesinnung, aber durch ihre Seele zog es
schmerzlich. War es nicht eine verlorene Sache, für die er kämpfte?
Das große Schicksal, das über die Planeten rollte, mußte es nicht
diese trotzigen Erdenkinder zermalmen? Ell hatte doch recht, die
Numenheit ist die Vernunft, ist die Freiheit, und ihr Sieg ist
gewiß, wie auch der edle Irrtum des einzelnen sich sträube. Und
dennoch! Was ist denn das Schicksal, wenn nicht die Festigkeit im
ehrlichen Willen der Person? Was ist denn die Freiheit, wenn nicht
der Entschluß, mit dem ein jeder nach seinem besten Wissen und
Gewissen handelt, was ihm auch geschehe? Und welch höhere Freiheit
konnten die Nume geben?

„Nein, Ell“, sagte La jetzt langsam, als Saltner auf Ells letzten
Vergleich nicht antwortete, „nein – nicht wie ein Kind. Saltner hat
wie ein Mann gesprochen. Ein Nume mag es besser verstehen, aber
besser wollen und fühlen kann man nicht. Und ich weiß, er wird auch
so handeln.“

Sie reichte Saltner die Hand. Ihre dunklen Augen schimmerten
feucht, als sie sagte:

„Warum muß es denn zum Streit kommen? Lassen Sie uns alles
versuchen, daß Nume und Menschen Freunde werden. Es ist ja doch nur
notwendig, daß sie sich kennenlernen, ehrlich kennenlernen. Lassen
Sie uns den Irrtum, die Verleumdung bekämpfen, die sich
einzuschleichen drohen. Noch ist es vielleicht Zeit! Nicht wahr,
auch Sie wollen es, Ell?“

„Was könnte ich Höheres wollen?“ erwiderte Ell warm. „Es war der
Wunsch meines ganzen Lebens, die Versöhnung, das Verständnis der
Planeten herbeizuführen, ihre Kulturarbeit zu vereinen. Seit ich
die Nume persönlich kennengelernt habe, ist mein Wunsch lebhafter
als je. Daß die Nume die Überlegenen sind, ist eine Tatsache. Wenn
es zum Kampf kommt, werden die Menschen unterliegen, das folgt
daraus. Daß ich trotzdem in diesem Fall auf der Seite der Nume
stehen würde, ist ebenso natürlich wie der entgegengesetzte
Standpunkt Saltners. Was ich nicht billige, ist nur das Mißtrauen,
mit dem die Menschen uns begegnen, weil sie von einem Teil der
Martier von oben herab behandelt werden. Aber diese Zeitungen sind
doch nicht die Marsstaaten. Ich hoffe wie Sie, daß die
entgegengesetzten Stimmen bald durchdringen werden. Hätte Saltner
andere Blätter gelesen, er wäre sicherlich weniger bitter
gestimmt.“

„Ich habe auch die andern gelesen“, sagte Saltner, „den ganzen
Vormittag habe ich mich mit den Zeitungen herumgeschlagen. Leider
haben sie einen schweren Stand, zu beweisen, daß die Menschen
anständige Leute sind. Was sie für uns sagen können, das müssen
ihnen die Martier halt glauben. Aber was sich gegen uns sagen läßt,
das haben sie in einem einzelnen Fall gesehen. Daran sind die
sakrischen Engländer schuld. Aber auch die beiden vorlauten
Matrosen vom Luftschiff und ihre Helfershelfer, die die Sache im
Theater aufgebauscht haben. Dagegen müßte die Regierung mehr tun,
als die bloße Berichtigung loslassen, die heute in den Zeitungen
steht.“

„Es wird auch geschehen“, sagte Ell. „Ich will eben deshalb jetzt
zu Ill, der gestern in Erwägung zog, ob sich nicht ermitteln lasse,
wie die Engländer dazu gekommen sind, unsere Leute anzugreifen.
Vielleicht lag nur ein Mißverständnis vor. Und wenn sich das
beweisen ließe, wenn sich außerdem zeigte, daß die Darstellung im
Theater und so weiter übertrieben ist, so wird die Gerechtigkeit
bei den Martiern siegen.“

„Wie wollen Sie das nachweisen, da Sie keine andern Zeugen haben
als die beiden Martier, von denen ich gar nicht behaupten will, daß
sie absichtlich übertreiben, die aber in ihrer Bedrängnis nicht
objektiv urteilen können?“

„Es käme darauf an zu sehen, was an dem Cairn – an dem Steinmann,
den die Engländer errichtet hatten – eigentlich vorging bis zu dem
Augenblick, in welchem die Seeleute dem Offizier zur Hilfe kamen.
Auch wäre es sehr gut, wenn unsere Landsleute sich durch den
Augenschein überzeugen könnten, wie europäische Matrosen und ein
europäisches Kriegsschiff eigentlich aussehen –“

„Das ist wahr“, sagte Saltner. „Am Ende ginge ihnen doch ein Licht
auf, daß die Menschen keine Wilden sind, mit denen zu spaßen ist.
Aber wie sollte so ein Nachweis möglich sein über einen Vorgang,
der in der Öde des Kennedy-Kanals vor Wochen stattgefunden hat?“

„Durch das Retrospektiv.“

Saltner machte ein erstauntes Gesicht.

„Das ist ein glücklicher Gedanke“, rief La.

„Ich habe dabei gar keinen Gedanken“, sagte Saltner
kopfschüttelnd.

La erklärte das Verfahren. Saltner wurde wieder kleinlaut. Bedrückt
setzte er sich nieder und murmelte für sich hin:

„Medizin! Wir sind ja doch arme Rothäute!“

Ell verabschiedete sich.

„Wenn es noch zur Anwendung des Retrospektivs kommt“, sagte La,
„dann müssen Sie mir aber einen guten Platz verschaffen.“

„Ich wäre glücklich, Ihnen gefällig sein zu können.“

Ell sprach es wärmer als gewöhnlich und ließ seinen Blick lange auf
La ruhen, die ihn lächelnd ansah. Dann ging er.

La wendete sich zu Saltner. Sie faßte seine Arme und blickte ihn
an. „Wie bin ich froh, daß ich dich hier habe, du geliebter
Mensch!“ sagte sie.

\section{34 - Das Retrospektiv}

Die Rüstungen der Martier für ihren Zug nach der Erde waren darauf
berechnet gewesen, sobald das Frühjahr für die Nordhalbkugel der
Erde gekommen sei, sich gleichzeitig mit ihren Luftschiffen über
sämtliche Hauptstädte der einflußreicheren Staaten zu legen und die
Regierungen zu zwingen, die vom Mars zu diktierenden Bedingungen
ohne weiteres anzunehmen. Es sollte dann unter einer Art
Protektorat der Marsstaaten den Erdbewohnern die Kultur der Martier
zugänglich gemacht werden, und man wollte abwarten, in welcher
Weise sich die Marsstaaten am besten aus den alten und neuen
Hilfsmitteln der Erde würden schadlos halten können.

Jetzt auf einmal sollte sofort und unter veränderten Umständen eine
Expedition abgesandt werden. Man hatte die Erfahrung gemacht, daß
die Erdbewohner vermutlich Widerstand leisten würden und daß sie
nicht ungefährliche Mittel der Verteidigung besaßen. Man konnte nur
wenige Luftschiffe auf einmal nach der Erde transportieren und
mußte darauf gefaßt sein, statt einfach ein Protektorat zu
erklären, in einen Krieg mit England, vielleicht mit der ganzen
Erde verwickelt zu werden.

Daher hatte Ill alle Ursache, in seinen Entschlüssen und Handlungen
sich nicht zu überstürzen. Je länger sich die Aktion gegen England
hinzog, um so eher konnte er hoffen, eine genügende Macht auf der
Erde zu versammeln, um nach dem ursprünglichen Plan eine Besetzung
aller Kulturstaaten sofort anzuschließen. Da sich die Planeten
jetzt voneinander entfernten, nahm die Reise immer längere Zeit in
Anspruch. Wenn sich die Absendung des Raumschiffs noch um einen
Monat verzögerte, so mußte wenigstens ein zweiter Monat ablaufen,
ehe es nach der Erde gelangte. Aber auch dann wollte er nicht
sogleich vorgehen, sondern zunächst weitere Verstärkungen abwarten.
Etwa im Januar – nach irdischer Rechnung – hoffte er stark genug zu
sein, seinen Forderungen den nötigen Nachdruck zu geben. Ließen
sich nun die Verhandlungen mit England noch einige Zeit
verschleppen, so hatte sich inzwischen die Raumschiffflotte auf der
Außenstation des Nordpols eingefunden, und Mitte März konnte man
dort mit der Indienststellung der Luftschiffe beginnen.

Ill hatte aber auch noch andere Gründe, die Absendung des
Raumschiffs nach dem Südpol zu verzögern. Es hatte sich ja gezeigt,
daß die Luftschiffe vor den Waffen der Erdbewohner keinen
genügenden Schutz besaßen. Einen solchen galt es erst herzustellen.
Wenn es gelang, die Luftschiffe gegen Geschosse jeder Art aus
irdischen Geschützen widerstandsfähig zu machen, so war dadurch der
Erfolg gesichert. Versuche darüber konnten erst jetzt angestellt
werden, nachdem man die Wirkungsart der Repetiergewehre und der
großen Marinegeschütze kennengelernt hatte. Und in dieser Hinsicht
war man einer neuen Entdeckung von ganz erstaunlicher Wirkung auf
der Spur. Dieses Argument schlug durch. Die oppositionellen
Parteien im Parlament wie in der Presse beruhigten sich darüber,
daß die Absendung der Expedition sich verzögerte. Die Wichtigkeit
der technischen Vervollkommnung der Luftschiffe leuchtete ebenso
ein wie die Schuldlosigkeit der Regierung, daß diese Erfahrungen
nicht früher gemacht werden konnten. Sobald es sich überhaupt um
die Lösung einer wichtigen technischen Aufgabe handelte, gab es
keine Parteikämpfe mehr. Dann waren alle einig, und alles Interesse
konzentrierte sich darauf. Da war ein Ehrenpunkt für jeden Martier
berührt, und das technische Problem drängte alle anderen Fragen in
den Hintergrund.

So kam es, daß die Hetze gegen die Erdbewohner und die zahllosen
Pläne über die Ausnutzung der Erde nach wenigen Wochen verstummten
und wieder eine ruhigere Auffassung Platz griff. Doch die Regierung
ließ sich dadurch nicht täuschen. Es war kein Zweifel, daß ähnliche
Gesinnungen wieder hervortreten würden, ja daß sie sich zu einem
chauvinistischen Übermut verdichten würden, sobald es feststand,
daß die martischen Schiffe durch menschliche Waffen unverletzbar
seien. Die Gefahr lag vor, daß der Gegensatz zwischen beiden
Planeten dann zu einer Vergewaltigung der Erde führen, daß die
Regierung zu kriegerischen Maßregeln gegen die ohnmächtigen
Menschen gedrängt werden könnte. Der Zentralrat war jedoch, in
voller Übereinstimmung mit Ill, fest gewillt, dies zu vermeiden und
die Würde der Numenheit in den Verhandlungen mit der Erde zu
wahren, indem er die Übermacht der Martier nur benutzen wollte,
Feindseligkeiten der Menschen ihrerseits unmöglich zu machen und
dadurch das friedliche Zusammenwirken der Planeten zu erzielen. Es
wurde daher versucht, ein Gesetz durchzubringen, das von vornherein
den Menschen die Freiheit der Persönlichkeit garantieren sollte,
indem es sie als Vernunftwesen erklärte. Doch war eine starke
antibatische Opposition dagegen vorhanden, und auch die
gemäßigteren Parteien erklärten, daß zuvor die Angelegenheit mit
England geordnet sein müsse.

Man bestrebte sich jetzt von seiten der Regierung wie der
Philobaten – so übersetzte Ell die Bezeichnung der
menschenfreundlichen Richtung –, nach Möglichkeit bessere Ansichten
über die Erdbewohner zu verbreiten. Dahin gehörte auch die
Aufklärung des Zwischenfalls mit dem englischen Kriegsschiff.
Namentlich war es für die beabsichtigten Unterhandlungen mit
England wichtig und erforderlich, genau aus eigenen Quellen zu
wissen, was am Cairn vorgegangen sei, womöglich auch, was aus dem
Kriegsschiff geworden. Infolgedessen entschloß sich der Zentralrat,
auf Antrag von Ill, einen Versuch mit dem Retrospektiv zu machen.

Die Einstellung des Apparates, um durch ihn ein bestimmtes Ereignis
in der Vergangenheit wieder zu erblicken, bedurfte einer längeren
Vorbereitung. Es war schwierig, genau die Richtung zu ermitteln, in
welcher die Achse des Kegels von Gravitationsstrahlen liegen mußte,
den man aussandte, um das zur Zeit des Ereignisses vom Planeten
zurückgestrahlte Licht auf seinem Weg durch den Weltraum einzuholen
und wieder zurückzubringen. Es kam dabei eine Menge von
Einzelheiten in Betracht, welche mehrtägige theoretische
Untersuchungen und langwierige Rechnungen erforderten. Alsdann
bedurfte es noch praktischer Versuche, um die passendste
Einstellung zu finden und zu korrigieren. Nachdem die
zurückkehrenden Gravitationswellen wieder in Licht verwandelt
worden waren und das optische Relais passiert hatten, erschien
endlich das Bild der aufgesuchten Gegend in einem völlig
verdunkelten Zimmer auf eine Tafel projiziert. Handelte es sich
nicht nur um eine Schaustellung, sondern um Konstatierung von
Tatsachen, so wurde das Bild, das sich natürlich fortwährend
veränderte, indem es den ganzen Verlauf des beobachteten
Ereignisses darstellte, durch eine ununterbrochene Folge von
Momentphotographien fixiert, die später im Kinematograph wieder in
ihrer lebendigen Folge betrachtet werden konnten. Die
Schwierigkeiten des Versuchs waren nun im vorliegenden Fall in noch
viel höherem Grad als sonst vorhanden, da man ein Ereignis
betrachten wollte, das sich auf einem andern Planeten vollzogen
hatte, und da man außerdem beabsichtigte, den Schauplatz, der
Bewegung des Schiffes folgend, zu wechseln. Es war das erste Mal,
daß man das Retrospektiv auf einen so komplizierten Fall anwendete,
und man durfte nicht erwarten, daß alle Teile des Versuchs
gleichmäßig oder überhaupt glücken würden. Das Experiment selbst
sollte daher nicht öffentlich sein. Es konnte nachträglich
wiederholt werden, in jedem Fall gaben die bewegten
Momentphotographien ein unwiderlegbares Protokoll über die
Beobachtungen, das jedermann zugänglich gemacht werden konnte.

\tb

Isma verzeichnete in ihrem Tagebuch bereits den 18. Oktober. Sie
mußte erst einige Zeit in ihrem Gedächtnis nachrechnen, ehe sie
sich des Datums vergewisserte, denn in den letzten Tagen hatte sie
keinerlei Aufzeichnungen gemacht. Sie fühlte sich sehr
niedergeschlagen. Zu ihren Besorgnissen kam eine körperliche
Verstimmung infolge der Veränderung ihrer Lebensverhältnisse.
Einige Tage hatte ihre Schwäche sie so überwältigt, daß sie ihr
Zimmer nicht verlassen konnte. Ihre Gastfreunde waren in
liebevollster Weise um sie besorgt und hatten sogar Hil den weiten
Weg von seinem Wohnort nach Kla machen lassen, um diesen besten
Kenner der menschlichen Konstitution auf dem Mars zu Rate zu
ziehen. Er hatte angeordnet, daß für Isma ein besonderer Apparat
gebaut werde, um die normalen Verhältnisse der Erde in Schwere und
Luftdruck für sie herzustellen; und seitdem sie sich die Nacht über
und einige Stunden des Tages in diesem künstlichen Erdklima
aufhielt, hatte sich ihr Zustand gebessert, und ihre Kräfte waren
wieder gestiegen.

Obwohl ihre Gastfreunde und befreundete Familien derselben, vor
allem La, sie in jeder Weise aufzuheitern suchten, obwohl sie
manchmal über Saltners harmlose Spöttereien und die Schilderungen
seiner Abenteuer auf dem Mars herzlich lachen mußte, zählte sie
doch sehnsüchtig die Stunden, in denen Ell bei ihr erschien. Er
hatte sie täglich aufgesucht und während ihrer Erkrankung, so oft
ihr Zustand es gestattete, sich durch den Fernsprecher mit ihr
unterhalten. Sein Verhalten gegen sie war stets unverändert
freundschaftlich und teilnehmend geblieben, sie hatte keine der
kleinen Aufmerksamkeiten vermißt, mit denen er sie seit Jahren
verwöhnt hatte. Ihre Wünsche suchte er zu erraten, fast nie kam er,
ohne ihr irgend etwas mitzubringen, von dem er glaubte, daß es sie
interessieren würde – einen Artikel in den Zeitungen, eine
Abbildung oder eine der tausend unterhaltenden Neuigkeiten der
Marsindustrie, und wenn er sie erblickte, ruhten seine Augen mit
der alten, zärtlichen Anhänglichkeit auf ihr. Sie hätte nicht sagen
können, worüber sie sich beklagen dürfte. Und dennoch – sie konnte
sich eines schmerzlichen Gefühls nicht erwehren, als wäre eine
Entfremdung zwischen ihr und dem Freund eingetreten. In seiner
Anwesenheit verschwand es, aber wenn er fort war, stellte es sich
wieder ein. Sie quälte sich selbst mit Grübeleien darüber, was sie
ihm vorzuwerfen habe.

Warum konnte er so gar nichts darin durchsetzen, daß ihr die
Erlaubnis erteilt werde, mit dem Raumschiff nach dem Südpol der
Erde abzureisen? Ihre Bitte war von Ill mit Bedauern, aber
entschieden abgeschlagen worden; die Verhältnisse gestatteten es
nicht. Ell hatte sich vergeblich für sie verwandt; man hatte
erklärt, so lange man sich in einer Art feindseligen Zustandes zur
Erde befinde, sei es nicht zulässig, daß einer der Erdbewohner
entlassen werde. Aber als Ell einmal in ihrer Gegenwart seinem
Oheim gegenüber aufs lebhafteste für ihre Zurücksendung nach der
Erde eingetreten war, hatte sie sich wieder durch den Eifer
verletzt gefühlt, mit dem er bemüht war, sie fortzuschaffen. Und er
wollte auf dem Mars bleiben. Es war gar keine Rede davon gewesen,
daß er sie begleite. Und jetzt wäre doch sein Platz auf der Erde
gewesen, jetzt hätte er zur Versöhnung tätig sein müssen! Was hielt
ihn auf dem Mars zurück?

Sie glaubte, es wohl zu wissen. Warum sprach er anfänglich so viel
und mit solcher Wärme und Bewunderung von La? Und jetzt suchte er
ihren Namen zu vermeiden. Was war zwischen ihn und Saltner
getreten, daß sie sich so kühl und förmlich begegneten, wo sie doch
mehr als je auf sich angewiesen waren? Und wenn Ell mit La bei ihr
zusammentraf, wie seltsam pflegte er sie anzusehen! Sie kannte
diesen Blick. Und warum sprach er manchmal so schnell und eifrig zu
La in ihrer Sprache, daß sie der Unterhaltung nicht zu folgen
vermochte? Sie mochte die beiden nicht zusammen sehen. Ein Gefühl
der Kälte durchzog ihre Seele und machte sie feindselig und
unwirsch gegen La wie gegen Ell. Es war ja nichts, das sie ihnen
vorwerfen konnte, und doch war ihr dieser Verkehr unbehaglich und
verstimmte sie. Wenn dann La gegangen war und Ell noch zurückblieb,
wenn er dann mit derselben Herzlichkeit zu ihr sprach, die sie eben
auch La gegenüber in seinem Ton gehört zu haben glaubte, so stieg
es wie Zorn in ihr auf, als wäre ihr etwas genommen, das ihr allein
gebührte. Sie war dann unfreundlich gegen Ell, sie machte ihm
Vorwürfe, die sie nicht verantworten konnte, und wenn er fort war,
bereute sie ihre Worte, ihre Blicke und schalt sich undankbar und
schlecht.

Ach, sie kannte auch diesen Zustand, dieses Gefühl der
Unzufriedenheit. Und sie konnte es doch nicht ändern. Es war
jedesmal so gewesen, wenn Ell an einer anderen Gefallen gefunden
hatte. Sie sagte sich selbst, wie töricht sie sei. Sie hatte jedes
Recht auf ihn aufgegeben, sie hatte es zur Bedingung ihrer
Freundschaft erhoben, daß er sich keine Hoffnungen mache, mehr von
ihr zu besitzen als diese Freundschaft. Wie durfte sie ihm
verwehren, eine andere zu lieben, da sie selbst verzichtet hatte?
Und doch, jedesmal, wenn diese Gefahr zu drohen schien, fühlte sie
sich von Eifersucht ergriffen, die sie sich nicht gestehen wollte
und die sie doch ohne ihren Willen ihm durch ihr Benehmen
eingestand. Warum auch mußte er ihr das jetzt antun, wo sie fremd
auf fremdem Planeten, eine Gefangene, krank und einsam weilen
mußte, wo er der einzige war, der sie verstehen konnte? Warum mußte
er jetzt –? Aber was warf sie ihm denn vor? Warum war sie selbst
nicht besser? Warum sagte sie ihm denn nicht, hier, frei von allen
Menschensatzungen, daß sie nicht ohne ihn sein wolle, daß sie ihn
nicht entbehren wolle, nicht könne? Warum? Weil sie ihn ja doch
nicht lieben wollte! Und warum konnte sie sich nicht von ihm
losreißen, da sie doch ihren Mann liebte, da sie ausgezogen war,
ihn zu suchen in den Öden der Polarnacht, und da sie zu ihm
zurückwollte durch die Leere des Weltraums? Und wenn Torm nicht
mehr war? Wenn sie zurückkam nach Friedau und er verschollen war,
ein Opfer der Forschung, wie so viele vor ihm? Wenn sie dann
verlassen war, hier wie dort? Sie ließ die Feder sinken und legte
den Kopf in ihre Hände. Ach, daß es kein Zeichen von ihm gab, keine
Nachricht! Und daß sie hier sitzen mußte, nicht mehr Tausende,
sondern Millionen von Meilen von ihm getrennt, und angewiesen auf
den Freund, der um ihretwillen gegangen war, allein mit ihm –
gerade alles, was sie hatte vermeiden wollen! Gerade in diese
Gefahr hatte sie sich gestürzt, der sie zu entfliehen gedachte. Und
sie sah sie vor sich, leibhaftig, jeden Tag in den großen treuen
Augen, die sie teilnehmend ansahen –. Ach, darum quälte sie ihn ja,
quälte sie sich –

Aber wäre es in Friedau besser gewesen, wenn sie nun doch von ihrem
Mann nichts erfahren konnte? Eines wenigstens war sie los, die
fortwährenden Fragen, die teilnehmend sein sollten und doch so
heuchlerisch waren, die hämischen Blicke, die widerwärtigen,
kleinlichen, schamlosen Klatschereien – –

Aus ihrem Nachsinnen weckte sie der Ton, der den Eintritt eines
Besuches durch das Gartentor meldete. Sie hörte den Wagen vor der
Veranda halten. Das war Ells Stimme, er sprach mit Ma. Isma strich
über ihr Haar, sie warf einen Blick in den Spiegel und ärgerte sich
über ihre Erregung. Gleich darauf trat Ell ein. Sie ging ihm
lächelnd entgegen. Er blickte sie an.

„Es geht Ihnen gut“, sagte er freudig. „Sie sehen wieder frisch und
kräftig aus.“

Er hielt ihre Hände. In ihren Augen las er ihre Freude. Es war
einer der Tage, an dem sie nicht verbergen konnte, wie lieb er ihr
war. „Ich weiß nicht“, sagte sie. „Es geht mir eigentlich gar nicht
gut. Ich kann von meinen Gedanken nicht loskommen.“

„Dann kommen Sie mit mir, Isma. Sie sollen etwas sehen, worauf wir
schon lange warten. Das Retrospektiv ist glücklich eingestellt –
der Cairn ist gefunden –“

„Ach, Ell! Noch einmal die schreckliche Geschichte! Ich bin ja
leider dabei gewesen. Soll ich sie wirklich noch einmal sehen?“

„Ich dachte, der Triumph der Technik würde Sie interessieren. In
die Vergangenheit zu blicken –“

Isma wollte eine abweisende Antwort geben. Aber sie sah, daß es Ell
erfreuen würde, wenn sie ihn begleitete. Sie wollte gut zu ihm
sein, sie wollte ihm nichts abschlagen.

„Nun denn“, sagte sie, „wenn es Ihnen lieb ist – kommen Sie. Es ist
doch etwas Neues in der Form, wenn auch nicht im Stoff. Ich habe
aber schon vor einigen Tagen den Platz abgelehnt, den Ihr Oheim mir
anzubieten die Güte hatte.“

„Ich habe noch einen für Sie reserviert, allerdings etwas mehr im
Hintergrund, wo La und Saltner sitzen, und wer sonst Verbindungen
mit der Erdkommission hat. Sie wissen, es handelt sich nur um einen
Versuch; außer dem Zentralrat, den Kommissionsmitgliedern und dem
Präsidium des Parlaments sind nur einige Vertreter der Presse da.
Aber unsre Plätze sind auch gut, und mit diesem Glas, daß ich Ihnen
mitgebracht habe, können Sie sicherlich die einzelnen Personen
erkennen – wenn wir sie überhaupt zu Gesicht bekommen. Allerdings
wird das Bild etwas aus der Vogelperspektive erscheinen, doch hat
man den Neigungswinkel so günstig genommen, als es die
atmosphärischen Verhältnisse nur immer gestatteten. Ich habe den
›Steinmann‹ vor mir gesehen wie von einem niedrig schwebenden
Luftschiff aus.“

Isma schwieg ein Weilchen. Also La war natürlich auch da. Sie
verdrängte den aufsteigenden Gedanken und sagte:

„Aber ich begreife nicht – wenn man so deutliche Bilder aus so
riesigen Entfernungen erzielen kann, warum betrachten Sie denn
nicht die Erde direkt, warum können wir nicht einmal nach Friedau,
nach unserm Haus sehen?“

„Mit Hilfe des Retrospektivs ginge das wohl an, aber Sie können
nicht verlangen, daß man dieses äußerst schwierige, zeitraubende
und kostspielige Experiment anstellt, um irgendeine Neugier zu
befriedigen. Was sollte Ihnen das nützen? Was wollte man damit
erfahren? Und selbst wenn eine Zeitung zufällig irgendwo
aufgeschlagen läge, mit neuen Nachrichten über die Verhältnisse auf
der Erde, und sie erschiene im Retrospektiv, so geht die
Deutlichkeit doch nicht so weit, daß wir sie lesen könnten.“

„Und mit Ihren Fernrohren können Sie so genau nicht sehen, daß Sie
Menschen auf der Erde erkennen könnten?“

„Das ist unmöglich. Beim Fernrohr haben wir mit den Lichtwellen zu
tun, da bekommen wir auf so riesige Entfernungen keine erkennbaren
Bilder von so kleinen Gegenständen. Das geht nur mit Hilfe der
Gravitationswellen. Sie müssen bedenken, daß es die
Gravitationsschwingungen sind, durch welche wir die ganze, vom zu
beobachtenden Ereignis ausgegangene Bewegung zurückbringen, und daß
die Umwandlung in Licht erst hier, innerhalb des Apparates,
geschieht. Da bilden sich wieder dieselben Schwingungen, wie sie
bei der Aussendung waren, abgesehen von den Störungen, die
inzwischen durch äußere Verhältnisse eingetreten sind. Wenn zum
Beispiel das Licht auf seinem Weg durch den Weltraum einen
Meteorschwarm passiert hatte, so erhalten wir kein deutliches Bild
mehr. Fernrohr und Retrospektiv verhalten sich etwa wie ein
Sprachrohr und ein Telephon. Direkt können Sie die Schallwellen
nicht weit senden, mit dem Telephon aber können Sie sprechen, so
weit die elektrischen Schwingungen reichen.“

Isma hatte sich inzwischen zu ihrem Weg zurecht gemacht. Ill und
seine Frau befanden sich schon im Retrospektiv-Gebäude. Eine halbe
Stunde später hatten auch Isma und Ell ihre Plätze eingenommen. La
und Saltner waren kurz zuvor gekommen.

Der große Saal war vollständig verdunkelt, trotzdem konnte man sich
in ihm unschwer zurechtfinden und die in der Nähe sitzenden
Anwesenden erkennen. Denn das erleuchtete Bild, von welchem das
Licht im Saal ausging, nahm an der einen Wand einen Kreis von zehn
Metern Durchmesser ein und erhellte dadurch die Umgebung. Es
stellte die Gegend an der Küste von Grinnell-Land dar, welche der
Schauplatz des englisch-martischen Konflikts gewesen war. Deutlich
erkannte man ziemlich in der Mitte des Bildes die Gruppe der beiden
englischen Matrosen, welche unter Leitung des Leutnants Prim mit
der Errichtung des Cairns beschäftigt waren. Es war überraschend zu
sehen, wie die etwa spannenlangen Figuren sich lebhaft
durcheinander bewegten. Die Deutlichkeit des Bildes wechselte,
mitunter erschien diese, dann jene Stelle etwas verschwommen,
mitunter verdunkelte sich ein ganzer Streifen, im allgemeinen waren
jedoch selbst Einzelheiten deutlich zu erkennen. Mit ihrem Glas
konnte sich Isma die Gestalten der Engländer so nahe bringen, daß
sie in dem Offizier denselben Mann wiedererkannte, den sie durch
ihr Fernrohr auf dem Verdeck des Kriegsschiffs gesehen hatte.

Da man den Apparat auf ein und dieselbe Stelle des Weltraums
eingestellt hielt und nur nach der sich verändernden Lage der
beiden Planeten regulierte, so gab das Bild den Verlauf der
Ereignisse in dem gleichen Zeitmaß wieder, wie er sich in
Wirklichkeit vollzogen hatte. Man befand sich offenbar noch am
Morgen, und wenn der ganze Tag in seinem Geschehen verfolgt werden
sollte, so stand eine lange und ermüdende Sitzung in Aussicht. Die
eintönige Arbeit der Engländer begann schon etwas langweilig zu
werden, und Saltner sagte zu La:

„Merkwürdig ist ja die Geschichte, und immerhin können die Herrn
Nume hier schon sehen, daß die Englishmen doch nicht ganz so wild
sind wie auf ihrem Theater. Aber läßt sich denn die Sache nicht ein
bissel beschleunigen?“

„Das geht allerdings“, antwortete Ell, der auf der andern Seite von
La saß, „und man wird es wohl nachher auch tun. Man braucht nur den
Apparat allmählich auf näher gelegene Stellen des Raumes zu
richten, so fängt man die Lichtstrahlen in immer früheren
Zeitmomenten ab und bewirkt dadurch, daß alles viel schneller zu
geschehen scheint. Aber es treten dabei andere Schwierigkeiten auf.
Und jetzt ist es nicht möglich, weil jeden Augenblick der
entscheidende Moment eintreten kann. Wir müssen uns also in Geduld
fassen.“

Es dauerte nicht mehr lange, so verstummte die Unterhaltung, denn
man sah, wie der Leutnant den Cairn verließ und den benachbarten
Hügel bestieg. Man konnte auch erkennen, daß er mit dem Feldstecher
nach einer bestimmten Richtung blickte. Es zeigten sich nun alle
die Ereignisse, wie sie sich abgespielt hatten. Unter lautloser
Spannung sah man die Matrosen sich entfernen, man sah mit Hilfe
einer kleinen Verschiebung des Bildes, wie sie verunglückten und
von den Martiern gerettet wurden, man sah den ganzen Konflikt sich
entwickeln – –

Die Martier waren von dem Versuch sehr befriedigt, da sich nun eine
Erklärung des Mißverständnisses ergab. Die Engländer hatten die
Martier in der Tat für Feinde halten müssen.

Man verfolgte das Schicksal der Gefangenen, bis sie auf dem
Kriegsschiff unter Deck gebracht worden waren. Es war nun nichts
mehr zu beobachten, da man wußte, daß man die Gefangenen nicht
wieder erblicken konnte bis zu dem Moment ihrer Auslieferung. Diese
achtzehn Stunden hindurch den Lauf des Kriegsschiffs und seinen
Kampf mit dem Luftschiff zu verfolgen, hatte kein Interesse für die
vorliegende Frage. Dagegen wollte man gern wissen, was aus der
›Prevention‹ nach ihrer Niederlage geworden sei. Es war daher
beschlossen worden, durch eine Umstellung des Apparats diese später
liegenden Ereignisse zu beobachten. Während der Vorbereitungen
hierzu, die einige Stunden in Anspruch nahmen, verließen die
Zuschauer den Saal. Isma erfuhr, daß erst in den Abendstunden die
Fortsetzung des Versuchs zu erwarten sei.

Saltner und Isma, ebenso wie Ell, brauchten daher ihre gewöhnliche
Tagesbeschäftigung nicht abzusagen, wie sie ursprünglich
beabsichtigt hatten. Diese bestand darin, daß sie auf Ersuchen der
Regierung es übernommen hatten, täglich einige Stunden mit dazu
ausgewählten höheren Beamten das Studium der wichtigsten
europäischen Sprachen zu treiben. Außer dem Deutschen hatte Ell den
Unterricht im Englischen, Saltner im Italienischen und Isma im
Französischen übernommen, den sie nur während ihrer Erkrankung
einige Zeit hatte aussetzen müssen.

Gegen Abend wurde Isma von Ell mit der Nachricht angesprochen, daß
der Apparat wieder eingestellt und das Kriegsschiff aufgefunden
sei. Man räume eifrig auf demselben auf, um die erlittenen
Beschädigungen zu beseitigen, und es scheinen daß das Schiff seine
Fahrt wieder aufnehmen wolle. Als Isma im Saal des
Retrospektivgebäudes erschien, zeigte indessen das Bild nur einen
Teil des Meeres und des felsigen Ufers; von einem Schiff war nichts
zu sehen. Sie hörte, daß es seinen Kurs nach Süden fortgesetzt
habe, dabei aber dem Gesichtskreis entschwunden sei. Infolge einer
vorübergehenden Trübung war es noch nicht gelungen, das Schiff
wieder aufzufinden. Jetzt war das Bild wieder hell, und in dem
Bemühen, das englische Schiff zu entdecken, ließ man die Fläche der
Bai und die Felsenufer vorüberziehen. Bald blickte man auf
treibende Schollen, bald in Buchten und Fjorde hinein.

Isma kam es vor, als befände sie sich wieder an Bord des
Luftschiffes und durchspähte die Gegend, der sie so schnell
entzogen worden war, nach Spuren von Hugo – –

Vielleicht war er gar nicht so weit von der Stelle entfernt, die
sie jetzt vor Augen hatte, vielleicht verdeckte nur jener
Berggipfel das Lager der Eskimos, bei denen ihr Mann weilte! Und da
– nein – ja doch – das war doch ein Boot, zwei, drei Boote, was
dort in dem Kanal unter dem Ufer sich bewegte –

Isma ergriff krampfhaft Ells Arm. „Sehen Sie doch – sehen Sie nicht
dort –?“

„Wahrhaftig“, rief Ell, „es sind Boote, Umiaks, sogenannte
Weiberboote der Eskimos. Sie scheinen mehrere Familien mit ihren
Habseligkeiten zu tragen. Man wird gewiß das Bild festhalten –“

In der Tat stand die Landschaft jetzt still, man wollte die Boote
betrachten, aber die Verschiebung war doch so weit gegangen, daß
sie schon durch das höhere Ufer verdeckt waren.

Dicht daneben zeigte das Bild das freie Wasser der Bai, in welche
der schmale Kanal mündete. Man erwartete, daß die Boote dort zum
Vorschein kommen müßten. Bis dahin wollten die Beobachter das
freiere Fahrwasser der Umgegend absuchen. Das Bild bewegte sich
wieder, man sah nur Meer – und da – am Rand des Lichtkreises
bewegte sich etwas Dunkles – es war das Kriegsschiff.

Bis jetzt hatte man ein größeres Gesichtsfeld angewendet, um einen
weiteren Umblick zu haben. Nun kam es darauf an, stärkere
Vergrößerung zu gewinnen, dabei mußte sich das Gesichtsfeld
einschränken. Man sah jetzt, allerdings so deutlich, daß man die
Stellung der Matrosen erkennen konnte, nur das Schiff und seine
nächste Umgebung; mit dem Glas konnte man den Kapitän und den
Leutnant Prim erkennen, der seine Hände, wie zur Übung, langsam hin
und her bewegte. Man bemerkte, daß eine Meldung gemacht wurde und
sich die Geschwindigkeit des Schiffes, dem der Apparat mit
wunderbarer Präzision und nur geringen Schwankungen folgte,
verringerte. Ein Boot wurde herniedergelassen. Die Ingenieure des
Retrospektivs waren zweifelhaft, ob sie dem Boot folgen oder das
Schiff im Auge behalten sollten. Das erstere wurde beschlossen, da
das Boot ja jedenfalls zum Schiff zurückkehren mußte. Alsbald war
nur noch das rasch rudernde, mit acht Matrosen bemannte Boot auf
der Wasserfläche zu sehen. Da erschien ein zweites Boot, ihm
entgegenfahrend. Man winkte von diesem aus. Die Fahrzeuge näherten
sich rasch, das fremde war jetzt deutlich als grönländischer Umiak
zu erkennen. An der Spitze desselben richtete sich ein Mann empor
und schwenkte seine Mütze – ein blonder Vollbart umrahmte das weiße
Gesicht – er war kein Eskimo –

„Hugo!“ gellte eine Stimme laut durch den Saal. Die Martier
blickten erstaunt auf, sie wußten nicht, was das bedeute.

„Es ist Torm!“ rief Ell erklärend zu Ill hinüber, indem er die
zusammensinkende Isma in seinem Arm auffing.

\section{35 - Die Rente des Mars}

„Es geht nicht, Saltner, es geht nicht!“

Ell legte den Brief in Saltners Hand zurück. Der kleine,
verschlossene Umschlag trug, von Ismas zierlicher Hand geschrieben,
die Adresse Torms.

„Ich darf es nicht“, sagte Ell noch einmal, als Saltner nicht
antwortete.

„Auch nicht, wenn Frau Torm Ihnen versichert, daß der Brief keine
politischen, keine auf die Operationen und Absichten der Martier
bezüglichen Mitteilungen enthält?“

„Auch dann nicht. Wir dürfen keinerlei Briefe von Erdbewohnern mit
diesem Schiff nach der Erde befördern, die dem Kommando nicht offen
eingereicht werden. Frau Torm verlangt, Sie verlangen von mir, daß
ich die Möglichkeit schaffe, diesen Brief heimlich nach der Erde zu
bringen. Sie verlangen etwas Unmögliches, den Ungehorsam gegen die
Gesetze. Es ist Kriegszustand; Sie verlangen von mir eine Handlung,
die als Hochverrat aufgefaßt werden kann. Und dann wollen Sie mir
zürnen, wenn ich ein für allemal ablehne? Und Frau Torm ist darüber
so entrüstet, daß sie mich nicht sehen, nicht sprechen will? Daß
sie sich Ihrer Person bedient, um mir ihren Wunsch noch einmal
vorzutragen? Sie hat ja doch an ihren Mann offen geschrieben, ein
ganzes Buch. Der Brief liegt bereits hier, mit der Genehmigung des
Kommandos versehen. Es steht alles darin, was sie ihm mitzuteilen
hat, daß sie in der Sorge um ihn mit meiner Hilfe das Luftschiff
benutzt hat, daß sie verhindert war, zurückzukehren, daß sie sich
sehnt, sobald es ihr gestattet wird, zurückzukommen – was will sie
mehr? Was hat sie dem Mann noch zu schreiben?“

„Das ist ihr persönliches Geheimnis. Wenn Frau Torm es Ihnen nicht
mitteilen kann, wie soll ich es wissen? Übrigens weiß sie nichts
von diesem Versuch meinerseits, auf Sie einzuwirken. Sie hatte mich
nur gebeten, La um Hilfe anzugehen.“

„La? Wie käme La dazu?“

„Sie hatte Grunthe einige areographische Angaben und Aufklärung
über verschiedene technische Fragen versprochen – ein kleines
Paket, das den Brief sehr gut aufnehmen kann.“

„Und La hat diesen Betrug natürlich von sich gewiesen?“

„Ich habe sie noch gar nicht gefragt. Zunächst bin ich ja den Tag
über von Pontius zu Pilatus gelaufen, um eine amtliche Erlaubnis
zu erhalten, dann habe ich La nicht angetroffen, als ich mit ihr
sprechen wollte. Ich mußte nun zunächst mit Ihnen als Freund und
Mensch reden. Ich sehe jetzt, daß es vergeblich wäre. Sie würden
diesen Brief an Torm von mir nicht befördern? Auch nicht einen an
meine Mutter?“

Ell schüttelte den Kopf. „Sie haben an beide schon geschrieben.“

„Aber offen. Es gibt Dinge, die man nicht vor andern sagen will. Wo
bleibt die gerühmte Freiheit, die versprochene Freiheit, wenn man
uns jetzt das persönliche Eigenrecht der Aussprache abschneidet?“

„Sie müssen bedenken, daß dies nur bis zu dem Augenblick geschieht,
in welchem unser Verhältnis zur Erde sich geklärt hat. Das ist eine
Ausnahme. Es ist ein Unglück, denn es ist allerdings ein Vergehen
gegen die sittliche Grundlage, gegen die persönliche Freiheit. Aber
sittliche Konflikte sind ein allgemeines Unglück, sie lassen sich
nicht vermeiden. Die höhere Pflicht, die Ordnung zwischen den
Planeten, erfordert diesen Verzicht des einzelnen auf seine
Freiheit. Und im Grunde genommen ist es doch nur der Ausdruck
individueller Gefühle, der eine Beschränkung erleidet.“

„Sie geschieht aber bloß aus einem Mißtrauen der Martier gegen die
Menschen.“

Ell sah Saltner durchdringend an.

„Geben Sie mir Ihr Ehrenwort“, fragte er, „daß in Ihren Briefen
nichts über unsre Maßnahmen steht?“

„Nein“, sagte Saltner.

„Und dann verlangen Sie von mir –“

„Ich verlange, was der Mensch vom Menschen, der Deutsche vom
Deutschen verlangen kann, daß er ihm hilft, eines übermächtigen
Gegners sich zu erwehren –“

„Ich aber stehe auf der Seite dieses sogenannten Gegners, der im
Grunde der beste Freund ist.“

„Dann haben wir uns nichts weiter zu sagen. Ich wollte mich nur
überzeugen, daß ich von Ihrer Seite für uns Menschen nichts zu
erwarten habe.“

„Sie wollen mich nicht verstehen. Nur in Ihrem einseitigen
Interesse kann ich nichts tun, sonst aber werden Sie mich stets
bereitfinden –“

„Leben Sie wohl.“

Saltner hörte nicht mehr auf Ells Worte. Er war schon auf den
Gleitstuhl getreten und löste die Hemmung. Der Stuhl sauste die
schraubenförmige Bahn um den Stamm des Riesenbaumes hinab nach dem
Erdboden. Das Gespräch hatte auf der Plattform stattgefunden,
welche einen der Riesenbäume in der Nähe von Ells Wohnung umgab,
dort, wo in einer Höhe von vierzig Meter über dem Boden die ersten
Äste ansetzten. Ein mechanischer Aufzug führte in einer
Schraubenlinie rings um den Stamm und beförderte ebenso leicht von
unten nach oben als von oben nach unten. Diese geschützten
Plattformen boten einen äußerst angenehmen Arbeitsplatz. Wie vom
Chor eines Domes blickte man zwischen den Säulen der Baumstämme
hindurch, über die niedrigen Häuser weit in die Anlagen. Die Luft
war hier frischer und kühler als unten.

Ell trat an die Brüstung vor und blickte hinab. Es begann zu
dämmern. An den Straßen entlang leuchteten schon die breiten
Streifen des Fluoreszenzlichtes, in den Häusern glühten die Lampen.
In tiefer Finsternis lag das Laubdach.

Ell seufzte. Also auch er hatte sich von ihm geschieden, der
biedere Saltner! Mochte es sein! Was galt ihm das alles noch, da er
sie verloren hatte! Finster zog sich seine Stirn zusammen. Das war
der Dank, ihr Dank für alles – – Ismas Dank!

Als sie auf der Tafel des Retrospektivs ihren Mann erkannt hatte,
wie er aus dem Umiak der Eskimos nach dem Boot des englischen
Schiffes winkte, da hatten sie ihre Kräfte auf einen Augenblick
verlassen. Auf einen Augenblick. Sie hatte sich sogleich wieder
zusammengerafft und mit fieberhafter Erregung die Vorgänge
verfolgt. Man hatte gesehen, wie beide Boote der ›Prevention‹
zusteuerten, wie Torm an Bord des Kriegsschiffes stieg, wie er dem
Kapitän Papiere überreichte, die dieser prüfte; man sah, wie der
Kapitän dann salutierte und Torm die Hand schüttelte, wie sich die
Offiziere um Torm versammelten; man sah, wie die Eskimos beschenkt
wurden und ihr Boot sich entfernte; wie die ›Prevention‹ ihre Fahrt
nach Süden wieder aufnahm. Eine Stunde lang konnte man sie
verfolgen. Maschine und Steuer waren offenbar nicht verletzt oder
wieder repariert; das Schiff dampfte schnell und leicht vorwärts.
Immer undeutlicher wurden die Umrisse desselben. Die Dämmerung
brach herein. Bald konnte man nichts mehr unterscheiden als die
Lichter. Man stellte den Versuch ein. Es war sicher, daß das Schiff
und Torm mit ihm in wenigen Wochen wohlbehalten London erreichen
würden. – –

Torm war gerettet. Er hatte ohne Zweifel jetzt schon die Nachricht
von Ismas Verschwinden. Man würde in Friedau dafür gesorgt haben,
daß ihm dasselbe unter dem Gesichtspunkt der Friedauer erschien.
Und sie, die nicht ohne ihn in Friedau bleiben wollte, nun mußte
sie ihn allein lassen –

Isma verbrachte eine schlaflose Nacht. Dann setzte sie noch einmal
alles in Bewegung, um ihre Mitnahme auf dem Raumschiff nach der
Erde zu erreichen. Es war unmöglich. Wenigstens einen Brief sollte
man mitnehmen. Ja, aber nur einen offenen. Sie schrieb, doch das
konnte ihr nicht genügen. Was sie ihm zu sagen hatte, das ging
niemand andern an; das konnte sie nicht lesen lassen. Sie wußte,
wie sie schreiben müsse und daß er sie nur so verstehen würde. Und
dies wurde versagt. Und hier ließ sie Ell im Stich! Sie bat ihn
flehentlich, ihren Brief zu besorgen. Es ginge nicht! Sie bat ihn,
selbst die Reise zu machen, ihren Mann aufzuklären. Er weigerte
sich, er wolle jetzt nicht auf die Erde zurückkehren. Die Martier
selbst hätten es vielleicht gern gesehen, aber er könne sich nicht
entschließen, jetzt den Mars zu verlassen. Warum nicht? Warum
wollte er nicht? Um sie, Isma, nicht allein zu lassen? Sie glaubte
es nicht, sie vermutete einen anderen Grund, den sie ihm nicht
verzeihen konnte. Sie sagte ihm Bitteres. Sie wollte ihn nicht
wiedersehen. Und er ging. Natürlich! La würde ihn wohl trösten – –

Ell versetzte sich in Ismas Seele, er sah deutlich, was in ihr
vorging – alles dachte er wieder durch, während er in die Nacht
hinausstarrte – das Gefühl der Bitterkeit verließ ihn, er konnte
ihr nicht zürnen. Nur traurig wurde er, tieftraurig.

Aber er mußte es tragen. Er konnte nicht anders handeln, es war
unmöglich. Stand sie auf der Erde, so stand er auf dem Mars. Die
Kluft überbrückte kein Raumschiff. Und selbst wenn die Planeten
sich versöhnten – würde er sie dann wiederfinden?

Er preßte die Hände gegen die Stirn und seufzte tief. Und seltsam,
mitten in den Kummer um Isma drängte sich das Bild Las vor Ells
Augen. Dieser Verkehr war so beglückend, so frei von dem dunkeln
Hintergrund irdischer Fesseln! Das war Numenart, zu geben und zu
nehmen! Die reizenden Stunden kamen ihm in den Sinn, in denen er
sich sagen durfte, daß sie ihn bevorzugte, und es schien ihm, daß
deren immer mehr geworden seien. Und doch! Er mußte sich gestehen,
wäre La ihm so geneigt, wie er hoffte, sie hätte sich ihm noch
anders zeigen müssen. Sie hatte sich in der letzten Zeit sichtlich
von Saltner zurückgezogen, aber gerade darin schien ihm eine
gewisse Absichtlichkeit zu liegen. Er konnte das Gefühl nicht
loswerden, daß La unter der gleichmäßigen Liebenswürdigkeit ihres
Wesens eine heimliche Sorge barg, und er sann nach, was sie wohl
bedrücken könne.

Gestern, als er bei ihr war, hatte er sie überrascht, wie sie in
Gedanken versunken saß, und er glaubte die heimlichen Spuren von
Tränen in ihrem Auge gesehen zu haben. Aber auf seine warmen Worte
erwiderte sie mit Scherzen, es war, als wollte sie nicht verstehen,
was sie doch längst wußte, wie er für sie fühle. Zum erstenmal war
er fortgegangen, ohne sie recht verstanden zu haben.

Und jetzt war Saltner auf dem Weg zu ihr. Es war ja nicht daran zu
denken, daß sie auf seine Bitte eingehen würde – Überhaupt –

Ell fiel es plötzlich ein – vielleicht war sie gar nicht in Kla. La
hatte mehrfach davon gesprochen, daß sie möglicherweise verreisen
würde, und Saltner hatte sie heute vergebens zu sprechen versucht.
– Er wollte sich doch überzeugen, ob La zu Hause sei. Auch die
Plattform war mit dem Haus telephonisch verbunden. Er sprach La an.
Sie war zu Hause, aber in großer Eile, wie sie sagte. Ell teilte
ihr mit, daß Saltner bei ihr vorsprechen werde mit einem Ansinnen,
das unmöglich zu erfüllen sei – darauf keine Antwort, trotz seiner
wiederholten Frage. Endlich die Worte, wie mit gezwungener Stimme:

„Befürchten Sie nichts. Leben Sie wohl.“ –

Nichts, nichts weiter! Ell wußte nicht, was er davon denken
sollte.

Er trat zurück an den Tisch, auf dem seine Papiere lagen, und ließ
die Lampe aufflammen. Er wollte versuchen, in der Arbeit zu
vergessen, und versenkte sich in das Studium des Etats der
Marsstaaten.

Die 154 Staaten, welche den Planetenbund des Mars bildeten, waren
an Einwohnerzahl sehr stark verschieden; es gab darunter Reiche,
die bis gegen hundert Millionen Einwohner zählten, und kleine
Staaten, die nicht einmal die Zahl von einer Million erreichten;
der kleinste von ihnen umfaßte nur zwanzig Bezirke mit zusammen
800.000 Einwohnern. Ebenso mannigfaltig wie die Größen waren die
Verfassungen der Einzelstaaten. Die republikanischen Staatsformen
herrschten vor, aber auch unter ihnen gab es eine bunte Musterkarte
von kommunistischen, sozialistischen, demokratischen und
aristokratischen Verfassungen. Die Monarchien waren besonders unter
den kleineren Staaten vertreten. Ganz wie es die historische
Entwicklung der lokalen Verhältnisse mit sich gebracht hatte, waren
auch in diesen die Verfassungen sehr mannigfaltig; im ganzen
unterschieden sie sich von den republikanischen nur dadurch, daß
das Staatsoberhaupt nicht durch Wahl, sondern durch Erbfolge
bestimmt war und sich eines größeren Einkommens und einer
glänzenderen Hofhaltung als die Präsidenten erfreute. Einen höheren
politischen Einfluß besaßen die Fürsten des Mars nicht, sie hatten
vornehmlich eine ästhetische Bedeutung. Die reiche Entwicklung,
welche die Verfeinerung des Lebens durch die Hofhaltung eines
intelligenten Fürsten erfahren konnte, und der Einfluß, den eine
hochsinnige Persönlichkeit hier zu entfalten vermochte, sollte auch
auf dem Mars nicht verlorengehen. Die individualistischen Neigungen
der Martier konnten daher nach jeder Richtung hin Befriedigung
finden, und dem Ehrgeiz wie dem Unabhängigkeitsgefühl eines jeden
war freier Spielraum gelassen. Zwischen allen Staaten herrschte,
durch das Bundesgesetz garantiert, vollständige Freizügigkeit und
Erwerbsfreiheit. Wem es in dem einen Staat nicht gefiel,
transportierte sein Haus in einen andern, und es genügte, daß er
dies bei der betreffenden Behörde anmeldete. Dadurch war eine
natürliche Regulierung dafür gegeben, daß kein Staat seine
Machtbefugnis mißbrauchte, denn er riskierte sonst, sehr bald seine
Einwohner zu verlieren. Die natürliche Verschiedenheit der
Individuen, ihre Gewohnheiten und ihre Anhänglichkeit für das
Hergebrachte sorgten andererseits dafür, daß den einzelnen Staaten
ihre Eigentümlichkeiten erhalten blieben und der Fluß der
Bevölkerung nicht in Unbeständigkeit ausartete. Jede Gegend hatte
ihre Vorzüge. Waren auch die wirtschaftlichen Lebensbedingungen in
den breiten, die Wüsten durchziehenden, durch künstliche
Bewässerung erhaltenen Kulturstreifen etwas erschwert, so boten
dieselben doch andere Vorteile. Die Gelegenheit zum gewerblichen
Gewinn war hier wegen der Nähe der großen Energiestrahlungsgebiete
günstiger, und ein reicherer Arbeitsertrag entschädigte für die
Störungen des äußeren Komforts, die dadurch entstanden, daß bei
eintretendem Wassermangel die schützenden Bäume binnen wenigen
Tagen ihr Laub verloren und die Vegetation unter ihnen
vertrocknete. Dafür waren aber auch die hier gelegenen Staaten
imstande, größere Zuschüsse den Privaten zu gewähren.

Gemeinschaftlich für den gesamten Staatenbund und unmittelbar dem
Zentralrat unterstellt, der seinerseits dem Bundesparlament
verantwortlich blieb, war die technische Verwaltung. Sie schied
sich in die beiden großen Gebiete des Verkehrswesens und des
Bewässerungswesens, wozu als drittes jetzt noch die Raumschifffahrt
gekommen war. Diese ungeheure Organisation hielt die Bundesstaaten
als ein untrennbares Ganze zusammen und machte es ebenso unmöglich,
daß sich einzelne, selbst mächtige Staaten, vom Zusammenhang des
Planeten ablösen konnten, als sich ein Organ des menschlichen
Körpers der Blutzirkulation zu entziehen vermag.

Unterhalten wurde der Riesenbetrieb durch ein stehendes Arbeitsheer
von sechzig Millionen Personen – ›Mann‹ kann man nicht gut sagen,
denn die allgemeine einjährige Dienstpflicht galt für beide
Geschlechter. Für besondere Fälle stand eine dreifache Reserve zur
Verfügung. Finanziert wurde der Betrieb durch die Sonne selbst. Der
Gesamtetat der Marsstaaten betrug – nach deutschem Geld gerechnet
für das Erdenjahr, also für ein halbes Marsjahr – 300 Billionen,
das sind 300.000 Milliarden Mark, also 100.000 Mark auf den Kopf
der Bevölkerung. Dabei hatte aber niemand eine Steuer, außer der
persönlichen Dienstleistung während eines Lebensjahres,
beizutragen. Das Privateinkommen der Martier belief sich außerdem
im Durchschnitt pro Kopf der Bevölkerung auf 100.000 Mark,
schwankte jedoch für den einzelnen zwischen dem Maximum des
zulässigen Einkommens von zwanzig Millionen und der Null. Die
Besteuerung des Einkommens der Privaten diente nur dazu, um jedem,
der nichts verdiente, wenigstens ein Minimum von Kapital pro Jahr
zu sichern, wodurch er sich wieder heraufarbeiten konnte. Ein
Notleiden aus Mangel an Nahrung, Wohnung und Kleidung konnte nicht
eintreten, da hierfür durch öffentliche Verpflegungsanstalten
gesorgt war. Aber es war natürlich jedem daran gelegen, dieser
Armenpflege nicht anheimzufallen. Der Gesamtbetrag, der vom Staat
und von den Privaten auf dem ganzen Planeten in einem halben
Marsjahr eingenommen wurde, belief sich also auf 600 Billionen
Mark. Dies war jedoch nur die Hälfte dessen, was bei völliger
Ausnutzung aller Kräfte hätte erzielt werden können.

Diese Summen erschienen Ell so ungeheuerlich, daß er sich damit
beschäftigte, sie nachzuprüfen und sich zu vergewissern, wie es
möglich sei, eine so kolossale Rente zu erzielen. Ell hatte bei
seinem ersten Versuch, den Geldwert auf dem Mars mit dem auf der
Erde zu vergleichen, seiner Umrechnung den Wärmewert der Kohle
zugrunde gelegt; er führte nun die Rechnung noch einmal so durch,
daß er als Vergleichseinheit die Pferdestärken nahm, welche durch
die Sonnenstrahlung pro Stunde als Arbeitseffekt erzielt werden
konnten. Wenn er den gegenwärtigen Stand der Technik auf der Erde
in Betracht zog, so glaubte er annehmen zu dürfen, daß selbst unter
den günstigsten Verhältnissen, bei Berücksichtigung der
Anlagekosten, die Pferdekraft in der Stunde nicht unter 0,8 Pfennig
oder 1 Centime geliefert werden könne. Um nun den geringsten Wert
der Sonnenrente für den Mars zu ermitteln, nahm er an, daß auch auf
dem Mars nur die direkte Wärmestrahlung seitens der Sonne – nicht
die anderen Wellengattungen – zur Arbeit verwertet werden. Er fand
dann, daß im Lauf eines Erdenjahres die Sonnenstrahlung dem Mars
soviel Wärme zuführt, daß, wenn sie vollständig in Arbeit
übergeführt wurde, ihr Wert pro Quadratmeter der Oberfläche
durchschnittlich 30 Mark betragen würde. Die zur Bestrahlung
ausgenutzte Oberfläche des Mars beträgt aber rund hundert Billionen
Quadratmeter, somit erhält der Mars eine Rente von 3.000 Billionen
Mark. Von diesem Strahlungsbetrag können jedoch nur etwa 40 Prozent
wirklich in Arbeit verwandelt und ausgenutzt werden – bei dem Stand
der Technik auf dem Mars –, so daß der Gesamtgewinn des Mars an
Arbeit (im Laufe eines Erdenjahrs) 1.200 Billionen Mark beträgt.
Tatsächlich benutzte man hiervon nur die Hälfte. Denn die
Gesamteinnahme der Marsstaaten betrug 300 Billionen, die der
Privaten ebensoviel. Es war also kein Zweifel, daß die Marsstaaten
über diese ungeheuren Mittel verfügten. Und dabei empfängt der Mars
nur etwa ein Neuntel so viel Wärme von der Sonne wie die Erde. Wie
weit also war die Erde zurück in der Ausnutzung der Mittel, die ihr
von der Natur verliehen waren! Wieviel konnte sie noch gewinnen,
wenn ihr die Erfahrung der Martier zugute kam!

Aufs neue fühlte sich Ell in der Ansicht bestärkt, daß gegenüber
dem immensen Fortschritt, der hier für die Menschheit in Frage
stand, die Rücksicht auf die Neigung der gegenwärtigen Menschheit,
dieses Geschenk anzunehmen, zu schweigen hatte. Noch viel weniger
aber durfte er sich seinen Handlungen durch persönliche Neigungen
irre machen lassen. Mochte man ihn als Überläufer, als Verräter an
der Sache der Erde betrachten, mochte man Schmach und Verachtung
auf ihn häufen – gleichviel! Er wußte, daß er zum besten der Kultur
überhaupt und so auch der Menschheit handle, wenn er voll auf der
Seite des Mars stand. Mochte er selbst seine persönlichen Freunde
verlieren, er mußte es tragen. Einst würden sie gerechter über ihn
urteilen. Und Isma! Er sah den traurigen Blick der blauen Augen, er
sah das schmerzliche Zucken ihrer Lippen und das verächtliche
Zurückwerfen des Kopfes –. Und noch einmal sprang er empor und
starrte trüben Blickes in die dunkle Nacht. Dort drüben, wo der
hellgrüne Schimmer des Straßenstreifens sich hinzog, da wohnte sie.
O könnte er hingehen und sie rufen, wie damals, als das Luftschiff
auf sie wartete, könnte er sie wieder zur Erde zurückführen und
dafür ihren dankbaren Blick erhalten! Doch es ging nicht. Sie
durfte nicht fort, sie konnte nicht, selbst wenn er versucht hätte,
sie fortzubringen. Aber er selbst! Ihm stand es frei, er besaß die
Erlaubnis, mit nach der Erde zu gehen, er hatte die Vollmacht hier
vor sich, die er eben mit den übrigen Briefschaften an Ill
zurückschicken wollte. In wenigen Tagen ging das Raumschiff. Ill
fuhr zu diesem Zweck selbst an die Polstation, um der Abreise
beizuwohnen. Er konnte mitreisen. Er konnte ihr den Wunsch
erfüllen, mit Torm selbst zu sprechen. – Nein doch, nein! Es war
unmöglich. Würde ihm Torm glauben können, wenn er ohne Isma kam?
Und in diese Verhältnisse! Unter diesen Umständen! Sich
gewissermaßen entschuldigen? Von allen Seiten beargwöhnt und
angefeindet, würde er überhaupt jetzt etwas zur Versöhnung
beitragen können? Nein, wenn er überhaupt zur Erde zurückging, da
konnte es nur sein, wenn die Menschen begriffen hatten, was die
Nume ihnen bringen und wie sie dieselben aufzunehmen haben. Er
wollte auf dem Mars bleiben, bis er zurückkehren konnte als ein
Herr und Beglücker der Menschen.

Ell schloß die Papiere für Ill in die Mappe und fügte seinen Paß
für das Raumschiff hinzu. Er brauchte ihn nicht.

\section{36 - Saltners Reise}

Saltner lenkte seinen Radschlitten, dessen er sich sehr bald zu
bedienen gelernt hatte, Frus Haus zu. Wie oft hatte er in diesen
zwei Monaten, die er schon auf dem Mars weilte, den Weg
zurückgelegt und die kürzeste Verbindung ausprobiert! Heute hatte
er weite Umwege gemacht und im nächtlichen Park seinen Gedanken
nachgehangen. Sonst konnte es ihm immer nicht schnell genug gehen,
wenn er über die schmalen Parkwege hinglitt, die nach Las Wohnung
führten. Wenn ihm das Verhältnis des Mars zur Erde Sorge machte,
bei La fand er Trost und Ermunterung, von ihr wußte er ja, daß sie
ihn nicht für gering hielt, weil er nur ein Mensch war – –. Sie
liebte ihn, die Nume, die herrliche. Sollte er nicht glücklich
sein? Und doch – das Wort: „Vergiß nicht, daß ich eine Nume bin“,
das sie zu ihm gesprochen, als sie zusammen auf die Erde
hinabblickten, es ging ihm nicht aus dem Sinn, was er damals kaum
beachtet, nicht verstanden hatte. Das Wort hatte er nicht
vergessen, aber vielleicht die Warnung, die es enthielt. Sollte er
jetzt daran erinnert werden? Durfte er es wagen, die Bitte
auszusprechen, die sie ihm versagen mußte? Warum war er seit zwei
Tagen nicht mehr bei La gewesen? Er hatte viel zu tun gehabt,
gewiß; die Erdkommission hatte von ihm verschiedene Gutachten
verlangt, auch Frau Torm hatte lange Unterredungen mit ihm, die
Briefe nach der Erde nahmen seine Zeit in Anspruch. Zweimal hatte
er auch La durch das Telephon angesprochen, doch beide Male war sie
nicht zu Hause gewesen. Er wußte nicht einmal, womit sie so eifrig
beschäftigt war. Seit acht Tagen war sie mit ihrer Mutter allein.
Fru hatte sich bereits nach dem Pol begeben, um die Ausrüstung der
Raumschiffe zu leiten. Es hatten lange Erwägungen in der
Erdkommission stattgefunden, welche Kapitäne und Ingenieure bei der
wichtigen und verantwortlichen Expedition nach dem Südpol der Erde
zu verwenden seien. Schließlich wollte man, obgleich an tüchtigen
Leuten kein Mangel war, doch des Rates Frus, als eines der
bewährtesten Erdkenner, nicht entbehren, und er hatte sich
entschlossen, die technische Leitung der Expedition zu übernehmen.
Es war auch davon die Rede gewesen, daß La ihn begleiten solle. Die
Aussicht, La so bald wieder zu verlieren, hatte Saltner schmerzlich
erregt, und er hatte nun befreit aufgeatmet, als er hörte, daß La
ihren Wunsch, auf dem Mars Liebe zu ihm der Hauptbeweggrund gewesen
sei, der sie hier zurückhielt – er hatte sich dessen geschmeichelt.
Aber warum war er in den letzten Tagen so zweifelhaft geworden?
Warum hatte er nicht die Zeit gefunden, sie aufzusuchen? zu
bleiben, durchgesetzt habe. Er schmeichelte sich, daß ihre

Er konnte es sich nicht verhehlen, er war eifersüchtig. Fast
jedesmal in der letzten Zeit hatte er Ell bei La getroffen, oder
sie war während seiner Anwesenheit von Ell aus der Ferne
angesprochen worden. Und wie begegnete sie Ell! Jedes Wort, jeder
Blick zwischen ihnen war sofort verstanden, ihren Gesprächen
vermochte er nicht zu folgen, es waren zwei Nume, die sich
unterhielten, die sich gefielen, die –. Es konnte ja gar kein
Zweifel sein, wer mußte nicht La lieben, der sie näher
kennenlernte? Und er, wie konnte er sich mit dem Martiersohn
vergleichen, der La ebenbürtig war und doch den eigentümlichen Reiz
des Menschentums besaß! Er hätte diesen Ell hassen mögen, er nannte
ihn einen Verräter an der Menschheit und einen Räuber seines
Glücks. Und doch, konnte man den einen Verräter nennen, der nur zu
seinem eigentlichen Vaterland zurückkehrte, das ihm durch ein
unverschuldetes Geschick geraubt war? Und welches Recht hatte er
selbst an La? Was entbehrte er überhaupt? Sie entzog sich ihm nicht
um Ells willen, sie war ebenso lieb und gut wie früher, ja
vielleicht sorgsamer und zärtlicher wie je, sie zeigte ihm in jedem
Augenblick, wie wert er ihr war. Aber sie zeigte es auch Ell. Das
störte ihn, das empörte ihn, sie aber fand es offenbar ganz in
Ordnung. Sie war eine Martierin. Sie hatte ihn ja gewarnt; wenn er
sie liebte, mußte er mit der Sitte der Martier rechnen. Er aber war
ein Mensch – –

Saltner näherte sich der breiteren Straße, wo La wohnte. In seine
Gedanken versunken hatte er nicht bemerkt, daß ein Transport der
Umzugsgesellschaft ihm entgegenkam. Er hatte nur gerade noch Zeit,
zur Seite auszuweichen und den Zug an sich vorüberzulassen. Ein
Haus, auf breiten Gleitkufen stehend, wurde von einer
Reaktionsmaschine vorwärtsgeschoben. Die Fenster waren geschlossen,
es war alles dunkel im Hause. Die Bewohner schliefen offenbar. Wenn
sie am Morgen aufwachten, stand ihr Haus viele Hunderte von
Kilometern entfernt. Nun war die Bahn wieder frei. Die Straße lag,
von den breiten Streifen des Fluoreszenzlichtes an beiden Seiten
erleuchtet, hell vor ihm. Noch eine Minute, und sein Schlitten war
vor ihrem Haus. Ob er sie heute noch würde sprechen können? Es war
schon ziemlich spät geworden. Ob er nicht seinen Besuch auf morgen
aufschieben sollte? Er hatte eine dringende Bitte an sie, aber wie,
wenn sie sich dadurch beleidigt fühlte? Er mochte gar nicht daran
denken, daß auch La ihn abweisen könnte.

Da war das Nachbarhaus, an seinen tulpenartig aufragenden Erkern
kenntlich, und hier –. Er hielt den Schlitten an. Frus Haus war
verschwunden, die Stelle war leer. Saltner traute seinen Augen
kaum. La war wirklich fortgezogen, ohne ihn zu benachrichtigen?

Auf dem Rasenplatz, wo das Haus gestanden hatte, zeigte sich eine
Tafel. Sie enthielt nur die Worte:

„Verzogen 29,36 nach Mari, Sei 614.“

Saltner stand ratlos. 29,36 – das war die Zeit der Abreise. Er
verglich den Kalender, den er sich zur Umrechnung der martischen
Zeit angelegt hatte, da ihm das duodezimale Zahlensystem und die
Angabe der Stunden und Minuten in Bruchteilen immer noch
Schwierigkeiten machte. Seine Uhr zeigte 29,37 – das war ein
Unterschied von zehn Minuten –, vor zehn Minuten erst hatte der
Transport des Hauses begonnen. So war es gewiß Las Wohnung gewesen,
die er an sich hatte vorüberschieben sehen. Sie konnte noch nicht
weit fort sein. Wenn er seinen leichten Schlitten in volle Eile
versetzte, konnte er den Transport vielleicht noch einholen, ehe er
die Gleitbahn erreichte, die ihn dann mit größter Geschwindigkeit
davontrug. Schon wandte Saltner sein Fahrzeug – doch – was hätte
dies genutzt? Er konnte doch La nicht in der Nacht aus dem Schlaf
stören. Nachreisen konnte er auch morgen noch. Er notierte sich die
Adresse. Mari – er wußte freilich nicht, wo dieser Staat oder
Bezirk lag, ob die Entfernung groß sei – doch das läßt sich
ermitteln. Also nach seiner Wohnung! Er war seit Mittag nicht zu
Hause gewesen. Gewiß, zu Hause würde er auch Aufklärung finden,
warum La so plötzlich verzogen war.

Saltners Wohnung war ganz in der Nähe. Als er die Tür öffnete,
flammten die Lampen im Haus auf, und das erste, was er beim
Eintritt ins Zimmer erblickte, war ein Zettel mit den deutschen
Worten: ›Ich sprach ins Grammophon. La.‹

Saltner eilte an das Instrument und löste den Verschluß. Das
leichte Klopfen ertönte, womit der Beginn der Rede angezeigt wird.
Dann vernahm er Las melodische, tiefe Stimme, er glaubte sie vor
sich zu sehen, wie sie mit zärtlichem Vorwurf sagte:

„Wo stecktest du denn, mein geliebter Sal, dreimal habe ich dich
angerufen, bei Frau Torm habe ich dich gesucht – du warst aber
fortgegangen und sie gleichfalls, da bin ich in deine Wohnung
geeilt, wo du auch nicht bist, und jetzt habe ich nur noch Zeit,
dir schnell ein paar Worte ins Grammophon zu sagen, damit du nicht
denkst, deine La wäre dir ohne Abschied davongegangen. Denn höre
nur! Wir ziehen in einer halben Stunde nach Mari, Sei 614. Mari
liegt ziemlich weit von hier nach Südwesten, am östlichen Rand der
Wüste Gol. Gern tu ich’s nicht, wie gern wäre ich bei dir geblieben
in unserm schönen Kla! In Mari ist es kühler, und das lockt meine
Mutter. Aber der Hauptgrund ist ein anderer. Ihr bösen Menschen
seid an allem schuld! Auf Gol werden die Versuche zum Schutz der
Luftschiffe gegen die Geschütze der Menschen abgehalten, und dort
kommt der Vater noch einmal hin, so daß wir vor seiner Erdreise
noch Abschied nehmen können. Bis hierhin würde es zu weit sein für
ihn. Dort werden wir auch Se noch einmal sehen. Leb also wohl, mein
lieber Freund! Wir können alle Tage miteinander sprechen. Morgen
zwischen drei und vier werde ich dich ansprechen, sei also zu
Hause. Ich erwarte dich vorläufig nicht in Sei, man würde deine
Reise dahin nicht gern sehen. Aber wenn erst die Raumschiffe fort
sind und mehr Ruhe bei uns herrscht, dann wirst du uns hoffentlich
besuchen. Also auf Wiederhören morgen! Deine La.“

Saltner hatte mit angehaltenem Atem gelauscht. Nun stellte er den
Apparat zurück und ließ sich die Abschiedsworte Las noch einmal
sagen. Dann dachte er lange darüber nach. Allerlei Fragen drängten
sich ihm auf.

An die Wüste Gol erinnerte sich Saltner; La hatte sie ihm gezeigt,
als das Raumschiff, das ihn nach dem Mars brachte, sich der
Außenstation näherte. Sie war der große helle Fleck, nicht sehr
weit vom Südpol, den die Astronomen der Erde die Insel Thyle I
nannten. Sein Weg vom Pol nach Kla führte nicht weit davon vorüber,
weil der direkte Weg damals im ersten Sommer noch durch Schnee
unbequem gemacht war. Er erinnerte sich, daß er auf seiner Fahrt
aus dem Fenster des Eilzugs zu seinem Erstaunen im ersten
Morgengrauen wolkenähnliche Gebilde gesehen hatte, fern im Westen
am Horizont, und daß man ihm gesagt hatte, daß dies die Morgennebel
auf dem Hochplateau der Wüste Gol seien. Auch daß die Versuche mit
den weittragenden Geschützen der Erdbewohner dort vorgenommen
wurden, hatte er gehört. Die Martier hatten für derartige
Schießplätze nur auf ihren Wüsten Raum, und Gol lag dem Südpol am
nächsten. Aber warum mußte La ihre Abreise so beschleunigen? Sie
sagte, um ihren Vater noch einmal zu sehen. Also mußte Fru sehr
bald, wohl morgen schon, dort erwartet werden, und daraus war zu
schließen, daß auch das Raumschiff bald abgehen werde. Er hatte
somit keine Zeit zu verlieren, wenn er La noch persönlich vor
Abgang des Schiffes sprechen wollte. Warum aber, wenn es sich bloß
um ein Zusammentreffen mit dem Vater handelte, war sie mit dem
ganzen Haus übergesiedelt? Es war doch noch ziemlich früh, um eine
so südlich gelegene Sommerfrische aufzusuchen. Und warum sollte er
ihr nicht nachkommen? Und was bedeutete diese hingeworfene
Bemerkung über Se?

Doch über diese Fragen nachzudenken, war noch Zeit auf der Reise;
denn La nachzueilen, um sie zu sprechen, dazu war Saltner sofort
entschlossen. Was er mit ihr zu beraten, von ihr zu erbitten hatte,
das konnte er nicht telephonisch erledigen, dazu mußte er ihr Aug’
in Auge sehen; fürchtete er doch mit gutem Grund, daß auch sie sich
weigern würde. Aber diesem Schritt, der ihm schwer genug wurde,
konnte und durfte er sich nicht entziehen, und er mußte sofort
geschehen, solange noch das Raumschiff den Mars nicht verlassen
hatte. Er hatte Isma das Versprechen gegeben, La um Hilfe
anzugehen, das mußte er halten. Wichtigeres jedoch lag ihm selbst
am Herzen. Er hielt es für seine Pflicht, die Staaten der Erde von
den Maßnahmen der Martier zu unterrichten. Er erinnerte sich jenes
Wortes von Grunthe, daß sie Kundschafter seien, an deren getreuen
Diensten vielleicht das Wohl und Wehe der zivilisierten Erde hinge.
Nicht von den Erklärungen allein, welche die Regierung der Martier
abzugeben belieben würde, sollten die Menschen erfahren, sondern
auch von den Ansichten, die hier auf dem Mars in der großen
Antibatenpartei herrschten, und von dem Urteil, das er, als Mensch,
über das Vorgehen der Martier sich gebildet hatte. Er mußte
versuchen, seine von den Martiern nicht kontrollierten Briefe nach
der Erde zu befördern, selbst in der schmerzlichen Aussicht, sich
La zu entfremden.

Sie hatte gesagt: „Ich erwarte dich vorläufig nicht in Sei, man
würde deine Reise hierher nicht gern sehen.“ Er ließ sich die Worte
noch einmal wiederholen. Das war also eine Meinungsäußerung Las,
ein Rat vielleicht, kein direktes Verbot. Warum hatte sie sich so
unbestimmt ausgedrückt, nicht mit der gewohnten Klarheit? Folgte
sie vielleicht einem fremden Wunsch, der mit dem eigenen nicht
übereinstimmte, oder war sie mit sich selbst im Zwiespalt? „Man
würde deine Reise nicht gern sehen.“ Wer ist das ›man‹? Sie hat
also nicht gesagt, daß sie selbst sie nicht gern sehen würde. Das
›man‹ aber, die andern, also wohl die Regierung, die Martier, Ill,
Ell und wer sonst, was ging ihn das an? Sie sollten nicht eher
davon erfahren, als bis er dort wäre; hatte er erst mit La
gesprochen, so war ihm alles übrige gleichgültig. Also vor allen
Dingen sofort nach Mari!

Saltner war müde, er hätte sich gern niedergelegt. Aber zum
Schlafen hatte er unterwegs Zeit. Er wußte, daß die
Personenbeförderung auf große Entfernungen mit den schnellen
Radbahnen alle Stunden stattfand, er konnte also jede Stunde
abreisen. Seine Vorbereitungen waren schnell erledigt, eine kleine
Handtasche, der Reisepelz, den er noch von der Erde mitgebracht,
und sein ›Energieschwamm‹, das ist sein Kapital, aus welchem er die
im Geldverkehr übliche Münze abzapfen mußte. Es war dies eine
Büchse mit einem äußerst feinen und dichten Metallpulver, das in
seinen Poren den höchst kondensierten Äther enthielt und dadurch
eine bestimmte Arbeitsmenge repräsentierte. Ein Gramm dieses
Pulvers hatte einen Wert von etwa fünftausend Mark, denn eine
gleichwertige Arbeitskraft konnte man in dem geeigneten Apparat
daraus entwickeln. Diese Währungseinheit hieß ein ›Eck‹ und war
zugleich das Zehntausendfache der Strahlungseinheit. Man pflegte
sich ein bis zwei Zentigramm, fünfzig bis hundert Mark, in die im
Kleinverkehr gebräuchliche Münze einzuwechseln, was in jedem
offenen Geschäft geschehen konnte.

Die Personenbeförderung auf den Radbahnen, die aber nur auf
Strecken über dreihundert Kilometer stattfand, war sehr bequem, und
Saltner wußte damit Bescheid. Um Fahrpläne, Anschlüsse und
dergleichen brauchte man sich nicht zu kümmern. Die Beförderung war
ungefähr in derselben Weise geordnet wie diejenige der Briefe auf
der Erde. Die Überführung der Passagiere an den Kreuzungsstrecken
fand ohne Zutun derselben auf dem kürzesten Weg durch die
Bahnverwaltung statt.

Saltner begab sich nach der nächsten Station, die er mit Hilfe der
Stufenbahn in einer Viertelstunde erreichte. Hier standen, in
langen Reihen aufgestellt, die Reisecoupés; Schalter, Billets,
Schaffner, alles dies gab es nicht. Ein einziger Beamter achtete
darauf, daß, sobald eine Anzahl Coupés besetzt war, sofort neue
herbeigeschoben wurden. Jede Person nahm ein solches Coupé für sich
in Anspruch. Sie waren etwa einundeinviertel Meter breit,
zweieinhalb Meter lang und drei Meter hoch. Sie bildeten also eine
Kammer von ausreichender Größe für eine Person und waren mit allen
Reisebequemlichkeiten versehen. Ein Handgriff genügte, um den
vorhandenen Sessel und Tisch in ein bequemes Bett zu verwandeln.
Auch ein Automat, der gegen Einwurf der betreffenden Münzen Speise
und Trank lieferte, fehlte nicht. Der Eingang zum Coupé war von der
schmalen Seite aus. Sie standen auf Gleitkufen und wurden vor
Abgang der Züge geräuschlos auf die Wagen der Radbahn geschoben.

Saltner trat vor ein unbesetztes Coupé, zog einen Thekel, eine
Goldmünze im Wert von etwa zehn Mark, aus der Tasche und steckte
sie in die hierzu angebrachte Öffnung an der Tür. Die bisher
verschlossene Tür sprang auf, und Saltner trat ein. Die Zeit des
Eintritts markierte sich selbsttätig an der Tür, und Saltner hatte
nunmehr das Recht erhalten, sich einen vollen Tag lang in dem Coupé
aufzuhalten und hinfahren zu lassen, wohin er Lust hatte.

Aus einem im Wagen befindlichen Kästchen nahm er ein kleines
Kärtchen, um die Adresse seines Coupés, sein Reiseziel,
daraufzuschreiben. Jetzt stutzte er einen Augenblick. Genügte auch
die Angabe ›Mari Sei‹? Wenn es vielleicht noch ein anderes Mari
gab, und er, statt in der Nähe des Südpols, sich am Äquator oder am
Nordpol wiederfand? Aber das Coupé war selbstverständlich mit der
erforderlichen Bibliothek versehen. Es fand sich da das Meisterwerk
statistischer und tabellarischer Kunst, das Mars-Kursbuch, in
welchem die Beförderungszeiten, Wege und Reisedauer angegeben
waren. Durch eine höchst scharfsinnig konstruierte, verschiebbare
Tabelle konnte man die Wegdauer zwischen je zwei beliebigen
Stationen sofort finden. Als Saltner ›Mari‹ nachschlug, fand er,
daß es allerdings noch einen Bezirk gleichen Namens auf der
nördlichen Halbkugel gab und daß er die Bezeichnung ›Gol‹
beizufügen hatte. Er schrieb also die Adresse auf das kleine
Kärtchen und steckte dies in einen hierzu bestimmten Rahmen im
Innern der Tür. Dadurch erschien die Adresse stark vergrößert und
hell beleuchtet außen an der Tür. Ein leises Summen begann
gleichzeitig. Dies dauerte so lange, bis der Wagen die Station
verlassen hatte, und diente als Merkzeichen für den Reisenden, daß
er nicht etwa bei der Abholungszeit übersehen war. Wenn es wieder
begann, so war es das Signal, daß das Reiseziel nach Angabe der
Adresse erreicht war.

Saltner hatte aus dem Kursbuch ersehen, daß seine Reise acht
Stunden in Anspruch nehmen würde, denn die Entfernung betrug etwa
3.000 Kilometer. Es war jetzt bald Mitternacht, er traf also am
Vormittag ziemlich zeitig auf der Station Mari ein. Übrigens
brauchte er sich nicht darum kümmern, ob er zur rechten Zeit
erwache, da sein Coupé so lange auf der Station halten blieb, bis
er die Adresse entfernt hatte oder der ganze bezahlte Tag
abgelaufen war. Aber er wußte nicht, wie weit er noch von der
Bahnstation nach Las Wohnort habe. Darüber wollte er sich am Morgen
während der Fahrt vergewissern, da die Bibliothek des Coupés genaue
Reisehandbücher über alle Teile des Mars enthielt. Früher als am
Nachmittag konnte er indessen nicht darauf rechnen, La anzutreffen,
weil die Beförderung des Hauses, die auf der Gleitbahn stattfand,
mindestens die doppelte Zeit in Anspruch nehmen mußte als seine
Eilfahrt.

Jetzt zog er den Handgriff, welcher das Coupé in ein Schlafzimmer
umgestaltete, und legte sich zu Bett. Kein Schienenrasseln, kein
Pfiff, kein Ruf und Signal störte ihn. Er merkte noch, daß das
leise Summen aufgehört und er somit seine Fahrt angetreten hatte.
Er dachte, es sei doch eine gute Einrichtung, daß hier jeder für
zehn Mark seinen eigenen Salonwagen haben könne, bequemer, als es
sich auf der Erde ein Fürst leisten kann. Dreitausend Kilometer –.
Und es fiel ihm ein, das war gerade die Entfernung von Ses Wohnort
–. Ob der wohl in der Nähe war? La wollte sie ja wieder sehen. Wie
lange hatte auch er sie nicht gesehen, obwohl gesprochen – aber
sehen –. Saltner entschlummerte, während sein Coupé, auf dem
Radwagen stehend, unter den Häuserreihen zwischen geradlinigen
Kanälen nach Südwesten jagte.

\section{37 - Die Wüste Gol}

Saltner hatte Se nicht wiedergesehen, seitdem er mit Frus die Reise
nach Kla angetreten hatte. Aber er hatte öfter mit ihr telephonisch
gesprochen – wenn sie ihn anrief, und auch dies war in der letzten
Zeit seltener geschehen. Solange er mit La zusammen war, verblaßte
der Eindruck, den sie auf ihn gemacht hatte, und La sprach mit ihm
nach ihrer Gewohnheit fast niemals über Se. Das letzte, was er von
Se gehört hatte, war ihre erneute Einberufung zum Dienst in der
chemisch-technischen Abteilung des Arbeitsheeres.

Nicht nur die Männer, sondern auch die Frauen bildeten sich auf dem
Mars für einen besonderen Beruf aus, doch bestand zwischen der Art
dieser Ausbildung und des Betriebes der Berufsarten zwischen beiden
Geschlechtern ein wesentlicher Unterschied. Nichts lag den Martiern
ferner als der Gedanke einer schablonenhaften Gleichmacherei;
Gleichheit gab es für sie nur im Sinne der gleichen Freiheit der
Bestimmung als Persönlichkeit, aber die tatsächlichen Verhältnisse
gestalteten sich durchaus verschieden nach dieser Selbstbestimmung.
Die Frauen erwählten daher Berufsarten, die ihren
Eigentümlichkeiten entsprachen und ihnen insbesondere eine gewisse
Freiheit in der Wahl der Arbeitsstunden gestatteten. Se hatte einen
wissenschaftlichen und praktischen Kursus in der Chemie
durchgemacht. Da die Herstellung aller Nahrungsmittel auf dem Mars
chemische Studien voraussetzte, war dies unter den Martierinnen
einer der verbreitetsten Berufszweige. In dieser Eigenschaft war Se
auch, als sie ihre einjährige Arbeitspflicht abzuleisten hatte, in
die chemische Arbeitsabteilung eingetreten und auf ihren Antrag der
Erdstation zugeteilt worden. Sie war nicht, wie La, in Begleitung
ihrer Eltern, sondern in ihrer eigenen Dienstleistung nach der Erde
gegangen. Auf Grund dieser besonderen Anstrengung konnte sie nach
der Rückkehr auf zwei Monate beurlaubt werden. Dieser Urlaub war
nun vorüber, und sie hatte noch einige Monate ihrer Dienstzeit zu
absolvieren. Sie war jetzt aber von der Abteilung für Lebensmittel
in die artilleristische Abteilung versetzt worden und bei den neuen
Versuchen beschäftigt, zu denen der Konflikt mit den Engländern die
Martier veranlaßt hatte. Saltner hatte davon nur soviel gehört, daß
man entdeckt hatte, wie das Repulsit in eine neue Verbindung mit
ganz wunderbaren Eigenschaften umgewandelt werden konnte, die man
jedoch, wenigstens ihm gegenüber, bisher als Geheimnis behandelte.
Se hatte damit zu tun, sie wohnte daher jetzt seit einer Woche
ebenfalls am Rand der Wüste Gol, zwar nicht in Mari, aber dicht an
der Grenze, im Bezirk Hed.

Als Saltner durch das Schütteln seines Kopfkissens erwachte, dessen
Rüttel-Wecker er auf eine Stunde vor seiner Ankunft – nach seiner
gewohnten Rechnung sieben Uhr morgens – gestellt hatte, zog er den
Fenstervorhang beiseite und sah zu seiner Verwunderung, daß der Tag
noch nicht angebrochen war. Er hatte nicht berücksichtigt, daß er
nach Westen fuhr und daher an seinem Reiseziel die Ortszeit um etwa
vier Stunden zurück sei. Er würde etwa um Sonnenaufgang in Sei
ankommen. Dennoch machte er Toilette, benutzte den
Frühstücksautomaten und begann, sich aus dem Reisehandbuch über den
Staat Mari zu unterrichten. Er erkannte daraus, daß Sei unmittelbar
am Abhang der Wüste Gol läge und die Station ebenfalls, aber
ungefähr hundert Kilometer südlicher. Die Radbahn zog sich in einer
Strecke von dreihundert Kilometern direkt am Ostabhang der Wüste
Gol hin, so daß er diese zur Rechten hatte. Um nach Sei zu
gelangen, wo die Radbahn nicht anhielt, mußte er von der Station
aus die letzten hundert Kilometer auf der Stufenbahn zurückfahren.
Da ihm die Wege und die Lage der Wohnung Las nicht genau bekannt
waren, mußte er eine Stunde auf den Weg von der Station bis zum
Haus rechnen. Es blieben ihm also noch ungefähr sechs Stunden zur
freien Verfügung, da er nicht eher bei La eintreffen wollte, als zu
der Zeit, die sie zur telephonischen Unterhaltung bestimmt hatte.
Er nahm an, daß sie diese Zeit gewählt habe, weil sie dann sicher
in ihrem neuen Wohnort angekommen sei.

Das Fenster seines Coupés, welches der Tür gegenüberlag, sah nach
Osten. Noch konnte er keinen Schimmer der Dämmerung erkennen, die
freilich auf dem Mars nur kurz und schwach war. Dennoch lag über
der Gegend ein rötliches Licht, das er sich nicht erklären konnte.
Die Monde des Mars gaben keinen derartigen Schein. Wo die Reihe der
Häuser, unter denen der Zug fortraste, unterbrochen war, und das
war in dieser Gegend mehrfach der Fall, sah er, daß das rötliche
Licht von Westen her auf die hier weniger dicht belaubten
Riesenbäume einfiel. Um nach der Seite zu sehen, auf welcher die
Wüste Gol lag, mußte Saltner die Tür seines Coupés öffnen. Sie
führte auf den schmalen Wandelgang, der sich durch den Wagen
hinzog. Hier konnten die Insassen der Coupés sich ergehen. Hier sah
man durch die großen Fenster, als der Zug eine Häuserlücke
passierte, die Felsenmauern der Wüste dunkel aufragen, über ihnen
aber lag eine rosig glänzende Lichtschicht. Die Nebel über der
Wüste, in ihrer Höhe von mehreren tausend Metern, waren bereits von
der Morgensonne beleuchtet.

Der Beamte, welcher den Radwagen begleitete, durchschritt den
Wandelgang und sagte zu jedem der wenigen sich hier aufhaltenden
Passagiere leise: „Bitte einzusteigen.“ Der Zug näherte sich der
Station, und während des Haltens auf dieser mußte sich jeder in
seinem Coupé befinden, er verlor sonst das Recht der
Weiterbeförderung. Denn sobald der Wagen hielt, klappte die ganze
Seitenwand herab und die einzelnen Coupés wurden mit großer
Gewandtheit sortiert, um je nachdem auf der Station zu bleiben oder
auf die kreuzenden Linien übergeführt zu werden. Bald verriet das
erneute leise Summen an seiner Tür Saltner, daß sein
Bestimmungsort, die Station Mari, erreicht war. Er packte seine
Sachen zusammen und trat aus dem Coupé ins Freie. Er fand die Luft
so kalt, daß er seinen Pelz umhing. Es waren nur wenige Coupés auf
der Station zurückgeblieben, und ihre Insassen waren noch nicht zum
Vorschein gekommen; sie schienen es vorzuziehen, ihren Schlaf nicht
vorzeitig zu unterbrechen. Während Saltner noch unschlüssig stand,
was er jetzt beginnen solle, trat jedoch aus einem der Coupés ein
Fahrgast, der, nachdem er einen Blick auf den Himmel geworfen
hatte, dem Ausgang der Station zuschritt wie jemand, der genau mit
der Örtlichkeit vertraut ist. Er trug das dunkle Arbeitskleid eines
Bergmanns und schien keine Zeit zu verlieren zu haben. Saltner
gedachte ihn anzureden und folgte vorläufig seinen Schritten. Der
Bergmann überschritt die hinter der Station vorüberführende
Stufenbahn auf einer Brücke und trat dann in den Eingang eines
Hauses. Da Saltner hier zögerte und der Martier bemerkte, daß ihm
Saltner gefolgt war, wandte er sich nach ihm um und sagte:

„Wenn Sie noch zum Sonnenaufgang hinaufwollen, müssen Sie sich
beeilen, der Wagen geht gleich ab.“

„Ich bin ganz fremd hier“, erwiderte Saltner. „Wenn Sie erlauben,
schließe ich mich Ihnen an.“

Der Bergmann machte eine höfliche Bewegung und ging voran. Sie
gelangten an einen gondelartig gebauten Wagen, welcher die
Aufschrift trug: ›Abarische Bahn nach der Terrasse‹. Saltner stieg
mit dem Martier ein, ein Schaffner nahm ihnen eine kleine
Fahrgebühr ab. Der Wagen, der nur schwach besetzt war, begann sehr
bald sich zu bewegen. Er glitt erst mit schwacher Steigung
aufwärts, dann, als die fast senkrecht abfallende Felswand der
Wüste erreicht war, sehr steil empor, indem er sich durch seine
Schwerelosigkeit erhob. Ein Drahtseil, an dem er hinglitt, schrieb
ihm die Bahn vor. Vorspringende Felswände verhinderten den Umblick.
Die ganze Fahrt dauerte nur wenige Minuten. Die Einrichtung war,
wie Saltner erfuhr, noch nicht lange in Betrieb.

Als Saltner den Wagen verließ, fand er sich auf einer kahlen
Felsstufe, die sich, so weit er sehen konnte, in nördlicher wie
südlicher Richtung einige hundert Schritt breit hinzog. Sie war mit
zahlreichen Baulichkeiten bedeckt, die meist elektrische
Schmelzöfen enthielten. In der ganzen Längserstreckung der Terrasse
lief ein Bahngeleis hin. Sie war eine Stufe am östlichen Abfall der
Wüste Gol. Nach Westen hin erhob sich das Gebirge noch weiter und
trug das Hochplateau der Wüste, die sich in einer Erstreckung von
etwa 600 Kilometer von Norden nach Süden und 1.000 Kilometer nach
Westen hin ausdehnte. Über derselben glänzten, in ihren oberen
Schichten hell beleuchtet, große Wolkenmassen, die sich in der
Nacht gebildet hatten, jetzt aber schon unter den Strahlen der
Sonne zu schwinden begannen.

Als sich Saltner dem Tal zuwendete, bot sich ihm ein herrlicher
Anblick. Sein Auge schweifte weithin über die Landschaft, die vom
Widerschein der erleuchteten Nebel schwach erhellt war. Nur im
Südosten erhob sich ein heller rötlicher Schimmer, das baldige
Nahen der Sonne anzeigend. Zwischen dem grünlichen Grau der
Baumkronen, auf die er hinabblickte, zogen sich, noch künstlich
erleuchtet, die geradlinigen Streifen breiter Straßen hin. Am
dunkeln, klaren Himmel standen die Sterne, einer aber von ihnen,
gerade im Osten, strahlte mit besonders hellem Licht, ein
glänzender Morgenstern. Saltner konnte sich von seinem Anblick
nicht losreißen. Ein tiefes Heimweh ergriff ihn. Zum erstenmal seit
seiner Landung auf dem Mars sah er die Erde wieder.

Die Stimme des Bergmanns, der sich zu ihm gesellte, weckte ihn aus
seiner Träumerei.

„Nicht wahr“, sagte dieser, „das ist schön. Da unten sieht man das
nicht vor lauter Bäumen, oder man muß erst zwischen die Maschinen
auf die Dächer steigen. Jetzt ist die Ba am hellsten, Sie haben sie
wohl noch nie so deutlich gesehen? Die letzten Monate hat sie zu
nahe an der Sonne gestanden.“

„Ich habe sie schon ganz in der Nähe gesehen“, sagte Saltner, „denn
ich bin schon dort gewesen.“

„So, so“, erwiderte der Bergmann lebhaft, „da sind Sie also ein
Raumschiffer. Das freut mich, daß ich einmal einen treffe, ich habe
nämlich noch keinen gesehen. Muß ein seltsames Handwerk sein! Sie
kamen mir gleich so fremdartig vor, einen solchen Mantel sah ich
noch nie.“

„Der ist von dem Fell der Tiere, wie sie auf der Erde leben.“

Der Bergmann befühlte neugierig das Pelzwerk.

„Da sagen Sie mir doch“, begann er wieder, „ist es denn wahr, was
die Zeitungen jetzt so viel schreiben, daß es dort auch Nume gibt?
Ich meine, so wie wir, mit Vernunft?“

„Etwas Vernunft mögen sie schon haben.“

Der Bergmann schüttelte den Kopf. „Viel wird es wohl nicht sein“,
sagte er. „Warum wären sie sonst nicht schon zu uns gekommen? Wir
glauben nämlich hier nicht recht daran, daß dort viel zu holen ist,
wir meinen, die Regierung nimmt nur jetzt den Mund recht voll, weil
nächstes Jahr Wahlen zum Zentralrat sind. Da heißt es, wenn wir auf
die Erde gehen, da können wir die Sonne sozusagen mit Händen
greifen, da bekommen wir soviel Geld, daß jeder den doppelten
Staatszuschuß erhält.“

Saltner zuckte plötzlich zusammen und wandte sich ab. Ohne daß die
Dämmerung sich merklich verstärkt hätte, hatte unvermittelt ein
blendender Sonnenstrahl seine Augen getroffen. Das aufgehende
Gestirn beschien die Terrasse, und bald verbreitete sich sein Licht
auch über die tieferliegenden Lande.

Der Bergmann verabschiedete sich, er müsse nun an die Arbeit.
Saltner begleitete ihn noch ein Stück. So stark wirkte die
Sonnenstrahlung, daß schon jetzt Saltner seinen Pelz nicht ertragen
konnte. Er ließ ihn auf der Station zurück.

Die Nebel von den Höhen hatten sich verzogen. Saltner wandelte die
Lust an, die felsigen Abhänge hinaufzuklimmen. Das Steigen in der
geringen Schwere des Mars schien ihm ein Kinderspiel. Zunächst aber
ging er mit dem Bergmann bis an den Eingang des Stollens, in
welchem dieser zu tun hatte. Überall sah man auf der Terrasse diese
Öffnungen, die zu den Mineralschätzen des Berges führten.

Im Gespräch erfuhr Saltner, daß der Bergmann auf einige Zeit unten
im Lande gewesen war, um seinen Sohn zu besuchen, der auf der
Schule studierte, und daß man sich hier in der Tat wieder ganz
andere Vorstellungen von der Erde machte als im politischen Zentrum
des Planeten. Man glaubte, daß man nur nach der Erde zu gehen
brauche, um alsbald mit unermeßlichen Schätzen zurückzukehren. Die
Jugend hatte sich daher massenhaft gemeldet, um nach der Erde
mitgenommen zu werden. Der Bergmann verhielt sich dagegen durchaus
skeptisch und hatte seine Reise hauptsächlich unternommen, um
seinen Sohn von der beabsichtigten Erdfahrt zurückzuhalten. Er sah
jetzt, daß er sich die Mühe hätte sparen können, denn die Regierung
hatte alle diese Meldungen rundweg abgeschlagen.

Eine andere Maßregel aber hatte die Erdkommission getroffen, von
der Saltner nur durch diese zufällige Unterhaltung erfuhr. Die
Marsstaaten besaßen zwar ein stehendes Arbeitsheer, aber keine
Soldaten, da Kriege und kriegerische Übungen bei ihnen als eine
längst veraltete Barbarei galten. Sie hatten nur eine Art
Polizeitruppe zur Aufrechterhaltung der Ordnung in besonderen
Fällen.

Es entstand nun die Verlegenheit, woher die Leute zu nehmen seien,
welche das technische Personal unterstützen sollten, falls es zu
einem wirklichen Krieg mit den Menschen, zu einer längeren
militärischen Aktion auf der Erde kommen sollte. Dazu gehörte eine
Gewöhnung an große körperliche Strapazen, eine Abhärtung, wie sie
die Martier im allgemeinen nicht besaßen. Man hatte deswegen an die
kühnen und rauhen Bewohner der Wüsten, an die Beds gedacht. Man
wollte dieselben anwerben und für den Dienst auf der Erde
ausbilden. Die Aufforderung an sie war ergangen. Diese Nachricht
erfüllte Saltner mit Besorgnis. Von diesen Leuten war zu
befürchten, daß sie als Sieger ein weniger zartes Gewissen haben
würden als die eigentlichen Träger der Kultur, die hochgebildeten
Nume. Er sah sich dadurch nur in seiner Absicht bestärkt, seine
Landsleute vor der Größe der drohenden Gefahr zu warnen.

Der Bergmann war an seinem Ziel. Er empfahl Saltner, wenn er das
Plateau der Wüste selbst besuchen wolle, bis zur nächsten Station
der Terrassenbahn zu fahren und die von dort nach oben führende
Bergbahn zu benutzen. Auf keinen Fall solle er sich vom Rand der
Wüste entfernen, da auf derselben nichts zu finden sei als die
großen Strahlungsnetze und in einigen schwer zugänglichen
Schluchten die ärmlichen Wohnsitze der Beds.

Saltner befolgte den Rat insofern, als er die Terrassenbahn
benutzte und mit dieser ein weites Stück nach Süden fuhr. Unterwegs
brachte er nämlich in Erfahrung, daß er hier eine Station ›Kast‹
erreichen könne, welche direkt über Sei lag, so daß er von da aus
abwärts nur noch eine Viertelstunde bis zu Las Wohnort hatte. Auf
diese Weise stand ihm genügend Zeit zur Verfügung, um das Plateau
zu ersteigen. Allerdings führte von hier keine Bahn hinauf, aber es
lag ihm viel mehr daran, durch eine Fußwanderung die seltsame
Gebirgsbildung kennenzulernen.

In einer steil herab ziehenden engen Schlucht klomm er rasch
aufwärts. Einige unten beschäftigte Leute riefen ihm etwas nach,
das er nicht verstand, es schien ihm eine Warnung zu sein, nicht
mit so großer Geschwindigkeit aufwärts zu springen; aber diese
Martier konnten ja nicht wissen, daß er auf Erden gewohnt war, ein
dreimal so großes Gewicht auf noch ganz andere Höhen zu schleppen.
Die Wände der Schlucht verdeckten ihm zwar die Aussicht nach der
Seite und, da die Schlucht nicht gerade verlief, auch nach oben und
unten, aber sie schützten ihn dafür vor den Strahlen der Sonne. Und
er sah bald, daß er ohnedies nicht weit gekommen wäre. Denn wo die
Sonne das Gestein traf, glühte es so, daß man es mit der bloßen
Hand kaum berühren konnte. Im Schatten aber war die Luft kühl.

Etwa dreiviertel Stunden mochte er so gestiegen sein, als die Wände
der Schlucht sich verflachten; er näherte sich dem Rand des
Plateaus. Mitunter war es ihm, als höre er in der Ferne ein
Geräusch wie Donner, er schob es auf Sprengungen in den Bergwerken.
Jetzt hörte der Schatten auf. Zwischen Felstrümmern mußte er sich
emporarbeiten. Der Schweiß rann ihm von der Stirn, er empfand
heftigen Durst, und noch immer wollte sich die ebene Hochfläche
nicht zeigen. Da endlich erkannte er einen Gegenstand, der wohl nur
das Dach eines Gebäudes sein konnte. Er eilte darauf zu, und
plötzlich blickte er auf eine weite Ebene, nur hier und da von
einzelnen Felsriegeln unterbrochen. Eben wollte er, aus den
Felstrümmern des Absturzes heraussteigend, den Rand des Plateaus
betreten, als er sich durch einen Draht von weißer Farbe gehemmt
sah, der an diesem Rand sich hinzog. Er achtete nicht darauf,
sondern überstieg ihn. Die Sonne, gegen die kein Schirm ihn
schützte, brannte so furchtbar, daß er jeden Augenblick umzusinken
fürchtete und nur daran dachte, ein schattenspendendes Dach zu
gewinnen. Er sah jetzt das Haus dicht vor sich, und einige eilende
Sprünge brachten ihn in den Schatten eines Pfeilers.

Nachdem er sich hier einen Augenblick erholt, blickte er sich
erstaunt um. Wenn das ein Haus war, so war es ein sehr seltsames.
Wie eine Brücke ruhte es schwebend auf zwei schmalen Pfeilern. Es
hatte die Gestalt eines Bootes, auf das man ein zweites mit dem
Kiel nach oben gesetzt hatte. Dazwischen war ein etwa meterhoher
Zwischenraum, nach welchem eine Leiter hinaufführte. Saltner
überlegte.

„Das Ding sieht beinahe aus“, sagte er bei sich, „wie das
Luftschiff am Nordpol, das ich freilich nur sehr von weitem gesehen
habe. Ob das hier vielleicht so eine Art Trockenplatz für frischen
Anstrich ist? Ich möchte mir das Ding einmal von innen
betrachten.“

Da er ringsum niemand bemerkte und ihm der schmale Schatten des
Pfeilers keinerlei Bequemlichkeit bot, beschloß er die Leiter
hinaufzusteigen und sich in dem seltsamen Bau umzusehen. Er fand
jetzt, daß das, was er für einen leeren Zwischenraum gehalten
hatte, von einer durchsichtigen Substanz verschlossen sei, die
jedoch eine Öffnung am Ende der Leiter freiließ. Er stieg hinein.
Niemand befand sich hier. In der Mitte war ein freier Raum mit
Sitzen und Hängematten. Ringsum, unten, oben und besonders an den
Enden des länglichen Baus, waren Verschläge mit unbekannten
Apparaten. Drähte liefen von dort nach unten und durch die Pfeiler
jedenfalls nach dem Erdboden, wo sie unterirdisch weitergeleitet
werden mochten. Saltner hütete sich wohlweislich, irgend etwas zu
berühren. Es wurde ihm einigermaßen unheimlich. Aber er fühlte sich
so matt, daß er jedenfalls erst frische Kräfte sammeln mußte, ehe
er den Rückweg antreten konnte. Vorsichtig zog er an einer der
Hängematten, und da sich nichts in dem Raum rührte, legte er sich
hinein.

„Ich bin doch neugierig, was das für eine Medizin sein wird“,
dachte er. „Jetzt nur nicht die Zeit verschlafen, bloß einen
Augenblick ruhen.“ Aber erschöpft schloß er die Augen.

\section{38 - Gefährlicher Ruheplatz}

Eine Viertelstunde mochte er so im Halbschlummer gelegen haben, als
ein gewaltiger Krach ihn emporschrecken ließ. Der ganze Bau war in
eine zitternde Bewegung geraten. Eilends sprang Saltner empor und
schaute sich um. Auf dem Felsboden, vielleicht hundert Meter hinter
ihm nach dem Rand des Plateaus zu, lag eine gewaltige Staubwolke.
Jetzt krachte es auf der anderen Seite. Eine neue Wolke von
Trümmern und Staub erhob sich vom Boden.

„Da hat eine Granate eingeschlagen!“ sagte sich Saltner. Im Moment
war ihm die Situation klar. Die Schießversuche der Martier auf der
Wüste Gol! Er hatte gehört, daß die Martier, ihren Erfahrungen und
den von Ell mitgebrachten Büchern folgend, Geschütze konstruiert
hatten, die, in ihren Wirkungen wenigstens, den auf der Erde
üblichen glichen. Nun schossen sie mit menschlicher Artillerie nach
ihren eigenen Luftschiffen. Er saß also gerade in dem Ziel selbst
drin! Die erste Granate war zu weit gegangen, die zweite zu nahe,
die dritte würde sicherlich treffen. Und jetzt sofort mußte der
Schuß erfolgen! Da hatte er sich ja einen recht geeigneten Ort zur
Ruhe ausgesucht! Ob noch Zeit war, hinauszuspringen? Instinktiv
wollte er es tun, aber er faßte sich. Draußen war es offenbar noch
gefährlicher – die Martier erwarteten ja wohl, daß das Ziel
Widerstand leiste. Freilich, diese dünnen Wände! Jetzt sah er, wo
das Geschütz stand. Es blitzte auf. Er empfahl seine Seele Gott und
richtete seinen Blick standhaft gegen die Schußrichtung. Er hörte
das Heransausen des Geschosses. Und wie ein Wunder schien es ihm,
was er sah. Etwa zehn Meter vor seinem Standpunkt, in gleicher Höhe
wie das Schiff, in welchem er sich befand, wurde die Granate
sichtbar, weil sie plötzlich langsam heranschwebte. Noch auf fünf,
auf vier Meter näherte sie sich – Saltners Züge verzerrten sich
krampfhaft, aber er konnte den Blick von dem Verderben drohenden
Geschoß nicht abwenden. Jetzt stand es still, ohne zu explodieren –
und vor seinen Augen verschwand die stählerne Spitze, der
Bleimantel, die Sprengladung löste sich unschädlich auf und der
Rest des Geschosses, zu einer mürben Masse zersetzt, senkte sich
langsam, wie ein Häufchen Asche, zu Boden.

Saltner glaubte zu träumen. Aber schon vernahm er das Heransausen
einer zweiten Granate. Dasselbe Schauspiel – nahe vor der Spitze
des Schiffes, gegen welche sie gerichtet war, verzehrte sie sich in
der freien Luft. Und so ein drittes und viertes Mal.

Für seine Person fühlte er sich jetzt im Augenblick sicher. Aber
wie gebrochen sank er auf eine Bank. Mit tiefem Schmerz gedachte er
der Menschheit, deren gewaltigste Kampfmittel vor der Macht dieser
Nume wirkungslos in nichts zerflossen. Er hatte wohl gesehen, daß
diese letzte Probe mit einem jener Riesengeschosse angestellt
worden war, denen die stärkste Panzerplatte nicht standhält. Aber
auch dieses war in der freien Luft vor seinen Augen verschwunden.
Es mußte sich in der Entfernung von drei bis vier Meter vor dem
Schiff eine unsichtbare Macht befinden, die jede Bewegung und jeden
Stoff vernichtete.

Ein eigentümliches Zittern hatte während der ganzen Beschießung in
dem Schiff geherrscht, und es schien ihm, als wenn auch die
Sonnenstrahlung rings um das Schiff matter wäre. Das hörte nun auf.
Bald sah er, wie sich über die Ebene eine Art von gedecktem Wagen
heranbewegte. Ohne Zweifel wollten die Schützen die Wirkung ihrer
Versuche in Augenschein nehmen.

Hier entdeckt zu werden, war Saltner im höchsten Grade bedenklich.
Er war sicher, daß man ihn als Spion behandeln und nicht glimpflich
mit ihm verfahren würde. Ehe er seine Unschuld dartun konnte, hätte
er mindestens viel Zeit verloren. Auf jeden Fall wäre seine Absicht
vereitelt worden, heute noch La seine Briefe zu überreichen. Und
doch war ihm jetzt mehr als je daran gelegen, seinen Landsleuten
mitzuteilen, daß ein kriegerischer Widerstand gegen die Martier
aussichtslos sei. Wenn er entfloh? Aber den Rand der Schlucht
konnte er nicht mehr erreichen, ohne gesehen zu werden. Und auf der
flachen Ebene war kein Versteck. Doch vielleicht im Schiff selbst?
Es war wenigstens das einzige, was er versuchen konnte. Es gab da
verschiedene Seitenräume – freilich, man würde sie wohl bei der
Untersuchung betreten. Sein Blick fiel auf den Fußboden. Hier war
eine Falltür. Zum Glück kannte er jetzt den üblichen Mechanismus
des Verschlusses. Er kroch in den unteren Raum, der offenbar zur
Aufbewahrung von Vorräten diente. Jetzt war er leer bis auf einige
Haufen eines heuähnlichen Stoffes, den Saltner nicht kannte. Aber
er hatte keine Wahl, er kroch in eine Ecke und versteckte sich.
Wenn man das Heu, oder was es war, nicht durchwühlte, konnte man
ihn nicht finden.

Inzwischen war der Wagen angelangt, und die Martier stiegen aus. Es
waren nur vier Männer und eine Frau. Sie betrachteten zufrieden die
Aschenrestchen der Geschosse, stiegen in das Schiff und überzeugten
sich, daß es vollkommen unversehrt war. Keines der feinen
Instrumente hatte einen Schaden erlitten. Saltner hörte, wie sie
das Schiff wieder verließen. Schon glaubte er sich gerettet. Er
lauschte aufmerksam, konnte aber nur hören, daß eine Unterhaltung
geführt und Anweisungen erteilt wurden, ohne daß er die Worte zu
verstehen vermochte. Dann vernahm er deutlich, wie der Wagen sich
wieder entfernte.

Er verließ sein Versteck. Alles war still. Vorsichtig öffnete er
die Falltür: das Schiff war leer. Er näherte sich der
Aussichtsöffnung und spähte nach dem sich entfernenden Wagen. Jetzt
konnte er versuchen, den Rand des Plateaus zu gewinnen. Er wandte
sich um und schritt nach dem Ausgang zu. In diesem Augenblick
erschien in demselben eine weibliche Gestalt. Saltner prallte
zurück, dann stürzte er wieder vorwärts – diese einzelne Martierin
konnte ihn nicht aufhalten. Er wollte an ihr vorüber, die,
ebenfalls erschrocken, zur Seite trat. Schon stand er an der
Öffnung, da hörte er seinen Namen.

„Sal, Sal! Was haben Sie hier zu tun?“

Er drehte sich um und erkannte Se. Sie faßte seine Hände und zog
ihn zurück.

„Oh“, sagte sie, „mein lieber Freund, warum müssen wir uns hier
treffen? Das durften Sie nicht sehen! Wie konnten Sie sich
hierherwagen?“

„Ich bin unschuldig, teure Se, glauben Sie mir, ich bin durch
Zufall hierhergeraten.“

„Wie sind Sie über den weißen Draht gekommen? Wissen Sie denn
nicht, was das bedeutet?“

„Ich bin einfach darübergestiegen –“

„Und haben die Gesetze verletzt und sich der höchsten Lebensgefahr
ausgesetzt.“

„Ich bedaure meine Unwissenheit. Und ich hoffe, ich darf Sie bald
in sicherer Lage wieder sprechen. Jetzt verzeihen Sie wohl, wenn
ich mich so schnell wie möglich davonmache.“

„Das geht ja nicht, Sal, das darf ich nicht zugeben – so sehr ich
es Ihnen wünschte. Aber ich bin hier nicht privatim, ich habe das
Nihilitdepot zu verwalten, ich darf Sie nicht freilassen, das hängt
nicht mehr von mir ab.“

„Aber von mir! Leben Sie wohl, auf Wiedersehen!“ Er schwang sich
auf die Leiter.

„Um Gottes willen, Sal!“ rief Se. „Keinen Schritt von hier, es ist
Ihr Verderben! Ich muß Sie festhalten!“

„Wie wollen Sie das?“ rief er lachend.

„Ich drehe diesen Zeiger, und der Nihilitpanzer bildet sich um das
Schiff. Es ist ein Spannungszustand des Äthers, der momentan jede
Kraft vernichtet, jedes Geschehen aufhebt. Alles, was in seinen
Bereich gerät, verzehrt sich, jede Energie wird ihm entzogen, es
schwindet in nichts. Da, sehen Sie!“

Das eigentümliche Zittern und die Trübung des Lichtes begann
wieder. Se ergriff einen Hammer, der im Schiff lag, und schleuderte
ihn durch die Öffnung hinaus. In etwa drei Meter Entfernung
verschwand er spurlos.

„Sie können nicht fort“, sagte sie. „Kommen Sie herein.“

Saltner setzte sich. Beide sahen sich traurig an. Er ergriff Ses
Hände. „Wenn ich Sie bitte“, sagte er. „Bei unserer Freundschaft!
Ich muß jetzt fort! Hören Sie mich!“

Er erzählte, was ihn herbeigeführt, daß er La sprechen müsse, was
er von ihr wünsche. Las Briefe nach der Erde würden nicht
kontrolliert, sie konnte die seinigen an Grunthe adressieren – –

Se schüttelte traurig den Kopf.

„Das kann La nicht tun, das wird sie nie tun, sie darf es
ebensowenig wie Ell. Bitten Sie sie nicht erst – Saltner, sie will
nicht darum gebeten sein.“

„Wie kann sie wissen?“

„Haben Sie das nicht herausgehört aus dem, was sie Ihnen sagte?
Wenn nun Ell mit ihr gesprochen hätte, ehe sie in Ihre Wohnung
ging, wenn er Ihre Absicht ihr mitgeteilt hätte – während Sie von
Ell nach Hause fuhren, war Zeit genug dazu. Und etwas Derartiges
hat sie sicher seit Tagen erwartet, das war doch leicht zu ahnen.
Warum ist sie fortgezogen, und warum sollen Sie nicht nach Sei
kommen? Weil La den Konflikt voraussah. Sie war in Widerspruch mit
sich selbst. Sie wollte die Bitte vermeiden, die sie Ihnen
abschlagen mußte. Und vielleicht – doch ich habe kein Recht, in Las
Gefühle zu dringen.“

Saltner klammerte sich an Ells Namen. Er also war ihm
zuvorgekommen! Und es erschien ihm, als gelte es nur Ells Einfluß
zu besiegen.

„Ich muß zu ihr!“ rief er verzweifelt. „Se, ich beschwöre Sie,
lassen Sie mich frei!“

„Ich darf ja nicht. Und Sie werden es mir noch danken, Saltner. La
liebt Sie, vielleicht mehr, als Sie ahnen, sie wird es nicht
ertragen, daß Sie in Trauer, in Zorn, in Verbitterung von ihr
gehen, weil sie Ihrem Wunsch nicht folgen kann. Wenn Sie an der
Ausführung Ihres Willens verhindert werden, so zürnen Sie lieber
mir!“

„Und wenn ich Sie bäte, Se, die Briefe zu befördern, würden Sie es
mir auch abschlagen?“

„Ich müßte es.“

Sie war aufgestanden und blickte auf die Ebene hinaus. Dann wandte
sie sich zurück und trat dicht an ihn heran, mit ihren großen Augen
ihn zärtlich anblickend.

„Mein lieber Freund, seien Sie vernünftig. Der Wagen mit meinen
Begleitern kommt zurück. Ich war hiergeblieben, um den
Nihilitapparat neu zu laden, und jene hatten nur frischen Vorrat zu
holen. Ihre Unwissenheit wird Sie entschuldigen. Man wird Sie
höchstens nach Kla zurückschicken. Aber ich darf nicht eigenmächtig
handeln. Zürnen Sie mir nicht!“

Saltner sah, daß der Wagen in der Ferne auftauchte. Fünf Minuten
mußten sein Schicksal entscheiden. Einen Moment zögerte er unter
Ses mächtigem Einfluß. Aber er raffte sich zusammen; sein Entschluß
war gefaßt.

„Ich zürne Ihnen nicht, geliebte Se“, sagte er. „Nur mögen Sie mir
nicht zürnen, aber ich kann nicht anders. Leben Sie wohl!“

Er umschlang sie fest mit seinem linken Arm, indem er mit der
rechten Hand den Zeiger des Nihilitapparates zurückdrehte. In ihrer
Überraschung und dem Bestreben, sich ihm zu entwinden, hatte Se
dies gar nicht bemerkt. Er drückte einen flüchtigen Kuß auf ihre
Stirn und schwang sich mit einem Satz aus der Öffnung. Da wußte
sie, was geschehen war. Im Augenblick, als Saltner den Boden
erreichte, berührte Ses Hand wieder den Zeiger. Drückte sie ihn
herum, so verzehrte das Nihilit den Freund. Und wenn sie es nicht
tat, so hatte sie einen Verräter entfliehen lassen.

Sie preßte die Hände an ihre Stirn – nur einen Augenblick – dann
schaute sie auf.

In weiten Sätzen entfernte sich Saltner und verschwand hinter den
Felstrümmern am Abhang der Wüste. – Wie er den Berg hinabgelangte,
er wußte es kaum. Am meisten fürchtete er, am Ausgang der Schlucht
von den dort beschäftigten Martiern angehalten zu werden. Er umging
ihn durch eine halsbrecherische Kletterei. Völlig erschöpft
gelangte er in die Restauration neben dem Bahnhof. Hier in dem
kühlen, separaten Speisezimmer, das er sich anweisen ließ, fand er
Zeit, sich zu erholen.

Wenn ihn Se verraten hatte, so war freilich seine Flucht nutzlos.
Man würde ihn in Sei, oder wohin er auch sonst sich wandte,
erreichen. Aber er vertraute darauf, daß Se nicht sprechen würde.
Niemand sonst hatte ihn oben gesehen. So benutzte er den zu Tal
gehenden Wagen nach Sei und fand nach einigem Umherirren die von La
angegebene Platznummer. Eben entfernten sich die Monteure, welche
das neu eingetroffene Haus an die verschiedenen im Boden liegenden
Leitungen angeschlossen hatten.

Es war die Zeit, um welche La mit ihm sprechen wollte, als Saltner
in ihr Zimmer trat.

„Da bin ich selbst!“ rief er. „Ich mußte dich wiedersehen!“

La stand wortlos. Dann atmete sie tief auf, preßte die Hände
zusammen und sagte leise:

„O mein Freund, warum hast du mir dies getan?“

„Warum nicht? Ich sehnte mich nach dir, La, und ich bedarf deiner
Hilfe.“

„Meiner Hilfe?“ sagte sie warm. Sie hoffte einen Augenblick, es
könne sich um etwas anderes handeln, als was sie fürchtete.

„Wenn es mir möglich ist, wie gern bin ich dir zu Diensten.“ Sie
zog ihn neben sich auf einen Sessel. Er hielt ihre Hand fest.

„Ich habe eine große Bitte, für Frau Torm und für mich.“

La wich zurück. „Sprich sie nicht aus! Ich bitte dich, sprich sie
nicht aus, damit dich meine Weigerung nicht kränkt. dots{}“

„Du weißt –?“

„Ich weiß, um was es sich handelt.“

„Von Ell!“

„Durch ihn. Sieh, das ist unmöglich! So wenig du damals am Nordpol
der Erde zögertest, die Pflicht für dein Vaterland zu erfüllen, so
wenig kann ich jetzt um deinetwillen das Gesetz durchbrechen. Das
Gesetz verbietet den Menschen, unkontrollierte Botschaft nach der
Erde zu senden. Hätte ich die freie Überzeugung, daß es ungerecht
und töricht sei, so dürfte ich mein Gewissen fragen, ob ich es
übertreten will. Es wäre ein Konflikt, aber ich könnte ihn auf mich
nehmen. Doch ich kann mich davon nicht überzeugen. Was ihr auch
berichtet, es kann nur Verwirrung anstiften, und Ismas private
Wünsche können nicht in Frage kommen.“

Saltner hatte ihre Gründe kaum gehört. Er blickte finster vor sich
hin.

„Durch Ell!“ sagte er dann bitter. „Natürlich, wann spräche er
nicht mit dir, wann träfe ich ihn nicht bei dir, wann hörtest du
nicht auf ihn mehr als auf mich?“

La seufzte. „Ich wußte es ja, daß es so kommen würde. Oh, hättest
du auf meinen Rat gehört und wärest nicht hergereist.“

„Ich werde dich nicht stören; sobald Ell kommt, gehe ich.“

„Warum? Er wird wohl kommen. Aber warum entrüstest du dich? Hast du
je bemerkt, daß ich dich weniger liebe?“

„Aber du liebst ihn?“

La sah ihn mit flammenden Augen an.

„Wie darfst du fragen“, sagte sie stolz, „was kaum das eigene Ich
sich fragt?“

Aber ihr Ausdruck wurde plötzlich unendlich traurig und zärtlich.
Sie faßte seine Hände und neigte sich zu ihm.

„Aber wie kann ich dir zürnen?“ sagte sie. „Mich nur müßte ich
schelten. Doch habe ich dir nicht gesagt: Vergiß nicht, daß ich
eine Nume bin? Ach, ich vergaß wohl, daß du ein Mensch bist, und du
weißt nicht mehr, was ich dir sagte: Liebe darf niemals unfrei
machen! Und du willst mich unfrei machen? Willst dem Gefühl
gebieten? Ist ein Nume so klein und einfach, daß ein einzelner
seinen Kreis erfüllen könnte? Ist nicht jedes Individuum nur ein
kleiner Ausschnitt, nur eine Seite von dem, was das Wesen des
Mannes, das Wesen der Frau ist? Wer kann sagen, ich repräsentiere
alles, was du lieben kannst?“

„Das also war es! Was vermag ich dagegen? Daß du eine Nume bist,
wußte ich, und ich wußte, daß du mir nicht angehören könntest fürs
Leben. Aber so dachte ich mir deine Liebe nicht. O La, ich weiß
nicht, wie ich ohne dich leben werde, aber deine Liebe teilen – mit
jenem –, das vermag ich nicht. Ich bin ein Mensch, und wenn du ihn
liebst, so muß ich scheiden.“

Saltner saß stumm. Er konnte sich nicht aufraffen zu gehen, es war
ihm, als müßte La ihn noch halten, er hoffte auf ein Wort von ihr.
Auch sie schwieg, sie atmete lebhaft, mit einem Entschluß kämpfend.
Dann sagte sie zögernd:

„Das glaube nicht, Sal, daß Ell dabei im Spiel ist, wenn ich dir
deine Bitte wegen der Briefe abschlage. Daß er mich
benachrichtigte, war nur zu unserm Besten, wenn du mir gefolgt
hättest. Ich wollte einer Auseinandersetzung ausweichen, weil ich
wußte, daß sie dich kränken müßte, daß du mich mißverstehen und an
meiner Liebe zweifeln würdest – nach Menschenart – und weil – weil
ich selbst nicht wußte, wie ich dies ertragen könnte. Ja, Sal, um
meinetwillen wollt’ ich dich nicht sehen –“

Saltner kniete zu ihren Füßen und schlang die Arme um sie.

„O La!“ rief er, „so habe ich noch die Hoffnung, daß du mich
erhörst, daß du meine Bitte erfüllst?“

„Du weißt nicht, was du verlangst, weißt nicht, welch namenlose
Qual diese Stunde mir bereitet. Du verlangst mehr als mein Leben,
du verlangst meine Freiheit, meine Numenheit. – Wenn ich dir
nachgebe, wenn ich diesem Rausch der Gegenwart unterliege – o mein
Freund –, dann bin ich keine Nume mehr, dann bin ich ein Mensch!
Aus dem reinen Spiel des Gefühls verfalle ich in den Zwang der
Leidenschaft, die Freiheit verlöre ich und müßte niedersteigen mit
dir zur Erde. Und kann deine Liebe das wollen?“

Saltner barg sein Haupt zwischen den Händen, seine Brust hob sich
krampfhaft.

„Verzeihe mir, La, verzeihe mir“, kam es endlich von seinen
Lippen.

La nahm seinen Kopf zwischen ihre Hände und blickte ihn an, ihre
Augen strahlten in einem verklärten Glanze.

„Du sollst es wissen, mein Freund“, sagte sie langsam, „ich liebe
Ell nicht, ich liebe nur dich.“

„La!“ hauchte er selig.

Tränen traten in ihre Augen, und mit gebrochener Stimme sagte sie:
„Und dies ist das Schicksal, das uns trennt.“

Er sah sie sprachlos an.

„Ich bin eine Nume, und weil ich ihn nicht liebe, weil ich fühle,
daß ich ihn nicht lieben kann, darum müssen wir scheiden. – Darum
müssen wir scheiden“, wiederholte sie leise, „denn in dieser Liebe
zu dir verlöre ich meine Freiheit. Was ich heute sprach, darfst du
nie wieder hören. Steh auf, mein Freund, steh auf und glaube mir!“

Saltner wußte nicht, wie ihm geschah. Er stand vor ihr, er begriff
sie nicht und wußte doch, daß es nicht anders sein konnte.

„Ob wir uns wiedersehen, weiß ich nicht. Jetzt nicht, jetzt lange
nicht.“ – Sie schluchzte auf und schlang die Arme um seinen Hals.
Lange standen sie so.

„Noch diesen einen Kuß! Leb wohl, leb wohl!“

La riß sich von ihm los.

„Leb wohl“, sagte er wie geistesabwesend. Dann schloß sich die Tür
hinter ihm. Mechanisch suchte er seinen Hut und schritt aus dem
Haus.

\section{39 - Die Martier sind auf der Erde!}

Auf der Erde hatte die Nachricht von der Besetzung des Nordpols
durch die Martier und der Existenz eines Luftschiffes, mit welchem
sie siebenhundert Kilometer in der Stunde in der Erdatmosphäre
zurückzulegen vermochten, ein Aufsehen erregt wie kaum ein anderes
Ereignis je zuvor. Der Bericht Grunthes und die von ihm vorgelegten
Beweise ließen keinen Zweifel zu, überdies war das Luftschiff in
Italien, der Schweiz, Frankreich und England gesehen worden, ja,
die Ankunft Grunthes und das Verschwinden Ells und Frau Torms waren
auf keine andere Weise zu erklären. Die Schriften Ells, welche
jetzt herauskamen, gaben eine hinreichende Auskunft über die
Möglichkeit technischer Leistungen, wie sie von den Martiern
vollzogen wurden.

Als daher Kapitän Keswick, sobald er mit der ›Prevention‹ die erste
Telegraphenstation berührte, seinen Bericht an die englische
Regierung abgab und Torm nach Friedau telegraphierte, daß er
glücklich gerettet sei, erregten diese Nachrichten schon nicht mehr
die Verwunderung, die man auf der ›Prevention‹ erwartet hatte. Wohl
aber wurde in England die anfänglich für die Martier vorhandene
Begeisterung stark abgekühlt und machte einer in der Presse sich
äußernden, etwas bramarbasierenden Entrüstung Platz, daß man diesen
Herrn vom Mars doch etwas mehr Respekt vor der britischen Flagge
beibringen müsse. Indessen fehlte es nicht an Stimmen, die zur
äußersten Vorsicht rieten und die Gefahren ausmalten, welche den
Nationen des Erdballs von einer außerirdischen Macht drohten, der
so ungewöhnliche und unbegreifliche Mittel zur Durchsetzung ihres
Willens zu Gebote ständen wie den Martiern.

Diese Sorge, die Bedrohung durch eine unbestimmte Gefahr,
beherrschte das Verhalten der Regierungen aller zivilisierter
Staaten. Man wußte weder, was man zu erwarten habe, noch wie man
einem etwaigen weiteren Vorgehen der Martier begegnen solle. Ein
äußerst lebhafter Depeschenwechsel fand statt, man erwog den Plan,
einen allgemeinen Staatenkongreß zu berufen, und konnte sich
vorläufig nur noch nicht über das vorzulegende Programm und den Ort
des Zusammentritts einigen. Während man sich auf der einen Seite
einer gewissen Solidarität der politischen Interessen aller Staaten
gegenüber den Martiern bewußt war, zeigten sich doch auf der andern
Seite sehr verschiedene Auffassungen über den zu erwartenden
kulturellen Einfluß der Martier. Die Presse aller Nationen
beschäftigte sich aufs eifrigste mit der Mars-Frage, und eine
unübersehbare Menge von Meinungen und abenteuerlichen Hypothesen
erfüllte die Blätter und erhitzte die Gemüter.

Die Quelle aller dieser Erwägungen war das Buch von Ell über die
Einrichtungen der Martier und die Erklärungen, welche Grunthe aus
seinen Erfahrungen am Nordpol dazu geben konnte. Ein Verständnis
derselben, wenigstens im größeren Publikum, war jedoch nicht zu
erreichen. Der Sprung von der technischen und sozialen Kultur der
Menschen zu der Entwicklung, welche diese bei den Martiern erreicht
hatte, war zu groß, als daß man sich in letztere hätte finden
können. Gerade die ersten Mahnungen Grunthes, man möge sich unter
keinen Umständen in einen Konflikt mit den Martiern einlassen, weil
ihre Macht alle menschlichen Begriffe überstiege, fanden am
wenigsten Gehör; dazu waren sie schon viel zu wissenschaftlich in
der Form.

Man stellte sich wohl vor, daß sich die Martier durch wunderbare
Erfindungen eine ungeheure Macht über die Natur angeeignet hätten,
aber man hatte keinerlei Verständnis dafür, wie ihre ethische und
soziale Kultur sie den Gebrauch dieser Macht benutzen, mäßigen und
einschränken ließ. Vor allem blieb das eigentliche Wesen ihrer
staatlichen Ordnung trotz der Erläuterungen in Ells Buch ein
Rätsel. Die individuelle Freiheit war so überwiegend, die
Entscheidung des einzelnen in allen Lebensfragen so ausschlaggebend
und so wenig von staatlichen Gesetzen überwacht, daß vielfach die
Ansicht ausgesprochen wurde, das Gemeinschaftsleben der Martier sei
durchaus anarchistisch. In der Tat, die Form des Staates war auf
dem Mars an kein anderes Gesetz gebunden als an den Willen der
Staatsbürger, und so gut ein jeder seine Staatsangehörigkeit
wechseln konnte, so konnte auch die Majorität, ohne in den Verdacht
der Staatsumwälzung oder der Staatsfeindschaft zu kommen, von
monarchischen zu republikanischen Formen und umgekehrt übergehen.
Keine Partei nahm das Recht in Anspruch, die alleinige Vertreterin
des Gemeinschaftswohls zu sein, sondern in der gegenseitigen, aber
nur auf sittlichen Mitteln beruhenden Messung der Kräfte sah man
die dauernde Form des staatlichen Lebens. Es gab keinen regierenden
Stand, so wenig es einen allein wirtschaftlich oder allein bildend
tätigen Stand gab. Vielmehr war zwischen diesen Berufsformen ein
stetiger Übergang, so daß ein jeder, ganz nach seinen Fähigkeiten
und Kräften, diejenige Betätigungsform erreichen konnte, wozu er am
besten tauglich war. Dies war freilich nur möglich infolge des
hohen ethischen und wissenschaftlichen Standpunktes der
Gesamtbevölkerung, wonach die Bildungsmittel jedem zugänglich
waren, aber von jedem nur nach seiner Begabung in Anspruch genommen
wurden. Natürlich bedeutete das nicht die Herrschaft des
Dilettantismus, sondern jede Tätigkeit setzte berufsmäßige Schulung
voraus, der Eintritt in höhere politische Stellen vor allem eine
tiefe philosophische Bildung. Aber der Fähige konnte sie erwerben.
Und dies beruhte wieder darauf, daß die Beherrschung der Natur
durch Erkenntnis die unmittelbare Quelle des Reichtums in der
Sonnenstrahlung erschlossen hatte.

Andere wieder behaupteten, die Staatsform der Martier sei durchaus
kommunistisch. Auch hierfür schien manches zu sprechen. Denn wenn
auch, was Ell nicht genügend hervorgehoben hatte, die Verwaltung
der großen Betriebe der Strahlungssammlung, des Verkehrs und so
weiter tatsächlich in der Hand von Privatgesellschaften lag, so war
doch das Anlagekapital Staatseigentum. Es existierte auch eine
staatliche Konzentration der wirtschaftlichen Tätigkeit, obwohl
diese der Arbeit des einzelnen völlig freie Hand ließ und
keineswegs die Güterproduktion durch Vorschriften regelte. Aber die
Zentralregierung, deren Mitglieder auf eine zwanzigjährige
Amtsdauer erwählt wurden, setzte unter Einwilligung des Parlaments
einen ›Strahlungsetat‹ fest, das heißt, es war dadurch für ein Jahr
im voraus bestimmt, welches Maximum von Energie der Sonne
entnommen, also auch welches Maximum mechanischer Arbeit auf dem
Planeten geleistet werden konnte. Sie setzte auch ein bestimmtes
Kapital fest, das jeder als ein zinsloses Darlehen in Anspruch
nehmen konnte, falls seine eignen Arbeitsmittel durch ungünstige
Verhältnisse in Verlust geraten waren. Im übrigen aber war ein
jeder auf seinen eigenen Fleiß angewiesen.

Auf dem Kulturstandpunkt der Menschheit erschienen die
Einrichtungen des Mars als Utopien, und mit Recht; denn sie setzten
eben Staatsbürger voraus, die in einer hunderttausendjährigen
Entwicklung sich sittlich geschult hatten und theoretisch an der
rechten Stelle alle die Mittel gleichzeitig zu benutzen wußten,
deren Gebrauch im Lauf der sozialen Lebensformen nach irgendeiner
Seite erprobt worden war. Ein Teil der Regierungen der Erdstaaten
befürchtete nun, daß das Beispiel der Martier die Veranlassung zu
übereilten Reformen, vielleicht zu gewaltsamen Umwälzungen geben
würde. Die agrarische Bevölkerung geriet in Bestürzung über die
drohende Konkurrenz der Lebensmittelfabrikation ohne Vermittlung
der Landwirtschaft. Auf der anderen Seite begrüßten die
Arbeiterschaft und alle für schnellen Kulturfortschritt
enthusiasmierten Gemüter die Martier als die Erlöser aus der Not,
deren Erscheinen nun bald bevorstünde. Durchweg aber war man im
unklaren, was geschehen würde und was geschehen solle.

Als im Oktober die Parlamente der meisten Staaten zusammentrafen,
gab es überall Interpellationen an die Regierungen über die
Marsfrage. Und überall lautete die Antwort ausweichend dahin, es
fänden Erwägungen statt über einen allgemeinen Staatenkongreß,
worüber man indessen Näheres noch nicht mitteilen könne. Überall
sprachen dann die Führer der verschiedenen Parteien die Ansichten
über den Mars aus, die sie vorher in ihren Blättern hatten drucken
lassen. Einige wollten die Martier enthusiastisch aufnehmen, andere
sie dilatorisch behandeln, andere sie überhaupt von der Erde
zurückweisen. Wie man das machen solle, wußte freilich niemand zu
sagen. Der Erfolg war jedoch in allen Staaten der gleiche: neue
Bewilligungen zur Vermehrung des Heeres und der Flotte.

Zum Glück für die Regierungen, die dadurch Zeit zur Beratung
gewannen, hörte man nun nichts mehr von den Martiern. Das
Luftschiff ließ sich nicht wieder sehen, die Martier schienen
verschwunden.

Da plötzlich kam im Januar die Nachricht vom Wiedererscheinen eines
Luftschiffs in Sydney. Am 2. Januar telegraphierte der Gouverneur
von Neusüdwales nach London, daß in Sydney mehrere Luftschiffe
eingetroffen seien, bestimmt, eine außerordentliche Gesandtschaft
der Marsstaaten nach London zu bringen, falls die englische
Regierung sich bereit erkläre, mit derselben wie mit der
bevollmächtigten Gesandtschaft einer anerkannten Großmacht zu
unterhandeln. Die Martier hatten sofort in Sydney einen berühmten
Rechtsanwalt als Agenten engagiert, der die Verhandlungen mit den
Behörden führte. Daß sie vom Mars mehr als 2.000 Kilogramm Gold in
Barren mitgebracht und bei der Bank of New South Wales deponiert
hatten, war eine so vorzügliche Empfehlung, daß ganz Neusüdwales
für sie eingenommen war.

Die diplomatischen Verhandlungen waren inzwischen nicht
weitergekommen. Auf Englands erneute Anregung einigte man sich
jetzt endlich dahin, daß man die Marsstaaten als politische Macht
anerkennen wolle, wenn sie gewisse Garantien gäben, daß sie sich
dem auf der Erde geltenden Völkerrecht unterwärfen. Daraufhin
beantwortete die englische Regierung die Depesche der Marsstaaten
im Prinzip bejahend, knüpfte aber verschiedene Bedingungen an die
Bewilligung weiterer diplomatischer Verhandlungen. Sie verlangte
von den Martiern außer der Anerkennung der völkerrechtlichen
Gewohnheiten der zivilisierten Erdstaaten, daß genau festgesetzt
werde, worüber mit der Gesandtschaft verhandelt werden solle, und
daß kein anderer Punkt zur Verhandlung käme, nachdem man die
Martier in London zugelassen habe. Ihrerseits versprach natürlich
die Regierung der Gesandtschaft den völkerrechtlichen Schutz auf
der Erde.

Der Bevollmächtigte der Marsstaaten, Kal, ging hierauf ohne
weiteres ein und stellte folgende Forderungen zur Verhandlung in
einer Depesche vom 22. Januar:

1) Formelle Entschuldigung der englischen Regierung wegen des
Angriffs, den die Mannschaft des Kanonenboots auf die beiden
Martier und der Kapitän auf das Luftschiff unternommen hatten.

2) Bestrafung des Kapitäns Keswick und des Leutnants Prim.

3) Entschädigung für die beiden Martier von je hunderttausend
Pfund.

4) Anerkennung der Hoheitsrechte der Marsstaaten auf die
Polargebiete der Erde jenseits des 87. Grades nördlicher und
südlicher Breite.

5) Anerkennung der Gleichberechtigung der Martier mit allen andern
Nationen in bezug auf Niederlassung, Verkehr, Handel und Erwerb.

Gleichzeitig depeschierte Kal an die Regierungen aller größeren
Staaten den Wunsch der Marsstaaten, über die beiden letzten Punkte
in Verhandlung zu treten.

Die Antworten ließen auf sich warten. Die Regierungen der Erde
verhandelten zunächst untereinander, da sie in ihren
vorangegangenen Verabredungen übereingekommen waren, gemeinsam
vorzugehen, falls die Martier mit allgemeinen Fragen des
internationalen Verkehrs an sie herantreten sollten. Die
Vereinigten Staaten, Frankreich, Italien und Japan traten dafür
ein, den Martiern entgegenzukommen, Deutschland, Österreich-Ungarn
und andere zögerten noch, Rußland verhielt sich ablehnend. Die
englische Regierung war zuerst geneigt, Verhandlungen einzuleiten.
Aber sobald die Forderungen der Martier in der Bevölkerung bekannt
geworden waren, erhob sich ein allgemeiner Entrüstungssturm. Das
Nationalgefühl forderte ungestüm die Ablehnung des Ansinnens der
Martier, das britische Selbstbewußtsein lasse nicht zu, daß man mit
einem Haufen Abenteurer in Verhandlungen über Entschuldigungen und
Entschädigungen trete. Es kam zu einer bewegten Parlamentssitzung,
in welcher das friedlich gestimmte Ministerium gestürzt wurde. Ein
Toryministerium, zu entschiedenem Vorgehen geneigt, trat an die
Stelle und erklärte sofort, daß es jede weitere Unterhandlung mit
den Marsstaaten zurückweise. Die ablehnende Note, welche nach
Sydney zur Mitteilung an den Gesandten der Marsstaaten geschickt
wurde, war in sehr kühlem und herablassendem Ton gehalten.

Die übrigen Staaten hatten jetzt, nachdem England eigenmächtig
vorgegangen war, keine Veranlassung, sich gegenseitig zu binden,
und erklärten nunmehr sämtlich im Prinzip sich zu Unterhandlungen
bereit, indem sie sich jedoch völlige Freiheit ihrer weiteren
Entschließungen vorbehielten.

Sobald die Martier in Sydney aus den Zeitungen, die sie aufs
sorgfältigste verfolgten, entnommen hatten, daß sie in England
vermutlich auf kein Entgegenkommen rechnen durften, sandte Kal nach
dem Mars die Lichtdepesche, derzufolge die verabredeten
Verstärkungen abzusenden seien. Ein Luftschiff vermittelte täglich
den Verkehr zwischen Sydney und dem Südpol, von dessen Außenstation
die Lichtdepeschen abgingen. Aber auch schon vorher hatte sich eine
ansehnliche Macht am Südpol angesammelt. Es waren drei neue
Raumschiffe angelangt, nachdem die früheren, um ihnen Platz zu
machen, zurückgegangen waren, und hatten neue Luftschiffe und
Mannschaften gelandet. Gegenwärtig befanden sich bereits
vierundzwanzig Luftschiffe am Südpol, sämtlich mit Nihilitpanzern,
Repulsitgeschützen und Telelyten ausgerüstet, eine furchtbare
Macht, deren militärischen Oberbefehl ein energischer Martier aus
dem Norden namens Dolf führte. Es ließ sich berechnen, daß binnen
vier Wochen die Streitmacht der Martier auf 48 Fahrzeuge
angewachsen sein würde. Mit dem letzten der Raumschiffe, dessen
Ankunft im März zu erwarten war, wollte Ill selbst eintreffen, um
die Leitung der Erdangelegenheiten zu übernehmen. Inzwischen hatte
man Kal eine Anzahl anderer bedeutender Männer zur Seite gestellt,
die als Gesandte an die Regierungen der Großmächte gehen sollten.

Als die Note der großbritannischen Regierung Kal übermittelt war,
telegraphierte sie dieser sofort nach dem Mars. Die Antwort traf
noch denselben Tag ein. Sie besagte nur, daß Kal genau nach den
Instruktionen verfahren solle, welche für den Fall einer
ablehnenden Haltung Englands festgesetzt seien. Am 15. März sei das
Hauptquartier nach dem Nordpol zu verlegen, woselbst im Laufe des
März nach und nach noch vierundzwanzig Raumschiffe mit
durchschnittlich je sechs Luftschiffen eintreffen würden. Damit
würde die Macht der Martier auf der Erde auf 144 große und eine
Anzahl kleinerer Luftschiffe mit 3.456 Mann gebracht sein, eine
Flotte, die den Martiern genügend schien, den Kampf im Notfall mit
der gesamten Erde aufzunehmen.

Die Note der englischen Regierung war vom 18. Februar datiert. Am
zwanzigsten erfolgte die Antwort Kals. Sie besagte, daß die
Regierung der Marsstaaten hiermit an die großbritannische Regierung
das Ultimatum richte, bis zum 1. März sämtliche gestellte
Forderungen zuzugestehen, widrigenfalls sich die Marsstaaten als im
Kriegszustand mit England betrachten würden. Diese Erklärung wurde
gleichzeitig allen andern Regierungen mitgeteilt.

Am 23. Februar drängte sich in Berlin auf der Wilhelmstraße, Unter
den Linden und vor dem königlichen Schloß eine ungeheure
Menschenmenge. Es hatte sich das Gerücht verbreitet, eine
Gesandtschaft der Martier sei eingetroffen, sie befinde sich im
Palais des Reichskanzlers und werde vom Kaiser empfangen werden.
Die Schaulust der Menge sollte jedoch nicht befriedigt werden,
dagegen wurde der gesamten Bevölkerung eine andere Überraschung
zuteil durch eine Nachricht, welche der Reichsanzeiger in einer
Extraausgabe brachte. Es wurde darin mitgeteilt, daß sich
allerdings in der Nacht eine Gesandtschaft der Martier in Berlin
befunden, die Stadt aber bereits am Morgen verlassen habe. Die
Beziehungen zur Regierung der Marsstaaten seien äußerst
freundliche, und man hoffe, daß auch ein Einvernehmen mit England
hergestellt werden würde. Bald darauf teilte der Telegraph aus
allen Hauptstädten ähnliche Nachrichten mit.

In aller Stille nämlich hatten die Martier mit den Mächten einzeln
verhandelt, und in der Nacht vom 22. zum 23. Februar waren
gleichzeitig in Washington, Paris, Berlin, Wien, Rom und Petersburg
Gesandtschaften der Martier heimlich eingetroffen, um durch
mündlichen Verkehr mit den leitenden Staatsmännern die Lage zur
Klärung zu bringen. In Berlin hatte ein Luftschiff mehrere Stunden
im Garten des Reichskanzlerpalais gelegen, und der martische
Gesandte hatte sich mit dem Reichskanzler besprochen. Aber weder
aus Deutschland noch aus irgendeinem andern Staat konnte man
erfahren, was der Gegenstand und das Resultat dieser Unterredungen
gewesen sei. Man vermutete, daß es sich um Erklärungen der Martier
über ihre Absichten und um die Vermittlung der Mächte zwischen den
Marsstaaten und Großbritannien handle. Man bezweifelte nicht, daß
die Martier friedliche Versicherungen gemacht hätten, aber man
setzte kein Vertrauen darauf, daß die Vermittlungsvorschläge der
Mächte bei England günstige Aufnahme finden würden.

Sie waren wohl auch hauptsächlich in der Absicht zugesagt, die
Geschäftswelt einigermaßen zu beruhigen; denn auf die erste
Nachricht vom Ultimatum der Martier hatten die Börsen aller Länder
mit einem gewaltigen Sturz aller englischen Werte geantwortet, und
die dadurch eingerissene Panik dauerte fort. Die Nachrichten aus
England aber wurden nicht günstiger. Die Stimmung war kriegerisch.
Nur wenige Blätter wagten einem Nachgeben gegen die Martier das
Wort zu reden, und sie wurden tumultuarisch überschrien. Krieg
gegen den Mars war die Losung geworden. Krampfhaft rüstete man in
Heer und Flotte, obwohl man nicht wußte, in welcher Form man einen
Angriff zu gewärtigen habe. Fieberhafte Tätigkeit herrschte in den
Arsenalen und Werkstätten, wo man hauptsächlich damit beschäftigt
war, die Konstruktion der Geschütze so umzuändern, daß sie eine
größere Elevation gestatteten. Denn man erwartete, den Kampf mit
einem Gegner führen zu müssen, der sich in der Luft befand. Man
tröstete sich mit der Sicherheit, daß die Martier jedenfalls nicht
imstande seien, außerhalb ihrer Luftschiffe irgend etwas
auszurichten, weil ihre Körper unter dem Einfluß der Erdschwere zu
Kraftleistungen, ja zur einfachen Bewegung untauglich seien. Man
hoffte daher, wenn man sich nur die Luftschiffe vom Halse halten
konnte, nichts Ernstliches zu befürchten zu haben und den auf der
Erde fremden Gegner bald zu ermüden.

\section{40 - Ismas Leiden}

Inzwischen war man auf dem Mars recht ungeduldig. Nachdem die
Abreise des ersten Raumschiffs sich bereits verzögert hatte,
vergingen weitere fünfundzwanzig Tage, bis die erste kurze
Lichtdepesche die glückliche Ankunft desselben auf der Außenstation
am Südpol der Erde meldete. Dann dauerte es wieder einige Tage, bis
man erfuhr, daß die übrigen Raumschiffe ebenfalls angelangt und die
Luftschiffe in Betrieb gesetzt seien. Die Verzögerung der Antwort
seitens der britischen Regierung wirkte verstimmend. Man war daher
angenehm überrascht, als man vernahm, daß die Regierung zu einem
tatkräftigen Vorgehen entschlossen war, und als das Ultimatum an
England bekannt wurde, wurden dem Zentralrat und insbesondere Ill
lebhafte Ovationen dargebracht. Die nach der Erde mit Verstärkung
abgehenden Schiffe wurden mit begeisterten Abschiedshuldigungen
gefeiert. Man bedauerte nur, daß die Nachrichten von der Erde so
kurz und spärlich waren, weil man auf den schwierigen Verkehr durch
Lichtdepeschen angewiesen war.

Mit Spannung sah man der Rückkehr des ersten Raumschiffes entgegen,
welches ausführlichere Nachrichten bringen mußte. Aber da die
Planeten jetzt von Tag zu Tag sich weiter voneinander entfernten,
dauerte die Überfahrt länger. Jetzt war seine Ankunft indessen
jeden Tag zu erhoffen. Ill wollte nur dieses Ereignis abwarten, um
sich selbst nach der Erde zu begeben.

Niemand aber ersehnte die Ankunft des Schiffes ungeduldiger als
Isma. Sollte es ihr doch Nachrichten von der Erde bringen. Sie
wußte zwar, daß sie mit diesem Schiff noch keinen Brief von ihrem
Mann erhalten konnte, denn es hatte die Erde verlassen, ehe eine
Antwort auf ihr Schreiben in Sydney eintreffen konnte. Aber sie
hoffte auf Zeitungen, die ja über die Rückkehr Torms Auskunft geben
mußten.

Isma lebte einsam und traurig in Ills Haus, und alle Bemühungen der
guten Frau Ma, sie zu erheitern, waren vergeblich. Ell begleitete
Ill auf seinen häufigen Reisen nach dem Südpol und der
Schiffsbaustätte. Bei Isma ließ er sich nicht mehr sehen, und
heimlich bereute sie ihre leidenschaftliche Trennung von dem alten
Freund. La war in der Ferne. Zu andern Martiern vermochte sie in
kein vertrauteres Verhältnis zu kommen. Ihr einziger näherer Umgang
war Saltner, der seinen Sprachunterricht in Kla wieder aufgenommen
hatte. Aber auch er war nicht mehr der übermütige, lustige Mann wie
früher, und Isma bemerkte wohl, daß ihn noch eine andere Sorge
drückte als das Heimweh und der Kummer um das Schicksal der
Menschen. Und doch war es schon schwer genug, hier in der
Verbannung zu leben, während das Vaterland in drohendster Gefahr
schwebte.

Und endlich, heute war die Depesche gekommen, daß das Raumschiff in
der Nacht gelandet sei. Kaum vermochte Isma ihre Aufregung zu
beherrschen. Doch die Aufgaben des Tages mußten erledigt werden,
sie zwang sich zur Ruhe, obwohl sie bei jedem Geräusch hoffte, man
bringe die ersehnten Nachrichten.

\tb{}
Die französische Konversationsstunde war beendet. Isma schloß die
Klappe des Fernsprechers und setzte sich an ihren Schreibtisch. Er
war ein Geschenk Ells, der ihn nach dem Muster ihres Schreibtisches
in Friedau aus der Erinnerung so gut wie möglich hatte herstellen
lassen, weil er wußte, daß Isma die Schreibmaschine und die Möbel
der Martier nicht sehr liebte. Sie zog wieder ihr Tagebuch hervor.
Die Zeitrechnung machte ihr Schwierigkeiten, denn der Marstag war
um 37 Minuten länger als der Erdentag, da sie aber stets einen
Marstag gleich einem Erdentag in ihrem Buch gerechnet hatte, so
mußte sie alle neununddreißig Tage einen Erdentag überspringen, um
nicht gegen den Kalender der Erde zu weit zurückzubleiben. Das war
nun jetzt zum viertenmal der Fall – so lange weilte sie auf dem
Mars! Sie fand, daß heute auf der Erde der 27. Februar sei, ein
Sonntag! Und der Geburtstag ihres Mannes! Wie glücklich hatte sie
diesen Tag sonst verlebt, und mit welchen Hoffnungen im vorigen
Jahr! Und wo mochte Hugo jetzt weilen? Der Trost, den seine Rettung
ihr gewährte, hatte nur auf kurze Zeit angehalten. Die
Unmöglichkeit, sich mit ihm so zu verständigen, wie es ihr Herz
verlangte, erhöhte nur ihre Sehnsucht und ihre Sorge. Was hatte er
von ihr gehört, in welchem Licht mußte sie ihm erscheinen, wie
würde er ihre Handlungsweise beurteilen? Konnte er ihr Glauben
schenken? Wie enttäuscht und einsam mußte er sich fühlen wenn er
das Haus leer fand, wo er sein Glück wiederzufinden hoffte!

Das Herabfallen der Fernsprechklappe schreckte sie aus ihren
Gedanken.

„Liebe Isma, sind Sie da? Ja? Ich bringe Ihnen etwas!“ Es war die
Stimme von Frau Ma. Im Augenblick war Isma aufgesprungen. Schon
erschien Ma an der Tür.

„Da, Frauchen“, rief sie, „da haben Sie die ganze Post für Sie. Ein
großes Paket, nicht wahr? Ill hat alle deutschen Zeitungen
aufkaufen lassen, die in Sydney zu haben waren. Und nun ängstigen
Sie sich nicht, es wird alles gut werden. Ich will Sie jetzt nicht
stören.“ Sie küßte Isma auf die Stirn und ging.

Das Paket, von einem leichten Korbgeflecht umhüllt, lag auf dem
Tisch. Ismas Hände zitterten, als sie den Verschluß auseinanderbog.
Ein Haufen Zeitungen lag vor ihr. Sie setzte sich und zwang sich
zur Ruhe. Systematisch nahm sie ein Blatt nach dem andern zur Hand,
sah nach dem Datum und entfaltete es. Die Blätter waren offenbar
schon von einer kundigen Hand geordnet. Das erste war vom 24.
September vorigen Jahres. Gleich nach dem Leitartikel enthielt es
in fettem Druck die Nachricht, daß das englische Kanonenboot
›Prevention‹ auf der Rückkehr begriffen sei. Es habe in der Nähe
von Grinnell-Land einen siegreichen Kampf mit einem Luftschiff,
angeblich den Bewohnern des Planeten Mars gehörig, bestanden. An
Bord befinde sich der Leiter der deutschen Nordpolexpedition, Torm,
der von wandernden Eskimos dahin gebracht sei – – –

Isma las nicht weiter. Sie ergriff ein neues Blatt. „Torm in
London.“ Sie überflog nur die Zeilen. „Tiefergreifend wirkten auf
den kühnen Forscher die Nachrichten über das Schicksal der übrigen
Expeditionsmitglieder, insbesondere die glückliche Heimkehr
Grunthes und die Rettung der wissenschaftlichen Resultate. Aber
alles tritt im Augenblick in den Hintergrund gegenüber der
Tatsache, daß die Martier –“ – Weiter – „Der Festabend der
geographischen Gesellschaft litt unter der getrübten Stimmung des
Gefeierten, den traurige Familiennachrichten niederdrückten –“

Isma seufzte tief. Sie vermochte kaum zu lesen. Jeden Augenblick
fürchtete sie auf ihren Namen zu stoßen und die Verleumdung
öffentlich ausgesprochen zu sehen. Aber es war nichts weiter
gesagt. Ein anderes Blatt! „Torm in Hamburg. Begeisterter Empfang.“
– Weiter! „Torm in Berlin. – Rührendes Wiedersehen von Torm und
Grunthe. – Allgemein bedauerte man die Abwesenheit Friedrich Ells,
des geistigen und pekuniären Vaters der Expedition, der sich
bekanntlich nach dem Mars begeben hat. – Wie wir hören,
beabsichtigt Torm, seinen Wohnsitz vorläufig in Berlin zu nehmen
–“

Isma atmete auf. Diese Zeitung wenigstens schien diskret zu sein –
man wollte offenbar den verdienten Forscher schonen.

Und sie, sie sollte schuld sein, daß man ihn schonen mußte? Was
mochten andere von ihr sagen? Und warum sagte man nicht offen,
weshalb sie fortgegangen war – Grunthe wußte es doch, er konnte sie
rechtfertigen.

„Es glaubt ihm niemand!“ Wie ein Schrei entrang es sich Isma.
Mechanisch blätterte sie weiter. Da haftete ihr Auge auf einer
Stelle.

„Infolge der gehässigen Angriffe, die von gewissen Blättern gegen
den Martier-Sohn Friedrich Ell gerichtet werden und die sich
bemühen, die Gattin unseres großen Landsmanns Torm zu verleumden,
sehen wir uns gezwungen, von unserm Grundsatz abzugehen, wonach wir
um persönlichen Klatsch uns nicht kümmern. Wir sind jedoch in der
Lage, aus bester Quelle jene schamlosen Hetzereien zurückzuweisen,
die, soviel wir wissen, ihren Ursprung aus einem Artikel des
Friedauer Intelligenzblattes genommen haben. Es war dort gesagt,
jedermann in Friedau wisse, daß zwischen Ell und Frau Torm intime
Beziehungen seit Jahren bestanden hätten. Die Polarexpedition, so
deutete man an, sei von Ell angeregt, um Torm zu entfernen. Auf die
Nachricht von seiner zu erwartenden Rückkehr habe Frau Torm ihr
Haus verlassen und sei aus Friedau verschwunden. Man vermute, daß
sie mit ihrem Freund nach dem Mars gegangen sei, und so weiter. –
Dies alles ist erbärmliche Lüge. Herr Dr. Karl Grunthe, der
Begleiter Torms, an dessen Wahrhaftigkeit wohl selbst das Friedauer
Intelligenzblatt nicht zu zweifeln wagen wird, schreibt uns, daß
Frau Torm in seiner Gegenwart in einer mit Ell geführten
Unterredung sich entschlossen habe, das Luftschiff der Martier zu
benutzen, um auf demselben Nachforschungen nach dem Verbleib ihres
verschollenen Gemahls anzustellen und die Rettung desselben zu
betreiben. Ohne Zweifel ist es dasselbe Luftschiff, welches in
Konflikt mit dem englischen Kanonenboot ›Prevention‹ geraten ist,
zu einer Zeit, als sich Torm noch bei den Eskimos befand. Nicht
aufgeklärt bleibt nur, warum das Luftschiff Friedau eher als
geplant, mitten in der Nacht, verlassen hat und warum es dann,
entgegen der Zusage des Befehlshabers, nicht nach Friedau
zurückgekehrt ist. Man kann hieraus die Befürchtung ziehen, daß ihm
irgendein Unglücksfall zugestoßen ist, und dies um so mehr, als der
Kapitän Keswick versichert, durch seine Beschießung das Luftschiff
beschädigt zu haben. Alle andern Schlüsse aber sind als
Verleumdungen zurückzuweisen. Der heldenmütige Entdecker des wahren
Nordpols, den der unerklärliche Verlust seiner geliebten Gattin
tief niederdrückt, verdiente wohl, daß man ihn im eigenen Vaterland
nicht noch in seinem Teuersten beschimpft.“

Die Nummer der Zeitung war bereits vom November des vorigen Jahres.
Die folgenden Nummern, die bis zum Anfang Januar dieses Jahres
reichten, schienen nichts weiter über diese Angelegenheit zu
enthalten. Wenigstens fand Isma beim eiligen Durchblättern keine
dahinzielende Notiz, und sie hoffte schon, die Erklärung habe ihre
Wirkung getan.

Isma saß lange unfähig ihre Gedanken zu ordnen, den Kopf in die
Hände gestützt. Dann begann sie weiterzusuchen. Es folgten jetzt
Exemplare anderer Zeitungen, sogar einige Witzblätter. Da sah sie
mit Abscheu und Entsetzen, daß man offenbar im großen Publikum sich
nicht an die gegebene Aufklärung kehrte. Wo von Ell die Rede war –
und sein Buch über die Martier wurde überall erwähnt – da fand sich
auch irgendeine hämische oder witzelnde Bemerkung. Was mußte Torm
dabei fühlen! Isma wollte nichts mehr sehen, sie ballte die Hände
zusammen. Da erblickte sie auf der halbgebrochenen Seite eines
Witzblattes unverkennbar das Gesicht Torms – sie schlug das Blatt
auf. Es war eine Karikatur – Torm in einem Luftballon auf dem
Nordpol, über ihm ein Luftschiff der Martier, worin Ell und Isma
ihm lange Nasen drehen – – Sie las nicht, was darunter stand, sie
sprang auf und ergriff den Rest der noch nicht durchblätterten
Papiere, um sie fortzuschleudern.

Da, was fällt da herab? Ein zusammengelegtes, geschlossenes Papier
– eine telegraphische Depesche – ein Formular des Telegraphenamts
in Sydney – die Adresse ist in englischer Sprache geschrieben – ›An
die Gesandtschaft der Marsstaaten für Frau Torm‹.

Isma reißt das Papier auf. Der Inhalt ist deutsch, mit lateinischen
Buchstaben von einer englischen Hand geschrieben. Die Buchstaben
tanzen vor ihren Augen, sie kann sie kaum entziffern.

„Berlin, den 6. Januar. Herzlichen Dank für die Aufklärung durch
dein langes, liebes Telegramm! Das Mißgeschick, das dich fernhält,
schmerzlich betrauernd, sende ich innige Grüße in treuer Liebe und
erhoffe baldiges, ungetrübtes Wiedersehen. Dein Torm!“

Das Telegramm entsank ihrer Hand, und ihre nervöse Spannung löste
sich in einem schluchzenden Weinen. Eine direkte Nachricht hatte
sie nicht erwartet. Sie wußte, daß die Martier am 2. Januar nach
Sydney gekommen waren und das Raumschiff bereits Mitte Januar die
Erde wieder verlassen hatte. In dieser Zeit konnte kein Brief nach
Berlin gelangen. Dein langes, liebes Telegramm! Also man war so
aufmerksam gewesen, ihren ganzen Brief an Torm zu telegraphieren.
Ein leichter Schreck durchzog ihr sparsames Hausfrauenherz, wenn
sie an die ungeheuren Kosten dieses Riesentelegrammes dachte. Aber
es versöhnte sie einigermaßen mit der Hartnäckigkeit der Martier,
nur offene Briefe zuzulassen. Sie war glücklich über das Telegramm,
das kein Wort des Vorwurfs enthielt, und doch wie wenig sagte es!
Aber was kann man auch in einem Telegramm sagen! Sie las die
wenigen Zeilen immer wieder.

Ma trat in das Zimmer.

„Sitzen Sie nun schon zwei Stunden über den Blättern, Frauchen? Und
geweint haben Sie auch? Ärgern Sie sich nur nicht. Was gibt es
denn?“

Isma versuchte zu lächeln. „Hätte ich nur das Telegramm eher
gefunden“, sagte sie, „so hätten mich die dummen Menschen weniger
gekränkt.“

„Aber Sie haben ja den Korb auf der falschen Seite geöffnet – es
hat doch wahrscheinlich obenauf gelegen. Und nun kommen Sie gleich
einmal mit mir! Saltner ist da, er hat auch Nachrichten, von seiner
Mutter und von Grunthe. Und Ell hat die Depesche hergeschickt, die
er von Ihrem Mann bekommen hat. Es ist doch nett von Ell, daß er
alle euere Briefe an ihre Adresse hat telegraphieren lassen und
sofortige telegraphische Antwort bestellt hat.“

Isma erhob sich. „Ich komme sogleich“, sagte sie.

Also Ell hatte sie es zu verdanken, daß sie schon eine Antwort
bekommen hatte! Während sie ihre Augen kühlte und ihr Haar ordnete,
bedrückte sie der Gedanke, daß ihr Brief zwanzig Seiten, eng
beschrieben – das waren gewiß an die viertausend Worte – enthalten
hatte. Wenn Ell das alles telegraphieren ließ, das war ja eine
Depesche für zwanzigtausend Mark! Früher hätte sie bei Ell
überhaupt nicht daran gedacht, daß zwischen ihnen ein Abwägen des
Gebens oder Nehmens bestehen könne, aber jetzt war es ihr peinlich,
sich so verpflichtet zu fühlen.

Bei ihrem Eintritt in das Empfangszimmer hielt ihr Saltner zuerst
freudestrahlend ein Telegramm entgegen, das sie gar nicht zu
entziffern vermochte. Es war von seiner Mutter. Aus den
abgebrochenen, nicht ganz dialektfreien Sätzen, welche die gute
Frau in der Absicht, recht kurz zu sein, gebaut hatte, war durch
den englischen Telegraphisten ein unmögliches Kauderwelsch
geworden. Saltner aber genügte es vollständig, daraus die Freude
der Mutter über sein Wohlbefinden zu ersehen, und jedes
verstümmelte Wort machte er mit rührender Sorgfalt zu einem
besonderen Studium.

Grunthe hatte nur kurz an Saltner telegraphiert, daß die plötzliche
Abreise Ells sehr störend für die Stimmung der Bevölkerung in bezug
auf die Martier sei, da er selbst die gegen Ells Schriften
erhobenen Bedenken nicht genügend widerlegen könne. Die politischen
Verhältnisse bezeichnete er als ziemlich trostlos; seine Ansicht,
daß man alle von den Martiern gestellten Forderungen bewilligen
müsse, um ihnen jede Veranlassung zu nehmen, sich in die
menschlichen Angelegenheiten einzumischen, finde wenig Anhänger.
Man unterschätze die Macht der Martier und baue auf ihre
Unfähigkeit, sich außerhalb ihrer Schiffe auf der Erde zu bewegen,
während doch rückhaltloses Vertrauen und reiner Wille die einzigen
Mittel sein würden, den Einfluß der Nume zum Besten zu lenken.

Isma hatte die Zeilen nur durchflogen, um nun in Ruhe Torms langes
Telegramm an Ell zu lesen. Es trug das Datum vom 8. Januar.
Zunächst war es rein geschäftlich gehalten, ein Bericht des Leiters
der Nordpolexpedition an deren Veranstalter. Was Isma am meisten
interessierte, die persönlichen Schicksale Torms, war nur kurz
geschildert. Dann aber hieß es:

„Ich bedauere tief, daß Sie den heldenmütigen, aber übereilten
Entschluß meiner Frau unterstützten und Friedau unter so
ungewöhnlichen Umständen verließen. Mir persönlich, wie dem
allgemeinen Interesse entstehen dadurch Schwierigkeiten, die sich
noch gar nicht absehen lassen. Bieten Sie allen Einfluß auf, um
Ismas Rückkehr zu ermöglichen, und kommen Sie selbst, um Ihre Sache
zu führen. Wirken Sie darauf hin, daß die Marsstaaten keine anderen
Bestrebungen verfolgen, als ganz allmählich einige ihrer
technischen Fortschritte uns zugänglich zu machen. Von jeder
direkten Einwirkung befürchte ich Unheil für die Menschen. Ich
bleibe vorläufig in Berlin. Leider scheint in den maßgebenden
Kreisen Entschlußlosigkeit zu herrschen. Ich bestätige dankend den
Empfang der von Ihnen für die nachträglichen Kosten der Expedition
angewiesenen Summe von 100.000 Mark. Torm.“

Isma ließ das Blatt sinken. Sie fühlte sich unsäglich elend. Um
ihren Mann zu retten, hatte sie sich zur Reise entschlossen, und
was hatte sie erreicht! Welche Qualen hatte sie ihm bereitet! Und
den Freund abgezogen von seiner höchsten Pflicht, für den Frieden
der Planeten zu wirken! Und sie selbst, einsam, machtlos, verbannt
– –

Sie sprang auf und faßte Mas Hände.

„Lassen Sie mich fort“, rief sie leidenschaftlich. „Ich muß nach
der Erde, ich muß zu meinem Mann! Ich muß Ell sprechen. Wo ist
er?“

„Aber Frauchen, was ist Ihnen? Zu Ell können Sie jetzt nicht, er
ist nach dem Pol gereist, um mit Ill zu konferieren. Aber beruhigen
Sie sich. Die nächsten Tage werden alles entscheiden. Ich darf
ihnen sagen, wir verhandeln mit den Mächten, auch mit ihrem
Vaterland. Sobald der Frieden gesichert ist, sollen Sie nach
Hause.“

„Ich gehe natürlich mit“, rief Saltner. „Auf Ell rechnen Sie nicht,
für ihn ist es jetzt zu spät, oder noch zu zeitig. Was er versäumt
hat, kann er jetzt nicht einholen. Er hätte mit dem ersten
Raumschiff nach dem Südpol gehen und sich sofort nach Deutschland
begeben müssen. Das wollte er nicht. Es war ein großes Unrecht.“

„Und wann“, seufzte Isma, „wann kommt endlich die Befreiung.“

Ma sprach einige tröstende Worte, als sie plötzlich abberufen
wurde. Schon nach wenigen Minuten kehrte sie zurück.

„Weinen Sie nicht mehr“, sagte sie zu Isma, „ich bringe Wichtiges
für sie, hoffentlich Gutes: Nachricht von Ill. Er hat
telegraphiert, weil es vertraulich ist, und beim Sprechen weiß man
nie, wer zuhört. Nun, ich lese ja schon, hören Sie nur: Soeben
meldet Lichtdepesche, daß sämtliche Großmächte, falls England unser
Ultimatum nicht annimmt, Neutralität erklärt haben. Wir
verpflichten uns gegen Verkehrsfreiheit, jeder Einmischung in
politische Angelegenheiten uns zu enthalten. Leider Annahme des
Ultimatums durch England aussichtslos.“

Saltner sprang auf. „Das ist doch etwas! So wird der Krieg
wenigstens lokalisiert, wenn man so sagen darf. England geht es
freilich an den Kragen, es ist ja traurig. Aber wir haben Frieden,
Gott sei Dank! Nun dürfen wir zurück, nicht wahr?“

„Ich zweifle nicht“, sagte Ma. „Gibt England nicht nach, so geht
übermorgen, sobald Ihr zweiter März anfängt, Raumschiff auf
Raumschiff nach dem Nordpol, und Sie dürfen sicher mitreisen. In
vier bis fünf Wochen können Sie daheim sein. Aber Frauchen, was
machen Sie, wie sehen Sie aus? Gleich kommen Sie mit mir, Sie
müssen in Ihr irdisches Schwerekämmerchen!“

Die Aufregung, die Sorge und nun die plötzliche Aussicht auf
Heimkehr hatten Ismas Widerstandskraft gelähmt. Alles Blut war aus
ihrem Gesicht entwichen, mit bleichen Wangen, einer Ohnmacht nahe,
lag sie auf ihrem Sessel. Ma umfaßte sie und führte sie schonend
auf ihr Zimmer.

\section{41 - Die Schlacht bei Portsmouth}

England hatte das Ultimatum abgelehnt. Hierauf ging an den
Befehlshaber der martischen Streitkräfte auf dem Südpol der Erde
die Weisung, mit Gewaltmaßregeln unnachsichtlich, doch ohne
Blutvergießen vorzugehen.

Am zweiten März erfolgte die Kriegserklärung.

Eine Mitteilung an die Regierungen und eine Proklamation an alle
Völker der Erde besagte, daß vom sechsten März mittags zwölf Uhr an
England und Schottland von jedem Verkehr abgeschnitten sein würden.
Von diesem Zeitmoment an werde die Blockade über die Küste dieser
Länder effektiv sein, und zwar in der Art, daß es keinem Schiff
gestattet sein solle, die Zone von fünf bis zu zehn Kilometer
Abstand von der Küste weder landwärts noch seewärts zu
überschreiten. Alle fremden Schiffe müßten bis dahin die englischen
Häfen verlassen haben.

Man lachte in England darüber als über eine Aufschneiderei der
Martier. Doch als es sich in der Nacht vom zweiten zum dritten März
herausstellte, daß sämtliche Kabel, welche England mit dem
Kontinent und mit Irland verbanden, unterbrochen waren und die
telegraphische Verbindung somit aufgehoben war, ohne daß eines der
vor der Küste kreuzenden Kriegsschiffe bemerkt hatte, wie die
hierzu erforderlichen Arbeiten ausgeführt worden seien, beeilten
sich die in den Häfen befindlichen fremden Schiffe, sich zu
entfernen. Die in England weilenden Ausländer ergriffen
scharenweise die Flucht.

Am Morgen des sechsten März hatten alle fremden Schiffe, die es
irgend ermöglichen konnten, England verlassen. Auch die Postdampfer
legten nicht mehr in den englischen Häfen an. Die Flotte war,
soweit sie nicht in den Kolonien gebraucht wurde, vor Portsmouth
versammelt. Von allen Schiffen, von allen Befestigungen am Land,
von den Anhöhen und den Landhäusern auf Wight spähte man nach dem
Gegner aus, der sich anheischig gemacht hatte, ein Land von 230.000
Quadratkilometern Fläche mit einer Bevölkerung von 35 Millionen,
geschützt von der stärksten Flotte der Erde, vom Weltverkehr
abzusperren. Nichts war zu sehen. Die zwölfte Stunde rückte heran.
Einige Schiffe, die von der Blockade noch nichts gehört hatten,
passierten ungehindert die zu sperrende Zone. Besonders lebhaft war
der Verkehr nach der Insel Wight. Zahlreiche Personendampfer waren
hier unterwegs, Boote aller Art belebten das Wasser. Noch fehlten
wenige Minuten zu zwölf Uhr. Die Kriegsflotte im Hafen ging unter
Dampf. Majestätisch verließ, allen voran, das neue
Riesenpanzerschiff ›Viktor‹ von 15.000 Tonnen mit seinen 30.000
indizierten Pferdekräften die Hafeneinfahrt. Die Kanonen donnerten
ihren Salut.

Nichts Verdächtiges zeigte sich nach der Seeseite zu. Aber eine
Minute vor zwölf Uhr erschienen plötzlich über dem Land sechs
dunkle Punkte, die sich schnell vergrößerten. Im Fernrohr erkannte
man sie als Luftboote. In eine Reihe aufgelöst hatten sie im
Augenblick alle Schiffe überholt und senkten sich dem Wasser zu. Es
schlug zwölf Uhr. In demselben Augenblicke wurde die bis dahin
ruhige See lebhaft bewegt. Am östlichen Ausgang der Spithead-Bucht,
dort wo der Abstand zwischen Wight und der englischen Küste die
Breite von zehn Kilometern erreicht, erschien eine gewaltige
Brandung, wie durch ein Seebeben aufgewühlt. Die Schiffe, welche
sich in der Nähe befanden, beeilten sich, den Wogen zu entgehen,
indem sie nach dem Land zurückkehrten.

Nahe über der Oberfläche des Meeres schwebend, markierte ein
Luftschiff der Martier den Punkt, bis zu welchem der
Absperrungsgürtel sich in die Bucht von Spithead hinein zog. Die
übrigen verteilten sich in der Nähe auf der Südseite von Wight und
östlich von Portsmouth. Die Martier hatten, indem sie das Wasser
durch eine Reihe von Repulsitschüssen aufregten, nur die weiter als
fünf Kilometer von der Küste befindlichen Schiffe vertreiben
wollen. Weiter durfte sich von jetzt ab kein Schiff vom Land
entfernen und keines näher als zehn Kilometer sich der Küste
nähern. Indessen blieb der Verkehr westlich von dem markierten
Punkt zwischen Wight und der Küste ungehindert, die Insel gehörte
mit in den blockierten Bezirk.

Ein großer englischer Dampfer, von Le Havre nach Southampton
zurückkehrend, wurde sichtbar. Schneller als ein Pfeil durch die
Luft schießend, erreichte ihn eines der Marsschiffe und rief ihm,
dicht an Bord hinschwebend, den Befehl zu, umzukehren. Wohl wußte
der englische Kapitän, daß er sein Schiff aufs Spiel setze, wenn er
dem Gebot nicht folge. Aber von dem Ausguck haltenden Matrosen war
ihm bereits gemeldet, daß die Kriegsflotte in der Bucht unter Dampf
sei und auf ihn zuhalte. Schon näherte sich der ›Viktor‹ dem
Luftschiff, welches die Sperrgrenze markierte; eine Granate sauste
unter dem schnell aufsteigenden Luftschiff fort. Unter diesen
Umständen glaubte der Kapitän, dem Befehl des Marsschiffes Trotz
bieten zu können, und setzte seinen Kurs fort. Aber sofort richtete
ein Schlag, der das Schiff an seinem Vorderteil traf, eine starke
Verwüstung auf dem Deck an, und von dem Marsschiff wurde ihm
zugerufen, daß, wenn er nicht sofort wende, sein Schiff auf der
Stelle in Grund gebohrt werden würde. Nun zögerte der Kapitän nicht
länger und entfernte sich wieder vom Land, in der Hoffnung, die
Flotte werde den Weg bald freimachen.

Inzwischen begann sich die Kriegsflotte in einer Stärke von gegen
dreihundert Schiffen, darunter zwanzig Panzerschiffe erster Klasse,
in der Bucht von Spithead zu entwickeln und schickte sich an, die
blockierte Linie zu forcieren, auf der man nichts bemerkte als drei
langsam hin- und hergleitende Luftschiffe der Martier. Auf diese
konzentrierte sich jetzt das Feuer von vielleicht fünfzig
Geschützen stärksten Kalibers. Geschoß auf Geschoß flog gegen die
in mäßiger Höhe schwebenden Ziele. Aber seltsam! Nicht ein einziges
Geschoß schien zu treffen. Völlig ruhig, als existierte für sie der
Angriff gar nicht, ließen die Martier die Flotte herankommen. Allen
voran dampfte die Riesenmasse des ›Viktor‹. Sein gepanzertes
Verdeck war, in Rücksicht auf die Erfahrungen der ›Prevention‹ mit
dem martischen Luftschiff, mit einer besonderen Konstruktion von
Schießscharten versehen, um einen in der Höhe befindlichen Gegner
mit Gewehrkugeln begrüßen zu können. Aber das Marsschiff, gegen
welches sich jetzt die Handfeuerwaffen richteten, schien gegen
dieselben gefeit zu sein. Unheimlich erschien diese Ruhe des
Feindes, den man bald direkt über sich erblicken mußte.

Jetzt konnte man an einem der aus dem Hafen dampfenden Schiffe die
Admiralsflagge unterscheiden. Sofort hißte auch eines der
Marsschiffe, um sich den Engländern, dem menschlichen Gebrauch
folgend, kenntlich zu machen, die Flagge, welche die Anwesenheit
des obersten Befehlshabers an Bord bezeichnete. Es war dasselbe
Schiff, das den von Le Havre kommenden Dampfer eben zurückgewiesen
hatte. In noch nicht einer Minute hatte es die zehn Kilometer
zurückgelegt, die es vom englischen Admiralsschiff trennten, und
hier legte es sich direkt zur Seite des Kommandoturmes, in welchem
sich der Admiral, ein königlicher Prinz, neben dem Kapitän des
Schiffes befand. Vergeblich richtete sich ein Hagel von Geschossen
gegen das kühne Luftschiff. Es schien in einem leichten Nebel zu
schwimmen, in welchem Granaten wie Langblei wirkungslos zerrannen.
Und nun geschah etwas ganz Unerwartetes. Immer näher rückte das
Luftschiff dem Kommandoturm, und lautlos, ein unerhörtes Wunder,
lösten sich die stählernen Platten des Panzerturms auf der Seite
des Luftschiffs und verdampften oder verschwanden in der Luft.
Schutzlos sahen sich die Befehlshaber dem schwebenden Feind
gegenüber. Aber kein Angriff auf sie erfolgte. Durch den Donner der
Geschütze der in der Front befindlichen Schiffe geschwächt, aber
deutlich verständlich vernahmen sie die englischen Worte: „Der
Oberbefehlshaber der martischen Luftflotte, Dolf, beehrt sich an
Ew. kgl. Hoheit die Bitte zu richten, sämtlichen unter Ihren
Befehlen stehenden Schiffen die Weisung zu erteilen, die Flagge zu
streichen und sich binnen einer Stunde in den Hafen von Portsmouth
zurückzuziehen. Ich würde mich sonst gezwungen sehen, jedes Schiff,
das nach zehn Minuten noch seine Flagge zeigt oder einen Schuß
abgibt und das nach einer Stunde sich nicht im Hafen befindet, zu
versenken, und müßte Ew. kgl. Hoheit für die entstehenden Verluste
verantwortlich machen.“

Ohne eine Antwort abzuwarten, war das Luftschiff verschwunden. Aber
ehe es noch in die Linie der Marsschiffe zurückgekehrt war, hatte
der ›Viktor‹ den Punkt erreicht, den nach der Instruktion der
Martier kein Schiff überschreiten durfte. Da ging das dort
befindliche Luftschiff aus seiner Wartestellung. Es senkte sich
direkt hinter dem Panzerschiff bis dicht über die Oberfläche des
Wassers und drängte sich an seine Rückseite. Die Nihilithülle des
Luftschiffes, die es gegen jeden Angriff schützte, zersetzte die
fünfzig Zentimeter dicken Panzerplatten binnen ebensoviel Sekunden.
Ein Repulsitschuß zerstörte das Steuer, ein zweiter schlug schräg
von oben nach unten durch das Schiff und zerbrach eine
Schraubenwelle. Das Riesenschiff war unfähig, sich zu bewegen.
Jetzt erhob sich das Luftschiff wieder und schmolz das Dach des
Kommandoturms ab. Mit Entsetzen sah der Kapitän das Schiff über
sich schweben, während die von seiner Mannschaft auf dasselbe
gerichteten Schüsse nicht die geringste Wirkung zeigten. Ratlos
starrte er in die Höhe. Diese Art des Kampfes mit einem
unverletzbaren Gegner mußte auch den Tapfersten entmutigen.

Aus dem Marsschiff kam eine Stimme: „Die gesamte Besatzung in die
Boote. Das Schiff wird versenkt. Wir müssen ein Exempel statuieren,
damit unsre Befehle künftig besser befolgt werden.“

Der Kapitän sah, daß er verloren war. Er ließ die Boote bemannen
und abstoßen. Er selbst blieb im Kommandoturm, entschlossen, mit
dem Schiffe, dessen Flagge im Winde flatterte, unterzugehen. Die
Boote entfernten sich. Das Marsschiff drängte seinen Nihilitpanzer
an die Seite des Panzerschiffes, dicht über der Wasserlinie. Die
eisernen Wände öffneten sich, während sich das Marsschiff in die
Luft erhob. Es wandte sich nach dem Kommandoturm, um den Kapitän an
seiner Selbstaufopferung zu verhindern. Aber schon neigte sich der
Koloß ›Viktor‹ zur Seite. Mit wehender Flagge sank er in die Flut,
die sich weitaufbrausend über ihm und seinem Führer schloß.

Der Kommandant des Marsschiffes trieb sein Boot dicht über den
schäumenden Wirbel hin, um nach dem Kapitän des ›Viktor‹ zu suchen.
Die Woge brachte ihn nicht zurück. Die Augen der Martier
verdüsterten sich, und finsterer Ernst lagerte über ihren Zügen.
Noch einmal umkreiste das Boot langsam die Stelle.

„Wir sollen den Willen der Menschen brechen“, sagte der Anführer,
den Gedanken der Seinigen Worte leihend, „aber kein Menschenleben
soll mit unserem Willen zugrunde gehen. Doch der Wille dieses
Tapfern war stärker als der unsere. Er konnte nicht leben, der das
stärkste Schiff der Erde nicht weiter als drei Seemeilen über den
Hafen hinausgebracht hatte. Gott verzeihe uns, wir wollten nicht
töten.“

Ein Signal weckte die Mannschaft aus ihrer Stimmung, die mehr der
eines Besiegten als eines Siegers glich. Das Luftboot des
Oberbefehlshaber Dolf war zurückgekehrt. „Vorwärts!“ rief er dem
ersten Marsschiff zu, „drei andere Panzerschiffe durchbrechen die
Linie. In den Grund mit ihnen!“

Der Offizier gehorchte schweigend. „Wir sind keine Mörder“,
murmelte es in der Mannschaft. Aber das Luftboot stürzte sich auf
ein zweites Panzerschiff und zerschmetterte ihm das Steuer und die
Maschine. Ein Gleiches taten die übrigen Boote mit den englischen
Schiffen, welche die Grenze der Blockade überschritten. Als ein
steuerloses, hilfloses Wrack trieben bereits sieben Panzerschiffe
erster Klasse auf den Wellen. Aber die Martier versenkten sie
nicht, weil sie jeden Augenblick erwarteten, daß der englische
Admiral das Signal zur Ergebung und zum Rückzug der Flotte geben
würde.

Doch nichts dergleichen geschah. Die zehn Minuten waren längst
abgelaufen. Die Flotte rückte weiter vor. Der Admiral konnte sich
nicht entschließen, so ruhmlos die Waffen zu strecken, obwohl ihn
ein Grauen vor dem unerreichbaren Gegner umfing.

Das Verderben nahm seinen Fortgang. Die Martier begnügten sich
überall damit, die Maschinen und Steuervorrichtungen zu zerstören.
Obwohl sie ihre sicheren Repulsitströme nur auf das Material wirken
ließen, traten trotzdem hier und da Explosionen und
Zerschmetterungen ein, denen auch Menschenleben zum Opfer fielen.
Doch waren die Verluste der Engländer an Mannschaft gering, ihre
Schiffe aber kampfunfähig. Bleiches Entsetzen bemächtigte sich
allmählich der Offiziere und Matrosen, als sie sahen, daß sie dem
Feind schutzlos preisgegeben waren. Ihre herrlichen Fahrzeuge waren
ein Spiel der Wellen. Von den Luftschiffen der Martier, die
unverletzlich blieben, verließ nur von Zeit zu Zeit eines den
Kampfplatz, um von einem in großer Höhe schwebenden Munitionsschiff
seinen Vorrat an Nihilit und Repulsit zu ergänzen. Eine halbe
Stunde mochte dies nutzlose Ringen gedauert haben, als auch das
Admiralsschiff manövrierunfähig wurde. Ein Luftschiff übersegelte
seine Masten, und die Flagge verschwand. Was sich von Schiffen noch
bewegen konnte, suchte in den Hafen zu fliehen. Aber dies nützte
nun nichts mehr. Ein großer Teil der Schlacht war direkt unter den
Kanonen der Festungswerke geschlagen worden. Sie konnten die
Vernichtungsarbeit der Martier nicht beeinträchtigen. Die
Luftschiffe gingen in den Hafen und zerstörten systematisch die
Bewegungsmechanismen sämtlicher Schiffe.

Nun wurde von den Engländern die Parlamentärflagge aufgezogen. Die
Martier verlangten als erste Bedingung, daß die Mannschaft der
kampfunfähigen Schiffe geborgen werde. Alles, was an
Handelsschiffen und Booten aufzutreiben war, wurde darauf nach der
Reede entsandt und brachte die Mannschaft der außer Bewegung
gesetzten Schiffe ans Land.

Die Engländer hatten jetzt eingesehen, daß es ganz nutzlos sei, ihr
Pulver zu verschießen. Sie konnten nur noch darauf bedacht sein,
das Leben der Seeleute zu schonen und weiteren Materialschaden zu
vermeiden. Als alle Menschen und die Hilfsflottille wieder im Hafen
angelangt waren, legten sich zwei der Marsschiffe vor die Mündung
und erklärten den Hafen für gesperrt. Die herrenlosen Schiffe
trieben unter einem leichten Westwind allmählich in den Kanal
hinaus und wurden nach und nach von französischen, holländischen
und deutschen Dampfern geborgen, die sich in großer Anzahl in
sicherer Entfernung von der Blockadelinie angesammelt hatten und
Zeugen des rätselhaften Vernichtungskampfes geworden waren.

Ähnliche Vorgänge wie bei Portsmouth, nur in kleinerem Maßstab,
spielten sich überall ab, wo sich Kriegsschiffe an der englischen
Küste vorfanden. Die Martier hatten Punkt 12 Uhr am 6. März die
gesamte Küste von England und Schottland in ihrer Ausdehnung von
fast 4.800 Kilometer mit ihren Luftschiffen besetzt, deren sie
vorläufig 48 zur Verfügung hatten. So kam im Durchschnitt eine
Küstenlänge von 100 Kilometer auf jedes Schiff. Doch dehnte sich
diese Strecke, je nach der Beschaffenheit der Küste, für manche
Schiffe auf 500– 600 Kilometer aus, während sich die Marsschiffe
vor den besuchten Häfen dichter gruppierten. Wo ein Schiff sich
zeigte, stürzte sofort ein Luftschiff der Martier herbei und zwang
es zur Umkehr oder vernichtete es im Fall des Ungehorsams auf eine
solche Weise, daß sich die Mannschaft gerade noch nach der Küste
retten konnte. Von außen kommende fremde Schiffe wurden einfach
durch einen ins Wasser abgegebenen Repulsitschuß zurückgetrieben.
Tatsächlich gelangte außer den einheimischen, kleineren
Fischerbooten, die man passieren ließ, kein Schiff mehr vom 6. März
an nach der englischen Küste, keines gelangte ins Ausland.

An diesem Tage ward die Macht Englands gebrochen. Die Flotte war
vernichtet. Wut und Bestürzung herrschten im ganzen Land. In London
war man ratlos. Niemand wußte, wie man sich gegen einen solchen
Feind verhalten solle. Das Ministerium trat zurück, aber es fanden
sich keine Nachfolger. Man wollte um Frieden bitten, aber die
aufgeregte Volksstimme rief nach Rache. Endlich entschloß man sich,
den Widerstand fortzusetzen, in der Hoffnung, daß sich Hilfe von
auswärts finden werde oder daß man irgendein Mittel entdecke, die
Blockade zu brechen. So vergingen Wochen, in denen man nichts
hörte, als daß die Martier in diesem oder jenem Hafen noch ein
armiertes Schiff entdeckt oder versenkt, daß sie hier eine Werft,
dort ein Dock vernichtet hätten. Alle Versuche, den gesperrten
Gürtel heimlich im Schutz der Nacht zu passieren, blieben
vergeblich. Die Marsschiffe, einen Weg von hundert Kilometern in
sieben bis acht Minuten durchsausend, beleuchteten mit ihren
Scheinwerfern den gesperrten Streifen taghell, und ehe ein Schiff
sich weit genug entfernen konnte, war es aufgefunden. Selbst der
Nebel schützte nicht vor Entdeckung. Denn nach einigen Tagen hatten
die Martier einen großen Teil der Küste mit einem dünnen,
schwimmenden Kabel umzogen, dessen Berührung durch ein Schiff ihnen
sofort die getroffene Stelle anzeigte. Und keine Nachricht von
außen! Der Handel unterbrochen, alle Arbeiter, deren Beschäftigung
von der Schiffahrt abhing, ohne Tätigkeit. Und schon begann die
mangelnde Einfuhr der Lebensmittel in einer drückenden Erhöhung der
Preise sich zu zeigen.

England war aus der Welt gestrichen. Aber die Welt ging weiter.
Neue Raumschiffe kamen an mit neuen Luftbooten. Diese gingen nicht
zur Verstärkung der Blockade ab, sondern sie suchten die englischen
Kriegsschiffe in den Kolonien auf und bedrohten sie mit
Vernichtung, soweit nicht die Befehlshaber sich in den Dienst der
Kolonien stellten. Letztere sahen sich plötzlich auf sich selbst
angewiesen. Indien, Kanada, die australischen Kolonien und das
Kapland erklärten sich für unabhängig und setzten selbständige
Regierungen ein. Dasselbe tat Irland. Die Marsstaaten erkannten sie
als souveräne und neutrale Staaten an, und so gewaltig war der
Eindruck, den die Vernichtung der englischen Flotte auf der ganzen
Erde gemacht hatte, daß kein Staat Einspruch gegen diese
Veränderungen erhob. Keine Hand rührte sich für England. Die
anderen Nationen beeilten sich vielmehr, die bisherigen
Handelsgebiete Großbritanniens für sich zu sichern. Von den
kleineren Kolonien zog jede Macht an sich, was sie zur Abrundung
oder zur besseren Verbindung ihres Besitzes für nötig hielt. Die
Beute war vorläufig so reich, daß man sich an diejenigen Gebiete
noch nicht machte, die zu Streit unter den Erbteilern hätten Anlaß
geben können. Im stillen verhandelten die europäischen Großmächte
über eine Teilung des englischen Besitzes am Mittelmeer und eine
Auflösung der Türkei.

Jetzt erst ließen die Martier Zeitungen der auswärtigen Staaten
nach England gelangen. Was man dort längst befürchtet hatte, war
eingetroffen. Die Völker teilten sich in die englische Erbschaft,
ohne sich viel darum zu bekümmern, ob der Erblasser wirklich tot
sei. Das gab den Ausschlag. Die Furcht, auch das Letzte zu
verlieren, bändigte den englischen Nationalstolz. Man bat um
Frieden.

Alles, was die Martier verlangt hatten, wurde zugestanden, nur den
Kapitän Keswick und den Leutnant Prim konnte man nicht mehr
bestrafen. Sie waren bei einem Versuch, die Blockade zu brechen,
mit ihrem Schiff untergegangen, von den Martiern aber gerettet
worden. Sie befanden sich als Gefangene bereits am Nordpol. Aber
auch den gegenwärtigen Zustand in den Kolonien und die Abmachungen
der Mächte über die Türkei mußte England anerkennen. Dafür
erklärten die Marsstaaten, das nun wehrlose England gegen alle
etwaigen weiteren Angriffe auf seinen nunmehrigen Bestand schützen
zu wollen. England hatte einen Protektor. – –

Nach einer durch ungeheuren Repulsitverbrauch beschleunigten Fahrt
von nur siebzehn Tagen war Ill auf dem Nordpol der Erde
eingetroffen. Am fünften April war der Präliminarfriede geschlossen
und die Blockade aufgehoben worden.

Aber nicht nur das gedemütigte England beugte sich dem Sieger, der
unter den Kanonen von Portsmouth dreihundert Kriegsschiffe binnen
drei Stunden durch ein halbes Dutzend Luftschiffe mit nur 144 Mann
Besatzung vernichtet hatte. Was die Nachrichten über die hohe
Kulturaufgabe der Martier nicht vermocht hatten, das Entgegenkommen
der zivilisierten Erdstaaten zu gewinnen, das brachte die
Bezwingung Englands durch Nihilit und Repulsit alsbald zustande. Es
begann ein förmlicher Wetteifer der Regierungen, die Gunst des
martischen Machthabers zu gewinnen, der aus dem reichen englischen
Besitz Länder und Meere verschenkte. Die Marsstaaten waren unter
dem Namen ›Polreich der Nume‹ nicht nur als ein Faktor im Rat der
Großmächte anerkannt, sie nahmen bereits tatsächlich die führende
Stellung ein. Unter dem Titel eines Präsidenten des Polreichs und
Residenten von England und Schottland übte Ill die Regierungsgewalt
im Auftrag der Marsstaaten aus. Alles dies war geschehen, ohne daß
ein Martier sein Luftschiff verlassen hatte. in dem großen, in
einem Park Londons auf weiter Wiesenfläche ruhenden Luftschiff
empfing Ill die Minister Englands und die Gesandten der fremden
Staaten. Es erregte daher trotz allem Ungewöhnlichen, das man im
letzten Jahr erlebt hatte, nicht geringe Spannung und Befriedigung,
daß der Präsident des Polreichs bei den Höfen und Regierungen in
Berlin, Wien, Petersburg, Rom, Paris und Washington um einen
persönlichen Empfang nachsuchen ließ. Es verlautete, daß sich daran
die Einsetzung ständiger Botschafter in diesen Hauptstädten und ein
von den Martiern einzurichtender regelmäßiger Luftschiffverkehr mit
dem Pol anschließen werde. Im stillen hoffte man, daß das
geheimnisvolle Grauen, welches die Personen der Martier für die
Menschen umhüllte, verschwinden werde, sobald man Gelegenheit haben
würde, sie außerhalb des Schutzes ihrer Luftschiffe unter der
natürlichen Schwerkraft der Erde sich beugen zu sehen.

Der einzige Mensch auf der Erde, der diese Hoffnung nicht teilte,
war vielleicht Grunthe. Er war überzeugt, daß Ill diesen Schritt
nicht getan hätte, wenn nicht die Martier zuvor ein Mittel entdeckt
hätten, sich auch außerhalb ihrer Schiffe vom Druck ihres
Körpergewichts zu befreien.

\section{42 - Das Protektorat über die Erde}

Torm bewohnte in Berlin zwei bequem eingerichtete Zimmer in einem
Hotel garni der Königgrätzer Straße. Nach seiner Rückkehr war er
überall der Held des Tages gewesen, den man nicht genug feiern
konnte und um so mehr feierte, als Grunthe sich sehr geschickt von
der Öffentlichkeit zurückzuziehen wußte. Seit der Ankunft der
Martier in Australien und dem Ausbruch ihres Krieges mit England
waren aber die beiden Polarforscher, deren Reise die eigentliche
Veranlassung war, daß die Martier mit den Staaten der Erde in
Verbindung traten, ziemlich in Vergessenheit geraten. Das
öffentliche Interesse hatte sich jetzt wichtigeren Gegenständen
zugewendet.

Am 20. März, dem Tag nach der Ankunft Ills am Pol, hatte Torm zwei
in Calais aufgegebene Depeschen erhalten, datiert aus Kla auf dem
Mars, vom 2. März. Die erste enthielt nur die Worte: „Ich komme mit
dem nächsten Raumschiff. Deine Isma.“

Die zweite war von Saltner und besagte, daß Frau Torm und er selbst
die Erlaubnis zur Heimreise erhalten hätten, da sie aber zum Abgang
des Regierungsschiffes nicht mehr zurechtkommen könnten, erst mit
dem nächsten Schiff reisen und daher vor Mitte April nicht bei ihm
eintreffen würden. Auch Ell habe sich entschlossen, sie zu
begleiten. Seitdem hatte Torm keine Nachricht mehr erhalten und
konnte auch keine erwarten. Denn kein anderes Raumschiff als der
›Glo‹ legte, wie Grunthe erklärte, bei der jetzigen
Planetenentfernung den Weg unter fünf Wochen zurück.

Heute schrieb man den 12. April. Es war ein Festtag in Berlin, das
in verschwenderischem Schmuck prangte. Die Gesandtschaft des Mars
sollte vom Kaiser empfangen werden. Unter Glockengeläut und
Kanonendonner drängte sich eine jubelnde Menge in den Straßen. In
goldigem Eigenlicht wie die Morgenröte strahlend, mit nie gesehenen
Verzierungen geschmückt, bewegte sich ein glänzender Zug kleiner
Luftgondeln, in Mannshöhe über dem Boden schwebend, durch die
Straßen; von den Fenstern aus überschütteten die Damen den Zug,
trotz der frühen Jahreszeit, mit kostbaren Blumen. Brausende
Hurrarufe betäubten das empfindliche Ohr der Martier.

Torm hatte seinen Platz auf der Tribüne im Lustgarten nicht
benutzt. Ihm waren diese Martier verhaßt. Hatten sie ihm doch den
Haupterfolg seiner Expedition und nun auch die Freude der Heimkehr
ins eigene Haus geraubt. Unruhig ging er in seinem Zimmer auf und
ab. Es klopfte, und Grunthe trat ein.

„Sie sind auch nicht draußen bei den Narren, ich dachte es mir“,
empfing ihn Torm.

Grunthe runzelte die Stirn und blickte finster vor sich hin.

„Es ist eine Schmach“, sagte er, „die Menge bejubelt ihre
Unterdrücker. Aber das tut sie immer. Morgen wird sie ebenso in
Paris, übermorgen in Rom jubeln, und noch viel ärger. Wenn man das
sieht, so kann man nur sagen, diese Menschen verdienen es nicht
besser, als von den Martiern vernichtet zu werden. Sie werfen sich
ihnen zu Füßen, und so werden sie als Mittel ihrer Zwecke zertreten
werden.“

Torm zuckte die Achseln. „Was sollen sie tun? Nihilit ist kein
Spaß.“

„Und ich sage Ihnen“, entgegnete Grunthe fast heftig, „kein Martier
vermag den Griff des Nihilitapparates zu drehen, keiner einem
Menschen seinen Willen aufzuzwingen, wenn ihm der Mensch mit
festem, sittlichem Willen gegenübertritt, mit einem Willen, in dem
nichts ist als die reine Richtung auf das Gute. Aber jene Engländer
– und wir sind nicht besser – hatten nur das eigene Interesse,
ihren spezifisch nationalen Vorteil, nicht aber die Würde der
Menschheit im Auge, und so sind sie Wachs in den Händen der
Martier. Sie können mir glauben, denn ich habe jenem Ill getrotzt,
vor dem jetzt Kaiser und Könige sich neigen. Ich weiß es freilich,
daß wir verloren sind. Ich habe Ill gesehen, wie er mit seinen
Martiern nur einige Schritte durch den Garten der Sternwarte von
Friedau schlich, auf Krücken gestützt und zusammenbrechend unter
der Erdschwere. Und ich habe ihn heute gesehen, durch den Garten
des Kanzlerpalais schreitend, aufgerichtet wie ein Fürst, im
schimmernden Panzerkleid; unter den Knien schützten ihn weit nach
den Seiten ausgebogene Schäfte und über dem Haupt, auf kaum
sichtbaren Stäben, von der Schulter gestützt, der glänzende
diabarische Glockenschirm gegen die Schwere. So haben sie es
verstanden, sich von dem Druck der Erde unabhängig zu machen. Aber
dies alles würde ihnen nichts nützen, wenn wir selbst wüßten, was
wir wollen.“

Auf der Treppe entstand Lärm. Man vernahm eine helle Stimme.

„Sakri, lassens mich los! Ich kenn’ mich schon aus.“

„Das ist Saltner“, rief Torm. Er stürzte zur Tür. Sie flog auf.

„Da bin ich halt wieder! Grüß Gott viel tausendmal!“

Er schüttelte beiden die Hände.

„Und meine Frau?“ war Torms erste Frage.

„Machens sich keine Sorge!“ sagte Saltner. „Die Frau Gemahlin wird
bald nachkommen, es geht ja jetzt alle paar Tage ein Schiff nach
der Erde.“

„So ist sie nicht mitgekommen?“ rief Torm erbleichend.

„Sie hat halt nicht gekonnt. Sie ist ein bisserl bettlägrig, aber
’s hat weiter nichts auf sich, nur daß sie der Doktor nicht gerad
wollt’ reisen lassen.“

„So hat sie geschrieben?“

„Schreiben konnte sie nicht. Aber grüßen tut sie gewiß vielmals.“

„So haben Sie sie gar nicht gesprochen?“

„Das war mir gerad in den Tagen nicht möglich, weil sie noch zu
schwach war. Aber der Doktor sagt, sie wird bald soweit sein, daß
sie reisen kann. Sie brauchen sich wirklich nicht zu ängstigen.“

Torm setzte sich.

„Und Ell?“ fragte er finster. „Wo ist Ell?“

„Er ist zurückgeblieben, bis die Frau Gemahlin reisen kann. Er
wollte sie nicht allein lassen. Es ist vielleicht unrecht, daß ich
allein gereist bin und nicht gewartet hab. Aber schauen Sie, die
Sehnsucht, und dann dacht’ ich, es wär doch besser, ich brächte
Ihnen selbst die Auskunft, als daß wir bloß schreiben sollten.“

„Es ist recht, daß Sie kamen“, sagte Torm, sich erhebend,
„verzeihen Sie, daß ich zuerst an mich dachte, ich habe Ihnen ja
soviel und herzlich zu danken. Und jetzt komme ich sogleich wieder
mit einer Bitte. Sie sollen mir einen Platz auf dem nächsten
Raumschiff erwirken, ich will nach dem Mars!“

Saltner und Grunthe blickten ihn erstaunt an.

„Das werden Sie doch nicht tun!“ rief Saltner. „Sie würden sich mit
der Frau Gemahlin verfehlen.“

„Das werde ich nicht. Ill ist hier. Grunthe wird mir die Bitte
nicht verweigern, er wird mit ihm sprechen, uns eine Lichtdepesche
zu gewähren. Wir werden erfahren, ob Isma noch dort ist, wir werden
uns verständigen. Und wenn ihre Krankheit noch anhält, so werde ich
reisen. Ich werde.“

„Das Reisen läßt sich schon machen. Ich bin jetzt mit der
Gesandtschaft, das heißt heute im Nachtrab, angekommen, daher weiß
ich’s. Von jetzt ab geht alle Wochen ein Luftschiff von hier nach
dem Pol, und von dort an jedem 15. des Monats ein Raumschiff nach
dem Mars, das Menschen als Passagiere mitnimmt. Man will den
Planetenverkehr eröffnen. Es kostet hin inklusive Verpflegung bloß
500 Thekel – 5.000 Mark wollte ich sagen.“

Torm sah ihn verwundert an. „Bloß?“ fragte er.

„Ja, wir haben Geld. Fünftausend Mark sind die Währungseinheit.“

Torm ergriff seine Hand. „Setzen Sie sich erst und erzählen Sie
dann.“

Saltner nahm Platz und begann zu sprechen. Grunthe fragte mitunter
dazwischen. Torm aber hörte nur halb, seine Gedanken waren auf dem
Mars. Sie war krank! Und immer wieder kam ihm die Frage, wie konnte
Saltner dessen sicher sein? War sie auch wirklich krank? Und wenn
sie nicht krank war?

„Ich muß reisen!“ rief er plötzlich.

„Nun, nun“, sagte Saltner beruhigend. „Im Moment können Sie nichts
tun. Ill ist jetzt gerade im Schloß.“

Torm sank auf seinen Platz zurück.

Erneuter Kanonendonner verkündete, daß sich der Kaiser neben dem
Präsidenten des Polreichs vor dem jubelnden Volk zeigte.

Grunthe stand auf und schloß das Fenster.

\tb

Isma lag bleich und angegriffen auf ihrem Sofa. Langsam genas sie
von der schweren nervösen Krankheit, die sie unter dem
Zusammenwirken der ungewohnten Lebensverhältnisse und der
seelischen Aufregungen ergriffen hatte.

Hil trat bei ihr ein.

„Wann kann ich reisen?“ war, wie immer, ihre erste Frage.

„Nun, nun“, sagte er, „sobald wir kräftig genug sind.“

„Ach, Hil, das sagen Sie nun schon seit vierzehn Tagen. Lassen Sie
es mich doch versuchen!“

„Erst müssen wir einmal einen Versuch machen, wie es Ihnen bekommt,
wenn Sie hier in Ihrem Zimmer anfangen, wieder ein wenig mit der
Welt zu verkehren. Es wartet da schon lange einer, der Sie gern
einmal sprechen und sehen möchte, aber ich habe bis jetzt nicht
erlaubt –“

„Und heute darf er kommen, ja?“ unterbrach ihn Isma lebhaft.

Hil lächelte. „Es ist ein gutes Zeichen, daß Sie selbst danach
verlangen. Aber hübsch ruhig, Frau Isma, und höchstens ein
Viertelstündchen! So will ich es ihm sagen lassen.“

Er verabschiedete sich.

Es dauerte nur wenige Minuten, bis Ell eintrat.

Eine leichte Blutwelle drängte sich in Ismas Wangen, als sie ihm
langsam die schlanke Hand entgegenstreckte, die er leidenschaftlich
küßte. Lange hielt er die Hand fest, bis sie sie ihm sanft entzog.

„Sie sind schon lange zurück?“ sagte sie endlich verlegen.

„Auf die Nachricht, daß Sie reisen dürften, kam ich hierher. Ich
hätte Sie nicht allein reisen lassen, obwohl – doch sprechen wir
von Ihnen. Ich fand Sie erkrankt. Es war unmöglich, Sie
wiederzusehen.“

„Und Sie sind mir nicht mehr böse?“

„Isma!“

„Ich habe es eingesehen, ich war ungerecht gegen Sie. Und ich war
doch schuld, daß Sie Ihren Posten auf der Erde verließen –“

„Sie wollten das Beste. Ich aber habe eine Schuld auf mich geladen
– und ich werde sie büßen müssen. Jetzt ist für mich auf der Erde
nichts mehr zu tun, aber die Zeit wird wieder kommen. Dann soll es
nicht an mir fehlen.“

„Und Sie wollen mich begleiten?“

„Wenn Sie reisen dürfen. Aber –“

„Was haben Sie, Ell? Seien Sie aufrichtig, ich beschwöre Sie –
sagen Sie mir die Wahrheit! Sie glauben, ich werde nie wieder –“

„Um Gottes willen, Isma, wenn Sie so sprechen, darf ich nicht
hierbleiben. Sie dürfen sich nicht erregen. Sicherlich ist Ihr
Gesundheitszustand in kurzer Zeit so vorgeschritten, daß Sie die
Reise antreten dürfen. Nein, ich dachte nur an Verzögerungen, die
möglicherweise aus anderen Gründen eintreten könnten, falls sich
der Antritt der Reise nicht bald ermöglichen läßt –“

„Verbergen Sie mir nichts. Man sagt mir sehr wenig von der Erde.
Ich denke, Ill ist mit so großartigem Jubel in Berlin aufgenommen
worden. Und mein Mann ist gesund –“

„Darüber können Sie beruhigt sein. Ich darf Ihnen noch mehr sagen,
Hil hat es jetzt erlaubt. Sollten Sie aus irgendeinem Grund an der
Reise verhindert sein, so werden Sie Ihren Mann doch bald
wiedersehen. Er ist an der Nordpolstation und erwartet dort die
Nachricht, ob Sie kommen oder ob er nach dem Mars reisen soll.“

„Nach dem Mars will er kommen! Und das wissen Sie? Und ich –?“

„Briefe können noch nicht hier sein. Es kam nur eine Lichtdepesche
von Ill. Aber Hil wollte Sie mit der Nachricht nicht aufregen – nun
seien Sie auch vernünftig und zeigen Sie, daß Sie die Probe
bestehen und uns nicht wieder kränker werden.“

„Er will kommen! Aber wozu? Ich möchte doch lieber nach der Erde!“

„Das sollen Sie ja auch. Nur für den Fall –“

„Was für einen Fall?“

„Wenn zum Beispiel die Verhältnisse auf der Erde in der nächsten
Zeit sehr unruhig werden sollten –“

„Ich denke, alles ist jetzt friedlich.“

„Die letzten Nachrichten sind weniger erfreulich.“

„Erzählen Sie, schnell! Unsre Viertelstunde ist bald um.“

„Die Mächte sind in Streit geraten. Was soll ich Sie mit den
politischen Einzelheiten ermüden, die ich selbst nur mangelhaft
hier kenne, weil bisher erst Lichtdepeschen hergelangt sind. Es ist
der Streit um die englische Erbschaft. Frankreich und Italien,
Deutschland und Frankreich, Österreich und Rußland rechten um ihre
Grenzen im Kolonialbesitz in Afrika, Asien und der Türkei. Am
Mittelmeer gibt es kaum einen Punkt, über den man sich einigen
kann. England ist ohnmächtig, die Marsstaaten schützen es in
einigen Punkten, und gerade diese möchten die andern haben. Die
Staaten rüsten gegeneinander, schon sind an den Kolonialgrenzen
Schüsse gefallen, man muß darauf gefaßt sein, daß ein Weltkrieg
ausbricht. Dies werden die Martier auf keinen Fall zugeben, und so
steht zu befürchten, daß wir zu neuen Gewaltmaßregeln gegen die
Menschen, diesmal auch gegen Deutschland, getrieben werden. Deshalb
wäre es gut, wenn Sie bald reisen könnten, ehe vielleicht wieder
eine Sperrung eintritt. Auf jeden Fall aber würde Torm
hierherkommen dürfen. Das hat Ill ihm zugesichert.“

Isma schüttelte den Kopf. „Was Sie da alles sagen, verwirrt mich,
ängstigt mich –“ Und nach kurzem Schweigen fuhr sie fort: „Aber ich
will gesund sein! Ich will gar nicht darüber nachdenken. ich fühle,
daß ich Ruhe brauche. Ich danke Ihnen herzlich, Ell, daß Sie
gekommen sind. Nun weiß ich doch wieder, daß ich nicht verlassen
bin.“

Sie reichte ihm die Hand.

„Leben Sie wohl, Isma. Sie können ganz ruhig sein. Sie werden bald
gesund sein.“

Er sah sie an mit den alten, treuen Augen und ging. Sie lächelte
müde und lehnte sich zurück. Die Lider fielen ihr zu.

„Ich will gesund sein“, dachte sie. Aber sie hörte schon nicht
mehr, daß Hil bei ihr eintrat und sie teilnahmsvoll betrachtete.

\tb

Eine Woche später, es war ein herrlicher Maitag, tobte eine
aufgeregte Volksmenge in den Straßen der europäischen Städte.
Überall hörte man Beschimpfungen der Martier. Wo man vor vier
Wochen gejubelt hatte und Hurra geschrien, ertönte jetzt: „Nieder
mit dem Mars!“ Die Geschäfte mit Marsartikeln, die wie Pilze in die
Höhe geschossen waren, sahen sich genötigt, ihre Läden zu
schließen. „Nieder mit den Glockenjungens“, hieß es in Berlin, wo
man die Martier ihrer diabarischen Helme wegen mit diesem
geschmackvollen Titel beehrte. Die Menge demonstrierte vor dem
Gebäude, das die Marsstaaten für ihre Botschaft gemietet hatten.
Auf dem flachen Dach ruhten die Luftschiffe, bereit, in der
nächsten Stunde die Hauptstadt zu verlassen.

Aber nicht weniger erregt, vielmehr erfüllt von einem heiligen
Zorn, war die Stimmung auf dem Mars. Die Nachricht von einem
ungeheuren Blutvergießen der Menschen untereinander war angelangt.
In der Türkei und in Kleinasien, wo man hauptsächlich nur aus
Furcht vor England sich soweit im Zaume gehalten hatte, daß die
europäischen Fremden sich sicher fühlen durften, war jetzt diese
Schranke gefallen. Der mohammedanische Fanatismus flutete über. Auf
einen heimlichen Wink der türkischen Regierung erhoben sich die
Massen. Ein entsetzliches Gemetzel begann gegen die Christen. Die
Gebäude der Botschaften wurden erstürmt, Männer, Kinder und Frauen
binnen einer Nacht in gräßlicher Weise gemordet. Und furchtbar war
die Rache. So weit die Kanonen der fremden Kriegsschiffe reichten,
wurden am andern Tag die blühenden Küsten, Paläste und Moscheen
Konstantinopels in Trümmerhaufen verwandelt. Und nicht genug damit.
Zwischen den europäischen Staaten selbst entbrannte die Eifersucht,
wer die Trümmer mit seinen Truppen besetzen sollte. Der Krieg war
so gut wie ausgebrochen, ehe er formell erklärt war.

Tiefe Empörung ergriff die Bevölkerung der Marsstaaten. Der
Antibatismus gewann die Oberhand. Das Parlament forderte von der
Regierung die sofortige Unterdrückung der Greuel und die
Herstellung des Friedenszustandes auf der Erde. Am 12. Mai beschloß
das Parlament unter Zustimmung des Zentralrats folgendes:

„Da die Menschen nicht fähig sind, aus eigener Macht unter sich
einen friedlichen Kulturzustand zu erhalten, sieht sich die
Regierung der Marsstaaten gezwungen, hiermit das Protektorat über
die gesamte Erde zu erklären und jede politische Aktion der
Erdstaaten untereinander, ohne vorherige Zustimmung der
Marsstaaten, zu verbieten. Der Präsident des Polreichs der Nume auf
der Erde wird beauftragt und bevollmächtigt, alle Maßregeln sofort
anzuordnen, die er für notwendig erachtet, um dem ausgesprochenen
Willen der Marsstaaten auf der Erde, und zwar zunächst in Europa,
Geltung zu verschaffen.“

Es war dieser Beschluß der Marsstaaten und die von Ill hinzugefügte
Erklärung, wodurch die Bevölkerung aller zivilisierten Staaten in
so außerordentliche Aufregung geraten war. Die Mitteilung an die
Regierungen war gleichzeitig in Form einer Bekanntmachung in den
europäischen Staaten von den Martiern verbreitet worden. Man zerriß
jetzt die Blätter, die sie enthielten, man entfernte die Plakate
von den Häusern. Die Bekanntmachung lautete folgendermaßen:

„Indem ich den vorstehenden Beschluß der Marsstaaten zur
allgemeinen Kenntnis bringe, übernehme ich mit dem heutigen Tage in
ihrem Namen die Schutzherrschaft über alle Staaten der Erde und
bestimme wie folgt:

Alle Regierungen und Nationen werden bis auf weiteres in ihren
verfassungsmäßigen Rechten bestätigt und sind in ihren inneren
Angelegenheiten frei, mit Ausnahme der unten angegebenen Bestimmung
über das Heerwesen.

Alle internationalen Verträge und Kundgebungen bedürfen zu ihrer
Gültigkeit der durch mich zu vollziehenden Bestätigung der
Marsstaaten.

Alle Kriegsrüstungen sind verboten. Die von den europäischen
Regierungen ausgegebenen Mobilisierungsbefehle sind aufzuheben. Die
Friedenspräsenzstärke ihrer Heere wird auf die Hälfte der
bisherigen herabgesetzt. Die Hauptwaffenplätze werden unter
Oberaufsicht eines von mir zu ernennenden Beamten gestellt.

Alle Regierungen werden eingeladen, bevollmächtigte Vertreter zu
der Weltfriedenskonferenz zu entsenden, die am 30. Mai unter meinem
Vorsitz am Nordpol der Erde wird eröffnet werden.

Von der Bevölkerung der Erde erwarte ich, daß sie die Bemühungen
der Marsstaaten, ihr die vollen Segnungen des Friedens und der
Kultur zu bringen, mit allen Kräften unterstützen wird.

Gegeben am Nordpol der Erde, den 15. Mai

Ill,

Präsident des Polreichs der Nume.

Bevollmächtigter Protektor der Erde.“

Mit klingendem Spiel und von der Menge mit Hochrufen begrüßt
rückten zwei Kompagnien der Garde vor das Gebäude der Botschaft der
Marsstaaten, um dasselbe gegen etwaige Übergriffe der aufgeregten
Bevölkerung zu schützen. Ein Adjutant begab sich in das Haus, um
dem Botschafter zu melden, daß die Regierung Seiner Majestät dem
Präsidenten des Polreichs nach dem bereits telegraphisch
übermittelten Protest nichts weiter mitzuteilen habe.

Eine Viertelstunde später erhoben sich die Luftschiffe der Martier
und richteten unter dem tobenden Gejohle der Menge ihren Flug nach
Norden.

\section{43 - Die Besiegten}

Es war an einem regnerischen Augustabend des Jahres, das auf die
tumultuarische Abreise der Gesandtschaft der Marsstaaten aus Berlin
gefolgt war, als ein Mann, in einen Reisemantel gehüllt, hastig die
menschenleere Straße hinaufstieg, die nach der Sternwarte in
Friedau führte. Ein dichter Bart und der tief ins Gesicht gerückte
Hut ließen wenig von seinen Zügen erkennen. Hin und wieder warf er
aus scharfen Augen einen scheuen Blick nach der Seite, als
fürchtete er, beobachtet zu werden. Aber niemand bemerkte ihn. Die
Laternen waren noch nicht angezündet, und der leise niederrieselnde
Regen verschluckte das letzte Licht der Dämmerung.

Je näher der Fremde dem eisernen Gittertor der Sternwarte kam, um
so mehr verzögerte sich sein Schritt, als suche er einen Augenblick
hinauszuschieben, den er noch eben so eilig erstrebte. Vor dem Tor
stand er eine Weile still. Er spähte nach den dunkeln Fenstern des
Gebäudes. Er nahm den Hut ab und trocknete die Stirn. Sein Gesicht
war tief gebräunt und trug die Spuren harter Entbehrungen und
schwerer Sorgen, die ihm das Haar gebleicht hatten. Mit einem
plötzlichen Entschluß zog er die Klingel.

Es dauerte lange, ehe sich ein Schritt hören ließ. Ein junger
Hausbursche öffnete die Tür.

„Ist der Herr Direktor zu sprechen?“ fragte der Fremde mit tiefer
Stimme.

„Der Herr Doktor Grunthe ist ausgegangen“, antwortete der Diener.
„Aber um halb neun kommt er wieder.“

„Ist denn Herr Dr. Ell nicht mehr hier?“

„Den kenne ich nicht. Oder – Sie meinen doch nicht etwa – aber das
wissen Sie ja –“

„Ich meine den Herrn Dr. Ell, der die Sternwarte gebaut hat.“

„Ja – der Herr Kultor residieren doch in Berlin –“

Der Fremde schüttelte den Kopf. „Ich werde in einer Stunde
wiederkommen“, sagte er dann kurz.

Er wandte sich um und ging. Der Herr Kultor? Was sollte das heißen?
Er wußte es nicht. Gleichviel, er würde ihn finden. Also Grunthe
war hier. Das war ihm lieb, bei ihm konnte er Auskunft erhalten.
Aber wohin inzwischen?

Einige Häuser weiter, in einem Nebengäßchen, leuchtete eine rote
Laterne. Er fühlte das Bedürfnis nach Speise und Trank. Er wußte,
die Laterne bezeichnete ein untergeordnetes Vorstadtlokal; von den
Gästen, die dort verkehrten, kannte ihn gewiß niemand, würde ihn
niemand wiedererkennen. Dorthin durfte er sich wagen.

Er trat ein und nahm in einer Ecke Platz. Das Zimmer war fast leer.
Er bestellte sich etwas zu essen.

„Wünschen Sie gewachsen oder chemisch?“ fragte der Wirt.

„Was ist das für ein Unterschied?“

Der Wirt sah den Fremden erstaunt an. Dieser bedauerte seine Frage,
da er sah, daß er dadurch auffiel, und sagte schnell: „Geben Sie
mir nur, was das Beste ist.“

„Das ist Geschmackssache“, sagte der Wirt. „Das Gewachsene ist
teurer, aber wer nicht für das Neue ist, zieht es doch vor.“

„Was essen Sie denn?“ fragte der Fremde.

„Immer chemisch, ich habe eine große Familie. Und – es schmeckt
auch besser. Aber, wissen Sie, man will es mit keinem verderben –
und das Gewachsene gilt für patriotischer. Ich habe sehr
patriotische Gäste.“

„Vor allen Dingen bringen Sie mir etwas, ich habe nicht viel Zeit.
Also chemisch.“

„Kohlenwurst, Retortenbraten, Mineralbutter, Kunstbrot, alles
modern, aus der besten Fabrik, à la Nume.“

„Was Sie wollen, nur schnell.“

Der Wirt verschwand, und der Fremde griff eifrig nach einer
Zeitung, die auf dem Nebentisch lag. Es war das ›Friedauer
Intelligenzblatt‹. Mit einer plötzlichen Regung des Ekels wollte er
das Blatt wieder beiseite schieben, aber er überwand sich und
begann zu lesen. Zufällig haftete sein Blick auf ›Gerichtliches‹.

„Wegen mangelhaften Besuchs der Fortbildungsschule für Erwachsene
wurden achtundzwanzig Personen mit Geldstrafen belegt; eine Person
wurde wegen dauernder Versäumnis dem psychologischen Laboratorium
auf sechs Tage überwiesen. Dem psychophysischen Laboratorium wurden
auf je einen Tag überwiesen: drei Personen wegen Bettelns, eine
Person wegen Tierquälerei, fünf Personen wegen Klavierspielens auf
ungedämpften Instrumenten. Die Klaviere wurden eingezogen. Der
ehemalige Leutnant v. Keltiz, welcher seinen Gegner im Duell
verwundete, wurde zu zehnjähriger Dienstleistung in Kamerun, die
beiden Kartellträger zu einjähriger Deportation nach Neu-Guinea
verurteilt. Allen wurden die bürgerlichen Ehrenrechte aberkannt.
Der vom Schwurgericht zum Tode verurteilte Raubmörder Schlack wurde
zu zehnjähriger Zwangsarbeit in den Strahlenfeldern von Tibet
begnadigt.“

Kopfschüttelnd sah der Fremde nach einer andern Stelle und las:
„Die Petition, welche mit mehreren tausend Unterschriften aus
Friedau an den Verkehrsminister gerichtet war und die Bitte
aussprach, unserer Stadt eine Haltestelle für das Luftschiff
Nordpol-Rom zu gewähren, hat wieder keine Beachtung gefunden.
Unsere Leser wissen wohl, warum unsere Stadt bei gewissen
einflußreichen Numen schlecht angeschrieben steht. Wir werden uns
trotzdem nicht abhalten lassen, immer wieder darauf hinzuweisen,
daß das rätselhafte Verschwinden unseres großen Mitbürgers und
Ehrenbürgers Torm im Mai vorigen Jahres noch immer nicht aufgeklärt
worden ist, wie unangenehm die Erinnerung daran auch für manche
sein mag.“

Das Blatt zitterte in der Hand des Fremden. Seine Augen überflogen
noch einmal die Stelle. Da trat der Wirt mit den Speisen herein.
Der Gast legte die Zeitung möglichst unbefangen beiseite.

„Der Retortenbraten ist leider ausgegangen“, sagte der Wirt. „Aber
die Kohlenwurst ist zu empfehlen, von richtiger Friedauer
Schweinewurst gar nicht zu unterscheiden. Bestes Mineralfett darin,
nicht etwa Petroleum, die Kohle ist aus atmosphärischer Kohlensäure
gezogen, der Wasserstoff aus Quellwasser, der Stickstoff ist
vollständig argonfrei, die Zellbildung nach neuester martischer
Methode im organischen Wachstumsapparat hergestellt mit absoluter
Verdaulichkeit –“

„Es ist wirklich sehr gut“, sagte der Gast, mit großem Appetit
essend. „Aber wo haben Sie denn Ihre Chemie her?“

„Ich? Was meinen Sie denn? Muß ich nicht jeden Tag zwei Stunden in
der Fortbildungsschule sitzen? Denken Sie, ich gehe nur hin, um
meine zwei Mark Lern-Entschädigung einzustreichen? Da war neulich
einmal so ein König oder Herzog vom Mars hier durchgereist, der
sich die Erde beschauen wollte, der wollte mich durchaus als
chemischen Küchenchef mitnehmen, habe es aber abgeschlagen, weil es
auf dem Mars keine Hühner gibt. Und ein richtiges Rührei, das ist
das einzige Erdengut, wovon ich mich nicht trennen kann. Soll ich
Ihnen vielleicht eins machen lassen?“

„Ich danke, geben Sie mir noch ein Glas Bier.“

„Sofort. Nicht wahr, das ist fein? Das exportieren wir sogar nach
dem Mars. So was haben sie dort noch gar nicht gekannt, wie das
Friedauer Batenbräu.“

„Verkehren denn auch Martier bei Ihnen?“

„Nume meinen Sie? Oh, ich könnte sie schon aufnehmen, habe ein paar
Extrazimmer. Gewiß verkehren sie hier, ich meine, sie werden noch
verkehren, ich werde auf dem Mars annoncieren lassen. Fritz, noch
ein Bier für den Herrn! Das ist mein Oberkellner. Ist so vornehm
daß er erst abends um acht Uhr antritt. Sie werden gleich sehen,
wie voll mein Lokal wird, jetzt ist nämlich die Fortbildungsschule
aus, dann kommen die Herren hierher.“

„Wo ist denn die Fortbildungsschule?“

„Die Kaserne ist gleich nebenan, in der nächsten Straße.“

„Das weiß ich, aber die Schule?“

Der Wirt machte wieder ein erstauntes Gesicht. „Entschuldigen Sie“,
sagte er, „sind Sie denn nicht aus Europa? Dann müßten Sie doch
wissen, daß die Kasernen so ziemlich alle in Schulen umgewandelt
sind?“

„Ich war allerdings zwei Jahre verreist, in China und Indien –“

„Zwei Jahre! Ei, da wissen Sie wohl gar nicht –. Militär haben wir
ja nicht mehr, bis auf fünf Prozent der früheren Präsenzstärke.
Dafür bekommt jeder eine Mark pro Stunde, die er in der
Fortbildungsschule sitzt. Ich sage Ihnen, gelehrt sind wir schon,
das ist kolossal. Nächstens gebe ich ein philosophisches Buch
heraus, auf das will ich Stadtrat werden, oder vielleicht
Regierungsrat. Nämlich wegen der Schwerkraft. Auf dem Mars ist doch
alles leichter. Nun schlage ich vor, wenn man schwer von Begriffen
ist, so geht man auf den Mars, und dort – ah, guten Abend, Herr von
Schnabel, guten Abend, Herr Doktor, guten Abend, Herr Direktor –,
entschuldigen Sie, ich will nur die Herren bedienen –“

Der Wirt wandte sich zu den eingetretenen Gästen, die sich an ihren
Stammtisch setzten.

Der Fremde hatte seine Mahlzeit beendet. Er sah nach der Uhr, es
war noch zu früh, um Grunthe zu treffen. Er rückte sich tiefer in
die Ecke, blickte in die Zeitung und wandte den Gästen den Rücken
zu. Sie waren ihm bekannt. Seltsam, dachte er im stillen, während
er, scheinbar in seine Lektüre vertieft, auf ihre Stimmen hörte,
wie kommen die Leute in diese Vorstadtkneipe? Früher hatten sie
ihren Stammtisch im ›Fürst Karl Sigmund‹, dieser Schnabel führte da
das große Wort. Er scheint auch jetzt wieder zu schimpfen.

Die halblauten Stimmen der Stammgäste waren deutlich vernehmbar,
insbesondere das hohe, quetschige Organ Schnabels.

„Haben Sie wieder den Knicks von der Warsolska gesehen“, sagte
Schnabel, „wie der Kerl, der Dor, rausging? Und wie die Anton die
Augen verdrehte? Und die haben am allermeisten geschimpft, als die
ersten Instruktoren herkamen. Und jetzt fletschen sie vor Vergnügen
die Mäuler.“

„Und bei Ihnen war’s umgekehrt, lieber Schnabel“, sagte Doktor
Wagner, mit einem Auge blinzelnd. „Jetzt schimpfen Sie, aber ich
kenne einen, der an den ersten Instruktor Wol einen großen
Rosenkorb geschickt hat mit den schönen Versen:

Sei mir gegrüßt, erhabner Nume,

Dich kränzet zu der Erde Ruhme

Ein Bat mit seiner schönsten Blume –“

„Ach, hören Sie auf“, rief Schnabel ärgerlich. „ich hatte mir die
Geschichte anders gedacht. Ich bin von den Numen enttäuscht worden
–“

„Und die Warsolska ist wahrscheinlich nicht enttäuscht worden.“

„Die verdammten Kerle. Aber die Anton ist doch eigentlich über die
Jahre hinaus –“

„Pst! meine Herren, Vorsicht!“ sagte der Fabrikbesitzer Pellinger,
den der Wirt mit Herr Direktor angeredet hatte. „Das Klatschgesetz
ist bereits in erster Lesung angenommen. § 1: Wer unberufenerweise
das Privatleben abwesender Personen beurteilt, wird mit
psychologischem Laboratorium nicht unter zwölf Tagen bestraft.“ Und
sein kahles Haupt über den Tisch beugend, richtete er seine
schwarzen Augen auf Schnabel und fuhr fort: „Wie sagt doch der
Dichter?

Denn herrlicher als Kant und Hume

Hebt uns die Weisheit hoher Nume

Empor zu freiem Menschentume.“

Darauf brach er in ein kräftiges Lachen aus.

„Seien Sie endlich still mit Ihren Versen, es ist gar nicht zum
Lachen“, brummte Schnabel.

„Es sind ja auch gar nicht meine Verse.“

„Na, meine auch nicht.“

„Ei, ei“, sagte Wagner, „von wem haben Sie sie denn machen
lassen?“

„Ich glaube, Sie wollen mich beleidigen!“ rief Schnabel.

„§ 2 des Klatschgesetzes!“ sagte Pellinger: „Der Begriff der
Beleidigung ist aufgehoben. Eine Minderung der Ehre kann nur durch
eigene unwürdige Handlungen, niemals durch die Handlungen anderer
erfolgen.“

„Das ist die richtige dumme Martiermoral. Wie kann der Reichstag
sich auf solche Gesetze einlassen? Die Demokraten haben ja freilich
die Majorität. Aber die Regierung! Sie dürfte sich nicht von den
Martiern einschüchtern lassen.“

„Die Regierung heißt Ell, Kultor der Numenheit für das deutsche
Sprachgebiet in Europa“, sagte Wagner.

„Dieser Schuft“, rief Schnabel. „Der Kerl hat elend vor meiner
Pistole gekniffen und ist auf den Mars ausgerissen. Und jetzt
spielt er hier den Diktator. Ich werde den Burschen –“

„Pst, meine Herren, Vorsicht!“ flüsterte Pellinger. „Schimpfen
können Sie, soviel Sie wollen, Herr von Schnabel. Sehen Sie, das
ist eben das Gute an der Numenherrschaft, das müßten Sie doch
dankbar anerkennen, es kann Sie niemand wegen Beleidigungen
verantwortlich machen. Aber um Himmels willen nicht vom Fordern
reden. Seien Sie froh, wenn Ell Gründe hat, nicht auf Ihre Affäre
vor zwei Jahren zurückzukommen. mit dem Laboratorium ist es dann
nicht abgetan, Sie kommen nach Neu-Guinea oder auf die neuen
Strahlungsfelder in der Libyschen Wüste.“

„Sie beschönigen natürlich alles, Herr Pellinger.“

„Wieso?“

„Wie war denn das, als wir neulich von Leipzig zurückkamen und
gemütlich in unserm Wagenabteil schliefen. In – Dingsda – auf
einmal wird die Tür aufgerissen – steht so ein Nume da in
abarischen Stiefeln mit seiner Käseglocke über dem Kopf und winkt
bloß mit der Hand. Im Augenblick ist alles hinaus, und der Kerl
setzt sich allein in unsern schönen Wagen. Wir mußten in die
vollgepfropfte dritte Klasse kriechen. Da haben Sie auch gesagt:
Das ist ganz in der Ordnung, als Nume kann der Mann ein Coupé für
sich allein beanspruchen.“

„Wo soll er denn sonst mit seinem Helm hin? Und wenn kein anderes
frei ist? Wir sind doch einmal die Besiegten!“

„Deswegen brauchen wir nicht so feig zu sein. Aber Sie haben ja
auch damals den Kerl, den Ell, verteidigt –“

„Das möchte ich wirklich wissen“, fiel Wagner ein, „ob er an dem
Verschwinden von Torm unschuldig ist. Man sagt doch, Torm habe ihn
gefordert und sei deshalb von den Martiern beseitigt worden.“

„Das ist nicht möglich“, rief Pellinger. „Damals existierte das
Duellgesetz noch nicht.“

„Aber es war Krieg, und die Martier brauchten unsere Gesetze nicht
anzuerkennen.“

„Da sehen Sie wieder einmal, wie ungerecht Sie urteilen“, sagte
Pellinger. „Torm ist verschwunden, ehe Ell überhaupt auf die Erde
zurückkam. Ich weiß es ganz genau. Ell ist erst nach dem
Friedensschluß am 21. Juni vorigen Jahres zurückgekehrt, Torm ist
aber schon beim Ausbruch des Krieges im Mai verschwunden. Die Sache
muß also anders liegen. Und Grunthe ist auch der Ansicht, daß Ell
unschuldig ist.“

„Ach, Grunthe!“ rief Schnabel. „Das ist ein Mathematikus, der sich
den Teufel um Weiberangelegenheiten kümmert. Davon versteht er
nichts. Und daß die Frau hier dahintersteckt, da will ich meinen
Kopf verwetten. Warum säße sie denn sonst jetzt in Berlin?“

Der Fremde hatte sich plötzlich auf seinem Stuhl bewegt, sich aber
sogleich wieder hinter seine Zeitung zurückgezogen. Doch war
Pellinger dadurch auf ihn aufmerksam geworden, er bedeutete
Schnabel, seine Stimme zu mäßigen.

„Erhitzen Sie sich nicht“, sagte er, „die Sache geht uns eigentlich
gar nichts an. Ich möchte auch nicht gern nach dem neuen
Klatschgesetz ins Laboratorium wandern.“

„Sie würden sich aber ausgezeichnet zu Durchleuchtungsversuchen des
Gehirns eignen“, sagte Wagner. „Das Opfer wären Sie eigentlich der
Wissenschaft schuldig. Sie brauchen sich den Schädel nicht erst
rasieren zu lassen.“

„Mein Gehirn ist zu normal“, antwortete Pellinger. „Aber Sie müssen
ja wissen, Herr Doktor, wie’s dort zugeht, Sie sind ja wohl schon
einen Tag drin gewesen. Haben Sie nicht Übungen machen müssen über
die Ermüdung beim Kopfrechnen? Was hatten Sie denn eigentlich
verbrochen?“

„Was Sie alles wissen wollen!“ sagte Wagner etwas verlegen. „Ich
wollte mir die Apparate einmal ansehen. Ich sage Ihnen, da habe ich
ein Instrument gesehen, mit dem kann man die Träume
photographieren.“

„Ach was“, rief Schnabel, „flunkern Sie doch nicht so, ich war ja
auch –“

„Ei, Sie waren auch schon einmal drin? Prosit, da gratuliere ich.“

Beide gerieten in ein Wortgefecht, während Pellinger aufmerksam den
Fremden am Nebentisch beobachtete. Dieser beglich jetzt seine
Rechnung mit dem Wirt, stand dann auf, setzte den Hut auf den Kopf
und verließ das Zimmer, ohne sich umzublicken.

„Den Mann sollte ich kennen“, sagte Pellinger vor sich hin.

„Wie meinen Sie?“

„Oh, nichts. Es kam mir nur so vor, als wäre der Herr, der eben
fortging, ein alter Bekannter. Aber ich habe mich wohl geirrt. Sie
wollten ja erzählen, wie Sie sich im Laboratorium amüsiert haben.
Hahaha!“

\section{44 - Torms Flucht}

Der Fremde war inzwischen auf die Straße getreten, wo jetzt der
Schein der spärlichen Laternen auf dem feuchten Pflaster glitzerte.
Bald war er wieder vor der Sternwarte angelangt und wurde in das
Haus geführt. Im Vorflur trat ihm Grunthe entgegen.

„Was wünschen Sie?“ fragte dieser, den späten Gast mißtrauisch
musternd.

„Ich möchte Sie in einer Privatangelegenheit sprechen“, sagte der
Fremde mit einem Blick auf den Hausburschen.

Beim Klang der Stimme zuckte Grunthe zusammen.

„Bitte, treten Sie in mein Zimmer.“

Der Fremde schritt voran. Grunthe schloß die Tür. Beide blickten
sich eine Weile wortlos an.

„Erkennen Sie mich wieder?“ fragte der Fremde langsam.

„Torm?“ rief Grunthe fragend.

„Ich bin es. Zum zweiten Male von den Toten erstanden. Ja, ich habe
noch zu leben, bis –“

Er schwankte und ließ sich auf einen Stuhl nieder.

„Wo ist meine Frau?“ fragte er dann.

„In Berlin.“

„Und Ell?“

„In Berlin.“

Torm erhob sich wieder. Seine Augen funkelten unheimlich.

„Und wie – wovon lebt sie dort?“ sagte er stockend. „Was wissen Sie
von ihr?“

„Ich – ich bitte Sie, legen Sie zunächst ab, machen Sie es sich
bequem. Was ich weiß, ist nicht viel. Ihre Frau Gemahlin ist
vollständig selbständig und lebt in gesicherten Verhältnissen. Sie
hat alle Anerbietungen von der Familie Ills und von Ell
zurückgewiesen und nur die Stelle als Leiterin einer der martischen
Bildungsschulen angenommen. Sie müssen wissen, daß sich vieles bei
uns geändert hat –“

„Ich meine, was wissen Sie sonst? Was sagt man –“

Er brach ab. Nein, er konnte nicht von dem sprechen, was ihm am
meisten am Herzen lag, am wenigsten mit Grunthe.

„Was sagt man von mir?“ fragte er. „Meinen Sie, daß ich mich zeigen
darf, daß ich wagen darf, nach Berlin zu reisen?“

„Ich wüßte nicht, was Sie abhalten sollte. Allerdings weiß ich ja
auch nicht, was mit Ihnen geschehen ist, wie es kam, daß Sie
plötzlich verschwunden waren –“

„Werde ich denn nicht verfolgt? Bin ich nicht von den Martiern, die
ja jetzt die Gewalt in Händen zu haben scheinen, verurteilt? Hat
man keine Bekanntmachung erlassen?“

„Ich weiß von nichts – ich würde es doch aus den Zeitungen erfahren
haben, oder sicherlich von Saltner, von Ell selbst – ich weiß wohl,
daß Ell sich bemüht hat, Ihren Aufenthalt ausfindig machen zu
lassen, aber ich habe das als persönliches Interesse aufgefaßt, es
ist niemals eine Äußerung gefallen, daß man Sie – sozusagen –
kriminell sucht. dots{}“

„Das verstehe ich nicht. Dann müssen besondere Gründe vorliegen,
weshalb die Martier schweigen – ich vermute, man will mich sicher
machen, um mich alsdann dauernd zu beseitigen. dots{}“

„Aber ich bitte Sie, ich habe nie gehört, daß Sie Feinde bei den
Numen haben.“

Torm lachte bitter. „Es könnte doch jemand ein Interesse haben –“

Grunthe runzelte die Stirn und zog die Lippen zusammen. Torm sah,
daß es vergeblich wäre, mit Grunthe von diesen
Privatangelegenheiten zu sprechen.

„Ich bin in der Tat“ sagte er leise, „in den Augen der Martier ein
Verbrecher, obwohl ich von meinem Standpunkt aus in einer
berechtigten Notlage gehandelt habe. Und in diesem Gefühl bin ich
hierhergekommen und schleiche umher wie ein Bösewicht, in der
Furcht, erkannt zu werden. Ich weiß nichts von den Verhältnissen in
Europa. Ich bin hierhergekommen, weil ich glaubte, Ell sei hier,
und mit ihm wollte ich ab–, wollte ich sprechen, gleichviel, was
dann aus mir würde. Mein einziger Gedanke war, nicht eher von den
Martiern gefaßt zu werden, bis ich Ell persönlich gegenübergetreten
war. Und das werde ich auch jetzt ausführen. Ich gehe morgen nach
Berlin. Ich habe noch Gelder auf der hiesigen Bank, aber ich habe
nicht gewagt, sie zu erheben, weil ich überzeugt war, man warte nur
darauf, mich bei dieser Gelegenheit festzunehmen.“

„Ich stehe natürlich zu Ihrer Verfügung, aber ich glaube, daß Ihre
Befürchtungen völlig grundlos sind. Und, wenn ich das sagen darf,
daß Sie auch Ell irrtümlich für Ihren Feind halten. Er hat sich
stets gegen Ihre Frau Gemahlin so rücksichtsvoll, freundschaftlich
und fürsorgend verhalten, daß ich wirklich nicht weiß, worauf sich
Ihr Argwohn stützt –“

„Lassen wir das, Grunthe, lassen wir das. Sagen Sie mir vor allem,
wie ist das alles gekommen, wie sind diese Martier hier Herren
geworden, wie sind die politischen Verhältnisse?“

„Sie sollen alles erfahren. Aber ich bitte Sie, erklären Sie mir
zunächst, worauf Ihre Besorgnis gegen die Martier sich gründet –
ich bin ja völlig unwissend über Ihre Erlebnisse. Wir hatten die
Hoffnung aufgegeben, Sie wiederzusehen. Wo kommen Sie her, wo waren
Sie, daß Sie so ohne jeden Zusammenhang blieben mit den
Ereignissen, welche die ganze Welt umgestürzt haben?“

„Nun gut, hören Sie zuerst, was mir geschehen ist. Ich kann mich
kurz fassen. Sie wissen, daß ich die Absicht hatte, selbst nach dem
Mars zu reisen, falls meine Frau nicht kräftig genug war, die Reise
nach der Erde anzutreten.“

„Gewiß. Ill bewilligte Ihnen eine Lichtdepesche, und Sie erfuhren
daraus, daß Sie nicht abreisen sollten, da Ihre Frau Gemahlin mit
einem der nächsten Raumschiffe nach der Erde käme. Sie gingen
darauf Anfang Mai nach dem Nordpol, um Ihre Frau dort zu erwarten.
Ich erhielt noch am 10. Mai einen Brief von Ihnen, in welchem Sie
die Hoffnung aussprachen, bald mit Ihrer Frau, die in den nächsten
Tagen zu erwarten sei, nach Deutschland zurückzukehren. Am 12. war
dann jener unselige Tag der Protektoratserklärung – und seitdem war
jede Spur von ihnen verschwunden.“

„So ist es“, sagte Torm. „An dem Tag, dem 12. Mai kam das
Raumschiff, aber es brachte weder meine Frau noch Ell, sondern die
Nachricht, daß der Arzt die Reise für die nächste Zeit noch
untersagt hätte. Ich geriet dadurch in eine gereizte Stimmung, die
sich noch steigerte, als ich erfuhr, daß die politischen
Verhältnisse sich bis zum Abbruch der friedlichen Beziehungen
zugespitzt hatten. Meine Pflicht rief mich nun unbedingt nach
Deutschland zurück; denn obwohl seit diesem Tag der Verkehr mit
Deutschland aufgehoben war, mußte ich doch annehmen und erfuhr es
auch bald aus ausländischen Blättern, daß das gesamte Heer
mobilisiert werde. Wie aber sollte ich in die Heimat gelangen? Die
Luftboote nach Deutschland verkehrten nicht mehr, und auf meine
direkte Bitte um Beförderung nach einem englischen Platz wurde mir
erwidert, daß ich in meiner Eigenschaft als Offizier der Landwehr
während der Dauer des Kriegszustandes zurückgehalten werden müßte,
es sei denn, daß ich mich ehrenwörtlich verpflichtete, mich nicht
zu den Waffen zu stellen. Das konnte ich selbstverständlich nicht
tun. Nach dem Mars zu gehen, war mir gestattet, aber damit war mir
nun nicht mehr gedient. Ich mußte nach Deutschland. Wiederholte
unangenehme Dispute mit den Offizieren der Martier, von denen die
Insel jetzt wimmelte, machten mir den Aufenthalt unerträglich. Ich
erwog hundert Pläne zur Flucht, selbst an unsern alten Ballon, der
noch immer dort lagert, dachte ich. Endlich beschloß ich auf eigene
Faust die Wanderung über das Eis zu versuchen, die mir ja schon
einmal gelungen war. Im Fall der vorzeitigen Entdeckung konnte
doch, wie ich meinte, meine Lage nicht verschlimmert werden.
Natürlich wurde ich entdeckt und zurückgebracht. Man kündete mir
an, daß ich wegen Verdachts der Spionage die Insel Ara nicht mehr
verlassen dürfe – vorher hatte man meinen Besuchen auf den
benachbarten Inseln nichts in den Weg gelegt – und daß ein
Kriegsgericht oder dergleichen über mein weiteres Schicksal
beschließen würde. Schon glaubte ich, daß man mich nach dem Mars
bringen würde; dann konnte ich wenigstens hoffen, meine Frau zu
treffen. Aber ich erfuhr bald, daß ich jedenfalls auf der
Außenstation interniert werden würde, von wo jede Flucht für mich
unmöglich war. In dieser Zeit dort untätig als Gefangener zu
sitzen, war mir ein entsetzlicher Gedanke. Ich faßte einen
Entschluß der Verzweiflung. Jetzt sehe ich ein, daß es eine Torheit
war. Doch Sie müssen sich in meine damalige Stimmung versetzen –
wenn Sie es können. Ich bildete mir außerdem ein, man werde mich
daheim für einen fahnenflüchtigen Feigling halten, wenn ich nicht
vom Pol zurückkehrte. Ich hatte auch keine Zeit zur Überlegung,
denn am nächsten Tag sollte das Kriegsgericht sein, worauf ich
sofort in den Flugwagen gebracht worden wäre. – Kurzum, ich
beschloß, die Zeit zu benutzen, während der ich mich noch auf der
Insel frei bewegen durfte. Die Luftschiffe zu betreten und zu
studieren, war mir immer erlaubt gewesen, ich kannte jetzt ihre
Einrichtung genau und erinnerte mich an das Abenteuer, das Saltner
auf dem Mars erlebt hatte, als er sich in dem Luftschiff des
Schießstandes versteckte. Ich wußte, welches Schiff im Lauf der
nächsten Stunden abgehen würde, denn sowohl nach dem Schutzstaat
England als auch nach andern Teilen der Erde fand lebhafter Verkehr
statt. So glaubte ich, wenn es mir gelänge, mich in dem nach
England gehenden Schiffe zu bergen, von den Martiern selbst
fortbefördert zu werden. Ich wollte es wagen. Ich versah mich mit
etwas Proviant, denn ich war entschlossen, im Notfall zwei Tage in
meinem Versteck zu verbleiben. Da fiel mir ein, daß ich ohne
Sauerstoff apparat unmöglich in der Höhe aushalten könnte, in der
die Schiffe zu fahren pflegen. Hier blieb mir nichts anderes übrig,
als zu stehlen. Ich eignete mir zwei von den Absorptionsbüchsen der
Martier an, mehr konnte ich nicht fortschaffen. Trübes Wetter – wir
hatten ja freilich keine Nacht – begünstigte mein Vorhaben durch
ein starkes Schneegestöber, so daß kein Martier, der nicht durch
sein Amt gezwungen war, sich auf dem Dach der Insel sehen ließ. So
gelang es mir leichter, als ich glaubte, mich in das noch gänzlich
unbesetzte Schiff einzuschleichen, dessen Wächter in einer der
Kajüten beschäftigt war. Es war ein ausnehmend geräumiges Boot, und
ich fand meine Zuflucht, wie damals Saltner, zwischen und hinter
dem Stoff, den Saltner für Heu hielt, der aber, wie Sie jetzt
wissen werden, den besondern Zwecken der Diabarieverteilung dient.
Bei gutem Glück rechnete ich, da noch drei Stunden bis zur Abfahrt
des Schiffes waren, in acht oder neun Stunden in England zu sein
und dann das Schiff ebenso unbemerkt verlassen zu können. Und
wirklich hatte man mich noch nicht vermißt oder nicht im Schiff
gesucht – das Schiff erhob sich. Stunde auf Stunde verging, und ich
schlummerte von Zeit zu Zeit in meinem dunklen Gefängnis ein. Nun
sagte mir meine Uhr, daß wir in England sein müßten. Aber aufs neue
verging Stunde auf Stunde, ohne daß das Schiff zur Ruhe kam. Ich
bemerkte die Bewegung natürlich nur an dem leichten Geräusch des
Reaktionsapparats und dem Zischen der Luft. So oft ich aus
Sparsamkeit mit dem Sauerstoffatmen aufhörte, fühlte ich alsbald,
daß wir noch immer in sehr hohen Schichten sein müßten, und ich
geriet in große Sorge, ob mein Vorrat ausreichen würde. Endlich,
nach mehr als zehnstündiger Fahrt, als ich schon überlegte, ob ich
mich nicht, um dem Erstickungstod zu entgehen, den Martiern ergeben
sollte, erkannte ich zu meiner unbeschreiblichen Freude an der
Veränderung der Luft, daß das Schiff sich senke. Bald hörte ich
auch aus dem veränderten Geräusch, daß es mit Segelflug arbeite.
Ich schloß daraus, daß man eine Landungsstelle suche und sich also
nicht einem der gewohnten Anlegeplätze nähere.

Können Sie sich meinen Zustand, meine nervöse Erregung vorstellen?
Seit zehn Stunden im Finstern eingeschlossen, zuletzt unter
Atemnot, in fortwährender Gefahr, entdeckt zu werden, ohne zu
wissen, wohin die Reise geht, wo ich das Licht des Tages wieder
erblicken werde und wie es mir möglich werden wird, unbemerkt dem
Schiff zu entfliehen! Ich suchte mir einen Plan zu machen – aber wo
würde ich mich dann befinden? Der Zeit nach zu schließen, mußten
wir sechs- bis siebentausend Kilometer zurückgelegt haben. Ich
konnte in Alexandria sein oder in New-Orleans, ebensogut auch in
der Sahara oder in China. Wie sollte ich dann weiterkommen, falls
ich den Martiern entfliehen konnte? Ich mußte alles vom Augenblick
abhängig machen.

Endlich verstummte das Geräusch der Fahrt, ich fühlte den
Landungsstoß, das Schiff ruhte. Es kam nun darauf an, ob es die
Martier verlassen würden. Wenn ich wenigstens gewußt hätte, ob es
Tag oder Nacht war. Aber das hing ja ganz davon ab, nach welcher
Himmelsrichtung wir gefahren waren. Aus der Landung selbst konnte
ich nichts schließen, da ich nichts von der Bestimmung des Schiffes
wußte, für welche ebensogut die Nacht als der Tag die passende
Ankunftszeit sein konnte, je nach den Absichten der Martier. Noch
eine Stunde vielleicht hörte ich über mir Tritte und Stimmen, dann
wurde es still. Ich schlich aus meinem Versteck nach der Drehtür.
Geräuschlos öffnete sich eine Spalte. Es war Nacht! Denn nur ein
ganz schwaches Fluoreszenzlicht schimmerte durch das Innere des
Schiffes. Man hatte also Grund, nach außen hin kein Licht zu
zeigen, man wollte nicht bemerkt sein. Nun öffnete ich die Drehtür
vollends und spähte in den Raum. Die Martier lagen in ihren
Hängematten und schliefen. Wachen befanden sich jedenfalls
außerhalb des Schiffes, aber nach innen konnten sie nicht gut
blicken und hatten auch dort nichts zu suchen. Ich konnte also ohne
Bedenken aus dem untern Raum heraussteigen und zwischen den
Hängematten nach dem Ausgang schreiten; selbst wenn mich jemand
hier bemerkte, hätte er mich doch für einen von der Besatzung
gehalten. So gelangte ich ungefährdet bis an die Treppe, die aufs
Verdeck und von dort ins Freie führte. Die Luke stand offen, aber
auf der obersten Stufe der Treppe saß ein Martier, der, von seinem
Helm gegen die Schwere geschützt, nach außen hin Wache hielt. An
ihm mußte ich vorüber. Ich stieg möglichst unbefangen und ohne mein
Nahen verbergen zu wollen die Stufen hinauf und drängte mich an ihm
vorüber, indem ich die gebückte Haltung der Martier ohne
Schwereschirm annahm. Ich hatte keine andre Wahl, durch List hätte
ich nichts erreicht. So stand ich schon auf dem Verdeck, als der
Martier mich anrief, wo ich hinwollte. Ich antwortete nicht,
sondern suchte nur nach der abwärts führenden Treppe. Sie war aber
eingezogen. Da faßte der Wächter mich an und rief: ›Das ist ja ein
Bat. Was willst du?‹

Zugleich drückte er die Alarmglocke. Was im nächsten Augenblick
geschah, weiß ich nicht mehr deutlich. Ich höre nur einen
Schmerzensschrei, den der Martier ausstieß, als er, von meinem
Faustschlag gegen die Stirn getroffen, die Treppe hinabstürzte. Ich
selbst fühle mich das gewölbte Dach des Schiffes hinabgleiten, doch
ich komme auf die Füße und laufe auf gut Glück vom Schiff fort, so
schnell meine Beine mich tragen wollen. Die Nacht war klar, aber
nur vom Sternenlicht erhellt. Der Boden senkte sich, denn das
Luftschiff war, wie nicht anders zu erwarten, auf einem Hügel
gelandet. Eine endlose Ebene schien sich vor mir auszudehnen; ich
fühlte kurzes Gras unter mir. Als ich es wagte, mich einen
Augenblick rückwärts zu wenden, bemerkte ich, daß hinter mir sich
eine Hügellandschaft befand, die zu einem schneebedeckten Gebirge
aufstieg. Ich hoffte, irgendwo ein Versteck zu finden, das mich vor
den ersten Nachforschungen der Martier verbarg, um mich dann im
Lauf der Nacht noch möglichst weit zu entfernen und bei den
unbekannten Bewohnern des Landes Schutz zu suchen. Da plötzlich
tauchte, wie aus der Erde gestiegen, eine Reihe dunkler Gestalten
vor mir auf, die sich sofort auf mich stürzten und mich
niederwarfen. Ich sah Messer vor meinen Augen blitzen und glaubte
mich verloren.

In diesem furchtbaren Augenblick wurde die Nacht mit einem Schlag
zum Tag. Das Marsschiff hatte seine Scheinwerfer erglühen lassen.
Wie eine Sonne in blendendem Licht strahlend erhob es sich langsam
in die Luft, jedenfalls um mich zu suchen. Dieser Anblick versetzte
die Eingeborenen, die mich überfallen hatten, in einen
unbeschreiblichen Schrecken. Zunächst warfen sie sich auf den
Boden, dann krochen sie, ohne sich um mich zu kümmern, auf diesem
fort und waren in wenigen Augenblicken ebenso plötzlich
verschwunden, wie sie gekommen waren. Ich war frei. Aber was sollte
ich tun? Wenn ich hier blieb, so mußte ich in wenigen Minuten von
den Martiern entdeckt werden. Ich sagte mir, daß sich dort, wo die
Eingeborenen verschwunden waren, auch ein Versteck für mich finden
würde. In der Tat, wenige Schritte vor mir zog eine trockene
Erdspalte quer durch die Steppe. Ich stieg hinab und schmiegte mich
in den tiefen Schatten eines Risses. Von oben konnte ich hier nicht
gesehen werden. Die Martier hatten natürlich bald die Spalte
bemerkt und schwebten langsam über derselben hin, aber ich wurde
nicht entdeckt. Noch mehrfach sah ich das Licht aufleuchten,
endlich verschwand es. Auch von den Eingeborenen sah ich nichts
mehr.

Etwa eine Stunde mochte ich so gelegen haben – es war unangenehm
kalt –, als der erste Schimmer der Dämmerung den Anbruch des Tages
verkündete. Ich verzehrte den Rest meines Proviants, und als es
hell genug geworden war, lugte ich vorsichtig über die Ebene. Das
Schiff mußte sich entfernt haben, es war keine Spur mehr zu sehen.
Ich wanderte nun am Rande des Spaltes weiter. Nicht lange, so
bemerkte ich, daß mir eine große Schar von Bewohnern des Landes
entgegenkam. Ich blieb stehen und suchte durch Bewegungen der Arme
meine friedlichen Absichten verständlich zu machen. Erst glaubte
ich, das Schlimmste befürchten zu müssen, denn die Leute liefen
unter lautem Geschrei auf mich zu und schossen ihre langen Flinten
ab, aber sie zielten nicht auf mich. Bald erkannte ich, daß dies
eine Freudenbezeugung sein sollte. Einige ältere Männer, offenbar
die Anführer, traten an mich heran und verbeugten sich mit allen
Zeichen der Ehrfurcht. Dann kauerten sie sich im Halbkreis um mich
nieder, und ich setzte mich ebenfalls auf den Boden. Allmählich
verständigten wir uns durch Pantomimen, und ich folgte ihrer
Einladung, sie zu begleiten. Nach einer langen Wanderung erweiterte
sich die Spalte zu einem kleinen Tal, und hier fand sich eine
Niederlassung, wo ich mit allen Ehren eines angesehenen Gastes
aufgenommen wurde. Ich blieb einige Tage dort und wurde dann von
meinen Gastfreunden nach Süden geleitet. Nach mehreren Tagereisen
erreichten wir eine ausgedehnte Stadt. Jetzt erst wurde mir nach
und nach klar, wo ich hingeraten war. Die Stadt war Lhasa, die
Hauptstadt von Tibet, der Sitz des Dalai-Lama. Die Tibetaner waren
durch die überirdische Erscheinung des lichtstrahlenden
Luftschiffes in ihrer Gesinnung völlig umgewandelt. Sie hielten
mich für ein wunderbares Wesen, das in einem leuchtenden Wagen
direkt vom Himmel gekommen war. Ich wurde auch in Lhasa sehr
ehrenvoll aufgenommen, aber alle Bemühungen, von hier
weiterzureisen, waren vergebens. Man gestattete nicht, daß ich mich
aus der Stadt entferne. Und so war ich fast ein Jahr in dieser
allen Fremden verschlossenen Stadt. Aber auch dies hatte
schließlich ein Ende.

Sie werden wahrscheinlich wissen, daß die Martier jetzt auf dem
Hochplateau von Tibet große Strahlungsfelder angelegt haben, um
während des Sommers die Sonnenenergie zu sammeln. Die Trockenheit
des Klimas bei der hohen Lage von 5.000 Meter überm Meer sagt ihrer
Konstitution am meisten zu von allen Ländern der Erde. Das Schiff,
mit welchem ich hingekommen war, stellte die ersten Nachforschungen
an, und bald hatten mehr und mehr Schiffe eine große Anzahl der
Martier, vornehmlich die Bewohner ihrer Wüsten, die Beds, dahin
gebracht. Die Tibetaner fühlten sich dadurch beunruhigt und wandten
sich an die chinesische Regierung. Zugleich aber glaubten sie, daß
meine Anwesenheit, die sie übrigens sorgfältig geheimhielten,
Ursache sei, weshalb die wunderbaren Fremden durch die Luft in ihr
Land kämen. So erhielt ich die Erlaubnis, mich einer Karawane
anzuschließen, die über den Himalaja nach Indien ging. Nach
mannigfachen Abenteuern, mit denen ich Sie nicht aufhalten will,
gelang es mir schließlich, mich bis nach Kalkutta durchzuschlagen.
Ich besaß noch eine nicht unbedeutende Summe deutschen Geldes,
durch das ich mich hier wieder in einen europäischen Zustand
versetzen konnte. Indessen wagte ich nicht, mich bei den Behörden
zu melden oder mich zu erkennen zu geben, da ich fürchtete, von den
Martiern verfolgt zu werden. Aus den Zeitungen ersah ich, daß das
Luftschiff, welches von Kalkutta allwöchentlich nach London geht,
in Teheran, Stambul, Wien und Leipzig anlegt. Von Leipzig benutzte
ich den nächsten Zug nach Friedau. Und mein erster Gang war
hierher. Ich habe es vermieden, mit jemand zu sprechen. Ich bin
entsetzt über die Veränderung der Verhältnisse. Nun sagen Sie mir
vor allem, was war unser Schicksal im Krieg mit dem Mars?“

Grunthe hatte ohne eine Miene zu verziehen zugehört. Jetzt sagte er
bedächtig, ohne auf Torms letzte Frage zu achten:

„Hatten Sie Ihren Chronometer und unsern Taschenkalender mit?“

„Ja, aber –“

„So haben Sie doch wohl einige Ortsbestimmungen machen können? Ich
meine nach dem Harzerschen Fadenverfahren, mit bloßem Auge?“

Torm lächelte trüb. „Ich hatte freilich Zeit dazu“, sagte er, „und
habe es auch getan. Sie können sie berechnen. Aber zuerst –“

„Oh, entschuldigen Sie“, unterbrach ihn Grunthe. „Sie wissen, ich
bin ein sehr unaufmerksamer Wirt. Ich hätte ihnen doch zuerst ein
Abendessen anbieten sollen. Allerdings habe ich nichts zu Hause,
doch – wir könnten vielleicht –“

Seine Lippen zogen sich zusammen. Das Problem schien ihm sehr
schwer. „Ich danke herzlich“, sagte Torm. „Ich habe gegessen und
getrunken.“

„Um so besser“, rief Grunthe erleichtert. „Aber logieren werden Sie
bei mir. Das läßt sich machen.“

„Das nehme ich an, weil ich mich nicht gern hier in den Hotels
sehen lassen möchte. Morgen fahre ich ja nach Berlin.“

„Wollen Sie denn nicht an Ihre Frau Gemahlin telegraphieren, daß
Sie kommen? Ich habe die Adresse, da ich wegen der Abrechnungen –
warten Sie, es muß hier stehen – ich kann unsern Burschen nach der
Post schicken

„Das ist nicht nötig“, sagte Torm. „Ich werde – doch die Adresse
können Sie mir immerhin geben.“

Grunthe suchte unter seinen Büchern.

„Ach, sehen Sie“, sagte er, „da finde ich doch noch etwas – im
Frühjahr hat mich Saltner einmal besucht – da ließ ich Wein holen,
und hier ist noch eine Flasche. Gläser habe ich von Ell. Sie müssen
da irgendwo stehen. Das trifft sich gut – wissen Sie denn, was
heute für ein Tag ist? Der neunzehnte August. Heute vor zwei Jahren
kamen wir am Nordpol an. Wie schade, daß Saltner nicht hier ist, er
könnte wieder ein Hoch ausbringen –“

Torm fuhr aus seinem Nachsinnen empor.

„Erinnern Sie mich nicht daran“, sagte er finster. „Mit jener
Stunde begann mein Unglück. Wie kam denn jener Flaschenkorb –“ Er
schlug mit der Hand auf den Tisch und sprang auf. Er unterbrach
sich und murmelte nur noch für sich: „Ich stoße nicht mehr an.“

„Geben Sie nur die Gläser her“, sagte er darauf ruhiger. „Ja, wir
wollen uns setzen. Und nun sind Sie daran zu berichten.“

Grunthe blickte starr vor sich hin.

„Wir sind in der Gewalt der Nume“, begann er nach einer Pause.
„Ganz Europa, außer Rußland. Wir beugen uns vor unserm Herrn. Wir
sind Kinder geworden, die in die Schule geschickt werden. Man hat
sogenannte Kultoren eingesetzt über die verschiedenen
Sprachgebiete. Der größte Teil des Deutschen Reichs, die deutschen
Teile von Österreich und der Schweiz stehen unter Ell. Man will uns
erziehen, intellektuell und ethisch. Die Absicht ist gut, aber
undurchführbar. Das Ende wird entsetzlich sein – wenn es nicht
gelingt –, doch davon später.“

Grunthe schwieg.

„Ich begreife noch nicht“, sagte Torm, „wie war es möglich, daß wir
in diese Abhängigkeit gerieten? Warum unterwarfen wir uns?“

„Entschuldigen Sie mich“, antwortete Grunthe. „Ich bin nicht
imstande, von diesen schmerzlichen Ereignissen zu sprechen. Ich
bringe es nicht über die Lippen. Lassen wir es lieber. Ich werde
Ihnen eine Zusammenstellung der Ereignisse in einer Broschüre geben
– hier sind mehrere. Lesen Sie selbst, für sich allein. Sie werden
auch müde sein. Lesen Sie morgen früh. Reden wir von etwas
anderm.“

Aber sie redeten nicht. Der Wein blieb unberührt. Das Herz war
beiden zu schwer. Einmal sagte Grunthe vor sich hin: „Es ist nicht
der Verlust der politischen Macht für unser Vaterland, der mich am
meisten schmerzt, so weh er mir tut. Schließlich müßte es
zurückstehen, wenn es bessere Mittel gäbe, die Würde der Menschheit
zu verwirklichen. Was mir unmöglich macht, ohne die tiefste
Erregung von diesen Dingen zu reden, ist die demütigende
Überzeugung, daß wir es eigentlich nicht besser verdienen. Haben
wir es verstanden, die Würde des Menschen zu wahren? Haben nicht
seit mehr als einem Menschenalter alle Berufsklassen ihre
politische Macht nur gebraucht, um sich wirtschaftliche Vorteile
auf Kosten der andern zu verschaffen? Haben wir gelernt, auf den
eigenen Vorteil zu verzichten, wenn es die Gerechtigkeit verlangte?
Haben die führenden Kreise sittlichen Ernst gezeigt, wenn es galt,
das Gesetz auch ihrer Tradition entgegen durchzuführen? Haben sie
ihre Ehre gesucht in der absoluten Achtung des Gesetzes, statt in
äußerlichen Formen? Haben wir unsern Gott im Herzen verehrt, statt
in Dogmen und konventionellen Kulten? Haben wir das Grundgesetz
aller Sittlichkeit gewahrt, daß der Mensch Selbstzweck ist und
nicht bloß als Mittel gebraucht werden darf? Oh, das ist es ja
eben, daß die Nume in allem vollständig recht haben, was sie lehren
und an uns verachten, und daß wir doch als Menschen es nicht von
ihnen annehmen dürfen, weil wir nur frei werden können aus eigener
Arbeit. Und so ist es unser tragisches Schicksal, daß wir uns
auflehnen müssen gegen das Gute! Und es ist das tragische Geschick
der Nume, daß sie um des Guten willen schlecht werden müssen!“

Er stand auf und trat an das Fenster.

„Es scheint sich aufzuklären. Vielleicht kann ich noch eine
Beobachtung machen. Wollen Sie mitkommen? Ich zeige Ihnen dabei Ihr
Zimmer.“

Torm ergriff die Broschüren und folgte ihm.

\section{45 - Des Unglück des Vaterlands}

Torm ging unruhig in seinem Zimmer auf und ab. Seine Liebe zu Isma,
das alte, feste Vertrauen, das sich wieder hervordrängte, die
Mitteilungen Grunthes über Ells freundschaftliches Verhalten, das
alles kämpfte in seinem Innern mit dem feindlichen Argwohn, in den
er sich in der Einsamkeit seiner Verbannung immer fester
hineingelebt hatte. Die stets erneute Verzögerung der Heimkehr
Ismas und das gleichzeitige Zurückbleiben Ells, wofür er keinen
Grund einzusehen vermochte, hatten allmählich in ihm den Verdacht
erweckt, daß es Ell doch nicht ehrlich mit ihm meine. Von nun ab
glaubte er überall die Hand Ells im Spiele zu sehen. Die
Verhinderung seiner Heimreise vom Pol schob er ebenfalls auf einen
Einfluß Ells. Wer konnte wissen, welche Lichtdepeschen zwischen den
Planeten, zwischen Neffe und Oheim, gewechselt wurden? Zu seiner
verzweifelten Flucht hatte er sich in einem Moment der Erregung
entschlossen, der noch einen andern Grund hatte, als er Grunthe
gegenüber aussprechen wollte. Bei seinen Disputen mit den Martiern
am Pol hatte er aus der hingeworfenen Bemerkung eines der
martischen Offiziere entnommen, daß man nach den Gesetzen der Nume
ihm überhaupt keinerlei Recht zuerkannte, die Rückkehr seiner Frau
zu verlangen. Die formale Gültigkeit seiner Ehe war auf dem Mars
nicht anerkannt. Niemand hätte es unter den vorliegenden Umständen
Isma verdacht, wenn sie sich als frei erklärt hätte. Dies hatte
Torm in die höchste Aufregung versetzt, und ein nagendes Gefühl der
Eifersucht hatte ihm einen Teil seiner ruhigen Besinnung geraubt.

Jetzt freilich mußte ihm Isma in anderm Licht erscheinen. Hatte er
denn irgendeinen bestimmten Vorwurf gegen sie zu erheben? Sie war
ja zurückgekehrt, und sie hatte sich damit offenbar zu ihm bekannt.
Sollte er nun zu Ell rücksichtslos vordringen und sich vielleicht
rettungslos der Gewalt der Martier ausliefern? War Ell unschuldig,
so war dieses Opfer ganz unnötig gebracht. War Ell aber schlecht,
so gab er sich in seine Hand. Als er seinen Entschluß gefaßt hatte,
zuerst zu Ell zu gehen, wußte er ja noch nicht, daß sich dieser in
einer so unerreichbaren Machtstellung befand. So schien es ihm
jetzt doch als das richtige, sich mit Isma in Verbindung zu
setzen.

Aber wie konnte das ohne Gefahr geschehen? Und vor seinem Geist
stieg die furchtbare Anklage auf, einen Nume bei der Ausübung
seiner Pflicht verletzt, vielleicht getötet zu haben – –. Was ihm
das Mittel werden sollte, Isma wiederzugewinnen, die rücksichtslose
Flucht, nun erschien es ihm als ein verhängnisvolles Schicksal, das
ihn für immer von ihr trennen sollte. Unter dem Druck der schweren
Anklage, die auf ihm lastete, durfte er vor ihre Augen treten? Was
sollte er tun? Mechanisch griff er nach einer der Broschüren, an
die er nicht mehr gedacht hatte. Sein Auge fiel auf die
Überschrift: „Das Unglück vom 30. Mai.“ Er begann zu lesen. Und der
Schmerz um das Vaterland drängte die eigene Sorge zurück.

„Ihr sollt es einst wissen, Kinder und Enkel“, so hieß es, „was uns
geschehen ist, damit ihr weinen könnt und zürnen wie wir. Darum
schreiben wir das Traurige auf, obwohl die Hand unwillig sich
sträubt.

Es war der Tag der großen Parade, an dem der oberste Kriegsherr
sein herrliches Heer musterte, das um die Hauptstadt
zusammengezogen war. Von der zahllosen, begeisterten Menge der
Zuschauer umgeben, waren die glänzenden Regimenter
vorübermarschiert an der ›einsamen Pappel‹. So hieß die Stelle nach
einem Baum, der sich einstmals hier befunden hatte, wo der Monarch,
umringt von der Mehrzahl der deutschen Fürsten und seinen
Generälen, die Heerschau hielt. Nun hatten sich die Truppen weiter
auseinandergezogen und die Gewehre zusammengestellt, während der
Kriegsherr den Führern seine Anerkennung aussprach.

Und da geschah es.

Vor der Hauptstadt des Reiches, an dessen Grenzen man nirgends die
Spur eines Feindes hatte beobachten können.

Im Augenblick der größten Machtentfaltung des stärksten
Landheeres.

Wie ein Schwarm von Raubvögeln schoß es vom Himmel hernieder,
geräuschlos, glänzende, glatte Ungetüme. Und im Moment, da man sie
bemerkte, waren sie auch schon da und hatten die Schar der Anführer
umringt.

›Zu den Truppen!‹ hieß es.

Die Kommandierenden stoben auseinander.

›Zurück! Ergebt euch! Der Weg ist gesperrt!‹ tönte es ihnen aus den
feindlichen Luftschiffen entgegen.

Die Offiziere kümmerten sich nicht darum, sie sprengten weiter.
Aber nicht lange. Keiner passierte den Kreis, den die Schiffe
absperrten. Von einer unsichtbaren Macht zurückgeworfen, stürzten
Roß und Reiter zusammen. Enger schloß sich der Ring der Schiffe,
die nur wenige Meter über dem Boden schwebten, um die Fürsten und
ihre Begleitung, so daß die gestürzten Offiziere jetzt außerhalb
des gesperrten Kreises lagen.

\tb{}
Die Truppen, soweit sie nahe genug waren, um den Vorgang zu
beobachten, waren sofort unter das Gewehr getreten. Als die
Bataillonsführer bemerkten, daß ihre Kommandierenden nicht zu ihnen
gelangen konnten, als sie sahen, daß die plötzlich erschienenen
Schiffe einen feindlichen Angriff bedeuteten, dem der oberste
Kriegsherr selbst mit allen Fürsten und Generälen ausgeliefert war,
da bebte ihnen wohl das Herz in der Brust unter der Verantwortung,
die sie auf sich gelegt fühlten. Aber nun bewährte sich der Geist
unseres Heeres in erhebender Weise. Nicht ein Augenblick der
Verwirrung, nicht ein Moment des Schreckens trat ein. Die Truppen
einer andern Nation, falls sie sich nicht in zuchtlose Flucht
aufgelöst hätten, wären vielleicht in wahnsinnigem Todesmut zur
Befreiung ihres Feldherrn vorgestürzt, um in den Repulsitstrahlen
und Nihilitsphären der Marsschiffe ihren Untergang zu finden, ohne
daß sie auch das Geringste hätten ausrichten können. Die deutschen
Offiziere indessen verloren ihre Instruktion auch in diesem
schrecklichen Moment nicht aus den Augen. Nach den Erfahrungen, die
man in England gemacht hatte, war es von der deutschen
Heeresleitung als erster Grundsatz ausgesprochen worden, unter
keinen Umständen Munition und Menschenleben gegen ein mit
Nihilitsphäre versehenes Marsschiff zu verschwenden, da man wußte,
daß dies ein völlig fruchtloses Beginnen sei. Die Truppen waren
überhaupt nicht zusammengezogen worden, um sich irgendwo in offenem
Kampf mit den Martiern zu versuchen. Man hatte vielmehr ein ganz
anderes System der Verteidigung aufgestellt, und von diesem auch im
Moment der äußersten Überraschung nicht abzuweichen, das war die
höchste Aufgabe, welche die Disziplin zu leisten hatte.

Man sagte sich, daß den Machtmitteln der Martier gegenüber eine
Armee im freien Feld wie in den Forts der Festungen ohnmächtig und
dem Untergang geweiht war, daß aber ihrerseits die Martier machtlos
sein würden, wenn sie verhindert würden, sich der Organe der
Regierung zu bemächtigen. Man hatte deswegen die Truppen lediglich
zum Schutz der Hauptstädte als der Zentralpunkte der
Staatsverwaltung zusammengezogen. Hier sollten sie verhindern, daß
die öffentlichen Gebäude von den Martiern besetzt und in Beschlag
genommen würden. Man nahm mit Recht an, daß in den Städten, mitten
zwischen den Häusern der friedlichen Bürger, die Martier von
gewaltsamen Zerstörungen absehen würden; daß sie, wenn sie einen
Einfluß auf die Regierung gewinnen wollten, gezwungen sein würden,
ihre schützenden Schiffe zu verlassen und den festen Boden zu
betreten. Und hier sollte dann die starke militärische Besatzung es
unmöglich machen, daß die Kassen, die Büros, die Archive und die
leitenden Amtspersonen selbst in feindliche Gewalt gerieten.
Deswegen hatte jedes einzelne Bataillon bereits seine bestimmte
Instruktion, wohin es sich beim ersten Erscheinen der Feinde sofort
zu begeben habe. Dies allein war auszuführen.

Die große Parade war zum Verderben ausgeschlagen. Aber in
Erinnerung an hergebrachte und liebgewordene Gewohnheiten hatte der
oberste Kriegsherr geglaubt, dieselbe ohne Gefahr anordnen zu
können, weil trotz des sorgfältigsten Nachrichtendienstes noch
keinerlei Spur einer feindlichen Annäherung gefunden worden war.

Nun war der Feind dennoch da. Jeder sah ein, daß man nichts tun
konnte, als der ursprünglichen Anordnung zu folgen. Auf die
feindlichen Luftschiffe schießen oder gegen sie anstürmen wäre
Unsinn gewesen. Das ganze große Feld war noch von Zuschauern
überflutet, die sich jetzt in eiliger Flucht nach der Stadt
zurückwälzten. Auf den Chausseen drängten sich die Wagen, darunter
die Equipagen, welche die fürstlichen Gemahlinnen und Prinzessinnen
vom Paradefeld fortführten. So taten die Truppen ihre einfache
Pflicht. Sie marschierten, so schnell sie konnten, auf den im
voraus festgesetzten Wegen nach ihren Bestimmungsorten. Nur das
erste Gardegrenadierregiment und das Gardekürassierregiment blieben
zur persönlichen Bedeckung des Kriegsherrn zurück.

Der Monarch blickte mit finsterem Ernst auf seine Umgebung, auf die
feindlichen Schiffe und die betäubt oder tot am Boden liegenden
Offiziere, um welche jetzt Ärzte und Krankenträger bemüht waren.
Dann riß er den Degen aus der Scheide und rief:

›Meine Herren! Hier gibt es nur einen Weg hindurch!‹

Er spornte sein Pferd an. Seine Begleitung warf sich ihm entgegen
und beschwor ihn, sich dem sichern Verderben nicht auszusetzen. Er
wollte nicht hören.

›Nun denn‹, rief da der greise General von Dollig, ›zuerst wir!‹

Und einen Teil der Offiziere mit sich fortreißend, jagte er im
Galopp gegen die unsichtbare Schranke, die sich nur durch eine
Staubschicht über dem Boden verriet.

Sobald man bei den außerhalb des Ringes der Marsschiffe haltenden
Schwadronen der Gardekürassiere die Bewegung in der Begleitung des
Feldherrn wahrgenommen hatte, ließen sie sich nicht länger
zurückhalten. Unter brausendem Hurraruf sprengten die glänzenden
Reitermassen heran, um ihren Feldherrn aufzunehmen oder mit ihm
unterzugehen. Es war ein furchtbarer Moment. Starres Entsetzen
faßte alle, die den Vorgang zu bemerken vermochten. Und als ob die
Kühnheit des Entschlusses den übermächtigen Feind bezwänge, so kam
jetzt neue Bewegung in seine Schiffe. Sie erhoben sich, als wollten
sie den Weg freigeben. Gleichzeitig aber senkte es sich von oben
herab wie eine dunkle, langgestreckte Masse, die eben erst auf dem
Feld erschien. Wie ein breites, schwebendes Band, von den
Luftschiffen begleitet, dehnte sich diese Masse jetzt in den kurzen
Sekunden aus, welche die heranstürmende Kavallerie zur Annäherung
brauchte. Und nun kam die erste Reihe der Reiter in den Bereich
ihrer Wirkung, und gleich darauf zog die seltsame Maschine über das
ganze Regiment hinweg.

Die Wirkung war so ungeheuerlich, daß die Schar der ansprengenden
Fürsten und Generale stockte und ein Schrei des Entsetzens vom
weiten Feld her herüberhallte. Kein einziges Pferd mehr stand
aufrecht. Roß und Reiter wälzten sich in einem weiten, wirren
Knäuel, eine Wolke von Lanzen, Säbeln, Karabinern erfüllte die
Luft, flog donnernd gegen die Maschine in der Höhe und blieb dort
haften. Die Maschine glitt eine Strecke weiter und ließ dann ihre
eiserne Ernte herabstürzen, wo die Waffen von den Nihilitströmen
der Luftschiffe vernichtet wurden. Noch zweimal kehrte die Maschine
zurück und mähte gleichsam das Waffenfeld ab. Keine Hand vermochte
Säbel oder Lanze festzuhalten, und wo die Befestigung an Roß und
Reiter nicht nachgab, wurden beide eine Strecke fortgeschleift. Die
Hufeisen wurden in die Höhe gerissen, und dadurch waren sämtliche
Pferde zum Sturz gebracht worden. Jene Maschine war die neue,
gewaltige Erfindung der Martier, eine Entwaffnungsmaschine von
unwiderstehlicher Kraft für jedes eiserne Gerät – ein magnetisches
Feld von kolossaler Stärke und weiter Ausdehnung. Mit Hilfe dieses
in der Luft schwebenden Magneten entrissen die Martier ihren
Gegnern die Waffen, ohne sie in anderer Weise zu beschädigen, als
es durch das Umreißen unvermeidlich war.

Während die Kavallerie aus ihrer Verwirrung sich aufzuraffen
versuchte, war der Luftmagnet schon weitergezogen und hatte sich
der Infanterie genähert. Vergeblich umklammerten die Soldaten mit
beiden Händen ihre Gewehre, eine unwiderstehliche Gewalt zerrte sie
in die Höhe, und mancher, der nicht nachgeben wollte, wurde ein
Stück in die Luft geschleudert, um dann schwer zu Boden zu stürzen.
In wenigen Minuten war das 1. Garderegiment entwaffnet. Die
Maschine flog weiter, um die auf dem Marsch befindlichen Regimenter
einzuholen und dasselbe Manöver an ihnen vorzunehmen. Binnen kurzem
mußte so selbst die stärkste Armee kampfunfähig gemacht sein. Auch
die Geschütze der Artillerie wurden fortgerissen.

Während der Monarch und seine Begleitung in tiefer Erschütterung
auf das Unfaßliche starrten, senkte sich aus der Höhe dicht vor
ihnen ein schlankes Schiff hernieder, das ein leuchtender Stern als
das Admiralsschiff bezeichnete. Demselben entstieg, während die
übrigen die Absperrung aufrecht erhielten, der Befehlshaber der
Martier. Zwei Adjutanten begleiteten ihn. Über ihren Köpfen
glänzten die diabarischen Helme. So traten sie langsam einige
Schritte vor, die großen Augen scharf auf die Offiziere gerichtet.
Unwillkürlich wichen alle zur Seite, eine Gasse öffnete sich, und
der Nume stand dem Monarchen gegenüber.

Der Martier grüßte mit einer ehrfurchtsvollen Handbewegung und
sagte:

›Mein Auftraggeber, der Protektor der Erde, lädt Ew. Majestät und
Ihre hohen Verbündeten zu einer Besprechung ein und bittet, zu
diesem Zwecke dieses Schiff allergnädigst besteigen zu wollen. Ich
bemerke, daß es unmöglich ist, diesen von unserer Repulsitzone
umgebenen Platz auf andere Weise zu verlassen.‹

Niemand wagte sich zu bewegen. Lange blickte der Fürst mit strenger
Miene in das Auge des Numen, der den Blick ruhig erwiderte; keiner
zuckte mit einer Wimper. Dann steckte der Monarch mit einer
entschlossenen Bewegung den Degen in die Scheide und sprach
nachdrücklich:

›Sie haben einen General gefangengenommen, nichts weiter. Seine
Majestät, mein Herr Sohn, befindet sich nicht unter uns.‹

›Ew. Majestät werden ihn im Schiffe finden‹, sagte der Nume mit
einer Verbeugung.

Der Feldherr schwang sich vom Pferd. Hoch aufgerichtet, die Hand
auf dem Griff des Degens, stieg er die herabgelassene Schiffstreppe
hinan.

Das Luftschiff, das bereits vor einer Stunde die Kommandierenden
der Armeekorps in Königsberg, Breslau und Posen aufgehoben hatte,
entfernte sich nach Westen – –“

Torm ließ die Blätter aus seiner Hand sinken.

Das also war das Unglück vom dreißigsten Mai!

Er nahm die Broschüre wieder auf, er blätterte weiter, er blickte
auch in die übrigen Hefte.

An demselben Tag waren die Festungswerke von Spandau durch die
Martier zerstört, die Kriegsvorräte unbrauchbar gemacht worden. Man
hatte die Fürsten nach Berlin geführt, die ganze Stadt wurde jetzt
zerniert. Wo sich Truppen im Freien zeigten, erschienen alsbald die
Elektromagnete der Martier und entrissen ihnen die Waffen. Nach
drei Tagen waren alle größeren Waffenplätze außer Funktion
gesetzt.

Jetzt liefen die Nachrichten aus Wien, Paris, Rom ein. Die Martier
waren überall in ähnlicher Weise vorgegangen. Zuerst hatten sie
sich der Personen der Fürsten, Präsidenten und Minister bemächtigt.
Man hatte den Kaiser von Österreich auf der Jagd, den König von
Indien während eines großen Empfanges aufgehoben, der Präsident der
französischen Republik spielte gerade mit dem Kriegsminister eine
Partie Billard, als er in das Luftschiff der Martier eingeladen
wurde. Die Kammer wurde im Palais Bourbon eingeschlossen, bis der
Friedensvertrag unterzeichnet war. Die gefangenen Fürsten dankten
zugunsten ihrer Thronerben ab, und die jungen Nachfolger konnten
zuletzt nichts anderes tun, als in die Friedensbedingungen der
Martier willigen, da ihre Armeen machtlos waren und ein längerer
Widerstand nur zu einer Auflösung der staatlichen Ordnung geführt
hätte.

Rußland allein war vorläufig von einem Angriff der Martier
verschont geblieben. Die Gründe dafür wußte man nicht, doch nahm
man allgemein an, daß die Martier nur eine günstige Gelegenheit
abwarteten, bis ihnen die Zustände in den westlichen Staaten mehr
freie Hand ließen. Das Protektorat über die Erde blieb erklärt, war
aber zunächst nur für die westlichen Staaten Europas durchgeführt.
Hier wartete in jeder Hauptstadt ein Resident der Marsstaaten und
ein Kultor seines Amtes. Zwar war die Freiheit der Verwaltung im
Innern garantiert, doch tatsächlich war auch in bezug auf
Gesetzgebung und Regierungsmaßregeln der Wille der Marsstaaten im
letzten Grunde ausschlaggebend. Die allgemeine Entwaffnung bis auf
eine Präsenzstärke von ein halb Promille der Bevölkerung war eine
der Friedensbedingungen gewesen. Trotz allen Sträubens mußten die
Fürsten sie annehmen, da es tatsächlich unmöglich gewesen wäre, den
technischen Machtmitteln der Martier gegenüber ohne ihren Willen
eine Truppe auszubilden.

Eine Reihe von Vorteilen in volkswirtschaftlicher Beziehung wurde
nun angebahnt. Produkte des Mars wurden eingeführt, neue
Betriebsformen von Fabriken, vor allem die Herstellung künstlicher
Nahrungsmittel. Die Landwirte wurden vorläufig damit beruhigt, daß
ihnen aus den Fonds der Marsstaaten große unverzinsliche Darlehen
gegeben wurden, um die Kosten der Umwandlung des Fruchtbetriebs in
Maschinenbetrieb durch Sonnenstrahlung zu bestreiten. Ingenieure
der Martier leiteten die Einrichtung der Strahlungsfelder, zu denen
vorläufig nur unfruchtbarer Boden benutzt wurde. Alles dies aber
waren bloß vorbereitende Schritte, die eigentlich mehr erziehen als
wirtschaftlich nützen sollten. Die Ausbeutung der Sonnenenergie
suchten die Martier auf den großen Wüsten und Steppen Asiens,
Afrikas und Nordamerikas. Sie hatten deshalb mit Rußland und den
Vereinigten Staaten neue Verhandlungen angeknüpft.

Inzwischen erstrebten sie in Europa rein ideale Ziele.
Kriegskostenentschädigung verlangte man nicht, die großen Summen,
die für das Militär erspart wurden, kamen den Fortbildungsschulen
zugute. Die Martier wollten die Menschheit für ihre höhere
Auffassung der Kultur und Sittlichkeit erziehen, und dem sollte die
Einsetzung der Kultoren, die Einrichtung obligatorischer
Fortbildungsschulen dienen.

Torm war zu abgespannt, um weiterzulesen. Er legte die Papiere
beiseite. Ein einzelnes Blatt schob sich vor. Er sah alsbald, daß
es ein Flugblatt sei, zu irgendeinem bestimmten Zweck verbreitet,
und sein Blick richtete sich nur noch einmal darauf, weil er mit
fetten Lettern die Worte gedruckt sah: „Glaubt nicht an ihren
Edelmut“, „Die Mörder von Podgoritza“, „Aber auch sie sind
sterblich“. Er las das Blatt jetzt durch, einmal, zweimal. Es
handelte von der sogenannten ›Bestrafung‹ von Podgoritza in
Montenegro. Diese Stadt war tatsächlich von den Martiern dem
Erdboden gleichgemacht worden. Allerdings hatte man den Einwohnern
Zeit gelassen, sie zu räumen, aber nicht alle hatten gehorcht; da
waren die Nume zum erstenmal auf der Erde schonungslos vorgegangen
und hatten ohne Rücksicht auf Menschenleben ihre Drohung
ausgeführt. Es waren wohl einige hundert Personen dabei umgekommen,
wütende Männer, die sich den Luftschiffen entgegengeworfen hatten.
Aber warum war dieses ungewöhnliche Strafgericht ergangen? Es war
kurz nach der Unterwerfung der westeuropäischen Staaten gewesen,
als ein großes Luftschiff der Martier, das von einer
wissenschaftlichen Expedition zurückkam und zum Zweck einer kleinen
Ausbesserung in der Nähe von Podgoritza anlegte, in der Nacht
unvermutet von bewaffneten Bewohnern der Stadt und Umgegend
überfallen worden war. Die Martier waren überrascht und bis auf den
letzten Mann, zum großen Teil im Schlaf, niedergemacht worden. Es
war der einzige Verlust, den die Nume bisher auf der Erde erlitten
hatten und die Empörung in den Marsstaaten war ungeheuer. Man war
nahe daran, die ganze Menschheit für die Bluttat unzivilisierter
Albaner verantwortlich zu machen. Etwas Derartiges war den Martiern
bisher undenkbar gewesen; und so wurde bestimmt, daß die Strafe
ausnahmsweise nach menschlicher Art, das heißt durch Vernichtung
des Gegners, vollzogen werde, weil man glaubte, sonst bei der
barbarischen Bevölkerung keinen Eindruck zu erzielen. Diese
Handlungsweise der Martier wurde nun in Europa ausgebeutet, um sie
in üblem Licht darzustellen.

Aber warum machte die Tat auf Torm einen so tiefen Eindruck? Immer
und immer wieder beschäftigte ihn die Frage, welches Schiff es wohl
gewesen sei, von dem kein Lebender zu den Martiern zurückkehrte.
Und eine Vermutung stieg in ihm auf, an die er kaum zu glauben
wagte.

\section{46 - Der Kultor der Deutschen}

„Unmöglich, Herr Kultor, unmöglich“, sagte der Justizminister
Kreuther, indem er seine hohe Stirn mit dem Taschentuch tupfte. „In
dieser Form, welche der Reichstag dem Gesetzentwurf zum Schutz der
individuellen Freiheit gegeben hat, ist er für uns unannehmbar. Sie
müssen das selbst zugeben. Es würden die Paragraphen 95 bis 101 des
Strafgesetzbuches hinfällig werden.“

„Und was schadet dies?“ fragte der Kultor kühl. Er lehnte sich
bequem in seinen Stuhl zurück und ließ seine großen Augen ruhig von
einem der beiden ihm gegenübersitzenden Herrn zum andern wandern.

Der Justizminister blickte ihn fassungslos an. Sein Begleiter, der
Minister des Innern, von Huhnschlott, richtete sich gerade auf und
zerrte an seinem grauen Backenbart.

„Was das schadet?“ sagte er mit mühsam zurückgehaltener Empörung.
„Das heißt, die Majestät schutzlos machen, das heißt, jeder
pöbelhaften Gemeinheit einen Freibrief ausstellen, das heißt, unsre
heiligsten, angestammten Gefühle angreifen und die Autorität
untergraben.“

„Sie irren, Exzellenz“, antwortete der Kultor mit einem überlegenen
Lächeln. „Es heißt nur, die Wahrheit festlegen, daß die Majestät
ebensowenig durch Äußerungen anderer beleidigt werden kann, wie die
Vernunft überhaupt, daß die sittliche Persönlichkeit dadurch nicht
berührt wird. Die Verleumdung bleibt strafbar wie jede Schädigung,
und die Autorität ist genügend geschätzt durch die
Unverletzlichkeit der Person der Fürsten. Wir können es aber als
keine Schädigung der Person erachten, wenn jemand ohne seine Schuld
lediglich beschimpft wird. Das ist eben die Grundanschauung, die
wir durchführen wollen, daß es keine solche Beleidigung gibt, daß
die Injurie nicht denjenigen verächtlich macht und in der
öffentlichen Meinung herabsetzt, den sie treffen soll, sondern
denjenigen, der sie ausspricht. Wir erstreben mit diesem Gesetz,
einen Teil unsres allgemeinen Erziehungsplanes durchzuführen. Die
Menschen sollen lernen, ihre Ehre allein zu finden in dem
Bewußtsein ihres reinen sittlichen Willens, und sie sollen
verachten lernen den äußern Schein, der dem Schlechten ebenso
zugute kommt wie dem Ehrenmann. Wir wollen die Erziehung zur innern
Wahrheit, indem wir den Schutz des Gesetzes dem entziehen, was dazu
verleitet, die Ehre im Urteil oder Vorurteil der Menge zu sehen.
Alle unsre Maßregeln, die volkswirtschaftlichen wie die ethischen,
haben nur das eine Ziel, den Menschen das höchste aller Güter zu
verschaffen, die innere Freiheit.“

Der Justizminister schüttelte den Kopf. „Das ist ein kindlicher
Idealismus“, dachte er, aber er wußte nicht gleich, wie er dies,
was er für beleidigend hielt, höflich ausdrücken solle.

„Herr Kultor“, sagte von Huhnschlott, „das bedeutet eine gänzlich
von der unsrigen abweichende Weltanschauung, das kann nur die
Umsturzideen fördern. Wir bitten Sie inständig –“

„Das ist keine neue Weltanschauung“, unterbrach ihn der Kultor
streng, „es ist nur der Kern der Religion, zu deren äußeren Formen
Sie sich so eifrig bekennen. Es ist die innere Freiheit im Sinne
des Christentums, nur daß sein Begründer, im Zusammensturz der
antiken Welt, machtlos im römischen Weltreich, sie allein finden
konnte in der Verachtung und Flucht der Welt und daß seine
angeblichen Nachfolger sie bloß verstanden als den Verzicht des
Armseligen zugunsten des Mächtigen. Wir aber sind die Herren der
Natur und der Welt und wollen nun der Pflicht nicht vergessen, für
jeden diese innere Freiheit zu ermöglichen, ohne daß er auf die
Güter dieses Lebens zu verzichten braucht. Und darum, meine Herren,
ist es ganz vergeblich, daß Sie sich weiter bemühen. Sie werden dem
Gesetz die Zustimmung der Regierung geben.“

Der Kultor erhob sich.

Die Minister standen sogleich auf und sahen sich verlegen an.

„Verzeihen Sie, Herr Kultor“, begann der Justizminister nach einer
Pause, „wir haben diese Unterredung als eine private nachgesucht;
ich sehe, daß sie leider erfolglos war. Was werden Sie tun, wenn
das Gesamtministerium Ihnen eine offizielle Vorstellung macht?“

„Ich werde auf der Sanktionierung des Gesetzes bestehen.“

„Und wenn der Bundesrat dennoch ablehnt?“

„Er wird es nicht.“

„Ich würde eher meine Demission einreichen, als die Annahme
empfehlen“, sagte Kreuther mit Haltung.

„Das Ministerium ist darin einig“, fügte Huhnschlott hinzu.

„Das täte mir leid, meine Herren, aber es würden sich andere
Minister finden.“

„Und wenn nicht?“ rief Huhnschlott auffahrend.

„Dann wird Ihnen der Herr Resident die Antwort erteilen. Bemühen
Sie sich nur zu ihm, ich weiß, was er Ihnen antworten wird. Hätten
Sie sich der Protektoratserklärung vom 12. Mai vorigen Jahres
unterworfen, so wäre eine Einmischung in innere Angelegenheiten
ausgeschlossen. So aber wird er sie auf Artikel 7 des nordpolaren
Friedensvertrages vom 21. Juni verweisen. Die Garantien für den
Rechtsbestand der Verfassung sind aufgehoben, wenn sich die
Regierung weigert, diejenigen Maßregeln zu unterstützen, welche die
Marsstaaten für notwendig zur wirtschaftlichen, intellektuellen
oder ethischen Erziehung der Menschheit hält.“

„Die Entscheidung des Kultors und des Residenten ist noch nicht
maßgebend“, erwiderte Huhnschlott finster. „Es steht uns der Appell
an den Protektor der Erde offen.“

„Appellieren Sie“, sagte der Kultor.

Die Minister verbeugten sich förmlich und verließen das Zimmer.
Langsam stiegen sie die breite Treppe hinab. In der Vorhalle
standen zwei riesenhafte Beds unter ihren diabarischen
Glockenhelmen Posten und senkten salutierend ihre Telelytrevolver.
Die Minister grüßten mechanisch und stiegen in den vor der Tür
haltenden elektrischen Wagen. Er rollte aus der bedeckten Auffahrt
auf den regennassen Asphalt der breiten Straße. Huhnschlott warf
einen Blick rückwärts auf das flache Dach des Gebäudes, wo die
glatten Rücken dreier Marsschiffe durch den grauen Schleier des
herabrieselnden Regens glänzten und ihre Repulsitrohre nach drei
Richtungen drohend der Hauptstadt entgegenstreckten. Kreuther war
dem Blick gefolgt und seufzte tief.

„Zum Kanzlerpalais“, rief Huhnschlott dem Wagenführer zu und
murmelte einen Fluch zwischen den Zähnen.

Der Kultor war an eines der hohen Fenster des Gemaches getreten und
blickte hinüber auf den Verkehr der Straße. Seine Stirn zog sich
finster zusammen. Das tut’s freilich nicht, so gingen seine
Gedanken, aber die Gängelbänder müssen fort, wenn die Kinder allein
und aufrecht zu gehen lernen sollen. Und diese Huhnschlotts sind
die gefährlichsten Feinde der Selbstzucht; doch ihre Macht ist
gebrochen. Sie werden nicht wagen, sich zu widersetzen.

In seinen Augen leuchtete es triumphierend auf. Es muß gelingen! Er
wandte sich nach seinem Arbeitszimmer.

„Die Berichte der Herren Instruktoren!“ sprach er ins Telephon.

Der Aufzug beförderte ein dickes Aktenbündel herauf. Er begann
darin zu blättern und sich Notizen zu machen. Sein Auge
verfinsterte sich wieder. Die Bestrafungen wegen Versäumnis der
Fortbildungsschulen vermehrten sich von Monat zu Monat. Auf dem
Land hatte man jetzt während der Erntezeit die Einrichtung
überhaupt pausieren lassen müssen. Und wie oft waren die Lehrpläne
falsch aufgestellt! Nicht wenige Instruktoren ließen Dinge lehren,
zu denen die Vorkenntnisse fehlten. Aber es fanden sich doch auch
erfreuliche Erfolge. In manchen Landesteilen, besonders bei der
industriellen Bevölkerung, drängte man sich nach den
Bildungsstätten. Merkwürdigerweise zeigte sich auch in
Süddeutschland, sogar in Tirol, ein Fortschritt in der Popularität
der Schulen. Hier stand den Bestrebungen der Nume die feste
Organisation der kirchlichen Macht feindlich gegenüber; es schien
zuerst, als würde es unmöglich sein, gegen den Fanatismus der von
der Geistlichkeit gelenkten Bevölkerung aufzukommen. Aber gerade in
diesen Gegenden wurde der Besuch, trotz der lokalen Schwierigkeiten
in den Gebirgen, immer lebhafter, es gründeten sich selbständige
Vereine, zahlreich wurden Lehrer verlangt.

Der Kultor sann lange über die Ursache dieser Erscheinung nach. War
es die natürliche Opposition gegen die Macht, die bisher das
Nachdenken geflissentlich vom Volk ferngehalten hatte? Brauchte man
dem menschlichen Geist nur die Freiheit und die Gelegenheit des
Denkens zu geben, um sicher zu sein, daß er seinen Aufflug gewinnen
werde? Oder waren die Instruktoren hier geschickter? Der Kultor las
einige der Einzelberichte, und er sah mit Vergnügen, wie gut es die
Sendboten der Numenheit verstanden hatten, sich vollständig nach
den kirchlichen Gewohnheiten der Bevölkerung zu richten. Nirgends
suchten sie Zweifel zu erwecken, nirgends gegen die traditionelle
Form zu verstoßen. Sie beschränkten sich zunächst auf rein
praktische Kenntnisse, deren Wirkung sich sofort in der Hebung des
wirtschaftlichen Lebens zeigte. So gewannen sie Vertrauen. Der Weg
ist lang, dachte der Kultor, aber er ist der einzig mögliche.

Der Kultor blickte auf seine Notizen und sprach eifrig in den vom
Mars eingeführten Phonographen, der ihm zum Festhalten seiner
Gedanken diente. Er entwarf eine Erläuterung zur Instruktion der
einzelnen Bezirkskultoren. Die süddeutschen Erfolge sollten zum
Vorbild genommen werden. Als er einiges aus der Statistik anführen
wollte, stutzte er bei einer Zahl, die von den übrigen auffallend
abwich. Wie kam es, daß in dem Bezirk von Bozen die Resultate so
ungünstig waren? Er suchte in den Akten den Bericht des
Instruktors. Es war die erste Arbeit eines neu hingekommenen
Beamten. Die Instruktoren mußten sehr häufig wechseln, das war ein
großer Übelstand; sie vertrugen das Erdklima nicht.

Eben begann der Kultor den Bericht zu lesen, als ihm gemeldet
wurde, daß der Vorsteher des Gesundheitsamtes von seiner
Inspektionsreise zurückgekehrt sei und anfrage, ob er ihn sprechen
könne.

„Ich bitte, sogleich“, war die Antwort.

Die Tür öffnete sich, und ein älterer Herr trat ein. Trotz der
diabarischen Glocke, die über seinem Haupt schwebte, ging er
gebückt und mühsam.

Der Kultor sprang auf und eilte ihm entgegen.

„Mein lieber, teurer Freund“, sagte er, seine Hände fassend, „was
ist Ihnen? Sie sehen angegriffen aus, sind Sie nicht wohl? Machen
Sie es sich bequem. Legen Sie den Helm ab, und setzen Sie sich hier
auf das Sofa unter dem Baldachin, dieses Eckchen ist auf
Marsschwere eingerichtet. Ihre Reise hat Sie gewiß sehr
angestrengt?“

„Es muß meine letzte sein. Sobald ich meinen offiziellen Bericht
abgegeben habe, spätestens in zwei bis drei Wochen, komme ich um
Urlaub ein. Ich hoffe, Sie werden mir keine Schwierigkeiten
machen.“

„Sie erschrecken mich, Hil! Selbstverständlich können Sie reisen,
sobald Sie wollen, sollen Sie reisen, wenn es Ihre Gesundheit
erfordert. Aber mir tut es von Herzen leid. Und wie sollen Sie
ersetzt werden? In diesem unausgesetzten Wechsel der Beamten – wir
haben nun schon den vierten Residenten – waren Sie mir die festeste
Stütze. Indessen, ich hoffe, es handelt sich nur um eine
vorübergehende Indisposition. Das feuchte Wetter –“

„Ja das Wetter! Sehen Sie, Ell – ich spreche im Vertrauen –, an dem
Wetter wird unsre Kunst zuschanden. Der Winter läßt sich allenfalls
ertragen, aber gegen diese feuchte Wärme kommen wir nicht auf. Oft
habe ich geglaubt, wenn unsre Beamten schon nach wenigen Wochen um
Urlaub einkamen, es liege an ihrer Willensschwäche. Ich habe jetzt
auf meiner Reise durch die Tiefebene und durch die feuchten
Waldtäler der Gebirge gesehen, daß dieses Klima für den Numen, der
sich wenigstens einen Teil des Tages im Freien aufhalten muß, wie
es doch auf Reisen unvermeidlich ist, in gefährlichster Weise
wirkt. Der Regen, der Regen! Wer diese Himmelsplage erfunden hat!
Bald prickelt es von allen Seiten in mikroskopischen
Wassertröpfchen, bald braust es in Sturzgüssen hernieder, bald
fällt es mit jener eintönigen, hypnotisierenden, tödlichen
Langeweile herab wie heute. Die Luft, mit Dampf gesättigt, lähmt
die Tätigkeit der Haut und läßt uns ersticken. Ich war manchmal wie
verzweifelt. Wir dürfen niemand länger als ein halbes Jahr im
Winter oder ein Vierteljahr im Sommer hier lassen, oder wir bringen
Lungen und Herz nicht wieder gesund nach dem Nu. Was nutzen uns die
trefflichen antibarischen Apparate, wenn das infame Wasser uns im
wahren Sinne des Wortes ersäuft? Da oben am Pol war das nicht so
merklich, wir lebten ja auch mehr nach unsrer eignen Weise. Aber
hier in Deutschland –. Warum mußten Sie sich auch gerade dieses
Volk zu Ihrem Experiment ausersehen? Es gibt doch Gegenden, in
denen wir einigermaßen besser fortkommen würden, zum Beispiel die
großen Steppen im Osten, und überall, wo es trocken ist –“

„Aber mein verehrter Hil! Unsre Kulturbestrebungen können wir doch
nur dort betreiben, wo wir die Bevölkerung am besten vorbereitet
finden, also wo die Volksbildung die vorgeschrittenste ist.
Deswegen mußte ich Deutschland wählen, und vornehmlich darum, weil
ich es am besten kenne. Höchstens an England hätte ich, aus andern
Gründen, denken können, aber dort ist es noch viel feuchter. Und
aus allen andern Staaten klagen die Kultoren und Residenten ebenso.
Hier liegt ein ganzer Stoß von Urlaubs- und Entlassungsgesuchen von
Leuten, die noch keine drei Monate im Lande sind. Doch Sie setzten
ja so viele Hoffnung auf das Anthygrin. Hat sich denn dieses
Heilmittel nicht bewährt?“

„Das Anthygrin ist in der Tat ein ausgezeichnetes Spezifikum gegen
das Erdfieber, und mit dem Chinin zusammen hält es uns einige Zeit
aufrecht. Aber es wird nicht lange vertragen, andere Organe werden
ruiniert. Ich habe es jetzt sehr stark anwenden müssen, und nun bin
ich hauptsächlich deswegen so schwach, weil ich nichts mehr essen
kann.“

„Sie sollten sich an Menschenkost gewöhnen. Man muß sich eben nach
dem Land richten. Im übrigen müssen wir uns damit abfinden, daß
unsre Beamten schnell wechseln. Wir wollen versuchen, ihnen öfter
einen kürzeren Urlaub in günstigere klimatische Verhältnisse, etwa
nach Tibet, zu geben. Dort hat sich ja jetzt eine vollständige
Marskolonie entwickelt. Und wissen Sie, Sie brauchen Ihren Bericht
nicht hier abzufassen, Sie können das tun, wo Sie sich wohler
fühlen, vielleicht in den Alpen, oder auch weiter fort. Ich stelle
Ihnen ein Regierungsschiff zur Verfügung.“

„Ja, wenn wir in der Lage wären, jedem ein Luftschiff mitzugeben –
das wäre freilich das beste Mittel. Zehntausend Meter in die Höhe,
das kuriert besser als Anthygrin und alle Mittel.“

„Das können wir freilich uns vorläufig nicht leisten, aber in
einigen Jahren, wenn wir die Energiestrahlung auf der Erde besser
ausnutzen können, wird es hoffentlich möglich sein. Etwas ließe
sich inzwischen schon tun. Man könnte einige Schiffe zu einer
Höhen-Luftstation einrichten und so doch abwechselnd den einzelnen
Erleichterung schaffen.“

„Tun Sie darin bald, was Sie tun können.“

„Ich kann jetzt nicht größere Geldmittel verlangen. Der Etat für
dieses Jahr ist erschöpft. Wir haben kolossale Anlagekosten
gehabt.“

„Ganz gleich, mögen es die Menschen bezahlen.“

Ell sah den Arzt erstaunt an.

„Nun ja“, lenkte Hil ein, „es klingt etwas roh. Schließlich wird es
doch darauf hinauskommen. Doch entschuldigen Sie meine – meine
Ausdrucksweise. Ich fühle selbst, daß ich jetzt so leicht heftig,
nervös gereizt bin. Man lernt ja die Menschen nicht gerade sehr
hoch schätzen – übrigens ist das die allgemeine Ansicht bei unsern
Beamten, daß es ganz gut wäre, lieber Steuern zu erheben als
Entschädigungsgelder zu zahlen.“

„Ich verstehe Sie gar nicht mehr, lieber Freund. Das wäre die
Ansicht bei unsern Beamten? Dagegen würde ich mich doch recht
ernstlich erklären.“

„Da es mir einmal so – wie man hier redet – herausgefahren ist, so
mag es denn auch gesagt sein“, erwiderte Hil, „obwohl ich erst in
meinem Bericht davon sprechen wollte, weil ich ihnen erst darin die
formellen Belege für meine Beobachtungen geben kann. Es ist
allerdings eine Gefahr da, eine moralische, die Ihnen in der
Auswahl der Beamten ganz besondere Vorsicht auferlegen wird. Es ist
mir im allgemeinen aufgefallen, daß die Instruktoren nach einigen
Monaten nicht mehr die Ruhe und das heitere Gleichmaß der Gesinnung
haben, die wir an den Numen gewohnt sind. Der Umgang mit den
Menschen, wenigstens in der autokratischen Stellung, die sie
einnehmen, wirkt – verzeihen Sie den Ausdruck – gewissermaßen
verrohend, und das äußert sich zunächst in der Sprechweise, in
einer Geringschätzung der ästhetischen Form, weiterhin in einer
Überschätzung der eigenen Bedeutung, schließlich in einer schon das
ethisch Statthafte überschreitenden Selbstherrlichkeit. Ja, ich
habe leider einzelne Fälle beobachtet, wo man direkt von einer
Psychose sprechen kann, ich möchte sie geradezu den ›Erdkoller‹
nennen.“

„Aber ich bitte Sie, da muß sofort eingeschritten werden. Darüber
werden Sie mir eingehend berichten.“

„Als Arzt, gewiß. Das andere wird Sache der revidierenden
Unterkultoren sein, wenn nicht gar des Residenten. Denn es können
politische Verwicklungen entstehen. Bis jetzt ist die Sache noch
verhältnismäßig harmlos, und ich werde die betreffenden Herren
schon morgen zur Beurlaubung vorschlagen. Da komme ich zum Beispiel
– ich weiß den Namen nicht auswendig – auf eine Kreuzungsstation,
wo ich umsteigen muß. Aber der neue Zug kommt nicht und kommt nicht
– er hat über eine halbe Stunde Verspätung. Ich erkundige mich dann
bei dem Zugführer und höre: Ja, der Herr Bezirksinstruktor ist ein
Stück mitgefahren. Ich frage, warum das so lange aufgehalten habe.
Der Herr Bezirksinstruktor habe einen eigenen Wagen verlangt, der
mußte erst geholt werden. Dann könne er aber den Lärm und Dampf der
Maschine nicht vertragen, und so mußte man den Wagen erst an das
Ende des Zuges bringen und noch einige leere Wagen
dazwischenschalten. Und endlich mußte man mitten auf der Strecke an
einem Dorf halten, weil es ihm beliebte, dort auszusteigen.“

„Und sagten Sie nicht, daß die Bahnbeamten solchem unberechtigten
Verlangen nicht nachgeben durften?“

„Die zuckten die Achseln und meinten, was will man tun? Man darf
sich den nicht zum Feind machen.“

„Die feigen Toren! Aber der Instruktor muß sofort von seinem Amt
suspendiert und vor das Disziplinargericht gestellt werden. Das ist
ja unerhört, wenn sich diese Angaben bestätigen, ich werde aufs
genaueste untersuchen lassen. Wie kann ein Nume seine
Berechtigungen so überschreiten!“

„Es würde ihm auf dem Nu nie einfallen. Hier achtet er niemand als
seinesgleichen. Die Theorie, daß Bate keine Numenheit besäßen, ist
ja sehr verbreitet.“

„Ich werde dafür sorgen, daß sich meine Beamten ihrer Pflicht
erinnern, die Gesetze dieses Staates als die ihrigen zu betrachten,
so lange sie hier sind, und sich keine privaten Vorrechte
anzumaßen. Wie sollen die Menschen lernen, sich dem Gesetz zu
fügen, wenn Nume solche Beispiele geben? Ich hätte das nicht
geglaubt. Warum aber beschwert sich niemand? Sobald die Presse über
einen derartigen Fall berichtete, würde ich sofort untersuchen
lassen.“

Hil zuckte mit den Achseln. „Die Untersuchung ist nicht immer sehr
angenehm. Es ist schwer, alle Einzelheiten zu beweisen. Übrigens
sind solche Fälle glücklicherweise noch vereinzelt. Sollten sie
sich wiederholen, so würde die Presse nicht schweigen. Das sehen
Sie ja an dem Fall Stuh.“

„Was meinen Sie da?“

„Haben Sie denn die heutigen Mittagsblätter nicht gelesen?“

„Es war mir bis jetzt unmöglich. Aber ich würde natürlich nachher
–. Doch was ist denn geschehen? Sie meinen doch nicht Stuh in
Frankfurt?“

„Die Sache spielt in der Nähe von Frankfurt. Der Bezirksinstruktor
ist vier Stunden im Regen gefahren – beachten Sie das –, kommt in
einen kleinen Ort und ist sehr hungrig. Er läßt vor dem Wirtshaus
halten. Es ist Sonntag, alle Zimmer sind überfüllt, der Wirt hat
selbst Taufe im Hause. Stuh geht in das Gastzimmer und bestellt
sich Essen. Die Bauern rücken auch zusammen und machen ihm eine
Ecke frei. Nun kommt das Essen. Stuh sagt dem Wirt, die Leute
möchten jetzt das Zimmer verlassen, er wolle essen. Der Wirt stellt
ihm vor, das könne er nicht verlangen, es sei kein Raum im ganzen
Hause frei; selbst der Hausflur war besetzt, und draußen regnete es
in Strömen. Da wird Stuh von Hunger und Regen wütend und herrscht
die Leute an, sie möchten sich hinausscheren, wenn ein Nume esse,
habe kein Bat zuzusehen. Die Bauern haben keine Ahnung, daß es uns
nicht möglich ist, so öffentlich den Hunger zu stillen. Sie halten
die Anforderung für eine Unverschämtheit und lachen Stuh einfach
aus. Ganz nüchtern waren sie auch nicht mehr. Kurzum, es kommt zum
Streit. Stuh will nun hinaus, jetzt aber verhöhnen ihn die Bauern
und klopfen ihm mit ihren Stöcken auf den Glockenhelm.
Unglücklicherweise hat Stuh an seiner Uhr ein kleines
Telelytstiftchen. Er nimmt die Uhr heraus, hält sie den Umstehenden
entgegen und sagt: ›Wenn ihr jetzt nicht macht, daß ihr
hinauskommt, so lasse ich hier einen Feuerregen heraus, daß ihr
alle verbrennen müßt.‹ Das war ja natürlich eine Aufschneiderei,
mit dem Stiftchen konnte er höchstens einem die Kleider versengen.
Und da nun nicht gleich Platz wird, so läßt er die Funkengarbe aus
dem Stiftchen sprühen. Nun denken die Leute wirklich, das Haus muß
anbrennen, und drängen sich zur Tür. Es entsteht ein Gewühl, und
eine Menge Verwundungen kommen vor. Das ganze Haus gerät in
Aufruhr. Stuh verriegelt die Tür und setzt sich ruhig zum Essen.
Als nun die Bauern sahen, daß weiter nichts geschehen war und sie
sich nur selbst gestoßen und getreten hatten, wurden sie wütend und
wollten die Tür einschlagen, um Stuh zu verhauen. Zum Glück war
inzwischen Polizei herbeigekommen und brachte Stuh unversehrt zum
Ort hinaus. Aber Sie können sich denken, welche Empörung jetzt in
dem Städtchen herrscht.“

„Das ist unangenehm, sehr unangenehm“, sagte Ell. „Und ich kenne
doch Stuh als einen ruhigen, menschenfreundlichen Mann.“

„Der Regen, Ell! Fahren Sie einmal vier Stunden im Regen – mit
Pferden, entsetzlicher Gedanke! Schon der Geruch kann einen
wahnsinnig machen. Aber freilich, das können Sie nicht so
nachfühlen –“

Ell war aufgestanden und auf und ab gegangen. Er blieb nun stehen
und sprach: „Aber das sind Zwischenfälle, die sich nicht vermeiden
lassen. Man muß sie korrigieren, ihnen jedoch kein großes Gewicht
beilegen. Unsere Aufgabe werden wir trotzdem erfüllen.“

„Ich zweifle nicht. Aber es sind Symptome. Möchten sie sich nicht
häufen! Indessen, sie sind nicht das Schlimmste. Es gibt eine viel
größere Gefahr. Deswegen kam ich her. Eine Gefahr für die
Menschen.“

„Sprechen Sie, Hil.“

„Wissen Sie, was bei uns Gragra ist?“

„Das ist, wenn ich mich recht erinnere, eine Kinderkrankheit auf
dem Mars, die ohne jede Bedeutung ist.“

„Ganz richtig, das ist sie jetzt, seit einigen tausend Jahren. Die
Kinder sind ein paar Tage müde, bekommen einen leichten Ausschlag,
und dann ist die Sache vorüber. Aber es war nicht immer so. Im
agrarischen Zeitalter war die Gragra eine furchtbare Plage, eine
entsetzliche Pest, welche ganze Landstriche bei uns entvölkerte,
nicht durch einen akuten Verlauf, sondern durch eine langsame,
chronische Vergiftung. Wir sind ihrer Herr geworden, teils durch
unsre Impfungen, teils durch die allmähliche Veränderung der
Ernährung. Und nun – die ersten Spuren dieser chronischen Form –,
doch setzen Sie sich her zu mir, ich muß leise sprechen.“

Ell ließ sich neben Hil nieder. Dieser sprach lange mit ihm. Ells
Gesicht war tiefernst geworden.

„Das ist ja furchtbar“, sagte er. „Und was können wir tun?“

„Noch weiß kein Mensch von der drohenden Gefahr. Die menschlichen
Ärzte sind noch nicht einmal auf diese leichten, ihnen unbekannten
ersten Symptome aufmerksam geworden. Und wenn die Krankheit
allmählich stärker unter den Menschen auftreten sollte, werden
Jahre vergehen, ehe sie erkennen werden, daß es sich um eine für
sie ganz neue Form von Bakterien handelt. Denn diese sind so klein,
daß sie nur durch unsere besonderen Strahlungsmethoden nachweisbar
sind. Ich habe die Überzeugung, daß die Krankheit in ihrer milden
Form vom Mars eingeschleppt worden ist und daß die Bazillen unter
den auf der Erde, respektive im menschlichen Körper herrschenden
Verhältnissen so günstige Bedingungen für ihre Vermehrung gefunden
haben, daß die alte perniziöse Form, die bei uns ausgestorben war,
wieder auftritt. In einigen Jahren werden wir die Verheerungen
sehen.“

„So müssen wir sofort die Ärzte auf diese Krankheit aufmerksam
machen –“

„Überlegen Sie das sehr sorgfältig, Ell. Wie gesagt, von selbst
würde kein Mensch auf Jahre hinaus auf die Ursachen der
Erscheinungen kommen, die sich zweifellos mit der Zeit zeigen
werden. Und bisher sind die Symptome selbst erst für uns
wahrnehmbar. Wollen Sie jetzt den Menschen sagen, wir haben Euch
ein furchtbares Übel auf die Erde gebracht, schlimmer vielleicht
als die Tuberkulose? Wäre das nicht der sichere Weg, unsern Einfluß
aufzuheben? Würde das nicht zu einem allgemeinen Aufstand führen,
den wir nur mit neuen Greueln unterdrücken könnten? Nein, es darf
kein Mensch ahnen, daß wir ihm nicht bloß Heilsames auf der Erde
verbreiten.“

„Aber wir müssen die Menschen vor dem drohenden Unheil schützen.“

„Es ist, wie ich überzeugt bin, möglich, aber es ist sehr
schwierig. Zunächst müssen die Nume sich jeder unmittelbaren
Berührung mit dem Körper der Menschen enthalten, es sei denn unter
den besonderen Vorsichtsmaßregeln, wie sie der Arzt bei einer
Untersuchung anwenden kann. Und es fragt sich, ob alle der Unseren
in dieser Hinsicht zuverlässig sein werden. Für die Menschen aber
ist zweierlei notwendig: Ernährung durch chemische Nahrungsmittel
und allgemein durchgeführte Impfung. Unter diesen Umständen würde
auch die Berührung mit den Numen nichts schaden können. Aber diese
Mittel werden nicht anwendbar sein.“

„Die allgemeine Verbannung der agrarischen Nahrungsmittel ist jetzt
noch nicht möglich, sie wird sich nur nach und nach einführen
lassen. Und bis dahin könnte schon viel Schaden geschehen sein. Die
Impfung ließe sich ja zwangsweise durchsetzen, aber man müßte doch
den Grund mindestens andeuten, und wir würden jedenfalls auf
Widerstand stoßen und Unwillen erregen. Indessen, geschehen muß
etwas. Ich erwarte baldigst die eingehenden Belege für die
Richtigkeit Ihrer Ansicht und werde dann mit dem Residenten und dem
Protektor konferieren. Es müßte wohl sicher international
vorgegangen werden. Ach Hil, was für eine neue große Sorge haben
Sie mir da gemacht!“

„Es war meine Pflicht.“

„Gewiß, mein verehrter Freund. Und vergessen Sie nicht bei Ihren
Besprechungen mit den Kollegen, daß es sich um ein Numengeheimnis
handelt. Es ist zu abscheulich! Nichts ist mir unangenehmer als der
Zwang, mit der vollen Wahrheit zurückzuhalten. Und doch muß hier
aufs sorgfältigste überlegt werden, ob wir reden dürfen. Darin
haben Sie leider recht.“

Ell trat an das Fenster und blickte, in Nachsinnen verloren,
hinaus.

Hil erhob sich, um sich zu verabschieden.

Plötzlich zuckte Ell, wie von einem innern Schreck ergriffen,
zusammen. Er drehte sich schnell nach Hil um und sagte:

„Noch eins, Hil, noch eine Frage. Schenken Sie mir noch einen
Augenblick. Ich möchte wissen –: Was halten Sie von der Gefahr, die
der Aufenthalt auf dem Mars für die Menschen bietet? Glauben Sie,
daß diejenigen, die dort waren, zum Beispiel unsre Freunde, den
Keim der Krankheit in sich aufgenommen haben könnten?“

Ein leichtes Lächeln spielte um Hils Züge, als er antwortete:

„Für Ihre Person können Sie ganz unbesorgt sein. Bei Ihrem
Numenblut und Ihrer Bevorzugung der chemischen Nahrungsmittel –“

Ell winkte mit der Hand. „Nicht doch, ich dachte wirklich nicht an
mich, ich dachte – zum Beispiel Saltner – und die Forschungs- und
Vergnügungsreisenden. Wir können ja jetzt kaum Raumschiffe genug
stellen. Glauben Sie, daß wir den Verkehr beschränken müßten?“

„In dieser Frage haben wir noch keine Erfahrung. Indessen könnte es
kein Bedenken erregen, wenn man die Impfung zum Beispiel für das
Betreten der Raumschiffe unter irgendeinem Vorwand zur Bedingung
machte.“

„Aber diejenigen, die nun schon zurück sind?“

„Saltner ist auf der Reise nach dem Mars geimpft worden, weil ihm
sonst das Ehrenrecht als Nume nicht hätte erteilt werden können.
Und was – was Frau Torm betrifft, so kann ich Sie ebenfalls
beruhigen. Ich habe es für gut gehalten, während ihrer Krankheit
nach und nach die bei uns vorgeschriebenen Impfungen zu vollziehen,
und ich halte sie jetzt überhaupt für vollständig
wiederhergestellt.“

Ell, der Hil gespannt angeblickt hatte, atmete auf. Er sagte jetzt
lächelnd: „Und halten Sie mich selbst für einen Ansteckungsherd?“

„Nein, ich halte Sie in dieser Hinsicht für ganz ungefährlich.“

„Ich danke Ihnen. Und wir wollen den Mut nicht verlieren. Ich will
nachdenken, was wir tun können. Leben Sie wohl, und schonen Sie
sich. Bestimmen Sie, wann Sie Höhenluft schöpfen wollen, das
Luftschiff soll zu Ihrer Verfügung stehen.“

Er begleitete Hil bis an die Tür und schüttelte ihm die Hand. Dann
kehrte er zurück. Ein Seufzer entrang sich seiner Brust. Lange
schritt er im Zimmer auf und ab. „Nur den Mut nicht verlieren!“
sagte er zu sich selbst. Dann glitt ein stilles Lächeln über seine
Züge. „Ja, das wird mir gut tun“, dachte er.

„Den Wagen!“ rief er ins Telephon.

\section{47 - Isma}

Die Martier besaßen ein Verfahren zur Herstellung von
Akkumulatoren, die nur ein sehr geringes Gewicht hatten. Diese
waren sehr bald auf der Erde eingeführt worden und hatten das
Fuhrwesen umgestaltet. In Berlin waren die Pferde vollständig aus
dem Verkehr geschwunden. Die Nähe größerer Tiere war den Martiern
wegen der damit verbundenen Unreinlichkeit und des Geruches ein
Abscheu, und der Umgang der Menschen mit ihren Haustieren erschien
ihnen als einer der barbarischsten Züge im Leben der Erde. In der
Hauptstadt waren jetzt nur noch elektrische Wagen und Droschken im
Gebrauch.

Der elegante Wagen des Kultoramts führte Ell durch einen großen
Teil der Stadt, vom fernen Südwest bis zum Südost. Als Ziel hatte
er die Bildungsanstalt 27 angegeben, die sich in der ehemaligen
Kaserne des dritten Garde-Infanterieregiments befand. Vor einer
Nebenpforte des großen Gebäudes verließ Ell den Wagen, der dort
warten sollte. Er trat in das Haus, aber er durchschritt nur einige
Korridore und Höfe und verließ es wieder durch einen Ausgang nach
der Zeughofstraße. Von hier kehrte er in die Wrangelstraße zurück
und trat nach wenigen Minuten in eines der dortigen Mietshäuser, wo
er in dem nach dem Garten zu liegenden Flügel drei Treppen
hinaufstieg.

Hier wohnte Isma. Sie saß an dem weitgeöffneten Fenster, aus
welchem ihr Blick über die regenfeuchten Bäume des Gartens nach den
dahinter anfragenden Häusermassen und Schornsteinen schweifte. Das
Buch, in dem sie gelesen hatte, lag neben ihr. Von Zeit zu Zeit,
wenn sie ein Geräusch von Tritten zu vernehmen glaubte, blickte sie
nach der Tür, als erwartete sie, daß sie sich öffnen werde.

Und nun klingelte es draußen. Sie stand auf und strich sich das
Haar aus den Schläfen. Dann ging sie auf die Tür zu, aus welcher
ihr Ell entgegentrat.

„Endlich“, sagte er, ihre Hand ergreifend, „endlich wieder einmal
bei Ihnen. Es tat mir zu leid, daß ich unsern letzten Abend nicht
einhalten konnte, aber ich durfte die Einladung da oben nicht
ablehnen. Fühlen Sie sich auch ganz wohl?“

Sein Blick ruhte mit zärtlicher Besorgnis auf ihren Zügen.

„Es geht mir besser wie je“, sagte Isma lächelnd.

„Fühlen Sie gar keine Beschwerden?“ fragte er weiter. „Kein
Kopfweh, keine Müdigkeit?“

„Gar nichts. Sie fragen ja gerade, als wenn Sie Hil wären. Was
haben Sie denn? Ich kann Ihnen wirklich nicht die Freude machen,
mich pflegen zu lassen. Aber wissen Sie, Ell, daß Sie mir
eigentlich gar nicht gefallen? Sie strengen sich offenbar zu sehr
an, Sie sehen angegriffen aus und sollten sich mehr schonen.“

„Ach, Isma, davon kann keine Rede sein“, erwiderte Ell, indem er
sich neben ihr niederließ. „Mir ist manchmal zumute, als wüchse mir
die Arbeit über den Kopf. Und dann die Sorge! Doch nichts davon!
Dann gibt es kein andres Heilmittel für mich, als hier die drei
Treppen hinaufzusteigen –“

„Das freut mich, daß Ihnen das Treppensteigen so gut bekommt. Ich
könnte ja auch noch eine Stiege höher ziehen.“

„Oh, es genügt. Wenn ich nur die schmale Hand fassen und Ihnen in
die lieben Augen sehen kann! Dann möchte ich wieder an die Menschen
glauben und wieder hoffen!“

„Sie dürfen so nicht sprechen, Ell, Sie ängstigen mich. Auf dem Weg
zu Ihrem hohen Ziel darf es kein Schwanken geben. Dazu waren unsre
Opfer zu groß, zu schmerzlich.“

Sie hob die Augen, die mit Tränen kämpften, wie bittend zu ihm
empor.

„Verzeihen Sie mir, Isma. Ich weiß es längst, daß ich für mich kein
Glück beanspruchen darf, der ich mir anmaßte, es der Menschheit zu
bringen. Aber wenn ich hier bei Ihnen sitze – und Sie wissen, daß
ich die neue Kraft hier schöpfe –, ach, dann ist es auch so
unendlich schwer, auf das einzige zu verzichten, was ich je vom
Leben für mich ersehnte. Und immer fester wird mir die Überzeugung,
daß beides zusammengehört, wenn ich meinen Lauf erfüllen soll.“

„Noch ist die Zeit nicht da, von uns zu sprechen. O Ell, sagen Sie,
was quält Sie, was ist geschehen? Ich kenne Sie kaum wieder, noch
vor kurzem waren Sie so siegesgewiß.“

„Es geht wohl vorüber. Gerade heute haben sich allerlei Nachrichten
gehäuft, die mir Schwierigkeiten machen. Die neuen Verhältnisse
wirken ungünstig auf die Nume, das ruhige Gleichgewicht, das sie in
den festen Kulturzuständen des Mars haben, wird zerstört, es
entstehen Konflikte, und das Ende vom Liede wird sein, daß ich von
beiden Seiten für alles verantwortlich gemacht werde.“

„Das müssen Sie tragen. Und Sie wußten es im voraus, Ell, als Sie
das verantwortliche Amt annahmen, daß Sie angefeindet werden
würden. Erinnern Sie sich noch? Es war kurz nach meiner Krankheit,
als ich wieder den ersten größeren Ausflug mit ihnen unternahm, zur
Probe, wie Hil sagte, ob ich das Reisen vertrüge. Wir waren nach
den großen Schleusen der Emm-Kanäle gefahren, dort zeigten Sie
mir, wie das Wasser auf das zweihundert Meter hohe Wüstenplateau
gehoben wird. Und da sagten Sie mir, daß der Zentralrat Ihren
Vorschlag über die Einsetzung von Kultoren angenommen habe und daß
Ihnen das Kultoramt für den deutschen Sprachbezirk angetragen sei.
Sie waren im Zweifel, ob Sie es annehmen durften, und Sie sprachen
ja ganz klar ihre Befürchtung aus. Ihre Landsleute, sagten Sie,
werden auf jeden Fall unzufrieden sein, weil sie die
Bildungsanstalten als einen Zwang empfinden werden, den die
Resultate doch erst nach Jahren rechtfertigen würden. Die Nume aber
würden es Ihnen nicht vergeben, daß ein Heer von Beamten unter
Ihnen stehen solle, der Sie auf der Erde geboren sind.“

„Ich weiß es, Isma, ich sehe Sie noch dort an dem Geländer lehnen
und in Nachsinnen verloren hinabblicken auf die Baumwipfel, und ich
höre Ihr Wort: Wenn ich glaube, daß die Nume solchen Vorurteilen
zugänglich sind, so sei es nicht notwendig, daß die Menschen von
ihnen lernen. Dann hätte ich meinen großen Kulturplan überhaupt
nicht fassen dürfen. Wenn ich aber an den Beruf der Nume glaube,
die Menschheit vom Druck ihrer Geschichte zu erlösen, so dürfe ich
auch keinen Zweifel hegen, daß die Nume um der Sache willen sich
gern und frei unterordnen würden. Wenn mich der Zentralrat zu einem
Amt beriefe, wie es noch niemals auf Erden ausgeübt worden, so
geschehe es, weil jeder weiß, daß ich der geeignetste,
gewissermaßen der geborene Vermittler sei zwischen den Planeten und
daß ich mein ganzes Leben lang auf eine solche Aufgabe mich
vorbereitet habe. Und darauf –“

„Oh, ich habe es nicht vergessen, Ell“, fiel Isma ein. „Ich
erinnere mich an jedes Wort. Denn in all meinem eignen Leid steht
mir jener Moment vor Augen als der größte meines Lebens. Unter mir
schwand mein eignes Dasein vor dem erhabenen Gefühl, daß wir der
Menschheit dienen müssen, und ich war stolz und glücklich, in dem
Augenblick bei Ihnen sein zu dürfen, da von Ihrem Entschluß der
Beginn eines neuen Zeitalters abhing. Sie wiesen hinab, wo zwischen
dem Laub die weiten Wasserflächen schimmerten, und sagten: Da
unten, wo die Schmelzwasser des Pols in ihrem natürlichen Bett sich
sammeln, sind sie klar und ruhig und versiegen nimmer. Aber wir
heben sie mit unsern Maschinen in den Sonnenbrand der Wüste, und
trübe verrinnen sie allmählich in dem Bett, das Tausende von
Kilometern sich hinzieht. Wer sagt uns, wie der heitere
Seelenspiegel des Numen sich trübt, wenn wir ihn künstlich auf die
Erde versetzen und auf unübersehbare Jahre seine Reinheit im
Schlamm der Menschheit vergraben? Und da erwiderte ich Ihnen: So
weit die Kanäle sich füllen, sproßt das Leben in der Wüste, und die
Kultur des Mars beruht auf diesen sich selbst verzehrenden Adern. –
Würden die Nume diese Riesenlasten von Wasser heben und verrinnen
lassen, wenn sie nicht glaubten, daß es seine begebende Kraft auch
behält in dem künstlichen Bett? Und wer schafft es herauf? Es ist
doch die Vernunft, die die Natur leitet. Glauben Sie nicht an die
Vernunft? Und als ich dies sagte, da blitzte es drunten auf über
den Bäumen, und helle Strahlen stiegen in die Höhe und mehrten
sich, und so weit der Blick reichte, zitterten die Lichtfontänen in
der Luft, und die Leute liefen durcheinander und riefen sich zu:
›Der Friede ist geschlossen! Die Erde gehört uns – –‹ Und Sie
faßten meine Hand und sagten: ›Ja, ich glaube an die Vernunft!‹ Und
sehen Sie, Ell, ich glaube! An die Vernunft und an Sie! Und wenn
ich das nicht mehr könnte –“

Sie brach ab. Ell aber ergriff ihre Hand und rief:

„Sie können es, Isma, Sie können es! Mein Glaube an die Vernunft
ist nicht erschüttert, und mich sollen Sie nicht weichen sehen aus
feiger Schwäche. Aber die Vernunft ist ewig, ich bin ein
vergänglicher Zeuge ihres zeitlichen Gesetzes, und ich muß gefaßt
sein, daß sie über mich hinwegschreitet. Denn ich habe mir angemaßt
zu beginnen, was zu vollenden Geschlechter gehören. Wenn ich mich
nun täuschte in den Mitteln, die ich für die richtigen hielt?“

„Es wird nicht sein. Es werden Fehler gemacht werden, das ist
natürlich. Aber die Grundlagen werden sich bewähren. Sie müssen
Geduld haben.“

„Wie danke ich Ihnen, Isma, für Ihr Vertrauen, das mich vor mir
selbst rechtfertigt. Einen Fehler habe ich begangen von Anfang an,
der mehr ist als ein Fehler, daß ich eine Zeitlang die Erde vergaß
–“

„O mein Freund, den büße ich für Sie –, davon nichts mehr –“

„Und das andere, wenn es ein Fehler ist, so weiß ich nicht, wie ich
ihn hätte vermeiden sollen. Wenn ich auf die Menschen wirken
wollte, konnte ich es anders als durch die Mittel, an die sie
gewöhnt sind, durch die Autorität der Macht? Und doch weiß ich, daß
hier ein Widerspruch liegt mit dem Zweck, den ich erstrebe, der
inneren Freiheit. Den Zustand will ich aufheben, daß irgendeine
Klasse der Bevölkerung ihre Macht dazu mißbraucht, durch
Einschüchterung und Beherrschung der übrigen die freie Entwicklung
aller Kräfte und Meinungen zu verhindern, und was tue ich? Ich übe
einen neuen Zwang aus, ohne zu wissen, ob ich die eingewurzelten
Vorurteile zu brechen vermag. Ich hoffe es, doch ob ich es erlebe?
Und was dann? Droht nicht eine neue Bürokratie über der alten?“

„Ell, vergessen Sie nicht den Glauben an die Nume! Es sind nicht
Menschen, es sind Nume, welche die Menschheit erziehen. Sie werden
ihre Zöglinge als freie Männer aus der Schule entlassen, sobald sie
sehen, daß ihre Lehrarbeit getan ist.“

„Das ist meine Hoffnung. Das ist ja das absolut Neue an der
Umwälzung der Verhältnisse. Die zur Macht gekommen sind, sind es
nicht, wie die Geschlechter der Menschen, in der Absicht, die Macht
um ihrer selbst, ihrer Klasse und Nachkommen willen zu erhalten,
sondern um sie als freies Gut der Menschheit, der geläuterten
Menschheit zurückzugeben.“

„Sie werden es.“

„Sie werden es, wenn sie Nume bleiben. Wenn aber die Berührung mit
der Erde sie ihrer Numenheit entkleidet und die Menschen sie
anstecken mit ihrem Eigennutz? Wenn die alte Kultur zurückschlägt
in die Barbarei der Erde und aus den Kultoren gewöhnliche Despoten
werden, wie Päpste aus Aposteln?“

Isma schüttelte den Kopf.

„Ich weiß nicht, Ell, was Sie im Sinn haben“, sagte sie. „Es mögen
auch solche Fälle vorkommen. Aber drüben, jenseits der Erde, kreist
der Mars mit seinen drei Milliarden Bewohnern. Diese sind der feste
Kern der Kultur, der jede Entartung wieder aufheben wird.“

Ell blickte schweigend vor sich hin. Er dachte daran, ob nicht in
den Menschen der Widerstand der Natur zu groß sein würde. Aber er
sprach es nicht aus. Seine Augen wandten sich auf Isma. Sie hatte
sich in ihrem Sessel zurückgelehnt und die Hände auf dem Schoß
gefaltet. Ein einfaches schwarzes Kleid umschloß ihre Gestalt, und
das feine Profil hob sich wie eine Silhouette gegen das Fenster ab,
vor welchem der Tag bereits in Dämmerung überging. Er wollte ihr
nicht neue Sorgen erwecken. Und doch, sie jetzt schon verlassen? Es
schien ihm unmöglich. Oh, wenn er sie immer so neben sich hätte,
wie ganz anders müßte sich der schwere Kampf des Lebens aufnehmen
lassen! Sie erschien ihm begehrenswerter wie je, so lieb in ihrer
treuen Freundschaft, so groß in ihrem einfachen Vertrauen.

„Isma“, kam es fast unbewußt über seine Lippen.

Sie reichte ihm ihre Hand hinüber mit dem milden, ernsten Lächeln,
das ihre Züge mitunter in seiner Nähe verklärte.

„Mein Freund“, sagte sie.

„Isma“, sprach er leise, „wollen Sie nicht bei mir bleiben?“

Sie drückte seine Hand, ohne sie ihm zu entziehen.

„Sie wissen, Ell“, antwortete sie ebenso leise, „daß ich es nicht
darf, ja auch nicht will, so lange noch eine Möglichkeit ist –“

„Aber wenn einmal die Zeit kommt, daß keine Möglichkeit mehr ist?“

„Dann sprechen wir wieder davon. Bis dahin –. Sie kennen meine
Bitte. – Wo ist die Grenze zwischen Gedanke und Wunsch? Und das ist
Frevel.“

„Aber ich darf annehmen, Isma –“

„Nehmen Sie an, was Sie wollen. Wenn mein Leben keinem andern
gehört, wem könnte es gehören als der Idee, der wir dienen? Und
dann mögen Sie nachdenken, wie das am besten geschehen kann.“

Sie entzog ihm sanft ihre Hand und trat an das Fenster. Er stellte
sich neben sie. Schweigend blickten sie hinaus, dann begann Ell:

„Die Nachforschungen ruhen niemals, und alles, was sich hat
ermitteln lassen, weist jetzt auf eine Vermutung hin, die jede
Hoffnung fast mit Sicherheit ausschließt.“

Isma zuckte zusammen. Ell schwieg.

„Sprechen Sie weiter“, sagte sie dann gefaßt. „Ich habe mir ja
soviel hundertmal gesagt, daß ich nicht mehr hoffen darf. Und doch
ist das Wort der Gewißheit wie ein Stahl, der ins Herz trifft. Aber
– sprechen Sie weiter.“

„Er konnte die Insel Ara nur verlassen durch Schwimmen nach einer
der Nachbarinseln, das war unsere Annahme. Dann mußte er in der
Umgebung des Pols aufgefunden werden; es ist jetzt dort kein
Fleckchen mehr ununtersucht, wo Menschen existieren können. Demnach
nahmen wir an, daß er unter das Eis geraten sei –“

Isma bedeckte die Augen mit der Hand.

„Eine Möglichkeit aber war noch da, so unwahrscheinlich, daß man
erst spät daran gedacht hat. Wenige Stunden, bevor man ihn
vermißte, ging ein Luftschiff ab, das nach Tibet bestimmt war, um
dort Vermessungen zur Anlegung von Strahlungsfeldern zu machen.
Wenn er sich unbemerkt in diesem versteckt hätte obwohl ich nicht
begreife, wie das geschehen konnte –“

„Ell“, rief Isma, „warum haben Sie mir das nicht gesagt!“

„Weil ich Ihnen keine Hoffnungen erwecken wollte, die nur zu neuen
Befürchtungen führen konnten. Jetzt haben Sie sich damit vertraut
gemacht, daß wir ihn verloren haben, und Gewißheit wird besser sein
als die Angst. Denn dieses Luftschiff – der Zusammenhang ist mir
selbst erst vor kurzem durch neue Untersuchungen klar geworden –
als das Unglück geschah, war ich selbst noch nicht auf der Erde,
die Akten über Torm waren abgeschlossen, und die Vermutung, daß er
sich auf dem Schiff befand, ist erst durch meine erneute Aufnahme
des Falles aufgetaucht –, jenes Luftschiff war dasselbe, das im
Juni vorigen Jahres bei Podgoritza von den Albanern zerstört und
dessen Besatzung bis auf den letzten Mann ermordet wurde. Also auch
diese Spur, wenn sie überhaupt eine war, blieb hoffnungslos. Sind
Sie mir böse, daß ich jetzt davon gesprochen habe?“

Isma seufzte tief. „Nein, Ell, Sie müssen mir alles sagen, und ich
muß es zu ertragen wissen.“

Sie blickte wieder stumm in den Abend hinaus. Plötzlich ergriff sie
mit einer krampfhaften Bewegung Ells Arm.

„Aber wenn er auf dem Schiff war, Ell, wenn er darauf war –“

„Es ist ja nicht sicher, Isma, ich bitte Sie, beruhigen Sie sich.
Niemand weiß es, es ist nur die einzige noch denkbare Vermutung –“

„Wenn er darauf war, wer sagt Ihnen, daß er auch noch in Podgoritza
darauf war? Konnte er nicht in Tibet das Schiff verlassen haben?“

„Wie sollte er es unbemerkt im fremden Land, in der Wüste
verlassen? Und wenn man ihn bemerkte, hätte man ihn gefangen
genommen, und das ist auch, falls die erste Vermutung überhaupt
zutrifft, das Wahrscheinliche. Er wird bei einem Fluchtversuch vom
Schiff entdeckt und als Gefangener unter der Besatzung –“

„Dann aber kann er bei dem Überfall entkommen sein“, unterbrach
Isma hastig. „Das ist sehr leicht möglich. O Ell, ich habe noch
Hoffnung. Er wird sich unter jene Halbwilden geflüchtet haben, dort
muß er gesucht werden. Das müssen Sie tun, Ell! Und wenn wir ihn
finden – o Gott!“

Sie warf sich auf einen Sessel und schluchzte. Endlich wurde sie
ruhiger.

„Er hat ja nichts mehr zu befürchten“, sagte sie, „nicht wahr? Mit
dem Frieden ist die Amnestie für alles ausgesprochen, was während
des Krieges geschehen ist.“

„Nicht gerade für alles.“

„Aber für seine Flucht kann er nicht mehr bestraft werden?“

„Nein, Isma. Aber ich bitte Sie, klammern Sie sich nicht wieder an
diese Unmöglichkeit. Oh, hätte ich doch nicht davon gesprochen!
Fassen Sie sich! Ich kann Sie so nicht verlassen.“

„Sie haben recht“, sagte sie endlich. „Ich bin so töricht.“ Sie
stand auf, schloß das Fenster und schaltete das Licht ein.

„Setzen wir uns noch ein wenig“, sagte sie dann. „Es ist ja alles
so unwahrscheinlich, bei ruhiger Überlegung. Aber wer klammert sich
nicht an einen Strohhalm?“

„Sehen Sie, Isma, Sie müssen sich mit dem Geschehenen abfinden, wie
Sie es bisher getan. Wäre er in Podgoritza entflohen, so wäre er
längst hier, oder wir hätten Nachricht. Er hatte ja nun nichts mehr
von den Martiern zu befürchten. Es ist seitdem über ein Jahr
vergangen, deshalb glaubte ich darüber sprechen zu dürfen.“

Sie reichte ihm wieder die Hand. „Ich weiß ja“, sagte sie, „daß Sie
es gut meinten. Aber eins müssen Sie mir doch noch sagen. Bei
wichtigen Ereignissen wenden Sie sonst das Retrospektiv an, um den
Vorgang zu beobachten. Warum ging es denn nicht – der Überfall von
Podgoritza zum Beispiel ist doch wichtig genug –, warum wurde er
nicht vorn Mars aus –?“

„Glauben Sie mir, Isma, ich hätte es durchgesetzt, um Ihretwillen,
das Retrospektiv anzuwenden, wenn ich mir den geringsten Erfolg
hätte versprechen können. Aber an dem Tag der Flucht lagen dichte
Wolken über dem Pol, die Landung des Schiffes in Tibet ist,
vermutlich wenigstens, in der Nacht erfolgt, jedenfalls aber wird
Torm, wenn er entfliehen wollte, die Nacht dazu benutzt haben. Auch
wissen wir gar nicht, in welcher Gegend des weiten Hochasien das
Schiff angelegt hat, und es ist doch unmöglich, diese großen
Landgebiete mit dem Retrospektiv abzusuchen. Der Überfall von
Podgoritza endlich fand ebenfalls in der Nacht statt, und ehe wir
etwas davon erfuhren, hatten die Räuber alle Spuren vernichtet. Die
Tat kam erst später durch den Verrat eines feindlichen Stammes an
den Tag. Da war also nicht die geringste Aussicht, etwas in den
Lichtspuren des Weltraums zu lesen.“

„Ich sehe es ein, Ell. Und es war recht, daß Sie sprachen. Was
haben wir auch Besseres in unsrer Freundschaft, als das volle
Vertrauen? Und nun –“

„Ich soll gehen?“

„Nein, nein, im Gegenteil. Sie sollen noch bleiben, und wir wollen
von gleichgültigeren Dingen reden, von gegenwärtigen, mein’ ich.
Sie haben mir noch nichts von der Politik erzählt. Wie steht es mit
dem Klatschgesetz? Was sagt denn Herr von Huhnschlott dazu?“

Jetzt lächelte Ell. „Er speit Feuer und Flammen“, sagte er.
„Natürlich, diese Herren haben nie gelernt, daß sich die Welt auch
anders regieren lasse als mit Polizeivorschriften. Ich wünschte,
Sie hätten das Gesicht unsres geschmeidigen Kreuther sehen können,
als ich ihm meine Auffassung der Lage auseinandersetzte. Ich bin
überzeugt, morgen bekommen wir die Sanktion. Sie werden nicht an
Ill appellieren, wenn sie klug sind, denn er ist viel
rücksichtsloser gegen die Vorurteile unsrer Regierungen als ich,
der ich ihren historischen Zusammenhang besser kenne. Ich gelte ja
natürlich bei den Konservativen als ein roter Revolutionär, auf dem
Mars sehen sie mich als einen schwachmütigen Leisetreter an.“

„Ich weiß wohl“, sagte Isma. „Ich lese ja die Marsblätter,
namentlich die ›Ba‹. Solche Dinge wie Zweikampf, Beleidigungsklagen
und dergleichen kommen den Numen gerade so vor, wie uns etwa die
Menschenfresserei oder die Blutrache bei den Wilden, und sie
meinen, das müsse man einfach mit Gewalt ausrotten.“

Ell erzählte, daß Hil von seiner Reise zurück sei, und schilderte
sein Entsetzen über den Regen. Mit stiller Freude sah er, daß Isma
ihre Ruhe wiedergewonnen hatte.

Es waren wohl zwei Stunden vergangen, als Ell sich endlich herzlich
von Isma verabschiedete. Als er auf die Straße trat, war es bereits
vollständig Nacht, und die Laternen brannten. Er schritt eilig die
Straße entlang und bestieg wieder seinen vor der Tür der
Bildungsanstalt haltenden Wagen. Er hatte den in einen Mantel
gehüllten Mann nicht bemerkt, der wie zögernd vor der Tür des
Hauses gestanden hatte, wo Isma wohnte. Bei Ells Erscheinen hatte
er plötzlich kehrtgemacht, dann aber schien es, als wolle er ihm
eilig nachgehen, um ihn anzureden. Doch bald blieb er wieder
zögernd zurück und blickte nur dem Wagen nach, der Ell schnell von
dannen führte.

\section{48 - Der Instruktor von Bozen}

Durch die engen Felsschluchten des Eisacktales brauste der von Wien
kommende Schnellzug nach Süden und überholte die schäumenden Fluten
des wild dahinstürmenden Baches. Der größere Teil der Fahrgäste
drängte sich an den Fenstern, um das von der klaren Septembersonne
vergoldete Naturschauspiel zu genießen. Einer jedoch, offenbar kein
Fremder in dieser Gegend, kümmerte sich wenig darum. Er saß in eine
Ecke gelehnt, mit geschlossenen Augen in seine Gedanken versunken,
unter denen seine Stirn sich von Zeit zu Zeit zu sorgenvollen
Falten zusammenzog. Dann blickte er nach seiner Uhr, als ob der Zug
ihn nicht schnell genug seinem Ziel zuführe.

„Noch zehn Minuten“, murmelte er.

Aus der Brusttasche seiner Joppe zog er einige Papiere, ein
Telegramm und eine Zeitung. Er hatte sie schon oft gelesen, dennoch
blickte er wieder hinein, als könnten sie ihm noch etwas Neues
sagen.

Das Telegramm war von einem seiner Freunde und enthielt nur die
Worte: „Komme sofort zu Deiner Mutter, sie bedarf Deiner.“

Die Zeitung war, wie das Telegramm, schon einige Tage alt. Aber er
hatte sie erst zu Gesicht bekommen, als er gestern von einer
vierzehntägigen Studienreise in einsamen Gebirgsgegenden nach Lienz
zurückgekehrt war. Sie enthielt die neuen Verordnungen, welche das
Kultoramt in Berlin mit Ermächtigung der Residenten in Berlin, Wien
und Bern und unter Bestätigung der Regierungen in der vorigen Woche
erlassen hatte.

Die Schwierigkeiten, auf welche die Martier bei der deutschen
Regierung in bezug auf das Gesetz zum Schutz der individuellen
Freiheit gestoßen waren, hatten den Protektor der Erde darauf
geführt, sie in künftigen Fällen auf eine sehr einfache Weise zu
umgehen. Er hatte gefunden, daß Bestimmungen über Beziehungen der
Menschen zu den Numen und Einrichtungen der Nume gar keiner
Gesetzgebung durch die Erdstaaten bedürfen, sondern auf dem
Verordnungsweg durch die Residenten erlassen werden können. Die
Regierungen aber mußten, wie gern sie es auch abgelehnt hätten,
sich der Macht beugen und ihr Ja dazu geben. Sie taten es immerhin
lieber, als sich einem Beschluß der Opposition in den Parlamenten
zu fügen.

Die Verordnung hatte im allgemeinen Mißstimmung hervorgerufen. Sie
bestimmte nämlich, daß jeder Mensch, ohne Unterschied des Alters,
sich einer von den Bezirksinstruktoren zu beaufsichtigenden Impfung
unter Leitung martischer Ärzte zu unterziehen habe. Bis diese
vollzogen sei, dürfe kein Ungeimpfter sich einem Numen bis zu einer
gewissen Distanz nähern, keine von Numen bewohnte Räume betreten
und die Luftschiffe und Fahrzeuge der Martier nicht benutzen.
Zuwiderhandlungen waren mit strengen Strafen bedroht.

Die Bestimmungen waren lediglich in Rücksicht auf die Menschen
getroffen, um sie vor den drohenden Verwüstungen der Gragra zu
schützen. Aber man hatte sich gescheut, diesen Grund anzugeben,
weil man fürchtete, dadurch eine größere Beunruhigung und
Unzufriedenheit zu erregen als durch die Maßregel selbst; man hatte
die Impfung nur durch einen allgemeinen Hinweis auf Besserung des
Gesundheitszustandes begründet. Der Beschluß war von den
europäischen Residenten gegen Ells Stimme gefaßt worden, der
eindringlich vor einem derartigen despotischen Eingriff gewarnt
hatte. Doch hatte sich bei den maßgebenden Numen auf der Erde mehr
und mehr die Ansicht herausgebildet, daß man die Menschen nur durch
Anwendung von Zwang zu ihrem Besten leiten könne. Ell fühlte sich
durch den Beschluß sehr bedrückt, hatte sich aber der Majorität
fügen müssen.

Saltner steckte das Blatt kopfschüttelnd wieder ein.

„Es muß da noch etwas im Hintergrund liegen, worüber sie nicht mit
der Sprache herauswollen“, dachte er bei sich. „Aber eine sakrische
Dummheit bleibt’s doch, die ich dem Ell nicht zugetraut hätte. Oder
vielleicht doch. Wie er sich damals aussprach –“

Er dachte an jene Stunde bei La, in der er sich gegen Ells Pläne
zur gewaltsamen Erziehung der Menschen aufgelehnt hatte. Und er sah
die Geliebte wieder vor sich mit der feinen Stirn unter dem
schimmernden Haar, er sah den tiefen Blick der dunklen Augen in
zärtlicher Achtung auf sich gerichtet und fühlte die unvergeßlichen
Küsse auf seinen Lippen. Wo mochte sie weilen? Ob sie seiner
gedachte? Ob sie wußte von dem Leid, das über die Menschen gekommen
war, ob sie es mit ihm fühlte? Verloren! Verloren!

Aus seinen Träumen weckte ihn der Pfiff der Maschine. Die Berge
waren zurückgewichen, grüne Hügel, auf denen Trauben und Kastanien
reiften, zogen sich zur Seite. Die Passagiere suchten ihr
Handgepäck zusammen, und der Zug hielt auf dem Bahnhof in Bozen.

Saltner stieg aus und drängte sich eilig durch die Menge. Am
Ausgang fiel ihm ein Plakat auf, das durch seine gelbrote Farbe
schon weithin als eine amtliche Bekanntmachung des martischen
Bezirksinstruktors kenntlich war. Er blieb stehen und las. Zuerst
war die allgemeine Verordnung über die Impfung mitgeteilt, die er
schon kannte. Daran aber schlossen sich spezielle Bestimmungen über
den hiesigen Bezirk, Er traute seinen Augen nicht. Nach Angabe von
Einzelheiten über die Ausführung der Impfung, die in den und den
näher bezeichneten Lokalen stattfinde, stand da: Die als
Bescheinigung der vollzogenen Impfung erteilte Marke ist sichtbar
an der Kopfbedeckung zu tragen. Wer sich ohne dieselbe einem Numen
auf mehr als sechs Schritt annähert, wird mit fünfhundert Gulden
Geldbuße oder entsprechendem Aufenthalt im psychologischen
Laboratorium bestraft. Unterredungen mit dem Instruktor finden nur
noch telephonisch statt. Jeder Anordnung eines Numen gleichviel,
worauf sie sich beziehe, ist ohne Widerspruch Folge zu leisten. Den
Numen steht das Recht zu, Menschen, welche sich ihnen ohne
Erlaubnis nähern, mit der Telelytwaffe zurückzuweisen. Das Halten
von Haustieren in menschlichen Wohnungen wird nochmals aufs
strengste untersagt.

Saltner ballte die Faust. Er wandte sich an einen neben ihm
stehenden Herrn und sagte: „Der hiesige Instruktor ist wohl
verrückt geworden?“

„Das ist schon recht“, antwortete der ernsthaft.

„Und das lassen Sie sich gefallen? Wie heißt denn der Kerl?“

„Der heißt Oß.“

„Der Name kommt mir bekannt vor. Haben Sie sich denn noch nicht in
Berlin beim deutschen Kultor beschwert?“

„Das wird geschehn. Aber es dauert halt eine Weile, und die
Verordnung ist erst von gestern.“

„Aber wenn Sie telegraphieren oder telephonieren?“

„Das wird nicht zugelassen. Es ist schon einer nach Innsbruck
gereist, aber sie haben’s auch dort nicht zugelassen. Sie meinen,
die Nume stecken halt alle unter einer Decke, und wenn es auch der
Kultor erfährt, so wird es doch nichts nutzen.“

„Es wird nutzen, das können Sie mir glauben. So etwas hat sich
keiner herauszunehmen und nimmt sich auch keiner woanders heraus.
Das ist nur eine Verrücktheit von diesem Oß, und der wird sehr bald
abgesetzt sein.“

„Das mag schon sein, so lange halten wir’s wohl aus. Aber die
Hauptverordnung bleibt doch bestehen, und dagegen ist nichts zu
machen. Ich mein’ so, den Oß werden sie schon wegjagen, vielleicht
gar bald, denn der Herr Bezirkshauptmann reist heute nach Wien und
wenn nötig nach Berlin. Aber inzwischen müssen wir folgen. Denn
wenn sich einer was gegen den Oß herausnähme und es ginge auch
nachher dem Oß schlecht, so ginge es uns doch noch schlechter. Wir
würden wegen Aufruhr nach Afrika oder sonstwohin geschickt. Also
lassen wir’s lieber. Habe die Ehre!“

Damit lüftete er den Hut und wollte sich entfernen. Gleich darauf
wandte er sich jedoch zurück und sagte mit einem fragenden Blick:
„Verzeihen Sie, ich irre mich doch wohl nicht, sind Sie nicht der
Herr von Saltner?“

„Mein Name ist Saltner.“

„Dann nehmen Sie’s nicht übel, wenn ich mir einen Rat erlaube – Sie
sind ja doch auf dem Mars gewesen, und da muß wohl irgend etwas
passiert sein –, nehmen Sie sich nur vor dem Oß in acht, ich weiß,
daß der sich schon mehrfach erkundigt hat, ob Sie nicht hier sind –
der muß irgend etwas gegen Sie haben. Lassen Sie sich lieber nicht
hier sehen, es kann ja nur ein paar Tage dauern, bis der Mann
abgesetzt ist.“ Und vertraulicher fuhr er fort: „Sie haben ja
vollständig recht, ich weiß, daß diese Bekanntmachung zu Unrecht
besteht und der Oß den Erdkoller hat – ich bin nämlich der Doktor
Schauthaler.“

„Ach, jawohl“, sagte Saltner, „ich erinnere mich jetzt sehr wohl,
entschuldigen Sie, daß ich Sie nicht gleich erkannte.“

„Bitte sehr. Nun also, solche Ausschreitung wird ja rektifiziert
werden. Aber lassen wir uns dadurch zu irgendeiner eigenmächtigen
Handlung hinreißen, so würde uns das trotzdem sehr schlecht
bekommen. Deswegen versuch ich mein Möglichstes, unsre Mitbürger zu
beruhigen. Wenn Sie indessen etwas tun wollen, so bringen Sie sich
selbst in Sicherheit, bis der Mann hier keine Gewalt mehr hat;
vorläufig hat er sie nun einmal, und Sie sind dagegen ohnmächtig.
Sie sind ja doch mit dem Herrn Kultor befreundet, reisen Sie sofort
zu ihm – in zehn Minuten kommt der Blitzzug von Venedig –, das
Luftschiff dürfen Sie jetzt nicht benutzen – aber auch so sind Sie
morgen in Berlin –“

„Ich danke Ihnen sehr für den Rat, Herr Doktor, nur kann ich ihn
leider nicht sogleich befolgen. Ich habe hier zunächst
unaufschiebbare Geschäfte –. Aber ich werde dann –“

„Dann, Herr von Saltner, dann? Sie wissen nicht, ob Sie dann noch
ein freier Mann sind –“

„Das wollen wir doch sehen! Da können Sie ganz unbesorgt sein!“

„Was nützt es Ihnen, wenn der Oß in ein paar Tagen vor das
Disziplinargericht gestellt wird, und Sie sind inzwischen irgendwie
verunglückt?“

„Ich verunglücke nicht so leicht. Aber was will denn der Mann von
mir?“

„Ich weiß es nicht. Ich weiß nur privatim durch den
Bezirkshauptmann, daß Sie gesucht werden, aber amtlich ist es
nicht. Es muß irgend etwas sein, worüber der Oß vorläufig nicht
reden will.“

Saltner runzelte die Stirn.

„Nun, wie gesagt, ich danke Ihnen und will mich vorsehen. Jetzt
entschuldigen Sie mich, ich darf nicht länger zögern.“

Er schritt eilend durch die Straßen der Stadt, ohne auf die
Umgebung zu achten. Was konnte dieser Oß von ihm wollen? Wo hatte
er ihn gesehen? Oß war ja der Name des Kapitäns gewesen, auf dessen
Raumschiff ›Meteor‹ Saltner die Reise nach dem Mars gemacht hatte,
und dann war er ihm manchmal in Frus Haus begegnet. Sollte es
derselbe sein? Er hatte sich mit ihm ganz gut unterhalten, und der
tüchtige, wenngleich etwas selbstbewußte Mann war mit La und Se
immer sehr vertraut gewesen. Mit Se? Sein Gewissen schlug ihm. Das
war das einzige, was er sich hatte zuschulden kommen lassen, die
Belauschung der Schießversuche und die Flucht aus dem als Ziel
dienenden Schiff. Aber dann hätte ihn Se verraten müssen, das war
unmöglich, ganz unmöglich.

Saltner hatte die Stadt durchschritten und betrat die Brücke,
welche über die Talfer führt. Drüben, jenseits des Flusses, wohnte
seine Mutter. Sie war diesmal schon früher als sonst von dem
kleinen Häuschen, das sie oben in den Bergen besaß, in die Stadt
herabgezogen, er selbst hatte noch den Umzug mit ihr besorgt und
war dann auf eine Studienreise gegangen. Was war nun geschehen?

Es fiel ihm auf, wie leer die Brücke war, auf der sonst um diese
Zeit, gegen Abend, ein reger Verkehr herrschte. Als er die Mitte
überschritten hatte, blieb er stehen und wandte sich nach alter
Gewohnheit rückwärts, um einen Blick auf das entzückende Panorama
zu werfen. Freudig hing sein Auge, über die altertümlichen Giebel
der Stadt wegschweifend, an den rötlich schimmernden Zacken und
Zinnen der Dolomiten, die der Rosengarten kühn in die Luft
streckte, und seine Seele schwebte über den freien Höhen. Aber er
durfte nicht lange weilen. Die Sorge um die Mutter trieb ihn
vorwärts.

Wenige Schritte hatte er zurückgelegt, als ihm einige Leute
entgegenkamen, die eilend an ihm vorüber der Stadt zuschritten und
ihn durch Winke zur Umkehr aufforderten. Er achtete nicht darauf,
sondern richtete seine Aufmerksamkeit auf einen seltsamen Aufzug,
der jetzt aus den Talfer-Anlagen herauskommend die Brücke betrat.
Eine Anzahl Neugierige, halbwüchsige Jungen, liefen voran, hielten
sich aber immer in respektvoller Entfernung. Dann folgte auf einem
Akkumulator-Dreirad ein Martier mit seinem diabarischen
Glockenhelm, ein riesiger Bed oder Wüstenbewohner, der hier
ähnliche Dienste verrichtete wie die Kawassen der Konsuln in der
Türkei. Er schwang ein langes Rohr mit einem Fähnchen in der Hand,
womit er die Begegnenden bedeutete, schleunigst zur Seite zu
weichen. Darauf folgte ein vierrädriger elektrischer Wagen, auf
dessen Polster in bequemer Stellung der Instruktor und zur Zeit
Tyrann von Bozen, der Nume Oß, ruhte, ebenfalls von dem Glockenhelm
gegen die Erdschwere geschützt. Den Beschluß bildete wieder ein Bed
auf seinem Dreirad.

Saltner erkannte auf den ersten Blick, daß er wirklich seinen alten
Bekannten, den ehemaligen Kapitän des Raumschiffs ›Meteor‹, vor
sich hatte. Er trat zur Seite in die halbkreisförmige Ausbuchtung
eines Brückenpfeilers, um den Zug an sich vorüberzulassen. Dem
voranfahrenden Bed erschien jedoch die Entfernung noch nicht
genügend, er winkte mit seiner Fahne und rief sein eintöniges:
„Entfernt euch!“ Saltner blieb ruhig stehen. Er streckte den linken
Arm gegen den Bed aus und wandte ihm die Handfläche mit gespreizten
Fingern zu. Der Bed stutzte. Das war ein nur bei den Numen
gebräuchliches Zeichen und bedeutete ungefähr soviel als: „Dein
Auftrag geht mich nichts an, ich besitze eine weitergehende
Vollmacht.“ Dann rief er ihm auf martisch in entschiedenem Ton zu:
„Fahr zu, ich bin ein alter Freund deines Herrn.“

Der Bed wußte nicht recht, was er davon denken sollte, ließ sich
jedoch in der Meinung, es vielleicht mit einem Numen zu tun zu
haben, einschüchtern und fuhr weiter. Saltner, den sein Stolz
verhindert hatte, sich fortweisen zu lassen, wollte doch lieber die
Begegnung mit Oß vermeiden und blickte über das Geländer der Brücke
in die Landschaft, indem er dem Wagen den Rücken zukehrte. Oß
dagegen hemmte den Wagen und herrschte Saltner an:

„Kann der Bat nicht grüßen!“

Saltner trat jetzt ohne weiteres auf den Wagen von Oß zu, grüßte
höflich nach martischer Sitte und sagte, ebenfalls martisch
sprechend, ganz unbefangen:

„Es freut mich sehr, einem alten Bekannten zu begegnen. Wie geht es
Ihnen, Oß?“

Dabei sah er ihn, die Augen soweit wie möglich aufreißend,
unverwandt an.

Oß hatte Saltner sofort erkannt. In seinen Augen blitzte es
unheimlich, indem er seinen Blick auf Saltner richtete, als ob er
ihn niederschmettern wolle. Aber Saltner kannte die Augen der Nume.
Dieses unruhige Funkeln war nicht der reine Blick des Numen, aus
dem der sittliche Wille sprach, er war getrübt von etwas
Krankhaftem, Selbstischem und besaß nicht mehr die Kraft, den des
Rechts sich bewußten Menschenwillen zu beugen. Er hielt den Blick
aus, während Oß in hochmütigen Worten ihn anherrschte:

„Was fällt dem Bat ein? Wer sind Sie? Wissen Sie nicht, daß Sie
sich sechs Schritt entfernt zu halten und überhaupt nicht mit mir
zu reden haben? Entfernen Sie sich sofort, oder –“

Er griff nach dem Telelytrevolver in seiner Tasche.

Saltner trat jede Bewegung von Oß genau im Auge behaltend,
vorläufig einen Schritt zurück und sagte laut, jetzt auf deutsch,
von dem er wußte, daß Oß, wie jeder Instruktor im deutschen
Sprachgebiet, es verstand, so laut, daß es bis zu den Neugierigen
vor und hinter dem Zug schallte:

„Sie scheinen mich nicht mehr kennen zu wollen. Gestatten Sie, daß
ich Ihrem Gedächtnis nachhelfe. Mein Name ist Josef Saltner,
Ehrengast der Marsstaaten auf Beschluß des Zentralrats mit allen
Rechten des Numen, und hier ist mein Paß, lautend auf zwei
Marsjahre, unterzeichnet von Ill, zur Zeit Protektor der Erde.
Bitte, mit dem gehörigen Respekt zu betrachten.“

Er zog aus seiner Tasche das nach Art der Marsbücher an einem Griff
befindliche Täfelchen und ließ es aufklappen.

„Der Paß ist noch nicht abgelaufen“, sagte er darauf leiser, „ich
denke, Sie lassen jetzt das Ding stecken. Erkennen Sie mich nun
wieder?“

Dabei trat er unmittelbar an den Wagen heran. Er sah, welche
Überwindung es Oß kostete, sich zu bezwingen, aber diesem Dokument
gegenüber blieb ihm kein anderer Ausweg. Oß versuchte jetzt
möglichst unbefangen zu lächeln und sagte:

„Ach, Sie sind Sal – entschuldigen Sie, daß ich Sie nicht gleich
erkannte. Das ist etwas anderes. Es freut mich sehr, Sie zu sehen.
Aber warum beehren Sie mich nicht in meinem Haus? Hier auf der
Straße bin ich gezwungen, sehr vorsichtig zu sein, Sie werden ja
wissen –“

„Die Begegnung überraschte mich, entschuldigen Sie daher diese
formlose Begrüßung auf der Straße. Ich konnte nicht annehmen, daß
der Unterzeichner jener Verordnung identisch sei mit dem Oß, den
ich –“

„Herr, Sie sprechen in einem Ton, den ich zurückweise.“

„Das nützt Ihnen nichts. Sie wissen so gut wie ich, daß derartigen
Befehlen niemand Folge zu leisten braucht.“

„Ich verbitte mir alle Einmischung in meine Angelegenheiten. Ich
bin hier der alleinige Befehlshaber und werde Ihren Widerspruch
bändigen. Wenn Sie auch durch Ihren Paß dagegen geschützt sind, von
meiner bisherigen Verordnung getroffen zu werden, so hindert mich
doch nichts, über Sie selbst einen speziellen Befehl auszusprechen,
so lange Sie sich in dem mir unterstellten Bezirk befinden. Merken
Sie sich das. Ich habe Sie im Verdacht, gegen amtliche Anordnungen
aufzuwiegeln. Sie werden sich deshalb noch heute zu verantworten
haben.“

Ohne Saltner Zeit zu einer Antwort zu lassen, hatte Oß bereits
seinen Wagen in Gang gesetzt und fuhr davon. Saltner blickte ihm
spöttisch nach und schritt dann eilig weiter. Wenige Minuten später
stand er vor dem Haus seiner Mutter. Es war ein altes, nicht großes
Haus. Im unteren Stockwerk wohnte Frau Saltner mit ihrer Bedienung,
einer älteren Frau. Das obere wurde im Winter an Kurgäste
vermietet, war aber jetzt noch unbesetzt. Saltner hatte den
Hausflur schnell durchschritten und die Tür des Wohnzimmers
geöffnet. Es war leer. Der Platz an dem nach dem Garten sich
öffnenden Fenster, an dem seine Mutter den größten Teil des Tages
zu sitzen pflegte, war unbesetzt. Saltner erschrak. Sollte sie
krank sein und zu Bett liegen? Er schaute vorsichtig, um nicht zu
stören, in das Schlafzimmer, aber auch hier war niemand. Besorgt
durchsuchte er nun das ganze Haus, weder seine Mutter noch ihre
alte Magd und Gehilfin, die Kathrin, waren zu finden. Aber auch der
Karo, der Hund, war nicht da, der sonst jeden Kommenden durch sein
Gebell anmeldete und ihm sicher zuerst entgegengesprungen wäre.
Wären die Frauen beide ausgegangen, so hätten sie gewiß das Haus
verschlossen. Doch vielleicht waren sie nur auf einen Augenblick in
den Garten gegangen. Eben wollte sich Saltner Gewißheit holen, als
sich die Hintertür des Hauses öffnete und die Kathrin hereintrat.
Der Korb mit Obst, den sie trug, entfiel fast ihren Händen, so
schnell setzte sie ihn zu Boden, als sie Saltner erblickte.

„Gelobt sei die heilige Jungfrau!“ rief sie aus. „Da ist ja der
Herr Josef.“

„Grüß Gott, Kathrin“, sagte Saltner. „Wo ist denn die Mutter? Es
fehlt ihr doch nichts?“

Die Dienerin brach sogleich in einen Tränenstrom aus.

„Sie haben sie ja, sie haben sie ja!“ rief sie unter Schluchzen.

„Was haben sie denn? So reden Sie doch schon! Kommen Sie hier
herein, Kathrin, und reden Sie vernünftig.“

Die Frau trat in das Zimmer, aber aus ihrem von Weinen
unterbrochenen Redeschwall konnte Saltner zunächst nichts verstehen
als unzusammenhängende Worte, wie „mit dem Karo hat’s angefangen“,
„in den Arm wollen sie stechen“, „den Hund haben’s genommen“, „mich
wollen’s auch impfen“, „sie haben sie“, „im Laboratorium“ und „wenn
sie der Herr Josef nicht schnell herausholt, so werden sie sie doch
noch braten“ und „fünfhundert Gulden sollt’ sie zahlen“. Endlich
beruhigte sie sich soweit, daß Saltner über den Zusammenhang
allmählich klar wurde.

„Mit dem Karo hat’s angefangen.“ Die Hunde waren den Numen ein
Greuel. War ihnen schon die Berührung mit Tieren überhaupt ein
Zeichen der Barbarei, so waren ihnen die Hunde wegen ihres
ekelhaften Treibens auf der Straße und ihres abscheulichen Gekläffs
ganz besonders verhaßt. Sie machten ihnen den Aufenthalt auf der
Erde um so unleidlicher, als sie auch ihrerseits gegen die Martier
eine besondere Abneigung zu haben schienen und sie überall mit
ihrem Gebell verfolgten. Es waren deswegen schon überall
einschränkende Bestimmungen über das Herumtreiben der Hunde auf der
Straße ergangen. Oß aber hatte kurzen Prozeß gemacht, nachdem er
einmal von einem Hund angefallen worden war, und die Tötung aller
Hunde befohlen. Dies war kurz nach Saltners Abreise geschehen, und
das erste Zeichen der bei Oß im Ausbruch begriffenen nervösen
Überreizung gewesen. Die Polizeimannschaften führten den Befehl
möglichst langsam und absichtlich ungeschickt aus und wußten es so
einzurichten, daß viele ihre Lieblinge rechtzeitig in Sicherheit
bringen konnten. Das Haus von Frau Saltner hatte sich aber Oß
einmal zeigen lassen und dabei den Hund bemerkt, ja, er hatte dann
gefragt, ob denn das Vieh noch nicht totgeschossen sei. So mußte
der arme Karo als ein Opfer zur Zivilisation der Menschheit fallen.
Das hatte nun die Frauen, die innigst an dem Hund hingen, in größte
Aufregung versetzt. Frau Saltner war ganz melancholisch geworden
und wurde von einer krankhaften Ängstlichkeit ergriffen, sobald
jemand in das Haus trat.

Nun war die Verordnung über das Impfen gekommen. Unglücklicherweise
war ihr Straßenviertel das erste gewesen, in welchem die Impfung
vollzogen wurde. Sie stellte sich dies als eine fürchterliche
Operation vor und schickte zu einem Freund Saltners, um sich Rat zu
holen, was sie tun solle. Alle seine Vorstellungen waren vergebens,
sie ließ sich nicht bereden, ebensowenig wie Kathrin, zu dem Termin
zu gehen, und der Freund wußte nichts Besseres zu tun, als an
Saltner zu telegraphieren. Inzwischen war der Termin verfallen, und
Frau Saltner wie ihre Dienerin wurden zu je fünfhundert Gulden
Strafe verurteilt. Nun gab es erst recht ein großes Wehklagen, das
Geld war, zumal in Saltners Abwesenheit, nicht zur Stelle zu
schaffen, und die beiden Frauen sollten in das psychophysische
Laboratorium zur Abbüßung der Strafe und zur Vollziehung der
Impfung abgeholt werden.

Die Beamten, welche die Anordnungen des Instruktors nur widerwillig
vollzogen, hätten es gern gesehen, wenn die Frauen sich auf
irgendeine Weise unsichtbar gemacht hätten. Und als sie endlich in
das Haus traten, hatte sich auch Kathrin versteckt und war nicht zu
finden. Frau Saltner aber saß auf ihrem Platz und sagte nur: „Ich
bin eine alte Frau und geh nicht eher hier fort, bis mein Sohn
kommt. Ihr könnt machen, was ihr wollt.“

Da sie keine andre Antwort erhielten und gegen die alte Frau, noch
dazu die Mutter eines in der ganzen Umgegend gekannten und
beliebten Mannes, keine Gewalt brauchen wollten, entfernten sie
sich wieder und brachten irgendeine Entschuldigung vor. Es war
aber, als ob der Instruktor alles heraussuchte, womit er Saltner
Kränkungen bereiten konnte, so daß er sich persönlich um die
Einzelheiten kümmerte, wenn Saltner in Frage kam. Er schickte einen
der Assistenten des Laboratoriums, einen jungen Nume, der hier
seine Studien machte, mit seinen beiden Beds ab, und Frau Saltner
wurde in einem Krankenstuhl in das Laboratorium geschafft.

„Sie haben es gewagt, diese Schufte?“ rief Saltner wütend.

„Eine fast siebzigjährige Frau! Und das nennt sich Nume! Und was
hat denn die Mutter gesagt?“

„Gar nichts hat sie gesagt“, antwortete Kathrin unter neuem
Schluchzen, „als nur immer, mein Josef, mein armer Josef, und, ich
überleb’s nimmer, und geweint hat sie, aber gesagt hat sie nichts
mehr.“

Saltner stand stumm und überlegte, was zu tun sei. Die Tränen
traten ihm in die Augen, wenn er an die Angst dachte, die seine
Mutter ausstand. Er wußte ja, daß ihr tatsächlich nichts geschehe,
daß man sie als eine Kranke behandeln würde und sie vielleicht
sicherer aufgehoben sei als zu Hause. Denn wenn auch Oß
unzurechnungsfähig war, der Leiter des Laboratoriums war ein Arzt,
ein wohlwollender Mann, der seine Aufgabe ernst im Sinn von Ell
nahm, und die Strafanstalt, als welche das Laboratorium diente, mit
Rücksicht auf jeden individuellen Fall leitete. Aber die Angst, die
Furcht, die Vorstellungen, die sich seine Mutter machen mochte, und
die Kränkung! Das konnte wirklich ihr Tod sein. Nicht eine Stunde
länger wollte er sie in dieser Besorgnis allein lassen, er mußte
sie herausholen.

Kathrin begann aufs neue zu jammern.

„Ist es denn wahr, Herr Josef, im Laboratorium, daß die Leute da
gebraten werden –“

„Reden Sie nicht so dummes Zeug, Kathrin, gar nichts geschieht
ihnen, als daß sie ein bißchen beobachtet werden, wie der Puls
geht, wenn sie so oder so liegen, oder wenn sie kopfrechnen –“

„Kopfrechnen, Jesus Maria, das könnt’ ich nun schon gar nicht.“

„Jedenfalls seien Sie still, und hören Sie, was ich sage, aber
passen Sie genau auf. Ich werde jetzt gleich die Mutter holen.“

„Ach Herr Josef, Sie werden sich doch nicht dahin wagen!“

Aber Saltner sprach nicht sogleich weiter. Er ging im Zimmer auf
und ab, während Kathrin lamentierte, und dachte seinen Entschluß
genau durch. Er dachte an die Warnung Schauthalers und an die
Begegnung mit Oß und sagte sich, daß er selbst keinen Augenblick
sicher sei. Aber die Mutter durfte er nicht ohne die größte Gefahr
für ihre Gesundheit länger in ihrer Angst und Einsamkeit lassen. Er
mußte sie und zugleich sich in Sicherheit bringen. Er war in
Sicherheit, sobald er das Gebiet verlassen hatte, das Oß
unterstellt war. Die Instruktoren der Nachbargebiete würden solchen
ungesetzlichen Forderungen nicht nachgeben, außerdem konnte er sich
auch einige Zeit im verborgenen halten. Er mußte sich nur hüten,
etwas zu tun, was von der Oberbehörde der Nume aus verboten war,
denn dadurch hätte er sich auf der ganzen Erde der Verfolgung
ausgesetzt. Sonst aber kam es allein darauf an, den Bezirk von Oß
zu vermeiden, bis dieser abgesetzt war. Dieser Bezirk erstreckte
sich über das westliche Südtirol, fiel aber nicht mit der
österreichischen Landesgrenze zusammen, sondern reichte nur bis an
die Grenzen des deutschen Sprachgebiets. Diese lief in wenigen
Stunden Entfernung im Westen, Süden und Osten über die Berge.
Dahinter war italienisches Sprachgebiet, das einem Kultor in Rom
unterstand. Über diese Grenze mußte er zunächst und auf der
Stelle.

Saltner ging an die Haustür, die er verschloß, ebenso verschloß er,
soweit dies Kathrin nicht schon getan hatte, die Fensterläden. Aus
einer Kassette in seinem Schreibtisch nahm er Papiere, die er zu
sich steckte. Dann ging er in den Garten und rief die Dienerin zu
sich.

„Kathrin“, sagte er, „nun seien Sie ganz still und tun Sie genau,
was ich sage. Ich werde die Mutter und Sie in Sicherheit bringen,
aber wenn Sie nicht genau alles tun, kommen Sie doch noch ins
Laboratorium. Schon gut! Jetzt gehen Sie – aber hier hinten zum
Garten hinaus – zum Rieser und sagen ihm, er möchte sogleich
einspannen und mit dem Wagen hinten am Tor, wo’s nach der Meraner
Straße geht, warten. In einer halben Stunde ist’s dunkel, dann
komme ich. Es wäre aber eine wichtige und geheime Sache, er wird
sich’s schon denken. Dann laufen Sie schnell – ist der Palaoro zu
Haus, der Sohn, mein ich?“

„Er wird schon zu Haus sein. Es gibt jetzt wenig Touren.“

„Er soll mit zwei zuverlässigen Leuten und zwei Maultieren mit
Frauensätteln sogleich nach Andrian aufbrechen, und wenn ich noch
nicht da bin, mich dort erwarten. Er soll auch den Schlüssel zur
kleinen Hütte mitnehmen. Dann laufen Sie gleich wieder nach Hause,
aber von hinten herein, und nehmen die Decken und etwas Zeug für
die Mutter und für sich, aber nur ein kleines Bündel – etwas zu
essen soll der Rieser besorgen –, und kommen wieder zum Rieser, wo
der Wagen hält. Und das weitere wird sich finden. Haben Sie alles
verstanden?“

„Ganz genau, Herr Josef, ich laufe bald.“

\section{49 - Die Flucht in die Berge}

Saltner verließ durch die Hintertür des Gartens seine Wohnung. In
wenigen Minuten stand er vor der Kaserne, die jetzt den Martiern
als Laboratorium, Schule und Strafanstalt diente. Er trat in das
Wartezimmer und verlangte den dirigierenden Arzt oder dessen
Stellvertreter zu sprechen.

Beide hatten bereits die Anstalt verlassen und sich in die Stadt
begeben. Der zweite Assistent, ein ganz junger Mann, der erst vor
kurzem vom Mars gekommen war, empfing ihn. Saltner stellte sich vor
und legitimierte sich durch seinen Paß. Der junge Nume wurde
außerordentlich höflich und etwas verlegen. Er sagte sogleich: „Sie
kommen gewiß wegen Ihrer Frau Mutter. Ich muß gestehen, ich weiß
nicht recht, wie es zusammenhängt, daß Ihre Frau Mutter hier
festgehalten wird, wir wissen ja doch alle, mit welchen Ehren Sie
als der erste Bat auf dem Nu empfangen wurden – aber es liegt ein
ausdrücklicher Befehl des Instruktors vor.“

„Das hängt einfach so zusammen“, sagte Saltner, „daß ich verreist
war und meine Mutter mit den Verhältnissen nicht Bescheid wußte,
auch während meiner Abwesenheit nicht über die Mittel verfügte, die
geforderte Geldstrafe wegen des versäumten Termins zu bezahlen. Ich
komme jetzt, um meine Mutter abzuholen, und deponiere hier
Obligationen im Betrag von tausend Gulden für meine Mutter und
unsere Dienerin Katharina Wackner, mit dem Vorbehalt, die
Gültigkeit der Verordnung auf dem Rechtswege zu bestreiten. Wollen
Sie die Güte haben, meine Mutter holen zu lassen.“

„Ich bin sehr gern bereit, Sie zu Ihrer Frau Mutter zu führen, aber
das Geld kann ich nicht annehmen, Sie müssen dasselbe auf der
Bezirkskasse deponieren, auf den erhaltenen Schein wird die
Entlassung verfügt werden. Ich bin dazu nicht ermächtigt.“

„Das ist aber äußerst fatal. Ich kann meine Mutter keinen
Augenblick länger hier lassen, sie wird dadurch im höchsten Grade
deprimiert, und es steht für ihre Gesundheit das Schlimmste zu
befürchten.“

„Ich muß zugeben, es wäre sehr wünschenswert, daß Ihre Frau Mutter
zu Ihnen käme – unsrerseits würden wir ja gern sofort –, wenn nicht
–“ Er zuckte mit einem bedeutungsvollen Blick die Achseln.
„Indessen, es wird sie beruhigen, wenn ich Sie inzwischen zu ihr
führe. Ich möchte Ihnen gern in jeder Hinsicht gefällig sein und
Ihnen daher folgendes vorschlagen. Um zehn Uhr kommt der Direktor
zurück, es sind dann noch einige Schlaf- und Traumversuche
anzustellen. Inzwischen fahre ich mit dem Geld nach der Kasse,
vielleicht treffe ich noch einen Beamten, ich besorge Ihnen den
Schein, und darauf wird der Direktor die Entlassung verfügen.“

„Sie sind außerordentlich liebenswürdig“, sagte Saltner. „Es ist
nur fraglich, ob es nicht schon zu spät am Tage ist – wollen Sie
mir nicht auf Ihre Verantwortung meine Mutter anvertrauen?“

„Das ist mir ganz unmöglich, so gern ich möchte.“

„Nun“, sagte Saltner mit einem Gesicht, das wenig Freude verriet,
„dann bleibt mir nichts anderes übrig, als ihr freundliches
Anerbieten anzunehmen.“

„Sehr gern. Sobald ich Sie zu Ihrer Mutter gebracht habe, fahre
ich, und in einer halben Stunde bin ich wieder hier.“

Saltner war in verzweifelter Stimmung. Er konnte das Anerbieten des
Numen nicht ablehnen, aber er konnte auch unmöglich diese
Entwicklung der Angelegenheit abwarten. Denn abgesehen davon, daß
sich heute vielleicht überhaupt nichts mehr erreichen ließ, so
mußten doch noch gegen zwei Stunden vergehen, ehe die Entlassung
vom Direktor zu erhalten war. Das war für Saltner so gut als die
Vereitelung seiner Rettung. Denn selbst wenn, was keineswegs
ausgeschlossen war, Oß von der Zahlung nichts erfuhr, so mußte doch
Saltner mit Gewißheit annehmen, daß noch in dieser Stunde Oß seine
Drohung ausführen und ihn persönlich zur Rechenschaft ziehen würde.
Vermutlich war sein Haus jetzt schon besetzt; wenn er nicht
zurückkehrte, so würde man ihn sicher bei seiner Mutter suchen; er
konnte jeden Augenblick erwarten, daß man ihn auf Grund einer
besonderen Order, die der Instruktor durchsetzen würde, hier
verhaften werde. Jede Minute war ihm kostbar. Das ging ihm durch
den Kopf, während er mit dem Assistenten durch die Korridore nach
dem Zimmer seiner Mutter schritt.

Der Nume blieb vor einer Tür stehen.

„Hier ist es“, sagte er, „gehen Sie allein hinein. Ich will
inzwischen in Ihrem Interesse eilen.“

Saltner schoß ein Gedanke durch den Kopf.

„Gestatten Sie noch eine Frage“, sagte er. „Wer vertritt Sie in
Ihrer Abwesenheit von hier?“

„Dr. Frank, der frühere Stabsarzt.“

„Ich kenne ihn. Ich möchte mit ihm über meine Mutter sprechen;
würden Sie die Güte haben, ihm sagen zu lassen, daß er sich hierher
bemühe?“

„Sehr gern.“ Der Nume verabschiedete sich.

Saltner blieb kurze Zeit pochenden Herzens vor der Tür stehen.

Leise klopfte er an. Es erfolgte keine Antwort. Er öffnete die Tür
geräuschlos und trat in das Zimmer. Es war fast dunkel, nur ein
letzter Schein der Dämmerung ließ noch einen unsichern Überblick
zu. Über einem Betstuhl in der Ecke brannte eine ewige Lampe. Davor
kniete Frau Saltner, in inbrünstigem Gebet begriffen. Er hörte sie
leise Worte murmeln.

Saltner wagte kaum zu atmen. Seine Augen füllten sich mit Tränen.
Und doch hing vielleicht alles an einer Minute.

„Mutter“, sagte er leise.

Ihre Lippen verstummten. Ihr Blick richtete sich wie verzückt nach
oben.

„Mutter“, wiederholte er. „ich bin’s, der Josef.“

Sie blieb in ihrer Stellung, als fürchtete sie, durch eine Bewegung
die Erscheinung zu verscheuchen.

„Es ist seine Stimme“, flüsterte sie. „Die heilige Jungfrau hat
mein Gebet erhört.“

Er kniete neben ihr nieder und umschlang sie mit seinem Arm. Jetzt
erst wandte sie ihm das Gesicht zu. Mit einem Freudenschrei fiel
sie ihm um den Hals.

„Steh auf, Mutter“, sagte er, „und komm schnell, ich bin hier, um
dich abzuholen. Wir müssen sogleich gehen.“

Er zog sie empor. Sie küßte ihn zärtlich. Sie sprach kein Wort. Nun
er da war, nun war es ihr wie selbstverständlich, daß sie fortgehen
konnte. Sie suchte ihre Sachen zusammen.

„Laß nur alles liegen“, sagte er, „es wird alles geholt werden. Nur
dein Tuch nimm um, es wird kühl. So, nun komm!“

Ihre Knie zitterten, er mußte sie stützen. Langsam gingen sie zur
Tür und betraten den Korridor.

Nach wenigen Schritten kam ihnen Doktor Frank entgegen.

„Guten Abend, Saltner“, sagte er herzlich. „Nun werden Sie ja
hoffentlich bald die liebe Frau Mutter wieder haben. Kommen Sie mit
mir in mein Zimmer, und essen Sie mit mir zu Abend, dort können Sie
alles gemütlich abwarten.“

„Lieber Freund“, antwortete Saltner, „ich danke Ihnen innig, aber
ich muß Ihnen eine Überraschung bereiten. Ich gehe jetzt mit meiner
Mutter sogleich fort. Ich habe Gründe, weshalb ich nicht warten
kann.“

„Haben Sie denn den Schein und das Attest vom Direktor?“

„Nein, das brauche ich nicht, wir gehen so.“

„Aber ich bitte Sie, bester Freund, das ist unmöglich, das darf ich
ja leider nicht zulassen –“

„Sie müssen es.“

„Es geht nicht. Sie bringen mich in Teufels Küche. Es geht mir an
den Kragen.“

„Ihnen kann gar nichts passieren. Kennen Sie die Verordnung von Oß,
wo es heißt: ›Jeder Anordnung eines Numen, gleichviel, worauf sie
sich beziehe, ist ohne Widerspruch Folge zu leisten‹, von den
Menschen nämlich?“

„Leider ja, ich kenne den Unsinn, muß mich aber danach richten.“

„Nun denn, führen Sie uns in Ihr Zimmer, ich will Ihnen etwas
zeigen.“

Sie traten in das Sprechzimmer des Arztes.

„Können Sie martisch lesen?“ fragte Saltner.

„Ich habe es einigermaßen lernen müssen.“

„Dann sehen Sie sich das an.“ Er zeigte seinen Paß.

„Erkennen Sie an, daß mir danach alle Rechte eines Numen
ausnahmslos zuerkannt sind?“

„Ich muß es anerkennen.“

„Demnach befehle ich Ihnen, meine Mutter und mich sogleich aus
diesem Hause zu entlassen.“

Der Arzt sah ihn verdutzt an. Dann blinzelten die Augen unter
seiner Brille, und ein vergnügtes Schmunzeln ging über sein ganzes
Gesicht. Endlich lachte er und rieb sich die Hände.

„Das ist gut!“ rief er. „Das nenne ich den Jäger in seiner eignen
Falle gefangen. Ja, wenn Eure Numenheit befehlen, so muß ein armer
Bat ja folgen. Aber um meiner Sicherheit willen möchte ich mir den
Befehl doch schriftlich ausbitten.“

Er rückte Papier und Feder zurecht.

Saltner schrieb eilig in martischer Sprache: „Auf Grund der
Verordnung des Instruktors von Südtirol vom 18. September kommt Dr.
Frank, in Vertretung des Direktors des Laboratoriums, meinem Befehl
nach, Frau Marie Saltner aus der Anstalt zu entlassen. Josef
Saltner, Ehrenbürger der Marsstaaten. Bozen, am 20. September.“

Frank verbeugte sich und nahm das Papier in Empfang. Er schüttelte
Saltner die Hand und sagte: „Nun wünsche ich recht glückliche
Reise, denn Sie werden sich wohl auf einige Zeit aus der Nähe
verziehen. Ich begleite Sie bis vors Haus.“

Langsam stiegen sie die Treppe hinab, denn Frau Saltner fiel das
Gehen noch immer schwer. Da kam ihnen ein Diener eilig entgegen.

„Herr Doktor“, rief er, „eben kommt der Instruktor vor die Tür
gefahren. Er wird gleich hier sein.“

Saltner stand erstarrt. Im letzten Augenblick sollte er scheitern?

„Haben Sie nicht einen Nebenausgang, durch den Sie uns führen
können?“ fragte er schnell.

Frank verstand. „Kommen Sie“, sagte er. Und zu dem Diener: „Sagen
Sie dem Herrn Instruktor, ich würde sofort zur Stelle sein. Sie
sehen, ich bin eben bei einer Kranken.“

Damit faßte er Frau Saltner unter den andern Arm, und sie gingen
schnell durch einen Korridor nach einer Nebentreppe und durch
einige Wirtschaftsräume in den Hof. Hier führte eine kleine Tür auf
einen schmalen Weg, der sich hinter dem Haus zwischen den
Weingärten hinzog. Schnell schloß Frank die Tür hinter Saltner und
seiner Mutter zu und eilte ins Haus zurück.

Jetzt tat Eile not.

„Wir müssen uns eilen, Mutter“, sagte er, „damit wir fortkommen,
denn in unserm Haus dürfen wir nicht bleiben. Ich habe einen Wagen
bestellt, wir wollen über die Berge, wo der Oß nichts mehr zu sagen
hat. Ich will dich deshalb das Stückchen tragen.“

„Du wirst es schon recht machen“, sagte sie.

Er nahm sie auf den Arm wie ein Kind und schritt rasch und ohne
Beschwerden zwischen den Mauern dahin. Der Weinhüter kam ihm
entgegen. Als er ihn erkannte, grüßte er ehrerbietig und öffnete
ihm die Türen. So kam er schnell an die Stelle, wo der Wagen hielt.
Kathrin saß schon darin, Rieser stand selbst bei den Pferden.
Saltner hob seine Mutter hinein, Kathrin wickelte sie in eine Decke
und bot ihr Wein an.

Saltner schwang sich auf den Bock. Der Weinhüter war herangetreten.
Hier kannte ihn jeder und liebte ihn, keiner hätte ihn verraten.
Saltner beugte sich zu dem Mann herab und sagte: „Die Nume sind
hinter uns her, sie dürfen uns nicht kriegen.“

„Schon recht“, sagte der Hüter, „ich habe nichts gesehen, hier ist
niemand gewesen.“ Damit tauchte er wieder in das Dunkel der Mauern.
Die Pferde zogen an, der Wagen rollte auf der Straße nach Meran
davon.

Saltner sprach zurück in den halbgedeckten Wagen. Er erkundigte
sich, wie Kathrin ihre Aufträge ausgerichtet habe. Palaoro war zu
Hause gewesen, er hatte gesagt, zwei zuverlässige Leute, die
besten, die da seien, würden gern mit ihm kommen, weil es für den
Herrn Saltner sei. Aber ob er die Maultiere gleich bekommen würde,
wüßte er nicht, doch werde er sein Möglichstes tun. Der Herr
Saltner möge sich nur nicht sorgen, wenn es etwas spät in der Nacht
würde. Dann wäre sie nach Hause gelaufen und hätte die Sachen
zusammengepackt. Als sie gerade wieder hinten zum Hause
hinausgewollt, hätte es vorn gepocht. Da hat sie das Licht schnell
ausgelöscht und zum Guckfenster hinausgeschaut. Dort ist der Wagen
des Herrn Instruktor gestanden, und noch eine Menge von Fahrrädern
mit den großen elektrischen Lampen sind dagewesen und wohl zehn
Leute mit Glockenhelmen, die haben ins Haus gewollt. Da ist sie
schnell hinten hinaus und hat die Tür verschlossen und ist zum
Rieser gelaufen, und der ist auch gerade mit dem Wagen gekommen.

Die Häuser des Ortes lagen hinter den Flüchtlingen. Die Nacht war
klar, und eine Spur von Dämmerung erleuchtete den Weg. Saltner
besprach sich mit dem Besitzer des Fuhrwerks und setzte ihm
auseinander, worauf es ankäme. Sobald Oß die Entführung aus dem
Laboratorium erfahren haben würde, und das war jetzt natürlich
schon geschehen, würde er sie jedenfalls verfolgen lassen. Er
konnte zwar nicht wissen, ob sie sich nicht in Gries versteckt
hielten, aber er würde jedenfalls auch seine fahrenden Gendarmen
die Hauptstraßen entlangschicken. Diese mußten mit ihren schnellen
elektrischen Rädern auf den glatten Chausseen den Wagen bald
einholen. Sie durften also nicht auf der Chaussee bleiben, wenn
auch die Fahrt auf diese Weise viel länger dauern mußte. Hatte Oß
nach den umliegenden Ortschaften telephonisch den Befehl gesandt,
sie aufzuhalten, so war ihnen die Nachricht doch in jedem Fall
vorangeeilt. Man mußte dann sehen, wie man durchkam.

„In der Hinsicht“, sagte Rieser, „brauchen Sie nichts zu
befürchten, wenn nicht gerade ein Nume in Andrian ist. Aber wie
sollte da einer hinkommen? Der Vorsteher sieht gern durch die
Finger, wenn er den Numen ein Schnippchen schlagen kann. Sie setzen
sich dann in den Wagen, und wenn ich mit dem Mann gesprochen habe,
wird er Sie gar nicht erkennen.“

Sie hatten jetzt die Straße verlassen und verfolgten einen
schlechten Feldweg, zwischen Obst- und Weingärten oder Rohrfeldern.
Die Schwierigkeit lag aber darin, über die Eisenbahn und die Etsch
hinüberzukommen. Dazu mußten sie bis Sigmundskron heran, und hier
galt es vorsichtig sein. Die Mitte des Bozener Bodens war noch
nicht erreicht, als sie hinter sich, wo die Straße nach Meran sich
etwas erhöht am Berg hinzieht, die unverkennbaren Lichter der
Martier sich in schneller Fahrt in der Richtung nach Terlan
hinbewegen sahen. Gleich darauf bemerkten sie auch vor sich
Lichter, die in derselben Richtung wie sie auf den dunkel
vorspringenden Felsen von Sigmundskron hineilten. Sie waren aber
auf der Chaussee ihnen bereits voraus und verschwanden bald hinter
den Bäumen und Baulichkeiten des Orts.

„Nun so schnell wie möglich ihnen nach“, rief Saltner. „Die fahren
sicher den Berg hinan, um zu sehen, ob wir über die Mendel wollen.
Bis sie zurückkommen, muß der Weg frei sein.“

„Sie werden aber nicht weit fahren“, sagte Rieser. „Denn das wissen
sie doch, daß sie uns in der ersten halben Stunde einholen müssen,
wenn sie auf dem richtigen Weg sind.“

„Wir müssen unser Glück versuchen.“

Ohne aufgehalten zu werden, passierten sie den Ort und den Fluß und
waren glücklich an der Stelle vorüber, wo links die Straße nach dem
Mendelpaß abgeht. Sie wandten sich rechts, um am Gebirge entlang
ihr Ziel zu erreichen. Jetzt durften sie hoffen, keinem Verfolger
mehr zu begegnen. Die Straße führte hier ein großes Stück
geradeaus, das sie schon zurückgelegt hatten, und Saltner spähte
vorsichtshalber noch einmal rückwärts. Da bemerkte er plötzlich,
wie hinter ihnen das elektrische Licht eines Rades auftauchte. Es
näherte sich nur langsam, da der Weg kein schnelles Fahren
gestaltete. Sie wurden verfolgt.

Saltner verlor die Geistesgegenwart nicht. Er sah, daß es nur ein
einzelner Bed war, der diese Straße einschlug; vielleicht hatte man
ihm gesagt, daß ein Wagen diesen Weg gefahren sei. Er durfte es
nicht darauf ankommen lassen, daß der Wagen erkannt wurde. Der Bed
hätte Hilfe herbeigeholt, und man hätte ihn jedenfalls noch in
Andrian erreicht. Er fühlte nach der Telelytwaffe, die er vom Mars
mitgebracht und heute zu sich gesteckt hatte. Ohne Rieser etwas von
dem Verfolger zu sagen, rief er ihm nur zu: „Fahren Sie weiter, ich
komme gleich nach!“und sprang während der Fahrt vom Wagen.

Er mußte den Bed abhalten, ihnen zu folgen, aber er durfte ihn auch
nicht zurückkehren lassen, um zu melden, daß er durch einen
Überfall verhindert worden sei, die Verfolgung fortzusetzen;
vielmehr mußte er es so einrichten, daß der Bed an einen zufälligen
Unfall glaubte. Und er hatte schon unterwegs daran gedacht, wie er
das einrichten könne. Saltner sprang hinter einen Baum, der ihn
gegen das Licht der Laterne deckte. Die Telelytwaffe ließ sich
ausziehen, daß man wie mit einem Gewehr genau zielen konnte. Das
Rad mit dem Bed näherte sich hell beleuchtet und mochte noch etwa
hundert Schritt entfernt sein. Saltner setzte eine kleine
Sprengpatrone ein und zielte an die Stelle, wo der diabarische
Glockenhelm von den beiden dünnen Stützen getragen wird, die ihn
mit der Fußbekleidung verbinden. Wird diese Verbindung
unterbrochen, so ist die Diabarität aufgehoben, da der zu
schützende Körper nach beiden Seiten gegen die Richtung der
Schwerkraft gedeckt sein muß. Es kommt beim Gebrauch des Telelyts
nicht wie bei einem Schuß auf eine einzige Entladung an, sondern
man kann die Wirkung, die sich wie das Licht in Ätherwellen
fortpflanzt, einige Zeit wirken lassen. Saltner war daher sicher,
wenn er auch bei den Schwankungen des Helms einigemal das Ziel
verlor, doch den Sprengerfolg zu erreichen. Und in der Tat, nach
fünf bis sechs Sekunden begann der Helm sich zu neigen, und die
eine Stütze brach. Der Bed hielt erschrocken sein Rad an. Diesen
Moment der Ruhe benutzte Saltner, um auch die andere Stütze zu
sprengen. Der Helm fiel herab, und der Bed bückte sich sichtlich
unter dem Druck der Erdschwere. Er konnte jedenfalls so bald weder
vorwärts noch rückwärts weit gelangen und war mit seinem Unfall so
beschäftigt, daß er nicht mehr auf den Weg achtete.

In schnellen Sprüngen eilte Saltner dem Wagen nach. Ohne ein Wort
zu sagen, schwang er sich wieder auf den Bock. Eine Stunde später
traf der Wagen in Andrian ein. Rieser ging voraus und überzeugte
sich, daß hier noch keine Nachforschungen angestellt seien. Der
Wirt brachte die Frauen in seiner eignen Wohnung unter, und auch
Saltner legte sich einige Stunden zur Ruhe, um die Ankunft der
Führer abzuwarten.

Um drei Uhr wurde er geweckt. Palaoro war mit zwei Führern und den
Maultieren eingetroffen. Alles wurde sogleich zum Aufbruch
vorbereitet. Frau Saltner fühlte sich vollkommen kräftig, die
Befreiung von ihrer Angst hatte ihr aufgeholfen. Nur die Füße
konnte sie nicht gut gebrauchen, aber auf dem bequemen Sattel des
Maultiers hatte sie keinerlei Beschwerden. Es war noch finster, als
der kleine Zug aufbrach und auf schmalen Pfaden durch eine enge
Schlucht zur Stufe des Mittelgebirges hinaufklomm. Sie waren erst
ein kurzes Stück vorwärts gekommen, als Palaoro in seine Tasche
griff und zu Saltner sagte:

„Da habe ich doch noch etwas vergessen. Gehen Sie nur ruhig
vorwärts, ich hole Sie bald wieder ein.“

Er schritt gemächlich den Weg zurück. Der Wirt, der zugleich
Ortsvorsteher war, trat eben ins Haus, um sich noch ein wenig aufs
Ohr zu legen, als Palaoro herankam.

Er überreichte ihm eine Depesche und sagte: „Das hat mir diese
Nacht der Postmeister in Terlan mitgegeben. Er hatte nach allen
Richtungen Boten ausschicken müssen, so daß er keinen mehr an euch
hatte; da hab ich gesagt, ich wolle das Telegramm mitnehmen.
Beinahe hätt’ ich’s vergessen. B’hüt euch Gott.“

Und schon war er mit raschen Schritten in der Dunkelheit
verschwunden. Der Wirt ging langsam ins Zimmer und entfaltete beim
Schein der Laterne das Telegramm. Es lautete:

„Josef Saltner mit Frau Marie Saltner und Katharina Wackner sind,
wo sie auch auf diesseitigem Gebiet betroffen werden, zu verhaften
und sogleich der hiesigen Gerichtsstelle zuzuführen.“

Der Ortsvorsteher faltete das Papier zusammen und sprach: „Das
hätte halt nachher schon vorher kommen gesollt.“ Dann ging er
wieder zu Bett.

Die Flüchtenden hatten das Mittelgebirge überschritten und
kletterten jetzt auf halsbrecherischen Pfaden die steilen Abstürze
des Gantkofels hinauf. Immer mit gleicher Sicherheit ging Palaoro
voran, die Saumtiere folgten an der Hand ihrer Führer mit festem
Tritt, und Saltner beschloß den Zug. Die Sonne ging auf und
vergoldete die Bergspitzen. Ohne Rast, den Abgrund zur einen, die
Felswand zur andern Seite, setzten die Reisenden ihren Anstieg
fort. Nach vier Stunden war der Rücken erreicht, mit welchem das
Mendelgebirge steil gegen das Etschtal abbricht. Dieser Rücken ist
die deutsch-italienische Sprachgrenze und das Ende des Oß’schen
Machtbereichs.

Menschen und Tiere blieben stehen und erholten sich. Der Blick
hatte sich nach Süden und Westen geöffnet. Auf den schlanken
Pyramiden der Presanella, auf den ewigen Schneemassen der
Ortler-Alpen glänzte strahlend das Sonnenlicht. Drunten im Tal
zogen Nebelstreifen, und über ihnen ruhten dunkel die bizarren
Formen der Dolomiten.

Saltner winkte einen Gruß zurück ins Tal.

„Auf Wiedersehen“, rief er, „wenn die Nebel vergangen sind. Jetzt
sind wir frei!“

Noch eine Viertelstunde mäßig bergab. Dann tat eine grüne, schmale
Talschlucht sich auf, von einem frischen Gebirgsbächlein
durchrieselt. Auf dem Rasen winkte eine Schutzhütte auf einem
verborgenen, selten besuchten Platz. Palaoro schloß auf.

„Hier werden wir wohnen“, sagte Saltner, indem er seine Mutter vom
Maultier hob, „bis das Recht wieder eingezogen ist in unser Land.“

„Wo könnte es schöner sein?“ sagte sie. „Und du bist hier.“

\section{50 - Die Luft-Yacht}

Die Strahlen der aufgehenden Sonne vergoldeten ein prachtvolles
Luftschiff, das aus den äußersten Höhen des Luftmeers von Norden
her herabschießend jetzt seine Geschwindigkeit mäßigte und seine
glänzenden Schwingen ausbreitend langsam und majestätisch, in
geringer Höhe über den Wogen, der nördlichen Küste von Rügen
entgegenschwebte.

Die Fischer in ihren Booten und die Badegäste, die am Strand
lustwandelten, verfolgten das Schiff mit erstaunten Blicken. An den
Anblick von Luftschiffen waren sie gewöhnt, denn der direkte Weg
vom Nordpol nach Berlin führte hier vorüber, wenn auch freilich
diese Schiffe in viel größeren Höhen zu ziehen pflegten. Aber ein
derartiges Fahrzeug hatten sie noch nicht gesehen. Es war keines
der furchtbaren Kriegsschiffe, deren farblose Einfachheit nur die
drohenden Öffnungen der Repulsitgeschütze unterbrach, es war auch
keines der langen und breiten Postschiffe, die den Personenverkehr
vermittelten. Für eines der Boote, die den höheren Beamten der Nume
zur Verfügung standen, war es zu groß und prächtig. Es war in der
Tat ein Schiff, wie es bisher auf der Erde nicht verkehrt hatte,
eine Privatyacht, von einem reichen Numen zu Vergnügungsreisen
erbaut. Seine glatte Oberfläche schimmerte rot und golden, auf
beiden Seiten wie auf den jetzt ausgebreiteten Flügeln glänzte
weithin sichtbar der Name des Schiffes, als wäre er von riesigen
Edelsteinen gebildet, ein nach rechts offener Halbkreis. Wer
martisch zu lesen verstand, erkannte darin den Namen ›La‹.

In der Mitte des Schiffes, auf dessen unterer Seite, befand sich
ein kleiner Salon, ausgestattet in einer ebenso kostbaren als
einfach wirkenden Eleganz und mit jeder Bequemlichkeit, die
martische Kunst zu erdenken vermochte. Eine hier zum erstenmal
angewandte Konstruktion ließ nach beiden Seiten erkerartige Ansätze
so hervortreten, daß sie, ohne die Bewegung des Schiffes zu
verhindern, eine freie Aussicht nach den Seiten und nach unten
gestatteten. Auf einem freihängenden Polster, wie auf einer
Schaukel halb liegend, ruhte hier eine graziöse weibliche Gestalt
in bequemem Morgenanzug, den der mit glänzenden Deli-Kristallen
bedeckte Lisschleier umhüllte. Es war Se. Sie beugte den schlanken
Hals herab, um das Meer zu betrachten. Sobald sie den Kopf bewegte,
spielten die braunen Locken in den lichten Farben des Regenbogens.
Von Zeit zu Zeit betrachtete sie Einzelheiten durch ein Glas, dann
ließ sie wieder den Blick rückwärts über die schaumgekrönten Wogen
in die uferlose Ferne schweifen. Sie konnte sich an diesem
Schauspiel nicht sattsehen. Daß es so viel Wasser gab, Wasser und
immer Wasser auf dieser Erde, wie wunderbar kam es ihr vor, die bis
jetzt nur das eisschollenbedeckte und beschränkte Meer am Nordpol
erblickt hatte.

Eine leise Berührung ihrer Schulter ließ sie aufblicken. Die Herrin
dieses fliegenden Wunderbaus stand vor ihr.

„Da bist du ja, La“, rief sie, sich aufrichtend, der ihr
zunickenden Freundin entgegen. „Hast du endlich ausgeschlafen?“

„Ich bin auch nicht so früh eingeschlafen wie du. Ich glaube, du
träumtest schon, als wir gestern vom Pol abreisten.“

„Ich war furchtbar müde. Ich hatte ja den ganzen Tag gearbeitet, um
mich noch rechtzeitig für dich freizumachen. Ach, La, das war doch
einer deiner gescheitesten Gedanken, mich zu dieser Reise
einzuladen. Aber diese Eile! In der Nacht kommst du mit dem ›Glo‹
an, ganz unerwartet. Früh läßt mich dein Vater nach dem Ring holen,
und abends muß ich schon mit dir fort nach Deutschland. Ich habe
noch gar keine Zeit gehabt, dich irgend etwas zu fragen.“

„Weil du gestern gleich eingeschlafen bist.“

„Ich bin ganz starr über diesen fabelhaften Luxus, das heißt für
ein Luftschiff. Sonst ist es ja gerade so wie zu Hause, aber das
auf einem Schiff zu haben, das ist eben das Überraschende. Wie bist
du nur dazu gekommen?“

„Das hat mir alles der Vater geschenkt.“

„Und das konnte er?“

La nickte.

„Aber du siehst gar nicht so vergnügt aus, wie es sich für eine
solche Prinzessin schickt. Komm, setz dich her und gestehe! Was ist
eigentlich mit euch vorgegangen? Ich versuchte vorhin in dein
Zimmer zu kommen, aber ich glaube gar, du hast es mit einer
akustischen Tür geschlossen, die nur auf das Stichwort aufgeht.“

La lehnte sich auf die schwebenden Polster und blickte zur Erde
hinab. Dann sagte sie:

„Du siehst, wir sind reiche Leute geworden. Der Vater hat eine
wichtige Erfindung gemacht, eine Verbesserung am
Fortbewegungsmechanismus der Raumschiffe.“

„Das weiß ich natürlich, den Fru’schen Gleitrepulsor, der das
Repulsit noch einmal so stark ausnutzen läßt. Das erspart dem Staat
Hunderte von Millionen im Jahr.“

„Nun ja, und einige davon haben wir als Ehrengabe bekommen. Dafür
hat mir der Vater dies schöne Schiff geschenkt und ein Reisejahr
für die Erde. Ich freue mich sehr darüber.“

„Wenn du es nicht sagtest, würde man es kaum glauben. Was hast du
also noch für Sorgen?“

„Weißt du, Se, schreiben oder in die Ferne sprechen kann man solche
Sachen nicht. Drum hab ich dich vor allen Dingen abgeholt, denn das
mußt du doch erfahren, daß wir mit Oß nicht mehr verkehren.“

„Aber Oß ist doch an der Erfindung deines Vaters beteiligt, er war
ja sein Assistent bei den Versuchen?“

„Ja, leider. Er hat auch vom Staat seine Million bekommen, und das
ist eben das Unglück, das ist ihm in den Kopf gestiegen.“

„Wieso? Ein bißchen exzentrisch freilich war er ja immer. Weißt du
noch? Damals am Pol, als Ill die Versammlung abhielt und Grunthe
und Saltner fortgegangen waren, da beantragte er doch, den Menschen
die persönliche Freiheit abzusprechen. Aber was hat er denn
getan?“

„Es war damals nach dem Friedensschluß mit der Erde, als der Vater
die Versuche machte, und Oß war deshalb viel bei uns, wir hielten
uns auf der Außenstation am Nordpol des Mars auf. Und da wollte er
mich binden.“

„Im Spiel? Ja? Nun, das ist doch noch kein Größenwahnsinn. Wer war
denn dabei?“

„Ich wollte aber nicht.“

„Und das hat er übelgenommen, das kannst du ihm nicht verdenken.
Warum wolltest du nicht?“

„Ich – ich war nicht in der Stimmung. Aber er hat das falsch
verstanden. Ich machte mir eben gar nichts aus ihm, und er bildete
sich ein, mir wäre das Spiel zu wenig. Er kam mit einem Antrag

„Im Ernst?“

La bewegte den Kopf bejahend. Ihre Augen blickten in die Ferne
hinaus, aber sie sah nichts von der anmutigen Landschaft, den
buchengekrönten Kreidefelsen zu ihren Füßen.

„Und du hast ihn abgewiesen? La! Das ist freilich schlimm. Das geht
doch nicht. Du mußtest das Spiel annehmen und dann so unausstehlich
sein, daß er von selber –. Aber La, du Liebling, ich glaube gar, du
weinst?“

Sie zog sie an sich und streichelte ihr die Wangen.

„Warum regt dich das so auf, macht dich so traurig? Du bereust? Du
liebst ihn? Darf ich es wissen?“

„Wirklich nicht“, sagte La mit so ruhiger Stimme, daß Se an ihrem
Wort nicht zweifeln konnte. „Ich konnte nicht anders, ich mochte
nichts von ihm wissen.“

„Ach so!“ Se faßte ihre Hand und drückte sie leise. „Also ein
anderer.“

Und bei sich dachte sie: „Also Ell!“ Aber das sagte sie nicht. Vor
solchen Gewissensfragen blieb auch die Freundschaft stehen.

La erhob sich heftig. „Lassen wir das nun“, sagte sie. „Es ist
nichts daran zu ändern. Ich hätte auch jeden andern abgewiesen –
das zu deiner Beruhigung. Ich wollte dir das nur mitteilen, damit
du dich nicht wunderst, wenn ich von Oß nichts mehr hören mag.“

„Und wo ist er denn jetzt?“

„Ich weiß es nicht, ich habe mich nicht darum gekümmert. Der Nu ist
groß. Er ist aus unserer Umgebung verschwunden.“

„Und deine Reise nach der Erde, nach Berlin? Hängt die damit
zusammen?“ fragte Se etwas neugierig.

„Indirekt ja. Ich habe mich über die ganze Sache geärgert. Ich war,
ich weiß nicht warum, in diesem Jahr recht wenig zufrieden mit mir.
Die Ärzte schickten mich hier- und dahin, aber ich war gar nicht
krank, ich war nur – ich weiß nicht. Da kam der Vater auf die Idee,
mich nach der Erde kommen zu lassen. Er mußte wieder hierher zu den
Erweiterungsbauten an der Außenstation. Und da schenkte er mir
vorher das schöne Schiff. Ich wollte die Mutter gern mitnehmen,
aber es wäre für sie zu anstrengend gewesen. Da dachte ich an dich.
Und nun hab ich dich ja.“

Sie küßte Se auf den Mund und sprach weiter: „Sei mir gut und tu
mir den einen Gefallen, wundere dich nicht über mich, ich weiß, was
ich tue, auch wenn es dir seltsam vorkommt. Ich will nämlich einmal
versuchen, wie es sich auf der Erde lebt, ob man überhaupt hier
leben kann.“

Se lächelte still für sich.

„In einem solchen Luftschiff läßt es sich schon leben“, sagte sie.
„Und im Palast des Kultors wird es sich wohl auch leben lassen.
Dort wirst du sicherlich diese La, ich meine die fliegende, in der
wir sitzen, unterbringen.“

„Nein, das werde ich nicht, ich will dir’s gleich verraten. Ich
habe nur dem Vater nicht widersprochen, als er es vorschlug. Aber
ich habe ganz andre Dinge vor. Ich will mir einmal die Bate in
ihrer Heimat ansehen, nicht als Nume, sondern wie ein Mensch möchte
ich unter Menschen verkehren. Wir wollen nicht in dem Schiff
wohnen, sondern in einem Hotel wie gewöhnliche Menschen.“

Se sah die Freundin erstaunt an.

„Was für Ideen du da ausheckst“, sagte sie. „Zur Abwechslung wäre
es vielleicht nicht übel, und ich wäre ganz gern dabei – wenn es
nur ginge. Aber die Schwere, La, die Schwere! Wenn wir als Menschen
auftreten wollen, können wir doch nicht mit den Helmen über dem
Kopf herumlaufen.“

„Könnten wir uns nicht ein bißchen an die Erdschwere gewöhnen? Ein
bißchen nur?“ fragte La, indem sie Se schelmisch ansah.

„Nein“, rief Se abwehrend, „dazu bekommst du mich nicht! Es ist ja
gar nicht dein Ernst!“

„Höre einmal“, sagte La, indem sie sich neben Se setzte und den Arm
um sie schlang. „Ich habe mir etwas ausgedacht und mir in Kla in
aller Stille anfertigen lassen. Darauf bin ich gekommen, wie ich in
einem Blatt die neuesten Moden auf der Erde gesehen habe. Sieh
einmal her.“

Sie holte vom Bücherbrett ein Journal der Erde und schlug es auf.

„Siehst du“, sagte sie, „man trägt jetzt diese merkwürdigen Hüte
mit breiten Krempen, die bis über die Schultern hinausragen, und an
beiden Seiten fallen Bänder herab. Ich vermute, daß unsre
diabarischen Glockenhelme das Muster dazu geliefert haben, unschön
genug sind sie dazu. Da dachte ich mir, so ein Hut müßte sich
diabarisch herstellen lassen, und ich ließ einige Modelle aus
Stellit anfertigen. Ich werde sie dir dann zeigen. Sie sehen aus
wie diese Hüte. Die Verbindung geht durch diese Bänder, die
allerdings an der Schulter befestigt werden müssen. Von dort geht
sie an den Seiten unter den Kleidern fort bis an die Stiefel, die
man aber unter den langen menschlichen Frauenkleidern nicht sieht.
Dieser Anzug schützt zwar nicht so gut wie der übliche Erdanzug mit
Helm, aber in der Hauptsache genügt er völlig. Nur die Oberkleider
und die Arme bleiben ohne Schutz, indessen das kann man schon
aushalten, es ist nicht so schwer; wir brauchen ja die Arme nicht
zu bewegen, sondern können sie meist am Gürtel oder an einem
Seitentäschchen aufstützen. Außerdem habe ich auch diabarische
Schirme gegen Sonne und Regen, die wir durch eine Stellitkette mit
dem Anzug verbinden können. Auf der Straße können wir also überall
ohne Beschwerde gehen, nur dürfen wir die Hüte nicht abnehmen. Aber
bei den menschlichen Damen ist es ja Sitte, bei vielen
Gelegenheiten auch im Zimmer die Hüte aufzubehalten.“

„Das ist fein. Man wird zwar gräßlich aussehen, doch wir sind ja
auf der Erde, da nimmt man es nicht so genau. Aber ich bitte dich,
wir können doch nicht zu Hause immer in Hüten sitzen und damit zu
Bett gehen.“

„Nein, das ist nicht zu verlangen. Trotzdem, im Schiff möchte ich
nicht wohnen, es braucht vorläufig niemand zu wissen, daß wir da
sind. Aber es gibt ja in Berlin Hotels für Nume, mit Zimmern, die
abarisch gemacht werden können. Dort mieten wir uns ein, daß wir
uns zu Hause erholen können. Das Schiff geht sofort weiter, daß die
Leute meinen, wir sind mit irgendeinem Mietschiff angekommen. Die
Schiffer nehmen mit dem Schiff in einem der Vororte Quartier, so
daß wir sie jederzeit herbeirufen können.“

„Das hast du alles sehr hübsch ausgedacht. Aber wie kommen wir denn
zu der nötigen menschlichen Toilette?“

„Das ist das wenigste! Es gibt doch in Berlin große Magazine, wo
man alles haben kann, was Menschen brauchen. Sobald wir im Hotel
angekommen sind, lassen wir uns von dort jemand kommen, und ich bin
überzeugt, in einer Stunde sind wir aufs Eleganteste
ausstaffiert.“

„Du bist gelungen! Was hast du für Ansichten von meinem
Geldbeutel!“

„Sei doch nicht töricht, Liebling. Du bist mein Gast, und ich habe
für dich zu sorgen. Das ist ganz selbstverständlich.“

„Nun, meinetwegen. Ich will dir deine Freude nicht verderben.“

„Ich danke dir, gute Se. Und nun komm, ich will dir die Hüte
zeigen. Wir wollen sie einmal probieren. Auf dem Verdeck ist
Erdschwere, und wir sind dennoch gegen den Luftzug geschützt.“

Die Probe wurde unter Lachen und Necken gemacht. Es ging alles nach
Wunsch, und Se erklärte, daß sie es wohl wagen würde, so spazieren
zu gehen. Aber Gesicht und Haar müßten unter einem Schleier
verborgen werden, und wenn sie so ein bißchen gebückt
einherhumpelten, werde man sie ja wohl für zwei alte Erdmütterchen
halten.

„Aber wenn wir Ell besuchen“, sagte sie fragend zu La, „da wirst du
doch nicht in diesem Aufzug hingehen?“

Sie waren wieder in den Salon getreten, und La war gerade damit
beschäftigt, ihren Hut abzulegen. Währenddessen antwortete sie
unbefangen: „Ell zu besuchen ist gar nicht meine Absicht.
Wenigstens nicht eher, als es die Höflichkeit unbedingt erfordert.
Weißt du, wen wir zuerst aufsuchen werden?“

„Nun dann vielleicht Grunthe?“

La lachte. „Das ist wahr“, sagte sie, „den müßten wir eigentlich
auch einmal heimsuchen. Aber im Ernst, ich will zuerst zu Isma. Wir
haben uns einigemal geschrieben.“

„Mir ist alles recht“, antwortete Se. Und nach einer Pause begann
sie ein wenig zögernd, indem sie La nur verstohlen betrachtete:
„Hast du denn eigentlich wieder einmal etwas von Saltner gehört? Er
ist doch so ohne Abschied vom Mars verschwunden.“

La ergriff das neben ihr liegende Fernglas und richtete es auf die
Landschaft. Dabei sagte sie mit möglichst gleichgültiger Stimme:

„Nur indirekt, hin und wieder. Er lebt, soviel ich weiß, bei seiner
Mutter da irgendwo in den Bergen. Übrigens hat er sich bei mir
verabschiedet, aber, du weißt ja, er hat sich damals auch mit Ell
überworfen wegen der Briefe –“

Se sah, wie Las Hand, die das Glas hielt, leise zitterte. Es war
unmöglich, daß sie etwas durch das Glas zu erkennen vermochte.

„Ach ja“, sagte Se, „ich weiß.“

Beide schwiegen. La sah wieder angelegentlich nach der Landschaft.
Se blickte zu ihr hinüber. Sie konnte aus der Freundin nicht klug
werden. Endlich sagte sie: „Übrigens, wenn wir ihn wiedersehen
sollten, die Bindung ist aufgehoben. Ich will nicht mehr.“

La antwortete nicht. Es war ganz still, man hörte das leise Zischen
der treibenden Maschine.

Plötzlich unterbrach der laute Pfiff einer Lokomotive die Stille.
Hundegebell wurde vernehmbar.

„Oh“, rief Se, „das ist Lärm, das ist die Erde!“

„Ich glaube, wir müssen schon weit über dem Binnenland sein. Ich
sagte dem Schiffer, er solle von Sonnenaufgang an ganz tief und
langsam fahren. Aber wir wollen nun etwas schneller vorwärts, die
Landschaft da unten ist recht eintönig.“

La rief den Schiffer. „Können wir in einer Stunde am Ziel sein?“

„In einer Viertelstunde, wenn Sie wollen.“

„Eine Stunde genügt.“ Der Schiffer ging.

„Wir wollen frühstücken und Toilette machen, ganz einfach“, sagte
sie zu Se.

Das Schiff zog die Flügel ein. Wie ein Pfeil durchschoß es die
Luft.

\section{51 - Martierinnen in Berlin}

In der glänzend ausgestatteten Vorhalle des neuen ›Marshotels‹ an
der Straße ›Unter den Linden‹ in Berlin standen zwei elegant
gekleidete Damen. In ihren gemessenen Bewegungen, mit denen sie die
Einrichtungen des Hotels aufmerksam musterten, machten sie einen
ebenso vornehmen Eindruck, als er dem Reichtum ihrer Toilette
entsprach. Ihr Gesicht war von einem dichten Schleier bedeckt, so
daß es schwer war, über ihr Alter ein Urteil zu gewinnen.

Als sie im Begriff waren, auf die Straße zu treten, näherte sich
ihnen ein Kellner und fragte ehrerbietig: „Befehlen die Damen
Plätze zur Table d’hôte?“

Se trat, entsetzt über diese Zumutung, einen Schritt zurück.
Schnell gefaßt sagte La:

„Wir können darüber noch nicht entscheiden.“

„Wagen gefällig?“ fragte der Portier.

La schüttelte nur den Kopf und ging vorüber.

Der Kellner und der Portier tauschten einen Blick, aus dem wenig
Hochachtung für die beiden Gäste sprach.

Die Damen schritten die Straße entlang nach dem Opernplatz zu. Sie
spannten ihre Sonnenschirme auf, und ihre Bewegungen wurden
sichtlich freier und lebhafter.

„Du hast doch nicht etwa die Absicht“, sagte Se leise, „wirklich
mit diesen Baten essen zu wollen? Das ist doch unmöglich.“

„Mit dem Hut und dem Schleier wird es nicht gehen, sonst aber – man
muß sich an alles gewöhnen.“

„Aber das ist doch zu unanständig.“

„Wir sind auf der Erde. In irgendeine der Restaurationen, die hier,
wie es scheint, in jedem Hause sind, wollen wir jedenfalls einmal
eintreten. Sieh nur, wo man hinblickt, sitzen Leute und trinken
Bier. Das nennen sie Frühschoppen.“

Sie schritten weiter durch das Gewühl der Menschen, über breite
Plätze, dann in engere, noch dichter belebte Straßen hinein. Ihre
Blicke schweiften über Gebäude und Denkmäler, über die begegnenden
Personen und Wagen oder verweilten auf den glänzenden Auslagen in
den Schaufenstern.

„Es gefällt mir gar nicht“, sagte Se. „Alles ist nüchtern, klein
und eng. Man sieht förmlich, wie die Schwere die Gebäude
zusammendrückt, die Dächer herabklappt. Die Wände, die Erker, alles
ist vertikal gezogen, eine horizontale Schwingung ins Freie scheint
es gar nicht zu geben. Sieh nur, wie dieser Balkon mühsam von unten
gestützt ist! Und wie ärmlich und geschmacklos all dies Zeug in den
Läden! Und das ist nun die Hauptstadt! Wie mag es auf dem Lande
aussehen? Denn diese ganze Herrlichkeit reicht nicht weit, selbst
wenn man zu Fuß geht, ist sie in ein paar Stunden zu Ende.“

„Du mußt doch nicht immer unsre Verhältnisse zum Vergleich
heranziehen“, entgegnete La. „Im ganzen ist es staunenswert, was
die Leute für ihre Kulturstufe leisten. Sie haben doch eine
Industrie. Natürlich müssen sie sich nach der Schwere richten und
können nicht wie wir in die Luft hinausbauen. Aber wie angenehm
kann man dafür hier im warmen Sonnenschein gehen, ohne verbrannt zu
werden. Und sieh nur, diese entzückenden weißen Wölkchen, wie sie
über den blauen Grund ziehen. Das gefällt mir besser als unser
ewiger grüner Baumschimmer oder der fast schwarze Himmel darüber.“

„Mir scheint, du willst dich zur Erdschwärmerin ausbilden. Mich
stößt schon dieser entsetzliche Lärm ab. Die Leute unterhalten sich
ja so laut, daß man es auf mehrere Schritte hört. Und dort zanken
sich gar zwei auf offener Straße. Auch die Wagen sind unausstehlich
geräuschvoll, man hört das Rollen der Räder auf weithin. Wie muß
das erst gewesen sein, als noch Pferde vor die Wagen gespannt
waren. Höre nur das unanständige Rufen der Wagenführer: He! He! Das
Klingeln und Pfeifen! Ich möchte mir die Ohren verstopfen.“

„Man gewöhnt sich daran.“

„Was kommt denn dort? Hoch oben sitzen Menschen, und unten ist ein
Tier mit vier Beinen. So was habe ich noch nie gesehen, das müssen
wir uns betrachten.“

„Es sind Reiter“, sagte La. „Sie sitzen auf Pferden. Es sieht gut
aus.“

„O nein, abscheulich! Diese Tiere, wie häßlich. Und wie das riecht!
O pfui! Komm, komm, das halte ich nicht aus.“

Aus der Tür eines Hauses trat ein Nume, mit dem großen, glänzenden
Glockenhelm über dem Kopf. Er schritt bis in die Mitte der Straße,
um sich nach seinem Wagen umzusehen. Ein Teil der Vorübergehenden
wich ihm in einem Bogen aus, andre, die gelbe Marken an der
Kopfbedeckung trugen, gingen zwar dicht an ihm vorüber, blickten
aber finster nach der andern Seite. Gerade jetzt waren die Reiter
bis hierher gelangt. Das Pferd des ersten scheute vor dem Helm des
Martiers, der, ohne an ein Ausweichen zu denken, in der Mitte der
Straße stand. Kerzengerade stieg es in die Höhe. Der gewandte
Reiter behauptete sich im Sattel, er wollte das Pferd an dem
Martier vorüberbringen. In unregelmäßigen Sätzen sprang es hin und
her und schlug aus. So drängte es in die Zuschauermenge hinein, die
sich schnell angesammelt hatte. Diese stob erschrocken auseinander,
auch La und Se wurden gestoßen, allgemeines Geschrei entstand.
Schreckensbleich sahen sie, in die Ecke einer Haustür gedrückt, der
Szene zu. Von den Sporen des Reiters getroffen, machte jetzt das
Pferd einen gewaltigen Satz nach vorn. Es streifte den Helm des
Martiers und riß diesen zu Boden. Die Reiter galoppierten davon,
und ein Hohngeschrei der angesammelten Straßenjugend begleitete die
Niederlage des Numen.

Wütend sprang der Nume in die Höhe, das Publikum beeilte sich, aus
seiner Nähe zu kommen. Ein Schutzmann hatte sich inzwischen
eingefunden und war dem Numen behilflich, in seinen Wagen zu
steigen.

„Wer waren die Reiter?“ fragte der Martier.

„Es waren Herren vom Rennklub.“

„Gut, diesem Unfug muß gesteuert werden.“

Der Nume fuhr davon.

„Das geht ja hier entsetzlich zu“, sagte Se schaudernd. „Man ist
seines Lebens nicht sicher. Ich gehe nicht weiter.“

„Nur noch bis an jene Ecke. Dort in der Restauration hinter den
großen Scheiben sehe ich Damen in Hüten sitzen, da wollen wir uns
ein wenig erholen. Und dann fahren wir direkt zu Isma.“

Sie traten in das reich ausgestattete Lokal ein und schritten
zwischen den Tischen, die Gäste musternd, hindurch, bis sie neben
einem der Fenster an einem noch unbesetzten kleinen Tisch Platz
fanden. Obwohl ihnen alle Verhältnisse fremd und ungewohnt waren,
so machte sie das doch in keiner Weise befangen; es waren ja nur
›Bate‹, die hier ihren barbarischen Sitten huldigten, und sie
wollten sich das nur einmal ansehen. So dachte wenigstens Se. Sie
rümpfte das Näschen und sagte:

„Eine furchtbare Luft! Diese Gerüche und dieser Lärm – wie kannst
du es nur hier aushalten.“

Das Gemisch von Düften nach Bier, Tabak und geräucherten Würstchen,
in Verbindung mit dem Geräusch der Stimmen, war für martische Sinne
betäubend.

„Wir können hier ein wenig das Fenster öffnen“, sagte La.

Sie befanden sich in dem großen Ausschank einer süddeutschen
Brauerei. Ein Kellner setzte unaufgefordert zwei Glas Bier vor sie
hin, und eine Kellnerin brachte ihnen die Speisekarte.

Se amüsierte sich. „Diesen Topf soll man austrinken?“ sagte sie.
„Aber wie macht man denn das, es ist ja kein Saugrohr dabei?“

La warf einen etwas verzweifelten Blick umher, dann hob sie das
Glas und sagte: „Wir müssen eben trinken wie die Menschen.“ Und sie
nahm einen tüchtigen Zug.

Se versuchte es gleichfalls, aber sie kam nicht recht damit zu
Rande. „Woher kannst du das nur?“ fragte sie lachend. „Ich glaube,
du hast dich auf deine Erd-Expedition vorbereitet!“

„Ich habe es wirklich eingeübt“, antwortete La. „ich habe mir nun
einmal vorgenommen, unter den Menschen so wenig wie möglich
aufzufallen.“

„Und das sagst du so ernsthaft – man möchte es wirklich glauben.
Nun, was steht denn auf dieser wunderbaren Speisekarte, die man mit
beiden Händen halten muß?“

„Ich werde nicht klug daraus. Doch, da –“ sie hielt inne, „– ich
werde mir – dies da –“

Ein wehmütiges Lächeln ging flüchtig über ihre Züge, dann wandte
sie den Kopf ab und blickte sinnend zum Fenster hinaus.

Se las die Stelle, die La mit dem Finger bezeichnet hatte, und warf
dann einen verwunderten Blick auf die Freundin. Sie suchte in ihrem
Gedächtnis, und nun hatte sie es gefunden. Ihre Augen blitzten
schelmisch auf, und plötzlich sagte sie, ganz mit Saltners Akzent:

„Ein Paar Geselchte mit Kraut, die wenn i’ hätt’, ’s wär’ schon
recht.“

La zuckte zusammen. Sie sah Se mit einem flehenden Blick an. Diese
ergriff ihre Hand und sagte, ihr Lachen unterdrückend: „Sei nicht
böse, liebe La, aber eine Nume, der bei der Erinnerung an ›ein Paar
Geselchte‹, die sie noch dazu nie mit ihren Augen gesehen hat, die
Tränen in diese schönen Augen treten, das ist doch ein Anblick, um
Götter zum Lachen zu bringen. Aber es ist wahr, diesen würdigen
Gegenstand müssen wir kennenlernen, aus Dankbarkeit an die lustigen
Zeiten. Und heute habe ich schon viel daraus gelernt“, setzte sie
im stillen für sich dazu.

Se bestellte. Und wieder mußte sie leise lachen. Sie sah sich mit
La und Saltner auf der Aussichtsbrücke des Raumschiffs stehen, als
sich die leuchtenden Flächen des Mars zum erstenmal vor den
Ankommenden im Sonnenschein ausbreiteten, und der Kapitän Oß, der
zu Saltners Ärger La nicht von der Seite wich, sagte: „Morgen
werden wir landen. Es ist ein hübscher Raumschifferglaube, daß der
Wunsch in Erfüllung geht, den man bei der Landung ausspricht; es
muß aber etwas Praktisches und etwas Kleines sein. Was werden Sie
denn sagen?“ Er blickte La schmachtend an, die aber nicht
antwortete. Da tat Saltner in seinem trockenen Ton den klassischen
Ausspruch von den Würstchen. La und Se hatten lange gefragt, was
denn dies sei, und er hatte sie immer mit diesem Geheimnis geneckt,
bis er es ihnen einmal erklärte, und dann war es eine scherzhafte
Redensart geworden.

„Das sind ein paar patente Frauenzimmer“, sagte ein Herr am
Nebentisch zu seinem Nachbar.

„Es sind Tirolerinnen, ich hab’ vorhin die eine sprechen hören“,
sagte der andre. „Sie sind gewiß von der Stürzerschen
Sängergesellschaft.“

Das Essen war gebracht worden. Die Würstchen dampften verlockend
auf den Tellern, nur nicht für die Freundinnen. Sie tauschten
verzweifelte Blicke miteinander.

„Es ist keine Waage unter dem Teller“, sagte La, „man weiß nicht,
wieviel man eigentlich zu sich nimmt. Willst du dir vielleicht
lieber etwas Chemisches geben lassen?“

„Ich bringe es überhaupt nicht fertig, vor allen diesen Leuten zu
essen. Ich schäme mich halbtot.“

„Es kommt ja kein Nume herein, und niemand kennt uns. Ich will dir
etwas sagen – entweder, oder! In dem Schleier können wir überhaupt
nicht essen. Wir drehen dem Publikum den Rücken zu und nehmen die
Schleier ab. Ich stelle mir jetzt vor, ein Mensch zu sein!“

Und mit einem kühnen Entschluß löste La den Schleier von ihrem
Gesicht und begann zu essen.

„Es ist wirklich gut“, sagte sie. „Es ist fett und schmeckt wie
Al-Keht. Versuch es nur!“

Se sah ihr gespannt zu. Sie bewunderte die Seelengröße der
Freundin, aber sie konnte sich nicht zu dem gleichen Opfer für die
Menschheit entschließen.

„Es ist zu viel“, sagte La.

„So wollen wir gehen. Die Leute sehen uns zu. Himmel, da draußen
geht ein Nume vorüber.“

Se drehte sich schnell um, indem sie den Schleier zu befestigen
suchte. Indessen bezahlte La, und sie verließen das Lokal.

\tb{}
Die beiden Herren waren ihnen gefolgt. Als Se und La auf der Straße
stehen blieben, um sich nach einer Droschke umzublicken, trat einer
der Herren an sie heran.

„Die Damen sind fremd und wissen den Weg nicht“, sagte er, den Hut
lüftend, „dürfte ich vielleicht die Ehre haben –“

Ohne ihn einer Antwort zu würdigen, wendeten sie ihm den Rücken zu
und setzten ihren Weg fort. Sie bemerkten alsbald, daß die beiden
ihnen unter anzüglichen Bemerkungen folgten.

„Das ist ja eine unverschämte Gesellschaft“, sagte Se, „es ist
wirklich recht nett hier unter den Baten, man kann sich nicht
einmal frei bewegen.“

„Du mußt bedenken“, bemerkte La entschuldigend, „das sind
ungebildete Leute, die nichts zu tun haben, sonst würden sie um
diese Zeit nicht im Gasthaus sitzen. Dort drüben stehen übrigens
Wagen.“

„Ich werde ihnen aber erst eine kleine Ermahnung geben. Paß auf,
wie sie verschwinden werden.“

Se nestelte an ihrem Schleier und blieb dann stehen. Als sich die
beiden Herren dicht hinter ihr befanden, drehte sie sich plötzlich
um und riß den Schleier herab. Der Glanz ihrer mächtigen Augen und
das Gebietende ihres Blickes zeigte den Abenteuerlustigen sofort,
daß sie vor einer Nume standen. Erschrocken prallten sie zurück.

„Macht, daß ihr in die Schule kommt!“ rief sie ihnen zu.

Beide entfernten sich aufs schleunigste.

Se lachte. „Aber nun habe ich wirklich Hunger“, sagte sie. „Isma
muß mir etwas zum Frühstück verschaffen.“

Eine Droschke brachte sie vor das Haus, wo Isma wohnte. Enttäuscht
sahen sie sich um, nachdem sie den Hof überschritten hatten. Kein
Aufzug im Haus, und drei Treppen! Es war eine mühsame Partie. Se
seufzte wiederholt.

„Man braucht ja nicht in einem solchen Haus zu wohnen oder nicht so
hoch“, sagte La begütigend.

„Man braucht glücklicherweise überhaupt nicht auf der Erde zu
wohnen, sollte ich denken.“

„Nun ja, ich meinte nur, wenn man – zum Beispiel amtlich –“

„Ach so.“

Endlich standen sie vor der Tür, welche die Aufschrift ›Isma Torm‹
trug. Sie hatten nun ihre Schleier abgenommen. Auf ihr Klingeln
öffnete sich die gegenüberliegende Tür, und eine ältere,
freundliche Dame sagte, Frau Torm sei nicht zu Hause. Jetzt
erkannte sie, daß sie zwei Damen vom Mars vor sich habe und
erschöpfte sich in Entschuldigungen. Frau Torm werde sogleich nach
Hause kommen, es sei jetzt ihre Zeit, und die Damen möchten nur
einen Augenblick warten, es sei alles geimpft im Haus, und sie
werde sie sogleich in Frau Torms Zimmer führen. Das geschah denn,
und die unterhaltende Dame ließ sie nun allein.

Die beiden Martierinnen sahen sich sorgfältig in dem freundlichen
und geräumigen Zimmer um. In dem lebensgroßen Porträt an der Wand
erkannte Se sogleich Ismas Gatten, dessen Bild ihr in allen
Schriften über die Erde begegnet war. Mit besonderem Interesse
betrachtete La die Einrichtung im einzelnen, nur irrte sie sich,
wenn sie glaubte, etwa hier den Typus des Wohnzimmers einer
deutschen Hausfrau vor sich zu haben. Denn wenn auch die Tätigkeit
der weiblichen Hand unverkennbar war, so enthielt doch die
Einrichtung nicht nur viele Züge des Studierzimmers eines Mannes,
sondern auch allerlei, was von den landesüblichen Gewohnheiten
abwich und an den Einfluß des Mars erinnerte. Da waren zahlreiche
Kleinigkeiten, die von Ismas Planetenreise erzählten, eine
Fluoreszenzlampe über dem Schreibtisch hing an einem Lisfaden, so
daß sie in der Luft zu schweben schien, ein Bücherbrett war ganz
nach martischem Muster eingerichtet, und es fehlte sogar nicht der
Phonograph, ein Geschenk Ells.

Unter den Drucksachen, die auf dem Tisch lagen, fiel La ein
Flugblatt auf, das in mehreren Exemplaren vorhanden war. Es trug
die Überschrift: ›An die Menschheit!‹ und begann mit den Worten:
„Numenheit ohne Nume! Das sei der Wahlspruch des allgemeinen
Menschenbundes, den wir aufrichten wollen unter allen Kulturvölkern
der Erde.“

La las weiter. Der Inhalt fesselte sie. In feurigen Worten war die
ideale Kultur der Martier gepriesen, aber ebenso entschieden
eifriger Protest erhoben gegen die Form, welche ihre Herrschaft auf
der Erde angenommen hatte. „Ergreifen wir“, so hieß es, „was sie
uns bieten, mit klarer Einsicht und offenem Herzen, so werden wir
ihrer selbst nicht mehr bedürfen. Zeigen wir, daß wir das große
Beispiel ihres Planeten begriffen haben, eine Gemeinschaft freier
Vernunftwesen zu bilden, in der die Ordnung herrscht nicht durch
die egoistische Gewalt einzelner Klassen, sondern durch das
lebendige Gemeinschaftsgefühl aller. Das neue Zeitalter ist
vorbereitet. Der Mars hat uns den gewaltigen Dienst geleistet, uns
zu zeigen, wie die Not des Daseins bezwungen werden kann durch eine
reichere Ausbeutung der Natur und eine größere Selbstbeherrschung
der Menschen. Er hat die historischen Fesseln gebrochen, die uns
verhinderten, die neuen Ideen in der Menschheit lebendig zu machen.
Er hat die Völker geeint in dem gemeinsamen Bewußtsein, daß sie als
Kinder der Erde zusammengehören und ihre häuslichen Streitigkeiten
zu begraben haben, um die Kräfte des Planeten zusammenzufassen; er
hat uns gezeigt, daß es gilt, dem überlegenen und geeinten Planeten
zu begegnen, nicht um ihn zu bekämpfen als einen Feind, sondern um
seiner Freundschaft würdig zu werden und ihn als Bundesgenossen zu
begreifen. Menschliche Wissenschaft und menschliche Arbeit möge
unser Leben mit dem Bewußtsein durchdringen, daß es nur nötig ist,
dem Gesetz der Vernunft zu folgen, um auch unsern Willen auf der
Höhe des sittlichen Ideals zu halten. Wagen wir es zu denken und an
uns selbst zu glauben! Fahren wir fort zu lehren und zu lernen,
damit wir verstehen, was menschliche Freiheit erfordert! Und aus
der Vertiefung des befreiten Menschengefühls heraus einigen wir uns
in einem großen geistigen Bund, daß wir von uns sagen können: Hier
ist die Menschheit, hier ist die Gemeinschaft des Willens, uns frei
unterzuordnen dem Gesetz der Vernunft, hier ist die Erde, um dem
Mars die Bruderhand zu reichen! Und dann, das laßt uns mit
Gewißheit glauben, wird der ältere Bruder uns ebenso frei die Hand
entgegenstrecken und sprechen: Ihr seid würdig der Freiheit, die
Ihr Euch gewonnen habt, nehmt sie hin, wir verzichten freiwillig
auf unsre Herrschaft. Unser Ziel ist erreicht, wenn Ihr Menschen
seid.“ – Daran waren Mitteilungen über die Organisation des Bundes
geknüpft.

Indessen hatte Se in den Zeitungen geblättert, als sie plötzlich
ausrief: „Höre, La, hier steht etwas, das dich interessieren wird,
von Oß und Saltner –“

La griff nach dem Blatt. Noch hatte sie kaum die Stelle gefunden,
als die Tür sich öffnete und Isma eintrat.

Ihre Überraschung war groß und die Begrüßung lebhaft. Und doch
fühlte sich Isma befangen. Warum hatte ihr Ell nichts von dem
bevorstehenden Besuch gesagt? Sie fühlte sich freier, als sie im
Lauf des Gespräches vernahm, daß Ell gar nichts von dem Eintreffen
Las wußte, und gewann bald die Überzeugung, daß es nicht der Wunsch
war, Ell wiederzusehen, der La nach der Erde geführt habe. La
erzählte von ihren Eindrücken und Erlebnissen auf dem Weg vom Hotel
zu Isma, und Se erhielt die ersehnte Kräftigung. Dann brachte Se
das Gespräch auf den Zeitungsartikel über Oß und Saltner.

Es war darin gesagt, daß auf Veranlassung des Instruktors für
Bozen, Oß, der bekannte Forschungsreisende Saltner steckbrieflich
verfolgt werde wegen öffentlicher Anreizung zum Ungehorsam gegen
die Gesetze, Widerstands gegen den vorgesetzten Numen, Bedrohung
des Instruktors, Mißbrauchs amtlicher Papiere und Befreiung von
Gefangenen. Es war noch hinzugefügt, daß hoffentlich die Schwere
der Anklage sich nicht bestätigen werde, da es bekannt sei, daß
gegen den Instruktor Oß selbst eine Untersuchung wegen
Überschreitung der Amtsgewalt schwebe und seine Abberufung
bevorstehe. Saltners Aufenthalt sei unbekannt, doch werde von allen
Behörden aufs angelegentlichste nach ihm geforscht.

La sagte kein Wort. Sie suchte ihre Erregung zu beherrschen. Aber
das Herz schlug ihr in Angst und Sorge. Gewiß hatte Oß Saltner
gereizt, um seine Rache an ihm nehmen zu können. Und sie fühlte
sich schuldig als die geheime Ursache dieser Gegnerschaft. Sie
horchte mit Bangigkeit auf die Erklärungen, die Isma jetzt auf Ses
Frage gab.

Ell hatte Isma am Tag vorher besucht, an demselben Tag, an welchem
er genauere Nachricht über die Vorgänge in Bozen erhalten hatte und
auch die ersten Mitteilungen an die Zeitungen gelangt waren. Die
Klagen über Oß waren zuerst beim Unterkultor in Wien erhoben
worden. Dieser befand sich in der schwierigen Stellung, daß er
amtlich dem Kultor des gesamten deutschen Sprachgebiets in Berlin
verantwortlich, in der Durchführung seiner Anordnungen aber an die
Zustimmung der politischen Oberbehörde, nämlich an das
österreichische Ministerium gebunden war. Infolgedessen konnte er
nicht ohne weiteres die Suspendierung des Oß von seinem Amt
verfügen, sondern es wurden Verhandlungen mit der Wiener Regierung
notwendig. Von dort aus konnte erst an Ell berichtet werden.

So waren mehrere Tage seit der Flucht Saltners vergangen, ehe Ell
von derselben erfuhr. Nun wurde auf Grund der Klage der Behörden
und der Einwohner des Bozener Instruktionsbezirks die
Disziplinaruntersuchung gegen Oß erhoben und der Unterkultor in
Wien angewiesen, sich persönlich nach Bozen zu begeben. Man konnte
annehmen, daß er heute daselbst eintreffen würde. Aber für Saltner
wurde der Stand der Sache dadurch nicht gebessert. Seine
Selbsthilfe war vom Standpunkt der Martier aus eine
Gesetzesverletzung, die eine eindringliche strafrechtliche
Verfolgung erforderte, weil man die Autorität der Nume unbedingt
aufrechterhalten wollte. Ell konnte daher nicht anders handeln, als
die Maßregeln zu bestätigen, durch welche die Verhaftung Saltners
erzielt werden sollte. Isma berichtete ausführlicher über die
Beschuldigung, die von dem Instruktor gegen Saltner erhoben wurde.
Danach erschien es, als hätte Saltner den Instruktor auf offner
Straße insultiert, die Einwohner zum Widerstand aufgefordert, seine
Mutter und die Magd endlich durch einen raffinierten Betrug aus dem
Laboratorium entführt.

Se dachte im stillen: Wie gut, daß man nicht weiß, was er schon auf
dem Mars verbrochen hat! La jedoch sagte mit künstlicher Ruhe: „Man
wird doch erst hören müssen, wie sich die Sache von Saltners Seite
aus ansieht.“

„Gewiß“, erwiderte Isma „und ich kann Ihnen auch darüber Auskunft
geben. Ell hat nämlich gestern einen Brief von Saltner selbst
erhalten, worin er ihm offen seine Handlungsweise darlegt und ihn
um Hilfe gegen die ihm drohende Verfolgung angeht.“

„Einen Brief? So weiß man also, wo er sich aufhält? So ist er in
Sicherheit?“

„Das kann man nicht sagen. Der Brief ist auf einer Station zwischen
Bozen und Trient aufgegeben. Die dortigen Einwohner sind natürlich
alle auf Saltners Seite und werden ihn nicht verraten. Jedenfalls
hat einer der Führer oder Träger, die ohne Zweifel bei Saltners
Flucht beteiligt waren, den Brief zur Station gebracht. Saltner
selbst hält sich wahrscheinlich im Hochgebirge auf irgendeiner
versteckt liegenden Hütte auf.“

Isma erzählte nun, was Saltner getan hatte, nach seiner eigenen
Schilderung in dem Brief, den Ell ihr gestern vorgelesen hatte.

Se schüttelte den Kopf und sagte: „Das sieht alles Saltner ganz
ähnlich. Aber die Sache steht doch recht schlimm. Wenn man ihn
bekommt, wird es ihm sehr übel ergehen.“

„Warum?“ fuhr La plötzlich auf. „Ich glaube jedes Wort, was Saltner
schreibt, und dann hat er sich gar nichts zuschulden kommen lassen.
Er hat Oß nicht angegriffen und sich seinen Befehlen nicht
widersetzt, denn es waren ihm noch keine zur Kenntnis gekommen; und
die Befreiung seiner Mutter hat er auf einem Weg bewirkt, der rein
formell nicht anzugreifen ist.“

„Ell ist doch anderer Ansicht“, erwiderte Isma. „Er entschuldigt
zwar Saltner, der in seiner Lage und nach seinem Charakter nicht
wohl anders handeln konnte, aber er glaubt doch, daß man ihn
verurteilen wird. Und jedenfalls muß er dem Gesetze freien Lauf
gestatten, und, so leid es ihm tut, Saltner aufheben lassen.“

La erblaßte in heimlicher Angst. „Und wie glaubt man seiner habhaft
zu werden?“ fragte sie.

„Ganz leicht wird es ja nicht sein, aber in einigen Tagen bekommt
man ihn sicher. Nur wenige der dortigen Führer kennen seinen
Aufenthalt, und von ihnen verrät ihn keiner. Auch die Kenner der
dortigen Berge werden sich nicht dazu hergeben. Nume können
überhaupt nicht auf diese Höhen steigen und die verborgenen
Schluchten durchsuchen. Aber der Wiener Unterkultor hat ein
Luftboot zur Verfügung, und auch Ell würde nicht anders können, als
ein solches bereitzustellen. Dann lassen sich die Berge mit
Leichtigkeit absuchen, und es ist nicht denkbar, wie Saltner
entkommen sollte.“

„Wenn er aber doch entkäme?“

„Wohin reichte die Macht der Nume nicht?“

„Es handelt sich zuerst nur um die Behörden des Mars auf der Erde.
Auf dem Nu selbst hört jede obrigkeitliche Gewalt der Kultoren oder
Residenten auf. Dann müßte erst der Zentralrat selbst die
Auslieferung beschließen. Und selbst dieser könnte nicht in den
Privatbesitz, in das Haus eindringen, um den Besitzer zu
verhaften.“

„Ich weiß wohl, aber wie sollte Saltner auf den Nu gelangen? Und
wenngleich, die Frage ist ja eben, ob man dem Paß, den Saltner
besitzt, die Bedeutung zuerkennt, daß ihm auch jetzt noch die
Rechte eines Numen zukommen. Man könnte ihn für ungültig
erklären.“

„Es gibt ein unverletzliches Asyl“, sagte La leise, den Blick wie
in weite Ferne gerichtet.

Isma verstand sie nicht. Se sah die Freundin an, als traute sie
ihren Ohren nicht. Dann legte sie ihr liebkosend die Hand auf die
Schulter und sagte lächelnd:

„Ich glaube, du siehst nun wieder zu schwarz. Saltner kann
überhaupt nur so weit verfolgt werden, als das martische
Schutzgebiet auf der Erde effektiv ist. Er wäre also in
außereuropäischen Staaten schon sicher, denn um von dort eine
Auslieferung zu erzwingen, wären Maßregeln erforderlich, die man um
einer solchen Kleinigkeit willen nicht ergreifen wird. Und was Ell
nicht geradezu tun muß, das wird er auch nicht tun.“

„Das glaube ich ja“, sagte Isma. „Unter uns gesagt, Ell äußerte
sich gestern: ich wünschte nichts mehr, als daß wir Saltner nicht
fänden, dann wird der Prozeß in absentia geführt, und in einem Jahr
kann bei der Amnestie die Sache eingeschlossen werden.“

„Nun denn, so wollen wir uns nicht weiter Sorge machen. Saltner
wird sich schon zu helfen wissen. Sagen Sie uns lieber, was wir bei
dem schönen Wetter hier machen sollen.“

„Ich möchte doch wissen“, sagte La zögernd, „wann etwa die
Verfolgung Saltners durch die Luftschiffe aufgenommen werden
könnte.“

„Heute und morgen sicher noch nicht, das weiß ich“, entgegnete
Isma. „Denn Ell sagte, daß der Kultor erst die Verhandlung gegen Oß
zu führen hat, und solange behält er sein Schiff bei sich. Soll ich
noch einmal bei Ell anfragen?“ Sie wies auf das Telephon.

„Ach nein“, sagte La, „wir wollen uns noch gar nicht beim Herrn
Kultor melden. Nun machen Sie Ihre Vorschläge.“

„Das Wetter ist eigentlich zu schön für Berlin.“

„Ach ja“, rief Se. „Wir wollen lieber hinaus. Haben Sie heute
nachmittag für uns Zeit?“

„Bis heute abend, gewiß.“

„Was meinst du, La, dann sehen wir uns einmal Ihren deutschen Wald
dort in der Nähe von Friedau an, den Sie uns so schön geschildert
haben.“

La sann nach. Dann nickte sie und sagte: „Das ist mir sehr recht.“

„Aber wohin denken Sie“, rief Isma. „Dazu brauchen wir ja allein
fünf bis sechs Stunden Eisenbahnfahrt, um nur bis hin zu kommen.“

Jetzt lächelte La. „In zwanzig, in fünfzehn Minuten, wenn Sie
wollen, sind wir da. Machen Sie sich nur zurecht, Sie sollen
sogleich unsre Reisegelegenheit sehen.“

„Sie haben ein Luftschiff?“

„Und was für eins!“ lächelte Se. „Wenn wir wollen, holt uns das
größte Kriegsschiff nicht ein.“

\section{52 - Im Erdgewitter}

Aus den Wipfeln des weiten Bergwaldes ragt ein Felsvorsprung und
blickt hinab auf das grüne Tal und die sanften Höhenzüge, die es
gegen die Ebene abschließen. Hier, zwischen dem blühenden
Heidekraut, hatten La und Se sich gelagert, während Isma, auf den
Ast einer verkrüppelten Fichte gelehnt, träumerisch in das Land
hinausblickte.

„Dies gefällt mir am besten von allem, was ich bis jetzt auf der
Erde gesehen habe“, sagte Se, die violetten Blüten der Erika zu
einem Kranz zusammenfügend. „Und zwar darum, weil es so still, ganz
still ist, fast wie auf dem Nu.“

„Und vieles ist noch schöner“, fügte La hinzu. „Daß wir im milden
Sonnenschein hier sitzen können, über uns das wunderbare Licht des
Himmels! Wie leichte Federn ziehen die weißen Wolkenstreifen dort
oben ihre zierlichen Figuren, und wie seltsam es sich da hinten
ballt über der dunklen Wand, die der sinkenden Sonne
entgegensteigt. Ach, seht doch, was ist das, drüben auf der Wiese
am Rande des Waldes? Ein vorsintflutliches Geschöpf.“

„Es ist ein Hirsch“, sagte Isma, „der auf die Wiese tritt. Sehen
Sie, wie er den Kopf hebt und die Luft einzieht, ob alles sicher
ist. Ach, er verschwindet wieder, vielleicht hat er uns bemerkt.
Übrigens, die Wolken gefallen mir am wenigsten. Es sieht aus, als
sollten wir ein Gewitter bekommen.“

„Ein Gewitter? Oh, davon haben wir gelesen. Das möchte ich einmal
erleben. Ich kann mir keine Vorstellung davon machen. Aber was
blicken Sie denn immer dort hinüber in die Ebene?“ fragte La.

„Sehen Sie dort hinten jenen dunklen Streifen?“ erwiderte Isma.
„Links davon erblicken Sie zwei Türme, das ist das Schloß von
Friedau. Und über dem Streifen – es ist ein bewaldeter Hügelrücken
– glänzt ein heller Punkt in der Sonne. Das ist die Sternwarte Ells
– –“

„Wo?“ rief La eifrig, nach ihrem Glas greifend. „Ja, ich sehe es
ganz deutlich. Den Turm und die Plattform des Hauses. Das möchte
ich einmal in der Nähe sehen. Es ist ja gar nicht weit.“

„Doch mehr als zwanzig Kilometer.“

„In drei Minuten sind wir drüben. Hätten Sie nicht Lust, Ihre
Heimat wieder einmal zu besuchen?“

„Jetzt?“ sagte Isma. „Was sollte ich dort? Alles würde mich nur
traurig stimmen. Nein, auf keinen Fall. Und noch dazu mit dem
Luftschiff, bei welchem die ganze Stadt zusammenlaufen würde. Oh,
Sie wissen nicht, wie man in Friedau über mich denkt.“

„Das ist schade. Ich möchte so gern –“ La zögerte einen Augenblick
und fuhr dann fort: „Ich möchte, offen gestanden, gern einmal mit
Grunthe sprechen. Wir hatten uns eigentlich vorgenommen, ihn zu
besuchen, nicht wahr, Se?“

„Natürlich“, sagte Se lächelnd. „Wir wollen einmal sehen, was er
für Augen macht. Und vielleicht weiß er, wo Sal –“

Sie unterbrach sich auf einen Blick von La.

„Ich aber muß, wie Sie wissen“, sagte Isma, „gegen sieben Uhr
wieder in Berlin sein, ich habe noch eine Vorlesung heute abend –
und jetzt – es ist schon fünf Uhr vorüber.“

„Nun, dann müssen wir Sie freilich nach Hause bringen. Oder noch
einfacher, wir können ja beides vereinigen – das Schiff führt Sie
nach Berlin und holt uns dann wieder hier ab. Es ist so schön hier,
und ich sitze sehr gern noch ein Stündchen im Freien.“

Isma überlegte. „Aber dann ist es doch besser“, sagte sie, „Sie
suchen einen geschützteren Ort auf, daß Sie eine Unterkunft finden
können, falls das Gewitter heraufkommt. Hier wäre es auch für das
Schiff nicht möglich, Sie während des Unwetters aufzunehmen, denn
dann ist alles dicht in Wolken gehüllt. Wollen Sie denn überhaupt
mit diesem auffallenden Schiff bei Grunthe ankommen?“

„Sie haben recht“, sagte La, „er ist imstande und macht sich vor
uns aus dem Staub, wenn wir ihn nicht überraschen. Sie kennen die
Gegend, geben Sie uns einen Rat, wo wir uns am besten wieder
abholen lassen können.“

„Sobald es dunkel ist“, antwortete Isma nach einigem Nachsinnen,
„findet Sie das Schiff nirgends besser als im Garten der Sternwarte
selbst. Dort hat sich Ill, als er Grunthe vom Pol zurückbrachte und
dann mit mir –, dort hat das Luftschiff Ills zwei Tage unbemerkt
von den neugierigen Friedauern gelegen.“

„Aber wie kommen wir dahin?“

„Wir fahren jetzt nach einer Stelle im Wald, von wo Sie in wenigen
Minuten nach einem bekannten Aussichtspunkt zu Fuß gelangen können.
Von dort fährt die Bahn nach Friedau, jede Viertelstunde geht ein
Wagen. In fünfundvierzig Minuten kommen Sie damit nach der Stadt
bis dicht an die Sternwarte. Daß auf der Sternwarte noch abends
Fremdenbesuch eintrifft, ist ja nichts Ungewöhnliches.“

„Gut, so wollen wir es machen. Von halb neun Uhr an soll mein
Schiff für uns im Garten der Sternwarte bereitliegen. Wenn Sie dem
Schiffer bei der Rückfahrt von weitem die Stelle zeigen und die
Lokalität ein wenig beschreiben, findet er sich zurecht. Er ist ein
sehr geschickter Mann. Nun lassen Sie uns zum Schiff gehen.“

Ein schmaler Fußweg zwischen dichtem, jungem Fichtengebüsch, auf
dem nur eine Person hinter der andern schreiten konnte, führte die
drei Damen nach einer Lichtung, wo die schimmernde Luftyacht ›La‹
ruhte. Kaum hatten sie diese betreten, als sie sich in die Lüfte
erhob und nach Ismas Weisung einem der bewaldeten Hügel zuflog, mit
denen der Höhenzug nach der Ebene hin abfiel. Hier fand sich wieder
eine Waldwiese, auf welcher das Schiff sich bequem niederlassen
konnte. Isma führte La und Se durch den Wald bis nach einem
sorgfältig gebauten Promenadenweg.

„Wenn Sie nun in dieser Richtung weitergehen“, sagte sie, „so sind
Sie in fünf Minuten an dem großen Gasthaus ›Zur schönen Aussicht‹,
und unmittelbar unter demselben liegt die Haltestelle der Bahn. Sie
können nicht mehr fehlen. Halten Sie sich aber nicht auf, denn das
Gewitter kommt näher, und auch ich muß mich eilen, damit ich vor
seinem Ausbruch fortkommen

„Seien Sie unbesorgt und reisen Sie glücklich!“ sagte La. „Wir
sehen uns bald wieder. Sind Sie einmal im Schiff, so kann Ihnen
kein Wetter etwas anhaben. Sie sind im Augenblick darüber oder so
weit, als Sie wollen.“

Nach herzlichem Abschied ging Isma durch den Wald zurück, während
La und Se auf dem bequemen Weg sanft bergab stiegen. Bald gelangten
sie an eine Bank, von welcher sich ein lieblicher Blick über den
Wiesengrund des Tales mit seinen Villen und kleinen Teichen und
weit in die Ebene hinaus eröffnete. La ließ sich nieder und sagte:
„Hier wollen wir so lange warten, bis wir das Schiff erblicken und
sehen, daß Isma glücklich abgereist ist.“

Längere Zeit saßen sie schweigend, während ihre Blicke bald über
das Land, bald über den Himmel schweiften. Der Sonnenglanz über der
Ebene war verschwunden. Nur die fernen Höhen im Osten leuchteten
noch in gelblichem Licht. Vergebens suchte La die Türme von Friedau
aus dem Gewirr der dunklen Flecken und Streifen herauszuerkennen.
Der Himmel hatte sich mit einer gleichmäßigen Schicht von Grau
überzogen, unter welcher jetzt von Westen her dunkelbraune
Wolkenmassen sich heranschoben.

„Das Schiff müßte längst sichtbar sein – ich glaube, wir dürfen
nicht länger warten“, sagte Se ängstlich, indem sie den drohenden
Himmel musterte.

„Ich glaube auch, wir warten vergebens“, antwortete La. „Sie werden
gleich bis über die Wolken gestiegen sein, und wir können sie daher
nicht sehen. Horch, was ist das?“

Ein dumpfes Rollen wurde vernehmlich, verstärkte sich und kehrte,
von den Bergen zurückgeworfen, mit erneuter Schärfe wieder.

Se faßte Las Arm. „Komm, komm“, sagte sie hastig.

La fühlte, wie ihr Herz lebhafter schlug, sie zwang sich, ruhig zu
bleiben.

„Wie wunderbar“, sagte sie, „das muß der Donner sein. Laß uns noch
lauschen.“

„Nein, nein, das ist nichts für mich.“

Ein Rauschen und Brausen kam durch den Wald. Plötzlich beugten sich
die Bäume unter der Gewalt eines Windstoßes, ringsumher wirbelten
Tannennadeln und dürre Zweige in einer Wolke von Staub. Die
Martierinnen griffen nach ihren Hüten und banden sie fester. Sie
zogen ihre fast unsichtbaren Listücher aus dem kleinen Futteral,
warfen sie über den Kopf und hüllten sich hinein. Lauter warnte der
Donner.

Von oben her ertönten eilende Schritte. Ein Herr, den Hut in die
Stirn gedrückt, mit einem Wettermantel um die Schultern, kam
schnell den Weg herab. Er grüßte, ohne die Damen genauer zu
beachten. Einige Schritte nachher drehte er sich noch einmal um. Er
wollte sie zur Eile mahnen, aber jetzt erkannte er, daß er
Martierinnen vor sich habe, und setzte seinen Weg ohne zu sprechen
fort.

Der Wind hinderte La und Se an der Bewegung. Jetzt hörte er
plötzlich auf, und sie schritten schnell aus. Der Weg zog sich in
engen Windungen bergab; an der Stelle, an welcher sie sich
befanden, hatten sie jetzt das Wetter mit seinen finstern
Wolkenmassen vor sich.

„Das sieht schrecklich aus“, sagte Se.

Sie hatte noch nicht ausgesprochen, als sie sich mit einem leichten
Schrei zurückwarf und an Las Arm klammerte, die ebenfalls
erschrocken stehenblieb. Ein blendender Blitzstrahl war drüben
jenseits des Tales niedergefahren. Während sie noch erschüttert
standen, begannen einige große Tropfen zu fallen, und nun kam der
Donner mit knatternden Schlägen, die sich in ein langes Rollen
auflösten, und ehe noch der Widerhall geendet, zuckte ein neuer
Blitz, näher und stärker – –

Sie sprachen nicht mehr, sie liefen den Weg hinab. Jetzt brach der
Regen in mächtigem Guß los, im Augenblick war der Weg mit
rieselnden Bächlein bedeckt.

„Ich kann nicht mehr!“ stöhnte Se.

La blieb stehen und sah sich um. „Da, dort!“ rief sie.

Der Weg machte wieder eine Windung. Hier stand, mit dem Blick ins
Tal, ein kleiner Pavillon, nur aus Fichtenstämmchen kunstlos
aufgezimmert und mit Baumrinde bedeckt; aus den ausgesparten
Fenstern hatte man dieselbe Aussicht wie oben, nur beschränkter,
jetzt aber blickte man auf nichts als strömende Wassermassen. Hier
fand man wenigstens einen notdürftigen Schutz gegen den Regen. Die
Freundinnen eilten in die Hütte.

Als sie eintraten, erhob sich von der Bank an der einzigen Seite,
die gegen den Regen und Wind geschützt war, der Herr, der vorhin an
ihnen vorübergegangen war.

„Oh, ich bitte“, sagte La, „lassen Sie sich nicht stören, wir
finden schon Platz.“

Der Herr verbeugte sich nur höflich, verließ aber die Hütte. Er
stellte sich vor derselben neben die Tür unter das vorspringende
Dach.

Ein Blitz, dem betäubender Donner im Moment folgte, ließ die
Martierinnen zusammenschrecken, sie sanken erschöpft auf die
hölzerne Bank.

„Das ist schrecklich“, sagte Se mit bebender Stimme. „ich zittere
am ganzen Körper. Ich will nichts mehr wissen von dieser Erde!“

La nahm ihre Hände zwischen die ihrigen.

„Zage nicht“, sagte sie. „Es ist leicht, ein freier Nume sein, wo
wir herrschen über die Natur und mächtig leben wie die Götter. Aber
hier, in der Gewalt der sinnlosen Mächte, die uns fremd sind und
ungewohnt, müssen wir den Mut des Willens erweisen. Sieh, dieser
Mensch hat uns seinen Platz eingeräumt, uns, die er vielleicht
haßt, und er steht draußen im Sturm und blickt furchtlos in das
tobende Wetter. Was der Mensch kann, muß auch ich können, oder ich
bin nicht wert der Erde. Und das will ich sehen!“

Sie erhob sich und trat an die Brüstung des offenen Fensters, unter
welcher der Fels ins Tal abfiel, so daß gerade nur noch die Wipfel
der hohen Tannen bis herauf reichten. Wind und Regen schlugen von
der Seite herein, La kümmerte sich nicht darum. Die Schulter an die
Pfosten des Fensters gelehnt, stand sie hochaufgerichtet, den
Elementen trotzend, in ihren Lisschleier gehüllt, dessen Zipfel der
Sturm zerzauste. Ihre großen Augen richteten sich gegen den Himmel,
als wollte sie den Wetterstrahl herausfordern. Und wie zürnend über
die Verwegenheit der Nume öffnete sich die Wolke, und die feurigen
Schlangen züngelten nach dem Talgrund, und gleichzeitig dröhnte ein
Donnerschlag, der die Luft erzittern machte.

Geblendet und betäubt hatten alle einen Moment die Augen
geschlossen.

„La, La“, rief dann Se, „was fällt dir ein, was soll das heißen?“

La stand aufgerichtet wie zuvor an ihrem Platz. Sie schüttelte
stolz das Haupt und sprach heiterer als vorher, fast jubelnd:

„Ich kann es, ich kann’s!“

„Wozu das alles!“ rief Se. „Komm her zu mir, du wirst völlig naß.“

„Ich will es. Dieser junge Planet tobt wie ein Jüngling in Launen
und Übermut, nicht achtend der Geschöpfe, die er behüten soll.
Unser Nu ist ein Greis, der uns verwöhnt hat in seiner sicheren
Ruhe. So verwöhnt, daß wir die Gefahr suchen mußten draußen im
Weltraum. Auf der Erde ist die Jugend mit ihrem Wetterunfug, mit
ihrer blinden Torheit, mit ihrem schwankenden Wechsel von Leid und
Glück. Zum Leid ward sie mir, zum Glück soll sie mir werden!“

Sie schwieg, noch einmal vom Rollen des Donners unterbrochen. Aber
sie hatte den Blitz nicht mehr gesehen, das Wetter war über ihrem
Haupt hinweggezogen. Se antwortete nicht. Das Geschick der Freundin
stand vor ihrer Seele wie eine Frage, deren Antwort mächtiger und
immer deutlicher sich ihr aufdrängte und die sie sich dennoch nicht
zu geben wagte. Jetzt lauschte sie wieder auf den Donner, dessen
Stärke sich verringert hatte. Sie fühlte sich freier. Der Nachlaß
der elektrischen Spannung oder die Entfernung der Gefahr, sie wußte
nicht, was es war, aber sie atmete auf. Ihr Blick richtete sich
nach dem Weg, wo sie das Knirschen von Tritten vernahm. Der Fremde
entfernte sich. Er hatte den Hut in die Hand genommen, deutlich sah
sie sein Profil, als er jetzt, einen Blick nach den Wolken werfend,
um die Ecke des Weges bog. Und wie ein Aufleuchten der Erinnerung
durchzuckte es sie. Das Bild hatte sie gesehen, oft gesehen, und
erst heute, die große Kreidezeichnung über Ismas Schreibtisch – nur
freilich, älter sah dieser Mann aus, abgehärmter und dennoch, es
konnte nicht anders sein dots{} doch es war ja nicht möglich – –

Sie wollte etwas zu La sagen. Aber diese stand ganz in den Anblick
der Gegend versunken. Und nun fing La aufs neue zu sprechen an, nur
mit ihrem eigenen Gedankengang beschäftigt. Und Se wandte wieder
der Freundin allein ihre Aufmerksamkeit zu.

Wie in einer stillen Freude begann La:

„Sieh, der Regen wird sanfter, drüben über dem Wald wird’s hell.
Und dort über dem Land, o welch ein frohes Wunder, in bunten Farben
flammt der Bogen über den Himmel, und grollend zieht der Donner
unter ihm hinweg.“

Se stand auf und trat neben La. Die Schritte des Fremden waren
längst verhallt. Sie schlang den Arm um die Freundin und fragte:
„Was ist es mit dir, La? Ich verstehe dich nicht!“

La blickte schweigend in die Ferne, wo die untergehende Sonne und
der abziehende Regen in wundersamer Farbenschlacht sich bekämpften.
Dann zog sie Se an sich und sagte: „Ich liebe die Erde.“

Se blickte ihr in die Augen. „Es wird wohl nicht die ganze Erde
sein“, sagte sie mit stillem Lächeln. „Komm, wir wollen uns auf
diese Bank setzen – der Regen rieselt noch immer im Gebirge –, bis
die Wasser von dem Weg sich ein wenig verlaufen haben, und du wirst
mir beichten, was du darfst, oder wenigstens, was du vorhast; denn
ich ahne wohl, was du fühlst, aber das Ganze, Ungeheure, was du zu
wollen scheinst, vermag ich nicht zu begreifen.“

„Du vermagst es wohl nicht zu begreifen“, sprach La mit kaum
hörbarer Stimme, indem sie Se folgte. „So hab ich auch eine, es ist
die der Vernunft im zeitlosen Willen, daß ich sein soll und daß wir
das eine, dasselbe Ich sein sollen – das ist die oft zu mir gesagt,
und wer vermöcht’ es wohl, der es nicht erlebt? Aber nun weiß ich,
daß es so sein muß. Glaube nicht, ich hätte vergessen, daß ich eine
Nume bin. Ich habe gekämpft um meine Freiheit, um meine Würde, und
mit bittern Tränen hab ich sie mir errungen, glaubt’ ich sie mir
errungen mit jenem Abschiedskuß in Sei. Ein Marsjahr ist
dahingegangen seitdem, zweimal hat die Erde ihren Sonnenlauf
vollbracht, aber frage nicht, wie ich die Zeit durchlebte! – Ich
habe mich aufgerieben in diesem nutzlosen Kampf. Ich hatte ja nicht
gesiegt, ich war geflohen vor mir selbst. Freiheit und Würde hatte
ich nicht gewonnen in meiner Seele, nur Weltraum und Sonne, die
trennenden Mächte der Planeten, hielten mich in dem leeren Schein,
daß der Nu meine Heimat und ich eine Nume sei. So lebt’ ich, mich
selbst betrügend und verzehrend, bis der Morgenstern wieder
leuchtete. Da trieb es mich her. Würde des Numen! Ist sie noch
Würde, wenn sie erhalten wird durch den äußeren Zwang? Nein, Se, es
wurde mir klar, Würde wie Freiheit wiedergewinnen konnte ich nur,
wenn ich selbst mich hingab, um sie in dieser Welt des Scheines zu
verlieren. Und wie ein Zeichen heiliger Bestimmung wurden mir die
Mittel der Macht, die in meine Hände gegeben war. Versuchen wollt’
ich, ob ich auf der Erde das sein kann, was der Geringsten Eine
unter den Menschen ihm hier sein könnte. Ihm! Se, dies eine Wort
verstehst du nicht – ihm? Warum ihm? Das ist das Geheimnis, das
unauflösliche, das weder Menschen noch Nume wissen. Ihm, weil ich
bin, weil wir so wollten, ehe noch Mars und Erde vom uralten
Sonnenschoß sich trennten. Ein lächerlicher Zufall, daß ihm der
Leib gebildet ward in diesem, mir in jenem Abstand vom Sonnenball!
Die Bestimmung ist nur Liebe. Dieser Bestimmung folgen ist
Freiheit. Dieser Bestimmung genügen ist Würde. Ich habe die Erde
versucht, ich kann ich gehe jetzt hin, ich hole ihn und rede zu
ihm, hier bin ich, und anders kann ich nicht sein. Als Nume oder
als Mensch, wie du mich haben willst, ich bin La, deine La. Und
nun, meine Se, schilt nicht, lästre nicht, es nutzt nichts. Komm
mit, laß uns zur Station hinabsteigen, Grunthe soll mir sagen, wo
er ist.“ ihren Mächten trotzen. Und damit du’s weißt, was ich will
–

„Ja, wer denn?“

„Wer? Es gibt nur einen Menschen.“

\section{53 - Schwankungen}

Die Wohnung Torms auf der Gartenstraße in Friedau stand noch immer
verschlossen, die Jalousien vor den Fenstern waren herabgelassen,
niemand betrat die Räume. Isma hatte sich nicht entschließen
können, die Wohnung aufzugeben, es war ihr, als gäbe sie damit die
letzte Hoffnung auf, noch einmal in ihre Häuslichkeit
zurückzukehren, als raubte sie ihrem Mann das Heim, das er
vielleicht in Schmerz und Elend suchte und ersehnte.

Und dennoch lebte Torm in Friedau in tiefster Verborgenheit. Er
wohnte bei Grunthe. Es war nichts Ungewöhnliches, daß fremde
Gelehrte sich längere Zeit auf der Friedauer Sternwarte zu Studien
aufhielten und dann Ells Gäste waren. So fiel es auch den wenigen
nicht auf, die darum wußten, daß bei Grunthe ein ausländischer
Astronom wohnte, der sich nirgends in der Stadt sehen ließ.

Torm war am Tag, nachdem er von Grunthe die erschütternden
Nachrichten über die Umwandlung der Verhältnisse in Europa erhalten
hatte, nach Berlin gereist. Die Sehnsucht trieb ihn, zu Isma zu
eilen, ihr die Sorge, die Trauer um den Verschollenen zu nehmen,
glücklich bei ihr zu sein und mit ihr vereint dann zu erwarten, was
sein Geschick über ihn bestimmen werde, wenn seine Rückkehr bekannt
geworden sei. Das war ja doch das Natürliche, zu ihr gehörte er, um
zu ihr zu gelangen hatte er sich in die neuen Gefahren gestürzt und
– in die Schuld. Seine Zweifel waren zerstreut, sein Vertrauen
zurückgekehrt. Wenn sie ihn nicht liebte, wenn sie nicht fest an
ihm hielt, was hätte sie gehindert, ihn zu verlassen, um den
mächtigen Freund zu wählen? Was sie offen tun konnte, warum hätte
sie es heimlich tun sollen? Nein, sie hatte es nicht getan, und da
sie es nicht getan, was ging Ell ihn an? Nicht zu Ell wollte er,
sondern zu ihr. Aber ohne vorherige Nachricht. Erst mußte er mit
ihr besprechen, was zu tun sei, wie sie es halten wollten, ehe
jemand erfahren durfte, daß er gerettet sei, wo er sich aufhalte.
Und in diesem Sinne hatte er Grunthe gebeten, das Geheimnis seiner
Wiederkehr zu bewahren.

Wie würde er Isma antreffen, wie würde sie ihm begegnen? Er konnte
sich kein Bild davon machen, vergebens versuchte er sich im Beginn
seiner Fahrt das Wiedersehen auszumalen. Noch immer lag der
Gedanke, als ein Geächteter zu reisen, wie ein Druck auf ihm,
unwillkürlich sah er die Mitreisenden darauf an, ob sie ihn wohl
erkannten. Mitunter erschien er sich als ein Fremder, der sich eine
Entschuldigung ersinnen müsse, um seinen Besuch zu rechtfertigen,
und er mußte sich erst daran erinnern, daß er als der Gatte zu
seiner Frau fahre, die seit zwei Jahren seine Wiederkehr erhoffte.
Dazwischen stellte er Betrachtungen über das Verhalten der
Passagiere an. Es fiel ihm eine Änderung darin auf, die seit seiner
letzten Fahrt durch Deutschland auf der Eisenbahn vor sich gegangen
war. Das war vor dem Antritt seiner Entdeckungsreise gewesen, denn
bei seiner Wiederkehr war er als Triumphator empfangen, überall in
Extrazügen eingeholt worden und nicht als einfacher Passagier
gereist. Ein Typus der Reisenden war ganz verschwunden, der
anspruchsvolle, geringschätzig auf die andern herabblickende,
hochmütige Elegant. Man sah vornehme Leute, aber keine Überhebung;
nicht nur ein höflicher, fast ein kameradschaftlicher Ton herrschte
überall; die verschiedenen Berufsarten und Stände hatten sich unter
dem allgemeinen Druck näher aneinander geschlossen, suchten sich
besser zu verstehen. Und ebenso auffallend war das Fehlen aller
Uniformen.

In Halle kaufte sich Torm eine Zeitung, er gedachte, sich den
übrigen Teil der Fahrt damit zu unterhalten. Aber alsbald stieß er
auf eine Nachricht, die ihm neue Sorgen erweckte. Die Zeitung
brachte die erste Mitteilung über den ›Fall Stuh‹ bei Frankfurt, wo
die Bauern in einem Ort sich an dem durchreisenden Instruktor
vergriffen hatten. Es war hinzugefügt, daß bereits vier der
Tumultuanten als Rädelsführer verhaftet seien und der Instruktor
die Überweisung an den Numengerichtshof verlangt habe. In diesem
Fall fürchte man sehr strenge Strafen für die Schuldigen. Im
Anschluß hieran behandelte ein Artikel die Frage nach der
Gerichtsbarkeit, sofern in einer Streitfrage Nume in Betracht
kämen. In dem Friedensvertrag war festgesetzt, daß Nume überhaupt
nur von martischen Richtern abgeurteilt werden konnten, aber es war
nicht ganz klar, in welchen Fällen Menschen, die sich gegen Nume
vergingen, vor die martischen statt vor die Landesgerichte kämen.
Sicher war dies in politischen Prozessen, aber ob ein Tumult, wie
gegen Stuh, zu den politischen zu zählen sei, war fraglich. Es
wurde nun darauf hingewiesen, daß der Protektor der Erde, als
oberste Instanz in Kompetenzkonflikten, in einem ähnlichen Fall
entschieden hatte, daß es sich um eine Auflehnung gegen martische
Anordnungen zur Kulturleitung der Menschen und somit um ein
hochverräterisches Unternehmen handle, das vor das Numengericht
gehöre.

Es handelte sich um einen jungen Mann namens Erbelein, der wegen
Versäumnis der Fortbildungsschule ins psychophysische Laboratorium
geschickt worden war und sich hier den Anordnungen nicht gefügt
hatte. Aus einem Schrank mit Chemikalien hatte er eine Flasche mit
einem Narkotikum entwendet, seinen Beobachter eingeschläfert und
sich entfernt. Von zwei Beds verfolgt und eingeholt, hatte er
dieselben plötzlich überrumpelt und einen von ihnen nicht
unbedenklich verletzt. Dies war als offene Empörung angesehen und
mit der schwersten Strafe belegt worden, mit lebenslänglicher
Deportation nach einem Dorf der Beds in einer der ödesten
Wüstengegenden des Mars.

Durch diesen Präzedenzfall, der in den Hauptsachen ganz mit seiner
eigenen Verschuldung übereinstimmte, fühlte sich Torm schwer
betroffen. Das alles und noch mehr hatte er sich ja auch zuschulden
kommen lassen. Er hatte sich dem Spruch des Kriegsgerichts
entzogen, Sauerstoff entwendet, ohne Befugnis ein Luftschiff
benutzt, endlich einen Wächter bei Ausübung seines Berufes
niedergeschlagen. Er konnte sich nun die Frage beantworten, was ihm
bevorstand, wenn er für seine Handlungsweise zur Verantwortung
gezogen wurde.

Und nun entdeckte er in demselben Blatt eine weitere Notiz, die ihn
stutzen ließ. Es war darin gesagt, daß die Regierung des Polreichs
der Nume auf der Erde infolge der allgemeinen Teilnahme, die das
Verschwinden des hochverdienten Forschers Torm hervorgerufen habe,
nochmals in allen Teilen der Erde sorgfältige Nachforschungen nach
seinem Verbleib anstellen ließe. Es läge die Möglichkeit vor, daß
er auf eine noch nicht aufgeklärte Weise doch im Mai vorigen Jahres
den Pol verlassen habe und sich vielleicht in unzugänglichen
Gegenden oder bei wilden Völkerschaften aufhalte.

Es war nicht gesagt, daß er eines Verbrechens wegen verfolgt würde.
Aber war das nicht vielleicht bloß eine Vorsichtsmaßregel, sollte
ihm nicht eine Falle gestellt werden? Und wenn er nun plötzlich
auftauchte, würde man dann nicht die Anklage gegen ihn erheben? Die
Flucht vor dem Kriegsgericht mochte durch die Amnestie beim
Friedensschluß von der Anklage ausgeschieden sein, die
Gewaltätigkeit bei der Flucht in Tibet aber jedenfalls nicht. Wenn
diese neuen Nachforschungen jetzt erst geschahen, weil vielleicht
diese seine Tat erst jetzt den Martiern bekannt geworden war?

Torm ließ sein leichtes Gepäck auf dem Bahnhof und trat unschlüssig
in den leise herabrieselnden Regen hinaus. Zu Fuß verfolgte er den
weiten Weg nach Ismas Wohnung, gleichsam als wollte er dem Zufall
noch eine Bestimmung über sein Schicksal einräumen. Dabei musterte
er die eilend einherschreitenden Fußgänger, und je näher er dem
Osten der Stadt kam, um so unruhiger fühlte er sein Herz schlagen.
So oft eine schlanke Frauengestalt ihm begegnete, immer durchzuckte
ihn der Gedanke, ob es nicht Isma sein könnte, und wenn er die
fremden Züge erkannte, wußte er kaum, ob er sich enttäuscht oder
befreit fühlte.

Es war bereits dunkel geworden, als er die Wrangelstraße erreichte
und nach den Hausnummern spähte. Jetzt stand er vor Ismas Haus. Er
mußte sich entscheiden. Er schämte sich seiner selbst. So kam er
nach Hause, den die gebildete Welt als den Entdecker des wahren
Nordpols gefeiert hatte? Heimlich wie ein Flüchtling, der das Licht
des Tages scheut, der die Schwelle des Hauses zu betreten zögert?
War es denn sein Haus? Nein, auch sie war ja geflüchtet –. Und
seine Frau? War sie es denn noch? Nicht mehr nach dem Gesetz des Nu
– wenn sie nicht wollte! Aber sie wollte doch wohl! Nein, nein,
nicht mehr diesen Zweifel! Aber er! Was brachte er ihr? Den
sonnigen Schein des Ruhmes, darin er vor sie zu treten hoffte, um
mit ihr auf den Höhen des Lebens zu wandeln? Konnte er sie
zurückführen in das verlassene Haus, in die friedliche Heimat?
Brachte er ihr den Frieden und die Ruhe, und nicht vielmehr neue
Sorge und rastlose Flucht? Riß er sie nicht heraus aus einem
stillen Glück, aus einer sich begnügenden Tätigkeit, um sie in
unübersehbares Leid zu stürzen? Das alles zog noch einmal, in einen
Moment sich zusammendrängend, vor seinem Bewußtsein vorüber, und
schon wandte er den Fuß, um wieder in das Dunkel der Straße
zurückzutreten.

Da öffnete sich die Tür. Der Portier hatte ihn durch sein Fenster
vor der Haustür stehen sehen. „Zu wem wünschen Sie?“ fragte er
mißtrauisch.

„Wohnt Frau Torm hier?“ fragte Torm heiser.

„Jawohl, im hinteren Flügel, drei Treppen.“

„Wissen Sie vielleicht, ob sie zu Hause ist?“

„Jawohl, es ist eben Besuch nach oben.“

Einen Moment zögerte Torm. Dann sagte er:

„Ich will wiederkommen.“

Die Tür schloß sich hinter ihm. Langsam schritt er die Straße
hinauf. Besuch? Wer war es? Gleichviel – sie mußte allein sein,
wenn er sie wiedersehen wollte. Besuch! Und er, der totgeglaubte,
nach drei Jahren heimkehrende, der überall gesuchte Gatte, er ließ
sich abschrecken durch das Wörtchen Besuch! Das trennte ihn von
ihr, der Heißersehnten. Warum? Er schauderte vor sich selbst.

Warum? Weil er nicht sagen konnte, hier bin ich, dein Hugo, mit dem
das Glück wieder einkehrt am Herd! Weil sie nicht sagen konnte,
hier ist er, den ihr jubelnd bewillkommt habt, hier ist mein Gatte!
Weil er vor ihr stehen mußte als ein Verbrecher, über welchem das
Schwert hängt, die lebenslängliche Verbannung. Weil er seinen Blick
niederschlagen mußte vor ihr, als ein unbesonnener Verletzer des
Gesetzes! Weil er wieder fort mußte von ihr auf immer, oder sie mit
sich ziehen ins Elend, wenn sie ihm folgte in die Wüsten des
feindlichen Planeten –. Nein, nein, dann lieber diesen Schmerz ihr
ersparen! Dann lieber sie in dem Glauben lassen, daß er verschollen
sei, unter dem Eis, oder wo auch immer – –

Und so schritt er die Straße hinab und wieder hinauf, und fragte
sich nochmals, welcher Besuch? Und die Tür öffnete sich jetzt, und
der heraustrat – – es war Ell. Ja, er durfte bei ihr sein, er, der
ihn hinausgelockt in die Gefahren des Pols, er –. Und nun war es
ihm, als müsse er sich auf ihn stürzen –. Doch der sah ihn nicht,
er schritt ruhig, aufgerichtet voran, ein glänzender Wagen hielt in
der Nähe, er stieg hinein.

Torm wandte sich um. Wieder suchte er durch den Regen den Weg nach
dem Bahnhof. Der Nachtzug führte ihn nach Friedau zurück.

Er sagte Grunthe, daß er erst noch nähere Aufklärung über die
Absichten der Martier und das Schicksal des nach Tibet gegangenen
Schiffes abwarten wolle, ehe er es wage, sich zu erkennen zu geben.
Solange wolle er versuchen, verborgen zu bleiben. Bereitwillig bot
ihm Grunthe das abgelegene stille Asyl der Sternwarte zum
Aufenthalt an. Hier weihte er Torm in seine schon längst
vorbereiteten Bestrebungen ein, einen allgemeinen Menschenbund zu
gründen, der durch eine freiwillige Aufnahme der von den Martiern
gebotenen Kulturmittel sich von der Fremdherrschaft der Martier
unabhängig zu machen suchen sollte. Von hier aus reichten die Fäden
der durchaus nicht geheimgehaltenen Verbindung zu den führenden
Geistern aller Kulturstaaten. Hier entwarf Grunthe mit Torm den
Aufruf mit dem Motto: ›Numenheit ohne Nume!‹

Und sie trafen damit einen Ton, der in der Seele der Völker
widerhallte. In Millionen und Abermillionen Köpfen und Herzen waren
dieselben Gedanken, dieselben Gefühle mächtig, es bedurfte nur der
Anregung, um sie zur lebendigen Bewegung auszulösen. Das Wort war
gefunden und gesprochen. Die Menschen waren ja einig, weil sie es
sein mußten; es war nur erforderlich, daß sie es nun auch
freiwillig sein wollten. Nicht Verbrüderung aus Schwärmerei,
sondern gleiche Ziele aus Vernunft. Zahllos strömten die
Zustimmungen in den organisierten Zentren der Vereinigung zusammen.
Es war klar, daß der Menschenbund bald eine Macht werden mußte, mit
der man zu rechnen hatte. Alle politischen und wirtschaftlichen
Parteien konnten sich an der großen Kulturaufgabe beteiligen, die
er sich gestellt hatte, mit Ausnahme einer extremen Gruppe, deren
oligarchische Interessen vor dem bloßen Gedanken der
Gleichberechtigung aller zurückscheuten. Aber ihr Grollen war
unschädlich, weil ihr Einfluß auf die Regierung gebrochen war und
die Verlockung fortfiel, welche so viele nach Macht und Karriere
strebende Kreise der Bevölkerung verleitet hatte, die
kulturfremden, kavaliermäßigen Gewohnheiten nachzuahmen.

Und selbst Anhänger von Lebensanschauungen, denen der Gedanke des
Menschenbundes anfänglich höchst unsympathisch gewesen war,
begannen sich damit zu befreunden. Der Fabrikbesitzer Pellinger,
der sich leicht für alles begeisterte, was einem versöhnenden
Ausgleich dienen konnte, hatte sich den Bestrebungen des Bundes
eifrig gewidmet und gehörte bald zu den Vertrauensmännern Grunthes.
Seine Vermutung, daß der Fremde, der auf der Sternwarte wohnte,
niemand anders wie Torm sei, war ihm bald zur Gewißheit geworden,
als er ihm bei Grunthe begegnete. Er verbarg dies Grunthe nicht,
und dieser hielt es für das beste, ihm gegen Zusicherung der
Verschwiegenheit zu sagen, daß Torm allerdings hier sei, aber aus
politischen Gründen sich versteckt halten müsse.

Herr von Schnabel setzte Pellingers Bemühungen, ihn für den
Menschenbund zu gewinnen, zuerst hartnäckigen Widerstand entgegen.
Mit Leuten, die auf dem Standpunkt eines Ell ständen, könne er sich
nicht befreunden. Er liebte es, sich als einen besonderen
Verteidiger der Ehre des verschollenen Torm aufzuspielen, indem er
behauptete, daß Frau Torm durch Ell kompromittiert sei, der sich
der Verantwortung in feiger Weise entzogen habe. Und da Torm nicht
gegen Ell vorgehen könne, so müsse wenigstens, seiner Ansicht nach,
jeder anständige Mensch sich von Bestrebungen fernhalten, die
darauf hinführten, daß niemand mehr für seine Ehre mit der eignen
Person eintreten könne. Die Gerüchte über Frau Torm seien noch
immer nicht verstummt, und wenn Torm da wäre, so müsse er, ob es
nun verboten sei oder nicht, durch irgendeine Herausforderung Ruhe
schaffen.

Pellinger lachte ihn aus. Er könne ihn versichern, daß alle diese
Gerüchte auf gänzlicher Unkenntnis der Verhältnisse beruhten. Das
sei ganz gleichgültig, meinte Schnabel, man dürfe eben die Gerüchte
nicht dulden.

„So?“ sagte Pellinger. „Und was, meinen Sie, würde dadurch
gebessert werden, wenn Sie zum Beispiel dergleichen behaupteten und
Torm Sie forderte? Ich will jetzt einmal gar nicht von dem
unentschuldbaren Frevel sprechen, der in der kulturwidrigen
Einrichtung des Zweikampfes selbst liegt, sondern die Sache rein
praktisch betrachten. Wird denn dadurch irgend etwas bewiesen?
Würde man nicht erst recht sagen, es muß doch etwas Wahres dran
sein?“

„Jedenfalls würde man Achtung vor dem Mann bekommen.“

„Meiner Ansicht nach müßte man ihn verachten; denn er hätte eine
unsittliche Handlung begangen. Ein Mann wie Torm kann auf die
Achtung derer verzichten, die sie an so verwerfliche Bedingungen
knüpfen. Und so jeder Mann von sittlichem Ernst. Der schiene mir
verachtungswert, der nicht seine eigne Würde und das Bewußtsein
seines Rechts so hochschätzte, daß sie nicht gekränkt werden können
durch das Gerede des Pöbels in Glacéhandschuhen.“

„Na, na, Sie sprechen da in einer Weise, die – die etwas
eigentümlich –“

„Ja, Herr von Schnabel, ich habe mich auch überzeugt, daß wir alle
mehr auf unsern eignen Wert und unser freies Urteil bauen müssen
als auf die sogenannte Ansicht der Gesellschaft, die sich auf
Irrtümern aufbaut. Dadurch sind wir im Begriff, den Wert dieser
Gesellschaft zu heben. Es müssen sich diejenigen zusammenfinden,
die der Unabhängigkeit ihres Urteils sich freuen. Das allein sind
die Gentlemen. Ich bin überzeugt, auch Sie werden sich noch bei uns
einfinden, wenn Sie sich die Sache überlegen. Daß Torm ebenso
denkt, darauf kann ich Ihnen mein Wort geben.“

Herr von Schnabel ging einige Tage in verdrießlichen Gedanken
umher. Auch Dr. Wagner war dem Menschenbund beigetreten. Die Zahl
derer, die seinen Ansichten beistimmte, wurde immer kleiner. Er
wälzte Pellingers Worte hin und her. Endlich suchte er Grunthe
auf.

Es war ein langes Gespräch, das sie führten. Vornehmlich drehte es
sich um die Persönlichkeit von Ell und die Ziele des
Menschenbundes.

Als Herr von Schnabel die Sternwarte verließ, war er Mitglied
geworden. Nicht irgendein besonderes, durchschlagendes Ereignis
hatte seine Sinnesänderung bewirkt. Der Sieg des Idealismus übte
eine assimilierende Kraft der Veredelung aus.

\section{54 - Auf der Sternwarte}

Es begann bereits zu dunkeln, als die beiden Freundinnen nach
kurzer Wanderung bergab die Haltestelle der elektrischen Bahn
erreichten. Sie nahmen sogleich in dem bereitstehenden Wagen Platz,
der sich nach wenigen Minuten in Bewegung setzte. Die helle
Beleuchtung im Innern des Wagens verhinderte sie, etwas von der
anmutigen Gegend, durch die sie fuhren, zu erkennen. Trotzdem
verging ihnen die Zeit rasch, denn La war glücklich, zum erstenmal
von der leidenschaftlichen Liebe und Sehnsucht sprechen zu können,
die sie so lange stillschweigend und duldend hatte im Herzen
verbergen müssen. Se hörte ihr teilnehmend zu, manchmal schüttelte
sie leise den Kopf, immer aber mußte sie wieder mit Bewunderung auf
die Freundin blicken, die mutig und entschlossen den unerhörten
Schritt vom Nu zur Erde wagen wollte. Wenn sie dann ihre Augen
glückstrahlend leuchten sah, so konnte sie nicht zweifeln, daß sie
alle Hindernisse siegreich zu überwinden wissen werde. Sie saßen
allein in ihrem Wagenabteil und konnten darum ungestört miteinander
plaudern. Und dabei fragte Se:

„Eines, liebste La, ist mir doch noch bedenklich. Du sagst, zwei
Jahre lang, zwei Menschenjahre, hast du ihn nicht gesehen, nicht
direkt mit ihm verkehrt. Das ist lange Zeit für einen Mann. Deiner
bist du sicher, aber weißt du denn, wie es mit ihm steht? Ob er
dich denn noch will? Hast du nie diesen Zweifel gehabt?“

„Niemals“, sagte La entschieden. „Niemals seit jenem Augenblick, da
ich ihn unter Tränen in meinen Armen hielt, da ich ihm gestand, daß
ich sein bin. Das war kein Spiel, das waren keine Küsse und
Liebesworte, die wie Frühlingsblumen im Sonnenschein sprießen und
über Nacht im Strauß verwelken. Das wissen wir beide, die unser
Wissen um das Glück mit dem Wissen um das Elend erkauften, daß wir
uns nie gehören können. O Se, du Kleinmütige, du weißt nicht, wie
stolz die Liebe macht; ich weiß jetzt, wie man es werden kann.
Glaubst du, daß der vergessen kann, um den diese Augen aus Liebe
weinten? Nein, ich bin La, ich bin seine La, und das denken wir
beide zu jeder Stunde, denken’s und fühlen’s in tausend Schmerzen,
und ob wir es uns auch niemals wieder sagen, wir zweifeln nicht.“

La schwieg und versank in Träumerei. Sie schloß die Augen und
wollte sich nach ihrer Gewohnheit im Sitz zurücklehnen. Aber der
unbequeme Hut erinnerte sie sogleich, wo sie war.

Se lächelte. „Ich habe mich schon lange darüber geärgert“, sagte
sie, „daß diese Bahn so unbequeme Sitze hat. Bei mir gehen die
kleinen Erdenleiden in keinem großen auf, und ich merke unter
anderm auch, daß die heutigen Strapazen und Erregungen uns ganz
schwach zur Friedauer Sternwarte werden kommen lassen. Aber ich
habe mich nicht wie heute früh auf die Erde verlassen, sondern mir
eine ganze Schachtel Energiepillen eingesteckt.“

„Ich auch“ sagte La und zog das Büchschen aus ihrem
Reisetäschchen.

„Ach sieh doch“, neckte sie Se. „Also hat das Zutrauen zu den
›Geselchten‹ doch seine Grenzen.“

„Närrchen, wozu haben wir denn unsre Vernunft? Doch nicht, um das
Kleine über dem Großen zu vergessen, sondern alles in seinem
richtigen Verhältnis als Zweck und Mittel abzuwägen.“

„Aha, du sprichst schon im Grunthe-Ton. Da werden wir wohl bald da
sein, hier sieht man bereits erleuchtete Straßen. Nun schnell die
Pillen geschluckt.“

Nicht lange darauf hielt der Wagen an der Endstation. Die Fahrgäste
in den übrigen Abteilen des Wagens waren alle schon unterwegs
ausgestiegen. Die beiden Martierinnen standen allein auf der Straße
und sahen sich ziemlich ratlos um. Der Wagenführer schaltete seine
Lichter um und verschwand in der benachbarten Restauration, um sich
in seiner kurzen Ruhepause zu stärken. Kein Mensch war auf der
Straße sichtbar.

Der Boden war noch feucht und teilweise mit den Resten des
Gewitterregens bedeckt. Die breite, von Vorgärten begrenzte Straße
endete hier in einem kleinen, mit Bäumen besetzten Platz, von
welchem dunkle Alleen nach drei Seiten ausgingen. Man konnte nicht
erkennen, wo sie hinführten, denn zwischen den dichtbelaubten
Bäumen verschwand das Licht der spärlichen Gasflammen, die sie
erhellten, und nur so weit konnte man sehen, als die Strahlen der
elektrischen Bogenlampen an der Endstation der Straßenbahn
reichten.

„So also sieht es in Friedau aus“, seufzte Se. „Und das ist noch
eine Residenzstadt! Wie mag es da erst auf dem Lande sein, wo –“

„Halte keine Reden“, unterbrach sie La, „sondern komm, die
Sternwarte wird schon zu finden sein.“

Sie spähte nach jemand aus, den sie nach dem Weg fragen könnte.
Eine Laterne tauchte in der Hauptstraße auf, es war die eines
Radfahrers, der in eine der Alleen einbog.

„Dort hinaus muß also noch irgend etwas liegen, denn es fahren noch
Menschen hin“, sagte La in unverwüstlicher Laune.

„Weißt du, wer das war?“ rief Se. „Als er bei der Bogenlampe
vorüberfuhr, erkannte ich ihn. Es ist derselbe Mensch, der während
des Gewitters bei dem Pavillon stand. Und – ich bin vorhin nicht
dazu gekommen, mit dir darüber zu sprechen – ist dir nicht eine
seltsame Ähnlichkeit aufgefallen?“

„Mit wem? Ich habe kaum auf ihn geachtet.“

„Mit Ismas Mann. Nach den Bildern. Ich bilde mir ein, es ist
Torm.“

„Wie töricht. Das würde doch Isma zuerst wissen –“

„Wenn er aber Gründe hätte, sich zu verbergen? Du hast ja gehört
–“

„Dann wäre er doch nicht nach Friedau gegangen, wo ihn jeder Mensch
kennt.“

„Und niemand sucht. Er sieht jetzt nicht mehr so aus, wie er damals
ausgesehen hat. Ich glaube gern, daß ihn kein Mensch wiedererkennt.
Der Bart ist anders, das Haar ergraut, die Gesichtsfarbe gebräunt,
die Wangen eingefallen – aber ich habe den Blick für den Charakter
der Physiognomie, ich sehe durch alle Veränderungen hindurch –“

„Aber warum sollte er sich vor seiner Frau verbergen?“

„Es ist mir auch ein Rätsel. Immerhin wäre es sonderbar, wenn es
zwei so ähnliche Individuen gäbe. Doch sieh, da kommt jemand.“

Der Wagenführer trat aus der Restauration. Seine Abfahrtszeit war
gekommen. Auf Las Frage gab er den Damen bereitwillig Auskunft. Die
Allee rechts, immer bergan, in ein paar Minuten kommt man an das
Gitter.

„Also die Allee, die dein Geistertorm hinaufgefahren ist. Wären wir
ihm nur gleich nachgegangen. Nun vorwärts“, sagte La.

Die Steigung war für die beiden Martierinnen beschwerlich. Sie
spannten jedoch ihre Schirme auf, und so kamen sie bald vor das
eiserne Gittertor, das von einer Glühlampe beleuchtet wurde.

Niemand war ihnen begegnet.

„Es ist furchtbar einsam hier“, sagte La.

„Das ist noch das Beste dabei“, sagte Se. „Es ist wenigstens auch
still. Wie spät ist es denn eigentlich?“

„Da oben leuchtet ja das Zifferblatt der Sternwartenuhr. Es ist
acht Uhr vorüber. Wir wollen schellen.“

Grunthe saß mit Torm, der soeben von seinem Ausflug zurückgekommen
war, bei ihrem frugalen Abendessen, als ihm der Besuch zweier Damen
gemeldet wurde. Sein Assistent, der sonst die Besucher der
Sternwarte herumzuführen pflegte, war nicht anwesend, und es war
ihm sehr unangenehm, jetzt sich stören zu lassen, zumal durch
Damen. Er ließ daher sagen, er bedauere, aber die Sternwarte könne
heute nicht gezeigt werden.

Der Diener ging hinaus, kam jedoch nach einer Minute in großer
Aufregung wieder herein.

„Was gibt es denn?“ fragte Grunthe.

„Zwei Damen vom Mars“, stammelte der Diener, indem er Grunthe
ehrfurchtsvoll eine schmale, zierliche Karte überreichte. Sie war
mit einer Nadel durchstochen, an der eine kleine goldene Medaille
hing. Diese Medaille war es, die den Diener in Aufregung versetzt
hatte. Jeder kannte diesen Weltpaß der Nume, das Wappen des Mars
auf der einen Seite, auf der andern die Worte: ›Im Schutze des Nu.‹
Sie öffnete dem Besitzer alle Türen.

„Nume?“ sagte Grunthe verwundert zu Torm. Er betrachtete die Karte.
Sie trug keinen Namen, sondern nur die flüchtig hingeschriebenen
martischen Zeichen: „Die Pflegerinnen von Ara bringen sich in
Erinnerung.“

Grunthes Stirn zog sich zusammen. Seine Lippen bildeten das in
Klammern gesetzte Minuszeichen. So las er noch einmal die Karte.
Dann lösten sich seine Züge wieder zu einem höflicheren Ausdruck,
und er sagte zu dem Diener: „Ich bitte in die Bibliothek. Ich werde
gleich kommen.“

„Es sind La und Se“, sagte er dann zu Torm, „die beiden Nume, die
Saltner und mich nach unserm Sturz gepflegt haben. Ich bin ihnen zu
großem Dank verpflichtet. Ich muß sie empfangen. Wollen Sie
mitkommen?“

„Es würde mich interessieren. Diese La war sehr freundlich gegen
meine Frau während ihres Aufenthalts auf dem Mars. Aber sie ist
auch eine Freundin Ells. Man weiß nicht, was sie herführt. Hören
Sie erst, was sie wollen.“

„Sie können nun einmal Ihr Mißtrauen nicht loswerden. Doch wie Sie
wünschen.“

Torm warf einen Blick durchs Fenster. „Es ist klar geworden“, sagte
er. „Ich will versuchen, am großen Refraktor einige Platten zu
exportieren. Die Damen kennen mich nicht, dort im Dunkeln können
sie mich überhaupt nicht erkennen. Wenn Sie sie herumführen, könnte
ich sie mir dort einmal –; übrigens, nun fällt mir ein, vielleicht
habe ich die Damen schon gesehen, heute, an der schönen Aussicht
bei Tannhausen –. Dort waren zwei Martierinnen, und kurz vorher sah
ich ein merkwürdiges Luftschiff aufsteigen –; nun aber gehen Sie,
wir werden ja sehen.“

Grunthe betrat die Bibliothek mit einem möglichst liebenswürdigen
Gesicht, sogar ein Lächeln machte einen Anlauf zum Erscheinen,
verunglückte aber in seinen ersten Zügen. La und Se enthoben ihn
der Schwierigkeit, ihnen die Hand zu reichen, indem sie ihn auf
martische Weise begrüßten.

Es gab bald ein lebhaftes Gespräch und kurze Erkundigungen und
Erklärungen herüber und hinüber. Grunthe wollte ausführlich auf die
wissenschaftlichen Ergebnisse zu sprechen kommen, die er mit Hilfe
der Mitteilungen gewonnen hatte, die ihm La vom Mars aus hatte
zukommen lassen, aber La ging nicht darauf ein, sie fragte direkt
nach Saltner.

„Ich will Ihnen mitteilen, was wir wissen“, sagte sie. „Er ist in
Bedrängnis, man wird ihn dieser Tage mit Hilfe von Luftschiffen
suchen und gefangennehmen. Ich bin aber von seiner Unschuld
überzeugt.“

Grunthe wurde sehr ernst. Er wagte es sogar, La jetzt anzusehen und
erkannte in ihren Zügen die Aufrichtigkeit der Teilnahme und die
herzliche Sorge um den Freund.

„Es ist für Saltners Freunde“, sagte er, „eine Freude, ein solches
Wort zu hören. Ich weiß, daß auch Ell ihm gerne helfen würde, wenn
er dürfte, aber er ist durch seine Amtspflicht gebunden. Leider
kann Ihre Überzeugung, selbst wenn sie nachträglich vom Gericht
geteilt werden sollte was ich bezweifle, Saltner nichts nützen. Ich
muß Ihnen gestehen, daß seine Lage eine verzweifelte ist. Er selbst
würde sich ja schließlich auch über die Verhaftung und das Urteil
hinwegzusetzen wissen. Aber Sie wissen, wie er an seiner Mutter
hängt. Und damit verknüpft sich sein Geschick. Die alte Dame würde
eine nochmalige Gefangennahme nicht überleben, das ist Saltners
Sorge. Und ihr Zustand gestattet ihm nicht, seinen Zufluchtsort
aufzugeben und etwa, was ihm sonst vielleicht gelingen könnte, sich
am Tag in den Wäldern zu verbergen und in der Nacht auf unwegsamen
Kletterpfaden in Sicherheit zu bringen. Wir sehen daher keinen Weg
vor uns, wie diese Gefahr vermieden werden könnte. Vielleicht schon
morgen geschieht das Traurige.“

„Morgen?“ unterbrach ihn La erschrocken. „Was wissen Sie?“

„Ich erhielt heute eine Depesche von einem seiner Freunde. Zwei
Luftschiffe sind zu seiner Aufsuchung ausgeschickt. Sie sollte
schon heute beginnen. Das Wetter, das die Berge in Wolken hüllt,
verhinderte sie jedoch. Wenn es morgen klar wird – und die
Wetterkarte läßt es vermuten –“

„Können Sie mir sagen, wo Saltner sich aufhält?“

„Genau wissen es nur wenige Eingeweihte. Wir wissen nur, was auch
den andern bekannt ist, in den Bergen, die sich südlich vom
Etschtal oberhalb Bozen, etwa nach dem Nonsberg, hinziehen, in
einer der dort befindlichen Hütten – hier können Sie die
Spezialkarte sehen.“ La ließ sich die Karte erklären.

„Können Sie mir die Karte leihen?“ fragte sie.

„Recht gern. Aber was wollen Sie damit?“

„Ich sagte Ihnen schon, ich bin mit meiner Freundin auf einer Reise
durch Europa. Vielleicht sehe ich mir diese Gegend einmal an.
Übrigens war ich so frei, mein Luftschiff hierher zu bestellen, um
uns abzuholen. Es müßte eigentlich schon hier sein. Frau Torm sagte
uns, daß Sie selbst hier im Garten gelandet seien, so glaubte ich
–“

La hatte ruhig gesprochen. Jetzt trafen sich ihre Blicke mit denen
Grunthes, sie ruhten eine Weile ineinander. Dann legte Grunthe
schweigend die Karte zusammen und überreichte sie La.

„Wünschen Sie eine Empfehlung an einen Kenner der dortigen Gegend?“
fragte er. „Sie dürften dort als Nume wenig Entgegenkommen
finden.“

„Wir brauchen keinen Führer“, erwiderte La. „Wir schweben ja über
den Höhen, da genügt uns die Karte. Ich danke Ihnen.“

Sie erhob sich.

„Wollen Sie nicht einen Gang durch unsere Arbeitsräume tun? Von der
Plattform aus würden wir die Ankunft Ihres Schiffes am besten
bemerken.“

Sie durchschritten mehrere Zimmer und betraten den Rundgang. Hier
und da sprach Grunthe einige erklärende Worte.

„Sie sehen“, sagte er, „wie wir uns Mühe geben, von Ihnen zu
lernen. Vieles hatte Ell bereits eingerichtet, ehe wir etwas von
den Numen wußten. Ich habe mich freilich schon damals gewundert,
wie er auf so viele neue Feinheiten hatte kommen können.“

An einer Stelle war die Seitenwand weit auseinandergeschoben. An
dem dort befindlichen, auf den Sternenhimmel gerichteten Instrument
war Torm beschäftigt. Er verbeugte sich flüchtig, ohne sich stören
zu lassen.

Se beobachtete ihn scharf, soweit es die matte Beleuchtung
gestattete, während sie scheinbar das Werk einer in der Nähe
stehenden Uhr studierte.

„Wissen Sie“, sagte sie plötzlich laut zu Grunthe, „daß wir Frau
Torm beinahe mitgebracht hätten? Wir waren mit ihr im Wald, nur
mußte sie leider nach Berlin zurück. Haben Sie denn etwas von den
Gerüchten gehört, daß Torm wirklich zurückgekehrt sei und sich nur,
man weiß nicht warum, hier verborgen halte? Wir haben mit Frau Torm
natürlich nicht davon gesprochen, aber Sie können wir ja doch
fragen.“

Torm hatte sich bei Ses Worten tief auf das Instrument gebeugt, und
Se sah deutlich, wie seine Hand an der Schraube des Apparats
zitterte.

„Welches Gerücht?“ fragte Grunthe, als hätte er nicht recht gehört.
In diesem Augenblick erhellte sich die Gegend plötzlich wie von
Sonnenlicht, und durch die geöffnete Wand drang auf kurze Zeit ein
tagheller Schein.

„Das Luftschiff“, rief La und blickte zum Fenster hinaus, während
Se ihren Blick auf Torm gerichtet hielt, der sich schnell
entfernte.

„Der Schiffer beleuchtet seinen Landungsplatz.“

„Und meinem Assistenten hat er die Aufnahme verdorben“, setzte
Grunthe hinzu.

„Das tut mir sehr leid“, sagte La. „Aber wir wollen Sie auch nicht
länger stören. Würden Sie jetzt die Güte haben, uns in den Garten
zu führen?“

Als La und Se mit Grunthe den Garten betraten, lag das Schiff schon
auf dem Rasenplatz. Nur zwei kleine Lichter machten es im Dunkel
kenntlich. Grunthe konnte die freundliche Einladung nicht
abschlagen, die Yacht zu besichtigen und einen Augenblick im Salon
Platz zu nehmen.

Se setzte sich ihm gegenüber, und ihn offen anblickend begann sie:

„Nun will ich Ihnen auch einmal etwas auf den Kopf zusagen,
Grunthe. Dieser Mann, den sie Ihren Assistenten nannten, war Hugo
Torm, und Sie wissen es. Warum steckt er hier im Verborgenen? Warum
ist er nicht bei seiner Frau, die ihn für tot hält? Warum läßt er
sie in ihrem Harm sitzen? Und das dulden Sie? Das ist ja ganz
unerhört. Und nun reden Sie die Wahrheit.“

Grunthe saß stumm mit eingezogenen Lippen.

„Sie wollen nicht reden?“ fragte Se.

„Ich darf nicht. Es sind nicht meine Geheimnisse.“

„Ach, also Torms! Das Zugeständnis genügt. Und billigen Sie dies
Verhalten?“

„Nein.“

„Warum benachrichtigen Sie nicht Frau Torm?“

„Das geht mich nichts an. Davon verstehe ich nichts. Das muß ich
Torm überlassen.“

„Und seine Gründe? Er muß Ihnen doch Gründe angegeben haben.“

„Ich kann nichts sagen.“

„So werde ich Isma –“

„Ich bitte Sie“, unterbrach sie Grunthe, „Sie können nicht wissen,
ob das gut wäre. Nehmen Sie an, er stünde unter dem Druck einer
Schuld, oder glaubte es wenigstens – er würde seine Frau nur ins
Unglück stürzen, wenn er jetzt käme, oder er scheue sich, vor sie
als ein Ausgestoßener zu treten, aber er hoffe, daß der Makel noch
von ihm genommen werden könnte, in einiger Zeit –. Nehmen Sie an,
er warte nur noch Nachrichten ab. Eine vorzeitige Mitteilung könnte
alles verderben –“

„Nehmen wir an, was wir wollen“ hub jetzt La an, „hier gibt es gar
keine andre Wahl, als die Frau in dieses Geheimnis zu ziehen, und
sie kann dann entscheiden –. Ihr haltet das wahrscheinlich für
besonders edel, daß der Mann die inneren Kämpfe in sich ausficht
und die Frau aus Schonung in der Angst der Ungewißheit läßt, weil
ihr denkt, sie könnte sich wieder durch rücksichtsvolle Gefühle
bestimmen lassen, das zu tun, was sie eigentlich nicht will.
Zartgefühl nennt ihr’s, und Hochmut ist es, weiter nichts. Der
Hochmut, daß ihr allein so außerordentlich fähig seid zu
beurteilen, wo und wieweit man sich aufopfern darf. Das kommt aber
alles davon, weil ihr nicht wißt, was Freiheit ist; Freiheit, die
das Gefühl anerkennt, wie es wirklich ist, aber nicht es
zurechtstutzt, wie es euch verständig scheint. Und weil eure
Vernunft zu blöde ist, um dieses ganze Gewirr von Gefühl und
Berechnung zu durchschauen so verderbt ihr das Leben aus lauter
Edelmut in der schönsten Selbstlüge.“

„Ich verstehe nichts davon“, sagte Grunthe wiederholt, indem er
aufstand. „Ich will nichts damit zu tun haben, das sind Sachen, die
sich nicht berechnen lassen. Ich bitte nur, wahren Sie ein
Geheimnis, das nicht das Ihrige ist, wie auch ich es tue.“

„Das versteht sich von selbst“, erwiderte Se. „Wir können nur von
dem Gebrauch machen, was wir mit eignen Augen gesehen haben.“

„Leben Sie wohl“, sagte Grunthe. „Und möge Ihre Reise zum Ziel
führen.“

„Sie werden uns in jedem Fall nächste Nacht wieder hier sehen.
Dürfen wir in Ihrem Garten übernachten?“

„Selbstverständlich. Indessen – ich kann mich nicht darum
kümmern.“

„Das beanspruchen wir nicht“, sagte La lächelnd. „Wenn wir aber
vielleicht Gäste mitbringen, die mit Ihnen sprechen möchten, wie
können wir Sie von unsrer Ankunft benachrichtigen?“

„An der Tür, die vom Garten nach dem Haus führt, ist eine Klingel.
Wir werden wahrscheinlich die nächste Nacht durcharbeiten, wenn es
klar ist.“

„Es wird klar werden“, sagte La, indem sie jetzt Grunthe die Hand
reichte.

Er nahm sie, er drückte sie sogar ein wenig. Dann ging er mit
steifen Schritten aus der Tür.

La sah ihm nach.

„Ich fürchte“, scherzte Se, „den hast du auch erobert. Er hat dir
ja beinahe die Hand gedrückt.“

„Ja“, sagte La, „er hat sich gebessert. Aber im Ernst, er ist einer
von den Menschen, die wert wären, auf dem Nu geboren zu sein. O Se,
wenn es Gott gäbe, daß wir morgen hier alle zusammen sind!“

„Laß uns hoffen und ruhen. Wir haben einen schweren Tag vor uns.“

„Ich will nur noch mit dem Schiffer sprechen. Eine Stunde vor
Sonnenaufgang wollen wir aufbrechen.“

Alle Luken wurden geschlossen, die Lichter gelöscht. Dunkel und
verschwiegen lag das Schiff auf dem Rasen, verborgen von den hohen
Bäumen des Parks. Ein fernes Wetterleuchten zuckte zuweilen im
Norden, im Süden aber, alle Sterne überstrahlend, zog der rötliche
Mars seine Bahn in ruhigem Licht.

\section{55 - In höchster Not}

Der getreue Palaoro war in der Nacht auf das Gebirge gestiegen, um
Saltner die Nachricht zu bringen, daß zwei Luftschiffe in
Bereitschaft seien, ihn zu suchen. Diese Nachforschung konnte nur
dadurch geschehen, daß die Luftschiffe Tal für Tal und Berghalde
für Berghalde absuchten und jedes einzelne Häuschen, jede Hütte
anliefen, um sich die Insassen anzusehen. Dies war allerdings eine
umständliche Sache, doch war das Gebiet, um das es sich handelte,
in bestimmter Weise eingeschränkt. Denn alle Täler, die den
Gebirgsstock umgaben oder aus ihm herausführten, waren sogleich am
Tag nach Saltners Flucht abgesperrt worden, die hier zerstreut
liegenden Ortschaften waren besetzt, und es wäre nicht möglich
gewesen, sie unentdeckt zu passieren. Ein einzelner Gebirgssteiger,
wie Saltner, hätte sich wohl vorüberschleichen können, nicht so
eine Gesellschaft, in der Saltners Mutter sich befand. Denn diese
mußte entweder reiten oder getragen werden, war also auf die
gangbaren Wege angewiesen. Die Hütten, welche in Betracht kamen,
waren entweder Unterstandshütten für Touristen, oder es waren
Sennhütten oder Zufluchtsorte für Hirten. Sie lagen stets an
hervorragenden Punkten oder offen auf Wiesen und Almen, so daß sie
von der Höhe aus leicht wahrgenommen werden konnten. Wollte Saltner
für ein Luftschiff unentdeckbar bleiben, so konnte es nur dadurch
geschehen, daß er sich in den Wald flüchtete, der die Abhänge der
Bergrücken bedeckte.

Da am ersten Tag nach Palaoros Ankunft des dichten Nebels wegen auf
den Bergen noch keine Gefahr der Entdeckung vorlag, brach Saltner
mit dem Führer auf, um in den Wäldern eine passende Unterkunft zu
suchen. Die Hütten, die sich hier für Köhler und Holzschläger
errichtet fanden, waren allerdings höchst primitiver Art. Es gelang
ihnen aber, doch einen Bau aufzufinden, der sich durch einige
Arbeit wenigstens für den Notfall bewohnbar machen ließ. Sie
setzten diese ärmliche Wohnung, so gut es ging, instand und kehrten
abends nach der sogenannten Kleinen Hütte zurück. In der Nacht wäre
es nicht möglich gewesen, Frau Saltner in das abgelegene Tal durch
den Wald zu transportieren, da sie getragen werden mußte. Sie
beschlossen also, es am Tag zu wagen. Gefährlich für die Entdeckung
war dies freilich, denn es mußte ein weites, baumloses Plateau,
dann eine steile Schutthalde und ein Felsabstieg passiert werden,
ehe man in den Wald gelangte. Sie hofften, daß der Nebel noch
anhalten werde.

Vor Sonnenaufgang verließ Saltner die Hütte und bestieg den
Bergrücken, der den Blick nach Norden und Westen gestattete. Hinter
den Zacken der Dolomiten strahlte der Himmel in leuchtendem Rot.
Ein Meer von weißen Nebeln wogte in den Tälern, und nur die Gipfel
der Berge blickten wie Inseln aus ihm hervor. Rosig glühten die
Schneeriesen im Westen, und ihre höchsten Häupter glänzten bereits
im Sonnenlicht. Saltner spähte nach der Gegend, wo Bozen unter
Nebeln verborgen lag. Und da – siehe –, aus den weißen Wolken
tauchten zwei dunkle Punkte auf, deutlich hoben sie sich jetzt
gegen den hellen Himmel ab. Er richtete sein Fernglas darauf. Es
war kein Zweifel, es waren die beiden Luftschiffe, die sich zu
seiner Verfolgung aufmachten. Er eilte den Berg hinab.

„Wir müssen fort“, sagte er zu Palaoro. „Sie suchen uns, und der
Tag wird klar werden. Aber sie fahren nach Südost, wir werden also
noch Zeit haben, ehe sie bis hierher kommen. Für den Anfang steigen
auch die Nebel noch herauf, wir müssen sehen, daß wir zur rechten
Zeit Deckung finden.“

Der Zug setzte sich in Bewegung. Saltner und Palaoro trugen den
Tragstuhl mit Frau Saltner. Katharina schritt, ebenfalls mit Gepäck
beladen, hinterher. Es ging die Bergwand im Südwesten hinauf, dann
über ein weites Plateau. Man kam nur langsam vorwärts. Oft mußte
geruht werden. Endlich waren die Felsen am Rande des Plateaus
erreicht, das sich von hier mit einer steilen Schutthalde in ein
Tal hinabsenkte. Dieses mußte passiert werden, um den Bergrücken
auf der gegenüberliegenden Seite zu gewinnen. Von dort führte der
Weg durch eine Scharte zwischen zwei Gipfeln nach einem zweiten,
engeren Tal, dessen waldbedeckte Abhänge sicheren Schutz boten. In
dem ersten Tal zogen sich die Nebel jetzt bis dicht an den Rand des
Plateaus. Ehe die kleine Expedition den schmalen, aber
verhältnismäßig leicht gangbaren Pfad betrat, der hier hinabführte,
spähte Saltner noch einmal nach den Schiffen aus, ohne eine Spur
von ihnen bemerken zu können. Dann bedeckten die Nebel die
Flüchtigen. Bevor der neue Aufstieg begann, wurde eine Ruhepause
gehalten und dann mit neuen Kräften vorwärts geschritten. Es waren
gegen vier Stunden seit dem Aufbruch vergangen, als sie aus den
Talnebeln herausstiegen und sich anschickten, die Höhe zu
passieren. Man hatte hier wieder einen weiten Umblick nach Westen
und Süden. Plötzlich blieb Palaoro stehen.

„Sie kommen“, rief er aus.

Er hatte in der Ferne, im Süden, einen dunkeln Punkt bemerkt, den
nun Saltners Glas als Luftschiff nachwies. „Sie nähern sich“, sagte
Saltner, „aber sie haben sich getrennt – es ist nur ein Schiff.“

„Sie werden von zwei verschiedenen Seiten anfangen. Diese wollen
wahrscheinlich hinüber nach den Hütten am Laugen. Hier können wir
nicht weiter, in wenigen Minuten müssen sie uns sehen. Wir müssen
den Berg zwischen uns bringen. Sie werden vorläufig jedenfalls auf
der Südseite bleiben.“

Man bog nach rechts ab und war bald durch die aufsteigenden, mit
Rasen und verkrüppelten Fichten bedeckten Felsabhänge des
Bergrückens gegen das herannahende Schiff gedeckt, so lange es sich
nicht über den Gipfel erhob. Es war aber anzunehmen, daß die
Martier zunächst die Abhänge im Süden absuchen würden. Der
beschwerliche Weg führte nun bergab nach einem Felsriegel zu, von
dem aus sich eine Schlucht in das Tal hinabzog. Doch war es
fraglich, ob diese von einem Wildbach durchströmte, in steilen
Abstürzen niedergehende Schlucht passierbar sein würde. Dies mußte
zunächst untersucht werden. Das Ende des Felsriegels, der nach
Norden fast senkrecht etwa hundert Meter abstürzte, war mit hohen,
flechtenbedeckten Fichten bestanden und bot unter diesen und
zwischen den Felstrümmern einen vorläufigen Zufluchtsort. Hier
wollte Saltner die Frauen verbergen, während Palaoro einen Weg nach
dem Tal auskundschaften sollte. Es galt nur noch die kurze Strecke
über den kahlen Rücken bis zum Beginn des Waldes zu durchqueren.

Vielleicht noch hundert Schritte bergab trennten die Flüchtigen von
dem schützenden Dickicht, als sie vor sich, nach Norden, über den
dort hervorragenden Berggipfeln ebenfalls einen Punkt bemerkten,
der unzweifelhaft ein Luftschiff war.

„Dort ist das andere Schiff“, rief Palaoro.

So schnell, als es möglich war, durchliefen sie die kurze Strecke
und suchten einen geschützten Platz unter den hohen Stämmen. Die
Sonne schien warm auf die harzduftenden Nadeln, in langen Bändern
hingen die graugrünen Flechten von den Ästen, und der aus
Felstrümmern bestehende Boden war mit weichem Moos bedeckt. Man hob
Frau Saltner aus dem Stuhl, und die Frauen ruhten an geschützter
Stelle in der stillen, sonnendurchwärmten Luft, während Saltner und
Palaoro bis an den Rand des Absturzes vorgingen, um vorsichtig nach
dem vermuteten Feind auszuspähen.

„Es ist mir nicht recht erklärlich“, sagte Saltner, „warum dieses
Schiff einen so seltsamen Weg eingeschlagen hat, daß es jetzt von
Norden kommt. Aber gleichviel, wenn sie uns nicht auf dem Weg
hierher erkannt haben, sind wir vorläufig sicher.“

„Sie können uns schon gesehen haben. Sie kommen ja gerade auf uns
zu.“

„Leider. Sie haben die ursprüngliche Richtung geändert. Man möchte
wirklich glauben, daß sie hierher wollen. Ach, sie steigen in die
Höhe und spannen die Flügel auf, sie werden eine Landung
versuchen.“

„Wenn sie wirklich uns gesehen haben und hier in das Wäldchen
wollen, so können sie nur draußen auf dem Bergrücken landen, von wo
wir gekommen sind. Sonst können sie nirgends heran, das verhindern
die Bäume.“

„Kommt, Palaoro. Wir wollen nach der andern Seite gehen, hier ist
nichts zu tun und nichts zu befürchten. Das Schiff ist so hoch, daß
man es nicht mehr sehen kann, ohne zu weit aus den Bäumen zu
treten. Was tun wir nun, wenn sie landen?“

„Wir steigen in die Schlucht hinunter, so weit es geht.
Nachklettern werden sie uns nicht. Bleiben Sie bei der Frau Mutter,
und ziehen Sie sich inzwischen nach der Schlucht zu. Kathrin kann
hier den Stuhl ein Stück tragen. Ich sehe inzwischen nach dem
Schiff.“

Saltner brachte seine Mutter mit Hilfe der Magd bis an die Stelle,
wo die Schlucht begann. Hier kletterte er selbst weiter, um den Weg
zu untersuchen. Es ging zunächst steil bergab, aber es schien ihm
möglich, doch noch hier herabzukommen. Nach einer kurzen Strecke
erweiterte sich die Schlucht zu einem kleinen, von fast senkrechten
Wänden umgebenen Felskessel. Den nahezu ebenen Boden, auf dem ein
kleines Bächlein entsprang, bedeckte kurzer Rasen. Im Sonnenschein
funkelten die Wassertropfen auf den Halmen, kleine blaue
Schmetterlinge und weißschimmernde große Apollofalter spielten in
diesem stillen Winkel. Die Quelle rieselte als schmales Rinnsal der
Felswand zu, die sie in einer kleinen Klamm durchbrach. Aber die
Neigung war gering, Saltner schritt durch das seichte Wasser und
überzeugte sich, daß sich dahinter der Boden des Tales erweiterte.
War man einmal bis hierher vorgedrungen, so mochte der weitere
Abstieg wohl gelingen. Nun beeilte er sich zurückzukehren.

Er hatte etwa zwei Drittel des Aufstiegs kletternd zurückgelegt,
als er zu seinem Erstaunen von Baum zu Baum ein Seil nach oben hin
ausgespannt fand. Bald begegnete ihm Palaoro, der Frau Saltner auf
einem Arm trug, während er sich mit Hilfe des Seiles vorsichtig den
steilen Abhang hinabarbeitete. Ihm folgte Katharina. Ohne ein Wort
zu sprechen unterstützte Saltner den Abstieg, bis sie das Ende des
Seils erreicht hatten. Hier setzte Palaoro Frau Saltner nieder und
sagte zu ihr beruhigend: „Hier sind sie ganz sicher, die dreißig
Meter können die Herren Martier nicht herabkraxeln. Wir wollen nur
das Seil holen.“

Er winkte Saltner und beide stiegen wieder den Berg hinauf.

Kurz vor der Höhe blieb Palaoro stehen und berichtete Saltner das
Geschehene. Als er vorhin den Rand des Waldes erreicht hatte und
die kahle Berglehne nach oben übersehen konnte, habe er das von
Norden gekommene Luftschiff bemerkt, das mit ausgebreiteten
Schwingen im Segelflug langsam über der Höhe schwebte. Es sei ein
ganz besonders großes, schönes Schiff gewesen. Da sei von der
andern Seite das kleine Regierungsschiff, das er als das Schiff des
Unterkultors in Wien erkannte, schnell herbeigekommen und hätte dem
andern Schiff Signale gemacht, die er nicht verstand. Darauf hat
das große Schiff die Flügel eingezogen, und er hat nicht sehen
können, was aus ihm geworden, da es hinter den Bäumen verschwunden
ist. Das kleine aber ist dicht vor dem Wald auf dem Bergrücken
gelandet. Nun ist der Pitzthaler, der Grenzjäger, aus dem Schiff
geklettert und nach dem Wald gegangen. Wie er gesehen hat, daß es
der Pitzthaler ist, hat er sich langsam zurückgezogen, und wie die
vom Schiff aus den Pitzthaler hinter den Bäumen nicht mehr sehen
konnten, ist er ihm so wie zufällig entgegengegangen. Hat ihn nun
der Pitzthaler gefragt, ob er nicht hier herum den Herrn Saltner
gesehen hat, der sollt’ mal gleich auf das Schiff kommen, denn sie
hätten von oben bemerkt, wie er um den Berg herumgegangen sei, und
da könnt’ er jetzt nirgends anders stecken als hier im Wald. Da
hätte er geantwortet, das wollte er dem Herrn Saltner schon sagen,
wenn er ihn halt zufällig hier treffen täte, wenn aber der Herr
Saltner nicht käme, was sie dann wohl tun würden. Dann würden sie
den Wald hier besetzen, daß er nicht herauskönnte, und er und der
Verpailer, der auch mit wäre, die müßten ihm halt nachgehen und ihn
herausholen, denn sonst kämen sie um ihr Brot. Er hätte sich aber
am Fuß was vertreten und könnte nur langsam den Berg
heruntersteigen. Und darauf wäre der Pitzthaler wieder
zurückgegangen. Nun sei er erst wieder bis an den Waldrand
geschlichen und habe gesehen, wie der Herr Unterkultor und vier
Beds mit Glockenhelmen aus dem Schiff gekraxelt und mit den beiden
Grenzjägern nach dem Wald zu gegangen seien. Da sei er rasch
zurückgesprungen, habe das Seil ausgespannt und sei mit den Frauen
herabgestiegen. Und er hat noch gesehen, wie die Grenzjäger mit den
Martiern erst nach der andern Seite gegangen sind.

Während des Berichts lösten Saltner und Palaoro das Seil und
stiegen die Schlucht wieder hinab. Sie beschlossen, sich bis in den
Felskessel hinabzuziehen und dort des weiteren zu warten. Beide
hofften, daß ihnen die Grenzjäger nicht sogleich folgen, sondern
die Martier unter irgendeiner Ausrede mit der Verfolgung hinhalten
würden.

Mit vielen Beschwerden gelang es, den übrigen Teil des Weges
zurückzulegen. Sobald sie hinter dem nächsten Felsblock
hervortraten, befanden sie sich am Rand der kleinen Wiese. Saltner
trug jetzt seine Mutter, Palaoro ging voran. Er stand am Eingang
zum Kessel. Da sprang er zurück. Erschrocken winkte er Saltner.
Dieser setzte seine Mutter sanft nieder und sprang zu ihm.

„Was gibt es?“ fragte er leise.

„Das große Luftschiff liegt auf der Wiese“ flüsterte Palaoro.

„Um Gottes willen! So sind wir verloren. Wir sind von beiden Seiten
eingeschlossen.“

Er warf einen Blick auf die seitlichen Abstürze der Schlucht, der
ihn belehrte, daß hier ein Entkommen mit den Frauen nicht denkbar
sei. Ratlos blickten die Männer sich an.

„Habt Ihr Leute bei dem Schiff gesehen?“ fragte Saltner.

„Ich hab mir gar nicht Zeit genommen“, antwortete Palaoro. „Sie
müssen von oben gesehen haben, daß hier der einzige Ausweg ist, und
haben ihn verlegt. Wenn sie sich jetzt hier umschauen, müssen sie
uns finden, auch wenn die von oben nicht herabkommen. Bergauf
werden die Nume nicht steigen, aber vielleicht haben sie auch
Grenzjäger bei sich. Wir wollen wenigstens das kleine Stückchen
zurück bis dort zwischen die beiden Felsen.“

„Es ist auch nur für den Augenblick“, sagte Saltner, „aber wir
wollen es tun. Möglich wäre es ja, daß die Grenzjäger nicht sehen
wollen und vorbeiziehen, wahrscheinlich freilich nicht, es ist zu
klar, daß wir hier stecken müssen. Ich werde mir dann das Schiff
ansehen, und wenn es nicht anders ist –“

„Ergeben?“ stammelte Palaoro.

„Ihr nicht, das hat keinen Zweck. Ihr könnt hier an der Seite
hinaufklettern. Ich aber kann die Frauen nicht verlassen.“

Er lehnte einen Augenblick wie gebrochen an dem Felsen.

„O meine Mutter!“ flüsterte er. Dann ging er zurück zu den Frauen.

„Ich muß euch noch ein paar Minuten hierlassen“, sagte er. „Dort
zwischen den Felsen wirst du besser sitzen. Es ist noch ein
Hindernis drunten, hoffentlich läßt es sich beseitigen.“

„Du mein lieber Josef, was ich dir für Mühe mache. Aber wenn sie
uns wieder fangen, das ist zu schrecklich“, antwortete Frau
Saltner.

Bald waren die Frauen untergebracht.

„Ich gehe jetzt“, sagte Saltner, sich beherrschend. „Ängstige dich
nicht, Mutter.“

Er küßte sie.

„Aber du kommst bald wieder?“

„Gott wird helfen.“

Saltner warf noch einen Blick zurück. Dann schlich er bis an den
Felsblock, der den Eingang zur Waldblöße deckte. Von oben konnte
man ihn nicht mehr sehen. Ein moosbedeckter Vorsprung am Felsen
bildete eine natürliche Bank. Hier ließ er sich einen Augenblick
nieder, um noch einmal zu bedenken, was er tun solle. Es war nichts
zu tun. Hierbleiben konnte er nicht. Vorüber konnte er auch nicht.
Er mußte sich gefangengeben. Auch das wäre ihm zuletzt gleichgültig
gewesen. Aber die Mutter! Sie überlebte den Schrecken nicht. Das
war das Ende! Und nun war alles verloren. Keine Rettung.

„Gnädiger Gott, hilf uns“, sagte er leise. „Doch Dein Wille
geschehe.“

Er erhob sich, er wollte um die Ecke des Felsens nach dem Schiff
ausspähen. Da war es ihm, als hörte er leises Rascheln der dürren
Zweige, die den Moosboden bedeckten. War es eine Eidechse? Kam
jemand? Er zögerte einen Augenblick. Die Spalte neben dem Felsen,
durch welche das Sonnenlicht in den Wald blickte, verdunkelte sich.
Eine Gestalt stand vor ihm.

Er richtete sich hoch auf. Das Herz schlug ihm, wie ein Nebel legte
es sich vor seinen Blick. Wer war das? Unter dem Schatten eines
breiten Hutes leuchteten ihm zwei Augen entgegen, glückstrahlend,
sonnenhaft. Schweigend standen sich beide gegenüber, bis es leise;
zögernd, als fürchte er, aus einem Traum zu erwachen, über Saltners
Lippen kam, eine einzige Silbe:

„La!“

Es war ihm, als müsse er zu Boden sinken. Da bewegte sich die
Gestalt. Zwei Arme umschlangen ihn, eine weiche Wange fühlte er an
der seinigen. La barg ihren Kopf an seiner Schulter und flüsterte:
„Sal! mein Sal!“

Er sank auf die Moosbank nieder und zog sie mit sich. Ihre Lippen
glühten aufeinander.

„Du bist es, du bist es“, sagte La selig.

Er zog sie aufs neue an sich.

Endlich stammelte er: „Und du, wie kommst du – O du mein Glück,
weißt du denn –“

„Ja, ja! Ich komme, um dich zu fangen und nie wieder freizugeben.
Ich komme vom Nu, und ich will bei dir bleiben auf der Erde, oder
wo du willst – nur nicht allein, nicht länger allein. Ich kann es
nicht!“ Sie sank aufs neue an seine Brust. Dann sprang sie auf.

Von oben hörte man das Klingen des Bergstocks. Palaoro wurde
sichtbar. Er prallte zurück, als er La erblickte. Dann rief er:
„Sie steigen von oben herab.“

Saltner blickte auf La. „Du kommst zu mir, Geliebte“, sagte er
hastig. „Aber ich bin gefangen und eingeschlossen. Du kommst, nur
zu sehen, wie ich dir entrissen werde.“

La lächelte glücklich. „Das ist unmöglich“, sagte sie. „Geh und
hole deine Mutter, und du wirst sehen.“

Saltner wirbelte der Kopf, aber er nahm sich keine Zeit, zu
überlegen, wie das alles möglich sei. Er prüfte nicht, er zweifelte
nicht, Las Wort glaubte er. Weiter bedurfte es nichts. Er sprang
mit Palaoro den Felsen hinauf.

„Wir sind gerettet, gerettet!“ rief er seiner Mutter zu. „Fürchte
dich nicht vor den Numen, zu denen ich dich bringe, es sind unsre
Freunde.“

„Wenn du es sagst, so ist es gut.“

In wenigen Minuten standen sie wieder bei La, die an dem Felsen
gewartet hatte.

„Das ist unsre Retterin“, sagte Saltner, auf La weisend.

La faßte ehrfurchtsvoll die Hand von Saltners Mutter und sprach:
„Sie sollen bald zufrieden sein.“

„Gott segne Sie“, antwortete die Mutter.

La schritt voran. Die nachfolgenden Menschen stutzten bei dem
Anblick, der sich ihnen bot. Kathrin stieß einen Schrei der
Verwunderung aus.

Wie eine goldene Schale in der Sonne leuchtend lag die Luftyacht
auf der Waldwiese. Niemand war zu sehen als am Fuß der breiten,
bequemen Schiffstreppe der Schiffer in seinem Glockenhelm, der
salutierend die Herrin des Schiffs erwartete. La eilte voran. Als
sie das Geländer erfaßte, flammte ein Funkenbogen über dem Eingang,
der die Aufschrift zeigte: „Willkommen im Schutze der La.“

Am Eingang zum Schiff blieb sie stehen und wiederholte die Worte.
Man stieg in das Schiff, der Schiffer folgte, im Augenblick war die
Treppe eingezogen.

Palaoro blieb vorläufig auf dem Verdeck. Saltner führte seine
Mutter und die Magd in den Raum, dessen Tür La öffnete.

„Hier ist Ihr Zimmer“, sagte sie, „und daneben das für Katharina.
Nun ruhen Sie sich recht aus. Und was Sie wünschen, sprechen Sie in
diese Öffnung, so wird es da sein.“

Frau Saltner war sprachlos. Ein weicher Polsterstuhl am Fenster
nahm sie auf. Sie blickte sich im Zimmer um.

„Das ist ja gerade wie daheim in unsrer Sommerwohnung“, sagte sie
endlich. „Die Täfelung ringsum, und die Gardine in der Ecke, und
dort, das Kruzifix und das Lämpchen –. Nur die Bilder, und die
Kissen, und die Teppiche das ist alles viel kostbarer – wie kommt
das nur – –“

„Das ist die Zauberin, die es gemacht hat“ sagte Saltner, gerührt
Las Hand ergreifend. „Sie hat nichts vergessen von allem, was ich
ihr von unserm Heim schildern mußte. Ihr gehört dieses Wunder von
einem Luftschiff.“

La sah dem geliebten Mann in die Augen.

„Uns beiden!“ sagte sie dann.

„Du willst? Du willst es wirklich?“ rief Saltner jubelnd und schloß
sie in seine Arme. Doch wie in einem tiefen Schreck verstummte er
plötzlich. „Aber ich bin ein Mensch“, sagte er tonlos.

„Sei, was du willst, ich bin dein, deine La.“

Er blickte auf die Herrliche, Königliche, deren Blick wie bittend
zu ihm aufgeschlagen war. Er wußte nicht, was mit ihm vorging. Der
plötzliche Übergang von der Verzweiflung zum höchsten Glück, von
der Not zur Sicherheit, vom Unerreichbaren zum Wirklichen verwirrte
ihn. Er schüttelte den Kopf, und sein Antlitz strahlte dabei von
Freude.

„Ich weiß ja nicht, was ich bin, wer ich bin, wo ich bin. Ich weiß
nur, daß ich namenlos glücklich bin. Schau, Mutter, das ist sie,
die ich liebe, der ich alles verdanke. Ich weiß nicht, wie man das
bei euch auf dem Mars macht, wenn man eine Frau haben will, und es
ist mir auch ganz egal, und du bist halt die La! Da, Mutter, gib
ihr einen Kuß, ich muß einmal einen Juchzer tun.“

Und mit einem Sprung war er aus dem Zimmer, und während die alte
Frau ihre Hände zitternd auf das von Liebe und Schönheit strahlende
Haupt der glücklichen Nume legte, schallte draußen ein Jodler laut
und jubelnd zu den Bergen empor, und das Echo der Felsen gab ihn
zurück – –

\section{56 - Selbsthilfe}

Kaum war der Widerhall verklungen, als noch eine andere,
unerwartete Antwort ertönte. „Holla! Wer da?“ Die Grenzjäger traten
aus dem Wald. Sie waren nicht wenig erstaunt, hier Saltner und
Palaoro auf dem Verdeck des fremden Schiffes zu sehen.

„Grüß Gott, Herr Saltner“, rief Pitzthaler, sich auf sein Gewehr
lehnend. „Da sind’s wohl gar schon gefangen?“

„Das bin ich schon“, rief Saltner lustig. „Es tut aber nichts. Es
ist ganz schön hier.“

„Aber um die Belohnung haben’s mich gebracht. Es sind hundert
Gulden ausgesetzt.“

„Darum sollt Ihr nicht kommen. Da habt’s an Hunderter, und da noch
einen.“ Die Scheine flatterten hinab.

Von innen rief eine Stimme: „Wollen die Herren ins Schiff kommen?
Wir werden bald aufsteigen.“

Saltner und Palaoro verschwanden. Die Luken schlossen sich.

La zog Saltner in den Salon. „Du sollst deine La sehen“, sagte sie,
sich an ihn schmiegend, „die fliegende und die wandelnde, denn
beide haben ihren Herren gefunden.“

Er blickte um sich, und von dem zarten Schmuck der Wände, von dem
Reichtum der Ausstattung schweiften seine Blicke nach der wonnigen
Gestalt, die ihn umschlungen hielt.

„Es ist ein Märchen“, sagte er. „Eine Fee hat mich in ihr
Zauberschloß geführt, und ich wundre mich über nichts mehr. Und ich
würde es nicht glauben, wenn ich nicht diese Lippen – –“

„Glaubst du es nun?“ fragte La, sich endlich aus seiner Umarmung
lösend.

„Was du willst. Aber ich habe dich so unendlich viel zu fragen. Wie
konntest du mich finden? Wie kamst du auf diese Stelle? Wie kamst
du überhaupt zu diesem Schiff? Und zu diesem Menschen? Und doch
habe ich noch keine Ruhe. Die Mutter wird sich ängstigen, sie ist
noch nie in einem Luftschiff aufgestiegen. Ich glaube, wir müssen
zu ihr gehen.“

„Sei ganz ruhig. Ich verließ sie, die Hände auf dem Schoß gefaltet,
mit geschlossenen Augen im Lehnstuhl liegend. Ich schob den
Fenstervorhang vor und schickte die Magd zu ihr. Sie wird jetzt
schlafen und merkt nichts von der Fahrt. Doch ich will schnell
sehen.“

Im Augenblick war sie zur Tür geschlüpft und wieder zurück.

„Sie schläft“, sagte sie. „Und nun kannst du fragen. Doch ich will
es dir sagen.“

Und sie begann zu erzählen, von ihrem Kampf mit sich selbst, von
ihrem Entschluß, von ihrer Prüfungsreise auf der Erde – und
inzwischen löste sich das Schiff von seinem Lager, langsam sanken
die Felswände hinab, heller strahlte die Sonne –

„Wir steigen“, sagte La, sich unterbrechend.

„Und sieh“, rief Saltner, „das Nächstliegende hab ich vergessen in
der Überraschung, dich zu haben. Was hast du mit dem Schiff des
Unterkultors abgemacht? Was tust du jetzt, wenn sie von dir unsere
Auslieferung verlangen? Wie kannst du überhaupt uns befreien?“

„Sie riefen mich an, als ich hierherkam, weil sie wußten, daß ich
deinen Zug und die Verfolgung gesehen hatte, und verlangten durch
Signale, daß ich sie unterstützen sollte. Ich ging darauf ein, um
bei der Hand zu sein, und besetzte den unteren Ausgang. Ich dachte
mir, daß du hier herabkommen würdest, wenn der Weg oben versperrt
ist. Und so hab ich dich gefangen. Aber ans Ausliefern denke ich
nicht, wenn du – du – bei mir bleiben willst.“

„Und wenn sie dich zwingen? Das Gesetz ist auf ihrer Seite.“

„Gesetz gegen Gesetz – wenn du willst, wenn du bestimmst, daß ich
dein bin und du mein nach dem Gesetz der Numenheit – dann darf ich
dir das Geheimnis sagen des unverletzlichen Asyls. Doch wisse, du
darfst es nur bestimmen, wenn es dein freier Wille ist, um deinet-
und meinetwillen, nicht aber um deiner Rettung willen. Darum darfst
du nicht sorgen; ich rette dich vor jeder Gefahr, auch wenn du frei
bleiben willst ohne mich – ich muß es dir sagen, damit kein fremder
Gedanke, keine Sorge dich beeinflußt. Dieses Schiff ist das
schnellste das je gebaut worden. Niemand kann es einholen. Ich
bringe dich mit der Mutter hinüber über den Ozean, wo du sicher
bist, und auch auf den Unterhalt brauchst du nicht zu denken. Denn
ich bin nicht zur Erde gekommen, um Freiheit aufzuheben, sondern
Freiheit zu bringen, dir und mir.“

Er hatte ihr zugehört, den Blick tief in ihre Augen versenkt und
ihre Hände in den seinen haltend, und dann antwortete er:

„Ich weiß nicht, ob ich alles verstehe, aber wenn’s darauf
hinauskommt, ob es mein freier Wille ist, daß du mein Weib sein
sollst –. O La, die du das getan hast, von der Höhe deines Nu
herabzusteigen zu diesem Jammertal, um diesem Menschen das Leben
zurückzugeben –. Wie kannst du das fragen, meine La, mein Glück und
mein alles – freilich will ich’s, bestimm’ ich’s, ich, Josef
Saltner, so wahr ich hier sitze und dich in meine Arme ziehe, ich
will’s.“

„Und ich“, sagte La feierlich, „auch ich will. Und nun ist es
Gesetz, und ich bin dein. Und damit du es beweisen kannst, so komm,
Ohr an Mund, und höre, was niemand wissen darf, außer uns beiden.“

Sie flüsterte in sein Ohr, und dann barg sie das Gesicht an seiner
Schulter.

Da klopfte es am Telephon.

„Das ist der Schiffer“, sagte La. Sie warf einen Blick aus dem
Fenster. „Ah, dort ist das Regierungsschiff. Laß uns hören.“

Als sich der Unterkultor überzeugt hatte, daß Saltner mit seiner
Begleitung unter Zurücklassung des Tragstuhls und Gepäcks in die
Schlucht hinabgestiegen war und somit entweder den Feldjägern oder
dem von ihm zu Hilfe gezogenen Luftschiff nicht entgehen konnte,
begab er sich mit seinen Beds wieder nach seinem Schiff zurück.
Sobald Las Schiff über der Berglehne erschien, signalisierte er
ihm, daß es sich zu ihm begeben solle, um die Gefangenen, die er
dort vermutete, an ihn auszuliefern. La wollte sich dieser
gesetzlich begründeten Forderung nicht entziehen und ließ daher ihr
Schiff in der Nähe des Kultorschiffes sich niedersenken.
Unmittelbar darauf erschien der Beamte selbst an Bord der ›La‹ und
wurde vom Schiffer in den Salon gewiesen, in welchem er La und
Saltner fand.

Der Unterkultor war ein vornehmer Mann mit entschiedenem Wesen.
Ohne Saltner weiter zu beachten, begrüßte er La höflich und sagte,
daß er den Kommandierenden des Schiffes zu sprechen wünsche.

„Er steht vor Ihnen“, sagte La, ihn mit ruhiger Würde anblickend.
„Ich war bis vorhin Besitzerin dieser Privatyacht, habe aber jetzt
das Eigentum und das Kommando derselben abgetreten an meinen
Gemahl, Josef Saltner, dessen Name Ihnen bekannt ist und den ich
mir hiermit vorzustellen erlaube.“

Der Beamte machte eine Bewegung des Unwillens und der Überraschung.
Seine Augen wanderten prüfend über La und Saltner. Dann sagte er
kühl: „Die Höflichkeit verbietet mir, Zweifel in Ihre Worte zu
setzen. Doch muß ich Sie bitten, mir die Papiere des Schiffes und
Ihre eigene Legitimation vorzuweisen.“

La trat an den Wandschrank und reichte ihm die Papiere, die er
sorgfältig prüfte. Sie enthielten die Schenkungsurkunde Frus über
die Luftyacht ›La‹, die zu Las vollkommen freier Verfügung gestellt
war; ferner einen Freipaß vom Verkehrsministerium des Mars, gültig
für das ganze Sonnensystem und bestätigt für die Erde von Ill, dem
Protektor der Erde, und alles, was für die Legitimation Las
erforderlich war.

Der Beamte gab die Papiere ehrfurchtsvoll zurück.

„Die Legitimation ist unanfechtbar“, sagte er. „Ich freue mich, in
Ihnen die Tochter eines Mannes begrüßen zu können, dessen
technischer Tätigkeit bei der Besitzergreifung der Erde wir zu so
großem Dank verpflichtet sind. Doch“, setzte er sehr ernst hinzu,
„ich habe, wie Sie hier sehen, den Auftrag von den Residenten der
europäischen Staaten, aufgrund der gesetzmäßig geführten
Untersuchung, Josef Saltner von Bozen nebst seiner Mutter Marie und
der Magd Katharina Wackner zu verhaften. Es ist nichts darüber
bekannt, noch aus Ihren Papieren zu entnehmen, daß Saltner ihr
Gemahl sei; auch kann weder dieser Umstand, der überdies zu
beweisen wäre, noch der Aufenthalt auf diesem Schiff die Verhaftung
aufheben oder verhindern. Ich bedauere daher, dazuschreiten zu
müssen –“

Er wandte sich zu Saltner, der an der gegenüberliegenden Wand des
Salons stand, und wollte auf ihn zuschreiten, um ihn zum Zeichen
der Verhaftung zu berühren. Doch La trat dazwischen, und auf einen
Wink von ihr flüsterte Saltner einige leise Worte gegen ein kleines
Schild, das rosettenartig in der Wand angebracht war. Sofort wich
die Wand an dieser Stelle auseinander und schloß sich wieder hinter
ihm.

„Die Verhaftung ist jetzt nicht mehr möglich“, sagte La.

Der Beamte warf einen finsteren Blick auf sie. „Ich muß Sie
bitten“, sprach er, „mir dieses Zimmer zu öffnen, oder ich müßte
die Öffnung erzwingen.“

La blickte ihn stolz an.

„Das werden Sie niemals wagen“, rief sie. „Haben Sie nicht gesehen,
daß die Tür eine akustische ist, die sich nur auf das Losungswort
öffnet? Und wenn ich Ihnen sage, daß dieses Wort niemand wissen
darf, außer mir und – ihm? Werden Sie nun glauben, wer er ist?“

„So ist es“, rief der Unterkultor zurückweichend, „das ist – Ihr
–“

„Mein Zimmer.“

„Dann allerdings. Der Beweis ist geführt. Dieser Raum ist
unverletzlich.“

Er lächelte gezwungen.

„Und ich glaube, unsere Unterhandlungen sind damit erledigt“, sagte
La kalt.

„Nicht ganz“, erwiderte der Beamte nach kurzem Schweigen. „Doch
fürchten Sie nicht, daß ich Sie aufhalte. Geben Sie nur Auftrag,
mich zu Frau Saltner und ihrer Magd zu führen. Diese Personen
können Sie nicht schützen.“

La wollte entrüstet erwidern. Doch erschrocken hielt sie inne.
Jetzt war das Gesetz auf seiner Seite. Sie stand stumm.

„Sie werden sich nicht weigern“, sagte er.

„Und wenn ich es tue?“

„So muß ich Gewalt gebrauchen. Ich werde das Schiff durchsuchen
lassen.“

Er schritt nach der Tür, um die Beds zu rufen, die vor dem Schiff
auf seine Befehle warteten. Zu diesem Zweck mußte er auf das
Verdeck steigen, von wo die Landungstreppe nach außen ging.

La klopfte das Herz. Was sollte sie tun? Bis jetzt hatte sie die
Gesetze nicht verletzt. Aber wie sollte sie die Mutter schützen?

\tb{}
Da öffnete sich die Tür ihres Zimmers. Saltner stand neben ihr.
Rasche Worte bestätigten die Vermutung, die ihn ohne Rücksicht auf
seine Sicherheit herausgetrieben hatte, um La und der Mutter zu
Hilfe zu eilen.

„Wir werfen die Leute hinaus!“ rief er.

„Beim Nu, ich bitte dich, das dürfen wir nicht.“

„Warum nicht? Ich darf mich ja doch nicht mehr hier sehen lassen.“

„Aber Gewalt, das ist etwas anderes. Es versperrt uns die Rückkehr
zum Nu, es beraubt dich deines Bürgerrechts.“

„Und doch sehe ich keinen andern Ausweg. Den Nu oder mich! Wenn der
Mann nicht freiwillig geht, wirst du wählen müssen.“

La blickte ihn an, die Hände zusammenpressend. Dann warf sie die
Arme um seinen Hals.

„Dich, dich!“ rief sie.

„Habe ich das Kommando?“

„Ja, ja.“

Saltner sprang dem Beamten nach. La folgte pochenden Herzens. Der
Unterkultor stand auf dem Verdeck, er winkte den Beds.

„Wie wird die Treppe aufgezogen?“ fragte Saltner La hastig.

„Vom Steuerraum aus automatisch.“

„Sage dem Schiffer, daß er sich bereithält. Hoffentlich verläßt der
Kultor das Schiff. Wenn nicht, bleibt doch nichts übrig, als ihn
hinauszuwerfen.“

„Sal – er ist bewaffnet – ich bitte dich –“

Leise stieg Saltner die Treppe zum Verdeck hinauf.

Die Beds hatten nicht sogleich die Winke des Beamten bemerkt, weil
sie ihre Aufmerksamkeit nach der entgegengesetzten Seite in die
Luft gerichtet hatten. Dort zeigte sich in großer Höhe von Südosten
her ein dunkler Punkt, das andre Schiff der Martier, das jetzt die
beiden Schiffe auf dem Bergrücken bemerkt hatte und, ohne sich zu
übereilen, auf sie zuhielt. Es war ein Stationsschiff aus Rom,
eines jener großen und furchtbar schnellen Kriegsschiffe, mit allen
Waffen ausgerüstet, wie sie in den Hauptstädten der Erde den
obersten Beamten zur Erhöhung der Autorität des Nu beigegeben
waren.

Ein paar rasche Schritte brachten Saltner hinter den Kultor. Dieser
wandte sich nach ihm um, aber in demselben Augenblick hatte Saltner
ihm mit einem raschen Griff den Telelytrevolver aus der Tasche
gerissen und ihn weit hinweggeschleudert.

„Was wagen Sie?“ rief der Kultor. „ich verhafte Sie –“

„Bedaure sehr – verlassen Sie sofort das Schiff, wenn Sie nicht
eine unfreiwillige Spazierfahrt machen wollen –“

Auf den Ruf des Kultors waren die Beds aufmerksam geworden, sie
blickten her. Saltner durfte ihnen keine Zeit lassen, sich mit dem
Kultor zu verständigen, denn wenn sie von ihren Telelytwaffen
Gebrauch machten, war er verloren. Er kommandierte: „Die Treppe
herauf! Aufsteigen! Schnell!“

Im Augenblick schlug sich die Treppe in die Höhe und schob sich auf
dem Verdeck ineinander, während das Schiff in die Höhe schoß. Die
Beds sahen ihm erstaunt nach, wußten aber nicht, was sie tun
sollten, da Palaoro gleichzeitig auf einen Wink Saltners den Kultor
ins Innere des Schiffes gezogen hatte. Sein Protest wurde nicht
beachtet.

„Was machen wir mit dem Mann?“ sagte Saltner. „Wir wollen ihn doch
nicht mitschleppen. Dort hinter dem Felsvorsprung können uns die
Beds und das kleine Schiff nicht sehen. Dort setzen wir ihn ab. Mag
er schauen, wie er heimkommt.“ Saltner erteilte dem Schiffer die
nötigen Befehle. Nach zwei Minuten lag das Schiff wieder still.

Der Kultor stieg in stummem Ingrimm die Schiffstreppe hinab, die
sich sofort wieder hob.

„Nehmen Sie’s nicht übel, Herr Kultor“, rief ihm Saltner nach.
„Aber es ging nicht anders. Habe die Ehre.“

Der Kultor wandte sich um. „Ich warne Sie“, rief er wütend.
„Ergeben Sie sich noch jetzt. Ich lasse Sie sonst rücksichtslos
durch das Kriegsschiff verfolgen und vernichten.“

„Tut mir leid“, antwortete Saltner. „Muß jetzt notwendig auf meine
Hochzeitsreise. B’hüt euch Gott.“

Man konnte nicht mehr verstehen, was der Kultor erwiderte, die ›La‹
war schon wieder zu hoch gestiegen. Aber man sah, daß das
Kriegsschiff auf den Ort zuhielt, wo es den Kultor bemerkt hatte,
der ihm mit den Armen winkte. Auch das kleinere Schiff erschien
jetzt.

La war neben Saltner getreten. „Komm herab“, sagte sie,

„wir müssen die Luken schließen und uns beeilen. Das Schiff dort
ist ein schnelles Kriegsschiff, wir können ihm nur durch
schleunigste Flucht entgehen.“

Saltner warf einen Blick zurück, dann umfaßte er La und sprang, sie
in die Höhe hebend, die Treppe hinab, auf der jetzt Marsschwere
herrschte.

„Wenn wir ausreißen müssen, so übernimm du wieder den Oberbefehl.
Ich weiß ohnehin nicht, wohin wir eigentlich wollen.“

„Die Luken zu!“ befahl La. „Volle Diabarie!“

Die ›La‹ schoß senkrecht in die Höhe. Schnell war sie bedeutend
höher als das niedrig schwebende Kriegsschiff. Aber dieses erhob
sich jetzt schräg und gewann, da es in voller Fahrt war, bald einen
Vorsprung nach Norden. Es kehrte nun in einem Bogen zurück, um der
›La‹ den Weg abzuschneiden. Es hatte gar nicht angelegt, um den
Kultor aufzunehmen, da inzwischen dessen eigenes Schiff
eingetroffen war, von dem aus er sich mit dem Kriegsschiff durch
Signale verständigte.

Die ›La‹ stieg weiter kerzengerade empor, während das Kriegsschiff
ihr in immer engeren Spiralen folgte. Der Horizont erweiterte sich
schnell, schon lagen die Bergriesen der Alpen tief unten, die
Eishäupter der Ortlergruppe erschienen als flache Schneehügel; im
Norden und Süden tauchten die Ebenen auf und verschwammen mit der
Luft des Himmels. Palaoro war bei dem zweiten Schiffer im
Steuerraum. In den drei Räumen, in denen sich Menschen befanden,
wurden die Sauerstoffapparate in Tätigkeit gesetzt, um die Luft
atembar zu erhalten. Die Höhe von zwölf Kilometern war erreicht.
Fern im Westen schien der Himmel von Wolken bedeckt zu sein. „Dort
müssen wir hin“, sagte La. „Im Nebel können wir die Richtung
ändern, ohne daß es bemerkt wird. Wir müßten sonst vielleicht die
Flucht so weit fortsetzen, bis wir in den Erdschatten kommen, und
das führt uns zu weit vom Ziel ab.“

„Und welches ist das Ziel?“

„Berlin.“

„Aber La?“

„Du sollst alles hören. Erst aber wollen wir einmal sehen, ob das
Kriegsschiff uns nachkommen kann. Richtung nach West! Voll
Repulsit!“ sagte sie zum Schiffer.

Das Schiff wandte seine Spitze nach Westen mit einer sanften
Neigung nach oben. Der Reaktionsapparat wirkte. Es sauste durch den
luftverdünnten Raum. Die Geschwindigkeit steigerte sich allmählich
auf 400 Meter in der Sekunde.

Das Kriegsschiff war der ›La‹ gefolgt. Sobald es erkannt hatte, in
welcher Richtung die ›La‹ zu entkommen suchte, schlug es
ebendieselbe ein. Aber nun zeigte sich die Überlegenheit der Yacht.
Die Entfernung von dem Verfolger wuchs schnell. Nach fünfundzwanzig
Minuten hatte das Schiff einen Weg von 600 Kilometern zurückgelegt.
Von der Erde erblickte man nichts, eine dichte Wolkendecke lagerte
hier unten. Das Kriegsschiff war nur noch als ein Punkt zu
erkennen. Nach weiteren fünf Minuten umhüllten Wolken die Yacht.
Alsbald wurde der Lauf gemäßigt. „Wenden Sie sofort“, sagte La zum
Schiffer, „und benutzen Sie die Wolken soweit wie möglich nach
Nordost. Kommen wir aus den Wolken heraus, und ist dann das
Kriegsschiff nicht mehr sichtbar, so fahren Sie so schnell wie
möglich nach Berlin. Dort wird man uns zunächst auf keinen Fall
suchen.“

„Das scheint mir doch fraglich“, sagte Saltner. „Sobald das
Kriegsschiff sieht, daß wir ihm entkommen, wird es nach der
nächsten Stadt hinabgehen und nach allen Richtungen telegraphieren.
Man wird uns, wo wir hingelangen, sofort erkennen.

Es wird wohl also nichts übrig bleiben, als bis über Europa
hinauszugehen.“

„Das ist wahr. Wir können erst in der Dunkelheit nach Berlin. Aber
wo bleiben wir so lange? Wir wollen doch nicht immerfort hier in
den Wolken herumfahren?“

„Warum willst du nicht sogleich nach Amerika?“

„Ich werde es dir dann erklären.“

„Wo sind wir denn eigentlich?“

„Wir müssen mitten in Frankreich sein. Wir wollen hinab und uns
einmal umsehen.“

„Dann laß uns doch lieber nach irgendeinem abgelegenen Gebirge
gehen, wo es einsam ist und so bald keine Nachrichten hinkommen,
dort können wir warten, bis es Zeit ist, nach Berlin zu reisen.“

„Du hast recht. Fahren Sie also weiter nach Südwest, mit mäßiger
Geschwindigkeit, und suchen Sie auf den Pyrenäen einen guten
Landeplatz. Dort warten wir bis gegen Abend. Dann gelangen wir
gerade zur rechten Zeit nach Berlin. Und jetzt komm! Wir wollen
einmal nach der Mutter sehen, und dann – ich habe dir noch soviel
zu erzählen. Und es ist auch noch jemand hier, den du begrüßen
mußt.“

Se trat ihnen im Salon entgegen.

„Sind wir endlich in Sicherheit?“ fragte sie. Und Saltner die Hand
reichend, fuhr sie lächelnd fort: „Sobald man mit Ihnen
zusammenkommt, ist man seines Lebens nicht sicher.“

„Seien Sie mir nicht böse. Ich werde von nun ab ganz vernünftig
werden.“

„Bei soviel Glück?“

„Ja, es macht mich bescheiden.“

\section{57 - Das Spiel verloren}

Die letzte Woche war für Ell im höchsten Grad aufregend gewesen. Er
arbeitete von früh bis spät in die Nacht und konnte doch die Last
verantwortungsvoller Entscheidungen nicht bewältigen, die ihn
bedrückte und seine ganze Tatkraft in Anspruch nahm. Er fühlte, wie
eine nervöse Abspannung sich seiner bemächtigte, deren er nicht
Herr zu werden vermochte. Selbst zu einem Besuch bei Isma, nach
welchem er sich sehnte, hatte er noch nicht Zeit finden können.

Die Übergriffe der Beamten hatten sich wiederholt. Es war nicht
immer ein krankhafter Zustand, ausgesprochener Erdkoller, wie bei
Oß, der dazu Veranlassung gab, sondern eine schärfere Tonart begann
Platz zu greifen, die leicht zu Konflikten führte. Und dies kam
daher, daß die auf der Erde angestellten Nume eine starke Partei
auf dem Mars hinter sich wußten. Die Antibaten, welche auf ein
härteres und entschiedeneres Vorgehen gegen die Menschen als eine
untergeordnete und nur durch Gewalt zu zügelnde Rasse drangen,
hatten im Parlament wie im Zentralrat an Einfluß gewonnen. Ells
Tätigkeit bot ihnen einen willkommenen Angriffspunkt, auf den sie
zunächst ihre Kräfte richteten. Die Strenge, mit welcher Ell jedem
Übergriff der Instruktoren und Beamten entgegentrat, wurde in der
Presse in übertriebener Weise erörtert und als eine
Voreingenommenheit für die Menschen hingestellt und getadelt. Die
Absetzung von Oß, die sofort nach der vom Wiener Unterkultor
vorgenommenen Untersuchung verfügt worden war, wurde besonders
aufgebauscht, da Oß eine angesehene und als Techniker um den Staat
verdiente Persönlichkeit war. Schon daß sich Saltner durch die
Flucht auf die Berge der Strafe entzogen hatte, war als ein Zeichen
von Nachlässigkeit gedeutet und Ell zum Vorwurf gemacht worden.
Unter diesem Druck, auf den Ell nicht Rücksicht nehmen wollte,
hatte der italienische Kultor das Kriegsschiff zur Verfügung
gestellt, um Saltner aufzusuchen.

Die gegen die Menschen gerichtete Strömung auf dem Mars war ja
nichts Neues. Ell hatte stets mit ihr rechnen müssen, und er hatte
ihr Anwachsen mit Besorgnis verfolgt. Doch vertraute er fest auf
die Macht der Vernunft in den Numen und die Reinheit seiner eigenen
Absichten, und in diesem Glauben hatte ihn Isma aus innerstem
Herzen bestärkt. Jetzt aber begannen die direkten Angriffe auf ihn
offener hervorzutreten, und er hatte zu seinen übrigen Arbeiten
seine Verteidigung in der Presse zu führen. Ein lebhafter Wechsel
von Lichtdepeschen, die alle über den Nordpol nach dem Mars gingen,
fand zwischen Berlin und Kla statt.

Aber ganz ohne Einfluß auf Ell war dieser vom Mars, das heißt von
einem Teil seiner Bevölkerung ausgeübte Druck doch nicht geblieben.
Er sah sich veranlaßt, die ihm zu Gebote stehenden Machtmittel
rücksichtsloser zu gebrauchen, und je mehr ihn das Mißverständnis
und der Tadel seiner Handlungen verdroß, um so mehr gewöhnte er
sich, auf seine eigenen Entscheidungen und Entschlüsse zu vertrauen
und jede anderweitige Beratung abzulehnen. Mit Erschrecken sagte er
sich zuweilen im stillen, daß die Furcht, er werde dahin kommen,
sein Kultoramt in autokratischer Weise zu handhaben, nicht
unberechtigt sei. Und immer wieder nahm er sich vor, unter keinen
Umständen sich dazu drängen zu lassen, als Selbstherrscher zu
verfahren oder den antibatischen Forderungen nachzugeben.

\tb{}
Der einflußreichste Teil der Martier ging ja, wie bei der
Besitznahme, so auch bei der Behandlung der Erde von rein idealem
Gesichtspunkt aus. Die Kultur des Mars, den Geist der Numenheit auf
der Erde zu verbreiten, war ihr Ziel; eine Beherrschung der
Menschheit, soweit sie notwendig schien, nur ein vorübergehendes
Mittel, eine Art notwendigen Übels. Aber gerade hier verstimmte es
vielfach, daß die Menschen im großen und ganzen so wenig
Entgegenkommen und Verständnis für das zeigten, was die Martier
ihnen bringen wollten. Man erkannte wohl Ells Tätigkeit in gewisser
Hinsicht an, aber man hielt seinen Weg doch für etwas umständlich.
Die Hebung der Bildung konnte natürlich nur allmählich geschehen,
und sie war die notwendige Vorbedingung für das Gelingen des
zivilisatorischen Werkes, das die Nume an der Erde ausführen
wollten. Aber man meinte, daß eine entschiedenere Wegräumung der
Hindernisse hinzukommen müsse. Ein solches Hindernis sah man in
Deutschland noch immer vor allem in der politischen Übermacht der
reaktionären Parteien. Man verlangte einen entschiedenen Bruch mit
den oligarchischen Traditionen, die sich von dem Gedanken einer
bevorzugten und herrschenden Klasse nicht trennen konnten. Man
wünschte hier ein entschiedenes Vorgehen, das aber wieder ohne
Anwendung von Gewaltmaßregeln, die Ell verhüten wollte, nicht
möglich gewesen wäre.

Ein weiterer Grund zur Unzufriedenheit, der sich allerdings gegen
die Regierung der Zentralstaaten selbst und gegen Ill als den
Protektor der Erde wendete, war die bisherige Beschränkung der
Kulturtätigkeit auf die westeuropäischen Staaten. Man verlangte die
effektive Ausdehnung des Protektorats auf die ganze Erde, vor allem
die Einbeziehung Rußlands und der Vereinigten Staaten von
Nordamerika. Ill sah voraus, daß dies zu neuen schweren Kämpfen
geführt hätte; er hoffte, sie zu vermeiden, wenn man es der Zeit
überließ, von selbst den Einfluß zu gewinnen, der bei der
kulturellen Überlegenheit der Martier auf die Dauer nicht
ausbleiben konnte. Aber diese weise Zögerung hatte doch zur Folge,
ungeduldigere Köpfe der antibatischen Strömung zuzuführen. Ihre
hauptsächliche Kraft indessen zog die Partei der Antibaten aus dem
Teil der Bevölkerung, in welchem die idealen Kulturziele von
eigennützigen Absichten getrübt waren. Zwar hatte man auf dem Mars
geglaubt, für immer der Gefahr enthoben zu sein, daß der reine
Wille der Vernunft zum Guten in Kampf geraten könne mit
selbstischen Interessen, mit dem Bestreben, wenn auch nicht für den
einzelnen, so doch für den Staat, Vorteile auf Kosten der
Gerechtigkeit gegen alles Lebendige zu gewinnen. Aber sobald mit
der Erschließung der Erde das Gefühl der Macht und die Möglichkeit
sich eingestellt hatte, Wesen, die man nicht für seinesgleichen
hielt, auszubeuten, erhoben sich in den weniger hochstehenden
Elementen der Bevölkerung, an denen es nicht fehlte, wieder jene
niederen Instinkte eines unter dem schönen Namen des Patriotismus
sich verbergenden Egoismus. Man erklärte es für eine nationale
Pflicht des Martiers, von der Erde alles zu gewinnen, was das
wirtschaftliche Interesse des Mars irgend daraus ziehen konnte. Mit
einem Worte, was man wollte, war nichts anderes als die Erhöhung
der Revenuen des Mars, aber nicht bloß durch den berechtigten
Handelsverkehr, sondern durch die direkte Arbeit der Menschen für
die Martier. Zwar hatte man schon bedeutende Energiemengen von der
Erde bezogen durch Anlegung von Strahlungsfeldern in Tibet, in
Arabien und in den äquatorialen Gegenden von Afrika. Aber diese
wurden von martischen Gesellschaften auf ihre Kosten, obwohl mit
hohem Gewinn, betrieben. Man wollte jedoch von Staats wegen zur
Erhöhung der Privatrente aller Bürger eine Besteuerung der
Menschen, um diese zu größerer Arbeit im Sinne der Martier zu
zwingen. Man führte aus, daß die Menschen bei ihrer natürlichen
Indolenz nur dann dazu gebracht werden könnten, sich die Technik
der Martier und damit ihre Kultur anzueignen, wenn man sie durch
eine hohe Steuer veranlasse, die von der Sonne ihnen zuströmende
Energie unter Anleitung der Martier auch wirklich auszunutzen.

Auf Grund dieser populären Erwägung wollte jetzt die
Antibatenpartei ihren ersten größeren Schlag führen. Er sollte, wie
sich wenigstens die Entschiedenen sagten, nur die Vorbereitung
sein, um die den Menschen zugesprochenen Rechte sittlicher
Persönlichkeiten überhaupt in Zweifel zu stellen und schließlich
aufzuheben. Die Erde sollte zu einer Werkstatt für die Erhaltung
des Mars durch eine Riesenrente erniedrigt werden. Diese letzten
Gesichtspunkte wurden zwar noch verschleiert, aber die Gegner
enthüllten sie in ihrer ganzen Unsittlichkeit und Torheit, ohne
doch die Anhänger einer Besteuerung der Menschen überzeugen zu
können, welche schiefe Bahn sie zu beschreiten im Begriff waren.

Gestern hatte Ell die Nachricht erhalten, daß der Antrag
eingebracht worden war – und nicht ohne Aussicht auf Annahme –, für
die westeuropäischen Staaten eine vorläufige Jahressteuer von 5.000
Millionen Mark anzusetzen, indem man anführte, daß die Hälfte davon
bereits durch die Verminderung der Militärlasten gedeckt sei.
Außerdem sollten von nun ab die Menschen die Kosten der martischen
Verwaltung selbst tragen, was man in Rücksicht auf die zu zahlenden
Schulgelder auf ebensoviel veranschlagen mußte.

Ell sagte sich, daß eine solche Maßregel, wenn sie sich
verwirklichen sollte, ihm die Fortführung seines Amtes unmöglich
machen würde. Sein ganzes Streben war auf die Versöhnung, auf die
freiwillige Anpassung der Menschen an die Kulturwelt des Mars
gerichtet. Der Aufruf zur Begründung eines allgemeinen
Menschenbundes, der zwar die Befreiung der Erde von der Herrschaft
des Mars anstrebte, aber durch ein Mittel, das zu demselben
versöhnenden Ziel führen sollte, das er ersehnte, war ihm daher
willkommen. Er tat nichts, um diesen Ideen und ihrer Ausbreitung
entgegenzutreten. Bei der Antibatenpartei auf dem Mars wurden
jedoch die Tendenzen des Menschenbundes unter ganz anderem
Gesichtspunkt betrachtet. Hier wollte man ja nicht das Kulturheil
der Erde, sondern ihre Ausnutzung, und man sah daher in dem Bund
eine große Gefahr, ein revolutionäres Unternehmen. Die neue
Steuerlast, die man den Menschen zudachte, sollte sie belehren, daß
sie auf keine freiwillige Aufgabe der Mars-Herrschaft zu rechnen
hätten. Siegten die Antibaten mit ihrem Vorhaben, so mußte dies auf
der Erde erneute Erbitterung hervorrufen und die Menschen über die
Absichten der Nume enttäuschen. Es würde also der von Ell
angestrebten Versöhnung entgegengewirkt werden, und seine eigene
Arbeit wäre nicht nur in Frage gestellt, sondern es wäre damit auch
sein vom Zentralrat gebilligter Plan gewissermaßen zurückgewiesen
worden. Ell mußte sich die Frage vorlegen, ob er dann seine
Tätigkeit noch für nutzbringend halten dürfte.

Heute nun brachten ihm die neuen Zeitungen vom Mars ihn persönlich
kränkende Nachricht. Man hatte ihm in einer Parlamentsrede seine
Abstammung von den Menschen mütterlicherseits zum Vorwurf gemacht
und die Regierung getadelt, daß sie einen Mann in eine so
verantwortliche Stellung eingesetzt habe, dem man als Halb-Numen
kein Vertrauen entgegenbringen könne.

Ell ging entrüstet in seinem Zimmer auf und ab. Sollte er nicht
diesen Leuten sein Amt vor die Füße werfen? Aber das hieße die
Sache aufgeben, der er sein Leben gewidmet hatte. Durfte er nicht
hoffen, wenn er selbst festhielt, doch seine Ansicht zum Sieg zu
führen und Gutes auf der Erde zu wirken? Ja, wenn er sich nur
selbst sicherer gefühlt hätte. Wenn nicht in jenem Vorwurf ein Kern
von Wahrheit gelegen hätte! War nicht seine Heimat auf zwei
Planeten, und hatte er die Kraft gehabt, im entscheidenden Moment
allein der Stimme der Numenheit zu folgen, die ihn hieß, nichts
anderes im Auge zu haben, als die große Aufgabe, das Verständnis
der Planeten anzubahnen? Hatte er nicht als ein schwacher Mensch
geschwankt in seiner Pflicht, ganz sich selbst zu vergessen um des
Ganzen willen, hatte er nicht seiner Neigung Gehör gegeben und der
Freundin zuliebe die Erde verlassen, wo er hätte wirken sollen?
Gewiß, es war nicht seine Absicht gewesen, sich dieser Pflicht zu
entziehen, äußere Umstände hatten ihn verhindert, rechtzeitig zu
ihr zurückzukehren. Aber eben diesen Umständen durfte er keinen
Spielraum des Zufalls gestatten, er hätte sich der Gefahr nicht
aussetzen dürfen, die Pflicht zu versäumen. Das war seine Schuld.
Er hatte eine Schuld auf sich geladen. Durfte er dann noch sich als
den Mann betrachten, der hoch genug stand, um die Kultur zweier
Planeten zu vermitteln? Durfte er sich die Kraft zutrauen,
gegenüber den Angriffen von beiden Seiten die Verantwortung zu
tragen und die Machtfülle nicht durch menschliche Leidenschaften zu
entstellen?

In solchen Gedanken wandelte sich seine Entrüstung in ernste
Selbstprüfung, und immer wieder erwog er die Frage, ob er der
Sache, die er durchzufechten entschlossen war, auch wohl an diesem
Platz noch die rechten Dienste zu erweisen vermöge.

Da wurde ihm der Unterkultor von Wien gemeldet.

Als das Luftschiff Las, von dem Kriegsschiff verfolgt, den Blicken
der Martier entschwunden war, hatte sich der Beamte sofort auf den
Weg nach Berlin gemacht. Drei Stunden später war er dort angelangt.
Er wurde sogleich vorgelassen. Empört beklagte er sich über die
Behandlung, die sich Saltner gegen ihn herausgenommen, und
verlangte die volle Strenge des Gesetzes gegen den Frevler, an
dessen Ergreifung er nicht zweifelte.

Ell glaubte seinen Ohren nicht trauen zu dürfen, so überraschte ihn
das, was er hören mußte.

„La?“ fragte er. „Sind Sie auch sicher, La, die Tochter von Fru,
des technischen Direktors im Ministerium für Raumschiffahrt? Sie
hat Saltner in aller Form für ihren Gemahl, nach dem Rechte des Nu,
erklärt und ihn in ihrem Luftschiff entführt?“

„Es ist kein Zweifel, die Papiere waren in Ordnung, der Beweis –
wie ich ihnen sagte. Und dieser Bat wagte es, mich anzufassen, mich
mit Gewalt ins Schiff hinabzuziehen, mich auszusetzen und mir
höhnische Worte nachzurufen. Aber Sie werden –“

„Ich werde dem Gesetz gemäß verfahren. Ich bedaure tief dieses
Ereignis. Entschuldigen Sie mich jetzt, aber halten Sie sich,
bitte, in der Nähe, daß ich Sie eventuell noch einmal sprechen
kann, ehe Sie nach Wien zurückkehren. Ich danke Ihnen für Ihren
Bericht, Sie haben Ihrerseits korrekt gehandelt, Sie konnten nicht
wissen, daß das Luftschiff Freunde und Helfer Saltners barg. Sorgen
Sie dafür, daß eine etwaige Nachricht von dem verfolgenden Schiff
mir sogleich mitgeteilt wird.“

Der Beamte hatte noch nicht die Tür erreicht, als das Signal am
Depeschentisch erklang und zwei Telegramme, die mit eilig
bezeichnet waren, sich auf die Platte desselben schoben.

Ell riß das erste auf und rief sogleich den Unterkultor zurück.

„Aus Lyon, vom Kommandanten des Kriegsschiffs“, sagte er. „Die
Luftyacht ›La‹, mit unerreichbarer Geschwindigkeit fliegend, ist in
einer unübersehbaren Wolkendecke verschwunden und konnte nicht mehr
aufgefunden werden.“

Der Beamte stand starr.

„Ihrer Rückkehr nach Wien steht nun vorläufig nichts entgegen“,
sagte Ell. „Das weitere werde ich veranlassen. Leben Sie wohl.“

Sobald Ell allein war, ließ er sich auf seinen Stuhl sinken und
stützte die Hände in den Kopf.

Das hatte La getan! Er konnte es nicht begreifen. Um Saltners
willen! Er sah sie vor sich, wie sie damals, als er auf dem Nu mit
ihm stritt, Saltners mannhaftes Eintreten für das Vaterland mit
einem Kuß belohnte, und eine Regung von Neid stieg in ihm auf, die
er gewaltsam zurückdrängte. Mochte sie! Der Vorgang hatte für ihn
eine ganz andere Bedeutung. Das war offne Auflehnung gegen die
Herrschaft der Nume auf der Erde. Was Saltner getan hatte,
freilich, das sah ihm ganz ähnlich, das mochte er selbst
verantworten, ja er konnte es ihm nicht einmal verdenken. Und er
hätte es ihm herzlich gegönnt, daß ihm die Flucht glücke. Gegen ihn
einschreiten zu müssen, war ihm ein peinlicher Gedanke. Ja, wenn es
Saltner aus eigner Kraft gelungen wäre, sich der Verfolgung zu
entziehen! Aber daß es durch Las Hilfe geschehen mußte! Daß sie
sich dazu hergab, den Schuldigen der Macht des Gesetzes zu
entreißen! Wie konnte sie das vor sich selbst verantworten? Mag
sein, daß sie sich keines ungesetzlichen Mittels bedient hatte, mag
sein, daß sie glaubte, in gutem Recht bei ihrer Selbsthilfe zu
handeln. Aber die Beihilfe zur Flucht war doch ein Faktum, das
blieb. Und diesen Mann band sie in aller Form an sich – La, die
Tochter Frus –, was mußte das wieder auf dem Mars für Aufsehen
erregen! Daraus würden die Gegner Kapital schlagen. Schließlich
würde man natürlich Ell verantwortlich machen, daß der Geist der
Widersetzlichkeit nicht nur bei den Menschen geduldet werde,
sondern sich durch die Berührung mit ihnen sogar auf die Nume
fortpflanze. Und was würde La tun? Wohin wollte sie sich flüchten?
Wenn sie nach dem Mars ging oder nach fremden Teilen der Erde,
welch schwierige Auseinandersetzungen, Verhandlungen, neue
Angriffspunkte ergaben sich da?

Gab es denn heute keine Ruhe für ihn? Er mußte sie suchen. Wo? Zu
Isma! Er wollte zu Isma. Er erhob sich. Da fiel sein Blick auf das
zweite Telegramm. Mochte es liegen bleiben! Doch nein, das ging
nicht, vielleicht war es doch wichtig. Er brach es auf. Oh, wie
lang!

„Kalkutta. dots{} Der Kommissar der Marsstaaten hat die Ehre zu
melden, daß es geglückt ist, unzweifelhafte Spuren des gesuchten
Hugo Torm aufzufinden und daß die Beweise vorliegen. Torm war der
Fremde, der wiederholt in den Verhandlungen mit Tibet erwähnt wurde
und sich längere Zeit in Lhasa aufgehalten hat. Es sind Leute
ermittelt worden, die mit ihm die Reise nach Kalkutta gemacht haben
und sich im Besitz von Gegenständen befinden, die sie von Torm
erhielten. Hier konnte festgestellt werden, daß Torm am 18. oder
19. August das Post-Luftschiff nach London benutzt hat. Sein
gegenwärtiger Aufenthalt konnte hier nicht ermittelt werden.“

Ell sank auf seinen Platz zurück.

Torm lebte! Daran war nun kein Zweifel mehr möglich.

Ell fühlte, wie sich ihm das Blut in den Kopf drängte, wie seine
Gedanken sich verwirrten – –. Und jetzt brauchte er Klarheit,
volle, nüchterne Klarheit!

Warum freute er sich denn nicht? Er mußte sich doch freuen, daß der
bewährte Freund, der verdiente Forscher, der Mensch überhaupt
gerettet war, und vor allem, daß – –

Ja, er wollte ja zu Isma. Er wollte bei ihr Ruhe suchen und Trost.
Jetzt konnte er sie ihr bringen. Jetzt konnte er ihre Hände fassen
und ihr sagen: „Freue dich, Isma, er lebt!“ Und er sah, wie sie die
Augen aufschlug und ihn ungläubig ansah und er wieder sagte: „Er
lebt“, und wie die blauen Augen sich mit Tränen füllten und sie
aufschrie: „Er lebt!“, und wie sie an seine Brust sank und den Kopf
an seine Schulter lehnte und schluchzte: „O mein Freund, mein
Freund! Ich bin so glücklich!“ Und es war ihm, als müßte er sie von
sich stoßen, und doch war es solche Seligkeit, sie an sich zu
pressen und die Lippen auf ihr Haar zu drücken, und zu sehen, wie
dies geliebte Wesen sich nicht zu fassen wußte im unerhofften Glück
– –. Warum freute er sich denn nicht? Warum zögerte er auch nur
einen Augenblick? Also vorwärts!

Er stand wohl auf, er schritt auf und ab, er blieb vor dem Telephon
stehen, aber er konnte sich nicht entschließen, nach dem Wagen zu
rufen. Nein, er konnte sich nicht freuen, er wollte nicht! Das
Glück war ihm so nahe, die erträumte Zukunft so schön – und es
sollte nicht sein? Aber was war denn geschehen? Würde es nicht so
sein, wie es immer gewesen war? Würde sie ihn weniger lieb haben?
Würde er sie nicht sehen, so oft er wollte? Hatte er sie je anders
begehrt? Wußte er nicht seit Jahren, daß sie ihm nie anders gehören
würde, und war er nicht glücklich gewesen trotz alledem mit der
treuen Freundin?

Doch, es war anders, es war eben nicht mehr so wie früher. Er wußte
es, sie selbst hatte sich frei gefühlt, sie hatte sich mit dem
Gedanken vertraut gemacht, daß sie den Gatten nie wiedersehen
würde, sie hatte den Schmerz durchlebt und langsam sich gewöhnt, an
den Verschollenen zu denken als an einen Verlorenen. Und wenn sie
je die Zukunft erwog, so sah sie einen andern neben sich. Und er,
Ell, er glaubte nur zu sicher zu wissen, daß diese Zukunft ihm
gehörte, zu fest hatte sich die Hoffnung in ihm gegründet, daß er
sie nun bald sein nennen würde in einem andern Sinne, ganz sein. Er
mochte das namenlose Glück nicht ausdenken, nur das wußte er, wie
viel leichter er dann die Schwere seines Ringens und Kämpfens
ertragen würde. – – Ja, es war anders geworden, er sah schon lange
nicht mehr in ihr die Freundin, der er geschworen hatte zu dienen
ohne Verlangen. In verzehrenden Flammen loderte in ihm die
Leidenschaft, sie zu besitzen! Sie wieder zurückkehren zu sehen in
die Arme eines andern – nein, es war nicht mehr möglich. Es konnte
nicht mehr so sein, nimmermehr konnte er neben ihr hergehen in
ehrlicher Entsagung. Wenn er jetzt die Geliebte verlor, so verlor
er auch die Freundin, so hatte er sie ganz und auf immer verloren
–. Dann mußte er fort, er durfte sie nicht mehr sehen – sie war ihm
verloren – verloren – –

Und das sollte er ertragen? Und das sollte er dulden? Und dabei
wissen, daß sie ihn liebte? Wo war denn der Mann? Er war ja nicht
da. Zurückgekehrt, ein Totgeglaubter, und der erste Schritt war
nicht zu seiner Frau? Warum kam er nicht und nahm sein Recht in
Besitz? Warum verbarg er sich? Kam er vielleicht doch niemals
wieder? Und wäre dieser Kampf mit sich selbst und der Sturm, den
die Nachricht in Ismas Herzen erregte, wären sie unnütz, zwecklos?
Doch nein, die Nachricht war zu sicher. Aus einem Postluftschiff
der Martier steigt man an einer Station aus, aber man verunglückt
nicht. Und wenn man in einem der zivilisierten Staaten ausgestiegen
ist, so verschwindet man nicht spurlos, wenn man nicht will, wenn
man nicht gute Gründe dazu hat. Warum also verbirgt sich Torm? Nur
in seinem Gewissen muß der Grund liegen, er muß etwas getan haben,
das ihn zur Flucht vor der Welt veranlaßt. Aber warum auch vor
Isma? Also auch vor ihr muß er sich scheuen? Er will vielleicht gar
nicht zu ihr? Offenbar, er will nicht! Und vor diesem Mann, der
vielleicht Ismas nicht mehr würdig ist, der sich vor ihr verbirgt,
sollte er, Ell, das Feld räumen? Wenn Torm sich gegen die Nume
vergangen hatte, so war es Ells Pflicht, dies zu sühnen. Welche
Rücksicht sollte er nehmen, wenn Torm selbst seine Rechte aufgab
oder das Recht der Nume sie ihm absprach? Und deshalb sollte Ell
den grausamen Verzicht auf sich nehmen, der ihm das Liebste, das
Teuerste entriß, der ihm die Wurzel im innersten Gemüt zerstörte,
aus dem seine Energie, sein Mut, sein Vertrauen, die ganze große
Aufgabe seines Lebens die besten Kräfte zog?

Ell ballte die Faust. „Hab ich mein Sein hingegeben für die Sache,
so will ich auch mein Glück mir erobern! Wo ist er, dem ich sie
geben soll, die ich mir verdient habe, die mir gehört? Wo ist er?
Verschwunden –, nun gut – er bleibe verschwunden!“

Er sank in seinen Stuhl zurück und verfiel in dumpfes Brüten. Tiefe
Stille herrschte in dem weiten Raum, nur von Zeit zu Zeit entrang
sich seiner Brust ein Seufzer, ohne daß er darum wußte. Und er
wußte nicht, daß die Zeit verging, daß das Dunkel des Abends sich
über die Stadt gelegt hatte – –

Und wenn Torm doch kam? Ja, verhindern konnte er es wohl, aber mit
diesem Wissen vor Isma treten – das konnte er nicht. Und sie sein
nennen um den Preis einer Schuld – das konnte er nicht, das war ja
unmöglich. Und wer weiß – er hatte Isma mehrere Tage nicht gesehen
–, wenn – wenn Torm schon gekommen wäre? Er fuhr in die Höhe, von
einem plötzlichen Schrecken aufgejagt. Wenn sie bei ihm wäre, und
ihm nichts davon gesagt hätte, wenn – –

Jetzt bemerkte er, daß es dunkel war. Ein Handgriff schaltete das
Licht ein. Dann stand er vor dem Telephon.

Wie es auch werden mochte, verbergen durfte er ihr nichts! Er
fragte an, ob Isma zu Hause wäre. Sie war da. Sie freue sich sehr,
ihn bald zu sehen.

Wenige Minuten später saß Ell in seinem Wagen. Er ließ so schnell
fahren, als es der Straßenverkehr ermöglichte, aber der Weg war
weit. Er sah es jetzt ein, er durfte ihr die Nachricht nicht
vorenthalten. Wenn Torm nicht zu ihr zurückkehrte, so mußte sie
trotzdem wissen, daß er hätte zurückkehren können.

Aber wie würde dies auf sie wirken? Nun sorgte er sich wieder um
die Freundin. Doch er hatte sich einmal angemeldet –. Er wollte sie
sprechen, er konnte ja vorsichtig sein, die neue Hoffnung für sie
nur andeuten – –

Der Wagen hielt, diesmal direkt vor der Tür. Ell eilte die Treppen
hinauf. Die Wirtin öffnete. Ell wollte mit flüchtigem Gruß an ihr
vorüber.

„Der Herr Kultor werden verzeihen“, sagte sie verlegen. „Frau Torm
sind nicht zu Hause.“

Ell prallte zurück. „Nicht zu Hause? Aber ich habe ja vor einer
halben Stunde mich angemeldet.“

„Ja, Frau Torm haben es mir auch gesagt, wir waren gerade bei
Tisch, aber dann – dann –“

„Was war dann?“ fragte Ell ungeduldig und hart.

„Der Herr Kultor mögen verzeihen, ich weiß es ja nicht –. Es kamen
die beiden Damen wieder –“

„Welche Damen?“

„Nun die Damen vom Mars, die gestern schon hier waren und die
Partie gemacht haben mit Frau Torm.“

„Welche Damen? Welche Partie? Sagen Sie alles!“

„Um Gottes willen, ich weiß ja nichts weiter, sie waren drin im
Zimmer, nur kurze Zeit, und auf einmal kam Frau Torm herausgestürzt
im Mantel und Hut, ganz eilig, und rief nur ›Ich muß fort‹, und die
beiden Damen gingen mit ihr, ich weiß ja nicht wohin. Und ich
wollte noch fragen, was ich denn nun sagen sollte, wenn der Herr
Kultor kämen, aber weil die beiden fremden Damen vom Mars dabei
waren, getraute ich mich nicht. Und ich bin noch die Treppe
hinuntergelaufen und habe gesehen, es stand ein Wagen vor der Tür,
in dem fuhren sie alle drei fort –“

„Wie lange ist das her?“

„Noch keine zehn Minuten.“

„Führen Sie mich ins Zimmer, ich werde warten.“

„Ach, entschuldigen Sie nur, Herr Kultor, das habe ich noch ganz
vergessen, Frau Torm hat mir zugerufen, sie käme die Nacht nicht
zurück.“

„So will ich doch nachsehen, ob sie nicht eine Nachricht für mich
hinterlassen hat.“

Ell sah sich vergeblich im Zimmer um. Kein Zeichen für ihn. Isma
hatte offenbar ganz vergessen, daß sie ihn erwartete. Er ging.

Er wußte nicht, was er denken sollte. Mit Torm mußte dieser
plötzliche Aufbruch zusammenhängen, das war das einzige, was er
sich sagte, aber weiter kam er nicht. Und so konnte sie ihn
verlassen, ohne auch nur mit einem Wort seiner zu gedenken? Und die
Damen vom Nu?

Völlig niedergeschlagen kam er zu Hause an. Neue Depeschen waren
eingetroffen. Er las sie durch – von Isma war nichts darunter. Er
fühlte sich nicht imstande, zu arbeiten.

Ein Gedanke drängte sich ihm immer wieder vor: Verloren! Verloren!
Das hatte er um sie verdient?

Es war zehn Uhr geworden. Da klang es noch einmal am
Depeschentisch. Die tiefe Glocke. Das war etwas Besonderes, eine
Lichtdepesche vom Mars.

Er öffnete das Stenogramm und sah nach der Unterschrift: „Für den
Zentralrat, der Präsident der Marsstaaten.“

„Ich habe die Ehre, Ihnen mitzuteilen, daß der Zentralrat Ihnen
seine ernste Mißbilligung aussprechen muß über die Nachsicht, mit
welcher im deutschen Sprachgebiet die Übergriffe der Menschen gegen
unsre Beamten behandelt werden. Der Zentralrat erwartet von Ihnen
sofortige entschiedene Maßregeln, wodurch den Menschen begreiflich
gemacht wird, daß sie der Herrschaft der Nume sich unter allen
Umständen ohne Widersetzlichkeit zu beugen haben. Zugleich mögen
Sie Vorbereitungen treffen, daß die nach dem nächstens zu
veröffentlichenden Gesetz auf das deutsche Sprachgebiet fallende
Kontribution von einer Milliarde Mark rechtzeitig erhoben werden
kann.“

Ell schleuderte das Blatt auf den Tisch.

„Das bedeutet den Sieg der Antibaten!“ rief er aus.

\section{58 - Lösung}

Zu derselben Zeit, als Ell in seinem Wagen nicht schnell genug
durch die Straßen Berlins jagen konnte, saß Torm an einem der
großen Tische des Bibliothekzimmers in der Friedauer Sternwarte. Er
beugte sich über seine Arbeit. Wohl zuckte es häufig in seiner
Hand, die Blätter mit den langen Zahlenreihen zurückzuschieben,
aber er bezwang sich; denn er wußte, daß ihn dann die bohrenden
Gedanken nur noch heftiger quälten.

Durfte er noch länger hier zögern? Und was sollte er tun? Grunthe
hatte sich an den Protektor Ill selbst gewandt, um zu erfahren,
welche Motive den neuen Nachforschungen nach Torm zugrunde lägen.
Aber die Antwort war noch nicht eingetroffen. Wie die Zeitungen
meldeten, hatte sich der Protektor, vom Zentralrat berufen, zu
einer wichtigen Konferenz nach dem Mars begeben. Ehe er
zurückkehrte, konnten, trotz der gegenwärtigen günstigen Stellung
der Planeten und der neuerdings erzielten kolossalen
Geschwindigkeit der Raumschiffe doch noch gegen zwei Wochen
vergehen. So lange noch hier auszuhalten, erschien Torm manchmal
als eine Unmöglichkeit. Und was dann, wenn die Antwort ungünstig
ausfiel?

Alle seine Willenskraft bot er auf, um die Sehnsucht nach Isma
zurückzudrängen. Und doch grübelte er immer wieder, ob es nicht
richtiger sei, ihr selbst die Entscheidung zu überlassen, sich zu
ihm zu bekennen oder nicht. Doch nein, das hieße, sie zu einem
verhängnisvollen Entschluß treiben. Aber er, er selbst, sollte er
nicht für sich auf die Entscheidung seines Schicksals dringen,
indem er Ell benachrichtigte? Er fand die Antwort nicht und
versenkte sich aufs neue in seine Rechnungen.

Da klang plötzlich durch die Stille des Raums aus dem Nebenzimmer,
in welchem Grunthe arbeitete – die Tür war nur angelehnt –, eine
helle Stimme, die Torm emporfahren machte.

„Grüß Gott, Grunthe!“ erscholl es.

„Saltner!“ hörte er Grunthe freudig überrascht rufen.

„Ja, ich bin’s. Und ich will Sie nur ins Schiff holen, hier getraue
ich mich nicht herein. Aber eins, sagen Sie gleich – ist Torm hier?
Na, machen’s keine Sperenzen, ich weiß, daß er bei Ihnen logiert.
Wo ist er?“

„Er arbeitet in der Bibliothek.“

„Dann heraus mit ihm, rufen Sie ihn. Frau Isma ist hier. Wir haben
sie mitgebracht.“

Da flog die Tür auf. Torm stand im Zimmer.

„Wo?“ fragte er bloß. Aber er wartete keine Antwort ab. Es konnte
ja nicht anders sein – sie war im Schiff, und das Schiff lag
natürlich im Garten. Mit einem Satz war er an der Tür der Veranda
und riß sie auf.

Hier lehnte Isma am Geländer der Treppe. Pochenden Herzens wartete
sie auf den Erfolg von Saltners Botschaft.

Einen Moment blieb Torm stehen, als er sie erkannte, nur einen
Moment. Dann hielt er sie in den Armen. Wie lange, sie wußten es
nicht.

„Komm herein!“ sagte er endlich. Noch vermochte er nichts anderes
zu sprechen. Er trug sie fast in das Zimmer. Es war leer. Grunthe
und Saltner hatten es durch eine andere Tür verlassen.

Sie hielten sich an den Händen und blickten sich an. Isma zitterte.
Die Tränen drängten sich in ihre Augen. Das war er! Der von ihr
geschieden war in der blühendsten Kraft des Mannes, hoffnungsfroh
und siegesgewiß – das Haar war ergraut, tiefe Falten hatten
Anstrengung und Sorge in seine Stirn gegraben –, sie hätte Mühe
gehabt, ihn wiederzuerkennen – aber die blauen Augen strahlten ihr
in der alten Innigkeit entgegen.

Sie schluchzte. „Ich habe dich wieder!“

Wieder warf sie ihre Arme um seinen Hals, er aber löste sich sanft
und sah sie nun an mit einem ernsten Blick voll Kummer und Liebe.

„Isma“, sagte er langsam, „du weißt nicht, wen du umarmst.“

„Ich weiß es, Hugo, ich weiß es! Die Freunde, die treuen, die mich
hierherbrachten, haben es mir gesagt. Ich weiß, warum du
fernbliebst, warum du nicht zu mir eiltest. Es war nicht recht,
doch ich versteh’ es – ich aber gehöre zu dir, drum bin ich hier
–“

„Über mir schwebt das Gericht und die Not, die Schande, die den
Frevler am Gesetz trifft. Du weißt nicht alles –. Ich brach das
Vertrauen der Nume am Pol, ich nahm von ihrem Gut, ich floh mit
Gewalttat und stieß den Wächter hinab ins Schiff Ich bin ein
Geächteter, solange die Nume herrschen. An dich aber hab ich kein
Recht, du stehst im Schutze des Nu, du bist frei. Warum kommst du,
mich in die furchtbare Qual zu stürzen, wieder von dir fliehen zu
müssen, nachdem ich dich gesehen – oh, es ist furchtbar!“

„Nein, nein“, rief sie, aufs neue sich an ihn schmiegend. „Ich
lasse dich nicht von mir, jetzt nicht wieder, und es ist nicht
furchtbar. Was du auch getan, du tatest es, um zu mir zu kommen,
nun trag ich mit dir, was geschehen soll. Aber du brauchst nichts
zu fürchten. Unsere Freunde führen uns, wohin der Arm der Nume
nicht reicht.“

Er schüttelte den Kopf. „Das geht nicht“, sagte er finster. „Ich
nehme keine Gnade an von denen, die ich als Feinde der Menschheit
betrachte, von den Vernichtern meines Glücks – das geht nicht!“

„Oh, wie kannst du so sprechen! Saltner ist in derselben Lage, er
hat nicht gezögert, Las Hilfe anzunehmen, er hat sie zur Frau
genommen nach den Gesetzen des Nu –“

„Dann kann er es tun, weil er sie liebt. Ich aber hasse diese Nume.
Und wir beide sind geschieden nach dem Gesetz des Nu –“

„Geschieden, wir? Wer hat das bestimmt? Dieses Gesetz ist nichts
ohne unsern Willen. Es schützt unsern Willen gegen fremden
Eingriff, aber gegen unsern Willen kann es weder fesseln noch
scheiden. Und ich habe niemals und werde niemals – o Hugo, wie
kannst du glauben, ich würde dich verlassen, ich, die ich selbst
die Schuld trage unsrer Trennung – hier stand ich, an dieser
Stelle, da beschwor ich Ell, mich mitzunehmen nach dem Nordpol,
denn binnen Tagesfrist gedacht ich dich zu finden, und es wurden
zwei Jahre – nicht durch meine Schuld –“

„Erinnere mich nicht an ihn“, unterbrach er sie hart. „Diese zwei
Jahre – oh! Als ich zurückkam und umkehrte vor deiner Tür, da trat
er heraus –“

„Hugo“, sagte sie flehend, „das Leid hat dich verbittert, sonst
würdest du so nicht reden. Ja, er ist mein Freund, der treueste,
beste, das weißt du, und das wird er uns immer beweisen. Eben
sagtest du, ich sei frei, wo aber findest du mich? In den
Prunkzimmern des Kultorpalais oder hier im Asyl des Geächteten, der
mich nicht will?“

Er blickte sie lange an, dann zog er sie an sich.

„Verzeih mir“, sagte er, „es ist wahr, ich habe dich ja hier, du
geliebte Frau. Was kümmert uns der Menschen Rede? Ich habe
gelitten, und das Elend war über mir. Aber die Philister sollen
nicht über uns sein. Wie wollen wir den Numen trotzen, wenn wir
nicht uns selbst die Freiheit im Gefühl zu wahren wissen? Mir aber
zerreißt es das Herz, daß ich dich nicht halten kann mit offnem
Trotz, weil ich selbst keine Stätte mehr habe, so weit die Planeten
kreisen. Denn eins will ich bewahren, den Stolz, und Rettung will
ich nicht durch ihre Gnade!“

Isma beugte sich zurück und sah ihm groß in die Augen.

„Wenn nicht durch ihre Gnade“, sagte sie langsam, „dann gibt es nur
eins: durch die Wahrheit!“

Seine Augen erweiterten sich, als er erwiderte: „Wenn ich dich
recht verstehe –“

„Vertraue dich Ell an. Sage ihm alles und höre, was er für richtig
hält. Und wenn es nötig ist, stelle dich ihrem Gericht. Ich aber
werde bei dir sein.“

Er zögerte. „Das heißt, ich gebe mich in seine Hand.“

„Er ist edel und groß.“

Torm runzelte die Stirn. Er dachte lange nach. Endlich sagte er:
„Ich sehe keinen andern Ausweg. Und nun du zu mir kamst, darf ich
nicht länger zögern, mein Schicksal zu entscheiden. Ich werde
gehen.“

Sie fiel ihm um den Hals. „Geh“, rief sie, „gehen wir, und
sogleich!“

„Jetzt? Auf der Stelle? Wie meinst du das? Es ist Abend – und ich,
in meiner Überraschung, ich habe noch nicht einmal gefragt, wie
kamst du her?“

„Komm mit zu La, und du wirst alles begreifen.“

Er schloß sie noch einmal in seine Arme. Dann gingen sie Hand in
Hand durch das Zimmer nach der Veranda, in den Garten.

Sie standen vor dem Luftschiff.

„Verzeih mir“, sagte Torm zu Isma, „aber jetzt in die Gesellschaft
der andern zu gehen, sie zu begrüßen, zu reden – es ist mir
unmöglich – und es ist doch schon zu spät, um Ell noch zu sprechen,
selbst wenn La uns wirklich so schnell und noch jetzt –“

„Ich werde La rufen.“

Die Beratung mit La dauerte nicht lange.

„Sie, Torm“, sagte sie, „wird Ell jederzeit empfangen, und Sie
haben nicht eher Ruhe, bis die Entscheidung gefallen. Für uns aber
ist es erwünscht, noch heute nacht alles abzuwickeln, denn der
Boden Europas brennt uns unter den Füßen, und wenn die Sonne
aufgeht, möchte ich hoch über den Wolken sein. In einer halben
Stunde können Sie in Ells Zimmer stehen.“

„Ihr Interesse entscheidet“, sagte Torm. „Meinetwegen dürfen Sie
sich nicht aufhalten. Ich bin bereit.“

La führte Torm und Isma ins Schiff. Sie sahen noch, wie La mit
Grunthe sprach, der das Schiff verließ. Dann blieben sie allein im
kleinen Salon. Was hatten sie nicht alles sich mitzuteilen! Sie
glaubten eben erst begonnen zu haben, als La eintrat und sagte:

„Wir sind auf dem Vorbau des Kultorpalais, auf dem Anlegeplatz für
die Luftschiffe, steigen Sie schnell aus und lassen Sie sich
melden. Da Sie an dieser nur für Nume zugänglichen Tür Einlaß
verlangen, wird man keine Schwierigkeiten machen. Unser Schiff
finden Sie am Akazienplatz, wohin Sie eine der vor dem Palais
haltenden Droschken in wenigen Minuten bringt. Und nun viel Glück
auf den Weg!“

Isma umarmte ihn schweigend, dann stieg Torm die Schiffstreppe
hinab. Von den Türmen der Stadt schlug die elfte Stunde, als der
diensttuende Bed Torm nach seinem Begehr fragte. Ein Besuch um
diese Zeit mußte wohl etwas sehr Wichtiges sein, darum zögerte er
nicht anzufragen, ob der Kultor noch empfange. Er arbeitete noch.

Ell erbleichte, als er die Karte las.

„In mein Privatzimmer“, sagte er.

Die beiden Freunde standen einander gegenüber. Beide fühlten sich
nicht frei. Beide hatten gegen die Macht eines Verhängnisses
gekämpft, das stärker war als sie, dem sie sich nun ergeben mußten.
Auch in Ells Zügen hatten Überarbeitung und Sorgen ihre Spuren
zurückgelassen. Es war nur ein Moment, daß ihre Blicke aufeinander
ruhten. Und jeder sah im andern ein stilles Leid, das an ihm
zehrte, und die Erinnerung stieg auf an die Jahre treuer,
gemeinsamer Freundesarbeit und kühner Hoffnung, und die Rührung des
Wiedersehens umschleierte ihre düsteren Blicke mit milder Freude.
Sie eilten aufeinander zu, und ihre Hände lagen ineinander.

„Sie werden vor allen Dingen wissen wollen, wo ich war“, begann
Torm endlich „ich aber komme, um von Ihnen zu hören – Sie empfangen
mich als Freund, wie aber empfängt mich der Kultor – was habe ich
zu erwarten?“

„Ich verstehe Sie nur halb“, erwiderte Ell betroffen. „Was
veranlaßt Sie zu der Frage? Sprechen Sie offen –. Kommen Sie aus
Tibet über Kalkutta?“

Torm zuckte zusammen. „Ach, Sie wissen? Doch nun hören Sie erst
alles.“

Er berichtete kurz über seine Flucht vom Pol und aus dem Luftschiff
und die Ereignisse, die sich dabei zutrugen. Er verheimlichte
nichts. Er erzählte, was ihn veranlaßt hatte, weder seine Frau noch
Ell aufzusuchen, sondern sich in Friedau verborgen zu halten, wo Se
ihn erkannt habe; daß ihn Isma infolgedessen aufgesucht hätte und
er jetzt hier sei, um den Rat Ells zu vernehmen und die Folgen
seiner Handlungen zu tragen.

Ell hörte schweigend zu, den Kopf sinnend auf die Hand gestützt. Er
unterbrach ihn mit keinem Wort, keine Miene verriet, was in ihm
vorging.

Das hatte er nicht gewußt. Die Tat gegen den Wächter des Schiffes
war verderblich für Torm. Ell, als oberster Beamter der Nume
hierselbst, mußte sie verfolgen. Der eben erhaltene Erlaß hatte ihm
seine Pflicht eingeschärft. Wenn er dieser Pflicht folgte, wenn er
die Mahnung des Zentralrats annahm, so war Torm verloren. Torms
Schicksal war in seine Hand gelegt. Ein Druck auf diese Klingel,
und er kehrte nicht mehr aus diesem Zimmer zurück, nicht mehr zu
Isma – –. Und dann? Isma war frei. Aber wo war sie? Ohne ein Wort
des Abschieds hatte sie ihn verlassen und war zu ihrem Mann geeilt.
Ein tiefer, bitterer Schmerz gekränkter Liebe durchzuckte ihn.
Durch Jahre hatte sie ihn in hoffnungsfroher Freundschaft gehalten,
bis die Erwartung des nahen Glücks ihn ganz eingenommen, und jetzt
– nun war er ihr nichts mehr. Das war Isma! Ja, er konnte sich
rächen. Er konnte auch – –. Und durfte er denn schweigen? Durfte er
Torm, nun er um sein Verbrechen wußte, unbehelligt ziehen lassen?
Ihn der Gattin zurückgeben und sie in ihrem Glück schützen? Und wie
dann den Gedanken an sie ertragen?

Torm hatte längst geendet. Ell saß noch immer, den Kopf in die Hand
gestützt, die seine Augen beschattete, ohne zu sprechen. Torm
wartete geduldig, obwohl sein Herz pochte. Denn jetzt mußte sich
alles entscheiden.

Endlich richtete sich Ill auf und blickte Torm an. Er begann ruhig,
fast gleichgültig:

„Ihr Prozeß am Pol und was damit zusammenhängt, die Entwendung des
Sauerstoffs – wovon übrigens nichts bekannt geworden ist –, die
unerlaubte Benutzung des Luftschiffs zur Flucht – darüber können
Sie beruhigt sein. Ich sehe dies als eine zusammenhängende einzige
Handlung an, die unter die Friedensamnestie fällt. Sie können
deswegen nicht verfolgt werden. Ich nehme es auf mich, diese Akten
kassieren zu lassen. Aber das andere! Das ist traurig, das ist
schwer! Wenn es zur Anzeige kommt, sind Sie verloren.“

Torm sprang auf.

„Sie wissen es, so bin ich verloren.“

Auch Ell erhob sich. Er schritt durch das Zimmer auf und nieder,
noch immer mit sich kämpfend. Dann blieb er wieder vor Torm
stehen.

„Wenn es zur Anzeige kommt, sage ich, und wenn Sie bei Ihrem
Geständnis stehen bleiben.“

„Wie kann ich anders.“

„Denn es ist nichts davon bekannt geworden. Es ist etwas geschehen,
was Sie nicht wissen. Das Schiff mit der gesamten Besatzung ist auf
der Rückkehr bei Podgoritza durch die Albaner vernichtet worden,
ehe irgendeine Nachricht von ihm zu uns gelangt ist. Niemand wurde
gerettet, alle Papiere und Aufzeichnungen sind verbrannt oder
verschwunden. Niemand kann beweisen, was Sie getan haben, außer
Ihnen – und mir!“

„O ich Tor!“ murmelte Torm; bleich und finster blickte er auf Ell.

„Wollen Sie widerrufen, was Sie mir gesagt haben? Es war vielleicht
nur eine poetische Ausschmückung ihres Abenteuers? Sie haben den
Wächter nur leicht beiseite gedrängt?“

„Ich schlug ihn vor die Stirn, ich hörte ihn mit einem Aufschrei
dumpf auf die Kante der Treppe schlagen. Hätte ich gewußt, was ich
jetzt weiß, ich hätte vielleicht geschwiegen. Lügen werde ich
nicht. Und doch – komme, was da kommen will, es ist besser so.
Gewißheit konnte ich nicht anders erlangen, als daß ich mit Ihnen
sprach. Gewißheit mußte ich erlangen, und die Wahrheit mußte ich
sagen, wenn ich überhaupt sprach. Und Sie müssen meine Bestrafung
einleiten.“

„Ich muß es, wenn –“, er unterbrach sich und ging wieder auf und
ab. Dann trat er an das Fenster. Torm hörte ihn leise stöhnen. Nun
wandte er sich um. Er schritt auf Torm zu. Er sah verändert aus.
Aus dem geisterhaft bleichen Gesicht leuchteten seine großen Augen
wie von einem überirdischen Feuer. Vor Torm blieb er stehen und
faßte seine Hände.

„Gehen Sie“, sagte er mit Bestimmtheit. „Gehen Sie, mein Freund,
ich werde die Anzeige nicht erstatten. Was Sie gesprochen haben,
der Kultor hat es nicht gehört – verstehen Sie –“

Torm schüttelte den Kopf.

„Sie werden es verstehen, binnen einer Stunde. Wohin gehen Sie?
Nach Friedau? Sie haben nichts mehr zu befürchten. Gehen Sie –
geben Sie sich zu erkennen – und seien Sie glücklich – gehen Sie
–“

Er drängte Torm zur Tür. Ein Diener nahm ihn in Empfang und zeigte
ihm den Weg durch die Gemächer und über die Treppen.

Sobald Ell allein war, sank er wie gebrochen auf einen Sessel. Er
schloß die Augen und preßte die Hände vor die Stirn. Nur wenige
Minuten. Dann stand er auf. Er wußte, was er wollte.

Mit fester Hand setzte er zwei Depeschen auf. Die eine war in
martischer Kurzschrift, sie war an den Protektor der Erde gerichtet
und trug den Zusatz: Als Lichtdepesche auf den Nu nachzusenden. Die
andre ging an Grunthe: Sofort zu bestellen.

\tb{}
„Besorgen Sie dies eilends“, sagte er zu dem Diener. „Und nun
wünsche ich nicht mehr gestört zu sein.“

Torm fand vor der Tür des Palais bereits einen Wagen halten, und
als er herantrat, winkte ihm Isma entgegen. Sie hatte keine Ruhe im
Schiff gefunden und wollte ihn hier erwarten. Angstvoll blickte sie
ihm entgegen.

„Alles gut!“ rief er und sprang in den Wagen, der sogleich sich in
Bewegung setzte.

„Ich bin frei, wir sind sicher! Nun habe ich dich erst wieder!“

„Gott sei bedankt“, flüsterte Isma, an seine Schulter gelehnt. „Und
was sagte Ell?“

„Gehen Sie nach Friedau, seien Sie glücklich!“

„Sonst nichts?“

„Nichts.“

Nach ihr hatte er nicht gefragt, für sie hatte er keinen Gruß,
keinen Glückwunsch, ihr Name war nicht über seine Lippen gekommen.
So klang es schmerzlich durch ihre Seele, während Torm, immer
lebhafter werdend, seine Unterredung mit Ell berichtete.

Am Akazienplatz verließen sie den Wagen. Alsbald senkte sich das
Luftschiff auf den menschenleeren Platz und nahm sie auf.

Gegen ein Uhr nachts ließ sich das Luftschiff wieder auf seinen
Ankerplatz im Garten der Sternwarte von Friedau nieder.

Grunthe hatte die Rückkehr erwartet. Saltner holte ihn herbei.

„Es ist zwar schon spät, aber das hilft heute nichts, und aus den
Beobachtungen wird auch nichts. Eine Stunde müssen Sie uns noch
schenken. Ich feiere nämlich meine Hochzeit, schauen’s, da müssen
Sie schon noch einmal lustig sein. Ich habe die ganze Expedition
eingeladen.“

Als er in den Salon des Schiffes trat, fand er eine Tafel für sechs
Personen nach Menschensitte gedeckt.

„Wir sind eigentlich zwei Brautpaare“, sagte Saltner zu Grunthe.
„Von Ihnen verlangen wir nicht, daß Sie das dritte abgeben, aber
eine Dame haben wir doch für Sie. Meine Mutter schläft freilich,
aber hier – die Se kennen’s ja, wir haben uns wieder versöhnt.“

„Ausnahmsweise“, sagte Se lachend, „werde ich mich heute
herablassen, mit euch fünf Menschen an einem Tisch zu essen, aber
nur zu Ehren der drei Entdecker des Nordpols.“

Unter lebhaftem Gespräch hatte man an der Tafel Platz genommen.
Torm wandte sich zu Se und sagte, sein Glas erhebend: „Die
Vertreterin der Nume gestatte mir, nach unsrer Sitte ihr zu danken.
Denn ihrem Scharfblick verdanke ich das Glück dieser Stunde.“

„Ich danke Ihnen“, erwiderte Se, „und ich freue mich, daß Sie nun
dem Bild wieder ähnlich sehen, nach welchem ich Sie erkannte.“

„Und jetzt“, rief Saltner, die Gläser neu füllend, „wie damals, als
wir zuerst den Pol erblickten, bring ich wieder ein Hoch aus auf
unsre gnädige Kommandantin, auf Frau Isma Torm, und diesmal stößt
sie selbst mit an, und das ist das beste. Und nun, Grunthe, können
Sie wieder sagen: Es lebe die Menschheit!“

Grunthe erhob sich steif. Sein Unterarm streckte sich im rechten
Winkel von seinem Körper aus, und seine möglichst wenig gebogenen
Finger balancierten das Weinglas wie ein Lot, mit dem er eine
Messung ausführen wollte.

„Es lebe die Menschheit“, sagte er, „so sprach ich einst. Ich sage
es jetzt deutlicher: Es lebe die Freiheit! Denn ohne diese ist sie
des Lebens nicht wert. Wenn die Freiheit lebt, so ist es auch kein
Widerspruch, wenn ich mich dessen freue, was meine verehrten
Freunde von der Polexpedition für ihre Freiheit halten, die
Vereinigung mit einem Vernunftwesen, das kein Mann ist. Um aber den
abstrakten Begriff der Freiheit durch eine konkrete Persönlichkeit
unsrer symbolischen Handlung zugänglich zu machen, sage ich, sie
lebe, die uns die Freiheit gebracht hat. Wie sie herabstieg von dem
Sitz der Nume und den Wandel seliger Götter tauschte mit dem
schwanken Geschick der Menschen, nur weil sie erkannte, daß es
keine höhere Würde gibt als die Treue gegen uns selbst, so zeigte
sie uns, wie die Menschheit sich erheben kann über ihr Geschick,
wenn sie nur sich selbst getreu ist. Denn es gibt nur eine Würde,
die Numen und Menschen gemeinsam ist, wie der Sternenhimmel über
uns, das ist die Kraft, nachzuleben dem Gesetz der Freiheit in uns.
Sie tat es, und so brachte sie die Freiheit diesen meinen Freunden,
und allen ein Vorbild, wie Nume und Menschen gleich sein können.
Darauf gründet sich unsre Hoffnung der Versöhnung, der wir
entgegenstreben. Ihr aber, die in so hohem Sinn uns genaht und die
Freunde der Not entrissen, ihr gelte unser Glückwunsch und Hoch.
Und so sage ich nun: Es lebe La!“

Er blieb stehen, wie in Nachsinnen verloren, sein Glas starr vor
sich hinhaltend, an das die andern mit Herzlichkeit anstießen.

Saltner küßte La und flüsterte: „Du kannst dir aber etwas
einbilden, das ist das erste Mal, daß er eine Frau leben läßt!“

„Und das letzte Mal“, murmelte Grunthe, sich niedersetzend.

Saltner aber sprang auf und trat zu Grunthe und umarmte ihn, ehe er
es verhindern konnte.

Grunthe wand sich verlegen. „Ich glaube“, sagte er, „ich meine ja
eigentlich diese La, in der wir sitzen, das schöne Luftschiff –“

„Oh, oh!“ rief Se, „das hilft Ihnen nichts mehr, Sie haben von
›Persönlichkeit‹ gesprochen – jetzt können Sie nichts mehr
zurücknehmen.“

„Nein, ich will es ja auch nicht“, sagte er ernsthaft.

Da öffnete sich die Tür. Der Schiffer trat ein.

„Eine Depesche für Herrn Dr. Grunthe ist eben gebracht worden“,
sagte er.

Grunthe stand auf und trat beiseite. Er las.

Dann kehrte er zum Tisch zurück. Er sah sehr ernst aus.

„Es ist etwas Wichtiges geschehen“, sagte er auf die fragenden
Blicke der andern. „Ell hat sein Amt niedergelegt.“

„Wie? Was? Lesen Sie!“

Er reichte Saltner das Blatt. Dieser las:

„Ich benachrichtige Sie hierdurch, daß ich soeben bei Ill um
Enthebung vom Kultoramt und um meine Entlassung aus dem Dienst der
Marsstaaten eingekommen bin. Unter den obwaltenden Umständen ist
mir die Fortführung unmöglich. Ich bitte Sie, meinen Besitz in
Friedau als den Ihrigen zu betrachten. Ich selbst gehe nach dem
Mars, um gegen die Antibaten zu wirken. Sie werden bald von mir
hören. Glück dem Menschenbund! Saltner und Torm meinen Gruß. Ihr
Ell.“

Isma wußte nicht, wohin sie blicken sollte. Sie fühlte, wie Blässe
und Röte auf ihrem Antlitz wechselten. In der allgemeinen Erregung
achtete man nicht auf sie.

„Also darum“, sagte Torm, „darum sagte er, in einer Stunde werde
ich verstehen, warum der Kultor meinen Bericht nicht gehört habe –
–. Lassen Sie uns des edlen Freundes gedenken!“

„Auf Ell!“ sagte Saltner. „Aber Sie müssen mir noch erklären –“

„Es muß ein politisches Ereignis eingetreten sein – vielleicht ist
der Antrag über die Steuern angenommen“, bemerkte Torm. „Also auf
Ell!“

Sie erhoben die Gläser. Ismas Hand zitterte. Als sie anstieß,
entglitt das Glas ihren Fingern und zerbrach.

La allein hatte gesehen, was in Isma vorging. Kaum klangen die
Scherben auf dem Tisch, als sie auch ihr Glas fallen ließ und mit
einem leichten Stoß Saltner das seine aus der Hand schlug.

„Das ist recht!“ rief sie.

„Fort alle mit den Gläsern und Flaschen! Auch der Nu will sein
Recht haben. Ehe wir Abschied nehmen, meine lieben Freunde, noch
einen Zug vom Nektar des Nu aus den Kellern der La!

Und dann hinauf in den Äther!“

\section{59 - Die Befreiung der Erde}

Zum zweiten Mal war es Herbst geworden, seit La und Saltner die
Freunde in Friedau verlassen hatten, um zunächst außerhalb des
Machtbereichs der Martier die Entwicklung der Dinge abzuwarten. Das
ganze Gebiet der Vereinigten Staaten von Nordamerika stand ihnen
zur Verfügung. Ihr Haus und ihr Glück führten sie mit sich. Ob in
den blühenden Gärten des ewigen Frühlings an den Buchten der
kalifornischen Küste, ob auf den Schneegipfeln der Sierra Nevada
oder unter den Wundern des Yellowstone-Parks, für La und Saltner
galt das gleich, das glänzende Luftschiff war ihre Heimat; ob es in
den Lüften schwebte oder unter Palmen ruhte, treu barg es die Wonne
der Vereinten und machte sie unabhängig von der Welt.

Nur über dieses Freigebiet hinaus durften sie sich nicht wagen. La
mußte sich den Wunsch versagen, die Ihrigen auf dem Mars oder auch
nur ihren Vater am Pol der Erde zu besuchen, und konnte ihn selbst
bloß zu kurzem Besuch einigemal bei sich sehen. Se war alsbald nach
dem Mars zurückgekehrt. Palaoro, der sich zum geschickten
Luftschiffer ausgebildet hatte, war bei Saltner zurückgeblieben.
Auch die beiden Martier, die in Las Diensten standen, blieben ihr
treu, selbst als sich das Verhältnis von Martiern und Menschen
schärfer zuspitzte.

Die Partei der Antibaten auf dem Mars hatte immer deutlicher ihre
Ziele enthüllt. Den Menschen sollte die Würde der freien
Selbstbestimmung abgesprochen, die Menschheit in eine Art
Knechtschaft zum Dienst der Nume gestellt werden. Die Erde wollte
man lediglich als ein Objekt der wirtschaftlichen Ausnutzung
zugunsten des Mars behandeln und das Kulturinteresse der
Menschheit nur insofern berücksichtigen, als es zum Mittel für die
größere Leistungsfähigkeit dieser tributären Geschöpfe dienen
konnte. Und diese Auffassung des Verhältnisses zur Erde war jetzt
auf dem Mars zum Sieg gelangt. Sowohl im Parlament als im
Zentralrat besaßen die Antibaten die Majorität. Die Abdankung von
Ell, die unglücklicherweise kurz vor die Neuwahlen fiel, durch
welche alle Marsjahre ein Drittel der Volksvertreter neu ernannt
wurde, hatte zum Sieg der Antibaten bedeutsam mitgewirkt. Sie war
als der sichtbare Beweis aufgefaßt worden, daß die bisherige
Methode in der Regierung der Menschen nicht die richtige sei. Man
verlangte ein rücksichtsloses Verfahren, höhere Revenuen, baldige
Unterwerfung Rußlands und der Vereinigten Staaten. Mit dem Sieg der
Antibatenpartei begann diese neue Politik. Maßregel folgte auf
Maßregel, um die Erde dem Dienst des Mars zu unterwerfen.

Oß und einige andere höhere Beamte der Martier auf der Erde waren
allerdings aus ihren Stellungen abberufen worden, da sie zweifellos
von jener nervösen Störung befallen waren, die man vulgär als
Erdkoller bezeichnete. Aber ihre Ersatzmänner verfolgten die
Politik der Unterdrückung nur mit größerer Klugheit. An Ells Stelle
war der Martier Lei gekommen, ein ausgesprochener Antibat, ein sehr
energischer Mann, der selbst vor gewalttätigen Eingriffen nicht
zurückscheute. Ell war auf den Mars gegangen und hatte dort mit
aller Kraft zugunsten der Erde zu wirken versucht, vorläufig ohne
erkennbaren Erfolg. Gleich ihm waren seine früheren Untergebenen in
das Privatleben zurückgetreten und agitierten nun auf dem Mars als
seine entschiedenen und gefährlichen Gegner.

Ells Oheim, der Protektor der Erde und Präsident des Polreichs,
Ill, kämpfte noch eine Zeitlang gegen die vom Zentralrat
gewünschten Maßregeln. Als man aber gegen seinen ausdrücklichen Rat
ihn beauftragte, die Vorbereitungen zu treffen, um im nächsten
Frühjahr die russische Regierung erforderlichenfalls mit Gewalt zu
veranlassen, auch in ihrem Gebiet die Einsetzung martischer
Residenten und Kultoren zuzulassen und einen jährlichen Tribut von
30 Milliarden Mark zu zahlen – um sie zu zwingen, die ausgedehnten
Steppen und Wüsten im Süden und Osten mit Strahlungsfeldern zu
bedecken –, da legte auch Ill mit schwerem Herzen sein Amt nieder.
Die Erde war nun der Gewalt einer den Menschen feindlich gesinnten
Partei ausgeliefert.

Rußland machte einen Versuch zum Widerstand. Aber der Geist, der
jetzt auf dem Mars herrschte, war weniger ›human‹ als in den ersten
Kriegen mit den europäischen Staaten. Die Martier scheuten sich
nicht, den Hafen von Kronstadt und das blühende Moskau ohne
Rücksicht auf Menschenleben zu zerstören. Der Zar gab nach, da er
sah, daß alles auf dem Spiel stand und seine Herrschaft zu
zerfallen drohte. Es gab keine Mittel für Rußland, der
vernichtenden Gewalt der Luftschiffe zu widerstehen. Der russische
Kaiser wurde Vasall der Marsstaaten. Das war im Sommer des dritten
Jahres nach der Entdeckung des Nordpols.

Schwer lag die Fremdherrschaft über Europa und den von ihm
abhängigen Ländern. Die Geldsummen, welche in Gestalt von Energie
nach dem Mars flossen, waren ungeheuer. Jedoch nicht diese
Leistungen waren es, die als drückend empfunden wurden. Zwar
erhoben die Staaten, um die auferlegten Tribute zu bezahlen,
Steuern in einer Höhe, die man vorher für unmöglich gehalten hätte.
Aber dies war nur die Form, in welcher ein Strom des Reichtums nach
dem Mars hin mündete, dessen schier unerschöpfliche Quelle in der
Sonne lag und nun zum erstenmal von den Menschen bemerkt und
benutzt wurde. Es fehlte nicht an Geld, vielmehr, der
Nationalreichtum stieg sichtlich, und zwar in allen Schichten der
Bevölkerung, die Lebenshaltung hob sich, und von wirtschaftlicher
Not war nirgends die Rede. Denn zahllose Arbeitskräfte fanden zur
Herstellung und Bearbeitung der Strahlungsfelder Beschäftigung, und
selbst die gefürchtete Entwertung von Grund und Boden trat nicht
ein. Mit dem Fortschritt in der Herstellung billiger chemischer
Nahrungsmittel fanden sich zugleich andere Methoden der
Bodenausnutzung. Der Verkehr blühte. Das Hauptzahlungsmittel
bestand in Anweisungen auf die Energie-Erträge der großen
Strahlungsfelder. Die aufgespeicherte Energie selbst kam nur zum
kleinen Teil in den Verkehr, die geladenen Metallpulvermassen, die
›Energieschwämme‹, wurden zum größten Teil direkt nach dem Mars
exportiert, die Scheine über diese Erträge aber wanderten von Hand
zu Hand und in die Regierungskassen als Steuern. Von hier wurden
sie als Tribut an die Marsstaaten verrechnet.

So hatten die Martier allerdings durch ihre Tributforderungen die
Menschen gezwungen, der neuen Quelle des Reichtums in der direkten
Sonnenstrahlung sich zuzuwenden und der Menschheit einen ungeahnten
wirtschaftlichen Fortschritt verliehen. Aber sie hatten dies nicht,
wie die Menschenfreunde auf dem Mars wollten, durch allmähliche
Erziehung zur Freiheit getan, sondern durch Zwang. Und dieser Zwang
war es, der die Menschen des äußeren Segens nicht froh werden ließ.
Es war Fremdherrschaft, die auf ihnen lag, und je leichter ihnen
der Gewinn des Unterhalts wurde, um so schwerer empfanden sie den
Verlust der Freiheit und Selbständigkeit. Und der gemeinsame Druck
ward wider Willen der Martier ein schnell wirkendes Mittel zur
Erziehung des Menschengeschlechts. Er weckte das Bewußtsein der
gemeinsamen Würde.

Je schwerer die Hand der Martier auf den Völkern ruhte, um so
rascher und mächtiger verbreitete sich der allgemeine Menschenbund.
Seine Prinzipien waren noch dieselben: Numenheit ohne Nume!
Erringung der Kulturvorteile, die der höhere Standpunkt der Martier
bieten konnte, um die Erde unabhängig von ihrer Herrschaft zu
machen – auf friedlichem Weg.

Aber was Ill und Ell als das eigene Ziel betrachtet hatten, darin
sahen die neuen Gewalthaber eine gefährliche Überhebung der
Menschen, die nur zu Unbotmäßigkeit führen würde. Und sie begingen
den großen Fehler, den Menschenbund zu verbieten.

Damit wurde aus dem Bund eine geheime Gesellschaft, die nur um so
fester zusammenhing. Er wurde ein wirklicher Bund der Menschen, der
aufklärend und verbrüdernd wirkte zwischen allen Nationen und
Stämmen, zwischen allen Gesellschaftsklassen und Bildungsstufen.
Ein jeder fühlte nun, daß er nicht bloß Franzose oder Deutscher,
Handarbeiter oder Künstler, Bauer oder Beamter sei, sondern daß er
dies nur sei, um ein Mensch zu sein, um eine Stelle auszufüllen in
der gemeinsamen Arbeit, das Gute auf dieser Erde zu verwirklichen.
Die Gegensätze milderten sich, das Verbindende trat hervor. In den
Staaten, in denen herrschende Klassen die hergebrachte Scheu vor
der Geltung des Volkswillens noch immer nicht überwunden hatten,
machte sich nun doch die Einsicht geltend, daß allein in der
Einigkeit des ganzen Volkes die Kraft zur Erhebung zu finden sei.
Längst erstrebte Forderungen einer volkstümlichen Politik wurden
von den Fürsten zugestanden. Man lernte, jeden eignen Vorteil dem
Wohl des Ganzen unterzuordnen. Und während ein ohnmächtiger Zorn
gegen den Mars in den Gemütern kochte, erhoben sich die Herzen in
der Hoffnung auf eine bessere Zukunft, und ein machtvoller, idealer
Zug erfüllte die Geister: Friede sei auf Erden, damit die Erde den
Menschen gehöre!

Der Menschenbund war der Träger dieser Ideen, aber man zweifelte
nun, sie auf friedlichem Weg durchführen zu können. Rettung, so
schien es, war nicht mehr zu hoffen vom guten Willen der Martier;
man mußte sie zu erobern suchen durch eine allgemeine gewaltsame
Erhebung gegen die Bedrücker – „zum letzten Mittel, wenn kein
andres mehr verfangen will, ist uns das Schwert gegeben“ –. Der
Menschenbund wurde eine stille Verschwörung zur Abschüttelung der
Fremdherrschaft. Aber wo war ein Schwert, das nicht vom
Luftmagneten emporgerissen, vom Nihilit nicht zerstört wurde, das
hinaufreichte zu den schwerelosen, pfeilgeschwinden,
verderbenbringenden Hütern der martischen Herrschaft?

Die herrschende Gärung konnte den Martiern nicht verborgen bleiben.
Die Partei der Menschenfreunde auf dem Mars machte sich die
Tatsache zunutze, daß die Unzufriedenheit auf der Erde nicht zu
leugnen war. Sie wies auf die Gefahren hin, die hieraus entstehen
mußten. Unermüdlich war Ell an der Arbeit, die Tendenzen der
Antibaten zu bekämpfen und die Nume mit dem Wesen, der
geschichtlichen Entwicklung und den Bestrebungen der Menschen
bekannter zu machen. Und seine Anhänger wuchsen an Zahl. Aber
gerade weil die Antibaten bemerkten, daß sie Gefahr liefen, an
Macht einzubüßen, wurden sie um so verblendeter in den Mitteln zu
ihrer Erhaltung. Aufs neue gewann die Absicht deutlichere Gestalt,
den Menschen durch ein Gesetz direkt das Recht der Freiheit als
sittliche Personen abzusprechen. Und gegen den Menschenbund wurde
ein System von Verfolgungen in Szene gesetzt. Die Erbitterung nahm
zu. Die Martier aber erkannten, daß die Fäden der Verschwörung nach
den Vereinigten Staaten hinwiesen. Der Sitz der Zentralleitung des
Bundes war nicht mehr in Europa, er befand sich in einem Land, das
ihrer Macht nicht unterworfen war.

Es kam zu einer stürmischen Sitzung im Parlament und im Zentralrat
des Mars, etwa ein Jahr nach der Unterwerfung Rußlands. Man
verlangte, daß nun auch die Vereinigten Staaten von Nordamerika
gezwungen werden sollten, sich der direkten Regierung durch die
Marsstaaten zu fügen. Eher werde man vor den Umtrieben des
Menschenbundes und der Widersetzlichkeit der Erde nicht sicher
sein. Und die Antibaten siegten wieder, obwohl mit geringer
Majorität. Den Vereinigten Staaten wurde die Forderung gestellt,
die Häupter des Menschenbundes, unter denen man Saltner als eines
der gefährlichsten bezeichnete, auszuliefern und Residenten und
Kultoren der Marsstaaten in die Hauptstädte der einzelnen Staaten
aufzunehmen.

Der Beschluß fand in einem großen Teil der Marsstaaten keineswegs
Billigung. Die Ansichten Ells, der bei einer Nachwahl in das
Parlament berufen worden war, wurden in weiten Kreisen geteilt. Man
sagte sich, daß ein etwaiger Widerstand der Vereinigten Staaten zur
Niederwerfung viel größere Mittel erfordern würde als die
Bezwingung des russischen Reiches. Denn hier war die Sache
entschieden, wenn der Zar sich beugte. In Amerika aber war
anzunehmen, daß, wenn auch die zentrale Regierungsgewalt aufgehoben
werde, jeder Staat für sich einen Widerstand leisten könne, der bei
der weiten Ausdehnung des Gebietes zu umfangreichster
Machtentfaltung und wahrscheinlich zu traurigen Verheerungen
zwingen würde. Aber der Beschluß war nun gefaßt und mußte
durchgeführt werden. Das Ultimatum wurde gestellt. Es enthielt die
Drohung, daß im Fall irgendeiner Feindseligkeit gegen die zur
Ausführung der verlangten Bestimmungen eintreffenden Luftschiffe
das Gesetz als sanktioniert zu gelten habe, wonach die gesamte
Bevölkerung der Erde des Rechts der freien Selbstbestimmung für
verlustig erklärt werde.

Die Vereinigten Staaten antworteten mit der Kriegserklärung.

Drei Tage darauf erfolgte von seiten der Marsstaaten die
Verkündigung des angedrohten Gesetzes: Die Bewohner der Erde
besitzen nicht das Recht freier Persönlichkeit.

Es war eine Zeit unbeschreiblicher Aufregung in allen zivilisierten
Staaten. Man empfand die Erklärung als eine Beschimpfung der
gesamten Menschheit. In Europa herrschte eine ohnmächtige Wut.
Jeder bangte davor, sich zu äußern oder zu widersprechen, weil er
den Schutz des Rechtes von sich genommen fühlte. Ein letzter Rest
der Hoffnung ruhte noch auf den Vereinigten Staaten. Aber die
Hoffnung war gering. Wie wollten sie der Macht der Martier
widerstehen? Und wirklich – die Überflutung der Staaten durch eine
Luftschifflotte von gegen dreihundert Schiffen ging vor sich, ohne
daß Widerstand versucht wurde. Die martischen Schiffe verteilten
sich auf die Hauptverkehrspunkte in dem ganzen ungeheuren Gebiet.
Eine merkwürdige Ruhe herrschte im Land, ein passiver Widerstand,
der unheimlich war. Die Kultoren und Residenten waren da, aber
außerhalb des Nihilitpanzers ihrer Schiffe wagten sie nichts zu
unternehmen. Die Martier stellten eine dreitägige Frist zur
Übergabe der Regierungsgewalt und drohten im Fall der Weigerung mit
Verwüstungen in großem Maßstab, vor allem auch mit Unterbrechung
des Verkehrs. Es schien keine Rettung. Mit Zittern und Bangen
verfolgte man auf der ganzen Erde die Vorgänge in Nord-Amerika. In
dumpfem Schmerz beugten sich die Gemüter. Sollte auch das letzte
Bollwerk der Freiheit auf der Erde vernichtet werden? Das war das
Ende der Menschenwürde!

Der gegenwärtige Protektor der Erde und Präsident des Polreichs,
Lei, war mit der Exekution gegen die Vereinigten Staaten beauftragt
worden. Sein Admiralsschiff lag auf dem Kapitol zu Washington. Am
11. Juli sollte die zur Unterwerfung gestellte Frist ablaufen. Es
war am Morgen dieses Tages, der die Geschichte der Menschheit
entscheiden mußte, als der Protektor durch den Lichtfernsprecher
der Außenstation am Nordpol den Auftrag geben wollte, eine
Nachricht durch Lichtdepesche nach dem Mars zu senden. Vergeblich
versuchte der Beamte zu sprechen. Der Apparat versagte – man mußte
auf der Außenstation den Anschluß nicht zustande bringen können. Es
wurde nun nach der Polinsel Ara telegraphiert. Die Leitung war
nicht unterbrochen. Aber lange erhielt man keine Antwort. Endlich
kam eine Depesche: „Anwesenheit des Protektors sofort
erforderlich.“ Das Schiff des Protektors raste nach dem Nordpol,
von einer kleinen Schutzflottille gefolgt. Im Lauf des Nachmittags
bemerkte man, daß die übrigen in Washington befindlichen
Luftschiffe der Martier ebenfalls nach Norden hin sich entfernten.
Gleiche Nachrichten liefen aus allen übrigen Städten ein. Sobald
das letzte Schiff der Martier die Hauptstadt verlassen hatte,
tauchten vorher in den Häusern verborgen gehaltene amerikanische
Truppen überall auf, die martischen Beamten, die allein den Verkehr
mit dem Pol hatten vermitteln dürfen, sahen sich plötzlich für
gefangen erklärt, und die nächste Depesche nach dem Pol lautete,
nicht mehr in martischer, sondern in englischer Sprache: „Wir sind
im Besitz des Telegraphen. Die feindlichen Schiffe sind fort.“

Und die Antwort, gezeichnet vom Bundesfeldherrn Miller, lautete:
„Großer Sieg! Die Außenstation ist erobert, achtzehn Raumschiffe
mit 83 Luftschiffen fielen in unsere Hände. Lei gefangen. Von den
zurückkehrenden Luftschiffen sind bereits über fünfzig genommen.
Ruft alle Völker zum Kampf auf!“

Das Unglaubliche war geschehen. Was niemand für möglich gehalten
hatte – die Macht der Martier war gebrochen, die Unbesiegbaren
waren gefangen in ihrem eigenen Bollwerk! Eine Vereinigung von
lange vorbereiteter Überlegung, von unerhörter Tatkraft und
schlauem Mut hatte es zustande gebracht. Die Nume waren vollständig
überrascht worden.

Tief verborgen in der Einsamkeit des Urwalds war ein Verein von
Ingenieuren seit Jahr und Tag tätig gewesen. Der Opfersinn
amerikanischer Bürger und die von der ganzen Erde
zusammenströmenden Mittel des Menschenbundes hatten hier eine mit
unbeschränktem Kapital arbeitende Werkstatt ins Leben gerufen. Man
hatte auf dem Mars die Technik des Luftschiffbaus schon längst
studieren lassen, und auf der Erde diente das Luftschiff ›La‹ als
Muster. Es war gelungen, durch schlaue Operationen große
Quantitäten von Rob, Repulsit und Nihilit einzuführen, und der
allmächtige Dollar hatte es in Verbindung mit Kühnheit und
Intelligenz fertiggebracht, daß hier in aller Stille eine Flotte
von dreißig Luftschiffen hergestellt worden war. Die nötige
Mannschaft war eingeübt worden. Das Letztere war hauptsächlich
Saltner zu verdanken, der diesen Dienst auf seinem eigenen
Luftschiff gründlich erlernt hatte. So war es gekommen, daß die
Vereinigten Staaten ohne Wissen der Martier über Luft-Kriegsschiffe
verfügten, die den martischen an Geschwindigkeit nichts nachgaben.

Freilich, diese wenigen Schiffe konnten gegen die Übermacht der
Martier und ihre überlegene Übung nichts ausrichten. Aber General
Miller, der Generalstabschef der Union, hatte einen Plan
ausgedacht, zu dessen Durchführung sie ausreichen konnten.

Sobald die Flotte der Martier zur Besetzung der Staaten
aufgebrochen war, hatte sich die kleine Unionsflotte unbemerkt in
das nördliche Polargebiet begeben. Äußerlich besaßen die Schiffe
ganz das Ansehen und die Abzeichen der martischen Kriegsschiffe. So
näherten sie sich unbefangen der Polinsel Ara. Keiner der hier
anwesenden Martier konnte vermuten, daß es sich um feindliche
Schiffe handeln könne. Die Insel war überhaupt nicht eigentlich
militärisch besetzt, denn sie war durch ihre Lage am Nordpol
vollständig gegen eine Überrumpelung geschützt gegenüber einem
Feind, der keine Luftflotte besaß. Außerdem ließ sich die ganze
Insel auf dem Meer durch einen Nihilit-Kordon gegen jede Annäherung
zu Schiff absperren. Es befanden sich daher nur einige Avisos zum
Nachrichtendienst hier. Auf den benachbarten Inseln waren noch
große Werkstätten errichtet, wo die vom Mars eingeführten
Luftschiffe montiert und bemannt wurden. Daneben befanden sich
ausgedehnte Werke zur Komprimierung von Luft, die nach dem Mars
verfrachtet wurde. Im ganzen hatte sich so hier eine Kolonie von
einigen tausend Martiern angesiedelt, die aber in keiner Weise auf
einen kriegerischen Angriff eingerichtet war.

Die Überrumpelung der Insel gelang vollkommen. Zwei Schiffe drangen
unmittelbar an den inneren Rand des Daches der Insel. Die Besatzung
dieser Schiffe bestand aus lauter Freiwilligen, die geschulte
Ingenieure waren und die Einrichtungen des abarischen Feldes
sorgfältig studiert hatten. Ehe man in den Maschinenräumen wußte,
was vorging, waren die martischen Ingenieure überwältigt oder durch
die vorgehaltene Waffe zur Ausführung der Befehle der Amerikaner
gezwungen. Sie wurden verhindert, eine Nachricht durch das
abarische Feld nach der Außenstation zu geben. Den nächsten
Flugwagen, der zum Ring der Außenstation auffuhr, bestieg General
Miller selbst mit einer auserwählten Schar von Offizieren,
Ingenieuren und Mannschaften. Eine Stunde später waren sie auf dem
Ring. Auch hier wurden die Ingenieure, welche das abarische Feld
bedienten, ohne Schwierigkeit überrumpelt und gefesselt. Dann drang
man in die obere Galerie, die große Landungshalle der Raumschiffe
vor. Hier lag die größte Schwierigkeit. Mehrere hundert Martier
waren damit beschäftigt, die Raumschiffe zu entladen, denn es waren
neue Raumschiffe gekommen mit Kriegsmaterial, vor allem mit neuen
Luftschiffen. Dies waren hauptsächlich Mannschaften der
Kriegsflotte, die mit Telelytrevolvern bewaffnet waren. Sobald sie
die erste Überraschung überwunden hatten, setzten sie sich zur Wehr
und zwangen das kleine Häuflein der Angreifer, sich schleunigst in
das untere Stockwerk zurückzuziehen. Hier erhielten diese zwar nach
einiger Zeit Verstärkung durch einen zweiten Flugwagen, dennoch
konnten es beide Teile nicht auf einen Kampf ankommen lassen – die
Telelytwaffen, die hier gegeneinander wirksam geworden wären,
hätten binnen wenigen Minuten zur vollständigen Vernichtung von
Freund wie Feind geführt. Die Menschen aber befanden sich im Besitz
des abarischen Feldes und der Elektromagneten der Insel – sie
drohten, den ganzen Ring durch Veränderung des Feldes zum Sinken zu
bringen und die Außenstation zu zerstören, wenn sich die Martier
nicht auf der Stelle ergäben.

Die Martier konnten zwar auf ihren Raumschiffen die Außenstation
verlassen, doch hätte es mehrere Stunden gedauert, ehe sie
dieselben klar zum Raumflug hätten machen können. In dieser Zeit
konnte, wenn die Menschen ernstmachten, das Kraftfeld der Station
und damit das Gleichgewicht des Ringes gestört werden. Überhaupt
sagten sie sich, daß sie bald Hilfe und Ersatz von den Ihrigen
bekommen müßten, und wollten deshalb nicht diese wichtigste ihrer
Anlagen auf der Erde gefährden. So blieb ihnen nichts übrig, als
sich gefangenzugeben.

Inzwischen hatten die übrigen amerikanischen Luftschiffe die
gesamte Kolonie auf den Inseln um den Pol eingeschlossen und
rücksichtslos mit ihren Nihilitsphären und Repulsitgeschützen
angegriffen. Die vollständig überraschten Martier waren wehrlos,
die wenigen Schiffe, die zum Gebrauch fertig waren, wurden sofort
durch die Angreifer zerstört, ehe sie soweit bemannt waren, daß sie
sich durch den Nihilitpanzer schützen konnten. Andererseits waren
diesmal die Menschen durch das Nihilit gegen einen Angriff durch
die Telelytwaffen geschützt. Auch hier war die Überrumpelung
gelungen, die Martier mußten sich ergeben. Sie wurden sämtlich auf
der Insel Ara untergebracht und hier bewacht.

Sobald die Insel im Besitz der Amerikaner war, wurde nach den
Städten der Union telegraphiert, gleich als ob es sich um Bitten
oder Anordnungen der Martier handle. Zunächst hatte man den
Protektor um sofortige Rückkehr gebeten, dann richtete man ähnliche
Ansuchen an die übrigen Schiffe der Martier. Einzelne Kapitäne
folgten ohne Bedenken, andere hielten weitere Umfragen, wodurch
eine allgemeine Verwirrung entstand. Es bestätigte sich jedoch, daß
der Protektor selbst mit einer Flottille nach dem Pol aufgebrochen
war. Endlich kam von der dem Pol zunächst gelegenen Station von
einem martischen Kapitän selbst ein in der amtlichen Geheimschrift
aufgegebenes Telegramm, das den tatsächlichen Vorgang meldete; die
Polstation sei von einer Luftschifflotte der Union überfallen.
Hierauf wurden sämtliche Schiffe zur Hilfeleistung nach dem Pol
berufen, und auch das letzte Stationsschiff verließ Washington. Der
Telegraph wurde nun von den Beamten der Union in Besitz genommen,
und die Menschen erhielten jetzt die Nachricht von dem unerhörten
Ereignis.

Ahnungslos war Lei mit dem schnellen Admiralsschiff allen andern
vorangeeilt, um nur sobald als möglich auf der Insel zu erfahren,
was geschehen sei. In seinem raschen Flug bemerkte er die
Zerstörungen in der Kolonie, konnte aber nichts anderes glauben,
als daß es sich um einen Unglücksfall, eine Explosion handle. Er
senkte sich auf das Dach der Insel, wo nichts Verdächtiges zu
bemerken war. Aber kaum berührte das Schiff das Dach, als es im
Augenblick erstürmt wurde. Der Protektor der Erde war
kriegsgefangen.

Nun erhob sich die kleine Luftflotte der Amerikaner und flog den
nach und nach eintreffenden martischen Schiffen entgegen. Diese
konnten in den sich nähernden Schiffen nichts anderes erwarten wie
entgegenkommende Boten. Sie mäßigten ihren Flug, um etwaige Signale
zu erkennen. Da zischten die Repulsitgeschosse, und ehe sich eine
Hand nach dem Griff des schützenden Nihilitapparates ausstrecken
konnte, wurden die Robhüllen zertrümmert, und die Schiffe der
Martier stürzten in die Wogen des Meeres oder zerschellten auf den
schwimmenden Eismassen. Es war eine furchtbare, erbarmungslose
Zerschmetterung der Feinde.

Noch mehrfach gelang es, vereinzelt ankommende Schiffe der Martier
durch Überraschung zum Sinken zu bringen. Dann hatten einige der
nachfolgenden Schiffe den Überfall bemerkt, die später
eintreffenden waren gewarnt und näherten sich in ihren
Nihilitpanzern. Zwischen zwei mit den Waffen und
Verteidigungsmitteln der Martier ausgerüsteten Schiffen konnte es
keinen Kampf geben, beide waren unverletzlich. Die Amerikaner zogen
sich daher auf die Insel zurück, deren Umkreis auf dem Meer sie
durch die Nihilitzone und deren Dach sie durch ihre Luftschiffe
schützten. So war es auch den Martiern, die nun im Verlauf des
Tages ihre ganze Flotte aus den Vereinigten Staaten um den Pol
konzentrierten, nicht möglich, einen Angriff zu wagen.

Während die Kapitäne noch berieten, brachte ein Schiff die
Nachricht, daß nach einer Depesche vom Südpol auch die Außenstation
an diesem Pol in den Händen der Menschen sei. Sie war gleichzeitig
mit dem Nordpol von zwei amerikanischen Luftschiffen überrascht
worden, die hier leichtes Spiel hatten. Der Südpol lag in der Nacht
des Winters vergraben, die Station war bis auf eine kleine Anzahl
Wächter verlassen, die den unerwarteten Besuch ohne Mißtrauen
aufgenommen hatten und sogleich überwältigt worden waren.

Die Nume auf der Erde waren somit von jeder Verbindung mit dem Mars
abgeschnitten.

Als die Nachricht nach Europa gelangte, brach ein Jubel aus, wie
ihn die Erde noch nicht vernommen. Aber auch hier war alles zu
einer Erhebung vorbereitet. Überall, wo sich die Beamten der
Martier nicht in ihre Luftschiffe retten konnten, bemächtigte man
sich ihrer Personen. Allerdings hielten die Luftschiffe ihrerseits
die Hauptstädte besetzt und bedrohten sie mit vollständiger
Vernichtung. Sie unterbrachen die Verbindungen der Länder mit dem
Pol, und zwei Tage lang schwebte Europa wieder in banger Sorge. Es
war der Rache der Martier schutzlos ausgesetzt, und die Regierungen
waren gezwungen, die eignen Staatsbürger zum Teil mit Anwendung von
Gewalt zu veranlassen, die gefangenen Martier wieder freizugeben.
Der erste Jubel verklang so schnell, wie er gekommen war, und eine
tiefe Niedergeschlagenheit trat an seine Stelle.

Doch welch Erstaunen ergriff die Bewohner der europäischen
Hauptstädte, als sie eines Tages die drohenden Kriegsschiffe auf
den Dächern der Regierungsgebäude verschwunden sahen. Zuerst wollte
man an keine günstige Veränderung glauben, man befürchtete
irgendeine unbekannte, neue Gefahr. Um Mittag erst erklärte eine
Bekanntmachung der Regierungen allen Völkern, was geschehen sei.
Der Waffenstillstand mit dem Mars war geschlossen worden.

Die Amerikaner hatten am Pol neben ungeheuren Vorräten an Rob und
Kriegsmaterial einige achtzig Luftschiffe erbeutet und diese durch
die gefangenen Martier instand setzen lassen. Dadurch waren sie in
die Lage gesetzt, nicht nur den Pol zu halten, sondern ihre Macht
auch über die ganze Erde zu erstrecken. Zwar konnten sie den
Schiffen der Martier nichts anhaben, aber ebensowenig konnten sie
von diesen aufgehalten werden. Sie begaben sich nach allen
denjenigen Punkten der Erde, wo die Martier große Anlagen zur
Verwertung der Sonnenstrahlung geschaffen hatten, und bedrohten
diese mit Vernichtung des martischen Eigentums. Zugleich drohte man
mit der völligen Zerstörung der Außenstationen an den Polen.
Hierdurch wäre nicht nur das Leben von einigen tausend Martiern,
sondern auch ein unermeßliches Kapital und die Verbindung zwischen
Erde und Mars zerstört worden.

Der gefangene Protektor korrespondierte von der Außenstation aus
durch Lichtdepeschen mit dem Zentralrat des Mars. Hier erkannte man
alsbald, daß die Gefahr ungeheurer Verluste und Verheerungen nur
durch einen friedlichen Ausgleich zu vermeiden war. Der Zentralrat
konnte nicht wagen, einen Vernichtungskrieg zu beginnen, der zwar
schließlich mit der Ausrottung der Menschen und ihrer Kultur
geendet, aber der Regierung der Marsstaaten die Verantwortung
aufgebürdet hätte. Es wurde daher zwischen den Marsstaaten und dem
Polreich der Erde einerseits, den Vereinigten Staaten, die auf
einmal die führende Macht der Erde geworden waren, und den
Großmächten Europas andererseits ein Waffenstillstand geschlossen,
dessen Bestimmungen im wesentlichen folgende waren:

Das Recht der Menschen auf die Freiheit der Person wird anerkannt.
Die Nume sollen auf der Erde keinerlei Vorrechte besitzen.

Das Protektorat über die Erde wird aufgehoben. Sämtliche bisherige
Beamte der Marsstaaten auf der Erde und sämtliche Kriegsschiffe
haben die Erde zu verlassen.

Die Kriegsgefangenen werden freigegeben.

Die Stationen der Martier auf den Polen sowie ihr gesamtes auf der
Erde erworbenes Vermögen bleibt ihnen erhalten, desgleichen ihre
Raumschiffe auf der Außenstation des Nordpols. Doch bleiben diese
Stationen so lange im Besitz der Amerikaner, bis durch einen
endgültigen Friedensvertrag das künftige Verhältnis der beiden
Planeten geregelt sein wird, und zwar nach Maßgabe obiger
Grundsätze.

Dieser Friedensvertrag ist innerhalb eines halben Erdenjahres
abzuschließen und soll den freien Handelsverkehr beider Planeten
als eine Bestimmung enthalten.

Der Sprung von der Not zur Rettung war so ungeheuer, daß man erst
allmählich fassen konnte, welches Heil der Menschheit zuteil
geworden. Und nun war die Freude unbeschreiblich.

\tb{}
Vom Mars kam Raumschiff auf Raumschiff und führte die Kriegsschiffe
der Martier und diese selbst nach dem Nu zurück. Die Staaten
ordneten aufs neue ihre Verfassungen und schlossen untereinander
ein Friedensbündnis, das die zivilisierte Erde umfaßte. Die
Grundsätze, welche der Menschenbund verbreitet und gepflegt hatte,
trugen dabei ihre Früchte. Ein neuer Geist erfüllte die Menschheit,
mutig erhob sie das Haupt in Frieden, Freiheit und Würde.

Am dritten August verließ das letzte Raumschiff der Martier die
Erde. Erst wenn der definitive Friede geschlossen war, sollte ein
regelmäßiger, friedlicher Verkehr wieder beginnen. Bis dahin
durften nur Lichtdepeschen gewechselt werden.

\section{60 - Weltfrieden}

Saltner hatte sich aus Rücksicht auf Las Eigenschaft als Martierin
an der kriegerischen Erhebung gegen die Martier nicht beteiligt. La
bedauerte innig die Trübung der Beziehungen zwischen den Planeten,
doch stand sie nicht bloß als Gattin ihres Mannes, sondern auch mit
ihrem Gerechtigkeitsgefühl auf der Seite der Menschen, die für ihre
Unabhängigkeit kämpften. Sie hörte nicht auf zu glauben, daß die
Vernunft auf dem Mars siegen und zu einem heilsamen Frieden führen
werde.

Sobald die Herrschaft der Martier über Europa aufgehört hatte,
begab sich Saltner mit La und den übrigen Angehörigen des
Luftschiffs in seine Heimat zurück. Er gab damit vor allem dem
Wunsch seiner Mutter nach, die von tiefer Sehnsucht nach ihren
heimatlichen Bergen befallen war. In der Nähe von Bozen, hoch über
dem Tal, erwarb La eine schloßartige Villa, um den Herbst und
Winter in diesem geschützten südlichen Klima und doch in Höhenluft
zuzubringen.

Der Verkehr durch Lichtdepeschen und die Friedensverhandlungen mit
dem Mars gestalteten sich nicht so einfach, wie man gehofft hatte.
Die Beamten, welche den Lichtverkehr zu vermitteln hatten, waren
wenig geübt, und als im Herbst die telegraphische Station auf die
Außenstation am Südpol verlegt werden mußte, gelang es nur mit
Schwierigkeit, den Apparat hier überhaupt zur Funktion zu bringen.
Eine Zeitlang fürchtete man, damit gar nicht zu Rande zu kommen,
und als dies endlich geglückt war, kamen nicht selten
Mißverständnisse im Depeschenwechsel vor, der infolgedessen von den
Martiern auf das Dringendste eingeschränkt wurde.

Und doch hätte man gerade jetzt auf der Erde, mehr als je, gern
Näheres über die Vorgänge auf dem Mars erfahren. Denn die letzten
Nachrichten waren beunruhigender Natur gewesen, und als über ein
Vierteljahr vergangen war, ohne daß die entscheidende
Friedensnachricht vom Mars eintraf, begannen beängstigende Gerüchte
über die Absichten der Martier sich auf der Erde zu verbreiten. Es
waren wiederholt in der Nähe der Station Raumschiffe beobachtet
worden, die sich allerdings in gehöriger Entfernung hielten, aber,
wie man fürchtete, die Vorboten irgendeiner feindlichen
Unternehmung sein konnten.

In der Tat stand das Schicksal der Erde vor einer furchtbaren
Entscheidung.

Die Niederlage der Martier, der Verlust der Herrschaft über die
Erde, hatte der Antibaten-Partei zunächst einen schweren Schlag
versetzt. Die Vertreter einer menschenfreundlichen Politik wiesen
darauf hin, wie allein das scharfe und ungerechte Vorgehen gegen
die Bewohner der Erde die Schuld trage, daß der Nume nun vor dem
Menschen sich demütigen müsse. Es sei dies aber eine gerechte
Strafe für die Fehler der Antibaten, die sich somit als unfähig zur
Führung der Regierungsgeschäfte erwiesen hätten. Die Idee der
Numenheit, die Gerechtigkeit gegen alle Vernunftwesen verlange als
die allein würdige Sühne die Bestätigung der Freiheit, welche die
Menschen sich erkämpft hätten. Es gäbe überdies kein Mittel, die
Menschen, seitdem sie sich im Besitz der Waffen der Martier
befänden, auf eine andre Weise zu bezwingen, als durch eine
vollständige Verheerung ihres Wohnorts; eine solche Barbarei aber
könne den Numen nie in den Sinn kommen. Sie seien der Erde genaht,
um ihr Frieden, Kultur und Gedeihen zu bringen, nicht um einen
blühenden Planeten zu vernichten, nur damit sie seine Oberfläche
zur Sammlung der Sonnen-Energie ausbeuten könnten.

Obwohl diese Ansicht wieder die öffentliche Meinung zu beherrschen
begann, war doch die Macht der Antibaten noch keineswegs gebrochen.
Es gab eine große Anzahl Martier, deren wirtschaftliche Interessen
durch den Verlust der von der Erde fließenden Kontributionen
geschädigt waren und deren Vernunft durch den Egoismus der
Herrschsucht Einbuße erlitten hatte. Sie stellten sich auf den
Standpunkt, daß die menschliche Rasse überhaupt nicht kulturfähig
im Sinne der Nume sei und daß es daher für die Gesamtkultur des
Sonnensystems besser sei, die Bewohner der Erde zu vernichten,
damit ihr Planet den wahren Trägern der Kultur als unerschöpfliche
Energiequelle diene. Der Wortführer dieser Ansicht war Oß, während
Ell an der Spitze der Menschenfreunde stand. Man warf ihm vor, daß
ja gerade durch seine Amtsführung als Kultor erwiesen wäre, wie
unfähig die Menschen zur Aneignung der martischen Kultur seien.
Habe er doch selbst sein Amt aufgegeben.

Ell gab zu, daß er sich über die Schnelligkeit getäuscht habe, mit
der seine Reformen zur Wirkung gelangen könnten. Die Nume seien zu
zeitig zur Erde gekommen, die Menschheit sei allerdings noch nicht
reif für die Lebensführung der Martier. Aber sie habe doch gezeigt,
daß sie zu vorgeschritten sei, um als unfrei behandelt zu werden.
Und deshalb sei es nunmehr der richtige Weg, durch einen
friedlichen Verkehr mit der Erde die Vorteile auszunutzen, welche
die Erde als Energiequelle biete, zugleich aber damit der
Menschheit das Beispiel einer überlegenen Kultur zu geben, die ihr
ein Vorbild sein könne. Nicht durch Unterjochung, sondern durch
freien Wetteifer müßten die Menschen erst auf die Stufe geführt
werden, die sie für die direkte Aufnahme martischer Kultur fähig
mache.

Diese entgegengesetzten Meinungen, die in den Marsstaaten zu
heftigen politischen Kämpfen führten, verzögerten die endgültige
Entscheidung über den Friedensschluß. Beide Parteien suchten den
Abschluß immer wieder hinauszuschieben in der Hoffnung, bei den
nächsten Wahlen zum Zentralrat eine entscheidende Majorität zu
bekommen. Man wußte dies auf der Erde und sah daher dem Ausfall
dieser Wahl mit Spannung und Furcht entgegen. Ell und Oß
kandidierten beide für den Zentralrat. Der Sieg Ells bedeutete den
Frieden. Der Sieg von Oß ließ befürchten, daß die Martier für ihre
Niederlage vom 11. Juli furchtbare Rache nehmen würden. Anfang
Dezember mußte die Wahl stattfinden, die Entscheidung fallen. Und
gerade jetzt versagte wieder der Lichttelegraph. Seit vierzehn
Tagen hatte man keine Depesche vom Mars erhalten, vergeblich
arbeitete und operierte man an dem Apparat – die Rechnungen wollten
mit den Beobachtungen nicht stimmen –, und jeden Tag depeschierte
man vom Südpol, daß man bestimmt hoffe, morgen mit der Einstellung
fertig zu werden.

Unheimliche Gerüchte über die Absichten der Martier durchschwirrten
die Erde. Eines vor allen nahm immer deutlichere Gestalt an und
erfüllte die Gemüter mit Grausen. Man sagte, daß sich in Papieren
der Martier, die nach der eiligen Entfernung der Beamten
aufgefunden worden seien, ausgearbeitete Projekte befunden hätten
zu einer völligen Vernichtung der Zivilisation der Erde. Der
ehemalige Instruktor von Bozen, Oß, der Kandidat der Antibaten für
den Zentralrat, bekannt als ein hervorragender Ingenieur, sollte
der Urheber eines Planes sein, wonach bei einem dauernden
Widerstand der Menschen die Oberfläche der Erde unbewohnbar gemacht
werden konnte. Einzelne Blätter brachten detaillierte Ausführungen.
Es handelte sich um nichts Geringeres als die Absicht, die tägliche
Umdrehung der Erde um ihre Achse aufzuheben. Diese Rotation der
Erde sollte so verlangsamt werden, daß der Tag allmählich immer
länger wurde und endlich mit dem Umlauf der Erde um die Sonne
zusammenfiele, daß also Tag und Jahr gleich würden. Dann würde die
Erde in derselben Lage zur Sonne sein wie der Mond zur Erde, das
heißt, sie würde der Sonne stets dieselbe Seite zukehren. Es gäbe
keinen Unterschied mehr von Tag und Nacht, die eine Seite der Erde
hätte ewigen Sonnenschein, die andere ewige Finsternis – die Sonne
bliebe für denselben Ort stets in demselben Meridian stehen. – Die
Folgen einer solchen Veränderung wären furchtbar gewesen. Der Plan
der Martier sollte angeblich dahin gehen, die Erde in eine solche
Stellung zu bringen, daß der Stille Ozean in ewiger Sonnenglut, die
großen Festlandmassen aber, der Hauptsitz der zivilisierten
Staaten, in ununterbrochener Nacht blieben. Dann mußte allmählich
eine Verdampfung des gesamten Meeres stattfinden. Denn die
Wasserdämpfe würden sich auf der immer kälter werdenden Nachtseite
der Erde niederschlagen und diese mit ewigem Schnee und
unschmelzbarem Gletschereis überziehen. Eine Eiszeit, der kein
Leben widerstehen könnte, würde auf die Schattenseite der Erde
hereinbrechen, während die Sonnenseite in Gluten verdorren würde.
Wohl nur auf einer schmalen Grenzzone könnte sich Leben erhalten.
Aber wer vermochte zu sagen, welch andere, verderbliche
Umwandlungen bei einer derartigen Änderung des Gleichgewichts von
Luft und Wasser auf der Erde noch eintreten mochten?

Wohl versuchte man diesen Plan als ein törichtes Hirngespinst
hinzustellen, als ein Schreckmittel, das die Martier wohl
absichtlich den Menschen zurückgelassen hätten. Doch konnte man die
entschiedenen Befürchtungen nicht genügend zerstreuen. Das Projekt
schien zu gut fundiert. Oß hatte die Energiemenge ausgerechnet, die
zur Hemmung der Erdrotation erforderlich ist. Sie ist allerdings so
groß als die Strahlungsenergie, die von der Sonne in 600 Jahren zur
Erde gelangt, wenn man nur die gegenwärtig den Menschen auf der
Erdoberfläche zugängliche Energie in Anschlag bringt. Viel größer
aber ist die Energiestrahlung unter Berücksichtigung aller
Strahlengattungen. Und wenn die Martier den von ihnen
aufgespeicherten Energieschatz aufbrauchten, so waren sie sicher,
ihn wieder ersetzen zu können. Oß hatte eine Methode ausgedacht –
er nannte sie die ›Erdbremse‹ –, wonach die Rotationsenergie der
Erde selbst die Arbeitsquelle sein sollte, um eine Hemmung
erzeugen, sie sollte zur Arbeit benutzt und somit die Erde durch
sich selbst gebremst werden. Zwanzig Jahre genügten seiner Rechnung
nach, um die Erdrotation auf das gewünschte Maß zu verringern.

Mit besonderem Bangen sah man dem 11. Dezember entgegen. An diesem
Tag fand die Opposition von Mars und Erde statt, es trat die
Stellung ein, in der die beiden Planeten sich am nächsten befanden.
Bei der Opposition am Ende des August vor vier Jahren war die
Anwesenheit der Martier auf der Erde entdeckt worden; die
Opposition im Oktober vor zwei Jahren hatte den Sieg der
Antibatenpartei gebracht; so bildete man sich ein, die nächste
Opposition im Dezember dieses Jahres müsse wieder durch irgendein
unheilvolles Ereignis sich auszeichnen. Daß sich dieses gerade an
den 11. Dezember, als den Tag der Opposition, knüpfen müsse, war ja
eine Art Aberglaube; daß aber die Zeit der größten Annäherung der
Planeten die günstigste für etwaige Unternehmungen der Martier
gegen die Erde war, ließ sich nicht leugnen. Und so fehlte es nicht
an düsteren Prophezeiungen für diesen Tag.

Das Aufhören des Depeschenverkehrs mit dem Mars vergrößerte nun die
Sorge. Man befürchtete, daß die Antibatenpartei gesiegt habe und
die Unmöglichkeit, den Apparat einzustellen, auf einer
absichtlichen Störung durch die Martier beruhe. Wenn das auch
seitens der Union, die im Besitz der Außenstationen war, nicht
zugegeben wurde, so traf man doch Anstalten, im Fall eines
unerwarteten Erscheinens von Raumschiffen der Martier die Station
sperren, ja im Notfall stürzen zu können. Seltsam war es gewiß, daß
auch auf der Station am Nordpol, wohin man trotz des Polarwinters
ein Luftschiff entsandt hatte, die Einstellung des Phototelegraphen
nicht gelingen wollte.

Inzwischen war die Entscheidung auf dem Mars gefallen. Ein
aufregender Streit der Meinungen, wie er seit Jahrtausenden in der
politischen Geschichte des Mars unerhört war, fand endlich seine
Schlichtung. Die Beweggründe, die Ell zuletzt ins Feld führte,
hatten einen durchschlagenden Erfolg. Der Plan von Oß, die Erde zu
bremsen, bestand wirklich, und Ell zeigte, zu welchen
unmenschlichen und verwerflichen Folgen diese wahnwitzige
Unternehmung führen müsse, deren Möglichkeit außerdem durchaus
fraglich sei. Und endlich deckte er einen Umstand auf, der bisher
noch immer als Geheimnis behandelt worden war – die Gefahr, die den
Menschen und vielleicht auch den Martiern bei einem dauernden
Aufenthalt auf der Erde drohte, das Wiederaufleben der furchtbaren
Krankheit Gragra. Selbst auf diese hatte Oß in einem geheimen
Memorial hingewiesen als auf ein Mittel, die Menschen zu
vernichten. Ell scheute sich nicht, dieses Aktenstück zu
veröffentlichen. Da erhob sich eine allgemeine Entrüstung in dem
überwiegenden Teil der Martier. Schon die ganze Methode geheimer
Pläne und Machinationen, die den Martiern als ein bedenkliches
Zeichen politischen Rückschritts erschien, noch mehr aber der
Verfall der Gesinnung, die Mißachtung des sittlich Guten und Edlen
empörte auch das Gemüt derer, die sich eine Zeitlang durch
Sondervorteile hatten zu Menschenfeinden machen lassen, und
erweckten sie zum Bewußtsein ihrer Würde als Nume. So brachte der
Tag der Wahl ein überraschendes Resultat. Der Registrierapparat der
telegraphisch abgegebenen Stimmen zeigte für Ell über 312 Millionen
Stimmen gegen etwa 40 Millionen für Oß.

Ell war mit einer erdrückenden Mehrheit in den Zentralrat gewählt,
mit ihm noch Ill und drei andere Führer der menschenfreundlichen
Partei. Die antibatische Bewegung war hierdurch endgültig
unterdrückt.

Schon am folgenden Tag genehmigte der Zentralrat den
Friedensvertrag mit den verbündeten Erdstaaten in der Fassung, wie
er längst sorgfältig ausgearbeitet von der menschenfreundlichen
Partei vorlag.

Aber ein unerwartetes Hindernis zeigte sich. Schon in den letzten
Tagen waren die Depeschen nicht mehr von der Erde erwidert worden.
Eine Störung des Apparats war vorhanden, und die Martier erkannten,
daß sie auf der Unfähigkeit der Menschen beruhte, ihren
Phototelegraphen zur Einstellung zu bringen. Trotz aller Bemühungen
war es unmöglich, die Friedensbotschaft der Erde durch
Lichtdepesche mitzuteilen.

Der Zentralrat hatte beschlossen, daß Ell, in Anerkennung seiner
Verdienste um die Erschließung der Erde und der nun erlangten
Versöhnung der Planeten, an der Spitze der Kommission nach der Erde
gehen sollte, die beauftragt war, den Friedensvertrag zwischen
beiden Planeten zu vollziehen. Aber es war im Waffenstillstand
bestimmt worden, daß kein Raumschiff auf der Erde landen sollte,
bis nicht telegraphisch die Annahme des Friedens durch die
Marsstaaten mitgeteilt sei. Und das war nun vorläufig unmöglich.

Ein Raumschiff, das man entsandte, um Aufklärung über die Ursache
der Störungen zu erhalten, und das mit der größten erreichbaren
Geschwindigkeit fuhr, kehrte nach zwölf Tagen unverrichteter Sache
zurück. Es hatte versucht, sich durch Signale mit der Außenstation
am Südpol zu verständigen, war aber nicht verstanden worden. Und
als es Anstalten traf, sich auf die Station hinabzulassen, wurde es
durch Repulsitstrahlen bedroht und an der Landung verhindert, so
daß es wieder umkehren mußte. Doch berichtete es, daß, soviel sich
bemerken ließe, die Station nicht in richtiger Verfassung zu sein
scheine und die Unmöglichkeit des telegraphischen Verkehrs
vielleicht an einer Verschiebung der Außenstation liege.

Hierauf nahm man seine Zuflucht zum Retrospektiv. Dies gestattete,
die Station genau zu beobachten. Und nun stellte sich für die
Gelehrten der Martier unzweideutig heraus, daß der Ring der
Außenstation seine Lage geändert habe. Die Berechnung zeigte, daß
binnen kurzem das Gleichgewicht des ganzen Kraftfeldes überhaupt
gestört werden müßte, wenn nicht bald eine Korrektur eintrat. Die
Menschen hatten es nicht richtig verstanden, die Korrektionen
vorzunehmen, die zur Erhaltung des Feldes und des Ringes notwendig
waren. Die Karte der Polargegend, die auf dem Dach der unteren
Stationen sich befand und den Entdeckern des Nordpols das erste,
unlösbare Rätsel über die Einrichtungen der Martier aufgegeben
hatte, diente nämlich dazu, eine Kontrolle für die feinen
Bewegungen der Außenstation infolge von Schwankungen der Erdachse
zu haben. Beide Stationen, im Norden wie im Süden, schwebten nun in
höchster Gefahr. Es mußte, sollte nicht der Verkehr mit der Erde
dauernd in Frage gestellt sein, sofort das Kraftfeld in den
richtigen Stand gesetzt werden, und dies konnte nur durch martische
Ingenieure geschehen.

Wie aber sollten die Martier dies rechtzeitig bewirken, da sie
jetzt kein Mittel hatten, die Menschen zu benachrichtigen, und ihre
Raumschiffe der Gefahr ausgesetzt waren, von den Menschen bei der
Landung zerstört zu werden? Und selbst, wenn es gelang, sich vor
der Landung mit den Menschen zu verständigen, so war es noch immer
sehr fraglich, ob bei dem Zustand der Station nicht diese Landung
mit unbekannten Gefahren verbunden sei. Jetzt das Band mit der Erde
neu zu knüpfen, indem man sich einem Raumschiff anvertraute, war
ein Unternehmen auf Leben und Tod. Wer wollte sich daran wagen? Der
Wille zum Frieden war auf beiden Planeten vorhanden, der Beschluß
der friedlichen Übereinkunft auf beiden Seiten gefaßt. Und nun
sollte der Weltfrieden daran scheitern, daß man die
Friedensbotschaft nicht verkündigen, die einzige Brücke, die
Außenstation, nicht vor der Vernichtung schützen konnte?

Da erbot sich Ell, das Rettungswerk zu unternehmen. Er wußte, was
er wagte. Aber er wußte auch, daß, wenn irgend jemand, so ihm die
Pflicht erwachsen war, die Verbindung zwischen den Planeten
herzustellen. Wieder stand er so nahe an der Erfüllung seines
Lebenszwecks, und noch einmal sollte seine Hoffnung fehlschlagen?
Aber es war auch die einzige Aufgabe, die er noch zu erfüllen
hatte. War der Friede geschlossen, so war alles getan, was er tun
konnte.

Eine freiwillige Gruppe geübter Ingenieure schloß sich ihm an. Das
Regierungsschiff ›Glo‹ sollte Ell mit seinen Genossen binnen sechs
Tagen nach der Erde bringen. Man hatte verschiedene Maßregeln
ausgedacht, um den Menschen die friedliche Absicht kundzutun,
insbesondere die Übermittlung von direkten Nachrichten durch
Hinabwerfen geeigneter Gegenstände auf die Erde. Die Hauptsorge für
Ell war, ob er noch zurecht kommen würde, den Einsturz der
Außenstation zu verhindern. Mit noch nie erlebter Geschwindigkeit
schoß der ›Glo‹ durch den Weltraum.

Die Störungen des abarischen Feldes und der Außenstation waren zwar
in der letzten Zeit auch von den Menschen wahrgenommen worden, doch
reichten ihre Kenntnisse und Mittel nicht aus, sie in ihren
Ursachen zu erkennen und ihre Bedeutung zu beurteilen. Man wußte
nicht, in wie großer Gefahr die Station schwebe, wenn nicht
schleunigst eine Korrektur eintrete. Als sich Ells Raumschiff der
Station näherte, bemerkte Fru, der genaueste Kenner dieser Technik,
der Ell freiwillig begleitet hatte, daß die Hilfe nur von der
Erdoberfläche aus zu bringen sei. Von dorther mußte das Feld
reguliert werden. Er bezweifelte, ob die regelrechte Beförderung im
Flugwagen überhaupt noch möglich sei oder es für die nächsten
vierundzwanzig Stunden bleiben werde, und da Ell fürchtete, viel
kostbare Zeit zu verlieren, ehe er sich vom Raumschiff aus mit der
Außenstation verständigen könne – denn dies war nur durch
unzureichende Signale möglich –, so beschloß er, überhaupt vom
Anlegen am Ring abzusehen. Er wollte vielmehr versuchen, sogleich
so weit in die Atmosphäre hinabzusteigen, bis die Dichtigkeit der
Luft das Aussetzen eines Luftschiffes gestattete, und mit diesem
wollte er nach dem Pol direkt sich begeben. Es war dabei wichtig,
der Erdachse so nahe wie möglich zu bleiben, obwohl er allerdings
hier befürchten mußte, von den Menschen angegriffen zu werden, ehe
er seine friedlichen Absichten darlegen konnte.

Das Raumschiff hatte sich bis auf zwanzig Kilometer der
Erdoberfläche genähert und kam nun in die Luftschichten, die
freilich bei ihrer geringen Dichtigkeit den Menschen noch nicht
gestatteten, sich in ihnen ohne Schutz aufzuhalten, aber doch die
Grenze bildeten, bis zu welcher sich dicht verschlossene
Luftschiffe allenfalls erheben konnten. Gern wäre Ell noch weiter
hinabgestiegen, indessen schon nahten sich Kriegsschiffe der
Menschen, deren Angriff er das schutzlose Raumschiff nicht
aussetzen durfte. Aber diese trauten ihrerseits dem Raumschiff
nicht und hielten sich in so weiter Entfernung, daß der Austausch
von Signalen nicht möglich war. Die Martier ließen ihre in Kapseln
eingeschlossenen Briefe durch eine besondere Vorrichtung aus dem
hermetisch geschlossenen Raumschiff herabfallen, doch war nicht
darauf zu rechnen, daß sie im Gewirr der Eisschollen des Bodens
gefunden werden würden. Inzwischen drängte Fru auf einen
entscheidenden Entschluß, da jede Stunde die Gefahr für die
Erhaltung der Station vergrößerte.

So entschloß sich Ell, das Raumschiff in einer Höhe zu verlassen,
zu der die Luftschiffe nicht emporsteigen konnten. Hier war
freilich das Luftschiff der Martier, auf welchem er das Raumschiff
verlassen mußte, selbst der Gefahr ausgesetzt, sich nicht schwebend
halten zu können. Dennoch war der Versuch, auf diese Weise zur Erde
zu gelangen, die einzige Möglichkeit, die übrig blieb. Und Ill
schwankte keinen Augenblick, die gefahrvolle Landung zu versuchen.

Um das Luftboot so leicht wie möglich zu machen, nahmen außer Ell
und Fru nur noch zwei Ingenieure in demselben Platz. Dann wurde es
verschlossen und die Entladungskammer des Raumschiffs geöffnet.
Sobald das Luftschiff von seinem Halt gelöst war, stürzte es mit
großer Geschwindigkeit abwärts. Sofort wurden die Flügel
ausgespannt und der Fall in eine schiefe Ebene übergeleitet, die in
der Richtung nach dem Pol hinführte. So gelangte das Luftschiff bis
in die Höhe von zehn Kilometern hinab und hatte sich damit dem Pol
so weit genähert, daß seine Bahn in eine Schraubenlinie verändert
werden mußte, damit es nicht zu weit vom Pol fortschösse. Jetzt
hatten die amerikanischen Kriegsschiffe das martische Schiff
bemerkt und näherten sich ihm, in ihre Nihilitpanzer gehüllt. Es
gelang den Martiern, ihr Schiff zu ruhigem Schweben zu bringen. Wie
jedoch sollte man sich bei den geschlossenen Schiffen verständigen?
Und sie zu öffnen, verbot die noch viel zu stark verdünnte Luft
dieser Höhe. Fru strebte danach, durch weiteres Sinken um einige
tausend Meter in dichtere Luftschichten zu gelangen. Deshalb zog er
die Flügel des Luftschiffs ein. Nun erst vermochten die
amerikanischen Schiffe die große weiße Fahne zu erkennen, die das
martische Schiff als Friedenszeichen führte. Sie näherten sich
trotzdem weiter. Das eine legte sich in die Fallinie des martischen
Schiffes und deutete damit an, daß es ein weiteres Sinken nicht
zulassen würde. Das andere zog zum Zeichen des Verständnisses
seinen Nihilitpanzer ein und kam dem martischen Schiff so nahe, daß
man die hinter den schützenden Robscheiben ausgeführten Signale
verstehen konnte.

Ell signalisierte: „Wir bringen den Friedensvertrag. Ich, Ell, bin
mit dem Abschluß beauftragt. Laßt uns sofort nach der Station.“

Der Kapitän antwortete: „Ich bin hocherfreut, darf Sie aber nicht
näher heranlassen, bis ich Instruktionen erhalten habe. Es werden
sogleich weitere Schiffe eintreffen.“

Darauf erwiderte Ell: „Es ist höchste Gefahr, die Außenstation ist
im Gleichgewicht gestört. Lassen Sie uns sogleich hin.“

Hierdurch wurde der Kapitän mißtrauisch. Er signalisierte: „Das
verstehe ich nicht.“

Ell war der Verzweiflung nahe. Der zähe Amerikaner antwortete
nicht, und alles konnte an einer halben Stunde hängen, um die man
zu spät zur Station kam. Auch Fru wußte nicht, was zu tun sei. Das
Signalisieren nahm zu viel Zeit in Anspruch. Ja, wenn man sprechen
könnte! Die Schiffe lagen jetzt dicht nebeneinander. Aber durch die
geschlossenen Hüllen konnte der Schall nicht dringen.

„Ich spreche hinüber!“ rief Ell. „Wir können nicht länger warten.“

„Unmöglich“, rief Fru.

„Es muß gehen.“

Ehe ihn die andern hindern konnten, hatte er den Verschluß, der zum
Verdeck führte, geöffnet und wieder geschlossen. Er stand auf dem
Verdeck in der eisigen dünnen Luft. Mit Erstaunen sah man vom
amerikanischen Schiff aus ihm zu. Ell winkte und rief durch ein
Sprachrohr. Man verstand, daß er sprechen wolle. Der Kapitän, in
seinen Pelz gehüllt, den Sauerstoffapparat vor dem Mund, trat
ebenfalls auf das Verdeck. Ell mußte, um zu sprechen, die
Sauerstoffatmung unterbrechen. Er mußte schreien, um in der dünnen
Luft gehört zu werden. So setzte er dem Kapitän die Tatsachen
auseinander. Dieser schüttelte einige Male den Kopf, dann begann er
zu verstehen, er nickte. Er hütete sich wohl zu sprechen. Mehrere
Minuten waren darüber vergangen. Ell fühlte, wie es ihm im Kopf
sauste, wie sein Herz schlug, wie seine Glieder erstarrten, seine
Augen nichts mehr erkannten. Aber der Amerikaner trat in sein
Schiff zurück, und im Augenblick darauf entfernte es sich nach dem
Pol zu.

Fru öffnete den Verschluß und zog Ell in das Innere des Schiffes.
Er faßte den Zusammensinkenden in seine Arme, ein Blutstrom brach
aus Ells Munde. Vergeblich bemühten sich die Martier um den
Leblosen, während ihr Schiff in rasender Eile dem Amerikaner nach
dem Pol folgte.

\tb

Die Mittagssonne eines klaren, windstillen Dezembertages lag auf
den Bergen, deren helle Landhäuser über das Etschtal und die
beschneiten Höhen weit nach Süden hin schauten. Es war warm wie im
Frühling auf der Veranda, an deren Geländer La lehnte. Ihre Blicke
waren auf den Fußweg gerichtet, der von der Stadt nach der Villa
emporführte. Dort, wo der Pfad aus dem Tannenwald hervortrat, um in
mehrfachen Windungen den steilen Rasenabhang vor dem Haus zu
erklimmen, wurde jetzt Saltners Gestalt sichtbar. Er kam aus der
Stadt. Mit Vorliebe pflegte er den Weg, obwohl er eine Stunde
tüchtigen Steigens in Anspruch nahm, zu Fuß zurückzulegen, um, wie
er sagte, nicht aus der Übung zu kommen. Sonst vermittelte das
Luftschiff den Verkehr in wenigen Minuten. Als er La erkannte,
schwang er den Hut und sprang schneller den Pfad hinauf. Bald stand
er auf der Veranda.

„Sind Nachrichten da?“ rief La ihm entgegen.

„Vom Mars noch nicht, aber vom Südpol“, sagte er, sie mit einem Kuß
begrüßend.

„So ist die Einstellung noch immer nicht gelungen?“

„Nein, aber man hat die Annäherung eines Raumschiffes beobachtet,
das der ›Glo‹ zu sein scheint. Es vermeidet jedoch die Station und
scheint sich unter dieselbe herab bis in die Atmosphäre senken zu
wollen. Die amerikanischen Luftschiffe bewachen die gesamte
Umgebung des Pols.“

La atmete auf. „Das ist ein gutes Zeichen“, sagte sie. „Hoffentlich
begegnet man ihm nicht feindlich, ein einzelnes Raumschiff ist
nicht zu fürchten, es wird Nachrichten bringen wollen.“

„Man kann das nicht wissen. Es ist gar nicht zu sagen, was die
Martier möglicherweise sich ausgedacht haben und womit sie uns
überraschen. Du warst selbst sehr besorgt.“

„Ja, wenn Oß gesiegt haben sollte, wäre allerdings alles zu
befürchten. Die ›Erdbremse‹ ist nicht bloß Phantasie, ich weiß, daß
er solche Gedanken schon mit sich herumtrug, als er noch Assistent
des Vaters war. Gebe Gott, daß das Schiff eine gute Nachricht
bringt.“

„Wir wollen uns nicht vor der Zeit ängstigen“, sagte Saltner, indem
er den Arm um ihre Schulter legte, um sie von der Veranda ins Haus
zu führen.

In diesem Augenblick hallte vom Tal ein Kanonenschuß herauf. Gleich
darauf ein zweiter und dritter.

„Was ist das?“ fragte La erschrocken.

Beide kehrten um und blickten auf die Stadt hinab. Wieder ertönten
die Schüsse. Sie spähten mit den Ferngläsern hinunter.

Saltner ergriff Las Hand.

„Es muß eine gute Nachricht sein“, rief er. „Schau dort, an den
Türmen und auf den Schlössern werden die Fahnen aufgezogen. Sollte
etwa –“

„O Sal, wenn es der Friede wäre!“

Saltner eilte ans Telephon. Er sprach das Telegraphenamt an. Eine
Weile mußte er warten, weil die Beamten voll beschäftigt waren.
Dann kam die Antwort.

„Botschaft vom Mars. Der Friedensvertrag nach Vorschlag der
Erdstaaten vom Zentralrat genehmigt. Ell mit dem Abschluß des
Friedens auf der Erde beauftragt. Nähere Nachrichten stehen noch
aus.“

La fiel ihrem Mann um den Hals. Tränen der Freude drängten sich in
ihre Augen. Er schloß sie in seine Arme. Er wußte, was in ihr
vorging. Jetzt, erst jetzt fand sie die volle Ruhe, nun war ihr
Bund bestätigt vom Geschick der Planeten.

„Wollen wir hinab, um die neuen Nachrichten in Empfang zu nehmen?“
fragte er.

„Laß uns hierbleiben. Ich möchte jetzt nicht gerade unter die
Menschen. Bleibe bei mir in unserm Haus!“

„So soll Palaoro mit dem kleinen Schiff hinab, um uns sogleich die
Extrablätter mit neuen Nachrichten heraufzubringen. Du hast recht,
geliebte La!“

Noch ehe Palaoro zurückkehrte, erfuhr Saltner durch ein
telephonisches Gespräch mit einem Freund den Hauptinhalt der neuen
Depeschen. Diese waren aber so unklar und zum Teil widersprechend,
daß La und Saltner nicht wußten, was sie davon halten sollten. Es
hieß, die Gesandtschaft unter Ells Führung sei zum Abschluß des
Friedens eingetroffen und habe die Friedensbotschaft selbst auf die
Erde gebracht. Sie sei aber an der Landung verhindert worden, weil
eine Beschädigung des abarischen Feldes vorläge. Eine spätere
Depesche besagte, die Außenstation sei im Begriff,
zusammenzustürzen, oder sei schon eingestürzt. Die Deputation der
Marsstaaten sei dabei verunglückt. Die letzte Nachricht meldete,
die Bestätigung des Friedensvertrages mit den Marsstaaten sei
bereits an die Regierungen telegraphiert. Der Erbauer der Station,
Fru, sei zur Rettung der Außenstation vom Mars herbeigeeilt.

La und Saltner tauschten noch ihre Ansichten über die Bedeutung
dieser Nachrichten aus, als Palaoro mit dem Luftboot anlangte. Das
erste, was er überreichte, war eine lange Depesche an La.

Sie riß den Umschlag auf.

„Vom Vater“, rief sie jubelnd. „Er kommt zu uns!“ Sie durchflog das
Blatt. Ihre Züge wurden ernst.

„Was ist geschehen?“ fragte Saltner besorgt.

„Der Vater ist gesund und die Station ist gerettet –“

„Gott sei Dank!“

„In der letzten Stunde. Mit Mühe gelang es dem Vater, das Unheil
noch abzuwenden. Daß die Unseren zurechtkamen, verdanken sie der
Aufopferung Ells. Und er –“

Saltner beugte sich über das Blatt. La hob ihre tränenfeuchten
Augen zu ihm auf, er küßte ihre Stirn.

„Das Andenken dieses Edlen ist unvergeßlich“, sagte er. „Er war der
Führer auf dem Weg, den die Welt nun wandeln kann zu Freiheit und
Frieden.“

\end{document}
