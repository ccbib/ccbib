\usepackage[ngerman]{babel}
\usepackage[T1]{fontenc}
\hyphenation{wa-rum Fracht-raum}
\hyphenation{schien}
\hyphenation{Tief-ebe-ne Tief-ebe-ne gro-ßen}


%\setlength{\emergencystretch}{1ex}

%\renewcommand*{\tb}{\begin{center}* * *\end{center}}

\newcommand\bigpar\medskip

\begin{document}
\raggedbottom
\begin{center}
\textbf{\huge\textsf{Befreiungsschlag}}

\medskip
Merlin Thomas

\end{center}

\bigskip
\begin{flushleft}
Dieser Text wurde erstmals veröffentlicht in:
\begin{center}
Die Steampunk-Chroniken\\
Geschichten aus dem Æther
\end{center}

\bigskip

Der ganze Band steht unter einer
\href{http://creativecommons.org/licenses/by-nc-nd/2.0/de/}{Creative-Commons-Lizenz.} \\
(CC BY-NC-ND)

\bigskip

Spenden werden auf der
\href{http://steampunk-chroniken.de/download}{Downloadseite}
des Projekts gerne entgegen genommen.
\end{flushleft}

\newpage

Der Diensthabende schaute auf. Der hektische Rhythmus, in dem die
Magnetstiefel durch den Korridor klickten, schwächte seine
Aufmerksamkeit für den Versuch, die wöchentliche Lektüre des
Clausewitz nicht schleifen zu lassen. Er platzierte das
Lesebändchen zwischen den Seiten und spannte das Buch unter eine
der Halteleinen seines Schreibtisches.

Seine Erwartung erfüllte sich: der Signalton der Tür erklang. Er
zog an dem Hebel, der die Tür in die Wand schwingen ließ. Der
Läufer klickte herein und salutierte. »Herr Major, melde
gehorsamst: æthertelegraphische Depesche von der Oberfläche
erhalten, Herr Major!«

Köpcke streckte den Arm aus, nahm das Klemmbrett und überflog die
Depesche. Sie stammte vom Büro des Gouverneurs und veranlasste den
Offizier, beim Aufspringen zu vergessen, dass er an seinen Stuhl
geschnallt war. Er löste den Gurt und reichte die Nachricht zurück.
»Bringen Sie das zum Kommandanten. Zack, zack!«

Der Gefreite salutierte und klickte hinaus. Der Major warf einen
Blick auf die Abdeckplatte an der Wand, aber er widerstand der
Versuchung, sie erneut abzuschrauben. Köpcke zog den Hebel, der den
Gefechtsalarm auslöste. Der Widerstand erschien ihm geringer, jetzt
da das Stahlseil, das zum Transæthersender lief, ausgehängt war. In
der gesamten Festung erklang das Alarmsignal in dreifacher
Wiederholung, riss die Nachtschicht aus ihrem Dämmerzustand und die
Übrigen aus ihrem Schlaf.

Major Köpcke verriegelte das Dienstzimmer und folgte dem Korridor
nach rechts, wo ihm einige Meter weiter ein Bewaffneter das
Doppeltor zur Operationszentrale öffnete. Durch die rückwärtige
Kuppel warf er einen Blick auf die Oberfläche hinunter. Ohne eines
der Teleskope zu nutzen, konnte er nichts Auffälliges erkennen. Im
Vorbeigehen ließ er sich eine Kopie der Depesche geben, dann
schnallte er sich auf seinen Platz am Taktiktisch. An der Wand lief
eine Uhr seit zwei Minuten und 17 Sekunden. Daneben war eine Reihe
von 24 Lampen, von denen ein Drittel nur blass schimmerte.

Bei drei Minuten und neun Sekunden leuchteten alle in kräftigem
Grün: Orbitalfestung 9 war gefechtsbereit. Der Major notierte die
Zeit im Wachbuch. Nach kurzem Zögern fügte er \emph{unbefriedigend}
hinzu. Die Verfahrensanweisung des Raumkommandos sah eine Zeit von
unter drei Minuten vor. Er würde zusätzliche Alarmdrills auf den
Plan setzen.

Das Schott zum Korridor öffnete sich und Oberst von Theeste trat
ein. Er stampfte auf Köpcke zu. Das auf den Tisch geschleuderte
Klemmbrett prallte ab und segelte davon. »Was ist denn das für eine
Sauerei, Köpcke? Sozialistenschwänze? Direkt unter meinem Arsch?
Denen müssen wir wohl mal auf den Kopf scheißen, was?«

»Sehr wohl, Herr Oberst!«

Der Festungskommandant schnallte sich seinem Wachoffizier gegenüber
auf einen Stuhl und schloss die goldenen Knöpfe seiner dunkelblauen
Uniform. »Dann sagen Sie Hansen mal Bescheid, dass er da unten
durchwischt, wenn die Hanseln von der Bodentruppe das nicht selbst
in den Griff bekommen.«

Köpcke winkte eine Ordonnanz herbei. Auf ein gerauntes »Hansen«
hin, klickte diese eilig zur Sprechanlage.

»Herr Oberst, ich muss mir erlauben, auf einen kritischen Punkt
hinzuweisen. Wir haben derzeit keine Kapazitäten, um Hansens
Kompanie auf die Oberfläche zu verbringen.«

Theeste starrte den Major mit zusammengekniffenen Augen an. Köpcke
wartete nicht auf die Nachfrage seines Vorgesetzten. Er schlug im
Wachbuch nach. »Die \emph{Kronprinz Ludwig} ist zum Befehlsempfang
bei der Admiralität auf Kaiser-Wilhelm-II.-Mond. \emph{Köpenick}
und \emph{Brandenburg} patrouillieren beim Zaren. Unsere
vorgeschobenen Teleskope haben verstärkte Flottenbewegungen
gemeldet. Und die \emph{Helgoland} ist raus, einen Kutter
aufzubringen, der lediglich 180 Raummeilen von hier eine Ladung
Unrat verklappt hat.« Köpcke deutete mit dem Buch in Richtung des
raumwärts gelegenen Observatoriums.

Theeste warf einen Blick hinaus und löste seinen Gurt. »Unrat?
Unser letztes Kriegsschiff geht raus, um einen Kutter wegen
Verklappung von Unrat zu jagen?« Er wiegte sich in der Hüfte, wie
es sich viele Offiziere angewöhnt hatten, die in Schwerkraft dazu
neigten, auf und ab zu laufen. »Nun, da danken wir aber ganz artig
unserer geliebten Kaiserin für ihre expliziten Befehle zur
Reinhaltung des Weltenraums.«

Hauptmann Hansen betrat die Zentrale und nahm neben dem Taktiktisch
Haltung an.

»Hansen, gut.« Der Oberst setze sich. »Wissen?s, die Sozialisten
haben drunten in Karlstadt eine Räterepublik ausgerufen und sich
nahe der Amtskuppel des Gouverneurs verschanzt.«

»Unverschämtheit, Herr Oberst. Erbitte Erlaubnis, die
Ratsversammlung aufzulösen, Herr Oberst!«

»Die hätten Se, mein Guter, die hätten Se. Aber leider …« Mit einem
Winken übergab er Köpcke das Wort.

»Leider haben wir keine Transportmöglichkeit für deine Jungs,
Frieder. Alle Schiffe sind draußen.«

»Sapperlot! Wann werden sie denn zurück erwartet?«

»Durch den Gefechtsalarm wurde ihnen automatisch über Transæther
die Rückkehr signalisiert. Die \emph{Ludwig} müsste …«, Köpcke
blickte auf die Uhren und überschlug einige Werte, »… in etwa
anderthalb Stunden wieder hier sein. Es sei denn, einer der Herren
Admirale fühlt sich bemüßigt, ihren Rückkehrbefehl zu widerrufen
Dann müssen wir auf die \emph{H?land} warten. Allerdings haben wir
deren genaue Position nicht.«

Hansen blickte ebenfalls auf die Chronographen. »Hm, da kann ich
den Wachtmeister ja noch mal die Spielkarten ausgeben lassen,
was?«

»Wer weiß, Frieder. Vielleicht hat die \emph{Helgoland} ihre Jagd
schon beendet. Dann könnte sie jeden Moment hier sein.« Köpcke
wandte sich an Theeste. »Wir sollten den Aufständischen nicht mehr
Zeit geben sich einzurichten als notwendig, oder Herr Oberst?«

»Richtig Köpcke. Hansen, machen Sie die Jungs schon mal fertig.
Eine Stunde auf die Verladung zu warten, fördert sicherlich den
Ehrgeiz, den Einsatz schnell zu Ende zu bringen.«

»Zu Befehl, Herr Oberst. Nun ja, wenn es bis zum Einsatzbeginn vor
Ort tatsächlich drei Stunden dauert, wird das sicherlich einer der
persistierendsten Räte in der Geschichte des Kaiserreichs sein.«

Der Oberst schmunzelte, riss sich aber schnell wieder zusammen.
»Aber was für eine Schande, dass das unter meiner Nase geschieht.«
Er atmete aus und schüttelte den Kopf. »Hansen, dafür müssen Se
umso gründlicher aufräumen, wenn Se verstehen, was ich meine.«

Hauptmann Hansen strich seinen Schnurrbart glatt. »Natürlich, Herr
Oberst, mit Vergnügen.« Er salutierte. »Harren wir also der
Rückkehr der \emph{Helgoland}. Ich mache meine Männer
marschbereit.«

Die Offiziere am Tisch erwiderten den Gruß, Hansen drehte sich um
und klickte zum Schott.

»Frieder?«

Hansen blieb stehen und blickte zurück.

»Pass auf dich auf!«

»Ach Kurt, während die noch einen Antrag auf Abstimmung über
Gegenwehr einbringen, sind meine Bajonette schon durch die
versammelte Mannschaft durch. Die werden quieken wie die Säue am
Schlachttag!« Er tippte sich mit zwei Fingern an die Schläfe und
verließ die Zentrale.

»Sagen Se mal, Köpcke, was war das denn? Haben Se Fracksausen vor
den Revoluzzerfatzken?«

»Nein, Herr Oberst, selbstverständlich nicht. Ich … « Er zögerte,
warf einen Blick über die Schulter auf das bodenseitige Fenster und
las einige Instrumente zur Bahnlage der Orbitalfestung ab. »Ich
sinniere nur über das, was Sie vorhin gesagt haben, Herr Oberst.
Dass wir denen auf den Kopf scheißen sollten.«

»Ja, und?«

»Nun ja, eine Salve 16-Zoll-Granaten sollte einen anständigen
Haufen abgeben, Herr Oberst.«

»Herr Major, ist ihrer Aufmerksamkeit entgangen, dass wir keine
bodenseitigen Geschütze haben?«

»Selbstredend nicht, Herr Oberst. Doch meiner Einschätzung zu Folge
liegt das Ziel lediglich ein paar Grad außerhalb unseres
Bestreichungswinkels. Ich vermute, die beiden Schlepper könnten uns
mit Hilfe unserer Lagekorrekturmaschinen so weit herumdrücken, dass
wir eine Feuerleitlösung für das Ziel bekommen.«

Der Oberst stemmte sich auf dem Tisch auf, so weit sein Gurt das
zuließ und kniff die Augen zusammen. »Hm, Se meinen, wir sollten
uns umdrehen, um auf die zu schießen? Hm, hm.«

»Ganz genau.«

»Das ist so töricht, das kann gelingen, Köpcke. Wer kommt schon auf
die Idee, eine Festung einfach umzudrehen?«

Der Major stand auf. »Ich werde das mit dem Geschützoffizier und
dem Leitenden erörtern, Herr Oberst.«

»Tun Se das, Köpcke, tun Se das.« Er schnallte sich ab und klickte
Richtung Schott. »Ich vertrete mir so lange die Beine.«

\tb

Als Oberst von Theeste die Zentrale ein Holsteiner Schnitzel später
wieder betrat, war das eigentlich raumseitige Observationsfenster
zum Teil von der roten Oberfläche des Planeten erfüllt. Er trat
heran und schwenkte ein Okular herbei, das ihm genügend
Vergrößerung bot, um Karlstadt in Augenschein zu nehmen. Anhand der
Skala am Rand konnte er erkennen, dass ein Drittel der Stadt in die
Reichweite seiner Geschütze geraten war. Er erlaubte sich ein
zufriedenes Knurren.

»Herr Oberst, ich hatte Sie nicht so schnell zurück erwartet!«
Major Köpcke trat neben ihn. »Neuausrichtung war erfolgreich, Herr
Oberst. Wir haben die Aufständischen im Sucher.«

Theeste klappte das Okular weg. »Gute Arbeit, Köpcke. Gab es schon
einen Respons auf den Kapitulationsbefehl des Gouverneurs?«

Köpcke versicherte sich mit einem Blick zu der Soldatin am
Lichtblitztelegraphen und verneinte.

»Dann setzen wir jetzt den geehrten Ratsherren mal einen kleinen
Gruß in ihren Vorgarten!«

»Sehr wohl, Herr Oberst.« Er warf einen Blick auf seine Uhr.
»Allerdings weisen die Artilleristen darauf hin, dass wir keine
Erfahrung mit planetarem Beschuss haben und schlagen vor, auf das
Eintreffen einer der Ætherfregatten zu warten.« Köpcke wandte sich
ab, ging ein paar Schritte auf das andere Observatorium zu und
suchte die Dunkelheit ab.

»Papperlapapp. Wenn wir auf die warten würden, könnten die auch
unsere Infanterie verbringen und wir bräuchten uns keine Gedanken
über die Artillerie zu machen. Die sollen nicht denken, sondern
schießen. Den Planeten können sie ja wohl kaum verfehlen. Und
Munition haben wir genug!«

Langsam kehrte der Major zu seinem Vorgesetzten zurück, salutierte
und griff nach dem Sprachrohr, das ihn mit dem kommandierenden
Offizier der Batterie David verband. Bevor er den Befehl des Oberst
weitergeben konnte, kam eine Späherin von der anderen Seite der
Zentrale herbeigeklickt.

»Herr Oberst, Herr Major, ein kleines Schiff nähert sich schnell.«

Köpcke hängte das Sprachrohr hastig ein. »Eines von unseren,
Mutzke? War eines näher, als wir dachten?«

»Nein, Herr Major. Es ist ein kleineres Schiff. Dem Anschein nach
ein großer Kutter. Könnte der sein, der gestern verklappt hat, Herr
Major.«

»Der gleiche Kutter?« Köpcke runzelte die Stirn. »Was hat denn das
zu bedeuten?«

»Vielleicht haben sie einen Koben Schweinemist gefunden, den sie
jetzt nachliefern wollen? Wir haben andere Prioritäten, da beißt
die Maus keinen Faden ab.« Theeste wedelte in die Richtung, in der
er das anfliegende Schiff vermutete. »Pusten Se das Ding um ihrer
Majestät Willen aus dem Æther und dann kümmern Se sich um unser
eigentliches Problem!«

Köpcke quittierte den Befehl mit einem Nicken und verglich
gemeinsam mit dem Geschützoffizier den Anflugvektor des Schiffes
mit allerhand Anzeigen und Skalen. Eine Diskussion brandete
zwischen den beiden auf, Leutnant Brunner deutete auf einen
Winkelmesser, Major Köpcke verschob ein Lineal auf einer Karte.

Der Oberst kurbelte das Stellrad einer Gaslampe von einem Anschlag
zum anderen und zurück. Zum dritten Mal erreichte das Licht seine
maximale Helligkeit. Theeste ließ die Kurbel los und klickte zu
seinen Offizieren. »Hätten die Herren wohl die Güte, meinen
Befehlen Folge zu leisten? Oder bedarf es eines ermunternden
Trittes in ihre Ärsche?«

Die beiden fuhren hoch, starrten den Oberst an, warfen sich einen
schnellen Blick zu. Köpcke schluckte. »Herr Oberst, der Kutter
nähert sich im toten Winkel unserer Geschütze!«

»Was?« Mit hinter dem Rücken verschränkten Armen klickte der Oberst
zum Observatorium und warf einen Blick auf den Kutter. Den
Angreifer, wie jetzt zu vermuten stand. »Woher sollten die gewusst
haben, dass wir da einen toten Winkel haben? Das wussten wir doch
selbst nicht, bis …«, er drehte sich um und deutete auf den Major,
»… bis Sie vorhin Ihre tolldreiste Idee hatten, Köpcke.« Seine Hand
schnellte zum Pistolenholster.

Köpckes Kugel riss ihm das Ohr weg und schlug hinter ihm in eine
Leitung. Sofort trat Dampf aus, der das Blut des Obersts durch die
Zentrale blies.

Dennoch zog Theeste seine Waffe. Köpckes zweite Kugel drang durch
das Auge ins Gehirn ein und enthob Jens Friedrich Freiherr von
Theeste seines Kommandos über Orbitalfestung 9.

Raumwebel Karoline Mutzke stieß sich von ihrem Pult ab und glitt an
dem Gefallenen vorbei zu dem Sperrhahn der lecken Leitung.

»Verflixt noch eins«, zischte Köpcke. Er ließ den Handgriff los,
der ihn davor bewahrt hatte, vom Rückstoß quer durch die Zentrale
katapultiert zu werden, wirbelte herum und half Brunner, die
Anwesenden in Schach zu halten. »Ich hatte nicht gedacht, dass der
Alte uns so schnell durchschaut.«

»Mach dir nichts daraus, Kurt. Wir mussten ihn sowieso los
werden.«

Köpcke betrachtete Brunner einen Moment aus den Augenwinkeln ohne
etwas zu erwidern. Dann wedelte er mit seiner Pistole in Richtung
der dem Ausgang gegenüberliegenden Seite. »Meine Damen und Herren,
bitte begeben Sie sich dort hinüber! Jan, nimm ihnen die Waffen
ab.« Er blickte zu der Frau, die am defekten Druckrohr arbeitete.
»Mutzke, wenn Sie fer …«

Das Tor öffnete sich. Köpcke fuhr herum, um den Eintretenden ins
Visier zu nehmen. Als er den Soldaten erkannte, der hereinklickte,
wandte er sich wieder den Gefangenen zu. Der bullige Ankömmling
blickte sich um. Den Anblick von Theestens Leiche quittierte er mit
einem Lächeln. Er wischte mit dem Ärmel seiner dunkelblauen Uniform
Blut von seinem Bajonett und verstaute die Klinge in der
Gürtelscheide.

»Hansens Kompanie ist in Ladebereich zwo festgesetzt, Herr Major!«

»Danke, Meier. Helfen Sie mir mit den Arrestanten. Jan,
signalisiere dem Kutter Ladebereich eins.«

Meier salutierte und ließ sein Gewehr von der Schulter gleiten.
Brunner aktivierte rechts der Kuppel einen Blitzgeber, richtete ihn
auf das einlaufende Schiff, das inzwischen mit bloßem Auge die
Größe einer Reichsmark erreicht hatte, und gab das verabredete
Signal. Das Antwortsignal blitzte auf. Gemeinsam mit seinen
Kameraden beobachtete Brunner den Anflug des Kutters.

Einige hundert Meter vor der Orbitalfestung raffte das Schiff seine
Æthersegel so weit, dass es seine relative Position hielt. Die
Einzelheiten des Rumpfes waren deutlich auszumachen. Es bereitete
Köpcke keine Schwierigkeiten, zu erkennen, dass sich eine
Geschützluke am Vorschiff öffnete und ein langläufiges Jagdgeschütz
ausgefahren wurde.

Auch Brunner bemerkte das merkwürdige Verhalten. Er schlug Köpcke
gegen die Schulter. »Was machen die denn da? Das gehört nicht zur
Vereinbarung, oder?«

Der schlanke Körper eines Raumtorpedos schoss aus der
Geschützöffnung, schraubte sich durch die kurze Distanz und
verschwand aus dem Sichtfeld der fassungslosen Beobachter.

»Einschlag!« Köpcke griff nach einer Konsole zu seiner Linken,
Brunner klammerte sich an eines der Rohre seitlich der Kuppel,
andere hielten sich an den Magnetstiefeln ihrer Kameraden fest, um
möglichst nahe am Boden zu bleiben.

Der folgende Schlag wirbelte alle durcheinander. Stiefel lösten
sich, Gegenstände flogen in alle Richtungen, Soldaten prallten
schreiend und keuchend von Wänden und Mobiliar ab, Bluttropfen
waberten durch die Zentrale. Köpcke wurde von der Konsole
fortgerissen und mit dem Kopf voran gegen die Decke geschleudert.
Benommen hangelte er sich an der Verschalung entlang zu einer Wand
vor und dort an einem Rohr hinunter auf den Boden. Mit einem satten
\emph{Klack} hafteten seine Magnetstiefel am Deck. Der Versuch, die
Benommenheit durch ein Kopfschütteln zu vertreiben, resultierte in
einem stechenden Schmerz, der den Offizier in die Knie zwang.
Brunner eilte an seine Seite, richtete ihn auf und stützte ihn.

Köpcke drückte die Schulter seines Kameraden, dann brachte er die
Kraft für Befehle auf: »Schadensbericht! Feindmeldung! Na los,
beeilen Sie sich, Herrschaften, wir sind im Gefecht!«

Das knappe Dutzend Mannschaften und Unteroffiziere im
Kommandozentrum überwand sein Zögern und begab sich an seine
Stationen. Meier sah sich verlegen um und suchte sich einen Platz
nahe dem Schott, wo er niemandem im Weg war.

»Kutter nimmt Fahrt auf und nähert sich den Dockanlagen, Herr
Major«, meldete ein Nahbereichsausguck.

»Und wie steht es mit unseren Schäden? Wo sind wir getroffen?«

»Der Treffer scheint genau in diesem Bereich erfolgt zu sein, Herr
Major.« Die Melderin an der Sprechanlage zuckte mit den Schultern.
»Von allen anderen Abteilungen habe ich Bereitschaftsmeldungen,
doch zum Dockbereich komme ich nicht durch.«

Der Major löste sich von Brunner und drückte ihn in Richtung seines
Postens. »Kümmere dich um die Geschütze, Jan, ich komme klar. Wir
müssen den Kutter aufhalten.«

»Aber unser Plan …«

»Ich mache mir mehr Sorgen um deren Plan! Los!«

»Nicht so schnell, die Herren Meuterer!« Pistolenläufe bohrten sich
in ihre Nacken. »Hände hoch!«

Die Männer gehorchten.

Der Major drehte seinen Kopf, um einen Blick auf die Frau hinter
ihnen zu werfen, aber sein Gesicht wurde durch deutlichen Druck mit
der Pistole wieder zurück gezwungen. »Keine Bewegung, Herr Major!
Herr Ex-Major!«

»Hören Sie Mutzke, Sie wissen doch gar nicht, …«

Die Raumwebel beendete den Einwand, indem sie die Pistole, die sie
dem Oberst abgenommen hatte, auf Köpckes Hinterkopf schlug. Wieder
sackte er zusammen und ließ sich frei schweben, soweit die
magnetische Verbindung zum Boden dies erlaubte.

Soldat Meiers hob sein Gewehr. Er brachte es in Anschlag, aber ehe
er Ziel nehmen konnte, schlug doppelter Tod aus Mutzkes Pistolen in
seinen Oberkörper.

Brunner wollte herumfahren, doch das Geräusch des Hahns, der
gespannte wurde, ließ ihn innehalten.

»Recht so!« Karoline Mutzke stabilisierte ihren Stand und nickte
der Soldatin am Bordvermittler zu: »Krieger, rufen Sie Oberleutnant
Martens hoch!«

Kriegers Finger zuckten zum Sprachrohr. Dann hielt sie inne und
blickte zwischen Mutzke und Brunner hin und her, der leicht den
Kopf schüttelte.

»Was ist denn, Krieger? Hopp, hopp!« Sie kniff die Augen zusammen.
»Oder gehören Sie zu diesen Saboteuren?«

»Nein, Frau Raumwebel, aber in einer Gefechtssituation sollte die
Kommandokette nicht derart durchbrochen werden! Wo doch schon der
Oberst tot ist, sollte da nicht der Major das Gefecht zu Ende
befehligen?« Sie deutete auf den driftenden Offizier, der jetzt
ebenfalls seinen Blick auf sie richtete und den Körper in Spannung
brachte.

»Sie hat Recht, Mutzke. Behalten Sie die Waffe, stellen sie mich
unter Arrest, aber lassen sie mich dieses Gefecht führen!«

»Das Gefecht, das Sie und ihre Freunde da draußen uns beschert
haben?«

»Es war nicht vorgesehen, dass es zu einem Gefecht kommt.« Er
deutete auf von Theestens Leiche, die vor der Observatoriumskuppel
trieb. »Oder dazu. Es tut mir leid deswegen, aber unsere Sache …«

Die Öffnung des Hauptschotts unterbrach die Diskussion. Ein
Sergeant trat ein und blieb steif stehen, als er zunächst von dem
aus Meiers Körper schwelenden Blut und dann von Raumwebel Mutzkes
vorgehaltener Waffe in Empfang genommen wurde. Er wischte sich das
Gesicht frei und kam etwas näher.

»Frau Raumwebel, was …« Verwirrt ließ er seinen Blick durch die
Zentrale gleiten. Major Köpcke und Leutnant Brunner von Mutzke mit
der Waffe bedroht, alle anderen reglos an ihren Stationen. Dazu die
zwei Leichen. »Was haben Sie getan? Sind sie von allen guten
Geistern verlassen?«

»Ich? Ich habe die Zentrale unter Kontrolle gebracht. Der Major ist
ein Reichsverräter. Er hat den Oberst getötet!«

»Der Major? Das ergibt doch keinen Sinn!«

Mutzke schwenkte die Waffe in ihrer Rechten zwischen Brunner und
dem Sergeant hin und her. »Brunner und Meier sind seine Komplizen.
Was ist mit Ihnen Topolski, auf wessen Seite stehen Sie?«

»Drei Mann, die gemeinsam einen Verrat angezettelt haben?«

»Ja, drei. Drei, die mir bekannt sind! Was ist mit Ihnen?«

»Zwei haben Sie doch schon in Gewahrsam!«

Mutzke schaute Topolski verständnislos an. Brunner warf Köpcke
einen beiläufigen Blick zu, den dieser mit einem fast unmerklichen
Nicken quittierte.

»Eins will ich Ihnen sagen, Frau Raumwebel …«

»Was denn?«

»Jetzt!«

Topolski beugte seine Knie und zog sich dem Boden entgegen; Köpcke
schmiss sich gegen Mutzkes rechten Unterschenkel, dabei gelang es
ihm, die Verankerung eines ihrer Magnetstiefel vom Boden zu lösen;
Brunner tauchte nach links aus der Schussbahn der Pistole und
schlug seinen Ellenbogen gegen Karoline Mutzkes Hüfte.

Instinktiv feuerte sie beide Waffen, traf aber keines ihrer
anvisierten Ziele. Die Kugel aus ihrer eigenen Waffe schoss am
Kragenspiegel des Leutnants vorbei, prallte an einer Wand ab und
schlug in eine Konsole ein. Das Projektil aus der Waffe des Obersts
verfehlte Topolski, wurde von einer Metallsäule abgelenkt und
durchdrang Kriegers Oberschenkel.

Der Schrei der Soldatin lenkte Raumwebel Mutzke stärker ab als
Sergeant Topolski, so dass dieser seine Waffe ziehen und der
Unteroffizierin eine Kugel in den Oberkörper schießen konnte. Mit
nur einem Stiefel am Boden arretiert, vollführte sie einen
aberwitzigen Todestanz.

Major Köpcke richtete sich auf, so schnell es sein mehrfach
malträtierter Schädel zuließ und entrang die beiden Pistolen dem
Griff der Sterbenden. Eine steckte er sich an seinem Rücken in das
Koppel, die andere ließ er zu Topolski gleiten, der sie elegant in
Empfang nahm.

»Jan, wie geht es dir?«

»Alles in Ordnung bei mir, Kurt. Und bei dir?«

»Ebenso.« Köpcke strich sich über den Hinterkopf. »Fast.«

»Gut.« Brunner schwebte zu Kriegers Station und versorgte die Wunde
der Verletzten.

Köpcke blickte sich in der Zentrale um. Alle blieben ruhig an ihren
Stationen. Im Moment hatten sie nicht mit weiteren Helden wie
Karoline Mutzke zu rechnen. Helden, die für ihren Einsatz nichts
weiter als einen vorzeitigen Tod erwarten konnten. Er warf einen
Blick auf das Wandchronometer. Der Aufstand dauerte noch keine
halbe Stunde und schon gab es drei Tote alleine hier in der
Zentrale.

»Entschuldigung, Herr Major!« Topolski räusperte sich.

Köpcke schaute ihn an. »Topolski. Gut, dass Sie gekommen sind.
Danke.«

»Keine Ursache, Herr Major. Sie und der Herr Leutnant haben zum
Glück schnell genug begriffen, was ich vorhatte.«

Brunner grinste verschmitzt hinter Kriegers Bein hervor. »Selbst
ich habe meine Momente, Topolski.«

»Jawohl, Herr Leutnant. Aber weswegen ich gekommen bin, Herr Major:
der Torpedo ist mitten rein in Ladebucht eins. Hat das Außenschott
weggerissen und Hansens komplette Kompanie zerstückelt. Wer die
Explosion überlebt hat, ist nach draußen geblasen worden.« Er
nickte mit dem Kopf in eine unbestimmte Richtung. Der Weltraum war
überall um sie herum.

Köpcke klickte zum Taktiktisch und suchte Halt. Drei? Hatte er eben
tatsächlich noch drei Tote beklagt? Jetzt hatte er dreihundert.
Dreihundert sinnlose Tode. Dreihundert Menschen für die er hatte
kämpfen wollen.

Sein Blick glitt zur Transpakuppel. »Wo ist der Kutter jetzt? Ich
sehe ihn nicht mehr.«

»Der Kutter hat an Ladebucht zwo festgemacht, Herr Major«, meldete
der Ausguck, der an einem neben der Kuppel angebrachten Omniskop
Dienst tat.

»Verflixt. Jan, Topolski, ihr haltet hier die Stellung. Ich gehe
runter und verhindere, dass die Situation noch weiter aus dem Ruder
läuft.«

»Und wenn du …« Brunner versagten die Worte. Er ließ Krieger halb
verbunden zurück und ging zu seinem Freund.

»Wenn ich nicht wiederkomme?«

Brunner nickte.

»Dann helfe euch Gott. Ich kann es dann nicht mehr.«

\tb

Köpcke betrachtete den Wachposten, der ihm vor weniger als einer
Stunde das Schott zur Zentrale geöffnet hatte. In seinen
Magnetstiefeln steckend hing er reglos mit zertrümmerter Nase in
der Schwerelosigkeit, die Arme trieben neben ihm, der Kopf baumelte
auf und ab. Der tiefe Schnitt durch die Kehle öffnete und schloss
sich dabei wie die groteske Parodie eines Mundes.

Der Major entriegelte seine Stiefel, stieß sich zu einer
Haltestange hoch und hangelte sich den Korridor entlang, bis er um
eine Kurve herum war. Er drückte die Stirn gegen die Wand, ließ die
Beine treiben und atmete einige Male tief durch. Keine Zeit,
ermahnte er sich, keine Zeit darüber nachzudenken. Runter zu den
Docks und dem nächsten Unheil die Stirn bieten.

Er zog sich an der Decke entlang, bis er zu einem Schacht kam, in
den er hinein schwebte. Er klinkte sich in das abwärts laufende
Band ein. Drei Decks tiefer löste er den Karabiner und ließ die
Trägheit den Rest erledigen, so dass er auf dem untersten Deck aus
dem Schacht hinaus trieb. Er verankerte sich am Boden. Vor ihm
teilte sich der Gang. Die linke Abzweigung war von einem Schott
verschlossen, über dem ein rotes Licht Atmosphärenverlust
signalisierte. Ladebucht eins, in der 300 Leichen tanzten.
Zumindest das von ihnen, was nicht hinaus geblasen worden war.
Vielleicht nicht allzu viel, wenn die Reparaturmannschaften später
Glück hatten. Vielleicht genug für eine anständige Beisetzung, wenn
die Angehörigen Glück hatten.

Köpckes Hand glitt zur Tür einer Kammer, in der Vakuumanzüge
verstaut waren. Aber dafür war keine Zeit. Er ließ von der Tür ab
und drehte sich um. Die eigentlichen Schwierigkeiten warteten jetzt
in Ladebucht zwei auf ihn. Und er musste sich beeilen, bevor …
Köpcke hob die Hände und trat einen Schritt zurück, um die
Situation zu erfassen.

Die Mündungen zweier kurzläufiger Schnellfeuergewehre, angelegt an
die Schultern rot uniformierter Riesen, schienen ihn anzustarren.
Er starrte zurück. Die Schwierigkeiten waren schon zu ihm
gekommen.

»Den nicht. Er ist unser … Verbündeter.« Eine Hand in einem weißen
Seidenstahlhandschuh bahnte sich ihren Weg zwischen den Hünen, ein
zierlicher Körper in einer stark verzierten Uniform folgte.

»Major, ich bin Ihnen zu tiefstem Dank verpflichtet, dass Sie die
Mühe auf sich nehmen, uns hier unten persönlich abzuholen.«
Gedehnte Vokale und harte Konsonanten tröpfelten aus dem reizvollen
Mund.

Köpcke ergriff die dargebotene Hand und beugte sich zur Andeutung
eines Kusses über sie. »Kommandantin Kanurova. Ich bin erfreut, Sie
zu sehen. Sie sind sehr … pünktlich.« Der Major schluckte.

»Danke. Sie ebenfalls, Herr Major. Ihr Plan war wie … wie sagt man?
Ein Uhrwerk? Ich muss Ihnen meine Anerkennung aussprechen.«

»Kommandantin, es scheint mir, dass es Ihnen schwer gefallen ist,
gewisse Aspekte des Planes so umzusetzen, wie wir es vereinbart
hatten.« Köpcke deutete auf das Schott hinter sich.

Die russische Kommandantin schwebte an ihm vorbei, um einen Blick
durch die Sichtluke zu werfen. »Nun ja, meine Erfahrung hat mir
gezeigt, dass es der sicherere Weg ist, jede Unwägbarkeit
auszuschließen. Man weiß nie, wer einem sonst in den Rücken fällt.«
Ruckartig drehte sie sich um. »Einerlei. Wenn Sie die Güte hätten,
uns zur Zentrale zu geleiten?«

»Einerlei? Sie haben 300 meiner Leute gemetzelt! Ich bin sicher,
dass ich Hauptmann Hansen hätte überzeu…«

Sie schnellte auf ihn zu und brachte ihr Gesicht dicht an seines.
Den Æetherschiffern gelangen solche Manöver deutlich eleganter als
den Festungskrabblern. Die Trägheit trieb eine ihrer
Kastaniensträhnen gegen seine Wange. »Sie hätten. Ich habe. Das ist
der Unterschied, Herr Major. Haben ist immer besser. Merken Sie
sich das!« Sie glitt eine Armlänge zurück. »Und nun, zur Zentrale,
wenn Sie gestatten.« Die Kommandantin deutete auf den
Schleppschacht. Hinter Köpcke räusperte sich einer der Riesen. Der
Major drehte sich um. Sein Blick fiel auf die kleine Gruppe
Soldaten, die im Gang wartete. Einer von ihnen trug die Abzeichen
eines Leutnants und nickte ihm ruhig, wie zur Begrüßung, zu.

Köpcke entspannte sich. Ein Lächeln sprang kurz in sein Gesicht. Er
wandte sich dem Schacht zu und klinkte sich in das Aufwärtsband
ein.

\tb

Köpcke ging zuerst durch das Schott, um zu verhindern, dass Brunner
aus Überraschung eine Dummheit beging. Die Situation war unter
Kontrolle, die drei Leichen aus der Zentrale geschafft worden.
Brunner hob fragend die Augenbrauen. Köpcke warf einen Blick über
die Schulter, um zu sehen, wie die Russen durch den Eingang
flossen.

Kanurova glitt in die Mitte der Zentrale, gefolgt von ihrem
Adjutanten, dem kleinen Leutnant mit dem gezwirbeltem Schnurrbart.
Einer ihrer Leute fürs Grobe postierte sich neben dem Schott, der
zweite etwa hundert Grad weiter raumwärts, ein dritter auf der dem
Planeten zugewandten Seite. Kreuzfeuerpositionen. Alle Anwesenden
waren schlagartig sehr eifrig mit ihren Posten beschäftigt.

Köpcke wandte sich seinem Mitverschwörer zu und zuckte mit den
Schultern. Sergeant Topolski trat zu den beiden.

Die Kommandantin winkte die Männer heran. »Gönnen Sie mir bitte
einen Moment Ihrer Aufmerksamkeit. Es gibt einige Sachverhalte, die
ich mich genötigt sehe, Ihnen darzulegen.«

»Und die wären?« Seine Kameraden postierten sich bei dieser Frage
wachsam hinter dem Major.

»Zunächst einmal, dass unsere gemeinsame Unternehmung ab diesem
Moment in sinnvollere Bahnen gelenkt wird. Anstatt Sie bei ihrem
törichten Aufstand zu unterstützen, werden wir selbst diese Festung
sowie Karlstadt übernehmen! Mein Zar hat sicherlich bessere
Verwendung für Deutsch-Ost-Marsien als es Ihre Räterepublik je
haben könnte.« Selbst das hämische Lächeln konnte sie nicht ihrer
Attraktivität berauben.

Brunner brauste auf. »Was erlauben Sie sich, Sie …« Das Geräusch
dreier Waffen, die feuerbereit in Anschlag gebracht wurden, ließ
ihn verstummen.

»Ruhig Jan!« Köpcke legte ihm die Hand auf die Brust.

Brunner hob abwinkend beide Hände. »Verzeihung.«

Köpcke wandte sich an die Russin. »Kommandantin, wir haben eine
Übereinkunft!«

»Ich habe die Übereinkunft geändert!« Auf ihr Zeichen hin schwebte
ihr Adjutant zur bodenseitigen Sichtkuppel. »Beten Sie, dass ich
sie nicht noch weiter ändere.« Sie ließ ihren Blick über die
Anwesenden schweifen. »Sie alle dürfen sich jetzt als
Kriegsgefangene des russischen Zaren begreifen. Wenn Sie der
Meinung sind, dies nicht akzeptieren zu können, werden Sie
Kriegstote sein. Einerlei.«

Kanurova schaute zu ihrem Adjutanten hinüber, der die Skalen an der
Kuppel abgelesen hatte. »Eingestellt und korrekt, Frau
Kommandantin.«

Sie quittierte die Meldung mit einem Nicken, betrachtete die
Kragenspiegel der Deutschen und fixierte Brunner. »So denn, Herr
Geschützoffizier. Wenn Sie die Güte hätten, das Feuer zu
eröffnen?«

»Was? Nein! Nein, niemals. Das war doch nur eine Finte für den
Oberst.«

Eine raffinierte Körperwelle brachte Kanurova dichter an Brunner
heran. Sie neigte den Kopf leicht zur Seite und lächelte die
Temperatur um zehn Grad herunter. »Lassen Sie es mich Ihnen so
verdeutlichen, Herr Leutnant: Sie tun Ihre Pflicht oder ich
erschieße Sie und Ihre Geschützmannschaften, hole meine an Bord und
lasse die ihre Pflicht erfüllen.« Sie zog den Kopf wieder zurück.
»Einerlei.«

Brunner blickte sich hilfesuchend um. Topolski wich seinem Blick
aus. »Kurt?«

Köpcke wiegte einen Augenblick die Hüften, dann schlug er dem
Artillerieoffizier auf die Schulter. »Tu es, Jan. Solange wir
leben, können wir kämpfen.«

Brunner zuckte von seinem Vorgesetzten zurück. »Was? Aber unsere
Kameraden dort unten können nicht mehr kämpfen.«

»Wir kämpfen für sie. Befiehl die Salve, Jan.«

Der Leutnant schloss einen Augenblick die Augen, atmete tief durch.
»Jawohl, Herr Major!« Er klickte hinüber zu seinem Posten und nahm
das Sprachrohr: »Batterie David für Zentrale!«

»David hört«, klang es aus dem Schalltrichter an der Wand. In der
Stille, die sich unter den Menschen in dem großen Raum ausgebreitet
hatte, war jedes Pfeifen, Quietschen und Rasseln, das sich in die
akustische Übertragung eingebettet hatte, überdeutlich zu hören.

»David, Feuerbereitschaft auf anvisiertes Ziel jetzt.«

»Feuerbereitschaft auf anvisiertes Ziel. Jawohl.« Ein Moment
Stille. »Feuerbereit auf anvisiertes Ziel jetzt.«

Brunner schaute Köpcke in die Augen. Der nickte langsam und kaum
merklich. Brunner führte das Sprachrohr zum Mund. »David, Feuer,
Feuer, Feuer!«

Orbitalfestung 9 schüttelte sich, als sechs Geschütze in schneller
Folge ihre Ladungen auf den Ort hinausspien, den zu beschützen sie
den Auftrag hatten. Kanurova schwebte zur Kuppel. Ihre Augen
schweiften hin und her, als sie versuchte, die Geschosse
auszumachen. Schließlich gab sie es auf, nutzte den auf das Ziel
ausgerichteten Vergrößerer und wartete auf den Einschlag.

Köpcke stellte dem russischen Leutnant eine stumme Frage und bekam
ein Kopfnicken zur Antwort. Er drehte sich, um die Konsole im Auge
zu behalten, von der aus die Dockanlagen kontrolliert wurden. Kurz
darauf änderten Leuchtsignale ihren Rhythmus und der dort
diensttuende Posten zog einen Papierstreifen aus einem
automatischen Schreiber. Er las ihn, und drehte sich erregt um, um
Meldung zu machen. Köpcke unterband dies mit einer energischen
Geste, wandte sich wieder von der Konsole ab. Einer der russischen
Wächter funkelte ihn misstrauisch an, unternahm aber nichts.

Einige Minuten später fand der Tod dreimal sein vorbestimmtes Ziel,
ein Geschoss traf ein Gebäude in der Nähe, eines eine Agrarkuppel
und eines ging außerhalb der Stadt nieder.

Freudestrahlend kehrte die Kommandantin zu ihren Gefangenen zurück.
»Meine Herren, damit dürfen Sie Ihren Aufstand als niedergeschlagen
betrachten.«

»Oder«, Köpcke trat vor, »Sie, Kommandantin, betrachten ihre
Usurpation als gescheitert.«

Sie starrte ihn entgeistert an, dann warf sie lachend den Kopf in
den Nacken. »Welchen Anlass hätte ich dazu?«

Der Major deutete auf das raumseitige Observatorium, vor dem
erschreckend nah der russische Kutter hing, die bedrohlichen
Mündungen der beiden ausgerannten Jagdgeschütze auf das
Kontrollzentrum der Basis gerichtet.

Ihr Lachen verklang so schnell, wie es gekommen war. »Was … wieso?«
Sie zögerte. »Das ist mein Schiff! Ich habe keinen Befehl
gegeben!«

»Nicht nur Sie sind in der Lage, unsere Übereinkunft zu ändern. Ich
habe sie ebenfalls geändert, Kommandantin. Hoffen Sie, dass ich sie
nicht weiter ändere!«

»So?« Sie hob in wiedergefundener Selbstsicherheit die Augenbrauen.
»Und welche Rolle sieht Ihre Übereinkunft für mich vor?«

»Ihre Rolle – Ihre und die Ihrer Kosaken«, er deutete einmal durchs
Rund, »ist es, zu kapitulieren!«

»Andernfalls?«

»Anderfalls wird ihr Kutter, der, wie Ihnen aufgefallen sein
dürfte, unter Kontrolle revolutionärer Kräfte steht, uns aus dem
Æether blasen!«

»Dann ist Ihre Revolution tot!«

»Dann bin ich tot. Dann sind Sie tot. Und dann sind einige hundert
Unbeteiligte tot. Die Revolution lebt. Die Revolution ist dort
unten.« Er deutete Richtung Oberfläche.

»Sollte Ihnen entgangen sein, dass ich Ihre Revolution dort unten
in den Marsstaub gebombt habe, Herr Major?«

Köpcke lächelte. »Nein, Ihnen ist entgangen, zu welchem Zeitpunkt
sich unsere Übereinkunft geändert hat! Die Geschütze waren nicht
auf unseren Rat gerichtet, sondern auf die Kuppel, in der Ihre
Raumlandekompanie auf ihren Einsatz gewartet hat, um unsere Kolonie
zu überrennen!«

Ihre Augen verengten sich für einen Moment, dann schüttelte sie ein
Lachen, das sie im freien Fall gegen den Taktiktisch trieb. »Sie
versuchen, mich hinter das Licht zu führen! Woher sollten Sie
gewusst haben, wo meine Kompanie einquartiert war?«

»Aus der gleichen Quelle, aus der ich wusste, dass überhaupt eine
da ist.«

»Das haben Sie erraten. Vermutet. Weil Ihnen bewusst ist, dass die
Übernahme Ihrer Station ohne Unterstützung am Boden strategisch
keinen Sinn ergeben würde.«

Der Major nickte. »Einerlei, Kommandantin.«

Sie starrte ihm in das gelassen drein blickende Gesicht. Ihre Linke
nestelte an einer Tasche ihres weißen Gürtels. Ein schneller Blick
über die Schulter auf das rote Segment hinter der Sichtscheibe. Ein
Kopfschütteln. »Nein. Nein, nein, nein. Markov hat doch die
Zielkoordinaten überprüft und er wusste, wo …«

Köpcke meinte, die Zahnräder in ihrem Kopf ineinandergreifen zu
hören. Sie fuhr herum zu ihrem Adjutanten. »Markov, Sie!«

Leutnant Markov trat zu Köpcke und Brunner. »Jawohl, Kommandantin.
Ich. Ich bin Mitglied der revolutionären Kräfte freier Mars! Genau
wie Subkommandant Tokoiev dort draußen.« Er deutete auf den
feuerbereiten Kutter hinter sich.

Sie schwieg, schluckte, suchte nach einer Lösung. Sie schwebte zu
dem Soldaten hinüber, der neben der bodenseitigen Kuppel
positioniert war und nahm ihm die Waffe ab.

»Los, los, an die Geschützkonsole. Schießen Sie die Chekov ab«,
übersetzte Markov die geraunten Befehle der Kommandatin, auf die
hin sich der bullige Soldat abstieß, um punktgenau vor Brunners
üblichem Platz zu landen. Er las Skalen ab, verschob Regler.

Ein rhythmisches Blitzen, das durch die Observationskuppel drang,
erregte die Aufmerksamkeit der Anwesenden.

»Das war die Aufforderung zur Bestätigung, dass die Festung unter
unserer Kontrolle steht«, erläuterte Köpcke. »Drei Minuten, dann
wird Ihr ehemaliger Stellvertreter das Feuer auf uns eröffnen.«

Kanurova wischte seine implizite Aufforderung mit einer
Handbewegung beiseite. »Soldat?«

Der Mann an der Geschützkonsole blickte auf und erwiderte etwas,
das nicht sehr zufriedenstellend klang.

»Ausgeschlossen, Kommandantin. Sie ist zu nah, im toten Winkel«,
ließ Markov seine Vertrauten wissen.

»Indiskutabel.« Kanurova wirbelte Pirouetten um alle Achsen.
»Absolut indiskutabel.« Sie ließ den Impuls vergehen und
orientierte ihren Körper in Richtung Köpckes. »Wie sind Ihre
Bedingungen, Herr Major?« Sie glitt auf die andere Seite des
Tatkiktisches, so dass dieser zwischen ihr und ihren Gegner war.

»Sie kapitulieren und werden inhaftiert.«

»Ganz simpel also.« Sie schlang ihr Bein um einen Holm des
Tisches.

»Ganz simpel, ja, Kommandantin.« Köpcke nickte. »Akzeptieren Sie?«

Sie wiegte den Kopf hin und her, präsentierte ruhige
Nachdenklichkeit während die Muskeln in ihrem Bein sich spannten.
»Njet!«

Schlagartig zog sie sich zum Deck hinunter und drehte sich
gleichzeitig um ihre Längsachse, stemmte sich mit dem Rücken gegen
den Tisch und brachte das Gewehr in Anschlag.

Eine Salve gegen eine Rohrleitung neben der Kuppel. Dampf trat
aus.

Eine, zwei, drei Salven gegen die transparente Sichtkuppel und der
Weltraum begann, Luft und Wärme unbarmherzig aus der Zentrale zu
saugen.

Die drei Männer auf der anderen Seite des Tisches schrien und
hielten sich fest, um die Magnetverriegelung ihrer Stiefel zu
unterstützen. Die Mannschaften an den Stationen rings herum
keuchten, als sie in die Gurte oder gegen die Lehnen ihrer Stühle
gedrückt wurden. Der Russe, der freischwebend an der
Geschützkonsole gearbeitet hatte, fand nicht schnell genug Halt,
wurde fortgerissen und knallte gegen die Observationsscheibe. Der
Aufprall zerbrach den durch die Durchschüsse geschwächten
Klarstahl. Durch die vergrößerte Öffnung riss das Vakuum an dem
kläglichen Rest der Atmosphäre.

Die anderen Russen hatten ihre Waffen losgelassen und klammerten
sich an Rohre. Eines war heiß vom Dampfdruck im Inneren und die
verbrannten Hände konnten ihren Griff nicht halten. Der Mann folgte
seinem Kameraden in das Nichts.

Köpcke versuchte noch die Situation zu erfassen. Er begriff, dass
die Schutzblende sich nicht schloss, weil Kanurova zunächst die
Druckleitung zum Auslöser zerstört hatte. Brunner und die
Kommandantin handelten schon. Kanurova ließ den Tisch los, löste
ihr Bein vom Holm und trieb mit dem Strom an Gas auf die Öffnung
zu. Im Sturz führte sie ihre linke Hand zum Gesicht, steckte sich
etwas in den Mund. Dann griff sie nach ihrem Kragen und löste eine
dünne Kapuze, die sie sich über den Kopf zog. Sie rollte sich
zusammen, schlang die Arme um die Knie und schoss durch den Riss
hinaus.

Brunner reagierte um Sekundenbruchteile langsamer. Er entriegelte
seine Stiefel und sprang über den Tisch, ließ sich mitreißen,
streckte Arme und Beine aus und wurde wie ein Gekreuzigter über die
Öffnung genagelt. Sein Rücken bog sich durch, als er sich gegen den
Sog stemmte, den linken Arm weit über seine Reichweite hinaus
streckte, wie ein Schatten an den Resten der Kuppel entlang glitt.
Es fehlten wenige Zentimeter bis zu dem Hebel, der das Absenken der
Armierung außerhalb der Scheibe auslöste. Noch pumpte das
Umwälzsystem genug Atmosphäre in die Zentrale, um die Anwesenden am
Leben zu erhalten. Lange würden die Vorräte nicht mehr ausreichen.
Brunner zog das rechte Bein nach, setzte den Fuß auf die Bruchkante
und stieß sich ab, riss auch den zweiten Arm in die Richtung seines
Ziels herum. Seine Hände griffen den Hebel und legten ihn um. Sein
Körper hing freischwebend vor der Öffnung, die jetzt von außen
verschlossen wurde. Seine Finger rutschten vom Griff, langsam, aber
nicht so langsam wie die Armierung herab fuhr.

Er blickte Köpcke in die Augen, lächelte und ließ los.

»Nein, Jan!« Köpcke entriegelte seine Stiefel und sprang hinterher.
Brunner und die Stahlplatte passierten den Riss zur gleichen Zeit:
die Panzerplatte drückte den Körper des Leutnants gegen den Boden
und durchtrennte ihn auf Höhe der Brust.

Der Sog brach ab, die Trägheit trieb Köpcke weiter und schlug
seinen Kopf gegen die heruntergefahrene Stahlwand. Benommenheit
zerrte an seinem Bewusstsein. Kälte kroch in seinen Körper. Sein
Blick glitt durch die Zentrale. Verschwommen nahm er die Menschen
wahr, die nach Luft schnappten. Brunners halbierten Körper, der
unter ihm an den Boden gedrückt wurde. Topolski, der auf ihn zu
kam. Markov, der zum raumseitigen Observatorium schwebte, um dem
Kutter das Signal zu geben. Das Signal für die erste gewonnene
Schlacht. Die erste Schlacht im Krieg für die Unabhängigkeit des
Mars. Ein Krieg, der schon in den ersten Stunden hunderte Opfer
gefordert hatte. Köpcke gab den Widerstand auf und sich der
Bewusstlosigkeit hin.
\end{document}
