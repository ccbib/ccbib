\usepackage[ngerman]{babel}
\usepackage[T1]{fontenc}
\hyphenation{wa-rum Fracht-raum}
\hyphenation{schien}
\hyphenation{Tief-ebe-ne Tief-ebe-ne gro-ßen}


%\setlength{\emergencystretch}{1ex}

%\renewcommand*{\tb}{\begin{center}* * *\end{center}}

\newcommand\bigpar\medskip

\begin{document}
\raggedbottom
\begin{center}
\textbf{\huge\textsf{Asaker von Asgard}}

\medskip
Niklas Peinecke

\end{center}

\bigskip
\begin{flushleft}
Dieser Text wurde erstmals veröffentlicht in:
\begin{center}
Die Steampunk-Chroniken\\
Geschichten aus dem Æther
\end{center}

\bigskip

Der ganze Band steht unter einer
\href{http://creativecommons.org/licenses/by-nc-nd/2.0/de/}{Creative-Commons-Lizenz.} \\
(CC BY-NC-ND)

\bigskip

Spenden werden auf der
\href{http://steampunk-chroniken.de/download}{Downloadseite}
des Projekts gerne entgegen genommen.
\end{flushleft}

\newpage

Doktor Hauptmann von Ranke hätte es als eine Feuersäule
beschrieben, eine fast massiv wirkende, flammende Emission, die das
dichte, dunkelrote Blätterdach des marsianischen Urwaldes
durchbrach wie eine haushohe Lanzette. Er hätte das grünliche
Gleißen im Inneren erwähnt, vielleicht auch das ohrenbetäubende
Zischen beschrieben, das damit einherging. Als Offizier des
Kaiserlichen Wissenschaftlichen Korps hätte er sich jeder
Gefühlsäußerung oder persönlichen Einschätzung bis zum Schluss
seines Berichtes enthalten, doch wäre er sicherlich im Resümee kurz
auf seine tiefe Ergriffenheit und ein intensives Gefühl der Furcht
eingegangen, als er eine geschwärzte Gestalt unversehrt aus dem
Flammenstrahl treten sah, selbst unheilvoll leuchtend, ein
menschgewordener Odin, ein Übermensch gar, zornig und die Rache auf
seinem Gesicht tragend.

Dies alles hätte von Ranke zusammengefasst, und er hätte
geschlossen, dass eine weitere Erforschung der Ruinen von
Muspelheim mit erheblichen Risiken für Leib und geistige Gesundheit
verbunden sei, dass es unumgänglich sei, die Stadt zu isolieren und
den Zugang zu sperren.

Unglücklicherweise las das vergöttlichte Wesen all dies in seiner
ängstlichen Miene und tötete ihn an Ort und Stelle, direkt vor dem
seltsamen Tempel im Wald auf dem Mars.

\tb

Der EO-55 war eine gute Maschine. Sein blankes Messinggehäuse
harmonierte hervorragend mit jedem eleganten Interieur, deshalb
wurde er bevorzugt in teuren Hotels wie dem Berliner Adlon
eingesetzt, und seine hochpräzise Staffelwalzensteuerung konnte für
diverse Aufgaben eingerichtet werden. Das Kalkül dieses speziellen
Modells steuerte die Maschine soeben durch den Salon des Hotels, in
dem sich Gäste aus Adel und Politik versammelt hatten, um über
Geschäfte der Politik oder des Adels zu verhandeln. Kerzenschein
vergoldete ihn, und Stimmgewirr umfloss seinen Orbit. Auf seinem
zylindrischen Körper balancierte er eine Auswahl importierter
Köstlichkeiten vom Mars, sowie einige hochprozentige Getränke sehr
irdischer Herkunft. Dieser EO-55 konnte dank einer hochsensiblen
Luftdruckmechanik nicht nur unbelebten Gegenständen ausweichen, er
navigierte auch sicher um all die Damen der Gesellschaft in ihren
Edelstahlkorsetten und die Botschafter und Konsuln mit ihren
Monokeln und spiegelnden Zylindern herum.

Einzig die Bahn von Heinrich Sangerhausen konnte er nicht
vorausberechnen, da Sangerhausen zu betrunken war, um seinen
eigenen Weg zu kennen.

Sie stießen klirrend zusammen, jedoch tat sich niemand dabei weh:
EO-55 besaß kein Schmerzempfinden und Sangerhausen war längst über
eine Empfindung wie Schmerz hinweg. Stattdessen griff er dankbar
nach einer Flasche mit klarer Flüssigkeit, die EO-55 mit sich
geführt hatte, und torkelte, verfolgt von pikierten Blicken der
versammelten Gesellschaft, vor die Tür des Hotels.

Eben dort ließ sich seine Verlobte Elisabetha von Bärlepsch von
einem Livrierten in den Mantel helfen. Als er aus der Tür stürmte
warf sie ihm einen verzweifelten, aber auch ärgerlichen Blick zu.

»Du bist wieder stockbetrunken, Heinrich!«, heulte sie und zog den
Mantel eng um die Schultern.

Sangerhausen hatte Schwierigkeiten, sie im Blick zu behalten, sie
schien im Mosaik der Lichter auf und ab zu hüpfen. »Nicht betrunken
genug für diese feine Gesellschaft«, lallte er.

»Du führst dich auf wie ein Bauernlümmel! Ich schäme mich so für
dich!«

Neben ihnen hielt ein Automobil, eine chromblitzende Karosse der
Allgemeinen Maschinenwerke München, wie Sangerhausen trotz des
Nebels in seinem Kopf bemerkte. Irritiert wandte er sich wieder
seiner Verlobten zu. »Ich erkenne dich kaum wieder, wennu mit
diesen Schnöseln z?sammen bist.«

Ein Fenster der Karosse wurde geöffnet und ein Mann lehnte sich
heraus. »Mylady, belästigt Sie dieser Flegel?«

»Botschafter!«, rief Elisabetha erfreut. Ohne weiter zu zögern,
riss sie die Seitentür des Gefährts auf und zwängte sich in die
Fahrgastzelle.

Aus ihrer Reaktion schloss Sangerhausen, dass es sich um den Wagen
des britischen Botschafters Sir Ian Householder handeln musste. Mit
aller Geistesgegenwart die ihm verblieben war, griff er nach der
Tür, ehe sie diese hinter sich zuziehen konnte.

»Du steigst also lieber wie eine Bordsteinschwalbe zu diesem Snob
in den Wagen, als mit mir gesehen zu werden?«

»Heinrich, mir graut vor dir!«

»Vor mir?«, er lachte gurgelnd. »Vor dir muss einem grauen! Zeig
ihm die Narben, schließlich hast du dir die Rippen extra entfernen
lassen, um in dein Korsett zu passen!«

»Sir, ich muss doch sehr bitten!«, mischte sich Botschafter
Householder ein und versuchte, Sangerhausen aus der Karosse zu
schieben.

»Householder«, grölte der, »halten Sie lieber irgendwelche Häuser,
bevor ich -» Weiter kam er nicht, da er sich geräuschvoll über die
hellen Lederpolster des AMM erbrechen musste.

Die darauf folgenden politischen Verwicklungen zwangen ihn, eine
lange Reise anzutreten. Eine sehr lange Reise.

\tb

Oft wird das menschliche Leben in einer Großstadt wie Berlin mit
dem Wimmeln eines Insektenstaates verglichen. Das geschäftige Hin
und Her der Passanten, Dampfwagen und Straßenbahnen zeigt auch
tatsächlich gewisse Parallelen. Der Versuch jedoch, das verborgene
Geschehen einer modernen Stadt mit einer Analogie aus der Natur zu
versehen, muss scheitern. Das Rattern und Prasseln der Rohrpost
kennt keine natürliche Entsprechung, sein Takt ist der Takt der
Maschinen. Getrieben von heißem Dampf pulsieren minütlich tausende
von Sendungen durch die Stahlkanäle der Städte. Selbst eine
kleinere Universitätsstadt wie Hannover ist durchzogen von Rohren,
wie ein Bimsstein von Löchern.

So sinnierte Doktor Oberstleutnant Ludolf von Störmer im
einundvierzigsten Stockwerk des neuen Universitätsturms, während er
seinen Blick über die benachbarten Gärten schweifen ließ, und wie
zur Bestätigung traf in diesem Moment eine Rohrpostsendung für ihn
ein, machte pfeifend auf sich aufmerksam und rollte in den
Eingangskorb.

Sicher, der Ausblick ist schön, dachte er, aber ich hätte lieber
mein altes Büro im Welfenschloss behalten. Und wer ist Schuld?
Letztlich die Franzosen! Das neue Institut für die technische
Verbesserung des Menschen breitete sich im prestigeträchtigen
Schloss aus. Alle waren besessen vom Gedanken an den Übermenschen,
den Mann, der durch an den Körper applizierte, dampfbetriebene
Vorrichtungen über seine Biologie hinauswächst.

Störmer schnaufte durch seinen grauen Bart, während er abwartete,
dass die Rohrpost tickend abkühlte.

Die Franzosen hatten es begonnen, mit ihren metallgeflügelten
Tänzerinnen, den beräderten Clowns. Und jetzt war auch das Deutsche
Reich davon befallen.

Er ergriff das Metallrohr und entnahm ihm zwei zusammengerollte
Depeschen aus Zelluloid.

Die erste enthielt ein Schreiben vom Oberkommando des
Wissenschaftskorps. Ihm wurde befohlen, den Verbleib von Doktor
Generalmajor Lauenfeld zu untersuchen. Der letzte Bericht des
Gelehrten war überfällig, und das Oberkommando verlor allmählich
die Geduld. Seit Jahren hielt sich Lauenfeld in den Ruinen von
Muspelheim auf, einer archäologischen Stätte außerirdischer
Herkunft auf dem Mars, die bisher nur von untergeordnetem Interesse
gewesen war.

Das Oberkommando sagte ihm einen Assistenten und großzügige Mittel
für technische Ausrüstung zu, ein mehr als ungewöhnlicher Schritt
angesichts der eher geringen Bedeutung der Mission.

Die zweite Depesche war als streng geheim gekennzeichnet. Störmer
riss das Siegel auf und überflog den Inhalt. Er runzelte die
Stirn.

\tb

Missmutig spießte Doktor Oberstleutnant Ludolf von Störmer mit
einer Zange ein längliches Objekt auf und hielt es gegen die
aufgehende Sonne. Es schien bläulich zu schimmern.

»Ich vermute, dass es sich um den Schenkelmuskel eines Amphibiums
handelt«, brummte er und pustete durch seinen Bart. »Die Franzosen
hatten schon immer eine Vorliebe für wasserbewohnende Fauna.«

Doktor Leutnant Heinrich Sangerhausen kaute sein Frühstück mit
offensichtlichem Genuss. »Es liegt mir fern, Sie kritisieren zu
wollen«, sagte er gutgelaunt, »doch ein französischer Koch könnte
aus einem alten Armeestiefel und einer Pickelhaube ein
Drei-Gänge-Menü zaubern, das des Kaisers würdig wäre.«

Störmer ließ den dubiosen Froschrest zurück in die Würzsauce fallen
und krümelte stattdessen an einem Weißbrot herum. »Ihr Enthusiasmus
für alles Exotische in Ehren, aber die Kochkünste der Franzosen
haben ihnen nicht ihre afrikanischen Kolonien gesichert.«

Sangerhausen beendete sein Mahl und fuhr sich durch die - der
aktuellen Mode entsprechend - mit Pomade zu Stacheln aufgestellten
Haare. Er lehnte sich zurück. »Um so besser für uns, sonst gäbe es
keinen Raumhafen in Deutsch-Guinea, der uns die Reise zum Mars
ermöglicht hätte.«

»Nur, um hier in Frankreich zu landen«, ergänzte Störmer
grießgrämig. Er ließ das Brot auf seinen Teller fallen und zog eine
Platinzwiebel aus einer Tasche seiner Expeditionsjacke. »Ich denke,
wir sollten weitere politische Diskussionen auf die Zugfahrt
verschieben, sonst müssen wir zu Fuß nach Asgard laufen.«

Sie erhoben sich.

»Sie meinen in die Anarchistische Volksrepublik Asgard?«,
korrigierte Sangerhausen lachend.

»Sie sind ein subversiver Revisionist!«

»Dann sind Sie ein jurassischer Anachronismus!«

Ihr gutmütiger Streit hielt die wenigen Schritte bis zum Bahnhof
an, dann verstummte er angesichts der monströsen Pracht des gare
centrale de Aresville. Die mit einem Pagodendach aus Stahl und
Adamantiumglas überspannte Fläche hätte eine Kleinstadt fassen
können, denn diese Abmessungen waren nötig, um die zehn Gleise des
Kopfbahnhofs aufzunehmen. Derzeit standen vier Lokomotiven unter
Dampf, hinter sich jeweils kilometerlange Züge, von denen nur die
Personenwaggons in die Halle passten.

Die Forscher gelangten über eine Bogenbrücke auf den Bahnsteig, wo
sie eine Lokomotive der deutschen Baureihe 10001 erwartete, ein
Leviathan aus Stahl mit drei Dampfmotoren von jeweils achttausend
PS Leistung. Sie überragte die französischen Behemoth? auf den
benachbarten Gleisen um gut das anderthalbfache.

»Ein Glanzstück deutscher Ingenieurskunst«, fand Störmer die
Sprache wieder.

»Dem gewiss kläglich die Puste ausgehen würde, wenn sich das
Kaiserreich nicht bald mit der AVA über neue Adamantiumerz-Verträge
einigt«, gab sich Sangerhausen unbeeindruckt.

»Politik, Politik«, grollte Störmer. »Seit dem zweiten
Herero-Aufstand geifern alle nur noch von der Rohstoffkrise! Sind
wir bereits so abhängig vom Adamantium? Haben wir nicht auf der
guten, alten Erde genug Erze? Ich sage: Lasst den Mars den
Franzosen und den Anarchisten, geben wir uns mit Afrika zufrieden!
Seit die Neger in die Kolonien verbracht wurden, ist es ein
irdischer Garten Eden.«

Wie stets bei diesem Thema lief Sangerhausen ein leichter Schauer
über den Rücken. »Sie unterschätzen die Lage. Ohne Adamantium
würden andere Mächte uns überrennen. Adamantium härtet Glas gegen
das Vakuum und ermöglicht den Bau besserer Dampfmaschinen und
Raketen, als jeder gewöhnliche Werkstoff es vermöchte. Adamantium
ist der Bindestoff, der das Deutsche Reich zusammenhält, ohne es
würden die Raumkreuzer in lose Teile zerfallen, wie das Reich in
seine Bundesstaaten Preußen, Bayern und Österreich.«

Sie bestiegen einen auf dem Bahnsteig haltenden Schmalspurzug,
dessen offene Waggons sie zur Mitte seines großen Bruders fuhren.
Sie als Gäste der ersten Klasse hatten ein Anrecht auf diese
Beförderung, die Fahrgäste der zweiten und dritten Klasse mussten
mehrere hundert Meter zu ihren Abteilen zu Fuß zurücklegen.

Ein Steward empfing sie im Waggon und geleitete sie in die dritte
Etage. Sangerhausen bewunderte die Aussicht im zentralen
Salonbereich des Waggons. Nach Art einer Orangerie war das Dach
hier durch eine verglaste Konstruktion ersetzt worden, die einen
Panoramablick ermöglichte. Elegant gekleidete Gäste nahmen hier
bereits vor der Abfahrt ihr Frühstück ein.

»Meine Herren, ihre Kabine«, sagte der Steward schließlich und
entriegelte eine Mahagonitür rechts des Ganges. Störmer drückte dem
Marokkaner einige Reichsmark in die Hand - seit der Revolution in
Asgard waren die meisten Bediensteten Nordafrikaner - dann
entfernte sich der Steward wieder. Sie betraten die Unterkunft.

Ein dumpfer Aufprall war zu hören, dann taumelte Störmer rückwärts
in den Gang zurück, die Hände um einen kurzen Wurfspieß verkrampft,
der aus seiner Brust ragte. Er keuchte. Sangerhausen reagierte
schnell. Er riss eine leichte Handfeuerwaffe aus seinem Rock, eine
Miniaturausgabe, lediglich so groß wie zwei Fäuste, und legte auf
den anscheinend von der Decke fallenden Gegner an.

Der Anblick ließ ihn Augenblicke zögern.

»Ein Breughelscher Dämon!«, schoss es ihm durch den Kopf. Dennoch
drückte er ab.

Die Kreatur wurde in die Brust getroffen und in die Kabine
geschleudert. Sie begrub krachend einen Beistelltisch unter sich
und blieb wie eine zerbrochene Gliederpuppe liegen.

Sangerhausen wedelte den scharfen Pulverdampf beiseite und beugte
sich über seinen Vorgesetzten. »Sind Sie schwer verletzt?«, fragte
er besorgt.

Störmer grunzte und rappelte sich etwas auf. Endlich gelang es ihm,
den Speer aus seiner Brust zu ziehen, er blutete augenscheinlich
nicht.

»Nun, mein Hemd ist ruiniert«, sagte er, »außerdem habe ich
unerträgliche Schmerzen in den Rippen, vermutlich eine Prellung.«
Er humpelte in die Kabine und legte seinen Gehrock und sein Hemd
ab.

Erstaunt registrierte Sangerhausen, dass er darunter eine Weste aus
Adamantiumgeflecht trug. »Wie konnten Sie wissen ...«

»Später!«, unterbrach ihn Störmer.

Derweil setzte sich der Zug stampfend in Bewegung. Selbst hier,
mehr als zweihundert Meter hinter der Lokomotive, war das Dröhnen
der Dampfmotoren zu spüren.

Störmer beugte sich über den verhinderten Attentäter. Seine Gestalt
schien, abgesehen von den dunkelbraun verfärbten Händen,
menschlich, jedoch war sein Schädel bizarr verlängert, die Augen
waren an Gottesanbeterinnen gemahnende, schräge Schlitze, eine Nase
war nicht vorhanden, der Mund darunter eine reißzahnbewehrte Raute.
Diesen Kopf ergriff Störmer zu Sangerhausens Entsetzen und löste
ihn leicht vom Körper.

»Eine Maske!«, erkannte er.

Störmer nickte. »Ein Askari. Sehen Sie die Uniform?« Er wies auf
die Bekleidung des Toten, eine beigefarbene Buschuniform der
deutschen Afrikatruppen, allerdings hatte man die Abzeichen
sämtlich entfernt und durch bizarre, barbarisch-bunte Stickereien
ersetzt. In dieser Uniform befand sich ein gewöhnlicher, wenn auch
schwarzer, Mensch.

»Ich dachte die Asaker stünden in Diensten des Reiches?«

»Nicht seit den Herero-Aufständen. Die meisten wurden zum
Arbeitsdienst nach Asgard verbracht, seit dem Aufstand jedoch fehlt
jede Spur von ihnen.«

\bigpar

Sangerhausen wählte am Telephon die Nummer des Stewards und
überließ diesem die Entfernung des Leichnams. Derweil zog sich
Störmer um, ihr Gepäck war glücklicherweise bereits an Ort und
Stelle. Der jüngere Wissenschaftler bewunderte ein wenig die
Kaltblütigkeit, mit der der Ältere das Geschehen hinnahm. Sie
beschlossen, den Salon aufzusuchen, bis die Kabine in Ordnung
gebracht sei.

Die Frühstücksgäste hatten sich verstreut, daher gehörte die
Aussicht über vorbeiziehende Marswälder ihnen allein. Dichtes,
rotes Laubwerk wechselte sich mit lichten violetten Auen an breiten
Kanälen ab.

Sangerhausen entnahm seinem Rock eine Zelluloid-Rolle und strich
sie auf dem Tisch glatt.

»Diese Karte konnte ich im archäologischen Institut in Aresville
erstehen. Sie geht auf die erste Expedition unter Pierre Colin
zurück.« Er tippte auf einen Punkt neben einer rötlichen Fläche.
»Wir befinden uns hier. Um die Ruinen von Muspelheim zu erreichen»
- sein Finger verfolgte eine kurze Bahn - »müssen wir uns noch etwa
zweihundert Kilometer dem Yggdrasil nähern. Beachten Sie, dass der
Berg auf dieser Karte den Namen Olympus Mons trägt.«

Störmer nickte ungeduldig.

»Den größten Teil der Strecke legen wir durch unsere Fahrt nach
Asgard zurück. Von dort müssen wir eine Expedition von etwa fünfzig
Kilometern durch den Urwald unternehmen. Unser Hauptproblem wird
dabei die unklare Lage in der ehemaligen Kolonie sein.«

»Verdammte Anarchisten«, knurrte Störmer.

»Ich denke, wir können von Glück sagen, dass die Reichsbahn noch in
unseren Händen ist und die Volksrepublik einer wissenschaftlichen
Zusammenarbeit zugestimmt hat«, beschwichtigte Sangerhausen.

»Ha!«, blaffte Störmer. »Die Marine wird ihren Teil
Überzeugungsarbeit geleistet haben.«

Sangerhausen bemühte sich, das Thema zu wechseln. »Wie dem auch
sei, ich denke, wir können Muspelheim in spätestens einer Woche
erreichen, vorausgesetzt, der von den Anarchisten gestellte
Kontaktmann hat die Expedition bei unserer Ankunft weitgehend
vorbereitet.« Er legte einen Finger an sein bereits von
Bartstoppeln gespicktes Kinn. »Ich frage mich, was für ein Mann in
einer verarmten Kolonie Interesse an Wissenschaft hegt?«

\bigpar

Sangerhausen irrte in zwei Punkten: Erstens war die ehemalige
Kolonie keinesfalls verarmt. Bei ihrer Ankunft im Bahnhof fanden
sie die Bewohner geschäftig wimmelnd wie einen Ameisenhaufen vor.
Sicher war die Kuppel der Station noch immer von den Luftangriffen
der Marine zerstört, und hier und da fand sich auch ein Bettler in
zerrissener Husarenuniform in der Volksmasse, doch der Großteil der
Bewohner wirkte, wenn auch nicht reich, so doch zumindest
wohlgenährt. Direkt auf den Bahnsteigen war eine Unzahl von
Marktständen aufgebaut, an denen man alles von der Bartcreme, über
Hühner und Lurche bis zum Telephon erstehen konnte. Kleidung aus
ungefärbter Baumwolle bestimmte das Bild, einfache, aber saubere
Hemden; und selbst die Frauen trugen Hosen, ein Umstand, der
Sangerhausen gleichermaßen verwirrte und faszinierte. Sie wirkten
dadurch sehr unweiblich, weil ihre Hüften so gar nicht nach der
Mode des Reichs betont wurden, andererseits ermöglichte der
zuweilen dünne Baumwollstoff Einblicke in den Körperbau, die er
bisher vorwiegend aus anatomischen Werken kannte.

Sein zweiter Irrtum betraf den Kontaktmann, es handelte sich
nämlich um eine Frau. Sie stellte sich ihnen als Clara Delbrück
vor, indem sie ihnen burschikos die Hände schüttelte.

Störmer setzte dabei einen angeekelten Gesichtsausdruck auf, den er
bisher für französisches Essen reserviert hatte. Sangerhausen war
indes sehr angetan von der Beobachtung, dass sie offenbar statt des
üblichen Korsetts eine sehr viel leichtere und knappere Bekleidung
aus Stoff unter dem leichten Baumwollhemd trug.

»Fräulein Delbrück«, unterbrach Störmer seine Begutachtung, »ich
bin verwirrt! Ich hatte erwartet, dass uns ein Wissenschaftler
ihrer - äh - Republik in Empfang nimmt.«

Delbrück lächelte kalt. »Ich habe vor dem Umbruch an der
Universität von Asgard einen Abschluss in Anthropologie und
Urgeschichte erworben. Reicht Ihnen das als Qualifikation?«

»Sie missverstehen mich«, polterte Störmer. »Ich dachte, Sie wären
... nun, äh ...« Er rang nach Worten, gab es dann auf und schnappte
nach Luft.

»Sie meinen ein Mann?«, fragte Delbrück.

Störmer nickte mit rotem Gesicht.

»Ich fürchte, Sie werden sich mit mir abfinden müssen.« Delbrück
wandte sich nun Sangerhausen zu und lächelte ihn an. Er errötete
und sah zu Boden.

Ihre Augen haben die Farbe der Baumwolle, dachte er. Warum
interessiert mich das? Sie sieht nicht einmal wie eine Frau aus!

»Die Universität wurde geschlossen«, fuhr sie fort, »nachdem der
größte Teil des Lehrpersonals wegen kontrarevolutionärer Umtriebe
hingerichtet oder ausgewiesen werden musste.«

»Barbarisch«, zischte Störmer, worauf Delbrück jedoch nicht weiter
einging.

»Ich habe ein bewaffnetes Kommando veranlasst, Ihr Gepäck in eine
Unterkunft zu bringen. Ich hoffe, das ist in Ihrem Sinne.«

»Bewaffnetes Kommando?«, erkundigte sich Sangerhausen.

Wieder zeigte sie ihm dieses Lächeln. »Sie vergessen, dass dies
eine Anarchie ist. Was denken Sie, wie lange Ihr Gepäck unbewacht
in Ihrem Besitz geblieben wäre?«

\bigpar

Delbrück hatte alles perfekt organisiert. Sie wurden im nördlichen
Stadtteil Neu-Kreuzberg untergebracht, einer Villengegend, die
ehemals dem Adel vorbehalten war. Jetzt war es ein
Vergnügungsviertel, ihr Quartier schien zur Hälfte ein
Stundenhotel, zur anderen Hälfte ein Freudenhaus zu sein.

Störmer war sichtlich erleichtert, als sie nach drei Tagen bereit
zum erneuten Aufbruch waren. »Ich hätte es keinen weiteren Tag in
diesem Sündenbabel ausgehalten!«

Sangerhausen hob einen Finger ans Kinn. »Die Leute scheinen auf
eine verdrehte Art am Adel zu hängen. Ist Ihnen aufgefallen, dass
alle Freudenmädchen wie Damen der Gesellschaft gekleidet sind?«

Störmer schnaubte. »Ich weiß nicht, in welcher Gesellschaft Sie
verkehren, ich jedenfalls habe noch keine Frau von Stand barbusig
in der Öffentlichkeit herumspazieren gesehen!«

»Ich bezog mich eher auf den Schnitt der Kleider.« Da Sangerhausen
bemerkte, dass dem Doktor Oberstleutnant das Thema unangenehm war,
schwieg er, während sie weiter ihre Ausrüstung verstauten.

In den strapazierfähigen blauen Uniformen des Wissenschaftskorps
samt Pickelhauben erschienen sie schließlich am Bahnhof. Delbrück
war mit einer ähnlichen Kluft in dunkelgrau angetan.

»Schön, dass Sie pünktlich sind«, begrüßte sie sie. »Sie können
gleich beim Beladen helfen.«

Sangerhausen ergriff sofort eine vor ihm stehende Kiste, während
Störmer mit verschlossener Miene zögerte, sich dann aber fügte. Sie
schoben die Kisten im Gepäckwagen eines kurzen Zuges zusammen, der
außer diesem Waggon nur aus einem Personenwagen und einer rostigen
Lokomotive der Baureihe 8500 bestand. Sangerhausen erinnerte sich,
dass diese Maschinen noch keinen Adamantiumkessel hatten, sondern
aus herkömmlichem Stahl konstruiert waren.

»Hervorragend!«, meckerte Störmer, als sie verschwitzt in die
zerschlissenen Ledersitze des Personenwagens fielen. »Wir fahren
mit einem Museumsstück auf Urwaldexpedition.«

Sangerhausen entnahm seinem Rock eine flache Metallflasche, trank
einen Schluck und bot sie dann Störmer an. »Branntwein?«, fragte
er.

»Nein danke«, knurrte dieser, ohne die Augen zu öffnen.

»Ich hätte gern einen Schluck«, sagte Delbrück und nahm dem
erstaunten Sangerhausen die Flasche aus der Hand. Er war nicht auf
den Gedanken gekommen, ihr überhaupt etwas anzubieten.

Gebannt beobachtete er, wie sie den Verschluss abdrehte und die
Flasche an die Lippen setzte.

»Haben Sie noch nie eine Dame trinken sehen?«, fragte sie, als sie
ihm den Flachmann zurückreichte.

»Ehrlich gesagt: Nein«, stotterte er. »Zumindest keine Dame der
Gesellschaft.«

Sie zuckte mit den Achseln. »Nun, unsere Gesellschaft ist ein wenig
anders, stimmt?s?«

\bigpar

Sie fuhren nun nordwärts auf den Yggdrasil zu, auf einer
eingleisigen Strecke, die zu einer aufgegebenen Bergbausiedlung
führte. Die Anarchisten unterhielten dort, wie Delbrück ihnen
erzählte, eine Poststation, die jedoch nur unregelmäßig von dem Zug
angefahren wurde. Sie beabsichtigten, sich dort mit einem
einheimischen Führer namens Wilhelm Mbeki zu treffen, um dann das
letzte Stück zu Fuß zu bewältigen.

Da der Zug weitgehend automatisch von einer Staffelwalzen-Maschine
gesteuert wurde, verbrachte der Zugführer, ein schweigsamer Mann
namens Carl Riese, viel Zeit mit ihnen im Personenwagen.

Es traf ihn daher ebenso unerwartet wie sie, als die
Staffelwalzen-Maschine einen plötzlichen Nothalt auslöste. Sie
wurden von ihren Sitzen geschleudert. Nicht befestigte
Ausrüstungsgegenstände polterten haltlos zwischen den Sitzbänken
herum. Mit kreischenden Bremsen kam der Zug zum stehen.

»Höllenhunde!«, fluchte Riese und klopfte seine Eisenbahnermütze
ab. »Nicht schon wieder!«

Vorsichtig stiegen sie aus. Sie waren mitten in einer
langgestreckten Rechtskurve stehengeblieben. Zu beiden Seiten
befand sich undurchdringlicher Urwald, die Sonne warf nur trübrotes
Zwielicht durch das dichte Blattwerk.

Sangerhausen brachte seine Schusswaffe in Anschlag und auch Störmer
zog seinen Säbel.

Störmer klopfte mit dem Säbel gegen einen jungen Baum, der bis zum
Bahndamm gewachsen war. »Es ist ein perfekter Platz für einen
Hinterhalt.«

»Mehr Asaker?«, fragte Sangerhausen.

Angespannt lauschten sie einen Moment, doch außer einem entfernten
Grunzen unbestimmbarer Waldtiere war nichts zu hören.

In einer Reihe gingen sie zur Lokomotive, die sich selbst unter
Dampf hielt. Nun konnten sie den Grund für den Nothalt erkennen:
Ein Baum von gut acht Metern Dicke lag quer über dem Gleisbett.

Störmers Schnauzbart sträubte sich regelrecht. »Was für eine
Schlamperei! Ein Baum dieser Größe hätte ausgeschlagen werden
müssen, bevor er die Strecke blockieren kann!«

Delbrück inspizierte das untere Ende des Urwaldriesen. »In diesem
Fall ist Ihr Einwand unberechtigt«, vermeldete sie. »Offenbar ist
der Baum woanders gefällt und dann mit Absicht hergebracht
worden.«

»Also doch Sabotage.« Sangerhausen steckte die Waffe wieder in den
Rock. »Aber wir können unmöglich zu Fuß weitergehen. Haben wir
Räumwerkzeuge für solche Fälle dabei?«

Riese nickte. »Allerdings müssten wir den Stamm zersägen, bevor die
Lok ihn beiseite schieben kann. Bei der Dicke dauert das Tage.«

Sangerhausen setzte sich auf das Gleis und ließ die Schultern
hängen. »Was für ein Schla-»

»Hören Sie auf zu lamentieren und helfen Sie mir lieber«,
unterbrach ihn Störmers Stimme aus Richtung des Güterwagens. Sie
drehten sich um und sahen, wie Störmer die längliche Kiste, die
Sangerhausen schon in Asgard bemerkt hatte, aus dem Waggon zerrte.
Er eilte hinzu, und gemeinsam hievten sie den Kasten in das
violette Gras neben dem Schotterbett. Riese kam mit einem Kuhfuß
hinzu und begann den vernagelten Deckel aufzusprengen.

»Sie hatten mich gefragt, worum es sich hierbei handelte«, begann
Störmer mit selbstzufriedenem Ausdruck zu erklären. »Nun, haben sie
jemals von kohärenten Lichtstrahlen gehört?«

»Kann ich nicht behaupten«, sagte Delbrück und auch Sangerhausen
schüttelte den Kopf.

Mit einem zufriedenen Schnauben begann der Doktor Oberstleutnant
nun, eine Reihe von verschieden großen Tuben der mit Samt
ausgeschlagenen Kiste zu entnehmen. Während er sie
zusammenschraubte, erklärte er: »Stellen Sie sich vor, Sie würden
von einem Schwarm Sperlinge attackiert. Ein jeder Vogel vermag nur
wenig, selbst ein Aufprall aus vollem Flug reicht kaum, ihnen mehr
als einen blauen Fleck beizubringen.« Er unterbrach sich kurz, um
die zusammengesetzte Röhre anzuheben. »Doktor Sangerhausen, wären
Sie so freundlich, den Dreifuß auszuklappen? Dankeschön!

Nehmen Sie nun aber an, die Spatzen kämen in synchronen
Angriffswellen, immer ein ganzes Geschwader gleichzeitig. Die
kombinierte Wucht ihres Anpralls würde Ihnen schwer zu schaffen
machen, ich behaupte sogar, dass sie Sie von den Füßen fegen würde
und Sie mit ein paar gebrochenen Rippen zurückließe.« Er zog einige
Flügelmuttern an und begutachtete sein Werk, eine einem Fernrohr
nicht unähnliche Apparatur aus poliertem Stahl, in deren Mitte eine
konzentrische Anordnung fast unterarmlanger Glasröhren zu sehen
war. »Gewöhnliches Licht ist wie ein ungeordneter Vogelschwarm,
kohärente Lichtstrahlen jedoch entsprechen der organisierten
Angriffsformation.« Störmer befüllte nun eine koffergroße
Adamantium-Dampfmaschine mit einem flüssigen Treibstoff und setzte
sie in Gang. Über ein dickes Kupferkabel war sie mit dem Apparat
verbunden. Er peilte kurz das Rohr entlang und justierte dann an
einer Winkelskala am Dreifuß. »Bitte treten Sie zurück und sehen
Sie nicht in den Strahlengang!«

Auf einen Handgriff Störmers hin ergleißten die Röhren in einem
unerträglichen Licht, so dass die Umstehenden unwillkürlich einen
Schritt zurücktraten und die Augen beschirmten. Gleichzeitig
flammte am Baumstamm eine handtellergroße Feuerzunge auf, die sich,
wie durch Magie der Bewegung des Apparats folgend, abwärts durch
das Holz fraß. Bereits nach wenigen Augenblicken war der Stamm
entlang eines sauberen Schnitts neben dem Gleis durchtrennt.

»Beeindruckend!«, kommentierte Delbrück. »Doch Sie sagten, es
handele sich um Licht. Warum war dann kein Strahl zu sehen?«

Störmer schien es unangenehm, der Frau zu antworten, doch
Sangerhausen und der Eisenbahner sahen ihn auch mit fragenden
Mienen an.

»Licht ist nur sichtbar, wenn es direkt ins Auge fällt. Kohärentes
Licht breitet sich nur entlang seines Vektors aus, doch würde ich
Ihnen nicht empfehlen, sich in diesen zu begeben«, dozierte er
schließlich.

Nach Anweisung des Eisenbahners Riese setzten sie den
Licht-Apparatus mehrmals um und trennten auf diese Weise drei
Baumscheiben von beträchtlicher Dicke aus dem Stamm. Dies dauerte
einige Zeit, weil die Dampfmaschine das Gerät immer von neuem
aufladen musste. Die letzte Scheibe rollte schließlich unter
brechenden Geräuschen den Bahndamm hinab in den Urwald.

Nur Sekunden darauf antwortete ein tierhaftes Brüllen, und ein
alptraumhafter Wirbel aus Hauern und Fell schoss heran, Riese unter
sich begrabend.

Störmer erstarrte vor Schreck, auch Sangerhausen besaß diesmal
nicht die Geistesgegenwart, seine Feuerwaffe zu ziehen. Einzig
Delbrück ergriff behände das herumliegende Brecheisen und drosch
damit auf die dämonische Kreatur ein, die noch immer nicht von
Riese abließ. Endlich trollte sich das Tier doch, eine dünne
Blutspur nach sich ziehend. Sangerhausen erhaschte einen Blick auf
das Wesen, es gemahnte ihn an ein schlankes, schwarz bepelztes
Flusspferd, jedoch mit muskulösen Beinen und vier gewaltigen Hauern
im Kiefer, unter einer buckligen Stirn. Es schien keine Augen zu
haben.

»Was in Gottes Namen war das?«, japste er.

Delbrück ließ das Eisen fallen und stützte sich auf die Knie. »Ein
Wühlschwein. Das ist seltsam!«

»Fürwahr!«, stimmte Sangerhausen zu und half Riese auf. Er blutete
aus einigen übel aussehenden Schürfwunden an Händen und Wangen. »Es
schien keine Augen zu haben.«

»Doch, doch. Unterhalb des Mauls«, sagte Delbrück. »Aber das meine
ich nicht. Es ist ein Aasfresser. Ich frage mich, was es hier getan
hat.«

Sie vergewisserten sich, dass Riese soweit wohlauf und in der Lage
war, die gelösten Holzblöcke mit der Lokomotive beiseite zu
schieben. Störmer begann seine Apparatur zu verstauen.
Währenddessen kamen Delbrück und Sangerhausen überein, die Richtung
zu untersuchen, aus der das Marsschwein aus dem Wald
hervorgebrochen war.

\bigpar

»Oh mein Gott!« Blut quoll unter Delbrücks Stiefel hervor. Sie
verzog das Gesicht und sprang einen Schritt zurück. Dabei stieß sie
gegen Sangerhausen. Sie fuhr herum. »Können Sie nicht aufpassen,
wohin Sie laufen? Da liegt eine Leiche!«

Sangerhausen starrte an ihr vorbei in die Baumkronen. »Und da hängt
eine weitere«, brachte er hervor.

Delbrück sah nach oben. »Beides Asaker. Was wollten die hier?«

»Wahrscheinlich handelt es sich wirklich um einen Hinterhalt, wie
der Oberstleutnant zuerst vermutete.« Er erzählte von dem Anschlag
im Zug. »Es fragt sich nur«, schloss er, »wer dieses Attentat hier
vereitelt hat.«

\bigpar

Die Stimmung nach ihrer Rückkehr zur Lokomotive war beklommen. Der
versuchte Überfall durch die Asaker war seltsam genug, doch ihr
unheimlicher Tod beschäftigte sie noch mehr.

Schweigend, die Schusswaffen im Anschlag beobachteten sie den
vorbeiziehenden Urwald. Doch die Fahrt verlief ereignislos, sie
erreichten die Poststation unbehelligt.

Wilhelm Mbeki erwartete sie auf dem hölzernen Bahnsteig.

»Noch ein Askari«, blaffte Störmer.

»Ich bin zwar Witbooi«, antwortete Mbeki freundlich, »aber ich
stand nie in Diensten des Militärs. Als Arbeitssklave kam ich auf
den Mars, doch nun bin ich ein freier Mann.«

»Hätte ich mir denken sollen, bei dem Namen«, murmelte Störmer,
während die anderen Mbeki die Hand gaben.

Sie verbrachten einen Tag damit, auszuruhen und Vorbereitungen für
die Weiterreise zu treffen. Abgesehen von Mbeki war die Station
unbesetzt, und wurde, wie sie erfuhren, meist lediglich von
Staffelwalzen-Maschinen gegen wilde Tiere beschützt. Die Minen
waren seit langem aufgegeben und die Ruinen waren nur für
Wissenschaftler von Interesse.

\tb

Am nächsten Tag befanden sie sich bereits mitten im Wald. Riese war
mit dem Zug auf dem Rückweg nach Asgard, er würde erst in zwei
Wochen zurückkehren.

Die Luft war verdichtet wie in einem Kolben, und es schien
Sangerhausen, als wateten sie eher, anstatt zu marschieren. Die
Dampfmaschine, die ihr Gepäck trug, rollte schnaufend hinter ihnen
her und warf zwischen den Bäumen hohle Echos.

»Herr Mbeki«, sagte er, »ich hörte, sie dienten bereits Doktor
Generalmajor Lauenfeld als Führer?«

»Das stimmt. Ich zeigte seiner Expedition eben diesen Weg.«

»Und Sie wissen nichts über seinen Verbleib?« Sangerhausen stützte
sich auf einen Baumstumpf und wischte Schweiß aus seinem Gesicht.

»Nachdem der Doktor Hauptmann Muspelheim erreicht hatte, ging ich
zur Poststation zurück. Wir kamen überein, dass ich sie nach zwei
Wochen mit Nachrichten und Vorräten versorgen sollte, doch das
Lager war verlassen.«

Störmer sah den Schwarzen mit zusammengekniffenen Augen an und
sagte: »Es würde mich nicht wundern, wenn er ...«

Mbeki tat, als habe er die Bemerkung überhört.

\bigpar

Lauenfelds Basislager erwies sich tatsächlich als verlassen. Sie
setzten das geräumige Hauptzelt instand und errichteten ein
kleineres daneben für Delbrück. Als die Sonne in violettem Feuer am
Horizonts versank, setzten sie eine von Staffelwalzen gesteuerte
Alarmanlage in Gang, die sie vor Tieren beschützen sollte.
Getränkte Kohlebriketts verbrannten in einem kleinen Ofen und
verbreiteten Wärme in der - im krassen Gegensatz zum Tag - kühlen
Nacht.

Mbeki verabschiedete sich als erster, Störmer folgte mit
sichtlichem Missbehagen, die Nacht mit dem Witbooi in einem Zelt
verbringen zu müssen.

Delbrück und Sangerhausen saßen schweigend vor dem Ofen und
starrten in die Glut.

»Wissen Sie«, begann Delbrück schließlich, »wenn Sie möchten, ich
meine, vielleicht, mögen Sie ja bei mir im Zelt ...« Sie brach ab
und drehte den Kopf zur Wand des Urwaldes.

»Nicht, dass mir das unangenehm wäre, aber ein solches Angebot -»

»Nein.« Sie senkte den Blick scheu. »Nicht was Sie denken. Es ist
nur diese Nacht, in der ich nicht allein sein kann. Die Witbooi
nennen sie die Nacht des Blicks, wegen der Monde.«

Sangerhausen sah zum Himmel. Tatsächlich standen Deimos und Phobos
wie zwei böse Augen dort. »Gut, in dem Fall ...«

Ohne sich zu entkleiden legten sie sich nebeneinander ins Zelt.
Bald darauf schlief Sangerhausen ein.

Er erwachte nur noch einmal kurz davon, dass Delbrück sich an ihn
kuschelte.

\bigpar

Am nächsten Morgen waren Delbrück und Mbeki mit der Zubereitung des
Frühstücks beschäftigt, als Störmer ihn beiseite nahm.

»Mein bester Doktor Leutnant Sangerhausen«, flüsterte er, »es ist
mir klar, dass ich ein Opfer von Ihnen verlange, aber ich denke,
wir sollten uns in zwei Gruppen aufteilen, um verschiedene
Grabungen zu überprüfen, die Lauenfeld in seinem letzten Bericht
dokumentiert hat.«

Sangerhausen zwinkerte. »Inwiefern ist das ein Opfer?«

»Nun«, wand sich der Oberstleutnant, »ich wage es nicht, die
Anarchistin und diesen Schwarzen zusammen ziehen zu lassen, sie
bringen es fertig und hecken irgendeine Teufelei gegen uns aus.
Daher muss ich Sie mit Fräulein Delbrück einteilen. Sie scheinen
besser mit ihr umgehen zu können als ich. Ich selbst werde
notgedrungen mit dem Witbooi arbeiten müssen.« Er pustete durch
seinen Schnauzbart.

Sangerhausen hatte einige Mühe, ernst zu bleiben. »Im Dienste des
Kaiserreichs nehme ich das auf mich.«

»Sie sind ein Patriot, trotz Ihrer zeitweiligen politischen
Umnachtung.«

»Umnachtung, ja, das trifft es«, sagte Sangerhausen und dachte an
die Nacht des Blicks.

Muspelheim zeigte erst bei Licht seine ganze, außerirdische
Großartigkeit. Eine Gruppe von acht Pyramiden, jede in der Größe
ihrer irdischen Schwester von Gizeh ebenbürtig, beherrschte eine
Fläche von fast zehn Quadratkilometern. Pierre Colins Expedition
hatte sie völlig unberührt vom Urwald vorgefunden, die kriechenden,
saugenden Organismen des Mars drangen nicht auf das Gebiet der
Ruinen vor. Alle Gebäude und die Straßen dazwischen schienen aus
massivem Eisen zu bestehen. Sie waren rostig, doch für ihr Alter,
das Colin auf etwa 14000 Jahre geschätzt hatte, erstaunlich gut
erhalten. Trotz des immensen Wertes, den eine solche Menge von
Eisen repräsentierte, hatte es niemand gewagt, diesen Schatz zu
bergen. Lediglich Wissenschaftler fühlten sich von Muspelheim
angezogen, Prospektoren, Glücksritter und Abenteurer mieden die
Stätte.

Lauenfeld hatte vierzehn Plätze dokumentiert, die seinem Bericht
nach interessante Artefakte enthalten könnten. Sie planten für jede
Gruppe einen Pfad, der je sieben der Punkte umfassen sollte, um
dort nach Anzeichen über den Verbleib Lauenfelds und seines
Assistenten von Ranke Ausschau zu halten.

\bigpar

Am späten Vormittag zogen Wolken auf und tauchten die Ruinen in
violettes Zwielicht. Missmutig schlitterte Störmer eine
rostüberzogene Rampe hinab, Mbeki dicht auf den Fersen. Sie waren
auf dem Weg zu Punkt Nummer drei, einer Gruppe von fünf exakten
Kugeln unbestimmten Zwecks, die inmitten einer Vertiefung zwischen
den Pyramiden lagen. Er konnte sie bereits ausmachen, jede war mehr
als doppelt so groß wie ein Mensch.

»Meinen Sie, die Erbauer dieser Anlage waren in irgendeiner Weise
menschenähnlich?«, fragte er.

Wilhelm Mbeki blieb stehen und sah ihn an. »Wer könnte das sagen?«,
antwortete er. »Kann man an den kulturellen Werken eines Stammes
seine Menschlichkeit ablesen?« Er schüttelte den Kopf. »Nehmen Sie
die Maschinen und Häuser von Berlin, und denken Sie sich alle
beweglichen Objekte hinfort. Könnte man nur aus diesen leeren
Hüllen auf ihre Bewohner schließen?«

Störmer schauderte beim Gedanken an ein entvölkertes Berlin, daher
wechselte er das Thema. »Sie waren in Berlin?«

Mbeki wandte sich ab und ging weiter. »Vor langer Zeit«, sagte er,
ohne sich umzuwenden, »lange vor den Aufständen. Sie können das
nicht glauben, nicht wahr?«

Störmer schloss wieder zu ihm auf. »Warum sollte ich das
bezweifeln?«

»Sie trauen mir nicht, weil ich schwarz bin. Deshalb halten Sie
auch Ihren wirklichen Auftrag geheim.«

Ein hohles Gefühl bemächtigte sich Störmers, doch er bemühte sich
ruhig zu bleiben. »Was meinen Sie damit? Mein wirklicher Auftrag?«

Mbeki lachte freundlich. »Sie unterschätzen die Anarchisten, auch
wir haben noch Verbindungen zur Erde! Ich meine Ihren Auftrag, den
Verbleib der Herero-Sklaven zu untersuchen. Sie zurückzubringen,
damit sie dem Reich dienen können, und eine neue deutsche
Marskolonie aufbauen können.« Seine Stimme gewann einen bitteren
Klang.

Störmer schwieg betroffen. Dann sagte er: »Ich will Ihnen nichts
vormachen. Ich habe tatsächlich den Auftrag, den Aufenthaltsort der
verschwundenen Herero zu ermitteln. Aber ich weiß nichts von
Plänen, die sich mit der Versklavung dieses Volks befassen.«

Sie hatten die erste der großen Kugeln erreicht und umrundeten sie
nun. »Ich bin Witbooi«, sagte Mbeki, »und die Herero waren nicht
immer unsere Freunde. Aber die Deutschen haben beide Stämme
betrogen, und ich werde es nicht zulassen, dass sie erneut in
Unfreiheit gelangen.«

»Wissen Sie denn, wo sich die Herero aufhalten?«

Mbeki hielt an und sah Störmer ernst in die Augen. »Ja«, sagte er,
»sie stehen hinter Ihnen.«

Störmer fuhr herum. Neben der Kugel standen etwa zehn Asaker mit
erhobenen Wurfspießen.

\bigpar

»Halten Sie mich ruhig für neugierig«, sagte Delbrück und stocherte
mit ihrem Klappspaten am Fuß einiger Säulen, »aber was hat Sie auf
den Mars verschlagen?«

Sangerhausen beschattete seine Augen mit der Hand und betrachtete
ausgiebig den oberen Teil der Obelisken. Es handelte sich um
vereinzelt stehende Monolithen aus demselben eisenartigen Material,
aus dem hier alles zu bestehen schien. Die Oberfläche der seltsamen
Gebilde war mit Mustern oder technischen Gravuren übersät, die
Sangerhausen an ein Luftbild von Berlin erinnerten.

»Was denken Sie, welchen Zweck hatten diese Obelisken?«, fragte
er.

»Sie lenken ab«, warf Delbrück ihm vor. Sie stützte sich auf den
Spaten und sah ihn direkt an. »Sind Sie auf der Flucht?«

Sangerhausen zögerte. »So könnte man es nennen«, gab er dann zu.
»Ich war mit Elisabetha von Bärlepsch verlobt.«

»Der Tochter des Grafen von Bärlepsch? Was haben Sie angestellt?
Sie geschwängert?« Sie kicherte.

»Wenn es das nur wäre«, seufzte er. »Ich habe im Suff das Automobil
des britischen Botschafters ruiniert. Von Bärlepsch konnte nur mit
Mühe einen politischen Eklat abwenden. Er sorgte dafür, dass ich
die Assistentenstelle bei von Störmer bekam.«

»Ich glaube, diesen Platz können wir abhaken«, sagte Delbrück.
»Wenn Lauenfeld hier war, hat er zumindest keine Grabung
veranlasst.« Sie schulterte ihr Marschgepäck, dann brachen sie auf.
»Der nächste Punkt ist eine Einrichtung, die von Ranke als
?Brunnen? bezeichnet. Soweit ich das sehe, ein etwa vierzehn Meter
tiefer Schacht. - Sie sind ja ein ganz Schlimmer«, setzte sie
hinzu.

»Ich hatte mir überlegt, dem Alkohol abzuschwören, allerdings ist
diese Expedition nicht gerade die ideale Gelegenheit dazu.« Er
entnahm seinem Rock die Taschenflasche und hielt sie Delbrück hin.
»Möchten Sie?«

»Gern.« Sie griff danach. »Mh!« Sie würgte und schluckte dann
krampfhaft.

»Was ist?« Sangerhausen blieb stehen. »Ist er zu stark?«

Delbrück hustete. »Nein, das ist es nicht.« Sie deutete auf den
Schacht, den sie inzwischen erreicht hatten: Ein quadratisches Loch
von gut zwei Metern Kantenlänge, umfasst von einem massiven
Eisenrahmen. Um den Schacht herum lag etwas, das Sangerhausen
zunächst für modrige Schlingpflanzen gehalten hatte, bevor ihm
einfiel, dass es auf dem Gelände von Muspelheim keine Vegetation
gab. Er trat näher und erkannte verrottete Uniformen, darunter
weißliche Knochen.

»Noch mehr tote Asaker«, flüsterte er. »Was mag sie getötet haben,
und warum?«

»Sie waren mir im Weg«, schnarrte eine Stimme hinter ihnen. Sie
fuhren herum. Aus dem Schatten eines Torbogens trat eine Gestalt,
die zunächst sehr menschlich wirkte, tatsächlich musste es sich,
wie Sangerhausen schloss, um Lauenfeld handeln. An seinem Körper
klebten die verkohlten Reste von etwas, das seine Korps-Uniform
gewesen war, der spezialimprägnierte Stoff bildete nun eine massive
Schicht von Kohle um ihn. Seine Haut war von einem tiefen,
glänzenden Schwarz, als wäre er durch irgendeinen unbegreiflichen
Prozess in ein Standbild aus Onyx verwandelt worden. Jedoch bewegte
er sich.

»Doktor Generalmajor Lauenfeld!«, rief Sangerhausen. »Was ist mit
ihnen geschehen?«

»Sorgen Sie sich nicht, mein Guter«, knarrte Lauenfeld. Auch seine
Stimmbänder schienen verändert, es klang, als versuche eine
Dampforgel zu sprechen. »Mir geht es besser als je zuvor.«

»Sie müssen sofort in ein Hospital! Und wo ist Ihr Assistent, von
Ranke?« Delbrück war sichtlich entsetzt: mit aufgerissenen Augen
starrte sie die Gestalt an, die einmal der Doktor Generalmajor
gewesen war.

»Genug Geschwätz für heute.« Lauenfeld begann zu glühen, seine
schwarze Haut bekam einen grünlichen Schimmer, als wäre ein Feuer
in seinem Inneren entzündet worden. Gegen seine rostrote Umgebung
bildete er einen außerweltlichen Kontrast und leuchtete nach
wenigen Augenblicken so stark, dass sie die Augen abwenden mussten.
»Ich habe ihm ein Ultimatum gestellt, dieselbe Wahl, vor der Sie
nun stehen: Werden Sie sich mir als Kaiser des Sonnensystems
unterwerfen, oder möchten sie noch am Platze sterben?«

»Was soll das?« Sangerhausen konnte die Hitze spüren, die nun von
Lauenfeld ausging. Gleichzeitig zitterte sein Körper angesichts des
seltsam veränderten Generalmajors. »Wir sind treue Untertanen des
Kaisers, wir werden uns keiner anderen Autorität beugen!«

Lauenfeld lachte, eine eigentümlich menschliche Regung an ihm, der
nun wie eine lebende Flamme wirkte. Ascheflocken seines vormaligen
Rocks lösten sich und trudelten in einer Wolke herum. »Das gilt
vielleicht für sie, aber wohl kaum für diese kleine Anarchistin!
Der Kaiser ist schwach, dass er die Existenz eines solchen
Staatsgebildes» - er spuckte das Wort aus - »überhaupt zulässt!
Geführt von Frauen und Negern!«

Delbrück umklammerte ihren Spaten, als könne er ihr als Waffe gegen
das Monstrum dienen. Sie nahm sich sichtlich zusammen, doch ihre
Stimme zitterte: »Sie fürchten die Schwarzen, nicht wahr? Sie töten
die Asaker, wo sie können.«

»Lächerlich!«, dröhnte Lauenfeld. Die Flammen bildeten mittlerweile
eine brüllende Säule von gut vier Metern Höhe. »Ich bin der
Übermensch, den Nietzsche voraussah, ich bin das biblische Omega!
Der Kaiser ist schwach, er konnte nicht verhindern, dass die Herero
revoltierten und so viele Deutsche töteten! Unter meiner Herrschaft
wird das nicht passieren, ich werde ein Reich des Friedens führen,
als gütiger, strenger Vater über meine Kinder.«

»Was für ein Vater tötet die Kameraden seiner Kinder? Die Herero
sind meine Landsleute!« Delbrück zitterte jetzt am ganzen Körper.

»Ein Neger kann nie Landsmann einer Deutschen sein, so können Sie
auch keine Deutsche sein. Sie wählen also den Tod.« Das Brüllen der
Flammen steigerte sich zu einem hohen Kreischen, und um die Mitte
der Flammensäule, in deren Zentrum die Gestalt Lauenfelds kaum noch
auszumachen war, bildete sich ein Ring aus Licht, der schnell zu
rotieren begann. Delbrück zog Sangerhausen heran und drückte ihn
schützend an sich.

Sangerhausens Ohren waren inzwischen taub vom sich stetig
steigernden Lärm. Er schloss die Augen, umarmte Delbrück und
wartete auf den Tod. Heiße Gase und Strahlung verbrannten sie, sie
kauerten sich zusammen. Er überlegte, was er ihr hätte sagen
wollen, aber bei dem sie umtobenden Inferno wäre jeder Versuch
etwas zu rufen vergeblich gewesen.

\bigpar

Dann war es plötzlich vorbei, und Sangerhausen fand sich zu seinem
Erstaunen am Leben. Vorsichtig erhob er sich, die verbrannte Haut
schmerzte an den Gelenken. Er blinzelte in Richtung der vormaligen
Feuersäule, doch weder von den Flammen, noch von Lauenfeld war
etwas zu sehen, lediglich ein sternförmiges Brandmuster war
zurückgeblieben.

»Mein Gott, Sangerhausen!«, sagte jemand. »Zum Glück sind Sie am
Leben, ich glaubte Sie schon verloren!«

Im Torbogen standen Störmer und Mbeki, zwischen sich den kohärenten
Licht-Apparatus.

Delbrück erhob sich nun ebenfalls. »Was ist passiert?«, fragte sie
und strich sich eine von der Hitze gekräuselte Strähne aus dem
geschwärzten Gesicht.

»Fräulein Delbrück, ich freue mich, auch Sie lebendig zu sehen«,
sagte der Doktor Oberstleutnant. »Was Ihre Frage betrifft«, fuhr er
fort, »ich denke, dass der Licht-Apparatus dem
verabscheuungswürdigen Doktor Generalmajor Lauenfeld mehr Energie
zugeführt hat, als gut für ihn war. Was für ein Schwätzer! Der
Kaiser ist ein Mann der Tat, und einer Tat bedurfte es, Lauenfeld
von seinem Toben zu erlösen.«

»Wahnsinnig war er sicher«, bestätigte Sangerhausen, »aber wie
konnte er überhaupt jene Kräfte erlangen?«

»Ich denke, das können wir zum Teil aufklären«, antwortete Störmer
mit einem Seitenblick zu Mbeki. Der lächelte ihn an. »Wir trafen am
fünften Markierungspunkt auf eine Truppe Asaker der Herero. Es wäre
gewiss mein Ende gewesen, wenn mein geschätzter Freund Wilhelm
nicht eingegriffen und unsere Situation erklärt hätte. Es stellte
sich heraus, dass in dieser Stadt ein Mechanismus existiert, der
einem Menschen außerirdische Kräfte verleihen kann, wie unser
verkohlter Nietzscheaner dies demonstriert hat.«

»Wenn er je Nietzsche gelesen hat, so hat er ihn nicht verstanden«,
warf Delbrück ein. »Der Philosoph forderte, dass sich der Mensch
auch moralisch weiterentwickelt, es ging nicht um bloße
Machtfülle.«

»Wohl gesprochen, Fräulein Delbrück«, brummte Störmer und
schmunzelte in seinen Schnauzbart. Sangerhausen war erstaunt, dass
er die Unterbrechung durch die junge Frau so gleichmütig hinnahm.

»Auf jeden Fall berichteten die Herero, dass nach ihrem Exodus aus
Asgard einigen ihrer Leute Ähnliches widerfahren sei. Sie
erkannten, dass die Ruinenstadt vor dem Zugriff von Menschen
isoliert werden muss - eine Maßnahme, der ich von ganzem Herzen
zustimme.«

Mbeki hatte inzwischen einen Verbandskoffer aus seinem Gepäck
gezogen und begann, Delbrück zu verarzten. Sangerhausen, der keine
Schwäche zeigen wollte, zog eine Brandsalbe aus seinem eigenen
Rucksack und bestrich, so gut es ihm möglich war, seine
Verbrennungen. »Wie konnten Sie uns rechtzeitig finden?«

»Eine Frage simpler Logik«, triumphierte Störmer. »Da der monströse
Lauenfeld uns bisher nicht gefunden hatte, mussten wir davon
ausgehen, dass er Ihnen auf der Spur war. Wir kehrten daher zum
Basislager zurück und bewaffneten uns mit dem Apparatus. Dann
folgten wir, so schnell es uns möglich war, Ihrer Route. Wie sich
zeigte, kamen wir gerade im rechten Moment.«

»Und dafür danken wir Ihnen«, sagte Delbrück.

\bigpar

Die Rückkehr nach Asgard verlief ohne Zwischenfall. Riese hatte in
der Poststation bereits ihre Ankunft erwartet. Er war entsetzt,
zwei Mitglieder der Expedition in solch bedauernswertem Zustand zu
finden. Dennoch waren sie glimpflich davongekommen. Sangerhausen
sann oft darüber nach, was hätte geschehen können, wäre Lauenfeld
nicht ein solcher Schwätzer gewesen.

Sie ordneten ihre Angelegenheit in der anarchistischen
Volksrepublik, währenddessen bezog Sangerhausen mit - wie er
hoffte\bigpar-\bigparder gebotenen Diskretion ein gemeinsames Zimmer mit Clara
Delbrück. Er zögerte eine Aussprache mit seinem Vorgesetzten so
lange wie möglich hinaus, jedoch kam der Punkt, an dem er ihn in
seinem Quartier in einem zweifelhaften Hotel aufsuchen musste.

»Doktor Oberstleutnant Störmer«, begann er, »ich muss ...«

»Ich denke, Sie müssen mich zunächst einmal Ludolf nennen. Nach
allem, was wir gemeinsam bestanden haben, halte ich das für
angemessen.«

Sangerhausen starrte den älteren Wissenschaftler einen Moment an,
wie er dort in seiner Korpsuniform an einem kleinen Tisch in seinem
Hotelzimmer saß. Vor sich hatte er einen Teller Sauerkraut mit
Amphibien, er hatte eine verdächtige Vorliebe für dieses Gericht
entwickelt.

»Gut«, setzte Sangerhausen erneut an, »Ludolf, ich möchte darum
bitten, auf dem Mars bleiben zu dürfen.« Er schluckte. »Ich
beabsichtige, mich mit Fräulein Delbrück zu verloben und dann
meinen Wohnsitz hier zu nehmen.«

»In diesem anarchistischen Staatsgebilde?«, fragte der Doktor
Oberstleutnant kauend.

»Nun, eine Gruppe von Idealisten möchte es den Herero gleichtun und
eine eigene Kolonie begründen. Ich hörte, sie wollen einen Staat
nach den sehr interessanten Lehren eines gewissen Karl Marx
errichten. Wir wollen uns ihnen anschließen.«

Störmer tupfte sich einige Soßenreste aus dem Bart. »Das sei Ihnen
gewährt, auch wenn ich bedauere, dann nicht mehr auf Ihre Dienste
zurückgreifen zu können. Jedoch werde ich nicht weit sein, denn
auch ich habe darum gebeten, auf dem Mars bleiben zu dürfen.«

Erneut war es an Sangerhausen seinen Vorgesetzten erstaunt
anzublicken. Dieser lehnte sich zurück und erklärte: »Die Existenz
jener Einrichtung, die Superkräfte verleiht, bedeutet eine nicht
unerhebliche Gefahr für das Kaiserreich. Schon die bloße Tatsache,
dass einige Herero sie benutzt haben, kann den Untergang des
Reiches herbeiführen - die Herero sind uns nicht eben wohlgesonnen,
auch wenn jene Gruppe, der Wilhelm und ich begegneten vorwiegend an
einer Geheimhaltung der Waffe interessiert war. Ich habe daher ein
Gesuch an Doktor Generaloberst Freiherr zu Lüchow gestellt, für
mich eine ständige Vertretung des Wissenschaftskorps in Asgard
einzurichten. Desweiteren werde ich versuchen, seiner Majestät dem
Kaiser bald persönlich Bericht zu erstatten - ich traue den
Bürokraten nicht, schon gar nicht, wenn es um eine solche
Machtfülle geht.«

»Und Sie wollen dann von hier aus die Isolation von Muspelheim
gewährleisten? Wie wollen Sie das alleine schaffen?«

»Nicht alleine, mein guter Heinrich. Mein Freund Wilhelm Mbeki wird
mir zur Seite stehen und meine Verhandlungen mit den Herero
unterstützen.«

Sangerhausen schüttelte den Kopf. »Ich hatte gedacht, Ihre
absonderliche Wandlung in Punkto Geschmack wäre das Erstaunlichste.
Jetzt muss ich lernen, wie sehr ich mich getäuscht habe.«

Wieder schmunzelte Störmer. »Sagen Sie, Heinrich, habe ich je
erwähnt, dass meine Lieblingstante die französische Gräfin Simone
de Texiers ist?«

\bigpar

Sangerhausen trat auf die Straße und ließ sich vom bunten Treiben
der Marsmetropole mitziehen. Ohne Ziel spülte es ihn durch die mit
Asphalt befestigten Straßen, auf denen Absonderliches feilgeboten
wurde, Edelsteine, die die Farbe abhängig vom Wetter änderten,
vielbeinige, augenlose Kreaturen für die Käfighaltung, zu
unbekannten Zwecken, dampfbetriebene Apparate und
Staffelwalzenmaschinen, hier auf dem Mars gebaut.

Eine neue Zeit brach an. Nach den Jahrzehnten der Stabilität drohte
dem Reich ein Umschwung, von dem die Revolution von Asgard nur ein
erster Vorgeschmack gewesen war.

Sangerhausen löste sich aus dem Strom der Menschen und betrat das
Haus, in dem Clara und er Wohnung genommen hatten.

Was auch immer geschehen würde, sie würden es erleben.

\end{document}
