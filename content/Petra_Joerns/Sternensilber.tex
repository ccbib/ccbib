\usepackage[ngerman]{babel}
\usepackage[T1]{fontenc}
\hyphenation{wa-rum}


%\setlength{\emergencystretch}{1ex}

%\renewcommand*{\tb}{\begin{center}* * *\end{center}}

\newcommand\bigpar\medskip

\begin{document}
\raggedbottom
\begin{center}
\textbf{\huge\textsf{Sternensilber}}

\medskip
Petra E. Jörns

\end{center}

\bigskip
\begin{flushleft}
Dieser Text wurde erstmals veröffentlicht in:
\begin{center}
Die Steampunk-Chroniken\\
Geschichten aus dem Æther
\end{center}

\bigskip

Der ganze Band steht unter einer 
\href{http://creativecommons.org/licenses/by-nc-nd/2.0/de/}{Creative-Commons-Lizenz.} \\ 
(CC BY-NC-ND)

\bigskip

Spenden werden auf der 
\href{http://steampunk-chroniken.de/download}{Downloadseite}
des Projekts gerne entgegen genommen. 
\end{flushleft}

\newpage


Anthony MacDonald lehnte sich an die kalte Mauer der engen Gasse.
Der übliche Londoner Nebel füllte die Straßen und brachte den
Gestank von brackigem Wasser, Unrat, Fisch und Fäkalien mit sich.
Seine Kleidung schien den Geruch bereits angenommen zu haben, was
daran liegen mochte, dass er mit einem Fischerboot im alten
Themseport angekommen war. Um ungesehen zu bleiben, war dies jedoch
der beste Weg, um das verlassene Stadtviertel zu erreichen.

Seit der Optimierung der Luft- und Ætherschiffe hatte der Hafen an
Bedeutung verloren. Und gäbe es nicht noch immer einige
Hinterwäldler-Staaten ohne Æthertechnik, wäre der ehemals
bedeutende Londoner Hafen nicht mehr als ein Friedhof.

Das Kopfsteinpflaster zu Anthonys Füßen glänzte feucht im matten
Schein einer nahen Gaslaterne. Obwohl es bereits Frühjahr war,
haftete dem Nebel klamme Kälte an.

Unwillkürlich zog er seinen Mantel enger. Er hasste es, hier zu
sein. Nicht nur weil es kalt war und stank, sondern vor allem wegen
der Enge der Gassen und der hohen Mauern der Lagerhäuser und
Kontore. Erst recht hasste er es, von seinen englischen »Freunden«
zu einem Auftrag gepresst zu werden. Er wünschte sich zurück auf
sein Schiff, das im Lufthafen vor Anker lag, zurück in die Weite
des Æthers, wo allein die Sterne seine Sicht begrenzten.

Sein Schiff, das eigentlich nicht ihm gehörte sondern seinen
Auftraggebern. Was er Sophie »verdankte«, die dafür gesorgt hatte,
dass sein vorheriges Æthergefährt zerstört worden war.

Süße Sophie … Der Gedanke an sie genügte, ihr Gesicht vor sich zu
sehen. Die Lippen geöffnet zum Kuss …

\bigpar

Ein Geräusch störte seine Gedanken. Er drückte sich von der Mauer
ab und ging einige wenige Schritte. Endlich! Am Ende der Gasse
schälten sich mehrere Gestalten aus dem Nebel. Ein Geräusch hinter
ihm machte ihn darauf aufmerksam, dass andere Helfer seines
Auftraggebers versuchten, ihm den Weg hinunter zum Wasser
abzuschneiden.

Resigniert unterdrückte er ein Seufzen. Er hatte damit gerechnet
und bereits Vorsorge getroffen. »Ihr solltet eigentlich wissen,
dass Ihr solche Spielchen unterlassen könnt, Mylord.« Dem letzten
Wort gab er einen vernehmbar ironischen Klang.

»Ich dachte, es amüsiert Euch, MacDonald.« Eine der Gestalten kam
ihm entgegen, entpuppte sich als hochgewachsener, hagerer Mann in
schwarzem Mantel, dessen Gesicht im Dunkel seiner Kapuze verborgen
blieb.

Wenn der Lord jedoch hoffte, unerkannt zu bleiben, täuschte er
sich. Anthony ahnte inzwischen, dass es sich um Lord Witherby, den
Vorsitzenden des Oberhauses höchstpersönlich handelte. Alles, was
er brauchte, um aus der Ahnung Gewissheit zu machen, war ein Blick
unter die Kapuze und den wollte er sich heute endlich verschaffen.
»Wollt Ihr Konversation betreiben oder habt Ihr einen Auftrag für
mich?«

»Weshalb so eilig, Captain?«

»Eilig hattet Ihr es und nicht ich. Also kommt zur Sache. Ich habe
Wichtigeres zu tun, als mir hier den Hintern abzufrieren.« Obwohl
Anthony es vermeiden wollte, war ihm seine Gereiztheit deutlich
anzuhören. Und sie wuchs mit jeder Sekunde, die er am Boden
verbringen musste.

»Dass Ihr hier stehen könnt – als Kapitän eines Ætherschiffes –
verdankt Ihr uns«, zischte der Mann im schwarzen Mantel. »Vergesst
das nicht!«

»Wie könnte ich? Ihr erinnert mich doch bei jeder sich bietenden
Gelegenheit daran.« Unwillkürlich tastete Anthony nach dem Griff
seines Degens. Eine dumme Angewohnheit, wie er sich sofort schalt,
denn die neuen Repetierschusswaffen, von denen er zwei Exemplare
bei sich trug, waren wesentlich effektiver. Sie waren eine Leihgabe
der englischen Krone – wie das Ætherschiff, das im Lufthafen lag.

Der Mann im Mantel kam einige Schritte auf Anthony zu, stand nun
fast auf Tuchfühlung. »Mir deucht, dass ich es nicht oft genug
erwähnen kann«, knurrte er.

»Habt Ihr nun einen Auftrag für mich oder wollt Ihr mich weiter
langweilen?« Anthony war zwar einen halben Kopf kleiner als sein
Gegenüber und er wusste, dass er aufgrund seiner blonden Haare und
des hübschen, glatt rasierten Gesichts meist unterschätzt wurde,
dennoch schaffte er es, auch größeren und breiteren Männern
gegenüber herablassend zu wirken. Insbesondere wenn jemand ihn
reizte – so wie sein Auftraggeber, der ihn immer wieder daran
erinnerte, dass er von ihm abhängig war. Sophies wegen, die ihn
verraten hatte. Der Stachel saß tief.

»Bastard!«

»Woher wisst Ihr das?«

Der Lord zischte vor unterdrücktem Zorn.

Anthony lächelte. »So wichtig also?« Wichtig genug, dass er es
geschafft hatte, den sonst so beherrschten Mann aus der Fassung zu
bringen. Dies schien der richtige Moment zu sein, um einen Schuss
ins Blaue zu wagen: »Oder hat der Meteorit, der heute Morgen auf
dem Mond einschlug, etwa auch Eure Sinne verwirrt?«

Die Hand des anderen schoss vor, um nach Anthonys Kehle zu fassen.
»Was wisst Ihr darüber?«

Treffer!

Beiläufig hob MacDonald seine Hand und schlug den Arm beiseite.
»Was wisst Ihr darüber?«

Der Mann im Mantel versetzte ihm einen Stoß vor die Brust. »Genug
der Spielerei. Holt ihn mir!«

Die Kapuze rutschte bei der Bewegung eine Handbreit nach hinten und
gewährte Anthony einen Blick auf Witherbys Gesicht.

Am liebsten hätte er gejubelt, weil sein Vorhaben so schnell zum
Ziel geführt hatte. Was er zu sehen bekam, erstickte seinen Triumph
jedoch im Keim. Scrimgeour hatte ihm erzählt, dass das Gesicht des
Lords bei einem mysteriösen Unfall entstellt worden war. Anthony
ahnte, dass es sich bei dem Unglück um die Explosion handelte, bei
der er unter anderem sein vorheriges Æthergefährt verloren hatte.
Aber auf den Anblick, der sich ihm nun bot, war er nicht
vorbereitet.

Ein Auge war durch eine mechanische Anfertigung ersetzt worden und
schien merkwürdig starr, ebenso wie die dazugehörige
Gesichtshälfte. Kein Wunder, dass der Lord ihn deswegen hasste.
Wenn auch eigentlich Sophie für die Explosion verantwortlich war.
Oder die Gier seiner Lordschaft nach mehr Macht – das war
Ansichtssache. Immerhin wusste Anthony jetzt, dass Witherby erneut
nach etwas gierte.

»Wem soll ich den Meteorit beschaffen? Euch?« Anthony vermochte
trotz des monströsen Anblicks Erstaunen zu heucheln. »Doch wohl der
Englischen Krone. Oder etwa nicht?«

Mit einem Ruck zog sein Gegenüber die Kapuze zurück an seinen
Platz. »Bringt ihn mir – im Namen der Englischen Krone.«

Anthony genoss den Zorn in der Stimme des Anderen. Nun wusste er,
was er hatte wissen wollen. Witherby arbeitete in die eigene
Tasche. »Wie viel?«

»Genug.« Die Hand verschwand unter dem Mantel. Mit einem prall
gefüllten Beutel kam sie wieder zum Vorschein.

»Ein kleiner Vorschuss – nehme ich an.«

Der Mann im schwarzen Mantel sog scharf die Luft ein. »Ich warne
Euch …«

Bevor er seinen Satz beenden konnte, griff der Ætherkapitän nach
dem Beutel und wog ihn in der Hand. Ein königliches Magnetsiegel,
das Witherby mit einem Fingerabdruck seines Zeigefingers öffnete,
hing an dem gegerbten Leder.

Ein kurzer Blick zeigte, dass es sich bei dem Inhalt um Goldmünzen
handelte. »Das Doppelte bei Lieferung.«

Die Hand, die aus dem Mantel ragte, ballte sich zur Faust. »Wie Ihr
wünscht.«

Zu billig, begriff Anthony im gleichen Moment. Aber genug, um
diverse Schulden zu zahlen, die ihn belasteten. Insbesondere die,
die er bei Scrimgeour hatte, seit er sich anlässlich der Explosion
aus französischer Haft hatte freikaufen müssen.

Aber was seinem englischen Auftraggeber derart viel Gold wert war,
musste etwas Besonderes sein. Es musste so besonders sein, dass man
zusätzlich mit dem Wissen darum Geld verdienen konnte. Oder mit
seinem Wissen, wer sein englischer Auftraggeber war und was dieser
mit dem Meteorit beabsichtigte.

Die Aussicht genügte Anthony. »Ihr hört von mir«, sagte er. Ohne
die Antwort des Anderen abzuwarten, ließ er ihn stehen.

Die Männer, die ihm den Rückweg abschneiden sollten, kamen aus
ihrer Deckung, um ihn aufzuhalten. Aber Anthony sprang auf die
Mülltonnen, die er sich zuvor positioniert hatte, und flankte von
dort über die Mauer, welche die Gasse vom nächsten Hinterhof
trennte.

Dilettanten, dachte er noch verächtlich, ließ sich an einem
Silberfadenseil hinab in den Hof und machte sich auf den Weg zum
Lufthafen.

\tb

Kaum dass sie die Zugleine ausgeklinkt hatten, mit denen die Winde
des Lufthafens sie durch die mit Rauch geschwängerten Schichten der
unteren Atmosphäre gezogen hatte, fühlte Anthony sich wohler. Als
sie endlich die Sonnensegel gesetzt hatten und nur noch die Sterne
ihren Weg zum Mond säumten, war er in seinem Element.

Leider nutzten nur die englischen Schiffe die Energie der Sonne zur
Fortbewegung im Æther. Auf den Luftgefährten des Handelssyndikats
und der Franzosen wurde Schwerbenzin verbrannt. Auf jenen einiger
rückständiger Staaten wurden gar noch Kohlen geschippt. All das
hatte die Atmosphäre so sehr verpestet, dass die englischen Schiffe
mit Winden in die sauberen Ætherschichten katapultiert werden
mussten. Und kein Wunder, dass alle Welt so sehr nach einem Antrieb
gierte, der von der Sonne oder fossilen Brennstoffen unabhängig
war. So groß war diese Gier, dass ein Lord des Oberhauses vor
einigen Monaten einen unbedeutenden, englischen Luftschiffer
inkognito angeheuert hatte, damit er mit dessen Hilfe bei den
Franzosen spionieren konnte – und dabei seine Gesundheit
riskierte.

»Also?«, fragte Anthony seinen ersten Maat Dante, als sie endlich
allein an Deck waren.

Der kratzte sich im Nacken, dort wo der Anschluss für seine
implantierten Datenplatten lag. Die Bewegung erinnerte den Captain
daran, dass die letzte Überholung schon lange her war – zu lange –
und die reaktive Luft des Æthers die kleinen Verbindungsbolzen
hatte rosten lassen. Trotzdem behielt Dante den Kurs fest im Blick.
»Sternensilber, das ist es also, was wir jagen, was?«

Anthony runzelte die Stirn. »Ich befürchte es.«

»Ich auch, Captain. Tatsache ist jedenfalls, dass ´ne Menge Leute
dahinter her zu sein scheinen.«

Und meinem englischen Freund ist das Zeug ziemlich viel wert,
ergänzte Anthony im Stillen. Ist das etwa die geheimnisvolle
Substanz, welche die Deutschen für ihren neuartigen Antrieb
benutzen? »Und was sagen die Gerüchte über seine Verwendung? Schau
doch mal in deiner Bibliothek nach!«

Dante kam der Anweisung seines Kapitäns nach, ohne auch nur einen
Moment lang den Kompass aus dem Auge zu verlieren. »Dies und das,
und nix Genaues. Einige sagen, man braucht es, um Waffen zu bauen.
Andere sagen für Tarnvorrichtungen von Ætherschiffen oder für einen
neuen Antrieb, mit dem man weder Sonne noch Benzin braucht.«

Also doch! »Und was glaubst du?«

»Wenn Ihr mich fragt: Keins davon. Wenn ich auch gehört habe, dass
die Franzosen tatsächlich Schiffe mit Tarnvorrichtungen haben
sollen.«

Dann funktionierte die mysteriöse Vorrichtung, die er damals mit
Sophie in der französischen Ætherschiffswerft gefunden hatte, also
tatsächlich. Eigentlich war das nur ein Vorwand gewesen, um sich
Sophies Loyalität zu sichern, während er für Witherby in der Werft
nach dem neuen Antrieb suchte. Bei der falschen Nation, soviel
wusste er inzwischen. Wirklich dumm, dass der Apparat damals
aufgrund der Diebstahlsicherung mitsamt seinem Schiff explodiert
war. Zum Glück war Sophie nichts zugestoßen. Ob ihr Gefährt
ebenfalls über eine Tarnvorrichtung verfügte?

»Und wofür soll es dann gut sein?«

»Keine Ahnung. Magie?«

Magie – um damit Ætherschiffe zu fliegen? Was, wenn das
Sternensilber magisch war und als Kern des neuen deutschen Antriebs
genutzt wurde? Die Deutschen waren führend in der Technikmagie. Es
würde passen. Sinnend starrte Anthony in das Nachtblau des Æthers,
der sie umgab. »Was weiß die Mannschaft?«

»Was sie wissen muss. Dass wir zum Mond fliegen. Aber ich glaube,
sie ahnen weshalb. Hat sich schnell herum gesprochen, dass alle
Welt hinter dem Sternensilber her ist.«

»Ein Rennen also.«

»So sieht es aus, Captain. Das wir aber gewinnen können.«

»Oh nein, Dante. Wir werden es gewinnen.«

»Aye, Captain.« Dante zeigte beim Grinsen einen Zahnstummel.

»Sag den Männern, dass derjenige, der zuerst ein anderes Schiff
sieht, diese Goldmünze erhält.« Anthony fischte bei den Worten eine
der Münzen aus der Tasche, die er von seinem Auftraggeber erhalten
hatte und warf sie Dante zu, der sie geschickt auffing. »Und wenn
wir das Rennen gewinnen, werden wir so viele davon erhalten, dass
sich jeder Einzelne zur Ruhe setzen kann.«

»Aye, aye, Captain. Ist mir ein Vergnügen, Captain.« Wieder war
Dantes Zahnstummel zu sehen, während er eifrig die Goldmünze an
seinem schmutzigen Hemd rieb und sie dann prüfend gegen die Sonne
hielt, um zu sehen ob sie bereits glänzte.

Der Kapitän war sich nicht sicher, ob sein Maat nur grinste oder
die nicht vorhandenen Zähne bleckte. Mit einem letzten Blick auf
die Sterne stieg er die Treppe hinab in die Kapitänskabine. Vor dem
Kartentisch blieb er stehen. Nachdenklich zog er einen kleinen
Brief aus der Innentasche seiner Jacke. Während er sich auf einen
der mit Leder bezogenen Stühle sinken ließ, erschnupperte er einen
dezenten Duft nach Veilchen, der das Papier immer noch umgab.

Sophie.

Nach kurzem Zögern entfaltete er das Schreiben und las ihn ein
zweites Mal, obwohl er den Inhalt bereits kannte.

»Mon cher Antoine, ich möchte dir ein Angebot unterbreiten. Triff
mich heute Abend im Hinkenden Einhorn. Es soll dein Schaden nicht
sein. Un doux baiser, Sophie.«

\bigpar

Sophie …

\bigpar

Allein der Duft genügte, um ihn an ihre letzte gemeinsame Nacht zu
erinnern. Augenblicklich fühlte er wieder ihre samtweiche Haut
unter seinen Fingern, das Streicheln ihrer nachtschwarzen Haare auf
seiner Brust, ihre Zunge in seinem Mund, mit dem Geschmack nach
Pfirsich und Vanille. Süß und verführerisch.

Er hätte alles dafür gegeben, um sie noch einmal kosten zu dürfen.
Sie noch einmal in den Armen zu halten. Ohne sich um sein Leben
oder seine Freiheit sorgen zu müssen – so wie beim letzten Mal, als
er so dumm gewesen war, Sophies Reizen zu erliegen, so dass sie ihn
mit einem Becher Wein und etwas Schlafmittel darin an die Franzosen
verkaufen konnte. Als Spion der englischen Krone. Was ihn den Kopf
gekostet hätte, wäre er nicht in der Lage gewesen, sich mit viel
Gold frei zu kaufen. Gold, das er nun Scrimgeour schuldete. Und
dessen Menge sich mit jedem Tag um eine Goldmünze vergrößerte.

Als Dank verkaufte er die Pläne der französischen Tarnvorrichtung,
die er gemeinsam mit Sophie gestohlen hatte, an die Englische
Krone, die ihm in Anerkennung seiner Dienste ein Ætherschiff
verliehen und ihn zwangsweise rekrutiert hatte. Und so war es
geradezu ein Wunder, dass Sophie ihm Briefe schickte, anstatt
Kugeln aus Blei. Zu erklären war das nur dadurch, dass sie einen
weiteren Betrug plante. Dass sie Kontakt mit ihm aufnahm, war der
beste Beweis dafür.

Aber dieses Mal, dachte Anthony, werde ich dir die Suppe versalzen,
ma Chére! Dieses Mal werde ich den Köder auslegen und du wirst
diejenige sein, die in die Falle läuft.

Wozu, fragte eine kleine, penetrante Stimme. Um dich zu rächen?
Oder um sie endlich wieder in deinen Armen halten zu können, du
verliebter Trottel?

Wider Willen atmete er erneut den Duft ihres Parfums ein.

\bigpar

Sophie …

\bigpar

Der Duft genügte, um alle Bedenken beiseite zu wischen. Dieses Mal
würde seine Rechnung aufgehen. Dieses Mal würde er triumphieren.
Dann wäre er endlich wieder frei – frei von der Englischen Krone
und seinen Schulden. Er war auf alles vorbereitet.

Frei – wofür? Etwa für Sophie?

Aber darauf blieb er sich die Antwort schuldig.

\tb

»Schiff achtern!«

Obwohl Anthony bereits seit Tagen auf diesen Ruf wartete, zuckte er
zusammen, als er schließlich durch das Schiff schallte. Nur mit
Hemd und Hose bekleidet und rasch angelegtem Degen stürzte er auf
die Brücke.

Einer der Männer hing im Ausguck und wies mit dem Finger zum Heck.
Dante stand bereits neben dem Rudergänger, um die Anweisungen des
Kapitäns abzuwarten.

Wortlos hob Anthony sein verstellbares Sehrohr an die Augen.
Tatsächlich. Schon nach wenigen Herzschlägen fand er die Silhouette
des fremden Schiffes vor dem Nachtblau des Æthers. Nein, da war
noch eines. Drei.

Sein Herzschlag beschleunigte sich. Um Ruhe bemüht suchte er ein
zweites Mal den Horizont ab. Ein Atemzug und seine Hände zitterten
nicht mehr. Äußerlich völlig beherrscht setzte er die Optik ab.

»Drei Schiffe. Sie versuchen, uns in die Zange zu nehmen.«

Als Antwort pfiff Dante durch die Zähne. »Könnte eng werden,
Captain.«

»Alle Mann an die Kanonen. Dante, schwenk die Marie Sophie um drei
Grad – stell sie voll in den Ætherstrom. Kurs zwei sieben null bei
null zwei null. Wir müssen uns eine Lücke erkämpfen.«

»Aye, Captain.«

Während der Maat die Befehle über Deck brüllte, beobachtete Anthony
weiter die gegnerischen Schiffe durch seine Rohroptik. Sie hatten
das interplanetarische Handelssyndikat geflaggt. Dieses
Hoheitszeichen hatte er auch im Flaggenschrank … Und um friedliche
Absichten zu hegen, waren die Bewegungen der Æthergefährte zu sehr
aufeinander abgestimmt. Zielgerichtet zogen sie die Schlinge enger.
Ohne das Teleskoprohr abzusetzen fühlte Anthony, dass Dante wieder
neben ihm stand.

»Die Männer sollen sich bewaffnen. Könnte gut sein, dass wir
kämpfen müssen.« Anthony nahm das Sehrohr mit einem Ruck herunter
und schob es zusammen. Die vernieteten Eisenplatten des Decks
fühlten unter seinen nackten Füßen gut an. Wie von selbst fand
seine Rechte den Griff des Degens.

»Aye, Captain.«

Während der Maat die Befehle mit lauter Stimme weitergab, trat der
Kapitän neben den Rudergänger. »Tom, drei Strich spinwärts.« Das
würde sie genau auf das Schiff an der rechten Flanke zusteuern
lassen.

»Aye aye.« Die Lippenbewegungen des Rudergängers waren durch den
dichten grauen Vollbart kaum zu erkennen.

»Dante, lass die Harpune vorbereiten.«

»Aye, Captain.« Der Maat grinste. Er schien die Absichten seines
Kapitäns zu erahnen.

Das Katz-und-Maus-Spiel schien Stunden zu dauern. Zeit genug, um
Stiefel und Jacke zu holen. Trotzdem blieb Anthony wie festgenagelt
an Deck. Er hatte seinen Degen und die beiden Pistolen, die Dante
ihm gereicht hatte, mehr brauchte er nicht. Völlig reglos wartete
er. Bis das Schiff an der rechten Flanke endlich in
Kanonenreichweite war.

Anthony hob die Hand.

»Captain, der versucht, uns zwischen sich und den andern zu
drängen«, brummte Tom.

»Ausweichen. Zwei Strich Sonnenlee … Lassen wir sie doch in die
Sonne blinzeln!«

»Aye aye.«

Der Kapitän sah, wie sich die Luken des gegnerischen Schiffes
öffneten. Sein Blick fiel auf Dante. Der Maat nickte.

»Sie feuern«, schrie der Rudergänger.

»Jetzt!« Anthonys Hand zerteilte die Luft.

»Harpune los«, brüllte Dante.

Ein Zischen ertönte. Ging von der riesigen mit Druckluft
betriebenen Harpune aus, die ein Seil hinter sich her schleppend
durch das Nachtblau des Æthers flog und sich in das gegnerische
Schiff bohrte.

Anthony griff nach dem Geländer. Keinen Moment zu spät. Ein Ruck
ging durch die Marie Sophie. Dann flog es im großen Bogen herum,
brachte sie auf Breitseite mit dem gegnerischen Gefährt. Die
Æthersegel knatterten, und wenn Anthonys Berechnungen falsch waren,
würden Segelbäume brechen

»Feuer!«

Einen Herzschlag später donnerten ihre Kanonen, während die
Kanonenkugeln des gegnerischen Schiffes wirkungslos hinter ihnen im
Æther verpufften. Wie ein Schwarm brennender Fackeln gingen die
ihren derweil auf dem anderen Schiff nieder.

Es handelte sich bei ihnen um ein weiteres Geschenk der englischen
Krone, das die Mannschaft immer noch mit Furcht erfüllte. Durch die
Reibung im Kanonenrohr entzündete sich die Imprägnierung ihrer
Kugeln und machte sie zu Brandbomben. Sie zerrissen Deck und
Spanten, zerschlugen einen Schornstein, nein, zwei und hüllten das
gegnerische Schiff in dichten Rauch, aus dem erste Flammen
schlugen.

»Nachladen!«

Das Schiff daneben drehte derweil bei und versuchte, sich hinter
sie zu setzen.

»Harpune los«, brüllte Anthony.

Die Männer reagierten sofort. Dante selbst hackte mit einem
Entermesser auf die Taue ein.

Ein Ruck ging durch das Schiff, als es frei kam. Gerade
rechtzeitig. Wie von der Sehne geschnellt, zogen sie an dem
getroffenen Gegner vorbei und gingen so vor dem Zweiten in
Deckung.

»Zusatzsonnensegel!«

Die Männer auf den Segelbäumen hatten nur auf den Befehl gewartet.
Das war MacDonalds letzte Geheimwaffe.

Wie an einer Schnur gezogen nahmen sie Fahrt auf, als das dritte
Gefährt hinter dem Havarierten auftauchte.

Anthony vergaß einen Moment zu atmen.

Im nächsten Moment ließ eine Explosion das Deck unter seinen Füßen
erbeben. Metallene Splitter pfiffen durch die Luft. Eine der
gegnerischen Kanonenkugeln hatte ihr Ziel erreicht und verteilte
ihren tödlichen Inhalt.

»Vorsicht, Captain!« Das war Toms Stimme.

»Sonnenlee«, schrie Anthony. Mehrere heiße Stiche trafen seine
linke Schulter. Flüssigkeit rann warm an seinem Arm herab.

Der Rudergänger gab keine Antwort. Blut tropfte aus seinem Mund und
aus unzähligen kleinen Wunden im Oberkörper. Lautlos brach er über
dem Ruder zusammen. Weitere Treffer erschütterten das Schiff.

Behutsam packte der Kapitän den Rudergänger, zog ihn vom Steuer und
ließ ihn zu Boden gleiten. Tom hatte ihm gerade das Leben gerettet,
indem er an seiner Stelle die Schrapnellsplitter mit seinem Leib
aufgefangen hatte. Bitterkeit erfüllte ihn angesichts des reglosen
Rudergängers. Mit dem Geschmack von Blut im Mund übernahm er das
Steuer, - »Feuer!« -, hörte die eigene Stimme das Chaos mühelos
übertönen. So ruhig, als habe er alles im Griff.

Ihr Heck schwenkte herum, um dem neuen Gegner weniger
Angriffsfläche zu bieten. Zu langsam.

Da erblühte ein Treffer auf dem gegnerischen Gefährt. Noch einer.
Metallsplitter zerfetzten metallene Schiffshaut und menschliche
Leiber.

Das waren nicht ihre Kanonen. Das war …

Anthony drehte sich um.

Ein vierter Æthersegler war aufgetaucht und hatte sich zu ihren
Gunsten eingemischt.

Weitere Kanonenkugeln trafen den gemeinsamen Gegner, der
hoffnungslos zwischen ihnen eingekeilt war. Kein weiterer Schuss
traf das ihre.

Jubelschreie ertönten.

Anthonys Blick suchte das helfende Schiff. Nichts. Da war nichts.
Als hätte es sich in Luft aufgelöst.

Die drei Gegner blieben hinter ihnen zurück.

Eine Tarnvorrichtung! Also doch. Nun wusste Anthony, dass Sophie,
denn nur um diese konnte es sich gehandelt haben, anscheinend
ebenso Geschenke der Franzosen empfing, wie er welche von der
Englischen Krone.

In diesem Moment trat Dante neben ihn. »Captain, Gegner außer
Gefecht gesetzt.« Und dann etwas leiser: »Euer Hemd ist blutig.«

Verwundert sah der Kapitän an sich herunter, registrierte erst
jetzt das Blut und betastete die Schrammen an seinen Rippen und
seiner Schulter. »Schaden sichten und Verwundete verbinden.« Sein
Blick fiel auf den Rudergänger. »Fang mit ihm an.«

»Aye, Captain.«

Anthony fasste das Ruder fester und korrigierte den Kurs. Schmerz
oder Ruhe würde er sich erst erlauben, wenn sie in Sicherheit
waren.

\tb

Der Meteorit war in den Bergen um das Meer der Stille nieder
gegangen. Weit ab von den Bergarbeitersiedlungen, wo nach
Meteoritenerz geschürft wurde, mit dem man die Außenhaut der
Æthergefährte verstärkte. Die Berge waren ein guter Ort zum
Anseilen des Schiffes, wenn dadurch auch ein längerer Fußmarsch
notwendig wurde, um die Einschlagstelle zu erreichen.

MacDonald hatte lange überlegt, ob er Begleitung mitnehmen sollte,
sich aber letztendlich dagegen entschieden. Ein Crewmitglied zu
gefährden, gefiel ihm nicht. Alleine würde er sein Vorhaben
wesentlich leichter umsetzen können, ohne dass falscher Heldenmut
ihm dazwischen funken konnte.

Allein genoss er die Stille und den feinen, knochenweißen Sand, der
unter seinen Schritten nachgab. Selbst in den Bergen bedeckte er
jede Ritze der Steine und Felsen. Eine gute Methode, um sich den
Knöchel zu verstauchen. Jeder Schritt wurde dadurch gefährlich und
anstrengend. Der englische Captain war froh, dass er sich dem Ort
endlich näherte.

Ein Krater tat sich so überraschend hinter einem steilen Bergkamm
auf, der wie das Rückgrat eines fossilen Tiers das Gelände
zerteilte, und brachte Anthony unwillkürlich dazu, stehen zu
bleiben. Der Krater betrug im Durchmesser sicherlich an die 300
Schritte. Der Staub war hier durch die Hitze des Einschlags zu
einer glatten Glasfläche geschmolzen. Der Meteorit selbst war auf
die Entfernung kaum auszumachen.

Vorsichtig begann der Engländer den Hang des Kraters hinunter zu
klettern. Der Boden unter seinen Füßen gab nach, so dass er ein
paar Schritte hinab rutschte. Geröll kullerte über seine Füße und
rollte weiter, bis es nahe des Mittelpunkts liegen blieb.

Als er näher kam, konnte der Kapitän endlich den Klumpen ausmachen,
der dort lag. Er war nur so groß wie eine Kanonenkugel. Lächerlich,
dass ein solch kleiner Stein so viel Wirbel verursacht, dachte
Anthony. Er bückte sich, hob ihn auf und wog ihn in der Hand.

Der Klumpen aus dunklem Silber fühlte sich warm an, fast lebendig.
Er war von schwarzen Adern durchzogen, Löcher, die wie geschmolzen
wirkten, durchsetzten ihn. Dennoch war er unerwartet schwer. Ihn
alleine zurück zum Schiff zu tragen würde ein Knochenjob werden.
Aber besser, als dass jemand aus der Mannschaft sah, was er
geborgen hatte.

Anthony nahm den Rucksack vom Rücken, steckte den Klumpen hinein
und lud ihn sich wieder auf. Die Riemen schnitten tief in seine
Schultern, als Antwort jagte durch die verwundete Linke ein heißer
Stich.

Erst jetzt entdeckte er, dass ein kleiner Brocken von dem
Meteoriten abgebrochen war. Er war kaum größer als ein Kinderring
und hatte ein Loch in der Mitte. Ohne nachzudenken steckte Anthony
ihn in seine Hosentasche. Auf, ermunterte er sich, als Belohnung
kann ich endlich meine Schulden abzahlen und dann kann mich die
Englische Krone kreuzweise.

Er hatte erst den Hang des Kraters hinter sich, als er bereits
keuchte. Der Schweiß rann ihm übers Gesicht. Die linke Schulter
protestierte bereits unter Schmerzen. Er musste sich dazu zwingen,
weiter zu gehen. Erschöpft stolperte er den steilen Bergkamm hinab.
Obwohl er bereits damit gerechnet hatte, hätte er das feine Klicken
fast überhört, aber nicht die weibliche Stimme.

»Antoine, mon cher. Bleib stehen. Du bist ja ganz erschöpft.«

Anthonys Herz machte einen Satz. Langsam drehte er sich um.

Da stand Sophie. In breitstulpigen Stiefeln, Männerhosen und einer
auf Figur gearbeiteten Langjacke, die ihre schmale Taille betonte.
Die Mündung der Repetierpistole, die sie in der rechten Hand hielt,
zeigte unmissverständlich auf seine Brust. So verführerisch und
doch so kaltherzig. Er hatte gehofft, sie hier zu treffen. Aber
nicht so.

»Ich nehme an, dass du nicht alleine bist.« Ich habe doch damit
gerechnet, dass Sophie hier auftaucht, dachte er. Wieso rege ich
mich dann darüber auf? Habe ich etwa gehofft, sie kommt nur, um mir
hallo zu sagen? Gut, dass ich darauf vorbereitet bin. Trotzdem tat
es weh, seine Vermutungen bewahrheitet zu sehen.

Sophie lächelte. »Non, mon amour. Im Gegensatz zu dir.«

Bei ihren Worten hörte Anthony Schritte hinter sich. Aus den
Augenwinkeln entdeckte er die drei Männer, die ihn mit gezogenen
Pistolen umringten. »Wie nachlässig von mir.«

»Oui, mon cher.« Die Französin machte einen Schritt auf ihn zu.
»Ich hoffe doch nicht, dass wir dich überzeugen müssen. Obwohl du
mich verraten hast, möchte ich dir nicht wehtun.«

Die Spitze erwischte ihn eiskalt. Unwillkürlich ballte der
Engländer die Fäuste. »Ich muss dich korrigieren. Du hast mich
verraten.«

»Antoine, lass das. Gib mir einfach den Rucksack und deine Waffen.
S´il vous plait.«

Mit fest zusammengebissenen Zähnen schnallte Anthony die Waffen ab
und ließ den Rucksack zwischen sich und Sophie zu Boden fallen.
Vier zu eins war eine sehr schlechte Ausgangsposition. Zudem hatte
er nicht vor, zu kämpfen. »Was hältst du von einem Geschäft? Halbe
halbe?«

Sophie lachte. »Ah non, mon ami. Für wie dumm hältst du mich?« Sie
winkte mit der Pistole. »Und geh ein wenig zurück. Ich habe nicht
vor, mich von dir als Geisel benutzen zu lassen.«

Langsam und mit einem erzwungenen Lächeln kam er Sophies
Aufforderung nach. Hilflos musste er mit ansehen, wie auf einen
Wink der Französin einer ihrer Männer sich seinen Rucksack auflud
und die Waffen einsteckte. »Und was bekomme ich als
Gegenleistung?«

Die Frau lachte. »Dein Leben. Und das.« Mit zuckersüßem
Augenaufschlag warf sie ihm eine Kusshand zu. »Merci beaucoup.«

Die drei Franzosen umgingen den Engländer in sicherer Distanz und
postierten sich zwischen ihm und die Frau.

»Ach, und versuche nicht, dein Schiff um Hilfe zu rufen. Dante - so
heißt doch dein Maat, nicht wahr? Er war sehr erfreut über das
viele Gold, das meine französischen Auftraggeber für dein
englisches Schiff zu zahlen bereit sind.«

»Dante …« Sprachlos starrte Anthony Sophie an. Nein, durchzuckte es
ihn, das kann nicht sein. Sie hat mich schon wieder verkauft.

»Ah, oui. Der Arme macht sich Sorgen wegen seiner Implantate. Er
braucht eine Operation. Und das ist teuer. Tut mir leid, mon cher.«
Die französische Agentin schickte sich an zu gehen.

»He, du kannst mich doch hier nicht einfach sitzen lassen!«

»Surement. Ich kann. So wie du damals. Aber keine Angst, ich werde
dir Proviant hierlassen, damit du nicht verhungerst, bis jemand
dich findet.« Auf einen Wink Sophies ließ einer ihrer Begleiter
seinen Rucksack vom Rücken gleiten. Einige Meter von Anthony
entfernt stellte er ihn auf den Boden. »Au revoir, mon cher.« Die
Französin schenkte dem Engländer ein letztes Lächeln, bevor sie mit
ihren drei Gehilfen hinter den Felsen zur Linken verschwand.

Sie waren kaum außer Sicht, da zog Anthony den Funkempfänger aus
seiner Jackentasche. Sophie würde nicht damit rechnen, dass er
eines dieser Wunder der Technik sein Eigen nannte. Streng genommen
gehörte es auch der Englischen Krone. Oder Lord Witherby. Das
wusste er nicht so genau.

»Dante, hörst du mich?«

»Laut und deutlich, Captain.«

»Wir hatten recht. Das Schiff mit der Tarnvorrichtung gehört
Sophie. Sie hat mir den Meteoriten abgenommen und ist mit drei Mann
Begleitung Richtung Bergarbeitersiedlung verschwunden. Schneid ihr
den Weg ab.«

Stille antwortete.

»Dante!«

»Captain, ich fürchte, das wird nicht gehen.«

Nein, dachte Anthony, das glaube ich nicht. »Sag mir, dass du kein
Verräter bist.«

In diesem Moment konnte der englische Kapitän sehen, wie sich sein
Schiff langsam über den Rand eines Bergkammes schob. Der Maat ließ
ein Flaggensignal geben: Kommt an Bord! Das galt wohl Sophie und
ihren drei Begleitern.

Obwohl Anthony mit Sophies Verrat gerechnet hatte, tat diese
neuerliche Variante unerwartet weh. »Weshalb«, fragte er mit
heiserer Stimme.

»Weil Ihre feine Miss mir eine Operation in einem französischen
Krankenhaus versprochen hat. Durchgeführt von einem
Spezialchirurgen. Nehmen Sie es mir nicht übel, Captain.«

Und ob ich es dir übelnehme, dachte dieser erbost. »Du solltest
dich bei unserer nächsten Begegnung in Acht nehmen!«

»Aye, Captain. Das werde ich.« Es knackte im Funkgerät. Dann
herrschte Stille.

Weit entfernt konnte Anthony sehen, wie vier ameisengroße Gestalten
auf sein Schiff zu liefen und über eine Hängeleiter nach oben
kletterten. Wie Phantome wurden hinter ihnen die Schemen von vier
getarnten Gefährten im Nachtblau des Æthers sichtbar, verfestigten
sich vor seinen Augen. Zwei davon waren deutlich beschädigt. Es
waren die gleichen Schiffe, die ihnen auf dem Weg zum Mond den Weg
hatten abschneiden wollen. Das vierte gehörte Sophie.

Voller Zorn ballte der englische Kapitän die Faust. Sein Blick
irrte von den Schiffen über den Rucksack mit Proviant zu den fernen
Bergarbeitersiedlungen.

Ich habe ein Funkgerät, erinnerte er sich. Mit etwas Glück hört
mich ein englischer Frachter und nimmt mich an Bord. Ich muss mich
nur beeilen, damit Sophie ihre Beute nicht verkauft hat, bevor sie
meinen Peilsender im Rucksack findet.

Vielleicht war ja doch noch nicht alles verloren.

\tb

»200 000 Goldmünzen. Rien ne vas plus, mon patron.« Sophie schlug
die langen Beine übereinander, so dass ihr Knie durch den Schlitz
ihres roten Abendkleides blitzte.

Das Hotel, in das der englische Stahlbaron sie bestellt hatte,
hatte Stil. Sophie liebte die bemalten Stuckdecken, die geblümten
Tapeten und schweren Brokatvorhänge, die edlen Stilmöbel und die
prachtvollen Blumengestecke auf den Tischen. Erst recht mochte sie
den Champagner, den man ihr angeboten hatte und der leise in dem
glitzernden Kristallglas perlte, das sie in der Hand hielt. Sie
nippte daran und erfreute sich an dem feinen Geschmack.

Der dicke Mann mit dem Monokel auf dem rechten Auge lachte abfällig
auf. Die geplatzten roten Äderchen in seinem Gesicht wiesen auf den
allzu häufigen Genuss illegaler Rauschkräuter von Jupiter hin.
»Meine Geduld ist am Ende. 150000. Und keinen Cent mehr.«

Die Französin ließ genießerisch den Champagner über ihre Zunge
rollen. Bedauernd sah sie auf das leere Glas und stellte es auf den
Tisch. Seufzend stand sie auf. »Dann werden wir wohl nicht ins
Geschäft kommen, mon patron. Ich bedaure.«

150000! Die Franzosen hatten ihr 120000 geboten. Und der liebe
Maxime gab immer ein wenig mehr, wenn sie ihm schöne Augen machte.
Eigentlich war es jammerschade, dass sie den süßen Antoine deswegen
hatte abservieren müssen. Er war hübsch, geschmackvoll und hatte
jede Menge Stehvermögen. Wenn er nur nicht so unzuverlässig und
verlogen wäre.

»Oh nein, nicht so schnell, meine Liebe.« Als hätten sie nur darauf
gewartet, verstellten ihr zwei muskelbepackte Leibwächter den Weg
nach draußen. »Wo ist das Paket?«

Sophie drehte sich zu dem Dicken um. »Kein Geld, kein Paket. So
lautet das Geschäft, mon patron.«

»Schnappt sie euch.«

Bevor Sophie reagieren konnte, packten die beiden Männer sie und
bogen ihr die Arme auf den Rücken, so dass sie vor Schmerz
aufstöhnte. Ihre Knie zitterten. Merde, war alles, was sie denken
konnte.

Der Dicke kam auf sie zu und fasste nach ihrem Kinn. »Ich bin mir
sicher, dass wir noch ins Geschäft kommen, Schlampe.« Seine andere
Hand fuhr in den Ausschnitt ihres Abendkleides und umfasste ihre
Brust.

Ohne zu zögern spuckte die Französin ihm ins Gesicht. »Niemals,
esroc.«

Eine Ohrfeige zerschellte als Antwort in ihrem Gesicht. Sophie
glaubte jeden Finger zu spüren. Ihre Wange brannte.

»Das werden wir sehen, sobald wir mit dir fertig sind, Schlampe.«
Der Dicke zeigte einen Goldzahn in seinem breit lächelnden Mund. Zu
den beiden Männern setzte er hinzu: »Bindet sie an einen Stuhl.«

»Das würde ich nicht tun.« Die Stimme war so vertraut, dass Sophie
nach Luft schnappte.

Antoine schwang sich auf den Sims des offenen Fensters. Mit dieser
lässigen Eleganz, die sie so sehr an ihm liebte, hielt er die
Spitze seines Degens an die Kehle des Dicken. »Lasst sie los!«,
sagte er.

Sophie hätte ihn dafür am liebsten geküsst.

Als die beiden Männer nicht gleich reagierten, verlieh er seiner
Drohung etwas mehr Nachdruck. Blut tropfte aus einer Schnittwunde
am Hals des Dicken, vermischte sich mit den Schweißperlen, die ihm
über das Gesicht rannen.

»Tut, was er sagt«, keuchte der Dicke.

Endlich gehorchten die beiden Leibwächter. Sophie riss sich mit
einem Ruck los und eilte zu ihrem Retter. Einen Herzschlag lang
vergaß sie sich und schlang die Arme um ihn. Ein Blick in das
Dunkel der Nacht hinter ihm zeigte ihr dabei das Seil mit dem
Rucksacksteigmotor, das er anscheinend benutzt hatte, um sich vom
Dach herab zu lassen.

»Bist du bereit?« Auf Antoines Gesicht lag dieses freche Grinsen,
das ihn so unwiderstehlich machte. Er bot ihr seine Linke.

»Oui, mon amour«, sagte Sophie und schwang sich an seiner Hand zu
ihm auf das Fensterbrett. Als sei es selbstverständlich, legte er
den Arm um sie und zog sie an sich. Sie fühlte seine Muskeln unter
dem dünnen Stoff seines Hemdes, roch seinen Duft nach Schweiß und
Dreck und einem herben Aftershave, und genoss die Wärme seines
Körpers. Wider Willen hauchte sie einen Kuss auf seine Wange.

Er erwiderte ihn, steckte mit nachlässiger Eleganz den Degen wieder
ein und deutete einen Gruß in Richtung des Dicken an. »Wir
empfehlen uns.« Sein Griff um Sophies Taille wurde noch eine Spur
fester, die andere Hand umfasste das Seil. Dann stieß er sich mit
ihr ab in die Nacht.

\tb

Sophies Kapitänskabine, wohin sie geflohen waren, war nahezu ebenso
dekadent eingerichtet wie das Hotelzimmer, aus dem Anthony sie
gerettet hatte. Nur ein paar Kerzen erhellten es. Der Duft nach
Veilchen hing in der Luft. Aufgrund des leichten Schwankens wusste
der Captain, dass sie sich bereits in der Luft befanden. Flucht war
also aussichtslos. Aber weshalb sollte er auch fliehen? Schließlich
war er der Löwin in ihre Höhle gefolgt, um das Sternensilber in
seinen Besitz zu bringen. Zumindest der Rucksack mit dem Peilsender
war noch an Bord. Soviel wusste er mit Sicherheit.

Sophie strich durch seine Haare, während ihr Mund sich dem seinen
näherte. »Antoine«, murmelte sie, »mon doux Antoine.«

Die Berührung ihrer beider Lippen jagte einen Schauer durch
Anthonys Körper, erinnerte ihn an die anderen Nächte, die sie
miteinander verbracht hatten. Sie endlich in den Armen zu halten,
endlich wieder bei ihr zu sein, ließ ihn schwindeln. Am Ziel meiner
Wünsche, dachte er, bin ich das? Es war ihm egal. Alles war ihm in
diesem Augenblick egal. Sie war hier. Wirklich und wahrhaftig. Sie
war hier und küsste ihn, streichelte ihn. Und wenn das alles war,
dann war es die Mühe wert gewesen. Hitze nistete sich zwischen
seinen Beinen ein und ließ sein Geschlecht schwellen. Unwillkürlich
drückte er Sophie etwas fester an sich. Seine Hände strichen über
ihren Rücken, hinab zu ihrem Hintern.

Als Antwort glitt ihre Zunge in seinen Mund auf der Suche nach der
seinen. Als sie sich berührten, schien ein Funke überzuspringen,
durchjagte Anthonys Körper und setzte ihn in Flammen. Mit einem
Keuchen schob er sie zu dem breiten Bett mit dem Baldachin. Ein
Ruck und sie lagen auf den seidenen Laken, er auf ihr. Fiebrig
schoben seine Hände ihr Kleid über ihre Schenkel nach oben, während
er den Kuss hungrig fortsetzte.

Sophie wehrte sich nicht. Im Gegenteil. Mit einem leisen Gurren
öffnete sie sein Hemd und seine Hose und zerrte letztere auf seine
Oberschenkel.

Vorsicht, mahnte Anthony sich noch.

Doch als Sophies Hände seinen Hintern streichelten, warf er alle
Bedenken über Bord. Keuchend drängte er sich gegen sie. Seine Hand
suchte nach dem Stoff einer Unterhose, die ihm das Vorankommen
erschweren könnte. Doch alles, was er fand, war glatte Haut.

Sophie stöhnte lustvoll auf und warf den Kopf in den Nacken. Ihre
Finger liebkosten seine Schultern. »Wo warst du so lange, cheri?«,
flüsterte sie in sein Ohr, bevor sie ihn erneut küsste.

»Auf dem Mond.« Zorn mischte sich ganz plötzlich in die Lust. Mit
einem harten Ruck drang Anthony in sie ein. »Wo du …« Noch ein
Stoß. »… mich zurück …« Noch einer. »… gelassen …« Und noch einer.
»… hast.« Anthony hielt inne. Alles in ihm drängte danach
weiterzumachen. Sie zu beackern, bis sie schrie vor Verlangen, um
sich dann erschöpft in sie zu ergießen. Aber der Zorn hatte seine
Sicht geklärt. Zuerst wollte er ein paar Antworten. Das Lederband
mit dem kleinen Meteoritenstein an seinem rechten Handgelenk musste
er dabei gut im Blick behalten. Nicht nur weil er sich davon
Antworten erhoffte, sondern auch weil Sophie dafür bekannt war,
Dinge wie durch Zauberei spurlos verschwinden zu lassen.

Sophie wand sich stöhnend in seinen Armen. »Antoine …«

»Warum?«, keuchte Anthony.

»Die Franzosen … Sie haben mich erpresst …« Sophie schlang die
Beine um ihn und drängte sich an ihn. Ihr Mund näherte sich ihm.
Sanft biss sie in seinen Hals. »Verzeih mir.«

Die Berührung jagte einen neuerlichen Schauer durch Anthonys
Körper. Biest, durchzuckte es ihn, dennoch drückte er sein Becken
noch eine Spur fester an sie. »Du hast meinen Maat bestochen.«

»Wäre es dir lieber gewesen, ich hätte erneut dein Schiff
zerstört?« Sophies Lippen zogen eine feuchtheiße Bahn seinen Hals
hinauf.

Anthony musste all seine Willenskraft aufbieten, um nicht
nachzugeben. »Ich hätte dort verhungern können.«

»Ah non, ich habe dir Proviant zurückgelassen.« Zärtlich begann
Sophie an seiner Unterlippe zu knabbern. Ihre Finger streichelten
seine Schultermuskeln.

Anthony unterdrückte ein Stöhnen. »Ich hatte Glück, dass mich ein
Schiff des Handelssyndikats aufgelesen hat.« Eigentlich war es ein
Frachter seines englischen Auftraggebers gewesen, aber das musste
Sophie nicht erfahren. Immerhin hatte er so die Gelegenheit nutzen
können, um seinem Auftraggeber auf den Zahn zu fühlen.

Seit er die Frachtpapiere und einige Schriftstücke in der Kabine
des Frachterkapitäns gelesen hatte, wusste er mit Sicherheit, dass
sein englischer Auftraggeber Lord Witherby mit dem Sternensilber
nur eines im Sinn hatte: die Macht im englischen Parlament an sich
zu reißen. Zudem war der Kapitän des Ætherschiffes ausgesprochen
auskunftsfreudig gewesen, nachdem er mit Anthony zwei Flaschen
Rotwein und einige Brandys getrunken hatte.

»Zum Glück.« Sophie küsste sein Ohrläppchen.

»Ja, zum Glück für dich.« Aus Anthonys zornigem Knurren wurde ein
leises Stöhnen. »Weshalb hast du deine Beute nicht an die Franzosen
verkauft?«

Sophie lachte auf. »Oh, mon cher. Der Dicke hat mehr geboten.« Ihr
Atem küsste seine Schulter.

Der Dicke war auch nur ein Strohmann Witherbys gewesen, das wusste
Anthony. Dank Scrimgeour. Der anscheinend noch eine Rechnung mit
Witherby offen hatte, so gut wie er über den Adligen informiert
war. »Anscheinend nicht genug.«

»Oui, nicht so viel, wie ich mir erhofft hatte.«

Sophies Hände waren überall. Die Berührung machte ihn schier
wahnsinnig. »Und jetzt?«

»Jetzt?« Sophie sah ihn schelmisch an.

»Wer kriegt das Sternensilber jetzt?«

»Du willst es sehen. Hab ich recht?«

»Ist es hier?« Anthony glaubte zu bersten vor unterdrückter Lust.
Mit fiebrigen Händen schob er die Träger von Sophies Abendkleid
über ihre Schultern herab und legte so ihre kleinen Brüste frei.

Neckend schnappte Sophie mit den Zähnen nach Anthonys Unterlippe.
»Oui, ist es.«

Anthony machte einen Kuss aus Sophies Biss. »Mein englischer
Auftraggeber bietet mehr als dein Dicker.«

Sophie lachte gurrend, während sie mit blitzenden Augen seine Haare
zerzauste. »Die Engländer! Mon ami, bevor ich das Sternensilber an
die Engländer verkaufe, muss eine Menge geschehen.«

»Dann der Meistbietende?« Jetzt, dachte Anthony, ich will sie! Ich
kann nicht mehr … Seine Hände legten sich auf Sophies Brüste, die
nur auf seine Berührung gewartet zu haben schienen. Hart und spitz
schmiegten sich ihre Knospen gegen seine Handflächen.

»Erhält der nicht immer den Zuschlag?« Sophies Hände glitten über
Anthonys Rücken.

Anthony stöhnte wohlig auf, während seine Finger ihre Knospen
umschlossen. »Halbe halbe?«

»Halbe halbe, mon amour. Aber nur, wenn du mich endlich beglückst.«
Sophies Stimme war ein dunkles, tiefes Stöhnen.

»Wie du befiehlst«, flüsterte Anthony.

\tb

Ein Hauch von Bedauern erfüllte ihn, als er das kleine Ruderboot an
den weißen Strand des Südsee-Eilands zog, wo Sophie sich mit ihrem
französischen Mittelsmann treffen wollte. Bedauern darüber, dass
das Objekt der Begierde nun fort war und darüber, dass er Sophie
erneut betrogen hatte.

Hatte ich denn eine Wahl, beschwichtigte er sich.

»Halt! Keinen Schritt weiter!«

Aufgrund des befehlsgewohnten Klangs der Stimme blieb Anthony
augenblicklich stehen und hob die Hände. Mit zusammengebissenen
Zähnen beobachtete er, wie ein Trupp englischer Soldaten in roten
Uniformen ihn umstellte. Der junge Lieutenant, der sie befehligte,
hatte die Pistole auf ihn gerichtet. Sie schienen nur auf ihn
gewartet zu haben.

Verrat, schrien alle seine Sinne. Nahm dieses Hin und Her denn nie
ein Ende? Es schien nur eine Antwort auf die Frage zu geben, wem er
dies zu verdanken hatte. Aber Anthony wollte diese Antwort nicht
wissen.

»Was wollt Ihr? Ich habe mich keines Verbrechens schuldig gemacht.«
Nur das Sternensilber heimlich von Sophies Schiff geschafft und
Scrimgeour die Papiere zukommen lassen, die er auf Witherbys
Frachter gefunden hatte. Aber das konnten diese Lackaffen doch
unmöglich wissen.

»Seid Ihr Captain Anthony MacDonald?«

»Wer will das wissen?«

»Lieutenant Christopher Darby. Im Auftrag der Englischen Krone. Und
nun antwortet mir!«

Die Südseeinsel lockte mit tropischer Schönheit. Die Wellen eines
azurblauen Meers schlugen an einen weißen, mit Palmen bestanden
Strand. Anthony hatte sich bereits ausgemalt, wie er hier den Abend
mit Sophie verbrachte, während er mit dem Boot zurück an Land
gerudert war. Dennoch war er sich immer noch sicher, das Richtige
getan zu haben.

»Ich muss Euch enttäuschen. Mein Name ist Antoine de Rochefort.«

Darby schlug so schnell mit der Pistole zu, dass Anthony viel zu
spät reagierte. Der Knauf der Waffe krachte gegen seine Wange.
Schmerz explodierte in seinem Kiefer. Anthonys Kopf wurde herum
gerissen von der Wucht des Aufpralls. Er schmeckte Blut und
taumelte. Da landete ein Gewehrkolben zwischen seinen
Schulterblättern. Mit einem Stöhnen sackte der Engländer auf die
Knie.

Trotz der Schmerzen taxierte er seine Fluchtmöglichkeiten. Wenn er
es schaffte, einen der Männer, am besten den Anführer, umzuwerfen,
konnte er es vielleicht schaffen, hinter den Hügeln zur Rechten in
Deckung zu gehen. Ein, zwei Schüsse mochten ihn erreichen. Aber
dahinter war er vorerst in Sicherheit.

Nur wohin dann? Zurück zu Sophie? Ohne das Sternensilber?

Wohin sonst? Genau das war seine Absicht gewesen. Mit einer guten
Lüge auf den Lippen, die alles erklären würde. Das Auftauchen der
englischen Soldaten änderte nichts daran, allenfalls machte es
seine geplante Lüge noch glaubwürdiger.

Doch. Es änderte alles. Er wollte es nur nicht wahrhaben.

Eine Hand fasste in Anthonys Haar und bog seinen Kopf zurück. »Wo
ist das Sternensilber, Mister MacDonald?«

Ein Handlanger Witherbys. Natürlich, das war des Rätsels Lösung!
Nur der Lord konnte ihm um an das Sternensilber zu gelangen einen
Trupp Rotröcke auf die Fersen gehetzt haben. Vor Erleichterung
unterdrückte Anthony ein Lachen. »Verkauft. An den
Meistbietenden.«

Während Unglauben sich in dem Gesicht des Lieutenants malte, warf
Anthony sich gegen ihn. Seine Faust landete im Magen des Gegners.
Eine Drehung des Unterarms und der Offizier ließ mit einem
Schmerzensschrei die Pistole fallen. Direkt in Anthonys Hand, der
den Lieutenant herumriss und die Mündung der Waffe auf die
gaffenden Rotröcke richtete.

»Waffen weg«, keuchte MacDonald.

»Tut, was er sagt.« Vom Gipfel des Hügels zu ihrer Rechten drang
eine weibliche Stimme.

Sophie. Die Erkenntnis versetzte Anthony einen schmerzhaften Stich.
Also steckte doch sie dahinter. Wie sonst konnte sie gewusst haben,
wo die Soldaten ihm auflauern würden?

Als er genauer hinsah, entdeckte er, dass sie mit einigen ihrer
Männer vom Hügel aus die Rotröcke in Schach hielt. Langsam ging er
mit dem Lieutenant als Deckung rückwärts, bis er die Französin
erreichte. »Du hast dir Zeit gelassen«, knurrte er. Auf keinen Fall
durfte er sich jetzt sein Misstrauen anmerken lassen.

»Du solltest das nächste Mal einen Zettel hinterlegen, wo du zu
finden bist, mon cheri. Wie soll ich dich sonst retten?«

Anthonys Blick fiel auf den Meteoritenstein an seinem Handgelenk,
der so matt war wie angelaufenes Silber. Der Anblick tat unerwartet
weh. Sie lügt, verriet ihm der Stein, sie war es, die mich an die
Engländer verraten hat. »Ich werde daran denken.« Bei den Worten
versetzte er dem Offizier einen Stoß, so dass dieser Hals über Kopf
einige Schritte den Hang hinab stürzte. Anthony wartete nicht, bis
dieser sich aufrappelte, sondern ging hinter der anderen Hügelseite
in Deckung.

»Ich kriege Euch, MacDonald. Das schwöre ich Euch«, hörte Anthony
den Lieutenant schreien. »Und es ist mir gleichgültig, wer mich
dafür bezahlt, verräterischer Hund.«

Schüsse ertönten.

Verrat bedingt Verrat, durchzuckte es den englischen Captain. Habe
ich etwas anderes verdient? Nun hatte er nicht nur Lord Witherby im
Besonderen, sondern auch die Engländer im Allgemeinen gegen sich
aufgebracht.

Und Sophie.

Vorbei. Alles vorbei.

Ohne sich umzudrehen, begann Anthony zu laufen. Bevor Sophie ihm
peinliche Fragen stellen konnte. Denn dass sie das tun würde, war
ohne Zweifel.

»Antoine, bleib stehen! Antoine!«

Anthony dachte nicht daran. Schüsse pfiffen um seine Ohren. Er lief
Haken wie ein Hase, wohl wissend, dass irgendein Schuss ihn
irgendwann treffen musste. Und wenn schon, dachte er, besser tot,
als in ihren Augen lesen zu müssen, dass sie mich wieder verraten
hat. Der Schweiß rann in seine Augen. Sein Herz hämmerte. Da traf
ihn ein Schlag gegen den Oberschenkel und warf ihn zu Boden.

Einen Herzschlag lang blieb er liegen und rang keuchend nach Atem,
während seine Hand nach dem Bein tastete. Warme Flüssigkeit rann
über seine Finger. Dann kam der Schmerz. Anthony biss die Zähne
zusammen und kämpfte sich auf die Füße. Ich will nicht mehr,
durchzuckte es ihn, ich will nicht mehr davon laufen und lügen und
betrügen. Ich will sie.

»Bleib stehen, mon cher.«

Nach Atem ringend hielt Anthony inne. Eine Faust traf seine Nieren
mit der Wucht eines Dampfhammers. Bevor er auch nur ansatzweise
reagieren konnte, traf ein zweiter Schlag seinen Magen. Während er
auf die Knie sackte, riss ein Schlag gegen das Kinn seinen Kopf in
den Nacken. Stöhnend fand er sich am Boden wieder.

»Wo ist das Sternensilber?«

Ein Tritt explodierte in Anthonys Unterleib. Schwärze überflutete
ihn, spuckte ihn schweißnass und am ganzen Leib zitternd wieder
aus.

Sophie kniete neben ihm und strich sanft durch seine Haare. »Rede,
mon amour. S´il vous plait. Ich will das nicht, Antoine. Sag mir
nur, wo das Sternensilber ist und wir sind wieder Freunde.«

»Halbe halbe«, quetschte Anthony zwischen zusammen gebissenen
Zähnen hervor. Das Sternensilber, natürlich! Wie konnte er das
vergessen? Nur darum ging es doch die ganze Zeit. Fast hätte er
gelacht über seine eigene Dummheit.

»Soll ich?«, fragte eine männliche Stimme.

»Non!« Sophies Stimme erlaubte keinen Widerspruch. Mit gerunzelter
Stirn wandte sie sich Anthony wieder zu. »Halbe halbe. So war es
ausgemacht.«

Halbe halbe? Hatte er sich eben verhört? Oder kam sie ihm
tatsächlich auf halbem Wege entgegen? Ein irrwitziger Gedanke
durchschoss ihn. Eine Möglichkeit, wie sich vielleicht wider alle
Vernunft und gegen jede Wahrscheinlichkeit doch noch alles zum
Guten wenden könnte.

Keuchend begann Anthony sein Wams auf zu knöpfen. Aber seine Finger
zitterten so sehr und waren so glitschig von seinem Blut, dass er
immer wieder an den Knöpfen abglitt.

»Scht!« Ihre Finger fingen die seinen ein. Behutsam öffnete sie die
Knöpfe seines Wamses, bis ihr der Beutel mit Gold entgegen fiel.
Erstaunt sah sie ihn an.

»Halbe halbe«, wiederholte Anthony außer Atem. Es musste einfach
gelingen.

»Halbe halbe.« Sophie wog den Beutel in der Hand. »Wieviel?«

»100000. Dein … dein Anteil.« Eigentlich war das der Vorschuss, den
Anthony von seinem Auftraggeber erhalten hatte. Er hatte ihn
behalten wollen – für ein neues Schiff. Aber manchmal konnte man
nicht wählerisch sein und sich dadurch eine Zukunft mit Sophie zu
erkaufen, war eine Möglichkeit, die er sich nicht entgehen lassen
konnte. Zumal Scrimgeour bereits auf andere Art und Weise bezahlt
worden war. Mit den Papieren, die Anthony auf dem Frachter gefunden
hatte und die Witherby vor dem englischen Oberhaus eindeutig
bloßstellen würden.

»Ich bin entzückt.« Sophie zog Anthony hoch in ihre Arme und
hauchte einen Kuss auf seine Stirn. »Und wer war der Meistbietende,
mon amour?«

»Die … die Deutschen.« Eine letzte Lüge. Nur dieses eine Mal noch.
Erschöpft ließ Anthony den Kopf gegen Sophies Schulter sinken. Sein
Blick fiel auf den kleinen Klumpen Sternensilber, den er an dem
Lederband um das Handgelenk trug. Es war noch genauso stumpf wie
zuvor. Hoffentlich waren Sophies Beziehungen nicht so gut, dass sie
seine Lügen herausfand.

\tb

Das erste, was Anthony fühlte, war Schmerz. Doch der war weit genug
entfernt, dass der Engländer seine Aufmerksamkeit auf seine
Umgebung lenken konnte. Er lag in einem Bett. Der dezente Geruch
nach Veilchen kitzelte seine Nase und eine Hand streichelte sanft
seine Haare.

Blinzelnd öffnete er die Augen und sah in Sophies besorgtes
Gesicht, das direkt über ihm schwebte.

»Mon pour cheri«, flüsterte sie und küsste zärtlich seine Stirn.

»Du hast mich verkauft. An die Engländer.« Trotz seines schlechten
Gewissens schaffte Anthony es, seiner Stimme den nötigen Vorwurf zu
verleihen.

»Oui.« Sacht strich Sophie über seine Wange. »Du musst dich
rasieren, mon cheri.«

Anthony schloss kurz die Augen, um nach Atem zu ringen. Noch wagte
er nicht, dem Frieden zu trauen. »Warum?«

»Tu ne sait pas? Du hast das Sternensilber gestohlen.« Sophies
dunkle Augen glitzerten gefährlich.

»Du irrst dich. Ich habe es verkauft. Für uns beide.« Anthonys Atem
ging schwer. Verzeih mir, dachte er, ich muss bei dieser Lüge
bleiben. Für uns beide.

»Und wo ist dein Anteil?« Kein Streicheln mehr.

»Scrimgeour. Du solltest das am besten wissen.«

»Scrimgeour, der alte Gauner?«

Anthony nickte nur erschöpft mit halb geschlossenen Lidern.

»Du hast ihm deinen Anteil gegeben?«

Wieder nickte er.

»Um deine Schulden abzuzahlen?«

»Ja.« Jede Menge Schulden hatte er bei Scrimgeour gehabt. Sophies
wegen. Aber die waren nun beglichen. Wenn auch nicht mit Geld.

»Du hättest mich einweihen müssen. Es hat nicht viel gefehlt und
meine Männer hätten dich umgebracht. Das war dumm von dir,
Antoine.« Sophies Finger berührten seine Wange. Es lag unerwartet
viel Zärtlichkeit in der Geste.

»Ich werde es mir merken.«

»S´il vous plait, mon amour.«

Anthony sah sie an. «Warum sagst du das?”

»Tu ne sait pas?« Sophie umfasste sein Gesicht.

»Sag es mir.«

»Weil ich möchte, dass du bleibst. Hier. Auf meinem Schiff. Fait
mon partenaire.«

Anthonys Herz klopfte hart und schmerzhaft vor Glück. »Egaux en
droits?«

»Oui.« Sophies Finger liebkosten sein Gesicht.

»Pourquoi?«

»Parce que …« Sophie hauchte einen Kuss auf seinen Mund. »… je
t´aime.«

Anthonys Blick fiel auf den kleinen Klumpen aus Sternensilber. Wie
jedes Mal, wenn Sophie oder er die Wahrheit sagten, leuchtete es in
sanftem Glanz. Nur, wenn jemand in seiner Nähe log, war es stumpf.
War dieser Klumpen nicht um vieles wertvoller als ein neuartiger
Schiffsantrieb? Nein, war das Wissen, dass Sophie ihn liebte, nicht
alles Gold dieser Welt wert?

»Ich liebe dich auch«, flüsterte er.

Der kleine ringförmige Klumpen glänzte immer noch.

»Paix?«

Anthony schloss die Arme um Sophie und zog sie an sich. »Frieden«,
flüsterte er. »Ich verzeihe dir.«

Mit einem Seufzen ließ sich Sophie neben ihm aufs Bett sinken und
schmiegte sich an ihn. »Je suis très heureuse.«

»Ich auch.« Auch ohne auf den Silberring zu schauen, wusste
Anthony, dass er leuchtete. Denn er war wirklich glücklich. Seine
Schulden bei Scrimgeour waren bezahlt, mit dem Wissen über seinen
englischen Auftraggeber anstatt mit Geld. Die englische Krone hielt
ihn für einen Verräter und würde ihn nie wieder zu einem Auftrag
pressen. In seinem Arm lag die Frau, die er liebte. Und das
verfluchte Sternensilber ruhte an der tiefsten Stelle des Meeres,
wo niemand es je finden würde. So dass niemand es für seine Zwecke
missbrauchen konnte. Denn nichts anderes konnte mit einem Material
passieren, das Lüge und Wahrheit anzeigte. Dieses Geheimnis würde
er alleine hüten.

Was wollte er mehr?
\end{document}
