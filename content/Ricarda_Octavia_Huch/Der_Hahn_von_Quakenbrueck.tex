\usepackage[german,ngerman]{babel}
\usepackage[T1]{fontenc}
\hyphenation{wa-rum Fracht-raum}
\hyphenation{schien}
\hyphenation{Tief-ebe-ne Tief-ebe-ne gro-ßen}


%\setlength{\emergencystretch}{1ex}

\begin{document}
\raggedbottom

\author{Ricarda Huch}
\title{Der Hahn von Quakenbrück}
\date{}
\lowertitleback{Diese Ausgabe basiert auf dem 
  \href{http://www.gutenberg.net/}{Project Gutenberg}
  EBook \#27446.}

\maketitle

\pagenum{[9]}Im folgenden wird gemeldet, was die Chroniken über
den staatswichtigen Prozeß wegen des eierlegenden Hahnes
überliefert haben, durch welchen eine freie Reichsstadt Quakenbrück
im Jahre 1650 ängstlich erschüttert wurde und leicht zu gänzlicher
Auflösung gebracht worden wäre.

Es hatte nämlich der Pfarrer an der Heiligengeistkirche, der der
Reformation anhing, mehrere Male auf der Kanzel vorgebracht, daß
der Hahn des Bürgermeisters, wider Natur und Gebrauch, als wäre er
eine Henne, Eier lege, darüber gewitzelt wie auch merken lassen,
daß dergleichen ohne die Beihilfe des Teufels oder teuflischer
Künste nicht wohl zu bewerkstelligen sei. Dies verursachte der
Zuhörerschaft des beredten Pfarrers teils Belustigung, teils
Grausen, und es wurde in den Bürgerhäusern hin und her darüber
geredet, besonders in den Kreisen der zünftigen Handwerker, welche
behaupteten, von Bürgermeister und Ratsherren aus dem Regimente
verdrängt worden zu sein, an dem sie vielerlei auszusetzen hatten.
Allmählich kam es so weit, daß die müßigen Buben, wenn der
Bürgermeister sich auf der Straße blicken ließ, anfingen zu krähen
und zu gackern und mit solchen Bezeigungen unehrerbietig hinter ihm
herliefen. Auch dem Stadthauptmann, der die Kriegsmacht von
Quakenbrück im Namen des Kaisers befehligte und eine\pagenum{[10]}
gewaltige Person war, kam etwas davon zu Ohren, und da er mit dem
Bürgermeister wie auch vorzüglich mit der Bürgermeisterin, Frau
Armida, befreundet war, begab er sich selbst in sein Haus, um ihn
deswegen zur Rede zu stellen. Bevor noch der Bürgermeister nach
Gewohnheit eine Kanne Wein auftragen lassen konnte, setzte sich der
Stadthauptmann auf einen Sessel, schlug auf den Tisch und sagte:
»Tile Stint« – denn so hieß der Bürgermeister –, »das mit dem Hahn
muß aufhören, oder du sollst sehn, daß ich nicht von Pappe bin!«

Tile Stint klopfte dem Stadthauptmann auf den Rücken, als ob er
einen Hustenanfall hätte, und sagte begütigend. »Wenn du mir sagst,
was es mit dem Hahne auf sich hat, so mag es meinetwegen aufhören,
da dir viel daran zu liegen scheint.« »Was,« rief der
Stadthauptmann noch lauter als zuvor, »so willst du zu der
Schandbarkeit deiner Tat noch die Dreistigkeit fügen, sie mir
abzuleugnen, da doch das Gelichter der Gasse ungestraft hinter dir
her kräht.« Diese Worte stimmten den Bürgermeister nachdenklich,
und er sagte: »Das Krähen der mutwilligen Buben ist mir in der Tat
aufgefallen, und es wäre mir lieb, den eigentlichen Grund desselben
zu erfahren. Ich dachte schon, es sei ein Symbolum und diene den
Reformierten, uns Altgläubige damit zu verspotten, doch will ich
sie gern einer derartigen Herausforderung und Tücke freisprechen,
wenn es sich anders verhält.« Der Stadthauptmann runzelte die
Brauen und brummte: »Firlefanz! Solltest du nicht wissen, daß auf
das niederträchtige Eierlegen deines Hahnes gezielt wird?«

Auf diese Insinuierung öffnet Tile von Stint seine matten blauen
Augen voll Staunen, indem er ausrief:\pagenum{[11]} »Der kann Eier
legen! Mache mir das nicht weis! Tun es doch nicht einmal meine
Hühner nach der Ordnung, so daß ich ihn schon habe abschlachten
lassen wollen, da er noch dazu die Federn läßt und schäbig wie von
einer Rauferei daherkommt; aber ich unterließ es, da er wegen
seiner Magerkeit keinen guten Bissen verspricht.«

Das Gesicht des Stadthauptmanns verdüsterte sich, und er herrschte
den Bürgermeister an: »Verlege dich mir gegenüber nicht aufs
Leugnen! Das Mistvieh legt Eier und gehört von Rechtens auf den
Scheiterhaufen. Du weißt, daß ich im Christentum unerbittlich bin
und meine besten Freunde nicht verschone, wenn ich sie bei
Frivolität und Gotteslästerung ertappe. Das Volk muß in Respekt
erhalten werden und an den Regierenden ein Beispiel sehen; deshalb
trage ich dir auf, dafür zu sorgen, daß der üble Leumund von dir
abgewaschen und künftig nichts Ungebührliches mehr von deinem Haus
und Hof vernommen wird, da ich zuvor meinen Fuß nicht wieder auf
deine Schwelle setzen werde.«

Über dies majestätische Auftreten seines Freundes heftig
erschrocken, rief der Bürgermeister: »Erlaubt wenigstens, daß ich
Frau Armida rufe!« und riß heftig an einem Klingelzuge, dessen
Geläut sich indessen noch kaum erhoben hatte, als die Erwünschte
schon in das Zimmer trat. Sie war eine prächtige Frau, die immer in
einem burgunderfarbenen Seidenkleide umherging und eine
hochaufgetürmte, weitläufige Frisur auf dem Kopfe trug, von deren
Spitze ein Kranz von weißen und hellblauen Federn herunternickte.
Infolge eines liebenswürdigen Temperamentes erglomm sie zwar leicht
zu großer Heftigkeit, besänftigte sich aber auch unversehens,
liebte die Geselligkeit und verscheuchte mit viel Geräusch die
\pagenum{[12]}Langeweile und üble Laune, weswegen sie wohlgelitten
und dem Stadthauptmann unentbehrlich war.

»Ihr seid es, Klöterjahn,« rief sie, als sie den erhabenen Mann
erblickte, und wollte mit einer angemessenen Begrüßung fortfahren;
allein der Bürgermeister schnitt ihr die Rede ab, indem er
kläglichen und gereizten Tones ausrief: »Warum meldet man es mir
nicht, wenn solche Unrichtigkeiten im Hühnerstall vorfallen? Du
bist die Hausfrau und solltest es wissen, wer bei uns die Eier
legt! Oder hält man es nicht für nötig, mich von so gröblichen
Mißständen in Kenntnis zu setzen?«

»Ereifere dich nicht!« sagte Frau Armida strenge, denn sie
mißbilligte es, wenn andere heftig wurden; »wenn du selbst nicht
weißt, was du sagst, verstehen es andere noch weniger.« Diese
Entgegnung brachte den Bürgermeister vollends auf, so daß er böse
rief: »Verstehst du nicht, daß es Sache der Hühner ist, Eier zu
legen, wie die der Weiber, Kinder zu gebären!« und hoffte mit
dieser Anzüglichkeit seine Frau zu ärgern, welche ihm keine Kinder
geschenkt hatte. Diese jedoch hielt an sich und lud nur durch einen
funkelnden Blick den Stadthauptmann ein, ihr unschuldiges Leiden zu
bezeugen. »Ich bin ein einfacher Kriegsmann, aber ein guter
Christ,« sagte von Klöterjahn, düster ihrem Blicke ausweichend;
»bevor ihr diesen Schandfleck nicht von euch abgewaschen habt, kann
ich eure Schwelle nicht mehr betreten. Was ich gesagt habe, kann
ich nicht zurücknehmen, also muß es dabei bleiben!« Damit stand er
eisern entschlossen auf und griff nach der Türklinke. »Klöterjahn!«
schrie Frau Armida auf und brauste hinter dem Entweichenden her,
willens, ihn mit ihren Armen festzuhalten, konnte ihn aber nicht
mehr einholen, der gerade die Gartentür\pagenum{[13]} hinter sich
zuwarf und mit starken Schritten sich ihrem Klageruf entzog.

Unterdessen bereute es Tile von Stint schon, daß er gegen seine
Frau ausgefallen war; denn er war keineswegs bösartig, vielmehr
sanft und verträglich, nur hatte er schwache Nerven, konnte Lärm,
Streit und Aufregung nicht vertragen und wurde zuweilen hitzig,
wenn es in seinem Kopfe durcheinanderzugehen anfing. Er besaß einen
mittelmäßigen Verstand, den er von jeher aus Bequemlichkeit nur
selten in Betrieb gesetzt hatte, und nun, seit er alterte und meist
schläfrig war, wie eine gute Stube mit überzogenen Kanapee und
Stühlen vermuffen ließ. Die Ratsgeschäfte liefen mehr oder weniger
von selber, und zu Hause bekümmerte er sich nur ein wenig um den
Garten und die Hühner, hauptsächlich aber um die Küche, in der er
sich gern aufhielt, um an den Töpfen zu schieben und mit der
blonden, rosigen und runden Köchin, welche Molli hieß, liebreich
umzugehen. Nachdem der Stadthauptmann und seine Frau das Zimmer
verlassen hatten, klingelte er sämtliche Dienstleute zusammen und
befragte sie wegen des Hahnes. Es war aus ihnen nichts
herauszubringen, als daß sie von der Munkelei schon vernommen
hatten; übrigens stotterten sie, verdrehten die Augen und kratzten
sich hinter den Ohren, was den Bürgermeister so aufregte, daß er
sie in großem Unwillen wieder fortschickte, sich in einen
Lehnsessel warf und einschlief.

Ganz anders war Frau Armida tätig: sie ließ die vertrautesten
Freunde ihres Mannes zu einem Plauderstündchen am häuslich
beschickten Tische bitten, nämlich die Ratsherren Lüddeke und
Druwel von Druwelstein und den Rechtsgelehrten Engelbert von
Würmling, der nur von den\pagenum{[14]} vornehmsten Familien als
Beistand gewonnen wurde. Es zeigte sich, daß auch diesen Herren das
häßliche Gerede bereits zu Ohren gekommen war, daß sie aber aus
verschiedenen Gründen gegen den Bürgermeister geschwiegen hatten,
der kleine Lüddeke, weil es eine heikle Sache und Tile Stint
vielleicht nicht genehm wäre. Druwel, weil es ihm schien, als wäre
eine Sache noch nicht ganz wahr, wenn man nicht davon spräche,
Würmling dagegen, der italienische Universitäten besucht hatte und
sehr aufgeklärt war, weil es ihm nicht wichtig vorgekommen war.
»Ich glaube nicht, daß ein Hahn Eier legen kann,« sagte er, »tut er
es aber dennoch, so mag er es meinetwegen, ich habe keine
Vorurteile. Es ist außergewöhnlich; gut. Es ist unnatürlich; gut.
Schadet es mir? nein. Überlassen wir es doch alten Weibern, über
Himmel und Hölle, Tugend und Laster zu disputieren.« »Indessen
doch,« wandte Druwel schüchtern ein, »da der Herr Stadthauptmann
seine Ungnade darüber ausgesprochen hat, möchte die Sache noch von
einem anderen Gesichtspunkte aus zu betrachten sein.« Herr
Engelbert schloß die Augen, wie wenn er sich davor behüten wolle,
den Anblick dummer und schwacher Menschen in sich aufzunehmen, und
sagte im Tone der Erschöpfung: »Die Meinung des Herrn
Stadthauptmann ist wohl, dem Volke das Maul zu stopfen, vor dessen
Unverstand und Aberglauben allerdings manches Ungewöhnliche
verborgen bleiben muß.«

Druwel war ein Kriegsmann und hatte sich bei allen Waffentaten der
Stadt hervorgetan, und wenn er daherkam mit steifem Knebelbart,
blitzenden Augen und sonnenverbrannter Haut, dick und steifbeinig
wie ein aufrechter Kanonenlauf, dachte ein jeder, es könne
Quakenbrück nicht fehlen, solange es seinen Druwel habe. Nur in
moralischen\pagenum{[15]} Dingen war er nicht beherzt, weil er
wohl Neigung dafür, aber keine Unterscheidung hatte und sich, so
gut es gehen wollte, nach irgendeinem ansehnlichen Manne, besonders
dem Stadthauptmann von Klöterjahn, richtete. Er hatte immer Angst,
daß er sich unversehens wider die Religion oder das Moralische
verfehlen könnte, ja schon daß er etwas sähe und hörte, was ihn bei
der Beichte in Ungelegenheiten bringen könnte. Der kleine Lüddeke
dagegen, ein munteres Männchen, ließ das Christentum auf sich
beruhen, wenn er nur das Vorschriftsmäßige absolviert hatte, und
freute sich schon des Abends beim Zubettgehen auf die Neuigkeiten,
die der folgende Tag bringen könnte. »Gestrenger,« sagte er, sich
ungeduldig am Bärtchen zupfend, »da wir nun doch einmal
daraufgekommen sind, so führe uns doch in den Garten und zeige uns
den Teufelsbraten, und laß ihn womöglich ein Pröbchen seiner Kunst
ablegen.« Obwohl Druwel zögerte unter dem Vorwande, es dämmere und
man könne doch nichts sehen, öffnete Tile Stint die Tür, um den
Herren voranzugehen: da kam Frau Armida durch dieselbe
hereingestoben und rief zornig, der Gärtner habe gekündigt, da er
in einem solchen Hause nicht bleiben könne, und Molli, die Köchin,
ließe das Trüffelgemüse in der Pfanne verbrennen, um nicht Schaden
an ihrer Seele zu nehmen. Hätte man doch der Bestie, dem Hahn, der
an allem schuld sei, zeitig den Hals umgedreht, wie sie gewollt
habe! Nun werde man heute abend vor leeren Schüsseln sitzen müssen,
oder sie werde kochen müssen, obwohl sie die Hitze des Herdes nicht
vertragen könne. Die ganze Gesellschaft begab sich darauf in die
Küche, wo Molli unter Händeringen erzählte, wie sie fünf Eier
bereits habe wegwerfen müssen, weil das Dotter in denselben nicht
gelb,\pagenum{[16]} sondern karminrot gewesen sei und noch dazu
das Ei fast ganz ausgefüllt habe, wie sie sich darüber bis ins Herz
entsetzt habe und nun die Geschichte glaube, was sie bisher nicht
habe tun wollen, wie sie keines von den verhexten Eiern mehr
anrühren werde und folglich die Trüffelomelette auch nicht zu Ende
bringen könne.

»Molli,« sagte der Bürgermeister sanft, indem er den Arm um ihre
Schulter legte, »was die Eier betrifft, so werde ich sie
zerklopfen, und wenn es mir gerät, hoffe ich von deiner Liebe und
Treue, daß du auch mir beistehst und die Trüffelomelette, die du so
geschmackvoll wie kein anderes Mädchen zu backen verstehst, wie
auch alle anderen Speisen in gewohnter Weise vollendest.« Darauf
teilte er mit ziemlichem Geschick ein Ei, obwohl ihm die Hände
zitterten, teils infolge seiner schwachen Nerven, teils weil Druwel
ihn durch Ziehen am Rocke von dem Geschäfte abzuhalten versuchte.
Als sie ihren Brotherrn so hantieren sah, wurde Molli weich, begann
laut zu weinen und erklärte, den Anblick seines Eierzerklopfens
nicht länger ertragen zu können; da außerdem die von ihm
aufgeschlagenen Eier recht und schlecht wie andere auch waren, nahm
sie ihm den Napf weg und schickte sich an, unter einem Stoßgebet
die Zurüstung selbst wieder in die Hand zu nehmen.

In dieser Zeit hatte Frau Armida ein großes Beil auf einem
Küchentische liegen sehen, bewaffnete sich damit und eilte in den
Garten, was das Zeichen zum allgemeinen Aufbruch gab, da die Herren
nicht zweifelten, sie wolle dem Hahn zu Leibe, und das Gefühl
hatten, als müßten sie eine rasche Tat verhindern. Der kleine
Lüddeke lief so schnell er konnte, und Druwel ließ sich so weit
hinreißen, daß er sie am Schweif ihres rotseidenen Kleides faßte,
um sie\pagenum{[17]} aufzuhalten, während der Bürgermeister und
Würmling langsamer nachfolgten. Eben hatte die Bürgermeisterin die
Tür des Hühnerstalles, der von einem hölzernen Zaun umgeben war,
erfaßt, und da sie glaubte, daß ihr Kleid an einer Latte festgehakt
wäre, suchte sie es ärgerlich loszureißen, wobei sie sich umdrehte
und den Druwel gewahrte, der sie beschwor, den Stall nicht zu
betreten, welcher vielleicht ein Bezirk des Teufels sei. »Wer ein
gutes Gewissen hat, fürchtet den Teufel nicht,« sagte Frau Armida
spitz, riß mit einer scharfen Bewegung ihre Schleppe aus den Händen
des Druwel und trat mit stiebendem Schritt unter die Hühner, die
erschreckt auseinanderflogen. Dem Hahn gelang es, sich mit
Aufopferung einer Schwanzfeder ihrem Griff zu entziehen und, an
einer Scheune hinaufflatternd, die den Hintergrund des Stalles
bildete, eine offene Luke zu entdecken, in der er sich niederließ.

Tile, Lüddeke und Würmling, die inzwischen näher gekommen waren,
versuchten der Frau zu erklären, man dürfe das Tier nicht töten, da
es so ausgelegt werden würde, als hätten sie ein verräterisches
Zeugnis aus der Welt geschafft; aber sie war Belehrungen nicht
leicht zugänglich, wenn ihr Gemüt in Aufruhr war, und forderte die
Herren mit Ungestüm auf, die Bestie herunterzuschießen, wenn anders
sie sie nicht für Feiglinge halten sollte. Herr Lüddeke blinzelte
mit seinen kleinen Augen bald Frau Armida, bald den Hahn an, der in
der viereckigen Luke saß, mit den Flügeln schlug, den Schnabel weit
aufreißend krähte und in der einfallenden Dämmerung größer als
natürlich aussah. »Er hat eine gellende Stimme und abscheuliche
Figur,« sagte er, »und es wäre nicht schade um ihn; allein wenn
Herr von Würmling uns rät, daß wir uns nicht mit\pagenum{[18]}
Übereilungen verdächtig machen, so müssen wir wohl unseren
berechtigten Groll und unsere Verwegenheit einstweilen zügeln.«

»Nun denn,« rief Frau Armida, welche dass Zureden und die Gründe
der Herren wie Wassertropfen an sich ablaufen ließ, »wenn die
Männer kein Herz in der Brust haben, so werde ich dem Federvieh
seinen Lohn geben,« raffte ein paar große Feldsteine auf, die
inmitten des Stalles einen Futtertrog bildeten, und warf sie weit
ausholend nach der Luke. Die Herren sputeten sich, aus dem Bereich
der niedersausenden Blöcke zu kommen, woran sie durch das Lachen
nicht wenig behindert wurden, in das sie über die Heftigkeit der
Dame geraten waren; doch kehrte der gute Tile wieder zurück, um
seine Frau darauf aufmerksam zu machen, daß sie leichter sich
selbst als den Hahn treffen würde. Da ihr das soeben selbst
eingefallen war, verließ sie den Kampfplatz, auf dem das Beil und
die Steine wild umherlagen. »Druwel,« sagte sie streng, indem sie
vor den Herren stehenblieb, »in manchem Korsett steckt ein Held und
in mancher Rüstung eine Memme.« »Das erste«, sagte der Druwel
demütig, »wird niemand bestreiten, der Euch kennt; was mich
betrifft, so ist mein körperliches System derart beschaffen, daß
ich vor geheimen Dingen, als Gespenster, Furien, Miasmen, Seuchen,
Visionen, Erdbeben und Gewittern, eine unüberwindliche, innere
Zurückhaltung und Grausen verspüre, während ein ganzes Kriegsheer
mein Herz nicht um einen einzigen Wirbel schneller schlagen läßt.«
»In Eurem Verzeichnis habt Ihr die Weiber vergessen,« bemerkte Frau
Armida, »und doch habt Ihr Ursache, auch vor ihnen die Augen
niederzuschlagen.« »Von dem Blick einer schönen und edlen Dame
überwunden zu werden,\pagenum{[19]} dessen braucht sich kein Mann
zu schämen,« antwortete Druwel und bot der nunmehr versöhnten
Bürgermeisterin den Arm, um sie in den Speisesaal zu führen.

Die charaktervolle Molli hatte nicht wie die übrigen Dienstboten
dem Auftritt im Garten zugeschaut, sondern war bei ihren Omeletten,
Pasteten und Bäckereien geblieben, so daß eitel Wohlgeschmack und
Üppigkeit die Gesellschaft an der Tafel empfing. Frau Armida, die
noch stark atmete, eröffnete das Tischgespräch, indem sie ausrief:
»Habe ich mich bisher nicht darum gekümmert, so bin ich jetzt
dessen sicher, daß der Bösewicht Eier legt, und schlau muß er es
anfangen, daß wir ihn nie dabei betroffen haben.« Von Würmling
sagte: »Gnädigste haben dem Armen ihre Huld entzogen und halten ihn
nun jeder Übeltat fähig: das ist die Art der Frauen.« »Ei
freilich,« entgegnete sie rasch, »die Art der Frauen ist es, sich
nicht verblenden zu lassen, weder durch ein geschabtes Kinn noch
durch einen langen Bart oder bunte Federn, sondern die schlechten
Faxen zu durchschauen und damit aufzuräumen.« Als sie bemerkte, daß
Herr Lüddeke sie der bedienenden Mädchen wegen durch Zublinzeln und
allerhand Zeichen zur Vorsicht zu mahnen suchte, blickte sie sich
herausfordernd um und sagte: »Warum soll ich in dieser Sache
schweigen, wie wenn ich die Eier gelegt hätte? Wir wollen schon
dahinterkommen und einen Stecken dabeistecken, so daß jedermann mit
unserer Justiz zufrieden sein muß.« Ja, sagte der Bürgermeister, so
sollte es wohl sein, aber die Zeiten wären nicht mehr so, sondern
es herrsche Mutwillen und Unbotmäßigkeit im Volke, es gebe freche
Leute, die sich ungestraft aufbliesen und den höheren Personen
etwas am Zeuge flickten. Der Stadthauptmann habe ihm ernstlich
aufgegeben, das Gerede Lügen\pagenum{[20]} zu strafen, als lege
sein Hahn Eier, wie sollte er das aber anstellen, wenn sein eigenes
Eheweib auf die Straße hinausriefe, daß es wahr sei?

Die Erwähnung des Stadthauptmanns stimmte Frau Armida nachdenklich
und trübe, so daß sie aus Schwermut und wachsender Besorgnis das
Knäuel der Unterhaltung sich entrollen ließ. Indessen wurden Herr
Lüddeke und der von Würmling immer lustiger; der letztere nämlich
fing an, wenn er eine Flasche guten Weins getrunken hatte,
umgänglich zu werden und Witz und Laune spielen zu lassen, wie wenn
das edle Feuerzeug ein Holz anzündete, das zuvor stumm und dumm
dagelegen hatte, nun aber knisterte, wärmte, leuchtete und
Wohlgeruch verbreitete. Sie versuchten auch den Druwel in die
Lustbarkeit hineinzuziehen; der aber, nachdem ihn das Essen zuerst
ein wenig ermuntert hatte, war wieder in Sorgen verfallen, die ihn
so drangsalierten, daß er sich zuweilen den Schweiß von der Stirne
trocknen mußte.

»Du weißt, Tile,« sagte er, »daß ich in allen Gefahren zu dir halte
und ein mannhafter Kriegsoberst immer gewesen bin, es ist dir aber
auch bekannt, daß ich im Christentum heikel bin, und wenn ich einen
Eid habe schwören müssen, am liebsten den Mund nicht wieder
auftäte, geschweige denn, daß ich dagegen anlöge. Wie soll ich mich
denn nun daraus ziehen, wenn ich wegen des Hahnes befragt werde?
Wenn ich auf die Folterbank gelegt und mit glühenden Zangen
gekneipt würde, ließe ich mir bei Gott über dich und das Eierlegen
nichts entschlüpfen; wenn sie mich aber mit drei Fingern gen Himmel
schwören lassen, so ist mir die Zunge wie vom Schlage gerührt und
geht keine Unwahrhaftigkeit mehr darüber.«

\pagenum{[21]}Alle blieben betroffen, nur Herr Engelbert lächelte
und sagte, indem er seinen schlanken blassen Zeigefinger über den
Tisch auf des Druwels Brust zu bewegte: »Habt Ihr denn den Gockel
Eier legen sehen?« Der Druwel rollte erstaunt seine Augen hin und
her und sagte endlich aufatmend mit großer Erleichterung: »Wenn ich
es recht bedenke, so habe ich gar nichts gesehen.« »Nun, so könnt
Ihr aussagen, was Euch beliebt, ohne Euer Gewissen zu verstricken,«
sagte der Rechtsgelehrte, »und die Wahrheit wird uns so wenig
schaden wie Euch die Lüge.« Jetzt brachte der Bürgermeister noch
ein Bedenken vor, nämlich, daß es doch etwa besser gewesen wäre,
das Tier abzutun, denn wenn es in der Untersuchung peinlich mit
Schrauben und Drehen behandelt würde, könnte es durch einen
unglücklichen Zufall doch noch Eier legen, wodurch sie dann ohne
Verschulden häßlich bloßgestellt sein würden; allein der Druwel
winkte heftig mit beiden Armen Schweigen und rief: »Redet mir nicht
mehr von dem verfluchten Viehzeug. Laßt mich über die ganze Sache
im Dunkeln, daß ich so wenig davon weiß wie von der unbefleckten
Empfängnis Mariä! Eure gelehrte Spitzfindigkeit, Herr Engelbert,
mögt Ihr vor dem Tribunal entfalten, einem einfachen Kriegsobersten
wird dadurch nur der Verstand verwirrt. Schenkt nur ein und füllt
mir den Teller, denn vorher hat sich mir jeder Schluck und Bissen
in Galle verwandelt.«

So begann der Druwel das Festmahl von neuem, nachdem die übrigen
bereits abgespeist hatten, und es ergab sich ein lautes Pokulieren
bis in die späte Nacht, wobei die Herren zum voraus ihren Sieg
feierten und beredeten, wie sie den alten Zustand wieder einführen,
den Zünften einen Denkzettel anhängen und die reformierte Sekte
\pagenum{[22]}hinausbefördern wollten, am liebsten durch Feuer und
Wasser, aber aus Mildherzigkeit und anderen Gründen durch
Verbannung, nachdem die Rädelsführer auf dem Markte wacker
ausgestäupt wären.

So plauderten die Herren beim Weine, indessen von weitem greuliche
Wetterwolken gegen sie dahergefahren kamen. Der Pfarrer
Splitterchen war ein unerschrockener und vorwitziger Mann, und da
er nun verklagt wurde, die Herrlichkeit des Bürgermeisters gröblich
verleumdet zu haben, als ob er ein Zauberer und Heide sei,
dermaßen, daß er einen eierlegenden Hahn auf dem Hofe hege, war ihm
keinerlei Beschämung oder Kleinmut anzumerken, im Gegenteil, er
trat noch dreister auf als sonst und führte eine ganze Sippe
seinesgleichen mit sich, die sich gebärdeten, als wollten sie den
Fürsten Beelzebub vom Throne stoßen und die betrogene Welt vom
Schwefelstanke räuchern. Er war annehmlich von außen, kraushaarig
und mager, mit so feurigen Augen, daß es zischte, wenn er sie
umherwarf, dazu voll loser Worte, die wohlgezielt geflossen kamen
wie ein Wasserguß, womit man kranke Gliedmaßen bearbeitet. Er hatte
auch einen rechtsgelehrten Beistand mitgebracht, den er aber nicht
an die Rede gelangen ließ und also füglich hätte daheim lassen
können, wenn er nicht in seinem breiten schimmeligen Gesichte ein
giftgrünes Lächeln versteckt gehabt hätte, das zuweilen anzüglich
herausspritzte und die Gegner zu ihrem großen Schaden und zum
Vergnügen der anderen Partei besabberte. Außerdem waren eine Reihe
von Zunftvorstehern und einige von der Kaufmannschaft gekommen,
welche aus alten Briefen ihr Recht erwiesen, einer solchen
Verhandlung beizuwohnen, während die Ratsherren lieber unter sich
geblieben wären.

\pagenum{[23]}Der Richter, welcher den Vorsitz führte, mit Namen
Tiberius Tönepöhl, hielt es im Herzen mit den Reformierten und
freute sich, wenn den Katholischen etwas aufgemutzt werden konnte,
aber er hatte gleichsam einen Pakt und Blutsbrüderschaft mit der
Gerechtigkeit abgeschlossen, wonach sein eigener Trieb so wohl
eingepfercht war, daß er nicht einmal die Schnauze durch die
Gitterstäbe zu stecken wagte; anstatt dessen war die göttliche
Themis bei ihm behaust und weissagte aus seinem Munde heraus, bis
auf einige Mußestunden, wo das Behältnis einmal aufgetan wurde und
das Herz sich ein wenig tummeln und verschnaufen durfte. Unter den
beisitzenden Richtern befanden sich auch ein katholischer und ein
evangelischer Pfarrer, da die Sache ebensosehr geistlicher wie
weltlicher Natur sei. Tiberius Tönepöhl bot zwar den Übergriffen
der Kirche die Stirn, ließ ihr aber andererseits das ihrige
zukommen und betonte, wenn Gelegenheit war, daß er als ein Laie von
den religiösen Mysterien nichts wisse noch wissen wolle, und jede
Konfession ihre Ketzer verbrennen lasse, soviel ihr zustehe, aber
nicht ein Titelchen mehr.

Tönepöhl eröffnete die Verhandlung damit, daß er sagte, er tue es
nicht ohne Bedauern und Schamgefühl, daß ein hochangesehener Mann,
wie der Bürgermeister und beinahe die höchste Person im
Gemeinwesen, öffentlich eines solchen Greuels habe geziehen werden
können, wie es sei, einen Hahn zu besitzen, der Eier lege. Das
wären anrüchige Dinge, die einen auf den Scheiterhaufen bringen
könnten, wenn er die geistliche Gerichtsbarkeit recht einschätze,
der er übrigens nicht vorgreifen wolle. Was man auch sonst für
Grundsätze haben möge, jeder müsse zugeben, sich mit dem Teufel
einzulassen, sei das Laster aller Laster, wie der Teufel
\pagenum{[24]}der Vater aller Sünde sei, und die Verehrung der von
Gott angeordneten natürlichen Leibesvorgänge deute auf einen
Auswuchs oder Monstruosität des Gewissens, die doppelt abscheulich
an einer Regierungsperson sei, die den Untergebenen beispielsweise
in fleckenloser Tugend voranleuchten solle. Er hoffe aber, es werde
dem Herrn Bürgermeister gelingen, sich von dem peinlichen Verdacht
zu säubern, und wenn der Pfarrer Splitterchen etwa jetzt schon
fühle, daß er in seinen Behauptungen zu weit gegangen sei, so möge
er dieselben sogleich zurücknehmen, was doch besser sei, als
hernach wie ein Ehrabschneider dazustehen. Verleumdung sei von
Moses in den zehn Geboten gerügt und sicherlich ein Haupt- und
Grundlaster, das scharf geahndet werden müsse, und das vorzüglich
Geistliche sich nicht sollten zuschulden kommen lassen. Man wisse
ja wohl, daß die Besorgnis um das Heil des Gemeinwesens
Splitterchen veranlaßt habe, von dem berüchtigten Hahn zu reden; um
so mehr könne er ja zugestehen, daß eben diese feurige Liebe des
Guten zu seiner Vaterstadt ihn hingerissen habe, etwas als Tatsache
hinzustellen, was eine zunächst nur unsicher begründete Vermutung
sei. Es sei freilich tadelnswert, überhaupt nur Anlaß zu einem so
gräßlichen Verdacht gegeben zu haben, aber man müsse bedenken, daß
einer dem Rechte nach auch des Teufels Buhle sein könne, solange es
ihm nicht nachzuweisen sei, und so solle sich niemand aufopfern,
indem er auf eine Wahrheit poche, die nicht ans Licht zu bringen
sei. Er fordere also pflichtgemäß den Pfarrer auf, seine
Unterschiebungen zurückzunehmen und dem Herrn Bürgermeister frei zu
gestehen, was zu gestehen sei; da sonst der Augenblick gekommen
sei, wo die Gerechtigkeit ihre eisernen Füße aufheben und
losmarschieren\pagenum{[25]} und ohne Ansehen der Person den
Schuldigen zermalmen werde.

Sogleich erhob sich der Pfarrer mit einer Handbewegung gegen seinen
Rechtsbeistand, Augustus Zirbeldrüse, des Bedeutens, er möge sich
wegen einer solchen Kleinigkeit nicht bemühen, und sagte freimütig,
daß er den heidnischen Unfug im Hühnerstalle des Herrn
Bürgermeisters bisher nur leichthin angedeutet habe, damit der Herr
Bürgermeister einlenken und die Schweinerei zudecken könne und das
Gemeinwesen nicht dadurch verseucht werde. Er befasse sich nicht
damit, die katholische Kirche anzutasten und die Obrigkeit zu
unterwühlen, teils aus natürlicher Friedfertigkeit, und dann auch,
um den Herrn Stadthauptmann, dem er wie jedermann treu ergeben sei,
nicht zu verstimmen, von dem man wisse, daß er in herzlich
vertraulichen Beziehungen zum Herrn Bürgermeister und seiner
Familie stehe, so sehr, daß er gewissermaßen mit ihm verschwägert
sei. Aus diesen Gründen habe er seine Entrüstung hintangesetzt und
zartsinnig geschwiegen, soweit es mit seiner Pflicht vereinbar
gewesen sei. Ob er ruhig hätte zusehen sollen, wie diejenigen, die
Gottes Gebote in den Staub, ja in den Dreck träten, mächtig am
Steuer säßen, während die guten Handwerker und Bürgersleute, die
ihre in Zucht und schlichter Frömmigkeit erworbenen Eier
verzehrten, das Maul halten und unter jeder Willkür sich ducken
müßten? Er habe trotzdem geschwiegen, solange er es vermocht habe;
nun aber der Bürgermeister ihn nicht verstehen wolle, sondern
trotzig gegen ihn vorrücke, um ihm eine Grube zu graben, der offen
und redlich an ihm gehandelt habe, wolle er denn das aufgeklebte
Blatt von Pietät und Rücksicht vom Munde reißen und die Wahrheit
herauslassen.

\pagenum{[26]}Bei den Worten des Pfarrers, die Beziehungen des
Stadthauptmanns zum Hause des Bürgermeisters betreffend, lächelte
sein Rechtsbeistand Augustus Zirbeldrüse, so daß sein Gesicht einem
auseinanderlaufenden Käse ähnlich wurde, und gab ein leises Pfeifen
von sich, das die Zuhörer kichern machte und ein erwartungsvolles
Schweigen im Saale verbreitete.

Tile Stint, der nicht bemerkt hatte, woher das Pfeifen kam, sah
sich erschrocken und ein wenig verlegen um in der Meinung, es sei
einem aus Versehen entwischt und als eine Unschicklichkeit
peinlich, und er räusperte sich, um zu antworten und zugleich den
kleinen Zwischenfall zuzudecken. Allein von Würmling drehte den
Kopf ein wenig nach ihm und sagte, ohne die Augenlider von den
Augen zu heben, er sowohl wie der Bürgermeister wären recht
neugierig, die Wahrheit kennenzulernen, die nun sollte vorgeführt
werden. Dieselbe sei als ein sprödes Frauenzimmer bekannt, die
viele Propheten und Potentaten vergebens um sich habe freien
lassen, Herr Splitterchen dürfe also billig stolz sein, daß er es
einer so wählerischen Person angetan habe. Freilich sei er ein
verdienstlicher Mann in den besten Jahren und brauche sich als ein
Reformierter auch um das Zölibat nicht zu kümmern.

»Zunächst«, antwortete der Pfarrer keck, »sollen einmal die
Kränzeljungfern und Brautführer antreten, zum Schlusse werde ich
dann die Braut zum Altare führen.«

Da begannen denn die Zeugen hervorzuströmen; es war, wie wenn die
Schleuse eines starken Stromes aufgemacht wird. Zuerst kam die
Köchin Molli, welche das Sacktuch an die Augen drückte und vor
Schluchzen nicht reden konnte, worauf Tiberius Tönepöhl sie einige
Minute weinen ließ,\pagenum{[27]} sodann sie gelinde tröstete,
dann sachte zu fragen anhub, wie sie heiße, wie lange sie beim
Herrn Bürgermeister im Dienst sei, und ob sie mit seinem Hahn
jemals etwas zu schaffen gehabt habe. Bei Erwähnung des Hahnes fing
die Molli, welche sich eben ein wenig erholt hatte, von neuem zu
weinen an und sagte nach erneuerter Tröstung, daß sie die Bestie
einige Male habe abstechen wollen, daß aber der Herr Bürgermeister
solches verhindert habe, weil er zäh und nicht schmackhaft sein
würde. Hier wurde das Verhör durch Augustus Zirbeldrüse
unterbrochen, der sich aufnotierte, daß der Hahn, weil zäh,
vermutlich sehr alt sei, und die Molli fragte, wie lange er sich
schon im Hause des Bürgermeisters befinde.

Auf die Frage des Vorsitzenden, warum sie die Bestie habe abstechen
wollen, besann sie sich eine Weile und sagte, daß es so Sitte sei,
von Zeit zu Zeit das Federvieh abzuschlachten, bevor es zu alt sei,
da sie ja auch dazu da wären und immer junge nachwüchsen; wurde
aber ermahnt, sich an die Wahrheit zu halten und auch ihres Eides
erinnert, da sie unzweifelhaft ein tieferliegender Grund zu der
sonst nicht gewöhnlich an ihr scheinenden Mordlust bewogen haben
müsse. Dies Zureden beängstigte die Köchin, und sie gab errötend
zu, daß sie in der Tat dem Hahne gram gewesen sei, da er eine
häßlich kreischende Stimme habe, von der sie oft vor Tage geweckt
sei. Wegen der Eier sagte sie aus, daß zwar letzthin mehrere Eier
durch eine sonderlich rote Farbe und Ausdehnung des Dotters ihr
Bedenken gemacht hätten, daß sie aber den Hahn niemals beim
Eierlegen betroffen habe, und daß sich etliche Hühner im Hühnerhofe
befänden, denen die vorkommenden Eier ihrer Zahl und Beschaffenheit
nach wohl zugeschrieben werden könnten.

\pagenum{[28]}Der Vorsitzende ging nun dazu über, die Molli zu
fragen, ob im Hause des Bürgermeisters viel Eier verbraucht, und ob
sie im Familienkreise oder mit Gästen genossen würden, und als sie
das letztere bejahte, wer die Gäste wären und wie sie sich
aufführten. Hierüber wurde Molli zornig und sagte, daß zu den
Gästen der Herr Stadthauptmann und der Herr Druwel von Druwelstein
gehörten, und daß diese von niemandem Lehren über ihr Betragen
anzunehmen brauchten, und daß sie, obwohl sie nur eine Köchin sei,
Bildung genug besitze, um zu wissen, daß es ungehörig sei, solche
Fragen stellen, auf welche sie nicht antworten würde. Tönepöhl,
welcher infolge seiner Gerechtigkeit sich niemals ereiferte, sagte:
»Liebes Kind, mir mußt du Rede stehen, als ob ich dein Beichtvater
wäre, sollte ich dich auch noch unziemlichere Dinge fragen, als
diese waren,« worauf Augustus Zirbeldrüse mit quiekender Stimme
einfiel, ihm stehe das Recht zu fragen nicht minder zu, und er
wolle denn auch gleich wissen, wie lange die Gesellschaft gemeinhin
bei Tafel gesessen habe, auf welche Weise Molli die Speisen,
insbesondere die Eierspeisen zubereitet, und ob die Frau
Bürgermeisterin dabei geholfen habe. Die eingeschüchterte Molli
erzählte, wie einmal der Herr Bürgermeister mit eigenen Händen die
Eier zerklopft habe, überhaupt zuweilen in die Küche gekommen sei
und ihr zugesehen habe. Bei diesen Worten hob Zirbeldrüse seinen
dicken Kopf ein wenig aus den Schultern und machte Kikeriki, was er
halb krähend, halb flötend überaus scherzhaft zuwegebrachte, um so
mehr, als er sein Gesicht dabei kaum bewegte und es schien, als ob
der Hahnenkraht wie ein Lebewesen eigenwillig aus seinem Munde
stiege. Nachdem der Pfarrer noch gefragt hatte, ob der Herr
Bürgermeister das Tischgebet\pagenum{[29]} spräche, und ob in
seinen Gemächern Heiligenbilder ständen oder hingen, wurde Molli
entlassen, von den wohlwollenden Blicken Tönepöhls und Zirbeldrüses
begleitet.

Tile Stints übrige Diener sagten aus, daß sie freilich den Hahn
nicht hätten Eier legen sehen, daß er aber etwas Widriges an sich
habe und sie ihm wohl allerlei Unrichtiges zutrauten; ferner, wie
oft der Stadthauptmann zu Besuch gekommen sei, wie oft der Herr und
die Frau Bürgermeister zur Kirche gegangen seien, daß sie keine
Kinder hätten und woran dies etwa liegen könne, was für Aufwand sie
trieben, wieviel Röcke, Unterröcke, Pelze und Hauben die
Bürgermeisterin hätte, daß sie alle ihre Bezahlung reichlich und
pünktlich erhielten und auch sonst, was ins Haus käme, auf den
Heller bezahlt würde.

Danach kamen die Freunde des Hauses, zuerst der Druwel, der sich
vorher mit einem Becher starken Weines Mut getrunken hatte und
deshalb mit gläsernen Augen und blauroten Backen daherkam, so daß
ein mißfälliges Murmeln durch die Reihe der Zunftvorsteher lief. Er
hatte indessen doch zu wenig getrunken und es wollte ihm mit dem
Schwören durchaus nicht glücken; der Schweiß trat ihm tropfenweise
auf die Schläfen, und er mußte um einen Stuhl bitten, wobei er sein
Alter, die Gicht und die ausgestandenen Feldzüge vorschützte. Wegen
des Hahnes wollte er sich von vornherein entschuldigen, daß er
durchaus nichts davon wisse und verstehe, überhaupt ein einfacher
Kriegsmann sei; allein der Vorsitzende erklärte ihm lächelnd, daß
er nur auf jede einzelne Frage der Wahrheit gemäß antworten müsse,
und da wurde er denn freilich ärger bedrängt, als er sich hatte
träumen lassen. Bald hatte er zugegeben, daß Frau Armida den Hahn
habe umbringen wollen, daß sie\pagenum{[30]} durch unüberwindliche
Abneigung dazu angetrieben worden sei, und daß der Bürgermeister
sie daran verhindert habe. Vollends aber machte es jedermann
stutzig, daß es der Frau Armida trotz ihres festen Willens nicht
gelungen war, den Hahn zu töten, was nach der Aussage mehrerer
Sachverständiger, die sogleich herbeigerufen wurden, kein schweres
Geschäft sei, sondern durch Halsumdrehen von jedem Kinde könne
bewirkt werden. Bei dieser Gelegenheit erhob sich Zirbeldrüse und
verlangte, daß die Köchin Molli noch einmal vorgeladen werde, damit
man erführe, ob es bei Bürgermeisters üblich gewesen sei, das
Geflügel durch Steinewerfen zu töten, widrigenfalls es sehr
auffallend und belastend sei, daß Frau Armida sich zu einer so
mühsamen und umständlichen Beförderungsart entschlossen habe.

Tönepöhl, der Vorsitzende, war mit dieser Wendung unzufrieden, weil
er bemerkt hatte, daß Zirbeldrüse auf Molli eine ebenso große
Zuneigung geworfen hatte wie er selbst, und zum Anwachsen eines
solchen Gefühls keine Gelegenheit bieten wollte, zumal er auch
fand, daß zu dergleichen verliebten Einfädelungen das Gericht in
seiner Würde der Ort nicht sei, und lehnte daher ab mit der
Begründung, ein jeder habe sich aus den Tatsachen, die Druwel von
Druwelstein beigebracht habe, genugsam seine Meinung bilden können;
denn wenn die Frau Bürgermeister häufiger Hühner durch Steinwürfe
getötet habe, beziehungsweise habe töten wollen, so würde es ihr
entweder bei dem Hahne besser gelungen sein, oder sie würde es
wegen der Ergebnislosigkeit für den gemeinen Gebrauch längst
aufgegeben haben. Während sich alle über den Scharfsinn des
Tönepöhl wunderten und freuten, ärgerte sich Zirbeldrüse dermaßen,
daß er grün anlief, und es bildete sich verdeckterweise eine
grimmige\pagenum{[31]} Feindschaft zwischen beiden, die sich nun
als Nebenbuhler erkannten.

Der Druwel wurde noch mehrere Stunden lang ausgefragt, erstens über
das Verhältnis des Stadthauptmanns zum Bürgermeister, über des
letzteren kirchliche Gewohnheiten, ob er die Fasten halte, ob er
zuweilen Ablaß kaufe, dann aber auch über seinen eigenen
Lebenswandel, wieviel Wein er im Keller habe, ob er schon einmal
Lotto oder Würfel gespielt habe und dergleichen mehr, so daß er, zu
Hause angekommen, sich auf der Stelle zu Bette legte und nicht mehr
zum Aufstehen zu bewegen war.

Nachdem alle Freunde des Bürgermeisters sowie alle Händler, die ihm
Waren lieferten, und alle Ratsangestellten vernommen waren, kamen
zum Schlusse noch ein Nachtwächter, welcher den Hahn des öfteren
zur unrichtigen Zeit, nämlich um Mitternacht statt um drei Uhr,
hatte krähen hören, und ein Dieb, welcher vor etwa einem Jahre in
einem dem Bürgermeister benachbarten Hause hatte einbrechen wollen
und jetzt seine Strafe im Gefängnis verbüßte. Dieser sagte aus, daß
in jener Nacht alle Fenster im Hause des Bürgermeisters erleuchtet
gewesen wären und ein großer Schall von Bankettieren in den Garten
und auf die Straße gedrungen wäre, daß es einen recht
gotteslästerlichen Eindruck auf ihn gemacht habe und er in Zweifel
gefallen sei, ob er sein Vorhaben ausführen solle, da doch nebenan
so viele Menschen wach wären. Er wäre aber doch dabei verblieben,
weil er sich gesagt hätte, daß in einem solchen Taumel und
Hexensabbat keiner auf sein gelindes Wesen merken würde, wie es
denn auch wirklich geschehen sei, so daß alles gut herausgekommen
wäre, wenn nicht im Hause, wo er es vorhatte, die Leute
\pagenum{[32]}durch ein schreiendes Kind auf ihn aufmerksam
geworden wären.

Hiermit, sagte der Vorsitzende, könne man wohl das Zeugenverhör
schließen. Es hätten sich zwar noch an hundert gemeldet, die
merkwürdige Dinge über den Bürgermeister und ihn Betreffendes
vorzubringen versprächen, er glaube aber, es sei nun übergenug
Stoff gesammelt, daraus man sich ein Urteil bilden könne, und er
wolle es dabei bewenden lassen damit der Prozeß doch einmal zu Ende
käme und auch übrigens wieder Gerechtigkeit gepflegt werden könne.
Etwa käme es noch in Frage, ob man den Stadthauptmann vorladen
solle, was er als ein tapferer und gerechtigkeitsliebender Mann
ohne weiteres tun würde, wenn dadurch mehr Licht in eine vorhandene
Dunkelheit gebracht würde. Er seinerseits sähe aber hell genug,
womit er indessen den anderen Richtern oder dem Kläger und
Beklagten nicht vorgreifen wolle. Da niemand in betreff des
Stadthauptmanns etwas wünschte, wollte oder meinte, erteilte am
folgenden Tage der Vorsitzende dem von Würmling das Wort, damit er
die Klage seines Klienten noch einmal kurz und faßlich begründe.

Herr Engelbert, der während der Zeugenvernehmung meist das blasse
spitzbärtige Gesicht in die schlanke Hand gestützt dagesessen
hatte, als ob er schliefe oder an etwas anderes dächte, öffnete die
Augen ein wenig und setzte auseinander, daß der Pfarrer überhaupt
höchst unbefugterweise auf der Kanzel etwas gegen den Herrn
Bürgermeister vorgebracht hätte, da den Reformierten das Predigen
nur unter der Bedingung gestattet wäre, daß sie sich in allen
Stücken ruhig und gehorsam verhielten und weder durch Tat noch
durch Wort sich gegen eine hohe Obrigkeit aufsässig zeigten,
\pagenum{[33]}welches zu beweisen er mehrere Erlasse aus
vergangener Zeit vorlas. Auch gab er einen schönen Abriß der
Verfassung und der Rechte von Bürgermeister und Ratsherren, welche
die Untertanen zu nichts anderem als zu schuldigem Gehorsam
verpflichteten, der durch den Pfarrer gröblich verletzt war, und
gab verschiedene Beispiele, wie in vergangener Zeit vorwitzige
Gesellen wegen loser Worte enthauptet oder gevierteilt wären,
welches zu beweisen er wiederum einige Abschnitte aus den Büchern
der Stadt vorlas. Da es nun den Untertanen und den reformierten
Pfarrern insbesondere verboten sei, der Obrigkeit etwas
Schmähliches vorzuhalten oder nachzusagen, selbst wenn es wahr wär,
so sei es über allen Ausdruck verbrecherisch und gemeingefährlich,
wenn dasselbe erfunden und erlogen sei; und das sei eben hier der
Fall. Der Bürgermeister sei über sechzig Jahre alt und in Ehren
ergraut, habe öfter kommuniziert und gebeichtet, sich niemals gegen
die Kirchenzucht verfehlt und wanke dem Grabe zu, so daß es jeden
rühren müsse, und es sei von vornherein widersinnig, einen solchen
Mann mit verdächtigem Teufelswerk in Verbindung zu bringen. Die
Hauptsache sei aber dies, daß das Eierlegen des Gockels nimmermehr
als bewiesen zu erachten sei, da er weder von irgend jemand dabei
betroffen sei noch auch vor versammeltem Gerichtshof eine Probe
seiner Unnatur abgelegt habe.

»Ei,« rief der Pfarrer aufspringend, »da möchte wohl jeder
Kirchenschänder und Muttermörder frei ausgehen, wenn die Richter an
seine Übeltat nicht glaubten, bis er sie in ihrer Versammlung als
ein Schauspiel vorgestellt hätte! Ist die Natur dieses Basilisken
nicht genugsam durch die hundertfachen Aussagen so vieler argloser
Menschen dargetan? Hat nicht eine unverdorbene Jungfrau, die Köchin
\pagenum{[34]}Molli, aus deren tränenden Augen abzulesen war, wie
ungern sie wider ihren Brotherrn zeugte, ihren unüberwindlichen
Abscheu vor der heillosen Bestie gestanden? Haben nicht alle, die
mit ihm in Berührung kamen, wes Alters, Standes und Geschlechtes
sie waren, dasselbe unerklärliche Gefühl des Grauens, gleichsam
einen inneren Warner, im Herzen gespürt? Hat nicht die
Bürgermeisterin selbst die Höllenausgeburt mit feindlichen Gefühlen
verfolgt, die sich bis zu einer der weiblichen Natur sonst fremden
Mordlust vergifteten? Selbst wenn der satanische Vogel niemals mit
Erlaubnis zu sagen Eier gelegt hätte, muß es doch jedem klar
geworden sein, daß er dies und noch viel anderes vermöchte, seiner
Abkunft und Konnexion, die ich nicht näher bezeichnen will,
gemäß.«

An dieser Stelle brüllte Augustus Zirbeldrüse so laut, daß ein
allgemeines Lachen und Beifallklatschen entstand und der Redner
erst nach einigen Minuten fortfahren konnte.

»O, schweigen wir«, rief er mit edler Betonung, »von diesen
unnennbaren, unkeuschen und unflätigen Dingen, da wir den
Unschuldschnee der Volksseele schon allzusehr mit Schlamm
durchmistet haben! Wie ungern habe ich meine Stimme in dieser Sache
erhoben! Wie leicht und lieblich ist es, die Nase wegzuwenden, wenn
wo Gestank ist. Uns Prediger aber hat Gott berufen, die Gemeinde
vor Übel zu bewahren, und uns mit einem wundersamen Harnisch
gerüstet, daß wir den Mächtigen der Erde furchtlos als Angreifer
und Entlarver entgegentreten. Liebe Freunde, ich weiß, daß die
Besten unter euch schon lange mit Murren zugesehen haben, wie das
Volkswohl, unbeachtet am Karren der Regierung hängend, durch den
Kot geschleift wird. Wir haben tüchtige\pagenum{[35]} Männer
genug, die zugreifen und die Ordnung herstellen könnten, die
löblichen Meister der Gilden, die Herren Bäcker, Kürschner,
Kupferschmiede und Gewürzkrämer, mit Herzen und Händen, die in
Entsagung und ehrlicher Arbeit geläutert sind, das Steuer zu
drehen; aber sie scheuen den Aufruhr und warten, bis das Maß voll
ist. Liebe Freunde, wir haben gehört, was für Aufwand im Hause des
Bürgermeisters getrieben wird. Wir wissen, wie überflüssig mittags
sowohl wie abends seine Tafel besetzt ist. Von dem übermäßigen
Eierverbrauch will ich nicht reden; aber führen wir uns noch einmal
alle die Speisen vor, die das zahlreich zusammengetriebene Gesinde,
im sauren Frondienst schwitzend, von früh bis spät herstellen
mußte: da folgen sich die mit Wein und Nelken gewürzte Suppe, die
Pastete voll Trüffeln, die schwer mit Äpfeln und Rosinen gespickte
Mastgans, der üppige Kapaun, der zartblätterige Salat, das
Mandelgebäck und die aus Pistazien, Mandeln und anderen fremden
Zutaten wie Mosaik gemusterte Magenmorselle. Und alle diese
Leckerbissen sind bezahlt! Bezahlt sind die Muskateller und
Malvasier, das böhmische Glas und der russische Hermelin! Wovon?
Das würde ein Rätsel bleiben, wenn die Lösung nicht in einer
anderen häßlichen Frage läge: Warum wächst der nördliche Turm der
Hundertjungfrauenkirche nicht, zu dessen Vollendung seit Jahren
unter der Bürgerschaft gesammelt wird? Da prahlt wohl ein
Baumeister mit seinen Plänen, da steigen Maurer an den Leitern auf
und nieder, da ist seit Jahren das Hauptportal mit Gerüsten
verstellt; aber an dem Turme ändert sich nichts, als daß ein Jahr
ums andere ein neues Kränzlein von Steinen auf die alten kommt.
Laßt mich nebenbei bemerken, daß die Hundertjungfrauenkirche, wie
\pagenum{[36]}schon in ihrem abgöttischen Namen liegt, der
katholischen Konfession vorbehalten ist, wir also einen
selbstischen Zweck an ihrer Vollendung nicht haben können und uns
nur aus unparteilicher Gerechtigkeitsliebe um eine diesbezügliche
Verwahrlosung und Unterschleif bekümmern. Diejenigen, die mich des
Parteihasses bezichtigen und wohl selbst dessen voll sind, werden
überzeugt sein, ich lachte in mir hämisch und schadenfroh, wenn ich
die Münstertürme der Papisten wie vom Blitz geköpft oder wie im
Frost verkohlte Strünke dem Untergang anheimfallen sehe. Nein,
meine Lieben, wo immer ich Mißstände und Treulosigkeit erblicke,
unter denen das Gemeinwesen leidet, rühre ich mich, dem Arzte
vergleichbar, der, wenn es an seinem Glöckchen läutet, sei es auch
um Mitternacht und zur Winterszeit, aus dem lauschigen Federbett
springt und über die dunklen Straßen durch Tümpel und Pfützen der
Pflicht nacheilt, die mit bescheidenem Lämpchen voranleuchtet an
das Wochenbett, an das Sterbelager, manchmal auch zu Besessenen,
die sich, unter dem Zwang ihres teuflischen Schmarotzers, gegen
den, der es gut mit ihnen meint und das Übel austreiben will, mit
Beißen und Kratzen zur Wehr setzen~…«

Weiter konnte der Pfarrer nicht reden; denn das Jauchzen und
Lebehochrufen der Gildenmeister und anderen Zuhörer verursachte ein
solches Geräusch, daß seine tapfere Stimme nicht mehr
hindurchzudringen vermochte. Als er sich wieder vernehmlich machen
konnte, wiederholte er den letzten Satz und fügte noch mehrere voll
rühmlicher Gesinnung hinzu, worauf er mit den Worten schloß: aus
allem diesem erhellte wohl für jeden, daß der Hahn des
Bürgermeisters wider göttliche Ordnung Eier lege, was er oben
behauptet habe, zu welcher Behauptung er, da sie gewissermaßen
\pagenum{[37]}wahr sei, nicht nur berechtigt, sondern sogar
verpflichtet gewesen sei, und wodurch er sich um den Bürgermeister,
für den es vielleicht noch Zeit sei, seine Seele zu retten,
verdient gemacht zu haben glaube.

Der Pfarrer hatte noch nicht ausgesprochen, als er von allen Seiten
unter Händeklatschen beglückwünscht wurde, da niemand mehr an
seinem Siege zweifelte. Eben forderte der Vorsitzende die anderen
Richter auf, sich mit ihm zur Findung des Urteils zurückzuziehen,
was sie, wie er bedeutsam fallen ließ, nun nicht mehr viel Zeit
kosten würde, als etwas Unerwartetes eintrat, das dem Verlaufe der
Sache eine andere Wendung gab.

Unter dem erweichenden Einfluß der sehnenden Liebe nämlich schien
es dem Stadthauptmann bald, als sei er allzu grausam gegen Frau
Armida gewesen; da er aber doch an seinem Worte, dem eine gewisse
Heiligkeit innewohnte, unerschütterlich festhalten mußte, ergrimmte
er gegen den Pfarrer, der das ganze unnütze Lärmen verursacht
hatte. Wie sich im Laufe des Prozesses merken ließ, daß der
Bürgermeister mit seiner Anklage abprallte, dagegen selbst und
vielleicht auch Frau Armida in eine gefährliche Malefizsache
geriet, wurde sein Zorn unbändig, und er schalt insgeheim auf seine
eigene Langmut, mit der er den Aufruhrgeist im Volke sich hatte
ausbreiten lassen, anstatt es von vornherein mit scharfen Mitteln
zu Bescheidenheit und Gehorsam anzuhalten. Da er ohnehin mit dem
Bischofe von Osnabrück, einem ausnehmend feinen Manne, Geschäfte
abzumachen hatte, reiste er zu ihm und stellte ihm die
Angelegenheit vor, ließ auch einfließen, wieviel ihm daran läge,
wenn der Bürgermeister aus der Falle gezogen würde, dem
reformierten Pfarrer und seinem Anhang aber eine\pagenum{[38]}
merkliche Belehrung für die Zukunft erteilt würde. Aus diesem
Grunde geschah es, daß der Bischof mit einem Male in den
Gerichtssaal zu Quakenbrück trat und begehrte vernommen zu werden,
da er etwas Wichtiges in der Sache des Herrn Bürgermeisters
auszusagen habe.

Die plötzliche Erscheinung des Kirchenfürsten wirkte so erbaulich,
daß einige auf die Knie fielen, die anderen wenigstens sich tief
und eilfertig verbeugten; einzig Pfarrer Splitterchen blieb
aufrecht stehen, und der von Würmling verneigte sich nur mit den
Augenlidern. Auf Grund seiner Vorurteilslosigkeit und
Gerechtigkeitsliebe zögerte Tönepöhl nicht, den Bischof in
höflichen Worten zum Sprechen aufzufordern, ja sogar ihm im voraus
für sein Kommen zu danken, falls er etwas Förderliches in diesem
schwierigen Handel beizubringen habe. Nachdem sich der Bischof, der
ein beleibter Mann war, mehrere Male nach rechts und links
umgesehen hatte, wurde ihm ein Sessel herbeigerollt, in den er sich
mit Anmut niedersetzte, und von dem aus er nun behaglich um sich
blickte und dem und jenem zulächelte, der ihm bekannt vorkam.
Unterweilen zog er eine funkelnde Schnupftabakdose hervor und sagte
lächelnd: »Euer Pflaster ist holperig, ich habe meinen Wagen am
Tore stehenlassen und mich in einer Sänfte hertragen lassen; so bin
ich zwar anständig hereingekommen, aber die guten Leute, die mich
trugen, ließen die Zunge zum Verdampfen aus dem Munde hängen, denn
sie mußten springen, damit ich zu rechter Zeit käme, und dazu zeigt
der Kalender noch den Hundsstern an.« Nachdem er sich noch einige
Male nach rechts und links umgesehen hatte, brachte man ihm auf
einem Brett eine Flasche Wein nebst einem Glase, das man auf ein
Tischchen neben ihm stellte, so daß er nun bequem und vergnüglich
eingerichtet\pagenum{[39]} war. »Es trifft sich gut,« sagte er,
indem er das Glas in die Hand nahm, »daß heute kein Fastentag ist,
sonst würde ich mir diesen Labetrunk versagen,« und ging dann
allmählich zu der schwebenden Sache über, indem er folgendes
erzählte: Er sei vor einem Jahre, um einen Ablaß für den Turmbau zu
verkünden, in Quakenbrück gewesen und habe bei der Gelegenheit Haus
und Hof des Bürgermeisters samt allen Bewohnern, Mensch und Vieh,
geweiht, und dieser Segen habe auch den fraglichen Hahn getroffen,
welcher dadurch entweder des teuflischen Charakters ledig geworden
sei oder niemals dergleichen an sich gehabt habe, da er sonst der
Weihespende ausgewichen sein würde, wie es böser Geister Sitte oder
Unsitte sei.

Tönepöhl unterdrückte eine leichte Verlegenheit und sagte, er wisse
als Laie in weltlichen Dingen besser als in kirchlichen Bescheid,
allein er achte auch die letzteren und sei fern davon, etwas in der
Kirche zu Recht Bestehendes antasten zu wollen. Hochwürden möge
ausdrücklich feststellen, ob wirklich der fragliche, des Eierlegens
bezichtigte Hahn und nicht ein anderer sich unter dem Geflügel
befunden habe, dem der Bischof die Weihe gütigst habe angedeihen
lassen. Ein Hahn sei dabeigewesen, sagte der Bischof leutselig, ein
hübsches Tier von stattlichem Betragen, der ihm wegen seines
übermäßig geschwollenen Kammes aufgefallen sei; er habe damals
diesen Kamm mit der päpstlichen Tiara verglichen und den Hahn
scherzweise Seine Heiligkeit genannt, wessen sich namentlich die
Frau Bürgermeisterin gewiß noch entsinnen würde.

Daß der Bischof mit so gewaltigen Dingen tändelte, machte auf
Tönepöhl, der ein Freigeist war, sich dessen aber doch nicht
getraut hätte, einen bedeutenden Eindruck, so daß\pagenum{[40]} er
begann, den Bischof als seinesgleichen zu bewundern. Er lächelte
ein wenig und sagte, daß man die Frau Bürgermeisterin gern hören
würde, wenn es ihr belieben sollte, der Darstellung des Bischofs
ihre Glossen hinzuzufügen. Als dann die Dame in ihrem
burgunderroten Kleide wie ein Windessausen dahergefahren kam,
winkte er nach einem zweiten Sessel, da der Bischof Miene machte
aufzustehen und ihr den seinigen anzubieten, wobei er sich aber
etwas langsam und schwerfällig bewegte.

Frau Armida dankte kurz mit Kopfnicken und sagte, daß der Hahn, der
die Weihe des Bischofs empfangen habe, derselbe sei, welcher jetzt
von Lästerzungen schmählich besudelt werde, leide keinen Zweifel;
denn sie besäßen ihn seit zwei Jahren und hätten inzwischen keinen
anderen gehabt. Es würde dann wohl das beste sein, den Hahn selbst
herbeizuholen, damit der Bischof ihn anerkennte und auch die
Richter ihn in Augenschein nähmen, ob etwas Verdächtiges an ihm zu
vermerken sei.

»Es soll mich freuen, das gute Tier wiederzusehen,« sagte der
Bischof liebenswürdig. »Und wie wäre es,« meinte er, »wenn man, um
ihn zutraulich zu machen und des Vergleiches wegen, ein paar Hühner
vom Hofe des Herrn Splitterchen dazu lüde? Es wäre merkwürdig zu
sehen, wie diese, die zweifelsohne natur- und ordnungsgemäße Hühner
sind, sich mit dem übelbeleumdeten Hahn vertragen, ob sie etwas
Anrüchiges an ihm wittern, oder ihn als einen tauglichen Hahn und
Herrn zulassen.«

Splitterchen erwiderte mit beißender Freundlichkeit, er wolle mit
seinen Hühnern nicht zurückhalten, halte aber dafür, daß es ein
schlechtes Appellieren sei von menschlicher Vernunft zu
tierischer.

\pagenum{[41]}»Nun,« entgegnen der Bischof, »es wird ja nichts
anderes von ihnen verlangt, als daß sie den Bösen wittern, wozu
man, meine ich, weder des Verstandes noch der Vernunft bedarf,
sondern des einfältigen Instinktes, womit die Tiere vorzüglich
behaftet sind.«

Nachdem noch einige Reden dieser Art zwischen den Parteien
gewechselt waren, entschied Tönepöhl, daß der beschuldigte Hahn den
Splitterchenschen Hühnern sollte konfrontiert oder
gegenübergestellt werden, jedoch erst am folgenden Tage, da die
Mittagsstunde sogar schon vorüber war und anzunehmen stand, daß
alle, besonders aber der Bischof, der unaufhaltsam gereist war,
einer Erfrischung bedürftig wären.

Inzwischen hatte Molli gekocht und gebraten, damit dem Bischof eine
ziemliche Bewirtung vorgesetzt würde. Während des Mahles wurde dem
hochwürdigen Manne ein Brief des Herrn von Klöterjahn überbracht,
der sehr vertraulicher Natur war, und nach dessen Lesung er sagte,
daß der Stadthauptmann bald wieder mit Freuden in diesem Hause
verweilen würde, wie denn jetzt schon sein gerechter Unwille sich
ein wenig verkühlt hätte und er dem Bürgermeister seine volle Liebe
und Gnade wieder zuwenden würde, wenn derselbe sein Christentum
sauber gereinigt vor aller Augen könnte glänzen lassen. Nachdem der
Bischof sich über den schönen Glaubenseifer des Stadthauptmanns,
über den unbotmäßigen Geist der Untertanen und der Reformierten
insbesondere und die Notwendigkeit, solchen zu dämpfen,
unwiderleglich geäußert hatte, ging er zu den auserlesenen Speisen
über, die wie die Sterne am Himmelsgewölbe nach einer weisen und
festen Anordnung die Tafel umliefen, erkundigte sich nach der
Herstellung der einen oder anderen bei der Hausfrau\pagenum{[42]}
und sprach den Wunsch aus, der verdienstvollen Molli seine
Zufriedenheit selbst in der Küche auszudrücken.

Da man sich am Schlusse der Traktierung dorthin begab, stand das
Gesinde am Wege aufgereiht und begehrte den Segen des Bischofs,
dessen Herablassung bekannt war; dazu war er fett und schön, mit
sicheren blauen Augen und einer erhabenen Nase und einer
Umgangsweise, als ob er gewohnt wäre, von einem Thron herunter mit
den Leuten zu reden. Molli empfing den hohen Gast in der Küche mit
Kniebeugung und Handkuß, worauf sie von ihm auf die Stirn geküßt
und sowohl wegen ihres Kochens belobt wurde, als auch weil sie sich
bei dem Verhör als ein tapferes, kluges und ihrer Herrschaft
ergebenes Mädchen erwiesen habe. Molli lächelte verschämt und
sagte, sie gehöre freilich nicht zu denen, die eine gute Herrschaft
im Unglück verließen. Zuerst sei sie wohl über die unanständigen
Dinge erschrocken gewesen, die man von dem Herrn Bürgermeister
gemunkelt habe, und als ihr dann noch die karminroten Eidotter in
die Hände geraten seien, habe sie den Kopf verloren, nachher aber
sich desto besser gefaßt und sich vorgesetzt, zu ihrem Herrn zu
halten, der doch einmal die Obrigkeit sei und bei der guten
katholischen Religion bleibe. Die Herren vom Gericht hätten sich
zwar recht darangehalten, um sie auf ihre Seite zu ziehen, sie
hätte gestern noch von Herrn Tiberius Tönepöhl sowie auch von Herrn
Augustus Zirbeldrüse je ein hübsch gemaltes Schreiben erhalten,
worin sie artig um das Vergnügen gebeten hätten, sie als Köchin in
ihr Haus einführen zu dürfen, wenn der Herr Bürgermeister, wie es
doch nun wohl nicht anders sein könnte, von Amt und Würden hinunter
in Schande und vielleicht gar Lebensverlust stürzte; aber sie hätte
nicht darauf geantwortet, da sie erst hätte erwarten\pagenum{[43]}
wollen, ob der Herr Bürgermeister wirklich so übel daran sei, und
dann auch aus den Blicken der beiden Herren den Argwohn gezogen
hätte, daß es ihnen nur darum zu tun wäre, die Ehre einer
unschuldigen Jungfrau zu Falle zu bringen. Diese letzten Worte
gingen in ein zartfühlendes Schluchzen über, das nur durch
liebreiches Zureden des Bürgermeisters und des Bischofs sowie durch
eine Geldspende von beiden endlich gestillt werden konnte.

Gegen Abend meldete sich Tiberius Tönepöhl zu einer Rücksprache bei
dem Bürgermeister und trug vor, daß es ihm ungeziemend vorkomme,
wenn das Geflügel im Saale des Rathauses vorgestellt würde, der
dadurch wie ein Stall mit Geschrei und Unrat erfüllt werden würde.
Man könnte den Garten des Bürgermeisters dazu verwenden, um diesem
gefällig zu sein; allein darin könnte Pfarrer Splitterchen eine
Benachteiligung erblicken, was er auch nicht scheinweise auf sich
laden möchte; sein Vorschlag gehe deshalb dahin, daß die Sitzung
vor dem Lindentore auf dem Anger abgehalten werde, wo nach altem
Gebrauch die städtischen Truppen eingeübt und auch Märkte und Feste
veranstaltet wurden. Wegen des Imbiß, zu dem Tile Stint den Richter
einlud, entschuldigte sich Tönepöhl, da er in seinem Amte sich der
weichen Regung, die ein trauliches Verkehren bei Tische anfache,
nicht unterstehen dürfe, vielmehr beständig das Bild des Rechtes
vor Augen haben müsse, gleichsam als den Nabel, auf den die
indischen Mönche ihr unentwegtes Augenmerk richteten, um zur
Gefühllosigkeit zu erstarren.

Am folgenden Morgen strömten Fußgänger, Wagen und Karren aus dem
Tore nach dem Stadtanger, der auf allen vier Seiten von alten, nun
blühenden Linden umrandet\pagenum{[44]} war. Wie ein Sternenkörper
in einer Lichtregion schwebt, die er von sich ausstrahlt, so
schwamm der Anger in einem Lindenduftgewoge, als ob ein elysisches
Seligenland aus der harten Erdenkruste hervorblühte oder daran
vorüberwehte. Wer der Zauberinsel nahekam, spürte eine reizende
Betäubung und wurde mitten in ein magisches Wohlgeruchsreich
hineingezogen, wo es eitel Scherz und Liebe und Wonnedasein gab.
Einzig Pfarrer Splitterchen und sein Rechtsbeistand Zirbeldrüse
gingen, wie wenn ihre irdischen Sinne mit Wachs verstopft wären, in
dieser sommerlichen Trunkenheit umher, als zwei Gerechte zwischen
ein Volk von Toren und Schelmen, und die wohl wissen, daß sie wegen
ihrer Überlegenheit und Tugend, deren sie sich nun einmal nicht
entbrechen können noch wollen, zuerst ausgelacht und dann
gekreuzigt werden müssen. Der Pfarrer rieb zuweilen die Zähne
aufeinander vor Verachtung und Ungeduld, oder er lachte, um
anzudeuten, er wisse wohl, daß er in einer Komödie mitspiele;
Zirbeldrüses Gesicht glich nicht mehr einem auseinanderlaufenden,
sondern einem hartgewordenen Käse, den man nicht schneiden,
höchstens zu einem grünlichen Pulver zerreiben kann. Sein Mund sah
aus wie ein Strick, an dessen Enden zwei schwere Gewichtsstücke
hängen, und er blinzelte von Zeit zu Zeit immer um sich wie ein
Hund, der ein Loch im Zaune sucht, durch das er entwischen könnte,
der aber zu voll im Bauch und zu träge ist, um davon Gebrauch zu
machen, selbst wenn er eins fände. Zwischen den Linden standen
einige Ratsbüttel, um dem zuströmenden Volke abzuwehren, allein sie
nahmen es nicht genau und ließen alt und jung lustwandeln, so weit
die Macht der alten Bäume schattete, sofern sie sich nur nicht in
den Ring des Gerichtes mitten auf dem Platze wagten.

\pagenum{[45]}Auf die Nachricht von dem hilfreichen Erscheinen des
Bischofs war Druwel von Druwelstein vom Bette aufgestanden und kam
mit festlich strahlendem Gesicht auf den Lindenanger, ohne sich
durch den Spott und Mutwillen Frau Armidas beirren zu lassen. »Da
war ich«, rief er, »im Getümmel unter mein Pferd geraten und sind
mir die Knochen arg zerquetscht worden; aber ich habe mich
hervorgearbeitet und sitze wieder aufrecht, bereit zu einem neuen
Gange.« »So laget Ihr unter dem Pferde, als man Euch allenthalben
vergeblich suchte?« erwiderte Frau Armida, »darunter ist man
freilich vor Stich und Kugel sicherer als darauf; aber ein Kavalier
geht nach Ehre aus, und die ist unter einem Pferdekadaver nicht zu
holen!« »Warum nicht!« rief Druwel frohmütig, »wenn man nur mit
Ehren darunter gekommen ist. Den möchte ich sehen, der den Druwel
von Druwelstein nicht da finden wird, wo der Herrgott und das Recht
ist, gleichviel ob einer in Ängsten ist oder florieret. Verzagt
nicht, gestrenge Freundin, solange Ihr mein Fähnlein flattern seht,
ist Eure Sache nicht verloren.« »Ei was, für den Herrgott brauche
ich keine Freunde, aber wider den Teufel,« sagte Frau Armida
ungeduldig, aber nicht herbe; denn sie ließ vielmehr ein
tröstliches Lächeln über Druwels bräunlichblinkende Wange und
seinen straffen Knebelbart gleiten.

Der Vorsitzende machte sich unterdessen mit der Einrichtung des
Tisches und mit dem Federvieh zu schaffen, das in Körben
herbeigeschafft war. Ratsherr Lüddeke, der Bürgermeister und die
Bürgermeisterin legten selbst Hand an, um den Hahn aus der Watte
herauszuwickeln, in die er wegen neuerlicher Gebrechlichkeit
verpackt war. Als davon nichts mehr an ihm und um ihn saß, glich er
einer Leichnammumie, von der soeben der Kalkbewurf abgekratzt ist,
welcher\pagenum{[46]} sie jahrhundertelang bedeckt hatte; der
kleine Lüddeke, der sich indessen nicht versehen hatte, geriet in
einige Verlegenheit und sah den Bürgermeister von der Seite an, der
gleichfalls die Augen niederschlug; denn hier draußen, wo der
lautere Sonnenglanz gleichsam in einem kristallenen Bade zwiefach
erglitzerte, stach das abgeschabte Jammergerippe widriger hervor,
als es sich zu Hause dargestellt hatte. Der Armselige hatte sich an
jenem Abend, als die Bürgermeisterin mit Steinwürfen nach seinem
Leben trachtete, zwischen das Dachgebälk der Scheune verkrochen und
war erst am vorhergehenden Tage wieder aufgefunden und gewaltsam
ans Licht gefördert. In dieser Zeit war seine Ernährung und
sonstige Pflege ungenügend gewesen: er sah nicht anders aus, als ob
der Böse ihn geholt, mit seinen rußigen Händen ihm das Gefieder
zerzaust und den Hals umgedreht hätte. Während der kleine Lüddeke
und der Bürgermeister sich unschlüssig ansahen, und der Druwel sich
räusperte, rief Frau Armida mit heller Stimme: »So ist der Arme in
der Zeit der Verfolgung heruntergekommen! Sollte er, was der Himmel
verhüte, tödlich abgehen, so werden wir auf Ersatz des Schadens
klagen, da wir nicht nur einen guten alten Haushahn, sondern auch
unseren Liebling mit ihm verlieren!« Auch der Bischof war nun
hinzugetreten und sagte: »Wie sehe ich Eure Heiligkeit wieder! So
kann es Gott gefallen, die Hohen dieser Erde zu erniedrigen.
Immerhin trägt er noch die Tiara, an der ich ihn wiedererkenne,
obwohl sie für seinen augenblicklichen Kräftezustand zu schwer ist
und trübselig wie eine Zipfelmütze von seinem Haupt herabhängt!«

Als der Bischof bei den Linden aus seiner Sänfte gestiegen war,
hatte sich das lustwandelnde Volk um ihn\pagenum{[47]} geschart
und im Schutze seines leutseligen Lächelns wie eine bunte und
brausende Schleppe hinter ihm hergewälzt. Eine solche hinter sich
herzuziehen, war er gewöhnt und hätte sich ohne das unvollkommen
bekleidet gefühlt, und ebensowenig dachten die Büttel daran, ihm
den Huldigungsschweif hinterrücks abzureißen. Demzufolge war der
Hahn im Nu von vielen Frauen und Kindern umgeben, die ihn
streichelten und ihm allerlei Futter beizubringen suchen, wovon er
schließlich etwas nahm und angstvoll hinunterschluckte. Die
beobachtende Menge begrüßte dies und andere Zeichen wiederkehrenden
Lebens mit frohem Geschrei; denn er schloß nun auch einige Male die
Augen ganz und öffnete sie wieder, als wollte er versuchen, ob die
Maschine noch ginge. Als er sogar mit dem Schnabel, wiewohl
schwächlich, unter die Körner stieß, die vor ihm ausgestreut waren,
mit den wackelnden Beinen nach hinten auszukratzen sich bemühte und
ein heiseres Krächzen von sich gab, kamen die Hühner, um die sich
niemand bekümmert hatte, erst schüchtern, dann eilfertiger
herbeigerannt und fingen um das Scheusal herum zu picken und zu
essen an. Hierüber erhob sich anhaltender Jubel, der mit leichten
Flügelschlägen den ausgebreiteten Lindenduft bewegte, so daß ein
seliges Jagen von Balsam und Schall sich zu Häupten des Volkes auf
und ab wiegte und als ein Baldachin der Freude über den Berauschten
schwebte.

Der Bürgermeister begann vor Rührung zu weinen, und auch dem Druwel
wurden die Augen feucht, als er seinem Freund und Frau Armida
kräftig die Hand schüttelte.

»Nun,« sagte der Bischof, auf die Hühner deutend, »das Völkchen hat
sich einträchtlich zusammengefunden, wie es nicht der Fall sein
könnte, wenn die Hölle dazwischen nistete.«

\pagenum{[48]}Tönepöhl ließ den Bischof aus Achtung den Satz zu
Ende bringen, fiel dann aber schnell ein, damit er ihm nicht
zuvorkäme, und schickte sich mit lächelndem Ernst zu einer Rede an.
»Wenn man sagt, daß die Stimme des Volkes die Stimme Gottes sei, so
kann man diesen Spruch wohl mit ebensoviel Recht auf die Tiere
anwenden, die noch mehr als das Volk aus der Tiefe untrüglicher
Grundgefühle heraus sich äußern. Hier haben wir nun beide, das Volk
und das Vieh, vernommen. Es hat sich vor unseren Augen ein
Gottesgericht abgespielt, markerschütternd und doch auch lieblich
in seiner Ahnungslosigkeit. Wenn wir heute vom strengen Gange der
Justiz abgewichen sind, so ist es mit Fug und durchdachter Absicht
geschehen, da zuweilen Freiheit Weisheit sein kann. Möge doch jeder
sich überzeugen, wie unberechtigt die Klage ist, daß in unserem
Gemeinwesen das Volk von der Regierung ausgeschlossen sei; wo es
ersprießlich ist, geben wir seinem Urteil Raum und Gehör.«

Hier wurde Tönepöhl durch einen Zwischenfall, der sich geräuschvoll
abspielte, unterbrochen. Es ertönte nämlich aus der Mitte der
Hühner ein lautes Kreischen oder Krächzen, dem auf der Stelle ein
Aufschreien der Bürgermeisterin folgte, eines von den
Pfarrershühnern habe Kikeriki gerufen. Sie bezeichnete das Huhn,
dem sie den Hahnenkraht zuschrieb, mit hindeutendem Finger und
sagte, rot vor Entrüstung, so komme denn Ungebührlichkeit und
Unnatur unter den Hühnern desjenigen vor, der ihren Hahn
teuflischer Umtriebe beschuldigt habe. Mit raschen Schritten
näherte sich der Pfarrer und sagte spöttisch: »Wenn irgendwo
Kikeriki gerufen wird, so schließt man daraus, daß ein Hahn
anwesend sei, und da in der Tat der Hahn des Herrn Bürgermeisters
hier vorhanden ist, so wird jeder Vernünftige der Ansicht sein, daß
\pagenum{[49]}er es getan habe.« »Freilich, freilich,« rief Frau
Armida, »so meint man auch, wenn irgendwo Eier gelegt werden, daß
es Hühner getan haben. Indessen habe ich mit meinen Augen gesehen,
daß das Kikeriki aus dem dünnen Halse jenes Huhnes kam, und stelle
es außerdem den Anwesenden anheim, ob unser armer schlotternder
Hahn imstande wäre, in so lauter, durchdringender Weise zu krähen,
wie eben geschehen ist.« »Gesehen habe ich nichts, aber daß eben
vernehmlich und deutlich gekräht worden ist, bestätige ich als
richtig,« sagte Tönepöhl. »Das kann jeder,« wandte Zirbeldrüse
hämisch ein. »Ich sage, daß von einem Hahn gekräht worden ist,«
wiederholte Tönepöhl aufgebracht, aber doch gemessen; »und zwar von
einem Hahn in der Gestalt eines eigentlichen Hahnes oder eines
wirklichen Huhnes.«

Jetzt meldeten sich Männer, Frauen und Kinder durcheinander, um zu
bezeugen, daß das von der Frau Bürgermeisterin bezeichnete Huhn den
vorgefallenen Hahnenkraht wirklich begangen habe. Auf den Befehl
Tönepöhls wurde das Huhn ergriffen und auf den Tisch gesetzt, wo es
verzweifelt herumstolperte, um zu entkommen, als ob es sich seiner
häßlichen Erscheinung schäme. Der Hals des Tieres war nämlich,
vielleicht durch die Arbeit von Ungeziefer, ganz von Federn
entblößt, und so schien es von einer grausamen Köchin lebendigen
Leibes gerupft, aber noch vor Beendigung des Geschäftes entsprungen
zu sein. »Das Tier ist ein Greuel!« rief Druwel von Druwelstein,
mit markiger Stimme das atemlose Schauen und Staunen der Menge
durchbrechend. »Man veranlasse es, noch einmal einen Ton von sich
zu geben,« sagte der Bischof heiter, »damit jeder sich von dem
Charakter desselben überzeugen kann.« Dieser Vorschlag wurde
unmittelbar als so einsichtig befunden, daß die Richter
\pagenum{[50]}ihre Gänsefedern ergriffen und das Huhn damit
stachen und belästigten, so gut sie konnten, wovon die Folge war,
daß der entsetzte Vogel hierhin und dorthin flatterte und endlich
auch in ein mißtönendes Kreischen ausbrach, dem sich ein nicht
schwächeres, sondern donnernd verstärktes Echo aus der Versammlung
anschloß. Als das Triumphgeschrei verhallt war, sagte Tönepöhl:
»Daß das Huhn krähen kann, halte ich hiermit für bewiesen,« in
welchem Sinne auch die übrigen Richter ihre Stimme abgaben; dann
wurde auf einen Wink des Vorsitzenden das gesamte Federvieh in die
Körbe gepackt und fortgeschafft.

Der Pfarrer, der bisher zähneknirschend und hier und da den Kopf in
den Nacken werfend, als rufe er Gott zum Zeugen solcher Dummheit
an, zugehört hatte, trat nun hastig vor und rief: »Und was folgt
daraus, wenn es bewiesen wäre, was ich nicht anerkenne? Es gibt
Tauben, die lachen, Pfauen, die trompeten, Papageien, die
menschlich schwatzen, warum soll ein Huhn nicht krähen? Hängt
solches doch nur von der zufälligen Bildung der Kehle ab!«

»Das Krähen,« entgegnete Tönepöhl mit nachdrücklicher Ruhe, die dem
Pfarrer seine unanständige Hitze beschämend zum Bewußtsein bringen
sollte, »das Krähen ist ein Abzeichen der Männlichkeit und kann auf
natürlichem Wege vom Huhne nicht erfolgreich nachgeahmt werden. Wir
haben vor mehreren Jahren eine Frau, die in Männerkleidern
einherging und auf ihrem Geschlecht ertappt wurde, öffentlich
ausgestäupt und des Landes verwiesen, da das Weib sich die Tracht
des Mannes, das ist des höhergeborenen Menschen, nicht anmaßen
darf. Wie soll man es da beurteilen, wenn ein Weibswesen sogar die
dem Manne angeborenen Eigenheiten, gleichsam die ihn auszeichnende
\pagenum{[51]}Naturtracht, nachahmen oder sich erwerben will? Wo
sollte bei einer solchen Vermischung die notwendige Zucht und
Botmäßigkeit bleiben, die im Hause wie im Hühnerstall herrschen
muß?« Wie nun der Pfarrer im hellen Ärger sich die Worte entfahren
ließ: »Wie konnte ich auch so albern sein, gegen papistischen
Aberglauben kämpfen zu wollen!« entstand ein unwilliges Murren in
der Menge, und sie hätten es ihm wohl übel eingetränkt, wenn nicht
der Bischof beschwichtigende Zeichen gegeben und Tönepöhl
aufgefordert hätte, den Pfarrer zu seinem Besten zu verhaften und
in ein gutes Gewahrsam zu bringen, damit ihm von dem zwar aus
verständlichen und schätzbaren Ursachen, aber doch über Gebühr
aufgeregten Volke nicht ein Leides zugefügt werde.

Dem Hahn war das hastige Fressen nach langer Enthaltsamkeit so
schlecht angeschlagen, daß Molli für gut fand, ihn abzuschlachten
und sein mageres und zähes Fleisch geschickt in eine lüsterne
Pastete verwurstete, welche bei dem Sieges- und Versöhnungsmahl,
das unter Teilnahme des Stadthauptmanns beim Bürgermeister
stattfand, verzehrt wurde.

\end{document}
