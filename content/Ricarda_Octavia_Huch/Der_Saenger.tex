\usepackage[german,ngerman]{babel}
\usepackage[T1]{fontenc}
\hyphenation{wa-rum}


%\setlength{\emergencystretch}{1ex}

\begin{document}
\raggedbottom

\author{Ricarda Huch}
\title{Der Sänger}
\date{}
\lowertitleback{Diese Ausgabe basiert auf dem
  \href{http://www.gutenberg.net/}{Project Gutenberg}
  EBook \#27446.}

\maketitle

\pagenum{[55]}Durch die breiten widerhallenden Gänge des
Gefängnisses San Callisto gingen an einem warmen
Frühlingsvormittage der Kardinal Mazzamori und der Meister der
päpstlichen Kapelle, Don Orazio, der seinen Stammbaum auf den
berühmten römischen Dichter zurückführte, beide Günstlinge des
Papstes Innozenz des Zehnten. Sie waren im Begriff, einen jungen
Menschen aufzusuchen, der des Mordes angeklagt und in Gefahr, das
Leben zu verlieren, dem Kardinal durch seine Geliebte, die schöne
Donna Olimpia, empfohlen worden war. Diese Dame, die durch Heirat
mit einem Ottobuoni aus kleinbürgerlichem Stande gehoben war, hatte
den Zusammenhang mit ihrer im Schatten weiterlebenden Familie nicht
verloren und pflegte ihn besonders, wenn sie sich in ihren neuen
Verhältnissen beeinträchtigt und unzufrieden fühlte. Als nun eine
ihrer Tanten zu ihr gekommen war und sie angefleht hatte, das
bedrohte Leben des einzigen Sohnes zu retten, wozu sie vermittels
ihres Freundes, des Kardinals Mazzamori, wohl imstande sei, war sie
nicht nur von Mitleid, sondern von Ehrfurcht für die Frau ergriffen
worden, die, von den Todesschmerzen der Mutter durchbohrt, ein
geheiligtes Schicksal zu erfüllen schien, während sie selbst nie
geboren noch die eheliche Treue bewahrt hatte und jetzt sogar an
ihrem geistlichen Freunde die Lust zu verlieren begann. Ihrem herb
erteilten Befehl hatte der\pagenum{[56]} Kardinal sich nicht
entziehen können, obwohl er die Möglichkeit, Hilfe zu schaffen, in
diesem Falle für ausgeschlossen hielt.

Es war nämlich der junge Lancelotto – so hieß der Vetter Olimpias –
durch seinen verstorbenen Vater, einen Kaufmann, der Gläubiger
eines Anverwandten des Papstes und hatte sich im Auftrage seiner
Mutter, nachdem verschiedene Mahnungen nicht gefruchtet hatten,
selbst in das Haus des Schuldners begeben, um ihn zur Zahlung
aufzufordern. Da der Herr sich kurzweg weigerte, seiner
Verpflichtung nachzukommen, oder sie gar leugnete, entstand ein
lebhafter Wortwechsel, in dessen Verlaufe der Nepot einige seiner
Leute herbeirief und ihnen befahl, den unverschämten Dränger zu
ergreifen und ihn durch das Fenster auf die Straße zu werfen. So
aufs äußerste gereizt, hatte Lancelotto, indem er sich der Männer,
die roh über ihn herfielen, zu erwehren suchte, einen derselben auf
den Tod verwundet. So viel Ursache der adlige Herr auch hatte, den
Vorfall zu verbergen, machte er ihn doch anhängig, um sich des
lästigen Gläubigers zu entledigen und zu seinem Glück trafen
mehrere Umstände zusammen, durch welche die Richter gegen den
Angeklagten eingenommen wurden.

Unter den Papieren Lancelottos fand sich außer allerlei verbotenen
philosophischen Schriften ein Spottgedicht auf den Papst, und so
liebenswürdig und empfindungsvoll Innozenz der Zehnte in mancher
Beziehung auch war, so hätten doch sogar seine verwöhntesten
Vertrauten sich jäher Ungnade versehen müssen, wenn sie ein gegen
ihn gerichtetes Witzwort zu verteidigen gewagt hätten. Vornehmlich
seine Schwäche, sich für einen Dichter zu halten, mußte von
jedermann geschont werden, und nichts hätte ihn davon abgehalten,
in demjenigen einen Mörder und Ketzer zu sehen, der mit viel Geist
\pagenum{[57]}und komischen Wendungen seine sapphischen Oden
parodiert hatte; denn dies war die Form, in die er die Ergießungen
seines Christenherzens vorzugsweise einzukleiden liebte.

Das Gedicht war »Die römische Sirene« betitelt und lautete etwa so:
»Segle nicht an der römischen Küste vorüber, Odysseus, oder tust du
es dennoch, so versäume nicht, deine Ohren mit Wachs zu verkleben,
damit du den Gesang des Papstes nicht vernimmst. Hörtest du ihn, so
würde dich ein solcher Schauer ergreifen, daß du nicht mehr
imstande wärest, dein Schiff zu lenken, und elend scheitern
würdest.« Es wäre tollkühn gewesen, sich eines Menschen anzunehmen,
der die Unvorsichtigkeit gehabt hatte, eine solche Keckheit nicht
nur aufzuschreiben und bei sich finden zu lassen, sondern sogar
seine Urheberschaft zuzugestehen.

Unter diesen Umständen schritt Kardinal Mazzamori mit bekümmerter
Miene neben seinem Freunde Orazio her, ihm seine Sorgen und
Bedenken mitteilend. »Ich kann Olimpias Teilnahme für die Tante
nicht anders als liebenswert finden,« sagte er, »obschon ich
darunter leide. Ihr Mitgefühl für ihre Verwandten macht ihr Herz
unzugänglich gegen meine Ansprüche, die ich ihrer düsteren Miene
gegenüber kaum geltend zu machen wage. Wie leicht wird die Tugend
zum Feinde des Glücks, und wie schwer ist es deswegen, zum Freunde
der Tugend zu werden. Ich kann mir keinen glücklichen Abschluß
dieser Angelegenheit vorstellen, da ich mich durchaus nicht in den
Prozeß einmischen kann, der schon zu viele Torheiten des Beklagten
ans Licht gebracht hat, und da doch die unberatene Olimpia ihre
Zärtlichkeit für mich an seine Rettung geknüpft hat.«

Orazio gab zu, daß es eine heikle Sache sei, und fügte bei, er
könne sich nicht genug freuen, daß seine Veranlagung ihn
\pagenum{[58]}vor dem unheilvollen Einfluß der Weiber beschützt.
»Nach meinem Dafürhalten«, sagte er, »wiegen die Vergnügungen, die
uns dies Geschlecht bereiten kann, die Ärgernisse und
Enttäuschungen nicht auf, die aus seinem Umgange fließen.«

Der Kardinal seufzte statt der Erwiderung, ohnehin war inzwischen
der ihnen vorangehende Wärter vor einer der vielen Türen
stehengeblieben, die auf den Gang führten, und gab ihnen ein
Zeichen, daß sie am Ziele seien.

Bei ihrem Eintritt richtete sich der Gefangene von seiner Pritsche
auf, sah die Fremden verdutzt und mißlaunig an, sprang dann auf und
sagte mit höflichem Gruß, daß er fest geschlafen habe und sich
nicht gleich auf seine Lage habe besinnen können. »Die barmherzige
Natur hat mir«, sagte er lachend, »die Gabe reichlichen Schlafes
verliehen, womit ich die Zeit verscheuchen kann, da mir keine
Gelegenheit gegeben wird, sie mir durch Arbeit oder Unterhaltung zu
befreunden.«

»Die Fähigkeit, zu schlafen, deutet auf ein freies Gewissen,«
bemerkte der Kardinal, worauf der junge Mann erwiderte: »Das habe
ich freilich; ich möchte den Mehlsack sehen, der sich von ehrlosen
Schurken mit Füßen treten ließe, ohne sich zu wehren. Wäre meine
Zunge so fehlerlos, wie meine Hände ohne Makel sind; aber die ist
so beschaffen, daß sie alles ausspricht, was durch mein Gehirn
zuckt, als ob sie eine Glocke wäre, an die der Schlegel der
Gedanken beständig anschlüge. Da wird denn manches laut, was den
Leuten Verdruß erregt; sprächen alle aus, was sie denken, so hätte
ich zu viele Gesinnungsgenossen, als daß man sie alle einsperren
oder ihnen allen den Kopf abschlagen könnte.«

»Ihr redet nicht eben wie ein bußfertiger Sünder,« sagte Don Orazio
nicht ohne Wohlgefallen an dem hübschen\pagenum{[59]} Jüngling,
dessen Munterkeit durch seine jammervolle Lage nicht gebrochen zu
sein schien.

»Was wollt Ihr, mein Herr!« entgegnen er zutraulich. »Einen Mord
habe ich nicht begangen; soll ich zerknirscht sein, weil ich nach
Maßgabe meines Verstandes über das Wunder des Daseins nachgedacht,
oder etwa gar, weil ich einen unbedeutenden Witz über Seine
Heiligkeit gemacht habe? Ich mache wohl auch einmal einen frechen
Scherz über die höchsten Herrschaften im Himmel, die doch
ehrwürdigere Häupter sind als der Papst, ohne daß ich mich deshalb
der Sünde zeihe; denn was für Abbruch tut es ihrer Herrlichkeit,
wenn ein Erdenwurm, ihnen fast unsichtbar, ein wenig daran zupft?
Ich bin nur ein armer Schlucker, den man leicht des Lebens und der
Ehre berauben kann, und bin doch den Richtern nicht böse, die mich
täglich als einen blutdürstigen Raufbold und Rebellen traktieren.«

Der Kardinal, welcher inzwischen verlegen seine weißen Nägel
betrachtet hatte, sagte, indem er eine ernste Miene annahm: »Die
Gerechtigkeit des Heiligen Vaters bürgt dafür, daß Euch kein Leid
widerfährt, wenn Ihr so schuldlos seid, wie Ihr behauptet. Hättet
Ihr nicht durch Euren Mutwillen die Anwartschaft auf Gnade
verscherzt, so möchte ich Euch raten, Euch mit Eurem Anliegen ganz
zu den Füßen Seiner Heiligkeit zu werfen.« Da der junge Mann nicht
sogleich antwortete, setzte Orazio im Tone wohlmeinender Überredung
hinzu: »Würdet Ihr nicht wenigstens das leidige Gedicht, das Euch
der Teufel eingegeben hat, zurücknehmen?«

»Warum nicht?« antwortete Lancelotto. »Am liebsten auch alle
Gedichte Seiner Heiligkeit, wenn ich es könnte.«

Es war Don Orazio unmöglich, das Lachen zurückzuhalten; der
Kardinal indessen spürte nur einen schwachen\pagenum{[60]} Anreiz
zur Heiterkeit, da das Bewußtsein der Widerwärtigkeit seiner Lage
in ihm fortwährend zunahm.

Wie der arme junge Mensch wahrzunehmen begann, daß der Besuch durch
ein gewisses Interesse an seiner Befreiung veranlaßt war, färbte
die erwachende Hoffnung seine blassen Wangen um einen Hauch röter,
und durch sein vorher so gelassenes Benehmen zitterte verhaltene
Unruhe. Ob es nicht wirksamer wäre, fragte er, indem er seine
Blicke zwischen den beiden Herren hin und her gehen ließ, wenn
seine Mutter sich an die Gnade des Papstes wendete? Sie würde alles
tun, was ihn retten und ihn ihr wiedergeben könnte. Auf ihr
Betreiben wären gewiß auch die beiden Herren mit so viel gütigem
Anteil zu ihm gekommen.

Der Kardinal nickte und ließ einige Worte fallen, wie die Liebe der
unglücklichen Frau zu ihrem Sohne nicht nachlasse, obwohl er ihr so
schweren Kummer bereite.

»Nicht durch meine Schuld,« sagte Lancelotto frei und freundlich;
»wäre ich aber noch so schuldig, so würde ich an ihrer Liebe doch
nicht zweifeln; denn ich bin noch in ihrem Herzen, wie ich einst in
ihrem Leibe war, und Gott selbst mit seiner Allmacht könnte mich
nicht herausreißen.«

Während er dies sagte, hatten seine Augen sich gefeuchtet und waren
dadurch glänzender und dunkler geworden, und den beiden Herren fiel
es jetzt auf, daß diese Augen ungewöhnlich schmal und lang waren,
so wie die älteren Maler die Augen der Cherubim und der verklärten
Heiligen zu bilden pflegten. Das geisthaft Schwebende, spielend
Süße, das ihnen eigen war, mochte dadurch bedingt sein, daß die
kleinen Pupillen den irdischen Leidenschaften keinen Versteck zu
gewähren und die Vielheit der bunten und veränderlichen Erdendinge
nur von ferne spiegeln zu können schienen.

\pagenum{[61]}Inzwischen hatten die scharfsichtigen Augen
Lancelottos erfaßt, daß die Herren doch nicht eigentlich darauf
ausgingen, etwas Wirksames für ihn zu tun, und die frohe Welle der
Hoffnung sickerte langsam wieder zurück. Wenn die Herren, sagte er
nach einer Pause, einen Auftrag an seine Mutter übernehmen wollten,
so möchte er sie bitten, ihr einen Büschel seiner Haare
zuzustellen, der ihr als ein lebendiges Stück von ihm teuer sein
würde. Ihm würde jedoch weder ein Messer noch sonst ein scharfes
Instrument in die Hände gegeben, so daß er ohne ihre Hilfe nicht
imstande sein würde, sich Haare abzuschneiden.

Der Kardinal zog ein mit Schmelz und farbigen Steinen verziertes
silbernes Büchschen aus dem weiten Ärmel, worin ein kleiner
Spiegel, eine Schere, ein Stückchen Wachs, eine Feile zum Glätten
der Nägel und dergleichen enthalten waren, öffnete es und blickte
unschlüssig hinein. Während er zögerte, bemächtigte sich Don Orazio
der Schere und beugte sich über den Kopf des jungen Mannes, um den
erforderlichen Schnitt zu tun, wobei er mit einer liebkosenden
Bewegung in die ein wenig geringelten kastanienbraunen Haare
griff.

Bei diesem Anblick fiel es dem Kardinal ein, daß die unglückliche
Mutter in der Besinnungslosigkeit ihres Schmerzes gewimmert hatte:
»Er ist ja noch ein Kind! Er hat Löckchen wie ein Kind!« und es
berührte ihn überaus peinlich, die Klage mit eigenen Augen
bestätigt zu sehen. Nachdem er sich leicht geräuspert hatte, sagte
er, daß er Sorge tragen werde, das Andenken in die Hände der Mutter
gelangen zu lassen, und daß er gern etwas tun würde, um die Lage
des Gefangenen zu erleichtern. Ob er Wünsche in betreff des Essens
habe? Oder womit ihm sonst gedient wäre?

Was das Essen angehe, sagte Lancelotto, so werde er durch seine
Mutter beköstigt, die ihm mehr Leckerbissen\pagenum{[62]}
vorsetzen lasse, als er bewältigen könne. Bücher, etwa ein Bändchen
Gedichte, würden ihn wohl erfreuen, am liebsten würde ihm aber
sein, wenn er einen Gefährten zugesellt bekäme, mit dem er ein
wenig plaudern und lachen könnte. Dieser Gedankengang brachte ihn
darauf, daß er seine Gäste, von denen namentlich der Kardinal
augenscheinlich sehr niedergeschlagen und durch die trübselige
Umgebung bedrückt war, bisher nicht eben gut unterhalten habe, und
er begann ein munteres Geschwätz, wobei aus den schmalen Augen die
unschuldige Schelmerei eines übermütigen Knaben blitzte. Er
erzählte Schulstreiche aus dem geistlichen Kolleg, das er besucht
hatte, von Lehrern und von dem Abt eines gewissen Klosters, der ihn
als einen jungen Heiligen angesehen habe und noch immer darauf
warte, ihn als Novizen eintreten zu sehen. Er habe den guten Mann
oft besucht und sich im Kloster wohl gefühlt; aber lange halte er
es in der Abgeschiedenheit nicht aus, dem Getümmel des Lebens
zuzusehen, sei ihm die liebste Beschäftigung; je toller es um ihn
her zugehe, desto stiller und behaglicher fühle er sich im Innern.

Dann sei die enge Zelle freilich nicht der rechte Aufenthalt für
ihn, meinte Orazio teilnehmend; aber der junge Mann erwiderte, es
sei immerhin so arg nicht, wie es den Anschein habe. Sein Fenster
gehe auf den Hof, wo die zur Gefangenschaft Verurteilten sich zu
gewissen Stunden ergehen dürften und untereinander handelten,
lachten, lärmten und zankten, wie wenn Viehmarkt auf der Piazza
Navona wäre. Zwischenhinein könne er schlafen, und schließlich
befinde sich in einer nicht weit entfernten Zelle ein
Untersuchungsgefangener, der eine so schöne Stimme besitze, daß man
sich einbilden könne, schon im Paradiese zu sein, wenn man ihn
singen höre.

\pagenum{[63]}Die Pause, die hierauf entstand, benutzte der
Kardinal, um Lancelotto zu fragen, wie es denn in Hinsicht der
Religion mit ihm bestellt sei. Ob er auf das Jenseits vorbereitet
sei, oder ob er etwa von einem verständigen Geistlichen über den
heiligen Glauben belehrt zu werden wünsche.

Der junge Mann schüttelte lachend den Kopf und sagten »Ich habe
Augenblicke, wo der Glaube mich mitten in Gottes Schoß trägt, und
ich habe Stunden, wo ich zweifle und denke, bis meine Gedanken an
jenes schwarze Tor stoßen, das sie nicht durchdringen und
übersteigen können. Das vermag kein Priester zu ändern, und ich
möchte es auch nicht. Den Platz, der in der weiten Welt für meine
Seele ist, werde ich erreichen; hat ja doch Gott dem Maultier
eingepflanzt, bei Nacht den rechten Weg, und einer Katze, das Haus
zu finden, wo sie hingehört. Die Herren müssen nicht um mich
besorgt sein, noch soll meine Mutter sich um mich grämen. Soll ich
sterben, so muß ich durch ein paar bittere Stunden hindurch, die
ebenso schnell vorübergehen werden wie manche andere, die ich auch
überstanden habe. Wie wohl wird mir aber hernach sein, wenn mir
Gott einen Anteil an der himmlischen Vollkommenheit gewährt. Dann
werden meine neugemachten, allgegenwärtigen und allwissenden Augen
auf die Verwirrung und das Händeringen und Zähnefletschen der
Menschen hinuntersehen und lachen, daß ich auch einmal mitten
dazwischen war und von armseligem Schlachtvieh zum Tode verurteilt
und auf das Schafott geschleppt wurde.« Sein frischer feuchter Mund
lächelte dabei mit besonderer Lieblichkeit, die man kaum wehmütig
nennen konnte, weil sie allzu unbefangen war.

Als die beiden Herren die Zelle verlassen und der Wärter die Tür
abgeschlossen hatte, winkten sie diesem, daß er\pagenum{[64]}
nunmehr entlassen sei, und gingen langsam den Gang hinunter. Der
Kardinal tupfte sich mit dem Taschentuch und sagte, es sei
jammerschade, daß ein solcher Knabe so zu Falle gekommen sei. Was
sei da zu machen? Der Mord könne schließlich nicht ungestraft
bleiben, er sähe keinen Ausweg.

»Ein allerliebster Junge,« sagte Don Orazio nachdenklich, »und
scheint durchaus nichts Strafwürdiges begangen zu haben. Ich hätte
Lust, mich seiner anzunehmen und ihn den Krallen dieses
gottvergessenen Tribunals zu entreißen, wenn sich nur ein
zweckmäßiger Weg dazu finden ließe.«

»Mir fehlt der Mut, mich vor Olimpia sehen zu lassen, wenn ich
keine Hoffnung bringe,« fuhr der Kardinal bekümmert fort. »Und wie,
wenn ich gar die Mutter mir vor Augen stelle! Ohnehin werde ich
diese Frau nie mehr vergessen können, die aussah, als ob sie
tausend Jahre gelebt und Schmerzen gelitten hätte. Sie sah aus wie
ein verwitterter Stein, und wenn sie zu weinen und zu schreien
anhub, so war es, wie wenn ein Berg sich bewegte und Feuer
auswürfe.«

»So, so,« sagte Don Orazio, »ich hatte sie mir als ein anmutiges
Weib vorgestellt mit süßen Lippen und zärtlichen Augen.«

Ohne diesen Einwurf zu beachten, ging der Kardinal in seinen
Betrachtungen weiter: »Was man ihr auch sagen mochte, sie schrie:
›Mein Kind, das ich geboren habe! Mein Paradiesvogel! Meines
Herzens Herz! Mein Eingeweide! Es ist mein, ich muß es
wiederhaben!‹, als ob das Gründe wären, mit welchen sich etwas
durchsetzen ließe.«

»Das ist«, fiel Don Orazio ein, »wie alle Frauen sind. Die setzen
ihren Eigenwillen der offenkundigen Notwendigkeit entgegen und
lassen sich durch Vernunft nicht belehren.«

\pagenum{[65]}Der Kardinal nickte verständnisvoll und nahm Anlaß,
in behutsamen Andeutungen über die Launenhaftigkeit der Donna
Olimpia zu klagen, die sie in letzter Zeit wie eine Krankheit
überfallen habe, während sie sonst liebevoll und verträglich wie
ein Engel gewesen sei. Was ihr sonst eine willkommene Zerstreuung
gewesen sei, gefalle ihr nicht mehr, sie liebe Einsamkeit und trübe
Gedanken, und die Verzweiflung jener unglücklichen Tante vermehre
ihren Tiefsinn. Sie werde es ihn entgelten lassen, wenn der Prozeß
des jungen Mannes übel auslaufe, als wenn er etwas dazu zu tun
vermöge; so sähe er einer unfrohen Zukunft entgegen. Da würde es
das beste sein, meinte Orazio, die launische Dame zu meiden und
bequemere Gesellschaft aufzusuchen; allein der Kardinal sagte, die
Frau habe sich doch um ihn verdient gemacht, und er halte sich
verpflichtet, nun, da sie offenbar krank und des Beistandes
bedürftig sei, bei ihr auszuharren. Er war beschämt, indem er dies
sagte, denn er fühlte, daß sein Freund seine Worte für eitel
Ausflucht und ihn für einen verliebten Toren hielt.

Als die Freunde, in dies Gespräch vertieft, in dem breiten und
kahlen, widerhallenden Gange auf und ab gingen, vernahmen sie
plötzlich den Gesang einer Männerstimme und blieben augenblicklich,
von dem Glanz derselben betroffen, stehen. Es war ein Volkslied,
das mit so viel Kraft und Sicherheit gesungen wurde, als ob es von
der Bühne eines großen Theaters her tönte, und mit so viel
Leidenschaft, als gälte es, ein zauderndes Mädchen zu einer
Entführung willig zu machen. Mazzamori und Orazio sahen einander,
vor Staunen und Vergnügen errötend, an, und als der Sänger dem
Abschluß einer Strophe eine Kadenz folgen ließ, hielten sie den
Atem an, besorgt, ob die schwindelnde Figur auch zu einem
glücklichen Ende gebracht würde.

\pagenum{[66]}Während der Dauer des Liedes näherte sich ein
wachehabender Soldat und machte Miene, dem Sänger Schweigen zu
gebieten, wie das den Vorschriften des Gefängnisses entsprochen
hätte, trat jedoch willig zurück, als die beiden Herrschaften ihm
einen Wink gaben, sich ruhig zu verhalten. Diesen riefen sie heran,
sowie das Lied zu Ende war, um Auskunft über die Wundererscheinung
zu erhalten. Der Sänger sei ein Bauer, meldete der Soldat, dem
wegen mehrfachen Mordes der Prozeß gemacht werde; er sei ein wilder
und böser Kerl, der den Mund nur zum Fluchen öffne, aber der
unerforschliche Gott habe für gut befunden, ihn mit einer Stimme zu
begnaden, wie kein Engel der himmlischen Heerscharen sie herrlicher
besitzen könne. Niemand habe den Mut, ein solches Wunder der Natur
zu unterdrücken, darum ließe man ihn singen, womit auch der
Direktor einverstanden sei, der manchmal selbst, wenn er in der
Nähe sei, stehenbleibe, um zuzuhören.

Ob man ihn nicht veranlassen könne, weiterzusingen? fragte Don
Orazio. Nein, sagte der Soldat, wenn man ihn um etwas bäte, würde
er es deswegen unterlassen, weil er bösartig und mißtrauisch sei.
Es könne ein Tag vorübergehen, ohne daß er die Stimme erhebe,
andere Male höre er stundenlang nicht auf; das sei von seiner Laune
abhängig.

»Ich kann mich nicht begnügen, von der Stelle zu gehen, ohne ihn
noch einmal gehört zu haben,« sagte Don Orazio, »sonst würde ich
morgen wähnen, daß mich meine Einbildungskraft geneckt hätte.«

Auch der Kardinal zeigte sich nach einer Wiederholung des Genusses
begierig. Sie erwogen eben, ob sie nicht dennoch versuchen sollten,
den Gefangenen zu einem Vortrage zu bewegen, als der Gesang von
neuem begann, um sie nicht minder als der erste zu entzücken.

\pagenum{[67]}»Ich habe einen Tenor wie diesen noch nie in meiner
Kapelle besessen,« sagte Don Orazio.

Der Kardinal stimmte ihm bei; er habe zwar die unvergleichliche
Schulung des berühmten Mignotta nicht, die Unfehlbarkeit des
Ansatzes und die Gleichmäßigkeit des Organs beim An- und
Abschwellen des Tones, aber an Kraft, Schmelz und Süßigkeit lasse
er alle anderen hinter sich. »Ich würde jederzeit«, so schloß er,
»eine Stunde lang auf einem Beine stehen, um ein solches Konzert in
mich aufnehmen zu können.«

»Mein Freund,« sagte Don Orazio, »ich habe keine Ruhe, bevor ich
nicht Näheres über diesen Mann erfahren habe; begleite mich
augenblicklich zum Direktor, damit wir Schritte tun können, um uns
dieser Kostbarkeit zu versichern.«

Der Direktor bestätigte die Aussage des Soldaten und führte sie
dahin aus, daß es sich in diesem Falle um einen erwiesenen
mehrfachen Mord aus Rachsucht handle; es habe nämlich der
Verbrecher, Ronco mit Namen, die Gewohnheit gehabt, nachts die Kühe
seines Nachbarn zu melken, und wie nun ein junger Bube, der Sohn
eines in der Nähe wohnenden Pächters, dem Geheimnis auf die Spur
gekommen sei und den Geschädigten, dessen rätselhafter Milchmangel
im Dorfe bekannt geworden sei, darauf aufmerksam gemacht habe, so
habe er sich anfänglich ruhig verhalten, als ob die Sache nur ein
Scherz und des Aufhebens nicht wert sei, aber nach acht Tagen nicht
nur den Buben, der ihn angegeben, sondern auch dessen Vater und
Mutter sowie eine alte Großmutter, die alle dieselbe Hütte
bewohnten, mit einem Messer umgebracht. Die Entrüstung über die Tat
sei allgemein, und der Mensch habe den Tod verdient und werde ihm
nicht entgehen; auch sei ihm nichts daran gelegen, der Kerl sei so
wild, daß er kaum einen Unterschied zwischen Leben und Tod zu
machen wisse.

\pagenum{[68]}Das sei ein seltsamer Bericht, sagte Don Orazio; man
müsse doch annehmen, daß ein alter Hader zwischen den Familien
bestanden habe, wie es bei solchen Rachehandlungen meistens der
Fall sei, und unüberlegt und empfindlich müsse der Mann auch sein.
Er hätte Lust, einmal selber mit ihm zu reden, um der Sache auf den
Grund zu kommen.

Der Direktor zuckte die Achseln und sagte, die Herren Richter
hätten sich schon genug Mühe mit ihm gegeben, die Bestie sei dessen
nicht wert; jedoch sei er bereit, die Herrschaften hinzuführen,
möchte ihnen aber raten, nicht ohne einen Wärter hineinzugehen, da
man sich von einem solchen Patron des Schlimmsten müsse gewärtig
sein.

An diesen Rat hätte der Kardinal sich gern gehalten, allein Don
Orazio lachte hoch aus und sagte, seine kräftige Gestalt reckend
und seine breite Brust aufblähend, er getraue sich wohl, es mit
einem maisfressenden Bauern aufzunehmen.

Auch war es in der Tat nicht eben Furcht, was den Meister der
Kapelle überlief, als er mit seinem Freunde dem Unhold
gegenüberstand, der sie mit einem schnellen Blick mißtrauischen
Hasses streifte, um sogleich wieder stumpfsinnig vor sich
hinzustieren; sondern vielmehr ein unwillkürliches Grauen vor dem
bösen Blick, dessen das Scheusal mächtig sein konnte. Vorher
getroffener Verabredung gemäß begann der Kardinal, nachdem er
verschiedenemal angesetzt hatte, und sagte, sie seien im Begriff,
die Ordnung der Gefängnisse zu untersuchen; ob er, der Gefangene,
über irgend etwas Klage zu führen habe? ob er den Besuch eines
Geistlichen empfangen habe? ob er geneigt sei, irgendwelche
Geständnisse zu machen oder seine Reue in den Schoß einer
vertrauenswürdigen Person geistlichen Charakters zu ergießen?

\pagenum{[69]}Die Antwort Roncos auf die sorgfältige Anrede des
Kardinals bestand darin, daß er knurrte und mit dem Daumen nach der
Tür deutete, worauf der Kardinal von neuem einigemal ansetzte und
fortfuhr, er, Ronco, sei eines grausamen und unerklärlichen
Verbrechens angeklagt; ob er vielleicht zur Erhärtung seiner
Unschuld oder zur Verminderung seiner Schuld etwas beizubringen
habe, was Verwirrung oder Scham den Inquisitoren gegenüber ihn
vielleicht gehindert hätte auszusprechen? Der Heilige Vater habe
viel mehr Freude an einer erlösten Unschuld als an der Bestrafung
eines Schuldigen und dehne seine Milde auch über diejenigen aus,
die durch Unbesonnenheit, Jähzorn oder Anstiftung des Teufels wider
ihren Willen zu einer bösen Tat hingerissen worden seien.

»Pest und Krebs über den päpstlichen Saustall!« zischte Ronco
zwischen den Zähnen hervor, indem er einen wilden Blick auf die Tür
warf und wiederum nach der Tür deutete, so daß der Kardinal
unwillkürlich einen Schritt zurückwich, wie um einen Platz jenseits
der Hörweite solcher Schimpfworte zu gewinnen.

Don Orazio, der das Bedürfnis fühlte, seinem Freunde zu Hilfe zu
kommen, sagte: »Weder der Heilige Vater noch seine Diener, mein
Freund, wollen dir übel, wie du vorauszusetzen scheinst. Wir wären
nicht an dieser Stelle, wenn wir deinen Tod suchten, den du
allerdings verdient zu haben scheinst. Gott der Allwissende hat
dich mit einer schönen Stimme begabt und dich dadurch nach
unerforschlichem Beschluß ausgezeichnet. Wie wäre es, wenn du uns
noch eine Probe dieser wunderherrlichen Kunst gäbest, der du
mächtig bist, und die beweist, daß mehr Göttlichkeit in dir wohnt,
als deine Taten, deine Worte und selbst dein Anblick vermuten
lassen.«

\pagenum{[70]}Ob nun Ronco diese Worte als eine Verhöhnung
auffaßte, oder ob er das Gespräch überhaupt als eine Belästigung
empfand, er schrie in ausgelassener Wut: »Hinaus! hinaus! Oder ich
werde euch etwas singen, daß euch die Lumpenschädel zerplatzen
sollen!« und begleitete die Aufforderung mit einer so drohenden
Gebärde, daß die beiden Herren es für das beste hielten, sich
zunächst zu bescheiden und den Rückzug anzutreten. Mit dem Schwunge
des Triumphes und der Verachtung spuckte Ronco hinter ihnen her.

Kardinal Mazzamori war so erschrocken, daß er nicht sofort
weitergehen konnte, sondern an dem nächsten Fenster des Ganges
stehenblieb, um Luft zu schöpfen und sich ein wenig zu erholen.

»Was für ein Tier!« sagte Orazio. »Man muß zugeben, daß unsere
Bauern nicht viel mehr als Vieh sind und die Anforderungen, die man
an sie stellt, danach bemessen.«

»Er hat eine wölfische Physiognomie,« sagte der Kardinal, »und ich
möchte wetten, daß er ein echtes Wolfsgebiß besitzt. Es scheint in
der Tat nötig, daß die Menschheit vor einem solchen Wüterich
beschützt werde.«

Seine letzten Worte wurden durch die Stimme Roncos übertönt, der
eben jetzt wieder zu singen begann, vielleicht aus Trotz, oder weil
ihm die Lobsprüche der vornehmen Herren dennoch geschmeichelt
hatten.

»Göttlich, göttlich!« flüsterte Don Orazio. »Dieses Wunderwerk von
Stimme darf nicht zerstört werden! Ich werde nicht Mühe noch Kosten
scheuen, sie mir zu retten.«

Unter den geschlossenen Augenlidern des lauschenden Kardinals
perlten Tränen hervor. »Welcher Wohllaut quillt noch aus dem Rachen
der Hölle!« hauchte er. »O Geheimnisse der Allmacht! Jeder Ton ist
rein, weich und lauter, wie ein\pagenum{[71]} Tropfen Tau, der
sich in der Frühe auf Knospen wiegt. Was wird Seine Heiligkeit
sagen, wenn sie diesen Gesang hört!«

In großer Erregung verließen die Herren das Gefängnis und ließen
sich in der bereitgehaltenen Sänfte zu dem Palast des Kardinals
tragen, um über die zunächst vorzunehmenden Schritte zu beraten;
denn darin stimmten sie überein, daß der rare Vogel für die
päpstliche Kapelle durchaus erworben werden müsse. Nachdem sie sich
bei einem Glase guten Weins in einem kleinen wohnlichen Gemach,
dessen Wände mit schönen Teppichen aus Arezzo verhängt waren, von
den verschiedenen heftigen Eindrücken des Vormittags erholt hatten,
schien es ihnen nicht unmöglich, das Tribunal zu einem Freispruch
des kostbaren Ronco zu bewegen. Sie hatten in Erfahrung gebracht,
daß ein gewisser Guidobaldo die Verteidigung des Verbrechers führe,
und mit diesem beschlossen sie sich zunächst ins Vernehmen zu
setzen. Don Orazio nämlich hatte ihn in einem befreundeten Hause
kennengelernt und sich gut mit ihm unterhalten, obwohl der Advokat
ein Freidenker und Feind des Klerus war. Da er aber seine Ansichten
nicht äußerte, außer wenn es am Platze war, die Formen der Religion
mit großem Anstand in acht nahm, sobald er sich beobachtet wußte,
und dazu ein fröhlicher und gewandter Mann war, so konnten auch
Geistliche seinen vorurteilslosen Verstand und seine geselligen
Gaben genießen und waren es zufrieden, einstweilen in gutem
Einvernehmen mit ihm zu bleiben. Es traf sich glücklich, daß der
Advokat gerade damals im Sinn hatte, eine Villa zu kaufen, deren
ausgedehnter Garten sich den Janikulus hinaufzog, daß er aber den
hohen Preis, der dafür gefordert wurde, nicht zahlen konnte oder
wollte; denn dadurch bot sich die erwünschte Gelegenheit, den
nützlichen Mann durch eine\pagenum{[72]} Gefälligkeit zu gewinnen.
Ohne Zaudern suchten die Freunde den Advokaten noch am selben Tage
auf und baten ihn, an die Bekanntschaft mit Don Orazio knüpfend,
ihm die Summe, deren er zum Erwerb der Villa benötige, vorstrecken
zu dürfen. Sie hofften, sagte Don Orazio, sich dadurch ein Anrecht
auf gütiges Entgegenkommen seinerseits zu verdienen, wenn sich das,
was sie von ihm wünschten, mit seiner Ehre und anderen Rücksichten
vereinigen ließe. Nach dieser Einleitung erzählte er von seinem
Funde im Gefängnis, sprach von der Vorliebe des Papstes für Musik,
insbesondere die menschliche Stimme, und von seinem Wunsch, eine so
überaus seltene Kraft für die päpstliche Kapelle zu gewinnen, zumal
damit ein Menschenleben gerettet und auf eine nutzbringende,
vielleicht ruhmvolle Bahn gebracht würde.

Der Advokat erwiderte, er habe bereits von der schönen Stimme des
Ronco gehört, sich aber nicht sonderlich dafür interessiert; er
trage jedoch gern dazu bei, dem Heiligen Vater ein Vergnügen zu
bereiten, auch sei es sowieso seines Amtes, die Verbrecher zu
verteidigen und womöglich zu retten. Immerhin sei das im
vorliegenden Falle schwierig, weil der Bauer überwiesen und
geständig sei und viel zu stumpfsinnig oder zu roh, um Schritte zu
seiner Rettung zu tun oder zu unterstützen, wenn solche überhaupt
erfindlich wären. Nach einigem Besinnen fuhr er fort, es ließen
sich wohl Wege zum Ziel ausdenken, wenn man fest entschlossen sei;
es sei schon mancher freigesprochen, der den Tod ebensowohl wie der
schlimme Ronco verdient hätte; von dem Präsidenten des Tribunals,
Monsignor Aloisio, sei es nur allzu bekannt, daß seine Stimme feil
sei, freilich um kein Geringes, wohingegen der weltliche Beisitzer
für ein billiges Trinkgeld zu haben sei. Da sei aber Don
\pagenum{[73]}Petronio, ein unzugänglicher Mann, dessen einzige
Eitelkeit und Liebhaberei seine Unbestechlichkeit sei, der stets
den Sittenrichter spiele und emsig aufpasse, damit ja nicht etwa
unter seiner Mitwirkung etwas Ungebührliches unterliefe. Wenn man
sich diesem mit wohlgemeinten Anerbietungen irgendwelcher Art
näherte, so würde man von vornherein alles verderben; wie man ihn
aber umgehen oder überlisten könne, dazu könne er noch keinen Plan
absehen, wolle die Sache aber bedenken. Ein Übelstand sei es auch,
daß der Prozeß im vollen Gange sei und nur noch ein Verhör
stattzufinden habe, worauf der Urteilsspruch bei der Klarheit des
Falles nicht auf sich warten lassen würde. Indessen ermutigte er
die beiden Bittsteller damit, daß guter Rat sich oft über Nacht
einstelle, und gab ihnen anheim, sich einstweilen mit dem
Präsidenten in Übereinstimmung zu setzen, offen gegen ihn zu sein
und etwa eine Art Mitwissen des Papstes anzudeuten, was seiner
Beflissenheit einen gedeihlichen Schwung geben würde.

Monsignor Aloisio war ein prachtliebender Mann und heiteren
Temperaments, der gern gut lebte und auch anderen Gutes gönnte,
wenn er nur Geld genug zur Verfügung hatte, dessen Mangel das
einzige war, was seine Laune auf die Dauer zu trüben vermochte. Als
er innewurde, daß Kardinal Mazzamori und Don Orazio ihm einen
erheblichen Zufluß des geschätzten Metalls zu eröffnen gedachten,
nahm er sie mit lauter und glänzender Gastlichkeit auf, führte sie
durch die pomphaft ausgestatteten Räume seines Hauses, zeigte ihnen
eine Sammlung chinesischer Porzellane und versprach für seine
Person, einem so billigen und harmlosen Wunsche ohne kleinliche
Bedenken entgegenzukommen, führte aber, wie der Advokat, den
unbestechlichen Don Petronio ins Feld, der, seiner schrullenhaften
Eitelkeit zuliebe, jeden Versuch, den armen Sünder\pagenum{[74]}
durchschlüpfen zu lassen, vereiteln würde. »Nach meiner Meinung«,
sagte er behaglich, »ist die Gerechtigkeit bei Gott, der es nicht
immer für gut findet, uns zu erleuchten. Wie kurz ist die Kette von
Ursache und Wirkungen, der wir folgen können! Nun ja, man urteilt
nach seiner Kurzsichtigkeit und glaubt, etwas Großes getan zu
haben, wenn man einen Dieb oder Räuber aufhängt. Wie oft dieser ein
frommes Herz im Busen hatte, ein guter Vater oder edler Freund war,
während sein sogenanntes Opfer auf dem Grunde der Seele, wohin kein
sterbliches Auge blicken kann, die Farbe der Hölle trug, wer mag
das wissen? Unser guter Petronio hingegen begreift nur den
Buchstaben und meint, unsere Erde zu verbessern, wenn die
Paragraphen recht in Anwendung kommen.«

Nachdem verschiedene Einfälle, um zum Ziele zu gelangen,
vorgebracht und verworfen waren, trennten sich die Herren, ohne zu
einem Schluß gekommen zu sein, mit der Befürchtung, daß ihnen der
Sänger dennoch entgehen würde. Indessen empfing Don Orazio noch zu
später Abendstunde einen Brief des Präsidenten, der so lautete: Es
sei ihm plötzlich ein eigenartiger, aber wohl tunlicher Einfall
gekommen. Wenn man nämlich den Petronio könnte glauben machen,
Richter und Advokat seien bestochen, um den Ronco, der zwar ein
ungebildeter Bauer, aber brav und nichts als das Opfer tückischer
Ränke sei, an den Galgen zu bringen, und sie alle ihre Rolle gut
spielten, auch der Advokat sich willig finden lasse, so sei zu
verhoffen, daß Don Petronio seine Kraft einsetze, die vermeintliche
Unschuld zu retten, so daß ihnen nach einem Kampfe von gewisser
Dauer nichts erübrige, als zu ihren eigenen Gunsten nachzugeben.

Das Einverständnis der Beteiligten wurde schleunig hergestellt.
Mazzamori und Don Orazio kargten nicht mit dem\pagenum{[75]}
Gelde, indem sie nicht zweifelten, der Heilige Vater würde ihnen am
Schluß reichlich ersetzen, was sie auf die Ausbildung eines so
erlesenen Sängers würden verwendet haben. Guidobaldo, der Advokat,
trug Sorge, daß Don Petronio durch eine anonyme Zuschrift auf das
Unwesen aufmerksam gemacht wurde, dem diesmal ein hilfloser Bauer
zum Opfer fallen sollte, und daß die Nachricht von seinem Hauskauf
sich verbreitete, und nahm den launigen Glückwunsch, den der
Präsident ihm in Gegenwart des versammelten Tribunals dazu machte,
händereibend entgegen. Er selbst, sagte der Präsident, habe auch im
Sinne, sich eine bescheidene Freude zu machen. Der französische
Gesandte, der von seinem König abberufen sei und Rom zu verlassen
gedenke, wolle eine goldene Karosse mit vier Pferden verkaufen, und
er habe ein Angebot darauf gemacht, wisse aber noch nicht, ob es
angenommen sei. Die Summe flüsterte er dem Advokaten lächelnd ins
Ohr, wie er denn überhaupt es so einrichtete, daß die Mitteilung
als eine vertrauliche, durch den zufälligen Lauf des Gesprächs
entlockte erscheinen mußte. Petronio, der die Herren scharf
beobachtete, säumte nicht, ihre so plötzlich auftretende
Verschwendung mit der schmachvollen Rechtsbeugung in Einklang zu
bringen, zu der sie sich dem empfangenen Briefe nach hatten
bereitfinden lassen. Um sicher zu gehen, lenkte er selbst die Rede
auf Ronco und sagte, mit dem würden sie hoffentlich heute zu Ende
kommen; denn man solle darauf halten, in einer so klaren Sache
wenigstens keine Zeit zu verlieren. Der Präsident stimmte bei, und
der Advokat fügte mit liebenswürdig scherzender Ironie hinzu, er
wisse wohl, daß er die Herren Richter, die gern Zeugen einer
breiten Entfaltung seiner Beredsamkeit wären, damit enttäuschte,
habe aber doch beschlossen,\pagenum{[76]} diesmal schlechtweg ohne
weitere Worte um ein gnädiges Urteil zu bitten, da er sich mit der
Verteidigung eines so verworfenen Übeltäters nicht verunzieren
wolle. Das töne anders, sagte Petronio mit Nachdruck, als er,
Guidobaldo, sich zuvor habe vernehmen lassen. Er habe damals
gesagt, der Grund, der den Ronco zum Morde getrieben haben solle,
sei zu geringfügig, um eine solche Untat zu erklären, auch würde
ein gemeiner Verbrecher die Bluttat zu leugnen versuchen, um sein
Leben zu retten; es lasse sich also erwägen, ob nicht der
augenscheinlich halb blödsinnig gemachte Bauer das Werkzeug
Mächtiger sei, die sich nebst ihren Absichten und Mitteln im
Hintergrunde hielten.

Der Advokat lachte künstlich verlegen: »Da sehen die Herren,« sagte
er, »wie weit ich den Pflichteifer getrieben habe! Nun aber scheint
es mir besser, an der Grenze der Höflichkeit haltzumachen, die ich
euch schulde, Männern, die leicht einsehen, daß Mächtigen nicht mit
der Ermordung einer unschuldigen Pächterfamilie noch mit der
Hinrichtung eines Ronco gedient sein kann, und daß das, was ich
damit vorbrachte, nichts als die Redensarten und Mutmaßungen waren,
die ein geübter Advokat stets im Vorrat haben muß.«

»Sie, mein Teurer,« sagte der Präsident mit heiterer Miene gegen
Don Petronio, »wittern überall Ungerechtigkeiten, weil Ihr
großmütiger Trieb, Verfolgten beizustehen, sich die Gelegenheit zum
Handeln schaffen muß. Ach, die Schlechtigkeit ist weniger
interessant, als Sie meinen! Erleben wir es nicht alle Tage, daß
das rohe Gesindel untereinander rauft und sticht? Wir brauchen
keine Fabeln zu erfinden, um das begreiflich zu machen.«

Don Petronio, den nichts mehr beleidigte als wenn man ihn nicht
ernst nahm, begann nun einen unmittelbaren Angriff,\pagenum{[77]}
wobei er sich fortwährend das Ansehen eines ruhigen,
unbeeinflußbaren Geistes zu geben suchte. An dem Schicksal des
Ronco sei nichts gelegen, sagte er, das sei sonnenklar. Er sei
nicht viel mehr als ein Tier, sei es nun, daß Stumpfsinn oder daß
Roheit seine Menschlichkeit gestört habe. Man möge nicht glauben,
daß er Teilnahme für Ronco habe, für ihn selbst wie für andere sei
es vielleicht das beste, wenn er der Zeitlichkeit enthoben würde.
Solche Rücksichten würden ihn aber niemals abhalten, der Wahrheit
nachzutrachten und das Recht in Ausübung zu setzen. Nur um Recht
und Wahrheit handele es sich für alle, nicht um das Wohl von
Klägern oder Beklagten, vor allen Dingen nicht um das eigene. Er
wolle nichts von jenem Ronco wissen, wolle nicht wissen, ob er Weib
und Kind, Verwandte oder Freunde habe. Eine derartige Gesinnung sei
zwar dem jetzigen Zeitalter fremd, um so mehr werde er daran
festhalten. Er werde niemals Landgüter kaufen oder Kunstsammlungen
anlegen können, vielleicht stifte er nicht einmal Gutes mit seiner
Handlungsweise; es sei ihm genug, der Wahrheit und Gerechtigkeit,
auf die er verpflichtet sei, ohne Gewinn gedient zu haben.

Die Gegner ließen eine gewisse Gereiztheit merken, und es entspann
sich ein Streit, der noch im Gange war, als Ronco vorgeführt wurde.
Dieser hatte sich zuvor nie so in Anspruch genommen gesehen, denn
Don Petronio ließ jeder Frage, die der Präsident an ihn richtete,
eine anders gesetzte folgen, die den Zweck hatte, die bisher listig
verschüttete Wahrheit ans Licht zu fördern. So viel hatte der Wilde
inzwischen gemerkt, daß man sich von hoher Seite seiner anzunehmen
begonnen hatte, ja geradezu auf seine Befreiung hinarbeitete, und
seine vielberedete Stumpf- und Roheit hinderte ihn nicht, diese
Aussicht als angenehm zu empfinden\pagenum{[78]} und seinen
Helfern, soweit er sie verstand, in die Hände zu arbeiten. Zuweilen
begriff er den Sinn der kreuz und quer an ihn gestellten Fragen so
gut, daß er Antworten gab, die die Richtigkeit seiner früheren
Aussagen in Frage stellten und eine böse Verwirrung in den bisher
so glatten Prozeß brachten. Bei solchen Gelegenheiten warf Don
Petronio jedesmal einen ernsten, leidenschaftlich forschenden Blick
auf seine Widersacher, die sich scheinbar mehr verwickelten und
erhitzten und gegen Ronco heftig und beinahe drohend losfuhren,
wodurch sie ihn in eine solche Gemütslage versetzten, die für ihren
Zweck eben die richtige war. Denn er fing allmählich an, sich für
eine ansehnliche Person zu halten, und wenn er schon vorher mit
sich und seiner Untat durchaus zufrieden gewesen war, so glaubte er
jetzt vollends, daß er sich nichts von dem großmäuligen Tribunal
brauche gefallen zu lassen, das keineswegs besser und
wahrscheinlich dümmer sei als er. Er gab nun zwar keine
Erklärungen, mit denen sich etwas hätte anfangen lassen, aber er
bejahte das, was ihm Don Petronio fleißig in den Mund legte, daß er
das Verbrechen nicht aus eigenem Antriebe begangen habe, sondern,
daß er dazu angestiftet, eigentlich gezwungen worden sei, aber
nicht sagen dürfe, von wem, und schloß damit, man möge ihn zum Tode
verurteilen, er sei es zufrieden, doch sei er unschuldig und
weniger ein Mörder, als diejenigen sein würden, die ihn an den
Galgen brächten.

So schnell indessen gaben die Verschwörer nicht nach, vielmehr
stellten sie sich an, als wären sie erpicht darauf, den Ronco zu
liefern, und erhoben gegen Don Petronio den Vorwurf, als habe er
dem Fuchs, der schon in der Falle gewesen sei, ein Türlein geöffnet
und halte sie unnützerweise bei einer nebensächlichen und üblen
Sache auf.\pagenum{[79]} Dadurch reizten sie diesen immer mehr, so
daß er sich fest vornahm, der hehren Wahrheit zum Siege zu helfen,
was es ihm auch für Mühe und Verdrießlichkeiten eintragen möchte.

Der Zufall wollte, daß es Don Petronio gelang, einen bisher nicht
bekannt gewordenen Umstand zu ermitteln, daß nämlich sowohl der
Mörder wie der Ermordete Besitzer freien Bauerngutes waren und vor
Jahren einmal mit dem Herrn, von dem sie das Land in Pacht hatten,
in Streit gewesen waren, weil er sie ganz und gar zu abhängigen
Pächtern hatte herabdrücken wollen. Es unterlag für Petronio kaum
einem Zweifel, daß dieser Herr, ein Aldobrandini, sich der beiden
halsstarrigen Männer, die ihm zu trotzen wagten, auf einmal zu
entledigen versucht hatte, indem er sie gegeneinanderhetzte, und
obwohl er darauf verzichtete, den Schuldigen zu entlarven, der
jedenfalls zu mächtig, schlau und gerissen war, um sich fangen zu
lassen, so wollte er ihm doch das Opfer abjagen, so wenig es an
sich der Teilnahme wert sein mochte.

Mittlerweile hatte der Meister der Kapelle, Don Orazio, eine
endgültige Aussprache mit Ronco, der sich auf vieles Zureden
bereitfinden ließ, wenn er freigesprochen sein würde, als Sänger in
den Dienst des Heiligen Vaters zu treten. Er vermochte nunmehr
seine Rolle besser zu spielen und gebärdete sich tagtäglich
dreister, so daß das gesamte Tribunal endlich den Augenblick
herbeisehnte, wo es die Bürde seines Schützlings würde abwerfen
können. Die Freude und Genugtuung war auf allen Seiten gleich groß,
als der Freispruch erfolgte, wenn auch der Präsident und der
Advokat sich nichts davon merken ließen, sondern Ingrimm und
Beschämung zu verhehlen schienen, um sich desto besser zu
belustigen, wenn sie unter sich waren.

\pagenum{[80]}Einzig Kardinal Mazzamori machte böse Zeiten durch;
denn die üble Laune seiner Herrin Olimpia nahm zu, seit er in der
Angelegenheit ihres jungen Vetters nichts ausgerichtet hatte. Er
mochte noch so sehr beteuern, daß er alles Erdenkliche unternommen
habe, ihn zu retten, daß aber die Gerechtigkeit ihren Lauf hätte
nehmen müssen, und daß es ihn nicht minder als sie betrübe: sie
beharrte dabei, er hätte es sich keine Mühe kosten lassen, weil
seine Liebe zu ihr selbstischer Natur sei und nur genießen, nicht
wirken oder opfern wolle, und sie bestrafte ihn durch eine durch
nichts zu erhellende Traurigkeit. Der Jammer ihrer unglücklichen
Tante, sagte sie, habe ihr auf einmal die Augen für das Elend des
Lebens eröffnet, so daß sie sich an den irdischen Dingen nicht mehr
ergötzen und nur in der Hingebung an Gott einigen Trost finden
könne. Wirklich war sie selten mehr zu Hause anzutreffen, sondern
hielt sich schwarz gekleidet in Kirchen auf, wo sie bald vor
diesem, bald vor jenem Altar sich in Tränen auflöste. Empfing sie
den Freund einmal, so forderte sie ihn auf, von geistlichen Dingen
mit ihr zu reden, und wenn er ihr auf diesem Gebiete nicht genugtun
konnte, hielt sie ihm mit großer Bitterkeit vor, daß er seinen
Beruf nicht verstehe, und ließ ihn merken, daß er nicht viel
Besseres als ein Heuchler und Betrüger sei. Sie fühlte sich
tiefunglücklich und alles dessen beraubt, was früher ihrem Dasein
Halt und Inhalt gegeben hatte. Es schien ihr, als sei ihr Mann, von
dem sie nun schon lange getrennt lebte, im Grunde viel annehmbarer
gewesen als der Kardinal, insofern, als er sich doch für nichts
Höheres ausgegeben hatte als er war. Wenn sie sich die Zeit
zurückrief, wo Mazzamori ihre Liebe erregt hatte, so vermochte sie
in allen jenen Szenen, die zwischen\pagenum{[81]} ihnen
vorgefallen waren, den Zauber der Poesie nicht mehr zu finden, der
sie früher in ihrer Einbildung vergoldet hatte. Was war daran, so
dachte sie nun, anderes und Edleres gewesen, als was sich
alltäglich in jedem Winkel abspielt und oft genug zu Gelächter und
Ekel reizt? Wie sehr sie sich bemühte, etwas Besonderes und
Ausgezeichnetes an dem Kardinal zu finden, ihr Gewissen wollte ihr
nichts sagen, als daß er eine unkeusche Bestie sei wie die anderen
Männer auch, mit dem Unterschiede, daß sein geistlicher Beruf ihn
noch dazu zu einem gleisnerischen Lügner machte. Sie hätte ihn am
liebsten nicht mehr gesehen; wenn sie ihn zuweilen dennoch
herbeiwünschte und seinen Besuch annahm, so tat sie es
hauptsächlich, um ihn fühlen zu lassen, was sie von ihm dachte, und
wie unglücklich sie sei.

Ein schweres Verhängnis war es für den Kardinal, daß der
verdüsterte Gemütszustand der schönen Olimpia sie ihm noch weit
liebenswerter erscheinen ließ als früher. Ihr Blick, da er so
seelenvoll geworden war, zog ihn mehr an, als es je ihr sinnlich
erglühender getan hatte, und ihre demütige Trauer, die ihn hätte
abwehren sollen, reizte mit seinem Mitleid und seiner Bewunderung
zugleich seine aufrichtigsten Liebesempfindungen. Wieviel reicher
und erhabener erschien sie ihm, seit sie seiner nicht mehr
bedurfte! Wenn er zusah, wie milde und verständnisvoll sie mit
armen Leuten umging – denn sie suchte nun Gelegenheiten, um sich
Notleidenden wohltätig zu erweisen –, wenn er hörte, wie klug und
frei sie über alle Verhältnisse des Lebens zu reden wußte, so kam
sie ihm wie eine Wiedergeborene vor, ihm weit entrückt und doppelt
begehrenswert. Er gab sich Mühe, auf ihre neuen Gedankengänge
einzugehen, ohne daß er etwas anderes als Spott und Bitterkeit
dabei\pagenum{[82]} geerntet hätte. Olimpia fand diese
Bestrebungen, die nicht der Sache galten, sondern nur ein Ausfluß
seiner Verliebtheit waren, lächerlich oder gar abstoßend und wurde
durch sie in der Meinung bestärkt, daß der Kardinal ein seichter
Heuchler sei.

In der Hoffnung, die ihm Entschlüpfende zu fesseln und ihre
weltlichen Interessen wieder ein wenig anzufachen, erzählte der
Kardinal ihr von dem wunderwürdigen Sänger, den sein Freund Don
Orazio kürzlich kennengelernt und für den päpstlichen Dienst
erworben habe. Dieser Sänger sei, erzählte er, durch widrige
Schicksalsfälle verfolgt und unter höchst seltsamen Umständen von
Don Orazio entdeckt worden, die auch ihm noch Geheimnis wären.
Gewiß sei, daß er die herrlichste Stimme besitze, die je ein
italienisches Ohr bezaubert habe, und die durch die sorgfältige
Ausbildung, der sie jetzt unterzogen werde, noch gewinnen solle.
Die Mittel dazu hätten Orazio und er hergegeben, da der Sänger
durch die erwähnten Schicksalsschläge mittellos geworden sei; es
reue sie aber das Opfer nicht, da jeder Ton aus der gesegneten
Kehle edleres Gold als das sei, was sie dafür ausgegeben hätten.
Wenn Olimpia ihn zu hören geneigt sei, so wolle er eine Gelegenheit
dazu in seinem Hause veranstalten.

Indessen Olimpia war zu sehr in ihre schwermütigen Anschauungen
versenkt, um sich irgendwelche Zerstreuungen gefallen lassen zu
mögen; nichts war ihr recht, was sie davon entfernte, nur das
willkommen, was sie darin bestärkte. Schöner Gesang wäre ihr wohl
lieb, sagte sie, aber zu teuer erkauft, wenn sie ihn im Beisein
anderer, etwa sogar unter geselligen Zurüstungen hören müsse. Könne
der Sänger sie aufsuchen und ihr seine Kunst vorführen, ohne sie in
ihrer einsamen Beschaulichkeit zu stören, so wäre es möglich, daß
sie Genuß daran hätte. Das wußte der Kardinal nun nicht
\pagenum{[83]}einzurichten; denn einmal wußte er nicht, ob Ronco,
der sich übermütig und habgierig entfaltete, sich ohne weiteres und
namentlich ohne die Aussicht beträchtlicher Vorteile dazu verstehen
würde, und zudem hätte er für das Benehmen seines Schützlings nicht
einzustehen gewagt, wenn derselbe ohne Zwang und Aufsicht allein
mit einer Dame gewesen wäre. So mußte er auf eine Gelegenheit
warten, um Olimpia mit dem Wundermanne bekannt zu machen, und eine
solche bot sich denn auch, als der Gesangmeister, der ihm
Unterricht erteilte, seine Stimme für so geschult erklärte, daß
seiner Vorstellung bei Hofe nichts mehr im Wege stehe.

Der Papst hatte zur Teilnahme an dem Konzert, das in seinen
Gemächern stattfinden sollte, einen kleinen gewählten Kreis
musikliebender Freunde um sich versammelt, unter denen Kardinal
Mazzamori, als um den Fund so sehr verdient, nicht fehlen durfte.
Auch war ihm bereitwillig gestattet worden, seine Freundin Olimpia
mitzubringen, die eine Einladung des Heiligen Stuhls auszuschlagen
sich denn doch nicht getraute. Sie wählte zwar eine Frisur und
Kleidung, die, dunkel und schlicht, gegen die früher von ihr
geliebte reichliche Ausschmückung weit abstach, und hielt sich auch
bescheiden und fast schamhaft im Hintergrunde; aber daß ihr zartes
Fleisch um so lieblicher aus dem Schatten hervorleuchtete, hatte
sie durch diese Veranstaltung doch nicht hindern können.

Innozenz war ein feiner, kleiner alter Herr mit einem zierlichen
Gesicht, mit etwas undeutlichen Augen, einer gebogenen zugespitzten
Nase und einem dünnlippigen, meist freundlichen Munde. Er nahm die
Huldigung der Gäste geschwind entgegen und ließ einem jeden einige
scherzende Worte zukommen, wobei ihm aber die Ungeduld wegen der
bevorstehenden Vorführung anzumerken war. Als es eine
\pagenum{[84]}Minute über die Zeit war, auf welche man den Sänger
bestellt hatte, wurde ein nervöses Zucken um seine Augen sichtbar,
und Mazzamori blickte ängstlich von der kostbar verzierten Uhr, die
auf dem Marmorkamine stand, zur Flügeltür; er atmete auf, als diese
sich öffnete und Don Orazio, eintretend, um die Ehre bat, den
Sänger Ronco vorstellen zu dürfen. Ronco hatte sich in der Zeit
seiner Vorbereitung eine Richtschnur seines künftigen Betragens
gemacht, die einfach, aber nichtsdestoweniger ausnehmend zweckmäßig
war: nämlich den Beifall des Papstes zu erwerben und einzig darauf
sein beständiges Augenmerk zu richten. Von diesem Vorsatz beseelt,
ging er stracks, die Augen mit einer gewissen eindringlichen
Heftigkeit auf das erhabene Ziel gestellt, auf den alten Herrn zu,
fiel vor ihm nieder, küßte ihm die Füße und verharrte in dieser
Stellung, indem er die Arme vor der Brust faltete. Diese kindliche
Gebärde inbrünstiger Hingebung rührte Innozenz so sehr, daß er
unwillkürlich die Lippen auf die Stirn und die Hände auf die
breiten Schultern des vor ihm knienden Mannes drückte, worauf er
ihn mit ermunternden Worten begrüßte und aufzustehen und sich zu
setzen aufforderte. Fast fürchtete der alte Herr, das erschütternde
Gefühl, sich zum erstenmal in seiner Gegenwart zu befinden, könne
den Sänger der Macht über seine Kehle berauben; es zeigte sich
aber, daß der starke Mann mit der Hingebung die Unbefangenheit
eines Kindes vereinte, denn ohne durch das leiseste Zittern
beeinträchtigt zu werden, rollten die ersten Töne wie große
glänzende Perlen in den Saal. Infolge einer Anordnung des Orazio
sang er zuerst das Volkslied, das er und Mazzamori im Gefängnis von
ihm gehört hatten, und das ohnehin als etwas Neues und
Befremdliches Aufsehen erregte und Eindruck machte.

\pagenum{[85]}Es war, als ob der Stab eines Zauberers die Herzen
der Zuhörer berührt habe; einem jeden tauchten liebe Träume auf,
allerschönste Augenblicke, deren sie sich erinnerten oder die sie
erhofften, und die einen süßeren Duft aushauchten, als der
zerkleinernden Wirklichkeit übrigzubleiben pflegt. Olimpia überkam
ein gewaltsamer Schmerz, der aber nicht niederdrückend war wie der,
dem sie seit vielen Wochen in wechselnder Art hingegeben war,
sondern durchdringend und angenehm, als eine Kraft, die sie über
das gemeine Leben emporzutragen schien. Sie fühlte, was sie einst
als Mädchen gewesen war, was sie von der Zukunft erwartet und was
sie selbst hätte leisten und erringen wollen, und mit der
schrecklichen Einsicht zugleich, wie weit sie von diesem Ziele
abgewichen war, glaubte sie zu wissen, daß es nur auf sie ankomme,
wieder die reine, starke und freudige Seele von damals zu werden.
Sie hörte nicht auf, ihre Beziehungen zu Mazzamori zu bereuen, aber
sie tadelte sich in diesem Augenblick, daß sie ihn mit Härte
behandelt hatte, da doch nicht er allein, sondern auch sie
gesündigt habe, und da er ihr doch nicht die Möglichkeit rauben
könne, aus den Irrwegen, auf die er sie geführt, zur Klarheit
aufzusteigen. Es erschien ihr wie ein Wunder, daß sie trotz ihres
Widerstrebens in die Nähe des Mannes gebracht worden war, dessen
Stimme ihr so tröstlich wurde, und der ihr dadurch fast wie ein
Abgesandter Gottes erscheinen wollte. Aus dem Winkel, wo sie Platz
genommen hatte, konnte sie ungestört seine heldenhafte Gestalt
bewundern und das dunkle Antlitz, dessen Wildheit sie erbeben
machte.

Ronco war bei der sorglichen Pflege, die er sich seit geraumer Zeit
hatte angedeihen lassen können, nur auf die Art schöner geworden,
wie aus einem schäbigen hungrigen\pagenum{[86]} Wolf ein
wohlgenährter wird; aber dies genügte um auf aller Augen einen
blendenden und überwältigenden Eindruck zu machen. Die Begeisterung
war allgemein; doch machte niemand dem Heiligen Vater das Recht
streitig, sie zuerst zu äußern. Der kleine Herr saß mit geröteten
Wangen da, klopfte hie und da in die Hände, rief: »Bravo! bravo!«,
wiegte den Kopf und unterbrach auch wohl den Gesang mit Ausrufen
des Entzückens: »Ah! Welcher Ansatz! Welche Süßigkeit! Welche
Erfindung!«, wenn die Kadenzen wie aus dem Füllhorn des Überflusses
aus seinem Munde strömten. Es vermehrte die Bewunderung Olimpias,
daß der Sänger keinen Blick auf die Gesellschaft, geschweige denn
in ihren Winkel warf; er schien nur für den Heiligen Vater da zu
sein und auf seinen Wink zu singen oder zu schweigen. Einem
Erzengel mußte sie ihn vergleichen, der, in voller Pracht gerüstet,
demutvoll den Befehl des Herrn der Heerscharen erwartet.

Erst als die Gesellschaft aufstand und auseinanderging, fiel ein
Blick des Sängers auf sie, der mehr als Gleichgültigkeit, der
niederschmetternde Verachtung auszudrücken schien. Sie schloß
daraus, daß er wisse, in welchen Beziehungen sie zu Kardinal
Mazzamori gestanden habe, ja seiner Meinung nach noch stehe, und
daß er sie aus diesem Grunde für eine Verworfene halte, was sie ja
im Grunde auch sei.

In Wirklichkeit hatte der Sänger weder sie noch sonst eine von den
Zuhörerinnen beachtet, da es ihm zunächst nur um den Papst zu tun
war und er überhaupt an den vornehmen Damen noch keinen Geschmack
gewonnen hatte. Allmählich jedoch stellte sich das Verständnis
dafür ein, und nunmehr konnte ihm die Verehrung nicht unbemerkt
bleiben, mit der die schöne Olimpia zu ihm aufblickte. Es
schmeichelte ihm nicht wenig, daß die Geliebte des Kardinals
Mazzamori\pagenum{[87]} ihn diesem angesehenen, einflußreichen,
liebenswürdigen und gebildeten Manne vorzog, den zu beleidigen er
ohnehin einen lebhaften Antrieb in sich verspürte. Je mehr seine
Stellung beim Papst und in Rom sich befestigte, desto unleidlicher
wurden ihm die beiden Herren, denen seine Vergangenheit so
wohlbekannt war, so daß er mit dem Gedanken umging, sie, wenn sich
ein Anlaß böte, aus Rom zu entfernen.

In den ersten Tagen, die dem Konzert folgten, war Mazzamori
hochbeglückt über den Erfolg. Schien es doch die geliebte Frau
weich und zugänglich gestimmt zu haben. Um so schneidender war
seine Enttäuschung, als sie ihm, wenn auch in gütigen Worten, ihren
unerschütterlichen Entschluß mitteilte, jeden Verkehr mit ihm
abzubrechen, da sie ein neues, reineres Leben in Gott nunmehr
beginnen wolle.

Da er sah, daß jeder Versuch, sie ihrem Vorsatz abtrünnig zu
machen, scheiterte, ergab er sich und rang bereits mit dem Plane,
ihr nachzueifern, um ihr wenigstens in den Regionen der Entsagung
wieder zu begegnen, als er durch Neckereien Bekannter auf die
zarten Fäden aufmerksam gemacht wurde, die zwischen dem Sänger und
der Büßerin hin und her gingen. War er auch überzeugt, daß bei
Olimpia nichts vorlag als die Schwärmerei einer empfänglichen Seele
für die Stimme, in der etwas Göttliches sinnfällig zu werden
schien, so zweifelte er doch billig, ob der gewalttätige Bauer
einer ähnlichen Erhabenheit der Empfindung fähig sei, dem er
vielmehr die Absicht zutraute, das Weib in die Niederungen seiner
Sinnlichkeit herabzuziehen.

Dies wurde ihm zur Gewißheit, als verlautete, der Sänger habe vor
einigen Tagen um Urlaub gebeten und solchen auch erhalten, um
irgendwo am Meere oder im Gebirge seine Stimme zu schonen, was zu
deren Erhaltung von\pagenum{[88]} den Ärzten für durchaus
notwendig erklärt worden sei. Außer sich eilte der Kardinal zum
Papste, um ihn darüber aufzuklären, welche Gefahr seiner Meinung
nach eine edle Freundin bedrohe, und wie freventlich die Güte des
huldvollen Gebieters mißbraucht werde.

Der alte Herr merkte kaum, daß es sich um einen Angriff auf seinen
Liebling handelte, als seine Lippen sich ärgerlich zusammenkniffen.
Er selbst litt unter dessen bevorstehender Abwesenheit, hatte
seiner Bitte aber dennoch willfahrt und ein Beispiel der
Selbstverleugnung gegeben; mußte er dem geistvollen Zauberer nicht
einmal ein Abenteuer, in dem er sich austobte, gestatten? War er
doch selbst jung gewesen! Und wieviel mehr als ein anderer bedurfte
der Feurige, der Zündende, der Verschwender Zufuhr neuer Kräfte,
die ihm, dem Papst, und allen, die ihn hörten, wieder zuteil werden
würden! Wenn er sich vorstellte, wie der löwenhafte Mann das
erstemal, die Arme über der Brust gekreuzt, vor ihm niedergekniet
war, so pflegten ihm Tränen in die Augen zu treten. Niemals war er
seitdem von dieser kindlichen und ritterlichen Hingebung
abgewichen. Obwohl hitzigen Temperaments und hochfahrenden Sinnes,
wie er denn im Umgang mit anderen Menschen oft durch zügellose
Laune und Grobheit überraschte, nahte er sich ihm, dem Heiligen
Vater, dem zierlichen kleinen Manne, nie ohne Unterwürfigkeit, nahm
er von ihm jeden Tadel mit Bescheidenheit und Geduld entgegen und
rief in jeder Angelegenheit sein Urteil als das höchste, gleichsam
von Gott selbst ausgefertigte an, dem sich zu beugen ihm
augenscheinlich sowohl Lust wie Pflichtgebot war.

Indem er sich in seinem Sessel zurücksetzte, betrachtete Innozenz
den Kardinal erstaunt und bat um eine Erklärung des Anteils, den er
an dem Urlaub und der Reise des\pagenum{[89]} Sängers nähme. Ein
wenig errötend sagte der Kardinal, es sei dem Heiligen Vater
vielleicht nicht bekannt, daß Ronco den Ausflug in Begleitung einer
Dame zu machen gedenke, einer Dame, mit der er weder in
verwandtschaftlicher noch in ehelicher Beziehung stehe, soviel ihm
bekannt sei.

»Und was weiter?« fragte der Papst kühl. »Sollten Sie, mein Freund,
niemals, eine Reise mit Damen ohne verwandtschaftliche Beziehung
unternommen haben? Oder wenn Sie, ein Geistlicher, ein Diener
Gottes, es nicht getan haben, warum sollten Sie einem Sänger diese
Freiheit mißgönnen?«

Der Kardinal zitterte vor Verlegenheit, Angst und Enttäuschung.
»Verzeihen mir Eure Herrlichkeit,« sagte er, »wenn die Sorge um
eine Frau, die mir teuer ist, und über deren Heil zu wachen ich
mich verpflichtet halte, mich zu weit hingerissen hat.«

Bevor er noch mehreres hinzufügen konnte, unterbrach ihn der Papst,
indem er sagte: »Gut, gut! Überlassen Sie es mündigen Frauen, sich
selbst zu schützen, wenn sie überhaupt des Schutzes bedürfen oder
ihn wünschen. Ich habe es stets so gehalten, daß ich meinen
Untergebenen in Familiensachen freie Hand ließ, denn dies ist der
Punkt, wo aus Herrschaft Tyrannei würde.«

Nach dieser Zurechtweisung wurde der Kardinal nicht ungnädig
entlassen; ja, der Heilige Vater zeichnete ihn beim nächsten
Empfang mit liebenswürdigen Worten aus; aber als er nach einigen
Tagen an die Spitze einer Mission zur Bekehrung der Heiden in Japan
gestellt wurde, konnte er nicht umhin, darin mehr den Wunsch des
Papstes, ihn zu entfernen, als einen Beweis seiner Hochachtung zu
sehen.

Das Bewußtsein seiner Untauglichkeit zu einer solchen Aufgabe war
so stark in ihm, daß er es wagte, dem Papst\pagenum{[90]} seine
Befürchtungen dieserhalb zu unterbreiten; doch beruhigte ihn dieser
mit dem Hinweis auf seine mannigfachen Talente, denen, wenn sie der
Glaubenseifer unterstütze, nichts unmöglich sein werde, und auf die
Märtyrerkrone, die er sich im besten Fall erwerben könne.

Don Orazio hielt sich etwas länger, schließlich jedoch wußte der
unentwegte Ronco auch ihn zu stürzen, indem er ihn durch
fortwährende Widersetzlichkeiten und Kränkungen dahin brachte, sich
beim Heiligen Vater über ihn zu beklagen. Als dieser ihn damit
abwies und ihm vielmehr empfahl, sich einem so herrlichen Künstler,
der Zierde seines Hofs, gegenüber nicht zu überheben, brauste
Orazio auf und rief aus: »Wie? Von diesem Vieh, das ich aus dem
Morast gezogen habe, soll ich mich ungerecht verhöhnen lassen?«,
mit welcher unbesonnenen Äußerung er die Gunst seines Herrn
vollends verscherzte. Denn wie er sich wegen des beleidigenden
Ausdrucks rechtfertigen wollte, bedachte er, daß er den wahren
Hergang seiner Bekanntschaft mit Ronco nicht wohl enthüllen konnte,
ohne sich in verhängnisvolle Mißhelligkeiten zu verwickeln, und
mußte, da er sich über sein Benehmen nicht ausreden konnte, als
verleumderischer Schwätzer oder ungezähmter Tollkopf dastehen. Die
es gut mit ihm meinten, waren der Ansicht, daß er es noch für Gnade
und Glücksfall ansehen müsse, als der Papst ihn nach dem kleinen
Hofe von Lucca empfahl, wo er zwar in schmalen Verhältnissen, aber
doch ohne Not und Gefährdung sein Dasein fristen konnte.

Schlimmer und besser zugleich erging es seinem Freunde Mazzamori,
der zwar mancherlei Entbehrungen und Todesgefahr zu bestehen hatte,
aber, wenn solche vorübergegangen war, auch Augenblicke bisher
unbekannter Seligkeit feierte,\pagenum{[91]} und über dessen liebe
und traurige Vergangenheit die vielen absonderlichen Eindrücke, die
er empfing, einen bunten Schleier webten, der sie undeutlich
machte. Zuweilen, wenn er in fremder Einsamkeit am Gestade des
Ozeans zwischen namenlosen Riesenbäumen und vorüberhuschendem
Getier in der Dämmerung sich erging, erinnerte ihn, er wußte nicht
wie, ein lieblicher Himmelsglanz zu seinen Häupten an die schmalen
länglichen Cherubsaugen jenes jungen Lancelotto, mit denen er frei
in paradiesischen Sphären auf die verlassene Erde hinabsehen
wollte. Vielleicht, dachte er, lächelt er über die Verworrenheit,
in die wir armen Toren verstrickt sind, wenn er sich nicht lange
schon ermüdet weggewendet hat zu den gelösten Geheimnissen der
Weltregierung. Auf Augenblicke schwieg dann das Heimweh nach der
goldenen Küste Italiens, das ihn in Stunden, wo er allein war, zu
beschleichen pflegte, und er dachte mit bänglicher Sehnsucht an die
Märtyrerkrone, die seine Arbeit unter den bösen Heiden ihm
eintragen konnte, und die vielleicht, von unsichtbaren Händen
bereit gehalten, schon über ihm schwebte.
\end{document}

