\usepackage[german,ngerman]{babel}
\usepackage[T1]{fontenc}
\hyphenation{wa-rum}


%\setlength{\emergencystretch}{1ex}

\begin{document}
\raggedbottom

\author{Ricarda Huch}
\title{Der neue Heilige}
\date{}
\lowertitleback{Diese Ausgabe basiert auf dem
  \href{http://www.gutenberg.net/}{Project Gutenberg}
  EBook \#27446.}

\maketitle

\pagenum{[95]}Im Küchengarten des Kapuzinerklosters in München gingen zwei der
vornehmsten Väter, Pater Gumppenberg und Pater Wildgruber, in
ernstlichem Gespräch über die schweren Zeitläufe zwischen den an
Stangen hochgezogenen Bohnen auf und nieder. »Es tut nicht gut,«
sagte Pater Gumppenberg, »wenn die Frau stärker ist als der Mann,
im Bürgerhause so wenig wie auf dem Fürstenthrone, das habe ich
immer gesagt und darum die savoyische Heirat widerraten. War es
nicht vorauszusehen, daß sie mit ihren welschen Dienern und ihrer
welschen Pracht einwirken und mit ihrem welschen Gespreiz alles und
jedes bei unserem guten Herrn durchsetzen würde?« Es hatte nämlich
vor einigen Jahren der Kurfürst Ferdinand Maria die schöne, stolze
und kluge Henriette von Savoyen geheiratet, die zwar in kirchlicher
Gesinnung niemandem nachstand, aber den Bedarf dazu von jenseits
der Alpen in Gestalt verschiedener Geistlicher italienischer und
französischer Herkunft mitgebracht hatte, unter denen ihr
Beichtvater Filiberto aus dem Orden der Theatiner der
hervorragendste war. »Der Anstand würde erfordern,« fuhr Pater
Gumppenberg fort, »daß diese Fremden sich in die Gebräuche unseres
Landes zu schicken suchten; anstatt dessen fahren sie naserümpfend
daher wie Eroberer und möchten das wohlerprobte einheimische Wesen
mit ihrem scheckigen Tändelkram austapezieren. Es sind bald hundert
Jahre her, daß unsere Stadt und unser Fürstenhaus unter dem Schutze
des heiligen\pagenum{[96]} Benno nicht nur in gutem Zustande
verharren, sondern erst recht zu florieren angefangen haben, wie
denn auch unser seliger Herr, der verstorbene Kurfürst, ihm
allezeit die Ehre gegeben und es an Dank und Verherrlichung nicht
hat fehlen lassen. Auch wir haben uns den Dienst des uralten
deutschen Heiligen angelegen sein lassen, obwohl wir absonderlich
auf den heiligen Franziskus Seraphikus, diesen hochberühmten,
eigentlich aus Gottes Geiste geborenen Himmelsmann, verpflichtet
sind, und haben also doppelte Ursache, von den Welschen, die über
uns gekommen sind, ohne daß wir ihrer bedürfen, die gleiche
Unterordnung zu verlangen. Mögen sie in der Stille verehren, wen
sie wollen; unerträglich aber müßte es jedem gläubigen Bayernherzen
sein, wenn wir in der altheiligen Frauenkirche einen neumodischen
Altar für einen gewissen Cajetan sich spreizen sähen, der uns so
wenig angeht wie ein Derwisch oder Mufti der Heiden im Orient.«

Pater Wildgruber nickte nachdrücklich und fügte hinzu: »Meine
sorgfältigen Erkundigungen haben bestätigt, was ich dir schon
sagte, daß dieser Cajetan erst kürzlich vom Heiligen Vater selig
gesprochen ist, auf das beschwerliche Drangsalieren der adeligen
Familie, die ihn zu ihren Verwandten zählt und ihre schäbige
Herrlichkeit mit seinem Namen ausputzen möchte. Dergleichen Adel
ist, wie du weißt, über den Bergen billig wie der Kies im Bache zu
haben. Es mag sein, daß der gute Mann aus Vicenza sich ein
Verdienst um jenes Völkchen erworben hat, als er den Theatinerorden
stiftete, in den nur Leute seinesgleichen aufgenommen wurden, die
dort eine ansehnliche Versorgung finden. Wenn sie es dabei bewenden
lassen und sich ruhig verhalten, wohl und gut, so mag man es ihnen
gönnen;\pagenum{[97]} ein anderes ist es, wenn sie ihren Familien-
und Standespatron uns hierorts als Heiligen aufdrängen und das Volk
zu dieser fremdartigen, übelberufenen Verehrung anlocken wollen.
Dergleichen Seltsamkeit dürfen diejenigen, die Gott zu Hütern
seiner Schafe bestellt hat, nicht einschleichen lassen.«

Die herzhafte Zustimmung seines Freundes erheiterte Pater
Gumppenberg, so daß er stehenblieb, eine Ranke der kletternden
Bohnen zu sich bog und den Ansatz der angenehmen Frucht, die sich
zeigte, auf ihr Wachstum untersuchte. »In acht bis vierzehn Tagen,
denke ich, können wir ein erstes Bohnengericht auf unserer Tafel
sehen,« sagte er behaglich. »Unser Himmel reift die Gottesgabe
langsam, ist es aber so weit, dann hat sie eine gediegene Würze,
die sich nach meinem Urteil über alle die gepriesenen Erzeugnisse
der Fremde erhebt.«

Auch hierin war Pater Wildgruber derselben Meinung. »Ich bin nicht
von denjenigen,« sagte er, »die ohne einen Tropfen französischen
oder welschen Weines die Mahlzeit fade finden; unser braunes Bier
mag es mit dem vielbeliebten Traubenstoff wohl aufnehmen, ja, indem
es das Blut nicht erhitzt, sondern kühlt, und den erwünschten
Schlaf herbeiführt, anstatt die Sinne zu kitzeln, ist es aus
erheblichen und auch gottgefälligen Gründen dem kostbaren
Nebenbuhler wohl noch vorzuziehen.«

Weiterschreitend gingen die frommen Männer zu der Erwägung über,
wie der von seiten der welschen Geistlichen drohenden Gefahr
füglich entgegenzuarbeiten sei. »Ich schrecke«, sagte Pater
Gumppenberg, »keineswegs davor zurück, unserem Herrn, dem
Kurfürsten, eine dringliche Vorstellung zu machen; habe ich doch
auch das Antlitz seines gestrengen Vaters nicht\pagenum{[98]}
gefürchtet, da ich mich im Schutze Gottes und seiner Heiligen
sicher fühle.«

»Du hattest es damals nicht mit einem Weibe zu tun,« gab Pater
Wildgruber zu bedenken, »einem Weibe, das ich ganz und gar der
Dalila vergleichen möchte, wenn auch unser hochgeliebter Herr, der
Kurfürst, nichts mit dem Helden Simson gemein hat als die
Bezauberung durch seine Gemahlin.«

»Es ist meine Pflicht, mich durch nichts von dem abhalten zu
lassen, was zum Heil unserer Kirche notwendig ist, am wenigsten
durch ein Weib,« sagte Pater Gumppenberg gefaßt, indem er seinen
schwärzlichen, mit vielen grauen Haaren durchschossenen Vollbart
über der breiten Brust auseinanderstrich; »es wird verhoffentlich
nicht ohne Eindruck bleiben, wenn ich den Geist seines hochseligen
Vaters berufe, um meine Vorstellungen zu unterstützen.«

Für den jungen Kurfürsten, der in seinem Vater das Abbild Gottes
auf Erden verehrt hatte, war die Anrufung desselben überwältigend,
und er wagte nicht leicht etwas durchzusetzen, wenn man ihn glauben
machte, daß es seiner Gesinnung zuwider sei. Freilich bekämpfte den
erhabenen Schatten seit seiner Vermählung das lebendige Auge der
geliebten Frau, allein er getraute sich in wichtigen Dingen noch
nicht immer dieser neuen, allzu reizenden Kraft nachzugeben, weil
es ihm unglaubhaft schien, daß irgend jemand, und nun sogar ein
junges Weib, seinem gewaltigen Erzeuger in etwas sollte überlegen
sein können. Demgemäß erwiderte er dem Pater Gumppenberg auf dessen
eindringlichen Vortrag, es sei ihm unbekannt gewesen, daß das Volk
der Einführung des neuen Heiligen so sehr entgegen sei; ein
vortrefflicher Künstler habe zwar den Entwurf zu einem Altare ihm
bereits vorgelegt, die Genehmigung habe er aber noch\pagenum{[99]}
nicht erteilt und werde die Sache einstweilen ruhen lassen, bis die
Vortrefflichkeit des vicentinischen Cajetan in Bayern besser
bekannt und bezeugt sein werde.

Ein wenig betreten begab er sich in das nach dem neuesten
italienischen Kunstgeschmack erst kürzlich fertiggestellte
Wohnzimmer seiner Gemahlin, bei der er zu seinem Troste Pater
Filiberto anwesend fand; denn obgleich dieser der Kurfürstin selbst
einflüsterte, was sie in bezug auf die Religion und die Geistlichen
ihrer Heimat verlangen solle, suchte er doch immer zu begütigen,
wenn der Kurfürst in seiner Gegenwart darüber verdrießlich wurde,
im Vertrauen darauf, daß der Streit sich hernach weiterspänne und
zu dem erforderten Zwecke führe. Henriette Adelaide nahm die
Botschaft ihres Mannes, die er unter mancherlei Scherzen
vorbrachte, unwillig entgegen und sagte: »Das kann ich am wenigsten
leiden, daß du deine eigene Einsicht geringer anschlägst als die
des Pater Gumppenberg, des ungeschlachten Dickschädels. Glichen
alle Menschen dir, so würde die Stadt München in Ewigkeit nichts
anderes werden als ein Häuflein bäurischer Hütten um die
barbarische Ausgeburt der Frauenkirche herum. Soll etwa jeder
Christ zwischen Spanien und Rußland zu jenem Benno beten, dessen
Knochen man in nordischen Urwäldern zwischen den Knochen von Bären
und Auerochsen zusammengelesen haben wird?«

»Er hat doch«, sagte Ferdinand Maria, »vor deinem Cajetan das
voraus, daß er ein regelrechter Heiliger ist, während jener, wie
ich höre, nur der Seligsprechung wert befunden wurde, ihm also doch
wohl etliche schätzenswerte Qualitäten abgehen müssen.«

»Das haben dir die Schelme weisgemacht!« rief Henriette Adelaide
heftig. »Die Heiligsprechung wird seinerzeit schon\pagenum{[100]}
erfolgen, und wenn Cajetan nur eine gute Tat getan hätte, die
beglaubigt ist, so gälte das mehr als hundert Wundertaten eines
Benno, die niemand mit angesehen hat, und von dem niemand beweisen
kann, ob er überhaupt gelebt habe.«

Hier legte sich Filiberto ins Mittel, indem er mit liebenswürdigem
Lächeln einschaltete, daß daran wohl kein Zweifel obwalten dürfe,
da Papst Hadrian im Jahre 1523 den würdigen Bischof und
Heidenbekehrer unter die Heiligen gestellt habe; indessen müsse er
auch bestätigen, was die Kurfürstin mit ihrem hohen Geiste bereits
festgestellt habe, daß die Heiligsprechung des Cajetan der
Seligsprechung sicher nachfolgen werde, so daß man sie schon für
geschehen annehmen könne. Er für seine Person müsse jedoch sagen,
wenn ihm eine Meinung gestattet sei, daß der Kurfürst
tiefdurchdachte Regierungsweisheit an den Tag lege, wenn er es
vermeiden wolle, durch stürmisches Vorgehen Ärgernis in seinem
treuen Volke zu erregen; die Kraft, die dem heiligen Cajetan
innewohne, sei so groß, daß sie sich selbst durchwirken und ihn bei
jedermann beliebt machen würde.

Es werde nie zu befürchten sein, sagte Henriette Adelaide, indem
sie den stolzen Mund ein wenig spöttisch verzog, daß ihr Gemahl
stürmisch vorgehen werde. Sie würde also darauf verzichten, die
vandalische Halle der Frauenkirche durch ein geläutertes Kunstwerk
verschönt zu sehen. Ihr könne es im Grunde gleich sein, da sie
diese Kirche, deren grobe nordische Bauart ihr nun einmal zuwider
sei, sowieso nicht besuche, und für die Theatiner und den heiligen
Cajetan könne anderweitig gesorgt werden, indem man ihnen eine
besondere Kirche baue, was denn besser und gründlicher sei als ein
bloßer, unwillig im fremden Raume geduldeter Altar. Sie warf diesen
überraschenden Plan nicht ohne\pagenum{[101]} Schelmerei hin, ließ
aber nur ein wenig davon aus den beredten Augen und von dem ernsten
Munde lächeln und trat voll Unbefangenheit an das Fenster, indem
sie sagte: »Es ist hier gegenüber Platz genug, um einen großen
Entwurf ins Werk zu richten. Eine weite Kuppel und ein paar
schmuckreiche Türme nach römischer Art wären mehr geeignet, unser
Auge zu erfreuen, als die Wüstenei, aus der wie die Buckel
erschöpfter Kamele hie und da ein paar steile Dächer steigen.«
Ferdinand Maria blickte zunächst nicht aus dem Fenster, sondern auf
die aufrechte Gestalt seiner Frau und sagte zwischen Erstaunen und
Bewunderung schwankend: »Meine Teure, du bist eine neue Semiramis,
und ich fürchte nur, daß mein armes München und vielleicht auch
dein armer Gatte dir zu klein seien. Wie soll ich eine Kirche
ausrichten, da ich schon mit dem Altar angestoßen habe?«

»Für den, der will,« entgegnete sie, »gibt es keine Hindernisse;
aber nicht ein jeder kann wollen.«

Der Beichtvater, dem es an der Zeit schien, die Gatten sich selbst
zu überlassen, pries sowohl die fürstliche Gesinnung seiner Herrin
wie die landesväterliche Bedächtigkeit des Kurfürsten, worauf er um
die Erlaubnis bat, sich zurückziehen zu dürfen.

»Möchtest du nicht auch lieber Beherrscher einer prächtigen Stadt
als eines veralteten Dorfes sein?« fragte Henriette Adelaide ihren
Gatten, als sie allein waren. »Wir genießen des Friedens, wir
können das Geld nach unserem Herzen ausgeben. Richten wir uns denn
eine schöne Wohnstätte her, wie sie uns behagt, nicht jenen
Mönchen, die keine Nasenlänge über ihren Rosenkranz hinausblicken
können.«

\pagenum{[102]}»Ach,« sagte der Kurfürst, indem er einen komischen
Seufzer ausstieß, »wenn ich bedenke, wie kurze Zeit wir in dieser
Wohnstätte bleiben werden, so scheint es mir, daß wir ein wenig zu
viel Lärm darüber vollführen, ob sie so oder so gestaltet ist.«

Die Kurfürstin maß sein feines junges Gesicht mit dem äußersten
Erstaunen und erst nach einer Minute mit aufgehendem Verständnis
seiner Rede. »Wenn du so willst,« sagte sie, »sollte man freilich
in Blockhäusern leben und nichts als Pyramiden zu dauerhafter
Versorgung seiner Leiche bauen.« Sie setzte sich bei diesen Worten
neben ihn auf die steife Lehne eines Damastsessels und küßte ihn
mit zärtlicher Behutsamkeit auf die wohlgebildeten roten Lippen.
»Du hast eine seltsame Art, das Leben anzusehen,« sagte sie, »mit
welcher man nicht viel ausrichten kann. Wer möchte denn Kinder
erzeugen, wenn man beständig vor Augen hätte, daß er sie vielleicht
morgen verlassen muß?«

Aus seinen dunklen Augen fiel ein warmer Strahl auf ihr lachend ihm
zugeneigtes, schönes Antlitz, und er sagte: »Du hast recht, weil du
stark und gesund bist und Gott und dir vertraust. Darum mag ich
auch entschuldigt sein, wenn ich dir schon mehr nachgegeben habe,
als deinem Fürsten und Eheherrn geziemt.« Unter mancherlei
Neckereien und Scherzen zog sie ihn an das Fenster, um ihm die
dürftige Umgebung zu weisen und den Einfall, den der Augenblick ihr
gebracht hatte, zu einem verlockenden Plane auszumalen. Obwohl das
Herz des Kurfürsten gestimmt war, sich durch die Unternehmungslust
seiner Gemahlin hinreißen zu lassen, so hielt er doch mit der
endgültigen Zustimmung vorsichtig zurück. Bei sich bedachte er, daß
sie beide vor seinem Volke anders dastehen würden, wenn er ihm erst
einmal einen\pagenum{[103]} Erben vorzustellen hätte, sprach dies
aber nicht aus, da er wohl wußte, wie übel sie es aufnahm, wenn man
sie daran wie an eine versäumte Schuldigkeit mahnte. Auch ging es
ihm zuweilen durch den Sinn, daß er vielleicht von Gott nicht
bestimmt sei, sich fortzupflanzen; denn er hatte sich von jeher im
Vergleich mit seinem bewunderten Vater als einen schwachen,
unwerten Sproß gefühlt; oder er dachte, daß es der Kurfürstin an
der rechten Liebe zu ihm fehle, und daß diese Ungeneigtheit der
Seele auch ihren Leib gegen ihn verschließe, so daß sie kein Kind
von ihm empfangen könne. Diese traurige Einbildung hatte er einmal,
einer wehmütigen Stimmung nachgebend, ihr gegenüber laut werden
lassen, da sie ihn aber ausgelacht und ihn einen Phantasten genannt
hatte, kam er nie wieder darauf zurück; im stillen hatte er
gehofft, sie werde ihm eine liebevollere Antwort daraus geben.

Die Zurückhaltung ihres Gemahls veranlaßte Henriette Adelaide
nicht, auf ihren Einfall zu verzichten; vielmehr schrieb sie ihrer
Mutter, der Herzogin von Savoyen, sie möge ihr ohne Säumen einen
erfahrenen Baumeister schicken, der die jüngsten Meisterwerke der
Kirchenbaukunst namentlich in Rom durch und durch studiert habe und
willens und fähig sei, seinen Geist auf eine außerordentliche
Erfindung dieser Art zu richten. Kaum war derselbe eingetroffen, so
mußte er den Platz, den sie ausgewählt hatte, besichtigen und die
Zeichnungen und Risse vorlegen, mit deren Besichtigung sie manche
Stunde in ihren Gemächern verbrachte, was alles den Personen, die
einige Wachsamkeit auf das öffentliche Leben richteten, nicht
verborgen blieb. Es währte nicht lange, so stellte sich Pater
Gumppenberg von neuem ein mit kummervoller Anfrage, ob es denn
\pagenum{[104]}an dem sei, daß dem seligen Cajetan nunmehr nicht
nur ein Altar, sondern ein ganzer Kirchenbau errichtet werden
solle, und zwar dem Schlosse preislich gegenüber, gerade als ob ein
sonderbares Einvernehmen zwischen dem Kurfürsten und dem
landsfremden Neuling bestehe? Was für Ursache denn der Kurfürst
habe, mit den bewährten Schutzheiligen des Bayerlandes, die bisher
alles so wohl geführt hätten, unzufrieden zu sein oder ihnen einen
hohlen Namen von jenseits der Berge vorzuziehen?

In dem Bestreben, die Verlegenheit, die ihn überkam, nicht merken
zu lassen, winkte Ferdinand Maria beruhigend mit der Hand und sagte
mit nachdrücklicher Unbefangenheit, da sei kein Anlaß zu Mißtrauen
und Empfindlichkeit! Die Sache sei so: Er habe dem heiligen Cajetan
gelobt, wenn ihm durch seine Fürbitte ein Erbe geschenkt werde, so
wolle er ihm unweit seiner Residenz eine Kirche aufrichten, so
schön er irgend vermöge, und damit er, wenn es so weit sei, sein
Wort lösen könne, habe er einstweilen die Örtlichkeit besichtigen
und einen Voranschlag machen lassen, so daß der Bau ohne
Zeitverlust könne in Angriff genommen werden, wenn der Erbe da
sei.

Bei sich selbst lachte der Kurfürst über die stattliche Antwort,
die ihm in der Not eingefallen war, und die den strengen Pater
augenscheinlich verblüffte und verstummen machte; allein er erholte
sich geschwind und ließ nun eine verdoppelte Mißbilligung und
strafenden Eifer ohne Hehl merken. Es habe zwar, sagte er, gewiß
dem Kurfürsten Gott eingegeben, daß er durch ein frommes,
christkatholisches Gelübde sich den Segen der Nachkommenschaft
erflehen wolle; wie er es aber verantworten wolle, von den Übungen
seiner löblichen Vorfahren eigenmütig\pagenum{[105]} abzuweichen?
Auch seine hochselige Mutter, die gottergebene Kurfürstin Maria
Anna, habe bei dem ehrwürdigen Alter ihres Gemahls um die
Fruchtbarkeit ihrer Ehe besorgt sein müssen; sie habe aber ihre
Zuflucht nicht zu ausländischen Meerwundern genommen, sondern sei
in Demut und Sitte nach Alt-Ötting gepilgert und habe der hölzernen
Maria in der alten Kapelle viele auserlesene Kostbarkeiten
gestiftet, worauf sie denn zum Troste des ganzen Bayerlandes
schwanger geworden sei und ihn selbst, Ferdinand Maria, geboren
habe. Wenn seine Gemahlin dem Beispiel seiner erlauchten Mutter
folgen wollte, so würden die erprobten Heiligen und Fürbitter an
ihr gewiß kein geringeres Wunder als an jener vollziehen, und die
bekümmerten Untertanen würden, der Sorge entledigt, wieder ruhig in
ihren Betten schlafen können. Wenn dann die erwartete Gnade nicht
einträfe, möchte die Kurfürstin es immerhin mit ihren einheimischen
Patronen versuchen, er würde der erste sein, seine Gebete zum
Gedeihen des Werkes mit denen der geliebten Herrschaft zu
vereinigen.

Das Heranziehen seiner in Gott ruhenden Eltern bewegte das Gemüt
des Kurfürsten so sehr, daß er für den Vorschlag des Pater
Gumppenberg völlig gewonnen wurde und erst, als dieser fortgegangen
war, mit leisem Bangen den Widerstand seiner Frau in Betracht zog.
Freilich zog sie die feinen schwarzen Brauen ärgerlich zusammen,
aber mehr deswegen, weil das Unternehmen nicht von ihr oder ihrem
Beichtvater ausgegangen war, als weil sie ihre Pflicht verkannt
hätte, dem Lande einen Kurprinzen zu geben und alles Erdenkliche zu
tun, um das hohe Ziel zu erreichen. Auch hätte ihr eine schöne
Wallfahrt, etwa nach Loreto, als das rechte Mittel voll
eingeleuchtet; aber daß ein\pagenum{[106]} armseliger bäurischer
Ort wie Alt-Ötting für eine Person ihrer Art das Angemessene sei,
hielt sie nicht für wahrscheinlich und lehnte dies mit
Entschiedenheit ab, ohne jedoch die Sache selbst von der Hand zu
weisen. Schließlich erklärte sie sich bereit, nach Andechs zu
pilgern, welches zwar bei weitem wilder und wüster gelegen war als
Alt-Ötting, für sie aber den Vorzug hatte, daß es ihr nicht von den
Kapuzinern aufgedrängt, sondern aus freien Stücken von ihr erwählt
war. Ebenso widersetzte sie sich allen anderen Anordnungen, die
Pater Gumppenberg zu vorschriftsmäßiger Veranstaltung der Reise für
nötig hielt. Er sagte nämlich, den letzten, beschwerlichen Aufstieg
zum heiligen Berg, welcher etwa eine Stunde dauerte, müsse die
Beterin allein, ohne ihr Frauenzimmer oder sonstige Begleiter
ausführen, damit aber keine Besorgnis über ihr Wohlergehen in
dieser Wildnis aufkommen könne, wolle er ihr als Wegweiser und
Beschützer einen Bruder seines Ordens mitgeben; denn einen
geistlichen Gesellschafter bei sich zu haben, sei ihr nicht nur
nicht verboten, sondern empfohlen.

Entrüstet sagte Henriette Maria, das Umherfahren in dem öden
Gebirgslande verspreche ohnehin keine Kurzweil, durch einen
Begleiter aus dem Orden, der ihr nun einmal widerwärtig sei, werde
es ihr vollends unerträglich gemacht, sie wolle ihren Beichtvater,
Pater Filiberto, mitnehmen, an den sie gewöhnt sei, und zu dem sie
Vertrauen habe. Gegen diesen wendete der Kurfürst ein, daß er Land
und Leute nicht kenne und nicht einmal der deutschen Sprache
mächtig sei; sie solle sich doch den Bruder, den Pater Gumppenberg
für sie ausgelesen habe, wenigstens einmal ansehen, es werde gewiß
ein bescheidener, verständiger Mann sein, der ihr nicht lästig
fallen werde.

\pagenum{[107]}Die Kurfürstin erstaunte nicht wenig, als sich ihr
bald darauf ein schöner Jüngling vorstellte mit einem Kranz
bläulich-schwarzer Haare um die breite, kindliche Stirn, mit Augen,
deren dunkelbraune Farbe eine starke Flamme zu verhüllen schien,
mit gerader Nase und schönem, großem Munde, der selten lächelte,
dann aber unversehens eine volle Perlenreihe gelblich-weißer Zähne
sehen ließ. Auf ihre Fragen teilte er mit, daß er aus dem südlichen
Tirol stamme und der jüngste Sohn adeliger Eltern sei, die ihn für
das Klosterleben bestimmt hätten. Sein Benehmen war, wenn auch
nicht höfisch, so doch voll sicherer Würde, die auf gutem Blute und
unantastbarer Unschuld zu beruhen schien. Er war in Blicken und
Worten bei aller Ehrerbietung so zurückhaltend, daß Henriette
Adelaide kaum wußte, wie sie ihn ermuntern sollte; denn sie fühlte,
daß sein kindlich strenger Sinn weder ihrem weiblichen Reiz noch
ihrer fürstlichen Hoheit zugänglich war, und fürchtete ebensosehr
seine Reinheit zu verwirren, wie sie wünschte, in seiner Seele
irgendeinen Widerhall zu erregen. Ihrem Gemahl sagte sie in guter
Laune, daß sie nicht geglaubt hätte, in einer Kapuzinerkutte könne
sich ein so hübscher, wohlerzogener Jüngling verstecken; sie habe
nichts an ihm auszusetzen, als daß er allzu schweigsam sei,
vielleicht indessen sei das Fräulein La Perouse, ihr erstes
Kammerfräulein, das sie nebst mehreren anderen mitzunehmen
beabsichtige, fromm genug, um von ihm eines Gespräches wert
gehalten zu werden. Der Kurfürst war hocherfreut, daß sich das
Unternehmen so friedlich und aussichtsvoll ordnete; doch ließ er
die Gesellschaft nicht abreisen, ohne ihr einen Teil seiner
Leibgarde zum Schutze beigeordnet zu haben unter der Führung ihres
Kapitäns, des Chevaliers La Perouse, der der genannten Hofdame
Bruder war.

\pagenum{[108]}Das Geschwisterpaar entstammte einer uralten,
französischen, in Savoyen ansässigen Familie, beide waren um
mehrere Jahre älter als Henriette Maria und seit sie denken konnten
am Turiner Hofe beschäftigt und beliebt. Beide waren in großer
Frömmigkeit erzogen, was bei dem Fräulein so gewirkt hatte, daß sie
trotz angeborener Lebhaftigkeit sich jede Lustbarkeit versagte und
auch die schuldloseste Ausgelassenheit, zu der Jugend und
Gelegenheit sie einmal fortrissen, durch anhaltende Bußübungen
wieder einzubringen suchte, bis sie zuletzt ein gedrücktes, leicht
geängstetes Wesen erhielt, aus dem die Wärme ihrer Natur zuweilen
rührend hervorbrach. Ihr Bruder hatte im militärischen Dienst und
bei den Gepflogenheiten des männlichen Lebens die klösterliche
Strenge aus dem Kinderdasein beiseitegesetzt und vergessen, nur daß
er die erlernten Formen getreulich beobachtete und das ganze
Glaubenswesen bei den Frauen seiner Familie und denen des achtbaren
Umgangs überhaupt als vornehmstes Erfordernis voraussetzte. Die
Kurfürstin war nach seinem Gefühl unter allen Frauen die in jedem
Betracht vollendetste, und seine Verehrung für sie war so
unbedingt, daß selbst der Kurfürst vor seinem rächenden Schwert
nicht sicher gewesen wäre, wenn er sie durch ihn verletzt gewußt
hätte.

Der Weisung gemäß, die er vom Kurfürsten empfangen hatte, hielt
sich der Chevalier mit seinen Leuten immer ein Stück von dem Wagen
entfernt, der seine Herrin einschloß, doch so, daß er das
umfangreiche Gefährt nie ganz aus den Augen verlor. Zuweilen sah
er, wie der Sommerwind, der über die Hochebene spielte, einen
lichten, aus dem Wagenfenster hängenden Schleier hob und gegen den
braunen Umhang des Kapuziners trieb, der zu Pferde neben der
\pagenum{[109]}Kutsche einherritt, oder ein helles Lachen der
Fürstin überzeugte ihn, daß sie einstweilen mit dem Verlauf ihrer
Wallfahrt nicht unzufrieden war. In Herrsching, wo im größten
Bauernhofe übernachtet wurde, lud Henriette sowohl den Chevalier
wie den Mönch ein, die Abendmahlzeit in ihrer und ihrer Fräulein
Gesellschaft einzunehmen; es wurde dazu eine Tafel vor dem Hause
gedeckt, von wo man das allgemach untertauchende Sonnenfeuer
rötliche Farben über den dämmernden See ergießen und das Leuchten
der fernen Alpen langsam in den Abend versinken sah. Der ländliche
Tisch war mit dem in einem Wagen mitgeführten fürstlichen
Silbergeschirr und Leckereien beladen, feingebackenem Brot, Obst,
Wein und Süßigkeiten; einzig die gebackenen Fische, den edeln
Amaul, den Kilch und die geschätzte Bodenrenke, lieferte der Wirt
als Erzeugnisse seines Landes. Das appetitliche Essen würzte die
Kurfürstin durch die fröhliche Laune, mit der sie ihrem alten
Freund und Diener La Perouse wiedererzählte, was der Mönch ihr
unterwegs von den Wundern des Heiligen Berges erzählt hatte: wie,
während zwischen den Andechsern und den Wittelsbachern eine Fehde
wütete, die Mönche alle Reliquien so gut und tief verscharrten, daß
sie in der Folge nicht wiedergefunden werden konnten, bis nach
vielen Jahren, als gerade ein Franziskaner die Messe feierte, eine
Maus über den Altar sprang, zwischen den Zähnen einen
Pergamentstreifen tragend, auf welchem nicht nur sämtliche
Reliquien, die diesen Wallfahrtsort einst berühmt gemacht hatten,
sondern auch der Ort ihres Verstecks verzeichnet waren.

Jetzt erst bemerkte der Chevalier, was für ein schöner Jüngling der
Mönch war, den er bisher nur flüchtig in Augenschein genommen
hatte, und auch mit wieviel zarter\pagenum{[110]} und bescheidener
Fürsorge die Kurfürstin ihn behandelte. Dies war nun freilich nicht
merkwürdig, insofern der junge Mann eine geistliche Person war,
allein bei seiner Schönheit fiel es schwer, sich dessen bewußt zu
bleiben, und der Chevalier ertappte sich immer wieder darauf, daß
er sich hinter dem unbekannten und unberühmten Jüngling
zurückgesetzt fühlte. An ihm war indessen kein Fehler irgendwelcher
Art zu entdecken: er aß mäßig, trank nichts als Wasser und blickte
mit unbefangenem Ernst auf seinen Teller, wenn er nicht der
Kurfürstin auf eine Anrede Auskunft zu geben hatte.

Den letzten Teil des Weges, der sich zwischen Tannen an einem
wildabrauschenden Bache steil hinaufwand, wollte die Kurfürstin mit
ihrem geistlichen Begleiter zu Fuß zurücklegen; da jedoch die
Frauenzimmer dringend baten, den Heiligen Berg gleichfalls
besteigen zu dürfen, und der Chevalier sich in ritterlicher
Ehrerbietung weigerte, die ihm Anvertraute ganz ohne Schutz der
Waffen zu lassen, gab sie allen die Erlaubnis, ihr in einiger
Entfernung nachzugehen, und schritt tapfer voraus, ohne sich um ihr
Gefolge zu bekümmern. Der Chevalier bekam sie erst am späten
Nachmittage wieder zu Gesicht, als sie alle die vorgeschriebenen
Gebete und die Verehrung der berühmtesten Reliquien vollendet hatte
und sich zum Abstieg anschickte. Sie war von der herben Luft und
der Anstrengung des Steigens gerötet, und es schien ihm, als blicke
sie zerstreut an ihm vorüber, wie es sonst nicht ihre Art war.
Obwohl an dem Kapuziner im Gegensatz zu ihr keine Spur von Erregung
wahrzunehmen, die bräunlichblasse Farbe seines edlen Gesichtes
unverändert, seine Miene ebenso kindlich strenge wie zuvor war, so
konnte der Chevalier doch eines\pagenum{[111]} feindseligen
Verdachtes nicht Herr werden, als habe er ihren Stolz und ihre
Überlegenheit durch irgendeine unerlaubte Einwirkung ins Wanken
gebracht. Während des Abendessens, das im selben Bauernhofe von
Herrsching verzehrt wurde, neckte die Fürstin ihn mehrmals, er habe
solchen Ernst auf die Wallfahrt gestellt, daß ihm das Lachen
abhanden gekommen sei und er künftig nur noch zu einem Führer auf
Pilgerfahrten taugen werde, und wenn er auch die Scherze sich
höflich und untertänig gefallen ließ, schnellte er doch unversehens
einen scharfen, zürnenden Blick auf sie, der sie befremdete und
leise in sich erschauern machte. Sie erhob sich zeitiger als am
vergangenen Tage von der Tafel, indem sie sagte, in dieser
unwirtlichen Gegend sei der Abend feucht und frostig, sie wolle mit
ihren Frauen das Lager aufsuchen, die Männer möchten es halten, wie
sie wollten.

So kam es, daß der Chevalier mit dem Kapuziner alleinblieb, der
sein Brevier aus einer Tasche seiner Kutte zog und, die
langbewimperten Lider über die umflorte Flamme seiner Augen
senkend, still für sich zu lesen anhub. Bei diesem Anblick schwoll
der schlecht bemeisterte Unwille des La Perouse so an, als sollte
er ihm die Brust zersprengen; wider sein Gewissen, das ihn
zurückhalten wollte, machte er seine Stimme stark und sagte
unvermittelt zu dem Lesenden: »Ihr habt da einen kurzweiligen
Auftrag von Euerem Kloster empfangen! Es muß eine Wonne für einen
jungen Mann sein, mit einer schönen Dame wie die Kurfürstin durch
die Wildnis zu lustwandeln.« Der Angeredete hob seine Augen ruhig
von dem Brevier auf und sagte: »Die Kurfürstin würde schöner sein,
wenn sie glücklicher, und glücklicher, wenn sie gehorsamer wäre.«

\pagenum{[112]}Die unerwartete Antwort versetzte den Chevalier in
ein so großes Erstaunen, daß er eine Weile still und steif auf
seinem Sitze blieb und erst, als der Kapuziner sich schon wieder
zum Lesen anschickte, gedämpfter als zuvor fragte, wie diese Worte
zu verstehen seien: Was einer so hochgestellten Dame zum Glück
fehle, und wer von ihr Gehorsam verlangen könne?

Der junge Mönch richtete die Augen seinerseits verwundert auf den
Kapitän und sagte: »Wäre sie glücklich, hätte sie die Wallfahrt
nicht zu machen brauchen, und wenn sie Gott widerstrebt, dem auch
die Kaiser und Könige unterworfen sind, wird ihr auch dieser
Bittgang nicht anschlagen; denn Gott vollbringt zwar Wunder, aber
nicht wider die Natur, und wird sie kein Kind gebären lassen, ohne
daß sie es zuvor empfangen habe. Wenn nun Gottes Ratschluß nicht
ihren Gemahl, sondern einen anderen dazu auserwählt hat, so sollte
sie nach dem Vorbild der allerseligsten Jungfrau Maria sich dem
Herrn unterwerfen, wohingegen sie sich störrisch erweist und durch
keinen als den Kurfürsten Mutter werden will.« Dunkelrot vor Zorn
sprang der Chevalier auf und rief drohend: »Ihr hingegen würdet
gehorsam sein und Euch nicht weigern, wenn Gott Euch zu diesem Werk
auserwählt haben sollte?«

Jetzt errötete auch das Marmorgesicht des Jünglings ein wenig, und
er sagte mit stolzer Gebärde abweisend: »Mich, einen Gottgelobten,
kann Gott dazu nicht befehlen. Ich habe der Fürstin die Meinung der
Kirche erklärt; eine andere Pflicht hat die Welt nicht von mir zu
fordern. Diene ich auch, soweit ich frei bin, Reichen und Armen mit
meinem Leben, so seid ihr alle doch, soweit ich Gottes bin, meiner
nicht mächtig.«

\pagenum{[113]}Dem Kapitän war so zumute, als wenn die Ordnung der
Dinge, so wie sie bis jetzt in seinem Kopfe gewesen war, sich
durcheinanderzudrehen begänne. Er hätte glauben mögen, daß der
Kapuziner ein Einfältiger sei, aber wenn auch aus seinem schönen
Gesichte nicht gerade überflüssiger Verstand sprach, so glänzte
doch in diesem Augenblick eine edle Entrüstung darauf, der
gegenüber er sich seines unsittlichen Argwohns schämte. »Verzeiht
mir,« sagte er, ihm gutmütig die Hand reichend; »Ihr habt mit
unserem Treiben nichts zu schaffen und seid wohl deshalb um so
glücklicher. Erlaubt mir aber, daß ich Euch um eine Erklärung
bitte: Wenn die Ehe ein Sakrament ist, wie kann die Kirche den
Ehebruch anraten oder billigen? Wir sind Katholiken, nicht aber
Ketzer oder Heiden!«

Den inständigen Blick des Chevalier freimütig erwidernd, sagte der
Bruder, hier handle es sich eben um keinen Ehebruch, insofern als
Gott, um dem kurfürstlichen Hause Erben zu schenken, was sonst
Sünde sei, in Recht umwandle. Wie dies möglich sei, das sei für
Menschen unfaßbar und könne auch von der Kirche nur ausgelegt,
nicht in seinem Wesen erklärt werden.

Woran man denn aber erkennen könne, fragte der Chevalier lebhaft,
wer der Gottgesandte sei? Könne sich nicht ein jeder für den
Auserwählten halten und der Kurfürstin mit strafbaren Gelüsten
nachstellen? »Wo strafbares Gelüsten ist,« sagte der Mönch ernst,
»da ist die Hand Gottes nicht. Wen der Geist treibt, den verwirren
keine Zweifel.« Nachdem er das gesagt hatte, beugte er sich mit
einer nachdrücklichen Wendung über sein Brevier, als wolle er zu
verstehen geben, daß er die Auseinandersetzung hiermit für beendet
halte.

\pagenum{[114]}Die Nacht brachte der Chevalier ohne Schlaf zu und
sah am folgenden Morgen bleich und hohlwangig aus, was an dem
blühenden Manne etwas so Auffälliges war, daß die Kurfürstin wieder
Gelegenheit nahm, ihn zu necken, während die Frühsuppe eingenommen
wurde. Wieder empfing sie neben der in angemessener Untertänigkeit
gegebenen Erwiderung den stolzen Blick, der sie seltsam
durchschauerte, obwohl sie sich anstellte, als habe sie ihn gar
nicht aufgefangen. Bei der Rückfahrt führte er seine Leibwache in
einiger Entfernung der Kutsche nach und sah den Kapuziner zu
Pferde, wie er den Kopf in sein Brevier neigte und ihn nur zuweilen
wendete, um einer Anrede der Kurfürstin zu entsprechen; ihr Lachen
indessen hörte er nicht so hell und häufig wie auf dem Hinwege, was
der Ermüdung zuzuschreiben sein mochte.

Als man wieder in der Residenz angelangt war, wurde der Kapuziner
von dem kurfürstlichen Paare liebreich und ehrerbietig entlassen,
war aber nicht zur Annahme eines anderen Geschenkes zu bewegen als
einer dem Kloster zu überweisenden Stiftung, die den Armen
zugutekommen sollte. Er sprach zugleich mit seinem Dank den Wunsch
aus, Gott möge die Wallfahrt der hohen Frau an ihrem Leibe segnen,
wobei sie ein wenig errötete, während der anmutige Ernst seiner
Miene sich nicht um einen Hauch veränderte.

Auch Pater Gumppenberg und Pater Wildgruber forschten vergeblich,
während sie den Bericht ihres jungen Abgesandten entgegennahmen,
nach der Spur eines Eindrucks, den die merkwürdige Reise ihm
hinterlassen habe. »Ich fürchte, wir haben kein Glück mit dem
jesuitischen Systema,« sagte Pater Gumppenberg nachdenklich: »ein
biderbes, altdeutsches Gemüt tut nicht wohl, sich mit den
spanischen Kniffen abzugeben.«

\pagenum{[115]}»Eben darum«, sagte Pater Wildgruber, »hatten wir
doch den Tiroler Buben ausgelesen! Mit diesem muß es einen Haken
haben, und wenn mich nicht alles trügt, sitzt derselbe in seinem
Verstande, sei es nun, daß er zu dumm oder nicht dumm genug für
eine so heikelige Konstellation ist.«

Während die Väter sorgenvoll den Erfolg der Wallfahrt erwarteten,
hatte der Chevalier La Perouse scharfe Schlachten in seiner Brust
geschlagen; denn die Leidenschaft für die Kurfürstin, die ihn
ergriffen hatte, nahm täglich zu und war um so schwieriger zu
bekämpfen, als sein Amt ihn in ihrer Nähe hielt. Es entging ihm
nicht, daß das Feuer, das in ihm wütete, auch sie erfaßte und ihre
Würde und ihren Hochmut schmolz, so daß sie auf Augenblicke wie
erweichtes Wachs erschien, das durch seine Hand geformt werden
sollte. Freilich war sie wiederum herb und launisch, wie sie es nie
zuvor gegen ihn gewesen war, und legte es darauf ab, ihn durch
nichtachtende Behandlung zu reizen. Eines Abends, als sie ihn, der
bestellt war, eine Partie Tricktrack mit ihr zu spielen,
ungebührlich lange im Vorzimmer hatte warten lassen, weil sie mit
ihrem Musikmeister den Entwurf zu einer Oper durchgenommen habe,
und er, ohnehin von Leidenschaft erregt und glühend, sich über die
Vernachlässigung beklagte, sagte sie: »Bei der Arbeit vergaß ich,
daß ich mit Euch spielen wollte,« und betonte die Worte so, daß er
den herabsetzenden Sinn wohl verstand, den sie hineinlegen wollte.
Seine Stirn färbte sich dunkelrot, und indem er rasch auf sie
zutrat, herrschte er sie drohend an: »Aber ich vergaß nicht, daß
ich Euch besiegen wollte!«, riß sie an sich und küßte sie so
gewaltsam, als ob er sie zermalmen wollte. Mit erlöschendem und
todesseligem Herzen ertrug sie die liebkosende Mißhandlung und war
ihrer sich kaum noch\pagenum{[116]} bewußt geworden, als er sie
freigab, vor ihr niederkniete und sagte: »Nun lege ich mein Haupt
vor Eure Füße.« Die Brust noch ungestüm wogend, das Antlitz bleich,
erschien er ihr herrlicher als je zuvor; Tränen brachen aus ihren
Augen, und mit bebender Stimme sagte sie: »Ich bin die Schuldige!«,
worauf sie schnellen Schrittes das Zimmer verließ.

Unter bitteren Schmerzen erkämpfte Henriette Adelaide am nächsten
Tage den Entschluß, den unseligen Mann zu sich zu befehlen und ihn
mit angemessenen Worten in die Stellung zurückzuweisen, die ihrer
und seiner Würde und Ehre entsprächen; jedoch, sowie sie ihn
eintreten sah, mit dem sieghaften Blick in den großen Augen, mit
den Händen, die, fein und gepflegt, sie doch grausam angefaßt und
ihr Schmerz zugefügt hatten, erlahmte ihre mühsam gesammelte Kraft
sogleich, und anstatt einer Herrin stand sie ihm, einer Sklavin
nicht ungleich, gegenüber. Ein solches Gefühl hatte sie noch
niemals vorher empfunden, und es war ein Rausch für sie, sich ihm
hinzugeben, in welchem Zustande es ihr erschien, als sei sie jetzt
in den Mittelpunkt des Lebens gerissen worden. War er nicht bei
ihr, so fühlte sie sich zwar geängstigt und gequält; allein sie
brauchte sich nur die Empfindung zu vergegenwärtigen, mit der sie
die Wange an seine Schulter schmiegte, um über jeden Zwiespalt hoch
emporgehoben zu werden.

Voll Sorge und Schrecken bemerkte das Fräulein La Perouse, was
vorging; ihr Bruder, an den sie sich mit flehenden und drohenden
Vorstellungen wendete, verwies ihr die Einmischung, und der
Kurfürstin gegenüber wagte sie keine Andeutung zu machen. Sie wußte
in ihrer Not kein anderes Hilfsmittel, als in der Kirche oder in
ihrem\pagenum{[117]} Schlafgemach vor dem Betpult zu weinen und zu
beten, und durch ihre Anwesenheit, zu der ihr Amt als
Kammerfräulein ihr Anlaß gab, den beiden Frevlern das Alleinsein zu
erschweren. Das verweinte Gesicht ihrer Gesellschafterin trug dazu
bei, die Kurfürstin trübe zu stimmen, wenn sie von dem Gegenstand
ihrer Leidenschaft entfernt war. Vor allem aber ergriff sie der
Anblick ihres Gemahls, so daß ihr zuweilen, wenn sie bei den
Mahlzeiten ihm gegenübersaß, Tränen in die Augen traten, ohne daß
sie einen Grund dafür angeben konnte. Eine solche Reizbarkeit war
ihr früher nicht eigen gewesen, da im Gegenteil der Strahlenglanz
ihres Blickes noch selten durch Weinen verwischt worden war, und
der Kurfürst besorgte ernstlich, ihre Gesundheit könne etwa bei der
Wallfahrt Schaden gelitten haben. Überhaupt, dachte er, hätte er
sie nicht dazu veranlassen sollen, weil vielleicht der Gedanke für
sie belastend sei, daß sie ihn im Hinblick auf den mangelnden Erben
enttäusche.

Eines Tages begab es sich, daß der Kurfürst, um seine Gemahlin zu
zerstreuen, sie einlud, ihn in den Grottenhof zu begleiten, wo die
beschädigte Figur einer Nymphe durch italienische Arbeiter
wiederhergestellt wurde. Ein etwa dreijähriges Kind, das zu diesen
gehören mochte, spielte am Rande des Beckens, aus dem die Schale
mit der Statue des Perseus aufsteigt, und bückte sich eben mit
ganzem Leibe so tief über das Wasser, daß der Kurfürst, im Gefühl,
es sei in Gefahr hineinzufallen, es rasch ergriff und auf seinen
Arm hob. Sogleich eilte einer von den Arbeitern erschrocken hinzu,
entschuldigte die Anwesenheit des Kindes und wollte es dem
Kurfürsten abnehmen; der jedoch machte eine beschwichtigende
Bewegung mit der Hand, als bedürfe es der\pagenum{[118]}
Entschuldigung nicht, und nickte dem Kinde zu, das über den
plötzlichen Eingriff des fremden Mannes zunächst ein wenig
entrüstet zu sein schien. Seine Versuche, mit ihm zu spielen, ließ
es sich gefallen, ohne sie gerade zu billigen, und betrachtete ihn
trotzig und aufmerksam untersuchend, wobei es einen mit Rubinen und
Diamanten besetzten Knopf entdeckte, der nebst anderen ähnlichen
sein Gewand schmückte. Nachdem es ihn eine Weile mißfälligen
Blickes angesehen hatte, ergriff es ihn plötzlich mit einer seiner
kleinen schmutzigen Hände und riß ihn, sich kräftig gegen die Brust
des Kurfürsten stemmend, mit entschlossenem Ruck los. Der Kurfürst
lachte, drückte einen Kuß auf den kleinen trotzigen Mund des Kindes
und sagte: »Behalte diesen, aber die anderen lasse mir!« worauf er
es niedersetzte und dem Vater empfahl, es besser zu beaufsichtigen,
damit es nicht in das Wasser falle.

Im Weitergehen schob Henriette Adelaide leise ihren Arm in den
ihres Mannes, und als sie im Münzhof angekommen waren, der kalt und
stillag, ergriff sie seine Hand, zog sie zärtlich und demütig an
ihre Lippen und sagte: »Nur dich habe ich lieb! nur dich, nur
dich!« mit anschmiegender Stimme. Nach einem Augenblick der
Überraschung zog er sie rasch an sich, küßte sie und sah ihr ins
Gesicht, wobei er wahrnahm, daß ihre Augen feucht waren, und
fühlte, daß ihr Herz unruhig klopfte. Wie sie so standen, kam ihr
der Gedanke, daß sie dem schmählichen Zustande, in dem sie lebte,
jetzt ein Ende machen müsse, bevor ihr die Kraft dazu wieder
entfiele, und sie sagte, den Arm ihres Mannes fest umfassend: »Ich
möchte dir, lieber Freund, den Vorschlag zu einer Veränderung in
meinem Hofstaate machen. Du weißt, daß mein Vater mir den
\pagenum{[119]}Chevalier La Perouse als Oberhofmeister mitgab, da
ich ihn seit meinen Kinderjahren kenne und an seinen Umgang gewöhnt
bin. Nun scheint es mir aber, daß für einen Mann seiner Art dies
Amt zu höfisch und weichlich ist, und daß er sich nach der
rühmlicheren Laufbahn des Soldaten sehnt, und ich möchte ihn in so
berechtigten Wünschen nicht hindern, vielmehr fördern.«

Der Kurfürst sah sie groß an und sagte mit fester Stimme: »Es ist
mir vor einiger Zeit einmal aufgefallen, daß La Perouse einen Blick
auf dich warf, der einem Funken glich. Hängt dein Gesuch mit der
Gefahr zusammen, die das Umherfliegen zündenden Stoffes mit sich
bringt?«

Sie erwiderte tapfer, indem sie sich ein wenig aufrichtete und den
Kopf hob: »Ja, so ist es,« und wollte die Bitte hinzufügen, er möge
den Chevalier nicht ein Gefühl entgelten lassen, das sie sich
anklagen müsse, nicht im Keim unterdrückt zu haben. Er jedoch
unterbrach sie, indem er sagte: »Was ich nicht gesehen habe, ist
nicht. Der Chevalier hat keine andere Schuld als jenen Blick, den
ich zufällig aufgefangen habe, und diese vergebe ich ihm.« In
seinem Wesen und seiner Haltung lag, wie er das sagte, etwas
Königliches, das Henriette Adelaide schweigen machte; sie lehnte
sich still an ihn, worauf er mit herzlicher Sicherheit einen Kuß
auf ihre Stirn drückte und ihr, durch das Antiquarium schreitend,
die Herkunft und den Wert verschiedener Kunstmerkwürdigkeiten
erklärte, die dort aufgestellt waren, ohne daß eines von ihnen die
schaurige Kälte empfunden hätte, die bei der herbstlichen
Jahreszeit in dem breiten Gewölbe herrschte.

Seit diesem Tage war der Kurfürst von zuversichtlichem Frohsinn
beseelt, und wenn er ohnehin stets bedacht war,\pagenum{[120]}
seine Gemahlin zu erfreuen, so legte er es jetzt vollends darauf
ab, den leichten Schleier von Niedergeschlagenheit von ihr
abzulösen, der die stolze Heiterkeit ihres Wesens seit einiger Zeit
verhüllte. Zu dem Zweck ermunterte er ihre frühere Baulust und
begann von der Kirche des heiligen Cajetan zu sprechen, die sie
hatte begründen wollen, was denn auch nicht verfehlte, ihre
Teilnahme zu erregen und ihre Unternehmungslust anzuschwellen. Bald
lag ihr künstlich eingelegter Tisch aus Lapislazuli und Korallen,
dessen Fläche ein dunkelblaues, von scharlachroten Segeln
durchflammtes Meer darstellte, voll von den Plänen römischer
Kirchen und den neuen Entwürfen des italienischen Baumeisters, die
den Sommer über in Vergessenheit geraten waren, und es wurde
beschlossen, sowie die günstige Frühlingszeit einträte, mit der
Arbeit zu beginnen.

Als im Scheine der kräftigen Maisonne auf dem der Residenz
gegenüberliegenden freien Platze ein frohgemächliches Hantieren von
Arbeitern sich zu entfalten begann und das Volk von der
jesuitischen Kirche munkelte, die dort errichtet werden sollte,
schürzte sich Pater Gumppenberg noch einmal und erschien mit
ernstem Vorwurf vor dem Kurfürsten. Dieser hörte die Klage ohne
Verlegenheit, vielmehr fragte er erstaunt, ob der ehrwürdige Vater
sich nicht erinnere, daß er selbst ihm, dem Kurfürsten, geraten
habe, es mit dem heiligen Cajetan zu versuchen, wenn die Wallfahrt
nicht fruchten werde? Er habe den ganzen Sommer und Winter hindurch
gehofft und geharrt, nun habe er betrübt auf den erflehten Segen
verzichtet und beschlossen, das Anliegen seines Hauses dem heiligen
Cajetan anheimzustellen, obwohl derselbe dem Bayerlande fremd und
in keiner Weise verbunden sei. In der Hoffnung, daß er sich
\pagenum{[121]}wundertätig erweise, errichte er ihm einen würdigen
Altar, damit er und seine Untertanen ihm Dank und Preis darbringen
könnten für die Hilfe, die er ihnen in der Bedrängnis erwiesen.

Diese Erklärung schlug zwar den Kampfesmut des Pater Gumppenberg
beträchtlich nieder, doch gab er seine Sache noch nicht völlig auf,
sondern rückte dem Kurfürsten einige Gegengründe vor: Ob man denn
in Bayern schon so rat- und mittellos sei, daß man gleich zum
Entlegensten greifen müßte? Wenn die Frau Kurfürstin, wie er
empfohlen hätte, nach dem Herkommen zu der hölzernen Maria in
Alt-Ötting gepilgert wäre, möchte alles andere gekommen sein. Ob da
nicht der heilige Franz Xaver, der heilige Michael und vor allen
Dingen der schon vielerwähnte heilige Benno wären? Hätte nicht der
heilige Benno sogar den blutdürstigen Schwedenkönig im Zaune
gehalten, daß er, ohne der Kirche und der Stadt einigen Schaden zu
tun, vorübergezogen wäre? Hätte nicht die heilige Mutter Gottes
mancher Pest Einhalt geboten, die sonst Land und Leute verzehrt
haben würde? Was aber würde in dieser Art von dem sogenannten
heiligen Cajetan berichtet? Und wenn er auch einmal ein Wunder
täte, so müßte man noch zweifeln, ob es zum Guten ausschlüge.

So weit wolle er sich nicht einlassen, entgegnete der Kurfürst,
wisse auch nicht, wie es zu verstehen sei, daß seine
Nachkommenschaft, die Enkel so hochberühmter Ahnen, übel
ausschlagen sollten.

Pater Gumppenberg entschuldigte sich, daß er so etwas, als des
kurfürstlichen Hauses treuester Knecht und Berater, niemals habe
andeuten wollen. Auch würde er der erste sein, dem heiligen Cajetan
zu danken, wenn durch seine Fürbitte\pagenum{[122]} die Kurfürstin
gesegnet werden sollte; nur gebe er zu bedenken, ob der Kurfürst
nicht erst die Gnade erwarten wolle, bevor er den Dank darbringe,
wie das jederzeit mit einem Gelübde gehalten worden sei. Nein,
erwiderte der Kurfürst freundlich, er gedenke nun einmal, den
Heiligen durch sein Vertrauen und seine Dienstwilligkeit zuvor zu
ermuntern. Suche doch ein Volk die Huld eines fremden Monarchen
durch Geschenke sich zu verdienen, anstatt daß es die empfangene
belohne; und nicht weniger ehrerbietig wolle er sich Gott und
seinen Heiligen gegenüber verhalten.

Es war erst ein kleines Kränzlein von Steinen an dem neuen Bau
ablegt worden, als die glorreiche Kunde sich verbreitete, die
Kurfürstin sei guter Hoffnung, und wenn Gott sich ferner gnädig
erweise, werde die Not des Landes binnen Jahresfrist ein Ende
nehmen. Nun wurde lustig gebaut, und als zu rechter Zeit ein
fürstlicher Knabe das heitere Licht der Stadt München erblickte,
stiegen die Mauern vollends hurtig empor, ohne daß fernerhin jemand
ein Mißfallen daran laut werden zu lassen hätte wagen dürfen.
Freilich erst wenige Jahre vor ihrem Tode konnte Henriette Adelaide
die im fröhlichen Triumphe thronende Kuppel vollendet sehen, von
der unermeßlich schwebenden Himmelsrotunde lächelnd umwölbt.

Nach der Geburt des Kronprinzen blieb die Gesundheit der Fürstin
gebrechlich, wiewohl sie in ungetrübtem Eheglück bis an ihr Ende
lebte und noch manches herrliche Lustgebäude errichtete und
auszierte. Indes sie glanzvolle Feste voll Musik und schöner
Symbole anordnete und in bedeutungsvollen Tänzen selbst ausführte,
übte das Fräulein La Perouse die Reue und Buße, die nach ihrem
Dafürhalten durch die strafbare Leidenschaft ihres Bruders zu
seiner\pagenum{[123]} Herrin verpfändet war. Abend für Abend las
sie stundenlang in Gebetbüchern und rollte, auf dem Betschemel
kniend, unermüdlich Rosenkränze ab, wodurch ihr Gesicht länger,
ihre Wangen hohler, ihre Augen tiefer wurden. Dieser Gewohnheit
frönend, verursachte sie in einer Nacht den Brand des Schlosses,
der nicht wenig Kunst und Pracht zerstörte und dazu noch das zarte
Leben Henriette Adelaides antastete; denn sie vermochte die Folgen
des Schrecks und der Flucht vor den nachjagenden Flammen nicht zu
überwinden und starb nach kurzem Kränkeln, ihren Gemahl nach sich
ziehend, der nur noch auf Erden zu verweilen schien, um ein Denkmal
unsterblicher Trauer über dem verschwindenden Leibe zu errichten
und sich dann für die Dauer der Grabesruhe ihm beizugesellen.
\end{document}

