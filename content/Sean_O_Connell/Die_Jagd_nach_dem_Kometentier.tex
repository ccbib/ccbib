\usepackage[ngerman]{babel}
\usepackage[T1]{fontenc}
\hyphenation{wa-rum Fracht-raum}
\hyphenation{schien}
\hyphenation{Tief-ebe-ne Tief-ebe-ne gro-ßen}


%\setlength{\emergencystretch}{1ex}

\newcommand\bigpar\medskip
\newcommand\gedanke\textit

\begin{document}
\raggedbottom
\begin{center}
\textbf{\huge\textsf{Die Jagd nach dem Kometentier}}

\bigskip
Sean O'Connell

\end{center}

\bigskip

\begin{flushleft}
Dieser Text wurde erstmals veröffentlicht in:
\begin{center}
Die Steampunk-Chroniken\\
Band I -- Æthergarn
\end{center}

\bigskip

Der ganze Band steht unter einer
\href{http://creativecommons.org/licenses/by-nc-nd/2.0/de/}{Creative-Commons-Lizenz.} \\
(CC BY-NC-ND)

\bigskip

Spenden werden auf der
\href{http://steampunk-chroniken.de/download}{Downloadseite}
des Projekts gerne entgegen genommen.
\end{flushleft}
\vfill


Sean O'Connell ist in Cromer, England, geboren und in London und
Lindau am süddeutschen Bodensee aufgewachsen. Derzeit wohnt er in
Ravensburg/Weingarten.

Das Schreiben und Arbeiten im künstlerischen Bereich hat ihn von
Anfang an begleitet: Radiomoderator und Redakteur, Mitinhaber einer
Videoproduktionsgesellschaft und ab 1998 Redakteur bei einer
österreichischen Landeszeitung. 2001 wurde er Leiter einer in
Österreich und Süddeutschland tätigen Werbeagentur und wechselte
schließlich 2002 in die Computerbranche.

2005 begann er mit dem Schreiben an dem phantastischen
Genre-Crossover-Roman »Tír na nÓg«. Das Hörbuch ist 2010 im
Action-Verlag erschienen, die Printversion wird Ende 2011 im
ACABUS-Verlag publiziert, und die Geschichte wird eine Fortsetzung
mit »Túatha Dé Danann« erfahren. Darüber hinaus hat er auch noch
zahlreiche Kurzgeschichten verfasst, von denen eine Reihe in der
eBook-Storysammlung »Verloren im Intermundium«
(Chichili-Satzweiss-Verlag) erhältlich sind.

\texttt{http://wortwellen.wordpress.com/}

\newpage

\section{Die Jagd nach dem Kometentier}

»Füllt die Segel mit Sonnenwind, volle Fahrt voraus!« Die HMS
\textit{Pequod} glitt aus den letzten Atmosphäreschichten in die finstere
Dunkelheit des Alls. Ein lautes Ein- und Ausatmen erfüllte den
Schiffsrumpf, als wäre er ein lebendiges Wesen. Die Pumpen im
Maschinenraum waren wie riesige Lungen, gigantische Blasebalge. Sie
saugten den sie nun umgebenden Æther in großen Mengen ein und
verwandelten ihn beim Ausatmen in Sauerstoff. Eine frische Brise
streifte ihre Gesichter. »Hart am Wind bleiben!« – »Aye, aye!«

Die Sonne kam in Sicht. Ein gigantischer, feuriger Ball im
glitzernden Feld der Sterne. Krakenarmige Protuberanzen griffen
hinaus in den Æther, verglühten im Dunkel. Leuchtende Bögen
gleißenden Lichts brannten sich scheinbar für immer in die
Netzhäute der Offiziere und Matrosen.

»Kurs nehmen. Vor den Wind drehen!« – »Aye, aye!« Die Sonne glitt
davon, kippte einfach nach Backbord weg. Das Schiff ächzte und
stöhnte unter dem Kurswechsel, aber es war in Wirklichkeit nur das
Geräusch des Sonnenwindes, der auf die blasenartige Ætherhülle
prasselte.

Kurz darauf zeigte der Bugspriet auf das bleiche Gesicht von
Enceladus, den fahlen Mond des Saturns, der – für das Auge noch
unsichtbar – in den Tiefen des Alls auf sie wartete. Der
Rendezvouspunkt für ihre Rückkehr.

Die Sonnensegel bauschten sich hoch über den Köpfen der Männer.
Bram- und Marssegel am Großmast füllten sich mit Sonnenwind. Einige
Matrosen enterten freudig auf, krochen in die Krähennester an
Fock-, Groß- und Kreuzmast und begannen mit der Observation von
herumirrenden Asteroiden, Meteoriten und anderen Himmelskörpern.
Jeder Mann der HMS \textit{Pequod}, der nicht im Inneren des Schiffs seinen
Dienst verrichtete, stand an Deck und sah hinauf zu dem silbernen
Flies der Segel, das in diesem Moment wie ein goldenes Feuer im
Licht der Sonne flackerte. Einige der Männer meldeten sich kurz von
ihren Posten ab, um nach Achtern zu eilen, einen letzten Blick auf
den blauen Planeten zu erhaschen.

»Kurs angelegt, Sir!«, rief Commander Binky. »Die Jagd kann
beginnen.«

Der Kapitän nickte. »Ein historischer Tag, Edward.«

»Ja, Sir. Waidmannsheil, möchte ich sagen!«

Der Kapitän lächelte. »Waidmannsdank, Edward! Ich bin
zuversichtlich, dass wir noch vor Weihnachten das Kometentier
erlegt haben und zurück in England sein werden.«

Binky nickte. Er summte leise vor sich hin und wippte mit den
Füßen.

»Sie lieben die Ætherfahrt, nicht wahr, Edward?«

»Ja, Sir. Ich bin an der Küste geboren. Schon als Kind liebte ich
es, den auf den Wellen dahin gleitenden Schiffen zuzusehen. Aber
hier zwischen den Sternen ist es noch viel schöner als auf dem
Meer, Sir.«

Die HMS \textit{Pequod} nahm endgültig Fahrt auf, brachte die Besatzung fort
von der Erde, ihrer alten Heimat, und führte sie hinaus an neue,
unbekannte Orte.

Als der Mars einige Tage später in Reichweite kam, war es fünf
Glasen der Mittelwache als Vollmatrose Willy Smith vom Oberdeck den
Niedergang hinunter stürmte und erst inne hielt, als er die Kajüte
des Kapitäns erreichte. Er klopfte laut und vernehmlich. »Sir, Sir,
wachen Sie auf, es ist soweit! Phobos und Deimos sind jetzt
Backbord!«

Der Kapitän trat Augen reibend auf den Gang, zog mit tapsenden
Bewegungen feste Plünnen an und folgte dem Matrosen an Deck.

Leewärts lag jetzt der Mars, eine blutig rote, riesige Kugel. Sein
Licht ließ das Deck rötlich erglühen. Der eiförmige Körper von
Phobos, einem der beiden Monde, schob sich kurz darauf bedenklich
nah vor die HMS \textit{Pequod}. Deimos, der kleinere Trabant, befand sich
noch weitgehend im Rücken des Roten Planeten, aber es würde
vermutlich nicht mehr lange dauern, bis auch er ganz sichtbar
werden würde. Die Mars-Trojaner, Asteroiden auf einer stabilen
Umlaufbahn an den Lagrange-Punkten $L_{4}$ und
$L_5$, blitzten im schwachen Licht der geschrumpften
Sonne.

\gedanke{Welch ein außergewöhnlicher Anblick}, dachte der Kapitän.

Der Zweite Steuermann, Christopher Pine, rief vom Poopdeck zu ihnen
hinunter. »Sir, wir werden über Ætherfunk gerufen! Die MS \textit{Utopia
Planitia} entbietet ihre Grüße. Sie haben von unserer Jagd auf das
Kometentier gehört und wollen sich uns anschließen.«

»Aye! Das konnte wohl nicht ausbleiben«, erwiderte der Kapitän
missmutig. »Aber wir werden es ihnen nicht leicht machen! Wecken
Sie die Matrosen und die Offiziere der Morgenwache, lassen sie alle
Segel setzen! Das Rennen ist noch nicht entschieden, und wir haben
bereits einen Vorsprung. Es wäre doch gelacht, wenn wir nicht ein
brauchbares Startfenster bekommen und das Kometentier vor diesen
hochnäsigen Kolonisten fangen würden! Schicken sie Männer hinunter
in die Last, man soll eine Extraportion Rum an Deck bringen – als
Belohnung, wenn es uns gelingt, der \textit{Utopia Planitia} davonzusegeln!«
Er lachte. »Alle Mann in die Wanten!« – »Aye, aye, Kapitän!«

Und so beeilte sich die \textit{Pequod}, den riesigen Mars zu umrunden, um
Schwung zu holen für den Sprung zu den äußeren Planeten.

Schließlich war es geschafft, die \textit{Utopia Planitia} blieb zurück und
wurde zusehends kleiner, während die \textit{Pequod} mit vollem Schub den
äußeren Planeten entgegeneilte.

Jupiter lag bereits ein geraumes Stück zurück, da sichtete einer
der Matrosen im Krähennest, es war Mr. Henry A. Mulhouse, das
seltene Kometentier. Saturn war in diesem Augenblick bereits zu
einer großen, bleichen Scheibe angeschwollen und lag auf elf Uhr.
Seine faszinierenden Ringe glitzerten frostig und abweisend.

Die Besatzung wurde in Alarmbereitschaft versetzt, die
Raketenharpunen über präzise Gewinde ausgerichtet, und der Kapitän
befahl dem Steuermann halbe Fahrt. Für die Offiziere wurde Tee an
Deck serviert.

»Nur nichts überstürzen jetzt. Das Kometentier ist launisch«, sagte
der Kapitän. »Wenn wir es zu früh aufschrecken, flieht es
vielleicht zurück in die Oortsche Wolke. Wenn es erst einmal dort
ist, können wir es nicht mehr fangen, weil das Sonnenlicht so tief
im Raum nicht ausreicht, um unsere Segel zu befeuern.« Er gab
Subalternoffizier Dingle ein Zeichen. »Ich erteile Ihnen hiermit
Feuererlaubnis für die Harpunen. Sagen sie den Männern, dass sie
die Beute ins Visier nehmen sollen.« Er wandte sich an Steuermann
Robert Collins, einen der zuverlässigsten Männer an Bord. »Bringen
Sie uns auf 10.000 Faden an die Kreatur heran! Und seien Sie auf
der Hut, die Bestie darf uns nicht entdecken.« – »Aye, aye.«

Und dann schlich sich die \textit{Pequod} wie ein Panther auf der Pirsch von
hinten an das riesenhafte Kometentier heran, von dem in diesem
Augenblick nur sein silberner Schweif erkennbar war. Der Kapitän
stand am Bug, trank seinen Tee, Earl Grey, und beugte sich
vorsichtig über das Schanzkleid des Schiffes, um besser sehen zu
können. Eine seltsame Unruhe hatte ihn erfasst.

»Sind auf Schussweite heran, Sir!«, rief der Steuermann von
Achtern.

Der Kapitän winkte seine Leutnants herbei. »Lassen Sie die Pinassen
bestücken. Die Männer sollen mobile Harpunen mitnehmen. Wir machen
es auf die alte Art und Weise.«

»Auf die \emph{alte Art und Weise}, Sir?«, fragte Leutnant Gram
stirnrunzelnd. »Das \ldots{} das \ldots{}«

»Keine Sorge, Mr. Gram, das ist schon in Ordnung. Kommen Sie, alter
Knabe, spüren Sie nicht auch das Jagdfieber?«

»Ja, Sir! Ganz deutlich, Sir!«, Gram nahm Haltung an, salutierte
und eilte davon, um die Beiboote bemannen zu lassen.

Kurz darauf lösten sich die vier Pinassen von der HMS \textit{Pequod} und
durchpflügten den dunklen Æther, der sich zwischen Schiff und
Kometentier spannte. Doch obwohl sie vorsichtig waren und sich aus
einem toten Winkel der Kreatur näherten, wurden sie bemerkt. Das
Kometentier entfaltete Teile seiner hauchfeinen segelartigen Haut,
streckte sich und glitt lautlos davon, direkt auf den Gasriesen
zu.

Der Kapitän ließ die Pinassen ausschwärmen. Die \textit{Pequod} folgte den
Beibooten in einem gebührenden Abstand.

»Wir kommen gefährlich nah an das Ringsystem heran«, gab einer der
Männer zu bedenken. Es war Mr. O’Reilly, der einzige Ire in der
Mannschaft. »Es lockt uns vielleicht in eine Falle. Die Felsen des
Ringsystems könnten die Pinassen zerstören!«

Der Kapitän sah auf. »Das Biest ist wahrlich gerissen, Mr.
O’Reilly, da haben Sie Recht. Als besäße es Intelligenz. Aber das
ist etwas, das wir ausschließen können. Machen Sie die Harpunen
bereit. Feuern sie auf mein Kommando!«

Die schlanken Pinassen eilten im Schub des Sonnenwinds wie Mücken
hinter dem Kometentier her und holten es langsam ein. Doch als der
Kapitän die Hand zum Feuern hob, tauchte das riesige Ætherwesen in
einem flachen Winkel direkt in das weitreichende Ringsystem des
Saturns ein und verschwand darin.

»Ausschwärmen!«, befahl der Kapitän. »Es kann noch nicht weit sein.
Wir werden es einkreisen und zur Strecke bringen. Haltet nach
seinem silbernen Schweif Ausschau!« – »Aye, aye!«

Die Pinassen navigierten vorsichtig durch die schier endlos vielen
Bruchstücke eines vor Äonen an den Gezeitenkräften des Saturns
zerbrochenen Mondes, und die Schiffsjungen gaben jetzt mit
ängstlichen Stimmen die immer geringer werdenden Abstände zu den
Felsbrocken an die hoch konzentrierten Steuermänner weiter.

»Ich hasse das, Sir«, sagte O’Reilly mit bebender Stimme, »man
sieht ja bald nicht einmal mehr seine eigene Hand vor lauter
Gesteinsbrocken.«

Der Kapitän kniff seine Augen zu Schlitzen zusammen. »Alle
Maschinen halt! Ich glaube, ich habe es entdeckt!« Er streckte die
Hand aus und deutete auf eine kleine Gruppe schwebender Felsen
direkt vor ihnen. »Da vorne!«

Die Harpunisten richteten ihre Waffe auf das neue Ziel und warteten
auf die Anweisungen des Kapitäns.

»Geben Sie Feuer!«

Die raketengetriebenen Harpunen durchschnitten die Nacht, und als
sie ihr Ziel erreichten, explodierten in der Ferne wunderschöne
kleine Feuerblumen. Ein unwirtlicher, urzeitlicher Schrei ertönte.
Das Kometentier war getroffen. Dann schoss es plötzlich zwischen
zwei Gesteinsbrocken hervor, direkt auf die vier Pinassen zu.

»Beidrehen!«, rief der Kapitän.

»Beidrehen!«, riefen die Männer und die Pinassen krängten, während
die Ruder herumgerissen wurden. Doch das Kometentier war schneller.
Zwei der vier Pinassen wurden mittschiffs getroffen und brachen
auseinander. Die Besatzung wurde davon geschleudert, und die
wenigen Überlebenden, die hilflos durch den Æther trudelten, wurden
von den herannahenden Gesteinsbrocken des Ringsystems getroffen und
allesamt getötet.

Das Kometentier hatte sich inzwischen freigeschwommen und nahm Kurs
auf die \textit{Pequod}.

»Folgt der Bestie!«, befahl der Kapitän mit ausdruckslosem Gesicht.
Seine Finger waren zu Fäusten geballt. »Ich will es erlegt wissen,
bevor es mein Schiff zerstört!« – »Aye!«

Die Pinassen drehten bei und nahmen die Verfolgung auf. Das
Ætherwesen hatte sich einen beachtlichen Vorsprung
herausgearbeitet, und so konnten der Kapitän und seine Männer nur
mehr stumm mit ansehen, wie die Kreatur die HMS \textit{Pequod} mit voller
Wucht rammte und ein riesiges Loch in den hölzernen Rumpf riss. Nur
der Größe des Schiffes war es zu verdanken, dass es nicht
vollständig auseinanderbrach. Das Kometentier, das immer noch wilde
Schreie von sich gab, flog eine Kurve und kam zurück, um sein Werk
zu vollenden. Die Pinassen waren noch nicht in Schussweite, und auf
dem Mutterschiff saß offenbar der Schock tief, denn kein Harpunist
feuerte seine Waffe ab.

\bigpar

In diesem Moment glitt ein fremdes Schiff heran und ehe das
Kometentier in Angriffsposition drehen konnte, wurde es von vier
Harpunen getroffen. Es gab ein letztes, markerschütterndes Stöhnen
von sich, dann herrschte Ruhe im Æther.

Die MS \textit{Utopia Planitia} schwebte majestätisch zwischen der HMS
\textit{Pequod} und den beiden verbliebenen Pinassen.

»Ahoi, Erdbewohner! Wir grüßen euch! Das Kometentier ist erlegt!«,
ertönte die Stimme des marsianischen Kapitäns über Ætherfunk. »Eine
wahrlich schöne Jagd, aber nun wird es Zeit, dass wir heil nach
Hause kommen.«

Das marsianische Schiff ging längsseits und sammelte die beiden
verbliebenen Pinassen ein, dann wurden die \textit{Pequod} und das tote
Kometentier ins Schlepptau genommen. In den Stunden bis zum
Erreichen des Rendezvouspunktes Enceladus, saßen die beiden
Kapitäne bei mehreren Gläsern marsianischen Whiskys zusammen und
waren sich einig, dass es der Besatzung der HMS \textit{Pequod} unbenommen
bleiben sollte, die Beute erlegt zu haben, und ob dieser
selbstlosen Haltung der Marsianer, hellte sich das finstere Gemüt
des Erdkapitäns auf und er schwor, dass er nie wieder etwas
Abfälliges über die marsianischen Brüder sagen würde.

\bigpar

Am Ende schaffte es die HMS \textit{Pequod} nicht mehr ganz nach Hause und
verglühte vollkommen steuerungsunfähig in der Erdatmosphäre wie ein
Zündholz, während seine Besatzung vom Aussichtsdeck der MS \textit{Utopia
Planitia} dem Schiff das letzte Geleit gab.

\bigpar

»Da geht sie dahin«, murmelte Commander Binky traurig und nippte an
seinem Whisky.

Der Kapitän rümpfte die Nase. »Ja, was für ein Jammer, Binky. Aber
uns bleibt immer noch das erlegte Kometentier. Die Marsianer
überlassen uns den Triumph. Stellen Sie sich vor, man wird uns auf
Monate hinaus feiern.«

Binky summte leise vor sich hin und wippte mit den Füßen.

»Was werden Sie jetzt tun, Commander?«, fragte der Kapitän nach
einer Weile. »Bleiben Sie dem Ætherraum treu?«

Binky schüttelte den Kopf. »Das war genug für ein ganzes Leben,
Sir. Einen wahrhaft prall gefüllten Sack voller Geschichten habe
ich jetzt für meine Enkelkinder. Mit Verlaub, Sir, vielleicht fahre
ich in Zukunft doch lieber zur See. Mein Schwager ist bei der
Marine. Er hat Beziehungen \ldots{} Sie wissen schon \ldots{} vielleicht kann
ich auf einem dieser neumodischen dampfbetriebenen Segelschiffe
anheuern. Intelligente Riesenkalamare jagen oder mit einem
Tauchboot die unterseeischen Städte im Pazifik aufsuchen. Das wäre
ein Spaß!«

\end{document}

