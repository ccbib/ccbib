\usepackage[ngerman]{babel}
\usepackage[T1]{fontenc}
\usepackage{textcomp}
\hyphenation{wa-rum Fracht-raum}
\hyphenation{schien}
\hyphenation{Tief-ebe-ne Tief-ebe-ne gro-ßen}


%\setlength{\emergencystretch}{1ex}

\renewcommand*{\tb}{\begin{center}* \quad * \quad *\end{center}}

\newcommand\bigpar\medskip

\begin{document}
\raggedbottom
\begin{center}
\textbf{\huge\textsf{Ruf der Sterne}}

\bigskip
Tanja Meurer
\end{center}

\bigskip

\begin{flushleft}
Dieser Text wurde erstmals veröffentlicht in:
\begin{center}
Die Steampunk-Chroniken\\
Band I -- Æthergarn
\end{center}

\bigskip

Der ganze Band steht unter einer
\href{http://creativecommons.org/licenses/by-nc-nd/2.0/de/}{Creative-Commons-Lizenz.} \\
(CC BY-NC-ND)

\bigskip

Spenden werden auf der
\href{http://steampunk-chroniken.de/download}{Downloadseite}
des Projekts gerne entgegen genommen.

\vfill

Tanja Meurer, geboren 1973 in Wiesbaden, ist gelernte
Bau-zeichnerin aus dem Hochbau, arbeitet seit 2001 in bauverwandten
Berufen und ist seit 2004 bei einem französischen Großkonzern als
Dokumentationsassistenz beschäftigt.

Nebenberuflich verdingt sie sich als Autorin und Illustratorin für
verschiedene Verlage.

\texttt{http://www.tanja-meurer.de/}
\texttt{http://www.nights-end.de/}

\end{flushleft}


\section{Ruf der Sterne}

Die mächtigen Glasfenster der Fertigungshalle wurden von einer
Druckwelle aus den Rahmen gesprengt. Ein Feuerball breitete sich im
Zentrum des Backsteingebäudes aus. Flammenspeere fauchten in die
mittägliche Sommerhitze. Durch die schiere Urgewalt dieser
Explosion hob sich das Dach. Gewaltige Metallplatten schleuderten
glühend durch die Luft. Einen Herzschlag später erschütterte eine
zweite Detonation das Werk. Die dicken Backsteinmauern brachen nach
außen. Ziegel wurden zu tödlichen Geschossen, die Tore, Fenster und
Türen der umstehenden Lager und Verwaltungsgebäude durchschlugen.

Die dritte Explosion riss die letzten Mauern ein. Gewaltige
Gesteinsbrocken wurden aus dem Erdreich in die Luft geschleudert.
Ein massives Stahlgerüst, das den Korpus eines schlanken Schiffes
trug, brach in sich zusammen, als würde es aus Streichhölzern
bestehen. Glühende Funken stoben in die flirrende Luft. Eine Wolke
aus Staub und Ruß wogte über dem Werksgelände. Leise klirrend
regneten feine Glassplitter auf das Kopfsteinpflaster nieder.

Die darauf folgende Stille wurde lediglich von dem leisen Knacken
überhitzten Metalls und brennenden Holzes durchbrochen.

\tb

Feuerwehr, Lazarettkutschen, Journalisten und Schaulustige
versperrten die Straße zu der Werkshalle. Über alle Köpfe hinweg
sah Anabelle den gen Himmel ragenden Rammsporn des Schiffes. Ruß
überzog seine Außenhaut. Der Leib des Schiffes lag vermutlich
zertrümmert auf dem Boden der Werkshalle.

Ein Schauer rann ihr über den Rücken. Ihr Maschinenkörper konnte
weder Hitze noch Kälte wahrnehmen. Dennoch sendete ihre menschliche
Seele diesen Impuls durch ihre Glieder. Instinktiv zog sie die
Schultern hoch und rieb sich die Arme.

Eine Hand berührte ihren Arm.

»Miss Talleyrand, Mr. Hailey erwartet Sie bereits.«

Anabelle sah zurück. Ihr Blick begegnete dem Sergeant Masters’.
Blankes Entsetzen stand in die Züge des blassen Mannes geschrieben.
Sie senkte den Kopf und nickte.

»Ist Madame Zaida schon anwesend?«, fragte Anabelle, während sie
ihm folgte.

Masters deutete ein Nicken an. Er führte Anabelle zu einem
prunkvollen Gebäude, dessen Front zur Themse zeigte. Die Schäden an
den verputzten Außenwänden zeugten von der Gewalt der Explosion.
Anabelles Blick glitt an dem imposanten Bau hinauf. Ein
Schmiedeeisenbalkon im ersten Stock überkragte, von wuchtigen
Säulen getragen, die gesamte Front. Die breite Freitreppe führte zu
kunstvoll bleiverglasten Türen. Auf einem Marmorschild konnte man
das Firmenlogo erkennen. Hier saß die Verwaltung der
Luftschiffwerft Erhardt \& Vock. Wortlos erklomm Anabelle die
Stufen. Unter ihren Stiefeln knirschten Glas und Steinsplitter.
Eine Krankenschwester in dunklem Kleid und Häubchen kümmerte sich
im Foyer um verwundete Personen. Anabelle schenkte ihnen wenig
Aufmerksamkeit. Ihr Hauptaugenmerk galt den gerahmten Werbeplakaten
der Werft. Die Druckwelle hatte viele von den Wänden gefegt. Auf
dem Boden lagen Bilder des neuen Schiffstyps. Anabelle blieb stehen
und kniete nieder. Neugierig hob sie zwei Plakate auf und
betrachtete den verzierten, schlanken Rumpf der neuen Ætherschiffe.
Bisher beschränkte sich die Werft lediglich auf heliumbetriebene
Starrluftschiffe. Seit sie zum ersten Mal von dem Konzept des
Himmelsseglers in der Zeitung gelesen hatte, grübelte sie über die
Technik hinter diesem Wunderwerk. Anhand der im Vorfeld
veröffentlichten Skizzen, baute sie in theoretischen Schritten
daran mit. Die Forschung an der neuen, eher historischen
Gestaltgebung irritierte und faszinierte Anabelle zutiefst. Mit
einer Art metallenem Wikingerschiff in die Luft zu schweben
erschien ihr unmöglich. Die Vorstellung, damit die Wolken zu den
Sternen zu durchbrechen, wirke noch weniger real. Ihr ausgeprägter
Verstand reichte nicht aus, sich dazu logisch herleitbare Techniken
einfallen zu lassen.

Sie rollte die Plakate ein und tippte sich nachdenklich mit dem
Rand gegen die Stirn.

»Mademoiselle Talleyrand?«, fragte Masters dicht hinter ihr. Nervös
fuhr er sich mit den Fingern durch das rote Haar. Sie erhob sich.

»Monsieur Masters«, flüsterte sie, darauf bedacht keinen zufälligen
Zuhörer auf sich aufmerksam zu machen. »Haben Sie sich nie die
Frage gestellt, wie dieses Prinzip funktioniert und ob es für solch
revolutionäre Technik nicht reichlich Neider gibt?«

Sie strich sich mit einer Hand den Rock glatt und klopfte den Staub
und die Splitter aus Saum und Schleppe.

»Doch, durchaus«, erklärte der Sergeant. »Wollen Sie andeuten, dass
Erhardt \& Vock möglicherweise ihrer Technik beraubt und damit aus
dem Wettbewerb ausgebootet wurden?«

Anabelle sah an ihm vorbei, aus der geborstenen Eingangstür.
Gendarmen trieben die Schaulustigen auseinander, um die
Leichenwagen durchzulassen.

»Es wäre in jedem Fall eine einzukalkulierende Möglichkeit,
Sergeant«, entgegnete sie.

»Inspektor Hailey vermutet ähnliches, einen Anschlag, keinen
Unfall«, erklärte Masters. »Aber wie skrupellos müsste man sein,
dafür ein Werk zu sprengen und hunderte Arbeiter zu töten?«

Anabelle reichte dem Sergeant ihre Hand. Masters ließ gern zu, dass
sie sich bei ihm unterhakte.

»Sind wir nicht zur Klärung hier, Sergeant Masters?«

\tb

»Monsieur le Inspecteur«, begrüßte Anabelle den stiernackigen
Hailey. Seine massige Boxerstatur hob sich neben der zierlichen
Gestalt einer älteren Dame in bieder hellgrauem Kostüm nahezu
ungeheuerlich ab. Der Inspektor sah kurz von dem Tisch vor sich
auf.

»Miss Talleyrand!«, rief er. Offenbar blieb es bei dieser etwas
mageren Begrüßung. Anabelle hob eine Braue.

»Es freut mich, Sie wiederzusehen«, gab sie betont pikiert zurück.
Die Spitze prallte an Haileys dickfelliger Ost-Londoner Natur ab.

»Milly Havelock ist mein Name.« Die Dame trat auf Anabelle zu.
Allerdings standen ihr Tränen in den Augen. »Ich bin Mr. Erhardts
Sekretärin.« Allein die Erwähnung dieses Namens ließ sie
aufschluchzen.

Anabelle ergriff ihre Hand und drückte sie leicht. Die emotionale
Art irritierte die Wissenschaftlerin.

»Anabelle de Talleyrand«, stellte sie sich knapp vor.

Rasch wandte sich Miss Havelock ab und presste das Taschentuch
gegen ihre dünnen Lippen. Verunsichert durch dieses Verhalten blieb
Anabelle in dem überladenen Büroraum stehen und beobachtete sie.

Masters löste sich von Anabelles Arm und trat zu der alten Dame
hinüber. Langsam geleitete er sie zu einem schweren Ledersessel.

»Mr. Masters …«, flüsterte sie mit tränenerstickter Stimme. »Was
soll nun aus uns allen werden? Mr. Erhardt ist tot und Mr. Vock …
seit Tagen im Ausland.« Sie schluchzte und vergrub ihr Gesicht in
den Händen.

Anabelle wusste, wie unhöflich und kalt ihr Verhalten auf diese
Frau wirken musste, konnte aber nichts daran ändern. Bereits jetzt
zog sie Schlüsse aus den Reaktionen Mrs. Havelocks. Die Beziehung
zwischen Mr. Erhardt und seiner Sekretärin schien nicht vollkommen
einwandfrei zu sein. Erhardt tot?, Anabelle legte die Stirn in
Falten. Zu ihm und Vock wollte sie gerade eine Frage stellen als
der Inspektor sie zu sich winkte.

»Anabelle, das ist Ihr Gebiet«, rief Hailey.

Sie wendete sich dem Inspektor zu. Dieser stand über einen
ausladenden Tisch gebeugt, und stützte sich mit seinen gewaltigen
Fäusten auf der dunklen Platte ab. Vor ihm lagen unzählige,
auseinander gerollte Pläne und Bauanleitungen für das Schiff, sowie
in Leder gefasste Schriftstücke.

Anabelle ergriff eines jener Bücher, und schlug es an willkürlicher
Stelle auf. Inhaltlich fanden sich statische Berechnungen für den
Rumpf des Schiffes. Neugierig überflog sie einige Zeilen. Die
Belastung, mit der Erhard und sein Partner zu rechnen schienen,
entsprach der Statik eines Güterdampfers. Irritiert runzelte
Anabelle die Stirn. Sollte das Schiff nicht ausschließlich
Passagiere transportieren? Nach den Berechnungen des Skeletts würde
der Luftsegler mit hoher Wahrscheinlichkeit auch schwere Last
transportieren können.

»Haben Sie etwas gefunden?«, fragte Hailey.

Anabelle wiegte den Kopf. »Ich weiß nicht genau«, murmelte sie.
Nachdenklich legte sie das Buch ab, um in den Risszeichnungen zu
blättern. Laut Plänen und Grundrissen sollte es drei Passagierdecks
und ein Aussichtsdeck geben. Die einfachen Kabinen besaßen mehr
Platz als eine Hotelsuite. Dieser Komfort fehlte den gängigen
Luftschiffen. Ebenso gab es Appartements mit Salons, großen Bädern
und Ankleidezimmern. Wozu solch eine Verschwendung?

Sie griff nach dem nächsten Buch und blätterte es durch.

»Anleitungen zur Herstellung von Leichtmetalllegierungen, die hoher
Reibungshitze widerstehen können«, murmelte sie. Ein Stahlskelett,
die Außenhaut aus leichtem Metall und die statischen Berechnungen
für ein Schlachtschiff passten nicht zueinander. Einige
Eintragungen erschienen ihr fundiert, andere sinnlos.

»Gibt es eine Angebotssammlung oder ein Auftragsbuch für die
Innenausstattung?«, fragte sie mit einem Seitenblick auf Hailey.
Der Inspektor nickte. Mit einer Hand hob er einen Katalog auf, der
neben dem Tisch auf dem Boden stand. »Das hier, denke ich«, sagte
er.

Anabelle schlug die ersten Seiten auf. Mit einem Finger fuhr sie
über das Register der einzubauenden Möbel, Lampen und
Sanitäranlagen.

»Haben wir das auch für die technischen Einbauten? Die müssten
ebenfalls als Katalog aufgenommen worden sein.«

Hailey hob die Schultern. »Da kann ich Ihnen nicht helfen.«

Anabelle nickte. »Mrs. Havelock?«, wendete sie sich an die
Sekretärin.

Die alte Dame saß zusammengesunken in ihrem Sessel. Erschrocken hob
sie den Blick. Erst als sie Anabelle ihre ungeteilte Aufmerksamkeit
schenkte, sprach die junge Wissenschafterin weiter. »Gibt es in
Ihren Unterlagen Kataloge über die bereits eingebauten
Sonderherstellungen?«, fragte sie.

Die alte Dame erwiderte Anabelles Blick mit vollkommenem
Unverständnis.

»Ich rede von Belüftungssystemen, Verrohrungen, Sanitäranlagen,
Gasleitungen, Luftumwälzern, Heizungen, Wasserwiederaufbereitung
und ähnlichem.«

Mrs. Havelock schlug die Augen nieder. »Ich glaube diese Bücher
befanden sich zum Zeitpunkt des Unglücks in der Maschinenhalle.«

Trotz der vorgeblichen Ahnungslosigkeit der alten Dame, fiel es
Anabelle schwer, ihr glauben zu schenken. Wie praktisch!, dachte
sie. Ihrer Ansicht nach war ein solcher Fall schier unmöglich.
Diese Bücher dienten lediglich der Ablage. Die Techniker arbeiteten
mit Duplikaten und Wasserpausen. Vorerst verschwieg sie ihre
Gedanken.

»Sehr bedauerlich«, sagte sie.

Nach der Mimik Haileys zu urteilen ließ ihr schauspielerisches
Talent zu wünschen übrig.

»Ich brauche mehr Zeit, um all diese Unterlagen gebührend zu
prüfen. Deshalb würde ich mich gerne über Nacht hier einquartieren
und arbeiten«, erklärte Anabelle.

Hailey warf Mrs. Havelock einen kurzen Blick zu.

Die Mimik der alten Dame gefror.

»Weshalb?!«, verlangte sie zu wissen.

»Miss Talleyrand ist wissenschaftliche Beraterin Schottland Yards,
Madam«, erklärte er steif. »Ihre Prüfung deckt möglicherweise mehr
als einen Unfall auf.«

Die alte Dame zögerte.

»Ich kann es befehlen lassen«, vertraute Hailey ihr wenig
freundlich an. Anabelle beobachtete die Sekretärin. Hölzern nickte
sie. »Also gut, wie Sie wollen.«

Steif erhob sich Mrs. Havelock und schritt zu der Bürotür. »Sie
gestatten, dass ich Mr. Vock telegraphisch informiere?!«

»Sicher«, bestätigte Hailey. Die Sekretärin verließ das Zimmer.
Anabelle wies hinaus. »Wissen Sie, wo sich Madame Zaida befindet?«

\tb

Journalisten bedrängten die Polizei mit ihren Fragen, während
Fotographen ihre unhandlichen Apparate abbauten.

Auch Anabelle wollte sich ein Bild über das Ausmaß der Zerstörung
machen. Mit raschen Schritten eilte sie in das Zentrum der
Explosion. Der Anblick ließ sie erneut schaudern. Riesige Teile des
Steinbodens fehlten. Geschmolzenes Metall verband sich mit Holz und
Ziegeln. Ketten, Zahnräder und große Platten der Dachabdeckung
lagen auf der Erde verstreut. Das Skelett des Schiffes glühte noch
immer. Die darunter begrabenen Arbeiter mussten ein grauenhaftes
Ende genommen haben. Noch immer wurden Leichen geborgen. Anabelle
bezweifelte, dass irgendeine Person so nah des Explosionsherdes
überlebt haben konnte. Die Ausmaße der Zerstörung lagen jenseits
jeder Vorstellungskraft.

Anabelle schritt langsam voran. Vorsichtig, bedacht auf das immense
Gewicht ihres Maschinenkörpers, umging sie instabilere Stellen des
Bodens. Sie suchte Zaida. Mit einer Hand raffte sie die Schleppe
ihres Kleides, um in den Krater unter dem Wrack zu klettern.

Geröll und Schutt knirschte unter ihren Absätzen. Ein Polizist hob
kurz den Blick, wandte sich aber wieder ab, als er sie erkannte.
Wenige Schritte entfernt stand die schlanke, hochgewachsene
Angolanerin. Sie hielt in ihrer unbehandschuhten Hand ihren Stock
mit dem silbernen Rabenkopf. Behutsam rieb sie über das Metall.
Langsam drehte sie sich um ihre Achse. Es schien, als wolle sie
sich einen Überblick verschaffen. Zaida sah durch die Schleier der
Wirklichkeit und den Filter der Magie.

Vorsichtig trat Anabelle an ihre Seite und wartete geduldig, bis
sich die Aufmerksamkeit ihrer Freundin auf sie richtete.

»Wie viel weißt du von Hailey?«, fragte die Zauberin leise.

»Pläne und Berichte habe ich mir angesehen und Mrs. Havelock kennen
gelernt«, entgegnete Anabelle.

»Deine Meinung?« Zaidas Mimik verriet nichts von ihren persönlichen
Eindrücken.

»Sie ist dem Seniorpartner sehr ergeben«, murmelte Anabelle. »Ihr
missfällt die Prüfung der Unterlagen.«

»Wir sind Fremde, die unangenehme Fragen stellen und den Ruf
Erhardts beflecken könnten«, entgegnete Zaida.

»In jedem Fall.« Anabelle straffte sich. »Anhand der Unterlagen und
dem Fehlen einzelner Dokumente möchte ich die ehrbaren Absichten
Mr. Erhardts ebenso in Frage stellen wie die von Mr. Vock.«

Zaida streifte ihren Handschuh über und umklammerte den Stock
fester. »Was hast du gefunden?«

Kommentarlos entrollte Anabelle die Plakate, die sie aus der
Eingangshalle mitgenommen hatte. Das silbrige Wikingerschiff
schnitt auf einer Illustration durch die Wolkendecke zu den
Sternen. Auf der anderen präsentierte es sich seriös als Luxusliner
der Lüfte. Mit einer knappen Kopfbewegung deutete sie zu dem Sporn,
der über Ihnen in den Krampen hing.

»Anhand der Statik und der Inneneinrichtung würde diese
Konstruktion nicht starten können«, erklärte Anabelle ruhig. Sie
sah den leisen Zweifel in dem Blick ihrer Freundin. »Die Forschung
ist hierfür nicht weit genug. Zurzeit brauchen wir Heliumballons
und Rotoren. Dieses Schiff hat Segel. Wenn sich im Bauch nicht
außergewöhnlich viel Gas für den Antrieb befinden sollte, wird es
nicht starten. Davon abgesehen fehlen ihm die relevanten Rotoren,
die es davor bewahren ins Trudeln zu kommen.«

»Dieses Schiff wird seit Monaten beworben«, murmelte Zaida tonlos.
»Laut der Times sollte morgen das erste Mal einen Testflug
stattfinden.«

»Unser Weg zu den Sternen«, zitierte Anabelle. Ihre Stimme troff
vor Ironie. »Nein. Dieses Schiff ist Betrug.«

Zaida senkte den Blick. »Kannst du diese Theorie beweisen?«

»Gib mir Zeit, Pläne und Berichte zu studieren, um sie anschließend
mit dem Wrack zu vergleichen.«

\tb

Der Verdacht des Betruges und der Veruntreuung von
Forschungsgeldern verdichtete sich für Anabelle, als Hailey von
einem Besuch bei der Rothschild-Bank zurückkehrte. »Anabelle!« rief
er schon auf die Entfernung. »Sie hatten offenbar den richtigen
Riecher!«

\bigpar

Anabelle hob den Blick. Sie kniete auf dem Boden, in ihren Händen
ein Kupferzylinder und Reste einer winzigen Kupferspule aus einer
Taschenuhr. Sie trug unterdessen den groben Hosenanzug eines
Arbeiters und Handschuhe. Sergeant Masters assistierte ihr. Er
stand an einem Tisch und katalogisierte, was Anabelle an
Einzelteilen fand und sofort zuweisen konnte. Sie legte den
Zylinder ab und nickte dem jungen Mann zu. »Notieren Sie,
Masters?«, bat sie ihn.

»Sicher, Mademoiselle!«

»Was haben Sie erfahren?«, wendete sie sich an den Inspektor, der
atemlos neben dem Tisch innehielt.

»Die Konten sind eingefroren worden. Aber die Einlage war
ungewöhnlich gering«, keuchte er. »Allerdings wurden laut den
Rothschilds große Beträge an eine Firma in Indien transferiert. Bei
allen anderen Unterlieferanten für dieses Projekt sind wiederum nur
geringe Anzahlungen geleistet worden.« Er tupfte sich mit einem
Taschentuch den Schweiß von der Stirn. »Was konnten Sie in der Zeit
finden?«

Anabelle deutete auf den Tisch und die Bücher hinter sich. »Dieses
Schiff sollte einen Verbrennungsmotor auf Basis von Diethylether
erhalten. Diese Idee ist gut. Die Energie, die aus diesem
hochexplosiven Stoff gewonnene wird, ist immens. Diese
Zusammensetzung entzündet sich bei 95° F.« Sie strich sich eine
Strähne aus dem Gesicht. »Diethylether ist hoch brennbar und
schwerer als Luft. Man kann sich ausrechnen, dass ein Leck in einem
der Versorgungstanks in Kombination mit einem Zündfunkengeber
tödlich endet.«

Hailey fuhr sich mit dem Handrücken über die schweißnasse Stirn.
Nachdenklich rieb er sich den Nasenrücken. Er überlegte. Anabelle
gewährte ihm die Zeit, seine eigenen Schlüsse zu ziehen. »Das ist
hier passiert?«

Sie nickte.

»Wie gut lassen sich die Tanks manipulieren?«, fragte er nach
einigen Sekunden.

Anabelle deutete auf die Kupferzylinder auf dem Tisch. »Diese
kleinen Spielzeuge hier hatten einen Zünder. Dafür dienten
wahrscheinlich billige Taschenuhren. Aus dem Schutthaufen habe ich
kleine Bestandteile eines Uhrwerks herausfiltern können. Ein
Zeiger, ein verschmortes Uhrblatt und eingeschmolzene Rädchen.
Diese zündeten das hier …« Sie hob das Kupferrohr an. »Sie dienten
als bewegliches Auflager unter den Gastanks im Keller.«

Sie wendete sich einem gewaltigen Krater im Boden zu. »Kommen
Sie!«

Mit raschen Schritten erreichte sie ihn und sprang hinein. Hailey
blieb nichts anderes, als ihr zu folgen. Dicht hinter Anabelle
federte er in den schlecht beleuchteten Gewölbekeller.

»Licht kann man keines machen, nehme ich an«, murmelte der
Inspektor. Anabelle nickte. »Der süße Geruch lässt auf einen Rest
Äther schließen, Monsieur Hailey«, erklärte sie. »Halten Sie sich
also besser ein Taschentuch vor Ihr Gesicht.«

Der Inspektor presste beide Hände gegen Mund und Nase. Anabelle
betrachtete ihn spöttisch.

»Kommen Sie.«

Mit gebührender Vorsicht schritt Anabelle vor ihm über Schutt und
verkohlte Reste von Kisten.

»Warum sind Sie sich so sicher, dass die Fabrik keinem neidischen
Konkurrenten zum Opfer fiel?«, fragte er.

»Die Pläne sind undurchführbar. Seit ich von dem Konzept gelesen
habe, Monsieur, überlege ich mir, wie ich ein Schiff dieser Art
bauen würde.«

Sie blieb vor einem weiteren schwarz verbrannten Krater stehen.

»Anabelle, Sie sind nicht das Maß aller Dinge!«, zischte Hailey
ärgerlich.

Sie hob den Blick. Die Beleidigung prallte an ihr ab.

»Das weiß ich durchaus. Allerdings ist mir auch bewusst, dass die
Menschen an Bord Luft zum Atmen und Wasser brauchen.« Sie hob beide
Hände. »Dazu würde ich die Außenhülle vollständig schließen und
eine Wasser- und Luftwiederaufbereitungsanlage einbauen.« Sie
lachte auf. »Davon abgesehen, wohin sollten die Menschen fliegen?
Zum Mond?«, fragte sie spöttisch.

Hailey schwieg.

»Wenn Sie sich den Mond in der Sternwarte betrachten, so können Sie
nicht mehr erwarten, als eine Ansammlung von Staub und Steinen.
Wissen wir, ob wir dort atmen oder leben können?« Sie wartete seine
Antwort nicht ab. »Nein!« Nach einer Sekunde Zögerns schüttelte sie
den Kopf. »Die Idee ist reizvoll, ein Traum für jeden Ingenieur,
Pionier und Abenteurer, nicht für reiche Menschen, die es leid
sind, um die Welt zu segeln.«

Die Konsequenz aus Anabelles Worten erschütterte Hailey zutiefst.

»Dann starben all diese Menschen für nichts«, flüsterte er tonlos.

»Wahrscheinlich.« Anabelle wies auf einen deformierten, zerplatzten
Tank, unter dem verkohltes Holz lag. »Alle Brennbarkeit und
Explosivität wäre nicht so verheerend gewesen, wenn nicht
zusätzliches Brandmaterial hier gelagert worden wäre.«

Hailey schluckte hart. »Grauenhaft!«

»Zaida und Masters haben hier unten den vollkommen zerfetzten Leib
eines Menschen gefunden.«

»Der Initiator?«, vermutete Hailey.

»Oder das Opfer«, überlegte Anabelle. »Mrs. Havelock sprach von Mr.
Erhardt, der offenbar starb.«

»Laut ihrer Aussage ging er jeden Tag für mehrere Stunden in die
Fertigung, um selbst die Arbeiten zu überwachen«, führte er aus
»Erhardt war der leitende Ingenieur und vertraute offenbar nicht
einmal seinem jüngeren Partner die Geheimnisse dieser Erfindung
an.«

»Er musste fürchten, dass Vock seine Konstruktion als Lüge
enttarnt.« Anabelle räusperte sich. »Wo ist Mr. Vock über-haupt?«

»Die Havelock konnte mir nur sagen, dass er bereits seit einer
Woche auf Geschäftsreise ist … in Delhi.«

»Sicher?«, fragte Anabelle misstrauisch nach. »Haben Ihre Männer
diese Spur schon überprüfen lassen?«

»Ja, aber sie verliert sich bereits in London«, gestand Hailey.

»Vielleicht ist der Tote Monsieur Vock.« Nachdenklich wiegte
Anabelle den Kopf. »Warten wir bis heute Nacht. Sicher wird Mrs.
Havelock weitergeben, dass ich hier bleibe. Ich rechne in jedem
Fall mit Besuch.«

»Kommen Sie allein klar?«, fragte Hailey. Sorge klang in seinen
Worten mit.

Mit einem Grinsen nickte Anabelle. »Sie wissen doch: ich bin eine
Maschine.«

\tb

Noch immer ragte der Sporn über ihr auf. Seit mehreren Stunden war
sie – bis auf wenige Bobbies, die darauf achteten, dass kein
Unbefugter das Gelände betrat – allein.

Im Licht von Karbidlampen und Fackeln las sie sich die Unterlagen
durch. Einige bemerkenswerte Punkte fielen ihr auf. Es gab geringe
Unterschiede im Strich von den technischen Rissen, den Details für
den Drechsler und den Darstellungen der Motoren, zu der Ausstattung
der Kabinen. Für Anabelle stand fest, dass an diesen Plänen
unterschiedliche Personen gearbeitet haben mussten. Stammte die
Grundidee von Vock – oder einer anderen Person? Vielleicht einem
unbedeutenden Ingenieur, der die Arbeit an dem Projekt aufgab?

Jemand räusperte sich hinter ihr. Ohne sich umzudrehen oder den
Blick von den Plänen zu nehmen fragte sie: »Monsieur Erhardt, wie
ich annehme?«

»Ja«, entgegnete der alte Mann knapp. Anabelle drehte sich zu ihm
um. Außerhalb des Lichtkreises erhob sich ein fast konturloser
Schatten. Lediglich der Lauf eines Gewehres hob sich schwach ab.

»Warum sind Sie zurück gekehrt?«, fragte sie ruhig.

»Ich kann Ihnen meinen Schatz nicht ausliefern«, entgegnete er.
Seine Tonlage klang androgyn und weich, aber auch alt. »Sie wissen
diese Pläne nicht zu würdigen, Mademoiselle Talleyrand.« Ein
gefährlicher Unterton schlich sich in seine Stimme. Die Schärfe
darin umriss ein Quäntchen Wahnsinn.

»Was glaubten Sie, mit diesen Dokumenten zu erreichen, Monsieur?«,
fragte Anabelle leise. »Sie wissen, dass diese Unterlagen ein
Luftschloss beschreiben …«

»Woher kommt nur diese französische Überheblichkeit?«, fragte er.

»Ich habe Wissen, keine Überheblichkeit«, entgegnete Anabelle
verärgert.

»Wissen?!«, zischte er. »Dieses Schiff ist ein Meisterwerk. Die
vorangegangenen Luftschiffe sind nichts im Vergleich hierzu!«

»Die bisherigen Bautypen sind in der Lage zu fliegen, sie bringen
Ihnen und Vock Ruhm und Geld ein …«

»Schweigen Sie!«, donnerte er.

Anabelle wusste, dass sie mit einem Verrückten sprach. Trotz allem
konnte sie nicht schweigen. »Warum? Weshalb haben Sie ein Projekt
verfolgt, dass zum Scheitern verurteilt war?«

Erhardt schoss. Die Kugel fetzte Erdreich und Stein aus dem Boden
vor Anabelles Füßen. Erschrocken wich sie zurück, bis sie gegen
ihren Tisch stieß. Auf Einschüsse in ihrer Kautschukhaut konnte sie
gut verzichten. »Monsieur …«, begann sie, wurde aber von seinem
unartikulierten Aufschrei unterbrochen. Die Wachen wurden sicher
gleich auf ihn aufmerksam!

»Schweigen Sie!«, brüllte er. »Die Pläne meines Sohnes waren
perfekt!«

Irritiert blinzelte Anabelle. Sohn? Wie hatte sie die menschliche
Seite aus ihrer Kalkulation heraus lassen können?! Es ging Erhardt
scheinbar nicht um Geld von Investoren.

»Die Differenzen in den Plänen«, murmelte sie. »Monsieur, Sie haben
die Arbeit ihres Sohnes fortgeführt?«

Die Milde in ihrer Stimme beruhigte Erhardt etwas. »Ja«, flüsterte
er. »Seine Idee war so brillant! Aber Vock wollte das Design
bestimmen. Es sollte prachtvoller sein als alle Schiffe, die je
gebaut wurden.« Er verstummte. Sein heiseres Schluchzen brach durch
die Stille zwischen ihnen. »Mein Sohn … Millys Sohn …« Wieder
versagte seine Stimme. »Vock hat sein Konzept ad absurdum geführt.
Dieses Schiff ist flugunfähig.«

Anabelle nickte. »Vock wollte Subventionen und schnellen Profit.
Vermutlich ist er mit dem Geld geflohen.«

Stein knirschte unter Erhardts Schuhen. Er trat in den Lichtkreis.
Sein eingefallenes, fahles Gesicht sprach von Entbehrung und Leid.
Wirr hingen seine grauen Haare in die Stirn. Offenbar trug er seit
Tagen den gleichen Anzug und fand keine Zeit sich zu rasieren.
Anabelles Seele zog sich zusammen. Sie empfand Mitleid.

»Mein Sohn wurde – Dank Vocks Habgier – während eines Unfalls im
Werk zu einem Krüppel. Wochen danach starb er an den Folgen. Ein
Leben für ein Leben!«

»Die zerfetzte Leiche war Vock«, vermutete Anabelle.

»Ja«, bestätigte Erhardt. Der Ingenieur sank ein Stück weit in sich
zusammen. Das Mitleid in Anabelle wuchs. Sie wusste, dass es für
den alten Mann keine Zukunft gab. Er war der Mörder von
einhundertfünfzig Arbeitern und seines Partners. Aber wie groß wäre
das Ausmaß der Katastrophe geworden, hätte dieses Schiff je
abgehoben. Ein Absturz, eine Explosion, gleichgültig welches
Szenario sie sich dafür ausmalte, hätte weitaus mehr Opfer
gefordert. Das Interesse an der Forschung und die Diskussion über
Sinn und Unsinn der Raumfahrt würde ohnehin neu angefacht werden,
ausgelöst durch das Fiasko des gestrigen Tages. Doch irgendwann
gelänge es den Menschen. Der Weg in das All war ihnen sicher.

\bigpar

»Ihr Sohn wird seine Ehre und Anerkennung bekommen«, flüsterte
Anabelle. »Auf Basis seiner Technik können andere Wissenschaftler
sein Werk fortführen. Seine Idee wird nicht ungehört bleiben. Er
teilt den Traum nach den Sternen mit uns allen. Wenn die Zeit reif
ist ein Sternenschiff zu bauen, wird auch er diese letzte Grenze
durchbrechen.«

\end{document}

