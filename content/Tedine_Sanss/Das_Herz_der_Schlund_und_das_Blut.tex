\usepackage[ngerman]{babel}
\usepackage[T1]{fontenc}
\hyphenation{wa-rum}


%\setlength{\emergencystretch}{1ex}

%\renewcommand*{\tb}[1]{\begin{center}#1\end{center}}

\newcommand\bigpar\medskip

\begin{document}
\raggedbottom
\begin{center}
\textbf{\huge\textsf{Das Herz, der Schlund und das Blut}}

\medskip
Tedine Sanss

\end{center}

\bigskip

\begin{flushleft}
Dieser Text wurde erstmals veröffentlicht in:
\begin{center}
Die Steampunk-Chroniken\\
Band I -- Æthergarn
\end{center}

\bigskip

Der ganze Band steht unter einer 
\href{http://creativecommons.org/licenses/by-nc-nd/2.0/de/}{Creative-Commons-Lizenz.} \\ 
(CC BY-NC-ND)

\bigskip

Spenden werden auf der 
\href{http://steampunk-chroniken.de/download}{Downloadseite}
des Projekts gerne entgegen genommen. 
\end{flushleft}

\newpage

»Extrablatt!« Der kleine, abenteuerlich verdreckte Zeitungsjunge,
dem die Schirmmütze immer wieder auf die Nase rutschte, schwenkte
die \textit{Times} hoch durch die Luft.

»Extrablatt! Die \textit{King Charles} greift erneut nach dem purpurnen
Band! \textit{King Charles} läuft heute noch zum Ganymed aus! Extra \ldots{}
Danke, Sir!« Er nahm den Schilling und kramte mit der anderen Hand
in seiner Umhängetasche, um das Wechselgeld herauszugeben.

\bigpar

»Es stimmt so, mein Sohn.« Algernon Holland hatte gute Laune. »Kauf
dir einen Becher Bier dafür und trink ihn auf mein Wohl. Von mir
werden deine Leser erfahren, ob die \textit{King Charles} tatsächlich in der
Lage ist, den Geschwindigkeitsrekord zu brechen.«

Der Junge machte große Augen und hob salutierend zwei Finger an die
Mütze, die dadurch erneut abstürzte.

»Danke, Mister Holland, Sir! Wenn Sie erlauben, nehme ich statt des
Biers einen Krug Milch für meine kleine Schwester, Sir. Sie sind
mein großes Vorbild. Ich will zur Zeitung gehen und Reisen machen
und Berichte darüber schreiben wie Sie.«

Lachend zog Algernon die Schirmmütze des Jungen wieder gerade und
klopfte ihm auf die Schulter. »Wenn ich wieder zurück bin, dann
melde dich doch bei der Daily. Mal sehen, ob ich etwas für dich tun
kann. Und hier hast du noch ein paar Pence. Kauf deiner kleinen
Schwester auch ein Püppchen.«

»Ja, Mister Holland, Sir, danke, Sir. Extrablatt!«

\bigpar

Algernon ging weiter und musterte nachdenklich seine Hand. Ob der
Junge Läuse hatte? Er musste sich bei der nächsten Gelegenheit
waschen und sein Hemd wechseln. Cremefarbene Seide war so
empfindlich, und der Staub in den Hafengassen schien beinahe
magnetisch davon angezogen zu werden.

Seit zwei Jahrzehnten hatte er nun schon die lukrative Aufgabe, auf
den schnellsten und luxuriösesten Ætherklippern die Strecke zum
Ganymed und zurück zu befahren. Jede entgegenkommende
Handelsschaluppe nahm einen seiner Berichte mit, die von seinen
Gesprächen mit den Gästen im Rauchersalon handelten, seinen
Beobachtungen auf den Kapitänsbällen und seinen Beschreibungen der
feinen Gesellschaft und ihrer Lebensweise auf der Erde und dem
erdähnlichsten der Jupitermonde, dem Ganymed. Regelmäßig lieferte
er seinen Lesern dabei eine Fülle von mythologischen Anspielungen
und die ätzenden Bonmots und klugen Sentenzen, für die er berühmt
war. Mit Genugtuung hatte er vernommen, dass seine Art zu schreiben
in Mode gekommen war und als »Hollandismus« bezeichnet wurde.
Weiter konnte man es wohl nicht bringen.

\bigpar

Die Zeitung unter den Arm geklemmt, wo sie vermutlich
Druckerschwärzeflecken hinterließ und das Hemd vollends ruinierte,
schritt er beschwingt zum Kai.

\bigpar

Die \textit{King Charles} war zu groß, um am Steg zu ankern. Sie lag ein
Stück weiter draußen im Hafenbecken. Als Algernon zu ihr hinaus
schaute, sträubten sich wie immer seine Nackenhaare. Sie war so
riesig. Selbst jetzt, da ihre Segel gerefft waren und die Masten
kahl in den Himmel ragten wie das Skelett eines Giganten – er zog
sein berühmtes blaues Heft hervor und notierte die Wendung –
vermittelte sie den Eindruck überwältigender Größe. Ihr
kupferbeschlagener Rumpf gleißte in der Sonne, er schien das Licht
beinahe auszustrahlen, anstatt es nur zu reflektieren. Wie ein
lebendiges, geschmeidiges Wasserwesen schaukelte sie ungeduldig auf
den Wellen, von den schweren Ankerketten nur mühsam gehalten. Der
Kessel, der das Schaufelrad am Heck antrieb, war schon vorgeheizt,
und aus dem schmalen Schornstein zwischen Hauptmast und Besanmast
kräuselte sich Rauch. Kleine Boote umkreisten sie wie Monde einen
Planeten, sie lieferten Wasserfässer, Tröge mit dem entsetzlichen
gepökelten Fleisch, das die Æthermänner als Hauptnahrungsmittel zu
betrachten schienen, Ballen mit Segeltuch, lebende Kühe, Schweine,
Ziegen und Hühner, die muhend, grunzend, blökend und gackernd auf
den wackligen Planken an Bord balancierten und eilig unter Deck
geschafft wurden, sowie eine Fülle weiterer Waren in Fässern,
Körben, Tragen, Säcken und Kisten. Es erschien ihm unglaubwürdig,
wie der Ætherklipper dies alles in sich hineinfraß und doch so
schlank blieb – ihm gelang so etwas nicht.

Erneut zückte er das blaue Heft, in dem er alle Hollandismen
festhielt. Dann winkte er einem der Bootsleute und dirigierte seine
eigene Habe, die Koffer, Schachteln, Taschen und schließlich sich
selbst an Bord.

\bigpar

Der Start, so aufregend er für die Æthermänner und Schaulustigen am
Kai auch war, wurde von den Gästen der Erste-Klasse-Kabinen
traditionell ignoriert. Sie wussten, dass auf dem Klipper unter
lauten Rufen die Segel gesetzt und die Anker gelichtet wurden, dass
der fauchende Kessel das Schaufelrad in Bewegung brachte, ohne
dessen zusätzliche Energie auch der prächtigste Klipper die
Atmosphäre nicht hätte überwinden können. Sie bemerkten auch, dass
das schlanke Gefährt immer schnellere Fahrt aufnahm und dann,
technisch zwar begründbar, aber für sie als Laien beinahe wie
Zauberei, in den Himmel aufstieg, durch die Wolkendecke stieß,
immer noch beschleunigte und sich schließlich in den Æther
aufmachte. Das war durchaus spektakulär, aber mit offenem Maul zu
staunen schickte sich nicht, und gescheit klingende Kommentare
trauten sie sich klugerweise nicht zu. Sie blieben also, wo sie
waren, schikanierten die Diener, verwendeten eine unglaubliche Zeit
auf die Wahl des richtigen Anzugs oder ließen sich von der Zofe zum
zehnten Mal das Haar hochstecken. Danach schritten sie zum
Dinnersaal. Erst dort, inmitten ihresgleichen, fühlten sie sich
wieder behaglich.

\bigpar

Algernon hatte das cremefarbene Seidenhemd gegen ein schneeweißes
ausgetauscht und dazu einen Smoking angezogen, dezent und
zurückhaltend für den ersten Abend, wie er befand. Dann war er
raschen Schrittes in den Saal geeilt und hatte sich in einer Ecke
platziert, um die übrigen Gäste beim Eintreten beobachten zu
können.

Vorläufig waren es nur sechs Stühle, die um einen Tisch aufgestellt
waren. Die weiteren Gäste der ersten Klasse würden erst bei der
Zwischenlandung im Mondhafen zusteigen.

\bigpar

Die erste die hereinkam, offensichtlich von Hunger getrieben, war
die hagere Baronin Hohenschön. Algernon war ihr trotz all seiner
zahllosen Reisen entgegen aller Wahrscheinlichkeit noch niemals
persönlich begegnet, aber die Geschichten über sie kannte er gut.
Von Jahr zu Jahr liefen neben den offiziellen Wetten auf die
Geschwindigkeit der Ætherklipper auch Nebenwetten darauf, ob die
Baronin dieses Mal wieder lebendig ihr Ziel erreichen, oder auf
halber Strecke zu Staub zerfallen würde. Bisher hatte noch immer
diejenigen gewonnen, die darauf setzten, dass die Dame viel zu gut
in Gin konserviert sei, um jemals zu modern.

In der linken Hand hielt sie einen Fächer, in der Armbeuge
balancierte sie einen winzigen Hund, der über den Spitzenbesatz des
Ärmels hinweg die Zähne bleckte, und mit der Rechten stützte sie
sich schwer auf eine blutjunge Gesellschafterin. Diese wurde von
Jahr zu Jahr ersetzt, und ein besonders bissiges Gerücht besagte,
die Baronin sauge die Lebenskraft ihrer Gesellschafterinnen aus und
halte sich nur dadurch am Leben. Natürlich war das Unfug. Sie
saugte darüber hinaus einige Nichten und Großnichten aus, bei denen
sie sich zwischen ihren Reisen einquartierte, des weiteren eine
Reihe von Anwaltsbüros, die die Vermögen ihrer diversen
verstorbenen Ehemänner verwalteten, und schließlich noch den
Verstand eines jeden, der sich mit ihr auf ein Rededuell einließ.

Algernon lächelte. Er war sicher, dass die Baronin auf dieser Fahrt
in ihm ihren Meister finden würde.

\bigpar

Nach ihr betrat Bert van Guellen den Saal, ein reicher Bankier, der
auf dem Höhepunkt seiner Macht den Fehler begangen hatte, eine
Schauspielerin zu heiraten. Blass und anämisch hing sie jetzt an
seinem Arm, als müsse sie demonstrieren, welch eine Last sie für
ihn war. Gesellschaftlich ohnehin in einer angreifbaren Position,
hatte er sich durch diese Mesalliance sämtliche Türen verschlossen,
die er nicht durch sein Scheckbuch offen halten konnte, und auch
sein Eheglück erschien Algernon fraglich, denn er ging gebeugt,
sein Gesicht war grau und eingefallen wie das eines Mannes, der vor
Kurzem sehr viel Gewicht verloren hatte.

\bigpar

Ihnen folgte John Shallow-Bargepole, ein Gentleman, was sich vor
allem darin zeigte, dass niemand genau hätte sagen können, womit er
sich beschäftigte. Gelegentlich wettete er auf Pferde, allerdings
mit wenig Glück, und meist lag er anderen Menschen auf der Tasche.
Mit langen Schritten überholte er die anderen und ließ sich auf den
einzigen Stuhl plumpsen, der mit dem Rücken zum Fenster stand.

\bigpar

Die Baronin verhielt den Schritt. Missbilligend stieg ihre rechte
Augenbraue in die Höhe. Algernon hätte schwören können, dass ihr
Gesicht dabei knisterte, als bestehe es aus brechendem Pergament.

Shallow-Bargepole grinste entschuldigend, was ihm nicht stand, denn
es ließ sein jugendlich-verlebtes Gesicht knittern wie ein billiges
Hemd. »Vertrage den Ausblick nicht. Das Auf und Ab. Muss davon
speien, wenn Sie verstehen, was ich meine.« Jetzt hatte er Algernon
entdeckt und ließ ihm einen gelangweilten Blick durch sein Lorgnon
zukommen. »Sie auch hier? Aussichten auf das purpurne Band, was
meinen Sie?«

»Auch Ihnen einen guten Abend«, erwiderte Algernon mit betonter
Höflichkeit. »Baronin, ich hoffe, Sie hatten eine gute Anreise?«

Sie musterte ihn eisig. »Ich glaube nicht, dass wir einander
vorgestellt wurden.«

»Algernon Holland, der Journalist«, warf Shallow-Bargepole
nachlässig ein, und es gelang ihm, dem Wort »Journalist« einen
Klang zu verleihen, als handele es sich um eine Art ansteckenden
Ausschlags.

»Sind Sie sicher, dass hier für Sie gedeckt ist?« Die Baronin
wandte sich ab. Mit dem Fächer versetzte sie ihrer
Gesellschafterin, einer gehetzten blassen Frau, einen gezielten
Hieb auf das Handgelenk und veranlasste sie damit, einen Stuhl
zurechtzurücken. Leise raschelnd setzte sie sich darauf nieder,
während der Hund, der um sein Gleichgewicht rang, einen enervierend
hohen Kläfflaut ausstieß. »Ruhig, Precious!«, befahl sie.

Achselzuckend begab sich Algernon auf die andere Seite des Tisches
und rutschte auf den freien Platz neben der Bankiersgattin.
Vergebens kramte er in seinem Gedächtnis nach ihrem Namen und
nickte ihr nur vage zu. Dann beschloss er, die Eröffnungsrunde
verloren zu geben und sich zunächst dem Essen zu widmen.

\bigpar

Als der erste Gang aufgetragen wurde, hatte seine Tischnachbarin
bereits das zweite Glas Wein geleert und schnippte ungeduldig mit
dem Finger, um das dritte in Angriff nehmen zu können.

\bigpar

»Cora, mein Liebling \ldots{}«, murmelte der Bankier.

Cora – jetzt fiel es ihm wieder ein. Cora Hock, für eine Saison der
Star des Londoner Westends. Er musterte sie verstohlen von der
Seite. Noch war sie schön, aber ihre klaren Züge hatten bereits
begonnen, schwammig zu werden. Kleine Fältchen sammelten sich in
ihren Mundwinkeln, ihre Kinnlinie verlor an Kontur.

Sie spürte seinen Blick und wandte sich um, seine Aufmerksamkeit
absichtlich missverstehend. Ihr Kleid war zu rot, ihr Ausschnitt zu
tief, ihr Schmuck ein wenig zu prunkvoll, um echt zu sein. »Gewiss
erinnern Sie sich an meinen Auftritt als Rosamunde im Adelphi?«,
fragte sie mit einem Augenaufschlag, der seelenvoll wirken sollte.

»Niemals wurde sie anmutiger verkörpert«, erwiderte Algernon ohne
zu zögern, obwohl er sicher war, sie nie auf der Bühne gesehen zu
haben.

»Reizend von Ihnen, dass Sie sich noch so genau entsinnen können.
Mein Kleid war \ldots{} Wissen Sie noch, welche Farbe es hatte?«

»Es passte genau zu Ihren Augen«, entgegnete er.

Shallow-Bargepole, der ihnen gegenüber saß, hatte seinen Trick
durchschaut und lachte garstig.

Verbittert griff Cora van Guellen nach ihrem Glas und leerte es in
einem Zug.

»Cora, mein Liebling \ldots{}«, drängte ihr Gatte.

Sie machte eine großartige Geste mit dem Arm, und es gelang
Algernon mit knapper Not, sein Gedeck in Sicherheit zu schieben.
Das Porzellan schepperte.

»Ich \ldots{} bin \ldots{} leer!«, begann sie mit Grabesstimme.

Bert van Guellen legte die Hand über die Augen, als habe er diesen
Monolog schon einmal zu oft vernommen.

»Ich bin nur mehr ein hohles Gefäß! Man hat mir mein Publikum
genommen, ich bin zur Sprachlosigkeit verurteilt. Ein Vogel in
einem goldenen Käfig hat es besser als ich, denn er darf doch
immerhin noch singen, und man hört ihm zu. Aber schauen Sie mich
an! Ohne die Bühne bin ich ein Nichts! Verstummt, verloren,
verstümmelt. Ein Schatten, das ist es, was ich bin.«

»Um Himmels willen, hören Sie auf zu prahlen!«, fuhr die Baronin
dazwischen. »Auf der Bühne haben Sie nicht getaugt, und hier taugen
Sie noch viel weniger. Nehmen Sie eine Valium und legen Sie sich
hin, dann richten Sie zumindest kein Unheil an.«

\bigpar

Mit einem Aufschrei sprang Cora van Guellen auf die Füße.

»Cora, mein Liebling \ldots{}«, flehte der Bankier, aber da stand sie
schon wie eine flammende Anklage vor der Baronin, den Zeigefinger
ausgestreckt.

»Was \ldots{} wissen \ldots{} Sie \ldots{}«

»Schluss damit!«, fauchte die Greisin. »Ich hatte zwar nicht das
zweifelhafte Vergnügen, Sie auf der Bühne agieren zu sehen, aber
ich erkenne eine drittklassige Mimin, wenn sie vor mir steht. Seien
Sie froh, dass Sie im gemachten Nest sitzen und halten Sie Ruhe!«

Der Hund, der ihre Erregung spürte, stellte die Ohren auf und
kläffte. Sie warf ihn ihrer Gesellschafterin zu, die sich bemühte,
die kleine Schnauze zuzuhalten und gleichzeitig den um sich
schlagenden Pfoten auszuweichen.

Cora stand wie erstarrt. Eine schwächere Frau, dachte Algernon,
wäre an ihrer Stelle schluchzend zusammengebrochen. Aber Cora war
nicht schwach. Erst jetzt, vollkommen isoliert und in die Enge
getrieben, entfaltete sie ihre eigentliche Stärke.

»Im gemachten Nest«, begann sie leise, aber mit klangvoller Stimme,
»da sitzen auch Sie. Sie haben Ihr Leben lang nichts weiter getan
als sich systematisch hochzuheiraten, von der unbedeutenden kleinen
Modistin zur Baronin. Ich habe immerhin auf der Bühne gestanden,
ich habe selbst etwas dargestellt, wenn auch nur für eine Saison.
Also wagen Sie es nicht, auf mich herabzusehen. Nicht auf mich, und
nicht auf meinen Mann!«

Sie warf den Kopf zurück und marschierte, wenn auch schwankend,
durch die Tür. Ihr Mann sprang auf und eilte hinter ihr her.

»Cora!«, hörten sie ihn durch den Gang rufen.

\bigpar

Shallow-Bargepole applaudierte. «Was für ein Abgang! Möchte
vielleicht noch einer der Herrschaften vor dem Hauptgang das Weite
suchen? In diesem Fall würde die dürre Ente nämlich für die
Verbleibenden ausreichen.«

»Bedienen Sie sich ruhig«, sagte die Gesellschafterin. »Ich hatte
ohnehin vor, nur von dem Gemüse zu nehmen.«

»Kein Wunder, dass Sie so saftlos sind, Mabel, bei dieser
Ernährung!«, schnappte die Baronin und zog sich die Ente heran.
»Mister Holland, nicht wahr? Sie sind doch hoffentlich keiner
dieser schmierigen Klatschreporter, die ihren einzigen Lebenszweck
darin sehen, auf der Vergangenheit anderer Leute herumzureiten \ldots{}
ihre Herkunft auszuposaunen, ihre Wurzeln aufzudecken, oder wie man
so etwas in Ihren Kreisen nennt.«

Sie sprach so schroff, dass der Hund erneut zu kläffen begann, aber
Algernon hörte die unterschwellige Panik in ihrer Stimme.

»Mir ist eher daran gelegen«, begann er vorsichtig, »in der
Gegenwart Freundschaft zu halten.«

Die Baronin begann auf die Entenbrust einzuhacken, als duelliere
sie sich mit einem Kontrahenten. »Und was verstehen Sie unter
Freundschaft?«

Algernon erblickte die Gelegenheit, einen seiner Hollandismen
unterzubringen. »Ach, wissen Sie \ldots{} eine Freundschaft unter
Literaten hält meist nur so lange, wie das Mischen des
Schierlingsbechers dauert.«

Mabel warf ihm einen bewundernden Blick zu, und Shallow-Bargepole
musterte betont gleichgültig seine Fingernägel, aber die Baronin
stieß nur ein kurzes, papiernes Lachen aus. »Oscar Wilde. Sie
vergessen, mein Lieber, dass ich vor ein paar Monaten wieder einmal
die Gelegenheit hatte, meinen hundertsten Geburtstag zu feiern. Als
junges Mädchen in London war ich mit der Familie befreundet.«

Algernons Lächeln blieb unverändert. »Freundschaften pflegen zu
können erscheint mir als die nobelste Tugend.«

»Es waren sehr gewöhnliche Leute, allesamt. In meinem Alter ist es
eher von Vorteil, Feindschaften zu pflegen. Es geht nichts über die
Genugtuung, die Todesanzeige einer alten Feindin zu lesen und zu
wissen, dass man das letzte Wort behalten hat. Und nun reichen Sie
mir das Salz.«

»Das ist das Mindeste«, spöttelte Shallow-Bargepole, »da er Ihnen
schon nicht das Wasser reichen kann \ldots{}«

Algernon drehte sich zu ihm um, eine schlagfertige Antwort auf den
Lippen. Dabei entglitt ihm der Salzstreuer, rutschte über den Tisch
und traf den Hund. Der sprang mit einem schrillen Laut vom Schoß
der Gesellschafterin, die von der plötzlichen Reaktion des Tieres
viel zu überrascht war, um es halten zu können, und rannte mit
eingekniffenem Schwanz durch die offengebliebene Tür.

»Schauen Sie nur, was Sie angerichtet haben!«, kreischte die
Baronin in einem Ton, nicht unähnlich dem Schreckensschrei ihres
Hundes. »Precious, komm zurück! Precious, zu Frauchen! Mabel, Sie
nutzloses Ding, nun laufen Sie schon hinter ihm her und bringen Sie
ihn zurück!«

Mabel sprang auf und wandte sich kopflos zuerst in die eine, dann
in die andere Richtung. Ritterlich eilte Algernon an ihre Seite.

»Lassen Sie nur, es ist ja meine Schuld, dass der Hund in Panik
geraten ist. Ich werde ihn suchen. Bleiben Sie hier und kümmern Sie
sich um die Baronin.«

Sie nickte erleichtert und kehrte zu ihrem Platz zurück.

\bigpar

Algernon stürmte aus der Tür, sah einen kleinen Schatten um die
Ecke verschwinden und setzte ihm nach, auf den Hauptgang zu den
Kajüten. Suchend blickte er sich um. Erneut schien etwas Kleines
hinter einer Biegung zu verschwinden. Er hetzte hinterher, in einen
Nebengang hinein, konnte von dem Hund aber nichts entdecken. Die
Wege verzweigten sich, er eilte zuerst den einen, dann den anderen
entlang, folgte diesem bis zum Ende und fand sich schließlich in
einem spärlich erleuchteten weiteren Gang wieder.

\bigpar

»Precious?«, rief er versuchsweise.

Von weither glaubte er ein Kläffen zu hören, das sich in der Ferne
verlor. Er lief dem Geräusch nach, bis der Gang unvermittelt an
einer steilen Treppe endete.

»Precious? Komm zurück!«

Noch während ihm klar wurde, dass der Hund unmöglich die schmalen
Stufen überwunden haben konnte, klang es erneut in seinen Ohren.
War das Tier vielleicht in seiner Panik gestürzt und lag dort
unten?

Behutsam kletterte Algernon rückwärts hinunter, tastete mit den
Füßen nach dem kleinen Körper. Als unter seinem Tritt etwas Weiches
aufquietschte, riss er das Bein hoch, strauchelte fast, klammerte
sich an das Geländer. Zischelnd, schnatternd huschte es davon. Eine
Ratte.

Algernon schluchzte auf, er wusste selbst nicht, ob vor
Erleichterung oder Enttäuschung.

»Precious, bist du hier unten?«

Der Gang, in den er nun geriet, war enger und dunkler als der
erste. Er zog den Kopf ein und horchte.

Das Geräusch hatte sich vervielfältigt. Aus mehreren Richtungen
zugleich quiekte, schabte, kläffte es. Verwirrt drehte er sich im
Kreis.

»Precious, möchtest du einen Knochen? Ich will nicht geizig sein,
ich serviere dir die ganze Baronin, wenn du brav herkommst.«

Der Weg teilte sich wieder. Nach kurzem Zögern wählte Algernon den
linken Gang. Gab es nicht einen Merksatz, mit dem man aus jedem
Labyrinth wieder herausfand? Wenn man in einer festen Reihenfolge
nach rechts und links abbog, konnte man nicht verloren gehen.
Sollte er die schmalen Durchgänge mitzählen? Ein weiteres Mal stand
er vor Stufen. Hinauf oder hinunter? Er hätte eine Münze geworfen,
hätte er nicht befürchtet, dass er sie auf dem dunklen Boden
niemals wiederfinden würde.

»Hinunter!«, entschied er. »Wenn der Hund nicht auf magische Weise
Flügel bekommen hat, ist es wahrscheinlicher, dass er auf dem Weg
in die Tiefe ist.«

Wieder kletterte er rückwärts die steile Stiege hinab, gab
zischende Geräusche von sich, um Ratten zu vertreiben, rief und
lockte, um den Hund zu einer Antwort zu bewegen, lief erneut einen
niedrigen Gang entlang, der sich teilte und verzweigte, hielt
schließlich inne und musste eingestehen, dass er sich hoffnungslos
im Bauch des Schiffes verirrt hatte.

»Precious? Irgendjemand?«

Es roch nach ängstlichen Tieren. Er entsann sich der Menagerie, die
an Bord gebracht worden war. Irgendwo hier unten mussten ihre
Ställe sein, dunkel, eng und erschreckend. Außerdem roch es nach Öl
und heißem Metall, nach verschwitzter Kleidung und zu vielen
Menschen am selben Ort. Die Luft war feuchtwarm und abgestanden, er
spürte, wie ihm das Hemd am Körper klebte.

Eine Vielzahl von Geräuschen umgab ihn, ein Schnauben und Grunzen,
das hoffentlich von den Tieren herrührte, ein Pfeifen und Stampfen,
das zum Schiff selbst gehören mochte, darüber schwebten kurze,
abgehackte Rufe.

Menschen, immerhin. Wesen, mit denen er sich verständigen konnte
und die ihm den Weg zurück zum Dinnersaal zeigen würden. Tief nach
vorn gebeugt, um nicht an die niedrige Decke zu stoßen, schritt er
in die Richtung, aus der die Rufe zu ihm drangen.

\bigpar

»Wahrschau!« Eine kräftige Faust packte ihn und riss ihn zur Seite.
Er starrte in das Gesicht eines vierschrötigen Æthermannes,
verschwitzt, mit Öl und Ruß verschmiert. Da, wo er eben noch
gestanden hatte, rasselte eine schwere Kette vorbei, an deren Ende
ein metallener Kasten hing.

Keuchend holte er Atem.

»Das«, brachte er schließlich heraus, »wäre möglicherweise knapp
geworden.«

Der Æthermann zuckte die Achseln und strich bedauernd über
Algernons Smokingärmel, auf dem sich eine hässliche Ölspur zeigte.

»Der feine schwarze Anzug!«

»Ja, er ist schwarz, nicht wahr?« Algernon lächelte schwach. »Als
ich ihn in Auftrag gab, habe ich extra vermerkt, er solle die Farbe
einer mondlosen Nacht über einem einsamen Bergsee haben, aber
herausgekommen ist dabei ordinäres Schwarz.«

Der Æthermann bedachte das Problem. »Möglicherweise gab’s die Farbe
die Sie wollten nicht im Stoffmusterkatalog«, schlug er dann vor.

Algernon nickte. »Ja, das wäre in der Tat eine Erklärung. Aber um
den Anzug, da er nun einfach nur ordinär schwarz ist, anstatt den
Farbton aufzuweisen, der mir vorschwebte, ist es nicht schade.
Statt dessen drücken mich zwei andere Probleme. Das eine ist klein,
laut und haarig, ein Hund namens Precious, der aus dem Dinnersaal
entlaufen ist, weil ihm die Ente zu mager war.«

Der Æthermann zuckte erneut die Achseln. »Wenn er hier unten ist,
kommt’s drauf an, ob er zäher ist als die Ratten.«

»Sofern es stimmt, dass Hunde auf die Dauer gewisse Eigenschaften
ihrer Eigentümer übernehmen«, sinnierte Algernon, »ist dieser Hund
zäher als das gepökelte Rind an Bord des Fliegenden Holländers.«

»Dann schafft er die Ratten«, konstatierte der Æthermann. »Und das
zweite Problem?«

»Bin sozusagen ich selbst. Ich bin diesem Hund nachgesetzt, ohne
auf den Weg zu achten, und befinde mich nun vermutlich sehr weit
von besagtem Dinnersaal entfernt. Würden Sie den Vergil machen, der
mich durch alle sieben Kreise der Hölle und sämtliche Abteilungen
des Fegefeuers zurück an die Oberfläche führt?«

Diesmal brauchte der Æthermann eine Weile, bis er Algernons Satz
durchschaut hatte. Dann leuchteten seine Augen auf, und er strahlte
über das ganze Gesicht.

»Ist mir eine Freude, Ihnen das Schiff zu zeigen, Sir! Meine
Freiwache hat ohnehin gerade begonnen. Mit dem Pfiff, der Sie
vorhin hergelockt hat, Sir. Simon Ross ist der Name, ich bin
Bootsmann hier auf der \textit{Charlie}.

»Algernon Holland«, stellte Algernon sich selbst vor und ergriff
herzlich die dargebotene Hand. Als keine Reaktion erfolgte, setzte
er hinzu: »Der Journalist. Ich schreibe die Berichte über die
Reisen der Klipper. Die Jagd nach dem purpurnen Band.«

Ross blies die Backen auf und machte Anstalten, auf den Boden zu
spucken, unterließ es aber nach einem raschen Seitenblick auf
Algernon.

»Diese Berichte \ldots{} hm.«

»Sie schätzen sie nicht besonders?«

»Na ja. Sie bringen meine Frau zum Lachen. Das ist nett, was Sie da
schreiben. Aber ich glaube, von den Ætherklippern, davon verstehen
Sie nicht viel, oder? Waren Sie schon mal hier unten?«

»Nein«, gab Algernon zu, und zu seinem eigenen Erstaunen röteten
sich seine Wangen.

»Haben Sie schon mal gesehen, wie so ein Klipper startet?«,
forschte der Æthermann weiter.

»Bisher noch nicht«, räumte Algernon ein. »Während des Ablegens,
wissen Sie, gibt es immer so viel vorzubereiten. Die Kajüten müssen
bezogen werden, dann will gut überlegt sein, in welcher Kleidung
man sich zum ersten gemeinsamen Dinner präsentiert – der Anzug in
seiner ordinär schwarzen Färbung zum Beispiel war eine fatal
falsche Entscheidung. Ich glaube nicht, dass ich in diesem Moment
hinter einer kleinen Bestie namens Precious her wäre, wenn ich mich
für das moosgrüne Jackett entschieden hätte.«

»Na, dann«, entschied Simon Ross und stopfte die Hände in die
Taschen seiner weiten Hose, »fangen wir doch am besten direkt im
Maschinenraum an. Hier entlang, Mister Holland.«

Der Bootsmann schlüpfte durch einen Durchgang, den Algernon
überhaupt nicht wahrgenommen hatte, duckte sich unter einem Balken
hindurch und stieg gewandt über ein Fallgitter. Algernon folgte ihm
schnaufend.

Unvermittelt war das Dröhnen und Stampfen viel lauter geworden, bei
jedem der rhythmischen Geräusche empfand er einen Druck auf den
Ohren, als presse ihm jemand die Hände darauf. Er wollte Ross
danach fragen, befürchtete aber, dass seine Stimme in dem Getöse
untergehen würde.

Der sah sich nach ihm um, packte den ohnehin ruinierten
Smokingärmel und zog ihn hinter sich her, zwischen zwei
gigantischen Behältern hindurch. Und plötzlich standen sie direkt
vor der Maschine.

\bigpar

Algernon verschlug es den Atem.

Sie reckte sich bis zur Decke des Raumes, stieß an beiden Seiten
an, schien mit jeder Bewegung wachsen und sich noch weiter
ausdehnen zu wollen. Ein mächtiges Fundament verband sie mit dem
Boden, darüber erhob sich ein Dickicht aus Röhren und Stangen.
Alles war in Bewegung und blieb doch am selben Platz, bäumte sich
auf, schnaubte, grollte, als kämpfe ein riesiges gefesseltes Tier
vergebens gegen seine Ketten. Eine glühende Hitze ging von diesem
Wesen aus – und eine ungeheure Kraft.

Er trat näher und streckte ihm die Hände entgegen.

Da öffnete sich ganz oben eines der Überdruckventile und stieß
kreischend Dampf aus.

Mit einem Aufschrei fuhr er zurück.

Beide Hände als Schalltrichter um die Lippen gelegt, wandte er sich
an Ross. »Ist dies der tiefste Kreis der Hölle?«

»Das Herz der \textit{Charlie}!«, schrie dieser zurück. »Hier herüber!«

Sie schlängelten sich an der Maschine vorbei, so dicht, dass
Algernon spürte, wie ihr Atem sein Nackenhaar versengte.

Ross öffnete eine schwere Tür, zog ihn hindurch und ließ sie hinter
ihnen wieder ins Schloss fallen. Unvermittelt war es ruhig.
Algernon klopfte ein paar Mal auf seine Ohren, bis der Druck
nachgelassen hatte.

»Das Herz der \textit{Charlie}?«, fragte er dann.

Ross lächelte fein. »Was glauben Sie«, begann er gemächlich, »wie
es die Ætherschiffe fertig bringen, in den Himmel aufzusteigen?«

Algernon zuckte die Achseln. »Bisher dachte ich, sie täten es
einfach. Wenn der Wind günstig steht.«

»Kein Wind brächte das fertig«, konstatierte der Bootsmann. »Ein
guter Wind und eine günstige Strömung können einen Klipper über den
Ozean treiben. Die Sonnenwinde und die Ætherströmungen treiben ihn
von einem Planeten zum nächsten. Aber um aufzusteigen, vom Meer
über die Wolken und in den Æther hinein, braucht man viel größere
Kräfte. Natürlich haben die Menschen auch vor der Erfindung der
Dampfmaschinen davon geträumt, zu den Sternen zu fliegen. Aber die
Kraft hat nicht ausgereicht. Und sehen Sie, hier kommt die Maschine
ins Spiel. Sicherlich ist Ihnen das Schaufelrad am Heck
aufgefallen?«

Algernon nickte eifrig.

»Wenn das Schiff gut im achterlichen Wind liegt, alle Segel
gesetzt, die Rahsegel, die Stagsegel und die Leesegel an den
Rahspieren, dann wird das Schaufelrad in Gang gesetzt. Erst das
gibt die zusätzliche Kraft für den Aufstieg, und der Klipper hebt
ab. Ohne die Dampfmaschine wäre die \textit{Charlie} nur ein Ozeanklipper,
und deswegen ist sie das Herz. Verstanden? Dann geht es jetzt
weiter zum Schlund.«

»Zum Schlund? Den hat Ihr Schiff auch?«

»Aber gewiss!«, entgegnete der Bootsmann und ging voraus. »Das
wissen Sie doch selbst: Wenn das Herz schlagen soll, braucht der
Körper Nahrung. Schauen Sie hier herein, aber treten Sie nicht zu
nahe.«

Er riss eine weitere Tür auf, anscheinend noch schwerer als die
erste. Algernon warf einen Blick direkt ins Höllenfeuer.
Rauchschwarze Seelen hantierten mit Schaufeln, warfen in einem
rasenden Tanz glänzende Kohlestücke in einen gewaltigen Ofen. Die
Hitze nahm ihm den Atem, und doch konnte er nicht anders als nur
immer weiter in die Flammen starren, bis ihn Ross schließlich sacht
zurückzog.

»Lassen Sie die Leute ihre Arbeit tun«, mahnte er. »Und jetzt, wo
Sie das Herz und den Schlund gesehen haben, können Sie mir sicher
auch sagen, was das Blut der \textit{Charlie} ist?«

»Sie alle!«, erwiderte Algernon ohne zu zögern. »Die Æthermänner,
die die \textit{Charlie} mit allem versorgen, was sie braucht. Aber Mister
Ross, an Ihnen ist ja ein Poet verlorengegangen!«

Der Bootsmann spuckte auf den Boden. »Jetzt geht’s wieder nach
oben. Kommen Sie, damit Sie den letzten Gang in Ihrem Dinnersaal
nicht verpassen.«

\bigpar

Sie kletterten eine der steilen Stiegen empor. Es war deutlich
kühler und weiter hier oben. Algernon streckte die Glieder.

»Von der Hölle ins Fegefeuer. Haben Sie Dank, Mister Ross. Aber was
ist das dort hinten?«

Durch eine Luke konnte man auf das Deck des Schiffes sehen. Darüber
erhoben sich die Masten, an denen sich die riesigen Segel blähten,
vierfach übereinander aufgetürmt und breit gefächert wie die
Schnupftücher einer Riesenfamilie mit Influenza. Wie Spinnweben
führten die Takelagen bis ganz nach oben zu den höchsten Rahen, und
dort, an den äußersten Spitzen, krabbelte und wimmelte es von
Æthermännern. Sie trugen silbrige Schutzanzüge, hatten ihre Füße
mit Tauen an die Masten gebunden und stürzten sich immer wieder in
den Æther.

»Mister Ross, was tun diese Leute dort?«

Der Bootsmann wand sich. »Sie \ldots{} springen.«

»Das sehe ich. Sie lassen sich halsbrecherisch von den Rahen fallen
und greifen nach irgendetwas. Warum? Was geht da vor?«

»Wir durchqueren gerade den Asteroidengürtel. Die Männer greifen
nach Spacium. Es kommt nur hier vor, auf kleineren Gesteinsbrocken,
es ist wertvoll, und wir brauchen es unbedingt.«

Algernon schüttelte verständnislos den Kopf. »Was die Männer dort
tun, sieht gefährlich aus. Wozu wird das Spacium benötigt?«

\bigpar

Ross wandte sich ab, als ergreife ihn eine leichte Übelkeit.

»Wozu, Mister Ross?«, beharrte Algernon. »Ich habe heute mehr über
die Ætherklipper erfahren als in all den Jahren zuvor. Sie haben
mir Herz, Schlund und Blut der \textit{Charlie} gezeigt. Seien Sie
versichert, dass ich mir alles gut gemerkt habe und meinen Lesern
davon erzählen werde. Von dem, was wichtig ist. Nicht von den
Gesprächen im Dinnersaal. Nun verraten Sie mir auch das letzte
Geheimnis. Was geht dort draußen vor?«

»Was glauben Sie, Sir«, begann der Æthermann leise, »weshalb Jahr
für Jahr ein neuer Rekord auf der Jagd nach dem purpurnen Band
aufgestellt wird?«

»Ich nehme an, die Klipper werden immer besser.«

»Oh, die Klipper sind immer gleich gut. Aber sie können nicht so
schnell fahren, wie es die Maschinen und die Besegelung zuließen.
Ihre Geschwindigkeit wird von äußeren Umständen begrenzt. Die
Reibung ist es, die ein guter Kapitän beachten muss.«

»Die Reibung also. Was tut sie?«

»Die Reibung, Sir, erzeugt Hitze. Wenn das Schiff mit voller Kraft
unterwegs ist, wird sein Rumpf so heiß, dass er Feuer fängt. Aber
es ist möglich, die Reibung zu vermindern, wenn man den Rumpf gut
schmiert. Mit Spacium. Dazu wird es benötigt. Aber es ist
gefährlich. Schauen Sie nur: Wenn einer der Männer zu kurz springt,
kracht er gegen die Masten. Oder es trifft ihn einer der größeren
Asteroiden. Damit Sie und Ihresgleichen, Sir, über immer neue
Rekorde berichten können, riskieren die Männer dort draußen ihr
Leben. Mitunter verlieren sie es. Zwei meiner Söhne, Sir \ldots{}«

Er sprach nicht weiter.

»Ich erkenne die Treppe wieder«, sagte Algernon sanft. »Bemühen Sie
sich nicht weiter, ich finde allein zurück. Ich danke Ihnen, Mister
Ross. Für alles.«

Damit wandte er sich ab und stieg hinauf, schritt den spärlich
erleuchteten Gang entlang, betrat an dessen Ende den helleren
Nebengang, erreichte den breiten Hauptgang und stand wieder vor der
Tür zum Dinnersaal.

Mit einem tiefen Atemzug, als müsse er unter Wasser tauchen, stieß
er sie auf und trat ein.

\tb

»Wo waren Sie denn nur?«, keifte ihm die Baronin entgegen.
»Precious ist längst wieder zurück, er hat den direkten Weg zur
Küche genommen und den Koch gebissen, eine kluge Wahl, wenn Sie
mich fragen, denn er war definitiv besser genährt als die Ente,
aber nun verlangt der Mann Schadensersatz, und diese nutzlose
Person« – sie hackte mit ihrem Fächer nach Mabel, die verängstigt
auf ihrem Stuhl kauerte – »ist auch noch geneigt, ihm Recht zu
geben und mir in den Rücken zu fallen.«

»Die Baronin würde sie auf der Stelle entlassen«, warf
Shallow-Bargepole mit träger Stimme ein, »aber dann hätte sie
niemanden mehr, den sie heruntermachen könnte, und das ginge über
ihre Kräfte. Holland, so wie es aussieht, werden Sie diese
Meinungsverschiedenheit schlichten müssen.«

\bigpar

Algernon lächelte.

\bigpar

»So wie es aussieht«, sagte er heiter, »habe ich zu tun.«

Er zog sein berühmtes blaues Heft hervor, schlug eine neue Seite
auf.

»Die wahnwitzige Jagd nach dem purpurnen Band«, schrieb er. »Von
Ihrem Korrespondenten Algernon Holland. Wer auf einem Ætherklipper
unterwegs ist, der nach einem neuen Rekord zu greifen versucht,
macht sich nur selten Gedanken darüber, dass diese Rekorde mit Blut
erkauft werden. Nicht mit dem der Reisenden, die behaglich im
Dinnersaal sitzen, sondern allein mit dem Blut derer, die auf den
Schiffen leben und arbeiten \ldots{}«


\section{Tedine Sanss}

Tedine Sanss ist das Alter Ego einer westfälischen Autorin und kam
als solches erst vor ein paar Monaten auf die Welt. Sie schreibt
Science Fiction und Steampunk. Dies ist ihre erste
Veröffentlichung.
\end{document}
