\usepackage[ngerman]{babel}
\usepackage[T1]{fontenc}
\hyphenation{wa-rum}


%\setlength{\emergencystretch}{1ex}

\newcommand\bigpar\medskip

\hyphenation{Gér-ard}

\begin{document}
\raggedbottom
\begin{center}
\textbf{\huge\textsf{Im Æther verloren}}

\medskip
Tedine Sanss

\end{center}

\bigskip

\begin{flushleft}
Dieser Text wurde erstmals veröffentlicht in:
\begin{center}
Die Steampunk-Chroniken\\
Geschichten aus dem Æther
\end{center}

\bigskip

Der ganze Band steht unter einer 
\href{http://creativecommons.org/licenses/by-nc-nd/2.0/de/}{Creative-Commons-Lizenz.} \\ 
(CC BY-NC-ND)

\bigskip

Spenden werden auf der 
\href{http://steampunk-chroniken.de/download}{Downloadseite}
des Projekts gerne entgegen genommen. 
\end{flushleft}

\newpage

Ich erwachte wie jeden Morgen in kaltem Schweiß gebadet und lag
noch eine Weile still, während ich mich daran erinnerte, wo ich
war. Die Nacht hatte mich erneut zu den Schlachtfeldern
Afghanistans geführt, Kameraden an meine Seite gestellt, von denen
ein Teil meines Bewusstseins erkannte, dass sie längst tot waren,
und ihnen deswegen das faulende Fleisch vom Körper fallen ließ, bis
ich die elfenbeinweißen Knochen darunter sah. Sie wandten sich mir
zu, um zu sprechen, aber ihre Kiefer fielen herab, und aus ihren
leeren Augenhöhlen krochen Maden, wälzten sich über die
Wangenknochen, stürzten auf mich herab wie ein Hagelschauer,
wimmelten über meine Arme und Beine, besudelten meinen ganzen
Körper.

\bigpar

Mit einem mühsam unterdrückten Schrei fuhr ich hoch und wischte mit
den Händen über meine Glieder, zerkratzte panisch meine Haut dabei
und rang nach Atem, während ich den Schlaf endgültig hinter mir
ließ.

Ich war daheim, in meinem neuen Zuhause, einem sonnendurchfluteten
Bauernhaus im Departement Calvados, dem schönsten Teil der
Normandie. Was ich im Traum für Kanonendonner gehalten hatte, war
Claudines und mein einziger Bediensteter Yves, der den
Samstagmorgen nutzte, um auf dem Platz unter meinem Fenster Holz zu
hacken. Das disharmonische Gellen der Alarmglocken stammte in
Wahrheit von den Kühen, die gemächlich über die Weide schritten.
Und was mir wie das Pfeifen von Schüssen geklungen hatte, waren die
Zweige des alten Apfelbaums, die am Fenster scharrten.

Die andere Seite des Bettes war leer. Claudine musste schon
aufgestanden und zum Frühstück hinuntergegangen sein. Dieser
Gedanke erleichterte mich zunächst, denn so hatte sie mein
gequältes Erwachen nicht miterleben müssen. Aber dann war mir umso
schwerer ums Herz, als mir bewusst wurde, an wie vielen Tagen ich
allein erwachte.

\bigpar

Ich hatte England erst vor einigen Monaten verlassen, als ich
begriff, dass ich mich dort niemals wieder heimisch fühlen würde,
nicht nach der Schlacht um Herat, nicht nach der grausamen
Belagerung von Chakcharan. Der Arzt hatte mir Luftveränderung
empfohlen, einen Ortswechsel und einen ausgedehnten Urlaub von Land
und Leuten.

In einer grimmigen Anwandlung von Gehorsam war ich auf den
Kontinent geflüchtet, in die Normandie, wo mich die Landschaft und
der Calvados heilen sollten, hatte überstürzt das hübsche
dunkeläugige Mädchen geheiratet, das im Café bediente, und von der
Invalidenrente den kleinen Hof gekauft. Aber wohin auch immer man
flüchtet, man nimmt sich selber mit, und so rang ich jeden Morgen
gegen mein nächtliches Ich wie Jakob gegen den Engel, und entfernte
mich dabei immer weiter von meiner liebenswürdigen, koketten
kleinen Ehefrau und dem einfachen Landleben, das ich gesucht
hatte.

\bigpar

Noch immer zitternd von dem Nachtmahr, schob ich die Decken
beiseite, erhob mich und ging ins Bad. Ein graues Gesicht starrte
mich aus dem Spiegel an. Schwarze wirre Locken, die tief in die
Stirn hingen, dunkle, weit aufgerissene Augen, die all den
Schrecken widerspiegelten, den sie gesehen hatten, ein schmaler
Mund, Wangen und Kinn mit dunklen Stoppeln bedeckt. Das Gesicht
musterte mich forschend. Rasch senkte ich den Blick, wusch mich,
rasierte mich flüchtig und kleidete mich an. Dunkle schmale
Reithosen, ein weißes Hemd, eine Weste darüber und hohe Stiefel. So
lief ich die Treppe hinunter in die Küche.

\bigpar

Yves, der mit dem Holzhacken fertig war, lächelte mir entgegen und
servierte mir frisches Brot, Käse und ein Glas warmer Milch.

»Das ist ein Frühstück für ein krankes Kind!«, protestierte ich
halbherzig, während ich mich auf dem Stuhl niederließ und schon
zugriff.

»Sie müssen doch zu Kräften kommen«, brummte Yves. »Die gnädige
Frau ist ins Dorf gegangen. Sie lässt Ihnen ausrichten, Sie sollten
nicht mit dem Mittagessen auf sie warten. Es gibt Tripes à la mode.
Wenn Sie mich dann nicht mehr brauchen – ich will noch ein Gatter
reparieren.«

Ich entließ ihn mit einem Nicken, während ich schon fieberhaft
überlegte, wie auch ich dem heutigen Mittagessen entgehen konnte.
Yves war ein guter Koch, aber auf das seltsame Gericht aus Kutteln
und Kalbsfüßen wollte ich gern verzichten.

Schließlich beschloss ich, mich bei meinem Nachbarn einzuladen. Ich
hatte Gérard Le Bon bei meinem Einzug kennen gelernt, den er
tatkräftig mitorganisiert hatte, und der breitschultrige, gelassene
Normanne war mir sofort sympathisch gewesen. Ich wusste, dass er im
Dorf als Sonderling galt, der in seiner Scheune an allerlei selbst
erfundenen Geräten herumbastelte. Da mein Interesse für alles
Technische ebenso groß war wie mein Abscheu vor allzu
traditionellen Gerichten, sah ich nun eine gute Gelegenheit für
einen Besuch. Ich beendete das Frühstück, klemmte mir eine Flasche
Cidre aus der Küche unter den Arm und verließ das Haus durch die
Küchentür.

\bigpar

Vom Garten aus konnte ich die Scheune meines Nachbarn sehen. Sein
Grundstück lag etwas unterhalb von meinem, in einem der Täler, die
sich in sanften Schwüngen zum Fluss hinab erstreckten, und war nur
durch einige Hecken von meinem Haus getrennt. Ich beschloss, den
direkten Weg zu nehmen. Die Augen fest auf die Scheune geheftet,
bahnte ich mir einen Pfad vorbei an den Kräuterbeeten, über die
Obstwiese, auf der die ersten Apfelbäume sich mit weißen und
rosafarbenen Blüten schmückten, über die Hecke zur Weide, auf der
träge wiederkäuend die Kühe umherwanderten, und schließlich über
das Gatter auf seine eigene, kurz geschnittene Wiese, in deren
Mitte die große alte Scheune stand. Davor lehnten zwei hölzerne
Kisten. Ein eifriges Klopfen drang durch das Tor, als hämmere
jemand mit präzisen kurzen Schlägen auf Metall ein.

Ich trat näher heran.

»Hallo? Ist es gestattet, einzutreten?«

Das Hämmern verstummte, und kurz darauf öffnete sich das Tor.
Gérard Le Bon stand vor mir, mit einem hellen Schutzanzug
bekleidet, und kämmte sich mit ölverschmierten Fingern das blonde
Haar aus den Augen. Er war einen halben Kopf größer als ich und
packte mich wie ein Bär, um mir die Hände zu schütteln.

»Henry, wie schön, dass Sie einmal vorbeischauen!«, rief er
herzlich. »Ich wollte Sie schon lange einladen, aber Yves sagte,
Sie müssten sich erholen.«

»Manchmal ist Yves fürsorglicher als eine Amme«, murrte ich. »Wenn
es nach ihm ginge, säße ich den ganzen Tag mit einer Decke über den
Beinen im Lehnstuhl. Wie steht es mit einer Erfrischung?« Einladend
schwenkte ich die Flasche.

Gérard nickte. »Gern. Irgendwo habe ich bestimmt Gläser.«

Er kehrte in die Scheune zurück, wühlte herum und kam mit einem
Weinglas und einem Cognacschwenker wieder heraus. Wir hockten uns
auf die Kisten, ich schenkte ein und nahm einen tiefen Schluck.
Heiter lehnte ich mich zurück und reckte die Beine in die
Frühlingssonne. Ich hatte mich lange nicht mehr so wohl gefühlt.

»Wie geht es Claudine?«, erkundigte sich Gérard.

Ich zuckte die Achseln. »Gut, vermutlich. Sie ist ins Dorf gegangen
und ließ mir ausrichten, ich solle nicht mit dem Mittagessen
warten.«

Möglicherweise hatte sich ein harter Unterton in meine Stimme
eingeschlichen, denn er blickte zu Boden, räusperte sich und zupfte
unbehaglich mit Daumen und Zeigefinger an seiner Unterlippe.

Wir schwiegen eine Weile, während ich wie ein Schuljunge auf der
Kiste herumrutschte und überlegte, wie ich das Gespräch auf das
Gerät in der Scheune lenken könnte, an dem er gehämmert hatte.
Gérard beobachtete mich aus den Augenwinkeln und ließ mich zappeln.
Schließlich, als ich kurz davor war, meine unbezähmbare Neugier
einfach herauszuschreien, stellte er das Glas ab und stand auf.

»Wollen Sie mir bei den letzten Nieten helfen?«

»Gern!« Sofort erhob ich mich ebenfalls. »Ich bin Ingenieur, das
heißt, ich war es, vor dem Krieg. Seit ich gehört habe, dass Sie in
der Scheune etwas bauen, brenne ich darauf, es zu Gesicht zu
bekommen.«

»Ingenieur also! Dann brauche ich Sie nicht mit langen Erklärungen
zu behelligen. Kommen Sie mit.«

Er ging voraus und stieß beide Flügel des Tores weit auf, um Licht
hineinzulassen. Ich folgte ihm und spähte über seine Schulter.

\bigpar

Im Inneren der Scheune ruhte, ein wenig zur Seite geneigt, ein
großer Zeppelin, etwa fünfzehn Meter in der Gesamtlänge. Der Korpus
war aus hölzernen Spanten gebaut und mit Leinenstoff bezogen.
Darunter schaute eine Kabine aus genietetem Kupfer hervor, mit vier
Bullaugen versehen wie ein Schiff. Eine Leiter führte zu einer Tür
hinauf. Unter der Kabine befand sich ein weiterer Raum, und auf
diesem hatte Gérard Kupferplatten festgenietet, als mein Klopfen
ihn unterbrach.

Beifällig glitten meine Augen über die sinnreichen Verstrebungen,
die Drahtseillenkung für Höhen- und Seitenruder und den stabilen
Propeller, aber dann ließ ein Kupferrohr mich stutzen, das von dem
untersten Raum geradewegs in den Korpus führte.

»Es funktioniert mit Heißluft«, erläuterte Gérard. »Die erwärmte
Luft ist ein Nebenprodukt des Dampfantriebs für den Propeller. Sie
wird durch das Rohr in den Korpus eingespeist, und der Druck \ldots{}
ach, am Besten steigen Sie ein und schauen es sich an.«

Ich ließ mich nicht zweimal bitten. Rasch erklomm ich hinter Gérard
die Leiter und folgte ihm durch eine schmale Schleuse ins Innere
des Luftschiffs.

Ein kleiner, äußerst komfortabler Raum tat sich vor mir auf. Direkt
hinter der großen Frontscheibe in ihrem Messingrahmen standen zwei
mit Leder bezogene Drehsessel, vor denen übersichtlich die
Instrumente angeordnet waren. Ich erkannte einen Kompass,
kardanisch aufgehängt in einem hölzernen Kasten, in dem er vor
allen magnetischen Ablenkungen geschützt war, die Anzeige eines
Höhenmessers und eine weitere, die die Geschwindigkeit in Knoten
auswies. Auch ein Morsegerät konnte ich ausmachen und eine Uhr in
einem äußerst stoßfest wirkenden Gehäuse. Vor dem linken Sessel,
offensichtlich dem des Piloten, befand sich ein Steuerknüppel mit
Mahagonischalen.

Rechts neben dem Sessel des Copiloten waren weitere, kleinere
Anzeigen. Ich beugte mich vor und musterte sie interessiert.

Gérard trat zu mir. »Wasserstand, Kohlevorrat, Gewicht,
Druckmesser, Seiten- und Höhenruderstand, Drehgeschwindigkeit«,
erläuterte er knapp, während er mit dem Finger darauf wies. »Das
muss alles bedacht werden, bevor man in den Æther startet.«

»In den Æther!«, entfuhr es mir. »Bei allen Göttern, Sie wollen
doch mit diesem Gefährt nicht etwa auf Sternenreise gehen!«

»Aber gewiss doch – was sollte mich aufhalten!«

Wie er dort stand, breitschultrig und selbstbewusst, war ich
beinahe bereit, ihm Glauben zu schenken.

»Aber«, wandte ich dennoch ein, »woher kommt die Atemluft während
der Reise?«

Er wies auf etwas, das ich für einen Ventilator gehalten hatte.
»Dieses von mir entwickelte Gerät wird während der Reise
kontinuierlich Sauerstoff in die Luft dieses Raumes einspeisen.
Sehen Sie – es hat sogar eine Kontrollanzeige.«

»Für eine Reise durch den Æther benötigen Sie eine enorme Menge an
Kohle«, wandte ich weiter ein.

»Lediglich für den Start, denn sobald die Anziehungskraft der Erde
überwunden ist, gleitet mein Heißluft-Ætherschiff in den
Sonnenwinden und dem Ætherstrom dahin. Aber schauen Sie selbst, ich
habe eine ungeheuer leistungsfähige Dampfmaschine dort unten.«

Wir kletterten über eine enge Leiter in den Raum unter der Kabine,
der beinahe ganz von einem glänzenden Kessel ausgefüllt war.

»Wie kamen Sie auf die Idee«, erkundigte ich mich während des
Abstiegs, »anstatt des Heliums mit erhitzter Luft zu arbeiten?«

Er zuckte in einer großartigen Geste die Achseln. »Es lag auf der
Hand, mon ami! Sollte der Korpus eines gewöhnlichen Zeppelins auf
dem Weg zu den Sternen einmal leck schlagen, dann könnten Sie das
ausströmende Helium nicht ersetzen und müssten auf ewig verloren
durch den Æther driften. Luft und Wärme hingegen produziert mein
sinnreiches Luftschiff kontinuierlich neu, sodass es ausreichen
würde, die Außenhülle zu reparieren, und dank dieses wunderbaren
Gerätes ginge die Fahrt alsdann ohne Verzögerung weiter.«

Er wies auf die Dampfmaschine, das Herz seiner Erfindung. Glatt und
blank lag sie da, und obschon ich wusste, dass sie nichts als ein
Konstrukt aus Zylindern, Ventilen und einem großen Kessel war,
erschien sie mir beinahe lebendig, als schlage ein Puls unter ihrem
glatten Äußeren, als atme sie und schmiege sich erwartungsvoll in
das ihr zugedachte Gehäuse.

»Die beste ihrer Art, und gerade gut genug für meine Zwecke. Ein
solches Ætherschiff, mon cher«, dröhnte Gérard, »erbaut man nicht
auf einer kleinen, engen Insel, die verloren im Ozean liegt. Dazu
braucht es den festen Untergrund eines Kontinents, den sicheren
Boden der Normandie.«

Ich entzifferte das Herstellerschild auf der Dampfmaschine.
»Fawcett \& Preston, Liverpool«, stand darauf. »Sicherlich die
beste ihrer Art«, lächelte ich still. »Und wann soll das
Heißluft-Ætherschiff vom Stapel laufen?«

»Jederzeit«, erwiderte er. »Sobald die letzten Nieten sitzen. Wie
wäre es mit jetzt gleich?«

Ich schüttelte den Kopf. »Sie sind ja verrückt!«

»Keineswegs!« Zusehends erwärmte er sich für diesen Gedanken. »Das
Schiff ist fertig. Wasser habe ich im Brunnen, die nötige Kohle
lagert im Keller. Dort sind auch ein paar Vorräte. Und Sie sind
Ingenieur, genau der Richtige für den Copilotenplatz. Einen zweiten
Schutzanzug habe ich auch – er wird Ihnen ein wenig zu groß sein,
fürchte ich, aber das macht sicher nichts. Wenn wir gemeinsam die
letzten Nieten anbringen, dann beladen und vorheizen, wären wir in
zwei Stunden reisefertig. Ein kurzer Ausflug nur, einmal um unsere
alte Erde und wieder zurück. Sind Sie dabei?«

»Claudine wird nicht wissen, wo ich bin«, wandte ich ein.

»Claudine ist mit ihren Freundinnen im Dorf. Sie werden Hüte
anschauen und Handschuhe anprobieren. Dann gehen sie in das Café
von Madame Fleury, essen Apfelkuchen und trinken Kaffee. Während
sie über die Hüte und die Handschuhe schwatzen, nehmen sie einen
kleinen Calvados, pour faire le trou normand – um ein wenig Platz
im Magen zu schaffen. Und schon ist es Abend. Bis dahin sind wir
längst zurück!«

»Yves wird sich sorgen. Er kocht Tripes à la mode.«

Ein Schatten glitt über Gérards Gesicht. »Welch ein Jammer! Aber
man kann es aufwärmen, morgen sind sie noch besser. Also, wie steht
es?«

»Nun gut!« Lachend hob ich die Hände. »Sie haben mich überredet.
Ein kurzer Ausflug in den Æther und wieder zurück. Nur um zu
schauen, ob Ihr Heißluft-Ætherschiff funktioniert.«

»Das wird es, seien Sie gewiss.« Er rieb sich die Hände. »Was für
ein Glück, dass gerade Sie heute Morgen vorbeigekommen sind!«

\bigpar

Wir hämmerten die Nieten fest, überprüften noch einmal die
Verstrebungen und testeten die Lenkung. Mit einer Reihe von
Seilwinden schafften wir anschließend das Luftschiff aus das
Scheune hinaus auf die Wiese.

Wir fühlten uns wie Schulbuben in einer geheimen Verschwörung als
wir die Vorräte aus dem Keller holten, ein paar weitere Flaschen
Cidre, ein frisches Brot, verschiedene Sorten Käse und eine Paté.
Dann schleppten wir die Kohle und das Wasser herbei und heizten den
Kessel an, steckten den Schutzanzug um, der mir viel zu groß war
und in dem ich versank, als sei ich tatsächlich ein kleines Kind,
schließlich, während der Kessel sich langsam erwärmte, hockten wir
noch eine Weile auf den Kisten vor der Scheune.

Gérard zupfte wieder einmal mit den Fingern an seiner Unterlippe
herum.

»Nervös?«, erkundigte ich mich.

Entschieden schüttelte er den Kopf.

»Nein, mon ami. Ich weiß, dass es funktionieren wird. Es ist
lediglich \ldots{} wie eine Vorahnung, verstehen Sie? Ich glaube, wir
werden ein Abenteuer erleben.«

Ich nickte. Genau dieses Gefühl hatte ich auch, es prickelte durch
meine Adern wie der Cidre.

Entschlossen erhob er sich. »Nun denn, steigen wir ein! Es ist noch
früh genug, der Wind steht günstig. In einer Stunde werden Sie die
Erde von oben sehen, eine große blaue Murmel inmitten des Æthers.
Kommen Sie!«

\bigpar

Wir kletterten ins Innere. Gérard nahm seinen Platz im
Pilotensessel ein, ich schlüpfte wie selbstverständlich auf den des
Copiloten, überprüfte aus alter Gewohnheit die Funktionsfähigkeit
der Geräte, indem ich mit dem Fingernagel gegen die Gläser
schnippte.

»Wasserstand und Kohlevorrat?«, fragte Gérard.

»Maximum!«

»Druck?«

»15 Pascal!«

»Ideal. Dann lassen Sie uns abheben.«

Wir lösten über einen Hebel die Leinen, holten sie ein und stiegen
gemächlich empor. Die Scheune schrumpfte zu einem Puppenhaus
zusammen, dann verschwammen die Obstwiesen und Hecken zu einem
Schachbrettmuster, schließlich breitete sich ein grüner Teppich
unter uns aus, nur gelegentlich durchbrochen durch die
schwärzlichen Flecken der Häuser. Gérard betätigte den
Steuerungshebel.

»Wir müssen zuerst ein Stück aufs Meer hinaus, damit wir genügend
Aufwind für den Weg in den Æther bekommen«, erläuterte er. »Unter
Ihrem Sitz liegt eine Karte. Machen Sie sich doch schon einmal mit
ihr vertraut.«

Ich ertastete das gefaltete Pergament, hob es auf meinen Schoß und
strich es glatt. Ein wirres Durcheinander aus Punkten und Linien
breitete sich vor mir aus, die noch dazu nicht stillzuhalten
schienen, sondern unter meinem Blick verschwammen, sich voneinander
entfernten und sich gleich darauf wieder anzunähern schienen.

»Gérard? Ich kann das nicht lesen.«

Ohne hinzusehen, griff er herüber und zog die Karte im Mittelfalz
eine Handbreit nach oben. Sofort sortierten sich die Punkte zu den
mir bekannten Sternbildern, die Linien bildeten Routen und
Verbindungsstrecken.

»Wie haben Sie das gemacht?« Sehr behutsam zog ich an den Rändern
der Karte, bis die Bilder wieder verwischten und in Bewegung
gerieten. Dann schob ich sie erneut zusammen, an einem anderen Falz
dieses Mal, probierte eine Weile herum, und die Bilder wurden
wieder klar. Verdutzt rieb ich mir die Augen. »Was für eine
seltsame Karte ist das?«

Gérard lächelte. »Im Æther, mon cher, gibt es kein Oben und Unten,
kein Rechts und Links. Unsere herkömmlichen Bezugspunkte verlieren
ihre Bedeutung. Deswegen würde uns eine gewöhnliche Karte nichts
nützen. Und so habe ich diese Sternenkarte entwickelt. Sie schafft
ihren eigenen Bezugspunkt, immer dort, wo man ihn braucht. Sehen
Sie die Buchstaben am Rand?«

Ich entzifferte S, M, V, E, M, M, J, S, U, N.

»Können Sie sich denken, mon ami, wofür sie stehen?«

»Gewiss!«, rief ich. »Sonne, Merkur, Venus, Erde, Mond, Mars,
Jupiter \ldots{} Aber Sie wollen doch nicht allen Ernstes behaupten, Sie
könnten mit diesem Heißluft-Ætherschiff bis zum Jupiter oder zum
Saturn, ins Äußere Sonnensystem!«

Er zuckte lediglich die Achseln. »Bisher noch nicht, aber wenn es
mir gelingt, die Außenhülle ein wenig stabiler zu machen, damit sie
größerem Druck standhält \ldots{} Im Augenblick komme ich nicht über 15
Pascal, aber ich denke über weitaus höhere Werte nach. Mit 500
Pascal beispielsweise könnte man es durchaus bis zum Uranus
schaffen. Vielleicht sogar noch weiter.«

»500 Pascal – Sie träumen!«, tadelte ich. »Welches Material gibt es
wohl, das solch einen Druck ertragen würde! Nun, immerhin der Mars
könnte erreichbar sein. Bei guter Vorbereitung und einem günstigem
Sonnenwind.«

Er hatte mich bei meinen Worten sehr aufmerksam angesehen. »Sie
wären also mein Copilot – sofern ich jemals eine Reise zum Mars
planen würde?«

»Nein. Ja! Beim Zeus, da haben Sie mich sauber in die Falle
gelockt!« Ich musste lachen. »Wenn Sie also ein Material finden
könnten, das einen wesentlich höheren Druck ertrüge als das
bisherige, und wenn Sie den Korpus Ihres Heißluft-Ætherschiffes
damit zu verkleiden wüssten. Wenn Sie außerdem einen Weg fänden,
genügend Vorräte und vor allem reines Trinkwasser an Bord zu
nehmen. Und wenn Sie außerdem Claudine und Yves erklären würden,
weshalb ich in den nächsten Wochen nicht pünktlich zum Essen käme –
dann, mein lieber Freund, wäre ich tatsächlich bereit, als Ihr
Copilot in diesem Gefährt bis zum Mars zu reisen.«

Gérard nickte, als habe er nichts anderes erwartet. »Nun gut, dann
ist das abgemacht, mon ami. Aber fürs Erste umkreisen wir nur die
Erde, der Mars muss noch eine Weile warten.«

\bigpar

Inzwischen waren wir über dem Meer, der grüne Teppich unter uns war
von einer unruhigen, silbrig schäumenden Oberfläche abgelöst
worden. Der Aufwind griff nach unserem Gefährt und drückte uns
empor, stärker als ich es für möglich gehalten hatte.

»Jetzt gilt es!« Gérard beschleunigte den Propeller auf höchste
Leistung und zog dann den Steuerungshebel mit aller Kraft auf sich
zu. Das Luftschiff richtete sich auf und preschte steil nach oben
in den Himmel. Die Aufwärtsbewegung presste mich in meinen Sitz,
ich klammerte mich an die Armlehnen.

Gérard warf mir einen seiner raschen Seitenblicke zu. »Es ist
gleich überstanden, mon cher. Je weiter wir uns von der Erde
entfernen, desto geringer wird ihre Anziehungskraft. Wenn wir
dieses Tempo halten können, werden wir in wenigen Minuten im Æther
schweben – dann lässt der Druck nach.«

Ich nickte krampfhaft. Das Atmen wurde mir schwer, als presse
jemand mit aller Gewalt meinen Brustkorb zusammen. Vor meinen Augen
tanzten Funken, in meinen Ohren pfiff es, als sei ich wieder in
Afghanistan, in den Schützengraben geduckt zum Schutz vor den
Gewehrkugeln, und meine Kameraden – oh Gott, meine Kameraden! – Der
Albtraum der vergangenen Nacht griff nach mir und schwemmte mich
davon wie eine Flutwelle.

»Henry? Es ist gleich vorbei!«

Mühsam kam ich zu mir, blickte in Gérards besorgte Augen.

»Es geht mir gut«, würgte ich hervor.

Er schüttelte nur den Kopf, griff unter seinen Sitz und holte eine
Flasche hervor. Als er sie öffnete und den Verschlussbecher füllte,
stieg mir der Duft von Äpfeln in die Nase.

»Calvados?«

»Selbst gebraut. Auf einen Schluck hinunter damit. Vertrauen Sie
mir!«

Ich tastete blind nach dem Becher und schüttete das Gebräu in meine
Kehle. Zuerst spürte ich nichts. Dann brannte sich flüssiges Feuer
eine Bahn durch meine Kehle hinunter. Ich hustete, rang nach Atem.
Ein konvulsivisches Zittern durchfuhr mich, heiße Tränen traten in
meine Augen.

»Atmen!«, befahl Gérard.

Als ich seiner Anweisung Folge leistete, sprengte ich dabei die
eisernen Reifen, die sich um meine Brust gelegt hatten. Mit einem
unterdrückten Schluchzen sackte ich in meinem Ledersessel
zusammen.

»Es ist gut, mon ami. Wir sind im Æther. Jetzt wird es Ihnen besser
gehen.«

Ich richtete mich auf und blickte durch das Bullauge. Alles um uns
her war dunkel und ruhig. Ohne einen festen Bezugspunkt war es
unmöglich, unsere Geschwindigkeit auszumachen. Wir schienen im
Nirgendwo stillzustehen.

»Es tut mir leid«, murmelte ich verlegen. »Ich weiß gar nicht, was
in mich gefahren ist.«

»Das ist die Ætherkrankheit«, stellte er fest. »Der Druck beim
Aufstieg führt zu Atemnot und Angstzuständen. Manche leiden mehr
darunter, manche weniger, aber verschont hat sie meines Wissens
beim ersten Ausflug in den Æther noch niemanden. Doch jetzt ist es
überstanden. Lassen Sie uns die Erde betrachten, unseren einmaligen
blauen Planeten. Die Aussicht ist atemberaubend.«

Er bewegte den Steuerknüppel, sodass wir in die Umlaufbahn
einschwenkten. Erwartungsvoll blickte ich hinaus, doch alles, was
ich erkennen konnte, war eine schmale goldene Sichel.

»Wir sind über dem Ozean«, erläuterte Gérard. »Die Sonne geht eben
über dem Horizont auf. Da – sehen Sie!«

Die Sichel wurde breiter, einzelne Strahlen tasteten sich vor und
glommen auf wie Glut im Kamin. Dann erkannte ich das Meer, blau und
wolkenverhangen. Immer weiter rundete sich die Erde unter meinen
Blicken, ich sah die bräunlichen Kontinente, die weißen Polkappen
und die Wolkenfetzen, die darüber hinwegtrieben.

»Was für ein unglaublicher Anblick!«, entfuhr es mir.

Gérard lächelte voller Besitzerstolz.

»Wir sind gleich über Nordfrankreich«, stellte er fest, »und können
das Morsegerät der Bahnstation von Calvados erreichen. Möchten Sie
Yves eine Nachricht senden?«

Damit schob er mir das Morsegerät zu.

Während ich weiter durch das Bullauge hinausschaute, tippte ich auf
die Taste ein. »Yves, warte nicht mit dem Essen, wir sind auf einem
längeren Ausflug und erst gegen Abend zurück«, sprach Gérard
halblaut mit. »Nun, immerhin weiß er jetzt, dass er sich keine
Sorgen machen muss.«

Ich nickte und wollte weiter hinaussehen, als ich aus den
Augenwinkeln eine Bewegung in der Kabine bemerkte. Der Kompass
schlug aus. In seiner kardanischen Aufhängung war er vor allen
magnetischen Einflüssen geschützt, dennoch geriet er immer heftiger
in Erschütterung, die Nadel begann schließlich sogar zu rotieren.
Irritiert blickte ich auf die übrigen Instrumente. Auch die
Anzeigen für Höhe und Druck spielten verrückt und schlugen
unkontrolliert aus.

»Gérard, schauen Sie doch! Hier ist etwas nicht in Ordnung!« Mit
dem Fingernagel schnippte ich gegen die Gläser, die leise
schepperten.

Für einen Augenblick schien er ebenso verwirrt wie ich. Dann schlug
er sich mit der Hand vor die Stirn. »Die Sonnenstürme!«, stöhnte
er. »Sie waren für heute Nachmittag angekündigt. Ich hätte die
Warnung ernster nehmen sollen. Nur gut, dass wir nicht höher
aufgestiegen sind und nach Sicht manövrieren können. Lassen Sie uns
den Ausflug abbrechen und zurückkehren, rasch, solange es nur die
Messinstrumente sind, die durch den erhöhten Magnetismus
beeinträchtigt werden!«

Er schob den Steuerknüppel nach vorn, um den Sinkflug einzuleiten.
Das Ætherschiff bäumte sich auf wie ein widerspenstiges Pferd und
drehte sich mit einem protestierenden Quietschen aus seiner
Umlaufbahn. Aber gerade als es sich vollkommen quergestellt hatte,
traf uns ein gewaltiger Stoß, erschütterte unser Gefährt, als sei
es ein Kinderspielzeug, und warf uns beinahe aus den Sitzen.

»Was war das?«, schrie ich.

Gérard ruckte am Steuerknüppel. Seine Knöchel wurden weiß von der
Anstrengung, aber der Knüppel bewegte sich nicht.

»Etwas hat unsere Lenkung verkeilt«, keuchte er, während er sich
wieder und wieder gegen das störrische Gerät stemmte. »Ein
Meteoritensplitter vielleicht. Helfen Sie mir!«

Nun zerrten wir beide mit aller Kraft an dem Hebel, aber er saß
fest und rührte sich um keinen Zoll.

»Nun gut!« Er stand auf, griff nach der Fliegerkappe, die unter
seinem Sitz deponiert war, und war schon auf dem Weg zur Tür. »Dann
gehe ich hinaus und versuche den Splitter in der Lenkung zu
lösen.«

Ich eilte ihm nach. »Nein, Sie werden hier drinnen gebraucht. Ich
kann das Ætherschiff nicht bedienen. Lassen Sie mich aussteigen und
den Schaden beheben.«

Er setzte zu einer Antwort an, da traf das Schiff erneut ein
furchtbarer Schlag. Ich verlor das Gleichgewicht und krachte gegen
Gérard, der dadurch mit dem Kopf gegen die Tür prallte und schwer
zu Boden ging. Eine weitere Erschütterung schleuderte mich durch
den Raum wie eine Lumpenpuppe, ich nahm noch wahr, wie unser Schiff
zu schlingern und zu trudeln begann, dann schwanden auch mir die
Sinne.

\bigpar

Ich erwachte mit schweren, verkrampften Gliedern, halb hinter dem
Sessel verkeilt. Von der Tür her hörte ich ein leises Stöhnen.

»Henry?«, fragte Gérard mit matter Stimme. »Mon ami, geht es Ihnen
gut?«

»Den Umständen entsprechend, ja«, erwiderte ich. »Und wissen Sie
was? Dies ist das erstemal seit Monaten, dass ich geschlafen habe,
ohne von Herat zu träumen. Was ist geschehen?«

Er rappelte sich hoch und blickte sich in der Kabine um. »Der
Sonnensturm scheint einen Meteoritenschauer ausgelöst zu haben,
dessen Ausläufer unser Schiff aus der Bahn geworfen haben. Nun, wir
leben noch. Das ist ein Zeichen dafür, dass die Kabine und unsere
Sauerstoffversorgung intakt geblieben sind. Die Geräte scheinen
ebenfalls wieder zu funktionieren. Was ist mit Ihnen? Sind Sie
verletzt?«

»Nein, ich denke nicht.« Ich bewegte Hände und Füße, zog dann die
Knie an und richtete mich schwerfällig auf. »Wie lange waren wir
ohne Bewusstsein?«

Er blickte auf die Uhr. »Wir sind gegen Elf gestartet, und jetzt
ist es kurz vor Drei. Es waren also mindestens vier Stunden. Nach
dem Hunger, den ich habe, könnten es allerdings auch sechzehn
Stunden gewesen sein.«

Als hätte er auf das Stichwort gewartet, knurrte mein Magen »Hunger
habe ich auch«, gab ich zu.

Gérard zog den Korb mit unserem Proviant hervor. »Bitte, bedienen
Sie sich. – Nanu, das Brot ist steinhart! Waren es sechzehn Stunden
oder am Ende gar noch mehr? Immerhin sind der Käse und die Paté
noch brauchbar.«

Ausgehungert machten wir uns über die Vorräte her und teilten eine
weitere Flasche Cidre.

»Wenn wir tatsächlich über sechzehn Stunden steuerlos durch den
Æther gedriftet sind«, fragte ich, »wo befinden wir uns dann?«

Gérard zuckte die Achseln und spähte durch das Bullauge. »Ich kann
keine mir bekannte Sternenkonstellation entdecken«, erwiderte er.
»Reichen Sie die Karte herüber.«

Ich gab ihm die Karte, und gemeinsam falteten und verglichen wir
das Pergament, bis Gérard plötzlich voller Stolz ausrief: »Hier ist
es! Schauen Sie nur, mon cher, wir haben es beinahe bis zum Mars
geschafft! Wir befinden uns in einer Umlaufbahn ungefähr auf halber
Strecke zwischen Phobos und Deimos, den beiden Marsmonden. Und die
Lufthülle hält, der Druck liegt immer noch bei 15 Pascal. Ich
wusste, es könnte gelingen. Das, mon ami, ist französische
Wertarbeit!«

Ich überprüfte die Instrumente. »Tatsächlich, der Druck ist noch
konstant«, stellte ich fest, »und auch um den Wasserstand brauchen
wir uns keine Sorgen zu machen. Allerdings muss die Dampfmaschine
neu angeheizt werden, die Kohle ist aufgezehrt.«

»Ah, was erwarten Sie von einer Dampfmaschine aus Liverpool!«, rief
er wegwerfend. »Sie findet doch immer eine Ausrede, um zu streiken.
Aber das ist bald behoben. Schlimmer ist der Ausfall unserer
Steuerung. Das Höhenruder ist beschädigt, und das Seitenruder
scheint von den Meteoriten komplett weggerissen worden zu sein. In
dieser Umlaufbahn bewegen wir uns fürs Erste zwar komfortabel, aber
wenn wir zur Erde zurückkehren wollen, müssen wir es irgendwie
ersetzen. Werkzeug für die Reparatur haben wir an Bord, aber kein
Holz von geeigneter Form und Größe. Immerhin sind wir nicht völlig
aus der Welt, der Mars liegt auf einer belebten Handelsroute, und
wenn wir Glück haben, kommt innerhalb der nächsten Wochen ein
Ætherschiff vorbei.«

»Innerhalb der nächsten Wochen?« Ich starrte betroffen auf die
spärlichen Reste im Vorratskorb. Schon jetzt, in halbwegs
gesättigtem Zustand, vermisste ich Yves’ Miesmuscheleintopf, und
beinahe sogar seine Tripes à la mode. »Was sollen wir bis dahin
machen?«

»Ausschau halten, mon ami!« Er drückte mir ein großes Fernrohr aus
Messing in die Hand. »Sie halten Ausschau und morsen das erste
Fahrzeug an, das in Sicht kommt, und ich versuche bis dahin,
wenigstens das Höhenruder wieder flott zu machen.«

Damit setzte er die Fliegerkappe auf, zog die Brille und das
Mundstück für die Sauerstoffversorgung vor das Gesicht und verließ
unser angeschlagenes Gefährt durch die Schleuse. Durch ein Bullauge
sah ich zu, wie er sich draußen mit einem Tau festmachte und zum
Höhenruder hangelte.

»Ausschau halten, aber wie?« Mit dem Rohr in der Hand ging ich von
Bullauge zu Bullauge, doch meine Sicht war in jedem Fall zu
beschränkt. Ich beschloss, dass ich selbst auch hinaus musste.
Unter meinem Sessel fand ich eine weitere Fliegerkappe, und mit
dieser ausgerüstet, folgte ich Gérard in den Æther.

Obwohl der Schutzanzug sich sofort aufblähte wie ein Federbett, war
es eiskalt, als sei ich in einen zugefrorenen Teich gesprungen. Es
gelang mir kaum, mit klammen Fingern das Haltetau festzuhaken, und
als ich über die Streben den Aufstieg zum Korpus begann, schien mir
die Kälte mit jedem Atemzug direkt in die Lungen zu dringen.

Ich sah Gérard am Höhenruder arbeiten und wollte ihm nichts
nachstehen, deswegen kroch ich weiter empor, das Fernrohr eng an
die Seite gepresst. Meine Augenlider schienen zu gefrieren, das
Zwinkern wurde mir schwer. Um der Kälte zu entgehen, atmete ich zu
flach. Ich bemerkte es, konnte es aber nicht verhindern. In meinen
Ohren begann es zu dröhnen wie Kanonendonner, mit einem Angstschrei
presste ich mich an den Leinenstoff, der mir gleich darauf wie ein
Leichentuch erschien, mit dem Gerippe der Spanten darunter, von dem
ich schaudernd wegstrebte.

Mit aller Kraft kämpfte ich gegen die Schreckensbilder, zwang mich,
das Fernrohr hochzuheben. Ich glaubte in weiter Ferne einen
Schatten vorüberziehen zu sehen, versuchte ihn genauer ins Auge zu
fassen, aber ein knackendes Geräusch neben mir ließ mich jammernd
zusammensinken. Ich fiel direkt in Gérards Arme.

Minuten später hatte er mich zurück in die Kabine geschafft und
flößte mir seinen selbstgebrauten Calvados ein. »Mon ami, das
Einzige, das größer ist als Ihre Torheit, ist Ihre
Hartnäckigkeit.«

»Aber es hat sich gelohnt!«, protestierte ich. »Da war etwas, ich
habe ein Schiff gesehen!«

Verdutzt blickte er mich an. »Sind Sie sicher, dass es nicht nur
ein Mondschatten war?«

»Ganz sicher. Ich habe die Masten erkannt. Ist es vorbeigefahren?«

»Gewiss ist es noch nicht außer Reichweite. Versuchen Sie es
anzumorsen.«

Ich morste CQD, die internationale Sequenz für einen Notfall.
Dreimal. Ein viertes Mal. Endlich kam eine Antwort. »Wo efind Sie«,
sprach Gérard amüsiert mit. »Der Funker hat einen lahmen
Zeigefinger.«

»Das ist in der Tat merkwürdig«, erwiderte ich. »Soweit ich weiß,
ist für sämtliche Offiziere an Bord eines Ætherschiffes eine
Morseprüfung zwingend vorgesehen. Aber vielleicht klemmt nur die
Taste.«

Ich gab unsere Position an und erklärte unsere Schwierigkeiten
wegen des abgerissenen Seitenruders. Schließlich bat ich darum, an
Bord kommen zu dürfen.

Die Antwort ließ lange auf sich warten. Endlich hörten wir:
»Schicken Niarkasse.«

Jetzt lachte Gérard hell auf. »Natürlich meinen sie die Barkasse.
Sie haben Recht, mon ami, die Taste klemmt.«

\bigpar

Wir blieben weiter in Kontakt, ich gab von Zeit zu Zeit unsere
Position durch, während das Ætherschiff sich näherte. Endlich kam
es in Sicht – und ich hielt den Atem an. Es war ein riesiger
Klipper unter voller Besegelung, vierfach übereinander türmten sich
an den Masten die Rahsegel auf, zu beiden Seiten waren zusätzliche
Leesegel gesetzt und an Bugspriet und Fockmast wölbten sich die
dreieckigen Stagsegel. Der schmale kupferbeschlagene Rumpf glänzte,
als sei er poliert, er schien durch den Æther zu schneiden wie eine
Klinge.

Der Klipper blieb in sicherem Abstand, ein Teil der Segel wurde
eingeholt, um die Fahrt zu verlangsamen. Wir sahen zu, wie die
Barkasse ausgesetzt wurde, ein geschlossenes kleines Beiboot, das
sich uns zielstrebig näherte. Es dockte an unserer Schleuse an, und
Gérard beeilte sich zu öffnen.

Vier Æthermänner standen vor uns, alle in kurzen engen Jacken, den
»Draught-nots«, und weiten, aufgekrempelten Hosen, die langen Haare
mit Tüchern aus dem Gesicht gebunden. In den Gürteln trugen sie
kurze breite Messer.

»Was für eine Art von Schiff ist das?«, fragte einer,
offensichtlich der Wortführer, anstelle einer Begrüßung. Er war
groß und hager, sein Gesicht war entstellt durch eine lange Narbe,
die von der Nasenwurzel knapp unter dem rechten Auge hindurch bis
zum Ohr reichte. Als er bemerkte, dass ich ihn anstarrte, machte
er: »Buh!« und rüttelte an seinem Messer.

»Dies ist ein Heißluft-Ætherschiff, das erste seiner Art«,
erwiderte Gérard freundlich. »Mein Name ist Gérard Le Bon, der Herr
dort ist mein Begleiter Henry Sutton. Willkommen an Bord. Mit wem
haben wir die Ehre?«

»James Hanway, Bootsmann auf der \emph{Olifant}«, entgegnete der
Æthermann. »Wir sollen Ihr \ldots{} Heißluft-Ætherschiff ins Schlepptau
nehmen, um den Schaden zu begutachten. Und Sie und Ihren Begleiter
bittet Kapitän Anstis an Bord.«

Ich griff nach meiner Fliegerkappe.

»Die Schutzanzüge können Sie hier lassen«, sagte Hanway. »Wir sind
komfortabel überdacht.«

Gérard und ich wechselten einen Blick. Auch wenn wir die
Schutzanzüge auf der \emph{Olifant} nicht brauchten, waren wir ohne
sie zugleich auf dem Klipper gefangen. Aber Hanways Worte hatten
nicht wie eine Option, sondern eher wie ein Befehl geklungen, und
in unserer derzeitigen Situation war es ratsam, ihn zu befolgen.

Wir verstauten die Schutzanzüge und die Fliegerkappen unter unseren
Sitzen, dann folgten wir Hanway durch die Schleuse auf die
Barkasse, während die übrigen Æthermänner unser Gefährt vertäuten.

Das Beiboot war eine Überraschung. Sein Inneres präsentierte sich
beinahe wie ein Salon, die Bänke waren fein gearbeitet und mit
Edelhölzern verziert, ein kristallener Lüster unter den
Deckenverstrebungen erfüllte den Raum mit Licht.

Hanway machte eine einladende Handbewegung. »Setzen Sie sich, wo
immer Sie mögen. Die Überfahrt wird mit Ihrem zusätzlichen Gewicht
ungefähr eine halbe Stunde dauern, also machen Sie es sich besser
bequem.«

Er selbst ließ sich mit einem Plumpsen auf eine der Bänke fallen,
und seine Leute, die ihm bald folgten, taten es ihm nach. Einer zog
das breite Messer aus dem Gürtel und begann an der Bank
herumzukratzen.

Ich hatte erwartet, dass Gérard plaudern und mit den Æthermännern
über sein Heißluft-Ætherschiff fachsimpeln würde, aber er war
ungewohnt schweigsam und spähte nur durch die hohen Fenster hinaus
zu unserem Gefährt, das hinter der Barkasse hergezogen wurde.
Deswegen schwieg auch ich.

\bigpar

Wie es der Bootsmann Hanway vorhergesagt hatte, erreichten wir nach
einer halben Stunde die \emph{Olifant}. Hatte sie aus der Ferne
schon riesig ausgesehen, so benahm sie mir jetzt vollends den Atem.
Sie war etwa 325 Fuß lang, besaß vier Masten, die so hoch in den
Æther aufragten, dass ich kaum ihre Spitze sehen konnte, war
wehrhaft ausgerüstet mit einer langen Reihe von Geschützen hinter
den Geschützklappen und zusätzlichen Drehbassen, die auf der Reling
montiert waren, und mit reichlichem Laderaum in den unteren Decks
ausgestattet.

Wir gingen längsseits und dockten an eine der Schleusen im
Zwischendeck an, durch die wir den Ætherklipper betraten.

»Ich habe Weisung, Sie sofort zu Kapitän Anstis zu bringen«, sagte
Hanway und schritt voraus, durch das Deck, eine Stiege hinauf und
zur Kapitänskajüte, an deren Tür er anklopfte.

»Aye!«, rief eine kräftige Stimme.

Hanway öffnete die Tür und stand stramm. »Kapitän Anstis, Sir! Die
Besatzung des Heißluft-Ætherschiffes, die Herren Le Bon und
Sutton!«, meldete er, trat dann zur Seite und winkte uns herein.
Hinter uns fiel schwer die Tür ins Schloss.

\bigpar

Wir standen vor einem großen, mit Schnitzereien verzierten Tisch
aus Mahagoniholz, auf dem einige Ætherkarten, ein Sextant, ein
Sternenglobus und mehrere Stechzirkel lagen. Dahinter saß ein
hochgewachsener, stämmiger Mann. Die Ætherreisen hatten sein Haar
beinahe weißlich blond gebleicht, seiner Haut einen dunklen
Kupferton verliehen. Seine Kleidung war elegant, wenn auch ein
wenig zu eng, und in der einen Hand hielt er ein Kristallglas mit
Rotwein, die andere hatte er fest um das Handgelenk einer jungen
Frau gelegt, die neben ihm stand, einer englischen Rose mit
milchweißer Haut und blassblondem Haar, deren Augen dieselbe
graublaue Farbe hatten wie das schlicht geschnittene Kleid, das sie
trug.

Gérard neben mir schnappte bei ihrem Anblick hörbar nach Luft. Der
Kapitän, der seinen Blick bemerkte, schloss voller Besitzerstolz
seine Finger noch fester um ihren Arm.

»Meine Herren, willkommen an Bord!«, dröhnte er. »Ich bin John
Anstis, Kapitän der \emph{Olifant}, und dies« – er zog die Frau an
sich, die leise aufstöhnte – »ist meine Ehefrau Madeleine. Was für
ein interessantes Gefährt hat Sie hergebracht? Ein
Heißluft-Ætherschiff, so etwas erscheint mir vielversprechend für
den interplanetaren Verkehr.«

Gérard schüttelte lächelnd den Kopf. »Leider ist es noch in der
Erprobungsphase und nicht besonders zuverlässig«, wehrte er ab.
»Schon ein leichter Sonnensturm hat unsere Steuerung abgerissen und
uns vom Kurs gebracht. Bis man es für den Ætherverkehr einsetzen
kann, werden wir wohl noch einige Jahre daran herumbasteln
müssen.«

»Nur keine falsche Bescheidenheit!«, beharrte der Kapitän. »Sie
werden mir später alles zeigen und genau erklären müssen, vor allem
den neuartigen Antrieb. Aber fürs Erste darf ich Sie zum Essen
bitten. Sie werden hungrig sein nach Ihrer Irrfahrt.«

Ich nickte lebhaft, aber Gérards Gesicht blieb unbewegt. »Ich danke
Ihnen für die Einladung, Kapitän Anstis«, erwiderte er lediglich.

Nach und nach trafen die übrigen Tischgenossen ein und wurden uns
vorgestellt – Evans, der Erste Offizier, der sich spreizte wie ein
Pfau, Barley, der Zweite Offizier, ein verschlossener Rotschopf,
und Doktor Scudamore, der kleine bewegliche Schiffsarzt.

Wir setzten uns zu Tisch, und der Steward trug auf. Angesichts der
übervollen Schüsseln gingen mir die Augen über. Es gab eine Reihe
von Fleischspeisen, dazu Knödel, kaltes eingelegtes Gemüse und
Käse.

Gérard musterte die Speisen. »Wie lange sind Sie schon unterwegs?«,
fragte er Kapitän Anstis, der seinen Teller voll häufte.

»Die \emph{Olifant} reist auf der Mars-Venus-Route«, erwiderte
dieser kauend, »aber wir haben vor einiger Zeit von einem
Versorgungsschiff frische Vorräte übernehmen können. Greifen Sie
ruhig zu, es ist genug da.«

»Für ein Handelsschiff sind Sie außergewöhnlich gut bewaffnet«,
forschte Gérard weiter, während er sich zurückhaltend bediente.

»Wegen der Ætherpiraten«, warf Barley ein, eine Bemerkung, die bei
dem Ersten Offizier Evans für ausgelassene Heiterkeit sorgte.

Doktor Scudamore warf einen mahnenden Blick in die Runde und führte
bedachtsam die Gabel zum Teller. »In der letzten Zeit haben sich
auf dieser Route Piratenüberfälle gehäuft. Die meisten
Handelsschiffe schließen sich daher in den Häfen zu Konvois
zusammen – und wenn das nicht möglich ist, weil sie Termingeschäfte
abgeschlossen haben, verstärken sie ihre Bewaffnung, in der
Hoffnung darauf, die Piraten abzuschrecken.«

»Was transportiert die \emph{Olifant}?«, fragte ich.

»Maschinenzubehör«, schnappte der Kapitän. Er wandte sich an
Gérard. »Ihr Heißluft-Ætherschiff erscheint mir außerordentlich
wendig. Gewiss wären Sie damit in der Lage, auf einem der Marsmonde
zu landen, nicht wahr?«

Gérard zuckte nur die Achseln. »Wir haben es noch nicht
ausprobiert. Und auch wenn eine Landung möglich wäre, sehe ich
Schwierigkeiten beim Starten voraus, denn der Druck auf dem Korpus
\ldots{}«

»Aber mein lieber Freund«, wandte ich ein, »wieso sollte der Start
von Deimos aus mit größeren Schwierigkeiten verbunden sein als der
von der Erde?«

»Die Druckverteilung«, behauptete Gérard.

»Was ist denn an der Druck \ldots{}«

»Der Käse!«, unterbrach er mich. »Würden Sie mir die Käseplatte
reichen, mon ami?«

Ich folgte seinem Wunsch, obgleich ich mich über ihn wunderte.
Seine schroffe Art war ungewöhnlich, und erst recht befremdete
mich, dass er mehr an dem Käse interessiert schien als an der
Erörterung einer Landung auf einem der Marsmonde.

Ungeduldig ruckte er auf seinem Platz herum, trommelte mit den
Fingern auf die Tischplatte, als wünsche er nichts sehnlicher, als
dass dieses Essen zuende gehe und wir uns zur Ruhe begeben könnten.
Auch ich fühlte mich ermüdet durch die Abenteuer des Tages,
versuchte aber die Form zu wahren.

»Ob es wohl möglich wäre«, sinnierte Evans gerade, »den Stauraum
eines solchen Heißluft-Ætherschiffes zu erhöhen?«

Gérard schwieg, deswegen warf ich ein: »Man könnte gewiss mehrere
dieser Schiffe aneinander koppeln.«

»Wie sollte das möglich sein?«, widersprach Gérard scharf. »Eine
starre Verbindung müsste zerbrechen, und eine bewegliche hätte zur
Folge, dass die nachfolgenden Schiffe zu schwer unter Kontrolle zu
halten wären!«

»Aber wenn man \ldots{}«, überlegte ich weiter, doch Madeleine, die Frau
des Kapitäns, fiel mir ins Wort.

»Es wäre mir lieb, wenn die Herren den Abend beenden könnten«,
erklärte sie und legte dabei die Hand an den Mund, um ein Gähnen zu
unterdrücken. »Gewiss gibt es im weiteren Verlauf der Reise noch
genügend Gelegenheiten, um über die technischen Details zu
fachsimpeln.«

Kapitän Anstis fuhr auf, als wolle er sie zurechtweisen, besann
sich dann aber eines Besseren. »Du hast Recht, meine Liebe«, lenkte
er ein und legte den Arm um ihre Schultern. »Wir werden morgen
weitersprechen. Lass uns zu Bett gehen.«

Damit erregte er bei Evans erneut einen Heiterkeitsausbruch. Auch
Barley und der Doktor wechselten einen amüsierten Blick. Gérard
dagegen erhob sich auf der Stelle und verneigte sich leicht vor dem
Kapitän. »Wir danken Ihnen für unsere Rettung und den interessanten
Abend. Könnten Sie uns ein Quartier zuweisen?«, fragte er.

Nach kurzer Diskussion wurde uns eine Kabine im Orlopdeck
zugeteilt.

Wir folgten einem Æthermann, der uns über ein paar Stiegen hinunter
führte, und fanden uns in einem engen Verschlag wieder, notdürftig
ausgerüstet mit zwei Hängematten und einer morschen Seekiste, die
zugleich als Tisch diente.

\bigpar

Sobald wir allein waren, streckte ich die Glieder. »Nun, mein
Freund, was halten Sie von unserem Gastgeber?«

»Es ist gewiss das schönste Geschöpf, das ich je gesehen habe!«,
sagte Gérard mit unerwartetem Ernst.

»Warum nur habe ich den Eindruck«, lächelte ich, »dass Sie nicht
von Kapitän Anstis sprechen \ldots{} Aber leider ist die schöne Dame
schon vergeben. Sie sollten sie sich aus dem Kopf schlagen.«

»Wir treffen Madeleine in einer halben Stunde im Geschützdeck«,
erwiderte er zu meiner Überraschung.

»Aber wie konnten Sie sich mit ihr verabreden?«, fragte ich
verdutzt, als es mir plötzlich klar wurde. »Sie haben vorhin nicht
auf den Tisch getrommelt, sondern gemorst – mit einer
außerordentlich hohen Geschwindigkeit.«

Gérard nickte. »Nach den stümperhaften Nachrichten, die wir
empfangen haben, ging ich davon aus, dass keiner dieser sogenannten
Offiziere des Morsens ausreichend mächtig ist. Nicht genug
jedenfalls, um einer derart schnell übermittelten Nachricht auf die
Spur zu kommen.«

»Und wieso glaubten Sie, dass ausgerechnet die Dame Ihnen folgen
könnte?«

Er zuckte die Achseln. »Das musste ich riskieren. Mit wem, mon ami,
übt ein Æthermann für die Morseprüfung? Mit seiner Frau. Deswegen
können die meisten Offiziersfrauen ganz ausgezeichnet morsen,
mitunter besser als ihre Männer, denen sie es mühsam beigebracht
haben. Und siehe da – mein Plan hat funktioniert, sie hat mich
verstanden. Das Handzeichen, wissen Sie noch?«

»Als unterdrücktes Gähnen getarnt«, begriff ich. »Das ist skandalös
– eine Verabredung zum Rendezvous direkt unter den Augen des
Ehemannes \ldots{}«

»Anstis ist nicht ihr Ehemann, das ist doch offensichtlich«,
erklärte Gérard. »Und dieses Treffen ist alles andere als ein
Rendezvous. Wir müssen uns bewaffnen, mon ami. Aber womit? Alles,
was uns von Nutzen sein könnte, ist im Heißluft-Ætherschiff
zurückgeblieben, und wir müssen noch heute Nacht zuschlagen,
solange die meisten der Æthermänner betrunken sind.« Fieberhaft sah
er sich in unserem Verschlag um und begann schließlich, die
Seekiste zu zertrümmern. Verbissen trat er auf sie ein, bis sie
zerbrach.

Mit Befremdung sah ich ihm zu. »Was ist denn nur in Sie gefahren?«,
fragte ich. »Gegen wen sollen wir kämpfen?«

»Gegen die Piraten natürlich«, erwiderte er. »Aber kommen Sie, wir
wollen Madeleine nicht warten lassen. Sie riskiert viel.«

Er öffnete die Tür, eine der Leisten aus der Seekiste zum Schlag
erhoben, und blickte zu beiden Seiten. Dann atmete er auf. »Die
Piraten scheinen sich ihrer Sache sehr sicher zu sein. Ich hatte
mit Wachposten gerechnet. Mon cher, nehmen Sie sich auch eine
Leiste. Es ist nicht viel, aber doch immerhin besser, als wenn wir
mit bloßen Händen gegen sie antreten müssten. Dort – die Stiege
hinauf. Und seien Sie leise! Wenn es misslingt, werden nicht nur
wir dafür büßen.«

Noch immer hatte ich die Zusammenhänge nicht verstanden, aber ich
ergriff die provisorische Waffe und folgte ihm hastig.

\bigpar

Wir erreichten ungesehen das Geschützdeck. Gérard schaute sich
suchend um, dann bemerkte er eine Bewegung und flüsterte:
»Madeleine?«

Die junge Frau eilte zu uns. »Es wird ihm auffallen, dass ich fort
bin«, wisperte sie, »und er wird mich suchen lassen, durch diesen
Bootsmann, Hanway!« Ein Beben durchlief sie, als sie den Namen
aussprach, und sie atmete tief ein, um sich wieder zu fassen.

»Sie sind im Kabelgatt eingeschlossen, die Mannschaft, sofern sie
nicht übergelaufen ist, und mein Mann.«

»Ihr Mann?«, vergewisserte ich mich.

»Der Kapitän der \emph{Olifant}. Thomas Roberts. Der Mann, der sich
Ihnen als Kapitän Anstis vorgestellt hat, ist ein Ætherpirat – und
ein Schuft. Er und seine Spießgesellen haben das Schiff vor zwei
Tagen in ihre Gewalt gebracht und alle loyalen Männer dort unten
eingesperrt. Mich betrachten sie anscheinend als Beute.« Sie
erschauderte und wandte sich an Gérard. «Ich konnte mein Glück kaum
fassen, als Sie mir morsten, dass Sie meine Zwangslage erkannt
hatten und mich hier treffen wollten. Anstis \ldots{} Er hat mich
bedroht und geschlagen. Ich sollte mir nichts anmerken lassen,
Ihnen diese Komödie vorspielen.«

»Das Kabelgatt – ist es bewacht?«, fragte Gérard.

»Von zwei Mann. Aber ich habe den Steward angewiesen, ihnen Wein zu
bringen. Bordeaux aus den Privatvorräten meines Mannes.« Sie
lächelte schmerzlich. »Wenn die beiden den ganzen Krug geleert
haben, müssten Sie leicht mit ihnen fertig werden.«

»Sie sind sehr mutig, Madeleine«, lobte Gérard.

»Das bin ich nicht«, widersprach sie, »aber ich liebe meinen Mann.
Helfen Sie ihm! Ich muss zurück; Gott stehe mir bei, wenn sie mein
Verschwinden schon bemerkt haben.«

Damit glitt sie davon wie ein Schatten.

\bigpar

»Zum Kabelgatt also!«, beschloss Gérard und wandte sich zum Gehen.

Ich hielt ihn zurück. »Möglicherweise bin ich der Falsche, um Ihnen
beizustehen«, begann ich beschämt. »Sie wissen, ich bin Invalide.
Sie haben es gesehen, mein Versagen – mehr als einmal, zuletzt
draußen im Æther, als Sie das Höhenruder reparierten und ich nicht
imstande war, Ihnen zu helfen. Im Gegenteil, ich war lediglich eine
Last.«

»Mon ami, Sie sind doch nur ein wenig Ætherkrank geworden, und
\ldots{}«

»Nein«, unterbrach ich ihn. »Es ist weit mehr als das. Wann immer
diese Erinnerungen mich heimsuchen, bin ich wehrlos. Ein Krüppel,
ein Klotz an Ihrem Bein.«

Gérard schaute mich mit großem Ernst an. »Mon cher, Sie halten sich
für einen Krüppel?«, rief er. »Ausgerechnet Sie? Sie haben Ihren
ersten Aufstieg in den Æther besser hinter sich gebracht als jeder
andere, den ich kenne, mich selbst eingeschlossen. Nach einem
Schluck Calvados waren Sie wieder auf den Beinen. Mon dieu, die
meisten verbringen wenigstens ihre erste Woche im Æther auf dem
Rücken liegend, mit Atemnot und Panikanfällen. Kurz darauf haben
Sie einen gehörigen Sonnensturm erlebt, aber obwohl Sie mindestens
sechzehn Stunden bewusstlos waren, sind Sie unbeschadet wieder
aufgestanden und haben sich sogar unvorbereitet in den Æther
gewagt, um mir zu helfen. Ohne Anleitung, ohne Atemtraining – mit
nichts weiter als nur Ihrem festen Willen. Schon als ich Sie kennen
lernte, wusste ich, dass ich mir keinen besseren Copiloten wünschen
könnte – nun, da ich an Ihrer Seite fliegen durfte, bin ich
überzeugt davon, dass ich mit Ihrer Hilfe, Ihrem Können und Ihrem
Mut alles erreichen kann.«

Ich war dunkelrot geworden vor Verlegenheit und starrte auf meine
Füße. »So denken Sie über mich? Aber die Bilder, die schrecklichen
Traumbilder, die mir jede Kraft zum Handeln rauben \ldots{}«

»Glauben Sie denn, mon ami, nur Sie allein litten unter solchen
Bildern?«, fragte er mit sanftem Tadel. »Ist Ihnen noch nie in den
Sinn gekommen, dass wir alle, jeder einzelne von uns, gegen die
Schatten unserer Vergangenheit ankämpfen? Jeder trägt seine eigene
Last, und ob wir der Mensch sind, der wir sein wollen, liegt nur
daran, mit wie viel Entschlossenheit wir sie überwinden.«

Nun schämte ich mich noch mehr. »Nachdem Sie mich so genau kennen –
meine Schwäche und meinen Jammer – wie können Sie mich dann immer
noch Ihren Freund nennen?«

»Weil Sie jetzt hier sind«, erwiderte er schlicht, »und weil Sie
mit mir ins Kabelgatt hinunter gehen werden, um den kleinen
aufrechten Teil einer Æthermannschaft zu befreien, der sich lieber
einsperren ließ, als sich der Piraterie schuldig zu machen, und der
gegen eine Übermacht kämpfen wird, um dieses Schiff
zurückzuerobern.«

»Wir haben keine Waffen«, wandte ich ein. »Mit diesen kümmerlichen
Brettern können wir nicht viel ausrichten.«

Gérard sah sich um. »Dies hier ist besser!«, rief er triumphierend.
Aus einer Ecke neben einer Bordkanone zog er Gerätschaften, die zum
Laden benötigt wurden – einen Stopfer, so lang wie eine Lanze, eine
Schaufel und einen mit Sand gefüllten Eimer.

»Damit können wir schon mehr ausrichten, meinen Sie nicht, mon
cher? Vielleicht finden wir noch mehr davon – genug, um unsere
Mitstreiter angemessen auszurüsten!«

Wir durchsuchten das Deck, so rasch es ging, da wir fürchteten, zu
viel Zeit zu verlieren. Zwei Putzer und eine vergessene Kartusche
waren allerdings unsere gesamte Ausbeute. So machten wir uns auf
den Weg zum Kabelgatt.

\bigpar

Vorsichtig tasteten wir uns über die Stiegen hinunter. Gérard ging
voraus, er horchte bei jeder Biegung. Aber der größte Teil der
Piraten schien bereits in weinseligem Schlummer zu liegen, die
übrigen feierten ihren Sieg und ihre künftigen Beutezüge mithilfe
unseres Heißluft-Ætherschiffes in den oberen Decks.

Ein letzter Niedergang. Wir hörten beseelten, aber unartikulierten
Gesang. Zwei Gestalten tauchten vor uns auf, die mit Pistolen und
Messern in den Gürteln einen Verschlag bewachten. Zweifelnd blickte
ich zu Gérard, der aber taumelte polternd aus der Deckung und rief:
»Holla, mes amis! Habt ihr den Krug etwa ohne uns geleert?«

Die beiden Wachposten zögerten und schauten einander an. Gérard
trat zwischen sie und fasste sie bei den Schultern.

»Wachablösung, mes chers! Ihr dürft euch nach oben trollen! Aber
was ist denn das für eine Art? Habt ihr gar nichts für uns übrig
gelassen?«

Die beiden zögerten.

»Weißt du etwas von einer Wachablösung?«, fragte der eine seinen
Genossen, mit schwerer Zunge zwar, aber offensichtlich noch mit
wachem Geist.

Der andere griff nach seiner Pistole, versuchte sie aus dem Gürtel
zu nesteln und brach in wilde Flüche aus, als es ihm nicht gelang.
Ich rammte ihn den Stopfer in den Magen, dass er zusammenklappte,
und schwenkte gleich darauf den Eimer gegen unseren zweiten Gegner.
Es schepperte, als das blecherne Gefäß sein Kinn traf und ihn von
den Füßen holte.

»Bravo!«, rief Gérard. Mit sicheren Griffen untersuchte er den am
Boden Liegenden und nahm ihm außer den Waffen auch einen
Schlüsselbund ab. Er öffnete das Vorhängeschloss, streifte es ab
und zog die Tür auf. Im selben Moment ließ er sich blitzschnell
nach hinten fallen.

Dort, wo gerade noch sein Kopf gewesen war, sauste ein schwerer
Gegenstand durch die Luft.

»Jetzt drehen wir den Spieß um!«, brüllte jemand und machte
Anstalten, sich auf Gérard zu stürzen.

»Halt!«, schrie dieser. »Madeleine schickt uns, wir wollen Sie
befreien!«

Verdutzt hielt der Angreifer inne. Es war ein hochgewachsener,
dunkelhaariger Mann, dessen üblicherweise wohl glattrasiertes Kinn
von einem dunklen Stoppelbart beschattet wurde. In der Hand hielt
er ein Holzbein.

»Mein Name ist Gérard Le Bon, mein Begleiter hier ist Henry
Sutton«, erläuterte Gérard hastig. »Die Ætherpiraten haben uns mit
unserem beschädigten Heißluft-Ætherschiff aufgefischt; wir wurden
nur deswegen nicht sofort eingesperrt oder über Bord geworfen, weil
sie neugierig auf unser Gefährt sind und darauf hoffen, dass wir
ihnen die Funktionsweise erklären. Madeleine hat uns berichtet,
dass Sie hier festgesetzt sind, und sie hat die Wache mit Bordeaux
traktiert, sodass wir den beiden nur noch den letzten Rest geben
mussten. Ihre Frau ist ungeheuer couragiert, Kapitän Roberts.«

»Ja, das ist sie!« Der Mann trat einen Schritt zurück und eröffnete
uns dadurch einen Blick ins Kabelgatt. »Wir sind nur eine sehr
kleine Schar«, erklärte er, »aber jeder von uns ist entschlossen,
alles zu geben, um die \emph{Olifant} zurückzuerobern. Dort ist
mein erster Offizier, Williams. Er ist verwundet. Nur gut, dass
unser Doktor Travis loyaler war als sein Assistent Scudamore.«

Er deutete auf einen blassen jungen Mann mit einer Bandage um den
Kopf und einem notdürftig geschienten Arm, neben dem ein Älterer
kniete und ihn stützte.

»Der Zweite Offizier, Miles«, er nickte zu einem ernsten, hoch
aufgeschossenen Dunkelhaarigen herüber, »und der dort neben ihm« –
ein vierschrötiger Mann in den mittleren Jahren – »ist unser
Zimmermann Harris, er hat übrigens das Holzbein geschnitzt, das Sie
soeben verfehlte. Die beiden Ætherleute da drüben sind Jones und
Barlow, und hier ist unser Smutje. Da, Omally, haben Sie Ihr Bein
zurück.«

Er reichte seine provisorische Keule dem Smutje, einem ungeheuer
dürren und zappeligen Kerl. Der schnappte eilig danach und
schnallte es sich wieder um, während er mit dünner Stimme zirpte:
»Sie haben verlangt, dass ich die gesamte Wochenration Fleisch auf
einmal zubereite! Das geht doch nicht, da kommt ja die gesamte
Vorratshaltung durcheinander! Dagegen muss man etwas unternehmen!«

Kapitän Roberts und Gérard wechselten einen amüsierten Blick.

»Nun gut«, fasste Gérard zusammen. »Wir sind zehn Mann – und eine
Frau, denn Madeleine wird sicher einen Weg finden, uns zu helfen.
Als Waffen besitzen wir zwei Pistolen, zwei Messer und ein paar
Gerätschaften zum Bedienen von Kanonen. Nicht eben viel gegen
mindestens fünfzig Piraten, selbst wenn wir sie überraschen
können.«

»Wir brauchen Waffen«, stimmte ihm Roberts zu. »Aber dieser Kapitän
von eigenen Gnaden, Anstis, wird sie in der Waffenkammer
eingeschlossen haben und den Schlüssel in jedem Augenblick gut
bewachen. Das ist es jedenfalls, was ich selbst tun würde.«

»Die Kartusche!«, warf ich ein.

Beide wandten mir die Köpfe zu.

»Wir haben eine Kartusche gefunden, das Pulver darin ist trocken«,
fuhr ich fort.

Ich klammerte mich an den Stopfer, den ich noch immer in meinen
Händen hielt. Kalter Schweiß rann mir den Rücken hinunter, meine
Hände begannen zu zittern.

»Damit können wir die Waffenkammer nicht aufsprengen«, erwiderte
Roberts. »Es stehen Pulverfässer darin. Ein einziger Funke würde
ausreichen, um sie zu entzünden, und die darauffolgende Explosion
könnte das ganze Schiff in den Æther jagen.«

»Ich weiß!«, schnappte ich.

\bigpar

Meine Knie gaben nach, die Bilder der Vergangenheit schlugen über
mir zusammen. Die Belagerung von Chakcharan, ein monatelanges
Ausharren im Niemandsland, Seuchen und Hunger dezimierten uns
schneller, als ein Feind es hätte tun können. Endlich ein Auftrag
für meine Abteilung, wir sollten die Wende herbeiführen, indem wir
durch einen Tunnel zum Munitionsdepot vordrangen, es leer räumten
und unsere Gegner mit den eigenen Waffen angriffen. Als der Tunnel
gegraben war, hätte es nur noch eine Routineaufgabe sein sollen,
denn wir waren ausgebildete Sprengstoffexperten. Wir erreichten das
Depot, legten eine Lunte. Aber meine Hand zitterte, und ich nahm
zuviel Pulver. Nur etwas zuviel. Und die Lunte war ein wenig zu
kurz.

Ich umklammerte meinen Kopf mit den Händen, als ich wimmernd
zusammenbrach. Die Explosion gellte in meinen Ohren, die Schreie
echoten durch meinen Kopf, und in meine Nase stieg der Geruch von
brennendem Fleisch.

Von weither nahm ich die Stimme von Kapitän Roberts wahr, »Doktor
Travis, kommen Sie – der Mann hat einen Anfall!«

»Ich kann nichts für ihn tun, wir haben kein Morphium. Rasch,
halten Sie ihn fest, damit er sich nicht verletzt!«

Die Finger, die mich hielten, sahen aus wie weiße Maden. Sie
krochen über meinen ganzen Körper. Ich versuchte zu schreien, aber
ich bekam keine Luft.

Dann hörte ich Gérard, dicht an meiner Seite. »Mon ami! Wir
brauchen Ihre Hilfe. Wenn es Ihnen gelingt, die Waffenkammer
aufzusprengen, dann sind wir gerettet. Henry, jeder von uns muss
gegen seine eigenen Schatten kämpfen. Sie sind ein Sprengmeister,
ich weiß es aus den Zeitungsberichten. Helfen Sie uns, mon cher.«

Langsam, sehr langsam kämpfte ich mich an die Oberfläche. Ich lag
im Kabelgatt, flach auf dem Rücken und mit einer Uniformjacke unter
dem Kopf. Über mich beugten sich besorgte Gesichter.

»Wenn ich diese Waffenkammer aufsprenge und versehentlich zuviel
Pulver benutze«, sagte ich matt, »dann könnte die \emph{Olifant} in
den Æther fliegen.«

»Sie werden die richtige Menge Pulver wählen«, erklärte Gérard
zuversichtlich. »Sie werden diese Tür öffnen, ohne dass einer von
uns verletzt wird. Was brauchen Sie dafür?«

Ich richtete mich auf. »Eine Lunte«, erwiderte ich, »fest
verdrillt, nicht weniger als zehn Zoll lang.«

Einer der Ætherleute, Barlow, bückte sich und zog einen
Schnürsenkel aus seinem Stiefel. »Wäre dies hier geeignet?«

Ich prüfte die Schnur und nickte.

»Außerdem brauche ich zwei Männer mit Wassereimern«, fuhr ich fort.
»Sie müssen sich bereit halten, rechts und links neben mir. Daran
hätte ich in Chakcharan denken sollen! Sobald die Explosion erfolgt
ist, müssen alle Funken zuverlässig gelöscht werden, und zwar
sofort, noch ehe sich die Tür öffnet.«

Williams und Miles nickten einander zu. »Einen Eimer haben Sie
hier«, stellte Miles fest, »ein weiterer steht neben den
Wasserfässern.«

»Dann also los!«, kommandierte Kapitän Roberts, und wir machten uns
auf den Weg.

\bigpar

Wir verließen das Kabelgatt, schafften die beiden immer noch
bewusstlosen Wachposten hinein und verschlossen die Tür, tasteten
uns langsam hinauf bis zur Waffenkammer. Meine Hände waren feucht,
ich zitterte am ganzen Körper und überließ mich beinahe ganz der
Unterstützung der beiden Offizieren Williams und Miles, die mit
ihren Wassereimern ausgerüstet neben mir herschritten, mit weitaus
mehr Entschlossenheit, als ich sie empfand.

\bigpar

Eine Welle der Erleichterung durchströmte mich, als wir endlich die
Waffenkammer erreichten, unbemerkt und sicher.

Meine Hände schienen zu wissen, was sie zu tun hatten, daher
überließ ich ihnen das Kommando, sah zu, wie sie das Pulver aus der
Kartusche ins Schloss füllten, zwang sie nur innezuhalten, als die
nötige Menge erreicht war. Ich legte die Lunte, hieß die Übrigen
zurücktreten, bis auf Williams und Miles, die sich bereit hielten.
Dann entzündete ich sie, trat selbst nur so weit beiseite, dass ich
den Offizieren nicht den Weg versperrte, und wartete.

Die Lunte brannte langsam und stetig ab, bis sie im Schloss
verschwand. Es gab eine gedämpfte Explosion. Im selben Augenblick
traten die Offiziere nach vorn und übergossen Schloss, Tür und
Boden mit Wasser, bis auch der letzte Funke gewissenhaft gelöscht
war. Das Schloss sprang auf.

»Rasch, bewaffnen Sie sich!«, befahl Roberts, mit vor Aufregung
heiserer Stimme. »Gewiss haben auch die Piraten die Explosion
bemerkt. Wenn wir sie überraschen wollen, bleibt uns nur wenig
Zeit.«

Die kleine Truppe strebte in die Kammer, rüstete sich mit Pistolen,
Äxten und den kurzen Messern der Æthermänner aus.

Gérard fasste meinen Arm. »Mon ami, ich wusste, dass Sie es
schaffen würden«, sagte er leise. »Ich bin sehr stolz, Sie in
diesem Augenblick an meiner Seite zu haben.«

Dann bewaffnete er sich mit zwei Pistolen und reichte mir zwei
weitere.

»Alle Mann mir nach!«, brüllte Roberts. »Stürmt die
Kapitänskajüte!«

Wir folgten ihm, aber die Piraten waren durch die Explosion
aufgeschreckt und stellten sich uns entgegen. Ich feuerte meine
Pistolen ab, drehte sie dann um und benutzte sie als Schlagstöcke,
bahnte mir einen Weg bis zu einem Niedergang, in dessen Schutz ich
nachladen konnte. Einer von Hanways Æthermännern stürzte sich mit
einem wilden Schrei auf mich, ich tauchte unter den Stufen hindurch
und schoss ihm ins Bein, sodass er schwer auf die Planken stürzte.
Einige der nachfolgenden Ætherleute, von Roberts’ Mannschaft zu den
Piraten desertiert, erfassten die veränderte Situation und
wechselten die Seiten, indem sie uns zur Hilfe eilten und auf
diejenigen losgingen, mit denen sie eben noch Kumpanei geschlossen
hatten.

In dem wilden Getümmel erblickte ich Gérard, der gegen Barley
focht. Hinter ihm näherte sich Evans und legte auf ihn an. Ich
stieß einen Warnschrei aus und stürzte mich auf den Ersten
Offizier. Miteinander ringend, gingen wir zu Boden. Ich kam zuerst
wieder auf die Füße und schlug ihn mit der umgedrehten Pistole
nieder.

Schon war Gérard an meiner Seite. »Zur Kapitänskajüte!«, rief er.
»Wir müssen Madeleine beistehen!«

Wir bahnten uns einen Weg durch die Kämpfenden, erreichten die Tür
und rissen sie auf.

Drinnen stand Madeleine, hoch aufgerichtet, den Sternenglobus in
der Hand.

»Wo ist Anstis?«, fragte Gérard hastig.

Sie wies auf einen zusammengesunkenen Haufen zu ihren Füßen, halb
von dem Schreibtisch verdeckt, ließ dann den Sternenglobus sinken
und brach in Tränen aus.

»Wir haben die Explosion gehört«, stammelte sie, »und als Anstis
hinauslaufen wollte, habe ich \ldots{} Ich wollte ihn nur aufhalten!«

»Gut gemacht!«, lobte Gérard und legte beruhigend den Arm um sie.
»Er lebt, er ist nur bewusstlos. Sie sind bemerkenswert tapfer,
Madeleine. Ihr Mann hat großes Glück, gerade mit Ihnen verheiratet
zu sein.«

Er verließ die Kajüte und rief mit Donnerstimme: »Kapitän Anstis
ist in unserer Gewalt!«

Das brachte die endgültige Entscheidung. Die Ætherpiraten, ihres
Anführers beraubt, ergaben sich und ließen sich ins Kabelgatt
abführen.

\bigpar

Am Tag darauf half uns die Mannschaft der \emph{Olifant} bei der
Reparatur unseres Heißluft-Ætherschiffes, und nachdem wir uns
verabschiedet hatten, konnten wir uns auf den Rückweg zur Erde
machen.

Gérard setzte sich in den Pilotensessel, ich nahm wie
selbstverständlich auf dem des Copiloten Platz. Mit dem Fingernagel
überprüfte ich die Funktionsfähigkeit der Geräte.

»Wasserstand und Kohlevorrat?«, fragte Gérard.

»Ausreichend!«

»Druck?«

»15 Pascal!«

»Dann, mon cher, lassen Sie uns nach Hause fliegen.«

Wir holten die Leinen ein und lösten uns von der \emph{Olifant}.

»Gérard«, fragte ich, »wie konnten Sie die Ætherpiraten so schnell
durchschauen? Woher wussten Sie, dass etwas nicht stimmte?«

Er lächelte versonnen und zupfte an seiner Unterlippe.

»Madeleines Augen«, sagte er dann. »Als Hanway uns in die
Kapitänskajüte führte und ich den Ausdruck ihrer Augen sah, wurden
mir die Zusammenhänge klar.«

Ich begann zu lachen. Ich konnte nicht anders, ein Kichern stieg in
meiner Kehle auf, wurde lauter und immer heftiger, bis ich mich vor
Lachen schüttelte.

Befremdet sah Gérard mir zu.

Endlich japste ich: »So seid ihr Franzosen: Ihr denkt eben immer
nur an Frauen!«

Jetzt stimmte Gérard in mein Gelächter ein. »Sie haben vollkommen
Recht, mon ami! Cherchez la femme!«

\end{document}
