\usepackage[ngerman]{babel}
\usepackage[T1]{fontenc}
\hyphenation{wa-rum}


%\setlength{\emergencystretch}{1ex}

\renewcommand*{\tb}{\begin{center}* \quad * \quad *\end{center}}

\newcommand\bigpar\medskip
\newcommand\gedanke\textit

\begin{document}
\raggedbottom
\begin{center}
\textbf{\huge\textsf{Gedanken an Schmetterlinge}}

\bigskip
Thomas Wüstemann
\end{center}

\bigskip

\begin{flushleft}
Dieser Text wurde erstmals veröffentlicht in:
\begin{center}
Die Steampunk-Chroniken\\
Band I -- Æthergarn
\end{center}

\bigskip

Der ganze Band steht unter einer
\href{http://creativecommons.org/licenses/by-nc-nd/2.0/de/}{Creative-Commons-Lizenz.} \\
(CC BY-NC-ND)

\bigskip

Spenden werden auf der
\href{http://steampunk-chroniken.de/download}{Downloadseite}
des Projekts gerne entgegen genommen.

\vfill

Geboren 1982 in Rostock und dort Kindheit und Jugend verlebt.
Angefangen mit Animation, dann sehr aktiv in der regionalen
Indie-Filmszene Mecklenburg Vorpommerns, wo ich lange Zeit Kurz-
und Experimentalfilme gedreht habe, um mich dann langsam
kommerziellen Projekten zuzuwenden.

\bigpar

2005 ging ich nach Berlin (das ich heute als meine Heimat ansehe)
und gründete hier mein eigenes Unternehmen »Morphium Film«, in dem
ich als Autor, Regisseur und Produzent auftrete. Konzentrierten
sich die Produktionen anfangs hauptsächlich auf Werbe- und
Imagefilme, ist das Themenspektrum inzwischen breit gefächert.

Ein starker Fokus liegt aber auf dem Thema Gaming, mit dem ich mich
in diversen kommerziellen und nichtkommerziellen Formaten
beschäftigt habe; das bekannteste wohl die inzwischen drei Staffeln
umfassende Webserie »Bubble Universe«. Hier soll die Reise auch
noch weiter gehen.

\end{flushleft}
\section{I. Die Drachenbändiger}

Mateo versuchte, wann immer er es einrichten konnte, ein Auge auf
den Jungen zu haben. Zwar war er die meiste Zeit damit beschäftigt
die Maschine in Betrieb zu halten, doch nutzte er jeden Gang zu den
Mechanikern um nach ihm zu sehen; und wenn Fredo, der Vorarbeiter
der Drachenbändiger, nach ihm verlangte, dann ließ er ihn gern ein
oder zwei Minuten länger warten, und unterhielt sich stattdessen
kurz mit dem Jungen. Sollte Fredo ihm beim Eintreten einen
missbilligenden Blick ob der Verspätung zuwerfen, dann brummte
Mateo nur in seinen Bart und sagte: »Auch du, Fredo, solltest
Interesse daran haben, dass es dem Kleinen gut geht. In deiner Haut
möchte ich nicht stecken, wenn ihm etwas zustößt. Seine Mutter
würde dich statt ihrer Ochsen vor den Karren spannen und lachend
dabei zusehen, wie du übers Feld jagst.«

Fredo wurde dann tatsächlich sehr ernst, stierte nachdenklich auf
die Berechnungen und Tabellen, mit denen sein Schreibtisch geradezu
tapeziert war, stemmte schließlich die Handflächen unter das Kinn,
und sagte mit fast verzweifelter, hilfesuchende Stimme: »Nun, dann
pass' gut auf ihn auf, Mateo. Unter den Drachenbändigern hat noch
keiner den anderen im Stich gelassen. Wir sind füreinander da, und
dir obliegt nun die Obhut des Jungen.«

\bigpar

Eigentlich hatte Mateo Emilio nicht aufnehmen wollen. Er war
fünfundfünfzig Jahre und hatte nicht umsonst ein kinderloses Leben
geführt. Er sah sich selbst als alten Griesgram und Eigenbrötler,
auch wenn seine Kollegen sehr wohl sein gutes Herz schätzten; als
niemand jedenfalls, dem man die Verantwortung für einen
Fünfzehnjährigen übertrug. Aber Martha, die Mutter des Jungen,
hatte noch nie viel von Argumenten gehalten. Bevor sie sich darauf
verließ, ihren Gegenüber im Dialog zu überzeugen, brachte sie die
ganze Wucht ihrer Person aufs Tapet – immerhin war sie die
Personifikation der öffentlichen Meinung im Dorf –, und so lenkte
man ein, bevor man sein Gesicht vor den anderen Dorfbewohnern
verlor.

»Mateo«, hatte sie ihm gesagt. »Ich sehe eine Zukunft für Emilio,
und es ist nicht die eines Viehhirten. Er ist zu verträumt, um als
Bauer zu enden. Nimm ihn mit, zeig ihm den Himmel. Und bring ihn
nicht wieder, ehe er nicht einer der Drachenbändiger geworden ist.«
Mateo wollte noch sagen, dass sein Beruf ein gefährlicher war und
er keineswegs den Tod des Jungen auf seinen Schultern brauchte,
aber im mexikanischen Volk war nach Jahren des Krieges endlich so
etwas wie Entspannung eingetreten. Gedanken an weitere Schlachten
fegte man hinfort. Nur von Ruhm und Ehre war zu hören. Er hatte
nichts entgegen zu setzen. Also hatte er den Jungen mitgenommen.

\bigpar

Er hatte Emilio zu den Weichenstellern gesteckt, jenen schrulligen
Ingenieuren, die über große Ventile den Verlauf des Dampfes durch
das Schiff lenkten. Dort, so dachte er sich, war der Junge mit
seinen naiven Spinnereien gut aufgehoben; sollte er doch
tagträumend durch die Gänge streifen. Außerdem konnte er so am
meisten über die Schifffahrt lernen, und blieb dem Drachen selbst
so fern wie möglich. Emilio hatte schnell Freude daran gefunden,
mit den feinmechanischen Apparaturen in die über das Schiff
verteilten Rohre zu spähen. Den Armreif mit dem Hitzedetektor trug
er wie einen königlichen Schmuck. Und wenn er bei seinen
Inspektionstouren durch das untere Deck zufällig einem Hoheitlichen
über den Weg lief, dann verbeugte er sich mit Anstand; sein Blick
aber ging nie so tief zu Boden, als dass er die Leitungen aus den
Augen verloren hätte.

\bigpar

Mateo selbst fütterte den Drachen. Es war ein respektabler aber
schmutziger Beruf, und er war stolz, ihn auf diesem besonderen
Schiff verrichten zu können. Die \textit{Acalli} war das erste Schiff, das
in Auftrag und alleiniger Ausführung der neu entstandenen Republik
gebaut worden war, und es stand ganz im Zeichen der Geschichte.
Mateo war ein \textit{Nahua}, ein Nachfahr der großen Azteken. So blickte er
mit dem Gedanken an seine Ahnen auf den Schlund des Drachen, jedes
Mal wenn er eine Schaufel Kohle ins Feuer beförderte: die metallene
Verkleidung der Maschine war verziert mit Ideogrammen aus der alten
Zeit, einer Darstellung von \textit{Mictlan}, der Unterwelt, und von
\textit{Mictlantecuhtli}, dem Herrscher über die Toten. Dies war die Art der
Konstrukteure, die Hierarchien an Bord zu versinnbildlichen – hier
unten herrschten der Tod und das Feuer. Oben an Deck, an der Spitze
des Schiffes, thronte \textit{Ehecatl}, der Herr des Windes, und lenkte sie
in der Kälte der Stratosphäre.

Mateo fühlte sich geborgen am Rande des Todes. Wenn die Spanier
kamen, dachte er sich jedes Mal wenn er auf die Ideogramme auf dem
Drachen blickte, dann war es an ihm – als Hüter von Mictlan – zu
bestimmen, wer in die Unterwelt kam, und wer verdammt wurde als
leere Hülle weiter durch die Lüfte zu driften.\\ Hier am Herzen der
Maschine hatte Mateo außerdem eine wahrhaft physische Erfahrung der
Vorgänge an Bord. Was auch auf den oberen Decks geschah,
transportierte sich durch die zylindrische Gestalt des Schiffes
direkt zu ihm und artikulierte sich in den unterschiedlichen
Vibrationen, die fast eine eigene Sprache – seine Sprache –
bildeten. Einmal hatte Emilio ihn gefragt, ob sie hier unten merken
würden, wenn die Spanier kämen. Er meinte damit, ob der Tod schnell
käme, aber Mateo glaubte nicht, dass die Chancen so schlecht
verteilt waren und überging den Unterton. So hatte er nur die Hand
an den nächstgelegenen Stahlpfeiler gelegt und gesagt: »Wenn sie so
schnurrt wie jetzt, dann sind wir auf geradem Kurs über den Ozean.
Wenn wir wenden oder Fahrt aufnehmen, dann merkst du es. Der Drache
spuckt dann Feuer und lässt es dich direkt in den Beinen spüren.
Hast du Angst, weiche Knie vor dem Feind zu bekommen? Nun, das
wirst du zwangsläufig. Die Maschine macht so einen Dampf, dass du
Probleme haben wirst auf den Füßen zu bleiben!«

\bigpar

Emilio war danach für eine Weile auf dem kleinen Balkon
verschwunden, der ihren Teil des Luftschiffs umrundete.
Nachdenklich starrte der Junge in die Ferne und ließ sich den Blick
nicht von der im Wind flatternden Folie trüben, die ihn von der
eisigen Luft der Stratosphäre trennte. Minus siebzig Grad Celsius
herrschten jetzt dort draußen und bald schon würde es noch
schlimmer werden. Wenn die Spanier kämen, würden sie aufsteigen in
die Weiten des Alls. Hier unten regierte der Zivilverkehr, und
keine Nation konnte es sich erlauben dieses Gesetz zu brechen oder
Unschuldige in ihre Kämpfe zu ziehen.

Dann aber würden sie auf Temperaturen stoßen, die einen Menschen in
Sekunden töteten – eine Arena der Kälte für den Schaukampf zwischen
den Nationen. Und doch, dachte Emilio, vertrauten sie ihr Glück
ganz diesem Ungeheuer an. Dem Drachen, der einem die Beine vom
Boden riss, wie Mateo fast mit Bewunderung konstatiert hatte. Ein
kleiner Riss im Körper des Schiffes – wie ihn eine Waffe selbst
primitiver Machart leicht verursachen konnte – dann war es vorbei.
Die Kälte strömte ins Innere, der Luftdruck zerriss die Wände, der
Fahrtwind fegte alles was noch lebte von Bord, wo es dann in
Eiskristallen zerstob.

Als Emilio den Blick nach unten senkte – er presste das Gesicht so
weit in die Folie und über die Reling, bis ihm das Atmen schwer
fiel – entdeckte er einen Vogel, der kaum einen Kilometer unter dem
Schiff und weit über den letzten Wolken die Schwingen ausbreitete.
Das Tier schien die letzten verträglichen Luftschichten unter
seinen Flügeln zu sammeln und wie auf einem Kissen zu thronen. Dann
stürzte es plötzlich nach unten (fast konnte Emilio den Tunnel
sehen, den es sich in die Atmosphäre bohrte), und durchstieß die
Wolkenwand, um sich endlich in lebendigere Regionen zu
verabschieden.

\gedanke{Der Vogel kann Reißaus vor dem Drachen nehmen}, dachte Emilio.
\gedanke{Ich muss gemeinsam mit dem Ungeheuer zurück zum Boden. Oder mit ihm
sterben.}

\section{II. Emilio}

Dank seiner schnellen Auffassungsgabe konnte Emilio bald nach
seinem Antritt an Bord allein durch die Gänge streifen. Wie alle
Gewerke hatten die Weichensteller einen eng abgesteckten
Arbeitsbereich, der auch ohne große Vorkenntnisse zu beherrschen
war, gesetzt den Fall man verstand sich auf die Bedienung der
entsprechenden Gerätschaften.

Neben dem Wärmearmreif, der mittels leicht erhitzbarer Chemikalien
die Richtung weisen konnte, hatte Emilio ein Okular, das ihm einen
Infrarotblick auf die Rohre ermöglichte. So schlich er denn mit
seinem besonderen Blick auf die Welt durch die Gänge: links ein
Dampfstrahl, den er zur nächsten Weiche verfolgen konnte, von dort
mehrere Verzweigungen, die über drei Etagen hinweg den runden
Basiskörper des Schiffs formten. Hier unten im Wartungsdeck waren
die Rohre frei und kreuzten sich offenen Blicks in den Wegen. Ein
Reparatureinsatz konnte schnell zur Kletterpartie werden. Zwar war
auch hier der ästhetische Geist der Bauherren zu spüren – die Rohre
waren häufig in Goldtönen verkleidet und an Wegkreuzungen mit
Ornamenten und Ideogrammen verziert – in den oberen Etagen aber
integrierten sie sich elegant in den Schick der Räumlichkeiten. Da
die Rohre auch zum Heizen genutzt wurden, lagen sie zwangsläufig in
den Zimmern der Hoheiten frei, bildeten aber im Zusammenspiel mit
vorgetäuschten Kaminen ein Bild, das vom opulenten Lebensstil der
Spanier nicht mehr zu unterscheiden war.

Nach ihrem langen Weg durch den Bauch des Schiffs erreichten die
Rohre schließlich die bauschige Oberfläche, auf der an einzelnen
Stationen Zelte aus Folie aufragten, um einen einfachen Ausblick zu
ermöglichen, und entließen dort ihren Dampf in die gespannten
Segel, die das Schiff Richtung Europa trugen.

\bigpar

Emilio kannte als einer der wenigen Drachenbändiger all diese Orte.
Oft hatten die Rohre Verstopfungen oder Lecks, und auch wenn ein
paar funktionsunfähige Dampfwege leicht kompensiert werden konnten,
ohne dass das Schiff aus der Bahn geriet, waren, wenn es in die
Schlacht ging, präzise Manöver vonnöten, und Emilio sorgte dafür,
dass der Fluss des Dampfes durch das Schiff stets frei war.

Er sah sich nicht als einen der Bändiger. Seine Hände waren nicht
schmutzig, und durch den freien Zugang zu den oberen Decks lag sein
Bewegungsradius deutlich über dem eines Arbeiters. Es hatte sich
nicht viel geändert im Vergleich zu seinem Leben am Boden. Er
konnte noch immer den Träumer spielen, der sich zwar für einen
Viehhirten hielt, seine Zeit aber lieber damit verbrachte,
Schmetterlingen auf der Wiese nachzujagen.

\gedanke{Wenn die rotglühenden Dampfzirkel, die durch die Rohre schießen,
Schmetterlinge wären}, dachte Emilio heute, \gedanke{dann wäre das Okular
mein Schmetterlingsnetz, mit dem ich sie in meinem Kopf
konserviere. Ich darf nur den Gedanken an sie nicht verlieren.}

\bigpar

Links vor sich entdeckte er endlich das Leck, nach dem er gesucht
hatte. Es war nicht groß, sonst hätte es ganze Gänge eingenebelt
und viel früher Alarm ausgelöst. Aber solche Lecks konnten, wenn
der Spanier kam, über die Wendefähigkeit des Schiffs, und damit
über Leben und Tod entscheiden. Kleine Fäden aus Dampf schossen in
Emilios Blickfeld, lenkten das Licht in tausend Richtungen und
machten diesen Teil des Schiffs zu einem Zerrbild aus verschiedenen
Perspektiven.

Emilio schloss die Weiche am Anfang des Korridors, und spürte, als
der Dampf versiegte, sofort eine leichte Neigung des Schiffes, die
sich schnell stabilisierte. Während er das Rohr mit
selbstwachsendem halborganischem Metall bearbeitete, dachte er
wieder über den unausweichlichen Angriff nach. Ein Luftschiff hatte
die Eigenschaft, jeden äußerlichen Einfluss direkt auf seine
Insassen zu übertragen. Kälte, die durch die Schiffswand in alle
Poren drang, das leichte Schwanken im Wind oder bei einer
Kursänderung, das sich direkt auf das Gleichgewicht übertrug (wie
sollte man in brenzligen Situationen die Nerven behalten, wenn man
unkontrolliert mit dem Schiff pendelte!); und ja: sicher übertrugen
sich Geschütze auf die Mannschaft. Und sie brachten nicht Schwindel
oder Kälte, sondern den Tod!

\bigpar

Emilio fühlte sich sehr verloren. In seinem alten Leben, da hatte
er die Dinge beeinflussen können. Wenn ihm eine Pflanze einzugehen
drohte, dann gab er mehr Wasser. Wenn ihm ein Pflug kaputtging,
dann reparierte er ihn mit allem, was eben greifbar war – zumeist
ein starkes Stück Seil. Hier war er den Geschehnissen ausgeliefert.
Er hatte die Kontrolle verloren.

Mit einem lauten Seufzen betätigte Emilio das Ventil und brachte
die Weiche wieder in Gang. Langsam nahm das Rot in seinem Sichtfeld
zu: die Hitze kroch zurück ins Rohr. Der Fluss des Dampfes war
wieder frei.

\section{III. Die Hoheiten}

»Der Junge hat kaum noch Zeit mich zu besuchen. Springt nur noch
auf den oberen Decks rum.«, sagte Mateo so laut, dass Emilio ihn
hören musste, als die Bändiger abends in ihrem Quartier zusammen
saßen. Emilio lag am Rande auf seiner Koje – wie so oft blieb er
den Geselligkeiten fern – und war vertieft in ein Handbuch.

Mit einem Lachen, das in einem Röcheln endete, rettete Arlo – ein
fähiger Mechaniker, wenn er nicht betrunken war – einen Becher
Mezcal aus seinen zittrigen Händen auf den Tisch. »Mateo, du wühlst
den ganzen Tag im Dreck. Zieh da nicht den Jungen mit rein; lass
ihm seine sauberen Hände.«

»Er ist eben von der zarten Art, deshalb spielt er auch den ganzen
Tag mit Zahnrädern!«, schickte Liron hinterher, der es im
Mezcal-Pegel scheinbar mit Arlo aufnehmen wollte.

\bigpar

Emilio war das Gespräch sichtlich unangenehm - tatsächlich las er
soeben einen Abschnitt über feinmechanische Zahnradgetriebe.
Verlegen klappte er das Buch zusammen, und schlich zum Tisch. Arlo
begleitete den Trab mit einem fletschenden Grinsen und fegte
Aschereste von dem klebrigen Eck des Tisches, an dem Emilio Platz
nahm. »Sei bloß froh«, sagte er zu dem Jungen indem er ihm
einschenkte, »dass du nicht bei den Hoheiten gelandet bist. Dort
geht es nicht so fröhlich zu. Ist leider so bei Menschen, die nur
mit dem Kopf arbeiten.«

Vorsichtig nippte Emilio an seinem Glas, nicht ohne einen
Seitenblick auf seinen Vormund zu werfen. Der aber sah in seinen
Obhutspflichten scheinbar kein Alkoholverbot für einen
Minderjährigen und war mit den Gedanken sowieso nicht am Tisch.
Nachdenklich ließ Mateo die Unterlippe am Glasrand kleben, nahm
dann einen beherzten Schluck und sagte:

»Macht es euch Sorgen, dass wir keine Meldung von oben bekommen?
Seit Tagen sind wir auf demselben Kurs. Ist fast so, als wollten
sie warten, bis die Spanier unseren Weg kreuzen. Warum gehen sie
nicht in die Offensive?«

Arlo blickte auf, sichtlich verärgert, dass das Gespräch einen
ernsten Weg einschlug: »Erstmal über den Ozean kommen. Dann können
wir ihnen immer noch Dampf machen.«

»Aber das wissen die Spanier auch.« Unruhig balancierte Mateo sein
Glas zwischen den Händen. »Und sie werden uns abfangen, bevor wir
über dem Kontinent sind.«

»Vielleicht haben sie von unserem Abflug noch gar nicht Wind
bekommen. Wir haben die Spanier verjagt, also haben wir auch ihren
Geheimdienst verjagt. Vielleicht sind sie völlig ahnungs-los.«

Da lachte selbst Liron hustend in sein Glas. Verschreckt schaute
Emilio zwischen den Alten hin und her. Sowieso fühlte er sich
unwohl in der Runde. Jetzt, da das Gespräch einen Verlauf nahm, der
ihm immer mehr die eigene Ohnmacht gegenüber der Zukunft vor Augen
führte, wäre er am liebsten aus dem Raum gestürmt. Sicher wusste er
um die Ereignisse der letzten Zeit, und wie es um ihre Nation und
ihre Mission bestellt war. Aber kam es zu Geheimdiensten, Taktik –
da war er der fünfzehnjährige Junge, der in seinem Dorf alles
glaubte, was ihm das Radio erzählte.

Er kannte nur das Hohelied auf Agustin de Iturbide, der dem
Freiheitskampf die Wende gebracht hatte und nun oberste Hoheit auf
der \textit{Acalli} war. Iturbide war zunächst Befehlshaber der
neuspanischen Armee gewesen und damit Herr über den technischen
Fortschritt der Eroberer. Die mexikanischen Kämpfer hatten ihm
nichts entgegenzusetzen, mit Schwertern und Pistolen standen sie
der geballten Macht der Dampfmaschine gegenüber. Luftschiffe und
gepanzerte Schienenwagen mähten eine blutige Schneise in die Reihen
der Mexikaner. Aber dann, so lobpreiste das Hohelied der
Radiosender, gewann der mexikanische Geist: Iturbide wechselte die
Reihen. Und mit ihm kam auch der technische Fortschritt zu den
Unabhängigkeitskämpfern.

Nur kurz nachdem die Spanier den Rückzug angetreten hatten und die
Republik offiziell ausgerufen war, wurde mit den Arbeiten an der
\textit{Acalli} begonnen. Die Pläne für das Luftschiff hatte Iturbide mit
einem Stapel weiterer Papiere entwendet, als er zu den
Revolutionstruppen überlief. Und nun segelten sie den Spaniern
entgegen, um ein für alle Mal klare Fronten zu schaffen, und die
Bedrohung über dem Ozean zu beseitigen.

Im Geiste blendete Emilio das Gespräch der alten Männer aus und
schaute zur Decke. Wenn er sich ganz stark konzentrierte, vermeinte
er die Fußtritte der Hoheiten zu hören. Wie sie unruhig in ihren
Zimmern hin und her tigerten, und sich auch keinen Rat wussten, als
weiter über den Himmel zu segeln bis die Spanier kamen. Irgendwo
dort war auch Iturbide und verfluchte wohl seinen Wagemut. Er
mochte der Held einer Nation sein, aber hier, als Kommandant der
Angriffsarmee, war er dem Verlauf der Ereignisse so ausgeliefert
wie jeder an Bord. Ohne seine Mannschaft war er verloren. So hatten
die Drachenbändiger am Ende vielleicht mehr Kontrolle als die
oberste Hoheit selbst. Wenigstens konnten sie das Schiff steuern
und notfalls die Flucht einleiten.

\bigpar

»Vermutlich sind die Hoheiten sehr einsam.« Bevor er noch merkte,
dass er laut gesprochen hatte, wurde Emilio harsch von Mateo
angefahren.

»Du magst das Gefühl haben, dass wir keine Führung haben, weil wir
hier unten wenig hören. Das verstehe ich. Aber Emilio, sei
vorsichtig, nicht zu hochnäsig zu werden. Wenn es ernst wird, sind
wir ganz auf die Entscheidungen der Hoheiten angewiesen!«

Damit hatte Mateo seinen Schlusspunkt für den heutigen Abend
gesetzt. Mürrisch leerte er sein Glas und verließ lautstark, indem
er keine Anstalten machte die Füße des Stuhls beim Zurücksetzen zu
heben, den Tisch. Arlo und Liron stierten ins Leere, ob aus
Trunkenheit oder aus Angst vor den von Mateo gehegten
Befürchtungen, wusste Emilio nicht. Jedenfalls sah er endlich eine
Möglichkeit zur Flucht und verzog sich zurück auf seine Koje.
Analysen, dachte er sich, mochten das unerträgliche Warten
vereinfachen. Beeinflussen konnten sie nichts.

Kurz bevor er über seinem Buch einzuschlafen drohte, warf er
nochmals einen Blick zu der Runde am Tisch. Die beiden dort
Verbliebenen schienen immer schwerere Köpfe zu bekommen, ließen den
Schopf tief ins Glas hängen. Flüsternd aber ging die Unterhaltung
weiter. Emilio spitzte die Ohren und ließ sich von den Bruchstücken
des Gesprächs, die er aufschnappen konnte, einlullen, und sich so
in die Dämmerung des Schlafes führen.

»Die \textit{Acalli}«, hörte er Arlo sagen »ist ein stolzes Schiff. Und,
egal was bei der ganzen Sache `rumkommt: stolz sollten wir
untergehen.«

\section{IV. Drachen am Himmel}

Als die Vibrationen durch das Schiff gingen, war Emilio auf
Wartungstour. Es riss ihn nicht von den Füßen, aber er spürte die
Angst im ganzen Körper. Da war es: das Wendemanöver.

Er war tief im Inneren des Schiffskörpers und so brauchte er eine
Weile, bis er den nächsten Balkon erreichte. Während er über Rohre
hechtete, pflanzte sich das Dröhnen des Drachens über die Wände und
Böden bis in seine Fingerspitzen fort. Er stellte sich Mateo vor,
der jetzt mehr denn je mit dem großen Drachen rang; mit jeder
Schaufel Kohle, die er in den Schlund der Bestie warf, auch sein
eigenes und Emilios Schicksal mehr in die Hände des Schiffes
legte.

Dann war er draußen. Zunächst sah er nur den dünnen Streifen
Abendrot, der sich tief unter ihm, in den letzten Wolken,
tausendfach widerspiegelte und sich in getönte Watte verwandelte.
Jetzt war auch das Rund des Planetenkörpers zu sehen. Sie verließen
die letzten Atmosphäreschichten. Und wie sie höher stiegen und die
Brechung des Lichts abnahm, ließ auch die strahlende Kraft des
Abendrots nach. Hier oben war es nur kalt und von einer Helligkeit,
die ein seltsam nüchternes Bild des Bevorstehenden zeichnete.

Dann schärfte sich sein Blick – was er für einen Fleck auf der
Folie gehalten hatte, war der winzige Bug eines Schiffs, das
schnell näher kam. Es sah überhaupt nicht nach Gefahr aus.

Langsam verschwanden die Spanier wieder aus dem Sichtfeld. Die
\textit{Acalli} machte eine scharfe Drehung nach links, um nicht mit voller
Breitseite im Schussfeld der Angreifer zu stehen. Sie wollten sie
kommen lassen. Das war die einzige Chance des mexikanischen Schiffs
– der Nahkampf.

Hinter Emilio wurde das Getrampel lauter. Das ganze Schiff war in
Aufruhr. Jeder erledigte seine Pflichten doppelt, nur um etwas zu
tun zu haben.

\bigpar

Jetzt erinnerte er sich auch seiner eigenen Aufgaben. Es war an
ihm, das Schiff kampftauglich zu halten; den Dampf zu lenken, wenn
Leitungen im Kampf beschädigt wurden. Aber noch war Zeit.

Als Emilio den Heizsaal betrat, hatte er das plötzliche Gefühl,
mitten in der Schlacht zu sein. Mateo war in Kohlestaub eingehüllt
und wirbelte in wilder Raserei durch den Raum. Er wurde jetzt von
zwei Heizern unterstützt, deren Gesichter Emilio kaum ausmachen
konnte: nur ein schmaler Streifen um die Augen und auf der Stirn
war vom Schweiß frei gewaschen, der Rest des Körpers war schwarz
und zeichnete sie als namenlose Höllendiener.

»Emilio! Mach dass du fortkommst. Die Republik braucht jetzt jeden
Mann auf seinem Posten!« Mateo schaute aus wilden Augen auf ihn. Er
schien vergessen zu haben, dass er nicht mit einem trainierten
Mitglied der Flotte sprach, sondern mit einem Jungen, der einen
Monat zuvor noch mit Kühen statt mit Drachen gekämpft hatte.

»Mateo..« Er wollte ihm sagen, welche Angst er hatte. Er wollte
Trost und Mut spendende Worte, und schließlich wollte er ihm wohl
Lebewohl sagen. In diesem Moment aber erschütterte der erste
Treffer das Schiff.

\bigpar

Ein Beben ging durch die Balken und Rohre, in einer Ecke des Saals
schoss Dampf aus einem winzigen Leck. Kohlestaub wurde
aufgewirbelt, verharrte in der zunehmenden Schwerelosigkeit in der
Luft und umtanzte die Heizer in dämonischen Figuren, bevor einer
von ihnen die Geistesgegenwart besaß, die Weiche zu schließen.

Noch einmal drehte sich Mateo zu ihm um: »Es ist nicht die Zeit,
Emilio. Versuche tapfer zu sein und wenn alles überstanden ist,
dann stimmen wir ein Siegeslied an und fliegen nach Hause.«

Unschlüssig stand Emilio inmitten des Chaos und wusste nicht ob er
gehen oder auf den Trost seines Obhuts bestehen sollte. »Wenn alles
überstanden ist – und falls wir uns wiedersehen.«

Ein weiterer Einschlag ließ das Schiff schwanken. Niemand kümmerte
sich mehr um ihn. Als er aber auf wackeligen Beinen einen
Entschluss zu fassen versuchte, wurde er von hinten gepackt.

»Wusst' ich doch wo ich dich finde!« Er wurde herumgewirbelt und
starrte in die geröteten Augen von Cuarto, dem Vorsteher der
Weichensteller. Sein Okular war ihm von der Nase gerutscht und hing
ihm nun wie eine eigenartige Waffe unter dem Kinn. »Wir brauchen
deine Hilfe!«

Ohne ein weiteres Wort wurde Emilio aus dem Saal und durch die
Eingeweide des Schiffes geschleift. Er ließ sich treiben. Cuartos
Hand hielt ihn fest gepackt – alles, was Emilio in diesem Moment
leisten musste, war, seine fahrigen Schritte, die in der geringen
Schwerkraft jetzt fast im Leeren ruderten, dem Tempo anzupassen und
nicht vor Angst durchzudrehen. Die Wände des Schiffes waren nun
nicht mehr die schützende Hülle vor der menschenfeindlichen
Außenwelt des Alls, sondern lediglich ein Hindernis zwischen zwei
Schauplätzen. Gekämpft wurde innen wie außen. Hier, im Bauch des
Schiffes, taten die Drachenbändiger ihr Bestes, alles zu reparieren
und einsatzfähig zu halten. Überall sah Emilio kleine Dampffontänen
und inmitten der Wirbel und Schwaden gesichtslose Menschen, die
gegen die Technik kämpften. Und wann immer Cuarto ihn an einem der
kleinen Balkonfenster vorüber zog, sah er vor dem dunklen
Hintergrund des Sternenhimmels auch dort draußen Fragmente von
Kämpfern. Die Gesichter waren hinter Apparaten verborgen, die sie
mehr dem Schiff anglichen, als sie noch menschlich erscheinen zu
lassen. Sie hingen an langen Schläuchen und wirkten wenig
selbstbestimmt. Hinter ihnen sah Emilio immer wieder kleinere
Explosionen.

\bigpar

\gedanke{So ist es also wenn Drachen kämpfen}, dachte er.
\gedanke{Man lässt alles Menschliche fallen und wird Teil der Bestie.}

Erst als Cuarto plötzlich stoppte, löste sich Emilio aus dem
Strudel der Eindrücke und blickte auf. Sie standen vor einem großen
Ventil und mit ihnen waren zwei weitere Weichensteller, die
betretene Gesichter machten.

»Dies ist die Hauptader der \textit{Acalli}, die das große Rahsegel
antreibt«, sagte Cuarto. »Sie ist leck. Wir sind ohne Antrieb.«

Jetzt trat – vielmehr: schwebte - einer der Weichensteller vor,
dessen Namen Emilio nicht kannte. Während er sprach, schraubte er
an etwas herum, dessen Zweck nicht sofort ersichtlich war, dessen
Form aber entfernt an das Gesicht eines Pantomimen erinnerte –
ausdruckslos und bleich (es war aus Silber), mit großen
Aussparungen für die Augen. Die Vorstellung, ein solches Gerät auf
dem Kopf zu tragen – denn das, wurde ihm jetzt klar, war der
bestimmte Einsatzort – verursachte in Emilio Übelkeit.

\bigpar

»Das Problem ist, dass wir durch die Kampfhandlungen nicht mehr in
die oberen Schiffsbereiche kommen. Der einzige Zugang ist das Rohr
selbst«, sagte der Weichensteller. »Und nur du bist klein genug, um
hinein zu passen.«

Er hob die Maske. »Setz' das auf.«

Ungläubig nahm Emilio das Gerät an sich. Im Innern sah er kleine
Polster, mit einem wabenartigen Netz überzogen. Als er es berührte
spürte er eine gelartige Substanz. »Sauerstoff«, sagte der
Weichensteller. »Es wird stickig da drinnen.«

Aller Widerstand strömte aus Emilio heraus. Dann sollte es halt so
sein. Er wurde einer der Gesichtslosen, ein Rad im Getriebe der
Drachenbändiger und erfüllte seine Pflicht. Vorbei waren die
Träumereien.

Er setzte die Maske auf und spürte sofort, wie kühle Luft sich über
die Waben auf seine Haut legte. Die Weichensteller öffneten eine
Klappe an dem Ventil und bedeuteten ihm mit ausladenden Gesten
hineinzukriechen.

\gedanke{Von jetzt an}, dachte Emilio noch, 
\gedanke{sehe ich alles aus dem Blickfeld des Drachen.}
Dann war er im Inneren und die Klappe flog zu.

Er wagte sofort den Blick nach oben. Der Schacht des Rohres ging
senkrecht durch das Schiff, so konnte er gut das Ende erkennen, das
sich zum Weltall öffnete. Dort hatte sich automatisch ein
Folienzelt aufgespannt, als das Schiff den Druckabfall registriert
hatte. Zwischen dem Ausgang und dem Boden zeichneten sich diverse
Lichtkegel – überall dort, wo die beständigen Erschütterungen ein
Loch in der Leitung verursacht hatten. Emilio machte sich an den
Aufstieg; er musste nur die Verkleidung des Rohres loslassen und
sich nach oben treiben lassen.

Die ersten Risse hatte er schnell repariert. Sie maßen nur
Millimeter und waren keine Hürde für das schnell wachsende Metall.
Langsam aber begann ihn die allgegenwärtige Stimme des Schiffes
aufzureiben. Durch jedes winzige Loch drangen die Klagen der
Besatzung zu ihm. Ein Wispern aus Angst, Hilflosigkeit und Scham,
der paar armen Seelen, die nichts an der Situation tun konnten, und
doch gewohnt waren, alles im Griff zu haben. Einmal spähte er in
ein dunkles Schlafgemach und sah den schluchzenden Körper eines
Offiziers. Er hörte nichts. In diesen Stunden glich das Weinen der
Stille des Alls.

\bigpar

Auf Höhe des Oberdecks, an einem größeren Leck, sah er ihn:
Iturbide. Es war tatsächlich die Kabine des Kapitäns. Iturbide
selbst war im Gespräch mit einer anderen Hoheit, aber die Maske
hinderte Emilio daran, alles zu verstehen.

»Wir sind manövrierunfähig und zahlenmäßig in der Minderheit«,
sagte eben die Hoheit, die nicht Iturbide war, aber dank der
unsäglichen Perücke fast das gleiche Aussehen hatte. »Und die
Spanier haben noch nicht einmal all ihre Waffen ausgepackt.«

Die Antwort des Kommandanten konnte Emilio nicht ausmachen. Er
sprach leise und sichtlich entspannt. Sofort nahm auch die
Nervosität der anderen Hoheit ab, schien sich aber alsbald in
Fatalismus zu wandeln: »Ich kann es versuchen. Aber alle Männer,
die wir noch aufbringen können, sitzen zitternd in ihren Kojen. Sie
werden die Schlacht nicht wenden.«

Wieder sprach Iturbide, und wieder strahlte er ruhige Souveränität
aus. Ermunternd klopfte er dem anderen auf die Schulter, und
verließ in schwebenden Schritten den Raum. Die Hoheit, deren Maske
aus Schminke bei näherer Betrachtung desolat wirkte, blieb allein
zurück und murmelte: »Durchhalteparolen werden uns nicht retten!«

Emilio arbeitete sich weiter durch den Schacht und war bald am
oberen Ende angelangt. Hier hatte der kurzzeitige Druckausgleich,
bevor die Folie aufgegangen war, immensen Schaden angerichtet. Der
Aufsatz des Rohres auf der Schiffsoberfläche, der den Dampf
richtete und das Segel antrieb, sah aus wie ein Wassertropfen, der
genau in dem Moment, da er auf einen See traf, eingefroren worden
war. Würde man jetzt die Weiche in Gang setzen, würde der Dampf
sich über das gesamte Schiff verteilen und die wenige Bewegung, die
durch die Trägheit noch vorhanden war, gänzlich abbremsen.

Emilio betrachtete das zersplitterte Ende des Schachtes. \textit{Calani}
hieß in der Sprache der \textit{Nahua} die Vibration des Metalls, die sowohl
erschaffen als auch zerstören konnte. Als Drachenbändiger hatte er
beides in seiner Macht. Er entfernte den zersprengten Kranz des
Rohres und begann langsam damit, neue Struktur aufzubauen.

\bigpar

Plötzlich wurde es hektisch um ihn. Überall bauschten sich Folien
auf, Luken öffneten sich und Männer in kupferfarbenen Metallanzügen
stampften zur Oberfläche. Kurz war ein Innehalten der
Kampfhandlungen am Rande des Schiffes zu spüren. Nicht dass die
Spanier die Verstärkung gefürchtet hätten. Das Signal aber, das die
\textit{Acalli} mit diesem Zug an ihre Männer gab, hatte genau den
motivierenden Effekt, der gewünscht war, und verschob kurzzeitig
das Kräfteverhältnis zugunsten der Mexikaner. Dann trat Iturbide
auf.

Nur durch die dünne Folie vor der Kälte des Alls geschützt,
erschien er, wie ihn die Legende versprach: in edler Uniform, mit
stolz gestrecktem Rücken, die Hände lässig hinter dem Körper
verschränkt.

Emilio stand genau zwischen ihm und der Truppe und versuchte so
wenig Aufmerksamkeit wie möglich auf sich zu ziehen, als Iturbide
bedächtig die Hand hob.

»Dieser Kampf«, sprach er mit einer solch eindringlichen Stimme,
dass Emilio kurz vergaß zu arbeiten. »hat nicht einmal richtig
begonnen. Ihr steht Mann gegen Mann. Das heißt: ihr habt es in der
Hand!«

Emilio hatte an dieser Stelle Jubel erwartet. So kannte er es aus
den Radiosendungen. Die Truppe aber war sehr verhalten. Vereinzelt
wurden Fäuste in die Höhe gestreckt, die meisten aber waren damit
beschäftigt, ihre Schutzkleidung anzulegen.

»Die Zukunft der mexikanischen Republik liegt auf diesem Schiff und
kein einzelner Mann wird sie bestimmen! Gemeinsam haltet Stand!
Gemeinsam siegt!«

Die ersten Folienzelte implodierten. Gesichtslose Kämpfer, die über
Schläuche mit dem Schiff verbunden waren, stürmten aus den Kokons
und warfen sich über den Rand des Schiffes. Die leichte
Gravitation, die noch immer vorherrschte, trieb sie nach unten –
ins Zentrum der Schlacht. Emilio vermeinte Blitze zu sehen, als sie
auf die spanischen Truppen trafen, wahrscheinlicher aber entstammte
diese Illusion seinem überlasteten Gehirn, das alles
überdramatisierte.

Vor den letzten Nachzüglern versuchte Iturbide nochmals Gehör zu
finden: »Viele von euch werden sterben. Das ist tragisch. Aber mit
eurem Tod werdet ihr dazu beitragen, dass die Republik überdauert.
Und mit ihr auch die Gedanken an die Männer, die für das Vaterland
gestorben sind – und es gerettet haben!«

Irgendwann sprach Iturbide nur noch zu Emilio. Im All wurden keine
Geräusche übermittelt sobald man die schalltragenden Elemente
verlassen hatte, und die Kämpfer hatten es nicht als wichtig genug
erachtet, weiter ihrem Führer zu lauschen. So senkte sich der Blick
des Kommandanten bei den letzten Worten auf den jungen
Drachenbändiger. Und plötzlich hatte Emilio das Gefühl, dass
Iturbide mehr zu sich selbst als zu seinen Männern gesprochen
hatte. Die Ruhe und Souveränität der Gesten spiegelte sich in den
Augen des Anführers als Leere wieder.

\gedanke{Er weiß auch nicht, was er tun soll}, wurde Emilio bewusst.
\gedanke{Er ist so verloren wie ich.}

Die scheinbare Gefasstheit, das taktische Kalkül, waren nichts als
Hilflosigkeit. Niemand an Bord hatte es als einzelne Person in der
Hand. Geschmeidig wie er gekommen war, fast schwebend in der
Schwerelosigkeit, verschwand Iturbide von der Oberfläche.

Als Emilio zu Boden schaute, stellte er fest, dass die Arbeit getan
war. Das Rohr war intakt und die \textit{Acalli} damit wieder
manövrierfähig. Er hatte seinen Teil geleistet. Einen Moment des
Durchatmens wollte er sich noch gönnen, bevor er sich an den
Abstieg machte.

Langsam drehte sich Emilio im Kreis. Durch die Folienkuppel hatte
er eine Rundumsicht auf die Schlacht. Es sah gut aus für sie.

Die Mexikaner zeichneten sich durch Wendigkeit aus. Selbst die
höchsten Ränge unter ihnen waren ehemalige Arbeiter, geschult durch
jahrelangen Widerstandskampf. Während die Spanier in den
Frontalangriff gingen, setzten sich die mexikanischen Kämpfer mit
einem geschickten Salto in der Schwerelosigkeit in ihren Rücken und
durchschnitten mit einer einzigen Bewegung die Schläuche der
Angreifer, die sie mit dem spanischen Schiff verbanden. Als Emilio
den ersten Blick auf das Schlachtfeld warf, wurde ihm die Sicht
bereits durch große Fontänen aus komprimierter Luft getrübt, die
nun langsam einen Nebel zwischen den Kämpfenden bildeten.

Die zahlenmäßige Überlegenheit der Spanier war immer noch deutlich,
reduzierte sich jedoch exponentiell. Wie Artisten sah er immer
wieder Kämpfer aus dem Malstrom der verlorenen Luft tauchen und
kurz das Messer ansetzen. Dann schwebte wieder ein neuer Leichnam
zwischen den Schiffen.

Waren die Toten bereits durch die sperrigen Raumanzüge
entmenschlicht, erschienen sie Emilio wie seltsame seelenlose
Apparate, als sie gemächlich dem Lauf des Schiffes folgten. Sobald
sie das Schlachtfeld verlassen hatten, nahmen sie an Fahrt auf,
ohne aber die Umlaufbahn der \textit{Acalli} zu verlassen. So sammelte sich
bald ein ganzer Friedhof um \textit{Ehecatl}, der die Spitze des Schiffes
als Galionsfigur schmückte und doch eigentlich als Hoffnungsträger
dienen sollte. Was er, wenn man es genau betrachtete und die
Spanier gänzlich entmenschlichte, auch erfüllte. War es doch ein
spanischer Friedhof vor seiner Nase.

\bigpar

Bevor sich Emilio noch einmal um die eigene Achse gedreht hatte,
war die Schlacht soweit gewendet, dass ein großer Teil der
Längsseite des Schiffes frei von Kämpfern war. Jetzt kamen die
Geschütze in Stellung. Beide Schiffe explodierten in Rauch, der
sich in Zeitlupe an der Bordwand entlang fraß. Die Kugeln bahnten
sich lautlos ihren Weg über die Kluft.

Emilio spürte die Einschläge, aber sie hatten nicht die gleiche
Wirkung, wie er sie aus Jahren des Befreiungskrieges vom Boden
kannte. Das Schiff erschütterte nicht – es schwankte. Die Trägheit
zeigte sich in den Weiten des Alls mächtiger und zugleich
langsamer.

Dann riss die gesamte Länge des spanischen Schiffes auf. Es sah
nach einem Volltreffer aus. Teile der Wandverkleidung schleuderten
ins All, die dahinter liegenden Verstrebungen stürzten in sich
zusammen. Staub schwebte vor der Szenerie, so dass der angerichtete
Schaden nicht auszumachen war. Emilio stellte sich die Spanier vor
– erstarrt in stummen Schreien, als die Dekompression sie kalt
erwischte. Ja, es sah gut aus für die \textit{Acalli}.

\bigpar

Emilio wandte sich zum Gehen. Er wollte Mateo suchen, wollte, jetzt
wo die Schlacht im Abklingen war, gemeinsam mit ihm auf den Tag
zurückschauen und über die Verwirrungen lachen, die in der Hektik
geschehen waren.

Als er noch einen Blick zurück warf, sah er, dass sich der Nebel
aus Staub vor dem spanischen Schiff gelichtet hatte. Er schaute in
eine Kammer, die von den eigentlichen Decks getrennt war. Die
Bruchkante um die Öffnung war sauber, fast als hätte sich eine Tür
ins Innere geöffnet. Dort warteten weitere Spanier, alle trugen
Raumanzüge und viele standen hinter großen Geschützen verborgen.

Diese Tatsache allein, so schockierend und lähmend sie auf manche
der Kämpfer wirkte, hätte wohl nicht gereicht das Kräfteverhältnis
zu verschieben. Die \textit{Acalli} hätte sich aus der Schussbahn gedreht
und die Männer wieder den Einzelkampf aufgenommen. Im Hintergrund
des Raumes aber sah Emilio nun den wahren Drachen der Spanier. Ein
kugelförmiges Gerät mit einem Durchmesser von drei Metern schob
sich durch die Angreifer und auf das Schlachtfeld. Es war aus
glänzendem Metall, mit deutlich hervorstechenden Nieten. An der
Vorderseite befanden sich ein kleines Rundfenster und mehrere
Greifer und Waffenarme. Eine Höllenmaschine, von der man erwarten
mochte, dass sie jeden Moment Feuer spuckte. An der Rückseite war,
wie Emilio es erwartet hatte, ein Luftschlauch, der ins Innere des
Schiffes auslief. Dieser aber war mit unzähligen Metall-Gliedern
verkleidet. Eine sichere Burg für den Aggressor.

Langsam schwebte der Drache über die Kämpfenden hinweg. Im
Vorbeiziehen riss er Schläuche, die in seinem Weg lagen, aus den
Halterungen und achtete wenig darauf, Freund von Feind zu
unterscheiden. Männer, die sich versuchten in seinen Rücken und auf
den Schlauch zu setzen, schüttelte er leichtfertig ab. Er steuerte
auf die \textit{Acalli} zu.

\bigpar

Kurz zuvor hatte Emilio tatsächlich einen Anflug von Hoffnung
gehabt. Jetzt stand er wieder im Angesicht des Drachens und dieser
hier war feindlich gesinnt.

\gedanke{Mateo, was würdest du zu der Bestie sagen? Würdest du es mit einem
Lächeln abtun, und mir erklären, dass dieses dampfspuckende Ungetüm
ganz so aussieht wie \textrm{Ayauhteotl}, die Göttin des Dampfes und des
Nebels, die auch die Göttin der Eitelkeit ist? Und dass Eitelkeit
noch nie zum Sieg geführt hat?} Emilio wusste, was er sagen musste.
Ein Blick auf seine Werkzeuge zeigte ihm, dass dieses eine Mal
tatsächlich eine Person das Schicksal der anderen in der Hand
hatte.

Er berührte die silberne Maske auf seinem Gesicht. Im Moment wollte
er sie lieber als Totem betrachten, als Schutzgeist und Talisman,
denn als technologischen Parasiten, der ihn seiner Mimik beraubte.

Mateo hatte ihm einmal von den \textit{Nahua} der Huasteca-Region erzählt,
die den Frühling als \textit{M\=ecoz}, als maskierte Geister der Unterwelt
begingen. Lieber hätte er jetzt sein Okular bei sich getragen, das
ihm erlaubte, die Schmetterlinge zu sehen. Aber vorbei war die
Träumerei. Er war jetzt einer der Unterwelt. Ein echter
Drachenbändiger.

Emilio ballte die Faust um sein Werkzeug. Vier Minuten hatte er
Zeit, bevor die Dekompression ihn tötete. Entschlossen trat er in
die Folie, die allzu leicht nachgab. Er hörte ein Reißen, zischend
entwich die Luft, dann war jedes Geräusch verschwunden.

Mit einem kräftigen Tritt stieß sich Emilio von der Oberfläche der
\textit{Acalli} ab. Der Drachen hatte sich an die Seite des Schiffes gesetzt
und bearbeitete die Rohre und Streben mit seinem Greifwerkzeug.
Nicht mehr lang, und er hätte die Struktur des Schiffskörpers
derart beschädigt, dass die \textit{Acalli} unrettbar in die Weiten des Alls
abgedriftet wäre.

Emilio beschrieb einen großen Bogen und sank dann langsam an der
Schiffswand entlang, eben dahin, wo die verbliebene Gravitation des
Planeten ihn hinführte: genau in die Arme der Bestie.

Viele der Bändiger aus den unteren Decks hatten sich an den
Balkonen versammelt. Ihre Arbeit war getan. Die Schlacht fand nun
in der Starre und der Stille des Alls ihr Ende. Sah er Mateo? Das
konnte er nicht mit Gewissheit sagen. Die Gestalten an der
Außenwand des Schiffs waren eingehüllt in Kohlestaub und ihre
schreckerstarrten Münder und die weißen Augen waren nicht zu
unterscheiden. Er wünschte es sich. Er wollte, dass Mateo ihn sah,
wenn er alles Träumen aufgab.

So sank er auf den Rücken des Ungetüms und landete genau auf seiner
Lebensader – dem metallverkleideten Schlauch. Er griff an seinen
Gürtel. Ein Soldat mochte sein Schwert haben und damit viel
bewirken. Aber nur ein Drachenbändiger konnte Metall beeinflussen,
es erschaffen und zerstören. Der Druck auf seiner Brust wurde
zunehmend stärker. Er hatte noch zwei Minuten.

Es ging ihm viel einfacher von der Hand als er gedacht hatte. Das
Metall der Spanier war weich. Emilio vermutete, dass es mit Gold
durchsetzt war. Überdauernd aber nachgiebig. Bald sah er einen
schmalen Streifen ausströmender Luft.

\gedanke{Mateo}, dachte er. \gedanke{Wenn du mich siehst, erinnere
dich an mich, wie wir uns das letzte Mal begegnet sind. Da war ich noch frei.}

\tb

Mateo stand an einem der zahlreichen Balkone und beobachtete den
Untergang der Spanier. Ihre Soldaten waren im Kampf unterlegen –
sie besaßen nicht die nötige Opfergabe – und hatten alles auf eine
Karte gesetzt. Aber nun saß Emilio, sein Schützling, rittlings auf
dem Drachen und vor ihm schoss eine Fontäne aus Luft in die Höhe.

Das silbern glänzende Gesicht sank langsam zur Brust. Mateo wusste
nicht, wohin Emilios Blick ging, spürte ihn aber unweigerlich auf
sich selbst.

Er konnte sich nicht abwenden. Auch als der Körper des Jungen
vollends erschlaffte und die Eiskristalle von der Maske über seinen
Kopf und Hals wanderten, um ihn in seinem letzten Ausdruck zu
konservieren: Mateo erschien es fast entspannt, als Emilio über den
Rand des Drachen in den Abgrund stürzte – ob aus innerer Ruhe oder
aus Hilflosigkeit, das wusste er nicht.

\section{V. Der Weg der Geschichte}

Da war sie also: die Geschichte von Emilio, der sich vom
Drachenbändiger zum -bezwinger aufschwang. Eine Geschichte, so
meinte Mateo, die Potential zur Legende hatte; nur seiner, das
wusste er auch, würde sich niemand erinnern. Dabei war er es, dem
die herzlose Aufgabe zukam, Martha zu unterrichten.

Einer Mutter waren Geschichten gleich. Ob ihr Sohn als Held starb
oder nur durch einen hässlichen Zufall zählte nicht. Anekdoten
würden ihr nicht helfen, die bittere Nachricht den Trauergästen zu
vermitteln, bei einem Begräbnis, das ohne Ehrung des toten Körpers
auskommen musste.

Gleichzeitig, dachte sich Mateo, würden die Begleitumstände des
Todes ihm die Aufgabe erleichtern. Ihm käme nur die Rolle des
Tröstenden zu. Die Fakten, oder das, was die Legende aus ihnen
machte, würde Martha schon vom Wind erfahren haben.

\bigpar

Hier oben begann die Geschichte; hoch über der Stratosphäre,
zwischen den Drachenbändigern (deren Sohn Emilio ebenso war) und
den Hoheiten. Hier würde sie bleiben, solange das Schiff aus den
Weiten des Alls in die Atmosphäre eintauchte. Sobald es aber am
Dock festmachte, und der erste Stiefel eines Matrosen mexikanischen
Boden betrat, trüge er die Geschichte mit sich. Und wispernd
zunächst und unter der Hand würde sie sich weiter verbreiten bis
sie den ersten Beamten erreichte. Der würde das Potential erkennen,
und sicher würde es nicht lange dauern, bis er Gesänge dichten
ließe, und bis ein Denkmal errichtet wäre, das Emilio – dem wahren
Emilio im Gegensatz zum historischen – nicht im geringsten gliche.

\bigpar

Mateo wusste nicht ob Martha zuerst die Gesänge hören würde
(wenigstens würde der Schock dann durch eine sanfte Stimme
überbracht), oder ob sie das Geschehene über den landesweiten
Rundfunk erfuhr.

\bigpar

Ihm, Mateo, aber bliebe nur, Martha in die Arme zu nehmen und zu
sagen:

\bigpar

»Emilio hat bis zum Ende gewagt zu träumen.«

\end{document}
