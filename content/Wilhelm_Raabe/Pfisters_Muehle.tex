\usepackage[german,ngerman]{babel}
\usepackage[T1]{fontenc}
\hyphenation{wa-rum}


%\setlength{\emergencystretch}{1ex}
\renewcommand*\dictumwidth\textwidth
\setkomafont{dictumtext}{\normalfont\normalcolor\sffamily\large}

\newcommand*\zusatz[1]{\dictum{#1}\medskip}

\begin{document}
\raggedbottom

\author{Wilhelm Raabe}
\title{Pfisters Mühle}
\subtitle{Ein Sommerferienheft}
\date{}

\lowertitleback{\textuppercase{UND IN DEM BLICK AUF DAS GANZE\\
IST DER DOCH EIN STÄRKERER GEIST,\\
WELCHER DAS LACHEN, ALS DER,\\
WELCHER DAS WEINEN NICHT HALTEN KANN\\}
Seneca, \textit{Von der Gemütsruhe}}

\maketitle

\section{Erstes Blatt}

\zusatz{Von alten und neuen Wundern}
Ach, noch einmal ein frischer Atemzug im letzten Viertel dieses
neunzehnten Jahrhunderts! Noch einmal sattelt mir den Hippogryphen;
– ach, wenn sie gewußt hätten, die Leute von damals, wenn sie
geahnt hätten, die Leute vor hundert Jahren, wo ihre Nachkommen das
»alte romantische Land« zu suchen haben würden!

Wahrlich nicht mehr in Bagdad. Nicht mehr am Hofe des Sultans von
Babylon.

Wer dort nicht selber gewesen ist, der kennt \emph{das} doch viel
zu genau aus Photographien, Holzschnitten nach Photographien,
Konsularberichten, aus den Telegrammen der Kölnischen Zeitung, um
es dort noch zu suchen. Wir verlegen keine Wundergeschichte mehr in
den Orient. Wir haben unsern Hippogryphen um die ganze Erde gejagt
und sind auf ihm zum Ausgangspunkte zurückgekommen.

Enttäuscht sind wir abgestiegen, und die Verständigen ziehen ihr
buglahmes, keuchendes Tier in den Stall, und wir haben es ihnen
schon hoch anzurechnen, wenn sie kopfschüttelnd und mit einem
betrübten Seufzer das still tun und sich nicht durch irgendeine
Redensart eines schlechte Geschäfte gemacht habenden Musterreiters
ob ihrer Enttäuschung rächen und grinsen:

\begin{center}
»Auf den Leim nie wieder!«
\end{center}

\noindent
oder:

\begin{center}
»Na, so blau!«
\end{center}

Jenseits dieser Verständigen sind dann einige, von denen wir, da
wir höchstpersönlich unter ihnen beteiligt sind, nicht wissen oder
nicht sagen können, ob sie zu den ganz Unverständigen gehören.
Diese stehen und halten ihr Vogelpferd am Zügel und wissen nicht
damit wohin, denken Kinder und Enkel und schütteln das Haupt. Durch
die Wüste, über welcher der Vogel Rock schwebte, über welche Oberon
im Schwanenwagen den tapfern Hüon und die schöne Rezia, den treuen
Knappen Scherasmin und die wackere Amme führte, sind Eisenschienen
gelegt und Telegraphenstangen aufgepflanzt; der Bach Kidron treibt
Papiermühlen, und an den vier Hauptwassern, in die sich der Strom
teilte, der von Eden ausging, sind noch nützlichere
»Etablissements« hingebaut: wer hebt \emph{heute} von
\emph{unseren} Augen den Nebel, der auf der \emph{Vorwelt Wundern}
liegt?

Wer? – Was? ist vielleicht die richtigere Frage. Ein leichter Hauch
aus der Tiefe der Seele in diesen Nebel, und er zerteilt sich auch
heute noch gradeso wie im Jahre siebenzehnhundertundachtzig. Das
»alte romantische Land« liegt von neuem im hellsten Sonnenschein
vor uns; wir aber erfahren mit nicht unberechtigtem Erstaunen, wie
uns jetzt der »Vorwelt Wunder«, die wir in weiter Ferne vergeblich
suchten, so nahe – dicht unter die Nase gelegt worden sind im Laufe
der Zeiten und unter veränderten Umständen.

Zehn Schritte weit von unserer Tür liegen sie – zehn, zwanzig,
dreißig Jahre ab~–, als die Eisenbahn noch keine Haltestelle am
nächsten Dorfe hatte – als der Eichenkamp auf dem Grafenbleeke noch
nicht der Separation wegen niedergelegt war – als man die
Gänseweide derselben Separation halben noch nicht unter die
Bauerschaft verteilt und zu schlechtem Roggenacker gemacht hatte –
als die Weiden den Bach entlang noch standen, als dieser Bach
selber~–

Nun, von diesem letztern demnächst recht vieles mehr! Er fließt zu
bedeutungs- und inhaltsvoll durch die Wunder der mir persönlich so
nahe liegenden Vorwelt, von welcher hier erzählt werden soll, als
daß über seine Existenz mit einem Sprunge oder in drei Worten
weitergeschritten werden könnte.

\tb

»Was schreibst du denn da eigentlich so eifrig, Mäuschen?« fragte
die junge Frau; und der junge Mann, das eben vom Leser Gelesene,
niedergeduckt durch die süße Last auf seiner Schulter, noch einmal
seitwärts beäugelnd, meinte:

»Eigentlich nichts, Mieze. Bei genauesten Betrachtung aber leider
nichts weiter als das, was du selber bereits längst durch gottlob
ziemlich eingehendes und eifriges Studium herausgefunden hast.
Nämlich, daß ein gewisser Jemand auch an einem so schönen Morgen
wie der heutige der graueste aller Esel, der ›erschröcklichste
aller Pedanten‹ und – kurz und gut eigentlich ›ein gräßlicher
Mensch‹ ist.«

»Dann klapp das dumme Zeug zu und komm herunter und erzähle mir das
übrige draußen. Ein schrecklicher Mensch bist du, und ein
himmlischer Morgen ist es. Die Wildtauben gurren immer noch in den
Bäumen, und von dir, mein Schatz, verbitte ich mir hoch und höchst
alles fernere Geknurre und Gedruckse. Komm herunter, Ebert~–

\begin{verse}
›Das Wasser rauscht zum Wald hinein,\\
Es rauscht im Wald so kühle;\\
Wie mag ich wohl gekommen sein\\
Vor die verlassene Mühle?‹«
\end{verse}

Mit heller, lustigster Stimme machte sich die liebe Kleine ihre
eigene Melodie zu dem wehmütig-schönen, melodischen Verse, und
\emph{mir} blieb wirklich nichts übrig, als unter meine
unmotivierte Stilübung dahin drei Kleckse zu machen, wo im Druck
vielleicht einmal drei Kreuze stehen, und mich hinunterziehen zu
lassen unter die alten Kastanienbäume, in deren Wipfel die wilden
Tauben immer noch in den Sommermorgen hineingurrten.

\section{Zweites Blatt}

\zusatz{Zu leeren Tischen und Bänken}
Es war ein eigen Ding um die Mühle, von der hier die Rede ist. Im
Walde lag sie nicht, und verlassen war sie grade auch nicht. Ich
hatte sie nur verkauft – verkaufen müssen~–, aber vier volle
Sommerwochen war sie noch einmal mein Eigentum. Dann erst traten
die neuen Besitzer in ihr ganzes Recht an ihr. Ich hatte mir das
nicht so ausbedingen und es mir schriftlich geben lassen können,
aber die jetzigen Herren hatten gegen meine »seltsame Idee« nichts
einzuwenden gehabt, sondern mich und meine Frau sogar recht
freundlich eingeladen, bis zum Beginn des Baus ihrer großen Fabrik
auf ihrem Besitz ganz zu tun, als ob wir daselbst noch zu Hause
wären. Einmal also sollte ich sie noch für mich haben, wie ich sie
seit meinem ersten Augenaufschlagen in dieser Welt kannte und in
meinen besten Erinnerungen mit ihr verwachsen war. Nachher durften
freilich die neuen Herren mit ihr anfangen, was sie wollten: ich
und mein Weib hatten weder ein Wort noch einen Seufzer
dreinzugeben. Ich wußte schon, daß sie, die nunmehrigen Eigentümer,
sich große Dinge mit ihr vorgenommen hatten, für mich aber konnte
leider Gottes mein Vätererbe nichts weiter sein als ein großes
Wunder der Vorwelt, ein liebes, vergnügliches, wehmütiges Bild in
der Erinnerung. Und ich hatte meine junge Frau dies Jahr, das erste
Jahr unserer Ehe, nicht nach der Schweiz, nach Thüringen oder in
den Harz in die Sommerfrische geführt, sondern nach meiner
verlassenen Mühle. Was sollte daraus werden, wenn das Weib dem
Manne nicht in seine besten Erinnerungen zu folgen vermochte?
Schnezlers Romanze hatte sie meinem »ewigen Gesumme« im
Eisenbahnwagen von Berlin her bereits so ziemlich abgelauscht und
abgelernt und mehr als einmal dabei gesagt: »Bald kann ich's auch
auswendig, Miezchen!«, wobei sie dann hinzusetzte: »Auf deine
väterliche Heimat bin ich aber doch sehr gespitzt, mein Herz.«~–~–

Meine väterliche Heimat! Daß ich gespitzt oder gespannt auf meinen
Aufenthalt und mein unwiderruflich Abschiednehmen dort gewesen sei,
kann ich nicht sagen. Der Ausdruck, selbst aus dem Munde der Liebe
oder grade aus diesem lieben, zärtlichen Mündchen, war mir auch gar
nicht zu Sinne, wenn ich gleich im Rädergerassel, in ein Geschrill
der Dampfpfeife und dem Getümmel der Bahnhöfe nicht wußte, wie ich
ihn verbessern sollte.

In den Wald hinein rauschte das Wasser nicht, das die Räder
\emph{meiner} Mühle in meinen Kindheits- und Jugendtagen trieb. In
einer hellen, weiten, wenn auch noch grünen, so doch von Wald und
Gebüsch schon ziemlich kahl gerupften Ebene war sie, neben dem
Dorfe, ungefähr eine Stunde von der Stadt gelegen. Aus dem Süden
kam der kleine Fluß her, dem sie ihr Dasein verdankte. Ein
deutsches Mittelgebirge umzog dort den Horizont; aber das Flüßchen
hatte seine Quelle bereits in der Ebene und kam nicht von den
Bergen. Wiesen und Kornfelder bis in die weiteste Ferne, hier und
da zwischen Obstbäumen ein Kirchturm, einzelne Dörfer überall
verstreut, eine vielfach sich windende Landstraße mit Pappelbäumen
eingefaßt, Feld- und Fahrwege nach allen Richtungen und dann und
wann auch ein qualmender Fabrikschornstein – das war es, was man
sah von meines Vaters Mühle aus, ohne daß man sich auf die Zehen zu
stellen brauchte. Aber die Hauptsache in dem Bilde waren doch, und
dieses besonders für mich, die Dunstwolke und die Türme im
Nordosten von unserm Dörfchen. Mit der Natur steht die Landjugend
auf viel zu gutem Fuße, um sich viel aus ihr zu machen und sie als
etwas anderes denn als ein Selbstverständliches zu nehmen; aber die
Stadt – ja die Stadt, das ist etwas! Das ist ein Entgegenstehendes,
welches auf die eine oder andere Weise überwunden werden muß und
nie von seiner Geltung für das junge Gemüt etwas aufgibt.

Was alles, worüber ich heute noch Rechenschaft ablegen kann, habe
ich erlebt in dieser Pappelallee, auf dem Wege von und nach der
Stadt!

Und sie stand noch dazu in einem ganz ausnahmsweise angenehmen
Verhältnis zu uns in der Mühle, diese Stadt!

Dutzende von nunmehr vermorschenden Tischen und Bänken unter unsern
Kastanien und Linden, in Gebüsch und Lauben, auf behaglichen
Rasenflecken zeugen noch davon. Heute haben Emmy und ich die
Auswahl unter allen diesen behaglichen Plätzen und das Reich allein
an allen Tischen und auf allen Bänken. Es hindert uns nichts mehr,
in meines Vaters Grasgarten, um der Sonne auszuweichen oder sie zu
suchen, mit dem Buch und der Zigarre, der Häkelarbeit und der
Kaffeekanne um ein paar Schritte weiterzurücken; aber einst war das
anders.

Es gab eine Zeit, wo Emmy mehr die Auswahl unter den Studenten aus
der Stadt als unter den Plätzen im Mühlengarten gehabt hätte. Aber
nicht bloß unter den Studenten. Es gab damals keinen angenehmern
Ruf als den meines Vaters mit seinem kühlen Bier, seinem heißen
Wasser zum billigen Kaffeekochen und seiner süßen und sauern Milch.
Sie kannten alle in der Stadt unsere Mühle, groß und klein,
Gelehrte und Ungelehrte, hohe Regierende und niedere Regierte.

Wir waren von Urväterzeiten die Leute darnach und lieferten den
Bauern im Dorf und den Bäckern in der Stadt nicht bloß das Mehl,
sondern auch noch einiges andere zu dem allgemeinen Behagen der
Welt. Soweit die deutsche Zunge klingt, sitzen heute noch Alte
Herren auf Kathedern, Richterbänken und an Krankenbetten, ganz
abgesehen von denen, die allsonntäglich auf Kanzeln stehen; und in
die Schulstube, den Schwurgerichtssaal, die Krankenstube und das
Räuspern und Schnauben der »christlichen Zuhörer« summt es ihnen
aus zeitlich und räumlich entlegener Ferne:

\begin{verse}
»Weende, Nörten, Bovenden\\
Und die Rasenmühle,\\
Das sind Orte, wo man kann\\
Sich behaglich fühlen.«
\end{verse}

Die Rasenmühle ist es freilich nicht, von welcher hier die Rede
ist; aber es wiederholt sich gottlob manches Gute und Erquickliche
an andern Orten unter andern Namen. Auch mein väterlich Anwesen hat
seine Stelle in mehr als einem ältern Studentenliede, und wir, die
Pfister von Pfisters Mühle, können nichts dafür, daß künftige
Generationen, wenn sie ja noch singen, nicht mehr von ihm singen
werden.

\section{Drittes Blatt}

\zusatz{Wie Sardes, Frau!}
Ich klappte das dumme Zeug zu, und es hatte wirklich keiner weitern
Überredungskunst und Kraft bedurft, um mich dazu zu bewegen. Emmy
hatte für den heutigen Morgen ihr und also auch mein Plätzchen in
einer zerzausten Laube dicht am Flusse gewählt, wo man im Schatten
saß und das Licht auf dem muntern Wasser und den Wiesen drüben im
vollen Morgenglanze vor sich hatte.

Die Wildtauben gurrten über uns, im Schilf schnatterte eine
Entenschar, hielt uns fest im Auge und achtete auf die Bissen, die
von unserm Frühstückstische für sie abfielen. Ein Storch ging am
andern Ufer in der Sonne spazieren, und Emmy sagte:

»Guck mal den! Eine volle halbe Stunde schon achte ich hier allein
in der Einsamkeit auf ihn, und manchmal guckt er auch hier herüber,
als wollte er sagen: Siehst du, ich stehe nicht bloß im Bilderbuche
und sitze im Zoologischen Garten gegen eine halbe Mark
Eintrittsgeld an Wochentagen, sondern~–«

»Ich bin eine Wirklichkeit, eine wirkliche, wahrhaftige
Wirklichkeit, und ich fange auch nicht bloß Frösche, sondern
Kinder; und weise Frauen und nicht bloß gelehrte, sondern auch
kluge Männer wollen nicht bloß nach der Tradition, sondern auch aus
eigener Erfahrung als ganz gewiß wissen~–«

»Du, höre mal, närrischer Dummrian«, meinte meine neunzehnjährige
blonde Matrone, mich jetzt ihrerseits wieder unterbrechend, aber
dabei doch noch ein wenig mehr sich annestelnd, »mit den
Kindergeschichten und Märchen, und was deine überweisen Frauen und
naseweisen Männer aus der Erfahrung und der Naturgeschichte und der
eigenen Tradition wissen wollen, rücke jetzt meinetwegen eine Bank
weiter. Die Auswahl haben wir ja; und ich habe auch darüber den
ganzen Morgen in meiner verlassenen Einsamkeit mir allerlei
Gedanken gemacht. Herzensmann, eine schöne Wirtschaft müßt ihr hier
vor meiner Zeit doch geführt haben!«

»Eine wunderschöne – wunderbare – wundervolle, Kind!«

»Das sieht man den Ruinen noch an; und es tut dir heute natürlich
nicht im geringsten leid, daß ich damals nicht auch schon mit
dabeiwar wie die Jungfer Christine und euch diese wunderbare,
wunderschöne, wundervolle Wirtschaft nicht mit führte?«

Und ich, Eberhard Pfister, frage jeden, das heißt jedes männliche
Erdengeschöpf, was er oder es auf diese Frage geantwortet haben
würde.

Glücklicherweise rief die Christine in diesem Augenblick in unseren
jetzigen hiesigen Haushaltsangelegenheiten nach der jungen Frau,
und zwar mit einer Milde und Lieblichkeit in Ton und Ausdruck, die
ich in meinen jungen Jahren nicht immer an ihrem Organ gekannt
hatte. Und Emmy flötete zurück: »Gleich, gleich, gute Seele!«, warf
mir ihr Nähzeug auf den Schoß und enttänzelte neckisch und
holdselig durch den Lichter- und Schattentanz unter den guten,
alten Kastanienbäumen unserer Mühle zu, mit zierlichem Knicks und
Kußhand mich in meinen Erinnerungen an die hiesige frühere
Wirtschaft zurücklassend.

Ach und wie nahe lagen sie noch, die Tage dieser früheren
Wirtschaft in der Mühle! Wie wenige Jahre war es her, daß mein
Vater dort in der Tür stand , in die eben mein Liebchen geschlüpft
war, und ebenfalls fröhlich und unschuldig »gleich!« rief, aber
hinzusetzte »meine Herrschaften!« im Verkehr zwischen dem Hause und
den Tischen und Bänken unter den grünen Bäumen den Fluß entlang und
auf den Rasenplätzen – der vergnüglichste Mensch der Welt. Ach,
wenn nur nicht grade die vergnüglichsten Menschen dann und wann das
bitterste Ende nehmen müßten!\ldots{}

Alle haben ihn gekannt. Patrizier und Plebejer, Philister,
Professoren und Studenten, die letzteren freilich nur neulich noch,
haben ihn gekannt, den Vater Pfister in seinem Haus- und
Gartenwesen; und wenn ich heute noch in jener vieltürmigen Stadt
dort von manchen Leuten gekannt bin und freundlich gegrüßt werde,
so habe ich das einzig und allein Pfisters Mühle, meinen Ahnen drin
und meinem verstorbenen Vater Bertram Gottlieb Pfister und seiner
ausgezeichneten Wirtschaft zu danken. Was unsern Familiennamen
anbetrifft, so hat der Ahnherr des Geschlechts sicherlich der
ehrsamen Bäckergilde angehört. Als Magister artium und Doktor der
Theologie ist ein der Familie zugehöriger, zu einem Pistor oder
Pistorius latinisierter Becker zwischen dem Schmalkaldischen und
dem Dreißigjährigen Kriege nachzuweisen; aber als Pfister haben wir
seit dem Anfang des achtzehnten Jahrhunderts eben auf Pfisters
Mühle gesessen, und verschiedene von diesen letzteren werten
Männern würden wahrscheinlich in ihrem Staub sich schütteln, wenn
die Nachricht zu ihren verschollenen Ruhestätten dränge, daß dem in
der Folge nicht mehr so sein werde.

Aber Emmy kümmert das ja gottlob nicht, und auch mich lange nicht
so viel, als es von Rechts wegen sollte. Das Kind ist reizend; und
gesund und jung sind wir beide, und Berlin ist eine große Stadt,
und man kann es darin zu vielem bringen, wenn man die Augen offen
und auch seine übrigen vier Sinne beisammen behält und nicht ganz
ohne Grütze im Kopfe ist. Wir zwei haben die Welt und unsere
hübschesten, feinsten und würdigsten und wertvollsten Hoffnungen in
ausgesuchter Fülle noch vor uns; wir haben das volle Recht, die
Mühle als nichts weiter als das uns nächstliegende Wunder der
Vorwelt zu nehmen. Und wenn einer nichts dagegen einzuwenden haben
würde, so ist das mein alter lieber Vater, der letzte Pfister auf
Pfisters Mühle unter seinem noch nicht eingesunkenen und
verschollenen grünen Hügel bei unseren Vorfahren auf dem Kirchhofe
unseres Dorfes.

Von dem, dem Vater Pfister, rede ich nun, an den denke ich nun,
während Emmy und Christine drinnen in dem Hause an seinem großen
Herde, auf welchem er einen so vortrefflichen Grog und Glühwein zu
brauen verstand, von welchem so viele sparsamere Familienmütter und
hübsche, junge Kleinbürgertöchter das kochende Wasser für ihren
Kaffeetopf holten, an welchem er so viele tausend glückselige
Kindergesichter vergnüglich tätschelte, – ihre Köpfe über mein
Mittagessen zusammenstecken.

»Vater Pfister, mir zuerst!«

Wie oft ist der Ruf durch den übrigen lustigen Lärm um uns her an
mein Ohr geklungen, seit ich aufwachte – auch ich unter den Gästen
von Pfisters Mühle – des Vater Pfisters verzogenster Stammgast!

Des Vaters! Meine Mutter hatten wir leider so früh verloren, daß
ich für mein armes Teil gar keine Erinnerung mehr von ihr hatte und
ich als Gast in der Mühle wie auf der Erde von frühester Kindheit
an auf den Vater angewiesen war. Und auf die Jungfer Christine! Die
hatte die Mutter bald nach ihrer Verheiratung mit dem jungen Müller
von Pfisters Mühle sich an die Hand und ins Haus gezogen und soll
auf dem Sterbebett zu ihr gesagt haben: »Mädchen, ich stürbe viel
weniger ruhig, wenn ich dich nicht kennte und wüßte, daß du ein
gutes Herz und eine harte Hand und weiter keinen Anhang in der Welt
hast. Die Wirtschaft und den Verkehr mit den Leuten hab ich dir
auch beigebracht, also rücke mir das Kissen zurecht in meiner
bittern Sorge und stehe fest für die Mühle und meinen Müller und –
nimm noch zum letztenmal einmal vor meinen leiblichen Augen mein
arm, verlassen Tröpfchen aus der Wiege und lege es trocken, auf daß
ich noch einmal sehe, daß du es in alle Zeit weich anfassen willst
und dein Bestes tun. Zurechtgeschüttelt hab ich dich wohl, wenn's
zu deinem Besten notwendig war, – jetzt küsse deine Frau in ihrer
höchsten Angst dafür zum Danke; und wenn's mir möglich sein wird,
passe ich auch ganz gewiß noch fernerhin aus der Ewigkeit auf dich
und dein Verhalten\ldots{}«

»Und den Kuß hab ich mit dir im Arme, mein Junge, an ihrem Bett auf
den Knieen ihr geben dürfen und mich so mit der Mühle verlobt und
auf kein Mannsbild nachher weiter geachtet, wenn ich auch wohl mal
wie andere die Gelegenheit gehabt habe, mich zu verändern, und ganz
gute Partien aus dem Dorfe und aus der Stadt!« hat mir die
Christine tausendmal mit immer sich gleichbleibender Rührung
erzählt, und ich werde wahrlich auch heute noch nicht darob
ungeduldig, auch wenn die treuherzige, melancholische Erinnerung
noch so sonderbar mit den Vorkommnissen – Ärgernissen und
Annehmlichkeiten – des laufenden Tages in Verbindung gebracht
wird.

Wie mein Vater die Jahre seit dem Tode meiner Mutter ohne die
Christine zurechtgekommen sein würde, weiß ich nicht. Er hätte es
wohl auch möglich gemacht, aber besser war besser, und so war auch
für die Stadt und Umgegend Pfisters Mühle ohne die Jungfer
Christine nicht mehr zu denken, und was demnächst in der großen
Stadt Berlin aus der Christine in unserm neuen Haushalt werden
wird, das wage ich nicht vorauszusagen, wenn ich mir gleich
vorgenommen habe, sie nach besten Kräften bei gutem Humor zu halten
und ihr das neue Leben so leicht als möglich zu machen. Daß Emmy
mir dabei helfen will und auch bereits einige Male ein
erkleckliches Maß von Selbstbeherrschung im Verkehr mit dem guten
alten Mädchen bewiesen hat, trägt viel zu meiner Beruhigung
bei.~–~–

Die Sonne steigt, und Vater Pfisters letzter Stammgast müßte um
eine Bank weiterrücken, um im Schatten seiner Erbbäume zu bleiben
mit seinen Morgenphantasien. Aber wir wohnen schon auf der
Schattenseite unserer Straße in der großen Stadt Berlin, und ich
habe mich daselbst allzu häufig nach dem Sonnenlicht der
Jugendheimat gesehnt, um demselben inmitten derselben in einer
solchen wohligen Frühe aus dem Wege zu gehen. Und ich habe den
Grundriß und sonstigen Entwurf der großen Fabrik, welche die
demnächstigen Eigentümer an diesem Orte aufrichten werden,
eingesehen und weiß, wie wenig Helle und Wärme im nächsten Jahre
schon die Ziegelmauern und hohen Schornsteine auch hier übriglassen
werden. Auch diese Vorstellung hält mich auf meinem Platze fest.
Ich fühle mich mehr denn je als Vater Pfisters letzter Stammgast in
dem heutigen Sonnenschein und Baumlaubschatten. Es hat sich manch
einer einen mehr oder weniger vergnüglichen kleinen Rausch an
diesen Gartentischen gezeugt; aber kein guter Trunk hat so einen
aus Licht und Schatten und Erinnerung gewebten, wie er mich in
diesen Tagen gefangen hält, einem andern Gast zuwege gebracht.

»Wie Sardes in der Offenbarung Johannis ist sie, meine Mühle,
Kind!« hatte ich noch neulich im Eisenbahnwagen zu Emmy geseufzt.
»Sie hat den Namen, daß sie lebet, und ist tot!«

»O Gott, dann weiß ich doch nicht, ob es trotz allem nicht besser
gewesen wäre, wenn wir woanders zu unserer Erholung hingegangen
wären!« hatte die Kleine unter dem Eindrucke dieses lugubern,
biblisch-gelehrten Zitats ängstlich erwidert und – nun gab es
nichts Lebendigeres für sie und für mich als Pfisters Mühle.

Für sie war es ein neues, liebliches, ungewohntes – unbekanntes
Leben, für mich ein konzentriertestes Dasein alles dessen, was an
Bekanntschaft und Gewohnheit gewesen war, von Kindheit an, durch
wundervollste Jünglingsjahre bis hinein ins früheste, grünendste
Mannesalter.

Alles um mich herum, bei gutem und schlechtem Wetter, bei
Sonnenschein und Regen, hatte in den Tagen und Nächten dieser
seltsamen Sommerfrische nicht bloß den Namen, daß es lebte, sondern
es lebte wirklich. Und wie hätte vor allem der letzte wirkliche
Herr und Wirt des guten Ortes sich in Nebel und Nichts auflösen
können, während sein letzter Stammgast noch seinen Platz auf der
Bank und am Tische festhielt?

\section{Viertes Blatt}

\zusatz{Herein von der Gänseweide}
»Einen Augenblick, meine Herren, es wird frisch angestochen!« Ich
höre den jovialen Ruf wie einer der durstigen Gäste im Garten, und
ich bin zugleich auf dem kühlen, gewölbten Flur mit dabei als
flachsköpfiger, dreikäsehoher Eingeborener von Pfisters Mühle und
beobachte den Vorgang mit stets sich gleichbleibendem Interesse.
Das geleerte Faß darf ich den Abhang hinter dem Haus hinab in den
Schuppen zu den übrigen rollen, und das Gaudeamus igitur aus der
großen Laube ist mir wie ein Gesang von der Wiege her. Seit
Väterzeiten kennen wir, alle Pfister in der Mühle, das Kommersbuch
auswendig, wenn ich gleich in neuester Zeit der einzige bin, der
auch in andern Lauben, Gärten, Schenken und Mühlen mit
Schankgerechtsame Gebrauch davon gemacht hat mit der
Verbindungsmütze auf dem närrischen, heißen Kopfe und dem Schläger
in der Faust.

Er setzte etwas auf seinen und seines Hauses und Gartens Ruf in der
Welt, mein Vater! Fast alle unsere Wände waren mit den
Verbindungsschildern, Silhouetten und Photographien seiner
akademischen Freunde bedeckt, und für mein eigenes Leben sind seine
Neigungen zu dem jungen gelehrten Volk und allem, was dazu gehört,
von dem größesten Einfluß gewesen. Der Umgang mit den jungen (und
auch den alten) Leuten, welche ihm die Stadt und die Universität
tagtäglich herausschickten und in deren mehr oder weniger
geräuschvolle Unterhaltung er gern auch sein Wort und seine Stimme
dreingeben durfte, hatte ihm in betreff meiner wohl allerlei in die
Phantasie gesetzt, was meinem Lebensgange jedenfalls eine andere
Richtung gab, als Pfisters Mühle seit Generationen an ihren
Erbeigentümern gewohnt war.

Ein weißlicher Müller und ein weiser Mann war er; aber alles auf
einmal konnte auch er nicht bedenken und das einander
Ausschließende miteinander in Gleichklang bringen. So trug denn
auch er sein Teil der Schuld, daß der augenblicklich letzte Pfister
nicht mehr als Müller auf Pfisters Mühle sitzt; und mein einziger
Trost ist, daß der Alte, als er auf seinem Sterbebett zum letzten
Male seinen Arm mir um den Nacken legte und mich zu sich niederzog,
sagen durfte: »Ist's nicht, als ob ich's vorausgerochen hätte,
lieber Junge, als ich dich von der Gänseweide holte und mit der
Nase ins Buch steckte? Die Welt wollte uns nicht mehr, wie wir
waren, zu ihrem Nutzen und Vergnügen. Aufdrängen muß man sich
keinem; und so ist's wirklich am besten so geworden, wie es sich
gemacht hat\ldots{}«

Es war richtig; auf Schulen ging ich zwar schon, nämlich in die
Dorfschule zum Kantor Busse, und am liebsten um den Kantor und die
Schule herum, als er, Vater Pfister, mich auf dem Gänseanger
nacktbeinig unter den übrigen flachsköpfigen Barfüßern
herauslangte, mich am Kragen nach Hause führte und mich in
genaueste wissenschaftliche Verbindung mit einem andern, etwas
ältern und gebildeteren, verwahrlosten Menschenkinde brachte, das
er gleichfalls am Kragen hielt, wenn auch mehr mittelbar, daß heißt
infolge des Pumpes, den es seit längerer Zeit bei ihm angelegt
hatte.

»Wenn Sie auf den Vertrag eingehen, Herr Asche, wird es vielleicht
für beide Parteien ein gutes Abkommen sein, und dünner sollen Sie
mir nicht dabei werden, wenn dies nicht so in Ihrer Natur liegt und
die Weltregierung Sie nicht schwerer auf der Waagschale haben will,
Adam«, sagte mein Vater.

Das aber ist die zweite Gestalt, die von Tisch und Bank, aus Licht
und Schatten, aus alle dem Tumult, den Klängen und Studentenliedern
um Pfisters Mühle sich loslöst und vertraulich-seltsam wie mit
Stroh im Haar, wenn auch keineswegs im Kopfe, in diese Traumbilder
hineinschlendert. Grade als habe auch sie bis jetzt den Tag auf der
Gänseweide hingebracht oder noch bequemlicher, auf dem Rücken
liegend zwischen den Roggengarben auf dem Felde jenseits der
Uferweiden, des Entengeschnatters und des Mühlwasserrauschens von
Pfisters Mühle.

»Können das Ding probieren, Vater Pfister! Geben Sie Ihren Bengel
her. Werden ja bald erfahren, wer die Langweilerei am ersten satt
kriegt, Sie, ich oder dies glückselige, quatschlige, weißfleischige
Geschöpf Gottes hier. Braten könnte ich es mir jeden Mittag;
weshalb sollte ich ihm nicht gegen zivilisiertere freie Beköstigung
und ein Taschengeld an jedem Mittwoch und Sonnabend die
Anfangsgründe des Lateinischen beizubringen versuchen? Die Sache
paßt mir vollkommen. Mürbe wollen wir ihn schon kriegen. So 'nen
jungen Römer zum Weichreiten unterm Sattel hab ich mir schon längst
zu Weihnachten oder zum Geburtstage gewünscht. Sollen wir heute mit
ihm anfangen, oder hat der Knabe auch eine Stimme bei dem Kontrakt
und zieht er's vielleicht vor, am nächsten Sabbat zum erstenmal
übergelegt zu werden?«

Ich habe damals erst meinem Vater in das freundliche, kluge,
vergnügte Gesicht gesehen und dann dem Studiosus der Philosophie
Adam Asche in das seinige, und, die Zähne zusammenbeißend, gesagt:
»Heute!« und nachher die volle Gewißheit erhalten, daß der letzte
wirkliche Besitzer von Pfisters Mühle auch bei dieser Gelegenheit
ganz genau wußte, wen er vor sich hatte und was er tat.

Emmy kennt die dämmerige, düstere Brutstätte meiner ersten
wissenschaftlicheren Betätigungen. »Brr!« hat sie zuerst gesagt,
den Kopf hineinsteckend, aber nachher, wahrscheinlich um mich in
meinen Gefühlen nicht zu sehr zu verletzen, hinzugefügt: »O, wie
hübsch kühl an einem heißen Tage wie heute!«, und das Liebchen
hatte vollkommen recht. Das Loch war recht schön kühl im Sommer,
und im Winter konnte man es leider heizen, und Studiosus Asche
bemerkte bei unserer ersten Niederlassung darin: »Würgen könnte ich
dich, Lümmel, ob deiner höchst unnötigen Existenz im Weltganzen! Da
soll nun ein Mensch Atem holen und Latein verstehen, mit dem vollen
Wissen davon, wieviel gemütlicher es draußen ist. Na, Gott sei dir
Esel gnädig in diesem Sack mit – Asche! Na, na, sieh mich nur nicht
so blödbockig an, Junge! Wir müssen's ja zusammen aushalten!«

Und wir haben es zusammen ausgehalten in dem Stübchen nach hinten
hinaus in Pfisters Mühle. Nach hinten hinaus, von der Lust des
Gartens so weit als möglich entfernt, aber doch nicht ganz von dem
Getön derselben und noch weniger von dem Geklapper und Rauschen der
Turbinenstube, hatte uns mein Vater den Tisch ans Fenster gerückt
und denselben mit allem nötigen Material an Dinte, Federn und
Papier versehen, und da habe ich nicht nur die Rudimente der
Römersprache, sondern noch manches andere von meinem – Freund Adam
Asche gelernt.

Was mir das Latein genützt hat, weiß ich so ziemlich genau heute;
aber wie nützlich mir das »andere« war, erfahre ich heute
tagtäglich so viel mehr, daß von einer sichern Berechnung noch
lange nicht die Rede sein kann.

Es war damals ein recht dürftiges, mageres Männchen, das mit einem
Kopf, der von einem äußerst schwarzstrubbelhaarigen Riesen ihm
zwischen die Schultern gefallen zu sein schien, mir gegenüber, wie
es sich ausdrückte, »die schönen Stunden vertrödelte« und mir nicht
selten energisch genug in die Flachswolle griff, um, wie es
seufzte, »wenigstens etwas« aus mir herauszuziehen. Von »zu braven«
Eltern, wie er meinte, war er – Studiosus philosophiae A.~A.~Asche
– Adam August Asche. »Ich gebe Ihnen mein Wort, Vater Pfister«,
sagte er, »ich würde hier wahrhaftig nicht sitzen müssen, um Ihr
Junges philologisch zu belecken, wenn mein Alter etwas mehr auf das
Wohlbehagen seines Jungen und etwa weniger auf die Wohlfahrt der
Welt und ihre gute oder schlechte Meinung von ihm gegeben hätte.«

»Reden Sie sich nicht um Ihren besten Trost in dieser Welt, Herr
Asche«, sagte mein Vater. »Weil ich Ihren Vater gekannt habe, habe
ich mir eben alleweile gedacht, allzuweit kann der Apfel nicht vom
Stamme gefallen sein, und Vertrauen zu Ihnen gehabt und Sie mir aus
dem Vivathoch da draußen im Garten und vom Verliegen da draußen auf
der Wiese und im Heu hereingeladen und Sie gegen einen Strich durch
Ihr Konto und eine übrige angemessene Entschädigung an meinen
eignen wilden Dorfindianer und eheleiblichen Tagedieb gesetzt.«

»Reden Sie sich nicht um Ihren Hals, Vater Pfister!« hat mein
Freund und Gönner, Doktor Adam Asche, gelacht.

\section{Fünftes Blatt}

\zusatz{Hinter dem Beutelkasten und unter den Kastanien}
Wie wunderlich das für mich heute ist, mit dem lieben jungen Weib
und der alten Christine in unserer alten Küche und unserm
wohlgegründeten behaglichen Heim in der großen Stadt in diesen
abgezählten Sommertagen von der guten alten Zeit in Pfisters Mühle
zu träumen und zu schreiben! Wie sind trotz der sonnigen,
hoffnungsreichen Gegenwart jene anderen, gleichfalls zu- und
abgezählten Tage und Stunden in dem muffigen, dunkeln Winkel nach
hinten hinaus gleichfalls zur »guten alten Zeit« für mich
geworden!

Von dem Latein, das mir darin, wie mein gelehrter Freund Asche das
nannte, »verzapft« wurde, werde ich reden müssen. Ich weiß heute
noch nicht, wie eigentlich meine Begabung dafür ist, aber das weiß
ich genau, daß wir uns damals in dieser Hinsicht auf das
Notwendigste beschränkt haben.

»Es ist Ihr Junge, Vater Pfister, und so haben Sie gewissermaßen
die Berechtigung, mit ihm anzufangen, was Sie wollen. Mensa bringe
ich ihm schon bei; was er nachher auf den Tisch zu stellen hat, ist
Ihre und seine Sache«, sagte Studiosus Asche. »Was mich anbetrifft,
so wissen Sie, daß mein Alter insolvent starb und Schönfärber
war.«

»Und daß von meines guten Freundes, Ihres Vaters, Kunst,
Wissenschaft und Sinnesart vielleicht grade das auf Sie
übergegangen ist, was Sie brauchen und was andern Leuten bei
Gelegenheit auch wieder nützlich werden kann. Auf einmal kann man
selten das Beste zugleich haben, so zum Beispiel den Verstand in
der Welt und das Glück in ihr. Sie ständen sich selber im Lichte,
wenn Sie von Ihrem seligen Vater mit der geringsten Despektion
reden wollten, Herr Asche.«

»Bei den unsterblichen Göttern!« ist die ruhige, gewissenssichere
Antwort gewesen. »Was würde aus mir armen Waisenknaben geworden
sein und werden, wenn nicht wenigstens ein Bruchteil vom Talent des
Alten, die Dinge in der Weit schönzufärben, auf mich übergegangen
wäre? Sie wissen, Vater Pfister, es ist so ziemlich das einzige,
auf was die Gläubiger beim Ausschütten der Masse keinen Anspruch
erhoben.«~–

Das ist wahr. Ich habe nicht einen zweiten Menschen kennengelernt,
der mit gleicher Fähigkeit, den Beschwerden dieser Erde eine
angenehme Färbung zu geben, versehen gewesen wäre, wie mein erster,
über den Dorfkantor hinausreichender Lehrmeister in unserm
Hinterstübchen. Auch die unvermutete, »aus dem blauesten Himmel
hereinbrechende« Störung seiner »Wald-, Feld-, Wiesen- und
Pfistersmühlen-Faulheit« überwand er, und die Stunden, während
welcher mein Vater uns beide hinter Schloß und Riegel hielt, gingen
viel glatter und behaglicher vorbei, als wir es uns beim Beginn der
ersten vorgestellt hatten. Es sitzt mehr als eine grammatikalische
Regel wahrscheinlich nur deshalb heute noch bei mir fest, weil ich
zugleich mit ihr noch das entfernte fröhliche Getön des Gartens und
das nahe Rauschen der Mühlräder im Ohre habe.

»Drei Viertel auf fünf! Noch fünfzehn Minuten, und das Elend liegt
wieder einmal hinter uns. Also noch einmal den Kopf zwischen beide
Fäuste und drücke dreist etwas fester am Gehirn, Knabe! Siehst du,
da haben wir das Gewürm schon draußen, und zwar wie gewöhnlich zum
Teil durch die Nase mit: der schwarze Rabe – corvus niger; der
angenehme Garten (es sind heute die Teutonen, die sich da den Hals
abbrüllen und die kleinen Mädchen anrenommieren!) – hortus amoenus;
das schwere Geschäft – negotium difficile. Der Eierkuchen mit
Schnittlauch, der uns für später in Aussicht gestellt wurde, ist
auch nicht gänzlich zu verachten. Noch einmal mit der Nase in den
Schoß der Weisheit! Drücke – drücke fest: der gierige Bauch?«

»Alvus a-vi-dus, Herr Asche.«

»Avida, Esel! Keine Regel ohne Ausnahmen, mein Sohn. Die große
Futterschwinge?«

»Vannus magna«!

»So machst du mir Freude. Und nun zum Schluß für heute den ganzen
Quark noch mal poetisch: Er ir ur us sind~–?«

\begin{verse}
»\ldots{} Mascula,\\
um steht allein als Neutrum da.«
\end{verse}

»Schön. Solltest du die nichtsnutzigen Ausnahmen auch noch in
dieser zum Herzen sprechenden Weise angeben können, würdest du mir
eine ebenso kindliche Freude bereiten wie dir selber. Leiere ab,
jugendlicher Kitharoede; aber bedenke, daß ich dich immer noch vor
Schluß der Stunde lebendig zu schinden imstande bin. Die Städt und
Bäume~–«

Und während Studiosus A. A. Asche am Tischrande die Faust im Kreise
dreht, als drehe er den Griff einer Straßenorgel, leiere ich her.

\begin{verse}
»Die Städt und Bäume auf ein us\\
Man weiblich nur gebrauchen muß.\\
Von andern Wörtern merke man\\
Sich alvus, colus, humus, vannus an.\\
Die Wörter virus, pelagus\\
Sind einzig Neutra auf ein us,\\
Und vulgus ist daneben auch\\
Als Neutrum meistens im Gebrauch.~–«
\end{verse}

»Hurra! Wieder hinein in das Vulgus, und zwar als möglichst
komplettes Neutrum!«

Es ist Freund Asche, der das nicht ruft, sondern mit merkwürdig
tonloser Stimme seufzt, als riefe ihn des Dorfes Abendglocke nicht
aus der tödlichsten Langeweile, sondern aus der innigsten
Versunkenheit in alle Freuden der Pädagogik ab. Und es ist mein
lieber, verstorbener Vater, der sein kluges, friedliches,
lächelndes Gesicht in die Tür steckt und ruft:

»Nun, Kinder? Hübsch fleißig gewesen? Brav was gelernt?«

»Sehr brav – alle zwei, Vater Pfister.«

»Na, dann seien Sie bedankt, Herr Asche, und kommt heraus. Es ist
wirklich ein recht amöner Abend und der Garten draußen voll bis zum
Platzen. Bis in die Hecken sitzen sie mir. Bringe auch noch euere
Stühle hier im Studio mit hinaus, Junge; bis ans Wasser haben sie
mir die letzten aus dem Hause hingerückt, und Ihre Herren Kollegen,
Herr Adam, haben die ihrigen schon lange höflich an die Damen
abgetreten und behelfen sich mit den leeren Fässern und ein paar
Brettern drüberhin. Hält diese Witterung so an, so bleibt uns
nichts anderes übrig, als daß wir noch ein zweites Stockwerk über
dem Pläsier etablieren, nämlich in den Baumästen. Einige von den
Herren sitzen schon drin und lassen sich das Getränk in die Höhe
reichen.«~–~–~–

Es ist alles vor allen meinen fünf Sinnen.

Es ist kein Zweifel mehr, es ist ein heißer Tag geworden; je mehr
die Sonne dem Mittage entgegengestiegen ist, durch desto
wolkenloseres Blau schwimmt sie, und die Grillen auf den Wiesen
jenseits des Baches hat sie allgemach vollständig berauscht; immer
vielstimmiger und schriller dringt deren Lust an mein Ohr herüber.
Die Enten rudern leise gegenüber im Schilfrohr; als der Schatten
eines großen Raubvogels, der mit schwerfälligem Flügelschlag einem
fernen Gehölz zuzieht, auf das Land fällt, hebt der letzte Gast in
dem einst so lebendigen, jetzt so verlassenen, stillen Garten von
Pfisters Mühle unwillkürlich die Hand und sieht sich erschreckt um:
Welch ein wunderlich Mittagsgespenst in der schwülen, grünen,
goldenen Einsamkeit von Pfisters Mühlengarten! Welch ein bunter,
fröhlicher und doch dem letzten Stammgast so sehr das Herz
beklemmender Abendzauber jetzt – jetzt zwischen elf und zwölf Uhr,
um die Mitte des Tages!\ldots{}

Der Garten voll bis zum Überquellen! Ist es nicht, als habe sich
die halbe Stadt ein Stelldichein in Pfisters Mühle gegeben? Alt und
jung bis zu den Allerjüngsten in der Wagenburg von mehr oder
weniger eleganten Kinderwagen! Männlein und Fräulein, und die
letztern in den zierlichsten, duftigsten Sommergewändern!
Lehrstand, Wehrstand und Nährstand! Die Herren Studenten von allen
Farben, und einige von ihnen – den Herren Studierenden – wirklich
bereits auf den bequemeren Baumästen, wahrscheinlich um von
denselben die Sonne bequemer untergehen zu sehen und einen
objektiveren Überblick über das Philisterium im ganzen, die
hübschen Mädchen und die Mütter der letztern im einzelnen zu
haben.

»Vater Pfister! Vater Pfister! Was soll denn das heißen, Samse, daß
sich kein Mensch von euch in dieser Region blicken läßt!«

»Es wird eben frisch angestochen, meine Herren«, brummt Samse –
unser Samse, ein Drittel Mühlknappe, ein Drittel Ackerknecht, ein
Drittel Dorf- und Gartenkellner, und also ganz und gar von der
Zipfelkappe bis zu den Nägelschuhen, mit Mehlstaubjacke und
Serviette, in Griff und Tritt und Ton vollkommen, unverbesserlich,
gar nicht anders zu denken und zu wünschen – \emph{Pfisters Mühle!}
Doktor Asche hat ihn heute in Berlin als alten, behäbigen,
weißköpfigen Herrn, hat ihm statt der Müllerjacke einen langen,
behaglichen, dunkelgrünen Rock, im Winter mit Pelzkragen
ankomplimentiert, ihm einen Lehnstuhl in eine gemütliche Wachtstube
neben der großen Eingangspforte hingestellt und gesagt: »Sie halten
die Augen wohl ein wenig offen, Samse, und passen mir hübsch auf
alles, was ein und aus geht, alter Knabe. Cave canem! Ist der Junge
aus den Windeln, so passen Sie mir auch auf den wohl ein bißchen
mit, lieber Freund.«

»Wie in Pfisters Mühle, Herr Asche«, hat Samse erwidert, und es ist
ganz gut so. Wie würde er uns verkümmert sein bei den gestellten
Rädern und zwischen den leeren Tischen und Bänken von Pfisters
Mühle! Wie schlecht hätte er sich, auch in meiner Gesellschaft, an
einem Morgen wie der heutige auf dieser Bank, an diesem Tisch gegen
das zu wehren vermocht, was vorbei war und niemals wiederkommen
konnte! Der alte Grobian und getreue Knecht hatte sich eben nur
unter den Menschen und nicht auch unter den Büchern umgetrieben. Er
hätte nicht seine Gefühle zu Papier gebracht; höchstens würde man
ihn nach längerm Suchen und Rufen aus dem Bach aufgefischt oder von
einem Strick in einem dunkeln Winkel von Pfisters Anwesen
abgeschnitten haben.

»Ich habe eine Vorahnung, daß dich nichts so sehr gegen deine
zukünftigen Erlebnisse abhärten wird als eine regelrechte
Beschäftigung mit den Wissenschaften, mein Junge«, sagte mein
Vater, und – es ist immer in diesem Augenblick noch Sommerabend und
Pfisters Mühle in ihrer Glorie ohne Schaden für Leib und Leben in
meiner \emph{abgehärteten} Phantasie. Wie freilich meine Stimmung
sein würde ohne Emmys Arbeitskörbchen auf dem Tische und ihr
Taschentuch auf der Bank neben mir und ohne die Gewißheit ihres
Vorhandenseins in dem stillen Hause unter den Kastanien und Linden
hinter mir, soll trotz aller Bücher und Wissenschaften in der Welt
eine offene Frage bleiben.

»Geh mir nicht so weit weg, daß ich dich nicht abrufen kann«, ruft
eben das süße Herz im weißen Küchenschürzchen von meines Vaters
verkauftem Hause her; ich aber habe wahrlich nicht die Absicht und
Neigung, jetzt weit wegzugehen.

\begin{verse}
Das Wasser rauschet neben mir hin,\\
Als wüßt es, was ich fühle,\\
Und nimmermehr will aus dem Sinn\\
Mir die verlassne Mühle;
\end{verse}
\noindent
es wäre auch ein wirkliches und dazu höchst
jämmerliches Wunder, wenn das trotz allem, was ich auf und vor
Schulbänken und Kathedern zur Abhärtung des »bessern Bewußtseins«
in Erfahrung brachte, möglich sein könnte.

Wie viele der Stimmen, die mich damals von allen Seiten her riefen,
können mich heute nicht mehr abrufen! Wie groß die Gefahr für
meines Vaters Sohn, sich in Stadtkuchen an Dutzenden von Tischen
aus Handtaschen und dem Papier der gestrigen Zeitung zu
überfressen! Und doch gehe ich den geputzten, feinen Stadtdamen und
den kleinen Fräuleins so gern aus dem Wege und ziehe am liebsten in
grinsender Dorfblödigkeit den Ärmel unter der Nase her, wenn man
mir zuwinkt und zulacht und das Behagen und Wohlgefallen an Vater
Pfister auch auf seinen Sprößling überträgt. Am liebsten halte ich
mich jetzt bereits so dicht als möglich hinter meinem vor kurzem
noch so sehr gefürchteten, gelehrten lateinischen Freund aus dem
Hinterstübchen, und es ist möglich, daß ich auch wie er die Hände
in die Hosentaschen geschoben halte und dasselbe Stück ihm
nachsumme oder zwischen den Zähnen pfeife, wie wir uns zwischen den
Tischen hinschieben und die heutigen Gäste von Pfisters Mühle einer
mehr oder weniger gemütlichen Betrachtung unterwerfen.

Wahrlich, ich habe nicht bloß die Grundlagen meiner Kenntnis der
Römersprache von meinem für einen Strich durch sein Kneipkonto,
fernerweitige gute Verköstigung und ein Taschengeld allmonatlich
angeworbenen, eigentümlichen Mentor! Freilich ist es in damals erst
kommenden Jahren, wo ich vollkommen einsehen lerne, was alles man
in Pfisters Mühle und Garten sehen, lernen, in die Erfahrung
bringen kann.

In den Tagen, von welchen jetzt die Rede ist, schiebt der gelehrte
Freund gewöhnlich so rasch als möglich irgendwo einen krassen Fuchs
vom Stuhl, schickt ihn, ganz gegen die Naturgeschichte, gleichfalls
am Baum in die Höhe auf den nächsten bequemen Ast und proklamiert
das riesigste Bedürfnis, mindestens sechs von den nächsten
wiederkäuenden Kamelen abzuschlachten und sie auf den Keller in
ihrem Innern zu prüfen.

An diesen Tischen, hinter diesen Stühlen und Bänken hielt ich mich
am liebsten auf, und Emmy meinte gestern: »Wenn ich bedenke, unter
welchen Gefahren und Verlockungen du hier von Kindesbeinen an
aufgewachsen bist, so habe ich meinem Herrgott eigentlich
tagtäglich dafür auf den Knieen zu danken, daß ich noch so ziemlich
gut davongekommen bin. Dies ist ja gräßlich! Und ein wahres Glück,
daß ich bis heute keine Ahnung hiervon gehabt habe und Papa und
meine liebe selige Mama ebenfalls nicht! Na freilich, Papa sein
Gesicht und seine vergnügte Freundlichkeit hinter seiner Pfeife
sind vielleicht auch nicht besser und moralischer, als sie von
Gottes und Rechts wegen sein sollten; aber was meine arme selige
Mama betrifft, so sollte ich es jetzt wirklich für einen Segen
halten, daß sie leider Gottes nicht uns hierher nach deiner
entsetzlichen Mühle begleiten konnte und ihre Vorgeschichte gehört
hat.«

»Beruhige dich, Kind. Wenn die Rede zu eingehend auf euch süße
Herzen, Trösterinnen im Erdenleben, kurz, bessere Hälfte des
Menschengeschlechts – Kalypso und ihre Schwestern gar nicht zu
erwähnen – geriet, wurde Telemachos vom Mentor stets mit einer
Bestellung ins Haus geschickt oder kurz und bündig aufgefordert,
sich weiterwegzuscheren.«

»Ich danke«, sagte Emmy, leider in einigem Zweifel, ob sie den
Trost wirklich als ein Kompliment aufzufassen habe.

»Und dann – manchmal wurde es ja auch unserm Freund Asche zu arg,
und er nahm mich am Arm und verzog sich selber mit mir aus der
Brüder wildem Reihen.«

»In den Frieden der Natur!« zitierte Emmy eine der mannigfachen
Redensarten ihres Freundes A.~A.~Asche.

\section{Sechstes Blatt}

\zusatz{Eine nachdenkliche Frage}
»Wo bleiben alle die Bilder?« das ist eine Frage, die einem auf
jeder Kunstausstellung wohl einige Male ans Ohr klingt und auf die
man nur deshalb nicht mehr achtet, weil man dieselbe sich selber
bereits dann und wann gestellt hat. Man sieht sich nicht einmal die
Leute, die das Wort aussprechen, drauf genauer an. Die Frage liegt
zu sehr auf der Hand: wo bleiben alle die Bilder?

Ein anderes mit dem Aufachten und der Beantwortung ist's freilich,
wenn einem vor all der unendlichen, bunten Leinwand in den goldenen
Rahmen die eigene junge Frau die Bemerkung macht und uns unsere
Meinung und Ansicht darüber nicht schenken will.

Mich persönlich ergreift sehr bald in einer solchen großen
Ausstellung ein melancholisches Unbehagen, das nicht die
gewöhnliche, aus dem »Bilderbesehen« hervorgehende, körperliche
Ermüdung ist. Und es ergreift mich um so mehr, als ich gottlob mich
zu denen zahlen darf, die wie der alte Albrecht von Nürnberg am
liebsten ihre Kritik in die Worte fassen: »Nun, die Meister haben
ihr Bestes getan!« – Wahrlich, es sind nicht immer die, welche vom
Publikum \emph{Meister} genannt werden und sich selber so nennen,
die ihr Bestes tun! Es gehört zu manch einer mutigen, heißen,
fieberhaft ihr Bestes geben wollenden Seele eine ungeschickte,
zaghafte Hand.~–

»Wo bleiben alle die Bilder? Man begegnet ihnen doch nie wieder
außerhalb dieser Wände. Meine Bekannten haben noch nie eines von
ihnen gekauft. Und immer malen die Herren Maler andere, wenn es
auch von Jahr zu Jahr so ziemlich immer die nämlichen bleiben. Für
ihren Spiegel und dergleichen wird so eine Künstlerfrau recht bald
keinen Platz übrigbehalten, und wenn sie sie nachher auch eins
übers andere an die Wand lehnt, so wird sie sich doch allmählich im
Raum recht beschränkt fühlen. Aber vielleicht werden sie übers Meer
verschickt, nach fremden Weltteilen, wo die Leute mehr Geld für so
was haben und mehr Gelegenheit an den Wänden und wo auch die
Fliegen im Sommer nicht so unangenehm werden.«

»Und wo die Leute vielleicht, abgesehen vom Geld, von den Wänden
und den Fliegen, mehr Geschmack und weniger Kunstverständnis haben,
mein Schatz. Du hattest eine Idee, Liebchen; aber ganz löst sie die
Frage doch nicht: Wo bleiben alle diese Bilder – alle diese Wälder
und Felder, Wasserfälle und italienischen Seen, diese angenehmen
Stilleben und schrecklichen Stürme zu Land und Meer, all das Genre,
all die Historie, diese Schlachten und Mordgeschichten? Komm du nur
noch ein paar Jahre unter meiner Führung hierher, um dein liebes,
kluges Alltagsnäschen und dein hübsches Sonntagshütchen hier mit
mir zum Besten der Kunst spazierenzuführen, und ein großes Licht
soll dir aufgehen.«

»Darauf bin ich neugierig, du Spötter.«

»Es sind nur die Umrisse und die Farben, welche wechseln; Rahmen
und Leinwand bleiben. Jaja, mein armes Kind, es würde uns, die wir
selber vorübergehen, den Raum arg beschränken im Leben, wenn alle
Bilder blieben!«

»Das ist mir zu hoch«, hat Emmy, Gott sei Dank, damals gesagt, und
es bleibt, jedenfalls noch für längere Zeit, eines der hübschesten
Bilder meines Lebensbilderbuches, sie in unsern Flitterwochen
glücklich, lächelnd, tänzelnd am Arm zu haben, sie aus den
heiligen, aber kühlen Hallen der bildenden Kunst in den warmen
Sonnenschein der menschenwimmelnden Straße und die nächste elegante
Konditorei zu führen, sie dort zierlich Eis essen zu sehen und das
Hin- und Herwogen der Tagesmoden draußen vor den glänzenden
Riesenspiegelscheiben mit den Bildern in ihrer Modenzeitung zu
Hause vergleichen zu hören.

Aber es regnet heute rund um Pfisters Mühle und auch auf
dieselbige. Derselbe Rahmen und dieselbe Grundfläche wie
vorgestern; aber ist das noch dasselbe Bild wie vorgestern? Ein
tüchtiger und, wie die Bauern meinen, sehr erwünschter Landregen
kommt seit gestern herunter. Wir haben es versucht, unterm
Regenschirm die Stadt zu erreichen, aber es hoffnungslos
aufgegeben. Nun sitzen wir im Oberstock des Hauses am geöffneten
Fenster und hören und sehen dem Regen zu, ich durch den Rauch
meiner Zigarre, Emmy über eine merkwürdig künstliche weibliche
Arbeit, die darin besteht, Löcher und Zacken in einen langen
Streifen weißer Leinwand zu schneiden und den angerichteten Schaden
vermittelst der Nadel eifrigst wiedergutzumachen. Von der
Landschaft jenseits des Flusses ist wenig zu sehen, große Sümpfe
stehen unter den triefenden Bäumen im Garten, es triefen die alten
Tische und Bänke, und alle Enten sind ans Land gestiegen und doch
in ihrem Elemente geblieben, wie Emmy sich ausdrückt. »Denen ist's
egal!« sagt sie und seufzt und schlägt die großen Sammetaugen von
ihrer Unterrocksborde auf und sieht mich mit einem solchen Ausdruck
von himmlischer, aber hoffnungsloser Geduld und Ergebung an, daß
mich eine unsägliche Armesünderstimmung und das ganz bestimmte
Gefühl überkommt, daß \emph{ich} dieses Wetter angerichtet habe,
daß \emph{ich} für es und alle seine Konsequenzen bedingungslos
verantwortlich bin.

»Auch in Baden-Baden, Wiesbaden und Baden bei Wien regnet es heute
vielleicht und vielleicht ärger als auf Pfisters Mühle, mein Herz«,
wage ich schüchtern zu flüstern; aber Emmy geht durchaus nicht
darauf ein.

»Ich mache dir ja gar keinen Vorwurf, mein Schatz«, sagt sie, »aber
leugnen mußt du es mir auch nicht: im Grunde ist es doch nur Wasser
auf deine Mühle, und ich merkte es dir gleich an, wie recht es dir
kam und wie wohl dir wurde, als sich der Himmel bezog und dich
unsrer Absicht, heute abend im Sommertheater in der Stadt Fatinitza
zu hören, entledigte. Es ist zwar wirklich unendlich lieb, so zu
sitzen und noch mehr als sonst auf uns allein und die Jungfer
Christine angewiesen zu sein; aber dann solltest du auch deine
Mappe zulassen und deine Dinte für Berlin und unser Nachhausekommen
sparen. Was habe ich heute davon, daß du alles das, was du da
Lustiges, Rührendes und Interessantes zusammenschreibst, mir
nächsten Winter vorlesen willst? Da war es ja fast auf Papas
Kirchhofe amüsanter.«

Auf Papas Kirchhofe!\ldots{} Wo bleiben alle die Bilder?\ldots{} »He, he,
he«, pflegte mein Schwiegervater, der damals, in jenen seligen
Tagen des Zweifels und der Erfüllung, noch nicht mein
Schwiegervater war, auf \emph{seinem} Kirchhofe zu kichern. »He,
he, junger Freund und Hosenpauker, nach getaner Arbeit ist gut
ruhn, he he? Könnten auch die Pferdebahn benutzen und weiter
draußen im Grün bei einer kühlen Blonden sitzen und halten sich
doch in der Stadt und gehen mit dem Alten von Aktenberge, dem alten
Spitzbuben Schulze, auf \emph{seinem} Landbesitz spazieren und
genießen den lieblichen Abend! Seltsam, aber – vielleicht nicht
unerklärlich. Ist in der Tat in der jetzigen Zeit was Neues, mal
beim Alten zu bleiben, he he he!«

Und es war in der Tat ein eigentümlicher Ort zum Lustwandeln, von
und auf dem der alte Herr damals sprach und von dem meine junge
Frau eben redete. Ein Kirchhof! Wenn nicht im Mittelpunkte der
beträchtlichen Stadt Berlin, so doch inmitten einer der Vorstädte,
und zwar nicht einer der ältesten! Ein grüner, busch- und
baumreicher Fleck, im Viereck von neuer, modernster Architektur
umgeben und von praktisch zwar noch imaginären, aber in der Theorie
fest auf dem Papier des Stadtbauplans hingestellten Straßenlinien
überkreuzt.

»Stehe auf meinem Schein, mich hier noch begraben lassen zu dürfen
und sie noch dreißig lange Jahre nach meinem Tode ärgern zu können,
die Fortschrittler«, grinste mein Schwiegervater. »Wenn Sie mich
einmal wieder besuchen, will ich ihn Ihnen zeigen, den Schein,
junger Herr, he he, he he. Andere Wertpapiere sind mir im Verlaufe
der Tage so ziemlich abhanden gekommen; aber dies habe ich sicher
in der Schublade hinter Schloß und Riegel, und sein Kurs ist
gestiegen und steigt, steigt – steigt. Ich habe es aber meiner
seligen Frau Mutter versprochen, mich meinerzeit neben ihr zur Ruhe
zu legen. Brave, aber eigensinnige alte Dame, die sich
merkwürdigerweise etwas darauf einbildete, noch einen
Kalkulationsrat, Steuerzahler, Hungerleider und Asthmatikus mehr in
die üble Luft dieser Welt gesetzt zu haben. Wie sie so sanft ruhn,
alle die Seligen, und – es ist mir in der Tat ein Vergnügen, hier
mit Ihnen zu promenieren, jugendlicher Freund, und Sie auf die
Lächerlichkeit mannigfacher Prätensionen des Menschen hinzuweisen.
Rauch ist alles irdische Wesen – und eine der größten
Lächerlichkeiten ist's, daß man hier nicht rauchen soll. Hier! –
Meiner seligen Frau in ihrer ewigen Ruhe war das Reglement an der
Pforte gegen Hunde und Zigarren freilich ganz aus der Seele
geschrieben. Der durfte ich natürlich nicht mit der Pfeife in die
beste Stube kommen und würde es mir also auch hier nicht erlauben,
sondern höchstens kalt rauchen oder lieber das Rohr an das Sofa
stellen oder es am besten ganz vor der Tür lassen.

»O Papa, wie kannst du nur so reden?« pflegte dann Emmy gegen den
Papa dieselbe Redensart zu gebrauchen, welche sie nun so häufig
gegen mich in Anwendung bringt. Mir aber würde es heute nicht das
geringste nützen, wenn ich es noch leugnen wollte, daß es nicht der
skurrile Alte war, dessen philosophischen, moralischen, ethischen
und asthmatischen Expektorationen zuliebe auch ich nur zu gern den
sonderbaren Erholungsplatz zum Frische-Luft-Schöpfen mir auswählte.
Herrn Rechnungsrat Schulzes blondes Töchterlein war's, dem zuliebe
ich kam, und – bei den unsterblichen Göttern – es gibt keinen
Rahmen, der golden genug ist, um mir das Bildchen für alle Zeit
einzulassen und festzuhalten!

Und ein wahres Glück war's, daß nicht jeder das gleiche Interesse
und verbriefte Eigentumsrecht des alten Spitzbuben Schulze an der
unheimlich-gemütlichen Lustwandelbahn besaß und daß die Büsche um
die alten hors de concours gesetzten Grabstellen sehr hoch und
dicht ineinander verwachsen waren und daß Emmy und ich ganz genau
sämtliche Flecke hinter ihnen zu kennen glaubten, wo man sich auch
gegen die Fenster und die Naseweisheit des umliegenden Stadtteiles
gedeckt hoffen konnte. Daß wir bald gern in diesen engen, grünen
Gängen dem Papa den Vortritt ließen und etwas hinter ihm
zurückblieben, vorzüglich an den Wendungen der Wege, ist eine
vergnügliche, wonnige Tatsache. Daß ich für meine Person es nie
gewesen bin, der den Herrn Rechnungsrat in seinen kuriosen
Betrachtungen durch Fragen oder gar den Ruf: So laufen Sie doch
nicht so, werter Greis! unterbrach, ist gleichfalls ein Faktum. Es
war schon störend genug, daß zuerst Emmy mich unterbrach und, das
rosige Mündchen scheu und schämig zurückbiegend, ängstlich
flüsterte:

»O, wie kannst du nur so sein!\ldots{} o bitte! Und gar hier auf dem
Kirchhofe!«\ldots{}

Ja, es ist eine historische Tatsache, daß ich damals so gewesen
bin, und glücklicherweise ändert nichts, was uns in Zukunft noch
begegnen mag, mehr das geringste dran. Und es ist richtig, daß ich
auf jenem Kirchhofe so war, nach welchem Emmy sich heute, während
der Landregen ununterbrochen auf Pfisters Mühle herabrauscht,
süß-schmollend, so sehr und dazu so lieblich schmeichelhaft für
mich zurücksehnt.

Und dessenungeachtet habe ich durchaus keine Lust, den ganzen
heutigen Tag mit ihr dort zuzubringen, welche Lust zu ähnlichem
Verweilen ich auch unter besagten Umständen damals dazu haben
mochte. Wohl fällt ein goldnes Licht, ein wonniglich Glänzen aus
der Zeit unserer jungen Liebe auf jenes Land Lemuria zwischen den
nüchternen Häusermauern und unter den neugierigen Fenstern der sich
ins Unbestimmte ausbreitenden Stadt Berlin; aber wir sind doch
eigentlich nicht nach Pfisters Mühle gekommen, um nach dem
Verbleiben jenes Bildes zu fragen.

Was für ein Gesicht ich zu der letzten Überlegung geschnitten haben
muß, erfuhr ich nicht dadurch, daß ich in den Spiegel sah, sondern
auf eine viel angenehmere Weise. Es fiel nämlich drüben an der
andern Seite des kleinen Tisches der langzackige Batist- oder
Leinwandstreifen in den Schoß, und eine kleine Hand kam über den
Tisch herüber und strich mir über die Stirn, nachdem mich zwei
ihrer Finger an der Nase gefaßt hatten; und Frau Emmy Pfister
geborene Schulze rief:

»Oh, nun guck ihn einer an!\ldots{} Willst du wohl!\ldots{} Daß du mir auf
der Stelle eine andere Miene machst! Das fehlte mir grade noch!
Drei Tage Regen draußen und drei auf deinem Brummbärengesicht sind
sechs, und das solltest du mir selbst jetzt, wo wir schon so lange
miteinander verheiratet sind, nicht antun wollen!« – Und ich tat es
der rechenkundigen Tochter meiner verstorbenen Schwiegermutter und
meines noch recht lebendigen Herrn Schwiegerpapas wahrhaftig nicht
an. Ich zog sofort meinen Stuhl um den Tisch herum an ihre Seite
und legte naturgemäß den Arm um sie; und sie hatte den Kopf an
meine Schulter gelegt, und der Regen regnete immerzu, und wir
ließen ihn glückselig dabei.

»O, wie konntest du nur so sein und denken, daß ich es nicht ganz
genau weiß, wie gut und lieb wir das jetzt hier haben in deiner
Mühle und wie traurig das ist, daß wir es hier nie so wieder haben
können!« flüsterte sie. »Und es ist auch ganz recht von dir, daß du
jetzt im letzten Augenblick noch einmal alles aufschreibst, was du
in ihr erlebt hast, und ich freue mich auch schon auf den Winter in
der Stadt, wo du es mir hoffentlich im Zusammenhang vorlesen wirst,
wenn auch Herr und Frau Asche dabeisein werden; aber ein klein,
klein bißchen mehr könntest du wirklich wohl jetzt mit mir darüber
reden, wo ich allein bei dir bin und wir alles rundum so himmlisch
behaglich und melancholisch für uns allein haben. Ob es dabei
regnet, schneit oder ob die Sonne scheint, das ist mir ganz
einerlei, du alter, scheußlicher Langweiler!«

Das liebe Wort oder vielmehr die reizende Strafpredigt des Kindes
hatte ihre Berechtigung; aber an »jenem Tage« hatte sie nur die
Wirkung, die das Buch Galeotto beim scheußlichen alten Langweiler
Dante Alighieri auf seinen Paul Böskopf aus Rimini und sein
zärtlich Fränzchen von Mehlbrei aus Ravenna ausübte. Wir fanden
etwas Besseres zu tun, als einander gegenüber oder nebeneinander zu
lesen, Putzmacherei zu treiben oder gar närrisches Zeug für den
Winterofen zu Papier zu bringen. Aber sein Recht und seinen Willen
bekam das liebe Herz zwischen gutem und schlechtem Wetter, zwischen
Tagen und Nächten, im Hause und draußen, unter den Gartenbäumen an
den stillen Tischen, unter den Weiden den Bach entlang, auf den
Wiesen und zwischen den Ährenfeldern. Ich habe es meiner Frau
ziemlich genau von Mund zu Ohr erzählt, was ich zwischendurch denn
doch auch auf diesen Blättern für den möglichen Winter meines
Lebens an lustigen und traurigen, tröstlichen, warnenden,
belehrenden Erinnerungen in meines Vaters Mühle dauerhaft in
bleibenden Bildern in goldenem Rahmen zusammensuchte und -trug.

\begin{verse}
Daß man der Dornen acht?,\\
Das haben die Rosen gemacht.
\end{verse}

\section{Siebentes Blatt}

\zusatz{Da trippelten den Bach entlang\\ Gar wunderliche Gäste}
heißt es in dem Liede, und zwar »bei Sonnenuntergang«, wie es in
demselben wunderlichen Liede heißt. Mir lag freilich noch die volle
Morgen- und Mittagssonne auf meines Vaters Hause und der Umgegend,
während um den Vater selbst die Schatten schon wuchsen. Aber es war
noch mein Recht, keine Ahnung davon zu haben oder doch nicht darauf
zu achten; ich habe noch nach der glücklichen Kindheit eine
glückliche Jugend in Pfisters Mühle gehabt und würde Bände
schreiben müssen, um ihr auf literarischem Wege gerecht zu werden,
und da könnte am Ende auch das Publikum wie meine Frau kommen und
fragen: »Wozu?«

Wenn es nur nicht gar zu verlockend wäre, von jenen Epochen zu
plaudern, zu den Zeitgenossen, zu der Frau, zu jedem beliebigen
ersten besten, der darauf hören mag, weil er seinerseits auch davon
zu reden wünscht und uns am Munde hängt, weil er mit zappelndem
Verlangen drauf paßt, uns endlich das Wort in dieser Hinsicht davon
abzufangen!

Nachdem ich die erste Stufe meiner wissenschaftlichen Bildung, die
vertraulichen gelehrten Unterhaltungen im Hinterstübchen mit
A.~A.~Asche hinter mir hatte, betrat ich die zweite Staffel der
Leiter. Auch die Herren vom städtischen Gymnasium besuchten
Pfisters Mühle, die älteren mit meistens zahlreicher Familie, die
jüngern neben der jungen Frau mit wenigstens einem Kinderwagen voll
und nur die jüngsten ohne Anhang und höchstens mit ihrem Ideal im
Herzen. Gewöhnlich am Mittwoch- und am Sonnabendnachmittag kamen
sie und bildeten dann an einem der längsten Tische des Gartens eine
große Familie, und eines schönen Mittwochnachmittags stellte einer
aus derselben, und zwar sogar das würdige Oberhaupt, der
weißlockige Patriarch, nämlich Direktor Pottgießer, aus blauer Luft
eine Art von kursorischem Examen mit mir an, dem mein Vater, mit
sämtlichen Schoppen der jüngern Kollegen in bunter Reihe leer auf
dem Tische, atemlos lauschte und dessen Resultat das Wort aus dem
Munde des gemütlichen Schultyrannen war: »Schicken Sie ihn mir zu
Michaelis, Pfister.«

Und zu Michaelis wurde ich ihm geschickt: das heißt, Vater Pfister
von Pfisters Mühle führte seinen zu einem höhern Ziel (das heißt
einem andern, als auch Vater Pfister aus Pfisters Mühle zu werden)
bestimmten Sprößling zu einem andern, mehr förmlichen und in die
Dinte und aufs Papier verlaufenden Examen in die Stadt. Das
Resultat hiervon war, daß ich nicht ein Stück Kuchen aus der
Handtasche der Frau Direktor Doktor Pottgießer wie beim ersten
bekam, sondern nur, daß mich der Doktor einen »mit wunderlichen
Allotriis vollgepfropften Tironen« nannte, mich aber doch in die
seiner wackern Obhut anvertraute Herde germanischer
Zukunftsgelehrtheit aufnahm und mich dem »passenden Pferch junger,
in gleichen Tritt zu bringender Böcke« zuwies, wie A.~A.~Asche sich
ausdrückte.

Ich bekam einen Platz in der Quinta, und mein Vater, der sein
ganzes liebes Leben durch in seinen Ansprüchen bescheiden war und
ein dankbar Gemüt dazu hatte, begabte zum Lohn für seinen Erfolg
meinen und seinen Privatgelehrten mit einer soliden silbernen
Taschenuhr, welchen höchst überflüssigen Zeitmesser Asche bereits
gegen Ende des laufenden Mondes nach dem Pfandhause trug und vor
dem Ablauf des Jahres für immer gegen »andere Werte und momentan
Nützlicheres« vertauschte. Daß er so ziemlich um diese Zeit seine
Studien, oder wie die Leute (nicht er!) es sonst nannten,
vollendete, rufe ich dazu mit einiger Schwierigkeit in die
Erinnerung zurück. Was er eigentlich studiert hatte, konnte kein
Mensch recht sagen und er selber vielleicht auch nicht.
Naturwissenschaften hieß es offiziell, und mit der Natur stand er
freilich auf bestem Fuße, legte sich aber noch lieber an schönen
Tagen, so lang er war, in dieselbe hin, mit den Händen unter dem
Kopfe und einer Zigarre oder kurzen Holzpfeife zwischen den Zähnen.
Wovon er in dieser Zeit lebte, das wußte außer den Göttern und
meinem Vater niemand; aber er lebte und wurde eines Tages auch
Doktor der Philosophie, und ich habe später die unumstößliche
Gewißheit aus verschiedenen Papieren in Pfisters Mühle gewonnen,
daß dieses gleichfalls nur unter Mitwissen und Beihülfe meines
Vaters und der Unsterblichen möglich gemacht worden war.

»Ich habe seinen Vater gekannt«, pflegte mein Vater zu sagen. »Der
war ähnlich und ist bis an seinen Tod mein bester Freund gewesen,
und es war schade, schade um ihn! Und wenn er von seines Berufs
wegen als Schönfärber sich auch die Welt für sein Fortkommen in ihr
ein bißchen zu hübsch gefärbt hat, so ist doch kein anderer Mensch
als er selber und höchstens sein Junge dabei zu Verdruß gekommen,
und der – deinen Doktor meine ich – der soll's in meinen Augen
nachträglich nicht auch noch entgelten. Dazu hat er mir zu viel
innerlich von seinem Alten, meinem guten Freund, seinem seligen
Vater. Und daß sein Umgang und seine Belehrung dir keinen Schaden
getan haben, das mußt du allgemach jetzt schon selber einsehen und
sagen können, Ebert.«

Und ob ich das schon selber einsah!\ldots{} Was ich damals aber noch
nicht wußte, war, daß ich es später sogar in meines Vaters Haus-
und Wirtschaftsbüchern finden sollte, wieviel Nutzen mein Freund
Adam Asche Pfisters Mühle schaffen konnte. A.~A.~Asche hat diese
Bücher jahrelang geführt in dem Hinterstübchen; und wäre der
Niedergang des guten, vergnüglichen Erdenflecks durch genaue
Buchhaltung zu verhindern gewesen, so würde heute wohl kein ander
Bild drüber hingemalt werden und würde der nüchterne Alltag um eine
grüne, lustige Feierabendstelle reicher geblieben sein für
\emph{die} Gegend.

Aber es hat alles seine Grenzen, und so hatte es auch das Zutrauen
meines Vaters in seinen Günstling.

»Nicht weiter als soweit ich ihn unter Augen haben kann«, meinte
der Alte. »Und daß ich dich ihm in der Stadt allein und
unbeaufsichtigt in die Pfoten oder nur in Kost und Wohnung geben
könnte, davon ist gar keine Rede. In einem von der Sorte hat die
Welt grade genug, und daß du, mein Sohn, dich unter seiner
speziellsten Obhut zur Anwartschaft auf den zweiten von der Art
herausbilden solltest, das paßt mir doch nicht ganz in die Mütze.«

Wo in seiner »grünen Salatzeit« Studiosus und Doktor Asche selber
seine Kost entnahm, war freilich etwas unbestimmt, und die
sonderbarsten Spelunken schienen ihm manchmal grade recht zu sein.
Was seine Wohnung anbetraf, so wechselte er häufig mit derselben,
und sie gehörte meistens zu den beschränktesten und erfreute sich
nicht immer der besten Luft und der erquicklichsten Aussicht. Am
liebsten hielt er sich in dieser Hinsicht wie in so mancher andern
in der Höhe, und ich habe ihn heute im Verdacht, daß er's in jener
vergnüglichen Zeit Mauernstraße Numero neunzig nur deshalb länger
als ein Jahr lang aushielt, weil er von seinem dortigen Fenster die
Hintergebäude der moralisch anrüchigsten Gasse der Stadt mit all
ihrem Leben und Treiben zum nachdenklichsten Zeitvertreib vor und
unter sich hatte.

Aber es war noch ein triftiger Grund vorhanden, der ein
Zusammenhausen mit ihm nicht bloß für mich, den Schulknaben,
sondern für jedermann sonst unmöglich machte. Er war zu häufig
nicht zu finden!\ldots{} Man vermißte ihn wochenlang im Kreise seiner
Freunde, und er blieb mondenlang für seine Hoflieferanten und
sonstigen Gönner und Gläubiger jenseits seines nächstumfriedeten
Wohnbezirks verschollen. Einmal ist er sogar länger als zwei Jahre
verreist gewesen.

Als er von dieser letzten Fahrt – einer wahren Weltfahrt, wie es
sich nachher auswies – von neuem im Lande erschien, war ich bereits
einer der verständigeren jüngeren Leute des Schulrats Pottgießer,
im Besitz eines Rasiermessers und des dazu gehörigen, glücklichen,
unverwüstlichen körperlichen und wissenschaftlichen Selbstgefühls,
zugleich mit der unvertilgbaren Neigung, noch andere
Wirtschaftsgärten als den von Pfisters Mühle, sowie allerhand
sonstige Kneipen zu besuchen. Ich war Primaner des löblichen
städtischen Gymnasiums und hatte schon mehr als eine erste Ahnung
davon, daß es eine Täuschung des Menschen ist, wenn er glaubt, daß
die Bilder der Welt um ihn her stehenbleiben. Und wie der Junge aus
Pfisters Mühle, so war auch das ganze deutsches Volk ein anderes
geworden; denn die Jahre achtzehnhundertsechsundsechzig und
-siebenzig waren ebenfalls gewesen und man zählte, rechnete und wog
Soll und Haben mit ziemlich dickem, heißem Kopfe so gegen die Mitte
der Siebenziger heran.~–

»Und das ist ein wahres Glück«, meinte Emmy, »hoffentlich kommen
wir jetzt endlich mehr zu Frau Albertinens Geschichte. Nimm es mir
nicht übel, Männchen, Freund Asche interessiert natürlich als dein
Freund auch mich ungemein, was seine Gelehrsamkeit und seine
nachlässige Toilette, seine Naseweisheit und seine Unruhe und
ewiges Umhertreiben in der Welt anbetrifft, aber auf seine
Liebesgeschichte bin ich doch am gespanntesten. Bis jetzt ist es
mir ein komplettes Rätsel, wie die beiden Leutchen zusammenkommen
konnten. Ich versetze mich ganz in ihre Lage und denke, zuerst muß
es sie doch schrecklich frappiert haben, als sie einander zum
ersten Male gegenseitig zu Gesicht bekamen. Du wirst natürlich
sagen daß wir hier ja in Pfisters Mühle sind und daß es eben ein
verzauberter Grund und Boden ist. Und wenn ich diesen Mondschein
ansehe, wie er so silbern durch die Baumzweige fällt und auf dem
Wasser, dem Gebüsch und dem Erdboden tanzt, und wenn ich mir
überlege, daß es auch damals wohl ebenso nette und warme Nächte gab
und daß Ehen im Himmel geschlossen werden und des Menschen Wille
sein Himmelreich ist und daß wir armen Mädchen nur allzuleicht vor
euch Übeltätern in Rührung und Aufopferung geraten und die
Contenance verlieren, so brauche ich eigentlich gar nicht an
Zauberei und Verzauberung zu glauben, sondern kann mich ganz
einfach an meine eigne klägliche Geschichte halten, du Bösewicht,
und wie du am hellen Mittag und beinahe vor aller Leute Augen die
Unverfrorenheit hattest~–«

»Die Sache endlich zwischen uns ins reine zu bringen und den Papa
so romantisch, wie es nur in Berlin möglich war, unter seinen
Gräbern, hinter seinen Taxusbüschen und unter seinem
Lieblings-Eibenbaum damit zu überraschen. Übrigens aber, mein Herz,
habe ich immer nach den besten Mustern mich zu bilden bestrebt:
dort auf des Papas Friedhofe hielt ich mich an das treffliche
Beispiel A.~A.~Asches, und in diesem Augenblicke schwebt mir Vater
Joachim Heinrich Campe als nachahmungswertes Exempel vor. Der brach
unter seinem Apfelbaum in seinen Historien von Robinson dem Jüngern
und seinem treuen Freund Freitag stets dann ab, wenn's in ihnen
›interessanter‹ wurde. Wie er, schlage ich vor: indem wir uns auf
unser eigenes, sicheres Lager strecken, wollen wir unsern freudigen
Dank dem guten Gotte bringen, der uns in einem Lande geboren werden
ließ, wo wir unter gesitteten, uns liebenden und helfenden Menschen
leben und nichts von wilden Unmenschen zu befürchten haben.«

»Lieber Himmel, was soll denn das nun wieder bedeuten?« rief Emmy
näher rückend und ganz bänglich nach allen Seiten in die nicht vom
Monde erhellten Gebüsche des verlassenen Gartens von Pfisters Mühle
scheue Blicke werfend. »Meinst du wirklich nicht, daß es hier, und
vorzüglich bei Nacht, doch ein bißchen zu einsam und zu weit
entlegen vom Dorf und andern Leuten ist?«

»Nichts meine ich, als daß morgen wieder ein schöner Tag wird und
daß, da uns die Tage auf Pfisters Mühle nur zu genau zugezählt
sind, wir uns die letzten nicht durch den Nachttau und den öfters
darauf folgenden Schnupfen verderben lassen wollen.«

»Jawohl«, meinte Christine, die seit einiger Zeit nach vollbrachten
Hausgeschäften am Tische gesessen hatte, »jawohl, ich denke auch,
daß es allmählich Zeit wird, zu Bette zu gehen, obgleich ich für
mein Teil Sie in alle Ewigkeit so erzählen hören könnte, Herr
Ebert. Es wird einem immer so kurios dabei, und je näher die Zeit
zum Abzug kommt, immer wehmütiger. Und wissen möchte ich grade in
diesem Augenblick, wie es Samse geht und ob er nicht bei diesem
Mondenschein nach Pfisters Mühle zurückdenkt! Ach Gott, ach
liebster Herrgott, und wie wird's mir sein, wenn auch ich in den
allernächsten Tagen schon hierher nur noch zurückdenken kann und
alles ist, als ob alles gar nicht gewesen wäre!«

\section{Achtes Blatt}

\zusatz{Wie es anfing, übel zu riechen in Pfisters Mühle.}
»Es ist Schnee in der Luft!« sagten die Leute und hatten
ausnahmsweise einmal vollkommen recht. Es war Schnee in der Luft,
und bald nach Mittag kam er sogar in einzelnen Flocken herunter und
zeigte sich zum erstenmal im Jahre unserm Stück Erde, und die Leute
darauf taten sich einiges darob zugute und fragten einander: »Haben
wir es nicht gesagt?«

Es war kurz vor den Weihnachtsferien im letzten Semester meines
Schülerlebens, und nie hatte mich der erste Schnee eines Winters in
gleich träumerischer Stimmung, ihn zu würdigen, zu empfinden,
gefunden wie dasmal. In gemütlicher Faulheit mit dem Kinn auf
beiden Fäusten in der Fensterbank zu liegen und in die trübe Luft
und auf die verschleierten Dächer zu starren und an dem Schulrat
Pottgießer, Pfisters Mühle und dem demnächstigen vir juvenis und
Studiosus der Philosophie Ebert Pfister bei diesem ersten Schnee zu
gleicher Zeit sein Behagen haben zu können, das war etwas, was bis
jetzt noch nicht dagewesen war, und ich genoß es ganz und gar und
zu allem übrigen eingehüllt in ein Gewölk nicht übeln Knasters.

Wenn ich mich wendete, lag die Stube in gleicher Dämmerung, im
gleichen Nebel wie die Gasse und die Dächer draußen. Wenn ich aus
einer Ecke der Bude zur andern querüber den langjährig gewohnten
Denkerpfad schritt, lebte und wogte es umher von Gestalten der
Vergangenheit und Genien der Zukunft, und – der Mensch ist nur
selten, selten so alt und so jung zu gleicher Zeit, wie in solchen
germanischen Zwischenlichtstunden, gleichviel mit welchem Datum er
im Kirchenbuche oder in der Standesamtsliste eingetragen sein mag!

Vor allem war es natürlich die nahe weihnachtliche Ferienzeit in
der Mühle, die ich in dieser Stunde vorkostete. Es war immer,
solange ich wenigstens zu denken vermochte, gut gewesen,
Weihnachten unter dem väterlichen Dach, Weihnachten in Pfisters
Mühle zu feiern und das neue Jahr darin anzufangen; aber so viel
Wohlbehagen wie diesmal hatte ich mir eigentlich noch nie davon
versprochen und in der Phantasie ausgemalt. Rechenschaft darüber
wußte ich mir nicht zu geben und gab mir auch keine Mühe, nach
Gründen dafür zu suchen.

Wie oft aber geschieht es im Leben, daß in dergleichen gute
Stimmungen ein Laut hineinklingt, ein Schritt auf der Treppe, ein
Klopfen an der Tür, die dem gemütlichen Träumer die Laune
vollkommen verderben würden, wenn er gleich wüßte, was sie für den
morgenden Tag, die nächste Woche, das folgende Jahr und so weiter
zu bedeuten hätten?

Diesmal aufhorchend vernahm ich einen gar wohlbekannten Fußtritt im
schweren Stiefel treppauf tappend draußen und ein Schnaufen und
Räuspern, das ich nie auf den Pfaden dieser Erde mit einem andern
verwechseln konnte, und so rief ich:

»Alle Wetter, das ist ja der Alte? Was will denn der Alte heute
noch und so spät am Tage in der Stadt?«

Ich kannte seinen Schritt, seinen Husten und sein Räuspern. Aber er
hatte noch eine andere Gewohnheit an sich: er sang stets, wenn er
eine Treppe stieg, vor sich hin; Pfisters fröhlicher Mühlengarten
schien immer mit ihm aufwärts zu steigen. Diesmal aber war dem
nicht so.

Weder einen Endreim aus einem Liede seiner Herren Studenten, noch
ein Stück vom Repertorium einer der vielen Sangesverbrüderungen der
Stadt, die sein Lokal allen übrigen zu ihren intimsten
Festlichkeiten vorzogen, brachte er heute mit die Treppe herauf.

»Was ist denn das?« murmelte ich, als ich ihm die Tür öffnete, um
ihn schon auf dem dunkeln Vorplatze in Empfang zu nehmen und zu
begrüßen.

Es war sehr dunkel bereits auf diesem Vorplatze, und Gaserleuchtung
gab es im Hause nicht. Der Alte hatte noch einige Stufen der
steilen Treppe zu erklimmen, und es schien mir, als mache das ihm
mehr Beschwerde als früher. Er atmete jedenfalls schwer dabei und
schnappte längere Zeit nach Luft, nachdem ich ihm die Hand gereicht
und ihn vollends emporgezogen hatte.

»Pfui Teufel!« rief er, nachdem er die Luft des Hauses noch einmal
mit gekrauster Nase geprobt hatte. »Auch eine angenehme Atmosphäre!
Nur um eine Idee lieblicher als Pfisters Mühle – der Satan weiß es.
Guten Abend, Junge.«

»Guten Abend, Vater«, sagte ich lachend. »Will der alte Sünder
seinen Sprößling ob der Wohlgerüche Arabiens, in die er ihn
gepflanzt hat, gar noch verhöhnen? Was kann denn dein Kind dafür,
daß Mutter Müller mit Käse, Heringen und Schellfisch aus zweiter
Hand handelt, daß Mutter Pape ihre Kinderwäsche wahrscheinlich zu
nah an den Ofen gehängt hat, daß Jungfer Jürgens heute mittag eines
kleinen Zwistes mit Schneider Busch halben ein wenig nachlässig mit
ihrem Sauerkraut auf dem Petroleumkocher umgegangen ist und daß
Meister Busch hinten hinaus soeben einen ziemlichen Teil der
Sonntagsgarderobe der Nachbarschaft auf Benzin traktiert? Na, komm
herein, Vater Pfister! Unter allen Umständen bringst du den neuen
Winter mit, also mach mir auch auf der Stelle dein gewohntes
vergnügtes Gesicht dazu und verkünde beiläufig, was dich eigentlich
zu so ungewohnter Stunde herführt.«

Ich hatte ihn in meinem Scholarenstübchen. Er saß in dem
Sorgenstuhl des Seligen der Witib, bei welcher er mich in Wohnung
und allerlei andere Verpflegung getan hatte. Hut und Stock hatte
ich ihm abgenommen und den wollenen Schal ihm vom Halse
abgewickelt. Einen Überrock hatte er nie getragen, und jetzt
knöpfte er kopfschüttelnd, dem Winter, den er mitgebracht hatte,
zum Trotz, die Weste über der breiten Brust und dem stattlichen
Bäuchlein auf, rang noch einige Zeit nach mehr Atem und sprach:

»Jaja, mein Junge, nur noch einen Augenblick\ldots{} das Fenster laß nur
zu; es kommt nichts Besseres herein, als hinausgeht. Jaja, in
Veilchen, Rosen und Hyazinthen bist du freilich hier nicht
gebettet, und so will ich auch nichts dagegen einwenden, daß du
dich auch wieder mal an meinen besten Varinas, wie ich merke,
gehalten hast, um dir die Lüfte zu verbessern. Es ist bei dir doch
nur ein Übergang in deinen jungen Jahren; aber ich bin zu alt dazu.
Ich halte es nicht länger aus, mich, ohne mich dagegen zu rühren,
zu Tode stänkern und stinken zu lassen, und heute ist dem Faß der
Boden ausgefallen, und du brauchst mich nicht so dumm anzustieren:
ich bin darum in der Stadt, und wenn es eine Wissenschaft und
Gerechtigkeit gibt, so soll sie jetzt für uns zwei – Pfisters Mühle
und mich – eintreten, oder wir schließen beide das Geschäft, sie
und ich, und für mich mag es ja wohl der beste Trost sein, daß du
dich nicht darum zu kümmern hast, sondern für was anderes auf
Schulen und Universitäten vorbereitet bist, grade als ob ich eine
Ahnung davon gehabt hätte, als ich dich aus der freien Luft
hereinrief und an die Bücher setzte und Doktor Aschen über dich!«

»Lieber Vater –«

»Jawohl, mein Sohn, wie dein lieber Vater es dir sagt, so verhält
es sich. Samse hat im Blauen Bock ausgespannt, und ich bin hier
vorhanden, um der Sache auf den Grund zu kommen, oder mit Ergebung
das Rad zu stellen und unser Schild einzuziehen. Können sie
Pfisters Mühle in der Welt nicht mehr gebrauchen, haben sie genug
von ihr, nun so muß es mir, ihr und dir am Ende ja wohl egal
sein.«

»So leicht geben wir und die Welt Pfisters Mühle doch wohl nicht
auf, Vater!«

»Das sage ich mir ja auch in jedweder schlaflosen Nacht, Ebert;
aber was kannst du am Ende noch weiter tun, als daß du dich bis
aufs äußerste wehrst, dir in der Mühlstube die Nase zuhältst, nur
an dein Handwerksgeschäft denkst und denkst: Freunde, Herrschaften,
gute Gevattern hin und her, was tut's, wenn sie dir ausbleiben,
Alter? Am Ende bist du doch von Rechts wegen eigentlich mehr ein
Müller als ein Krugwirt, und solange sich dir das Rad dreht, hast
du noch nicht den richtigen Grund, deinen Herrgott wegen
Ungerechtigkeit anzuklagen. Aber wenn sie dir auch in der Mühlstube
aufwerfen und sprechen: ›Meister Pfister, daß Sie uns recht sind,
das wissen Sie; aber aushalten tut das bei Ihnen keiner mehr, der
Parfüm ist zu giftig!‹ Was dann?«

»Deine Leute haben dir gekündigt?«

»Bis auf Samse, und den sehe ich immer nur darauf an in stiller
Verwunderung und zerbreche mir den Kopf über die Frage, ob er aus
Dummheit oder Anhänglichkeit bleibt. Ja, sie haben allesamt außer
ihm ihre Kräfte in Nase und Lunge taxiert und sind zu dem Beschluß
gekommen, daß sie über Weihnachten und Neujahr wohl noch reichen
müßten, aber daß sie zu Ostern komplett damit zu Ende seien. Sie
gehen alle zu Ostern von Pfisters Mühle!«

»Zum Teufel auch! Der Henker soll sie holen!«

»Fluche nicht, mein Sohn«, sprach der alte Herr, melancholisch den
Kopf schüttelnd. »Du bist seit vierzehn Tagen nicht draußen gewesen
und hast schon bei deinem letzten Aufenthalt und Besuch genug
geflucht.«

»Und es ist seitdem noch schlimmer geworden?«

Der Alte erhob sich aus seinem Stuhl, weitbeinig stellte er sich
fest, beide Hände in die Seiten stemmend. Sechsmal blies er aus
vollen Backen vor sich hin und schlug dann mit voller Faust auf
mein Schreibpult, daß rundum das ganze Gemach zitterte, und so
keuchte er wütend:

»Der lebendige Satan soll mich frikassieren, wenn ich für mein Teil
es bis zum Heiligen Christ aushalte! Sie haben am Ende
Anhänglichkeit an mich und prätendieren es also ein bißchen länger;
aber was kann ich denn noch an mir haben bei so bewandten
Zuständen?\ldots{} Ob es ärger geworden ist?\ldots{} Bücher könnte man
darüber schreiben und soll es auch, wenn ich was dazu kann! Die
besten alten Freunde und urältesten treuen Stammgäste – gelehrte
und ungelehrte – gucken nur noch über die Hecke oder in das
Gartentor seit Mitte vorigen Monats oder klopfen höchstens ans
Fenster vom Klubzimmer und sagen: ›Mit dem besten Willen, es seht
nicht länger, Vater Pfister; das bringt kein Doppelmops, kein
Kardinal, kein Pariser Numero zwei, keine Havanna und kein Varinas
oder sonstig Kraut in keiner Nase und Pfeife mehr herunter, dieser
Gestank kriegt alles tot! Und wenn wir es auch aushielten, Pfister,
so will man doch des Sonntags auch gern seine Damens mit
herausbringen und es frißt uns das Herz ab, aber – sie danken,
sobald wir Sie jetzt in Vorschlag bringen, alter Freund. Unsre
Weibsleute, die doch sonst von Gottes und Natur wegen jeglichen
übeln Geruch in der Welt am besten ausdauern können, werden von
einem einzigen Nachmittag bei Ihnen, Meister Pfister, ohnmächtig,
verlangen unterwegs auf dem Heimwege eine Droschke und räsonieren
die ganze nächste Woche; und so nehmen Sie es uns wohl nicht übel,
Pfister, wenn wir am Ende nur können, wie wir müssen, Ihnen
vorbeipassieren und unsere Unterkunft bei der Konkurrenz im Dorfe
suchen, bis die Lüfte bei Ihnen wieder reiner sind. Sie sollten
aber wirklich sich da recht bald mal an den Laden legen, die
Konkurrenz und der üble Geruch verdirbt überall leider Gottes nur
zu rasch das allerbeste Geschäft.‹«

Der Alte setzte sich wieder, und ich klopfte ihm zärtlich und so
beruhigend als möglich den braven, breiten Rücken; aber schwer
war's in der Tat, einen Trost für ihn zu finden. Ich kannte ja die
jetzigen Düfte um und in Pfisters Mühle selber nur zu gut, und
wußte, daß sie alle vollkommen recht hatten, der Meister Müller und
seine Knappen wie seine Gäste. Es war schwer auszuhalten für einen,
der's nicht unbedingt nötig hatte, es zu ertragen.

»So bin ich nun jetzt hereingekommen, um mich an den Laden zu
legen«, seufzte der Vater. »Die Herren Studiosen sind und bleiben
mir zwar allewege eine Ehre und ein Vergnügen; aber wenn sie nicht
ausbleiben, so pumpen sie mir doch alleweile ein bißchen zu arg auf
den odeur de Pfister hin, wie sie sich ausdrücken. Von den Bauern
habe ich nur noch diejenigen, so am wenigsten zahlungsfähig sind,
und so – wenn der Mensch sich gar nicht mehr zu helfen weiß, dann
geht er eben zum Doktor, und dieses werde ich jetzt auch besorgen,
Ebert.«

»Zum Doktor?« fragte ich in einiger Verwunderung.

»Jawohl! Er ist ja wohl wieder im Lande, und wenn ein Mensch sich
vor keinem Stank in der Welt fürchtet, so ist er das. Und er kriegt
sein Stübchen im Oberstock und seine Verpflegung, bis er's
herausgebracht hat, was mir mein Wasser, meine Räder und alle meine
Lust am Leben so verschimpfiert und schändiert. In der Stadt hat er
ja doch noch immer nicht allzuviel zu verlieren an Wohlleben und an
Liebe und Vertrauen unter den Leuten. Beides soll er aber noch mehr
als sonst schon dann und wann in Pfisters Mühle finden, solange er
sie in der Kur hat. Mein allerletztester Trost ist er! Und er muß
es mir herauskriegen, an wem ich meine Wut auszulassen habe, wem
ich in dieser pestilenzialischen Angelegenheit mit einem Advokaten
zu Leibe steigen kann! Meinen Widerwillen gegen Prozesse kennst du,
Junge; aber den infamen Halunken, der uns dieses antut und mir
meiner Väter Erbe und ewig Anwesen und Leben so verleidet, den
bringe ich mit Freuden an den Galgen. Ein schönes Erbe werde ich
dir an Pfisters Mühle hinterlassen, mein armer Junge, wenn der
Doktor uns gleicherweise wie alle übrigen vor dem Duft ohnmächtig
wird und bleibt!«\ldots{}

Ich hatte sie richtig in den Schlaf erzählt.

Emmy nämlich.

Sie hatte zwar nicht geschworen, mich von meinem »nichtsnutzigen«
Kopfe ganz zu befreien, wenn ich sie diesmal nicht außergewöhnlich
interessieren würde; aber sie hatte mir doch fest versprochen, mich
bei diesem eben bezeichneten Kopfe zu nehmen. Und wie Scheherezade
hatte ich das möglichste geleistet; Schahriar schlummerte süß und
lächelte wie ein Kind in seinem Schlummer.

In Berlin war es noch früh am Tage; aber nebenan in unserm Dorfe
schlug die Kirchuhr schon zehn, und niemand schien dort mehr wach
zu sein als auf den an der Landstraße gelegenen Gehöften einige
Hunde, die über den Zaun ihre Gedanken über ein verspätetes
Wagengerassel oder einige der Stadt zueilende Fußgänger
austauschten.

Ich lächelte ebenfalls. Weniger in Betracht als in Betrachtung
meines unumschränkten Herrschers über Indien mit allen seinen
großen und kleinen Inseln bis an die Grenzen von China – mein Herz
für immer und Pfisters Mühle, solange es sich tun ließ,
eingeschlossen. Das Kind sah in seiner lieblich-ergebenen Hingabe
an mein Erzählertalent – in seinem tiefen, unschuldigen Schlaf zu
reizend aus! Was blieb mir dieser Flut von blonden Locken
gegenüber, die über die hübschen Schultern und die Stuhllehne
rollten, anders übrig, als leise, wie in den Brauttagen, eine von
ihnen, den Locken nämlich, zu fangen und verstohlen einen Kuß
darauf zu drücken? Wozu hat man eine Frau, wenn sie nicht in allem
recht hat – selbst in ihrem Entschlummern bei Mitteilung unserer
kuriosesten vorehelichen Erlebnisse und Betrachtungen a priori und
a posteriori darob?!

»Du brauchst nicht zu denken, daß ich nicht zuhöre, wenn ich auch
einmal die Augen für einen Augenblick zumache«, hatte das Herz
mehrere Male gesagt. »Erzähle nur ruhig weiter; aber eigentlich
begreife ich den seligen Papa nicht so recht. Wir wohnen doch nun
über vierzehn Tage schon hier in deiner verzauberten Mühle; aber so
arg, wie er es eben dir schilderte, ist es doch nicht. Es mag eine
Täuschung von mir sein, weil ich eben selten oder nie aus Berlin
herausgekommen bin; aber die Bäume rundum und die Wiesen drüben und
das Heu duften ganz hübsch, und das Wetterleuchten da hinten ist
auch ganz reizend, wenn nur das Gewitter nicht wieder näher kommt.
Das habt ihr Gelehrten auch noch nicht heraus, warum alle diese
wunderhübschen hundert Tiere, Mücken und Schmetterlinge, sich ihre
Flügel an der Lampe verbrennen wollen, sowie man sie angezündet
hat, und das sage ich dir, auf eine Jagd wie gestern mit der
Fledermaus lasse ich mich nicht wieder ein; mir zittern – noch –
die Glieder, und – es – war sehr unrecht – von – dir~–«

Ich erfuhr es nicht, was sehr unrecht von mir am vergangenen Abend
gewesen war; ich ließ das liebe, seidene Geflecht, auf welches das
geflügelte Nachttier gestern so erpicht gewesen war, leise aus der
zögernden Hand gleiten und legte mich noch einen Augenblick in das
offene Fenster des Oberstocks von Pfisters Mühle und blickte in die
Sommernacht hinein. Eigentlich ist das freilich nicht das richtige
Wort; ich roch vielmehr in sie hinaus und mußte augenblicklich Emmy
vollständig recht geben, wenn sie vorhin den letzten Wirt von
Pfisters Mühle in seiner Verzweiflung und meiner Erzählung gar
nicht begriffen hatte.

\section{Neuntes Blatt}

\zusatz{Wie es eben bei dem Doktor Adam Asche noch viel übler
roch.}
Lieblich düftevoll lag die Sommernacht vor den Fenstern über dem
alten Garten, dem rauschenden Flüßchen und den Wiesen und Feldern.
Ein leiser Hauch von Steinkohlengeruch war natürlich nicht zu
rechnen; aber er genügte doch, um mich bei den gewesenen Bildern
festzuhalten, wenn ich gleich am heutigen Abend nicht mehr meinem
Weibe davon weitern Bericht gab.

Es war eben ein Herbst- und Wintergeruch, den weder die dörflichen
und städtischen Gäste, noch die Mühlknappen und die Räder und mein
armer, fröhlicher Vater ihrerzeit länger zu ertragen vermochten.
Und die Fische auch nicht –
\emph{jedesmal, wenn der September ins Land kam}.

Damit begann nämlich in jeglichem neuen Herbst seit einigen Jahren
das Phänomen, daß die Fische in unserm Mühlwasser ihr Mißbehagen an
der Veränderung ihrer Lebensbedingungen kundzugeben anfingen. Da
sie aber nichts sagten, sondern nur einzeln oder in Haufen, die
silberschuppigen Bäuche aufwärts gekehrt, auf der Oberfläche des
Flüßchens stumm sich herabtreiben ließen, so waren die Menschen
auch in dieser Beziehung auf ihre eigenen Bemerkungen angewiesen.
Und ich vor allem auf die Bemerkungen meines armen seligen Vaters,
wenn ich während des Blätterfalls am Sonnabendnachmittag zum
Sonntagsaufenthalt in der Mühle aus der Stadt kam und den Alten
trübselig-verdrossen, die weiße Müllerkappe auf den feinen grauen
Löckchen hin- und herschiebend, an seinem Wehr stehend fand:

»Nun sieh dir das wieder an, Junge! Ist das nicht ein Anblick zum
Erbarmen?«

Erfreulich war's nicht anzusehen. Aus dem lebendigen, klaren Fluß,
der wie der Inbegriff alles Frischen und Reinlichen durch meine
Kinder- und ersten Jugendjahre rauschte und murmelte, war ein träge
schleichendes, schleimiges, weißbläuliches Etwas geworden, das
wahrhaftig niemand mehr als Bild des Lebens und des Reinen dienen
konnte. Schleimige Fäden hingen um die von der Flut erreichbaren
Stämme des Ufergebüsches und an den zu dem Wasserspiegel
herabreichenden Zweigen der Weiden. Das Schilf war vor allem übel
anzusehen, und selbst die Enten, die doch in dieser Beziehung
vieles vertragen können, schienen um diese Jahreszeit immer meines
Vaters Gefühle in betreff ihres beiderseitigen Hauptlebenselementes
zu teilen. Sie standen angeekelt um ihn herum, blickten
melancholisch von ihm auf das Mühlwasser und schienen leise
gackelnd wie er zu seufzen:

»Und es wird von Woche zu Woche schlimmer, und von Jahr zu Jahr
natürlich auch!«

»Sieh dir nur das unvernünftige Vieh an, Ebert«, sagte der Alte.
»Auch es stellt die nämlichen Fragen an unsern Herrgott wie ich.
Experimentiert er selber so schon damit im Erdinnern, na, so kann
man ja wohl nichts dagegen sagen und muß ihn machen lassen; denn
dann wird er's ja wohl wissen, wozu es uns gut ist. Aber –
vergiften \emph{sie} es, da weiter oben, in nichtsnutziger
Halunkenhaftigkeit \emph{ihm} und mir und uns, na, so müßte er denn
wohl am Ende mit seinem Donner dreinschlagen, wenn nicht
meinetwegen, so doch seiner unschuldigen Geschöpfe halben. Guck, da
kommen wiederum ein paar Barsche herunter, den Bauch nach oben; und
daß man einen Aal aus dem Wasser holt, das wird nachgrade zu einer
Merkwürdigkeit und Ausnahme. Kein Baum wird denen am Ende zu hoch,
um auf ihm dem Jammer zu entgehen; und ich erlebe es noch, daß
demnächst noch die Hechte ans Stubenfenster klopfen und verlangen,
reingenommen zu werden, wie Rotbrust und Meise zur Winterszeit. Zum
Henker, wenn man nur nicht allmählich Lust bekäme, mit dem warmen
Ofen jedwedes Mitgefühl mit seiner Mitkreatur und sich selber dazu
kalt werden zu lassen!«\ldots{}

»O, ich habe alles gehört«, sagte Emmy. »Erzähle nur ruhig weiter;
ich höre alles. Es ist bloß ein Erbteil von meinem armen Papa, wenn
den etwas sehr interessierte, was Mama erzählte, und er in seiner
Sofaecke saß, und Mama grade wie du sagte: ›Kind, wozu rede ich
denn eigentlich?‹ – Er wußte nachher so ziemlich alles, wovon die
Rede gewesen war, wenn er auch mit geschlossenen Augen darüber
nachgedacht hatte. Und du brauchst mich nur zu fragen, lieber
Ebert, ob ich dir nicht auch alles an den Fingern aufzählen kann
von dir und den Fischen in Pfisters Mühle – nein, von Pfisters
Mühle und deinem Papa und den Enten und allem übrigen, den
Studenten und den Gästen aus der Stadt, und wie alles so sehr übel
roch jedesmal, wenn seit dem Kriege mit den Franzosen und dem
allgemeinen Aufschwung der Herbst kam. Und eben hatten die Leute
schon gesagt: ›Es ist Schnee in der Luft!‹ und du saßest in deiner
Schülerstube am Fenster und wartetest drauf, und da war dein Papa
in die Stadt gekommen, und ihr hattet wieder von den
entsetzlichsten Gerüchen euch unterhalten, daß es einem allmählich
ganz unwohl dabei wird. Siehst du wohl, ich weiß alles ganz genau,
und zuletzt waret ihr grade in euerer äußersten Verzweiflung auf
dem Wege zum Doktor Asche, und das ist eigentlich mehr, als du von
mir verlangen kannst, denn du hattest seinen Namen noch durchaus
nicht genannt; ich habe es mir aber gleich gedacht, auf wen die
Sache hinauslief.«

»Ein Prachtmädchen bist du und bleibst du!« stotterte ich ein wenig
verwundert und in einigem Zweifel darob, wieviel eigentlich unser
Herrgott den Seinigen im Traum zu geben vermag. Aber einerlei,
woher das liebe Seelchen es hatte; es war seinem eigenen Ausdruck
zufolge vollkommen au fait und blieb helläugig und munter und
schlauhörig bis weit über Mitternacht hinaus.

Ein Grund zur Eifersucht war gottlob nicht vorhanden; aber es gab
glücklicherweise außer mir keine andern Individuen innerhalb und
außerhalb meiner Männerbekanntschaft, die mein Weib so ausnehmend
interessierten wie Doktor A.~A.~Asche und so gut Freund mit ihr
waren wie derselbige Herr, Weltweise und Berliner
Großindustrielle.

»Ja, setz deine Mütze auf«, sagte mein Vater. »Du kannst mitgehen
und anhören, was seine Meinung ist und ob er auf meine Vorschläge
in Anbetracht euerer Weihnachtsferien und Pfisters Mühle eingehen
will. Es ist mir sogar recht lieb, wenn ich dich als Zeugen habe,
der mir im Notfall dermaleinst vor dem Weltgericht bestätigen kann,
daß ich mein möglichstes getan habe, um deiner Vorfahren uralt Erbe
vor dem Verderben zu bewahren und es vor dem Ausgehen wie Sodom und
Gomorra in Schlimmerem als Pech und Schwefel und in Infamerem als
im Toten Meere zu erretten. Deine selige Mutter, wie ich sie kenne,
stünde schon längst als Salzsäule dran; und in der Beziehung ist es
ein Glück, daß sie das nicht mehr erlebt hat. O du lieber Gott,
wenn ich mir Pfisters Mühle von heute und deine selige Mutter
denke!«

Ich hatte meine selige Mutter nicht gekannt. Ich wußte von ihr nur,
was mir der Vater und Christine von ihr berichtet hatten und immer
noch erzählten, und ich wußte es in der Tat schon, daß sie und
Pfisters Mühle »von heute« nicht mehr zueinander paßten, und daß
ihr, meiner jungen, zierlichen, reinlichen, an die beste Luft
gewöhnten lieben Mutter, viel Ärgernis und Herzeleid erspart worden
war durch ihr frühes Weggehen aus diesem auf die höchste Blüte der
Kunst- und Erwerbsbetriebsamkeit gestellten Erdendasein.

Ich setzte meine Mütze auf und nahm den Arm meines alten, einst so
fröhlichen Vaters. Er hatte mich sorgsam und nach bestem
Verständnis geführt, solange er die alte Lust, das alte Behagen an
seinem Leben hatte. Heute abend auf der steilen Treppe, auf dem
Wege zu unserm beiderseitigen Freunde, Doktor Adam Asche, überkam
mich zum erstenmal die Gewißheit, daß in näherer oder fernerer Zeit
an mir wohl die Reihe sein werde, sorgsam und liebevoll seine
Schritte zu unterstützen. Es war kein kleiner Trost, daß das
lichte, liebe Bild, das er eben durch Erwähnung meiner Mutter
wachgerufen hatte, uns freundlich und ruhig und lächelnd
voranglitt.

Die Witterung draußen war längst nicht so behaglich, wie sie sich
vom Fenster aus ansehen ließ. Der Wind blies scharf, und ich hatte
häufig die Kappe mit der freien Hand zu halten auf dem Wege zu
»unserm Freunde«.

Der pflegte, wie gesagt, häufig mit seinen Wohnungen zu wechseln,
wenn er im Lande war, das heißt, wenn er sich in seiner Vaterstadt
aufhielt. Diesmal hatte er sein Quartier in einer entlegenen
Vorstadt aufgeschlagen, und zwar, wie immer, nicht ohne seine
Gründe dazu zu haben; und ich, der ich, um die Schülerredensart zu
gebrauchen, die Gegend und Umgegend natürlich wie meine Tasche
kannte, hatte zwischen den Gartenhecken und Mauern, den
Gartenhäusern und Neubauten in dem nur hier und da durch eine
trübflackernde Laterne erhellten Abenddunkel mehr als einmal
anzuhalten, um mich des rechten Weges zu ihm zu vergewissern.

Ein enger Pfad zwischen zwei triefenden Hecken brachte uns zu einer
letzten Menschenansiedlung, einem dreistöckigen, kahlen Gebäude,
mit welchem die Stadt bis jetzt zu Ende war und hinter welchem das
freie Feld begann. Aber Lichter hie und da in jedem Stockwerk
zeigten, daß auch dies Haus schon bis unters Dach bewohnt war, und
mancherlei, was umherlag, -hing und -stand, tat dar, daß es nicht
grade die hohe Aristokratie im gewöhnlichen Sinne war, die hier
ihren Wohnsitz aufgeschlagen hatte.

Bei einer halbwachsenen Jungfrau, die in sehr häuslicher
Abendtoilette eben einen Zuber voll Kartoffelschale über den Hof
trug, erkundigte ich mich, ob Herr Doktor Asche zu Hause sei, und
erhielt in Begleitung einer Daumenandeutung über die Schulter die
eigentümliche Benachrichtigung:

»In der Waschküche.«

»Wo, mein Herz?« fragte mein Vater ebenfalls einigermaßen
überrascht; doch ein ungeduldiges Grunzen und Geschnaube aus einer
andern Richtung des umfriedeten Bezirkes nahm das Fräulein so sehr
in Anspruch, daß es nichts von fernerer Höflichkeit für uns übrig
behielt. Zu dem Behälter ihrer Opfertiere schritt die vorstädtische
Kanephore; und wir, wir wendeten uns einer halboffenen Pforte zu,
aus der ein Lichtschein fiel und ein Gewölk quoll, welche beide
wohl mit dem Waschhause der Ansiedlung in Verbindung zu bringen
waren.

»Du lieber Gott, er wird doch nicht – es ist zwar freilich morgen
Sonntag; aber er wird doch nicht jetzt noch sein frisches Hemde
selber drauf zurichten?« stotterte Vater Pfister, und ich – ich
konnte weiter nichts darauf erwidern als.

»Das müssen wir unbedingt sofort sehen!«

Ich stieß die Tür des angedeuteten Schuppens mit dem Fuße weiter
auf. Das vordringende Gewölk umhüllte uns und~–

»Alle Wetter!« husteten und prusteten zurückprallend sowohl der
Müller von Pfisters Mühle wie sein Kind, – der Dampf, der uns den
Atem benahm, stammte wohl von noch etwas anderm als unschuldiger
grüner Seife und Aschenlauge; und wie eine menschliche Lunge es
hier aushielt, das war eine Frage, zu der wir erst eine geraume
Zeit später fähig wurden.

Dagegen begrüßte uns sofort aus dem vielgemischten, entsetzlichen
Dunst eine wohlbekannte Stimme:

»Holla, nicht zuviel Zugluft bei obwaltender Erdenwitterung
draußen! Tür zu, wenn ich bitten darf! Olga, bist du es, so muß ich
dir doch sagen, daß mir so ein Unterrock während meiner ganzen
wissenschaftlichen Praxis noch nicht vor Nase und Augen gekommen
ist.«

»Olga ist es grade nicht; wir sind's, Doktor Asche«, keuchte mein
Vater. »Ich bitte Sie um des Himmels willen~–«

Und aus dem vom Herd und aus dem Waschkessel aufwirbelnden Greuel
hob sich, wie das Haupt eines mittelalterlichen Alchimisten, der
schwarze Struwelkopf unseres letzten Trösters in unseren übeln
Erdengerüchen; und Doktor A.~A.~Asche mit aufgestreiften Ärmeln, in
einem Schlafrock, der wahrscheinlich seinesgleichen nicht hatte,
sagte gelassen:

»Sie sind es, Vater Pfister? Und der Junge auch? Na – dann kommt
nur herein und machen Sie auch die Tür zu, wenn das Ihnen lieber
ist.«

»Den Teufel auch!« ächzte der alte Herr von Pfisters Mühle. »Aber
Asche – Doktor – Herr Doktor~–«

Doktor Asche ließ sich gegenwärtig nicht so rasch stören, wie es
für unsern freien Atem wünschenswert sein mochte.

Mit einem langen hölzernen Löffel fuhr er in den Kessel vor ihm,
vermehrte durch längeres Suchen und Rühren Gedämpf und Gedüft um
ein erkleckliches, holte ein unheimliches Etwas empor, packte das
brühheiße Scheußliche mit abgehärtet verwogener Gelehrtenfaust,
hielt es, ließ den stinkgiftigen Sud abträufen und sprach wie mit
bescheidener Ergebung unter die eben vom Genius auferlegte Last
eines ewigen guten Rufes und unsterblichen Namens:

»Meine Herren, Sie kommen zu einem großen Moment grade recht! Ich
glaube wirklich in diesem Augenblick sagen zu dürfen: Bitte, treten
Sie leise auf!\ldots{} Vater Pfister, halten Sie sich die Nase zu, aber
stören Sie gefälligst das Mysterium nicht. Und du, Bengel – ich
meine dich, Eberhard Pfister, mein Zögling und mein Freund, tritt
heran, glücklicherer Jüngling von Sais, werde mir bleich, aber
nicht besinnungslos – ekle dich meinetwegen morgen mehr und soviel
du willst, doch gegenwärtig beuge in schaudernder Ehrfurcht dein
Knie: so geht man im zweiten Drittel des neunzehnten Jahrhunderts
zur Wahrheit!«\ldots{}

Jedenfalls ging er mir um den Herd herum zwei Schritte näher,
schlug mir den triefenden, furchtbaren Lappen, den Fetzen vom
Schleier der Isis, fast ums Gesicht und grinste:

\begin{verse}
»Gewichtiger, mein Sohn, als du es meinst,\\
Ist dieser dünne Flor – für deine Hand
\end{verse}

Zwar leicht, doch zentnerschwer für meinen – Beutel; ich meine,
Sie, meine Herren, bei der in diesem Raume obwaltenden Atmosphäre
nicht darauf weiter hinweisen zu dürfen, daß es keine Kleinigkeit
ist, der Natur nicht aus dem Tempel zu laufen, sondern den Stein
der Weisen weiter zu suchen, auch auf die Gefahr hin, ihn wieder
nicht zu finden.«

Vater Pfister, der seit längerer Zeit von seiner Mühle doch schon
an allerlei obwaltende Atmosphäre gewöhnt war, kam vor
Atmungsbeschwerden noch immer nicht dazu, die nötige Frage zu
stellen. Ich brachte es zu dem gekeuchten Wort:

»Ich bitte dich um alles in der Welt, Asche!« – Doch Doktor Asche
ließ sich fürs erste noch nicht stören.

Er hielt jetzt sein geheimnisvolles Gewandstück zwischen beiden
Fäusten. Er wrang es aus zwischen beiden Knieen – schweißtriefend.
Er entfaltete es, hielt es gegen eine trübe Petroleumflamme, rollte
wie wütend es noch einmal zusammen und rang von neuem mit ihm, wie
der Mensch eben mit der alten Schlange, dem Weltgeheimnis als Ideal
und Realität a priori und a posteriori zu ringen pflegt, seit er
sich, sich auf sich selber besinnend, erstaunt in der Welt vorfand.
Aber er gelangte, wie immer der Mensch, auch diesmal nur bis zu den
Grenzen der Menschheit, und er nahm das \emph{Ding}, nachdem er es
zum drittenmal auseinandergebreitet und wieder zusammengewickelt
hatte, \emph{an sich}, das heißt, er nahm es jetzt unter den Arm,
bot uns die biedere, wenn auch augenblicklich etwas anrüchige
Rechte und meinte: »Zu Ihrer Verfügung, meine Herren! Ich hatte
doch eben das Laboratorium dem schnöden Alltagsgebrauch zu
überlassen. Es wollen noch andre Leute am heutigen Abend im Hause
waschen, und das wissenschaftliche Trocknen besorge ich in meinem
Falle lieber am eigenen Ofen. Olga!\ldots{}.. Witwe Pohle!\ldots{}
Stinchen!\ldots{} Frau Börstling!\ldots{}.. Fräulein Marie – das Lokal ist
frei. Krallen in die Höhe und munter in die Haare einander! Vater
Pfister, gehen wir?«

Wir gingen gern; denn schon drängte es sich in die Pforte dieser
Waschküche dieser vorstädtischen Mietskaserne – ein zürnend giftig
Gewoge aufgeregter, nevösester Weiblichkeit, das, wie wir im
eigenen Durchzwängen noch vernahmen, schon seit Mittag auf das Ende
der Schmiererei in seinem eigenen, angeborenen Reiche und Bereiche
gewartet hatte. Und ein Gewimmel unmündiger Nachkommenschaft war
natürlich auch vorhanden, begleitete uns mit teilweise höhnischen,
teilweise aber auch wohlwollenden Gefühlsäußerungen über den Hof
und verließ uns auch im Innern des Hauses auf den Treppen nicht.

»Tausend Donnerwetter«, ächzte mein Vater, meinen Arm fester
fassend. »In Kannibalien an 'ne Insel geworfen werden muß ja ein
Labsal hiergegen sein. Hat man denn gar nichts, was man unter sie
schmeißen könnte? Hier, halte meinen Stock, Ebert; vielleicht löse
ich uns mit meinem Kleingeld aus! Da wage ich mich doch nie in
meinem Leben wieder hierher ohne polizeiliche Begleitung heraus. Da
ist ja die reine Kommunewirtschaft, Asche; und Sie mitten drin,
Doktor, und zwar ganz in Ihrem Esse, wie's den Anschein hat? Das
fasse ein anderer!«

»Mein Versuchsfeld, Vater Pfister«, sprach lächelnd Doktor
A.~A.~Asche. »Sie haben mir an jedem andern Orte nach dem zweiten
Experiment die Miete aufgesagt. Als ob ich etwas dafür könnte, daß
die Wissenschaft in ihrer Verbindung mit der Industrie nicht zum
besten duftet. Gleich sind wir aber oben, und zwar in mehr als
einem Sinne. Wie sagte man zu Syrakus, Knabe, als die Geldnot am
höchsten und der Küchenschrank am leersten war? ›Gib mir, wo ich
stehe, und ich setze mich sofort‹ – wenn ich nicht irre! Und das
nämliche sage ich jetzt, und – hier stehe ich, und von hier aus
hoffe ich in der Tat die Welt aus den Angeln zu heben und allen
Sambuken und Argentariern zum Trotz dem Jammer ein wohlgesättigt,
ja vollgefressen behaglich Ende zu bereiten, solide Platz zu nehmen
auf Erden und Ihnen, Vater Pfister, ganz speziell alles Gute, was
Sie an mir vollbracht haben, mit dem eigenen Keller- und
Speisekammerschlüssel in der Tasche gerührt zu vergelten.«

Wir standen nämlich jetzt in seinem absonderlichen Daheim,
Schlehengasse Numero eins, im Ödfelde, und selbst hier nicht im
ersten Stockwerk. Es war aber ein ziemlich umfangreiches Gelaß, in
dem er jetzt noch, in Erwartung alles Bessern, sich und seine
kuriosen wissenschaftlich-industriellen Studien und Bestrebungen
untergebracht hatte. Und Vater Pfister kam noch einmal aus einem
übeln Dunst in den andern und hatte Grund, von neuem sich die Nase
zuzuhalten und nach Atem zu schnappen.

Ein überheißer, rotglühender Kanonenofen bösartigster Konstruktion
war von einem Gegitter von allen vier Wänden her durch den Raum
ausgespannter Bindfäden und Wäscheleinen umgeben. Was aber auf den
Fäden und Stricken zum Trocknen aufgehängt war, das entzog sich
jeglicher genauern Beschreibung. Ich brauche nur mitzuteilen, daß
jede Familie im Hause ein Stück ihrer Garderobe dazu geliefert zu
haben schien und daß Doktor Adam Asche Olgas Gewand eben auch
dazuhing, und darf hoffen, genug gesagt zu haben.

»Und nun, Kinder, setzt euch«, rief der Doktor, im vollsten Behagen
sich die Hände reibend und in überquellender Gastfreundlichkeit
unter und zwischen seinen Leinen und Lumpen und Fetzen männlicher
und weiblicher Bekleidungs- und Hausratsstücke nach
Sitzgelegenheiten hin und her fahrend, auf und ab tauchend. »Das
ist ja reizend von Ihnen, Vater Pfister. Ein Abend, ganz darnach
angetan, um wie in Pfisters Mühle beim Schneetreiben und einem
Glase Punsch zusammenzurücken! Nur einen Moment, meine Herren;
kochendes Wasser stets vorhanden! Störe mir meine Kreise nicht, das
heißt, reiß mir meine Feigenblätter menschlicher Eitelkeit und
Bedürftigkeit nicht von der Linie, Ebert, sondern greif behutsam
hin und drüber weg: die Zigarrenkiste steht auf dem Schranke gerade
hinter dir. Vater Pfister~–«

»Jetzt will ich Ihnen mal was sagen, Asche, und zwar am liebsten
gleich wieder draußen vor der Tür«, sprach mein Vater, und zwar mit
einer wütenden Gehaltenheit in Ton und Ausdruck, die nur selten bei
ihm zum Vorschein kam. »Sie werden sich doch nicht einbilden, Adam,
daß ich, der ich grade wegen ziemlich gleichem Geruch und noch dazu
bei dieser Tages- und Jahreszeit als älterer Mann mich auf meinen
weichen Füßen zu Ihnen herausbemüht habe, hier jetzt in diesen
infamen Odörs ein pläsierlich Konvivium bei Ihnen halten will?
Behalt deine Mütze auf dem Kopfe, Junge; das haben wir zu Hause
auch. Komm wieder mit; ich sehe ein, es ist nicht anders und soll
nicht anders sein. Die Welt will einmal in Stank und Undank
verderben, und wir Pfister von Pfisters Mühle ändern nichts daran.
Bringe mich mit möglichst heilen Knochen wieder hin nach dem Blauen
Bock. Samse mag sofort wieder anspannen; wir fahren nach Hause. Es
ist wohl nicht das letzte Mal, daß dein Vater sich in das
Unabänderliche geschickt hat, Ebert.«

»Holla! Halt da! Nur noch fünf Minuten Aufenthalt«, rief der
Doktor. »Was ist es denn eigentlich, Vater Pfister? Das klingt ja
verflucht tragisch. Um was handelt es sich, Knabe Eberhard?\ldots{}..
Wenn die Herren sich vielleicht einbilden, daß ich, Doktor
A.~A.~Asche, vorhin aus inniger Neigung in meinem angeborenen
Element plätscherte, daß ich hier wie 'ne Kölnische Klosterjungfer
gegenüber dem Jülichsplatz in meinem Eau de Cologne schwimme und
mich selber mit Wonne rieche, so irren Sie sich. Auch der Gelehrte,
der Chemiker bleibt am Ende Mensch – Nase – Lunge! Es ist zwar
schön, aber durchaus nicht angenehm, auf dem Gipfel seiner
wissenschaftlichen Bestrebungen dann und wann ohnmächtig zu werden;
und – wißt ihr was, Leute? Feierabend ist es doch – ich gehe am
besten mit euch nach dem Blauen Bock und vernehme dort in
gesünderen atmosphärischen Verhältnissen das, worüber Sie meinen
bescheidenen Rat einzuholen wünschen, Vater Pfister.«

»Das ist wenigstens ein Wort, was sich hören läßt«, sagte mein
Vater. »Das ist sogar ein vernünftiges Wort, Adam, und ich nehme
Sie und warte mit dem Ebert so lange draußen auf der Treppe, bis
Sie sich hier drinnen gewaschen und angezogen haben. Nicht wahr,
Sie nehmen das einem alten Manne, der sonst schon tief genug im
Morast sitzt, nicht übel?«

»Durchaus nicht!« lachte der Doktor, und nach fünf Minuten befanden
wir uns auf dem Wege nach dem Blauen Bock. Wieviel Verdruß, Ärger
und leider auch herzabfressenden Kummer Vater Pfister noch von
Pfisters untergehender Mühle haben sollte: das ist mir wenigstens
ein Trost, daß er dabei zur Rechten wie zur Linken jemand hatte,
der, wie treue Söhne sollen, Leib und Seele hingegeben hätte, ihm
seine letzten Schritte durch die schlimme Welt behaglicher zu
machen. Er ist doch noch mehr als einmal zu einem vergnüglichen
Knurren und herzlichen Lachen in seiner alten Weise gekommen, ehe
es aus mit ihm war.

Wo bleiben alle die Bilder?

\section{Zehntes Blatt}

\zusatz{Der Blaue Bock und ein Tag Adams und Evas in der
Schlehengasse}
Ich nahm Emmy nicht weiter mit in den Blauen Bock; wir gingen denn
doch endlich lieber zu Bett in der stillen Mühle, und das Kind mit
seinem unschuldigen, besten Gewissen entschlummerte auch sofort und
drehte sich nur einmal auf die andere Seite, wie es schien, von der
seltsamen Wäsche ihres guten Freundes Doktor Adam Asche träumend.

Ich aber, wenngleich ebenfalls in »Nacht und Kissen gehüllt«, blieb
in der Erinnerung noch ein wenig im Blauen Bock und saß mit dem
verstorbenen Vater und dem Freunde und – Samse, dem treuen Knecht,
in der wohlbekannten Wirtsstube der weitbekannten
Ausspannwirtschaft und frischte alte Bilder auf.

Der alte Herr zahlte selbstverständlich uns hungrigem jungen Volk
die Zeche, und Samse griff in die Schüssel wie in die Unterhaltung
ein und gab nicht nur einen wackern Durst, sondern auch mehr als
ein verständig Wort dran und dazu. Über seine eigenen übelduftenden
Augiasstallstudien und seine sich möglicherweise daran knüpfenden
Absichten und Aussichten, Pläne und Hoffnungen ließ sich der Doktor
wenig aus, murmelte nur einiges von: Berliner Schwindel! und tat
selbst mir gegenüber zurückhaltender, als sonst seine Gewohnheit
war. Aber seinem alten Gönner lieh er ein williges Ohr und ließ,
mit Messer und Gabel beschäftigt, Vater Pfister so ausführlich
werden, als das demselben in seinen Nöten und Ängsten ein Bedürfnis
sein mochte.

»Den Braten habe ich lange gerochen!« seufzte er, Asche, mit einem
fetten Stück Kalbsniere auf der Gabel, und ließ es ungewiß, was für
einen »Braten« er eigentlich meine. Das Wort wird ja wohl immer
noch dann und wann in Verbindung mit der Nase des Menschen
figürlich genommen.

»Sie hören mir doch auch zu, Adam?«

»Mit vollstem Verständnis, würdigster Gastfreund. Bis über die
Ohren in diesem Salat!« lautete die Antwort. »Erzählen Sie ruhig
weiter, Vater Pfister; es gehört mehr in der Welt dazu, mir in
gegenwärtiger Stunde den Appetit zu verderben. Dich ersuche ich um
den Pfeffer dort, Sohn und Erbe von Pfisters Mühle. Hoffentlich hat
man es dir in der klassischen Geographie beigebracht, daß Grade
durch das Land Arkadien der Fluß Styx floß und daß jeder, der im
neunzehnten Jahrhundert einen Garten und eine Mühle an dem
lieblichen Wasser liegen hat, auf mancherlei Überraschungen gefaßt
sein muß. Schade, daß ich dich meinerzeit nicht schon darauf
aufmerksam machen konnte in unserm Hinterstübchen! Sie waren dort
sehr gastfrei, Vater Pfister – in Arkadien nämlich – und sie
beteten den Gott Pan an, und in der Poesie und Phantasie wird es
immer ein Paradies bleiben – grade wie Pfisters Mühle mir! – was
auch in der schlechten Wirklichkeit daraus werden mag. Ob ich Ihnen
zuhörte, Vater Pfister? In Ihrer Seele sitze ich! Als Sie in
harmloser Heiterkeit in gewohnter, lieber Weise Ihre Nase noch hoch
unter Ihren Gästen herumtrugen, habe ich Ihr und unserer alten
guten Mühle Schicksal bereits vorausgerochen. Zu Weihnachten also
das Weitere, und zwar so wissenschaftlich, als es Ihnen beliebt;
vorläufig nur das Wort: Krickerode!«

Krickerode!

Es war nur ein Wort, aber es wirkte, wie ein einziges Wort dann und
wann zu wirken pflegt. Es schlug ein; und mein Vater, nachdem er
auf den Tisch geschlagen hatte, sprang auf, legte sich vorwärts
über Gläser, Schüsseln und Teller, faßte mich, hielt mich an beiden
Schultern, schüttelte mich und rief:

»Was habe ich mir gedacht?\ldots{} In schlaflosen Nächten und am wachen
Tage!\ldots{} Was hab ich dir gesagt, Junge? Bezeuge es dem Doktor da,
was ich dir schon längst gesagt habe!«

»Was verlangen Sie denn sonst noch von dem Zucker, als daß er uns
das Leben versüße, Vater Pfister?« fragte Doktor Asche mit
behaglich gesättigter Grabesstimme. »Allzuviel davon in der Welt
Feuchtigkeiten kann einem freilich – hie und da zuviel werden. Ich
gebe Ihnen da wie gewöhnlich vollkommen recht, alter Herr und
Gönner.~–

»Also doch – Krickerode!« murmelte mein Vater, jetzt schlaff und
erschöpft auf seinem Stuhle sitzend und wie abwesend (an seinem
Wasserlauf und in seiner Mühle) von einem zum andern blickend. »Wer
mir \emph{das} in meiner unschuldigen Jugend prophezeit hätte, wenn
mich meine selige Mutter mit dem Sirupstopf ins Dorf schickte und
sich jedesmal wunderte, daß der Kaufmann so wenig fürs Geld gab!\ldots{}
Also Krickerode!\ldots{}«

»Zuviel Zucker – zuviel Zucker – viel zuviel Zucker in der Welt, in
der wir leben sollen!« seufzte Asche.

»Rübenzucker«, sagte mein Vater, matt die brave, breite Hand auf
den Tisch legend; und Adam Asche meinte jetzt mit wirklicher,
aufrichtiger Teilnahme:

»Wozu ich Ihnen und der Mühle unter diesen Umständen werde nützlich
sein können – wozu ich Ihnen verhelfen kann: ob zu Ihrem Recht oder
nur zu größerem Verdruß, kann ich nicht sagen; aber daß ich zu
Weihnachten nach Pfisters Mühle kommen werde, darauf können Sie
Gift nehmen, Vater Pfister.«

»Letzteres ist gar nicht notwendig, Adam«, meinte der alte Herr
melancholisch. »Bloß auch wissenschaftlich möchte ich es jetzt gern
zum Heiligen Christ von Ihnen haben, Doktor. Anspannen, Samse!\ldots{}«

Ehe Samse hinausging, um anzuspannen, setzte der gute Knecht mir
unterm Tisch den nägelbeschlagenen Gamaschenschuhabsatz in einer
Art auf die Fußzehen, die nur bedeuten konnte:

»Komm mal mit in den Stall.«

Und im Stall neben dem treuen, die letzten Haferkörner in der
Krippe beschnaubenden Hans von der Mühle legte er, Samse aus der
Mühle, mir die arbeitsharte, treue Hand schwer auf die Schulter und
sagte:

»'s ist die höchste Zeit, daß Ihr was dazu tut, Ebert. Seht ihn
Euch an! Er wird mir umfänglicher, aber auch weichlicher von Tag zu
Tage. Da will er mir des Morgens nicht mehr aus dem Bette, und
heben wir ihn heraus, so sitzt er uns hin im Stuhl am Fenster und
schnüffelt und schnüffelt und schnüffelt. Und steht er und geht er
um, so ist es noch schlimmer mit der Mühle – von uns gar nicht zu
reden. Er schnüffelt drinnen, er schnüffelt draußen; an mir mag er
riechen, was und so viel er will, aber an dem übrigen riecht er
sich noch seinen Tod an den Hals, und die Christine ist da auch
ganz meiner Meinung. Ja, die hat sich auch in Geduld zu fassen und
das Ihrige zu leiden! Nichts riecht ihm an ihr mehr recht. In Küche
und Kammer, auf dem Boden und im Keller schnüffelt er uns; aber das
Schlimmste ist doch sein Stehen im Garten und sein Atemholen
dorten, so viel ihm noch davon vergönnt ist, und das ist leider
Gottes wenig genug. Daß ich ihm heute morgen unsern Herrn Doktor
Adam aufs Tapet gebracht habe, das ist mein Verdienst; aber nun
sorgen auch Sie, Ebert, nach Kräften dafür, daß der als
Übergelehrter das Seinige an uns tut. Es ist ja diesmal wirklich,
als ob uns die Doktoren zu unserm einzigsten Troste in die Welt
gesetzt wären: ohne unsern andern von der Art, Sie wissen es, wen
ich meine, stünde es an manchem gegenwärtigen Winterabend noch
tausendmal elender um Pfisters Mühle, und einen schlimmen Zahler
muß unser Meister ja mal zu jeder Zeit auf dem Konto haben. Das ist
eben sein absonderlich Privatvergnügen, zu dem er unter Millionen
allein auf die Welt gekommen scheint. Und dann Fräulein
Albertine~–«

Ich wußte es natürlich, von wem der Alte redete; aber ehe ich ihm
meine vollständige Übereinstimmung mit seiner Meinung kundgeben
konnte, rief mein Vater derartig ungeduldig von dem Hausflur des
Blauen Bockes her nach seinem getreuen Knechte, daß dieser allen
Grund hatte, sich und den braven Mühlen-Hans zu beeilen.

Zehn Minuten später standen Adam und ich in dem Torbogen und sahen
dem Vater Pfister nach, wie er heimwärts fuhr und wenig Trost aus
der Stadt mit nach Hause nahm. Mit den Augen konnten wir ihm und
dem Gefährt nur wenig über die nächste Laterne am Wege folgen; aber
wir standen in der scharfen Zugluft und dem feuchten Niederschlag
des Winterabends unter dem Tor und Schilde des Blauen Bockes, bis
sich das letzte Rädergerassel des Müllerwagens von Pfisters Mühle
in der Ferne verloren hatte.

Dann meinte Doktor A. A. Asche:

»Ein Mensch wie ich, der die feste Absicht hat, selber einen
sprudelnden Quell, einen Kristallbach, einen majestätischen Fluß,
kurz, irgendeinen Wasserlauf im idyllischen grünen Deutschen Reich
so bald als möglich und so infam als möglich zu verunreinigen, kann
nicht mehr sagen, als daß er sein Herzblut hingeben würde, um dem
guten alten Mann dort seinen Mühlbach rein zu erhalten. Ich bin,
wie du weißt und nicht weißt, seit ich dir im Hinterstübchen von
Pfisters Mühle die Anfangsgründe nicht nur des Lateinischen,
sondern auch der Menschenkenntnis beibrachte, unter den Menschen
viel und an vielen Orten gewesen; aber einen zweiten seinesgleichen
habe ich nicht unter unsersgleichen gefunden. Da ist kein Wunsch,
den ich dem nicht zum Heiligen Christ erfüllen möchte, aber leider
Gottes werde ich ihm nur in einem zu Willen sein können. Erfahren
soll er, wer ihm seinen Bach trübt. Wissenschaftlich soll er's
haben bis zur letzten Bakterie! Schriftlich soll er's haben – zu
Gericht soll er damit gehen können! Ich werde ihm sein Wasser
beschauen, und kein anderer Doktor wird ihm die Diagnose so sicher
stellen, wie sein alter, verlungerter Schützling und Günstling Adam
Asche.«

»Du bist doch ein guter Mensch, Asche!« rief ich.

»Das bin ich gar nicht«, schnarrte mir aber der chemische Vagabund
und Abenteurer zu. »Komm nach Hause, \emph{junger} Mensch! Wende du
deine Windeln auf dem Zaune um, das heißt, setze dich an deine
Bücher. Mich verlangt's anjetzt dringlich zu der Wäsche zurück, die
mir, wie du vorhin bemerken konntest, auf der Leine hängt. Ich habe
viel zu tun die nächsten Wochen hindurch und du auch einiges; also
beschränke deine Erkundigungen nach meinem Ergehen auf das
geringste Maß der Höflichkeit. Am liebsten ist's mir, du kommst am
Tage Adam und Eva, am vierundzwanzigsten Dezember, so um vier Uhr
nachmittags, und holst mich ab nach Pfisters Mühle. Das soll
übrigens allem Erdenstank und -drang zum Trotz die gemütlichste
Weihnacht werden, die ich seit manchem widerwärtigen Jahr gefeiert
habe. Den Wind im Rücken auf der Landstraße, Abenddämmerung, Nacht
und Nebel auf den Feldern rundum und in seiner Mühle der Vater
Pfister: ›Christine, da kommen sie! Brenne die Lichter an der Tanne
an!‹ – Das wäre wahrhaftig eine Sünde, ihm seinen Wunsch nicht zu
erfüllen. Bis auf das letzte Atom soll er's wissen, wieviel Teile
Ammoniak und Schwefelwasserstoff der Mensch dem lieben Nachbar
zuliebe einatmen kann, ohne rein des Teufels zu werden ob der Blüte
des nationalen Wohlstandes und lieber alle Viere von sich zu
strecken, als noch länger in \emph{diese} Blume zu riechen. Guten
Abend, Ebert.«

Er nahm hiermit nach seiner Art einen kurzen Abschied, und ich sah
ihn wirklich nicht eher wieder als bis am Tage Adam und Eva und
ließ ihn bis dahin ungestört bei seinen mysteriösen Studien und
Arbeiten. Der vierundzwanzigste Dezember dämmerte dann ganz wie ein
Tag nach seinem Wunsche – dunkel und windig vom ersten grauen
Schein – über den Dächern an; nur daß wir den Wind, einen recht
wackern Nordost, nicht im Rücken, sondern gradeaus im Gesicht und
nur hie und da an einer Wendung der Landstraße scharf in der Seite
haben sollten.

Ich holte ihn ab und hatte das Vergnügen, ihm beim Packen seines
Reisebündels behülflich zu sein und auch sonst für die Tage seiner
Abwesenheit sein städtisches Heimwesen zu einem Abschluß bringen zu
helfen, was auch nicht ohne seine drolligen Schwierigkeiten war.
Er, der behauptete, einer der freiesten der Menschen zu sein, war
nach so vielen und verschiedenen Richtungen hin gebunden und so
erfinderisch den kuriosen Einzelheiten seiner Existenz gegenüber,
daß es nur einem Normalphilisterkopf ein wahres Übermaß der
Schadenfreude gewähren konnte, ihn sich in seinen Verlegenheiten
abzappeln zu sehen. Schadenfroh war ich nicht, aber daß ich bei
seinen Versuchen, die Bande und Knoten, welche ihn an die
Schlehengasse fesselten, möglichst ohne arges Gezeter und sonstige
Ärgernisse zu lösen, in Mitleid und Wehmut verging, kann ich
freilich auch nicht sagen.

Er hatte, als ich kam, seiner Mietsherrin bereits mitgeteilt, daß
er für einige Zeit vom Hause abwesend sein werde, und ich traf
mehrere bei ihm anwesend, die dringend genügende Garantie für sein
Wiederkommen verlangten, ehe sie ihn losließen. Merkwürdigerweise
hatten die Gewerbtreibenden im Hause sämtlich ihre Frauen oder
Töchter geschickt und warteten selber lieber auf ihrem
Schusterschemel oder Schneidertisch das Resultat ihrer
Verhandlungen ab. Und Meister Börstling hatte Weib und Kind
gesendet. Mit Madame lag Fräulein Olga dem unseligen, gelehrten
chemischen Wäscher auf dem Halse, und Olga hatte ganz intime Stücke
weiblicher Garderobe mitgebracht und hielt sie dem Hausgenossen
unter die Nase:

»Wie Zunder, Herr Doktor! Zwischen den Fingern zu zerreiben! Und
hinten und vorn versengt! Und frage ich Sie, wer steht mir nun für
den Schaden, den wir in unserer Herzensgüte uns haben anrichten
lassen?«

Fräulein Marie hatte nur »eine kleine Note vom Papa« gebracht, der
aber doch grade auf das Fest Besseres zu tun hatte, als mit seinem
Schneiderkonto faulen Kunden in die weite Welt nachzulaufen. Aber
die Furchtbarste war doch die dem Doktor Nächste, seine
Stubenwirtin, Witwe Pohle. »Vollständig unbezahlte Rechnung seit
Anmeldung auf der Polizei«, sperrte sie uns die Tür und den Weg
nach Pfisters Mühle.

Und es war ihnen allen nicht zu verdenken! Sie hatten meistens
sämtlich Kinder, und zwar viele. Es war der Tag Adam und Eva, der
Heilige Abend dämmerte bereits, und sie hatten sämtlich Geld nötig
aufs Fest.

Mitleid mit dem Sünder konnte aber, wie schon bemerkt, dreist für
dringendere Fälle aufgespart werden; guter Rat wäre gänzlich an ihn
weggeworfen gewesen.

»Nur sachte, immer sachte, Kinder«, sprach mit höchstem Gleichmut
Doktor Adam Asche, nur von Zeit zu Zeit beide Hände auf beide Ohren
drückend. »Bin ich Orpheus, daß ihr mich zu zerreißen wünscht, ihr
kikonischen Weiber? So schlimm ist's doch nicht mit dem Peplos, wie
Sie's mir einbilden wollen, Olga! Einmal tut er doch noch seine
Schuldigkeit mit Weinlaub und Eppich im Orpheon, liebes Kind!\ldots{} So
halten Sie mir doch die Krabben vom Leibe, Madam Börstling! Zahlung
hoffen Sie und werden in Ihrer Hoffnung nicht getäuscht werden;
fragen Sie den jungen Mann hier, ob er nicht noch einmal bluten
wird – sein Erzeuger nämlich! Wir haben beide die besten Absichten,
nicht umsonst Weihnachten in Pfisters Mühle zu begehen – Silvester
feiern wir hier, und ich gebe dem ganzen Hause eine Bowle!\ldots{} O
Fräulein Marie, von Ihnen und Papa hätte ich doch etwas anderes
erwartet als dieses! Haben wir – der eine wie der andere – Papa,
ich und Sie, nicht höhere Bildung, nicht andere Interessen, nicht
größere Ziele? Darf ich Sie nicht noch ein einziges Mal auf unsere
Ideale verweisen, Maria? Ich darf es, ich sehe es Ihnen an, daß
Papa auch diesmal noch sich bis nach Neujahr gedulden wird!\ldots{} Mit
Ihnen, Mutter Pohle, sollte ich eigentlich gar nicht zu reden
brauchen. Sie wissen es, daß ich es weiß, wie sicher ich Ihnen bin,
und daß es Ihnen gar keinen Spaß machen kann, Ihren angenehmsten
Stubenherrn, seit Sie auf dergleichen als Witwe angewiesen sind, in
anderthalb Stunden an den Christbaum zu hängen. Ich setze Ihnen
hier diesen Jüngling zum Pfande, daß ich zu Neujahr wieder zurück
bin von Pfisters Mühle. Daß bis Ostern vielleicht sich alles –
alles gewendet haben wird, Knabe Ebert, ist etwas, was ich
gegenwärtig so wenig diesen Herzen hier wie dir plausibel machen
könnte. Ein Poet mit der gültigsten Anweisung auf die
Unsterblichkeit ist da dem vorhandenen Moment gegenüber nicht übler
dran als ich, und nun, Kinder, tut mir den Gefallen und verderbt
euch und mir nicht länger die Gemütlichkeit des Abends vor dem
Heiligen Christ! Hier – auf den Tisch – mein letztes Zehnmarkstück!
Das ist vom Onkel Asche für die \emph{Kinder} Schlehengasse Numero
eins. Da, Toni ist die Vernünftigste, die und Hermann nehmen den
größten Handkorb, aber alle übrigen gehen mit in die Stadt zum
Zuckerbäcker, und – euch älteres und ältestes Gesindel mache ich
darauf aufmerksam, daß ich zu Neujahr wieder hier am Platze bin und
fürchterliche Rechenschaft fordern werde, wenn der geringste
Krakeel wegen ungerechter Verteilung im Hause entstanden sein
sollte.«

Damals stand ich ob dieses Erfolges dieser Wendung der Rede
A.~A.~Asches nur stupifiziert. Wie ich heute, bei reiferen Jahren,
die Sache ansehe, kann ich mir nur sagen: Hier war der Charakter,
den Durchlaucht Otto Fürst Bismarck, Kanzler des Deutschen Reiches,
wenigstens so ungefähr im Sinne haben konnte, wenn er den Reichstag
ersuchte, sich gütigst für einen andern Mann auf dem harten Stuhl
zu sammeln.

Sie entfernten sich, und wir blieben noch einige Augenblicke. Sie
liefen, und wir hörten ihren jauchzenden Tumult auf allen Treppen –
Kinderjubel und Kindergekreisch treppauf, treppab: »Onkel Asche!«
von oben bis unten durch das Haus Schlehengasse eins im Ödfelde.
Aus der Atemlosigkeit eines Lachkrampfes, dessen ich mich heute
noch schäme, riß mich das gelassene Wort Doktor Adam Asches:

»Wie ich glaube, können wir allmählich auch gehen.«

\section{Elftes Blatt}

\zusatz{Auf dem Stadtwege nach Pfisters Mühle}
Der Tag Adams und Evas! – Fürs erste war es ein Morgen über und um
Pfisters Mühle, so blau und so grün, so lau und doch so frisch, so
sonnenklar und so voll lieblichen Schattens, wie vielleicht der, an
welchem in dem großen Tiergarten der Erde die erste Eva
verschämt-zärtlich zum erstenmal leise die Hand dem Adam auf die
Schulter legte und flüsterte:

»Da bin ich, lieber Mann!«

Es steht nicht im Buch der Genesis und wird natürlich nur von der
Bank stammen, auf der die Spötter sitzen – nämlich, daß unser aller
Stammvater in der dem süßen Wunder vorhergehenden Nacht bedenklich
schwer geträumt habe, und zwar apriorisch von unendlichen
Katzbalgen mit und unter seinesgleichen, und daß er in jener Nacht,
und zwar im Traume, noch einem Dinge seinen Namen gegeben habe. Es
ist unbedingt nicht wahr, daß zu dem Begriff Rippenstoß in jener
Nacht das Wort gefunden worden sei.~–

Was nun das Fleisch von meinem Fleisch, das Bein von meinem Bein
anbetrifft, so gelang es dem an diesem schönen Morgen nicht wie
sonst wohl, scherzhaft mich durch einen Nasenstüber zu erwecken und
dabei in eine seiner wunderwollen blonden Flechten kichernd mir zu
insinuieren:

\begin{verse}
»Drei Teile seines Lebens\\
Verschläft der Dachs vergebens~–
\end{verse}
\noindent
sieh doch nur die Sonne, Ebert! Wir sollten schon seit einer
Stunde draußen unter den Bäumen sein. Du bist doch eigentlich ein
zu furchtbarer Faulpelz, liebstes Männchen!«

Seit einer Stunde schon saß ich unter den Bäumen meines alten
Mühlgartens und hatte den wonnigsten Morgen unserer Sommerfrische
für mich allein.

»Mit dem Kaffee warte ich wohl, bis unser Frauchen kommt?« hatte
Christine gemeint, und ich hatte selbstverständlich durchaus kein
Bedürfnis gehabt nach dem Kaffee in Abwesenheit meines »Fräuleins«,
wie Doktor Martin Luther übersetzt.

Endlich hatte das Fenster geklungen und der Vorhang sich bewegt. In
rosiger Verschlafenheit hatte sich mein Kind, meine holdselige
Sommerfrischlerin, herausgebeugt in der Sicherheit, daß keine
fremden Leute, keine frühen Gäste, Brunnentrinker und Lustwandler
aus der Stadt mehr von den Tischen und Bänken des alten Gartens aus
sie belauschen konnten.

»Nun seh? einer den Durchgänger! Gott, wie lange sitzest du denn da
schon, Ebert? Himmel, wie spät ist's denn eigentlich?\ldots{} Laß dir
nur den Kaffee bringen; in fünf Minuten bin ich bei euch!«

Der weiße Vorhang war von neuem zugefallen, und wirklich nicht
länger als eine gute halbe Stunde hatte es gedauert, bis mir meine
zweite, noch lieblichere Sonne aufging an dem neuen Lebenstage
unter den Bäumen, den verwirkten Paradiesesbäumen von Pfisters
Mühle.

Sie – Emmy Pfister, geborene Schulze – trippelte daher vom Hause im
leichten, lichten Morgenkleid und verlor einen zierlichen Pantoffel
auf dem Wege und kehrte sich um, ihn aufzuheben, hüpfte mit ihm in
der Hand – natürlich in meine Arme, und – weg hatte ich ihn – den
Klaps mit dem ersten Kuß am Tage:

»Weißt du wohl, daß du mir gestern abend ganz dumme Geschichten
erzählt haben mußt, Ebert? So unruhig wie in vergangener Nacht habe
ich lange nicht geschlafen und so schwer geträumt auch nicht.«

»Armes Vögelchen! Na, jedenfalls kannst du sie mir
wiedererzählen.«

»Meine Träume? Ja\ldots{} warte mal\ldots{}«

»Nein, meine Geschichten meine ich!«

»O die! Ja natürlich! Selbstverständlich vom Anfang bis zum Ende!«

Ich meine jetzt noch etwas, nämlich, daß es mehr als bloß mich
gibt, die es aus Erfahrung wissen können, daß die letzte Behauptung
meines Weibes eine von der Weiber siegessichersten Lügen war und es
gewesen wäre, selbst wenn sie im Buch der Bücher auch schon von
Frau Eva vorgebracht worden wäre.

Widerstand zu leisten, war also nicht von mir zu verlangen an dem
schönen Morgen. Ich nahm ihn mit allem, was er an süßen Reizen
brachte, hielt mich durchaus nicht länger beim gestrigen Abend auf,
sondern fragte nur im logik-vergessensten Behagen:

»Herz, mein Herz, was sagst du heute zu unserm Leben und zu
Pfisters Mühle?«

»Himmlisch ist's, Männchen, und bei solchem Wetter, ehe der Tag zu
heiß wird, wirklich schade, daß es so bald damit aus und vorbei ist
– eure Pfisters Mühle meine ich natürlich. Läge sie nur ein bißchen
näher bei den Leuten, so wär's zu hübsch, alle paar Jahre uns
wieder mal in die Stille hinzusetzen! Ja, wovon ich geträumt habe,
fragtest du? Natürlich von schlechten Gerüchen, von ganz
greulichen, und von großer Wäsche bei uns in Berlin, und von Doktor
Asche, aber wie gesagt, hauptsächlich von schrecklichem Gestank,
grade wie du mir vorher davon erzählt hast. Habe ich nicht geächzt
im Schlafe? Nicht? Na, dann ist es einfach zu arg darin gewesen,
und ich habe nicht gekonnt. Übrigens begreife ich jetzt an diesem
reizenden Morgen keinen von euch allen – deinen seligen Papa nicht,
dich nicht und eure Gäste auch nicht mit ihrem Naserümpfen. Doktor
Asche hatte ganz recht, daß er gar nichts auf eure Querelen gab,
sondern sich bloß ganz einfach über euch lustig machte mit seinem
eigenen gelehrten, scheußlichen und wissenschaftlichen Geruch zum
Besten der Welt und der Industrie\ldots{} Aber heiß wird es heute
werden, und da wird es heute in Berlin schrecklich sein, und es ist
wirklich himmlisch, Ebert, daß wir hier jetzt so in der wonnigen
Kühle und der Sonne und dem Tau sitzen und uns auch den ganzen Tag
über von einem schattigen Sitz auf den andern ziehen können. Wie
schade, daß wir nicht Frau Albertine und den Doktor bei uns haben!
Die werden heute auch genug von der Hitze in Berlin ausstehen
müssen.«

Die Kleine hatte wie gewöhnlich recht. Es wurde sehr heiß an diesem
Tage, so heiß, daß wir uns nach Mittag aus dem schwülen Garten doch
ins Haus und im Hause an den kühlsten Platz verzogen.

Der kühlste Platz aber war die Mühlstube oder, wie der
wissenschaftliche Mühlengelehrte das heute nennt, die
Turbinenstube.

Ich bin ein ungelehrter Müllerssohn und sonst im Leben ein
einfacher Schulmeister, der sich bescheiden wegduckt und in den
Winkel drückt mit seinem Griechischen und Lateinischen, wenn die
Tagesherrin, die reale Wissenschaft, mit ihren philosophischen
Ansprüchen und gelehrten Ausdrücken kommt. Ich fand es wie Emmy
ebenfalls am kühlsten in der Mühlstube von meiner Väter Mühle und
ließ in urältester Weltweisheit den Wassern draußen ihren
rauschenden Weg vorbei an den nutzlosen, gestellten Rädern.

In der Mühlstube von Pfisters Mühle habe ich Emmy von Frau
Albertine Asche und ihrem Mann, da wir sie in Person leider nicht
bei uns haben konnten in der Kühle, weiter erzählt und – mir auch.

Es standen die Türen aller Räume des verkauften Hauses offen, und
so hatten wir von dem Tischchen aus, das wir uns in unsern
Zufluchtsort getragen hatten, den Ausblick über den Flur auch in
das alte, jetzt vollständig leere Gastzimmer. Das Beste war, man
brauchte sich in \emph{dieser} Sommerfrische gar keinen Zwang
anzutun. Hemdärmelig ging ich im Schwatzen mit meiner Zigarre herum
um Trichter und Beutelkasten, um Ober- und Untermühlstein, lehnte
am Kammrad und trat auch wohl auf den Hausflur und schritt in der
Gaststube auf und ab. Letzteres aber nie allzulange. Die Schritte
klangen zu hohl in dem geleerten Raume.

Wo bleiben alle die Bilder?

Nun waren wir, Emmy und ich, wieder auf der Landstraße mit dem
Freunde und chemischen Doktor Adam Asche, und Emmy meinte:

»Daß die Geschichte im Winter liegt, ist heute wirklich sehr
angenehm bei der schrecklichen Temperatur. In der Wüste Sahara oder
unter dem Äquator hielte ich es selbst in der Idee nicht aus.~–

Im Winter lag freilich die Geschichte. Es war auf der Chaussee bei
jener Wanderung zu meinem damaligen Christbaum in Pfisters Mühle
ganz das Wetter, welches sich Freund Asche für den Weg gewünscht
hatte. Der Wind pfiff uns schneidend um die Ohren, und wir hatten
nicht wenig zu lavieren, um ihm die beste Seite abzugewinnen und
immer querüber weiterzukommen. In der Stadt herrschte, als wir sie
hinter uns ließen, all das Leben, welches der letzten Stunde vor
dem Anzünden des ersten Lichtes an der Tanne voranzugehen pflegt.
Sie liefen noch in den Gassen – die Landstraße hatten wir für uns
allein, nachdem wir die Fabriken am Wege, die ihre Tätigkeit auch
heute abend nicht aussetzten, die Region der »Bockasche«, passiert
hatten.

Die Fabriken erstrecken sich heute schon so ziemlich bis an das
Dorf hin und die Region der Bockasche also ebenfalls. Damals waren
zwei Drittel des Weges noch frei davon, und nur vereinzelte
Häuschen kleiner Leute lagen an diesem Wege, im Rücken das freie
Feld.

In dem letzten dieser Häuschen, nach dem Dorfe zu, sahen wir die
ersten flimmernden Weihnachtskerzen durch das beschweißte Fenster.
Als wir die Mühle erreichten, war es vollkommen Nacht.

»Schwefelwasserstoff! und\ldots{} Gänsebraten!« ächzte A.~A.~Asche unter
dem guten, alten Schenkzeichen, in vollster Gewißheit seines
chemischen und kulinarischen Verständnisses mit erhobener Nase den
Duft in der Haustür einziehend. »Keine andere Diagnose möglich am
Krankenbette!\ldots{} Vivat die Wissenschaft!\ldots{} Gänsebraten heute
gottlob vorherrschend! Pfisters Mühle mit allen ihren Gerüchen in
Ewigkeit!«

»Ich danke, Doktor Adam«, sagte mein Vater auf der Schwelle seiner
gastlichen Pforte.~–~–

Wo bleiben alle die Bilder und – die Gerüche in dieser Welt? Es
riecht heute nicht nach Gänsebraten und (da es Sommer ist) auch
nicht nach Schwefelwasserstoff, Ammoniak und salpetriger Säure. Ein
feiner, lieblicher Wohlduft hat eben die Oberhand und stammt von
Emmy, aus ihrem Nähkasten und dem Gewölk feinen Weißzeuges, das sie
auf Tisch und Stuhl um sich versammelt hat, und wirkt berauschender
und mächtiger als sonst ein Duft aus der alten Hexenküche, Erde
genannt. Die heiße Julisonne fällt durch jeden Ritz und Spalt in
die kühle aufgegebene Mühle. Die Stuben sind, wie gesagt,
ausgeleert von Gerätschaften, und selbst die Fliegen haben nur ihre
vertrockneten Leichname in den staubigen Fenstern der Wohn- und
Gaststube zurückgelassen. Es ist ja ein Wunder, wie Christine das
Notwendige für unsere wunderliche, mir so märchenhafte Villeggiatur
für uns zusammengebracht hat und wie uns, meinem jungen Weibe und
mir, eigentlich nichts, \emph{gar nichts} mangelt, obgleich wir
allstündlich so manches vermissen.

»Das spricht eigentlich doch für vieles, Emmy – was?«

»Du dummer Mann – natürlich!\ldots{} aber ärgerlich ist's doch, daß ich
nicht damals schon mit dabeigewesen bin. Jetzt erzähle nur zu,
närrisches Menschenkind. Da, fädele mir aber erst meine Nadel ein.
Die Nähmaschine hätten wir doch mit herausbringen sollen.~–

»Na, da seid ihr ja endlich! Seit Stunden guckt man nach euch aus«,
sagte mein Vater, mit einer Laterne und einem Korb voll Flaschen
eben aus der Kellertiefe und -tür emporsteigend. »Halt mal das
Licht, Junge«, sagte er, mir die Laterne reichend und mit der
freigewordenen Hand meinen Begleiter am Oberarm packend und ihn
unter dem Tor festhaltend. Ȁrger denn je! Na, was meint Ihr,
Doktor?«

»Ganz wie ich es mir gedacht habe«, meinte grinsend Freund Asche.
»Es war Gott sei Dank immer eine nahrhafte Hütte, Vater Pfister.
Der Vogel gehört vollkommen in den heutigen Abend, und wenn ich
sagen würde, daß ich nicht auf ihn in der Bratpfanne gerechnet
hätte, so löge ich.«

»Sapperment, meine ich \emph{den} Geruch?« brummte der alte Herr.
»Was geht in diesem Ärgernis von Gedünsten mich das an, was aus
meiner Küche kommt?«

»Hm«, sprach Doktor Asche, »von dem übrigen lieber morgen. Jaja –
industrielle Blüte, nationaler Wohlstand und – Ammoniak nicht zu
verkennen, trotz aller Füllung mit Borsdorfer Äpfeln.~–

Einen Augenblick sah Vater Pfister seinen Günstling und Gastfreund
an, als wisse er nicht recht, ob er ihm nicht noch etwas zu
bemerken habe; dann aber, seine Müllerzipfelkappe vom rechten aufs
linke Ohr schiebend, meinte er mit dem alten, behaglichen, guten
breiten Lächeln: »Na, im Grunde habt Ihr recht, und so will ich's
auch noch mal versuchen, mir den Appetit nicht verderben zu lassen.
So kommt herein in die Stube, junge Leute, und seid willkommen in
Pfisters Mühle.«

Umsprungen und umwedelt von allen Hunden des Hauses traten wir in
die Stube und nahmen den Flaschenkorb mit hinein. Oh wie die Mühle
an jenem Abend noch voll war von allem, was zur Behaglichkeit des
Lebens gehört! Und wie angenehm es war, aus der Kälte in die Wärme,
aus der Dunkelheit in den Lampenschein, von der Landstraße in die
Sofaecke hinter geschlossenen Fensterläden zu kommen!

Meiner Väter Hausrat noch überall an Ort und Stelle – die
Kuckucksuhr im Winkel, die Bilder an den Wänden (nur die Herren
Studenten und der Liederkranz hatten ja ihre Massengruppierungen in
Lithographie bis jetzt weggeholt), der ausgestopfte Wildkater in
seinem Glaskasten über der Kommode und die zahme Hauskatze am Ofen,
sich bis über die Ohren putzend, weil Gäste kommen sollten! Es ist
nicht auszusagen, wo alle die Bilder bleiben.~–~–

Die Gäste, die kommen sollten, waren wir – ich, der Haussohn, und
Doktor Asche, der gerufen worden war, um dem Behagen von Pfisters
Mühle den Puls zu fühlen; aber es waren auch schon Gäste vorhanden,
derentwegen Miez am Ofen sich dreist über die Ohren putzen durfte.
Der lange Tisch, der sich sonst unter gewöhnlichen Umständen die
eine Wand entlang vor der Bank herzog, war in die Mitte der
Gaststube gerückt, mit einem weißen Laken überzogen und mit allem
versehen, was in Pfisters Mühle zu einer festlichen Tafel gehörte.
Auf der Bank, die sie demnächst an den Tisch nach sich rücken
sollten, saßen die Knappen und der Junge in ihren reinlichen
Müllerhabitern (wie die weißen Tauben auf dem Dachfirst, meinte
Asche), und hinter einer geputzten Tanne stand Samse (wie der
Feuerwerker hinter der Kanone, meinte Asche), bereit, auf den
ersten Wink von Vater Pfister loszubrennen, das heißt, die gelben,
grünen, roten Wachslichter zwischen den vergoldeten Nüssen und
Äpfeln, den Zuckerherzen und allem, was sonst Christine aus der
Stadt zum Zweck mitgebracht hatte, anzuzünden. Christine selbst
freilich scharwerkte in Verbindung mit den beiden Mägden der
Wirtschaft noch aufgeregt in der Küche und hatte mir vorerst nur
eine feuchte und nach einem Gemisch von Zwiebeln und Zitronen
duftende Hand zum Willkommen durch die Türspalte reichen können.

Es war ihnen gottlob allen lieb, daß wir endlich da waren. Sie
kamen sämtlich bei unserm Eintritt in Bewegung.~–

»Ran mit der Lunte, Samse!« kommandierte mein Vater, und über alles
Begrüßungsgetöse von Vater Pfisters Weihnachtsgesellschaft klang
eine tiefe, klangvolle Stimme:

»Willkommen im Hafen, meine Herren!«

Man muß sich immer erst eine Weile an das Licht gewöhnen, wenn man
von der Landstraße, aus der Nacht und dem scharfen Nordost kommt.
Wir hielten beide noch die Hände über die Augen; aber jene Stimme
kannten wir seit lange bei Nacht und bei Tage.

»Je, auch der Sänger!\ldots{}.. Vater Pfister, Sie sind wie immer der
Meistermann!\ldots{} Lippoldes! Natürlich – zu dem Guten bringt er das
Beste! Guten Abend, göttergeweihter – alter Freund.«

»Ich erlaube mir, Ihnen meine Tochter vorzustellen, Asche. – Meine
Tochter – Herr Doktor Asche! – Herr Eberhard Pfister junior – meine
Tochter Albertine! Ja, Ihr Herr Vater war so freundlich, uns zu dem
heutigen Festabend einzuladen, lieber Ebert!«

\section{Zwölftes Blatt}

\zusatz{Unter Vater Pfisters Weihnachtsbaum}
Ich habe mein Teil, und glücklicherweise ist es auch seine oder
ihre Meinung, daß das ein Glück sei! Da sitzt es oder sie in der
Turbinenstube mit dem Nähzeug im Schoß und läßt sich von mir in
Ermangelung eines Interessanteren von Pfisters Mühle erzählen in
der Villeggiatur. Reizend sieht es aus, mein bescheiden lieblich
Teil, neben dem Beutelkasten. Ich weiß nichts Hübscheres in aller
weiten und nahen Welt als mein mir beschieden Teil, wie es dasitzt
an unserm Tischchen vor dem stillen Kammrad und den unbeweglichen
Mühlsteinen, mit dem heißen Tag draußen und dem Fluß, der für jetzt
noch munter fort und fort rauscht durch den jetzt so nutzlosen
Mühlrechen. Um den Wellbaum herum sucht sich die Sonne aber doch
wieder ihren Weg in unsern kühlen Schlupfwinkel und zu meinem
jungen Weibe, grad als ob auch sie mir eben mein wonniglich Teil
vom Glücke dieser Erde in das beste Licht zu stellen den Auftrag
erhalten habe.

Ganz unnötigerweise. Sie sind ja, Gott sei Dank, die besten
Freundinnen geworden – Frau Doktor Pfister und Frau Doktor Asche.
Sie (Frau Doktor Emmy) wünscht es ja, daß ich ihr von ihr (Frau
Doktor Albertine) mehr und genauer Bericht tue als von irgend etwas
anderm aus der Zeit des Niedergangs von Pfisters Mühle. Es ist
keine Gefahr für unsern häuslichen, ehlichen Frieden dabei, daß
auch andere ihr hübsches Teil von der Welt bekommen sollen. Ich
kann weiter erzählen von Fräulein Albertine Lippoldes und dem armen
Schelm, ihrem Papa, unter meines Vaters Christbaum und an seinem
Weihnachtstische und leider auch im, trotz aller Christfestdüfte,
nicht wegzuleugnenden Ammoniak- und Schwefelwasserstoff-Geruch von
Pfisters Mühle.~–

Ach, daß es so häufig, wenn man der nicht mehr vorhandenen Bilder
gedenkt, nötig ist, so pragmatisch als möglich zu sein, sobald man
von ihnen reden oder gar schreiben will! Wie strahlte Samses Visage
in dem Lichte, das von ihm selber ausging – welch eine Gloriole
umgab Fräulein Albertinens müdes, freundliches Gesicht vor dem
grünen, leuchtenden Tannengezweig – wie hübsch war das Bild im
ganzen, meines Vaters Weihnachtsstube mit allen ihren Hausgenossen
und Gästen in Pfisters verstänkerter Mühle! Wie ließe sich davon
singen und sagen – märchenhaft wundervoll: ich aber habe nüchtern
von Felix Lippoldes und seiner Tochter zu berichten.

Nüchtern von Felix Lippoldes! Es gibt noch einige Leute, die noch
wissen, wie schwer das, aller Pragmatik in der Welt zum Trotz,
seinerzeit war. Am einfachsten ist's hier, ich erzähle nicht, wie
ich meiner Frau erzählte, sondern ich schreibe ab aus einer andern
Biographie, einem Buche, welches durchaus nicht von meines Vaters
Mühle und von Felix Lippoldes handelt, in welchem aber der Name des
früheren Besitzers, Doktor Felix Lippoldes, auf der ersten Seite
stand und welches \emph{nicht durch Zufall} unter die wenigen Bände
meiner Reisebibliothek geraten war.

»Mittlerweile hatte einer auf die Straße gesehen und rief nun:
»›Sieh, da geht er hin!‹ – ›Wo, wo?‹ Und alles drängte sich an die
Fenster. Und er ging dahin, ein trauriger Aufzug. Seine Kleidung
schien sehr abgetragen und saß sehr nachlässig; der braune
Frackrock war hinten am Ellenbogen schon ziemlich weiß geworden,
und die weite, schwarze Hose wehte sehr melancholisch um seine
dünnen Beine; die dunkle Weste war bis unter dem Halse zugeknöpft,
seine grobe Halsbinde ließ nichts Weißes sehen, und auf dem Kopfe
trug er eine alte grüne Mütze. In seinem ganzen Körper war kein
Halt, er wankte so, daß man fast befürchten mußte, er möchte
umfallen; nur langsam bewegte er sich fort nach seiner Weise, wo er
die Spitzen der Füße wie zufühlend voraussetzte.~–

›Gott, wie betrübt! Nein, so traurig hätt? ich mir's nicht
vorgestellt!‹ sagte man. ›Der lebt keinen Monat mehr, es ist aus
mit ihm. Übrigens ist es nur gut, er sehnt sich gewiß auch selbst
nach dem Tode. Er hat offenbar die Schwindsucht. Der verfluchte
Rum!‹~–

Inzwischen kam er an ein paar Knaben vorbei, welche ihm aus dem
Wege gingen, ihn anstaunten und die Mützen zogen.

Als er durch das Abnehmen seiner Mütze wieder grüßte, konnte man
wahrnehmen, wie sehr ihm das Haar ausgegangen war, sein Kopf war
beinah kahl, nur hin und wieder flatterte eine einsame Locke im
Winde. Dabei lag auf seinem abgemagerten Gesichte eine tiefe
Blässe, eine dicke Finsternis lagerte sich auf seiner hohen Stirne,
ein Gewitter um den Olymp, aber die Blitze seiner Augen waren sehr
matt.

›Sieh, er fällt vor Mattigkeit.‹ ›No, no; es geht noch einmal. Ach,
grad wie ein Landläufer.‹«

Er ist ja leider keine vereinzelte Tragödie in der Welt und der
Literatur, der verlorengehende Tragöde, und er hat trostloserweise
nicht immer das Glück, so unbemerkt, unbeschrieen und unbeschrieben
vorbeizutaumeln wie der arme Felix. Sie haben sie nur zu häufig in
ihrem Spiritus aufbewahrt, in Detmold, in Leipzig, in Braunschweig
und an mancher andern größern und kleinern »Kulturstätte«, diese
Hohenstaufen- und Französische-Revolutions-Dramatiker – die
verunglückten Terzinen- und Stanzenepiker, die unausgegorenen
Lyriker – all das ruhelose, unglückselig-selige Zwischenreichsvolk,
von dem Annette Droste-Hülshoff meinte, daß es dann und wann viel
mehr wert sei und bedeute, als – viele andere. Es konnte wahrlich
nicht in meiner Absicht liegen, den Dichter der Thalatta, des
Alarich in Athen usw. usw., Felix Lippoldes, in meinen Geschichten
von Pfisters Mühle auch noch meiner Emmy als abschreckendes
literarisches Beispiel aufzustellen; unter manchen, die das nicht
leiden würden, ist eine vor allen, seine liebe Tochter Albertine,
die seinetwegen aus England zurückgekommen war und mit ihm in
unserm Dorfe sein letztes armseliges Obdach teilte.

Wir hatten alle, in der Stadt, an der Universität, auf den
gelehrten Schulen, längst genug von ihm gewußt, aber eigentlich
nicht das geringste von dieser Albertine, seiner klugen, braven,
tapfern Tochter, obgleich selbst wir, die wir noch »auf Schulen
gingen«, unsere Glossen so gut darüber machten wie die älteren
Herrschaften, denen vor Zeiten seine Verheiratung so unendlichen
Stoff zur Unterhaltung gab, sowohl im wissenschaftlichen Kränzchen
wie hinter dem Bierkrug und am Tee- und Kaffeetische. Zu welchen
Hoffnungen er in seinen jüngern, besseren Jahren im Kreise seiner
Altersgenossen und als Dozent der klassischen Philologie an unserer
Universitas litterarum berechtigt haben mochte: die schlimmsten
Befürchtungen, die man in betreff eines zu gescheiten, zu nervösen
und zu phantasiereichen Menschen haben kann, waren eingetroffen.
Nun vegetierte er in unserm Dorfe in einer Bauernstube, die im
Sommer auf den Landaufenthalt der unbemittelten Honoratioren der
Stadt sich eingerichtet hatte, und seine Tochter war aus England,
wohin sie als Gouvernante gegangen war, zurückgekommen, um ihm –
leben zu helfen.

Ich tue mein Bestes, ihn Emmy zu schildern, wie er vor mir steht,
nicht der dramatische Poet Felix Lippoldes, sondern dieser Heilige
Abend, bei dem auch noch der arme Felix zugegen war. Ach, wie meine
Schritte hohl widerhallen in den ausgeleerten Räumen der
verkauften, verlassenen Mühle! Wie leuchtete Samses wetterfestes
Gesicht unter den Lichtern, die er auf den grünen Zweigen
angezündet hatte, wie gab mein Vater alles her, was er an
Wohlwollen und Fröhlichkeit in seinem guten Herzen hatte! Unter der
Tanne saß er, mit seinem Samse als Hofmarschall hinter sich, und um
ihn her alles, was Weihnachten mitfeierte in Pfisters Mühle. Wie
die Welt, wie die Zeit, die sich augenblicklich die neue nannte,
andringen mochten, wie es draußen riechen mochte: in Vater Pfisters
Gaststube war noch einmal alles beim alten.

Sehr merkwürdig war das Verhalten Asches.

Noch bis vor die Tür hatte er augenscheinlich die feste Absicht
mitgebracht, sich so toll, ausgelassen und närrisch wie nur möglich
zu behaben und den Ironiker beim bis Feste an die Grenzen des
Hanswursts hinan zu agieren. Viel Gewissen hatte er seinerzeit in
dieser Hinsicht eigentlich nicht, und ein erklecklich Teil von dem,
was er heute in der Beziehung sein nennt, kommt vielleicht auch mit
auf Rechnung jenes Weihnachtsabends.

Es kam einmal wieder ganz anders, als wie der Mensch sich's gedacht
und vorgenommen hatte; der erste Anblick des Poeten aber tat
wahrlich nichts, die Lust des Schalks am Spiel mit der Welt zu
dämpfen. Im Gegenteil, nachdem er alle übrigen in seiner Weise
begrüßt und sich von der Christine einen Rippenstoß und die Weisung
geholt hatte: »Gehn Sie weiter, Nichtsnutz!« schien er fernerhin
sich ganz dem Dichter widmen zu wollen.

»Ne, wie Sie mich freuen, Lippoldes!« rief er. »Sie hat mir das
Christkind ganz speziell für mich unter den Baum gelegt, und den
Stuhl da neben Ihrer Sofaecke kriege ich natürlich auch. Vater
Pfister, Sie wissen doch immer zu dem Guten das Beste zu finden;
schmerzten mich nicht noch meine Rippen so sehr, hätten Sie jetzt
schon den Kuß, den Jungfer Christine eben so schnöde verschmäht
hat!\ldots{} Das hätte ich schon auf dem Wege ins
Schwefelwasserstoffhaltige wissen sollen, daß ich Sie in dem
Gewölke schwebend erblicken würde, Lippoldes; meine Schritte und
die des Knaben an meiner Seite würden sich um ein beträchtliches
beschleunigt haben. Das ist ja zum Radschlagen gemütlich! Seit
einer halben Ewigkeit hat man sich nicht gesehen. Nun, Olympier,
wie ging es denn während der ganzen Zeit im ewigen Blau?«

Seit er uns zu unserer glücklichen Ankunft im Hafen beglückwünscht
hatte, hatte Doktor Lippoldes sich nur in seiner Sofaecke geregt,
um aus dem Schatten und Tabaksrauch eine dürre, zitternde Hand nach
dem dampfenden Glase auszustrecken; jetzt erwiderte er mit matter
Gleichgültigkeit:

»Wenden Sie sich mit der Frage an meine Tochter, lieber Asche.«

»Mein lieber Vater!« sagte Albertine Lippoldes, auf ihrer Seite
näher an den armen Mann rückend und den Arm um seinen Nacken
legend. Dabei hat ein bei weitem mehr gleichgültiger als drohender
Blick meinen guten Freund Asche gestreift, und von diesem
Augenblick an ist der ein verlorener, das heißt gewonnener Mensch
gewesen und hat sich, wie gesagt, selten an einem fidelen Festabend
so anständig betragen wie an diesem. Wer dies aber gegen
Mitternacht hin nicht mehr vermochte, das war Doktor Felix
Lippoldes.

Um jene späte Zeit stand Felix Lippoldes nicht etwa bloß auf einem
Stuhl, sondern mitten auf dem Weihnachtstische in Pfisters Mühle,
das graue Haar zerwühlt, das schäbige Röcklein halb von den
Schultern gestreift, und deklamierte mit finsterm Pathos:

\begin{verse}
»Einst kommt die Stunde – denkt nicht, sie sei ferne~–\\
Da fallen vom Himmel die goldenen Sterne,\\
Da wird gefegt das alte Haus,\\
Da wird gekehrt der Plunder aus.\\
Der liebe, der alte, vertraute Plunder,\\
Viel tausend Geschlechter Zeichen und Wunder:\\
Was sie sahen im Wachen, was sie sannen im Traum,\\
Die Mutter, das Kind, die Zeit und der Raum!\\
Kein Spinnweb wird im Winkel vergessen,\\
Was der Körper hielt, was der Geist besessen,\\
Was das Herz gefühlt, was der Magen verdaut;\\
Und \emph{Tod} heißt der Bräutigam, \emph{Nichts} heißt die Braut!«
\end{verse}

Mit offenem Munde, den Bowlenlöffel in der Hand, stand mein Vater
vor seiner größten Punschschale. Sie hatten alle die Stühle
zurückgeschoben oder waren von ihnen aufgesprungen und drängten
sich um den leider in gewohnter Weise außer sich geratenen Poeten
halb lachend, halb verblüfft – mit vollem Verständnis für das Ganze
wohl nur Asche, ich und – eine leise, klagende, bittende Stimme in
dem lustigen Lärm:

»Vater! Lieber, lieber Vater!«

»Gott bewahre mich in seiner Güte«, rief \emph{mein} Vater, »hab
ich Sie darum in meiner Bedrängnis höflich um ein vergnügtes
Weihnachtspoem ersucht, Doktor Lippoldes, um mir so von Ihnen den
Teufel noch schwärzer an die Wand von meiner Mühle malen zu lassen?
Da kommen Sie doch lieber runter vom Tische und lassen Sie ihren
Kollegen in der Phantasie rauf! Adam, so reden Sie doch mit ihm!
Sie haben doch sonsten das gehörige Getriebe zur Verfügung und
sitzen mir heute den ganzen Abend da, als wären Ihnen Bodenstein
und Laufer zugleich geborsten, der Fachbaum gebrochen und das
Wasser überhaupt ausgeblieben. O Fräulein Albertine, beruhigen Sie
sich: wir sind ja ganz unter uns! Das ist ja das einzige Gute
jetzt, daß Pfisters Mühle meistens ganz unter sich ist und ihren
Spaß in jeder Art für sich allein hat.«

Unter den Gläsern und Schüsseln des Weihnachtstisches vor der
erloschenen Tanne von einem Fuße auf den andern springend,
kreischte Felix Lippoldes:

\begin{verse}
»Wie schade wird das sein! Dann kehrt man dort\\
Den guten Kanzeleirat weg und seinen Stuhl,\\
Auf dem er fünfzig Jahr lang kalkulierte.\\
Vergeblich wartet mit der Suppe seine Alte,\\
Nicht lange doch; denn plötzlich füllt ein mächtiges\\
Gestäub die Gasse, dringt in Tür und Fenster~–\\
Der Kehrichtstaub des Weltenuntergangs.«
\end{verse}

»Hm«, murmelte Adam Asche, an meiner Seite beide Ellenbogen auf das
Tischtuch stützend.

\begin{verse}
»Sehr drollig wird das sein für den, der da zuletzt lacht,\\
Sieht er im Wirbel fliegen, was ihn quälte,\\
Bis selber ihn der letzte Kehraus faßt.«
\end{verse}

Zwei Stunden später saß er trotz der kalten Nacht noch längere Zeit
in unserer Kammer unter dem Dache auf dem Bettrande, und einmal
hörte ich ihn vor sich hinbrummen:

»Das ist wirklich ein merkwürdig nettes Mädchen – ein ganz liebes
Kind und, wenn der erste Eindruck nicht vollkommen täuscht, auch
gar nicht dumm!«

\section{Dreizehntes Blatt}

\zusatz{Vater Pfisters Elend unterm Mikroskop}
Am andern Morgen begannen wir (nicht Emmy und ich: wir halfen den
Bauern im Dorfe beim Heumachen und kamen erst am Abend zu den
Geschichten von Pfisters Mühle zurück) die wissenschaftlichen
Forschungen und beschäftigten uns mit den ersten Vorbereitungen zu
der Diagnose, behufs welcher Doktor Asche von meinem Vater an das
Krankenbett seiner einst so gesunden, fröhlichen Wirtschaft berufen
worden war.

»Es ist freilich arg!« sagte der sonderbare Mühlenarzt und
Wasserbeschauer, als er die Nase aus dem Fenster unterm Dachrande
in den grauen, feuchtkalten Morgen hinausschob und sie niesend
wieder zurückzog. »Hm, und auch nur, weil die Menschheit ihre Welt
nicht süß genug haben kann!«

Wir stiegen hinab in die Weihnachtsstube und fanden sie zwar gefegt
und zurecht rückt, aber doch auch voll seltsamer Dünste, die nicht
bloß von dem vergangenen lustigen Abend her an ihr hafteten. Die
Tanne war bereits in den Winkel geschoben, und am Tische saß mein
Vater in seiner Hausjacke, wenig festtäglich gestimmt.

»Die Leute und die Weibsleute gehen ins Dorf in die Kirche, und ich
würde auch hineingehen und euch zwei Heiden mitnehmen, wenn es mir
noch so wäre wie vor Jahren und als deine selige Mutter noch bei
uns war, Ebert; aber das Gemüte ist mir nicht mehr darnach, und
ändern kann ich's leider nicht. Setzt euch und trinkt Kaffee. Wir
haben seit Jahrhunderten in unserer Mühle unsern Stolz an unserm
Oster-, Pfingst- und Weihnachtskuchen gehabt, aber auch er ist mir
nicht mehr derselbige, sondern riecht und schmeckt mir nach
Vergiftung und Verwesung; und alle blutigen Tränen, die mir die
Christine hinweint, wenn ich ihr den Teller zurückschieben muß,
helfen nichts dagegen. Freßt euch hinein, liebe Jungen, und Gott
segne euch euern bessern Appetit und eure grünere Hoffnung! Nachher
wollen wir dann in Teufels Namen in der Mühlstube die Nase so voll
als möglich nehmen und sehen, ob es wirklich von Nutzen ist, was
Sie gelernt und getrieben haben die langen Jahre durch, Adam. Uh,
das wäre dann \emph{meine} Weihnachtsbescherung!«

Über unsere Würdigung ihres Feiertagsgebäcks hatte unsere Christine
keine Tränen zu vergießen. Wir fraßen uns tief genug hinein in die
Berge, die sie vor uns aufgehäuft hatte und – hoffentlich wird sie
mir noch zu manchem Feste in Berlin denselben Kuchen backen, wegen
dessen Pfisters Mühle vordem so berühmt war.

In er Turbinenstube hatten wir dann mit Vater Pfister das Reich und
den Geruch ungestört zu unserer gelehrten Disposition. Ob die
Knappen wirklich sich in der Kirche befanden, wie der alte Mühlherr
voraussetzte, kann ich nicht sagen; aber gegenwärtig waren sie
nicht, und das Rad stand, und wir standen auch und schüttelten die
Häupter.

Es war sehr arg!

»Mit der Nase brauche ich keinen draufzustoßen«, ächzte mein Vater;
»aber die Augen und das Gefühl sollen ja auch das ihrige haben! Ja,
sehen Sie sich nur um, Doktor, und dann seien Sie hier mal der
Müller, der seit Jahrhunderten das klar wie 'nen Kristall und
reinlich wie 'ne Brautwäsche gekannt hat! Da, guck, Junge, und
streif dir meinetwegen den Ärmel auf und greif in das
Einflußgerinne und fühle, was für einen Schleim und Schmier deiner
Vorfahren hell und ehrlich Mühlwasser mir heute in meinem Gewerk
und Leben absetzt. Ja, holen Sie sich dreist eine Handvoll vom
Rade; es ist mehr davon vorhanden und wird gern vermißt. Und,
junges Volk, ihr lacht darüber, oder wenn ihr das jetzt nicht wagt,
so haltet ihr mich für einen alten Narren; aber mir ist das doch
wie ein Lebendiges, zu dem ich den Doktor habe rufen müssen, um ihm
den Puls zu fühlen. Und der Puls von Pfisters Mühle geht langsam,
Ebert Pfister! Und wer weiß, wie bald er ganz stille steht!«

Bei Gott, mir war nicht lächerlich zumute diesem alten, vor Ingrimm
und Betrübnis zitternden braven Manne und noch dazu meinem Vater
gegenüber und auf meiner Väter in Ehren, Leiden und Freuden von
Geschlecht zu Geschlecht vererbtem Grund und Boden! Da rauschte
milchigtrübe, schleimige Fäden absetzend, übelduftend der kleine
Fluß unbeschäftigt weiter in den ersten Christtag. Christtäglich,
weihnachtsfestlich war mir nicht zu Sinne, und in Spannung und fast
in Angst sah ich auf meinen chemisch und mikroskopisch gelehrten
Freund und Exmentor, der eben die schleimschlüpfrige Masse, die er
aus dem Getriebe entnommen hatte, von der Hand abspülte.

»Asche, du weißt offensichtlich, an was und an wen wir uns zu
halten haben?« rief ich. »Ich bitte dich, Adam, treibe keinen Spaß
zur unrechten Zeit«, flüsterte ich ihm zu.

»Liegt durchaus nicht in meiner Absicht. Weniger weil, sondern
obgleich ich der Sohn eines Schönfärbers bin«, meinte der Doktor
mit der vollen Ruhe und Gelassenheit des Mannes der Wissenschaft,
des an ein Krankenbett gerufenen sichern Operators. »Das Ding kommt
mir viel zu gelegen, um es scherzhaft aufzufassen. Vater Pfister,
vielleicht hätten Sie mich nicht gerufen und zum Christbaum
eingeladen, wenn Sie eine Ahnung davon hätten, wie sehr ich Partei
bin diesen trüben Wellen und kuriosen Düften gegenüber. Aber ich
habe Pfisters Mühle viel zu lieb, um nicht völlig objektiv meine
Meinung über ihr Wohl und Wehe begründen zu können. Augenblicklich
erkenne ich in der Tat eine beträchtliche Ablagerung niederer
pflanzlicher Gebilde, worüber das Weitere im Verlaufe der Festtage
das Vergrößerungsglas ergeben wird. Pilzmassen mit Algen überzogen
und durchwachsen, lehrt die wissenschaftliche Erfahrung. Aber was
für Pilze und welche Algen bei gegebener Verunreinigung der Adern
unserer gemeinsamen Mutter? \emph{Das} herauszukriegen im eigenen
industriellen Interesse, würde dann wohl \emph{meine}
Weihnachtsbescherung sein, mein Sohn Eberhard!«~–

Wir stellten das Mikroskop in die wenigen, hellen Stunden des
ersten Christtages, und der Doktor begab sich an die genauere
Untersuchung des Unflats mit der Hingebung, welche Vater Pfister
aus früherm, schönerm Verkehr mit der Universitas litterarum nur
als »Biereifer« bezeichnen konnte. Und begreiflicherweise taten
Vater Pfister und sein Stammhalter nicht das geringste, diesen
Eifer zu dämpfen. Sie hielten sogar die Stubentür verriegelt und
saßen stumm, mit den Händen auf den Knieen, und hielten dann und
wann sogar den Atem an, wenn der Mann der Wissenschaft zu einem
neuen Resultate gelangt war und uns daran teilnehmen ließ.

»Wie ich es mir gedacht habe, was das interessante Geschlecht der
Algen anbetrifft, meistens kieselschalige Diatomeen. Gattungen
Melosira, Encyonema, Navicula und Pleurosigma. Hier auch eine
Zygnemacee. Nicht wahr, Meister, die Namen allein genügen schon, um
ein Mühlrad anzuhalten?«

»Das weiß der liebe Gott«, ächzte mein Vater.

»Jawohl, groß ist sein Tiergarten«, meinte ruhig Adam Asche. »Was
die Pilze anbetrifft, so kann ich leider nicht umhin, Ihnen
mitzuteilen, daß sie den Geruch, über den Sie sich beklagen, Vater
Pfister, durch ihre Angehörigkeit zu den Saprophyten, auf deutsch:
Fäulnisbewohnern, vollkommen rechtfertigen. Was wollen Sie denn
eigentlich, alter Schoppenwirt? Ein ewig Kommen und ein ewig Gehen!
Haben die Familien Schulze, Meier und so weiter den Verkehr in
Pfisters Mühle eingestellt, so haben Sie dafür die Familien der
Schizomyceten und Saprolegniaceen in fröhlichster Menge, sämtlich
mit der löblichen Fähigkeit, statt Kaffee in Pfisters Mühle zu
kochen, aus den in Pfisters Mühlwasser vorhandenen schwefelsauren
Salzen in kürzester Frist den angenehmsten Schwefelwasserstoff zu
brauen. Lauter alte gute Bekannte – Septothrix, Ascococcus
Billrothii, Cladothrix Cohn und hier~–«

Er richtete sich auf von seinem Instrument und seinen
Vergrößerungsobjekten. Er fuhr mit beiden Händen durch die Haare.
Er blickte von dem Vater auf den Sohn, legte lächelnd dem Vater
Pfister die Hand auf die Schulter und sprach, was ihn selber
anbetraf, vollkommen befriedigt und seiner Sache gewiß:

»Beggiatoa alba!«

»Was?« fragte mein Vater. »Wer?« fragte er.

»Krickerode!« sagte Doktor Adam Asche, und der alte Herr faßte
seine Stuhllehne, daß der Sitz unter ihm fast aus den Fugen ging:

»Und daran kann ich mich halten mit meiner Väter Erbe und unseres
Herrgotts verunreinigter freier Natur? Und darauf darf ich mich
stellen mit meinem Elend? Ich zahle Ihnen alle Ihre Schulden für
das Wort, Adam!\ldots{} Wie nannten Sie es doch?«

»Beggiatoa alba. Von einem von uns ganz speziell für Sie erst
neulich zu Ihrer Beruhigung in den Ausflüssen der Zuckerfabriken
entdeckt, alter Freund. Was wollen Sie? Pilze wollen auch leben,
und das Lebende hat Recht oder nimmt es sich. Dieses Geschöpfe ist
nun mal mit seiner Existenz auf organische Substanzen in möglichst
faulenden Flüssigkeiten angewiesen, und was hat es sich um Pfisters
Mühle und Kruggerechtsame zu kümmern? Ihm ist recht wohl in Ihrem
Mühlgerinne und Rädern, Meister, auch das gebe ich Ihnen
schriftlich, wenn Sie es wünschen; und Kollege Kühn, der zuerst auf
das nichtsnutzige Gebilde aufmerksam wurde und machte, setzt Ihnen
gern seinen Namen mit unter das Attest.«

»Und die Krickeroder Fabrik halten Sie also wirklich und wahrhaftig
einzig für das infame Lamm, so mir mein Wasser trübt? I, da soll
doch~–«

»Ja, was da soll, das ist freilich die Frage, welche wir Gelehrten
unseres Faches nicht berufen sein können zu lösen. Übrigens habe
ich bis jetzt nur das Behängsel Ihres Rades untersucht und einige
Tropfen den Garten entlang aus dem Röhricht dazu entnommen.
Selbstverständlich werden wir den Unrat den Bach aufwärts bis zu
seiner Quelle verfolgen. Aber, Vater Pfister, was ich Ihrem Jungen
da gesagt habe, wiederhole ich Ihnen jetzt: es interessiert mich
ungemein, dieser Sache einmal so gründlich als möglich auf den Leib
zu rücken; aber – ich bin grenzenlos Partei in dieser
Angelegenheit, und der Dienst, den ich Ihnen im besondern und der
Welt im allgemeinen vielleicht tue, kann mir nur das höchst
Beiläufige sein. Ihren Ärger, Ihre Schmerzen und sonstigen lieben
Gefühle in allen Ehren, Vater Pfister!«

»Jeder Mensch ist Partei in der Welt«, seufzte mein alter, lieber
Vater, »nur ist es schlimm, wenn der Mensch das auf seine alten
Tage ein bißchen zu sehr einsieht und sich zu alt fühlt, um noch
mal von neuem mit mehr Aufmerksamkeit in die Schule zu gehen. Was
Sie aus meinem ruinierten Mühlwasser noch zu lernen haben, weiß ich
nicht, Adam Asche – für den vorliegenden Fall möchte ich, ich hätte
meinen Jungen da weniger auf das Griechische und Lateinische
dressieren lassen und mehr auf ihr Vergrößerungsglas. Da könnten
Sie mir denn auch nur ein angenehmer Gast sein, ohne daß ich Sie
weiter um ihre Wissenschaft zu bemühen brauchte.«

In dieser oder einer ähnlichen Weise gerieten sie bei jedem längern
Zusammensein aneinander, aber es war nicht nötig, daß der
nächstbeste gute Freund oder in diesem Falle der Sohn des Hauses
beruhigend zwischen sie trat. Sie kamen gottlob stets bald wieder
zu einem Verständnis, und zwar dem innigsten.

»Es ist heute der erste Weihnachtstag, Vater Pfister, und aus Abend
und aus Morgen wird sicherlich der zweite, also meine ich, wir
lassen's für heute bei den gewonnenen schändlichen Resultaten
bewenden und gehen morgen der Scheußlichkeit bis zu ihrer Quelle
nach«, sagte Doktor Asche, erhob sich seufzend von seinem
Mikroskop, trat zu der halb geplünderten Tanne im Winkel und griff
nach einem vergessenen Zuckerherzen an einem der höchsten Zweige.
Sonderbarerweise aber schob er es nach einiger melancholischen
Betrachtung nicht in den Mund, sondern in die Tasche. »Es ist
Weihnachten, alter, lieber Vater Pfister, und ich wollte, Sie
wüßten es ganz genau, wie leid mir Ihr betrübtes Gesicht tut. Wer
kann denn was dagegen, daß es so viel Bitterkeit und – schmutzige
Wäsche auf dieser Lumpenerde gibt? Und ich habe Ihnen noch so
manche famose Geschichte aus der Stadt und der Welt mitgebracht.
Sie rauchen mir auch schon viel zu lange kalt. Stopf dem Papa eine
frische Pfeife, Ebert. Wir haben wahrhaftig genug für heute.«~–

Auch mir schien's genug zu sein an dem Abend nach dem Heumachen am
heißen Sommermorgen auf den Wiesen gegenüber von Pfisters Mühle.
Tauschwer hatten sich alle Blumen, die wir auf ihren Stengeln
gelassen hatten, mit denselben geneigt. Es war entzückend kühl
unter meinen alten väterlichen Bäumen; aber der Tau fiel auch auf
meine eigenste Herzensblume, und wer sagte mir, ob er für die nicht
ungesund sei? Sie hatte mit allen ihren Schwestern – die
Nachtviolen ausgenommen – die Äuglein geschlossen und in unserer
Laube am murmelnden Bach das Haupt an meine Schulter gesenkt, und
es hob und senkte sich auch an meiner Brust wie leise, ungestörte
Wellen und dazu murmelte es:

»Erzähle nur ruhig immer weiter, ich höre genau zu, ich höre alles;
aber bitte, wenn es möglich ist, werde nur ein klein bißchen nicht
noch zu gelehrter, Herz! Es ist recht schlecht von mir, aber in der
Geographie- und der Naturgeschichtsstunde habe ich immer am
wenigsten aufgepaßt, und vielleicht waren die Tiere in Latein, von
denen du gesprochen hast, zu meiner Zeit wohl gar noch nicht
erfunden. Frau Albertine weiß viel mehr in der Hinsicht, und ich
nehme dir es gewiß nicht übel. Ich habe ja aber auch zu Hause bei
Papa eigentlich nur mit ihm auf seinem Kirchhofe botanisieren
können, und da – da – du weißt ja selber, wie auch du mir
dazwischen gekommen bist!«

\section{Vierzehntes Blatt}

\zusatz{Krickerode}
Ich trug mein sommertagsmüdes, schlaftrunkenes Weiblein mehr, als
daß ich's führte, in unser Sommernest, das noch vor Sommersende wie
ein ander Schwalben- oder sonstiges Wandervögelnest mit einer
dummen, langen Stange unterm Dachrande weg für alle Zeit
herabgestoßen werden sollte. Und nun ist es mir heute auf dem
langweiligen Papier, als trage ich sie in den Herzpunkt, die volle
Mitte meiner acta registrata, der Regesten von Pfisters Mühle.

Es wurde aus Abend und Morgen der zweite Weihnachtstag, und Felix
Lippoldes, der sich und uns versprochen hatte, dem Greuel mit auf
den Grund zu kommen, das heißt uns auf unserer unheimlichen
Entdeckungsfahrt stromauf von Vater Pfisters Mühlwasser zu
begleiten, ging wirklich mit.

Er kam unter dem dritten Glockengeläut durch einen dichten Nebel
nach der Mühle und wartete, an meines Vaters Schenktische auf einem
Faß sitzend, blödselig in Geduld oder Stumpfsinn darauf, daß der
Nebel sich lege, und wir, Doktor Adam Asche und ich, bereit seien.

Das letztere war bald der Fall, auf das erstere hätten wir den
ganzen Tag vergeblich warten können. Der graue Dunst stieg weder,
noch fiel er. Er blieb liegen, wie er lag, und es war ihm kein Ende
abzusehen; ich aber habe selten ein verdrossener, grimmiger Gesicht
erblickt als das meines Freundes Adam bei seiner ersten Begrüßung,
sowohl mit dem armen Poeten drinnen wie mit der grauen,
feuchtfrostigen Welt draußen.

»Das sage ich Ihnen, Dichter, Denker und Doktor«, brummte er, »auf
den Tisch steigen wir heute morgen nicht. Und du, Junge, bilde dir
ja nicht ein, daß ich nach Pfisters Mühle herausgekommen sei, um
mir Weltuntergangsgefühle aus deines Vaters verstänkerter
Kneipidylle herauszudestillieren. Idylle hin, Idylle her; trotz
Weihnachten, Ostern und Pfingsten in \emph{einer} Wehmutsträne habe
ich jetzt die Absicht, ruhig unter den Philistern auf gegebenem,
bitter realem Erdboden so gemütlich als möglich mit zu schmatzen,
zu schlucken, zu prosperieren und möglicherweise auch zu
propagieren. Zum Henker, am liebsten wär mir's jetzt, ihr zwei
Phantasienarren säßet mit Vater Pfister im Gotteshause, lobtet den
Herrn und alle seine Werke und hättet mir allein diese gegenwärtige
Auseinandersetzung mit den Lebens- uns Kulturbedingungen des
Moments überlassen. Da ihr aber einmal da seid, also vorwärts –
hinein in den Schmaratz! Nehmen Sie die Rumflasche und das Glas da
fort, Samse, und geben Sie mir Ihren Arm, Don Feliciano. Das
Mikroskop brauchen wir heute nicht, Ebert; aber da, Samse, den
Flaschenkorb können Sie schleppen – Sie, Lippoldes, brauchen aber
nicht aufzuhorchen, die Pullen sind leer, und der Stoff, mit dem
ich sie jetzt zu füllen gedenke, stammt nicht aus dem Brunnen
Melusinens, auch nicht aus dem fons Bandusius und am wenigsten aus
Ihrer Hippokrene.«

Wir verließen den Mühlgarten nunmehr durch ein mir seit meinen
frühesten Lebensjahren wohlbekanntes Loch in der Hecke und
wanderten am Uferröhricht über feuchtes Wiesen- und holperichtes
Ackerland den Fluß aufwärts. Drei oder vier Anbauerhäuser des
Dorfes lagen noch etwas weiter hinauf und reichten mit ihren Gärten
bis an den Bach.

Das eigentliche Dorf liegt, wie jeder weiß, der Pfisters Mühle
kennt oder kannte, einige Büchsenschüsse weit unterhalb derselben.
Hoffentlich wird es noch ungezählte Jahre länger als meines Vaters
liebes Haus an seiner Stelle zu finden sein.

»Ist denn das Ihr Fräulein Tochter, Doktor Lippoldes?« fragte
plötzlich Asche, eine Flasche blaugrauen, schleimigen Flußwassers,
die ihm Samse eben zwischen dem dürren, mit »chlorophyllfreien
Organismen« behängten Uferschilf gefüllt hatte, in unsern
Flaschenkorb versenkend.

Eine weibliche Gestalt war's, die im graublauen Nebel in dem vor
der letzten Häuslingswohnung sich herziehenden übelzerzausten
winterlichen Kohlgarten unter einem Baume stand.

»Singt Weide, grüne Weide!« schrillte der Poet. »Seid Ihr es,
Fräulein, mit Fenchel, Raute und Aglei – mit Hahnfuß, Nesseln,
Maßlieb, Kuckucksblumen – mitten im dänischen Winter? Bist du es,
mein Kind Albertine?«

Die schlanke Gestalt im kümmerlichen Kleidchen, dicht gehüllt in
ein graues Tuch, näherte sich durch den melancholischen Dunst,
neigte sich vornehm unseren Grüßen, und Albertine Lippoldes sagte
lächelnd:

»Aber, Papa, dein Husten! Nach allen vier Weltgegenden habe ich dir
wieder meine Sorge um deinen Katarrh nachtragen müssen! Es ist sehr
unrecht von dir.«

»Jaja«, greinte der Dichter, »ich wollte euch auch ein paar
Veilchen geben, aber sie welkten alle, da mein Vater starb. Sie
sagen, er nahm ein gutes Ende. Na, natürlich! Was sollte er sonst
noch nehmen können? Und – da – sieh dir nur die Herren genau darauf
an, Kind: sie scheinen auch das nutzbare Ergebnis meines
Menschendaseins in dieser vergnüglichen Welt in mehr als gelinden
Zweifel zu ziehen.«

»Hören Sie jetzt auf, mit diesem Unsinn wenigstens, Doktor
Lippoldes!« schnarrte Doktor Asche. »Fräulein hat vollkommen recht,
und in der warmen Stube sind Sie am besten aufgehoben. Ihre
Veranlagung zur Unsterblichkeit und zum Schnupfen ist mir seit
lange zur Genüge bekannt. Bleiben Sie mir mit Ihrem Esel von Hamlet
dem Dämel gefälligst vom Leibe, und in Ihrem eigenen Interesse auch
von Vater Pfisters Mühlwasser weg. Was, Maßlieb und Veilchen bei
\emph{der} Jahreszeit? Dänische Tropfen werde ich Ihnen morgen
anzuraten haben, und deutsche Kamille wird alles von Florens
Kindern sein, was Fräulein~O – Fräulein Albertine Ihnen zu bieten
hat, wenn Sie wieder einmal nicht auf den guten Rat Ihrer besten
Freunde hören und nicht auf der Stelle nach Hause gehen.«

Die junge Dame griff mit einem fast bösen Blick auf meinen armen
Freund Asche, aber doch zugleich angstvoll nach der Hand ihres
Vaters:

»O bitte, komm mit mir! Der Herr sagt es ja auch, daß es dir besser
sein wird.«

»Nachher – mit den jungen Leuten, Kind! Sie sind selbstverständlich
zum Frühstück bei uns eingeladen.«

»Oh!« rief Fräulein Albertine leise, nun nicht zornig und
ängstlich, sondern im wirklichen Schrecken. »Aber Vater – die
Herren – du weißt~–«

»Wenn die Zeit langt, Lippoldes«, brummte Adam Asche gröblicher
noch denn zuvor. »Jedenfalls drängt sie, wenn Vater Pfister bei
seiner Rückkehr aus der Kirche \emph{seine} Gastfreundschaft gegen
mich nicht zu allen seinen übrigen Plagen rechnen soll. Doktor
Lippoldes – lieber ein andermal! Mein Fräulein – ich habe die
Ehre!«

Er hob den zerdrückten, langgedienten Filz ein wenig von dem
seltsamen, zerzausten Haarwulst und ließ ihn wieder darauf
zurückfallen. Sodann beförderte er den ahnungslos gaffenden Samse
mit seinem Flaschenkorbe vermittelst eines Winkes, der fast einem
Rippenstoß glich, auf unserm Pfade stromaufwärts weiter und sich
ihm nach, die handschuhlosen Fäuste tief in den Taschen seines
Überrocks. Doktor Lippoldes aber nahm meinen Arm und sagte:

»Dieser Mensch ist ohne Zweifel ein Grobian! Nun, aber der erste
nicht, der mir im Leben begegnete. Ich mag ihn schon seit langen
Jahren ganz gern, junger Pfister; unter den Flegeln mit Gemüt ist
er mir einer der liebsten, und so mag auch er unter meiner bessern
Bekanntschaft weiter mitlaufen. Kommen Sie, junger Mann, daß wir
ihn nicht aus dem Gesicht verlieren. Er hat selbstverständlich
keine Ahnung, wie sehr ich eben res mea agitur sagen kann an Ihres
Vaters vergiftetem Lebensquell. Mädchen, die Herren haben deine
Einladung angenommen. Leihe mir deinen Arm, Knabe Lenker.«

Er hatte es wirklich nötig, daß er nicht nur geführt, sondern auch
gelenkt wurde. Über die Schulter zurückblickend, sah ich noch, wie
Fräulein Albertine die Hand an die Augen hob, ihr Tuch dichter um
sich zusammenzog und dann zögernd der armseligen Behausung
zuschritt.

Als wir die Vorangegangenen wieder erreicht hatten, meinte Adam:

»Sie hätten was Besseres tun können, als Ihrer armen Tochter diesen
Schrecken einzujagen, Lippoldes.«

»He he he«, kicherte der unzurechnungsfähige Gastfreund der
Olympier. »Es soll mich in der Tat wundern, wie sie es anfangen
wird, sich nicht zu blamieren. Merken Sie sich's, Eberhard Pfister,
und halten Sie sich an ein solides Kopf- und Handwerk. Kinder von
meinesgleichen, und wenn es die besten, lieben Mädchen wären, sind
leider nicht cour- und tafelfähig da oben – über den Wolken und
Krähenschwärmen. Beim Zeus und allen seinen Redensarten nach der
Teilung seiner Erde, mein Kind und gutes Mädchen hat wenigstens
auch seine Freude an reinem Wasser auf dieser Erde, und ich halte
es nicht weniger als mich und Ihren Papa, Vater Pfister,
berechtigt, durch die chemischen Kenntnisse des Menschen da vor uns
zu erfahren, wer uns dieses hier verpestet. Da kommt wieder ein
halb Dutzend toter Fische herunter, Asche.«

Der Wasserbeschauer zuckte nur verdrossener denn zuvor die Achseln,
antwortete dem Poeten aber nicht. Doktor A.~A.~Asche hielt sich
jetzt einfach an seiner Aufgabe und teilte nur mir dann und wann
ein Minimum seiner Beobachtungen mit.

Mir aber kam es nicht zu, meinem Weibe in der Sommerfrische das
Verständnis zu öffnen für saures Kalzium und saures
Magnesiumkarbonat, für Kalziumsulfat und Chlorkalzium, für
Chlorkali, Kieselsäure und Chlormagnesium.

»Ich bitte dich, bester Mann, hör auf«, sagte sie, meine Emmy, nach
dem ersten Versuch meinerseits. »Großer Gott, und das mußtet ihr
alles riechen? Ja, da riecht es zu Weihnachten ja selbst bei uns in
Berlin besser! Verliere nur weiter kein Wort mehr; ich kann mir
wirklich Frau Albertine und deinen armen seligen Papa ganz genau
vorstellen, auch ohne Doktor Asches gräßliche gelehrte
Apothekerredensarten.«

Ich tat, offen gestanden, mir nicht weniger als ihr einen Gefallen
damit, aufzuhören und uns den Sommertag nicht auch noch gar durch
unverständliche termini technici einer uns doch nur vom Hörensagen
bekannten unheimlichen Wissenschaft zu verderben.

Kurz, wir sahen meines Vaters Mühlwasser je höher hinauf, desto
unsaubrer werden, wir sahen noch mehr als einen auf der Seite
liegenden Fisch an uns vorbeitreiben, und wir füllten, die Nasen
zuhaltend, Samses Flaschenkorb und versahen jede einzelne Flasche
mit einer genauen Bezeichnung der Stelle, wo wir die geschändete
Najade um eine Probe angegangen waren.

Zweiundeinhalb Kilometer von Doktor Lippoldes Behausung gelangten
wir dann nach der Welt Lauf und Entwicklung wie zu etwas ganz
Selbstverständlichem zu dem Ursprung des Verderbens von Pfisters
Mühle, zu der Quelle von Vater Pfisters Leiden; und Doktor Adam
Asche sprach zum ersten Male an jenem Morgen freundlich ein Wort.
Auf die Mündung eines winzigen Nebenbaches und über eine von einer
entsetzlichen, widerwärtig gefärbten, klebrig stagnierenden
Flüssigkeit überschwemmte Wiesenfläche mit der Hand deutend, sagte
er mit unbeschreiblichem, gewissermaßen herzlichem Genügen: »Ici!«

Jenseits der Wiese erhob sich hoch aufgetürmt, zinnengekrönt,
gigantisch beschornsteint – Krickerode! Da erhob sie sich,
Krickerode, die große, industrielle Errungenschaft der Neuzeit, im
wehenden Nebel, grau in grau, schwarze Rauchwolken, weiße Dämpfe
auskeuchend, in voller »Kampagne« auch an einem zweiten
Weihnachtstage, Krickerode!

»Der reine Zucker!« rief Asche. »Da schwatzen die Narren immerfort
über die Bitterkeit der Welt. Da können sie sie niemals süß genug
kriegen, und da – stehen wir, das Leid der Erde wiederkäuend, vor
dem neuen Tor. Sie sind nicht Aktionär, Lippoldes – Vater Pfister
auch nicht, und von dir jungem Bengel ist es ebenfalls noch nicht
anzunehmen~–«

»Du bist es aber auch nicht, Adam«, meinte ich, das ungeheuchelte
Pathos des großen Chemikers unterbrechend; aber der – A.~A.~Asche –
sprach ruhig: »Ich wollte, ich wäre es schon.«

Der arme Tragöde hing sich stumpfsinnig lächelnd mir fester an den
Arm, und so umschritten wir den wohl zwanzig Morgen bedeckenden
künstlichen Sumpf und gelangten unter der Mauer der großen Fabrik
zu dem dunklen Strahl heißer, schmutzig-gelber Flüssigkeit, der
erst den Bach zum Dampfen brachte und dann sich mit demselben über
die weite Fläche verbreitete, die meine nächsten Vorfahren nur als
Wiese gekannt hatten.

»So ist es nicht unerklärlich, daß beim Wiedereintritt des
Wässerleins in deines Vaters Mühlwasser, mein Sohn Ebert, das
nützliche Element trotz allem, was es auf seinem
Überflutungsgebiete ablagerte, stark gefärbt, im hohen Grade
übelriechend bleibt. Das, was ihr in Pfisters Mühle dann, laienhaft
erbost, als eine Sünde und Schande, eine Satansbrühe, eine ganz
infame Suppe aus des Teufels oder seiner Großmutter Küche
bezeichnet, nenne ich ruhig und wissenschaftlich das Produkt der
reduzierenden Wirkung der organischen Stoffe auf das gegebene
Quantum schwefelsauren Salzes«, sagte Adam Asche. »Und nun, denke
ich, können wir wieder nach Hause gehen«, fügte er hinzu, indem er
die letzte Flasche aus Samses Flaschenkorb gefüllt mit warmem,
leise dampfendem Naß aus der Abflußrinne von Krickerode mit fast
zärtlicher Kennerhaftigkeit gegen den grauen Feiertagshimmel und
vor das linke, nicht zugekniffene Auge prüfend erhob.

»Es ist freilich recht frostig und auch nicht der Humor in dem
Dinge, den ich mir davon versprochen hatte«, murmelte Doktor Felix,
in seinem abgetragenen Winteroberrock die Schultern
zusammenziehend. »Ich habe Sie vor nicht allzu langer Zeit auch
noch als einen andern gekannt, Adam, und ich werde mich auch Ihnen
nicht mehr einer derartigen Expedition in den allzu gesunden
Menschenverstand als Begleiter und Chorus anhängen. Ich hatte mich
auch in dieser Angelegenheit auf Sie gefreut, Asche; aber mein
Gedächtnis ist leider schwach geworden, und ich habe mich alle Tage
von neuem darauf zu besinnen, wie alt ihr junges Volk und wie
vernünftig und langweilig ihr seid.«

Nun krallte er sich mit der Linken in meinen Kragen und streckte
den dürren rechten Arm und die Faust aus dem schäbigen Ärmel weit
vor gegen das phantastischer als irgendeine Ritterburg der
Vergangenheit mit seinen Dächern und Zinnen, seinen Türmen und
Schornsteinen im Nebel des Weihnachtstages aufragende große
Industriewerk und rief hell und heiser:

»Sieh es dir an, Knabe, und finde auch du dich mit ihm ab, wie der
da – wissenschaftlich oder als Aktionär. Kind, habe dreist wie die
andern Furcht, dich ihm gegenüber lächerlich zu machen, und renne
dir ja den Schädel nicht dran ein mit irgend etwas drin, was über
der Zeit und dem Raume liegt. Folge du unserm Rate, so wirst du
etwas vor dich bringen; nur sieh dich nicht um nach dem, was du
vielleicht dabei hinter dir liegen lässest. Ich aber werde jetzt
eurem Rate folgen, nach Hause gehen und unterkriechen und mich mit
nützlicher Festtagsnachmittagslektüre beschäftigen. Meine eigene
Bibliothek ist mir, wie du weißt, Asche, mit mehrerem andern im
Laufe des Lebens abhanden gekommen, ich bin bei meinem jetzigen
Landaufenthalt einzig auf die meines Bauern angewiesen, auf den
Kalender vom laufenden Jahr und auf ein altes Buch im Fach über der
Tür, das mir mein Mädchen herunterholen mag. Uralte jüdische
Weisheit und Prophezeiung, auf die ihrerzeit auch niemand geachtet
hat! Rate dir ebenfalls zu der Lektüre, wenn dir einmal alle andere
abgestanden, stinkend und voll fauler Fische vorkommen wird, wie
deines Vaters Mühlwasser, Ebert Pfister! Zephanja im ersten Kapitel
Vers elf: ›Heulet, die ihr in der Mühlen wohnet, denn das ganze
Krämervolk ist dahin, und alle, die Geld sammeln, sind
ausgerottet!‹«

»Hoffentlich fürs erste noch nicht«, brummte mein Freund Adam, wie
es schien, gänzlich unberührt von dem unmächtigen Pathos unseres
beklagenswerten Begleiters. »Was aber das Heulen in den Mühlen
anbetrifft, na, so stehen wir ja grade deswegen hier mit blauen
Nasen im Erd- und Ätherqualm. Ich kann deinem Vater leider nicht zu
seinem alten, fröhlichen Dasein verhelfen, Ebert; Sie aber,
Lippoldes, dürfen sich schon ganz ruhig mit Ihren Idealen zum Vater
Pfister auf die harte Bank in der harten Schule des Lebens setzen.
Was beiläufig mich angeht, Ebert Pfister, so meine ich, der beste
Mann wird immer derjenige sein, welcher sich auch mit dem
schofelsten Material dem gegenüber, was über der Zeit und dem Raume
liegt, zurechtzufinden weiß. Zu Ihrem ›Alarich in Athen‹ und
›Schneider in Straßburg‹ konnten Sie meinen Senf nicht gebrauchen,
Doktor; der Vorschlag, in Kompanie mit mir aus Pfisters Mühle ein
Gedicht zu machen, würde Ihnen heute nur lächerlich vorkommen; Sie
sind mein Mann, Samse, nehmen Sie mir den Korb da in acht, und
marsch nach Hause. Die unsterblichen Götter aber mögen mir meinen
Willen lassen, ich – lasse ihnen ja auch den ihrigen.«

Er stiefelte dem getreuen Knecht Samse voran, flußabwärts, und ich
suchte mit dem verschollenen Poeten nachzufolgen. Das Wort, daß es
besser gewesen wäre, wenn der letztere zu Hause und im Warmen sich
gehalten hätte, bewahrheitete sich in bedenklicher Weise immer
mehr.

Ach, er paßte ganz, nur zu sehr in den Tag die Witterung, die
Beleuchtung, und deshalb um so dringlicher an den warmen Ofen und
unter die lieben, hellen, sorglichen Augen seiner Tochter! Immer
tiefer schien ihm der Frost in die vorzeitig mürben Knochen zu
dringen, und mit zitterndem Finger wies er auf den jüngern,
gesundern Mann im Nebel vor uns, und mit vor Erregung bebender
Stimme rief er:

»Und ich habe ihn einmal mit zu denen gezählt, für die ich in
meinen guten Stunden zu leben glaubte! Ich habe ihn, als er in
deinem Alter war, mit glänzenden Augen vor meiner Tür gehabt und
mit Tränen in den Augen regungslos auf seinem Stuhl an meinem
Tische! Nun bin ich ihm der kindische Narr, der blöde Wirrkopf, der
schwache Phantast, und er schnauzt mich an und glaubt, verständig
zu mir zu reden und mich zur Vernunft zu bringen, und er überhebt
sich mehr, als ich mich je in meinen besten Tagen überhoben habe.
Wie es ihn heute kitzelt, wenn er sich für sein junges, dummes
Pathos rächt und den alten Lippoldes unter seine Kuratel nimmt und
ihn seinerseits zum Schluchzen bringt! Rufe ich ihn jetzt um und er
hält es der Mühe wert, sich umzusehen, so wird er von
pathologischen Vorgängen reden und ganz genau wissen, was mir auf
Nerven oder Tränendrüsen wirkt, und er hat recht; recht hat er, der
junge Mann! Zehn Jahre jünger – zwanzig Jahre jünger, und mit den
jüngsten Erfahrungen des Lebens von vorn beginnen! O Eberhard
Pfister, wenn nur nicht diese schöne Festtagslandschaft, die Welt
um uns her, allerlei Staffage zur künstlerischen Vollendung nötig
hätte! Und wenn es nur nicht so entsetzlich gleichgültig wäre, von
welchem Hintergrunde wir uns abheben und wie wohl oder übel wir uns
persönlich auf dem Bilde fühlen!«\ldots{}

Dies war nun ganz wie Emmys tiefsinniges Wort: »Wo bleiben alle die
Bilder?« – Der arme, gequälte, verloren gegangene Mann, der Poet,
und mein liebes, unpoetisches, gutes kleines Mädchen standen vor
derselben Frage, und – ich mit A.~A.~Asche und den übrigen
ebenfalls, was wir uns auch sonst einbilden mochten.~–

Sie hatte sich seit Stunden nicht gerührt in unserm Sommernest
unter dem Dachrande von Pfisters Mühle – Emmy. Sie hatte auch im
glücklichsten, unschuldigsten, gesunden Vormitternachtsschlaf
gelegen, aber wer sagt es, wieviel von den Bildern, die mir
nächtlicherweise am Tisch im Stübchen neben der Kammer über das
Papier gegangen waren, ihr im Traum zu eben solchen Wirklichkeiten
wurden, wie die wirklichsten Ergebnisse des wachen, lebendigen
Tages?

Ein Faktum ist, daß sie (immer meine Frau), als bald die Hähne im
Dorfe krähen wollten und der erste kühle Hauch aus Morgen den
Vorhang neben mir bewegte, sich auf ihrem Bett regte und sich auf
die Hand stützte und murmelte:

»Ich wollte wirklich, du brächtest ihn jetzt bald endlich wieder an
den warmen Ofen, Herz!\ldots{} Die arme Albertine!\ldots{} Aber so seid ihr
Männer, einerlei, ob ihr unsere Väter oder ob ihr unsere Männer
seid. Papa machte es gradeso improvisiert, wenn er mir am liebsten
meinen höchsten Abscheu, seinen sogenannten jungen Freund
Buckendahl, zum Frühstück mitbrachte. Wir hätten uns gegenseitig
auffressen können, und er, Assessor Buckendahl, mich aus wirklich
ernst gemeinter Zuneigung. Wie zog sich denn aber Albertine aus der
entsetzlichen Verlegenheit, und was hatte sie euch vorzusetzen in
ihren damaligen Umständen?«

Ich ging auf den Zehen hin und sah das Kind wieder im tiefsten,
lächelndsten Schlummer liegen, und ich ging trotz dem ersten Streif
grauen Morgenlichtes im Osten noch einmal zu meinem Schreibgeräte
zurück. Ja, so sind wir Männer dann und wann, selbst bei den
behaglichsten Verlockungen, wenn uns etwas auf den Nägeln und der
Seele brennt: ich \emph{mußte} in dieser Nacht noch mit der
Geschichte von unserm Weihnachtsgange nach Krickerode zu Ende
kommen, gleichviel, ob ich Emmy mündlich oder mir schriftlich davon
erzählte!~–

Ach, wäre es an jenem Wintertage nur so leicht gewesen, den Doktor
Lippoldes zum warmen Ofen zurückzubringen, wie Emmy es sich in
ihrem Sommernachtstraum vorzustellen schien! Zu meinem Schrecken
merkte ich, daß ich allein den Mann nicht weiterzuführen vermochte.
Er schnatterte jetzt vor Frost und sprach immer seltsamere Dinge.
Es blieb mir nichts übrig, als Asche um Beistand anzurufen.

Der blieb denn auch stehen, zuckte die Achseln, sah sich den Poeten
von neuem an und murmelte:

»Kann man es den Leuten verdenken, wenn sie sich was drauf zugute
tun, daß sie stets ganz genau wissen, was unsereinem gegen Schluß
der Komödie zu passieren pflegt?«

Er legte mit einer wahrhaft nichtswürdigen Fratze den
grimmig-possierlichen Akzent auf die Worte »Leute« und
»unsereinem«, und meinte dann mit vollkommen gleichgültiger Miene:

»Wir haben ihn natürlich so rasch als möglich – lebendig oder tot –
nach Hause zu schaffen; ich kann dem armen Mädchen nicht darüber
weghelfen. Nur betrunken ist er diesmal nicht. Stellen Sie den
verdammten Kober weg, Samse. Es wird ihn heute am heiligen Feste
hoffentlich niemand uns stehlen. Laufen Sie vorauf zu Fräulein
Lippoldes und bestellen Sie ein Kompliment – zum Henker, nein,
warten Sie; hier bin ich doch zu wenig nütze, Ebert; – greifen Sie
dem Elend unter die Arme, Samse; \emph{ich} werde vorausgehen, das
Bett zu wärmen und das Fräulein vorzubereiten.«

»Ein Wort noch, Herr Doktor!« sprach Samse. »Was meinen Sie
hierzu?« fragte er, aus der Tasche seiner Zottenjacke eine flache
Flasche mit einer Flüssigkeit vorlangend, die nicht meines Vaters
Mühlwasser entnommen war. »Ich habe wohl gehört, Herr Doktor~–«

»Recht haben Sie gehört! Alter Praktikus, weshalb haben Sie davon
nicht gleich gesagt? Alle Wetter, selbstverständlich! Lassen Sie
riechen – jawohl, Vater Pfisters echtester Nordhäuser. Wir brauchen
ihm ja das nur zu zeigen, um ihn gegen jede See von Plagen
wenigstens für den Moment mit Wehr und Waffen auf die Beine zu
bringen.«

Es verhielt sich leider Gottes wirklich so. Der kranke Mensch in
dem unseligen, genialen Menschenkinde griff mit einem fast
tierischen Laut nach Samses »Buddel«, zog den Inhalt des letztern
gierig in sich hinein und – fühlte sich wieder als Mensch, wie er
sich selber ausdrückte.

»Ich gebe dir mein Wort darauf, Eberhard Pfister«, murrte Adam
Asche mir ins Ohr, »der Mann geht auch nicht an Krickerode
zugrunde. Ich will es keine Lüge nennen, wenn er derartiges
behauptet, aber er irrt sich unbedingt. Ich wollte, ich könnte
dieses auch von deinem Vater sagen. Nun, kommt jetzt ruhig mit dem
Unglück nach; ich werde doch etwas rascher voraufgehen und dem
armen Mädchen ein Wort zur Beruhigung sagen.«

Er verschwand im Nebel flußabwärts, und Samse flüsterte schlau, mit
dem Finger an der Nase:

»Ebert, ich bin doch nicht umsonst, seit ich vernünftig denken
kann, Knappe, Sommergarten- und Winterpläsier-Garçon, und was sonst
so zu unserm Meister und Anwesen gehört, gewesen! Herr Doktor, na,
es ist Ihnen jetzt wohl'n bißchen besser zumute? Also denn, wenn's
beliebt, die paar Schritte noch aushalten!\ldots{} Ich denke, den Korb
mit dem Giftwasser nehmen wir doch lieber mit, Ebert; – der Satan
trau dem Fabriklervolk da hinter uns, selbst am hochheiligen
Festtage. Es treibt sich immer was von ihnen an unserm ruinierten
Nahrungsquell im Busch und Röhricht um, und wär's auch nur auf dem
Anstande nach unserm krepierten Fischstande. Dem Jammervolk muß ja
jedwede Viehseuche, wie Herr Doktor Asche vorhin sagte, reiner
Zucker sein. Sie wären imstande und söffen uns ihre eigne
Schandbrühe aus, bloß wegen Vater Pfisters alten Etiketten an den
Flaschen!«

Felix Lippoldes hatte weder von dem Gemurr des Chemikers noch von
Samses Zufriedenheit mit sich und seinen klugen Bedenken in betreff
anderer Notiz genommen; er zitierte aus seinen Dramen und hielt
meinen Arm jetzt nur deshalb fest, um eindringlicher auf mich
hineinzitieren zu können. In sonoren Jamben redete er von Sonnen,
Palmen, Zinnen, Türmen, Frauen, Helden und Heeren; und die Leute,
von denen vorhin Adam Asche redete, würden sicherlich gesagt haben:
»Wie gut er sich jetzt auf seinen Beinen hält!«, wenn sie bei uns
gewesen wären unter den Weiden am faulen Strom, auf dem Rückwege
von Krickerode nach Pfisters Mühle. Einige würden sich vielleicht
auch des Wortes »Stelzen« bedient und sich einiges auf den witzigen
Doppelsinn zugute getan haben. – Ich aber gedachte meiner Kindheit
und frühesten Jugend, und wie in jenen Tagen Felix Lippoldes über
meinem Gesichtskreise wie eine Sonne leuchtete, wenn ich von
Studiosus Asche und der Grammatik freigegeben und in meines Vaters
bunten, wimmelnden, fröhlichen Lebensgarten von neuem losgelassen
wurde.

Ja, er war in seinen glücklichen Tagen dann und wann auch ein Gast
Vater Pfisters und hatte merkwürdig ungestört und ununterbrochen
das große, phantastische Wort in Pfisters Mühle. Philister mit
Frauen und Töchtern, Bürger und Bürgerinnen mit ihren Kindern wie
ich damals, höhere und niedere Beamtete mit ihren Damen und
Kinderwagen, selbst die Vorstände und Vorsteherinnen der
respektabelsten Vereinigungen: für öffentliche Gesundheitspflege –
für Verschönerung der Umgegend der Stadt – für Verbesserung des
Loses entlassener Strafgefangener – gegen den Mißbrauch geistiger
Getränke – gegen die Überhandnahme des Vagabundentums – für, für,
für und gegen, gegen, gegen – ließen ihn reden, hörten ihm, wenn
auch erstaunt, so doch nicht ungern zu und waren so ratlos und
ungewiß in ihren Gefühlen und ihrer Stimmung gegen ihn wie ich nun
als erwachsener junger Mensch im Nebel und Rauhfrost des
Wintertages auf diesem Wege zum Anfang des Endes von Pfisters
Mühle.

Ja, sie hatten beide ihre guten Tage hinter sich, der Müller und
der Poet. Die Quellen und Ströme ihres Daseins waren ihnen beiden
abschmeckend, trübe und übelriechend geworden, und es war ihnen
wenig damit geholfen, daß wir wußten, womit das zusammenhing und
wie es durchaus nicht etwa geschah, weil die Welt aus ihrem Geleise
geraten wäre.

Das sind nun freilich Reflexionen, wie sie der Mensch beim
nachträglichen Aufzeichnen seiner Erlebnisse macht, wie sie ihm
aber nur selten in Begleitung der Erlebnisse selber kommen. Ich war
damals ganz einfach auf dem Rückwege zu meines Vaters verödetem
Haus und Garten dem armen Felix behülflich, seine Wohnung zu
erreichen, und es war mir sehr angenehm, daß mir Adam und Albertine
entgegenkamen, um mir die Verantwortlichkeit für das letztere von
der Schulter zu nehmen.

Mein Weib in seinem Kinderschlaf und lieblichen Tagleben hat
gottlob kaum eine Ahnung davon, wie gut sie es gehabt hat gegen
ihre nunmehrige beste Freundin Frau Albertine. Es war gerade nicht
angenehm, zur Erholung mit auf Papas sonderbares
Kirchhofs-Spaziervergnügen angewiesen zu sein; aber einem toten
Mann selber auf seinen unheimlichen Spaziergängen durch den kalten,
klappernden, rasselnden, klirrenden, mitleidlosen Werkeltag
Gesellschaft leisten zu müssen, war doch noch etwas schlimmer, und
Fräulein Albertine Lippoldes hatte nur dazu auf ihrem eigenen Wege
durch die Welt haltgemacht und war nur deshalb aus der Fremde nach
Hause zurückgekehrt.

»Da kommt Fräulein Tochter, Herr Doktor, und nun sehen Sie nur mal,
welche Angst sie wieder um Sie hat!« rief Samse. »Und Herr Doktor
Asche hinter ihr sollte sich wirklich die Mühe, sie zu beruhigen,
nicht machen. Es hilft ihm ja doch ganz und gar nichts. Nun sehen
Sie nur das liebe Gesicht! Ich bin gewiß für Pfisters Mühle in
ihrem Jammer, aber diese Angst- und Unglücksmiene der lieben Dame
geht doch noch drüber, Ebert.«

»Da bist du ja, Kind – und Sie auch, Freund Adam! Also – ein Glas
Madeira und eine Gabel Hummersalat, meine Herren. Du hast
vorgesorgt, Tochter deines Vaters – Hebe unter dem Strohdach? Meine
Herren, wenn es der feinste und höchste Egoismus ist, sich zu
sagen: Du machst ein Kunstwerk für hundertundfünfzig durch die Welt
verstreute Seelen, die für dich sind, so ist's ungemein angenehm,
sich nach einem Morgen wie der heutige zu vier zu Tische zu setzen.
Was schneiden Sie mir wieder für eine Fratze, Adam? Es wird uns
alles zugeteilt; ich habe mir mein Leben und Dasein sowenig selbst
gegeben, wie Sie sich das Ihrige. Kannst dich darauf verlassen,
Ebert; jeder bekommt das Kostüm und Werkzeug, das er nötig hat zu
seiner Rolle in der Welt. Niemand ist da ausgenommen. Niemand! Ich
auch nicht. Auch nicht die Kinder, die in limbo infantum schwimmen;
nicht die flüchtigste Erscheinung und nicht die dauerndste. Es gibt
nur aufgedrungene Pflichten, Genüsse und Versündigungen. Die
Richter sitzen zu Gericht, aber es hat noch nie ein Tribunal oder
einen Menschen gegeben, die über einen andern Menschen hätten
Urteil und Recht sprechen können. Ehrbar, ehrbar, wenn ich bitten
darf; – nicht zu dumm aussehen, Samse – nicht zu gescheit, ihr
andern! Aber was kommt es auf eure Gesichter an? Die kleine,
hülflose, offene Hand am schlafenden Kinde ist's, die die Welt von
Generation zu Generation sicher weitergibt. Also ein Glas old dry,
meine Herren. Da sind wir ja wohl wieder angelangt an den Grenzen
unseres Reiches und fordern Euch gnädigst auf, Adam Asche, unsere
Prinzessin Tochter über die Schwelle zu führen. Ei, es weiß kein
Mensch genauer als ein König und ein Poet, wie wenig der Erde
Pracht und Herrlichkeit bedeutet. He, he, da läge noch ein Buch,
Asche: De tribus imperatoribus – Von den drei großen Herren! Der
König – der Dichter und – der Vorstand der Irrenanstalt, und der
letzte als der größeste! Was sind alle Weltherrschaften gegen das
ungeheure Reich, das sich dem letztern in den Köpfen seiner
Untertanen in Wundern, Schönheiten und Schrecknissen ausbreitet und
das er zusammenhalten und regieren muß. An die Zigarren hast du
hoffentlich auch gedacht, Albertine?\ldots{}«

So ging das fort und fort unter dem frostigen, grauen Himmel, und
an dem trüben Fluß zwischen den Schlehenhecken und Büschen –
Gemeinplätze, seltsame Gedankenblitze, Erinnerungen an vergangene
üppige Tage und Genüsse. Für uns aber handelte es sich nur darum,
dem alten, schlafwandelnden Kind mit der wahrlich hülflosen,
offenen Hand in seinen gegenwärtigen Nöten so gut als möglich zu
helfen und seiner Tochter noch mehr. Wir konnten wirklich jetzt von
keiner seiner vielfachen Begabungen, das Leben »groß aufzufassen«,
Gebrauch machen. Es handelte sich nur darum, ihn in der ärmlichen
Bauernstube, die ihm und seinem Kinde zum letzten Unterschlupf
diente, im schlechten Tagelöhnerarmstuhl hinter dem gottlob warmen
Ofen niederzudrücken.

Wie seine Tochter das Leben auffaßte, davon konnte damals nicht die
Rede sein; doch am Nachmittag, es fing eben an zu schneien, führte
mich A.~A.~Asche noch einmal unter die Kastanienbäume von Pfisters
Mühlengarten, faßte mich an der Schulter, schüttelte mich und
sagte:

»Das ist ein prächtiges Mädchen, und es scheint mir die höchste
Zeit zu sein, ein wohlhabender Mann zu werden. Entschuldige mich
nachher bei deinen Leuten da drinnen; ich fahre heute abend noch
ab, denn ich halte es wirklich für die Pflicht der anständigeren
Menschen, die Ströme dieser Welt nicht bloß den andern zu
überlassen. Deinem Vater werde ich das ihn betreffende Ergebnis der
Erfahrnisse des gestrigen und heutigen Tages von Berlin aus
schicken. Überlege es dir, überlege es mit ihm, ob es ihm das
brave, gute Herz viel erleichtern wird, wenn er sich damit an einen
Advokaten wendet.«

\section{Fünfzehntes Blatt}

\zusatz{In versunkenen Kriegesschanzen}
Wie es trotz des Sommersonnenscheins hier schneit auf diese
Blätter! Wie der Nordwind kalt herbläst trotz der Julihitze! Ich
aber habe mir ja wohl vorgenommen, die Zähne zusammenzubeißen und
die Leute nichts merken zu lassen von meinem innerlichen
Frösteln?~–

Die Tage in der Mühle schienen immer schöner zu werden, je mehr sie
sich ihrem Ende näherten. Und sie näherten sich unwiderruflich,
unwiederbringlich ihrem Ende.

Von dem leeren Hause, dem toten Rade hatte ich bereits Abschied
genommen, aber rundum zu beiden Seiten des jetzt im Sommer wieder
so reinlichen Flüßchens lag noch mancherlei, was ich noch zum
letztenmal sehen und grüßen mußte – war noch vieles vorhanden, was
ich, wenn ich allein oder mit meiner Frau zu ihm ging, sicherlich
auch zum letzten Male sah; denn – was konnte mich je wieder nach
der Stelle locken, wo (nächsten Monat schon) Pfisters Mühle einmal
gestanden hatte?

Emmy begriff es dann und wann durchaus nicht, wenn ich sie hie und
dort mit hinzog, wo es – wo es ja eigentlich gar nichts zu sehen
gab und wohin auch der Weg eigentlich gar nicht hübsch, zumal bei
dem wolkenlosen Himmel, war.

Da gab es, zwanzig Minuten von der Mühle und eine halbe Stunde vom
Dorfe entlegen, eine nur mit vereinzelten Büschen bedeckte kuriose
Bodenerhöhung und Vertiefung, von wo aus man ganz gewiß noch
weniger als gar keine Aussicht hatte und wo ich ganz gewiß die
Verantwortung dafür auf mich nehmen mußte, wenn ich gar keine
Gründe hatte, an solchen heißen Nachmittagen mein erschöpftes Lieb
dort unter einem der Dornbüsche zum Sitzen einzuladen. Ich hatte
wohl meine Gründe in meiner Stimmung, aber sie waren dem Kinde in
der seinigen freilich ziemlich schwer begreiflich zu machen. Für
die letzten Tage auf meines Vaters und meiner Väter Habe entfaltete
grade dieser Ort seinen Zauber, und es gab keinen bessern, um
darauf von diesem verlorenen Erbe weiterzuplaudern.

Nämlich es gab eine Zeit, wo ganz andere feindliche Mächte als die
moderne Industrie sich auch nicht viel um das Wohl und Wehe von
Pfisters Mühle gekümmert hatten. Der Dreißigjährige Krieg hatte
grade hier in der Gegend dem Kundigen recht interessante Spuren
zurückgelassen. Alte Dämme und Verschanzungen diesseits und
jenseits des Flüßchens waren den Sachverständigen stellenweise noch
deutlich zu erkennen zwischen den Wiesen und Ackerfeldern, und die
viereckige Erdvertiefung, in der jetzt mein Weibchen zierlich in
der die roten Knospen öffnenden Heide unterm Hagedorn saß, war eine
solche Stelle, wo die schwedische oder kaiserliche Bellona den Fuß
fest hingestellt hatte. Die einen meinten, die Schweden hätten
diese »Kuhle« gegraben, diesen Wall aufgeworfen; die andern
behaupteten, kaiserliches Kriegsvolk sei's gewesen; Emmy war's ganz
einerlei und mir auch; denn recht behalten hatte heute doch nur der
Thymian, wie Emmy meinte. Es sei sehr gleichgültig, sagte sie, wer
hier gegraben und geschanzt habe, du er, der Quendel, noch lebendig
vorhanden und jener Wirrwarr nur den Gelehrten dunkel gegenwärtig
sei.

Wenn ich doch nur nicht selber zu sehr zu den Gelehrten zu rechnen
gewesen wäre!

Noch dazu in den letzten Tagen dieser sonderbaren, süßwehmütigen,
märchenhaften Sommerfrische mit meinem jungen Weibe – in den
letzten Tagen von Pfisters Mühle!

Denn hier, hinter den alten, versinkenden, grasbewachsenen
Böschungen und Stockaden Pikkolominis oder Torstensons, fern vom
Auge meines Vaters, dem fröhlichen Lärm seines Gartens und dem
Klappern seiner Mühle wie vom Turmuhrschlag unseres Dorfes, unter
den Weißdornbüschen, den Feldastern, Ginstersträuchen und
Steinnelken, bei den flatternden blauen Motten und den fetten
Raupen des Wolfsmilchschwärmers, hatte ich mit meinem Freund und
speziellsten Privatlehrer A.~A.~Asche, mit dem verlumpten Studenten
Adam Asche, mehr Geschichte, Philosophie der Geschichte und
Geschichte des Auskommens des Menschen mit seinesgleichen und
seinen Um- und Zuständen auf dieser Erde getrieben als sonst
irgendwo und mit irgendeinem andern.

Nun saß ich mit meiner Frau unter demselben Buschwerk, mit
denselben Lerchen über uns, denselben Kräutern und Blumen um uns,
und so~–

\begin{verse}
»\ldots{} gedacht ich nun der Ewigkeit,\\
Der längst entschwundnen, toten, wie der jetzigen\\
Lebendgen Zeit und ihres Lärms. In dieser\\
Unendlichkeit versank mein ganzes Denken,\\
Und süß war's mir, auf diesem Meer zu scheitern.«
\end{verse}

Ich hatte die ganze Kanzone, die Hände unteren Hinterkopf mit
halbgeschlossenen Augen vor mich hingesprochen: und~–

»Hast du das eben gemacht, Männchen?« fragte mein unliterarisches
Mädchen so freundlich und vergnüglich, daß ich mich rasch offenen
Auges auf den Ellenbogen stützte und rief:

»Du dummes Närrchen, habe ich das eben selber gemacht? Von einem
kleinen, buckligen Italiener ist's. Recanati hieß sein Dorf, in
dessen Umgebung wohl eine ähnliche Hecke gewesen sein muß wie diese
hier, hinter welcher er es, wie deine Volksgenossen sich
auszudrücken pflegen, unter der Feder hatte. Er war sogar ein Graf,
mein Herz, wenn auch mit zu wenig Taschengeld~–«

»Und er war sicher ein ebenso närrischer Patron wie du, wenn du
gottlob auch keinen Buckel hast und noch weniger ein Graf bist, und
mein Haushaltungsgeld mußt du mir unbedingt erhöhen, Ebert, wenn
wir wieder nach Berlin kommen und zu Hause sind. Ich habe eben
alles noch einmal ganz genau zusammengerechnet und komme wirklich
für den Herbst nicht weiter aus. Und höre mal, in den nächsten
Tagen müssen wir doch wohl anfangen, unsere Sachen so leiseken
zusammenzusuchen in deiner Mühle. Die Herren aus der Stadt, die
gestern wieder mit ihren Maßstäben und Notizbüchern dawaren, und
der Wagen mit Schubkarren und Schaufeln und Hacken, der heute
morgen kam und abgeladen wurde, deuten doch wohl darauf hin, daß
unsre Stunden hier gezählt sind.«

Und statt Giacomo Leopardi zu deklamieren in unserer alten Schanze
aus der Schwedenzeit, sang mit heller Stimme mein fröhliches,
sonniges Lebensglück von G.\,K.\,Herloßsohn und mit Franz Abt:

\begin{center}
»Wenn die Schwalben heimwärts ziehen«,
\end{center}
\noindent
und alle die
Schwalben, die noch in sommerlichster Lust zwitschernd über uns und
der alten Schlachtenstätte sich im Kreise schwangen, schienen diese
Kreise zu verengern um meine klarstimmige Sängerin, während die
Lerche ihr zu Häupten im Blauen fest hing.

Ach und wie gut das weichmütige Abschiedslied in die Stunde paßte!
Sie hatten den Wagen mit den Schubkarren, Hacken und Schaufeln der
nächstens nachrückenden Erdarbeiter wirklich am Morgen unter unsre
Kastanienbäume geschoben. Die Schaufeln, Hacken und Äxte waren fürs
erste noch in der Turbinenstube niedergelegt worden; aber die
Schubkarren waren schon draußen geblieben und standen in zwei
langen Reihen zwischen den Gartentischen unter den lieben, dem
Verhängnis verfallenen Bäumen.

Das Kind hatte vollkommen recht: es wurde unheimlich in der Mühle
und Zeit, daß die Schwalben heimwärts zogen; denn nicht einmal
waren die Karren und Schaufeln die einzigen Anzeichen, daß es mit
der Lust und dem Behagen am Leben an dieser Stelle zu Ende ging.
Der Maurer und Zimmerleute Handwerksgerät war auch bereits auf dem
Wege nach meiner Väter lustigem Erbe, und unbedingt war's besser,
in der versunkenen Schanze des großen Krieges von Pfisters Mühle
und ihren Schicksalen weiterzuerzählen als unter ihrem Dache in der
öden Gaststube, wo der Architekt der neuen, großen
Fabrikgesellschaft schon seine Planrollen in den Winkel gestellt
hatte.

»Nun bist du schon wieder bei deiner dritten Zigarre und redest
nichts und sagst nichts als kuriose italienische Verse«, seufzte
Emmy, ihr Schwalbenlied mit dem ersten Verse endigend. »Wir stecken
noch immer in euerm ungemütlichen und übelriechenden Winter damals.
Wie wurde es denn nun weiter mit Albertine und Doktor Asche und dem
Herrn Doktor Lippoldes und deinem seligen Vater?«

Ja, wie wurde es denn eigentlich weiter? Wie waren die Bilder, nach
deren Verbleiben das Kind hinter dem Schwedenwall hier
augenblicklich sich erkundigte? Freund Asche war so gut als sein
Wort, das heißt, er sendete richtig sein gelehrtes Gutachten von
Berlin aus ein an meinen Vater, und als es nachher in einer
Berufszeitung gedruckt erschien, fand es sich, daß es eine Arbeit
von höchstem wissenschaftlichem Werte war, was ihn sicherlich
durchaus nicht überraschte und ihn also auch nicht in übermäßiges
Erstaunen versetzte. Große Ehre legte er damit ein bei den
Fachgenossen und sonstigen Kennern, bei den Poeten und sonstigen
sinnigen Gemütern und vor allem bei allen den Bach- und
Flußanwohnern, die in gleicher Weise wie der alte Mühlherr von
Pfisters Mühle und Krugwirtschaft zu dulden hatten. Aber wenig
Anerkennung und gar keinen Dank fand er bei den Leuten von
Krickerode und ähnlichen Werkanstalten, die das edelste der
Elemente als nur für ihren Zweck, Nutzen und Gebrauch vorhanden
glaubten. Diese stellten sich selbstverständlich auf einen andern
Standpunkt dem unberufenen, überstudierten Querulanten gegenüber
und ließen es vor allen Dingen erst mal ruhig auf einen Prozeß
ankommen.

Und das war denn der erste und der letzte Prozeß, den mein armer
Vater zu führen hatte, trotzdem daß er schon eine so erkleckliche
Reihe von Jahren in dieser bissigen, feindseligen Welt gelebt
hatte. Er war immer gut, friedlich und vergnügt mit eben dieser
Welt ausgekommen, sowohl als Müller wie als Schenkwirt, und hatte
jetzt also sein ganzes freundliches, braves Wesen umzuwenden, ehe
er seinerseits in den großen Kampf eintrat und im Wirbel des
Übergangs der deutschen Nation aus einem Bauernvolk in einen
Industriestaat seine Mülleraxt mit bitterm Grimm von der Wand
herunterlangte. Noch häufig sah ich ihn damals bis Ostern, ehe er
seinerseits zum Advokaten ging, in meinem Schülerstübchen und mit
immer wachsendem Herzeleid. Von Woche zu Woche kam er auf müderen
Füßen und in verdrießlicherer Stimmung. Zwar war, wie das immer
ist, vom Februar an, wo die Zuckerkampagne beendigt wird, sein
Mühlwasser wieder klar und die Luft über seinem Anwesen und in
seinem Hause wieder rein; aber die Gewißheit, daß im nächsten
Oktober das Elend von neuem angehe und Krickerode ihm ungestraft
von jeglichem Jahr die Hälfte streichen und stehlen dürfe, nagte zu
sehr an seiner Seele und an seinem Rechtsgefühl, als daß er noch in
der alten Weise die alte, lustige Schenke für den Sommer hätte
putzen und seinen fröhlichen, grünen Maienbaum zu Pfingsten vor
ihre Tür hätte pflanzen können.

»Reden Sie ihm nur um Gottes willen jetzt nichts mehr darwider,
Herr Ebert«, flüsterte mir Samse zu. »Es ist der leidige Satan,
aber es ist nicht anders, der Advokate bleibt anjetzo noch das
einzige, was uns in dem Jammer eine Ableitung geben kann!«

So begleitete ich nun den Alten zu dem juristischen Weisen, wie ich
ihm zum chemischen das Geleit gegeben hatte; aber es war doch noch
ein anderes, diesen als jenen nach Pfisters Mühle herauszuholen,
und da konnte es noch für ein Glück in allem Unheil gerechnet
werden, daß ich wenigstens den richtigen Mann für die Sache in
Vorschlag zu bringen wußte.

Diesmal war's ein sonniger, windiger Morgen im staubigen Monat
März, als ich den Vater durch die verkehrsreichsten Gassen der
Stadt zum Doktor Riechei begleitete. Und der ließ auch nicht mehr
seine Beine in Kanonen von einem der Baumäste in Pfisters Garten
auf den Zechtisch der Kommilitonen herabbaumeln, sondern hatte sie
in schäbigen schwarzen Büchsen stecken und trug einen von den
unberechenbaren, unbezahlten Bäuchen drin, über die ungezählte
Anekdotensammlungen seit Urväterzeiten zu scherzen wissen.

»Vater Pfister!« rief er, bei unserm Eintritt besagte Lastträger
immer noch mit merkwürdiger Behendigkeit von einem hohen Dreibein
herabschwingend und sie in grünen Pantoffeln auf dem zerschabten,
aber doch noch schreiend bunten Teppich vor uns feststellend. »Beim
Zeus, der Vater Pfister – der Müller und sein Kind! Leben Sie denn
wirklich noch? Ja, gottlob! Aber das ist ja riesig, das ist ja
reizend, das ist wirklich ganz famos!\ldots{} Du liebster Himmel, wie
lange hängt man hier im Spinnweb, ohne zu Ihnen hinausgekommen zu
sein!\ldots{} Und beinah noch ganz unverändert – ganz die liebe, alte,
heitere Kneipenseele und Kommersidylle! Vivat Pfisters Mühle~–«

»Jawohl, vivat Pfisters Mühle«, seufzte mein Vater. »Hat sich was
mit vivat Pfisters Mühle, Doktor. Na ja, Sie haben freilich
seinerzeit mit ihren Herren Studentenbrüdern manch liebes Vivat auf
mancherlei Dinge bei mir ausgebracht, und so kann ich wohl nichts
dawider haben, daß Sie's noch mal tun auf das alte Lokal, Herr
Doktor. Und mehr als ein Pereat haben Sie auch ertönen lassen beim
Vater Pfister seinerzeit, und – das ist jetzt die Parole. Pereat,
Herr Doktor! Und von wegen Pereat Pfisters Mühle sind wir heute
morgen zu Ihnen gekommen, und Sie erlauben wohl, daß ich mir für
einen Augenblick einen Stuhl nehme, denn es will doch nicht mehr
ganz so wie früher fort mit Ihres frühern, alten Schoppenwirts
unteren Beweggründen. Mein Junge da hat Ihnen die Papiere
mitgebracht, lieber Herr.«

Seinen besten, weichsten Sessel schob Rechtsanwalt Doktor Riechei
seinem neuesten Klienten zu, nahm ihm zärtlich Hut und Stock ab und
sagte gedehnt – nicht ohne wirklich freundschaftliche Teilnahme:

»Jawohl! Ja so! Ei freilich! Hm hm – nicht die größte, aber eine
von den größern Fragen der Zeit. Deutschlands Ströme und
Forellenbäche gegen Deutschlands Fäkal- und andere Stoffe.
Germanias grüner Rhein, blaue Donau, blaugrüner Neckar, gelbe Weser
gegen Germanias sonstige Ergießungen. Pfisters Mühle gegen
Krickerode! Und die Papiere für den Spezialfall bringt ihr sogleich
mit, das ist ja sehr schön – na, dann zeigt mal her. Setze dich
jedenfalls aber auch, Sohn Eberhard, so rasch wird das wohl nicht
gehen – Kinder, steckt euch vor allen Dingen erst mal eine Zigarre
an; – links von deinem Ellenbogen, würdiges Pennal.«

Ich hatte Asches Resumptio in die Hand Riecheis gegeben; und sich
von neuem auf seinen Dreifuß schwingend, fing er an zu blättern.

Eine gute Viertelstunde blätterte er, dann wickelte er plötzlich
das Schriftstück in blauer Pappe zu einer Rolle auf, sprang, hoch
sie über den etwas kahl werdenden Scheitel erhebend, in die Mitte
seines »Bureaus«, klopfte meinen anscheinend teilnahmslos
dasitzenden Vater auf die Schulter und rief:

»Und doch – und – abermals und zum drittenmal Vivat Pfisters Mühle,
Vater Pfister! Pereat Krickerode! Das ist ja der Fall, auf den ich
seit Jahren warte, um mich in die Mäuler der Leute zu bringen. Also
endlich auch mal ein richtiges Fressen für mich! Wären Sie ein
anderer, als Sie sind, Vater Pfister, so würde ich es Ihnen
sicherlich nicht so auf die Nase binden, daß ich mich hierauf seit
Lustren hingehungert habe. Kurzum, diese Sache führe ich, mit Asche
in der Tasche, und zwar glänzend, glorreich und zu einem guten
Ende. Vivat Pfisters Mühle!«

Wie würde mein Vater sonst in diesen Ruf eingestimmt haben! Heute
sagte er nur gedrückt:

»Tun Sie wenigstens Ihr Bestes für uns, Herr Doktor – für mich und
die alte Mühle! Glanz und Gloria käme wohl bei uns zwei immer an
die Unrechten; aber ein gutes Ende bleibt immerdar etwas recht
Wünschenswertes auch für einen, der seinen Knacks für alle Zeit
weggekriegt, hat, wie der alte Pfister von Pfisters Mühle.«

Für alle Zeit sehe ich das Gesicht vor mir, mit welchem Doktor
Riechei jetzt die Tür seiner Schreiberstube (es saß ein einziger
drin, und der bis zu jenem Tage auch nur mehr zur Zierde als zum
Nutzen) zuzog, auf den Zehen zu uns zurückkam und sprach:

»Das wäre denn in schönster Ordnung. Ich führe und gewinne Ihnen
Ihren Prozeß, würdiger Freund und Gönner; aber nun auch im vollsten
Vertrauen – jetzt sagen Sie mir mal um Gottes willen, weshalb haben
Sie eigentlich Krickerode nicht mitgegründet?«

\section{Sechzehntes Blatt}

\zusatz{Emmy auf dem Schubkarren in meinem versinkenden Paradies}
»Ja, das wollte ich eigentlich auch schon längst einmal fragen,
Herz – wirklich, weshalb hat denn dein armer Papa nicht mit auf die
große Fabrik unterschrieben, da alles ihm doch so bequem lag, und
hat keine Aktien genommen, sondern ist leider gestorben, obgleich
die Herren Asche und Riechei ihm doch seinen Prozeß gewonnen
haben?« fragte Emmy hinter dem alten Kriegswall unterm
Weißdornbusch.

»Weil er nicht anders konnte, Lieb.«

»Ach ja, es muß wohl so sein; obgleich es recht schade für uns ist
und obgleich auch mein Papa seine Gründe bis heute nicht recht
begriffen hat.«

»Hm, Kind, nach dessen Anhänglichkeit an seinen letzten grünen
Spazierfleck inmitten seiner Umgebung von Stein, Mörtel, Kalk und
Stuck möchte ich das doch nicht allzu fest behaupten. Jedenfalls
haben er und ich einander in dieser Hinsicht immer recht gut
begriffen.«

»Ja, Gott sei Dank, in diese seine Schrullen hast du dich immer
recht gut zu finden gewußt, und ich bin dir auch sehr dankbar dafür
gewesen; aber daß du's nicht bloß aus Liebe zu mir, sondern
wahrhaftig aus wirklicher Liebhaberei zu seinen sonderbaren Ideen
getan hast, das habe ich doch erst während unseres jetzigen
merkwürdigen Sommeraufenthaltes in eurer merkwürdigen Mühle
erfahren. Nun ja, es ist ja auch so recht schön, und es hat sich ja
auch, gottlob, alles nach des Himmels Willen recht passend
zusammengeschickt, und die Vorsehung weiß eben alles doch am
besten, wenn ihr Gelehrten das auch manchmal leugnen wollt. Erzähle
nur weiter. Eine Weile dauert es wohl noch, ehe die Sonne auf
deinem schrecklichen Feldwege erträglich wird und du deinen
spaßhaften, langen Schatten auf dem Felde vor dir her wirfst auf
dem Rückwege nach deiner närrischen, lieben, armen Mühle. Ja, ihr
seid richtig Vögel aus einem Nest, du und mein armer, lieber Papa!
›Schnurren, Miezchen, müßte der Mensch können und dabei
wiederkäuen; nachher wäre mein Ideal von ihm fertig‹ pflegte er
dann und wann zu bemerken, wenn er mich nach Tische am Kinn nahm.
Ach, ich fühle seine liebe, arme Hand noch immer um die
Mittagszeit, obgleich ich jetzt freilich dir zuliebe meine eigene
Küche habe in Berlin!«

Selbstverständlich erzählte ich nicht weiter. Spinnen und schnurren
wie Miez am Ofen oder in der Sonne und wiederkäuen konnte auch ich
noch nicht, obgleich ich das Ideal meines klugen und vernünftigen
Schwiegervaters wohl begriff und es wirklich vielleicht dann und
wann nicht ungern zur Darstellung gebracht haben würde. Aber am
Kinn konnte ich sein liebes Kind, mein liebstes Weibchen, auch
nehmen; und am Kinn fassen mußte ich es jetzt beim Heimchengezirp,
im Thymianduft, in der blühenden Heide im Hagedornschatten, allem
verjährten Verdruß und Elend und allen gegenwärtigen Schubkarren,
Äxten, Schaufeln, Hämmern und Sägen unter den Kastanienbäumen und
in der leeren Wirtsstube von Pfisters Mühle zum Trotz.

Es waren ja doch auch noch andere Dinge zu besprechen als die
überwundenen Erlebnisse der Leute in und um Pfisters Mühle! Hatten
wir denn nicht in der lebendigen Wirklichkeit dort in der Ferne,
jenseits des grünen Schanzenwalls, jenseits des Friedens von Wiese
und Ackerfeld unser selbstgebautes Nest nicht nur so weich als
möglich auszufüttern, sondern auch zuzeiten mit Schnabel und Klaue
im bittersten Sinne des Wortes gegen die große, unruhige Stadt
Berlin zu verteidigen? Waren wir nicht bereits mehrfach mit unserm
Hauswirt und einmal sogar auch mit der Polizei in Konflikt geraten,
und hatte nicht Emmy schon das innigste Verlangen, mal ganz
persönlich mit dem Präsidenten der letztern zu reden und ihm ihren
und seinen Standpunkt zum Besten der allgemeinen Behaglichkeit
klarzumachen? Und war vor allem nicht noch die große Frage zu
lösen, wo wir »bei unsern beschränkten Räumen« einen Zuwachs an
Raum für einen (»sieh mich nicht so närrisch an, bitte, bitte, du
dummer Peter!« flüsterte Emmy) einen anderen ahnungsvollen,
glückseligen, wunderbaren Zuwachs hernehmen sollten?

»Da hat es Frau Albertine doch gewiß besser«, seufzte Emmy, als nun
wirklich auf dem Heimwege und auf dem engen Feldpfade unsere
Schatten ganz spaßhaft lang, aber glücklicherweise ineinander
fielen. »Oh, die kann sich ausdehnen! Oh, wenn ich an die denke und
dann an uns, so wird mir ganz schwindlig!\ldots{} Gleich zuerst
Zwillinge und jetzt bald das vierte! Aber wenn der das Gelaß nicht
reicht, so baut der Doktor ganz sicher auf der Stelle an. In dieser
Hinsicht hat die Frau es viel besser als ich!«

»Aber sie hat es vorher vielleicht nicht so gut gehabt wie du, mein
Herz!« wagte ich meiner kleinen Melancholikerin in ihren bedrückten
Umständen als einen kleinen, möglichen Trostgrund ganz heimlich
zuzustecken, und glücklicherweise gelang es, und dies beruhigende
Wort fand vollen, zustimmenden Widerklang.

Aus der Tiefe ihres guten, mitleidigen Herzens aufatmend, meinte
meine Frau:

»Das ist freilich auch wahr! Ja, das arme Mädchen! Sie hat es recht
schlimm gehabt, ehe sie es besser bekam. Komm doch mit unter meinen
Sonnenschirm, Mann; die Sonne sticht noch immer recht sehr, und ich
möchte dich doch nicht ganz als geschälte Zwiebel nach Hause
bringen. Du hast mich auch ohne das heute schon mehrmals zu Tränen
und zur Rührung gebracht. Erzähle weiter, aber zapple nicht so,
sondern bleib mit unter meinem Schirm.«

Ich bemühte mich nach Kräften, beim Weiterwandern nicht zu sehr zu
zappeln und in dem lieben blau-rosigen Schatten zu bleiben, den
mein junges Weib auch auf diesen Weg unseres Lebens warf.~–

Als der Tag im veränderlichen Monat April eintrat, der Tag, an
welchem ich zum erstenmal von meinen nächsten Heimatsumgebungen für
längere Zeit Abschied zu nehmen hatte, um in die Ferne und auf die
Universität zu ziehen, war der Prozeß meines Vaters gegen
Krickerode bereits im Gange, und wie uns um und in Pfisters Mühle
däuchte, stand das Universum auf den Zehen, das Resultat
erwartend.

Asche hatte nichts mehr von sich hören lassen. Der war schon in
Berlin. Aber an einem sonnigen, windigen, dann und wann von einem
Regenschauer besprengten Tage kam ich in sehr seltsamer Weise doch
wieder zu der Gewißheit, daß er noch in der Gegend spuke und in
innigster Art mit ihr in Verbindung zu bleiben sich bemühe.

Unser Fluß im April war wie je vorher, ehe Zucker an seinem
rauschenden, murmelnden Laufe gemacht wurde. Die Vorfrühlingsfluten
vom Gebirge her hatten allen Schlamm und Wust aus Krickerode von
seinem sonnenbeleuchteten Grund und von seinem Ufergebüsch weg- und
abgespült. Es lag der erste lenzgrüne Hauch auf Baum und Strauch,
auf Wiese und Feld. Daß allerlei Blumen blühten und einige Arten
bereits verblüht waren, achtete ich durchaus nicht. Ich hatte an
andere Dinge zu denken, als ich nochmals jenen Pfad am Bache
aufwärts hinschlenderte, den wir an jenem zweiten Weihnachtstage
mit Samse und dessen ominösem Flaschenkorbe gingen.

Es gehörte zwar alles dazu, aber – im einzelnen, was waren Blumen,
was Frühlingsgrün, was Krickerode, was Prozesse, ja, was Pfisters
Mühle für das erlöste Pennal, für den angehenden Fuchs, für den
freien, von den Göttern auf seine eigenen Füße in das unermessene
Dasein hingestellten Menschen, kurz, für den demnächstigen
studiosus philologiae Eberhard Pfister?

Grün mochte die Welt sein, blau mochte sie sein; so blau, so grün
wie ich, Ebert Pfister, war sie nicht um diese Zeit, in diesen oder
– jenen Tagen. Und es war, den Unsterblichen sei Dank, mein volles,
unbestrittenes Recht, in mir grüner, blauer, bunter mich zu
empfinden als irgend etwas anderes rings um mich her!

Doch da trat nun aus dem Frühling, aus dem Licht und Schatten, aus
dem großen Andern um mich her eine Gestalt, die meinem unbefangenen
und gleichmütigen Mitatmen im übrigen doch wenigstens für einige
Zeit ein Ende machte. Albertine Lippoldes redete mich an auf dem
Buschpfade an meines Vaters Mühlwasser.

In demselben abgetragenen grauen Kleide wie an jenem
Weihnachtsfeiertage stand sie unter dem nämlichen Baum an der Hecke
wie damals, wo sie auf ihren Vater und unsere Expedition zur
Erforschung der Gründe vom Untergange von Pfisters Mühle wartete.
Als ich, betroffen ob ihrer bleichen und kränklichen Erscheinung,
stehenblieb und die Mütze zog, kam sie auf mich zu und reichte mir
die Hand.

Sie lächelte auch dabei, aber es war das Lächeln einer, die ein
schweres Leid auf der Seele trägt und ein schwerwiegend Wort
auszusprechen hat.

»Sie wollen uns nun auch verlassen, Herr Pfister? Und Sie gehen
jetzt auch nach Berlin?« fragte sie, und als ich dieses stotternd
bejahte, sagte sie mit leiser, beklommener Stimme:

»Dann hätte ich wohl eine Bestellung dort, Herr Ebert, und Sie
würden mir einen rechten Gefallen tun, wenn Sie dieselbe ausrichten
wollten.«

»Mit dem größten Vergnügen, Fräulein! Alles, was Sie wünschen. Was
und an wen? Mit der Rapiditat eines Mokkakäf – ja wirklich und auf
Ehre, Fräulein Albertine, mein Herzblut würde ich~–«

»Das nicht, Sir Childe«, sagte das Fräulein und lächelte noch
einmal dabei. »Nur ein Wort an Ihren Freund, Herr Doktor Asche,
auszurichten, möchte ich Sie freundlich bitten.« Und damit
verschwand das Lächeln aus ihren feinen, müden Zügen, als würde es
nie wieder dahin zurückkehren. Mit einer bittenden Bewegung beider
Hände, doch mit einem fast zornigen Blick über mich weg in die
grüne, eben wieder im Sonnenlichte glänzende Ferne, flüsterte sie
mit unterdrücktem Schluchzen:

»Sagen Sie – bestellen Sie Ihrem Freunde, daß Albertine Lippoldes
ihm vom ganzen Herzen dankbar sei für seine Güte gegen ihren Vater,
daß er aber kein Recht – daß er es unterlassen müsse, sie so rat –
sie noch ratloser zu machen durch seine – Teilnahme. Sagen Sie
Ihrem Freunde, daß mein armer Vater freilich nicht mehr das Mitleid
von der Anerkennung zu unterscheiden wisse, aber daß mich mein
Leben, vielleicht vor der Zeit, alt und sehr klug gemacht habe und
daß Albertine Lippoldes nicht mehr so leicht sich der bestgemeinten
Täuschung hinzugeben verstehe. Bestellen Sie Ihrem weisen, treuen,
guten Freunde~–«

Ob ich es damals schon ganz genau wußte, was ich eigentlich sagen
und bestellen sollte, weiß ich auch heute noch nicht, aber daß auch
mir die Tränen in den Augen standen und daß ich, dieselben
hinterschluckend, versprach, alles ganz genau auszurichten, weiß
ich heute noch sehr genau. Ich habe in der Erinnerung ein Flimmern
vor dem Gesicht, das ich vielleicht auch auf einen eben
niederrauschenden Regenschauer jenes Apriltages schieben könnte.
Durch dieses Flimmern sah ich, wie Fräulein Albertine ihr Tuch
fröstelnd zusammen- und über ihr Haupt zog und rasch, doch
unsichern Fußes, zu dem verwahrlosten Anbauerhaus zurückeilte, zu
dem kümmerlichen Dach, unter welchem Doktor Felix Lippoldes
wirklich nur noch von dem Mitleiden und nicht mehr von der
Anerkennung der Welt lebte oder vegetierte.

Und trotzdem, daß ich damals noch ein recht junger Mensch und sehr
dumm und unerfahren in den meisten, und zwar innerlichsten
Angelegenheiten des Lebens war, fühlte ich doch in aller
Verblusterung durch, weshalb ich grade dem Doktor A.~A.~Asche in
Berlin diese mir eben von dem Fräulein aufgetragene Bestellung
ausrichten sollte. Gegen Vater Pfisters hülfreiche Hand hatte
Albertine Lippoldes nimmer mit ihren zwei hülflosen, tapfern Händen
eine abwehrende Bewegung gemacht.

Ich sah das Fräulein vor meiner Abfahrt zur Universität nicht
wieder, aber wohl den Papa Lippoldes. Diesen traf ich noch einmal
in der Stadt, doch will ich nicht genauer beschreiben, in welchen
Zuständen. Auf dem Hausflur des Blauen Bockes unter den
Marktleuten, Ausspanngästen und städtischen Kutschern und
Straßenvagabunden fand ich ihn vor dem Schnapsschank. Da hängte er
sich an mich, redete mit schwerer, stammelnder Zunge auf mich ein
und gab mir seinerseits Grüße an seinen liebsten Freund , seinen
einzigen Freund Asche, seinen besten Freund Adam, seinen letzten
Trost und seine letzte, einzige, wahre Stütze in dieser »Lausewelt«
mit. Am andern Tage ging ich mit beiden Bestellungen aus Pfisters
melancholischer Mühle in die so lachende, sonnige, aller Wunder und
Hoffnungen volle Welt hinein nach Berlin.

»Jott sei Dank, da sind wir denn endlich!« seufzte Emmy mit
echtestem Berliner Akzent und erinnerte mich dadurch aufs
hübscheste und vergnüglichste, daß ich nicht ohne Erfolg auf die
Suche nach Abenteuern, Wundern und verzauberten Prinzessinnen von
meines Vaters Hause ausgezogen sei. Ob sie aber mit ihrem Ausruf
ihre Vaterstadt Berlin oder unsern Mühlgarten meinte, kann ich
nicht sagen. Jedenfalls waren wir wieder unter den schattigen, grün
und treu aushaltenden Kastanien und unter den stillen Tischen und
Bänken des letzteren angelangt. Das Kind aber war nicht auf einer
der Bänke niedergesunken; es hatte sich, mit dem Taschentuche sich
Kühlung zuwehend, auf einem der Schubkarren, die man behufs der
demnächst beginnenden Erdarbeiten unter den unschuldigen, lieben,
vertrauensvollen Bäumen zusammengefahren hatte, hinsinken lassen.

\section{Siebzehntes Blatt}

\zusatz{Fräulein Albertine hat etwas nach Berlin zu bestellen}
Der Architekt für den neuen Fabrikbau an Stelle von Pfisters Mühle
ist gar kein übler Mann, obgleich er keineswegs jenem berühmten
Kollegen in den Wahlverwandtschaften gleicht und durchaus nicht
»ein Jüngling im vollen Sinne des Worts« zu nennen ist, sondern als
ein weniger wohlgebautes als wohlbeleibtes Individuum mit der
Veranlagung zu einer Kümmelnase sich darstellt. In Berlin hat er
den Doktor Asche kennengelernt, und in unserer Stadt, am
entgegengesetzten Ende unserer Pappelallee, gehört Doctor juris
Riechei zu seinen behaglichsten Bekanntschaften, und der Herr
Baumeister weiß ganz genau anzugeben, weshalb es gar nicht anders
möglich war, als daß jene beiden Herren sehr wohlhabende Leute
wurden, »wahre Fettaugen auf unsern bekannten dünnen
Bettelsuppen«.

»Es sind beide Phantasiemenschen«, meinte er, der Architekt, »aber
alle zwei mit dem richtigen Blick und Griff fürs Praktische. Und,
lieber Pfister und gnädige Frau – das Ideale im Praktischen! Das
ist auch meine Devise. Verlassen Sie sich drauf, bester Doktor, Sie
sollen auch noch Ihre Freude an dieser Stelle erleben, wenn Sie uns
– mir noch einmal mit der Frau Gemahlin übers Jahr hier das
Vergnügen Ihres Besuches schenken wollen. Das Schöne, das
Großartige im innigen Verein mit dem Nützlichen! So hält's auch
unser gemeinschaftlicher Freund Asche, den ich, wie gesagt,
ebenfalls in seinen Anfängen kannte. Und Sie, Pfister, konnten gar
nichts Gescheiteres tun, als Ihr an hiesiger Stelle überflüssig und
nutzlos gewordenes Kapital in seinem Unternehmen anzulegen.
Gigantisch – einfach gigantisch das! Und daneben – in feinster
Renaissance dieses Lippoldesheim! Wundervoll!\ldots{} Nun, ohne mir
schmeicheln zu wollen, wir werden jedenfalls unser Bestes tun,
unsere Gesellschaft und ich, Ihnen etwas ähnlich Imponierendes auch
hier auf Ihres seligen Papas idyllisches Besitztum hinzustellen.
Wir verlassen uns fest darauf, daß Sie sich die Geschichte übers
Jahr wenigstens mal flüchtig ansehen.«

»Wenn es mir möglich ist«, sagte ich müde. Der Architekt mit dem
Zirkel in der Hand und der Bleifeder im Munde beugte sich von neuem
über seinen in meines Vaters leerem Gastzimmer ausgebreiteten Plan,
indem er meine Frau, soweit ihm das möglich war, tiefer sowohl in
das Ideale wie das Praktische, das Schöne wie das Nützliche, das
Grandiose, das Imponierende und das Idyllische desselben mit sich
zog.

»Ich komme gleich wieder heraus unter die Bäume, Ebert« sagte Emmy
über die Schulter; und unter den Bäumen und zwischen den
Schubkarren hatte ich eine geraume Zeit allein für mich mit der
erloschenen Zigarre zwischen den Zähnen auf und ab zu wandeln, ehe
sich mein Weib wieder zu mir fand.~–

Es läßt sich nicht leugnen, großartig ist das wasserverderbende
Geschäft am Ufer der Spree, in welchem Freund Adam heute als
leitende Seele waltet; als Fräulein Albertine mich mit ihrer
Bestellung zu dem Phantasiemenschen mit dem merkwürdigen Blick fürs
Praktische schickte, traf ich ihn freilich noch auf den unteren
Stufen der Leiter des Glücks, aber doch schon im Begriff, drei
Staffeln für eine nach der Höhe hinauf zu nehmen.

Nun kam es mir zutage, weshalb er sich vordem so eingehend mit der
schmutzigen Wäsche des Ödfeldes im allgemeinen und der
Schlehengasse im besonderen beschäftigt hatte. Schmurky und
Kompanie hieß die Firma, unter der er augenblicklich noch seine
wissenschaftlichen Erfahrungen im Fleckenreinigen im großen genial
zur Geltung brachte. Und wenn er selber in der umfangreichen Stadt
Berlin noch etwas schwierig zu finden war, so fand ich Schmurky und
Kompanie doch sofort und mich, grade wie bei Krickerode, vor
gotischen Toren und Mauern, hinter denen sich ganz etwas anderes
tummelte als Ritter, Knappen, Edelfräulein, Falkoniere und
Streitrosse.

Betäubt schon durch die sonstigen Erlebnisse meines ersten Tages in
der Hauptstadt wurde ich willenlos, vom Türhüter aus, sozusagen von
Hand zu Hand weitergegeben, und zwar durch den größten Tumult und
die übelsten Gerüche, die jemals menschliche Sinne überwältigt
hatten. Über Höfe und durch Säle – wie selber erfaßt und
fortgewirbelt von dem großen Motor, dem Dampfe, der um mich her die
Maschinen – Zentrifugalen, Appreturzylinder, Rollpressen, Kalander,
Imprägnier-, Kräusel-, Heft-, Näh- und Plisseemaschinen – in
Bewegung setzte, taumelte ich; – durch Wohldüfte, gegen welche
meines Vaters Bach in seinen schlimmsten Tagen, gegen welche die
Waschküchen und sonstigen Ausdünstungen der Schlehenstraße im
Ödfelde gar nichts bedeuteten, mußte ich; – und in einem von dem
ärgsten Getöse nur durch eine dünne Wand geschiedenen Raum fand ich
den Freund, nicht mehr über Olgas Unterrock, sondern über ein
zahlen-, buchstaben- und formeln-bedecktes Papierblatt mit seinem
Leibe und seiner Seele, mit all seinem Wissen und Können gebeugt
und – richtete ihm Albertine Lippoldes Bestellung aus!\ldots{} Ich darf
ihm aber das Zeugnis geben, daß er alles ihm eben Vorliegende
beiseite und über den Haufen warf, als die letzte führende Hand
mich ihm in das Allerheiligste seiner großen –
\emph{chemischen Waschanstalt} schob.~–

»Mein Telemachos!\ldots{}.. Ebert – mein Sohn Ebert Pfister von Pfisters
Mühle!\ldots{}.. Bengel – Knabe – Jüngling, welch ein Hauch und Licht
aus bessern, besten Tagen! Was zum Henker, richtig – seit 'nem
halben Jahre schon angemeldet hier im Morast, im Pechsumpf, in
Malebolge. Na, so kann ich dir nur wiederum raten, stehe nicht so
dumm da, sondern stürze in meine Arme, Kind.«

Ich stürzte, warf mich in seine Arme, das heißt, wir schüttelten
herzhaft und mit wahrhaftiger Freude einander die Hände, und dann
zog mein Exmentor vor allen Dingen seinen Rock an und meinte:

»Du kommst im Fleisch aus einem Reiche, in dem ich mich eben im
Traum temporär aufhielt. Du wirst mir allerlei erzählen wollen, und
wir können dann ja unsere Notizen vergleichen. Gefrühstückt wirst
du haben, zum Mittagessen fahren wir in die Stadt – vor dem
verdammten Gelärm nebenan hört man sein eigen Wort nicht und noch
weniger das eines andern: vielleicht würdest du vorziehen, bei
etwas geringerem Getöse und etwas reinerer Luft von euch zu
berichten?«

»Ja, es riecht hier in der Tat wie bei uns im Winter nach allerlei,
aber vorzüglich nach Benzin, wie damals in deiner Schlehengasse.«

»In der Tat? Merkst du das wirklich?« schmunzelte Asche
geschmeichelt. »Benzin! Grandioser Fortschritt, riesige
Errungenschaften, stupifizierende Neuerungen! Ich hoffe, dir an
deiner eigenen Garderobe demnächst zu beweisen, welche
Gigantenschritte wir auf dem Wege zur höchstmöglichen
Vollkommenheit in unserm Fache gemacht haben! Dreh dich mal um; –
wie wär's, wenn du auf der Stelle deinen Rock auszögest und ihn in
jene Klappe reichtest? Wir stellen dir sofort die allein aus dem
Kragen extrahierten Fetteile als Rosenpomade und
Kokusnußölsodaseife wieder zu! Du möchtest lieber nicht? Nun, so
rede mir jedenfalls mit Achtung von allem bei siebenzig bis hundert
Grad destillierendem flüssigen Kohlenwasserstoff; aber da die
Verwendung desselben freilich mit einigem Lärm verknüpft ist, so
komm mit. Wandeln wir auch hier ein wenig an \emph{unserm}
Wasserlauf auf und ab, denke dich völlig nach Pfisters Mühle und
erzähle mir so viel als möglich von – \emph{euch!}

Er führte mich durch eine zweite Tür seines Arbeitsgemachs
merkwürdigerweise durch ein von gotischen Kreuzgängen im Viereck
umgebenes Klostergärtchen in einen andern Korridor, zu einem andern
Flügel des ungeistlichen Fabrikgebäudekomplexes und von da aus
platt auf die Landstraße an der, wie es schien, halb ohnmächtig vor
Ekel auf niedergetretenen »Parisern« gen Spandau schlurfenden
Spree.

»Es hindert dich durchaus nichts, dir einzubilden, wir schritten
wiederum still und friedlich, wenn auch mit einiger Sehnsucht nach
der Ferne, an den Bächen deiner Heimat. Nun singe mir dein Lied von
Pfisters Mühle! Was macht der alte Herr? Gedenkt die Jungfer
Christine meiner noch mit dem alten Wohlwollen? Und vor allen
Dingen, wie steht der große Prozeß Pfisters Mühle gegen
Krickerode?«

Ich dankte für alle diese gütigen Nachfragen und war aus Bedürfnis
ziemlich ausführlich. Mein Exmentor nahm alles mit Gleichmut hin
und machte mir den Eindruck, als ob er stellenweise bei meinem
Berichte abwesend sei, und zwar in dem kleinen Kabinett, dem
Maschinenlärm, dem destillierten Kohlenwasserstoff und über den
Bogen mit den Zahlen, Buchstaben, Formeln und Figuren von Schmurky
und Kompanie auf der andern Seite der Straße.

»Und dann habe ich zuletzt noch eine Bestellung an dich, Asche.«

»Die wäre?\ldots{} Schwach opalisierend\ldots{} nicht flüchtige Substanzen\ldots{}
11,36~Prozent Chlor – du weißt, wie du mir durch die kleinste Notiz
aus dem alten, lieben Leben das Herz erregst~–«

»Von Fräulein Albertine Lippoldes nämlich.«

Da tat der Mann an meiner Seite und am Ufer des graufarbigen
Stromes einen Schritt zur Seite, um mich besser ansehen zu können.
Er packte mich auch am Arm, und zwar gar nicht sanft, und
schnarrte:

»Was sagst du? Was hat sie gesagt? Was hatte sie mir durch dich
dummen Jungen zu bestellen? Menschenkind, bei den unzählbaren
Wohltaten, die ich dir vordem erwiesen habe~–«

»Sie läßt dir sagen, Adam – o, ich wollte, ich könnte dir malen,
wie sie dabei aussah~–«

»Gar nicht nötig; aber ich tauche dich sofort dort in die
schleichende Brühe, wenn du mir das Geringste von dem Deinigen zu
ihrer Meinung tust!«

»Nun, sie läßt dir, zitternd, ich weiß nicht, ob vor Verdruß oder
Unglück, aber jedenfalls mit verschluckten Tränen bestellen, daß
sie dir von Herzen dankbar sei, daß du aber doch lieber unterlassen
mögest, sie ferner so sehr zu kränken. Sie wisse noch das Mitleid
von der Anerkennung zu unterscheiden, aber ihr Papa nicht mehr. Und
sie sagt, daß es sie recht elend mache, dir auch noch und nicht
bloß meinem Vater und anderen verpflichtet zu werden. Wir standen
an der Hecke, grade an der Stelle, wo du die erste Flasche aus
Samses Flaschenkorb mit dem Wasser aus Krickerode fülltest; und
sie, wie gesagt, mit Frösteln, und ich weiß nicht, ob sehr zornig
auf dich oder sehr dankbar. Dann fing es wieder an zu regnen, und
sie ging auf unsichern Füßen nach Hause, grade wie an dem Morgen,
wo du mit uns ihr so zweifelhaft nachsahst, nachdem ihr Vater uns
zum Frühstück eingeladen hatte. Und den Papa Lippoldes habe ich
kurz vor meiner Abreise auch noch gesprochen, und zwar im Blauen
Bock. Du seist sein letzter und einzigster Trost, läßt er dir
bestellen, und er halte dich auch für den einzigen, der ihn je
begriffen, verstanden und vor allem seinen ›Eulogius Schneider‹
gewürdigt habe, und die Nachwelt werde das dir anerkennen, und er
werde in seinem literarischen Nachlaß auch auf dich hinweisen und
dich in das Gedächtnis des kommenden Menschengeschlechts mit
hinübernehmen.«

»Den lauten, schreiigen Hals hätte man dem Narren bei seiner Geburt
umdrehen sollen. Das wäre eine Wohltat für mich, für ihn und für
die Welt und Nachwelt gewesen! Zum Henker mit seinem Bombast, Quark
und quäkigen Egoismus. Na, die Seife, die ich mir daraus koche!
Ebert Pfister, mein lieber Sohn, du wirst heute und noch manch ein
andermal mein Gast sein, aber den Appetit hast du mir für diesmal
gründlich verdorben. Komm mit und laß sehen, wo du in dem räudigen
Nest dort unter der Rauchwolke untergekrochen bist. Es ist mir ein
Trost, daß ich wenigstens dich aus den alten, besseren Tagen wieder
in der Nähe habe. Daß ich mein Mentoramt unter veränderten
Umständen hie und da von neuem aufnehme, wird dich nicht hindern,
deine eigenen Wege zu gehen. Hm, diese albernen, braven
Frauenzimmer – diese Weiber – diese dummen, guten Mädchen mit ihren
verschluckten Tränen und – sonstigem Unsinn. O Krickerode, Felix
Lippoldes und Pfisters Mühle – o Schmurky und Kompanie!«

Das letztere murrte er kaum verständlich in sich hinein. Wir fuhren
sodann in die Stadt, und der Freund machte sein Wort gleich wahr
und nahm seine Mentorschaft mit der alten, närrisch versteckten
Hingebung auf. Er führte mich auch in seine dermalige
Privatwohnung, die sich um ein beträchtliches in Ansehung
menschlichen Behagens von der in der Schlehenstraße unterschied.
Ich ließ einige Bemerkungen darüber fallen, in wie verhältnismäßig
kurzer Zeit jeglicher Duft und Schein von Vagabundentum um ihn her
verschwunden sei, und er meinte ruhig:

»Es ist besser, nie und nirgend zu laut von dem zu reden, was man
auf der Spindel hat. Merke dir das für kommende, verständigere
Jahre, Kind. Beiläufig, du wirst wahrscheinlich bald nach Hause
schreiben, um deine glückliche Ankunft und deinen ersten Eindruck
hier zu melden?«

»Ich täte jedenfalls meinem Vater eine Liebe damit.«

»Dann tue sie ihm ja, und von mir laß einfließen, du habest deine
Botschaft richtig ausgerichtet.«

»Weiter nichts, Asche?«

»Stelle keine überflüssigen Fragen in betreff der Schicksale
anderer an die Zukunft, sondern beschäftige dich fürs erste
möglichst intensiv mit dem, was vor deiner eigenen Nase liegt, vir
juvenis.«~–~–~–

»Du, dem Herrn Baumeister seine neue Anlage imponiert mir aber doch
wirklich sehr!« sagte Emmy, unter den Kastanien von Pfisters Mühle
wieder ihren Arm in den meinigen hängend.

\section{Achtzehntes Blatt}

\zusatz{Ausführlicher über Jungfer Christine Voigt}
»Es ist doch heute eigentlich recht sonderbar, daß du so lange dich
in Berlin aufhieltest, ohne daß ich eine Ahnung davon hatte und
wahrscheinlich auch ohne daß wir uns je einmal auf unsern
Schulwegen begegneten«, sagte Emmy.

»Einige Semester war ich ja auch auf andern Schulen«, meinte ich.
»Aber~–«

»Aber das Schicksal legte es dir doch vor die Nase, daß es in
Berlin am besten für dich zum Studieren sei – was?«

Es ging nicht anders; ich mußte dem Kinde mit einem Kuß die
Versicherung geben, daß sie wie in vielen andern Sachen meines
Lebens, so auch in diesem Dinge vollständig recht habe. Das geschah
in unserm Stübchen unterm Dach, während es draußen wieder einmal
regnete, und unter den ersten Vorbereitungen zum Packen und zur
Abfahrt von Pfisters Mühle.

Die Zeichen, daß unsere flüchtige Sommerlust hier zu Ende sei,
mehrten sich zu sehr. Der Architekt in der Gaststube unter uns
pfiff Tag für Tag über seinen Plänen das Beliebteste aus den
neuesten Sommertheateroperetten. Bruchsteine wurden ununterbrochen
angefahren und in Quadraten aufgeschichtet. Es war ein ewiges
Kommen und Gehen, Schimpfen und Lärmen von allerlei Volk, und meine
alte Christine war zu nichts mehr zu gebrauchen in der alten,
verlorenen Mühle!\ldots{}

Ach, es ist eigentlich viel zu wenig die Rede gewesen in diesen
Blättern von der alten Christine. Ach, wenn was mit in die Bilder
gehörte, die ich hier von Pfisters gewesener Mühle malte, so ist
das meine arme, greise, liebe Wärterin und Pflegemutter, so ist das
die harte, arbeitsselige Hand, die traute, treue, weibliche Seele
von meines Vaters Haus und Hof, Küche und Keller, Feld und Garten,
die letzte »schöne Müllermaid« des Ortes.

Ich hatte Latein, Griechisch, moderne Sprachen und sonst allerlei
erlernt. Ich war in Berlin, Jena und Heidelberg auf Schulen gewesen
und auch sonst noch ein gut Stück in die Welt hinein, in Ländern,
wo Menschen die modernen Sprachen zum Hausgebrauch haben. Ich hatte
mir ein ander Hauswesen in der großen Stadt Berlin gegründet und
ein jung Weib hineingenommen – und ich und mein Weib, wir waren,
wenn ich gleich der juristisch unanfechtbare Erbe meines Vaters
war, doch nur die letzten Gäste, wenn auch Stammgäste, von Pfisters
Mühle.

Aber die alte Christine hatte nichts weiter in der Welt gehabt und
kannte nichts weiter als die Mühle, und so hatte sie nun, da es
bitterer, blutiger Ernst auch mit ihrem Abschiednehmen wurde, so
ziemlich alles verloren, und wenn ein Mensch in der Wüste um sie
her sanft und vorsichtig mit ihr umgehen mußte, so war ich das –
ich, Ebert Pfister, meines verstorbenen Vaters Sohn und Erbe.

Nun waren die Tage, wo ich sie hier und da sitzend fand,
zusammengekauert auf einer Treppenstufe, in einer Bodenkammer, am
leeren Mühlkasten oder am Fluß, trotz des warmen Sommers fröstelnd,
die beschäftigungslosen Hände in die Schürze gewickelt. So manches
Jahr durch hatte sie die lustigen Bänke und Tische unter den
Kastanien ihres Meisters fröhlichen Gästen überlassen: jetzt hatte
sie dieselben für sich allein, und so fand ich sie eben wieder auf
einem der Sitze in einer der Lauben am Bach, während der linde
Sommerschauer leise auf das dichte Blätterdach niederrieselte.

Und den schweren, alten Kopf mit beiden Händen fassend und den
Oberkörper in Angst und Ruhelosigkeit hin- und herwiegend,
schluchzte sie, als ich zu ihr trat:

»O Ebert, daß ich das auszustehen habe! Daß ich dieses erleben
muß!\ldots{}«

\begin{verse}
»Da öffnet sich ein Fensterlein,\\
Das einzige noch ganze,\\
Ein schönes, bleiches Mägdelein\\
Zeigt sich im Mondenglanze\\
Und ruft vernehmlich durchs Gebraus\\
Mit süßer Stimme Klang hinaus:\\
Nun habt ihr doch, ihr Leute,\\
Genug des Mehls für heute!«
\end{verse}
\noindent
so summte es mir schauerlich aus dem Liede des
untergegangenen Dichters, aus der schönen Allegorie, in der sich
Gleichnis und Dichtung so vollkommen decken, durch den Sinn. In
seinem Liede meint der Sänger mit dem bleichen, schönen Mädchen die
Poesie selber, die ihre Mühle im romantischen Walde in die Hand der
Tagesspekulanten übergehen sieht; und ich bin Philologe genug, um
mich hier darüber auszulassen, aber ich war auch Poet genug, um
auch bei grauem Tageshimmel und leisem Regenfall den wundervollen,
innersten Herzschlag des Erdenlebens da zu erhorchen, von wo er mir
in diesem Augenblicke wirklich herklang. Ich hielt die dürre, harte
Hand, ließ das trostlose Greisenhaupt an meiner Schulter lehnen und
horchte kaum hin, als hinter uns in Pfisters Mühle sich eines der
heute noch ganzen Fenster öffnete und mein junges, rosiges
Mägdelein sich vorbeugte und rief.

»Aber Kinder, ihr werdet ja bis auf die Haut naß bei dem Regen. Was
sitzt ihr denn da auf der Bank am Wasser und rührt euch seit einer
halben Stunde nicht?«

Ich hatte während dieser halben Stunde das alte Weiblein neben mir
zu trösten gesucht, so gut ich konnte, und was das Naßwerden
betraf, so boten ja an diesem Abend noch die alten Bäume ihren
Schutz der Poesie und dem juridischen Rechtsnachfolger in Pfisters
Mühle.

»O Ebert, laß mich hier! Ich möchte doch hierbleiben und mich in
den Grund, den sie übermorgen ausheben wollen, verschaufeln lassen!
In meiner Kinderzeit erzählten sie, daß sie immer ein lebendiges
Kind mit vermauert hätten, um ein festes Haus zu haben: ich möchte
mich nun als ein altes Weib mit vergraben lassen, um ihnen
allmählich an ihrem Mauerwerk zu rütteln. Ach Ebert, lieber Ebert,
so habe ich es mir doch nicht vorgestellt, und überleben tu ich es
nicht und will es auch nicht!«

»Samse hat es aber ja auch überlebt, arme, liebe Christine.«

»Ja, der auch! Aber dein seliger Vater nicht! Und dem wurde ja noch
nicht einmal das Dach über dem Kopfe und der Boden unter den Füßen
weggerissen, sondern er hatte nur seinen Ärger und Kummer an den
bösen Gerüchen von Krickerode und unseres Doktor Asches dummen
Pilzen mit den grausamen lateinischen Namen.«

»Christine, es müssen die Menschen so vieles ertragen und kommen
mit ihren Schmerzen durch. Denke nur an Fräulein Albertine, unsre
liebe Freundin, wie schlimm es der in Pfisters Mühle und mit
Pfisters Mühlwasser ging und was sie Schreckliches dadurch erlebte,
und nun wohnt sie ja auch in Berlin, und es geht ihr dort recht
gut, und du wirst viel Vergnügen an ihren hübschen, gesunden
Kindern haben, und – höre, Christine, wir, als wie Emmy und ich,
wir können dich ja gar nicht entbehren in unserer jungen,
unerfahrenen Haushaltung! Hast mich ja von meiner Mutter Armen
genommen und großgepäppelt, und – wer weiß, was die Familie Pfister
in dieser Hinsicht noch alles von dir erwartet, und wer alles auf
deine Gegenwart an seiner Wiege fest rechnet!«

Ich mochte wohl die richtige Saite in der Alten betrübtem Gemüte
angeschlagen haben. Sie trocknete sich die Tränen mit der Schürze
ab und seufzte und rückte sich zurecht auf der Bank. Der Regen
rauschte immer heftiger auf unser Blätterschutzdach nieder und fing
doch an nun durchzuschlagen.

»Wir werden wirklich wohl noch naß, wenn wir noch länger hier
sitzenbleiben, Ebert. Und dein kleines Frauchen wird wunders
denken, was für Geheimnisse wir uns hier anzuvertrauen haben. Und
das, was du eben von Fräulein Albertine gesagt hast, hat ja leider
seine Berechtigung. Viel Schmerz und Elend seit, wie sie sagen,
manchen hundert Jahren hat Pfisters Mühle auch gesehen, trotz aller
Lust und guter Kost und Liedersingen und Gläseranklingen rundum. O
Gott ja, es ist dies ja derselbige Ort, wo wir ihn fanden, den
armen Herrn! Dort der Busch halb im Wasser, an dem er sich gefangen
hatte, ist auch noch vorhanden, und hier in diese Laube zogen ihn
dein seliger Vater und Doktor Asche zuerst, nachdem sie ihn aus dem
Wasser gezogen hatten. Und hier zu unsern Füßen lag er, bis Samse
und die Knappen kamen, um ihn in die Gaststube tragen zu helfen.
Gütiger Himmel der Gast da und der Abend und die Nacht und die
darauffolgenden Tage könnten einen freilich schon mit dem Abbruch
von Pfisters Mühle aussöhnen! Hast du denn eigentlich deiner
kleinen Frau schon das Nähere davon erzählt, wie es kam, daß der
berühmte Herr Doktor Lippoldes von unserer Wirtschaft aus begraben
wurde, und wie es kam, daß Fräulein Albertine von der Mühle aus
Hochzeit machte?«

Ich schüttelte den Kopf:

»Wir sind hier in der Sommerfrische, wie man das in der Stadt
nennt, gewesen, Christine. Ich habe Emmy hergebracht, um ihr die
Sonne, die Bäume, die Wiesen und den Bach von Pfisters Mühle und
meiner Jugend noch zu zeigen. Sie würde nicht so harmlos und
vergnüglich diese Wochen durch in der für sie doch schon so
sonderbaren Mühle gewohnt haben, wenn ihr dieses Trauerspiel drin
gespukt hätte. Aber unsere Zeit hier zählt ja nun nur noch nach
Stunden. Das Kind wird nicht fortgehen, ohne auch dieses Letzte von
dem guten, alten Hause und Garten an Ort und Stelle zu wissen
bekommen zu haben.«

»Es gehört auch wohl dazu«, meinte die Greisin, und dann liefen wir
doch ein wenig, um das altersschwarze Ziegeldach unseres verlorenen
Erbes zwischen uns und den feuchten Segen vom Himmel zu bringen.~–

Gegen sechs Uhr hörte es auf mit diesem Segen, und die Abendsonne
kam herrlich hervor. Es war zwar ein wenig naß auf den Wegen um das
Dorf, aber die Chaussee nach der Stadt binnen kurzem wieder
vollkommen trocken. Dorthin richteten wir unsern Abendspaziergang,
allen Lustwandlern, die aus der Stadt kamen. entgegen. Es begegnete
uns der Architekt, diesmal in Begleitung einiger der vermöglichen
Herren, die das neue, »lukrativere, zeitgemäßere« Unternehmen an
Stelle von meines Vaters Haus aufrichten wollten.
Selbstverständlich standen wir einige Augenblicke zusammen, die
gebräuchlichen Höflichkeiten auszutauschen.

»Es tut uns wirklich sehr leid, die Frau Doktor nunmehr aus ihrer
hiesigen, hoffentlich recht heitern Dorfgeschichte mit feurigem
Schwert vertreiben zu müssen«, sagte freundlich einer der Herren.
»Aber da wir vor Herbstes Ende das Etablissement jedenfalls bis
unter Dach in die Höhe zu bringen haben, so läßt sich die Sache
leider nicht anders einrichten, gnädige Frau.«

»O wir sind ganz bereit, Ihnen den Platz auch ohne Ihr feuriges
Schwert, Herr Stadtrat, zu räumen!« rief meine gnädige Frau
fröhlich. »Schon heute habe ich alle unsere Siebensachen so
ziemlich gepackt, und es war wirklich sehr hübsch und behaglich,
und ich sage Ihnen, und auch sicherlich im Namen meines Mannes,
unsern besten Dank für diese angenehmen Wochen. Und so ruhig!\ldots{}
und so gesund!\ldots{} Ich bin ganz gewiß dieses Jahr viel lieber in
Ihrer Mühle als in Thüringen, im Harz oder in der Ramsau gewesen.
Das Wetter war ja auch meistens ganz prächtig, und, Herr
Baumeister, wenn Sie wieder einmal nach Berlin kommen, müssen Sie
jetzt auch uns jedenfalls in unserm dortigen Heimwesen aufsuchen.«

»Werde gewiß nicht verfehlen, gnädige Frau«, sagte der Herr
Baumeister.

Sie wanderten weiter nach \emph{ihrer} Mühle, wir gingen in die
Stadt, um einige Einkäufe zu machen. Auf dem Heimwege begegneten
wir einander nochmals in der Dämmerung, grüßten uns jedoch bloß,
ohne uns nochmals miteinander aufzuhalten. Emmy meinte:

»Es sind doch recht nette Leute, und es freut mich, daß ich nun in
Berlin doch wissen werde, wer eigentlich hier sitzt und deiner oder
unserer lieben, kuriosen Mühle ein Ende gemacht hat.«

»Mich auch!« seufzte ich. –

Unter den Bäumen im Garten war's an diesem Abend natürlich zu
feucht für uns. Die Mühlstube war schon vollgepfropft mit
Handwerksgerät; in der Gaststube hatte, wie berichtet, der
Architekt seine Pläne ausgebreitet liegen, und – ich kann nicht
sagen, daß ich nicht gewußt hätte, wie es zuging, daß es sich grade
jetzt mit schärfster Deutlichkeit in die Erinnerung drängte, wie
Doktor Felix Lippoldes da gelegen hatte; – es war das beste, daß
wir uns wieder an unser Stübchen im Oberstock hielten und nur die
laue Luft und, wieder einmal, das Wetterleuchten von ferne zu uns
ließen durch die weit offenen Fenster.

Ich hielt meine alte, melancholische Pflegerin in diesen unsern
letzten Tagen und Nächten in Pfisters Mühle so viel als möglich in
meiner Nähe. Sie saß also auch jetzt am Tisch mit ihrem Strickzeug.
Ich und mein Weibchen lagen wieder Seite an Seite im Fenster und
atmeten den wohligen Duft der Nacht ein.

Es war, als rauschte der kleine Fluß munterer denn je, und auch
Emmy fand das und stieß mich an und sagte: »Hör nur, wie lebhaft
dein Bach diesen Abend ist! Es muß im Gebirge wohl noch stärker als
hier im flachen Lande gegossen haben.«

»Das müßte dort gestern oder vorige Nacht gewesen sein«, meinte
Christine. »So lange dauert es wohl an, ehe so ein Wolkenbruch aus
den Bergen bei Pfisters Mühle anlangt.«

»Die Zeitung heute abend weiß schon davon«, sagte ich.

»Ja die Zeitung, die Zeitung«, murmelte die Alte am Tische. »Was
wissen die Zeitungen alles! Wie schnell oder wie viel zu spät
wissen sie alles und schreiben über alles, was sie wissen und nicht
wissen. Erinnerst du dich wohl noch, Ebert, wie sie damals nach
geschehenem Unglück über den armen Papa von Frau Albertine redeten?
Dein seliger Vater las es uns vor, und uns allen standen die Tränen
in den Augen, die blutigen Reuetränen, daß wir ihn in der Welt so
wenig ästimiert hatten, da er es doch so sehr verdiente. Selber ich
in meiner armen, dummen Seele mußte mit in Wehmut in das Gefühl
einstimmen, daß wir alle so sehr zu der schlechten, unverständigen,
undankbaren Welt gehörten, die keinen großmächtigen, berühmten
Menschen zu taxieren wüßte.«

»Was sagten denn diese dummen Zeitungen, Christine?« fragte Emmy,
lächelnd sich umwendend.

»Nun, im Grunde wuschen sie nachträglich sich nur selber die Hände
in Unschuld und schoben alles auf uns, die schlechte, unvernünftige
Welt, daß er bei Pfisters Mühle aus dem Wasser gezogen worden
sei.«

»Barmherziger Himmel – Ebert?!« stammelte die arme Kleine. »Aus
\emph{unserm} hübschen Bache da? \emph{Hier} aus dem Wasser? Oh das
mußt du mir auf der Stelle ganz genau erzählen. Das ist ja zu
schrecklich interessant! Mein Gott, dann hat er aber auch wohl hier
in eurer Mühle auf dem Stroh gelegen? Ich habe bei Berlin auch
einmal ein junges Ding von Mädchen auf dem Stroh liegen sehen. Ich
hatte den Papa endlich auch einmal von seinem Kirchhofe
weggekriegt, und wir hatten eine Pfingsttour nach Pichelswerder
gemacht, und ich vergesse das in meinem ganzen Leben nicht!«

Ich hatte doch wohl die Nerven der Großstädterin und der lieben
Weiberchen überhaupt ein wenig zu sehr unterschätzt, da ich ihr wie
alle andern den unheimlichen Spuk von Pfisters Mühle verheimlichte.
Nun durfte ich schon mit ziemlichem Gleichmut sagen: »Es hängt mit
dem übrigen zusammen, Liebste, – ganz genau mit der Geschichte von
Adam Asche und Albertine, und da Christine und du einmal daran
gerührt habt, so kann ich die Tragödie Felix Lippoldes? wohl auch
zu Ende erzählen, ohne dich zum Gruseln zu bringen in den letzten
Nächten auf meines Vaters Erbe.«

»Na, na, Närrchen! Bist du nicht bei mir? Etwas andres wäre es
wohl, wenn ich hier ganz allein säße mit deinen Gespenstern. Und
dann, erinnere dich nur, Papa hat mich doch lange genug auf seinem
lächerlichen Kirchhofe spazieren geführt, als daß ich nicht mit den
Geistern auf dem besten Fuße und du und du stehen sollte. Und noch
dazu als geborene vernünftige Berlinerin!«

Sie nahm meine Hand von der Fensterbank auf, hob sie zu ihrem Munde
und ließ ihren lieblichen, warmen, lebendigen Atem drüberwehen und
lächelte:

»Erzähle nur dreist zu. Grade weil es unsere letzten Stunden hier
bei euch sind, paßt es um so besser drein, Und erzähle im einzelnen
– halte mich nicht für zu dumm in euern Wissenschafts- und
Literaturgeschichten; im großen ganzen wußte ich ja auch schon ohne
dich und die Christine davon. Papa las ja auch die Zeitungen und
manchmal ein Stück laut, und ich gab darauf hin und wieder acht,
wenn ich damals auch nur ein albernes Schulkind war und an andere
Dinge zu denken hatte. Nur daß es grade eure Mühle war, die durch
Frau Albertinens armen Papa so romantisch und interessant werden
sollte, wußte ich nicht.«~–~–

Ich weiß nicht, ob die Geschichte vom armen Felix Lippoldes so
romantisch gewesen ist wie die des jungen Mädchens bei
Pichelswerder; jedenfalls erzählte ich sehr gelassen weiter, und
auch mir selber rede ich hier auf diesen Blättern noch einmal
davon.~–

Ich hatte in Berlin die ersten Semester meiner Studienzeit
zugebracht, und ich war auf andern Universitäten Studierens halber
gewesen. Nun saß ich wiederum ernstlicher über den Büchern in
Berlin und verkehrte wieder mit meinem frühern Mentor A.~A.~Asche.
Und wie früher verschwand er auch jetzt dann und wann aus der Mitte
seines energischen Tun und Treibens, wenn auch auf kürzere Zeit.
Aber er verschwand nicht mehr in die weite Welt, sondern ich wußte
stets genau, wohin er ging, nämlich nach Pfisters Mühle.

Ich habe es nachher mit tiefer Rührung sehr eingehend erfahren, wie
die beiden, der Vater und der Freund, nicht nur ihre klugen Köpfe,
sondern auch ihre braven Herzen zusammengelegt haben, und zwar
nicht bloß zum Besten des großen Prozesses Pfisters Mühle contra
Krickerode. Letztern betrieb Doktor Riechei von Instanz zu Instanz
mit wechselndem Erfolg, und es ging wieder einmal gegen
Weihnachten, als wir vor der letzten standen und ihn gewannen, ohne
daß das Abendrot über Pfisters vordem so fröhlicher Mühle dadurch
eine Stunde länger am Himmel hätte festgehalten werden können.

Es war ein Nachmittag, wie ich schon einmal beschrieben habe in
diesem Sommerferienheft: Schnee in der Luft, Wind in den Gassen,
die Gedanken in der Ferne und mancherlei unbestimmt Bangen und
allerlei übler Geruch nahebei und umher. Wie damals meine
Schuljahre, so lag jetzt meine Studentenzeit so ziemlich hinter
mir. Am Fenster saß ich wieder, wenn auch nicht das Kinn auf beide
Fäuste stützend und an den Schulrat Pottgießer in Verbindung mit
all den vergangenen lustigen Christbäumen von Pfisters Mühle
denkend. Aber an Pfisters Mühle, Vater Pfister und seine fröhlichen
Weihnachtstannen dachte ich, und – wieder – wie damals – kam ein
Schritt die Treppe herauf, und jemand klopfte an meine Tür – und
beinahe hätte ich im Zwischenlichtshalbtraum wieder gerufen:

»Alle Wetter, das ist ja der Alte! Was will denn der Alte heute
noch und so spät am Tage in der Stadt?«

»Ich bin's, mein Junge«, sagte Doktor A. A.~Asche, und er legte mir
seine Hand fast so schwer auf die Schulter wie damals mein
verdrußgequälter, sorgen- und kummervoller Vater. »Eberhard
Pfister, du bist ein belesener junger Mensch, Philologe noch dazu,
– erinnerst du dich vielleicht eines der kleinern Meisterwerke
erzählender deutscher Dichtung, welches beginnt: Ein Knabe aß, wie
viele Knaben, die Datteln für sein Leben gern~–«

\begin{verse}
»Und um der Datteln viel zu haben,\\
Pflanzt er sich einen Dattelkern«,
\end{verse}
\noindent
stammelte ich.

»Ganz richtig, Telemachos, oder so ungefähr. Nun denn, jener Knabe
war ich, aber wenn auch nicht ethisch aufgepusteter, so doch um ein
erkleckliches schlauer, als mir der Fabulist in seinen Reimen
nachsagte.«

»Du redest wahrlich in Rätseln, Adam.«

»Keineswegs für den nur mit einigem Weltverständnis Begabten. Wer
nicht seiner Palmen Keime in ein Mistbeet pflanzt, wird sehr selten
Datteln davon in seine eigene Tasche, für sein eigen Maul
herunterholen. Non olet, wie der römische Allezeitmehrer sagte. Ich
werde es durchsetzen, und wie Mr. François Marie Arouet, genannt de
Voltaire, werde ich Geld machen, um meine Meinung und jedem Lumpen
das, was er wert ist, sagen zu können. Im nächsten Frühjahr legen
wir den Grundstein zu A.~A.~Asches eigenem
Erdenlappenlumpenundfetzenreinigungsinstitut am Ufer der grauen
Spree. Du reisest morgen nach Hause, und ich fahre mit dir und
feiere noch einmal, mit gewaschenen Händen, mit euch Weihnachten in
Pfisters Mühle.«

Ich tat einen jauchzenden Schrei:

»Asche, das ist ja wundervoll!«

»Durchaus nicht«, seufzte der Freund und Exmentor. »Mir ist
ziemlich öde und katzenjämmerlich zumute.«~–~–

Man kann nicht immer auf den Ellenbogen in der Fensterbank liegen,
wenn die Nacht draußen auch noch so schön und duftig ist. So traten
wir in den Lichtkreis von Christinens kleiner Lampe zurück; aber
wir saßen nicht nieder am Tisch, wir saßen auf unsern Reisekoffern
einander gegenüber und verplauderten so den Rest des Abends.

\section{Neunzehntes Blatt}

\zusatz{Felix Lippoldes? erste durchschlagende Tragödie}
»Höre mal, Ebert«, meinte Emmy, »es ist ein wahres Glück, daß ich
meinen Freund, den Doktor Asche, so sehr genau kenne. Im Grunde
hast du doch während unseres hiesigen Aufenthalts dein
allermöglichstes getan, ihn mir recht zuwider zu machen mit seinen
ewigen, gräßlichen Redensarten und alledem, was ihr Männer unter
euch und auch nur viel zuviel gegen uns arme, weiche Seelen eure
Philosophien zu nennen pflegt. Na, an einer guten Vorschule hat es
mir freilich gottlob ja auch nicht gefehlt: Papa in Berlin ist in
dieser Hinsicht das Seinige vollkommen wert.«

»Kind, wir leben eben in einer Welt, in der ein jeglicher bei
weitem mehr auf die Schwächen, Untugenden und Laster des andern
angewiesen ist als auf seine Tugenden. Und bedenke, was konnte es
für einen fahrigen, unerfahrenen jungen Menschen, der demnächst aus
innigstem Herzensgrunde die intimste Bekanntschaft deines Papas zu
machen wünschen sollte, außerdem Wünschenswertes geben, als einen
Patron zur Seite zu haben, der ihn so eines andern lieben Mädchens
wegen (denn darauf lief es doch hinaus) zu der letzten
Weihnachtsfeier in Pfisters Mühle abholte?«

»Da magst du recht haben«, sagte Frau Emmy Pfister nach einem
längern Nachdenken, und ich – fahre fort, wie ich angefangen habe
und wie mich diese guten Sommertage so zwischen Traum und Wachen,
zwischen Gegenwart und Vergangenheit gleich leise schaukelnden
Wellen getragen haben bis an das Ende meiner Schulferien und den
Beschluß der Geschichte von Pfisters Mühle – und so gehe ich noch
einmal unsern kleinen Fluß aufwärts den Weg nach Krickerode, und
zwar mit meinem frühern Lehrmeister und jetzigen Freunde
A.~A.~Asche.~–

Meinen Vater fanden wir kränkelnd, kümmerlich, apathisch trotz
Riechei und Riecheis vollständigem Siege in Sachen Vater Pfister
contra Krickerode. Vielleicht auch grade darum. Es ist schon recht
viel auf der Erde, wenn der Mensch für einen zu spät kommenden
Triumph noch ein sauersüßes Lächeln übrigbehalten hat.

»Jawohl, wie es beliebt, wenn es dir Vergnügen macht, ziehe wieder
in den Oberstock, Adam«, sagte mein Vater, mit einem Male seinen
Schützling wieder mit dem vertraulichen Du aus den Kinderjahren
desselben beehrend. »Aber mit der Weihnachtsfeier wird es wohl
wenig werden. Wenn der Mensch seinen Knick und Knacks weg hat, soll
er keine Vergnügungskomödie spielen, wenn er's nicht absolut nötig
hat.«

So wohnten wir, der angehende Kapitalist und der Student der
Schulweisheit dieser Erde, noch einmal beim ersten Schneefall in
Pfisters Mühle, jeder in seiner Weise an den Bildern dieser Welt
weiter malend. Was Adam Asche anbetraf, so erklärte er sich selber
für den größten Pinsel des Universums, und zwar in seinem
Verhältnis zu der armen Albertine Lippoldes und ohne im geringsten
damit renommieren zu wollen.

»Sie will mir keine Last sein, gibt sie als offiziellen Grund an,
indem sie mir den Stuhl vor die Tür setzt!« murrte er grimmig. »Ist
es nicht zu dumm?\ldots{} Mir eine Last?\ldots{} Mehr Ballast, Kind, oder
Fräulein, oder Gänschen, oder gnädiges Fräulein, wenn die Brigg
nicht beim ersten Umsegeln von Landsend kentern soll! – Hilft alles
nichts! Nichts bockbeiniger als Lottchen, Laura oder Beatrice, oder
wie sie sonst heißen, die lieben Seelen, diese kleinen, braven
Feminina, wenn sie das Bedürfnis fühlen, in weiße Schleier
drapiert, über unsereinem im Blau dahinzusegeln, wenn sie, um in
ihre guten, dummen Herzen hineinzuweinen, ihren Kopf aufsetzen zu
müssen glauben!\ldots{} Da stehe ich nun mit meiner innigsten
Überzeugung, auch einen Schwiegervater zu einer Frau und Familie
ernähren zu können. Du hast mich in der Schlehengasse waschen sehen
– ich bitte dich um alles in der Welt, du Tropf, sieh mich nicht so
sekundanerhaft an! – Du hast mich bei Schmurky \& Kompanie am Werk
gefunden, und da sitze ich nun von neuem in Pfisters Mühle,
abermals abgeblitzt, und würde ein Königreich mit Vergnügen geben
für die Gefühle von Adalbert von Chamissos alter Waschfrau. Ich
versichere dir, Bursche: ohne dieses Mädchen wird mir das Resultat
meines Lebens so stinkend, so widerwärtig, so über alle Maßen
abgeschmackt sein, daß mir nichts übrig bliebe, als eines schönen
Morgens mich mittellos wie Papa Lippoldes und seelenlos wie seine
sämtlichen tragischen Helden im fünften Akt in Monaco an einem Öl-
oder Lorbeerbaum hängend oder an der Riviera mit ›nichts im Herzen
als einer Kugel‹ finden zu lassen. Sie muß, sie muß! Und nun frage
ich dich um Gottes Willen, weshalb sollte sie nicht müssen? Habe
ich es denn besser als sie in dieser infamen Lappen-, Lumpen- und
Fetzenwirtschaft der Mutter Erde? Bei dem reinen Äther über dem
rauchverstänkerten Dunstkreis über Pfisters Mühle und Umgegend von
Pol zu Pol, ich liebe dieses Frauenzimmer und will es bei mir haben
und es so gut als möglich halten in dieser Welt des Benzins und der
vergifteten Brunnen, Forellenbäche und schiffbaren Flüsse. Und die
Närrin fürchtet sich bloß, mir das Ideal meiner Jugend, das Pathos,
die Tränen und das Herzklopfen meiner Knabennächte, ihren Papa zur
Aussteuer mit in den Haushalt aus der Schlehengasse und dem Ödfelde
zu bringen! 's ist, um das Herze durchzuprügeln, da es sich nicht
abküssen lassen will! Komm mit an deines Vaters Bach, Ebert; man
spürt immer die Neigung, draußen Atem zu holen, wenn man innerhalb
von vier Wänden dem, was man sein Herz nennt, Luft gemacht
hat.«~–~–~–

Nun hatte ich Emmy von dem schlimmsten Tage, den Pfisters Mühle,
wenigstens bei Menschengedenken, erlebt hatte, zu berichten, und
zwar auf Wunsch der teilnahmvollen Schönen »so genau und so ins
einzelnste wie nur möglich« – Es hatte Mühe gekostet, unsere etwas
zu vollen Koffer zu schließen, und nun saßen wir ein wenig
erschöpft auf ihnen einander gegenüber und plauderten weiter über
vergangene Bilder und Tage, und Jungfer Christine Voigt gab auch
ihr kunst- und lebensverständiges Wort darein in der lauen
Sommernacht. In meiner Seele und im Rauch meiner Zigarre war es
wieder der Tag Adam und Eva, der Tag vor dem Heiligen Christ, und
ich stand wieder im dichten Nebel an dem Mühlwasser meines Vaters
und wieder mit Adam Asche.

Es war zwischen drei und vier Uhr nachmittags; die Abenddämmerung
kroch schon leise heran; zu unserer Linken ragte das Dach, unter
dem Albertine ihre Tage kümmerlich verlebte, über das kahle
Buschwerk, und Asche sagte:

»Hindern kann sie uns wohl nicht, ihrem Vater einen Besuch zu
machen. Sie wird dies zwar von meiner Seite taktlos finden; aber
bin ich in die Welt gekommen, um feine Gefühle oder mit Feingefühl
zu poussieren? Ich, der Ismaelit – unter den Büschen ausgehungert?
Der wirkliche geflickte Lumpenkönig mit diesen Pfoten des
Kehrichtfegers? Ich, dem man sein stänkrig Handwerk auf eine Stunde
Weges anriecht? Komm mit, Knabe, es ist mir jedenfalls lieb, daß
ich dich vorangehen lassen kann. Es ist lächerlich, aber ich habe
eine schändliche Angst vor jedem Nasenrümpfen des lieben, nobeln
Herzensmädels!«

Der Nebel war wieder so dicht wie an jenem zweiten Weihnachtstage,
wo wir ausgingen, um Krickerode in ihm zu suchen; und zwanzig
Schritte weiter flußaufwärts blieb der Freund von neuem stehen und
brummte: »Was war denn das eben? Dieser Qualm liegt einem nicht
bloß vor dem Auge, sondern auch im Ohr. Kam das aus der Luft, vom
Lande oder aus dem Wasser?\ldots{} Du hast es doch auch gehört?«

»Gewiß. Es war ein kurioser Laut und schien mir von dort her aus
der Richtung der Gärten und Anbauerhäuser zu kommen.«

»Mir nicht!« murmelte Asche, mich hastig weiter aufwärts am Bach
durch das Ufergebüsch mit sich ziehend; – das Bett von Vater
Pfisters Mühlwasser war wie gewöhnlich um diese Jahreszeit bis zum
Rande voll, und die trübe Flut stand an manchen Stellen bis in den
engen Fußpfad hinein.

Noch einmal hielten wir an und horchten –

»Dummes Zeug!« meinte Asche, und einige Augenblicke später klopften
wir an Doktor Felix Lippoldes? Tür in seinem letzten, kläglichen
Aufenthaltsort unter den Lebendigen auf dieser Erde.~–

Fräulein Albertine erhob sich von ihrem Stuhl am Fenster, und wenn
mein Exmentor sich vor der jungen Dame so sehr fürchtete, so
geschah doch augenblicklich nicht das geringste, was ihm fernerhin
Gründe dazu hätte geben können.

Ruhig reichte das Fräulein \emph{uns beiden} ihre Hand:

»Sie sind dem Vater nicht begegnet, Herr Doktor? Er hatte die
Absicht, Sie in der Mühle aufzusuchen, Herr Pfister – wollen die
Herren sich nicht ein wenig setzen?«

Sie wies uns an die zwei schlechten Bauerschemel mit der
Handbewegung einer königlichen Prinzessin, die sie auch war. So
unbefangen, wie nur die vornehmste Dame unter den bänglichsten
gesellschaftlichen Umständen sein kann, nahm sie selber wieder
Platz. Ihre schöne, mutige Seelenkraft trat in der ärmlichsten,
kahlsten, trostlosesten Umgebung nur um so glorreicher hervor, und
sogar lächelnd wiederholte sie ihre Handbewegung.

Aber Adam Asche, der vor Minuten noch alles, was er in der Welt
bedeutete, für einen dieser Stühle hingegeben haben würde, zögerte
jetzt in sonderbarer Unruhe, Besitz zu nehmen.

Er fingerte nervös an der Lehne des seinigen.

»Nach Pfisters Mühle?\ldots{} Dann müßte er uns doch begegnet sein!\ldots{}
Sollte er nicht wieder einmal den Weg nach Krickerode gegangen
sein, Fräulein~A – gnädiges Fräulein?«\ldots{}

Nun war es eine Tatsache, daß der arme Tragödiendichter seit
längerer Zeit mit Krickerode auf dem vertrautesten Fuße lebte.
Unter dem jüngern Beamtenpersonal der großen Fabrik, den Kommis,
Buchhaltern und Technikern, hatte er Freunde gefunden, die, wenn
nicht zu seinem Wohlergehen, so doch zu seinem Wohlbehagen, wie er
das jetzt leider verstand, ein Erkleckliches beizutragen
vermochten. Mit einer gewissen respektvollen Scheu noch machten
sich die Herren über ihn lustig; denn noch immer kamen Momente, in
denen er die jungen Leute durch sein Pathos, seinen grimmigen Witz
und Sarkasmus und vor allem durch sein Talent, seine Dichtungen
selber vorzutragen, in Enthusiasmus und auch Rührung versetzen
konnte. Und da die Herren fast sämtlich Lebemänner im kleinen Stil
waren, so fand er auch immer in ihrer Gesellschaft das, was er
jetzt allem übrigen vorzog, trotz ästhetischer Leidenschaft,
Erhabenheit, Empfindung und hoher Ironie, nämlich eine Flasche mit
feinem Rum oder dergleichen. Es war auch in dieser Hinsicht nicht
gut, daß Krickerode sich so nahe bei Pfisters Mühle angesiedelt
hatte, und schon der Name des gewinnbringenden Institutes aus
Asches Munde wirkte beängstigend auf die Tochter von Felix
Lippoldes.

Selbst zu einem gleichgültigen Gespräch über das Wetter und das
nahe Fest, wie es sich der Freund vorgestellt haben mochte, kam es
nun nicht mehr mit der jungen Dame. Adam setzte sich endlich wohl,
aber er rückte unruhig auf dem Stuhle hin und her, und bald sagte
er, hastig von neuem aufspringend:

»Es liegt mir doch daran, den Papa heute noch zu sprechen,
Fräulein. Seien Sie unbesorgt – nur eine
Feuilletonredaktionsangelegenheit, eine Zeitungsverlegersache,
Fräulein Albertine. Die Leute machen Reklame für A.~A.~Asche \&
Kompanie, und kurz – was meinst du, Ebert, wenn wir dem Doktor ein
wenig nach Krickerode entgegenliefen?«

»O tun Sie es, meine Herren!« rief Albertine mit gefalteten Händen
und mit einem Dankesblick auf meinen Exmentor, für den sie nicht
verantwortlich war, weil sie nichts dafür konnte, der aber wie ein
Blitz aus dem Reiche alles Lichtes auf die Firma A.~A.~Asche \&
Kompanie fallen mußte.

»So gehen wir, Knabe!« rief der »eminente« Gewerbschemiker mit
merkwürdig erstickter Stimme und sich nach der Gurgel greifend, wie
um dem Organ auch von außen zu Hülfe zu kommen. Vor der Haustür sah
er sich scheu nach dem Fenster des Fräuleins um, und als wir so
weit von dem Hause im Garten entfernt standen, daß der Nebel uns
jedem möglichen Nachblicken entzog, packte er mich an der Schulter,
schüttelte mich und rief:

»Mensch, hast du jemals etwas an oder in mir bemerkt, was auf das
hindeutete, so man zweites Gesicht, Ahnungen nennt, oder wie die
Altweiberhirngespinste sonst heißen mögen?«

»Nicht, daß ich wüßte!«

»Nun, so nenne du mich jetzo, wie du willst; aber seit einer
Viertelstunde fühle ich mich auch diesem Menschlichen nicht mehr
fremd. Ebert, es wäre nicht unfolgerichtig, aber doch greulich,
wenn da eben eine menschliche Tragikomödie in einer Weise zum
Abschluß gelangt sein sollte, die freilich diesmal sensationell
genug wäre, um das Publikum für längere Zeit mit Felix Lippoldes zu
beschäftigen!«

»Ich begreife dich nicht –«

»Etwa ich mich?\ldots{} Es ist ja wohl auch nur eine verrückte
Einbildung von mir, der nichtsnutzige Nebel wird mir auf den Nerven
liegen, aber eine Wohltat würde es unbedingt sein, wenn ich jemand
persönlich für diesen neuen Zug in meiner Seele verantwortlich
machen könnte. Nun, die Genugtuung, mich selber in fünf Minuten zu
maulschellieren, bleibt mir wenigstens; aber es hilft in diesem
Moment nichts, komm also rasch mit an den Fluß, euern verteufelten
Provinzialstyx. Zum Henker, ich würde viel drum geben, wenn wir
auch diesmal Samse wieder zur Begleitung hätten.

»Aber –«

»Der Ruf von vorhin klingt mir jetzt von Sekunde zu Sekunde mehr
wie seine Stimme auf dem Trommelfell nach.«

»Samses Stimme?«

»Ärgere mich nicht!« schrie der wunderliche Mann grimmig. »Felix
Lippoldes? Gekräh, ohne Pathos, aber in wirklicher dramatischer
Not. Beim Zeus, ich bin ein Narr, ein Esel, meine selige Tante
Kassandra, aber ich wollte, wir begegneten der Unglückskreatur bald
– einerlei, in welchem Zustande.«

»Asche?«

»Ja, Asche, Asche! Komm jetzt mit hinaus, gen Krickerode zu und
möglichst rasch und so dicht als möglich am Wasser. Ich traue jetzt
diesem Pfisterschen Familien-Phlegethon durchaus nicht. Ich habe
mich wohl vordem ein wenig zu unbefangen, familiär gegen seine
heimtückischen Nymphen und Nixen benommen – bis an den Hals steigt
mir die unheimliche Brühe. Vorwärts!

Wir drangen nun durch das Buschwerk, dann und wann in den in den
Weg getretenen Sümpfen steckenbleibend, einer den andern in seiner
Aufregung steigernd. Und plötzlich hatte ich einen Schreckenslaut
auszustoßen. Unter einer steil abfallenden Böschung, an der das
Wasser wie in einem Miniatur-Hafen sich lautlos im Kreise drehte,
wurde in diesen winzigen Wirbeln ein mir seit Jahren bekannter,
zerdrückter, abgetragener, weitkrempiger Filzhut mit herumgezogen.
Und ein Arbeiter aus Krickerode, der von der Fabrik her grade im
Nebel uns entgegenkam, gab uns dazu die Nachricht, daß der Herr
Doktor an diesem Nachmittag wohl in Krickerode und mit den Herren
sehr laut und lustig gewesen sei, daß er aber vor mehr als einer
Stunde schon Abschied genommen habe, und zwar nicht auf recht
gesunden Füßen: »Na, na, Sie werden schon wissen, was ich
meine\ldots{}«

»Es ist einfach entsetzlich«, sagte Emmy auf ihrem Koffer, die
Hände im Schoße zusammendrückend. »Und die Art und Weise, wie wir
uns das jetzt so hier an unserm vorletzten Tage, hier in deiner
Mühle erzählen, macht mich auch wirklich ganz nervös. Und du malst
das alles so deutlich, wie du da in Hemdsärmeln auf unserm Gepäck
sitzest, daß es dadurch fast noch schrecklicher wird. O Gott, wie
froh mußte die arme Albertine sein, als sie endlich auch so weit
war, wie wir heute, nämlich fertig zur Abreise aus Pfisters Mühle!
Sie hat doch, trotz aller Schönheit der Gegend und Lieblichkeit der
Natur rund umher fast zu viel hier erleben und ertragen müssen, und
es war sehr lieb vom Doktor Asche, daß er sie endlich doch daraus
wegnahm, und zwar – so bald als möglich!«

»Und, Kinder, nun nehmt doch einen Rat von der Alten an«, sagte
Christine, die Hände über ihrem Strickzeug faltend . »Laßt die
Sonne oder wenigstens den hellen Tag auf den Rest von der
Geschichte scheinen. Die junge Frau hat ganz recht: Herr Doktor
Asche hat seine Sache wohl recht schön gemacht; aber du bist nun
daran, deinem lieben Frauchen zu berichten, was dein seliger Vater
von dem Seinigen dazu getan hat, Ebert; und dazu solltest du die
Morgensonne abwarten – wir kriegen gewiß morgen das beste Wetter! –
und unsern letzten Tag in Pfisters Mühle dazu anwenden. Der Wächter
im Dorf hat schon längst gerufen, und es hat auch schon elf vom
Kirchturm schlagen, o Gott, o du mitleidiger Herrgott, und ich
werde nun nimmer und nimmermehr darauf zuhorchen können!«

Ich ließ den Hut des auf dem Wege von Krickerode her
verlorengegangenen genialen Dramatikers auf meines Vaters trübem
Mühlwasser im Kreise sich drehen, und – gottlob, mein junges,
weichherziges Weib sprang lebendigst empor, legte bestürzt,
zärtlich der Alten den Arm um den Nacken, küßte töchterlich sie auf
die gebeugte Stirn und trocknete ihr mit dem Taschentuch, immer
liebe, abgebrochene Trostworte flüsternd, die Tränen aus den Augen
und von den runzligen Backen.

\section{Zwanzigstes Blatt}

\zusatz{Alte, schöne Lieder von ferne; die letzte schöne, alte
Müllerin auf dem Haustürtritt}
Es ist in Wahrheit ein Sommerferienheft, zu dessen losen Blättern
ich jetzt die letzten zusammensuche, ehe ich es mit einem blauen
Umschlag versehe, zusammenrolle, von meiner jungen Hausehre ein
rotes Bändchen drumbinden lasse und es in die tiefsten Tiefen
meines Hausarchivs versenke. Wie ist das Gekritzel
zusammengekommen? Die Buchstaben, die Kleckse, die Gedankenstriche
und Ausrufungszeichen müssen selber ihr blaues Wunder in der
Dunkelheit ihrer Truhe unter meinem Schreibtisch in der großen
Stadt Berlin haben! Das wurde unterm Dach geschrieben, das unterm
Busch auf der Wiese; auf diese Seite fiel der helle, heiße
Julisonnenschein, hier ist die Schrift ineinander geflossen und
trägt, solange das Papier halten will, die Spuren, daß das Ding mit
Not aus einem plötzlichen Platzregenschauer in Emmys Handkörbchen
gerettet wurde. Gar glatt liegen die Bogen nicht aufeinander; der
Wind hat dann und wann allzu lustig damit gespielt; und – hier ist
eine Seite, auf der ich alles mitnehme, was mir von dem Erdboden
auf meines Vaters Erbe übriggeblieben ist. Der Wind trieb es vor
sich her durch Vater Pfisters Mühlgarten, und ich hatte ihm lange
genug um die Kastanienbäume nachzujagen, bis ich es unter der
letzten Bank am Wasser wiedererhaschte.

Wo bleiben alle die Bilder?

Wie ich die Sache im »Spiel der Gedanken« angefangen habe, so muß
ich sie nun beenden, und der bitterste Ernst wird sich auch auf
diesen letzten Blättern in die seltsame Form finden müssen, welche
ihm nur eine solche ungewöhnliche Sommerfrische geben konnte.

Die Morgensonne, auf welche uns Jungfer Christine hingewiesen
hatte, fiel lachend in unser Gemach, und wir hatten den letzten Tag
unseres Aufenthalts in Pfisters Mühle vor uns. Noch einmal
\emph{diese} Welt in voller Schöne!

Der nächste Morgen sah uns mit unseren kuriosen
Vagabunden-Haushalts-Habseligkeiten auf der Fahrt, zurück in den
Alltag, zu dem »eigenen Herd«, den lateinischen Exerzitien und
regelrechten deutschen Aufsätzen – kurz, allen normalen Stilübungen
und soliden Lebensbedingungen, und wie Emmy sich ganz richtig
ausdrückte, zu »unserm jetzigen eigentlichen Dasein auf dieser
Erde«. Es ging nicht, es ging nicht an, es war eine Unmöglichkeit,
diesen letzten Heimatsonnentag, wie ich es mir vorgenommen hatte,
ganz den vergangenen, verblichenen Bildern zu widmen! Blieb uns
doch auch noch der letzte Abend, wenn nichts dazwischen kam und
mich hinderte, die Geschichten vom Ausgang von Pfisters Mühle
meiner Frau zu Ende zu erzählen.

Es ging, solange diese letzte Sonne mir über meines Vaters Hause
stand, nicht an, von neuem mit Adam Asche nach dem Hut in der
trüben Schlammflut von Vater Pfisters Mühlwasser fischen zu gehen.
Emmy kannte ein Gehölz, wo »wundervoller Efeu« wuchs, und wir waren
schon im Tau dort, einen Busch mit Wurzeln für unsern Fenstergarten
in Berlin auszugraben.

»Laß es mit Albertines armem Papa, bis wir zum letztenmal wieder zu
Tisch hier nach Hause kommen«, meinte das Kind. »Dieser Morgen ist
noch einmal zu wonnig und die Geschichte zu traurig. O, und ich
hoffe, dies soll anwachsen, und dann ziehen wir die Ranken um
deinen dummen, langweiligen Schreibtisch und haben so immer etwas
Grünes aus deiner so lustigen und traurigen Heimat und von deines
Vaters Mühle um uns; und ich werde dabei ganz gewiß noch manch
liebes Mal an diese im ganzen doch so reizenden Wochen hier
denken.«

Wir kamen mit dem Busch nach Hause, das heißt diesmal noch nach
Pfisters Mühle heim und fanden den Garten voll Lärm und Gezänk und
den Architekten sehr erbost inmitten seiner Fuhrleute und
Bauführer. Wie war es da möglich, unter den Kastanien, selbst auf
der entlegensten Bank, zu einem stillen, letzten Worte über die
vergangenen Bilder des Ortes zu gelangen? Der Nachmittag wäre
vielleicht geeignet gewesen, doch den verschlief mein Weibchen,
ermüdet von dem frühen Ausflug in den Wald, vom Blumenpflücken und
Efeuausgraben, zum größten Teil.

So blieb uns nur der letzte Abend in Pfisters Mühle übrig, wenn
nicht wiederum etwas dazwischen gekommen wäre; nämlich gegen fünf
Uhr ein Billett vom Doktor Riechei, der sich darin, wie er sich
ausdrückte, uns zur Gesellschaft für die uns vielleicht sonst
ziemlich ungemütlichen, letzten Stunden auf Vater Pfisters
vielbedrängtem und seinerzeit glorreich in integrum restituiertem
Erbe anmeldete.

»Famos!« meinte der Baumeister. »Da bleibe ich auch! Und das beste
ist in diesem Falle, da hier doch wohl schon Schmalhans ein wenig
Küchenmeister ist, wir machen ein Picknick draus, Frau Doktor. Ich
jage einen Boten in die Stadt mit einer Notiz an unsern Advocatus
diaboli, einen anständigen Tropfen mit herauszubringen. Im übrigen
begnügen wir uns mit dem, was das Dorf liefert, und damit werden
sich die gnädige Frau und Jungfer Christine gern beschäftigen. So,
meine ich, kann Ihnen, lieber Eberhard, der Seiger allhier die
letzten Sandkörner noch am behaglichsten ausrinnen lassen. Morgen,
wenn Sie und Frau Gemahlin uns verlassen haben, werde ich die Uhr
sofort umkehren, und der Sand mag von neuem laufen; – und aber nach
fünfhundert Jahren will ich desselbigen Weges fahren. So sagt ja
wohl der selige Rückert?«

»So sagt er!« sagte ich. –

Wie hatte ich mich im tiefsten Grunde meines Herzens vor diesem
allerletzten Abend unter dem Dache meines Vaters und meiner Väter
gefürchtet! Und nun war er da und ging vorüber in der trivialsten
Weise, bei der angenehmsten, aber auch allergewöhnlichsten
Unterhaltung. Die beiden Herren, meine sehr guten Freunde, taten
das Ihrige, daß das kuriose Abschiedspicknick so vergnüglich als
möglich ausfiel. Sonst begnügten sie sich gern mit dem, was wir zu
geben hatten, und waren vor allen Dingen noch mal gesprächig heiter
in der Gewißheit, daß ich damals doch ein recht gutes Geschäft bei
dem Verkauf von meines Vaters Anwesen gemacht hätte und daß ich,
eins ins andere genommen, heute im innersten Gemüte herzlich froh
sei, es von der Seele und aus der Hand los zu sein. Die
Bereitwilligkeit des »Konsortiums«, mir und meiner Frau noch einmal
einige Wochen einer vergnügten Villeggiatura in Pfisters Mühle zu
gestatten, wurde denn auch von mir von neuem gebührend anerkannt
und von Emmy auch sehr gewürdigt. Dann redeten wir Bismarck,
Kulturkampf, soziale Frage und was sonst so dazu gehört, um einen
Abschiedsabend unter guten Freunden hinzubringen, ohne zu sehr zu
merken, wie die Zeit läuft.

Ich tat wahrlich nichts dazu, die Unterhaltung wieder auf Pfisters
Mühle zu bringen. Die alten Baumkronen über unserm vergnügten,
letzten Gartentisch waren auch ganz still. Viel Sterne flimmerten
am dunkeln Himmel. Nicht der leiseste Lufthauch bewegte die Flamme
unter der Glaskuppel unserer aus dem Dorfe entliehenen Lampe. Ich
hörte in die Unterhaltung hinein wie in das Rauschen des Flusses,
der immer noch von Krickerode herkam, aber nächste Woche schon zum
letzten Male an Pfisters Mühle vorbeirauschen sollte.

»Das sind die Teutonen drüben in der neuen Schenke jenseits des
Dorfes«, sagte Riechei. »Wie oft haben wir das hier unter diesen
Bäumen – auch an diesem Tische – bei deinem Vater – dem guten,
alten Vater Pfister – gesungen, Ebert~–

\begin{verse}
›Und dem Wandersmann erscheinen\\
Auf den altbemoosten Steinen\\
Oft Gestalten zart und mild!‹«
\end{verse}

»Gaudeamus igitur«, summte der Architekt. »Krambambuli, das ist der
Titel~–

\begin{verse}
Die Mühlen können nichts erwerben,\\
Sobald das Wasser sie nicht treibt~–«
\end{verse}

Ich aber hielt es bei dem fernen Singen der alten Couleur und bei
dem nahen Potpourri des Baumeisters nicht länger aus in der
Gemütlichkeit der Stunde. Ich schlich vom Tische dem Hause zu, wo
auf dem Türtritt der alten Mühle, die das Wasser nicht mehr trieb,
noch jemand kauerte und den letzten Abend auf Vater Pfisters
Anwesen zu überwinden suchte.

Wenn sie nichts mehr im Hause zu schaffen und sorgen hatte und die
Gartenbewirtung ihr ebenfalls freie Hand ließ, pflegte an schönen
Abenden Christine Voigt immer da zu sitzen und die müden Hände in
die Schürze zu wickeln. Und ich saß jetzt nieder zu ihr, wieder wie
sonst als Kind und als Knabe, als das Lied von der Saale hellem
Strande und das Gaudeamus noch unter \emph{unsern} Kastanien im
vollen Chor erklang und ich mit klopfendem Herzen horchte.

Nun hatte ich die alte, blaue Schürze der alten Pflegerin von den
Augen zu ziehen:

»Mutter, wir bleiben ja zusammen!\ldots{} Ich wollte mein Herzblut darum
geben, wenn ich's hätte ändern können! Aber selbst der Vater sah
es, daß es nicht anders ging, und es war so sein Wille, wie es
gekommen ist heute! Er wußte es ja auch, daß wir noch übrigblieben
und beieinander – auch in fremdem Lande, wo es auch sei!«

»Wohl bis zu Ende, wenn du mich mitnehmen willst, Ebert; aber, o
Gott, wenn ich nicht gedächte, daß deine liebe Frau und du mich
doch noch wenigstens als Aushülfe gebrauchen könntet, ließe ich
mich am liebsten hier vergraben. Der Kirchhof, wo dein Vater und
deine Mutter liegen, wäre mir nicht lieber.«

»Natürlich, hier sitzt er wieder bei seiner Alten, Frau Doktor!«
rief Riechei, von dem Tisch am Wasser mit den Händen in den
Hosentaschen auf uns zuschreitend, vergnüglich über die Schulter
zurück. »Wenn ich an Ihrer Stelle wäre, Frau Pfister, würde ich
doch allgemach ein wenig eifersüchtig. Na, wo steckst du denn,
Pfister? Man vermißt dich ungewöhnlich lange mit deinem
Pfropfenzieher. Den solltest du zum Angedenken an diese
urgemütlichen Abschiedsstunden doch von deinem Reisegepäck zurück-
und mit dem Grundstein von Neu-Pfisteria verscharren lassen. Ich
werde dann jedenfalls eine vidimierte Abschrift des
Schlußerkenntnisses in Sachen Vater Pfister contra Krickerode
beilegen und der Baumeister dort seine Visitenkarte.«

»Geh nur hin, geh nur wieder zu deiner kleinen, guten Frau, Ebert«,
flüsterte mir meine Pflegemutter zu. »Ja, der Meister, dein seliger
Vater, hatte ganz recht, als er einsah, daß es nicht anders ging.
Die Herren haben auch ganz recht, daß sie sich nicht mehr, als
nötig ist, aus dem letzten Abend von Pfisters Mühle machen.«

Ich nahm ziemlich fest den lustig dargebotenen Arm des
wohlberufenen Advokaten und rechtsgelehrten Beistandes und Siegers
in unserem Prozeß gegen Krickerode~–

\begin{verse}
»Schön Müllerin schließt's Fenster zu,\\
Und alles liegt in tiefer Ruh,\\
Des Morgens Nebel haben\\
Die Mühle ganz begraben«;~–~–~–~–
\end{verse}

- – – – – – – der nächste Morgen sah uns auf dem Bahnhofe.

»Den Rest mußt du mir nun doch lieber im Eisenbahnwagen erzählen,
oder noch besser zu Hause im ganzen und der Ordnung nach vorlesen«,
meinte Emmy, als wir in meines Vaters Hause uns zum allerletzten
Male schlafen legten. Sie erinnerte sich, todmüde von dem
fröhlichen Abend, nicht daran, daß sie im Eisenbahnwagen stets
leicht Kopfweh bekommt und unfähig wird, auf das Interessanteste
hinzuhorchen.

\section{Einundzwanzigstes Blatt}

\zusatz{Auf dem Schub und im Frieden}
Wir stiegen grade in den Wagen, der uns mit unsern Hutschachteln
und Koffern und meiner alten Christine nach der Stadt und dem
Bahnhof bringen sollte, als die erste Kastanie unter der Axt fiel.
Der Architekt stand am teilweise schon niedergelegten Zaun von
Pfisters Garten und winkte uns mit dem Hute vergnügt nach. Nun
hatte ich nur noch am Bahnhof den schönen Strauß zu überwinden, den
Dr.~jur.~Riechei, welcher den berühmten Prozeß Pfister gegen
Krickerode so glänzend ausfocht und gewann, meiner Frau ins Coupé
reichte, und dann – war Pfisters Mühle nur noch in dem, was ich mit
mir führte auf diesem rasselnden, klirrenden, klappernden Eilzuge,
vorbei an dem Raum und an der Zeit.

Da brauchte ich dann wohl nicht mehr zu fragen: Wo bleiben alle die
Bilder?\ldots{} Die von ihnen, welche bleiben, lassen sich am besten
wohl betrachten im Halbtraum vom Fenster eines an der bunten,
wechselnden Welt vorüberfliegenden Eisenbahnwagens.~–

Wie unauslöschlich fest steht Pfisters Mühle gemalt in meiner
Seele!

Mir gegenüber hatte ich die geröteten Augen meiner alten
Pflegemutter; meine junge Frau lehnte meistens ihr Häuptlein an
meine Schulter. Von den wechselnden Wagengenossen und den kleinen
Abenteuern der Reise ist mir diesmal nichts in der Erinnerung
hängen geblieben: ich begrub den armen tragischen Poeten, Doktor
Felix Lippoldes, noch einmal von Pfisters Mühle aus; ich trug
meinen lieben Vater – den guten Vater Pfister – von seiner Mühle
aus zu Grabe und hatte nicht zu suchen und zu fragen, wo die Bilder
geblieben waren. Wie könnte ich zum Exempel den Ton vergessen, mit
dem mein Vater, als wir die Leiche des Poeten dicht vor unserm Wehr
fanden, sagte:

»Kinder, es stimmt ganz mit mir!«

Aber er sagte auch, und zwar mit einem ganz andern Ton und
Ausdruck:

»Doch das arme Mädchen gehört mir auch an. Ihr zwei, du, Ebert, und
du, Adam, vor allem, werdet euch am besten wohl aus dem Hause
scheren und euch wo anders unterbringen, im Dorf, in der Stadt, und
wenn ihr mir in den nächsten paar Tagen mit dem Schriftlichen zur
Hand gegangen seid, auch wieder in euerm Berlin. Ich hab es Ihnen
wohl vorausgesagt, Doktor Asche, daß es nichts mehr werden würde
mit den Weihnachten in Pfisters Mühle.«

Nun war es rührend, auch von fern aus anzusehen und halb zu ahnen,
wie zart der alte Mann, Müller und Schenkwirt mit der jungen Dame
in seinem Haus und winterlichen Garten umging.

In dem Anbauerhause, in dem Albertine Lippoldes ihren Vater bei Tag
und Nacht in Dürftigkeit und Scham mit ihren klugen, unruhigen
Augen bewacht hatte, ohne ihn vor seinem endlichen Schicksal
bewahren zu können, war nichts mehr, was ihr gehörte, wie sich
sofort nach Verbreitung des Gerüchts vom Tode des berühmten Mannes
durch Wort und Zeitung fand. Aber mein Vater sagte, auf mich
zeigend:

»Das da ist mein Erbe; aber du, liebes Kind, bist mein letzter
Gast. Hole eine Leiter und nimm das Schild von der Tür, Samse. Wir
schließen mit heute die Wirtschaft; laß mir deine Hand, arm
Mädchen, gute Tochter – Vater Pfisters letzter, liebster Gast in
dieser lustigen Welt!\ldots{}«

Auf dem Wege nach dem Dorfwirtshause, hinter dem Schubkarren her,
der unser Reisegepäck trug, schnarrte Asche grimmig und mit dem
Regenschirm an die niedere Mauer des Kirchhofes, an welchem wir
eben vorbeischritten, klopfend:

»Eberhard Pfister, sie werden wieder mal keine Ahnung davon haben,
welchen großen, wirklichen Dichter sie mit Rasen bedecken, wenn sie
deinen Vater – den Vater Pfister hier neben dem Doktor Felix
Lippoldes seinerzeit verscharren werden. Der Himmel wende es noch
lange ab!«

Das hat nun der Himmel freilich nicht getan, aber er hat dem einst
so fröhlichen und allezeit hülfreichen Herzen des letzten Wirtes
von Pfisters Mühle Zeit gelassen, noch ein oder zwei gute Werke zu
verrichten und ein heiter glänzend Licht vor die dunkle Pforte zu
stellen, die sich hinter ihm so bald, leider so bald, für immerdar
schließen sollte.~–

»Es ist meiner Frauen Bette, das dir die Christine in der Kammer
unterm Dach aufschlagen soll, Kind«, sagte der alte Meister. »Bleib
bei mir, Herz; wenigstens, bis du wieder mehr Ruhe hast. Was willst
du, obgleich du eine vornehme junge Dame und eine junge, schöne
Gelehrte bist und alle Sprachen kannst, in der Fremde? Bleib bei
mir, denn hier hast du mit keinem weiter zu schaffen als mit meiner
seligen Frau und mir, der auch mit keinem mehr zu tun haben will.
Die Christine da kannst du, wenn du sie erst besser kennengelernt
haben wirst, auch zu uns zweien rechnen. Und sieh mal, wen findest
du obendrein da draußen, der deinen Papa besser kannte und mehr
ästimierte als der alte Pfister von Pfisters Mühle? Wenn sie vor
Jahren auf ihn sahen wie auf ein Wunder, wenn er uns mit seiner
Gegenwart im Garten oder in der Gaststube beehrte: wer hat bei
seinen hohen, fließenden Worten das Herz höher in seinem Hals
gefühlt als wie ich? Da unter den kahlen Bäumen, wenn sie in
Blüten, im Laube und im Mondlicht standen, und in der Winternacht,
wenn er so gegen zwei Uhr morgens ging und noch keiner aus der
Stadt seinetwegen die Beine unterm Tisch vorziehen konnte: wer hat
da mehr als ich seinen Stolz an dem Herrn Doktor gehabt, als er
selber noch seinen Stolz hatte? Wenn er so deklamierte, liebes
Kind, seine Ehre und sein Ruhm ist da manch liebes Mal meine Ehre
und Glorie gewesen, wenn ich hinter seinem Stuhl stand oder mit am
Tische sitzen konnte. Nun hat er seinen Prozeß verloren, und mir
hat Doktor Riechei den meinigen gewonnen, und es ist ganz ein und
dasselbige; – weiß Gott!\ldots{} Ich fühle mich, wie er da liegt, und du
tätest ein Werk der Barmherzigkeit, wenn du bei mir bliebest. Ich
weiß es ja wohl, du hast mich gar nicht nötig; – du kannst morgen
schon als kluge, studierte junge Dame in die Welt gehen und findest
dein Brot überall; aber – tue es deines Vaters guten Stunden in
Pfisters Mühle zuliebe, bleibe hier fürs erste. Ich gebe dir mein
Wort, es soll dir keiner – weder mein Junge noch – sonst wer in den
Weg kommen, solange du selbst was dagegen hast. Also, bleibe bei
uns für jetzt und mache mit mir den Beschluß von Pfisters Mühle,
mein armes, liebes Mädchen.«

Fräulein Albertine hat da ihr schmerzendes Haupt an die Brust des
alten Herrn gelegt und hat dem Vater Pfister sein Mitleid und seine
Güte vergolten bis an den Tod – seinen Tod. Ja, bis zu Vater
Pfisters ruhigem Abscheiden aus dieser ihm so sehr übelriechend und
abschmeckend gewordenen Welt hat Albertine Lippoldes ihr Bestes
getan, ihm seine letzten Tage leicht und freundlich zu machen, da
sie dem eigenen Vater nicht mehr helfen konnte.

Der liegt auch in seiner Ruhe auf dem unbekannten Dorfkirchhofe
unter einem grünen Hügel, auf welchem kein Epitaphium mit Namen,
Jahreszahlen und sonstiger Steinmetzarbeit drückt, welchen also
kein Literaturgeschichtenschreiber und Interviewer post mortem so
leicht wohl finden wird.«

Mein Vater blieb fest bei seinem Wort. Er steckte, nachdem Samse
sein Schild von unserer Tür herabgenommen hatte, nicht wieder einen
grünen Busch über seinen Torweg. Nicht zu Ostern und auch nicht zu
Pfingsten. Fräulein Albertine hatte den Mühlgarten den nächsten
Sommer ganz für sich allein.

»Nur mit dir, Ebert, wenigstens während eines Teils, als du vor
deinem Examen saßest, und ich hätte wohl Grund, heute noch ein
wenig eifersüchtig zu sein«, sagt Emmy, fügt aber hinzu: »Nun, da
ist es denn freilich ein Glück gewesen, daß Doktor Asche schon
vorhanden war.«~–

Doktor Adam Asche ließ sich den ganzen Sommer über nicht in
Pfisters Mühle blicken. Er baute am Ufer der Spree weiter an seinem
Vermögen und seiner sonstigen nähern und fernern Zukunft und ließ
nur von Zeit zu Zeit in etwas unbestimmter Weise in seinen Briefen
an mich »alle unter Vater Pfisters Dache freundlichst« grüßen.

Merkwürdigerweise schrieb er damals ziemlich häufig an mich, er,
der sonst in dieser Hinsicht (außergeschäftlich) alles für seine
Korrespondenten zu wünschen übrigließ. Ich aber häufte nun für
seinerseits früher begangene Unterlassungssünden feurige Kohlen auf
sein Haupt, antwortete rasch und ausführlich und unterhielt ihn
stets aufs genaueste über meine Zustände, Hoffnungen und
Befürchtungen.

Darüber wurde er dann von Brief zu Brief immer anzüglicher und
gröber und schien es wirklich als ein Recht zu verlangen, daß ich
ihn wenigstens dann und wann \emph{zwischen den} \emph{Zeilen}
lesen lasse. Mein Vater, der »diesen schnurrigen Patron und Freund
Hechelmaier« fast ebenso gern schreiben als reden hörte, ließ sich
jeden Brief vorlesen, und nicht immer nahm Fräulein Albertine ihre
Arbeit und verschwand unter dem Vorwand, daß sie vom Hause oder aus
dem Garten her gerufen werde.

Tat sie es, so stieß mich Vater Pfister jedesmal in die Seite,
rückte mir näher und meinte kopfschüttelnd, aber doch lächelnd:

»Nun sieh mal. Soweit meine Menschenkenntnis hier von unsrer Mühle
und Pfisters Vergnügungsgarten aus reicht (und es sind mancherlei
Hochzeiten in unserer Kundschaft hier unter diesen Bäumen und an
diesen Tischen zustande gebracht worden), meint er es doch ungemein
gut mit ihr – seelengut! Und ein so ganz übler Bursche ist er ja
auch nicht, wenngleich eine feine junge Dame wohl allerlei Kurioses
an ihm auszusetzen haben mag. Sieh mal, und es wäre doch sehr
hübsch und eine wahre Beruhigung für mich, wenn ihr alle
dermaleinst, so gut es gehen will, noch zusammen- und
aneinanderhieltet, wenn mit dem alten Pfister auch seine Mühle
nicht mehr auf Gottes verunreinigtem Erdboden und an seinen
verschlammten Wasserläufen gefunden wird. Was der Mann da zum
Beispiel von seinem stinkigen Berufe und Geschäfte schreibt,
braucht dich gar nicht zu hindern, dein Kapital mal mit
hineinzustecken. Wie lieb wäre es mir aber dazu, wenn dann das
liebe Kind da einen Strauß und Duft von meinen Wiesen euch mit
darzu täte! Du holst dir dann deine Frau mit ihrem Strauß und
Blumengeruch von einem andere Garten weg; die Christine und den
Samse verlaßt ihr mir auch nicht, und so ist, wenn ich nicht mehr
bin, der Schaden vielleicht für Kinder und Kindeskinder nicht ganz
so groß, wie ich ihn mir dachte, als sie mir Krickerode auf die
Nase bauten und mir meine Lust an meinem Rade, meinem Bach, mein
Leben und Wohlsein auf deiner Väter Erbe verekelten.«~–

Und die Räder unter uns rasselten, klirrten und klapperten, und es
war ein Rauschen dazu, daß ich, wenn ich auch die Augen schloß wie
mein Weib neben mir oder die alte Christine mir gegenüber, wohl
meinen mochte, die Jahre seien nicht hingegangen, ich sei noch ein
Kind in meines Vaters Mühlstube und höre das Getriebe um mich und
das Wehr draußen. Ich hielt sie aber mit Gewalt offen, die Augen;
ich hatte zu wenig Zeit mehr, mich dem Traum hinzugeben und mit dem
Vergangenen zu spielen – die Tage in Pfisters Mühle waren vorüber,
und Arbeit und Sorge der Gegenwart traten in ihr volles, hartes
Recht.

Wir waren auch in Berlin viel eher, als wir es dachten. Und
obgleich es heute nicht mehr die Kirchtürme der Städte sind,
sondern die Fabrikschornsteine, die zuerst am Horizont auftauchen,
so hindert das einen auch heute noch nicht, gesund, gesegnet und –
soviel es dem Menschen auf dieser Erde möglich ist – zufrieden mit
seinem Schicksale, ergeben in den Willen der Götter
\emph{nach Hause zu kommen}.

»Gott sei Dank!« seufzte Frau Emmy Pfister, sich aufrichtend und
die Äuglein reibend. Gluhäugig, dann – fröhlich und glücklich
blickte das Kind umher und dann mir mit einiger dunkel
aufsteigenden Befangenheit und Ängstlichkeit ins Gesicht. Wie
konnte ich da anders, als meinerseits so vergnügt und behaglich als
möglich auszusehen?

Dichter drängte sich mein junges Weib unter dem schrillen Gepfeife
der Lokomotive an mich heran und kümmerte sich gar nicht um die
Leute und flüsterte:

»O Herz, liebster, bester Mann, ich kann ja nichts dafür; aber ich
freue mich so sehr, so unendlich auf unsere eigenen vier Wände und
deine Stube und meinen Platz am Fenster neben deinem Tische! Bist
wohl manchmal recht böse auf mich gewesen, aber ich konnte ja
wirklich nichts dafür und habe mir gewiß selber Vorwürfe genug
gemacht, wenn ich in den letzten Wochen nicht alles gleich so
mitsehen und mitwissen und mitfühlen konnte wie du. Es war ja
wirklich so wunderschön und das Wetter auch und die guten Stunden
unter den Hecken und auf deinen Wiesen, aber – o bitte, bitte,
nicht böse sein! – auch manchmal so bänglich für dein arm,
närrisches Mädchen, deine dumme, kleine Frau in deiner verzauberten
Mühle, die dir gar nicht mehr gehörte, und bloß mit unsern
mitgebrachten Koffern und Petroleumkocher, den wir freilich nicht
gebrauchten, und den geliehenen Stühlen und Tischen und Betten aus
dem Dorfe, die wir so sehr nötig hatten! Und wie wird sich mein
Papa freuen, daß er mich wieder in der Nähe hat bei seinem fatalen
Kirchhof, wenn er es uns auch nur auf seine Art merken läßt und ein
paar schlechte Witze macht. Sieh nur gleich scharf, daß sie dir
nicht die letzte Droschke wegschnappen, und ich will es dir auch so
behaglich bei dir und mir machen, daß du doch denken sollst, das
Beste habest du doch mitgebracht nach Berlin von Pfisters Mühle.
Und wenn dein armer, lieber Papa es sehen könnte, würde er sich
auch freuen, und deine gute, alte Seele, deine Christine, haben wir
ja auch zu uns geholt aus deiner Verwüstung, und sie wird mir
helfen in meinem jungen Hausstande – nicht wahr, Christine?!«

»Helfe mir Gott – so gut ich kann!« schluchzte meine greise
Pflegerin, betäubt, willenlos in das Gewühl der Großstadt
starrend.

Und mein Weib, meine Frau! War sie nicht in ihrem Rechte, wie ich
vordem in Wirklichkeit in Pfisters Mühle und während der letzten
vier Wochen im Traum?

Sie war während meines Sommerferientraumes nicht in ihrem Elemente
gewesen, und nun fand sie sich wieder darin, und ich – wußte
gottlob, weshalb ich sie auf ihres sonderlichen Papas düsterm
Spazierplatz gesucht und für mich hingenommen und festgehalten
hatte. Sie war wieder bei sich zu Hause und in meinem Hause (wenn
es auch nur eine moderne, unstete Mietswohnung war) ganz meine
Frau, mein Weib, mein Glück und Behagen. Was ging sie eigentlich
mit vollkommen zureichendem Grunde Pfisters Mühle oder gar der
große, unbekannte dramatische Dichter Doktor Felix Lippoldes an, da
\emph{wir uns} hatten und »die gute Albertine ja gottlob auch ihren
Adam und ihre neue, feste Heimat!«~–?

\section{Zweiundzwanzigstes Blatt}

\zusatz{Von Vater Pfisters Testament, der Mühlen Ausgang und
Fortbestehen, und wozu doch am Ende das Griechische nützt.}
Und da sitze ich wieder an meinem feststehenden, soliden
Arbeitstisch, den ersten Packen korrigierter blauer Schulhefte auf
dem Stuhl neben mir. Nun könnte ich mich selber literarisch
zusammennehmen, auf meinen eigenen Stil achten, meine Frau und alle
übrigen mit ihren Bemerkungen aus dem Spiel lassen und wenigstens
zum Schluß mich recht brav exerzitienhaft mit der Feder aufführen.
Wenn ich wollte, könnte ich jetzt auch noch das ganze Ding über den
Haufen werfen und den Versuch wagen, aus diesen losen
Pfisters-Mühlen-Blättern für das nächste Jahrhundert ein wirkliches
druck- und kritikgerechtes Schreibekunststück meinen Enkeln im
Hausarchive zu hinterlassen.

Und es fällt mir nicht ein – es fällt mir im Traume nicht ein! Ich
werde auch jetzt nur Bilder, die einst Leben, Licht, Form und Farbe
hatten, mir im Nachträumen solange als möglich festhalten!

So schreibe ich weiter, während ich Emmy nebenan fröhlich lachen
und meine alte Wärterin und Pflegemutter »einen wahren Trost im
Dasein« betitulieren höre.

Das alte, tapfere Mädchen, die Christine! Sie hat gottlob ihre
Beschäftigungen gefunden, die auch in Berlin nicht leicht zu Atem
und vielem Nachdenken über das Vergangene kommen lassen! Wir haben
alle unsere Beschäftigung: Emmy in ihrem Haushalt und,
merkwürdigerweise, in merkwürdig viel Nachdenken über die nächste
Zukunft, ich in ebendem und meiner Quinta und Doktor A.~A.~Asche
auf Lippoldesheim oder, wie er sonst sein großes »Etablissement« zu
benamsen beliebt: Rhakopyrgos, arx panniculorum – Lumpenburg. Frau
Albertine Asche, geborene Lippoldes, hat auch ihre Beschäftigung
vom Morgen bis zum Abend in Lippoldesheim.~–

»Lippoldesheim!« brummt der berühmte chemische
Universalfleckenreiniger, Schön- und Neufärber. »Klingt es dir
nicht auch etwas affektiert, Pfister, wenn man das deutsche Drama
im allgemeinen und den wackern, armen, guten Teufel, meinen seligen
Schwiegervater, im besondern dranhält. Ja, aber wie kommen Namen in
die Welt!«

»Jawohl, wie kommen Namen in die Welt? Das ist eben eine solche
Frage wie die: Wo bleiben alle die Bilder, Freund Adam!«

Da ist er selber, Doktor Adam Asche aus Lippoldesheim und von
Rhakopyrgos. Er hat Geschäfte in der Stadt gehabt, sogar
Börsengeschäfte, und ladet sich bei uns ein auf kleinbürgerlich
Tagesglück und setzt Emmy und Christine glücklicherweise durchaus
nicht dadurch in Verwirrung. Uns ladet er ein, am Nachmittag mit
ihm hinauszufahren und den Abend und den morgenden Sonntag in der
»schönen Natur« zu verbringen. Er hat die Stirn, die Umgebung
seiner großindustriellen Fabrik eine »schöne Natur« zu nennen, und
wir freuen uns wirklich sehr auf dieselbe und sind bereit zu der
Fahrt, auch Jungfer Christine, auf die Samse sich unmenschlich
freut.

Übrigens fängt mein Exmentor merkwürdig rasch an, beleibt zu
werden, und das steht ihm gar nicht übel. Seine Nachmittagsruhe
hält er seit lange nicht mehr unter jedem beliebigen Busch im
Felde. Diesmal liegt er auf meinem Sofa nach Tisch; aber er hält
die Arme noch nach alter Weise dabei unterm Hinterkopf und behält
die Zigarre auch im tiefsten, süßesten Schlummer zwischen den
Zähnen – einem bemerkenswert intakten Gebiß.

Die Stunden des Sonnabendnachmittags gehören mir mehr als alle
übrigen der Woche; nun schreibe ich in ihnen, während das Leben
weiter wühlt, von \emph{Vater Pfisters letzten Tagen}.~–

Krickerode war rechtskräftig verurteilt worden. Das Erkenntnis
untersagt der großen Provinzfabrik bei hundert Mark Strafe für
jeden Kalendertag, das Mühlwasser von Pfisters Mühle durch ihre
Abwässer zu verunreinigen und dadurch einen das Maß des
Erträglichen übersteigenden übeln Geruch in der Turbinenstube und
den sonstigen Hausräumen zu erzeugen, sowie das Mühlenwerk mit
einer den Betrieb hindernden, schleimigen, schlingpflanzenartigen
Masse in gewissen Monaten des Jahres zu überziehen.

Das ist sehr gut für andere Flußanwohner, ob sie eine Mühle haben
oder nicht; aber Vater Pfister macht wenig Gebrauch mehr von dem
durch Doktor Riechei für ihn erfochtenen Sieg. Das hätte früher
kommen müssen – an jenem Tage schon, an welchem er sich zum
erstenmal fragte, wo eigentlich sein klarer Bach – der lustige,
rauschende, fröhliche Nahrungsquell seiner Väter seit Jahrhunderten
– geblieben sei und wer ihm so die Fische töte und die Gäste
verjage. Zu lange hat zuerst der alte Mann das widerwärtige Rätsel
selber sich lösen wollen. Zu sehr hat er sich ärgern müssen
innerhalb und außerhalb seines sonst so lustigen Besitzes auf
dieser Erde. Der Ärger über seine Nachbarschaft, seine Knappschaft
und seine Gäste hat ihm das Herz abgefressen, und so mußte es ihm
sogar zu einem Troste werden, daß »sein Junge doch nicht die alte
Ehre, den alten Ruhm von seiner Vorfahren wackerm Erbteil aufrecht
und im Getriebe halten könne, sondern, Gott sei Dank, einen Abweg
ins Gelehrte durch die Welt eingeschlagen habe«.

Und noch ein schönerer Trost ist ihm gegeben worden, daß die Sonne
im Scheiden, wenn nicht so vergnüglich wie sonst, doch ebenso
schön, ja noch schöner als sonst über Pfisters Mühle leuchte: des
armen, untergegangenen Poeten Kind, Albertine Lippoldes!

Es war im Herbst des Jahres, das der schlimmen Weihnacht folgte,
nach welcher das heimatlose Mädchen als letzter, liebster Gast
unter meines Vaters freundliches Dach eingeladen und in Zartheit
und Sicherheit gebettet wurde. Ich hatte eben die Bekanntschaft
meines jetzigen Schwiegervaters gemacht, und zwar infolge eines
andern Miteinanderbekanntwerdens, über das sich Emmy heute noch
nicht wenig verwundert stellt, wenn die Rede auf jene Zeiten
kommt.

»Und wir dachten doch damals noch gar nicht aneinander«, pflegt
mein Liebchen zu sagen; aber – dem sei nun, wie ihm wolle – ich
ging eben schon in jenem Herbst zuerst mit Rechnungsrat Schulze auf
seinem sonderbaren Spazierplatze lustwandeln, dachte aber freilich
dabei an ihn selber nur so viel, als unumgänglich nötig war, was
der Unterhaltung jedoch nicht Abbruch tat, sondern mich sogar
bewog, so gesprächig als möglich zu sein und stets der Meinung des
grauen, skurrilen Humoristen bei jedem Thema, welches er neben
seinem Taxus und seinen Trauerweiden knarrend aufs Tapet brachte.

Es war zu Anfang Oktobers, und warme, sonnige Tage waren, wie die
Götter sie nicht immer um diese Jahreszeit über Norddeutschland
hinzubreiten belieben. Die Bäume schienen in diesem Jahre länger
als sonst ihre Blätter, die Blumen, sowohl in den Gärten wie auf
Vater Schulzes Friedhofe, länger ihre Blüten festzuhalten. Die
Zeitungen brachten unter ihrem Vermischten in dieser Hinsicht
merkwürdige Einzelheiten, und Fräulein Emmy Schulze sagte zu mir:

»Nein, Herr Doktor, Papa hat ganz recht, es ist eigentlich zu
angenehm so! Und, Papa, rede nur nicht, das weiß ja jeder schon
selber, daß es so hübsch nicht bleiben wird.«

Auf Vater Schulzes Kirchhofe hatte mich der Briefträger aus einem
der Treppenfenster der umliegenden Häuser erspäht und kam, um mir
den letzten Brief meines Vaters aus Pfisters Mühle über das Gitter
zu reichen. Einen Brief in sehr veränderter Handschrift, doch im
vollkommen unveränderten Stil des alten Herrn:

\begin{quotation}
»Mein Junge, tust mir 'nen Gefallen, wenn Du für acht Tage Urlaub
nimmst. In Familienangelegenheiten, kannst Du vorschieben. Und
bring Doktern Asche möglichst mit. Hätte mit ihm auch einiges zu
besprechen. Neuigkeiten nicht zu vermelden als eine Kuriosität, die
ich aber auch schon öfters erlebt habe. Eine der Kastanien am
Wasser, dritter Tisch in der Reihe rechts, blüht zum andernmal.

Wir grüßen Dich alle. Fräulein Albertine auch. Und sind recht
gesund. Aber komm doch lieber auf ein paar Tage.

\textit{Dein Vater.«}
\end{quotation}

Doktor Adam Asche hatte wie immer »alle Hände voll« in seinem
merkwürdigen, aber gewinnbringenden Geschäft; als ich ihm jedoch
diesen Brief aus der Heimat zu lesen gab, wunderte mich die Hast,
mit der er ihn nahm, die Langsamkeit, mit der er ihn zurückreichte,
und der Eifer, mit welchem er seine Bereitwilligkeit, mich zu
begleiten, kundgab.

Er fragte durchaus nicht: Was kann der Alte mir zu sagen haben? Er
nahm mich an der Schulter, schob mich aus seinem modernen
Alchemistengewölbe und rief:

»Packen! Sofort packen! Du tust sofort die nötigen Schritte bei Abt
und Prior; ich mit meinem Reisesack bin unter allen Umständen
morgen abend auf dem Bahnhof und fahre ab. Wir benutzen den
Nachtzug und sind bei guter Zeit in der Mühle. Jetzt halte mich und
dich nicht länger auf, Mann! Packe dich und packe so rasch als
möglich!«~–

Wir kamen diesmal bei hellem, klarem Himmel zu Hause an. Der
leichte Dunst auf der sonnigen Ferne deutete tausendmal eher auf
einen neuen Frühling als auf den nahen Winter hin. Aber man hatte
uns Samse mit dem Mühlenfuhrwerk nach dem Bahnhof geschickt, und
obgleich der getreue Knecht niemals ein allzu fröhlich Gesicht
machte, erschrak ich doch heftig, als ich ihm jetzt in es
hineinsah.

»Wie steht es daheim, alter Freund?«

»Schlimm«, antwortete Samse kurzab. »Hat er denn gar nichts davon
geschrieben?«

»Daß er mich und den Doktor Adam sprechen will, daß ihr alle gesund
seid und daß die Kastanien in unserm Garten zum zweiten Male
blühen.«

»Du lieber Himmel!« seufzte Samse. »Da bleibt uns denn wohl nichts
anders übrig, als daß wir machen, daß wir möglichst bald nach Hause
kommen, um ihn leider Gottes in der Hauptsache Lügen zu strafen.
Vor der Apotheke muß ich doch noch mal anhalten.«

Wir warfen in aller Hast unser weniges Gepäck in den wohlbekannten
Korbwagen und fuhren im Trabe rasselnd durch die wohlbekannten,
auch schon in der Morgensonne lebendigen Gassen der Stadt. Vor der
Apotheke ließ mir Samse die Zügel, kam mit einer giftig aussehenden
Arzneiflasche aus dem Hause wieder zum Vorschein und brummte
seufzend:

»Wenn \emph{das} was helfen könnte! Ja, wenn sie es ihm vor Jahren
in seinen Bach bei Krickerode hätten schütten und sein Leben und
Gemüte dadurch reinlich hätten halten können! Der Doktor weiß es
auch selber gut genug, daß es nur eine Komödie damit ist, und der
Meister selber weiß es erst recht. Ihr Herren, fragt mich nur nicht
weiter; ihr werdet ja bald selber sehen, wie es mit uns steht,
trotzdem daß die Bäume in unserm Garten zum zweiten Male blühen.«

Wir kamen an in Pfisters Mühle, und wir sahen selber. Das heißt,
wir fanden den lieben alten Vater zum Sterben krank in seinem
Lehnstuhl, in heftigen Atembeschwerden nach Luft ringend, und doch
bei unserer Ankunft aus der Welt des Lärms, der pädagogischen
Experimente, des Lumpenreinigens und des Gelderwerbens gottlob
wieder mit dem alten, guten Lächeln um die trostlos blauen Lippen.
Wir fanden ihn reinlichst in seinem hellen Müllerhabit in seiner
Urväter altem gepolstertem Eichenstuhl und zu seinen Füßen auf
meiner Mutter Schemelchen Albertine Lippoldes mit einem Buche auf
den Knieen.

Sie hatte ihm daraus vorgelesen – aus einem von ihres Vaters
Geschichtsdramen nämlich, denn – »er tat in seiner letzten Zeit
nichts Lieberes als das anzuhören«, meinte Christine später.
»Unsereinem hielt es den Atem an, wenn man auch nur das wenigste
davon verstand; aber er atmete besser dabei, und es war ihm eine
Beruhigung, daß es selten einem Kaiser und König und grausamen
griechischen und römischen Soldaten und allen vornehmsten Damen
gegen Ende ihrer Komödien besser ergehe als dem Müller von Pfisters
Mühle.«~–

Als bei unserm Eintritt das Fräulein erschreckt und errötend sich
erheben wollte, legte ihr Vater Pfister die Hand auf die Schulter
und drückte sie sanft wieder nieder. Die andre Hand streckte er uns
entgegen:

»Guck mal, so schnell seid ihr da? Das ist schön! Und du auch,
Doktor Adam – trotzdem daß man keine Zeitung umwenden kann, ohne
dich hinten drin zu finden unter Pauken und Posaunen mit deinem
Mordgeschäft von Allerweltswäsche. Das ist brav! Und du, Junge,
Ebertchen, nun zieh mir nur keine Gesichter; bin ganz zufrieden mit
mir und ebenso mit unserm klugen Herrgott, wenn der mal wieder das
Beste wissen sollte und den alten Pfister, Jacke wie Hose, in seine
wirkliche, gründliche große Wäsche nähme. Ein gar lustiges
Trockenwetter schickt er ja dazu schon im vorauf – die beste Luft,
die er hat, für 'nen Patienten wie ich. Offene Fenster den ganzen
Tag und zu Mittag im Rollstuhl unterm blühenden Baum im Oktober!
Was will da unsereiner mehr?\ldots{} Nun legt ab und macht's euch
behaglich und spielt nicht die Narren, wenn's euch auch
einleuchtete, daß ihr zum letzten Kommers in Pfisters Mühle
verschrieben seid, Kommilitonen! Helft mir Kontenance behalten und
tragt's euerm alten Schoppenwirt nicht nach, wenn er die letzten
Jahre durch zu muffig den Philister herausgekehrt hat. Willkommen
denn zum letztenmal im Bund – und sieh, Ebert, das liebe Fräulein
und mein liebes Kind hier hat mich noch in die Schule genommen; und
dich, Adam, habe ich diesmal nicht berufen, mir meinen Mühlbach auf
Krickerode zu untersuchen, sondern dich mit allen deinen
Wissenschaften und Chemikalien und richtigen Begriffen von unserm
Verkehr auf der Erde auch noch mal in die Schule zu geben.«

»O, wie gern kniee ich mit umgehängtem Esel auf Erbsen, Vater
Pfister!« rief Adam Asche mit sehr unsicherer Stimme, und das liebe
Fräulein fuhr nun doch auf und trat hinter den Stuhl des kranken
Greises, wie um ihn als eine Schutzwehr oder als ein Katheder zu
benutzen: ein Lachen, das ganz Pfisters Mühle in ihren besten Tagen
war, verklärte das fieberheiße Gesicht des guten, schlauen letzten
Wirtes von Pfisters Vergnügungsgarten.~–

Zu Mittage am andern Tage, als dann wiederum diese Herbstsonne wie
im vollen Sommer den leeren Garten anlachte, saßen wir am dritten
Tisch in der Reihe rechts unter dem noch einmal so kurz vor dem
ersten Schneefall blütentragenden Kastanienbaum, alle, die wir nach
gestelltem Rade und abgenommenem Schenkenzeichen noch dazu
gehörten: unser lieber Meister und Vater Bertram Pfister, Fräulein
Albertine Lippoldes, Doktor A.~A.~Asche, Jungfer Christine Voigt,
Samse und ich, Doktor Eberhard Pfister; und der Vater Pfister hielt
in Atemnot und bei von den Füßen aufwärts steigender Wassersucht
seine letzte Tischrede in seinem Garten. Sie floß ihm leider damals
nicht so leicht hin, wie mir jetzt hier aus meiner Feder.

»Kinder«, sagte er, »'s ist meine Devise gewesen: Vergnügte
Gesichter! Und wenn ich meine letzte Zeit durch selber keins
gemacht, sondern konträr mich als ein richtiger Narr und Brummkopf
aufgeführt habe, so denkt nicht daran, sondern denkt an den alten,
richtigen, fidelen Vater Pfister von Pfisters richtiger Mühle, wenn
ich euch später mal bei einem Liede oder bei Tische oder in einer
andern Wirtschaft, oder wenn ihr mal still bei euern lieben Frauen
und Kindern sitzet, durch den Sinn gehe. Es ist manch ein Lied hier
gesungen und manch eine Rede gehalten, lustig und ernsthaft; manch
eine Bowle habe ich hier auf den Tisch gestellt, und manch einer
ist auch mal drunter gefallen und gelegen gewesen, und die andern
haben weitergesungen und Sonne und Mond ihren Weg unbesehen gehen
lassen. Nun, Ebert, mein armer Junge, und ihr andern, liebste
Freunde, macht euch gar nichts draus, wenn auch ich jetzo das
letzte Beispiel nachahme und unter meinen eigenen Gasttisch
rutsche!\ldots{} Rede mir keiner drein; wie es gekommen ist, weiß ich in
meiner jetzigen Verfassung selber nicht ganz genau anzugeben; aber
'n bißchen zuviel habe ich, und es ist ein Glück, daß ich nicht
weit nach Hause habe. Der Nachwächter, der mich unterm Arm fassen
soll, steht, vom Herrgott abgeschickt, mir hinterm Stuhl und hat
schon mehrmals gesagt: ›Na, wenn's nun beliebt, Herr Pfister!‹ –
Laß das Tuch von den Augen, Herzmädchen, dich meine ich eben nicht
mit dem ›Wächter‹ mein liebes Leben! Denkt an meine Devise, ihr
andern! Ja, es beliebt mir durch alle Knochen und durch die ganze
Seele. Und weil ich's weiß, daß es mit mir zu Ende geht, so wird es
euch ein Trost sein, zu wissen, daß es mir eine Beruhigung ist, daß
kein Fremder da unter dem Dach und hier unter den Bäumen sich auf
meinen Ruf und Namen setzt, sondern daß mit dem alten Pfister es
auch mit der Pfister uralter Mühle aus ist. – Nun höret mein
Testament. Ihr werdet's zwar auch aufgeschrieben im Pult finden,
und ich hätte auch wohl den Doktor Riechei dazu berufen können, um
es euch vor meinem Bett vorzulesen; aber pläsierlicher ist mir
pläsierlicher, und der Baum hier über uns soll nicht vergebens zum
zweitenmal seine Maienkerzen aufgesteckt haben. Es soll sein, als
ob durch ihn mein Garten mir das letzte vergnügte Gesicht zu meinem
letzten Willen mache! Denn sintemalen ich stets ein Mann der
Ordnung gewesen bin, trotzdem daß die Welt und die Herren Studiosen
mich nur als den rechten Wirt zu Pfisters Mühle ästimieret haben,
so wird ja auch wohl jetzt alles nach seiner Ordnung zugehen.

\begin{verse}
›Wer selig will sterben,\\
Soll lassen vererben\\
Sein Allodegut\\
Ans nächstgesippt Blut ;–‹
\end{verse}
\noindent
das ist ein Reim,
den die juristischen Herren Studenten mir oftmals auch an diesem
Tische zitieret haben, wenn unter ihnen die Rede kam auf ihrer
Herren Väter Güter und so ein kleines Konto bei mir. Und so komm
her, mein eigen nächstgesippt Blut, mein lieber Sohn und
Philosophiedoktor Ebert Pfister, und tritt mit Verstand und
Gleichmut, mit einem vergnügten Herzen, wenn auch im Moment nicht
fidelen Gesichte, die Erbschaft an von Pfisters Mühle mit allem,
was dazu gehört und was zu deinem Vater in Treue gehalten hat in
guten und bösen Tagen, durch Sauer und Süß, durch Sommer und
Winter, durch Wohlduft und Gestänke. Darauf gib deine Hand nicht
mir, sondern der Christine da und dem Samse; oder, noch besser, leg
jedem, wie sie da bei dir sitzen, den Arm mal um die Schulter und
denke: Ich weiß, wie es der alte Mann meint!

Wollen sie am Orte, im Dorfe bleiben, was ich aber nicht vermute,
so kriegt die Jungfer Christine Voigt eine volle
Altjungferaussteuer an Bett, Geschirr und Geräte nach Wahl aus
ihrer Frauen, deiner seligen Mutter, Nachlaß, Samse Wagen und Pferd
und item sein Bett und Notwendiges an Tisch und Gestühl und ein
jegliches die Zinsen von einem Kapital, so dreihundert Mark
abwirft, so lange sie leben. Das Nähere im Pulte schriftlich –
deine sonstigen Verpflichtungen gegen meine zwei allergetreuesten
Helfershelfer im Erdenvergnügen ungeschrieben auf deine Seele,
Eberhard! Denn wie gesagt, ich glaube nicht daran, daß sie sich
hier am Orte halten werden, da es aus und zu Ende sein muß mit
meinem, deinem und ihrem Haus, Hof und Garten. Ich täte es auch
nicht und lebte unter diesen Umständen fort im Dorfe. Und nun – den
schwersten Sack in den Trichter! Nämlich, da mein eingeborner
Junge, Namens- und Erbeserbe gänzlich aus meiner und seiner Väter
Art schlug und kein Müller wurde, wofür ich jetzt nur dem Himmel
danke, so wünsche ich, daß Herr Doktor Adam Asche, meines alten
verstorbenen Freundes Schönfärber Asches aus und wieder in die Art
geschlagener Sohn und meines Jungen erster Lehrmeister in der Welt,
sich auch hier möglichst der Sache annimmt und Pfisters Mühle mit
allen Rechten, Werk und Zeug zu einem für alle Parteien
gedeihlichen Abschluß verhilft. Denn wenn auch Doktor Riechei den
Prozeß gegen Krickerode recht glorreich gewonnen hat, so fällt mir
doch grade jetzt des alten seligen Rektor Pottgießers öfteres Wort
hier am Mittwochsnachmittagskaffeetisch ein, wenn einer zu einer
Ehre gewünscht wurde, der nicht da war. ›Ist kein Dalberg da?‹
fragte er dann jedesmal im Kreise herum unter den Herren
Oberlehrern und Kollaboratoren und ihren lieben Damen. Es tat dann
nie einer den Mund auf und rief ›Hier!‹ und so auch in meinem Fall.
Was helfen mir alle ersiegten Gerechtigkeiten, wenn kein Dalberg
und kein Pfister vorhanden ist, sie auszunutzen. So meine ich,
Samse und Christine halten sich hier auf dem Altenteil und Adam
Asche liegt auf der Lauer und wartet ab, bis ihm die neue Welt und
Zeit das Rechte honorig bietet für die Stelle und den Wasserlauf;
dann schlägt er ein, und wenn der Doktor Eberhard sein Kapital in
seines Freundes neuem Geschäft anlegt, ist's mir auch recht. Für
seine Mühe aber vermache ich dem Adam Asche meine Mülleraxt, die er
sich über meinem Bette herunterholen soll, wenn sie mich
herausgehoben haben, und wobei er manchmal in seinem besagten neuen
Geschäft gedenken mag, wie viele Pfister die seit vielen
Jahrhunderten mit Ehren in der Faust hielten.«

»Hier, Vater Pfister!« rief mein Freund mit bebender Stimme, dabei
mit merkwürdig unsicherer Hand die Hand des Greises fassend, und
nun doch, als habe aus der neuen Zeit heraus jemand in eine
versinkende hinein auf den fragenden Ruf: »Ist kein Dalberg da?«
geantwortet.

»Gedacht hätte ich es wahrhaftig nicht, wenn ich dich in meinen
Bäumen über dem Gelage hängen oder auf meiner Wiese im Heu liegen
sah, und noch weniger, als ich dich mir mit deiner Wissenschaft zur
Hülfe rief gegen Krickerode«, sagte mein Vater kopfschüttelnd,
lächelnd.

Die Augen feucht, voll Tränen, doch auch voll wundervoll anmutigen
Glänzens, legte Albertine Lippoldes das Kissen hinter dem alten,
müden Haupte zurecht, und der alte Mann sah zu ihr auf und
streichelte ihr leise den hülfreichen Arm und sagte:

»Ja, Kind, ich habe nicht ganz ohne Nutzen an diesen Tischen hinter
meinen Gästen im Dasein gestanden. Zu meinem Vergnügen an der
verschiedenen Unterhaltung ist es mir auch ein Vergnügen gewesen,
zu lernen und zuzulernen. Und so ist es mir jetzt der beste Trost,
daß ich genau weiß, weshalb wir nicht mehr recht aufkommen gegen
Krickerode, trotz aller gewonnenen Prozesse. Aus jedweder
Unterhaltung im Gastzimmer und hier unter den Kastanien zwischen
alt und jung, Gelehrten und Ungelehrten, Bürger, Professor, Bauer
und Bettelmann, Weib und Mann, wie das der Herrgott bis zu den
Kindern mit dem Kreisel oder im Kinderwagen herunter durcheinander
gehen ließ in Pfisters Mühle, habe ich allgemächlich angemerkt,
weshalb wir nicht mehr bestehen vor Krickerode. Und, Fräulein
Albertine, meines seligen Freundes Schönfärber Asches Junge hat mir
das letzte Verständnis dafür eröffnet. Denn das ist derjenige, von
dem ich mir am festesten gedacht habe, daß er eher sein Herzblut
hergeben werde, als die Wirtsstube und den Garten, die Wiesen, den
Fluß und die Sonne von Pfisters Mühle! Denn ich habe ihn ja
aufwachsen und hinbummeln sehen und auf meinem Konto gehabt von
Kindsbeinen an, und es ist keiner gewesen, auch dein armer seliger
Papa nicht, Kind, der mit solchem Sinn fürs Ideale seine Beine
unter meine Tische oder sich ganz der Länge nach auf die Bänke oder
in die Gräserei gestreckt hat wie meines alten Kumpans Schönfärber
Aschen nachgelassener Phantastikus, Adam Asche! Da \emph{der}
Partei genommen hat für die neue Welt und Mode und hergekommen ist
und den Kopf nicht nur in die Wissenschaft, sondern auch in die
doppelte Buchhaltung, das Fabrikwesen gesteckt und Krickerode nicht
bloß für mich ausgespüret, sondern es in anderer Art für sich
selber an euerm Berliner Mühlenbach aufgepflanzet hat, so gebe ich
klein bei und sage: dann wird es wohl der liebe Gott für die
nächsten Jahre und Zeiten so fürs beste halten. Fräulein Albertine,
wer dieses strubbelköpfige Geschöpfe in seinem seligen Schlummer am
Feldwege unterm Hagedorn bekopfschüttelt und es nachher an der
chemischen Wäsche gesehen hat und es heute in seinem Wesen und
Treiben, Spaß und Ernst sieht, der muß sich bekennen, der richtige
Mensch hat am Ende auch nicht die reine Luft, die grünen Bäume, die
Blütenbüsche und das edle, klare Wasser von Quell, Bach und Fluß
nötig, um ein rechter Mann zu sein.

Hast es dem Vater Pfister kurios beigebracht, Freund Adam, wie dem
Menschen auf dieser Erde alles Wasser auf seine Mühle werden kann;
und auf daß du siehst, daß er's dir nicht übelgenommen, wenn du
auch mal in betreff von des alten, närrischen Kerls Idealem zu sehr
pläsierlich den Gleichmut herauskehrtest, so will er dir jetzt zu
deinem Ideal, höchstem Sehnen und schönstem Wunsch, in deinem
Schornsteindampf und Waschkesselqualm verhelfen – im heiligen
Ernst! Nämlich es ist wohl von vorigen Weihnachten bis jetzt in
diesen Oktober zwischen mir und meinem lieben Kinde hier so von
Zeit zu Zeit die Rede auf dich gekommen, Doktor, und da habe ich
denn, wie gesagt, manchmal behauptet, grade Leute von deinem
Schlage würden wohl noch am ersten die Traditionen von Pfisters
Mühle auch unter den höchsten Fabrikschornsteinen und an den
verschlammtesten Wasserläufen aufrechterhalten; und, Doktor Asche,
Fräulein Albertine hat wirklich meiner Meinung beigepflichtet, und
– na, was ist mir denn dieses? Paß auf das Geschirr, Samse; da
fängt's an, heiß herzugehen unter den Kastanien – dritter Tisch,
Reihe rechts!\ldots{}..«

Wenn je ein Mensch zu Stein auf einem Stuhle geworden war, so war
das mein guter Freund Doktor A.~A.~Asche. Aber nur einen Augenblick
starrte er regungslos von dem alten Vater Pfister auf das junge
Fräulein; und wenn je ein Mann ein hübsches, tapferes, kluges
Mädchen fest in die Arme gefaßt hatte, so war das mein närrischer
Freund Adam ebenfalls.

»Ja, es war so auch meine Meinung«, flüsterte das Kind des
verlorengegangenen Poeten schluchzend. »Du bist sehr gut gegen mich
und meinen Vater gewesen; ich aber habe zuerst dich nicht recht
gekannt und nachher nicht mehr gewußt, wie ich dir danken sollte.«

Die Stimme, mit der Adam Asche jetzt nichts weiter als: »Vater
Pfister!« rief, klang nicht im Alltagston des Gründers von
Rhakopyrgos, und Vater Pfister sagte trübe lächelnd:

»Das ist nicht die erste Hochzeit, die in Pfisters Mühle verabredet
worden ist; aber es wird wohl die letzte gewesen sein. Halte dein
Weib in Liebe und meine Axt in Ehren, Adam. Räum den Tisch ab,
Samse, zieh mir die Decke um den Leib, Christine; und du, mein
lieber Junge, schieb den letzten hiesigen Müller und Wirt aus
seinem Garten, roll ihn ins Haus. Du hattest gottlob deiner Väter
Ehrenstab und Waffe nicht vonnöten bei deinem Kopf- und Handwerk.
Halte du in deiner Schule nur einfach diejenigen beim Rechten, zu
denen von ihren Vätern her der Ruf von Pfisters Mühle im Liede
kommen sollte!«\ldots{}

Sieben Tage später ist er nach schwerem Leiden in unser aller
Gegenwart sanft und friedlich eingeschlafen, mein liebe Vater, der
gute, fröhliche Vater Pfister. Nachher haben Adam und Albertine
geheiratet, und Vater Schulze hat seine Einwilligung zu meiner
Verlobung mit Emmy, wie ich vermute, mit Vergnügen,
selbstverständlich jedoch nicht ohne absonderlichstes Gesperr,
Gezerr und Gespreize erteilt.

Wo bleiben alle die Bilder? – – –

Freund Asche hat wieder einmal seinen Nachmittagsschlaf auf meinem
Sofa beendet; wir sind mit ihm nach Lippoldesheim hinausgefahren
und sind am Sonntag abend wieder nach Hause gekommen. Wo bleiben
alle die Bilder? Hier halte ich das letzte des bunten Buches fest;
für das Schicksal des Blattes Papier, auf welches es gemalt wird,
übernehme ich auch diesmal keine Verantwortung.~–~–

Die zwei Frauen sitzen in der Veranda von Lumpenburg-Lippoldsheim
unter der Klematisblüte und im Kinderlärm; die beiden Männer
wandern am Ufer der Spree, wie vordem zwischen dem Weidengebüsch am
Ufer von Vater Pfisters Mühlbach.

Noch ein Mann wandelt von der Villa her auf uns zu und überbringt
uns zarten Wunsch in nicht grade ausgelassen vergnügter Art:

»Die Herrens möchten zum Tee kommen.«

Das ist Samse. Er und Christine gehören vollständig zu uns; wir
können uns weder Lippoldesheim noch unser Heimwesen in der Stadt
Berlin noch die Bilder, die einst waren, ohne die zwei vorstellen –
denken.

Wir gehen zum Tee unter der Veranda. Nebenan klappert und lärmt die
große Fleckenreinigungsanstalt und bläst ihr Gewölk zum Abendhimmel
empor fast so arg wie Krickerode. Der größere, wenn auch nicht
große Fluß ist, trotzdem daß wir auch ihn nach Kräften
verunreinigen, von allerlei Ruderfahrzeugen und Segeln belebt und
scheint Rhakopyrgos als etwas ganz Selbstverständliches und höchst
Gleichgültiges zu nehmen.

Aus der Wiege des jüngsten Asche schallt plötzlich ein heftigeres
Geschrei, und Vater Asche spricht:

»Der versteht's auch! Nun hör ihn nur und richte dich auf Ähnliches
ein, Knabe Telemachos. Höre nur das intensive Bedürfnis der Krabbe,
ihren Willen zu kriegen! So was hilft. Das ist kein Knüzäma oder
Wimmern, keine Ololügä oder Weinen, kein Klauma, keine Oimogä, kein
Odürmos – dies ist das Richtige: eine Blächä, Geblöke, ein Orügmos,
Geheul, kurz eine Korkorügä, die dem Lümmel sofort zu seiner Mutter
Brust verhelfen wird. Da ist sie ja schon mit aufgehobenen Armen
und fliegendem Hyakinthosgelock. Na, Pfister, ich denke, der Junge
wird ferner gut werden, nicht aus der Art schlagen und seinem Alten
keine Schande machen.«

»Bei allen Göttern von Hellas, wie kommst du aber zu dieser
Nomenklatur des Menschen- und Kindergeschreis, von den
Hyakinthoslocken deiner Albertine ganz abgesehen, Adam?«

»Ja, siehst du (der junge Molch und Reklamerich hat sich an meiner
Frau so fest verbissen, daß sie nicht sieht und hört), weißt du,
das Handwerk ist doch zu stinkend, und selbst eine solche
Hausidylle wie die unsrige reicht gegen den Überdruß nicht immer
aus. Es ist eben nicht das Ganze des Daseins, alle Abend aus der
Wäsche von alten Hosen, Unterröcken, Ballroben, Theatergarderobe
und den Monturstücken ganzer Garderegimenter zu der besten Frau und
zum Tee nach Hause zu gehen. Da habe ich mir denn das Griechische
ein bißchen wieder aufgefärbt und lese so zwischendurch den Homer,
ohne übrigens dir hierdurch das abgetragene Zitat von seiner
unaustilgbaren Sonne über uns aus dem Desinfektionskessel heben zu
wollen.«~–

\end{document}
