\usepackage[ngerman]{babel}
\usepackage[T1]{fontenc}
\usepackage{textcomp}
\hyphenation{wa-rum}


%\setlength{\emergencystretch}{1ex}

\newcommand{\uebersicht}[1]{#1\medskip\par\noindent}
\newcommand\wurzel{$\sqrt{-1}$}

\begin{document}
\raggedbottom
\begin{center}
\textbf{\huge\textsf{Wir}}\\
\bigskip
{\large Jewgenij Samjatin}\\
\bigskip
Übersetzung von\\
\texttt{http://nemesis.marxists.org}\\
\bigskip
Nicht-kommerzielle Nutzung erlaubt.
\end{center}

\section{EINTRAGUNG NR. 1}

\uebersicht{\emph{Übersicht:} Eine Zeitungsnotiz. Die weiseste
aller Linien. Ein Poem.}
Ich schreibe hier genau ab, was ich in der heutigen Staatszeitung
lese:

„In hundertzwanzig Tagen ist unser erstes Raketenflugzeug Integral
vollendet. Es naht die große historische Stunde, da sich der
Integral in den Weltraum aufschwingen wird. Vor einem Jahrtausend
haben eure heroischen Vorfahren diesen Planeten dem Einzigen Staat
Untertan gemacht. Ihr seid es, deren gläserner, elektrischer, Feuer
speiender Integral die unendliche Gleichung des Alls integrieren
wird. Eure Aufgabe ist es, jene unbekannten Wesen, die auf anderen
Planeten — vielleicht noch in dem unzivilisierten Zustand der
Freiheit — leben, unter das segensreiche Joch der Vernunft zu
beugen. Sollten sie nicht begreifen, dass wir ihnen ein
mathematisch-fehlerfreies Glück bringen, haben wir die Pflicht, sie
zu einem glücklichen Leben zu zwingen. Doch bevor wir zu den Waffen
greifen, wollen wir es mit dem Wort versuchen. Im Namen des
Wohltäters wird sämtlichen Nummern des Einzigen Staates bekannt
gegeben: jeder, der sich dazu befähigt glaubt, ist verpflichtet,
Traktate, Poeme, Manifeste, Oden und andere die Schönheit und
erhabene Größe des Einzigen Staates preisende Werke zu verfassen.

Diese Werke werden die erste Botschaft sein, die der Integral in
den Weltraum trägt. Heil dem Einzigen Staat! Heil dem Wohltäter!
Heil den Nummern!“ Mit glühenden Wangen schreibe ich diese Worte
nieder. Ja, wir werden diese herrliche, das ganze Weltall um-
fassende Gleichung integrieren! Wir werden die wilde, krumme Linie
geradebiegen, sie zur Tangente, zur Asymptote machen. Denn die
Linie des Einzigen Staates ist die Gerade. Die große, göttliche,
weise Gerade, die weiseste aller Linien.

Ich, Nr. D-503, der Konstrukteur des Integral, ich bin nur einer
der vielen Mathematiker des Einzigen Staates. Meine an Zahlen
gewöhnte Feder vermag keine Musik aus Assonanzen und Rhythmen zu
schaffen. Ich kann nur das wiedergeben, was ich sehe, was ich
denke, genauer gesagt, was WIR denken. WIR — das ist das richtige
Wort, und deshalb sollen meine Aufzeichnungen den Titel WIR
tragen.

Aber sind sie nicht eine von unserem Leben, von dem mathematisch
vollkommenen Leben des Einzigen Staates abgeleitete Größe, und wenn
das stimmt, müssen sie nicht ganz von selber zum Poem werden? Ja,
sie müssen es, ich glaube, ich weiß es.

Ich schreibe diese Zeilen und fühle meine Wangen dabei glühen. Das
ist wahrscheinlich das gleiche, was eine Frau empfindet, wenn sie
zum ersten Mal den Herzschlag eines neuen, noch winzig kleinen
Menschenwesens in sich spürt. Dieses Werk — das bin ich, und doch
bin ich es nicht. Viele Monate noch muss ich es mit meinem Blut
nähren, bevor ich es unter Schmerzen gebären und dem Einzigen Staat
darbringen kann. Aber ich bin bereit wie jeder von uns, oder fast
jeder.

\section{EINTRAGUNG NR. 2}

\uebersicht{\emph{Übersicht:} Das Ballett. Die quadratische
Harmonie. X.}
Frühling. Aus der wilden, unbekannten Weite jenseits der Grünen
Mauer weht der Wind gelben Blütenstaub herüber. Dieser süßliche
Staub macht die Lippen trocken — man muss sie alle Augenblicke mit
der Zunge anfeuchten —, alle Frauen, die mir begegnen, haben diese
süßen Lippen (die Männer natürlich auch). Das verwirrt das logische
Denken ein wenig.

Doch was für ein Himmel! Tiefblau, von keiner einzigen Wolke
befleckt (was für einen jämmerlichen Geschmack müssen unsere
Vorfahren gehabt haben, wenn diese dummen, unförmigen Dampfklumpen
ihre Dichter begeistern konnten). Ich liebe einen sterilen,
peinlich sauberen Himmel. Nicht ich allein, wir alle, ich täusche
mich nicht, lieben ihn. An einem Tag wie heute ist die ganze Welt
aus dem unzerbrechlichen ewigen Glas gegossen, aus dem die Grüne
Mauer und alle unsere Gebäude bestehen. An solchen Tagen sieht man
die blauste Tiefe dieser Dinge, nimmt unbekannte Größen, wunderbare
Gleichungen wahr — man entdeckt sie im Allergewöhnlichsten,
Alltäglichsten\ldots{}

Heute morgen zum Beispiel war ich auf der Werft, wo der Integral
gebaut wird. Plötzlich fiel mein Blick auf die Maschinen. Mit
geschlossenen Augen, selbstvergessen, drehten sich die Kugeln der
Regulatoren. Die blitzenden Hebel neigten sich nach rechts und nach
links, stolz wiegte sich die Balancierstange in den Schultern, der
Meißel der Stemmmaschine knirschte im Takt einer unhörbaren Musik.
Da ging mir die Schönheit dieses prächtigen, von bläulichem
Sonnenlicht überfluteten Maschinenballetts auf.

Unwillkürlich fragte ich mich dann: Warum ist das schön? Warum ist
der Tanz schön? Die Antwort: Weil er eine unfreie, eine gebundene
Bewegung ist, weil sein tieferer Sinn die vollkommene ästhetische
Unterwerfung, die ideale Unfreiheit ist. Wenn es stimmt, dass
unsere Ahnen in Augenblicken der höchsten Begeisterung sich dem
Tanz hingaben (religiöse Mysterien, Militärparaden), dann kann das
nur das eine bedeuten: der Trieb zur Unfreiheit ist dem Menschen
angeboren, und wir in unserem heutigen Leben tun nur bewusst\ldots{}

Ich werde unterbrochen, in meinem Numerator ist eine Klappe
gefallen. Ich blicke auf: O-90, natürlich. In einer halben Minute
ist sie bei mir, sie will mich zum Spaziergang abholen.

Die liebe O! Ich fand schon immer, dass sie genau wie ihr Name
aussieht: sie ist zehn Zentimeter unter der Mutternorm, ganz rund,
wie gedrechselt, und bei jedem Wort, das sie sagt, formt ihr Mund
ein rosiges O. Am Handgelenk hat sie tiefe runde Grübchen wie ein
Kind. Als sie in mein Zimmer kam, kreiste das Schwungrad der Logik
noch in mir, und das Trägheitsgesetz wollte es, dass ich O von der
Formel erzählte, die ich eben gefunden hatte, die Formel, die alles
umfasst, uns, die Maschinen und den Tanz. „Wunderbar, nicht wahr?“
fragte ich. „Ja, wunderbar, der Frühling!“ antwortete O mit
strahlendem Lächeln.

So etwas! Der Frühling\ldots{} sie redet vom Frühling! Ach, diese
Frauen\ldots{} Ich schwieg.

Drunten auf der Straße. Der Prospekt ist von Leben erfüllt: bei
solchem Wetter verwenden wir unsere persönliche Stunde nach dem
Mittagessen gewöhnlich zu einem Ausgleichsspaziergang. Wie immer
erklang aus
sämtlichen Lautsprechern der Musikfabrik der Marsch des Einzigen
Staates. In mustergültig ausgerichteten Viererreihen marschierten
die Nummern im Takt zu den feierlichen Klängen — Hunderte,
Tausende, alle in blaugrauen Uniformen, mit goldenen Abzeichen an
der Brust — die uns vom Staat gegebene Nummer, die wir tragen. Und
ich — wir vier in dieser Reihe, wir sind nur eine der unzähligen
Wellen des gewaltigen Stromes. Zu meiner Linken geht O-90 (wenn
einer meiner behaarten Ahnen diese Aufzeichnungen vor tausend
Jahren geschrieben hätte, dann hätte er vielleicht „meine O-90“
gesagt), rechts zwei andere, mir unbekannte Nummern, eine weibliche
und eine männliche.

Strahlendes Glück des blauen Himmels, die goldenen Abzeichen
blinken wie winzige Sonnen, nirgends ein Gesicht, das verdüstert
ist, überall heller Glanz, alles aus einer leuchtenden, lächelnden
Materie gewoben. Und die ehernen Takte: Tra-ta-ta-tam,
tra-ta-ta-tam, sind sonnenbeglänzte eherne Stufen, mit jeder Stufe
steigt man hinauf, immer höher hinauf ins schwindelnde Blau\ldots{}
Plötzlich sah ich alle Dinge wieder so wie heute morgen auf der
Werft. Mir war, als erblickte ich dies alles zum ersten Mal in
meinem Leben: die schnurgeraden Straßen, das lichtfunkelnde Glas
des Straßenpflasters, die langgestreckten Kuben der durchsichtigen
Wohnhäuser, die quadratische Harmonie der blaugrauen Marschblöcke.
Nicht eine Generation nach der anderen war nötig gewesen: ich
allein hatte den alten Gott und das alte Leben besiegt. Ich hatte
das alles geschaffen, ich war wie ein Turm, und ich wagte nicht,
die Ellbogen zu bewegen, damit die Mauern, Kuppeln und Maschinen
nicht einstürzten und zersplitterten \ldots{} Im nächsten Augenblick —
ein Sprung durch die Jahr-
hunderte, von Plus zu Minus. Mir fiel ein Bild im Museum ein
(wahrscheinlich eine Assoziation der Kontraste): eine Straße des
20. Jahrhunderts, ein verwirrend buntes Gewühl von Menschen,
Rädern, Tieren, Plakaten, Bäumen, Farben und Vögeln\ldots{} Aber das hat
es tatsächlich gegeben! Mir erschien das alles so unwahrscheinlich
und absurd, dass ich mich nicht beherrschen konnte und in lautes
Gelächter ausbrach. Sogleich kam das Echo — ein Lachen zu meiner
Rechten. Ich blickte nach rechts und sah weiße, ungewöhnlich weiße,
scharfe Zähne im Gesicht einer mir unbekannten Frau.

„Verzeihen Sie“, sagte sie, „aber Sie haben alles so entzückt
betrachtet wie ein gewisser mythischer Gott an seinem siebten
Schöpfungstag. Sie sehen aus, als wären Sie sicher, dass Sie und
kein anderer auch mich geschaffen haben. Sehr schmeichelhaft für
mich\ldots{} “ All das sagte sie ganz ernst, fast mit einem gewissen
Respekt (vielleicht wusste sie, dass ich der Konstrukteur des
Integral bin). Und dennoch — in ihren Augen oder in ihren Brauen
war ein merkwürdig aufreizendes X; ich konnte diese Unbekannte
nicht erfassen, sie nicht in Zahlen ausdrücken. Ich war sehr
verlegen und versuchte verwirrt, mein Lachen logisch zu begründen.
Es sei völlig klar, sagte ich, dass dieser Kontrast, dass diese
unüberbrückbare Kluft zwischen der Gegenwart und der
Vergangenheit\ldots{} „Aber warum soll diese Kluft unüberbrückbar sein?“
unterbrach sie mich. Wie weiß ihre Zähne waren! „Man kann eine
Brücke über sie schlagen. Stellen Sie sich vor: Trommeln,
Bataillone, Menschen in Reih und Glied — das hat es auch damals
gegeben, folglich\ldots{} Nun, das ist doch ganz klar!“ rief sie. (Was
für eine seltsame Gedankenübertragung: sie gebrauchte die gleichen
Worte, die ich vor dem Spaziergang niedergeschrieben
hatte!) „Sehen Sie“, sagte ich, „wir haben die gleichen Gedanken.
Wir sind eben keine Einzelwesen mehr, sondern jeder von uns ist nur
einer von vielen. Wir gleichen einander so sehr\ldots{} “ „Sind Sie ganz
sicher?“

Ihre hochgezogenen Brauen bildeten einen spitzen Winkel zur Nase,
es sah aus wie ein exakt gezeichnetes X, und das verwirrte mich von
neuem. Ich blickte nach rechts, nach links, wieder nach rechts\ldots{}
Da schritt sie, schlank, sehnig, geschmeidig wie eine Gerte, I-330
(jetzt erst sah ich ihre Nummer); links ging O, die so ganz anders
war, nur aus Kreisen und Kurven zu bestehen schien, und am Ende
unserer Reihe eine mir unbekannte männliche Nummer — zweifach
gekrümmt wie ein S. Keiner glich dem anderen. Wir waren alle
verschieden\ldots{} Die I-330 hatte offenbar meinen zerstreuten Blick
bemerkt, denn sie sagte seufzend: „O weh!“ Dieses „O weh“ war
durchaus angebracht, doch wieder war etwas in ihrem Gesicht oder in
ihrer Stimme\ldots{} Ich entgegnete scharf: „Kein O weh! Die
Wissenschaft schreitet voran, und es ist klar, dass wir alle, wenn
auch nicht jetzt, so doch in fünfzig oder hundert Jahren\ldots{} “ „Dass
wir dann alle die gleichen Nasen haben\ldots{} “ „Ja, die gleichen
Nasen!“ Ich schrie es fast. „Denn die Verschiedenheit der Nasen ist
ein Grund zum Neid\ldots{} Wenn ich eine Knollennase habe, und ein
anderer\ldots{} “ „Was wollen Sie? Ihre Nase ist geradezu klassisch, wie
man früher sagte. Aber wie ist es mit Ihren Händen\ldots{} Zeigen Sie
mir einmal Ihre Hände. Zeigen Sie sie doch!“ Ich kann es nicht
ausstehen, wenn man meine Hände betrachtet. Sie sind dicht behaart,
haben einen richtigen Pelz. Das ist ein verrückter Atavismus. Ich
hielt ihr meine Hände hin und sagte gleichgültig: „Affenhände.“

Sie sah sie an und dann mein Gesicht. „Das ist wirklich ein
interessanter Zusammenklang.“ Sie maß mich mit einem abschätzenden
Blick und zog wieder die Brauen hoch.

„Er ist auf mich eingetragen“, flöteten die rosigen Lippen O.s
voller Stolz.

Sie hätte lieber schweigen sollen. Ihre Bemerkung war überflüssig.
Überhaupt, die liebe gute O\ldots{} wie soll ich nur sagen\ldots{} Es stimmt
etwas nicht mit der Geschwindigkeit ihrer Zunge. Die
Sekundengeschwindigkeit der Zunge muss stets ein wenig geringer
sein als die Sekundengeschwindigkeit des Denkens; umgekehrt ist es
von Übel. Vom Akkumulatorenturm am Ende des Prospekts schlug die
Uhr fünf. Die persönliche Stunde war um. I-330 ging mit jener
S-ähnlichen männlichen Nummer fort. Er hat ein Achtung gebietendes
Gesicht, das mir irgendwie bekannt vorkommt. Ich war ihm schon
begegnet, ich konnte mich im Augenblick nur nicht darauf besinnen,
wo. Beim Abschied lächelte I mir rätselhaft zu. „Schauen Sie doch
morgen einmal im Auditorium 112 herein“, sagte sie.

Ich zuckte die Achseln: „Wenn ich eine Order erhalte, ich meine,
für dieses Auditorium, das Sie mir genannt haben\ldots{} “

Mit einer Bestimmtheit, die mir unverständlich war, sagte sie: „Sie
werden eine Order bekommen.“ Diese Frau wirkte auf mich ebenso
unangenehm wie ein unlösbares irrationales Glück, das unvermutet in
einer Gleichung auftaucht, und ich war froh, dass ich mit der
lieben O, wenn auch nur für kurze Zeit, allein blieb. Arm in Arm
gingen wir bis zu der vierten Straßenkreuzung. An der Ecke musste
sie links, ich rechts abbiegen. „Ich würde heute so gern zu Ihnen
kommen und die

Gardinen herunterlassen. Gerade heute, jetzt, in diesem
Augenblick\ldots{} “, sagte O und sah mich schüchtern mit ihren runden,
kristallblauen Augen an. Was sollte ich dazu sagen? Erst gestern
war sie bei mir gewesen, und sie wusste genauso gut wie ich, dass
unser nächster Geschlechtstag erst übermorgen war. Ihre Zunge war
wieder einmal schneller als ihr Denken, ähnlich der (manchmal so
schädlichen) Frühzündungen eines Motors. Zum Abschied küsste ich
sie zweimal, nein, ich will genau sein, dreimal auf ihre
wundervollen blauen, von keiner Wolke getrübten Augen.

\section{EINTRAGUNG NR. 3}

\uebersicht{\emph{Übersicht:} Der Rock. Die Mauer. Die
Gesetzestafel.}
Ich habe meine gestrigen Aufzeichnungen noch einmal durchgelesen,
und es kommt mir so vor, als ob ich mich nicht klar genug
ausgedrückt hätte. Uns Nummern ist das alles freilich sonnenklar.
Doch wer weiß, vielleicht haben Sie, unbekannte Leser, denen der
Integral meine Aufzeichnungen bringen wird, das große Buch der
Zivilisation nur bis zu der Seite gelesen, bei der unsere Vorfahren
vor 900 Jahren stehen geblieben sind. Es ist durchaus möglich, dass
Sie nicht einmal solch elementare Dinge wie die
Stunden-Gesetzestafel, die Persönlichen Stunden, die Mutternorm,
die Grüne Mauer und den Wohltäter kennen. Ich finde es lächerlich
und zugleich sehr schwierig, Ihnen dies alles auseinanderzusetzen.
Es ist genauso, wie wenn ein Schriftsteller, nun, sagen wir des 20.
Jahrhunderts, in seinem Roman erklären müsste, was ein Rock, eine
Wohnung, eine Gattin ist. Übrigens, wenn
sein Buch für unzivilisierte Völker übersetzt wurde, dann konnte
man kaum ohne eine Anmerkung zu dem Wort Rock auskommen.

Wenn der Wilde Rock las, dachte er gewiss: „Wozu das? Das ist doch
nur eine Last.“ Ich glaube, auch Sie werden große Augen machen,
wenn ich Ihnen sage, dass seit dem 200jährigen Krieg keiner von uns
in dem Land jenseits der Grünen Mauer gewesen ist.

Aber, verehrter Leser, denken Sie nur einmal ein wenig nach: Die
ganze Geschichte der Menschheit, soweit wir sie kennen, ist die
Geschichte des Übergangs vom Nomadentum zu wachsender
Sesshaftigkeit. Daraus folgt, dass die Lebensform der zähesten
Sesshaftigkeit (nämlich die unsere) auch die vollkommenste ist
(wiederum die unsere). Nur in prähistorischen Zeiten, als es noch
Nationen, Kriege und Handel gab, als mehr als nur ein Amerika
entdeckt wurde, zogen die Menschen sinn- und planlos von einem Ende
der Welt zum anderen. Aber wozu, wer braucht das jetzt noch?

Ich gebe zu, die Gewöhnung an diese Sesshaftigkeit wurde nicht
sofort und auch nicht ohne Mühe erreicht. Im 200jährigen Krieg, als
alle Landstraßen zerstört und mit Gras überwuchert waren, musste es
anfangs recht unangenehm sein, in Städten zu leben, die durch grüne
Einöden voneinander abgeschnitten waren. Aber was hat das schon zu
bedeuten? Nachdem der Mensch seinen Affenschwanz verloren, hat er
wahrscheinlich auch nicht sofort gelernt, die Fliegen ohne dieses
Hilfsmittel zu verjagen. Anfangs kam er sich ohne Schwanz
zweifellos sehr kläglich vor. Anfangs hat er seinen Schwanz
bestimmt schmerzlich vermisst. Jetzt aber — können Sie sich
vorstellen, dass Sie einen Schwanz hätten? Oder dass Sie nackt auf
der Straße herumliefen, ohne Rock (vielleicht
tragen Sie noch einen Rock) Mir geht es ebenso: Ich kann mir
keine Stadt ohne die Grüne Mauer denken, kein Leben, das nicht in
das Zahlengewand der Gesetzestafel gekleidet ist.

Die Gesetzestafel\ldots{} Von der Wand meines Zimmers blicken ihre
purpurnen Zahlen auf goldenem Grund mir wohlwollend-streng in die
Augen. Unwillkürlich muss ich an das denken, was die Alten Ikone
nannten, und ich möchte Verse schreiben oder beten (was übrigens
das gleiche ist). Ach, warum bin ich kein Dichter, um dich würdig
zu preisen, o Gesetzestafel, du Herz und Puls des Einzigen Staates!
Wir alle (vielleicht auch Sie) haben schon als Schulkinder das
größte aller uns erhaltenen Denkmäler der alten Literatur gelesen,
den Eisenbahnfahrplan. Vergleichen Sie ihn einmal mit der
Gesetzestafel, und Sie werden sehen: Das eine ist Graphit, das
andere Diamant, beide bestehen aus dem gleichen Element, C,
Kohlenstoff, aber wie durchsichtig-klar ist der Diamant, wie
leuchtet er! Ihnen geht gewiss der Atem aus, wenn Sie die Seiten
des Fahrplans entlangjagen. Die Stunden-Gesetzestafel hingegen
verwandelt jeden von uns in einen stählernen sechsrädrigen Helden
des großen Poems. Jeden Morgen stehen wir, Millionen, wie ein Mann
zu ein und derselben Stunde, zu ein und derselben Minute auf. Zu
ein und derselben Stunde beginnen wir, ein Millionenheer, unsere
Arbeit, zur gleichen Stunde beenden wir sie. Und zu einem einzigen,
millionenhändigen Körper verschmolzen, führen wir in der gleichen,
durch die Gesetzestafel bestimmten Sekunde die Löffel zum Mund, zur
gleichen Sekunde gehen wir spazieren, versammeln uns zu den
Taylor-Exerzitien in den Auditorien, legen uns schlafen \ldots{} Ich
will ganz offen sein: Die absolute, endgültige Lösung
des Problems Glück haben selbst wir noch nicht gefunden: Zweimal am
Tag, von 16 bis 17 und von 21 bis 22 Uhr, spaltet sich der
gewaltige Organismus in einzelne Zellen auf — das sind die von der
Gesetzestafel festgesetzten Persönlichen Stunden. Zu dieser Zeit
sehen Sie folgendes Bild: Die einen sitzen hinter geschlossenen
Gardinen in ihren Zimmern, andere gehen im Takt zu den ehernen
Klängen des Marsches auf dem Prospekt spazieren, wieder andere
sitzen am Schreibtisch, wie ich in diesem Augenblick. Aber ich
glaube — man mag mich einen Idealisten oder Phantasten nennen —,
ich glaube ganz fest daran, dass wir irgendwann, früher oder
später, auch für diese zwei Stunden einen Platz in der allgemeinen
Formel finden werden, dass dann die Gesetzestafel sämtliche 86.400
Sekunden des Tages umfassen wird. Viel Unwahrscheinliches habe ich
von jenen Zeiten gelesen und gehört, da die Menschen noch in
Freiheit, nicht organisiert und wie die Wilden lebten. Aber am
unbegreiflichsten war es mir immer, wie der damalige Staat, so
unvollkommen er auch gewesen sein mag, es dulden konnte, dass die
Menschen ohne Verordnungen lebten, die denen unserer Gesetzestafel
vergleichbar wären, ohne Pflichtspaziergänge, ohne genau
festgelegte Essenszeiten, dass sie aufstanden und zu Bett gingen,
wann es ihnen gerade einfiel; einige Historiker berichten sogar,
dass damals die ganze Nacht Lampen in den Straßen brannten, dass
die Leute nachts durch die Straßen gingen und fuhren.

Ich kann das einfach nicht fassen. Wie beschränkt ihre Einsicht
auch war, sie mussten doch erkennen, dass dieses Leben Selbstmord
war, ein langsamer Selbstmord. Der Staat (die Humanität) verbot,
einen Menschen zu töten, verbot aber nicht, Millionen umzubringen.
Einen zu töten, das heißt die Summe aller Menschenleben um 50 Jahre
zu verringern, war ein Verbrechen, aber die gleiche Summe um 50
Millionen Jahre zu verringern, war keines. Ist das nicht
lächerlich?

Jede beliebige zehnjährige Nummer unseres Staates kann dieses
mathematisch-moralische Problem in einer halben Minute lösen; sie
aber vermochten es nicht, nicht einmal all ihre Kants zusammen
(weil keiner dieser Kants draufkam, ein System wissenschaftlicher
Ethik zu schaffen, einer Ethik nämlich, die auf Substraktion,
Addition, Division und Multiplikation beruht).

Und ist es nicht absurd, dass der Staat von damals (dieses Gebilde
wagte sich Staat zu nennen) das Geschlechtsleben ohne jegliche
Kontrolle ließ? Die Menschen konnten sich vergnügen, wann und wie
sie wollten, und sie zeugten Kinder wie die Tiere, in blinder Lust,
ohne sich um die Lehren der Wissenschaft zu kümmern. Ist das nicht
lächerlich: Sie kannten sich in Gartenbau, Geflügelzucht,
Fischzucht aus (wir haben zuverlässige Quellen darüber) und
vermochten dennoch nicht, die letzte Sprosse dieser logischen
Leiter zu erklimmen: die Kinderzucht. Sie kamen nicht auf unsere
Vater- und Mutternorm. Alles, was ich bis jetzt geschrieben habe,
ist so töricht und unwahrscheinlich, dass Sie, unbekannter Leser,
mich vielleicht für einen üblen Witzbold halten. Sie werden denken,
dass ich mich über Sie lustig mache und mit todernster Miene den
größten Unsinn von mir gebe.

Aber erstens bin ich gar nicht fähig, einen Witz zu machen — jeder
Witz ist eine unklare Funktion, also eine Lüge —, und zweitens
behauptet die Wissenschaft des Einzigen Staates, dass das Leben
unserer Vorfahren so und nicht anders war, und die Wissenschaft des
Einzigen Staates kann sich nicht irren. Woher hätte damals, als
die Menschen in Freiheit, nämlich wie Tiere, wie Affen in Herden,
lebten, die Staatslogik herkommen sollen? Was konnte man von ihnen
erwarten, wenn man sogar noch in unseren Tagen irgendwoher aus der
Tiefe, aus dem wilden Abgrund, das wilde Echo des Affen vernimmt?
Zum Glück vernehmen wir es nur selten. Zum Glück sind das nur
unbedeutende kleine Schäden, die wir leicht beheben können, ohne
den ewigen Lauf der ganzen Maschine zu stoppen. Wenn wir einen
verbogenen Bolzen entfernen müssen — dazu haben wir die geschickte,
starke Hand des Wohltäters und die scharfen Augen der Beschützer\ldots{}
Übrigens, da fällt mir ein, diese S-ähnliche Nummer von gestern
habe ich, glaube ich, einmal aus dem Beschützeramt herauskommen
sehen. Jetzt begreife ich, warum ich unwillkürlich Ehrfurcht vor
ihm empfand und warum mir so unbehaglich zumute wurde, als die
sonderbare I-300 in seiner Gegenwart\ldots{} Ich muss gestehen, diese
I\ldots{} Es läutet zum Schlafengehen: 22.30 Uhr. Bis morgen.

\section{EINTRAGUNG NR. 4}

\uebersicht{\emph{Übersicht:} Der Wilde und das Barometer.
Epilepsie. Wenn\ldots{}}
Bis zum heutigen Tag war mir alles im Leben völlig klar (ich habe
wohl nicht zufällig eine gewisse Vorliebe für das Wort klar). Heute
aber\ldots{} Ich kann es nicht fassen.

Erstens: Ich habe tatsächlich Order erhalten, zum Auditorium 112 zu
kommen, wie sie mir sagte. Obgleich die Wahrscheinlichkeit dafür
nur 1500 : 10,000.000 = 3 : 20.000 war (1500 = Anzahl der
Auditorien, 10,000.000 = Anzahl der Nummern). Und drittens\ldots{} Aber
ich will alles der Reihe nach erzählen. Das Auditorium. Eine
riesige, sonnendurchglühte Halbkugel aus massivem Glas. Zahllose
kugelförmige, glattrasierte Köpfe. Ich blickte mich etwas beklommen
um. Ob hier nicht irgendwo über den blauen Wogen der Uniformen ein
rosiger Halbmond schwebte, die lieben Lippen von O? Da — eine Reihe
ungewöhnlich weißer, scharfer Zähne\ldots{} Nein, es war etwas anderes,
das ich suchte. Heute Abend um 21 Uhr wird O zu mir kommen; der
Wunsch, sie hier zu sehen, war also ganz natürlich. Ein
Klingelzeichen ertönte. Wir erhoben uns, sangen die Hymne des
Einzigen Staates, und auf dem Podium begann der goldfunkelnde
Lautsprecher des Phonolektors: „Verehrte Nummern! Vor kurzem haben
die Archäologen ein Buch aus dem 20. Jahrhundert ausgegraben. Der
Autor erzählt darin von einem Wilden und einem Barometer. Der Wilde
hatte entdeckt, dass, sooft das Barometer auf Regen stand, es
tatsächlich regnete. Da der Wilde Regen haben wollte, kratzte er so
viel Quecksilber heraus, bis das Barometer auf Regen stehen blieb.“
Auf der Leinwand sah man einen federgeschmückten Wilden, der das
Quecksilber aus dem Barometer entfernte. Gelächter. „Sie lachen,
aber meinen Sie nicht auch, dass der Europäer jener Epoche weit
lächerlicher war als dieser Wilde? Der Europäer begehrte ebenfalls
Regen, aber wie hilflos war er dem Barometer gegenüber! Der Wilde
hingegen besaß Mut, Energie und Logik, wenn auch eine recht wilde
Logik: er stellte fest, dass es eine Verbindung zwischen Ursache
und Wirkung gibt. Indem er das Quecksilber herauskratzte, tat er
den ersten Schritt auf jenem großen Wege, den wir\ldots{} “

Hier (ich wiederhole, ich will in diesen Aufzeichnungen die volle
Wahrheit sagen), hier wurde ich gleichsam wasserdicht,
undurchdringlich für die belebenden Ströme, die dem Lautsprecher
entquollen. Plötzlich war mir, als wäre es sinnlos, dass ich
hierher gekommen war (wieso sinnlos? Ich musste kommen, ich hatte
ja den Befehl erhalten!). Alles erschien mir leer und hohl. Mit
großer Mühe gelang es mir, mich wieder zu konzentrieren, als der
Phonolektor bereits zum Hauptthema gekommen war, zu unserer Musik,
zur mathematischen Komposition (die Mathematik ist die Ursache, die
Musik die Wirkung), zur Beschreibung des kürzlich erfundenen
Musikometers. “\ldots{} Man dreht einfach an diesem Knopf und kann bis
zu drei Sonaten in der Stunde komponieren. Welche Mühe machte das
Ihren Vorfahren! Sie konnten nur dann schaffen, wenn sie sich in
einen krankhaften Zustand, in
\glq{}Begeisterung\grq{}, versetzten, was nichts
anderes ist als eine Form der Epilepsie. Ich gebe Ihnen jetzt ein
äußerst komisches Beispiel von dem, was man damals zuwege brachte.
Sie hören Musik von Skrjabin, 20. Jahrhundert. Diesen schwarzen
Kasten“ — der Vorhang auf dem Podium teilte sich, wir sahen ein
altmodisches Musikinstrument — „diesen Kasten nannte man damals
Flügel, was wiederum beweist, wie sehr ihre ganze Musik .,.“ Was er
dann sagte, habe ich vergessen, wohl deshalb, weil\ldots{} nun, ich will
es offen gestehen, weil sie, I-330, zu dem schwarzen Kasten ging.
Wahrscheinlich hatte mich ihr unerwartetes Erscheinen auf der Bühne
verwirrt. Sie trug ein seltsames Kostüm, wie es damals Mode war,
ein enganliegendes schwarzes Kleid; es betonte das Weiß der
entblößten Schultern und Brüste und den warmen zuckenden Schatten
dazwischen\ldots{} und ihre blendend weißen, fast bösen Zähne \ldots{}

Sie lächelte uns zu. Ein bleckendes, beißendes Lächeln. Dann setzte
sie sich und begann zu spielen. Es klang exaltiert, wild und wirr,
wie alles aus jener Zeit — bar der Vernunft des Mechanischen. Und
alle, die hier saßen, hatten recht: sie lachten. Nur einige wenige
.. aber warum auch ich\ldots{} ich?

Ja, die Epilepsie ist eine Geisteskrankheit, ein Schmerz\ldots{} ein
brennender, süßer Schmerz, wie ein Biss, und ich will, dass er
tiefer in mich eindringt, dass ich ihn noch stärker spüre. Und da
geht langsam die Sonne auf. Nicht unsere Sonne, die mit
kristallblauem, gleichmäßigem Schein durch die gläsernen Wände
dringt, nein, eine wilde, unaufhaltsam dahinjagende, alles
versengende Sonne — nichts mehr bleibt von mir —, alles zerfällt in
kleine Fetzen\ldots{} Die Nummer links von mir sah mich kichernd an. Ich
kann mich noch deutlich erinnern, dass an seinen Lippen ein
winziges Speichelbläschen hing und zerplatzte. Dieses Bläschen
ernüchterte mich. Ich war wieder ich. Wie die anderen hörte auch
ich nur noch das wirre, tosende Rauschen der Saiten. Ich lachte,
und alles war plötzlich so leicht und einfach. Was war geschehen?
Nur dies: Der Phonolektor hatte jene unzivilisierte Epoche
heraufbeschworen. Mit welchem Genuss lauschte ich dann unserer
zeitgenössischen Musik (sie wurde zum Schluss als Kontrast
gespielt). Die kristallenen chromatischen Tonleitern ineinander
verschmelzender und sich wieder lösender unendlicher Reihen, die
Akkorde der Formeln Taylors und MacLaurins, die schweren
Ganztonschritte der quadratischen Pythagorashosen, die
schwermütigen Melodien verebbender Schwingungsbewegungen\ldots{} Welch
erhabene Größe! Welch unerschütterliche Gesetzesmäßigkeit! Wie
kümmerlich wirkte dagegen die eigenwillige, sich nur in wilden
Phantasien ergehende Musik unserer Vorfahren!

Wie sonst gingen alle in Viererreihen durch die breiten Türen des
Auditoriums hinaus. Eine mir wohlbekannte, zweifach gekrümmte
Gestalt huschte vorüber; ich grüßte respektvoll.

In einer Stunde würde O zu mir kommen. Ich war in einem Zustand
angenehmer und zugleich nützlicher Erregung. Zu Hause ging ich
sofort zur Hausverwaltung, zeigte mein rosa Billett vor und erhielt
die Genehmigung, die Vorhänge herabzulassen. Dieses Recht haben wir
nur an Geschlechtstagen. Sonst leben wir in unseren durchsichtigen,
wie aus leuchtender Luft gewebten Häusern, ewig vom Licht umflutet.
Wir haben nichts voreinander zu verbergen, und außerdem erleichtert
diese Lebensweise die mühselige, wichtige Arbeit der Beschützer.
Wäre es anders, was könnte dann alles geschehen! Gerade die
sonderbaren, undurchsichtigen Behausungen unserer Vorfahren können
es bewirkt haben, dass man auf diese erbärmliche Käfigpsychologie
verfiel: „Mein Haus ist meine Burg!“

Um 22 Uhr ließ ich die Vorhänge herunter, und da trat O auch schon
ins Zimmer. Sie war ein wenig außer Atem und hielt mir ihr rosiges
Mündchen und ihr rosa Billett hin. Ich riss den Talon ab — und
dann\ldots{} Erst im allerletzten Augenblick, um 22.15 Uhr, löste ich
mich von dem rosigen Mund.

Ich zeigte ihr meine Aufzeichnungen und sprach von der Schönheit
des Quadrats, des Würfels und der Geraden, wobei ich mich exakt und
gewählt ausdrückte. Sie hörte schweigend zu, und plötzlich tropften
Tränen aus ihren blauen Augen und fielen auf mein Manuskript (Seite
7). Die Tinte färbte sich wasserblau und zerfloss. Ich muss die
Seite also noch einmal schreiben. „Lieber D, wenn Sie nur\ldots{}
wenn\ldots{} “

„Was, wenn?“

Die alte Leier; sie möchte ein Kind haben. Oder ist es vielleicht
etwas Neues, weil\ldots{} weil jene andere\ldots{} ? Sie ist freilich\ldots{}
Nein, das kann nicht sein, es wäre zu unsinnig.

\section{EINTRAGUNG NR. 5}

\uebersicht{\emph{Übersicht:} Das Quadrat. Die Herren der Welt.
Eine angenehm-nützliche Funktion.}
Wieder drücke ich mich unklar aus, wieder spreche ich mit Ihnen,
lieber Leser, als wären Sie — nun, sagen wir, mein alter
Schulfreund R-13, der Dichter mit den wulstigen Negerlippen, den
alle kennen. Sie aber leben auf dem Mond, auf der Venus, auf dem
Mars oder dem Merkur, wer weiß, wo und wer Sie sind.

Also, stellen Sie sich ein Quadrat vor, ein lebendiges, schönes
Quadrat. Und es soll von sich, von seinem Leben erzählen. Sehen
Sie, dem Quadrat würde es nie einfallen, davon zu sprechen, dass
alle seine vier Seiten gleich sind, das sieht es schon gar nicht
mehr, so selbstverständlich erscheint es ihm. Ich bin die ganze
Zeit in einer ähnlichen Lage. Nehmen wir zum Beispiel die rosa
Billetts und alles, was damit zusammenhängt: für mich ist das so
selbstverständlich wie die vier gleichen Seiten für das Quadrat,
Ihnen jedoch erscheint es vielleicht raffinierter als ein
Newtonsches Binom.

Nun, hören Sie zu. Irgendein alter Philosoph hat, natürlich rein
zufällig, ein kluges Wort gesagt: „Liebe und Hunger regieren die
Welt.“ Ergo: um die Welt zu beherrschen, muss der Mensch die
Beherrscher der Welt bezwingen.
Unsere Vorfahren haben einen hohen Preis gezahlt, um den
Hunger auszurotten, ich meine den 200jährigen Krieg, den Krieg
zwischen Stadt und Land. Wahrscheinlich hielten die wilden Heiden
nur aus religiösen Vorurteilen hartnäckig an ihrem Brot fest.
(Dieses Wort wird heute nur noch als Metapher gebraucht, die
chemische Zusammensetzung dieses Stoffes ist uns nicht bekannt.)
Aber fünfunddreißig Jahre vor der Gründung des Einzigen Staates
wurde unsere heutige Naphtha-Nahrung erfunden. Es waren freilich
nur 0,2 Prozent der Bevölkerung der Erde übrig geblieben. Doch
dafür erstrahlte das von tausendjährigem Schmutz gereinigte Antlitz
der Erde in neuem, ungeahntem Glanz, und diese 0,2 Prozent genossen
das Glück im Paradies des Einzigen Staates. Es bedarf wohl keiner
Erklärung, dass Glück und Neid Zähler und Nenner jenes Bruches
sind, den wir Zufriedenheit nennen. Welchen Sinn hätten die
unzähligen Opfer des 200jährigen Krieges gehabt, wenn es in unserem
Leben noch immer einen Grund zum Neid gäbe? Und doch existiert er
noch, da es immer noch „Knollennasen“ und „klassische Nasen“ gibt
(ich erinnere an das Gespräch auf dem Spaziergang), weil viele um
die Liebe der einen werben, während um die andere sich keiner
kümmert.

Nachdem der Einzige Staat den Hunger besiegt hatte, führte er einen
Krieg gegen den zweiten Beherrscher der Welt, die Liebe.
Schließlich war auch dieser Feind geschlagen, das heißt,
organisiert, mathematisch festgelegt, und vor rund 300 Jahren trat
unsere Lex sexualis in Kraft. Jede Nummer hat ein Recht auf eine
beliebige Nummer als Geschlechtspartner.

Alles weitere war dann nur noch Technik. In den Laboratorien des
Amtes für sexuelle Fragen wird man sorgfältig
untersucht, der Gehalt an Geschlechtshormonen wird genau bestimmt,
und dann erhält jeder eine seinen Bedürfnissen entsprechende
Tabelle der Geschlechtstage und die Anweisung, sich an diesen Tagen
der Nummer Soundso zu bedienen, und man händigt ihm zu diesem Zweck
ein Heftchen mit rosa Billetts aus.

So gibt es nun keinen Grund mehr zum Neid, denn der Nenner des
Bruches Zufriedenheit ist Null geworden — und der Bruch wird zur
großartigen Unendlichkeit. Das, was bei unseren Vorfahren eine
Quelle unzähliger, sinnloser Tragödien war, haben wir zu einer
harmonischen, angenehm-nützlichen Funktion gemacht, ebenso wie den
Schlaf, die körperliche Arbeit, die Nahrungsaufnahme, die Verdauung
und alles übrige. Darin zeigt sich, wie die große Kraft der Logik
alles reinigt, was sie berührt. Ach, mögen auch Sie, ferner
unbekannter Leser, diese göttliche Kraft erkennen und lernen, ihr
in allem zu folgen. Seltsam, ich habe heute von den Gipfelpunkten
der Menschheitsgeschichte geschrieben, ich habe die ganze Zeit die
reinste Höhenluft des Geistes geatmet, doch in mir selbst ist alles
düster, von dunklen Wolken, von Spinnweben verhangen, irgendein
vierfüßiges X hat von mir Besitz ergriffen. Vielleicht kommt das
nur von meinen Händen, ich hatte sie so lange vor Augen, meine
behaarten Hände, die wie Pfoten aussehen. Ich spreche nicht gern
von ihnen, ich liebe sie nicht, sie sind ein Überbleibsel aus jener
längst vergangenen, unzivilisierten Epoche\ldots{} Eigentlich wollte ich
dies alles ausstreichen, weil es nicht zum Thema gehört, aber dann
habe ich mich entschlossen, es doch stehenzulassen. Meine
Aufzeichnungen sollen wie ein Seismograph selbst die
geringfügigsten Schwankungen meines Gehirns registrieren, denn
mitunter sind solche Schwankungen eine Warnung\ldots{}

Nein, das ist ja absurd, ich hätte es durchstreichen müssen: wir
haben alle Elemente gebändigt, es kann keine Katastrophe mehr
geben.

Jetzt ist mir plötzlich alles ganz klar: dieses seltsame Gefühl
kommt nur von der sonderbaren Lage, in der ich mich Ihnen gegenüber
befinde. Es gibt kein X in mir (das ist unmöglich), im Gegenteil,
ich befürchte, dass in Ihnen irgendein X zurückbleibt, lieber
Leser. Aber ich glaube, Sie werden mich deswegen nicht verurteilen.
Sie werden verstehen, dass es für mich viel schwieriger ist, zu
schreiben, als es für alle Schriftsteller in der ganzen Geschichte
der Menschheit je gewesen ist. Die einen schrieben für ihre
Zeitgenossen, die anderen für ihre Nachkommen, aber keiner hat für
seine Vorfahren oder für Wesen geschrieben, die seinen ungesitteten
Ahnen aus grauer Vorzeit glichen\ldots{}

\section{EINTRAGUNG NR. 6}

\uebersicht{\emph{Übersicht:} Ein Zufall. Das verfluchte klar. 24
Stunden.}
Ich wiederhole: Ich habe es mir zur Pflicht gemacht, in meinen
Aufzeichnungen nichts zu verschweigen. Darum muss ich an dieser
Steile bemerken — so betrüblich es auch ist —, dass selbst in
unserem Staat der Prozess der Verhärtung, der Kristallisation des
Lebens noch nicht abgeschlossen ist. Wir sind noch einige Schritte
vom Ideal entfernt. Das Ideal ist dort, wo nichts mehr geschieht
(das ist klar), bei uns hingegen\ldots{} Was sagen Sie dazu: Heute las
ich in der Staatszeitung, dass übermorgen der Tag der Gerechtigkeit
auf dem Platz des Würfels stattfindet. Also hat wieder irgendeine
Nummer den Lauf der großen Staatsmaschine gehemmt, wieder ist etwas
Unvorhergesehenes, nicht Vorausberechnetes geschehen. Auch mit mir
ist etwas geschehen, zwar in der Persönlichen Stunde, das heißt, in
jener Zeitspanne, die für Unvorhergesehenes bestimmt ist,
dennoch\ldots{}

Ich kam um 16 Uhr, genau gesagt, 10 Minuten vor 16 Uhr, nach Hause.
Plötzlich rasselte das Telefon. „D-503?“ fragte eine weibliche
Stimme. „Ja.“

„Ich bin?s, I-330. Ich hole Sie gleich ab, wir gehen zusammen zum
Alten Haus. Einverstanden?“ I-330\ldots{} Diese I hat etwas
Aufreizendes, sie stößt mich ab, erschreckt mich fast. Aber gerade
darum sagte ich ja. Fünf Minuten später saßen wir im Flugzeug. Ein
majolikablauer Maihimmel. Die warme Sonne folgte uns summend in
ihrem goldenen Flugzeug, ohne uns einzuholen und ohne
zurückzubleiben. Aber dort, vor uns, hing eine milchig-weiße Wolke,
hässlich und rund wie die Backen eines antiken Cupido, und das
störte mich. Das vordere Fenster des Flugzeuges war geöffnet, ein
scharfer Wind wehte, meine Lippen wurden trocken. Unwillkürlich
feuchtete ich sie an und dachte an nichts anderes. In der Ferne
tauchten trüb-grüne Flecken auf, das Land jenseits der Mauer. Dann
erfasste mich ein leichter Schwindel, es ging hinab, immer tiefer,
einen steilen Hang hinunter: wir landeten vor dem Alten Haus. Das
morsche, düstere Gebäude ist rings von einer gläsernen Hülle
umschlossen, sonst wäre es natürlich längst eingestürzt. Vor der
Glastür hockte ein uraltes Weib; ihr Gesicht war über und über mit
Runzeln bedeckt, der Mund bestand nur aus Falten, er sah aus wie
zugewachsen, und man konnte nicht glauben, dass sie ihn noch öffnen
könne. Doch die Alte sagte:

„Na, Kinderchen, wollt ihr euch mein Häuschen ansehen?“, und ihr
verrunzeltes Gesicht strahlte. „Ja, Großmutter, es hat mich wieder
einmal hierher gezogen“, antwortete I.

Die Runzeln lachten: „Ja, die Sonne! Ach, du Teufelsmädchen! Ich
weiß schon, ich weiß! Na gut, geht nur hinein, ich will in der
Sonne sitzen bleiben\ldots{} “ Meine Begleiterin war offenbar kein
seltener Gast in diesem Haus. Ich wollte etwas von mir abschütteln,
es war wohl immer noch jenes bedrückende Bild — die Wolke am
blanken Majolikahimmel.

Als wir die breite, dunkle Treppe hinaufgingen, sagte I: „Ich habe
die alte Frau sehr gern.“ „Warum?“

„Ich weiß nicht. Vielleicht ihres Mundes wegen. Vielleicht auch aus
keinem besonderen Grund. Ich habe sie einfach gern.“

Ich zuckte die Achseln. Lächelnd fuhr sie fort: „Ich fühle mich
schuldbewusst. Selbstverständlich darf es kein
\glq{}Einfach-Gern\-ha\-ben\grq{} geben, sondern nur ein
Gernhaben, das begründet ist. Alle Urelemente müssen\ldots{}“ „Klar“,
sagte ich, ertappte mich sogleich bei diesem Wort und blickte I
verstohlen an. Hatte sie es bemerkt oder nicht?

Sie blickte zu Boden, ihre gesenkten Lider glichen
heruntergelassenen Vorhängen. Dabei fiel mir ein: Wenn man abends
um 22 Uhr durch die Straßen geht, sieht man in den
hellerleuchteten, durchsichtigen Glashäusern hier und dort dunkle
Zimmer mit zugezogenen Gardinen, und dahinter\ldots{} Was sie dann wohl
tut? Warum hat sie mich heute angerufen, und was soll das alles
bedeuten? Ich öffnete eine schwere, undurchsichtige, knarrende Tür,
und wir traten in einen dunklen Raum (so etwas nannten
unsere Vorfahren Wohnung). Ein seltsames Musikinstrument stand
dort, ein Flügel, und in dem ganzen Zimmer war der gleiche
regellose Wirrwarr von Formen und Farben wie in der alten Musik.
Eine weiße Decke, dunkelblaue Wände, alte Bücher in roten, grünen,
orangefarbenen Einbänden, gelbe Bronzen — zwei Leuchter und eine
Buddha-Statuette —, die Linien der Möbel epileptisch verzerrt,
zuckend, in keine Gleichung zu bringen. Nur mit größter Anstrengung
vermochte ich dieses Chaos zu ertragen. Meine Begleiterin hingegen
hatte offenbar eine kräftigere Konstitution.

„Diese Wohnung gefällt mir am besten von allem im Alten Haus“,
sagte sie, und plötzlich schien sie sich auf irgend etwas zu
besinnen; ihr Lächeln — ein Biss, die scharfen weißen Zähne
blitzten: „Sie ist die hässlichste aller Wohnungen von einst.“

„Der hässlichste aller Staaten von einst“, verbesserte ich sie.
„Jener Tausenden von mikroskopisch kleinen Staaten, die beständig
miteinander Krieg führten, grausam wie\ldots{}“

„Ja, ich weiß\ldots{} “, unterbrach mich I ernst. Wir gingen durch ein
Zimmer, in dem Kinderbetten standen (in jener Epoche waren die
Kinder noch Privateigentum). Und dann ein Raum nach dem anderen —
blitzende Spiegel, riesige Schränke, unerträglich bunte Sofas, ein
ungeheurer Kamin, ein großes Mahagonibett. Unser schönes
durchsichtiges, unzerbrechliches Glas sah ich nur in Form von
armseligen, trüben Fensterscheiben. „Seltsam, hier haben die
Menschen \glq{}einfach so\grq{} geliebt, geglüht, sich
gequält\ldots{}“ (wieder schlug sie die Augen nieder), „welch eine
sinnlose, unwirtschaftliche Verschwendung menschlicher Energie —
nicht wahr?“ Damit sprach sie meine eigenen Gedanken aus, doch in
ihrem Lächeln las ich die ganze Zeit dieses aufreizende X. Hinter
ihren gesenkten Lidern ging irgend etwas vor, das mich aus dem
Gleichgewicht brachte.

Ich wollte ihr widersprechen, wollte sie anschreien, aber ich
musste ihr zustimmen — sie hatte recht. Wir blieben vor einem
Spiegel stehen. In diesem Augenblick sah ich nur ihre Augen. Da kam
mir der Gedanke: Der Mensch ist genauso unzivilisiert wie diese
scheußlichen Wohnungen; sein Kopf ist undurchsichtig, nur zwei
kleine Fenster gestatten einen Blick ins Innere: die Augen.
Anscheinend hatte sie meine Gedanken erraten, denn sie wandte sich
um:

„Nun? Das sind meine Augen.“ (Sie sagte das natürlich nicht laut,
sondern nur mit den Blicken.) Vor mir zwei traurig-dunkle Fenster,
dahinter ein unbekanntes, fremdes Leben. Ich sah nur, dass dort
drinnen ein Feuer brannte, irgendein Kamin, und da waren Gestalten,
ähnlich wie\ldots{}

Es war ganz natürlich: ich sah mein Spiegelbild in ihren Augen.
Aber dieses Spiegelbild war unnatürlich und glich mir ganz und gar
nicht (das kam gewiss von der bedrückenden Umgebung) — ich fühlte
deutlich ein Entsetzen, fühlte mich gefangen, in diesem Käfig
eingeschlossen, vom wilden Wirbel des einstigen Lebens
fortgerissen. „Ach“, sagte I, „gehen Sie doch bitte für eine Minute
ins Nebenzimmer.“

Ich ging hinaus und setzte mich. Von einem Bücherbrett lächelte mir
das stupsnasige, asymmetrische Gesicht eines antiken Dichters
entgegen (ich glaube, es war Puschkin). Wie kommt es nur, dass ich
hier sitze und dieses Lächeln gelassen hinnehme? Ja, warum bin ich
überhaupt hier? Woher kommt dieser sonderbare Zustand? Diese
aufreizende, abstoßende Frau, dieses seltsame Spiel.

Nebenan klappte eine Schranktür, Seide rauschte leise, es kostete
mich große Selbstbeherrschung, nicht hineinzugehen und ihr sehr
harte Worte zu sagen. Doch da kam sie schon. Sie trug ein kurzes,
altmodisches gelbes Kleid, einen schwarzen Hut und schwarze
Strümpfe. Das Kleid war aus dünner Seide, ich konnte hindurchsehen
und erkennen, dass die Strümpfe bis übers Knie reichten. Ihr Hals
war entblößt, ich sah den Schatten zwischen ihren Brüsten\ldots{}

„Das soll wohl originell sein, aber glauben Sie denn wirklich\ldots{}“

„Ja“, unterbrach mich I, „originell sein heißt, sich von den
anderen unterscheiden. Folglich zerstört die Originalität die
Gleichheit\ldots{} Das, was in der idiotischen Sprache unserer Ahnen
banal sein bedeutete, das heißt bei uns: seine Pflicht erfüllen.
Denn\ldots{} “

Ich konnte nicht mehr an mich halten. „Das brauchen Sie mir gar
nicht erst zu sagen\ldots{} “

Sie trat zur Büste des stupsnasigen Dichters, schlug die Augen
nieder und sagte, diesmal anscheinend ganz ernst (vielleicht um
mich zu besänftigen), etwas sehr Kluges: „Finden Sie es nicht
höchst sonderbar, dass die Leute einmal solche Gestalten geduldet
haben? Nicht nur geduldet, dass sie sie sogar verehrt haben? Welch
sklavischer Geist! Nicht wahr?“

„Klar\ldots{} das heißt, ich wollte\ldots{} “ (O dieses verfluchte klar!)

„Schon gut, ich verstehe. Und diese Dichter waren mächtiger als die
gekrönten Häupter jener Zeit, Warum hat man sie nicht isoliert,
nicht ausgerottet? Bei uns\ldots{}“ „Ja, bei uns\ldots{}“, begann ich.
Plötzlich lachte sie spöttisch. Ich erinnere mich, dass ich am
ganzen Leib zitterte. Am liebsten hätte ich sie gepackt, und ich
weiß nicht mehr, was\ldots{} Ich musste irgend etwas tun. Ich öffnete
mechanisch mein goldenes Abzeichen und sah auf die Uhr. 10 Minuten
vor 5.

„Meinen Sie nicht, dass es Zeit ist für uns?“ sagte ich so
unbefangen wie möglich.

„Und wenn ich Sie bitten würde, mit mir hier zu bleiben?“ „Wissen
Sie, was Sie damit sagen? In zehn Minuten muss ich im Auditorium
sein.“

„Alle Nummern sind verpflichtet, an dem Kurs für Kunst und
Wissenschaft teilzunehmen“, sagte I mit meiner Stimme. Dann blickte
sie auf und sah mich an; hinter den dunklen Augenfenstern glühte
das Feuer. „Ich kenne einen Arzt vom Gesundheitsamt, er ist auf
mich abonniert. Wenn ich ihn bitte, schreibt er Ihnen ein Attest,
dass Sie krank waren. Nun?“ Ich begriff endlich, wohin dieses ganze
Spiel führte. „Auch das noch! Sie wissen wohl, dass ich, wie jede
anständige Nummer, eigentlich sofort zu den Beschützern gehen
müsste und\ldots{} “

„Aber uneigentlich“ — sie lächelte spöttisch — „wüsste ich zu gern,
ob Sie hingehen oder nicht.“ „Sie bleiben hier?“ Ich griff nach der
Türklinke. Sie war aus Messing, wie meine Stimme. „Einen Augenblick
bitte. Darf ich?“

Sie ging zum Telefon, wählte eine Nummer — ich war so aufgeregt,
dass ich sie mir nicht merkte — und sagte: „Ich erwarte Sie im
Alten Haus. Ja, ich bin allein.“ Ich drückte auf die kalte
Messingklinke: „Gestatten Sie, dass ich Ihr Flugzeug nehme?“
„Natürlich, bitte\ldots{} “

Am Ausgang träumte die Alte wie eine Pflanze in der Sonne vor sich
hin. Wieder wunderte ich mich, dass der zugewachsene Mund sich
öffnete und sprach:

„Und Ihre Freundin — ist sie allein im Haus?“ „Ja, allein.“

Die Alte schüttelte schweigend den Kopf. Selbst ihr schwaches Hirn
schien zu begreifen, wie tollkühn, wie wahnwitzig diese Frau
handelte.

Punkt fünf Uhr war ich in der Vorlesung. Da fiel mir plötzlich ein,
dass ich der Alten nicht die Wahrheit gesagt hatte — I war jetzt
nicht allein. Vielleicht quälte und lenkte gerade dies mich jetzt
ab, dass ich die Alte gegen meinen Willen belogen hatte. Ja, sie
war nicht allein.

21.30 Uhr — ich hatte eine freie Stunde. Ich hätte noch heute zu
den Beschützern gehen und Anzeige erstatten können. Aber diese
dumme Geschichte hatte mich müde gemacht. Außerdem beträgt die
gesetzliche Frist für eine Anzeige zweimal vierundzwanzig Stunden.
So hatte es noch Zeit bis morgen.

\section{EINTRAGUNG NR. 7}

\uebersicht{\emph{Übersicht:} Die Augenwimper. Taylor. Bilsenkraut
und Maiglöckchen.}
Nacht. Grün, orange, blau, ein Flügel aus Mahagoni, ein
zitronengelbes Kleid. Ein bronzener Buddha. Plötzlich hob er die
metallenen Lider — und Saft entströmte ihm. Auch aus dem gelben
Kleid rann Saft, an dem Spiegel hingen kleine Tropfen, das große
Bett und die Kinderbettchen tröpfelten, und im nächsten Augenblick
werde auch ich\ldots{} Ein banger, süßer Schreck überkommt mich\ldots{} Ich
wachte auf. Gleichmäßiges bläuliches Licht, das Glas der Wände
leuchtete, die gläsernen Stühle, der gläserne

Tisch. Das beruhigte mich, mein Herz pochte nicht mehr so ungestüm.
Sanft, Buddha\ldots{} was für ein Unsinn! Mir ist klar, dass ich krank
bin. Früher habe ich nie geträumt. Träumen — das soll bei unseren
Vorfahren etwas ganz Normales und Alltägliches gewesen sein. Ihr
ganzes Leben war ja ein entsetzliches Karussell: Grün-orange
Buddha-Saft. Wir aber wissen, dass Träume eine gefährliche
psychische Krankheit sind. Und ich weiß: Bis zu diesem Tag war mein
Gehirn ein chronometrisch regulierter, blitzender Mechanismus ohne
das geringste Stäubchen, jetzt aber\ldots{}

Ja, ich spüre in meinem Gehirn einen Fremdkörper wie ein feines
Wimpernhaar im Auge; man fühlt sich im ganzen recht wohl, aber
dieses Haar im Auge — nicht für eine Sekunde lässt es sich
vergessen. In meinem Kopfkissen ertönte ein heller, kristallklarer
Ton: 7 Uhr, aufstehen. Durch die gläsernen Wände rechts und links
sah ich gleichsam mich selbst, mein Zimmer, meine Kleider, meine
Bewegungen — tausendfach wiederholt. Das gab mir neuen Mut, ich
empfand mich als Teil eines gewaltigen, einheitlichen Organismus.
Und welch exakte Schönheit: keine überflüssige Geste, Neigung,
Drehung. Ja, dieser Taylor war zweifellos der genialste Mensch der
alten Zeit. Er kam freilich nicht darauf, seine Methode auf das
ganze Leben auszudehnen, auf jeden Schritt, auf sämtliche
vierundzwanzig Stunden des Tages, er vermochte nicht, sein System
von einer zu vierundzwanzig Stunden zu integrieren. Dennoch — wie
konnten die Menschen von damals ganze Bibliotheken über einen
gewissen Kant schreiben, während sie Taylor, diesen Propheten, der
zehn Jahrhunderte vorausblickte, kaum erwähnten! Das Frühstück war
zu Ende. Wir hatten die Hymne des Einzigen Staates gesungen und
marschierten in Viererreihen zum Lift. Die Motoren summten leise,
es ging rasch nach unten, tiefer, immer tiefer, und ich spürte
einen leichten Schwindel\ldots{}

Da war wieder dieser unsinnige Traum oder irgendeine unklare
Funktion, die von diesem Traum herrührte. Gestern, als wir mit dem
Flugzeug landeten, hatte ich das gleiche Gefühl. Im übrigen ist das
nun alles vorbei: Punkt.

Und es ist sehr gut, dass ich ihr gegenüber so entschieden und
schroff war.

In der Untergrundbahn fuhr ich zur Werft, wo der noch unbewegliche,
noch nicht vom Feueratem durchglühte schlanke Leib des Integral auf
Stapel lag und in der Sonne blitzte. Ich schloss die Augen und
träumte in Formeln: in Gedanken rechnete ich noch einmal aus, wie
groß die Anfangsgeschwindigkeit des Integral sein musste. In jedem
Atom einer Sekunde verändert sich die Masse des Integral (es gibt
Explosionshitze ab). Ich gelangte zu einer höchst komplizierten
Gleichung mit transzendentalen Größen.

Wie im Traum sah ich, dass sich jemand neben mich setzte, mich
leicht mit dem Ellbogen anstieß und „Verzeihung“ murmelte.

Ich öffnete die Augen, und zuerst war mir, als jagte etwas durch
den Raum (eine Assoziation vom Integral): ein Kopf — er flog, weil
er abstehende, rosige Ohren hatte, die wie Flügel aussahen. Dann
die Kurve des gebeugten Nackens, der krumme Rücken, zweifach
gebogen, wie ein S\ldots{}

Und durch die gläsernen Mauern meiner algebraischen Welt drang
wieder ein Wimpernhaar; es ist doch sehr unangenehm, dass ich
heute\ldots{} „Bitte sehr, bitte sehr.“ Ich lächelte meinem Nachbarn
zu und verbeugte mich. Auf seinem Abzeichen funkelte die Nummer
S-4711 (darum also hatte ich ihn von Anfang an mit dem Buchstaben S
in Verbindung gebracht! Es war ein vom Bewusstsein nicht
registrierter visueller Eindruck). Seine Augen blitzten, sie waren
zwei spitze Drillbohrer, die sich schnell drehten und immer tiefer
in mich eindrangen; im nächsten Augenblick würden sie auf den Grund
stoßen und das sehen, was ich sogar vor mir selbst verbarg\ldots{}

Mit einemmal wusste ich, was das Wimpernhaar bedeutete: Er war ein
Beschützer, und es war am einfachsten, wenn ich ihm sofort alles
sagte.

„Wissen Sie, gestern war ich im Alten Haus\ldots{}„Meine Stimme klang
fremd und gepresst. Ich räusperte mich.

„Das ist ja ausgezeichnet!“ entgegnete er. „Das gibt uns Material
für sehr lehrreiche Schlüsse.“ „Aber ich war dort nicht allein, ich
begleitete die Nummer I-330, und da\ldots{} “

„I-330? Gratuliere! Eine sehr interessante, begabte Frau. Sie hat
so viele Verehrer.“

Jetzt fiel es mir ein, er hatte sie damals auf dem Spaziergang
begleitet. Vielleicht war er sogar auf sie abonniert. Nein, ich
konnte ihm nichts davon sagen, das war unmöglich, soviel war mir
klar.

„Ja, das stimmt! Sehr interessant.“ Ich lächelte, immer breiter und
törichter, und fühlte, dass dieses Lächeln meine ganze Nacktheit
und Torheit offenbarte \ldots{} Die Bohrer fraßen sich in mich hinein,
dann schnellten sie zurück. S lächelte zweideutig, nickte mir zu
und ging zur Tür. Ich entfaltete meine Zeitung (mir war, als
blickten alle mich an). Eine Notiz sprang mir in die Augen, die
mich so sehr erregte, dass ich darüber das Wimpernhaar, die Bohrer
und alles übrige vergaß. Es war nur eine kurze Notiz: „Wie aus
wohlinformierten Kreisen verlautet, wurden erneut die Spuren einer
bisher nicht fassbaren Organisation entdeckt, deren Ziel die
Befreiung der Nummern von dem wohltätigen Joch des Staates ist.“
Befreiung? Es ist wirklich erstaunlich, wie stark die
verbrecherischen Instinkte im Menschen sind. Ich sage ganz bewusst:
verbrecherisch. Denn die Freiheit und das Verbrechen sind so eng
miteinander verknüpft wie\ldots{} nun, wie die Bewegung eines Flugzeugs
mit seiner Geschwindigkeit: ist die Geschwindigkeit eines Flugzeugs
gleich Null, bewegt es sich nicht. Ist die Freiheit des Menschen
gleich Null, begeht er keine Verbrechen. Das ist völlig klar. Das
einzige Mittel, den Menschen vor dem Verbrechen zu bewahren, ist,
ihn vor der Freiheit zu bewahren. Kaum ist uns das gelungen, da
kommen ein paar erbärmliche Narren\ldots{}

Nein, ich begreife einfach nicht, warum ich nicht sofort, gestern
noch, zu den Beschützern gegangen bin. Heute suche ich sie bestimmt
auf, gleich nach 16 Uhr. Um 16.10 Uhr verließ ich das Haus, und an
der nächsten Ecke begegnete ich O. Gut, dass ich sie treffe, sie
kommt mir wie gerufen, dachte ich, sie hat einen gesunden
Menschenverstand. Sie wird mich gewiss verstehen und mir helfen.

Aber ich brauchte ja gar keine Hilfe, ich war fest entschlossen.

Die Schornsteine der Musikfabrik bliesen donnernd den Marsch des
Einzigen Staates, den gleichen, täglichen Marsch. Wie beglückend
ist diese alltägliche Wiederholung!

O nahm meinen Arm: „Wir wollen Spazierengehen.“ Die runden blauen
Augen waren weit geöffnet, sie glichen
blauen, klaren Fenstern, und ich konnte ungehindert
hindurchblicken, ohne an etwas hängenzubleiben. Dahinter verbarg
sich nichts, nichts Fremdes, nichts Überflüssiges jedenfalls.

„Nein, ich habe keine Zeit, ich muss\ldots{} “ Ich sagte ihr, wohin ich
gehen wollte. Zu meiner großen Verwunderung sah ich: der
kreisrunde, rosige Mund wurde zum Halbmond, dessen Spitzen nach
unten zeigten zeigten, als hätte sie Säure getrunken. Ich fuhr
auf:

„Ihr weiblichen Nummern seid anscheinend unheilbar von Vorurteilen
verseucht, ihr seid völlig unfähig, abstrakt zu denken!
Entschuldigen Sie, aber das ist einfach dumm!“

„Sie gehen zu den Spionen\ldots{} pfui! Aber ich, ich war im Botanischen
Museum und habe Ihnen Maiglöckchen mitgebracht\ldots{} “

Warum dieses „aber ich“? Typisch weiblich. Wütend (ja, ich war
wütend, ich gebe es zu) nahm ich die Maiglöckchen.

„Na, riechen Sie einmal an Ihren Maiglöckchen. Gut, ja? Also haben
Sie wenigstens so viel Logik, festzustellen: Maiglöckchen riechen
gut. Aber können Sie von dem Geruch selbst, von dem Begriff Geruch,
sagen, er sei gut oder schlecht? Sie können es nicht? Es gibt
Maiglöckchengeruch und es gibt den widerlichen Geruch des
Bilsenkrauts; beides ist Geruch. Der Staat unserer Vorfahren hatte
Spione — und wir haben auch welche. Ja, Spione. Ich fürchte dieses
Wort nicht. Denn es ist klar, dass der Spion damals ein Bilsenkraut
war, während er bei uns ein Maiglöckchen ist. Jawohl, ein
Maiglöckchen!“ Der rosige Halbmond zuckte. Ich glaubte, sie
lächelte, jetzt aber weiß ich, dass mir das nur so vorkam. Und ich
sagte noch lauter:

„Ja, Maiglöckchen! Da gibt es nichts zu lachen, gar nichts!“ Runde,
glatte Kugelköpfe kamen vorüber; sie wandten sich erstaunt um. O
nahm mich zärtlich am Arm: „Sie sind heute so merkwürdig. Sie sind
doch nicht krank?“ Der Traum — das gelbe Kleid — der Buddha\ldots{}
Richtig, ich musste zum Gesundheitsamt gehen. „Ja, ich bin wirklich
krank“, sagte ich sehr erfreut (ein unerklärlicher Widerspruch:
wieso freute ich mich?). „Dann müssen Sie sofort zum Arzt gehen.
Sie wissen doch, Sie haben die Pflicht, gesund zu sein — es wäre
lächerlich, wenn ich Ihnen das klarmachen wollte.“ „Sie haben
natürlich recht, liebe O, völlig recht!“ Ich ging nicht zu den
Beschützern. Es half nichts, ich musste mich zum Gesundheitsamt
begeben. Dort wurde ich bis 17 Uhr festgehalten. Am Abend kam O zu
mir (übrigens ist dort abends ohnehin geschlossen). Wir zogen die
Gardinen nicht zu, sondern lösten Aufgaben aus einem alten
Mathematikbuch, denn eine solche Beschäftigung beruhigt und klärt
den Geist. O-90 saß über ihrem Buch, den Kopf ein wenig zur Seite
geneigt, und stieß vor Anstrengung mit der Zunge gegen die linke
Wange. Wie kindlich, wie bezaubernd das war! Auch in mir war alles
gelöst, exakt, einfach\ldots{}

Sie ging. Ich war wieder allein. Ich holte zweimal tief Atem (das
ist sehr gut vor dem Schlafengehen), und plötzlich spürte ich einen
seltsamen Geruch, der mich anwiderte\ldots{}

Bald wusste ich, woher er kam — in meinem Bett war ein
Maiglöckchenstängel versteckt. Alles in mir empörte sich, alles war
von neuem aufgerührt. Es war wirklich sehr taktlos von ihr, mir
diese Maiglöckchen ins Bett zu stecken\ldots{} Nun, ich bin nicht zu den
Beschützern gegangen. Aber es ist nicht meine Schuld, dass ich
krank bin.

\section{EINTRAGUNG NR. 8}

\uebersicht{\emph{Übersicht:} Die irrationale Wurzel. R-13. Das
Dreieck.}
Vor langer Zeit, ich ging noch zur Schule, begegnete ich zum ersten
Mal \wurzel{}. Ich kann mich noch genau an alle Einzelheiten erinnern, an
den hellen, kugelförmigen Schulsaal, an Hunderte von runden
Knabenköpfen, an Plapa, unseren Mathematiklehrer. Wir hatten ihm
den Spitznamen Plapa gegeben; er war schon ziemlich abgenutzt, und
wenn der diensthabende Schüler ihm den Stöpsel in den Rücken
steckte, sagte der Lautsprecher immer „Pla-pla-pla-schschsch —“,
und dann begann sofort die Mathematikstunde. Einmal erzählte uns
Plapa von irrationalen Zahlen, und ich entsinne mich noch deutlich,
dass ich mit den Fäusten auf den Tisch hämmerte und schrie: „Ich
mag \wurzel{} nicht! Reißt \wurzel{} aus mir heraus!“ Diese irrationale Wurzel
wuchs in mir, sie war ein Fremdkörper, ein furchtbares Gewächs, das
an mir zehrte, mich verschlang. Man konnte diese Wurzel nicht
definieren, sie auch nicht unschädlich machen, weil sie außerhalb
der Ratio war.

Und nun meldete sie sich plötzlich wieder, diese Wurzel. Ich las
meine Aufzeichnungen durch und erkannte, dass ich mich selbst zum
Narren gehalten, mich selbst belogen hatte, nur um \wurzel{} nicht zu
sehen. Es ist alles dummes Zeug, dass ich krank bin usw. — ich
hätte sehr wohl zu den Beschützern gehen können. Vor drei Tagen
noch hätte ich gewiss nicht lange überlegt und wäre sogleich
hingegangen. Warum aber jetzt\ldots{} warum? Heute geschah das gleiche
wie gestern. Punkt 16 Uhr stand ich vor der blitzenden Glasmauer.
Über mir flammten die goldenen Buchstaben des Schildes in der
Sonne. Durch die gläsernen Wände sah ich in dem Gebäude eine lange
Reihe blaugrauer Uniformen. Ihre Gesichter strahlten wie die Lampen
in den Kirchen der alten Zeit. Sie waren gekommen, um eine große
Tat zu vollbringen — um ihre Lieben, ihre Freunde und sich selbst
auf dem Altar des Einzigen Staates zu opfern. Und ich — es trieb
mich, zu ihnen zu eilen, es ihnen gleichzutun. Doch ich vermochte
es nicht, meine Füße waren tief in das gläserne Pflaster
eingesunken, ich stand wie angewurzelt, stumpf vor mich
hinstarrend, unfähig, mich von der Stelle zu rühren\ldots{} „He,
Mathematiker, träumst du?“

Ich fuhr zusammen. Ich erblickte lackschwarze, lachende Augen und
wulstige Negerlippen. Der Dichter R-13, mein alter Freund, stand
vor mir, und neben ihm die rosige O. Ich wandte mich verärgert ab
(wenn sie mich nicht gestört hätten, wäre es mir vielleicht
gelungen, \wurzel{} mit Stumpf und Stiel auszureißen und hineinzugehen).
„Ich träume nicht, ich habe nur etwas eingehend betrachtet“, sagte
ich schroff.

„Nun ja, mein Lieber, Sie sollten nicht Mathematiker sein, sondern
Dichter! Kommen Sie doch zu uns, den Poeten. Wenn Sie wollen,
arrangiere ich das sofort.“ R-13 spricht unglaublich viel und
schnell, die Worte sprühen nur so aus dem wulstigen Mund heraus;
jedes p ist eine Fontäne.

„Ich bin ein Diener der Wissenschaft und werde es bleiben“,
entgegnete ich mit finsterem Gesicht. Ich liebe solche dumme Späße
nicht, ich verstehe sie nicht einmal; aber R-13 hat die hässliche
Angewohnheit, Witze zu machen.

„Gehen Sie mir mit Ihrer Wissenschaft! Diese Wissenschaft ist
nichts als Feigheit! Ihr wollt einfach das Unendliche mit einem
Mäuerchen umgeben und fürchtet euch, hinter diese Mauer zu blicken.
Ja! Und wenn ihr hinüberseht, dann kneift ihr die Augen zu!“

„Die Mauern sind der Anfang jener menschlichen\ldots{}“, begann ich. R
spritzte mir eine ganze Fontäne ins Gesicht, O lachte übermütig.
Ich winkte ab: Lacht nur, das macht mir nichts aus. Mir war nicht
zum Lachen zumute. Ich musste irgend etwas tun, um diese verfluchte
\wurzel{} zu betäuben.

„Wie wäre es“, sagte ich, „wir gehen auf mein Zimmer und lösen
Rechenaufgaben.“ (Ich dachte an die stille Stunde von gestern;
vielleicht würde sie heute wiederkehren.)

O sah R an, dann richtete sie ihre klaren, runden Augen auf mich,
und ihre Wangen nahmen das zarte Rosa unserer Billetts an.

„Heute?\ldots{} Ich habe ein Billett für ihn“ — sie deutete mit dem Kopf
auf R —, „und am Abend ist er beschäftigt, so dass\ldots{} “

Die feuchten, wie Lack glänzenden Lippen schnalzten: „Wir kommen
auch mit einer halben Stunde aus, nicht wahr, O? Ihre
Rechenaufgaben interessieren mich nicht. Gehen wir doch zu mir und
unterhalten uns ein wenig.“ Ich hatte Angst, mit mir allein zu
bleiben, oder, genauer gesagt, mit diesem neuen, mir fremden
Menschen, der durch einen seltsamen Zufall meine Nummer trug —
D-503. Ich folgte R. Es fehlt ihm zwar der exakte Rhythmus, er hat
eine verdrehte, lächerliche Logik, dennoch sind wir Freunde. Nicht
umsonst haben wir uns vor drei Jahren diese reizende, rosige O
ausgesucht. Das verbindet uns noch stärker als die gemeinsame
Schulzeit. Wir saßen in Rs Zimmer. Auf den ersten Blick sah alles
genauso aus wie bei mir: die Gesetzestafel, gläserne Stühle, Tisch,
Schrank und Bett aus Glas. Doch kaum war R hereingekommen, da hatte
er die beiden Sessel zurechtgerückt — und die Flächen waren wie
weggefegt, das feste, dreidimensionale Ordnungssystem zerstört,
alles wurde uneuklidisch.

R ist immer noch der gleiche wie früher. In Taylor-Kunde und
Mathematik war er stets der Letzte in der Klasse. Wir erzählten von
dem alten Plapa, wie wir seine gläsernen Beine mit kleinen
Dankesbriefen beklebt hatten (wir liebten Plapa sehr). Dann
sprachen wir von unserem Religionslehrer. (Im Religionsunterricht
haben wir natürlich nicht die Zehn Gebote unserer Vorfahren
gelernt, sondern die Gesetze des Einzigen Staates.) Er hatte eine
ungewöhnlich laute Stimme, sie drang wie Sturmesheulen aus dem
Lautsprecher, und wir Kinder wiederholten den Text mit schreienden
Stimmen. Einmal hatte ihm der freche R-13 gekautes Löschpapier ins
Sprachrohr gestopft, und bei jedem Wort schossen Papierklümpchen
heraus. R wurde selbstverständlich bestraft, es war auch wirklich
ein hässlicher Streich gewesen, aber wir lachten alle — ich muss
gestehen, ich auch.

„Und wenn er lebendig gewesen wäre wie die Lehrer in alten Zeiten,
dann hätte er erst gespuckt\ldots{}“ Eine Fontäne spritzte über die
dicken, schnalzenden Lippen. Die Sonne drang durch Decke und Wände
und spiegelte sich im Fußboden. O saß auf Rs Knien, in ihren Augen
schimmerten kleine Sonnenflecke. Mir war warm geworden, und ich
verabschiedete mich. Die irrationale Wurzel rührte sich nicht
mehr.

„Na, wie steht?s mit dem Integral? Können wir bald zu den
Marsmenschen fliegen und sie beglücken? Beeilt euch, sonst
schreiben wir Dichter so viele Verse, dass euer Integral nicht mehr
hochkommt. Jeden Tag von 8 bis 12 Uhr\ldots{}“ R-13 schüttelte den Kopf
und kratzte sich auf dem Rücken. Er hatte einen viereckigen Rücken,
der wie ein hinten aufgeschnallter Koffer aussieht (ich musste an
ein altes Bild denken: Im Reisewagen). Ich wurde mit einemmal
lebendig:

„Ach, Sie schreiben auch für den Integral? Über welches Thema? Was
haben Sie zum Beispiel heute geschrieben?“ „Heute habe ich keinen
Strich getan, ich war mit anderen Dingen beschäftigt\ldots{}“ „Womit?“

R machte ein düsteres Gesicht: „Womit, womit? Nun, wenn Sie es
unbedingt wissen wollen — mit einem Urteilsspruch. Ich habe ein
Urteil poetisiert. Da hat so ein Idiot, einer von unseren
Dichtern\ldots{} Zwei Jahre saß er neben mir, war ganz normal, und auf
einmal fängt er an zu schreien: \glq{}Ich bin ein Genie,
für mich gibt es kein Gesetz!\grq{} und ähnlichen Unsinn\ldots{} Na
ja\ldots{} ach!“ Die dicke Unterlippe hing herab, der Lack der schwarzen
Augen wurde stumpf. R-13 sprang auf, kehrte uns den Rücken zu und
starrte durch die gläserne Wand. Ich blickte auf seinen Rücken, der
wie ein fest verschlossener Koffer war, und dachte: Was mag er wohl
jetzt in diesem Koffer suchen?

„Zum Glück sind die vorsintflutlichen Zeiten der Shakespeares und
Dostojewskijs — oder wie man sie damals nannte — vorbei“, sagte ich
betont laut. R wandte sich mir zu. Die Worte sprühten, sprudelten
wie immer aus seinem Mund hervor, doch mir schien, dass seine Augen
allen Glanz verloren hatten. „Ja, lieber Mathematiker, zum Glück,
zum Glück! Wir sind vollkommen glückliche arithmetische
Durchschnittsgrößen\ldots{} Wie heißt das doch bei euch? Von Null zu
Unendlich integrieren, vom Kretin zu Shakespeare\ldots{} Ja, so ist
das!“

Ich weiß nicht, warum mir plötzlich I-330 einfiel, ihre Stimme;
irgendein hauchdünner Faden verband sie mit R-13. Aber was für ein
Faden? Wieder regte sich die irrationale Wurzel. Ich öffnete mein
Abzeichen: 17.35 Uhr. O hatte laut Billett noch 45 Minuten Zeit.
„Ich muss gehen\ldots{}“ Ich küsste O, drückte R die Hand und ging zum
Lift.

Draußen überquerte ich die Straße und blickte zurück: In dem
hellen, sonnendurchfluteten Glasblock waren hier und dort
undurchsichtige, blaugraue Zellen, Zellen eines rhythmischen,
taylorisierten Glücks. Ich spähte zum 7. Stock hinauf, wo sich das
Zimmer von R-13 befand; die Gardinen waren zugezogen.

Liebe O\ldots{} Lieber R\ldots{} In ihm ist etwas, das ich nicht recht
begreife. Und trotzdem bilden ich, er und O ein Dreieck, wenn auch
kein gleichschenkeliges, so doch ein Dreieck. Wir sind, mit den
Worten unserer Vorfahren ausgedrückt (vielleicht verstehen Sie,
lieber Leser, auf fernen Planeten diese Sprache besser als die
unsere), eine Familie. Und es tut wohl, für eine kleine Weile
auszuruhen, sich in einem einfachen, starken Dreieck von allem
abzuschließen.

\section{EINTRAGUNG NR. 9}

\uebersicht{\emph{Übersicht:} Liturgie. Jamben und Trochäen. Die
eiserne Hand.}
Ein strahlend klarer Tag. An solch einem Tag vergisst man seine
Sorgen, Unzulänglichkeiten und Gebrechen, alles ist kristallen,
ewig, wie unser neues Glas\ldots{} Auf dem Platz des Würfels.
Sechsundsechzig riesige konzentrische Kreise — die Tribünen. Und
Sechsundsechzig

Reihen — die Gesichter leuchten still wie die Lampen in den Kirchen
der Alten, die Augen spiegeln den Glanz des Himmels wider, oder
vielleicht auch den Glanz des Einzigen Staates. Blutrote Blumen,
die Lippen der Frauen. Zarte Girlanden, Kindergesichter in der
vordersten Reihe, ganz nahe dem Ort der heiligen Handlung. Tiefe,
feierliche, gotische Stille.

Die uns erhaltenen Schilderungen beweisen, dass unsere Ahnen
während ihrer Gottesdienste nichts dergleichen erlebten. Doch sie
dienten einem dummen, unbekannten Gott — und wir verehren eine
weise, bis in die kleinsten Einzelheiten bekannte Gottheit. Ihr
Gott gab ihnen nichts außer einem ewigen, qualvollen Suchen, ihm
fiel nichts Besseres ein, als sich aus einem unbekannten Grund für
sie zu opfern — wir aber bringen unserem Gott, dem Einzigen Staat,
ein Opfer dar, ein genau durchdachtes, vernünftiges Opfer. Ja,
dieses Opfer war eine feierliche Liturgie für den Einzigen Staat,
eine Erinnerung an die schweren Tage und Jahre des 200jährigen
Krieges, der erhabene Feiertag des Sieges der Masse über den
einzelnen, der Summe über die Zahl\ldots{} Auf den Stufen des
sonnenblitzenden Würfels stand ein einzelner. Sein Gesicht war
weiß, nein, nicht weiß, es hatte keine Farbe mehr, es war gläsern,
durchsichtig wie die Lippen. Nur die Augen waren zwei schwarze
Schlünde, sie sogen gierig jene Welt ein, zu der er noch vor
wenigen Minuten gehört hatte. Das goldene Abzeichen mit der Nummer
hatte man ihm abgenommen, seine Hände waren mit einem Purpurband
zusammengebunden (das ist eine uralte Sitte. Sie ist gewiss damit
zu erklären, dass die Menschen einst, als dies alles nicht im Namen
des Einzigen Staates geschah, sich berechtigt glaubten, Widerstand
zu leisten, und darum mit Ketten gefesselt wurden).

Oben auf dem Würfel, neben der Maschine, reckte sich stumm wie aus
Erz gegossen jener, den wir den Wohltäter nennen. Sein Gesicht war
von unten nicht zu erkennen, man sah nur seine strengen,
majestätischen, quadratischen Umrisse. Aber die Hände\ldots{} Sie
erinnerten an Hände, wie man sie bisweilen auf Photographien sieht;
weil sie der Kamera zu nahe sind, werden sie riesig und verdecken
alles andere. Diese schweren Hände, die noch müßig auf den Knien
ruhten, waren wie Stein, die Knie konnten kaum ihr Gewicht
tragen\ldots{} Plötzlich hob sich die eine ganz langsam — eine
gemessene, strenge Geste —; der gehobenen Hand gehorchend, löste
sich eine Nummer von den Tribünen und schritt zu dem Würfel. Ein
Staatsdichter, dem das Glück beschieden war, den Feiertag mit
seinen Versen zu weihen. Der göttliche, eherne Rhythmus dröhnte
über die Tribünen hin und über jenen Narren mit den gläsernen
Augen, der dort auf den Stufen stand und die logischen Folgen
seiner Wahnsinnstat erwartete\ldots{} Feuersbrunst. In den Jamben
erzittern die Häuser, sprühen als flüssiges Gold empor, stürzen
donnernd zusammen. Die grünen Bäume krümmen sich, sinken, das Harz
fließt — und schon sind sie schwarze, verkohlte Skelette. Und da
erschien Prometheus (damit sind natürlich wir gemeint) :

Und zwang das Feuer in Maschin? und Stahl, das Chaos in die Fesseln
des Gesetzes.

Alles war neu, stählern: die Sonne war Stahl, die Bäume, die
Menschen. Plötzlich kam ein Wahnsinniger und „befreite das Feuer
von seiner Kette“ — und wieder wird alles zugrunde gehen\ldots{}

Leider habe ich ein schlechtes Gedächtnis für Verse, aber an diesen
erinnere ich mich noch: eine lehrreichere und schönere Metapher
kann es kaum geben. Wieder eine langsame, strenge Bewegung, und ein
zweiter Dichter stand auf den Stufen des Würfels. Ich wäre beinahe
von meinem Sitz aufgesprungen — war das ein Spuk? Nein, er war es,
mein Freund mit den wulstigen Lippen\ldots{} Warum hat er nichts davon
verraten, dass ihm die große Ehre\ldots{} Seine Lippen zuckten, sie
waren aschfahl. Ich begriff: vor dem Wohltäter, vor den Beschützern
zu stehen, das ging fast über die Kraft, dennoch, wie konnte man
sich nur so erregen\ldots{}

Schneidende, rasche Trochäen, messerscharf. Beilhiebe. Sie kündeten
von einem unerhörten Verbrechen, von gotteslästerlichen Versen, in
denen der Wohltäter mit Namen belegt wird, wie\ldots{} Nein, ich bringe
es nicht über mich, die Worte zu wiederholen.

R-13 war totenbleich; mit niedergeschlagenen Augen (ich hätte nicht
vermutet, dass er so schüchtern ist) stieg er die Stufen hinab und
setzte sich auf seinen Platz. Eine halbe Sekunde lang sah ich ein
Gesicht neben ihm — ein scharf umrissenes schwarzes Dreieck —, und
im gleichen Augenblick war es wie weggewischt: meine Augen,
Tausende von Augen, waren auf die Maschine dort oben gerichtet. Die
übermenschliche Hand machte eine dritte Bewegung. In einem
unsichtbaren Wind schwankend, stieg der Verbrecher hinauf, eine
Stufe, noch eine — ein Schritt, der letzte seines Lebens — und er
lag, das Gesicht zum Himmel gekehrt, den Kopf zurückgeworfen, auf
seinem letzten Lager. Schwer, ehern wie das Schicksal, schritt der
Wohltäter um die Maschine herum, legte die riesige Hand auf den
Hebel\ldots{} Totenstille. Aller Augen hingen an dieser Hand. Welch
feurig glühender Sturm, welch
innerer Aufruhr — das Werkzeug, die Resultate von 100.000 Volt zu
sein! Welch großes Los! Eine unendliche Sekunde. Die Hand hatte den
Strom eingeschaltet und sank herab. Die unerträglich helle Schneide
des Strahls blitzte auf — ein Zittern, ein kaum vernehmliches
Geräusch in den Röhren der Maschine. Der ausgestreckte Körper war
in eine dünne, leuchtende Rauchwolke gehüllt — und da zerschmolz er
vor unseren Augen, zerfloss, löste sich mit erschreckender
Schnelligkeit auf. Nichts blieb von ihm als eine kleine Pfütze
chemisch reinen Wassers, das noch eben rot im Herzen pulsierte\ldots{}
All das war höchst einfach, jedem von uns vertraut. Es war nichts
weiter als die Dissoziation der Materie, die Spaltung der Atome des
menschlichen Körpers. Dennoch war es jedes Mal von neuem ein
Wunder, ein Zeichen der übermenschlichen Macht des Wohltäters. Dort
oben, vor Ihm, standen zehn weibliche Nummern mit glühenden Wangen
und vor Erregung halbgeöffneten Lippen. Die Blumen in ihren Händen
schwankten sacht im Wind. (Diese Blumen stammten natürlich aus dem
Botanischen Museum. Ich kann an Blumen nichts Schönes finden, wie
auch an allen anderen Dingen der unzivilisierten Welt, die wir
längst hinter die Grüne Mauer verbannt haben. Schön ist nur das
Vernünftige und Nützliche: Maschine, Stiefel, Formeln, Nahrung
usw.)

Nach altem Brauch schmückten diese zehn Frauen die noch nasse
Uniform des Wohltäters mit Blumen. Mit dem majestätischen Schritt
eines Oberpriesters stieg Er langsam die Stufen herab, ging langsam
an den Tribünen vorbei — die Frauen streckten Ihm die Arme wie
zarte weiße Zweige entgegen, donnernde Hochrufe, ein Sturm aus
Millionen von Kehlen. Dann die gleichen Rufe für die Beschützer,
die irgendwo unsichtbar in den Reihen sitzen.

Wer weiß, vielleicht hat die Phantasie der Menschen von einst
unsere Beschützer vorausgeahnt, als sie die wohlwollend-strengen
Schutzengel schuf, die jedem Menschen von seiner Geburt an
zugesellt waren. Etwas von der alten Religion, etwas Reinigendes,
wie Sturm und Gewitter, war in der ganzen Feier spürbar. Ihr, die
ihr diese Zeilen lesen werdet, kennt ihr solche Augenblicke? Ihr
tut mir leid, wenn ihr sie noch nicht erlebt habt\ldots{}

\section{EINTRAGUNG NR. 10}

\uebersicht{\emph{Übersicht:} Der Brief. Die Membrane. Das zottige
Ich.}
Der gestrige Tag war für mich wie jenes Papier, durch das die
Chemiker ihre Lösungen filtrieren: alle schweren Teilchen, alles
Überflüssige bleibt darauf zurück. Am Morgen ging ich sauber
destilliert und durchsichtig zur Halle hinunter.

Die Kontrolleurin saß an ihrem Tischchen, sah auf die Uhr und trug
die Nummer der Hinausgehenden in eine Liste ein. Sie heißt U\ldots{} ich
will ihre Nummer lieber nicht nennen, denn ich fürchte, ich könnte
etwas Hässliches über sie schreiben. Obwohl sie eigentlich eine
anständige, nicht mehr ganz junge Frau ist. Das einzige an ihr, das
mir nicht gefällt, sind ihre Hängebacken, die wie Kiemen aussehen.

Ihre Feder kratzte, ich sah meine Nummer, D-503, auf dem Papier und
daneben einen Tintenklecks. Ich wollte sie gerade darauf aufmerksam
machen, als sie plötzlich aufblickte und mir süß-säuerlich
zulächelte: „Hier ist ein Brief für Sie.“

Ich wusste, dass dieser Brief, den sie bereits durchgelesen hatte,
noch von den Beschützern zensiert werden musste (ich glaube, es ist
überflüssig, diesen ganz natürlichen Vorgang zu erklären), und dass
ich ihn nicht vor 12 Uhr erhalten würde. Aber dieses sonderbare
Lächeln hatte mich verwirrt, so sehr verwirrt, dass ich mich später
bei der Arbeit am Integral nicht konzentrieren konnte und sogar
einmal einen Rechenfehler machte, was mir früher nie passiert ist.

Um 12 Uhr stand ich wieder vor den bräunlich-roten Fischkiemen, sah
wieder das gleiche Lächeln — und endlich hielt ich den Brief in der
Hand. Ich weiß nicht, warum ich ihn nicht sofort las, ich steckte
ihn in die Tasche und eilte in mein Zimmer. Ich riss den Umschlag
auf, überflog die Zeilen und — setzte mich\ldots{} Der Brief war die
offizielle Benachrichtigung, die Nummer I-330 habe sich auf mich
eingetragen, und ich müsse heute um 21 Uhr bei ihr erscheinen\ldots{}
Adresse umstehend\ldots{}

Und das, obwohl ich ihr so unmissverständlich gezeigt hatte, wie
ich zu ihr stehe! Außerdem wusste sie gar nicht, ob ich nicht doch
bei den Beschützern gewesen war — sie konnte von keinem Menschen
erfahren haben, dass ich krank war —, ich hatte wirklich nicht
hingehen können\ldots{} Und dennoch\ldots{}

In meinem Kopf rotierte, heulte ein Dynamo. Der Buddha, das gelbe
Kleid, die Maiglöckchen — ein rosiger Halbmond\ldots{} Auch das noch:
heute Abend wollte O zu mir kommen. Soll ich ihr diese
Benachrichtigung zeigen? Sie wird mir wohl nicht glauben (wie
sollte sie das auch?), dass ich nichts damit zu tun habe, dass ich
völlig unschuldig bin. Es wird gewiss zu einer sinnlosen, völlig
unlogischen Auseinandersetzung kommen\ldots{} Nein, nur
das nicht! Mag alles ganz mechanisch seinen Lauf nehmen, ich werde
ihr einfach eine Abschrift dieser Benachrichtigung schicken.

Ich steckte den Brief hastig in die Tasche, und dabei sah ich meine
hässliche Affenhand. Da fiel mir ein, wie sie, I-330, damals auf
dem Spaziergang meine Hand genommen und betrachtet hatte. Dachte
sie denn wirklich\ldots{} Es ist Viertel vor neun. Eine weiße Nacht.
Alles ringsum ist wie grünes Glas. Doch das ist ein anderes, dickes
Glas, nicht das unsere, nicht das richtige, sondern eine dünne,
gläserne Schale, und darunter wirbelt, brodelt, heult etwas\ldots{} Ich
würde mich nicht wundern, wenn jetzt die Kuppeln der Auditorien als
runde Rauchwolken langsam emporstiegen, wenn der alte Mond
verkniffen lächelte — wie jene Frau heute morgen hinter ihrem Tisch
— und wenn in sämtlichen Häusern die Vorhänge zugezogen würden und
dahinter\ldots{}

Ein seltsames Gefühl: Ich spüre plötzlich meine Rippen, sie waren
eiserne Bänder und beengten, bedrückten mein Herz. Ich stand vor
einer Glastür mit den goldenen Ziffern I-330. I saß mit dem Rücken
zu mir am Tisch und schrieb. Ich trat ein\ldots{}

„Da“, ich hielt ihr mein Billett hin. „Ich habe heute morgen eine
Benachrichtigung erhalten und bin gekommen\ldots{} “

„Wie pünktlich Sie sind! Einen Augenblick bitte. Setzen Sie sich,
ich bin sofort fertig.“

Sie blickte wieder auf ihren Brief — was mochte wohl jetzt in ihr
vorgehen? Was würde sie in der nächsten Sekunde sagen, was würde
sie tun? Wie sollte man das erfahren, berechnen, da bei ihr alles
aus dem wilden, längst versunkenen Land der Träume kam? Ich
betrachtete sie schweigend. Meine Rippen waren eiserne Spangen,
drückten\ldots{} Wenn sie spricht, gleicht ihr Gesicht einem wirbelnden,
blitzenden Rad, man kann die einzelnen Speichen nicht
unterscheiden. Doch im Augenblick stand das Rad still, und ich sah
eine seltsame geometrische Figur: Die hochgezogenen Brauen bildeten
ein spitzwinkeliges Dreieck, zwei tiefe, spöttische Falten liefen
von den Nasenflügeln zu den Mundwinkeln. Diese beiden Dreiecke
standen im Widerspruch zueinander und zeichneten das ganze Gesicht
mit jenem unangenehmen, aufreizenden X, das einem Kreuz glich. Ein
kreuzweise durchgestrichenes Gesicht.

Das Rad drehte sich, die Speichen verschmolzen\ldots{} „Sie waren also
nicht im Beschützeramt?“ „Ich war\ldots{} ich konnte nicht, ich war
krank.“ „Das habe ich mir gleich gedacht, irgend etwas musste Ihnen
dazwischenkommen — ganz gleich was.“ Sie lächelte, die scharfen
Zähne blitzten. „Aber dafür habe ich Sie jetzt in der Hand.
Vergessen Sie nicht: Jede Nummer, die nicht binnen 48 Stunden
Anzeige erstattet, wird\ldots{} “

Mein Herz klopfte so wild, dass es die Rippen zu sprengen drohte.
Ich kam mir wie ein dummer Junge vor, den man bei einem Streich
ertappt hat, und schwieg. Ich war völlig verwirrt, konnte weder
Hand noch Fuß rühren\ldots{} Sie stand auf und reckte sich wohlig. Dann
drückte sie auf einen Knopf, und die Gardinen schlossen sich mit
leisem Rauschen. Ich war von der Welt abgeschnitten, allein mit
ihr.

I stand irgendwo hinter mir, vor dem Schrank. Ihre Uniform
raschelte, glitt zu Boden — ich lauschte gespannt. Da dachte ich,
es durchzuckte mich wie ein Blitz \ldots{} Neulich musste ich die
Krümmung einer neuen Straßenmembrane berechnen (jetzt hängen die
elegant dekorierten

Membranen in allen Straßen und registrieren die Gespräche der
Passanten für das Beschützeramt), und plötzlich musste ich denken:
die konkave, rosige, schwingende Membrane ist eigentlich ein
seltsames Wesen, es besteht aus einem einzigen Organ, dem Ohr. Eine
solche Membrane war ich jetzt.

Leise klirrte ein Knopf am Kragen — an der Brust — noch weiter
unten. Die gläserne Seide glitt über die Schultern, die Knie — zu
Boden. Ich hörte, deutlicher als man es sehen konnte, wie das eine,
dann das andere Bein aus dem blaugrauen Seidenhaufen stieg. Die
straff gespannte Membrane lebte und registrierte: Stille. Nein,
scharfe Hammerschläge gegen die Rippen. Ich hörte, sah: Sie
überlegte eine Sekunde lang.

Die Schranktür klappte, schloss sich wieder — Seide, Seide\ldots{}
„Bitte.“

Ich wandte mich um. Sie trug ein leichtes, safrangelbes
altmodisches Kleid. Das war tausendmal schlimmer, als wenn sie
nichts angehabt hätte. Zwei spitze rosige Punkte schimmerten in dem
dünnen Gewebe wie zwei Kohlen in der Asche. Zwei zärtlich gerundete
Knie\ldots{} Sie setzte sich in den niedrigen Sessel. Auf dem
quadratischen Tischchen vor ihr standen eine Flasche mit einer
giftgrünen Flüssigkeit und zwei Gläser mit dünnem Stiel. Im Mund
hatte sie eine dünne Papierpfeife, wie man sie in alten Zeiten
rauchte (den Namen dafür habe ich vergessen).

Die Membrane schwang immer noch. Der Hammer in mir schlug auf
rotglühende Eisenstäbe. Ich hörte deutlich jeden einzelnen
Schlag\ldots{} Konnte auch sie es hören? Nein, sie rauchte ruhig weiter,
blickte mich an und schüttelte die Asche achtlos auf mein rosa
Billett. Ich fragte sie so kaltblütig wie möglich:

„Warum haben Sie sich eigentlich auf mich eingetragen? Und warum
haben Sie mich gezwungen, hierher zu kommen?“

Sie tat, als hörte sie nicht, goss die Gläser voll und nahm einen
Schluck.

„Ein ausgezeichneter Likör. Trinken Sie auch einen?“ Da erst
begriff ich, dass es Alkohol war! Wie ein Blitz zuckte mir ein
Erlebnis von gestern durch den Kopf, die steinerne Hand des
Wohltäters, der blendend helle Lichtstrahl, der ausgestreckte
Körper mit dem weit zurückgebogenen Kopf. Ich erbebte.

„Aber“, sagte ich, „wissen Sie denn nicht, alle, die sich mit
Nikotin und besonders mit Alkohol vergiften, wird der Einzige Staat
unbarmherzig\ldots{} “

Die dunklen Brauen schnellten empor, bildeten ein spöttisches
Dreieck. „Einige wenige schnell zu vernichten ist vernünftiger, als
vielen die Möglichkeit zu geben, sich selbst umzubringen; und die
Degeneration usw\ldots{}. Das ist die nackte Wahrheit.“ „Ja, die nackte
Wahrheit\ldots{}“

„Und wenn man die ganze Gesellschaft solcher glatzköpfigen, nackten
Wahrheiten auf die Straße hinausließe\ldots{} stellen Sie sich nur vor,
meinetwegen meinen hartnäckigen Verehrer — Sie kennen ihn doch —
stellen Sie sich vor, er würde die ganze Lüge seiner Verkleidung
abwerfen und sich in seiner wahren Gestalt in der Öffentlichkeit
zeigen\ldots{} ach!“

Sie lachte. Aber ich konnte das untere, traurige Dreieck in ihrem
Gesicht deutlich erkennen — zwei tiefe, bittere Falten von der Nase
zu den Mundwinkeln. Diese Falten verrieten mir, dass er, der
zweifach gekrümmte, bucklige Kerl mit den abstehenden Ohren sie
umarmt hatte, sie, eine solche Frau\ldots{} er\ldots{}

Ich will versuchen, die anomalen Gefühlen, die ich in diesem
Augenblick empfand, zu schildern. Jetzt, da ich diese Zeilen
schreibe, weiß ich genau: All das musste so sein; er hatte das
gleiche Recht auf Glück wie jede andere anständige Nummer, und es
wäre ungerecht von mir\ldots{} I lachte sonderbar und lange. Dann sah
sie mich durchdringend an: „Wissen Sie, vor Ihnen habe ich nicht
die geringste Angst. In Ihrer Gegenwart bin ich ganz ruhig. Sie
sind ein netter, lieber Mensch — davon bin ich überzeugt —, und Sie
denken nicht daran, zu den Beschützern zu laufen und anzuzeigen,
dass ich Likör trinke und rauche. Sie werden krank sein oder sonst
etwas. Noch mehr — ich bin sogar überzeugt, dass Sie gleich mit mir
dieses berauschende Gift trinken werden.“ Welch spöttischer,
dreister Ton! Jetzt hasste ich sie wieder. Wieso „jetzt“? Ich habe
sie doch die ganze Zeit gehasst. Sie trank ihr Glas in einem Zug
aus, machte ein paar Schritte — die rosigen Punkte unter dem gelben
Kleid schimmerten — und blieb hinter meinem Sessel stehen.
Plötzlich schlang sie die Arme um meinen Hals, presste ihre Lippen
auf meinen Mund\ldots{} Ich schwöre, das kam völlig unerwartet für mich,
und vielleicht nur, weil\ldots{} Denn, was dann geschah, das konnte ich
doch nicht wollen — soviel ist mir jetzt klar.

Ich fühlte unerträglich süße Lippen (ich glaube, es war der
Geschmack des Likörs) — und ein Schluck des brennenden Giftes floss
in meinen Mund, ein zweiter, ein dritter\ldots{} Ich löste mich von der
Erde und stürzte wie ein selbständiger Planet, mich rasend um meine
Achse drehend, hinab, immer tiefer hinab — in einer unberechenbaren
Bahn\ldots{}

Alles weitere kann ich nur ungefähr beschreiben, nur mit Hilfe mehr
oder minder passender Analogien.

Früher wäre mir das nie in den Sinn gekommen, aber es ist so: Wir
Menschen auf der Erde gehen stets über einem brodelnden Feuermeer,
das sich tief im Schoß der Erde verbirgt und an das wir nie denken.
Mit einemmal war mir, als wäre die dünne Rinde unter meinen Füßen
zu durchsichtigem Glas geworden, als könnte ich plötzlich sehen\ldots{}
Ich wurde zu Glas. Ich konnte in mein Inneres blicken.

Da waren zwei Ich, das alte D-503, die Nummer D-503, und das
andere\ldots{} Früher hatte es sehr selten seine behaarten Hände aus der
Schale emporgestreckt, doch jetzt war es ganz herausgekrochen, die
Schale krachte, gleich würde sie in tausend Stücke zerspringen, und
— was dann?

Ich klammerte mich mit aller Kraft an einen Strohhalm, an die
Sessellehne, und fragte, nur um die Stimme jenes früheren Ich zu
hören:

„Wo\ldots{} wo haben Sie dieses\ldots{} dieses Gift her?“ „Von einem Arzt,
einem meiner Freunde.“ „Einem Freund! Wer ist das?“

Und mein anderes Ich sprang plötzlich auf und schrie: „Das erlaube
ich nicht! Ich will, dass keiner da ist, außer mir! Ich schlage
jeden tot, der\ldots{} weil ich Sie\ldots{} “ Ich sah, wie jener andere sie
grob mit seinen behaarten Händen packte, ihr die Seide vom Leib
riss, die Zähne in ihre Schulter grub, ich weiß es noch genau, die
Zähne! I befreite sich, wie, weiß ich nicht mehr. Sie stand mit
gesenkten Lidern gegen den Schrank gelehnt und hörte mir schweigend
zu.

Ich kniete auf dem Fußboden, umschlang ihre Beine, küsste ihre
Knie. Ich flehte: „Bitte, bitte, sofort — jetzt, in diesem
Augenblick!“ Die scharfen Zähne blitzten, die Brauen zuckten
spöttisch.

Sie neigte sich zu mir und knöpfte wortlos mein Abzeichen mit der
Nummer ab. „Ja, Liebste\ldots{}!“ Ich nestelte hastig an meiner Uniform.
Aber I hielt mir, immer noch schweigend, die Uhr auf meinem
Abzeichen hin. Fünf Minuten vor halb elf.

Ich erstarrte. Ich wusste, was es bedeutet, sich nach 22.30 Uhr auf
der Straße zu zeigen. Alle Wahnideen waren wie weggeblasen, ich war
wieder ich. Mir wurde das eine klar: Ich hasse sie, ich hasse sie!

Ohne mich zu verabschieden, ohne mich umzusehen, stürzte ich aus
dem Zimmer. Im Laufen steckte ich irgendwie das Abzeichen an die
Uniform, rannte die Nottreppe hinunter (ich fürchtete, jemandem im
Lift zu begegnen) auf die verlassene Straße.

Alles war an seinem Platz — alles ganz einfach, alltäglich,
gesetzmäßig: die hellerleuchteten, gläsernen Häuser, der gläserne
bleiche Himmel, die grünliche regungslose Nacht. Aber unter diesem
stillen, kühlen Glas toste unhörbar etwas Wildes, Rotes, Zottiges.
Auch ich jagte keuchend dahin, um nicht zu spät zu kommen.

Plötzlich merkte ich, dass mein Abzeichen sich gelockert hatte, es
löste sich von der Uniform und fiel klirrend auf das gläserne
Pflaster. Ich bückte mich, um es aufzuheben — und in der
sekundenlangen Stille hörte ich leise, schlurfende Schritte hinter
mir. Ich wandte mich um; etwas Kleines, Krummes bog um die Ecke. So
kam es mir wenigstens damals vor.

Ich rannte, so schnell ich konnte. Am Eingang meines Hauses blieb
ich stehen, die Uhr zeigte eine Minute vor halb elf. Ich lauschte —
niemand folgte mir. Alles war nur Einbildung, eine Nachwirkung
jenes Giftes. Die Nacht war qualvoll. Das Bett unter mir hob sich,
senkte sich, hob sich von neuem und beschrieb eine Sinuskurve.
Ich sagte mir immer wieder: „In der Nacht müssen alle
Nummern schlafen. Das ist ebenso eine Pflicht wie die Arbeit am
Tage. Das ist notwendig, damit man tagsüber arbeiten kann. In der
Nacht nicht zu schlafen, ist ein Verbrechen.“

Trotzdem konnte ich kein Auge zutun. Ich gehe zugrunde. Ich bin
nicht mehr fähig, meine Pflichten dem Einzigen Staat gegenüber zu
erfüllen\ldots{} Ich\ldots{}

\section{EINTRAGUNG NR. 11}

\uebersicht{\emph{Übersicht:}\ldots{} Nein, ich kann nicht. Also ohne
Übersicht.}
Abend. Leichter Nebel. Der Himmel ist mit goldenmilchigen Schleiern
verhangen. Unsere Vorfahren wussten, dass dort ein gelangweilter
Skeptiker wohnte, der größte ihrer Skeptiker — Gott. Wir wissen,
diese kristallblaue Leere ist das nackte Nichts. Ich freilich weiß
nicht, ob sich dahinter etwas verbirgt, denn ich habe zu vieles
erfahren. Das Wissen, das von seiner Unfehlbarkeit überzeugt ist,
nennt man Glauben. Ich besaß einen festen Glauben an mich selbst,
ich glaubte mich bis in die letzten Winkel zu kennen. Und da \ldots{}

Ich stehe vor dem Spiegel, und zum ersten Mal in meinem Leben sehe
ich mich klar und bewusst. Ich betrachte mich verwundert wie einen
Fremden. Das bin ich — nein, das ist jener andere: schwarze, gerade
Brauen, dazwischen eine steile Falte wie eine tiefe Schramme (ich
kann mich nicht entsinnen, ob sie früher schon da war). Stählerne
graue Augen, darunter dunkle Schatten von der schlaflos verbrachten
Nacht, und hinter diesem Stahl\ldots{} Ich wusste nie, was dort war. Aus
diesem „Dort“ — es ist ganz
nahe und doch unendlich fern — blicke ich auf mich — auf den
anderen, und ich bin sicher, dass jener mit den schnurgeraden
Brauen ein Fremder ist. Ich kenne ihn nicht, ich begegne ihm zum
ersten Mal im Leben. Aber das wahre Ich, das bin ich, nicht er\ldots{}
Nein, Unsinn! Diese Anwandlungen sind nur Fieberphantasien, eine
Folge der gestrigen Vergiftung\ldots{} Womit habe ich mich vergiftet —
mit der grünen Flüssigkeit oder mit ihr? Darauf kommt es nicht an.
Ich schreibe dies alles nur nieder, um zu zeigen, auf welch
seltsame Abwege der exakte Verstand des Menschen geraten, wie er
sich verlieren kann. Der gleiche Verstand, der es vermochte, die
unseren Ahnen so schreckliche Unendlichkeit begreiflich zu machen
durch\ldots{}

Im Numerator fällt eine Klappe — R-13. Nun, mag er kommen. Ich
freue mich sogar auf ihn. Ich möchte jetzt nicht allein sein.

20 Minuten später.

Auf der Fläche des Papiers, in der zweidimensionalen Welt, stehen
diese Zeilen untereinander, doch in jener anderen Welt\ldots{} Ich
verliere das Gefühl für Zahlen: 20 Minuten — das ist vielleicht 200
oder gar 200.000. Es kommt mir höchst sonderbar vor, dass ich mein
Gespräch mit R ruhig, gleichmäßig, jedes einzelne Wort genau
abwägend niederschreiben soll. Mir ist dabei zumute, als säße ich
mit übergeschlagenen Beinen in einem Sessel vor meinem Bett und
beobachtete neugierig, wie ich mich auf diesem Bett hin und her
werfe.

Als R-13 hereinkam, war ich völlig ruhig. Ehrlich belobte ich die
Trochäen des Urteils, die sein Werk waren, und sagte, jener
Wahnwitzige sei vor allem durch diese Trochäen bezwungen und
vernichtet worden.

„Wenn man mir den Auftrag gegeben hätte“, setzte ich hinzu, „eine
schematische Zeichnung von der Maschine des Wohltäters zu machen,
dann hätte ich auf jeden Fall Ihre Verse beigefügt.“

Rs Augen verloren plötzlich allen Glanz, und seine Lippen wurden
grau. „Was haben Sie?“

„Was ich habe? Ich habe es satt! Alle reden nur noch von diesem
Urteil. Ich mag nichts mehr davon hören.“ Er machte ein finsteres
Gesicht und rieb sich den Rücken, diesen Koffer mit dem seltsamen
Inhalt, den ich mir nicht zu deuten wusste. Pause. Er hatte etwas
in seinem Koffer gefunden, zog es heraus, rollte es auf; seine
Augen lachten, er sprang auf:

„Ich schreibe etwas für Ihren Integral, da ist es!“ Er war wieder
ganz der alte, seine Lippen schnalzten, die Worte sprudelten wie
ein Springbrunnen. „Es ist die alte Legende vom Paradies\ldots{}
natürlich auf uns, auf die Gegenwart übertragen. Jene beiden im
Paradies waren vor die Wahl gestellt: entweder Glück ohne Freiheit
— oder Freiheit ohne Glück. Und diese Tölpel wählten die Freiheit —
wie konnte es anders sein! Und die natürliche Folge war, dass sie
sich jahrhundertelang nach Ketten sehnten. Darin war das ganze
Elend der Menschheit beschlossen — sie gierte nach Ketten.
Jahrhundertelang! Und wir erst sind dahinter gekommen, wie man das
Glück wiedergewinnen kann\ldots{} Unterbrechen Sie mich nicht. Der alte
Gott und wir sitzen am gleichen Tisch. Jawohl! Wir haben Gott
geholfen, endlich den Teufel zu überwinden — denn der Teufel war es
ja, der die Menschen dazu trieb, das Verbot zu übertreten und von
der verderblichen Frucht zu kosten, er, die höllische Schlange. Wir
aber haben ihm den Kopf zertreten und sind so in das Paradies
zurückgekehrt, sind wieder einfältig und unschuldig wie Adam und
Eva. Es gibt kein Gut und Böse mehr. Alles ist unkompliziert und
einfach geworden. Der Wohltäter, die Maschine, der Würfel, die
Gasglocke, die Beschützer — all das ist erhaben und kristallklar.
Es erhält unsere Freiheit, und unsere Freiheit ist unser Glück. Die
Menschen von einst hätten sich lange den Kopf zerbrochen, ob das
eine Ethik sei oder nicht. Aber genug davon. Ist das nicht ein
prächtiges Gedicht über das Paradies? Und ein so ernstes dazu!“ Ich
erinnere mich noch genau, dass ich dachte: Welch scharfer, klarer
Verstand in dieser hässlichen, asymmetrischen Gestalt! Darum stand
er mir auch so nahe, meinem wahren Ich (ich sehe mich trotz allem
als mein früheres, wahres Ich; alles Gegenwärtige ist nur eine
Krankheit). R hatte diesen Gedanken offenbar auf meiner Stirn
gelesen, denn er schlug mir auf die Schulter und rief lachend:
„Ach, Sie\ldots{} Adam! Übrigens, Ihre Eva\ldots{} “ Er kramte in seiner
Tasche und zog ein Notizbuch heraus. „übermorgen, nein, in drei
Tagen, hat O ein rosa Billett für Sie. Wie wollen wir es machen?
Wie immer? Soll sie mit Ihnen allein\ldots{} “ „Selbstverständlich.“

„Das meine ich auch. Sonst geniert sie sich, wissen Sie\ldots{} Eine
merkwürdige Geschichte! Mit mir hat sie nur eine
Rosa-Billett-Affäre, aber mit Ihnen\ldots{} Übrigens, sagen Sie, wer hat
sich eigentlich als vierter in unser Dreieck eingeschlichen? Wer
ist es — gestehen Sie, Sie Verführer!“ In mir tat sich ein Vorhang
auf, ich hörte das leise Rauschen von Seide, sah die Flasche mit
der grünen Flüssigkeit, ein Lippenpaar\ldots{}

„Sagen Sie“, entfuhr es mir, „haben Sie schon einmal Nikotin oder
Alkohol gekostet?“

R schürzte die Lippen und blickte mich forschend an. Ich konnte
seine Gedanken deutlich hören: Und du bist mein Freund?

„Eigentlich nicht“, antwortete er. „Aber ich kannte eine Frau\ldots{} “
„I“, rief ich.

„Wie? Sie waren auch mit ihr zusammen?“ Er schüttelte sich vor
Lachen. Mein Spiegel hängt hinter dem Tisch, und von meinem Sessel
aus konnte ich nur meine Brauen sehen. Sie zogen sich zusammen, und
mein wahres Ich hörte einen wilden, widerwärtigen Schrei: „Was
meinen Sie mit \glq{}auch\grq{}? Ich verlange\ldots{} “ R
riss die Augen weit auf. Mein wahres Ich stürzte sich auf das
andere, das Wilde, behaarte, keuchende, und sagte zu R: „Verzeihen
Sie mir, um des Wohltäters willen. Ich bin schwer krank, ich kann
nicht mehr schlafen. Ich begreife einfach nicht, was mit mir ist.“
Die dicken Lippen lächelten flüchtig:

„Ich verstehe! Das ist mir alles bekannt — theoretisch natürlich.
Leben Sie wohl!“

In der Tür wandte er sich um, kam noch einmal zurück und warf ein
Buch auf den Tisch.

„Mein letztes Werk. Ich habe es für Sie mitgebracht, hätte es fast
vergessen, Ihnen zu geben. Auf Wiedersehen!“

Ich war allein, oder richtiger: unter vier Augen mit diesem andern
Ich. Ich saß mit übergeschlagenen Beinen im Sessel und beobachtete
voller Neugier, wie ich mich auf dem Bett hin und her warf.

Wie kommt es nur, dass O und ich drei volle Jahre einträchtig
miteinander gelebt haben — und jetzt bedarf es nur eines einzigen
Wortes über sie\ldots{} über I und\ldots{} Gibt es diesen ganzen Unsinn von
Liebe und Eifersucht
denn nicht nur in den Büchern unserer Ahnen? Und das muss mir
widerfahren — mir! Ich bestehe doch nur aus Gleichungen, Formeln
und Zahlen — und nun plötzlich dies! Ich begreife es nicht! Ach,
ich werde morgen zu R gehen und ihm sagen\ldots{}

Nein, ich gehe nicht, weder morgen noch übermorgen, ich gehe nie
mehr zu ihm. Ich kann, ich will ihn nicht mehr sehen. Aus! Unser
Dreieck ist zerstört. Ich bin allein. Abend. Leichter Nebel. Der
Himmel ist mit milchig-goldenen Schleiern verhangen. Wenn ich nur
wüsste, was sich dahinter verbirgt. Und wenn ich nur wüsste, wer
ich bin!

\section{EINTRAGUNG NR. 12}

\uebersicht{\emph{Übersicht:} Begrenzte Unendlichkeit. Die Angel.
Gedanken über Dichtung.}
Ich glaube, ich kann wieder gesund werden. Ich habe ausgezeichnet
geschlafen. Keinerlei Träume oder andere krankhafte Symptome.
Morgen kommt die liebe O zu mir, und alles wird so einfach,
regelmäßig und begrenzt wie ein Kreis sein. Begrenztheit — ich
fürchte es nicht, dieses Wort, denn die Arbeit des Größten, das der
Mensch besitzt, die Arbeit des gesunden Verstandes, besteht ja in
dem unablässigen Streben, die Unendlichkeit zu begrenzen, sie in
bequeme, leicht fassliche Portionen, in Differentiale aufzuspalten.
Darin liegt die göttliche Schönheit meines Faches, der Mathematik.
Und ihr, jener I, fehlt jegliches Verständnis für diese Schönheit.
Das ist übrigens eine rein zufällige Assoziation. All das ging mir
bei dem rhythmischen, metrischen Räderrollen
der U-Bahn durch den Kopf. In Gedanken skandierte ich das
Stoßen der Räder und die Verse von R (ich las in seinem Buch, das
er mir gestern gegeben hat): Plötzlich merkte ich, wie jemand
hinter meinem Rücken sich vorbeugte und über meine Schulter auf die
aufgeschlagene Seite des Buches blickte. Ohne mich umzudrehen, sah
ich mit einem Blick aus den Augenwinkeln rosa abstehende Ohren und
etwas doppelt Gekrümmtes\ldots{} ihn! Ich wollte ihn nicht stören und
tat, als bemerkte ich ihn nicht. Wie er hierher gekommen war,
wusste ich nicht; als ich einstieg, saß er, glaube ich, noch nicht
in dieser Bahn. Dieser an sich recht unbedeutende Vorfall hatte
eine starke Wirkung, ich möchte fast sagen, er gab mir neue Kraft.
Es ist so beruhigend, den Blick eines scharfen Auges zu fühlen, das
einen liebevoll vor dem kleinsten Fehler, vor dem kleinsten
Seitensprung bewahrt. Vielleicht klingt das sentimental, aber mir
fiel wieder jene Analogie ein: die Schutzengel, von denen unsere
Ahnen phantasierten. Ja, so vieles, von dem sie nur träumten, ist
in unserem Leben Wirklichkeit geworden. In dem Augenblick, als ich
meinen Schutzengel hinter mir wusste, las ich gerade ein Sonett mit
dem Titel Glück. ich glaube, ich täusche mich nicht, wenn ich
dieses Werk in seiner Schönheit und Gedankentiefe als wahrhaft
einzigartig bezeichne. Die ersten vier Zeilen lauten:

Ewig Verliebte sind zwei mal zwei, ewig vereint als selige vier,
heißeste Liebe auf Erden hier — die unzertrennlichen zwei mal
zwei\ldots{}

Und so geht es weiter von dem weisen, ewigen Glück des Einmaleins.
Jeder echte Dichter ist ein Kolumbus. Amerika hat schon vor
Kolumbus jahrhundertelang existiert, doch erst Kolumbus hat es
entdeckt. Das Einmaleins existiert ebenso schon viele Jahrhunderte
vor R-13, doch erst er vermochte im jungfräulichen Dickicht der
Zahlen ein neues Dorado zu entdecken. In der Tat, gibt es irgendwo
sonst ein weiseres, wolkenloseres Glück als in dieser Wunderwelt?

Der alte Gott schuf den alten Menschen, das heißt, einen Menschen,
der die Fähigkeit besaß, zu irren — folglich hat auch Gott selbst
geirrt. Das Einmaleins ist weiser und absoluter als der alte Gott,
es irrt sich niemals, hören Sie, niemals! Und niemand ist
glücklicher als Zahlen, Nummern, die nach den harmonischen, ewigen
Gesetzen des Einmaleins leben. Keine Unklarheiten, kein Irren. Es
gibt nur eine Wahrheit, nur einen rechten Weg — diese Wahrheit ist
zwei mal zwei, und dieser Weg ist vier. Wäre es nicht absurd, wenn
diese beiden glücklich und ideal miteinander multiplizierten zwei
plötzlich anfingen, an Freiheit, an einen Fehler zu denken? Für
mich ist es ein Axiom, dass R-13 das Grundlegende, das\ldots{} Da spürte
ich wieder den warmen Atem meines Schutzengels, zuerst im Nacken,
dann am linken Ohr. Er hatte offenbar bemerkt, dass das Buch auf
meinen Knien zugeklappt war und dass meine Gedanken in weite Fernen
schweiften. Ich war sogleich bereit, alle Seiten meines Gehirns vor
ihm aufzuschlagen: das ist ein sehr beruhigendes, beglückendes
Gefühl. Ich erinnere mich, dass ich mich sogar umdrehte, ihm
hartnäckig, flehend in die Augen blickte; doch er verstand nicht,
er wollte nicht verstehen — und sagte kein Wort\ldots{} Mir bleibt nur
eines: ich muss Ihnen, lieber Leser, alles erzählen (Sie sind mir
jetzt ebenso teuer und nah — und unerreichbar fern — wie er
damals). Der Weg, den ich in Gedanken zurücklegte, führte mich
vom Teil zum Ganzen. Der Teil ist R-13, das erhabene Ganze ist
unser Institut staatlicher Dichter und Schriftsteller. Ich stellte
folgende Überlegungen an: Wie hatten die Menschen von einst nicht
erkennen können, dass ihre ganze Literatur und Dichtung ein
einziger Unsinn war? Die majestätische Kraft des dichterischen
Wortes wurde sinnlos vergeudet. Jeder schrieb, was ihm gerade
einfiel. Das ist genauso lächerlich und dumm wie etwas anderes aus
der alten Zeit: Damals schlug das Meer volle vierundzwanzig Stunden
stumpfsinnig gegen die Küste, und die in den Wogen eingeschlossenen
Millionen Kilogrammmeter dienten nur dazu, die Gefühle der
Verliebten zu erwärmen. Wir aber haben aus dem verliebten Geflüster
der Wellen Elektrizität gewonnen, wir haben die rasende, schäumende
Bestie zum Haustier gemacht, und genauso haben wir das einst wilde
Element der Poesie gezähmt und gesattelt. Heute ist die Dichtung
kein süßliches Nachtigallenschluchzen, sie ist Dienst am Staat, sie
ist etwas Nützliches.

Nehmen wir zum Beispiel unsere berühmten Mathematischen Nonen —
hätten wir ohne sie in der Schule die vier Grundrechnungsarten so
aufrichtig lieben gelernt? Oder die Dornen — ein geradezu
klassisches Bild: Die Beschützer sind die Dornen an der Rose, sie
schützen die zarte Blume des Staates vor rohen Händen\ldots{} Nur ein
Herz aus Stein bleibt ungerührt, wenn unsere unschuldigen Kinder
wie ein Gebet die Worte lallen: „Der böse Bub wollte die Rose
brechen, aber der stählerne Dorn stach ihn wie eine Nadel. Au, au!
Der Schelm, er läuft nach Hause\ldots{}“ usw. Und die Täglichen Oden auf
den Wohltäter. Jeder, der sie gelesen hat, neigt sich in frommer
Ehrfurcht vor der selbstlosen Arbeit dieser Nummer aller Nummern.
Und die roten Blüten der Gerichtsurteile, die unsterbliche Tragödie
Zu spät zur Arbeit gekommen und das Volksbuch Stanzen über
Geschlechtshygiene. Das Leben in all seiner Mannigfaltigkeit und
Schönheit ist auf ewig in das Gold dieser Werke gefasst. Unsere
Dichter schweben nicht mehr in höheren Regionen, sie sind zur Erde
herabgestiegen. Im gleichen Schritt marschieren sie mit uns unter
den Klängen der strengen, mechanischen Marschmusik der Musikfabrik.
Ihre Leier ist das morgendliche Surren der elektrischen
Zahnbürsten, das drohende Funkenknistern in der Maschine des
Wohltäters, das intime Plätschern im kristallklaren Nachttopf, das
erregende Rauschen der sich schließenden Gardinen, die fröhlichen
Stimmen des neuesten Kochbuchs und das leise Geflüster der
Straßenmembranen. Unsere Götter sind hier auf Erden, sie stehen
neben uns im Büro, in der Küche, in der Werkstatt, im Schlafzimmer;
die Götter sind geworden wie wir, also sind wir wie Götter
geworden. Liebe Leser auf fernen Planeten, wir werden zu Ihnen
kommen, damit Ihr Leben ebenso göttlich-vernünftig und exakt wie
das unsere werde.

\section{EINTRAGUNG NR. 13}

\uebersicht{\emph{Übersicht:} Nebel. Du. Eine dumme Geschichte.}
In der Morgendämmerung erwachte ich und blickte zu der starken,
rosigen Himmelsfeste auf. Alles war gut. Am Abend würde O zu mir
kommen. Ich war gewiss genesen. Ich lächelte und schlief wieder
ein.

Der Wecker rasselt, ich stehe auf, und alles ist verändert. Hinter
dem Glas der Decke, der Wände, überall sehe ich bleichen Nebel.
Wilde Wolken, immer schwerer, immer
näher — und schon ist die Grenze zwischen Himmel und Erde
verschwunden, alles fliegt, fällt, zerfließt, findet nirgends einen
Halt. Es gibt keine Häuser mehr, die gläsernen Mauern haben sich im
Nebel aufgelöst wie Salzkristalle im Wasser. Wenn man von der
Straße her in die Häuser blickt, gleichen die Menschen da drinnen
den unlöslichen Teilchen in einer gärenden, milchigen Lösung. Und
alles raucht — vielleicht rast irgendwo eine Feuersbrunst.

11.45 Uhr. Vor Beginn der täglichen körperlichen Arbeit, die das
Gesetz vorschreibt, ging ich rasch auf mein Zimmer. Plötzlich
läutete das Telefon\ldots{} eine Stimme, die sich wie eine lange, feine
Nadel in mein Herz bohrte: „Ah, Sie sind zu Hause? Freut mich sehr.
Warten Sie an der Ecke auf mich. Ich gehe mit Ihnen\ldots{} wohin, das
sage ich Ihnen später.“ „Sie wissen, dass ich jetzt zur Arbeit
muss.“ „Sie wissen, dass Sie tun werden, was ich Ihnen sage. Auf
Wieder sehn. In zwei Minuten.“

Zwei Minuten später stand ich an der Ecke. Ich musste ihr doch
beweisen, dass der Einzige Staat über mich zu bestimmen hatte und
nicht sie. „Sie werden tun, was ich Ihnen sage\ldots{}“ Sie war wirklich
davon überzeugt, ich hatte es an ihrer Stimme gehört. Nun, ich
würde ihr ungeschminkt sagen, was ich dachte\ldots{} Graue, aus feuchtem
Dunst gewebte Uniformen huschten vorbei und lösten sich nach
wenigen Schritten im Nebel auf. Ich starrte auf die Uhr — zehn,
drei, zwei Minuten vor zwölf. Zu spät, um zur Arbeit zu gehen. Wie
ich diese Frau hasste! Aber ich musste ihr ja beweisen\ldots{} Im
blassen Nebel schimmerte etwas Blutrotes — ein Mund. „Ich glaube,
ich habe Sie warten lassen, aber jetzt haben Sie sich ohnehin
verspätet.“

Wie ich sie\ldots{} Übrigens hatte sie recht, es war tatsächlich zu
spät.

Sie trat dicht an mich heran, unsere Schultern berührten sich, wir
waren allein. Irgend etwas strömte aus ihr in mich hinein, und ich
wusste, es musste so sein. Ich wusste es mit jedem Nerv, mit jedem
schmerzlich-süßen Schlag meines Herzens. Mit unsäglicher Freude
überließ ich mich diesem Gefühl. So freudig muss ein Eisenstück
sich dem unabänderlichen, ewigen Gesetz unterwerfen und sich an
einem Magneten festsaugen. So muss ein emporgeschleuderter Stein
eine Sekunde lang stillstehen und dann in steilem Flug zur Erde
hinabstürzen. So muss ein Mensch nach schwerer Agonie Atem
schöpfen, ein letztes Mal — und dann sterben.

Ich erinnere mich, dass ich zerstreut lächelte und ganz
unvermittelt sagte: „Es ist neblig\ldots{} “ „Liebst du den Nebel?“

Dieses alte, längst vergessene Du, mit dem die Herrin einst ihren
Sklaven anredete — auch das musste sein, auch das war gut.

„Ja, gut\ldots{}“, sagte ich laut vor mich hin. Und dann zu ihr: „Ich
hasse den Nebel, ich fürchte ihn.“ „Also liebst du ihn. Du
fürchtest ihn, weil er stärker ist als du, du hasst ihn, weil du
ihn fürchtest, du liebst ihn, weil du ihn nicht bezwingen kannst.
Denn man kann nur das Unbezwingbare lieben.“

„Ja, das ist wahr. Und zwar darum, weil\ldots{} weil ich\ldots{} “ Wir gingen
zu zweien, allein. Irgendwo in der Weite schimmerte die Sonne kaum
sichtbar durch den Nebel; alles füllte sich mit etwas Weichem,
Goldenem, Rosigem, Rotem. Die ganze Welt war eine riesige Frau, und
wir ruhten in ihrem Schoß, wir waren noch nicht geboren, wir
reiften freudig heran. Ich wusste — die Sonne, der Nebel, das
Rosige, Goldene, all das war für mich, nur für mich\ldots{}

Ich fragte nicht, wohin wir gingen. Mir war alles gleich, ich
wollte nur gehen, gehen und reifen\ldots{} „Wir sind da“, sagte I und
blieb vor einer Tür stehen. „Heute hat gerade einer meiner Freunde
Dienst. Ich habe dir damals im Alten Haus von ihm erzählt.“ Ich sah
ein Schild Gesundheitsamt und begriff alles. Ein gläsernes, von
goldenem Nebel erfülltes Zimmer. Gläserne Wandregale mit
buntschillernden Flaschen und Fläschchen. Elektrische Leitungen,
bläuliche Funken in den Röhren. Und ein winzig kleiner Mensch. Er
sah aus, als hätte man ihn aus Papier ausgeschnitten, und wie er
sich auch drehte, er hatte immer nur ein Profil, ein scharfes
Profil: eine blitzende Schneide — die Nase, eine Schere — die
Lippen.

Ich hörte nicht, was I zu ihm sagte, ich sah nur, wie sie sprach,
und fühlte, dass ich glücklich lächelte. Die scherenartigen Lippen
blitzten, und der kleine Doktor antwortete: „So, so. Ich verstehe.
Eine höchst gefährliche Krankheit, die schlimmste, die ich
kenne\ldots{}“ Er lachte, die winzige, papierene Hand schrieb irgend
etwas und reichte jedem von uns ein Blatt Papier. Es waren Atteste,
dass wir krank seien und nicht zur Arbeit gehen könnten. Ich hatte
den Einzigen Staat um meine Arbeit betrogen, ich war ein
Verbrecher, ich würde durch die Maschine des Wohltäters enden. Doch
das alles war jetzt so fern, so gleichgültig\ldots{} Ich nahm das Blatt,
ohne zu zögern; ich wusste, mit Augen, Lippen und Händen wusste
ich, dass es so sein musste.

In der halbleeren Garage an der Ecke mieteten wir ein Flugzeug. I
setzte sich ans Steuer, drückte den Starter auf

Vorwärts, und wir lösten uns von der Erde, wir schwebten. Hinter
uns rosig-goldener Nebel, Sonne. Das winzige, scharfe Profil des
kleinen Doktors war mir mit einemmal unendlich lieb und nah. Früher
hatte sich alles um die Sonne gedreht: jetzt, wusste ich, drehte
sich alles um mich\ldots{}

Wir standen vor der Tür des Alten Hauses. Die alte Pförtnerin
lachte uns entgegen. Ihr runzliger Mund war wohl die ganze Zeit
fest verschlossen gewesen, wie zugewachsen, nun aber öffnete er
sich und sprach lächelnd: „Nein, so etwas! Statt zu arbeiten wie
alle anderen\ldots{} Nun, wenn irgend etwas ist, komme ich herein und
sage euch Bescheid.“

Die schwere, undurchsichtige Tür fiel knarrend zu, und zugleich
öffnete sich mein Herz, öffnete sich schmerzlich weit. Ihre Lippen
und meine. Ich trank, trank, riss mich von ihrem Mund los, blickte
stumm in ihre großen Augen — und küsste sie wieder.

Im halbdunklen Zimmer. Blau, safrangelb, dunkelgrünes Leder, das
goldene Lächeln des Buddha, der blitzende Spiegel. Und mein Traum
von damals — wie klar wurde er mir jetzt: alles in mir war mit
golden-rosigem Saft durchtränkt, im nächsten Augenblick musste er
überfließen, versprühen\ldots{}

Und unausweichlich, wie Eisen vom Magneten angezogen wird, floss
ich in sie, mich dem unabänderlichen, ewigen Zwang des Gesetzes
beugend. Es gab kein rosa Billett, keinerlei Berechnung, keinen
Einzigen Staat mehr; auch ich hatte aufgehört zu existieren. Da
waren nur noch spitze, zärtliche, zusammengepresste Zähne,
weitgeöffnete Augen, durch die ich langsam in die Tiefe hinabstieg.
Totenstille — nur in der Zimmerecke, tausend Meilen entfernt,
tröpfelte das Wasser im Waschbecken, und ich war
das Weltall, zwischen dem Fall jedes einzelnen Tropfens lagen ganze
Epochen\ldots{}

Ich warf hastig meine Uniform über, sah I an und nahm sie ein
letztes Mal mit den Blicken in mich auf. „Ich wusste es, ich
wusste, wie du bist\ldots{}“, sagte sie leise. Sie erhob sich, kleidete
sich an, und das bissige Lächeln zuckte wieder um ihren Mund:

„Nun, Sie gefallener Engel? Jetzt sind Sie verloren. Haben Sie
keine Angst? Leben Sie wohl! Sie werden allein zurückkehren.“

Sie öffnete die Tür des Spiegelschranks, blickte mich über die
Schulter an und wartete, dass ich ging. Gehorsam verließ ich das
Zimmer. Doch kaum stand ich auf der Schwelle, da fühlte ich, dass
sie noch einmal ihre Schulter an meine lehnen musste\ldots{}

Ich lief ins Zimmer zurück, wo sie wahrscheinlich vor dem Spiegel
ihre Uniform zuknöpfte — und blieb wie angewurzelt stehen. Ich sah,
dass der Ring am Schlüssel des Schranks noch hin- und herpendelte,
aber I war verschwunden. Sie konnte nicht hinausgegangen sein, das
Zimmer hatte nur eine Tür — und trotzdem war sie nicht mehr da. Ich
suchte in allen Ecken und Winkeln, ich machte sogar den Schrank auf
und befühlte die bunten, altmodischen Kleider — niemand.

Es ist mir sehr peinlich, lieber Leser, dass ich Ihnen von diesem
höchst unnatürlichen Vorfall berichten muss. Aber was soll ich tun,
da es nun einmal so gewesen ist? Der ganze Tag war ja vom frühen
Morgen an voller Unwahrscheinlichkeiten, er war wie jene alte
Krankheit, der Traum. Im übrigen bin ich fest davon überzeugt, dass
es mir früher oder später gelingen wird, jede Art von Widersinn in
irgendeinen Syllogismus zu fassen. Das beruhigt mich, und
hoffentlich auch Sie.

\section{EINTRAGUNG NR.. 14}

\uebersicht{\emph{Übersicht:} „Mein“. Unmöglich, ich kann nicht.
Der kalte Fußboden.}
Noch immer von den gestrigen Ereignissen. Gestern war ich in meiner
Persönlichen Stunde vor dem Schlafengehen beschäftigt und kam nicht
zum Schreiben. Am Abend wollte die liebe O zu mir kommen, es war
ihr Tag. Ich ging zum Hausmeister, um mir eine Bescheinigung zu
holen, die mich berechtigte, die Gardinen zu schließen.

„Was haben Sie denn?“ fragte mich der Hausmeister, „Sie sind heute
so sonderbar.“ „Ich\ldots{} ich bin krank.“

Es war die Wahrheit, ich bin wirklich krank. Das alles ist nur eine
Krankheit, nichts weiter. Plötzlich fiel mir ein: Ich hatte ja ein
Attest\ldots{} Ich griff in meine Tasche — da knisterte etwas, Ich hatte
das also nicht geträumt\ldots{} Ich reichte dem Hausmeister das Attest.
Ich fühlte, wie meine Wangen zu brennen begannen; ohne aufzublicken
wusste ich, dass der andere mich verwundert betrachtete. 21.30 Uhr.
Im Zimmer links sind die Vorhänge zugezogen. Im Zimmer rechts sehe
ich meinen Nachbar sitzen. Er hat seinen kahlen, mit kleinen
Pickeln und Pusteln übersäten Kopf über ein Buch geneigt, und seine
Stirn ist eine riesige gelbe Parabel. Ich gehe gequält im Zimmer
auf und ab: Was soll ich, nach allem, was geschehen ist, mit O? Und
mein Nachbar — ich fühle, dass seine Blicke auf mir ruhen, ich sehe
die Runzeln auf seiner Stirn, es sind unklare Zeilen, und mir
scheint, diese Zeilen beziehen sich auf mich. Um 21.45 Uhr kam ein
heiterer, rosiger Wirbelwind in
mein Zimmer, zwei rosige Arme umschlangen meinen Hals. Doch der
Ring um meinen Nacken lockerte sich immer mehr. O ließ die Arme
sinken. „Sie sind so anders, nicht wie sonst, Sie sind nicht mehr
mein!“

„Was ist denn das für ein unzivilisierter Ausdruck — mein? Ich war
niemals\ldots{}“ Ich stockte. Früher hatte ich niemandem gehört, fuhr es
mir durch den Kopf, aber jetzt\ldots{} Jetzt lebte ich nicht mehr in
unserer vernünftigen Welt, sondern in der alten, phantastischen, in
der Welt der \wurzel{}.

Die Vorhänge schlossen sich. Mein Nachbar ließ sein Buch fallen,
und durch die kleine Spalte zwischen Fußboden und Vorhang sah ich,
wie seine gelbe Hand das Buch aufhob. Ich hätte mich am liebsten an
diese Hand geklammert.

„Ich dachte, ich würde Sie heute Abend beim Spaziergang treffen.
Ich habe Ihnen so viel zu erzählen\ldots{} “ Liebe, arme O! Ihr rosiger
Mund war ein Halbmond, dessen Enden nach unten zeigten. Aber ich
konnte ihr unmöglich alles erzählen, was geschehen war, schon
deshalb nicht, weil ich sie damit zur Mitwisserin meiner Verbrechen
machen würde. Ich wusste, dass sie nicht Kraft genug besaß, um zu
den Beschützern zu gehen, und darum\ldots{}

O legte sich hin, ich küsste sie flüchtig. Ich küsste die kindliche
Falte an ihrem Handgelenk; ihre blauen Augen waren geschlossen, um
ihre Lippen spielte ein Lächeln, ich bedeckte ihr Gesicht mit
Küssen.

Mit einemmal wurde mir bewusst, wie sehr ich mich am Morgen
verausgabt hatte. Nein, ich konnte nicht. Ich musste — doch ich
konnte nicht. Meine Lippen wurden starr und kalt\ldots{}

Ich saß neben dem Bett auf dem Fußboden — welch unerträgliche
Kälte! — und schwieg. Die gleiche stumme Kälte herrschte
wahrscheinlich auch dort oben, in den blauen, stummen
Himmelsräumen.

„Begreifen Sie doch, ich\ldots{} ich wollte nicht\ldots{}“, stammelte ich,
„ich habe mit aller Kraft versucht\ldots{}“ Das war keine Lüge, ich,
mein wahres Ich, wollte sie nicht kränken. Und dennoch musste ich
ihr jetzt alles sagen. Aber wie sollte ich ihr klarmachen, dass das
Eisenstück nicht wollte, obwohl das Gesetz unabänderlich ist? Sie
hob das Gesicht aus den Kissen und sagte mit geschlossenen Augen:
„Gehen Sie, gehen Sie.“

Von eisigem Frost geschüttelt, ging ich in den Korridor. Jenseits
der gläsernen Mauer sah ich ferne, feine Nebelschleier. In der
Nacht würde sich dieser Nebel verdichten und alles einhüllen.

Und was würde nach dieser Nacht sein? O eilte stumm an mir vorbei
zum Lift und schlug die Tür zu.

„Einen Augenblick!“ rief ich ihr nach. Mir war elend zumute. Doch
der Lift glitt summend nach unten\ldots{} Sie hat mir R genommen, sie
hat mir O genommen. Und trotzdem, trotzdem\ldots{}

\section{EINTRAGUNG NR. 15}

\uebersicht{\emph{Übersicht:} Die Glocke. Das Spiegelmeer. Ich muss
ewig brennen.}
Auf der Werft, wo der Integral gebaut wird, kam mir der zweite
Konstrukteur entgegen. Sein Gesicht war wie
immer, rund, weiß, flach, einem Porzellanteller ähnlich, und auf
diesem Teller wurde mir nun etwas widerlich Süßes serviert. Er
sagte:

„Sie beliebten krank zu sein, und während Ihrer Abwesenheit, in
Abwesenheit des allerhöchsten Chefs, ist gestern etwas geschehen\ldots{}
“ „Was?“

„Stellen Sie sich vor, als es zur Mittagspause läutete und wir
hinausgingen, erwischte einer von uns einen umnumerierten Menschen!
Ich kann einfach nicht fassen, wie er hereingekommen ist. Man hat
ihn ins Operationsbüro gebracht, dort werden sie aus dem Täubchen
schon herausholen, wieso und warum\ldots{} “ Er lächelte süßlich. Im
Operationsbüro arbeiten unsere erfahrensten Ärzte unter der
unmittelbaren Aufsicht des Wohltäters. Dort gibt es allerlei
Vorrichtungen und Geräte, vor allem die Gasglocke. Sie beruht im
wesentlichen auf dem gleichen Prinzip wie die wohlbekannte
Glasglocke unserer Ahnen: Man setzte eine Maus unter die Glocke,
saugte die Luft heraus, und so weiter. Nur ist unsere Gasglocke ein
weit vollkommenerer Apparat, sie arbeitet mit verschiedenen Gasen,
und außerdem dient sie nicht dazu, kleine schutzlose Tiere zu
quälen, sondern sie schützt den Einzigen Staat und damit das Glück
von Millionen Menschen. Vor etwa fünfhundert Jahren, als das
Operationsbüro gerade erst mit seiner Arbeit begonnen hatte, wurde
es von ein paar Narren mit der Inquisition unserer Vorfahren
verglichen, doch das ist genauso absurd, wie wenn man einen
Chirurgen, der eine Tracheotomie-Operation durchführt, auf eine
Stufe mit einem Mörder stellen wollte. Beide gebrauchen vielleicht
das gleiche Messer, beide führen die gleiche Operation durch — sie
schneiden einem Menschen die Kehle durch: Der eine ist ein Helfer
der Menschheit
und der andere ein Verbrecher. Der eine trägt ein Pluszeichen, der
andere ein Minuszeichen. Das war so klar und einfach, dass ich es
in einer Sekunde, in einer einzigen Umdrehung der logischen
Maschine begriff. Doch plötzlich blieben die Zahnrädchen an einem
kleinen Minus hängen, und ein anderer Gedanke drängte an die
Oberfläche: der Ring an der Schranktür hatte hin-und hergependelt.
Also war die Tür gerade erst zugeschlagen worden, aber I war
spurlos verschwunden. Das konnte die Maschine nicht kontrollieren.
Ein Traum? Doch ich spürte ja noch einen seltsam süßen Schmerz in
meiner rechten Schulter. An diese Schulter gelehnt, war I mit mir
durch den Nebel gegangen\ldots{}

„Liebst du den Nebel?“ Ja, auch den Nebel, ich liebte alles, alles.
Und alles war neu und wunderbar\ldots{} „Alles ist gut\ldots{}“, sagte ich
vor mich hin. „Gut?“ Die runden Porzellanaugen starrten mich
erschrocken an. „Was ist gut? Wenn dieser Kerl ohne Nummer hier
herumschnüffelt\ldots{} Sie sind überall, die ganze Zeit sind sie hier
beim Integral, sie\ldots{} “ „Wer?“

„Wie soll ich das wissen? Aber ich fühle, dass sie unter uns sind,
die ganze Zeit.“

„Haben Sie schon gehört, dass man jetzt die Phantasie wegoperieren
kann?“ (Ich hatte tatsächlich vor kurzem davon gehört.)

„Ich weiß. Aber was hat das mit dieser Sache zu tun?“ „An Ihrer
Stelle würde ich zum Arzt gehen und mich operieren lassen.“

Er zog ein säuerliches Gesicht. Der Gute, selbst die kleinste
Anspielung darauf, dass er Phantasie haben könnte, kränkte ihn
zutiefst. Vor einer Woche hätte auch mich das beleidigt, jetzt ist
es anders, denn ich weiß, dass ich

Phantasie habe, dass ich krank bin. Und ich weiß auch, dass ich
nicht gesund werden will.

Es verlangt mich einfach nicht danach. Wir stiegen die gläserne
Treppe zum Integral hinauf. Die Werft unter uns lag wie auf der
flachen Hand ausgebreitet. Lieber unbekannter Leser, wer Sie auch
sein mögen, auch über Ihnen scheint die Sonne. Wenn Sie schon
einmal so krank waren, wie ich es jetzt bin, dann wissen Sie, was
Morgensonne ist, was sie sein kann. Sie kennen es, dieses rosige,
warme Gold. Die Luft selbst scheint rosig, alles ist vom warmen
Sonnenblut durchtränkt, alles lebt. Die Steine sind weich und
lebendig, das Eisen lebt und glüht, die Menschen sind voller Leben
und Freude. Schon in einer Stunde wird das alles vielleicht nicht
mehr sein, aber noch ist es da.

Auch in dem gläsernen Leib des Integral pulsiert etwas; der
Integral dachte an seine große, furchtbare Zukunft, an die schwere
Last des unvermeidlichen Glückes, die er zu Ihnen hinauftragen
soll, lieber Unbekannter, der da ewig sucht und niemals findet. Sie
werden finden und glücklich sein, Sie haben die Pflicht, glücklich
zu sein, Sie brauchen nicht mehr lange zu warten, der Rumpf des
Integral ist fast vollendet, ein anmutiges Ellipsoid aus unserem
Glas, dauerhaft wie Gold und biegsam wie Stahl. Ich beobachtete,
wie man die Spanten und Längsrippen in dem gläsernen Leib
befestigte, wie man im Heck das Lager für den gigantischen
Raketenmotor einmontierte. Alle drei Sekunden eine Explosion, alle
drei Sekunden wird der Integral Flammen und Gase in den Weltraum
speien und unaufhaltsam vorwärtsstürmen, ein feuriger Tamerlan des
Glückes\ldots{}

Ich blickte hinunter auf die Werft. Nach Taylors Gesetz, rhythmisch
und schnell, im gleichen Takt, genauso wie die

Hebel einer riesigen Maschine, bückten die Menschen sich, richteten
sich auf, drehten sich. In ihren Händen blitzten dünne Stäbe: mit
Feuer schnitten und löteten sie Platten, Winkelmaße, Spanten und
Winkelknie. Gläserne Riesenkrane rollten langsam über gläserne
Schienen, drehten und neigten sich ebenso gehorsam wie die Menschen
und senkten ihre Last in den Leib des Integral. Und diese
vermenschlichten Krane und diese vollkommenen Menschen waren eins.
Welch eine ergreifende, vollkommene Schönheit, Harmonie, Musik\ldots{}
Schnell hinunter zu ihnen, ich musste bei ihnen sein!

Ich arbeitete Schulter an Schulter mit ihnen, im gleichen
stählernen Rhythmus\ldots{} gleichmäßige Bewegungen, straffe rote
Wangen, spiegelklare Augen und Stirnen, ungetrübt vom Wahn des
Denkens. Ich schwamm in einem Spiegelmeer. Da sagte jemand zu mir:
„Geht es Ihnen heute wieder besser?“ „Wieso besser?“

„Sie waren doch gestern nicht da. Wir dachten schon, Sie seien
ernstlich krank.“ Seine Augen strahlten, er lächelte
kindlich-unschuldig.

Mir schoss das Blut in die Wangen. Ich konnte diese Augen nicht
belügen, ich konnte es nicht. Ich schwieg\ldots{} In der Luke über mir
erschien ein lachendes, porzellanweißes Gesicht:

„Hallo, D-503! Bemühen Sie sich bitte einmal herauf! Wir bauen
gerade\ldots{} “

Was er weiter sagte, hörte ich nicht mehr, ich stürzte Hals über
Kopf nach oben, ich rettete mich schimpflich durch die Flucht. Ich
hatte keine Kraft, aufzublicken; die gläsernen Stufen unter meinen
Füßen schwankten, und mit jeder Stufe wurde meine Lage
hoffnungsloser: ein versuchter Verbrecher wie ich hatte hier nichts
zu suchen. Nie mehr kann ich in den exakten, mechanischen Rhythmus
einfließen, nie mehr über das stille Spiegelmeer schwimmen. Ich
muss ewig brennen, ruhelos hin und her jagen, einen Winkel suchen,
in dem ich meine Augen verbergen kann — ewig, bis ich endlich die
Kraft aufbringe, hinzugehen und\ldots{}

Ein eisiger Schauer packte mich: es ging ja nicht allein um mich,
ich musste auch an sie, an I, denken. Was würde dann mit ihr
geschehen?

Ich stieg durch die Luke aufs Deck und blieb stehen. Ich wusste
nicht, weshalb ich heraufgekommen war. Ich blickte auf. Über mir
die trübe, matte Mittagssonne, unter mir der Integral, grau, starr,
leblos. Das hellrote Blut, das in diesem riesigen Körper pulsiert
hatte, war hinausgeflossen, und ich erkannte, dass meine Phantasie
mir einen Streich gespielt hatte, dass alles wie früher war.
„Hallo! 503! Sind Sie taub? Ich rufe und rufe\ldots{} Was ist denn mit
Ihnen?“ sagte der zweite Konstrukteur dicht neben mir. Er musste
schon lange gerufen haben, doch ich hörte ihn nicht. Ja, was ist
mit mir? Ich habe das Steuer verloren, das Flugzeug rast weiter,
aber ich habe das Steuer verloren, ich weiß nicht, ob ich
hinabstürze zur Erde oder ob ich hinaufstürme, höher, immer höher,
in die feurige Sonne\ldots{}

\section{EINTRAGUNG NR. 16}

\uebersicht{\emph{Übersicht:} Gelb. Mein Schatten. Die Seele — eine
unheilbare Krankheit.}
Tagelang habe ich keine Zeile geschrieben. Wie viele Tage es sind,
weiß ich nicht, denn alle Tage sind wie ein Tag, sie haben alle die
gleiche Farbe, Gelb, wie ausgedörrter, glühender Sand. Nirgends
Schatten, nirgends ein Tropfen Wasser, und ich wandere ohne Ende
über den gelben Sand. Ich kann nicht mehr ohne sie leben, sie
aber\ldots{} Seit jenem Tag, an dem sie auf so geheimnisvolle Weise im
Alten Hause verschwunden war, bin ich ihr nur einmal beim
Spaziergang begegnet. Sie eilte an mir vorüber und belebte meine
gelbe, öde Welt für einen kurzen Augenblick. Arm in Arm mit ihr, an
ihre Schulter gelehnt, sah ich den buckligen S, den papierdünnen
Doktor und eine weitere männliche Nummer. Ich entsinne mich nur
noch an seine Finger, sie waren sonderbar weiß, lang und dünn und
schossen wie ein Lichtbündel aus dem Uniformärmel hervor. I hob
die Hand und winkte mir zu. Dann wandte sie sich zu dem
Langfingrigen, und ich hörte deutlich das Wort Integral. Alle vier
blickten mich an. Sie verschwanden im graublauen Himmel, und ich
schritt wieder den trockenen, gelben Weg entlang.

Am Abend dieses Tages hatte sie ein rosa Billett für mich! Ich
stand vor dem Numerator und flehte ihn zärtlich, hasserfüllt an, er
solle die Klappe fallen lassen, damit ich die Nummer I-330 auf dem
weißen Feld sehen könne. Eine Tür fiel ins Schloss, hochgewachsene
brünette Frauen kamen aus dem Lift, in allen Zimmern ringsum wurden
die Vorhänge zugezogen. Sie war nicht gekommen. Und in dieser
Minute, um Punkt 22 Uhr, da ich dies
niederschreibe, lehnt sie sich vielleicht mit geschlossenen Augen
an die Schulter eines anderen und fragt ihn, wie sie mich fragte:
„Liebst du das?“ Wer ist er, wer? Der mit den langen, schmalen
Fingern oder R mit seinen wulstigen Negerlippen? Oder gar S?

S\ldots{} Warum habe ich in den letzten Tagen stets seine schlurfenden
Schritte hinter mir gehört, die klingen, als patschte er durch eine
Pfütze? Warum ist er mir wie ein Schatten gefolgt? Vor mir, hinter
mir, an meiner Seite erschien sein graublauer, krummer Schatten.
Die Vorübergehenden schritten durch diesen Schatten hindurch,
traten darauf, doch er blieb unverändert, wich nicht von meiner
Seite, als wäre er durch eine unsichtbare Nabelschnur mit mir
verbunden. Vielleicht ist I diese Verbindung, ich weiß es nicht.
Oder haben die Beschützer bereits gemerkt, dass ich\ldots{} Mein
Schatten sieht mich, sieht mich die ganze Zeit! Wissen Sie, wie das
ist? Ein seltsames Gefühl: Meine Arme scheinen mir nicht mehr zu
gehören, sie sind mir im Wege, ich ertappe mich dabei, dass ich
sinnlos mit den Armen schlenkere, dass ich aus dem Takt gekommen
bin. Oder mir ist, als müsste ich mich umsehen — aber ich kann es
nicht, denn mein Genick ist starr, wie festgeschmiedet. Ich laufe,
laufe immer schneller — der Schatten hinter mir läuft gleichfalls
schneller, ich kann ihm nicht entrinnen\ldots{} In meinem Zimmer —
endlich allein. Aber hier ist etwas anderes, das mir keine Ruhe
lässt — das Telefon. Ich nehme den Hörer ab: „Bitte I-330.“ Ich
höre ein leises Geräusch — eilige Schritte vor ihrer Tür - dann
tiefe Stille\ldots{} Ich werfe den Hörer auf
die Gabel, ich kann nicht mehr. Ich muss sie sehen. Das war
gestern. Ich strich eine volle Stunde, von 16 bis 17 Uhr, um das
Haus herum, in dem sie wohnt. Unzählige Nummern marschierten in
Reih und Glied vorbei, Tausende Füße im gleichen Schritt, ein
millionenfüßiger Leviathan. Ich aber war allein, vom Sturm auf eine
öde Insel verschlagen, und späte suchend in die graublauen Wogen
hinaus. Plötzlich sah ich spöttisch hochgezogene Brauen und dunkle
Augenfenster, hinter denen ein Feuer brannte. Ich eilte zu ihr und
sagte:

„Du weißt doch, dass ich ohne dich nicht leben kann! Warum kommst
du nicht?“

Sie schwieg. Plötzlich wurde mir die tiefe Stille ringsum bewusst.
Es war längst 17 Uhr. Alle waren zu Hause, nur ich war noch auf der
Straße, ich hatte mich verspätet. Um mich eine gläserne, von gelbem
Sonnenlicht durchglühte Wüste. In der glatten Fläche spiegelten
sich die blitzenden Häuserblocks, sie standen auf dem Kopf wie ich.
Ich musste sofort, in dieser Sekunde noch, zum Gesundheitsamt gehen
und mir ein Attest holen, dass ich krank sei; sonst würde man mich
verhaften, und dann\ldots{} Ach, vielleicht wäre es das beste. Hier
stehen bleiben und warten, bis sie mich aufgreifen und in den
Operationssaal schleppen — dann wäre alles zu Ende, wäre alles
gesühnt.

Ein leises, patschendes Geräusch — ein S-förmiger Schatten stand
vor mir. Ohne aufzublicken spürte ich, wie zwei stahlgraue Augen
sich in mich hineinbohrten. Ich riss mich zusammen und sagte
lachend — ich musste doch etwas sagen —:

„Ich\ldots{} ich will zum Gesundheitsamt\ldots{}“ „Warum stehen Sie dann hier
herum?“ Ich schwieg, meine Wangen glühten vor Scham. „Kommen Sie
mit!“ sagte S streng.

Ich folgte ihm gehorsam, mit den Armen schlenkernd, die mir nicht
mehr gehörten. Ich vermochte nicht die Augen zu heben, und so
bewegte ich mich die ganze Zeit
in einer grotesken, Kopf stehenden Welt. Ich sah Maschinen, die
verkehrt aufragten, sah Menschen, die wie Antipoden mit den Füßen
an der Zimmerdecke klebten, sah den Himmel, der mit dem gläsernen
Straßenpflaster verschmolz.

„Entsetzlich“, dachte ich, „dass ich das alles auf dem Kopf stehen
sehe.“ Aber ich konnte nicht aufblicken. Wir blieben stehen. Vor
mir Stufen. Ich tat einen Schritt und glaubte, Gestalten in weißen
Arztkitteln und eine riesige stumme Glocke wahrzunehmen\ldots{} Mit
größter Anstrengung riss ich meine Augen von dem gläsernen Boden
los — und vor mir leuchteten die goldenen Buchstaben
Gesundheitsamt\ldots{} Weshalb er mich hierher und nicht in den
Operationssaal geführt, weshalb er mich geschont hatte, darüber
machte ich mir jetzt keine Gedanken; ich nahm die Stufen mit einem
Satz, schlug die Tür hinter mir zu und atmete auf.

Zwei Ärzte: der eine, klein und krummbeinig, schaute die Patienten
finster an; der andere, schmächtig und dünn, hatte eine
messerscharfe Nase\ldots{} Er war es! Ich stürzte auf ihn zu, als wäre
er mein Bruder, ich stammelte etwas von Schlaflosigkeit, Träumen,
Schatten, von einer gelben Welt.

Seine schmalen Lippen lächelten: „Schlecht, schlecht. Bei Ihnen hat
sich offenbar eine Seele gebildet.“ Eine Seele? Das ist ein
uraltes, längst vergessenes Wort. Wir sagen wohl manchmal noch „ein
Herz und eine Seele“, „Seelenruhe“, „Seelenverderber“, aber Seele,
nein!

„Ist das\ldots{} ist das sehr gefährlich?“ stotterte ich. „Unheilbar“,
erwiderte er.

„Aber — was ist das eigentlich, eine Seele? Ich kann mir das nicht
richtig vorstellen.“

„Ja, wie soll ich Ihnen das erklären? Sie sind doch Mathematiker,
nicht wahr?“ „Ja.“

„Stellen Sie sich eine Fläche vor, zum Beispiel diesen Spiegel.
Blicken Sie hinein — auf dieser Fläche sehen Sie uns beide, Sie
sehen einen blauen Funken in der Leitung, und jetzt huscht der
Schatten eines Flugzeugs vorüber. Nehmen wir an, diese Fläche sei
weich geworden, jetzt gleitet nichts mehr darüber hin, sondern
alles versinkt in jener Spiegelwelt, die wir als Kinder voller
Neugier bestaunten. Glauben Sie mir, die Kinder sind gar nicht so
dumm. Die Oberfläche ist also zu einem Körper geworden, zu einer
Welt, und im Innern des Spiegels — und in Ihnen selbst — ist eine
Sonne, der Propellerwind Ihres Flugzeugs, Ihre bebenden Lippen und
ein zweites Lippenpaar. Sehen Sie, der kalte Spiegel reflektiert
die Gegenstände, jener andere aber absorbiert sie, und alles lässt
für immer eine Spur zurück. Vielleicht haben Sie einmal in einem
Gesicht eine ganz feine Falte entdeckt — und schon ist sie für
immer in Ihnen. Sie haben einmal gehört, wie in der Stille ein
Wassertropfen fiel, und Sie hören ihn auch jetzt\ldots{}“

„Ja, ja, genauso ist es!“ unterbrach ich ihn und ergriff seine
Hand. „Ich habe einmal gehört, wie der Wasserhahn am Waschbecken
leise in der Stille tropfte, und ich werde das nie mehr vergessen.
Dennoch ist mir nicht klar, warum ich plötzlich eine Seele habe.
Ich hatte keine, nie, nie\ldots{} und plötzlich ist sie da. Warum habe
ich allein eine Seele, während die anderen\ldots{} “

Ich klammerte mich noch fester an die dünne Hand, ich fürchtete,
den Rettungsring zu verlieren. „Warum? Nun, warum haben wir weder
Federn noch Flügel, sondern nur noch Schulterblätter, die
Fundamente
der Flügel? Weil wir keine Flügel mehr brauchen — wir haben ja
Flugzeuge, und Flügel wären uns nur hinderlich. Flügel sind zum
Fliegen da, wir aber brauchen nirgendwohin zu fliegen, wir haben
uns in die höchsten Höhen emporgeschwungen, wir haben gefunden, was
wir suchten. Nicht wahr?“

Ich nickte zerstreut. Er blickte mich an und lachte spöttisch. Der
andere Arzt hatte unser Gespräch gehört, er kam aus seinem Büro und
warf dem schmächtigen Doktor und mir einen wütenden Blick zu.

„Was ist denn hier los? Was heißt Seele? Eine Seele, sagen Sie?
Weiß der Teufel, was das ist! Wenn das so weitergeht, werden wir
bald eine richtige Epidemie haben! Man muss bei allen die Phantasie
herausschneiden, exstirpieren. Wie oft soll ich Ihnen das noch
sagen?“ Er funkelte den kleinen Doktor böse an. „In solchem Fall
hilft allein die Chirurgie\ldots{} “

Er setzte eine riesige Röntgenbrille auf, betrachtete meinen Kopf
von allen Seiten, spähte durch die Schädelknochen in mein Gehirn
und schrieb etwas in ein Notizbuch. „Sehr interessant! Wirklich
sehr interessant! Hören Sie, haben Sie etwas dagegen, dass wir Sie
in Spiritus setzen? Das wäre äußerst nützlich für den Einzigen
Staat\ldots{} Es könnte uns helfen, eine Epidemie zu verhüten.“ Der
Schmächtige sagte: „D-503 ist der Konstrukteur des Integral, und
das, was Sie da tun wollen, würde bestimmt verhängnisvolle Folgen
haben.“

„Ach so!“ brummte der andere und stapfte in sein Büro zurück.

Wir waren unter vier Augen. Die papierdünne Hand legte sich
tröstend auf meinen Arm, das scharfgeschnittene, winzige Gesicht
neigte sich ganz nahe zu mir, und er flüsterte mir ins Ohr:

„Mein Lieber, Sie sind nicht der einzige Fall. Mein Kollege redet
nicht umsonst von einer Epidemie. Besinnen Sie sich einmal, ob Sie
nicht bei anderen sehr ähnliche Symptome bemerkt haben.“

Er sah mich forschend an. Wem galt diese Anspielung? Ich sprang von
meinem Stuhl auf.

Doch ohne darauf zu achten, fuhr er laut fort: „Und was Ihre
Schlaflosigkeit und Ihre Träume betrifft, so kann ich Ihnen nur den
einen Rat geben: Gehen Sie öfter zu Fuß. Machen Sie gleich morgen
früh einen längeren Spaziergang, vielleicht zum Alten Haus.“

Wieder sah er mich prüfend an und lächelte leise. Und ich glaubte
das Wort deutlich zu erkennen, das in das feine Gewebe dieses
Lächelns gehüllt war, es war ein Buchstabe, ein Name\ldots{} Oder
täuschte mich meine Phantasie? Er schrieb mir ein Attest für zwei
Tage. Ich schüttelte ihm noch einmal die Hand und eilte davon. Mein
Herz ging leicht und schnell, so schnell wie ein Flugzeug, und trug
mich empor, immer höher empor. Ich wusste, der nächste Tag würde
mir große Freude bringen. Aber was für eine Freude?

\section{EINTRAGUNG NR. 17}

\uebersicht{\emph{Übersicht:} Blick durch die Grüne Mauer. Ich bin
gestorben. Korridore.}
Ich bin wie betäubt. Gestern, gerade in dem Augenblick, als ich
dachte, ich hätte alles entwirrt, alle x gefunden, tauchte eine
neue Unbekannte in meiner Gleichung auf. Der Punkt, von dem alle
Koordinaten dieser Geschichte ausgehen, ist natürlich das Alte
Haus, es ist der Ausgangspunkt 
der x-, y- und z-Achse, auf denen seit einiger Zeit meine
ganze Welt ruht.

Auf der x-Achse (dem 59. Prospekt) ging ich zu Fuß zum Alten Haus.
Meine Erlebnisse von gestern kreisten gleich einem wilden Wirbel in
mir: die Kopf stehenden Häuser und Menschen, meine mir nicht mehr
gehörenden Arme, das scharfe Profil des Doktors, das Klicken der
Wassertropfen im Waschbecken. Das alles hatte ich erlebt und konnte
es nun nicht mehr vergessen. Unablässig brodelte es unter der
aufgeweichten Oberfläche, dort drinnen, wo die Seele war.

Ich hatte die schmale Straße erreicht, die an der Grünen Mauer
entlang führt. Aus dem unabsehbar weiten grünen Ozean jenseits der
Mauer wälzte sich mir eine Woge von Wurzeln, Blüten, Ästen und
Blättern entgegen, sie bäumte sich hoch auf und drohte, mich
wegzuspülen und mich, einen Menschen, den exaktesten aller
Mechanismen, in ein Tier zu verwandeln. Doch zum Glück trennte mich
die Grüne Mauer von diesem wilden, grünen Meer. O große,
göttlich-begrenzende Weisheit von Mauern und Schranken! Die Mauer
ist wahrscheinlich die bedeutendste Erfindung der Menschheit. Der
Mensch hat erst dann aufgehört, ein unzivilisiertes Geschöpf zu
sein, als er die erste Mauer errichtete. Zum Kulturmenschen wurde
er erst, als wir die Grüne Mauer erbauten und unsere vollkommene
Maschinenwelt von dieser unvernünftigen, hässlichen Welt der Bäume,
Vögel und Tiere isolierten.

Durch das trübe, milchige Glas gewahrte ich die stumpfe Schnauze
eines Tieres, seine gelben Augen starrten mich verwundert an. Wir
blickten einander lange in die Augen — in diese Schächte, die von
der Welt der Oberfläche in jene andere, tiefere führen. Und da
sprach eine Stimme in mir: „Vielleicht ist dieses gelbäugige Tier
in seinem
schmutzigen Blätterhaufen, in seinem ungeregelten Leben glücklicher
als wir.“

Ich machte eine wegwerfende Handbewegung, die gelben Augen
blinzelten, wichen zurück und verschwanden im grünen Dickicht.
Erbärmliches Wesen! Es sollte glücklicher sein als wir? Welch
absurder Einfall! Vielleicht glücklicher als ich; aber ich bin eine
Ausnahme, ich bin krank. Ich war beim Alten Haus angekommen. Die
alte Pförtnerin stand vor der Tür. Ich lief auf sie zu und fragte:
„Ist sie hier?“

Der eingefallene Mund öffnete sich langsam: „Wer — sie?“

„I natürlich\ldots{} Ich bin doch damals mit ihr hergekommen — im
Flugzeug.“

„Ah! So, so\ldots{} Ja, sie ist hier, ist eben gekommen.“ Sie war hier!
Die Alte saß neben einem Wermutstrauch, ein Zweig berührte fast
ihre Hand; sie streichelte die silbernen Blätter, und auf ihren
Knien lag ein gelber Sonnenstreifen. Und mit einem Male waren wir
eins, die Sonne, die Alte, der Wermutstrauch, die gelben Augen.
Eine geheimnisvolle Ader verband uns unlösbar, und in dieser Ader
pulsierte wildes Blut\ldots{}

Ich schäme mich sehr, die folgenden Worte niederzuschreiben, aber
ich habe nun einmal versprochen, nur die Wahrheit zu sagen. Ich
beugte mich zu der Alten nieder und küsste ihren weichen, runzligen
Mund. Sie wischte sich lachend die Lippen ab.

Durch die vertrauten halbdunklen Räume ging ich dann zum
Schlafzimmer. Ich hatte schon die Türklinke in der Hand, da
durchzuckte es mich: „Vielleicht ist sie nicht allein!“ Ich
lauschte. Aber ich hörte nur ein dumpfes Pochen ganz in meiner
Nähe, nicht in mir, sondern neben mir — mein Herz.

Ich betrat das Zimmer. Da war das breite, unberührte Bett, der
Spiegel, der Schrank mit dem altmodischen Ring am Schlüssel. Sie
war nicht da.

Leise rief ich: „I, bist du hier?“ Und noch leiser, mit
geschlossenen Augen und angehaltenem Atem, als läge ich auf den
Knien vor ihr: „I! Liebste!“

Stille. Aus dem Wasserhahn klickten Tropfen munter in das
Waschbecken. Mich störte das Geräusch, ich vermag nicht zu sagen
warum, und so drehte ich den Hahn fest zu und ging wieder hinaus.
Da sie bestimmt nicht hier war, musste sie in einer anderen Wohnung
sein. Ich lief die dunkle Treppe hinunter, rüttelte an einer Tür,
an einer anderen: verschlossen. Alles war verschlossen, außer
„unserer“ Wohnung, und dort war kein Mensch. Dennoch zog es mich
dorthin. Langsam und schwerfällig ging ich Stufe um Stufe hinauf,
meine Füße wurden plötzlich so schwer wie Blei. Ich kann mich noch
genau erinnern, dass ich dachte: Es ist ein Irrtum, dass die
Schwerkraft konstant ist, und folglich sind meine sämtlichen
Formeln\ldots{}

Da zuckte ich zusammen: ganz unten schlug eine Tür, und jemand
schlurfte über die Fliesen im Hausflur. Das Gefühl der Schwere war
jäh verschwunden, ich flog zum Treppengeländer. Ich beugte mich
hinab, wollte in dem einen Wort „Du!“ alles aus mir hinausschreien
— und erstarrte. Abstehende rosa Ohren, ein doppelt gekrümmter
Schatten\ldots{} S!

Ohne lange zu überlegen, kam ich zu dem Schluss: Er darf mich um
keinen Preis hier sehen! Ich drückte mich an die Wand und schlich
auf Zehenspitzen zu der unverschlossenen Wohnung. Eine Sekunde
blieb ich vor der Tür stehen. Er stapfte die Treppe herauf, er kam
hierher! Dass nur die Tür nicht knarrte! Ich bat, ich flehte sie
an, aber sie
war ja aus Holz, sie knarrte und kreischte in den Angeln. Grünes,
Gelbes, Rotes, der Buddha flog an mir vorbei, ich stand vor dem
Spiegelschrank. Mein Gesicht war totenbleich, meine Augen und
Lippen waren in angstvoller Spannung weit geöffnet\ldots{} Durch das
Rauschen meines Blutes hindurch hörte ich, wie die Wohnungstür
klappte\ldots{} Das war er, er!

Ich fasste nach dem Schrankschlüssel, und der Ring daran begann zu
pendeln. Das erinnerte mich an etwas — „Damals war I\ldots{}“ Ich riss
die Schranktür auf, stieg in den dunklen Schrank. Ein Schritt — und
ich verlor den Boden unter den Füßen. Langsam, ganz langsam sank
ich in die Tiefe, mir wurde schwarz vor Augen, ich verlor das
Bewusstsein, ich starb.

Später, als ich all diese seltsamen Erlebnisse niederschrieb,
zerbrach ich mir lange den Kopf und schlug in vielen Büchern nach.
Da wurde mir alles plötzlich klar: Ich hatte mich in einem Zustand
befunden, den unsere Vorfahren Ohnmacht nannten und der, soviel ich
weiß, bei uns völlig unbekannt ist.

Wie lange ich ohnmächtig war, weiß ich nicht, es mögen fünf oder
zehn Sekunden gewesen sein; als ich wieder zu mir kam, war es um
mich herum stockfinster, und noch immer glitt ich langsam nach
unten. Ich streckte die Hand aus und fühlte eine raue Mauer, an der
sich mein Finger blutig riss. Also war dies nicht nur ein Spiel
meiner krankhaften Phantasie. Aber was war es dann? Was? Ich hörte
meinen keuchenden Atem, ich bebte vor Angst. Eine Minute, zwei,
drei — immer noch ging es hinunter. Endlich spürte ich einen
leichten Stoß; ich hatte wieder festen Boden unter den Füßen. Ins
Dunkel tastend, fand ich eine Türklinke, ich öffnete die Tür, und
trübes Licht fiel in den schwarzen Schacht, in dem ich stand. Ich
wandte mich um und sah, wie ein kleiner Fahrstuhl hinter mir sich
rasch nach oben entfernte. Ich wollte ihn anhalten .— zu spät, ich
war abgeschnitten\ldots{} Ich wusste nicht, wo ich mich befand.

Vor mir ein Korridor. Drückende, bleierne Stille. In den runden
Gewölben brannten kleine Lampen, eine unendliche Linie flimmernder
Lichtpunkte. Dieser lange Gang erinnerte an die Kanäle unserer
Untergrundbahn, nur bestand er nicht aus unserem dicken,
unzerbrechlichen Glas, sondern aus irgendeinem altertümlichen
Material. Vielleicht war das einer der unterirdischen Gänge, in die
sich die Menschen während des 200jährigen Krieges geflüchtet
hatten\ldots{} Aber mochte es sein, was es wollte — ich musste weiter.

Ich ging gut zwanzig Minuten. Dann bog der Korridor nach rechts ab,
wurde breiter, und die Lampen leuchteten heller. Ich hörte ein
wirres, dumpfes Geräusch. Ob es menschliche Stimmen oder der Lärm
von Maschinen war, konnte ich nicht feststellen, jedenfalls fand
ich mich einer schweren, undurchsichtigen Tür gegenüber, hinter der
das Geräusch erklang.

Ich klopfte, zuerst leise, dann lauter. Plötzlich wurde es still.
Dann klirrte etwas, und die Tür öffnete sich langsam. Ich weiß
nicht, wer von uns beiden verdutzter war — vor mir stand der kleine
Doktor.

„Sie? Hier?“ rief er erschrocken. Ich starrte ihn schweigend an und
verstand kein Wort von dem, was er sagte, als hätte ich noch nie
den Laut einer menschlichen Stimme vernommen. Er wollte wohl, dass
ich wieder ging, denn er nahm mich mit seiner papierdünnen Hand am
Arm und führte mich zurück in den Korridor. „Erlauben Sie“, sagte
ich, „ich wollte\ldots{} ich dachte, dass sie, ich meine I-330\ldots{} “

„Warten Sie hier!“ fiel er mir ins Wort und verschwand. Endlich,
endlich! Sie war hier, ganz in meiner Nähe! Ich stellte mir ihr
gelbes Seidenkleid vor, ihr spöttisches Lächeln, ihre gesenkten
Wimpern, und meine Lippen bebten, mir zitterten Hände und Knie, und
mir kam ein ganz törichter Gedanke: Schwingungen sind Töne, also
muss mein Zittern auch klingen. Aber warum höre ich es nicht? Da
kam sie. Ihre Augen waren weit geöffnet, und ich versank in
ihnen\ldots{}

„Ich konnte es nicht länger ertragen! Wo waren Sie die ganze Zeit?“
Ich starrte sie verzückt an und stammelte wie im Fieber: „Ein
Schatten\ldots{} hinter mir\ldots{} ich war wie tot — der Schrank\ldots{} Ihr
Freund, der Doktor, sagt, ich hätte eine Seele\ldots{} unheilbar\ldots{} “
„Eine Seele! Unheilbar! Du Armer!“ antwortete I lachend. Der
Fiebertraum war verflogen, überall hörte ich ein helles,
spöttisches Lachen, und das tat mir wohl. Der Doktor bog um die
Ecke und kam auf uns zu. „Nun?“ fragte er sie.

„Kein Grund zur Aufregung. Er ist ganz zufällig hierher gekommen,
ich werde Ihnen später alles erzählen. In einer Viertelstunde bin
ich wieder da.“

Der Doktor ging. Sie wartete eine Weile. Dumpf schlug die Tür zu. I
legte die Arme um meinen Hals und schmiegte sich eng an mich; die
Berührung ihres Körpers war wie ein Nadelstich, der tiefer, immer
tiefer in mein Herz drang. Wir waren zu zweit, allein\ldots{} Arm in Arm
gingen wir dunkle Stufen hinauf, die kein Ende zu nehmen schienen.
Wir schwiegen beide; ich konnte es zwar nicht sehen, aber ich
wusste, dass sie mit geschlossenen Augen, zurückgeworfenem Kopf und
geöffneten Lippen ging wie ich und dass sie einer leisen Musik
lauschte: meinem kaum hörbaren Beben.

Wir gelangten zu einem der vielen Nebenhöfe des Alten Hauses; ich
sah einen Zaun und nackte steinerne Rippen und gelbe Zähne
eingestürzter Mauern. Sie schlug die Augen auf, sagte: „Übermorgen
um 16 Uhr“, und ging. Ist das alles wirklich geschehen? Ich weiß es
nicht. Übermorgen werde ich es erfahren. Nur eine einzige,
wirkliche Spur ist zurückgeblieben: eine Schramme an den Fingern
meiner rechten Hand. Aber heute, als ich beim Integral war,
versicherte mir der zweite Konstrukteur, er habe selber gesehen,
wie ich mit den Fingern den Schleifring streifte. Vielleicht war es
tatsächlich so. Ich weiß es nicht, ich weiß überhaupt nichts mehr.

\section{EINTRAGUNG NR. 18}

\uebersicht{\emph{Übersicht:} Im Dickicht der Logik. Wunden und
Pflaster. Nie wieder.}
Gestern legte ich mich zu Bett und versank sogleich im grundlosen
Meer der Träume, wie ein zu schwer beladenes Schiff. Ich fühlte
deutlich den Druck der schwankenden grünen Wellen. Dann wurde ich
langsam emporgetragen und öffnete auf halbem Weg die Augen: mein
Zimmer, kaltes, grünes Morgenlicht. Auf der Spiegeltür meines
Schrankes ein schmaler Sonnenstreifen, der mich blendete. Dieses
Licht hinderte mich, die gesetzlichen Schlafstunden genau
einzuhalten. Vielleicht sollte ich die Schranktür öffnen? Doch ich
hatte nicht die Kraft aufzustehen, ich war von einem Spinnennetz
gefesselt, und in meinen Augen klebten Spinnweben\ldots{}

Ich erhob mich trotzdem, öffnete den Schrank — und plötzlich
erblickte ich hinter der Spiegeltür I, die gerade
ihr Kleid auszog. Ich bin jetzt so sehr an die unwahrscheinlichsten
Dinge gewöhnt, dass ich nicht einmal erstaunt war und keine Frage
stellte. Ich stieg in den Schrank, schlug die Tür zu und umarmte I,
keuchend, blind, gierig. Da drang ein greller Sonnenstrahl durch
die Türspalte und fiel wie eine scharfe, blitzende Schneide auf I.s
zurückgebogenen, entblößten Hals\ldots{} Ich erschrak darüber so sehr,
dass ich die Nerven verlor und laut schrie. Ich schlug die Augen
auf.

Mein Zimmer. Noch immer kaltes, grünes Morgenlicht. Die Sonne
spiegelte sich in der Schranktür. Ich lag im Bett. Es war also nur
ein Traum. Aber immer noch klopfte mein Herz zum Zerspringen, ich
fühlte einen stechenden Schmerz in den Fingerspitzen und in den
Knien. Also doch kein Traum. Ich wusste nicht, ob ich schlief oder
wachte. Die irrationalen Größen verdrängten alles Dauerhafte,
Gewohnte, und statt fester, geschliffener Flächen sah ich ringsum
nur raue, zottige Massen. Es dauerte noch lange, bis der Wecker
rasselte. Ich liege da, denke nach und gelange zu einem höchst
merkwürdigen Ergebnis.

Jeder Gleichung, jeder geometrischen Figur entspricht eine krumme
Linie oder ein Körper. Für die irrationalen Formeln, für meine \wurzel{},
kennen wir keine entsprechenden Körper, wir haben sie nie
gesehen\ldots{} Aber das Entsetzliche ist, dass diese unsichtbaren
Körper existieren, dass sie unbedingt existieren müssen, denn in
der Mathematik huschen ja ihre seltsamen, stachligen Schatten, die
irrationalen Wurzeln, wie auf einer Leinwand an uns vorbei. Und die
Mathematik und der Tod haben noch nie geirrt. Wenn wir aber diese
Körper in unserer Welt, in der Welt der Fläche, nicht sehen können,
dann müssen sie in einer eigenen, gewaltigen Welt leben, die
dahinter liegt\ldots{}

Ohne das Rasseln des Weckers abzuwarten, sprang ich aus dem Bett
und ging im Zimmer auf und ab. Meine Mathematik, die bis jetzt die
einzige feste, unerschütterliche Insel in meinem sonderbaren Dasein
gewesen war, riss sich los und tanzte auf den Wogen. Bedeutete das
nicht, dass diese lächerliche Seele ebenso wirklich war wie meine
Stiefel, obgleich ich sie im Augenblick nicht sehen konnte, weil
sie hinter der Spiegeltür des Schranks standen? Und wenn meine
Stiefel keine Krankheit waren, warum war dann die Seele eine
Krankheit?

Ich tappte im Kreis und fand keinen Ausweg aus diesem unheimlichen
Dickicht der Logik. Das waren die gleichen unbekannten,
schrecklichen Abgründe wie jene hinter der Grünen Mauer, und in
ihnen lebten gleichfalls sonderbare, unbegreifliche Wesen. Hinter
einer dicken Glasscheibe glaubte ich ein unendlich großes und
zugleich unendlich kleines skorpionenhaftes Etwas zu erkennen,
dessen Minus-Stachel verborgen und doch die ganze Zeit fühlbar war:
die Wurzel aus minus eins\ldots{} Aber vielleicht war dies nichts
anderes als meine Seele, die sich gleich dem Skorpion unserer
Vorfahren mit all dem stach, was\ldots{} Der Wecker rasselte. Es war
heller Tag. All diese Gedanken waren nicht tot, waren nicht
verschwunden, sondern nur vom Tageslicht verdeckt, so wie die
sichtbaren Gegenstände in der Nacht nicht sterben, sondern nur vom
Dunkel verhüllt werden. In meinem Kopf wogte ein dünner Nebel.
Durch diesen Nebel sah ich lange gläserne Tische und im Takt
kauende Kiefer. Irgendwo in der Ferne tickte ein Metronom, und zu
dieser gewohnten, zärtlichen Musik zählte ich mechanisch fünfzig —
fünfzig Kaubewegungen sind für jeden Bissen gesetzlich
vorgeschrieben. Mechanisch den Takt schlagend, ging ich hinunter
und trug mich ins Ausgangsbuch ein wie die anderen.

Doch ich fühlte, dass ich von den anderen getrennt, dass ich allein
war, von einer weichen, alle Laute dämpfenden Mauer umgeben, und
hinter dieser Mauer lag meine Welt. Aber wenn diese Welt nur mir
gehört, was hat sie dann in diesen Aufzeichnungen zu suchen? Warum
erzähle ich hier von den törichten Träumen, den Schränken und
endlosen Korridoren? Ich merke bekümmert, dass ich statt eines
ausgewogenen, streng mathematischen Poems zum Preise des Einzigen
Staates einen phantastischen Abenteuerroman schreibe. Ach, ich
wünschte, es wäre nur ein Roman und nicht mein jetziges Leben, in
dem es von unbekannten Größen, von \wurzel{} und von schmählichen
Entgleisungen wimmelt.

Vielleicht ist es doch besser, dass es so gekommen ist, denn Sie,
unbekannter Leser, sind im Vergleich zu uns wahrscheinlich die
reinsten Kinder (wir sind ja von dem Einzigen Staat erzogen worden
und haben daher die höchstmögliche menschliche Entwicklungsstufe
erreicht). Und wie die Kinder werden Sie alles Bittere, das ich
Ihnen reiche, nur dann ohne Geschrei schlucken, wenn es dick
verzuckert ist\ldots{}

Abends:

Kennen Sie dieses Gefühl: Man stürmt im Flugzeug himmelan, das
Fenster ist offen, der Wind peitscht das Gesicht; es gibt keine
Erde mehr, man hat sie vergessen, denn sie ist so fern wie Saturn,
Jupiter und Venus? So lebe ich jetzt, der Wind braust in meinen
Ohren, ich habe die Erde vergessen, habe die liebe O vergessen.
Doch die Erde existiert, früher oder später muss man im Gleitflug
zu ihr zurückkehren\ldots{} ich schließe ja nur die Augen vor dem Tag,
an dem ihr Name, O-90, auf meiner Geschlechtstabelle steht\ldots{} Heute
Abend brachte sich die ferne Erde
in Erinnerung. Gemäß der Vorschrift des Arztes (ich bin fest
entschlossen, gesund zu werden), wanderte ich volle zwei Stunden
durch die schnurgeraden, menschenleeren Prospekte. Alle waren in
den Auditorien, wie das Gesetz es befahl, nur ich nicht. Ein
widernatürliches Bild: stellen Sie sich einen Finger vor, der vom
Ganzen, von der Hand, abgeschnitten ist, einen einzelnen Finger,
der geduckt mit langen Schritten über die gläsernen Bürgersteige
eilt. Dieser Finger war ich. Und das Seltsamste und Unnatürlichste
war, dass er nicht die geringste Lust verspürte, zu der Hand
zurückzukehren. Er zog es vor, entweder allein zu bleiben, oder —
nun, ich habe jetzt nichts mehr zu verbergen — bei jener Frau zu
sein, sich an ihre Schulter zu lehnen, ihre Hand zu halten, sich
ganz in sie zu verlieren. Als ich nach Hause kam, war die Sonne
untergegangen. Auf den gläsernen Mauern, auf der goldenen Spitze
des Akkumulatorenturms, in den Stimmen und im Lächeln der
vorübergehenden Nummern lag die rosa Asche des Abendlichtes. Ist
das nicht merkwürdig: die Strahlen der erlöschenden Sonne haben den
gleichen Einfallswinkel wie die Morgensonne, und doch sind sie
völlig verschieden voneinander. Die Abendröte ist ganz still, fast
ein wenig bitter, die Morgenröte aber klingt und braust. Ich stand
vor dem Kontrolltisch im Vestibül. U, die Aufsichtsbeamtin, zog aus
einem Haufen von Briefen einen Umschlag heraus und reichte ihn mir.
Ich wiederhole: U ist eine sehr anständige Frau, und ich bin
gewiss, dass sie es gut mit mir meint. Trotzdem habe ich jedes Mal
ein unheimliches Gefühl, wenn ich ihre kiemenähnlichen Hängebacken
sehe.

U hielt mir mit ihrer knochigen Hand den Brief hin und seufzte
tief. Doch dieser Seufzer streifte den Vorhang, der mich von der
Außenwelt trennt, nur ganz leicht, denn
ich dachte allein an das Stück Papier in meinen zitternden Fingern.
Gewiss war es ein Brief von I! U seufzte zum zweiten Mal, so laut,
dass ich verwundert aufblickte. Schamhaft schlug sie die Augen
nieder und verzog die Hängebacken zu einem süßen, betörenden
Lächeln. Dann sagte sie:

„Ach, Sie Ärmster —“ und deutete dabei auf den Brief. Sie kannte
natürlich seinen Inhalt, sie war ja verpflichtet, alles zu
zensieren. „Wieso? Ich bin wirklich\ldots{} “

„Nein, nein, mein Lieber, ich weiß das besser als Sie selber. Ich
beobachte Sie schon lange und sehe, dass Sie jemanden brauchen, der
Arm in Arm mit Ihnen durchs Leben geht, jemanden, der das Leben
kennt\ldots{}“ Ihr Lächeln legte sich wie ein Pflaster auf die Wunden,
die mir dieser Brief in meiner Hand gleich schlagen würde.
Schließlich sagte sie leise:

„Ich will mir überlegen, wie ich Ihnen helfen kann. Seien Sie ganz
ruhig, wenn ich genug Kraft in mir fühle, dann werde ich\ldots{} nein,
nein, ich muss das alles noch genau überlegen.“

Großer Wohltäter! Ist das wirklich mein Schicksal? Will sie damit
sagen, dass sie ein Auge auf mich geworfen hat?

Alles verschwamm mir vor den Augen, ich sah tausend Sinusoide, der
Brief begann zu tanzen. Ich trat näher zur Wand, zum Licht. Die
Sonne erlosch, und die dunkelrote Asche auf mir, auf dem Fußboden
und auf dem Brief in meinen Händen verfärbte sich grau. Ich riss
den Umschlag auf, warf einen Blick auf die Unterschrift, und da
öffnete sich die tiefe Wunde — der Brief war nicht von I, sondern
von O! In der unteren rechten Ecke sah ich einen blassblauen
Klecks, der von einem Wassertropfen kam. Ich kann Kleckse nicht
ausstehen, ganz gleich, woher sie kommen, ob von Tinte oder von\ldots{}
Früher war mir ein solcher Fleck nur unangenehm. Warum kommt er mir
jetzt wie eine Wolke vor, warum verdüstert er alles? Oder ist das
wieder die Seele? O schrieb:

Ich verstehe mich zwar nicht aufs Brief schreiben, aber das ist
gleich. Sie wissen jedenfalls, dass ich nicht einen Tag, nicht eine
Stunde, nicht einen Frühling ohne Sie leben kann. R-13 ist für mich
nur\ldots{} nun, das ist unwichtig für Sie. Ich bin ihm jedoch sehr
dankbar, denn ich weiß nicht, wie ich die letzten Tage ohne ihn
hätte überstehen können. In diesen Tagen und Nächten bin ich um
zehn, zwanzig Jahre gealtert. Mir war, als wäre mein Zimmer nicht
quadratisch, sondern rund, ich tappte unablässig im Kreis und fände
nirgends eine Tür.

Ich kann nicht ohne Sie leben — weil ich Sie liebe. Ich weiß, dass
Sie jetzt niemanden in der Welt brauchen außer ihr, der anderen,
und gerade weil ich Sie liebe, muss ich auf Sie verzichten. In
zwei, drei Tagen, wenn ich aus den Fetzen meines Ichs etwas
zusammengeflickt habe, das der früheren O ungefähr gleicht, will
ich mein Abonnement auf Sie kündigen, dann wird Ihnen leichter ums
Herz sein. Ich werde nie wiederkommen. Verzeihen Sie mir. O.

Das war natürlich das beste, sie hatte recht. Aber warum, warum\ldots{}

\section{EINTRAGUNG NR. 19}

\uebersicht{\emph{Übersicht:} Eine unendlich kleine Größe dritter
Ordnung. Die gerunzelte Stirn. Blick über das Geländer.}
In dem unheimlichen Korridor mit den flackernden Lampen — nein,
nicht dort, sondern später, als ich mit ihr in einem versteckten
Winkel im Hof des Alten Hauses stand, hatte sie gesagt:
„Übermorgen.“ Dieses „Übermorgen“ ist heute, und alles hat Flügel.
Der Tag fliegt, und auch unsere Integral wird sich bald
emporschwingen; der Raketenmotor ist eingebaut; heute haben wir ihn
ausprobiert. Welch herrliche, gewaltige Salven! Jede einzelne
empfand ich als Salut für mein Heute.

Bei der ersten Explosion standen ungefähr zehn schlafmützige
Nummern vor dem Auspuff — und es blieb von ihnen nichts übrig als
ein Häufchen Asche. Zu meiner Genugtuung kann ich hier
niederschreiben, dass dieser Vorfall unsere Arbeit nicht im
geringsten aufhielt. Keiner von uns zuckte auch nur mit der Wimper,
wir und unsere Maschinen setzten unsere geraden und kreisenden
Bewegungen so exakt fort, als wäre nichts geschehen. Zehn Nummern —
das ist der hundertmillionste Teil der Masse des Einzigen Staates,
also eine unendlich kleine Größe dritter Ordnung. Unsere Ahnen
kannten ein arithmetisch-analphabetisches Mitleid, das wir
lächerlich finden.

Übrigens kommt mir auch etwas anderes lächerlich vor, dass ich
nämlich wegen eines armseligen grauen Flecks, wegen eines Kleckses
gestern so nachdenklich geworden bin und das in meinen
Aufzeichnungen erwähnt habe. Auch das kommt eben von der
„Aufweichung der Oberfläche“, die diamantenhart sein muss wie
unsere gläsernen Mauern.

16 Uhr. Ich ging nicht zum Gemeinschaftsspaziergang. Wer weiß,
vielleicht fiel es ihr ein, gerade jetzt zu kommen\ldots{} Ich war fast
allein im Hause. Durch die sonnenfunkelnden Wände konnte ich die
lange Flucht der in der Luft schwebenden leeren Zimmer rechts,
links und unter mir überblicken. Ein dichter, grauer Schatten kam
die bläulich schimmernde Treppe herauf, deren Stufen im hellen
Sonnenlicht kaum sichtbar waren. Ich hörte Schritte und sah, wie U
an meiner Tür vorbeiging, mir zulächelte und dann die andere Treppe
hinunterlief. Die Klappe des Numerators fiel. In dem schmalen
weißen Feld erblickte ich eine mir unbekannte männliche Nummer (sie
begann mit einem Konsonanten, daher wusste ich, dass es ein Mann
war). Der Lift surrte, und die Tür wurde zugeschlagen. Vor mir
finstere Brauen und eine vorspringende Stirn, die einem tief ins
Gesicht gedrückten Hut glich, so dass man kaum die Augen sehen
konnte. „Hier ist ein Brief für Sie“, sagte der Fremde. „Von ihr.
Sie bittet Sie, alles genauso zu machen, wie sie Ihnen schreibt.“

Er spähte verstohlen nach allen Seiten. Aber es war niemand da.
Endlich reichte er mir den Brief und ging hinaus. Ich war wieder
allein.

Nein, nicht allein, der Umschlag strömte einen feinen Duft aus,
ihren Duft, und er enthielt ein rosa Billett! Sie kommt, sie kommt
zu mir! Schnell, schnell den Brief überfliegen, damit ich mich mit
eigenen Augen überzeugen, es wirklich glauben kann\ldots{} Aber was
stand denn da? Ich las ihre Zeilen noch einmal:\ldots{} Billett. Und
schließen Sie auf alle Fälle die Gardinen, so, als wäre ich
tatsächlich bei Ihnen. Es tut mir unsäglich leid\ldots{}

Ich zerriss den Brief in kleine Fetzen. Im Spiegel sah ich meine
zusammengezogenen Brauen. Ich nahm das Billett, um es gleichfalls
zu zerreißen. „Sie bittet Sie, alles genauso zu machen, wie sie
Ihnen schreibt\ldots{}“

Meine Hände sanken kraftlos herab, das Billett fiel auf den Tisch.
Sie war stärker als ich, ich musste tun, was sie befahl. Musste ich
es wirklich? Nun, bis zum Abend war noch viel Zeit\ldots{} Das Billett
lag auf dem Tisch. Wie ärgerlich, dass ich kein ärztliches Attest
für heute hatte. Ich wollte gehen, endlos lange gehen, die ganze
Grüne Mauer entlang, und mich dann auf mein Bett werfen, mich
fallen lassen. Aber ich musste zum Auditorium 13, musste zwei
Stunden, zwei geschlagene Stunden auf einem Fleck sitzen, ohne mich
zu rühren.

In der Vorlesung. Sonderbar — aus dem blinkenden Apparat klang
nicht die gewohnte metallische Stimme, sondern eine andere, weich
und zart wie Moos. Es war eine Frauenstimme, sie erinnerte mich an
die der Alten im Alten Haus.

Das Alte Haus,.. alles schlug wie eine Woge über mir zusammen, ich
musste an mich halten, um nicht laut aufzuschreien.

Ich lauschte der weichen Stimme, ohne die einzelnen Worte in mich
aufzunehmen. Ich war wie eine fotografische Platte, alles zeichnete
sich seltsam scharf darauf ab: die Lichtreflexe auf dem
Lautsprecher, das Kind darunter — die lebendige Illustration der
Vorlesung —, der kleine Mund, der an einem Zipfel der winzigen
Uniform lutschte, die geballten Fäustchen, die Falten im
Handgelenk. Ich registrierte: Jetzt baumelt ein nacktes Bein über
den Tischrand, die
kleinen Hände greifen in die Luft, gleich wird das Kind
herunterfallen. Da — ein Schrei. Eine Frau fliegt zum Podium, fängt
das Kind auf, legt es auf die Mitte des Tisches und kehrt zu ihrem
Platz zurück. Ich sah rosige, sanft geschwungene Lippen, feuchte
blaue Augen. Es war O! Jäh erkannte ich die fast mathematische
Gesetzmäßigkeit, die Notwendigkeit dieses unbedeutenden
Zwischenfalls.

Sie saß eine Reihe hinter mir. Ich wandte mich um; gehorsam blickte
sie von dem Kind auf dem Tisch weg und sah mich an. Sie, ich und
der Tisch auf dem Podium waren drei durch eine Linie verbundene
Punkte, und diese Linie bildete die Projektion unvermeidlicher,
noch unsichtbarer Ereignisse.

Als ich nach Hause ging, lagen die Straßen in grünlicher Dämmerung,
die Lampen glühten wie feurige Augen. In mir tickte eine Uhr. Im
nächsten Augenblick würden die Zeiger eine gewisse Zahl
überschreiten, und dann würde ich etwas Unerhörtes tun. Sie wollte,
die Leute in ihrem Haus sollten denken, sie sei bei mir. Ich aber
wollte sie, und was gingen mich ihre Wünsche an! Ich hatte keine
Lust, um fremder Leute willen die Vorhänge zuzuziehen.

Tappende, schlurfende Schritte hinter mir. Ich drehte mich nicht
um, weil ich wusste, dass es S war. Er folgt mir bis zur Haustür,
dachte ich, dann wird er wahrscheinlich von der anderen
Straßenseite aus mein Zimmer beobachten, bis ich die Vorhänge
schließe, die das Verbrechen dieser Frau verbergen sollen ;.. Nein,
er, mein Beschützer, mein Schutzengel, hatte die Sache entschieden
— ich war fest entschlossen, die Vorhänge nicht zuzuziehen! Als ich
in meinem Zimmer Licht machte, sah ich O, die an meinem Tisch
stand. Sie hatte sich unheimlich ver-

ändert, die Kleider schlotterten um ihren Leib, ihre Arme hingen
kraftlos herab, ihre Stimme klang matt und brüchig.

„Ich bin wegen meines Briefes gekommen. Haben Sie ihn erhalten? Ich
muss eine Antwort haben — gleich.“ Ich zuckte die Achseln und
blickte vorwurfsvoll in ihre blauen Augen, als wäre sie an allem
schuld. Nach langem Schweigen sagte ich boshaft, jedes Wort scharf
betonend: „Eine Antwort? Sie haben ja recht, vollkommen recht. In
allem.“

„Das heißt also\ldots{}“ Sie versuchte ihr Zittern hinter einem
krampfhaften Lächeln zu verbergen, aber ich bemerkte es dennoch.
„Gut. Ich werde sofort\ldots{} sofort gehen.“

Doch sie rührte sich nicht vom Fleck, sondern blieb mit
niedergeschlagenen Augen und mit hängenden Schultern stehen. Auf
dem Tisch lag noch das zerknüllte rosa Billett der anderen. Ich
versteckte es schnell unter einer Seite meines Manuskripts
(vielleicht mehr vor mir selber als vor O).

„Da sehen Sie, ich schreibe ununterbrochen. 170 Seiten sind es
schon\ldots{} Es wird etwas ganz anderes, als ich selber vermutet
habe\ldots{} “ Ihre Stimme — nur der Schatten einer Stimme: „Wissen Sie
noch\ldots{} die siebente Seite\ldots{} ich weinte, und eine Träne fiel auf
diese Seite — und Sie\ldots{} “ Sie hielt inne. Aus ihren großen blauen
Augen stürzten Tränen. Erregt sagte sie:

„Ich kann nicht mehr, ich gehe\ldots{} ich werde nie wiederkommen. Aber
ich möchte\ldots{} ich muss ein Kind von Ihnen haben. Schenken Sie mir
ein Kind, und dann gehe ich für immer.“ Ich sah, wie sie unter der
Uniform am ganzen Leibe
zitterte, und ich fühlte, dass auch ich in diesem Augenblick\ldots{} Ich
legte die Hände auf den Rücken und sagte lächelnd: „Sie wollen wohl
auf der Maschine des Wohltäters enden?“

„Das kümmert mich nicht! Aber ich fühle es doch schon in mir, fühle
es ganz deutlich. Und wenn ich es nur einen Tag bei mir habe, nur
ein einziges Mal die kleine Falte an seinem Ärmchen sehen kann, so
wie dort auf dem Tisch.“

Ich musste wieder an die drei Punkte denken, sie und ich und das
geballte Fäustchen auf dem Tisch im Auditorium.

Als Schuljungen wurden wir einmal auf den Akkumulatorenturm
geführt. Auf der obersten Plattform beugte ich mich über das
gläserne Geländer und blickte hinunter. Die Menschen drunten waren
winzige Punkte. Ein leichter Schwindel überkam mich: Wie, wenn ich
jetzt hinunterstürze? Damals klammerte ich mich mit aller Kraft an
das Geländer, jetzt aber würde ich hinunterspringen. „Wollen Sie es
wirklich? Obwohl Sie genau wissen\ldots{} “ Sie schlug ihre blauen Augen
auf und lächelte unter Tränen: „Ja, ich will es!“

Ich zog das rosa Billett der anderen unter dem Manuskript hervor
und lief zur Aufsicht hinunter. O fasste mich am Arm und schrie
etwas, das ich aber erst verstand, als ich zurückkam. Sie saß auf
der Bettkante, die Hände im Schoß gefaltet.

„Beeilen Sie sich\ldots{} “ Ich packte sie grob am Handgelenk (morgen
wird sie blaue Flecken haben, genau an der Stelle, wo die kindliche
Speckfalte ist). Dann wurde der Ausschalter gedreht, die Gedanken
erloschen, Finsternis, Funken — und ich stürzte über die Brüstung
in die Tiefe.

\section{EINTRAGUNG NR. 20}

\uebersicht{\emph{Übersicht:} Entladung. Ideenmaterial. Die
Null-Klippe.}
Entladung — das ist der passendste Ausdruck. Jetzt sehe ich, dass
das alles eine elektrische Entladung war. In den letzten Tagen war
mein Pulsschlag immer lauter, rascher, schmerzhafter geworden — die
Pole kamen einander näher und näher — ein trockenes Knacken — noch
ein Millimeter. Explosion, dann tiefe Stille. In mir ist es jetzt
still und leer, wie in einem Hause, wenn alle Bewohner ausgegangen
sind, während man selber krank zu Bett liegt und dem metallischen
Hämmern der Gedanken lauscht.

Es mag sein, dass die „Entladung“ mich endlich von der mich
quälenden Seele geheilt hat und dass ich nun wieder so bin wie wir
alle. Wenigstens sehe ich jetzt im Geist O auf den Stufen des
Würfels, sehe sie unter der Gasglocke — und empfinde dabei nicht
den geringsten Schmerz. Es ist mir auch gleichgültig, ob sie im
Operationssaal meinen Namen angibt, in meiner letzten Minute werde
ich dankbar und ergeben die strafende Hand des Wohltäters küssen.
Dem Einzigen Staat gegenüber habe ich das Recht, eine Strafe zu
erleiden, und dieses Recht lasse ich mir keinesfalls nehmen. Keine
von uns Nummern kann und darf es wagen, auf dieses ihr einziges und
darum um so teures Recht zu verzichten\ldots{} \ldots{} Leise,
metallisch-klar hämmerten meine Gedanken; irgendein Flugzeug trug
mich in die blauen Höhen meiner geliebten Abstraktionen empor. In
der reinen Höhenluft zerplatzten meine Betrachtungen über dieses
„Recht“, und ich erkannte, dass es nur Reminiszenzen an lächerliche
Vorurteile unserer Ahnen, an ihre Rechtsideen waren.

Es gibt Ideen, die einem irdenen Topf gleichen, und es gibt Ideen,
die für die Ewigkeit aus Gold oder aus unserem kostbaren Glas
gegossen sind. Um das Material einer Idee zu bestimmen, braucht man
es nur mit einer stark wirkenden Säure zu beträufeln. Eine dieser
Säuren war unseren Vorfahren bereits bekannt, die reductio ad
finem. Ich glaube, so nannte man das damals; doch sie fürchteten
dieses Gift, sie wollten irgend etwas vor sich sehen, ganz gleich,
was es war, sie zogen einen Spielzeughimmel dem blauen Nichts vor.
Wir aber — dem Wohltäter sei Dank — sind erwachsen, brauchen kein
Spielzeug. Nehmen wir an, man würde die Idee Wahrheit mit Säure
benetzen. Schon in alten Zeiten wussten die größten Geister, dass
die Quelle der Wahrheit die Macht, die Wahrheit also eine Funktion
der Macht ist. Oder stellen wir uns zwei Waagschalen vor: auf der
einen liegt ein Gramm, auf der anderen eine Tonne, auf der einen
das Ich, auf der anderen Wir, der Einzige Staat. Dem Ich
irgendwelche Rechte dem Einzigen Staat gegenüber einzuräumen, wäre
das gleiche, wie wenn man behaupten wollte, dass ein Gramm eine
Tonne aufwiegen könne. Daraus ergibt sich der Schluss: Die Tonne
hat Rechte, das Gramm Pflichten, und der einzige natürliche Weg von
der Nichtigkeit zur Größe ist: Vergiss, dass du nur ein Gramm bist,
und fühle dich als millionsten Teil einer Tonne. Ihr, üppige,
rotwangige Venusbewohner, und ihr, Uranusmenschen, rußig wie
Schmiede — ich höre in meiner blauen Stille euer Murren. Doch
wisset: alles Erhabene ist einfach, wisset: unerschütterlich und
ewig sind nur die vier Grundregeln der Arithmetik. Und nur die
Moral wird erhaben, unerschütterlich und ewig bleiben, die sich auf
diese vier Regeln gründet. Das ist die letzte Weisheit, die Spitze
jener Pyramide, welche die

Menschen rot vor Anstrengung, ächzend und stöhnend jahrhundertelang
zu erklimmen versuchten. Und wenn man von diesem Gipfel in die
Tiefe blickt, wo gleich nichtigen Würmern noch etwas wimmelt, das
von unseren unzivilisierten Ahnen her in uns fortlebt — wenn man
von diesem Gipfel hinabblickt, sind alle gleich: O, die
ungesetzliche Mutter, der Mörder und jener Wahnsinnige, der sich
vermaß, den Einzigen Staat mit seinen Versen zu schmähen. Sie alle
erwartet das gleiche Gericht: ein vorzeitiger Tod. Das ist nichts
anderes als jene göttliche Gerechtigkeit, von der die
Steinhausmenschen, von dem rosigen, naiven Morgenschein der
Geschichte erleuchtet, phantasierten: ihr Gott — Schmach und
Schande über die Heilige Kirche — strafte, indem er mordete.

Nun, ihr Uranusbewohner, düster und schwarz wie die alten Spanier,
die sich so gut darauf verstanden, Menschen auf Scheiterhaufen zu
verbrennen — was schweigt ihr? Ich glaube, ihr seid auf meiner
Seite. Doch ich höre die rosigen Venusbewohner etwas von Foltern,
Hinrichtungen, von der Rückkehr zu barbarischen Zuständen murmeln.
— Ihr tut mir leid, meine Lieben, ihr seid nicht fähig,
philosophisch-mathematisch zu denken. Die Geschichte der Menschheit
bewegt sich in Kreisen nach oben, genau wie ein Flugzeug. Es gibt
verschiedene solcher Kreise, goldene und blutige, aber sie sind
alle in 360 Grad eingeteilt. Und nun geht es vom Nullpunkt
vorwärts: 10, 20, 200, 360 Grad — und dann wieder zu ihm zurück.
Ja, wir sind zum Nullpunkt zurückgekehrt! Aber für meinen
mathematisch geschulten Verstand ist es völlig klar, dass dieser
Nullpunkt etwas gänzlich anderes, Neues ist. Wir sind von Null nach
rechts gegangen und kehren von links nach Null zurück, und deshalb
haben wir statt plus null minus null. Verstehen Sie?

Ich sehe diese Null als ein riesiges, stummes, messerscharfes
Felsenriff. In wilder, undurchdringlicher Finsternis stießen wir
mit angehaltenem Atem in unserem Boot von der schwarzen Nachtseite
der Klippe Null ab. Jahrhundertelang trieben wir, gleich Kolumbus,
dahin, wir umfuhren die ganze Erde, und endlich — hurra! Salut! Vor
uns lag die andere, bisher unbekannte Seite der Nullklippe im
kalten Nordlicht des Einzigen Staates, ein blauer Klumpen, Funken
eines Regenbogens, Sonne — Hunderte von Sonnen, Millionen
Regenbogen\ldots{} Was hat es schon zu bedeuten, dass nur ein
messerdünner Grat uns von der anderen, der schwarzen Seite der
Nullklippe trennt? Das Messer ist das Dauerhafteste, Unsterblichste
und Genialste von allem, was der Mensch geschaffen hat. Das Messer
war eine Guillotine, das Messer ist ein Universalmittel zur Lösung
aller Knoten, und der Weg der Paradoxe führt auf des Messers
Schneide entlang — der einzig würdige Weg eines furchtlosen
Geistes.

\section{EINTRAGUNG NR. 21}

\uebersicht{\emph{Übersicht:} Die Pflicht des Autors. Das Eis
schwillt an. Die schwierigste Form der Liebe.}
Gestern war ihr Tag, aber sie kam nicht, sie schickte nur ein paar
unverständliche Zeilen, die nichts erklärten. Doch ich war ruhig,
völlig ruhig. Wenn ich tat, wie sie mich in ihrem Brief geheißen,
wenn ich ihr Billett zum Hausmeister brachte und hinter
geschlossenen Vorhängen allein in meinem Zimmer saß, so geschah das
selbstverständlich nicht deshalb, weil ich zu schwach war, mich
ihrem Willen zu widersetzen. Lächerlich! Es war einfach so: Die
Vorhänge isolierten mich von einem gewissen Lächeln, das sich wie
ein heilendes Pflaster auf meine Wunden legen wollte, und so konnte
ich diese Seiten ungestört niederschreiben; das war das erste. Und
das zweite: Ich fürchtete wohl, in I den Schlüssel zu allen
Geheimnissen zu verlieren (die Geschichte mit dem Schrank, meine
Ohnmacht usw.). Aber ich fühle mich verpflichtet, diese Rätsel zu
lösen, allein schon darum, weil ich der Autor dieser Aufzeichnungen
bin, ganz abgesehen davon, dass alles Unbekannte dem Menschen von
Natur aus zuwider ist; der homo sapiens ist nur dann ein Mensch im
vollen Sinne des Wortes, wenn es in seiner Grammatik keine
Fragezeichen, sondern nur Ausrufzeichen, Punkte und Kommata gibt.
Um meiner Pflicht als Autor zu genügen, nahm ich heute um 16 Uhr
ein Flugzeug und flog zum Alten Haus. Ich hatte starken Gegenwind.
Die Maschine kämpfte sich mühsam durch das Luftdickicht,
durchsichtige Zweige schlugen mir pfeifend ins Gesicht. Die Stadt
unter mir schien aus blauen Eisblöcken zu bestehen. Da — eine
Wolke, ein schneller, schräger Schatten — das Eis verfärbte sich
bleigrau und schwoll an wie im Frühling, wenn man am Ufer steht und
darauf wartet, dass im nächsten Augenblick alles kracht,
zerspringt, sich losreißt, dahintreibt. Eine Minute nach der
anderen vergeht, doch das Eis bleibt starr, und in einem selbst
schwillt etwas an, immer rascher, immer ungestümer pocht das
Herz\ldots{} (Übrigens, warum schreibe ich eigentlich von solchen
Dingen, und woher kommen diese seltsamen Gefühle? Denn es gibt ja
keinen Eisbrecher, der das reine, feste Kristall unseres Lebens
brechen könnte\ldots{}) Am Tor des Alten Hauses war kein Mensch zu
sehen.

Ich ging um das ganze Gebäude herum und fand die Alte bei der
Grünen Mauer; sie hielt die Hand schützend über die Augen und
blickte nach oben. Über der Mauer schwebten spitze, schwarze
Dreiecke — Vögel. Krächzend warfen sie sich mit der Brust gegen das
Schutzgitter aus elektrischen Wellen, kehrten um und kamen wieder
zurück.

Über das dunkle, runzlige Gesicht der Alten huschten flüchtige
Schatten, sie warf mir einen kurzen Blick zu: „Niemand da! Sie
brauchen gar nicht erst hineinzugehen\ldots{} “

Was sollte das? Und welch merkwürdiges Benehmen — mit mir
umzugehen, als wäre ich nur ein Schatten. Vielleicht seid ihr alle
nichts weiter als meine Schatten, dachte ich, denn ich habe euch
das Leben gegeben, indem ich euch auf diesen Seiten ansiedelte, die
noch vor kurzem rechteckige, weiße Wüsten waren. Wenn ich nicht
wäre, würden alle jene, die ich durch die engen Pfade meiner Zeilen
führe, euch niemals sehen!

Ich sagte ihr natürlich kein Wort davon, denn ich weiß aus eigener
Erfahrung, dass es die größte Qual für einen Menschen ist, wenn man
seine Realität bezweifelt, seine dreidimensionale Realität, nicht
irgendeine andere. Ich bemerkte nur trocken, sie sei dazu da, die
Tür zu öffnen, worauf sie mich in den Hof ließ.

Leer. Still. Hinter der Grünen Mauer heulte der Wind, fern wie
jener Tag, da wir Schulter an Schulter aus den unterirdischen
Gängen emporstiegen — wenn das wirklich geschehen ist. Ich ging
durch steinerne Arkaden; in den hohen, feuchten Gewölben hallten
meine Schritte wider, und es klang, als folgte mir jemand dicht auf
den Fersen. Gelbe verwitterte Backsteinmauern blickten mir aus
dunklen, quadratischen Fensterhöhlen nach, beobachteten, wie ich
kreischende Scheunentüren öffnete, wie ich in verlassene Winkel,
Sackgassen und Seitengänge spähte. Ich ging durch ein kleines Tor
im Zaun — dahinter ein Trümmerfeld, ein Andenken an den Großen
Zweihundertjährigen Krieg: Aus der Erde ragten nackte, steinerne
Rippen, gelbe ausgebrannte Mauern, ein uralter Ofen mit einem
langen Rohr — ein für alle Zeiten versteinertes Schiff inmitten
starrer Wogen aus gelbem Stein und roten Ziegeln.

Mir war, als hätte ich diese gelben Zähne schon einmal gesehen,
undeutlich, als lägen sie unter dichten Wassermassen begraben — und
ich begann zu suchen. Ich stürzte in Gruben, stolperte über Steine,
dornige Ranken krallten sich in meine Uniform, dicke Schweißtropfen
rannen über mein Gesicht\ldots{} Umsonst! Ich konnte den Ausgang des
unterirdischen Korridors nirgends entdecken, er war verschwunden.
Aber vielleicht war es ganz gut so, denn damit wurde es mir
wahrscheinlicher, dass ich das alles nur geträumt hatte. Erschöpft,
ganz von Spinnweben und Staub bedeckt, kehrte ich in den Haupthof
zurück. Plötzlich ein leises Rascheln, tappende Schritte, und ich
erblickte die abstehenden rosa Ohren und das listige Lächeln von S.
Er runzelte die Stirn und sah mich durchdringend an: „Gehen Sie
spazieren?“

Ich antwortete nicht. Meine Arme waren mir im Wege. „Fühlen Sie
sich ein wenig besser?“

„Ja, danke. Ich glaube, ich bin bald wieder ganz gesund.“

Er trat ein wenig zur Seite und blickte angestrengt nach oben. Sein
Kopf war weit zurückgebogen, und ich sah zum ersten Mal seinen
Adamsapfel. Über uns brummten Flugzeuge, nicht sehr hoch, etwa
fünfzig Meter. An der geringen Flughöhe und an den zur Erde
gerichteten Fernrohren erkannte ich, dass es Maschinen der
Beschützer waren. Aber es waren nicht zwei oder drei wie
gewöhnlich, sondern zehn oder zwölf.

„Warum so viele?“ fragte ich.

„Warum? Hm\ldots{} Ein guter Arzt beginnt bereits beim gesunden Menschen
mit der Behandlung. Das nennt
man Prophylaxe!“

Er nickte mir zu und lief über die steinernen Fliesen des Hofes
davon. Dann wandte er sich noch einmal um
und rief: „Seien Sie vorsichtig!“

Ich war allein. Stille. Leere. Über der Grünen Mauer flatterten
Vögel, der Wind heulte. Was wollte er damit sagen?

Mein Flugzeug glitt rasch dahin. Die Wolken warfen schwere
Schatten, die blauen Kuppeln, die Würfel aus
gläsernem Eis tief unter mir wurden bleigrau, schwollen an\ldots{}

Abends:

Ich schlug mein Manuskript auf, um einige Gedanken über den Tag der
Einigkeit festzuhalten, den wir in Kürze begehen. (Ich glaube, dass
diese Gedanken für meine Leser recht nützlich sind.) Aber ich
konnte nicht schreiben. Ich lauschte die ganze Zeit, wie der Wind
mit dunklen Schwingen gegen die gläsernen Mauern des Hauses schlug,
ich schaute mich in einem fort um. Ich wartete\ldots{} Worauf wartete
ich? Ich wusste es nicht. Und ich freute mich wirklich, als ich die
roten Hängebacken von U in meinem Zimmer sah. Sie setzte sich, zog
den Rock über die Knie, lächelte mir aufmunternd zu, und ihr
Lächeln war ein Pflaster für meine Wunden. „Da komme ich heute früh
in meine Klasse“ (sie arbeitet
in der Erziehungsfabrik) „und entdecke eine Karikatur an der Wand.
Stellen Sie sich vor, eine Karikatur von mir, auf der ich wie ein
Fisch aussehe. Nun, vielleicht habe ich wirklich etwas von einem
Fisch.“ „Wie können Sie so etwas sagen!“ erwiderte ich rasch. (Aus
der Nähe betrachtet, hat sie wirklich nichts von einem Fisch, und
was ich von ihren „Kiemen“ geschrieben habe, trifft durchaus nicht
zu.) „Im Grunde genommen ist es ja auch gleichgültig. Nur, dass man
es überhaupt gewagt hat! Ich habe es natürlich gleich den
Beschützern gemeldet. Ich liebe Kinder sehr, und ich glaube, dass
die schwierigste, höchste Form der Liebe die Grausamkeit ist,
verstehen Sie?“ Wie sollte ich das nicht verstehen? Es deckte sich
genau mit meinen eigenen Gedanken. Ich konnte nicht umhin, ihr
einen Abschnitt aus meinen Aufzeichnungen vorzulesen, und zwar
folgende Stelle:

„Ganz leise, metallisch klar, hämmerten meine Gedanken\ldots{} “

Ihre roten Wangen begannen zu zittern, sie kamen immer näher, und
schon umklammerten ihre trockenen, harten Finger meine Hand.

„Geben Sie mir das! Ich lasse es auf Schallplatten aufnehmen, damit
die Kinder es auswendig lernen können. Wir brauchen das ebenso
nötig wie ihre Venusbewohner, vielleicht noch nötiger.“ Sie schaute
sich um und flüsterte:

„Haben Sie schon gehört? Es wird erzählt, am Tag der Einstimmigkeit
soll\ldots{}“ Ich sprang auf:

„Was\ldots{} soll am Tag der Einstimmigkeit?“ Meine gemütlichen vier
Wände existierten plötzlich nicht mehr. Mir war, als wäre ich ins
Freie geschleudert worden, wo ein wilder Sturm über die Dächer
fegte, wo düstere Wolken in steilem Flug zur Erde hinabstürzten. U
legte mir den Arm um die Schultern und sagte mit fester Stimme:

„Setzen Sie sich, mein Lieber, und regen Sie sich nicht unnütz auf.
Was wird nicht alles erzählt\ldots{} Wenn Sie wollen, werde ich an dem
Tag bei Ihnen sein. Ich bitte irgend jemanden, auf meine
Schulkinder aufzupassen, und komme zu Ihnen. Denn Sie sind auch ein
Kind, mein Lieber, und brauchen\ldots{} “

„Nein, nein!“ unterbrach ich sie und machte eine abwehrende Geste.
„Auf keinen Fall! Sonst denken Sie wirklich, ich sei ein Kind und
könne nicht ohne Aufsicht sein. Nein, auf keinen Fall!“ (Ich muss
gestehen, dass ich für diesen Tag andere Pläne hatte.)

Sie lächelte, was offensichtlich bedeutete: „Ach, du eigensinniger
Junge!“ Dann setzte sie sich. Sie schlug die Augen nieder, zog den
Rock schamhaft über die Knie und sprach von etwas anderem:

„Ich denke, ich muss mich nun doch entschließen\ldots{} um
Ihretwillen\ldots{} Nein, bitte, drängen Sie mich nicht, ich muss mir
alles genau überlegen\ldots{} “ Ich drängte sie ja gar nicht. Obwohl ich
einsah, dass ich darüber glücklich sein musste und dass es eine
große Ehre für mich war, den Lebensabend eines Menschen krönen zu
dürfen\ldots{}. Die ganze Nacht hörte ich das Schlagen schwerer Flügel,
und ich legte die Hände vors Gesicht, um es vor diesen Flügeln zu
schützen. Dann — ein Stuhl. Doch es war keiner unserer modernen
Stühle, sondern ein altmodischer, aus Holz. Er trabte wie ein Pferd
— rechtes Vorderbein und linkes Hinterbein, linkes Vorderbein und
rechtes Hinterbein —, der Stuhl näherte sich meinem

Bett, sprang auf die Bettdecke, und ich umarmte den hölzernen
Stuhl. Das war sehr unbequem und tat weh. Gibt es denn wirklich
kein Mittel, das dieses krankhafte Träumen beseitigt oder es
vernünftig, vielleicht sogar nützlich macht?

\section{EINTRAGUNG NR. 22}

\uebersicht{\emph{Übersicht:} Erstarrte Wogen. Alles wird
vollkommen. Ich bin eine Mikrobe.}
Stellen Sie sich vor, Sie ständen an einem Ufer: die Wellen steigen
langsam, und plötzlich verharren sie still, erkalten, erstarren.
Genauso schrecklich und unnatürlich war, was bei unserem gesetzlich
vorgeschriebenen Spaziergang geschah:

Die Reihen der Nummern stockten, gerieten in Verwirrung und blieben
stehen.

Etwas Ähnliches hat sich zum letzten Mal vor 119 Jahren ereignet,
als ein Meteor zischend und rauchend vom Himmel herabstürzte und
mitten unter die Spaziergänger fiel.

Wir marschierten wie immer, gleich einer Armee, so, wie sie auf
assyrischen Reliefs dargestellt ist: Tausend Köpfe, zwei Beine,
zwei emporgestreckte Arme. Vom Ende des Prospekts, wo der
Akkumulatorenturm drohend summte, kam uns ein Viereck entgegen: an
beiden Seiten, vorn und hinten Wachen, in der Mitte drei Männer,
denen man das goldene Abzeichen mit der Nummer abgenommen hatte —
alles erschreckend klar zu deuten. Das riesige Zifferblatt am Turm
war wie ein Gesicht, es neigte sich aus den Wolken nieder, spie die
Sekunden nach unten und wartete gleichgültig. Und da, um 13 Uhr 6
Minuten, geriet das Viereck in Verwirrung. Es geschah ganz in
meiner Nähe, und ich erinnere mich deutlich an einen langen, dünnen
Hals und an Schläfen, die von einem Geflecht blauer Adern
durchzogen waren, wie Flüsse auf der Karte einer kleinen,
unbekannten Welt, und diese unbekannte Welt war ein junger Mann. Er
hatte offenbar jemanden unter den Spaziergängern entdeckt, denn er
reckte den Hals und blieb stehen. Eine der Wachen schlug ihn mit
der elektrischen Knute so heftig, dass blaue Funken sprühten. Er
wimmerte leise wie ein junger Hund. Und dann alle zwei Sekunden:
Schlag — Wimmern — Schlag — Wimmern.

Wir gingen ruhig weiter, und als ich die anmutigen Zickzacklinien
der Funken sah, dachte ich: Alles in der menschlichen Gesellschaft
wird unendlich vollkommen werden, und so muss es auch sein. Welch
hässliche Waffe war die Knute unserer Ahnen, und wie viel
Schönheit\ldots{} In diesem Augenblick löste sich eine schlanke,
biegsame Frauengestalt aus unserer Gruppe und warf sich mit dem
Schrei: „Genug! Rührt ihn nicht an!“ dem Viereck entgegen. Das
wirkte wie das Meteor vor 119 Jahren: Der ganze Spaziergang
stockte, und unsere Reihen wurden zu grauen Wogenkämmen, die der
Frost jäh hatte erstarren lassen.

Sekundenlang blickte ich diese Frau mit dem gleichen Entsetzen an
wie alle anderen: sie war keine Nummer mehr, sie war ein Mensch,
sie existierte nur noch als metaphysische Substanz einer
Beleidigung, drehte sich um und wiegte sich dabei leicht in den
Hüften — und plötzlich wurde mir klar, dass ich diesen
gertenschlanken, biegsamen Körper kannte — meine Augen, meine
Lippen, meine Hände kannten ihn! In diesem Augenblick war
ich jedenfalls fest davon überzeugt! Zwei Mann von der Wache traten
ihr entgegen. Gleich würden sie zuschlagen, ihre erhobenen Knuten
spiegelten sich im klaren Glas des Pflasters\ldots{} Mir stand das Herz
still, und ohne lange zu überlegen, stürzte ich hinzu\ldots{}

Ich fühlte, wie Tausende von Augenpaaren erschrocken auf mir
ruhten, doch gerade das gab dem Wilden mit den behaarten Händen,
der sich von mir losriss, noch mehr verzweifelte Kraft, und er lief
um so schneller. Noch zwei Schritte — da wandte sie sich um. Ein
sommersprossiges Gesicht, rötliche Augenbrauen\ldots{} Sie war es nicht!
Es war nicht I!

Eine tolle Freude stieg in mir auf. Ich wollte laut schreien:
„Packt sie! Haltet sie!“ oder etwas Ähnliches, doch ich brachte
keinen Ton heraus. Schon legte sich eine schwere Hand auf meine
Schulter, und man führte mich ab. Ich versuchte, der Wache
klarzumachen\ldots{} „Hören Sie! Begreifen Sie doch! Ich dachte,
dass\ldots{}“ Aber wie sollte ich ihnen erklären, was in mir vorging,
meine Krankheit erklären, die ich in diesen Aufzeichnungen
beschrieben habe! Ich verstummte und folgte ihnen ergeben\ldots{} Ein
Blatt, das ein jäher Windstoß vom Baum reißt, sinkt gehorsam zur
Erde, aber im Fallen dreht es sich verzweifelt, klammert sich an
jeden vertrauten Zweig, an jedes Ästchen — so klammerte ich mich
hilfesuchend an jeden der schweigenden Kugelköpfe, an das
durchsichtige Eis der Mauern, an die goldene Spitze des
Akkumulatorenturms.

Genau in dem Augenblick, da der Vorhang des Schweigens mich für
immer von dieser herrlichen Welt zu trennen drohte, tauchte in der
Nähe ein bekanntes Gesicht auf. Der Mann fuchtelte erregt mit
seinen rosigen Händen, und dann sagte eine vertraute Stimme:

„Ich halte es für meine Pflicht, zu bezeugen, dass D-503 krank und
nicht fähig ist, seine Gefühle zu kontrollieren. Ich bin überzeugt,
dass er sich von einem ganz natürlichen Unwillen hat hinreißen
lassen\ldots{} “

„Ja, ja“, unterbrach ich ihn. „Ich hab? sogar geschrieen: Haltet
sie!“

Hinter mir sagte jemand: „Gar nichts haben Sie geschrieen!“

„Aber ich wollte schreien — ich schwöre beim Wohltäter, dass ich
schreien wollte.“

S warf mir einen kalten, durchdringenden Blick zu. Ich weiß nicht,
ob er tief in meinem Inneren sah, dass ich beinahe die Wahrheit
sprach, oder ob er mich nur noch für eine Weile schonen wollte,
jedenfalls schrieb er ein paar Worte auf einen Zettel, gab ihn
einem der Wachleute — und ich war frei, war wieder in die
geordneten, endlosen assyrischen Reihen eingegliedert.

Das sommersprossige Gesicht und die blaugeäderten Schläfen
verschwanden um die Straßenecke — für immer. Wir marschierten
weiter, ein Körper mit Millionen Köpfen, und in jedem von uns war
jene stille, demütige Freude, in der wahrscheinlich Atome, Moleküle
und Phagozyten leben. In der alten Welt wussten die Christen als
einzige unserer wenn auch sehr unvollkommenen Vorgänger, dass Demut
eine Tugend, Stolz hingegen ein Laster ist, dass Wir von Gott
stammt und Ich vom Teufel. Ich marschierte im gleichen Schritt mit
den anderen und war trotzdem von ihnen getrennt. Dieser
Zwischenfall hatte mich so erregt, dass ich noch am ganzen Leibe
zitterte. Ich fühlte mich. Alle jene, die sich fühlen, sind sich
ihrer Individualität bewusst. Doch nur das entzündete Auge, der
verletzte Finger, der kranke Zahn machen sich
bemerkbar, das gesunde Auge, der gesunde Finger, der gesunde Zahn
scheinen nicht vorhanden zu sein. Man ist also bestimmt krank, wenn
man sich der eigenen Persönlichkeit bewusst wird!

Ich bin vielleicht schon keine Phagozyte mehr, die eifrig die
zerstörenden Mikroben verschlingt (solche wie der junge Mann mit
den blauen Adern und die sommersprossige Frau), ich bin selber eine
Mikrobe, und vielleicht gibt es bereits Tausende unter uns, die wie
ich nur so tun, als wären sie Phagozyten.

Was aber, wenn dieser heutige an sich recht unbedeutende
Zwischenfall nur ein Anfang ist, nur der erste Meteor eines ganzen
Hagels glühender Steine, der aus der Unendlichkeit auf unser
Paradies herabstürzt?

\section{EINTRAGUNG NR. 23}

\uebersicht{\emph{Übersicht:} Blumen. Die Auflösung des Kristalls.
Wenn\ldots{}}
Man sagt, es gäbe Blumen, die nur alle hundert Jahre blühen. Warum
sollte es dann nicht auch solche geben, die nur einmal in tausend
Jahren oder gar in zehntausend Jahren blühen? Vielleicht haben wir
nur deshalb nichts davon gewusst, weil dieses
Einmal-in-tausend-Jahren gerade heute ist.

Trunken vor Glück ging ich ins Vestibül hinunter, und vor meinen
Augen öffneten sich überall tausendjährige Knospen und erblühten —
Sessel, Stiefel, goldene Abzeichen, elektrische Lampen, dunkle,
sanfte Augen, die blitzenden Glasstäbe am Treppengeländer, ein
buntes Tuch, das jemand auf der Treppe verloren hatte, der kleine
Tisch in der Pförtnerloge, die zarten rotgesprenkelten Wangen von
U. Alles war ungewöhnlich, neu, zart, rosig, frisch.

U nahm mein rosa Billett entgegen; über ihrem Kopf — ich blickte
durch die durchsichtige Glasmauer — da schwebte der Mond an einem
unsichtbaren Zweig, blau und duftend.

Ich hob die Hand und sagte zu U: „Der Mond — sehen Sie doch!“

Sie schaute zuerst auf mich, dann auf das Billett und zog mit einer
bezaubernd-verschämten Geste den Rock über die Knie.

„Sie sehen heute anomal und krank aus, mein Lieber. Anomalität und
Krankheit sind ein und dasselbe. Sie richten sich zugrunde, aber
niemand sagt Ihnen das, niemand.“

Mit diesem „niemand“ ist natürlich die Nummer auf dem Billett
gemeint: I-330. Liebe, gute U! Sie haben natürlich recht, ich bin
unvernünftig und krank, ich habe eine Seele, ich bin eine Mikrobe.
Aber ist denn das Blühen keine Krankheit? Schmerzt es nicht, wenn
die Knospen aufbrechen? Meinen Sie nicht auch, dass das
Spermatozoid die entsetzlichste aller Mikroben ist?

Ich ging auf mein Zimmer. In der flachen Schale des Sessels saß I.
Ich warf mich ihr zu Füßen, umschlang ihre Knie und legte meinen
Kopf in ihren Schoß. Wir schwiegen. Stille. Mein Herz klopfte zum
Zerspringen. Mir war, als wäre ich ein Kristall, ich löste mich in
I auf. Ich fühlte, wie die geschliffenen Facetten, die mich daran
hinderten mich auszudehnen, schmolzen, und ich verschwand in ihrem
Schoß, ich wurde immer winziger — und zugleich immer breiter,
größer, unermesslicher. Denn sie war nicht mehr I, sie war das
Weltall. Eine Sekunde lang waren wir allein, ich und dieser von
Freude erfüllte Sessel neben
meinem Bett — und das strahlende Lächeln der Alten vom Alten Haus,
das wilde Dickicht hinter der Grünen Mauer, silbrige Ruinen auf
schwarzem Grund, die vor sich hin träumten wie jene Alte, und eine
Tür, die in unendlicher Ferne zuschlug — all das war in mir und
hörte meinen Pulsschlag\ldots{}

Mit wirren, unzusammenhängenden Sätzen versuchte ich ihr zu
berichten, dass ich ein Kristall sei, dass eine Tür in mir
zuschlage; doch was ich sagte, war so widersinnig, dass ich
beschämt innehielt und dann murmelte: „Verzeih mir Liebste. Ich
weiß wirklich nicht, weshalb ich so dummes Zeug schwatze\ldots{} “

„Warum hältst du dieses dumme Zeug für etwas Schlechtes? Wenn man
die Dummheiten der Menschen jahrhundertelang so hegte und pflegte
wie die Vernunft, dann würde man vielleicht etwas sehr Kostbares
erhalten.“ „Ja\ldots{}“ (Ich glaubte, dass sie recht hatte. Wie hätte
sie in diesem Augenblick unrecht haben können?) „Und um deiner
Dummheit willen, um dessentwillen, was du gestern auf dem
Spaziergang getan hast, liebe ich dich noch mehr, viel mehr als
zuvor.“ „Aber warum hast du mich so gequält, warum bist du nicht
gekommen, warum hast du mir ein Billett geschickt und mich
gezwungen\ldots{} “

„Vielleicht wollte ich dich auf die Probe stellen, vielleicht
musste ich die Gewissheit haben, dass du alles tust, was ich will,
dass du mir ganz gehörst.“ „Ja, ganz!“

Sie nahm mein Gesicht in ihre Hände und zog meinen Kopf zu sich
empor.

„Nun, wie steht es mit den Pflichten, die Sie wie jede anständige
Nummer haben?“ Die weißen, scharfen Zähne blitzten.

Ja, die Pflichten\ldots{} Ich blätterte im Geist die letzten Seiten
meines Manuskripts durch: tatsächlich, nichts steht darin von
meinen Pflichten\ldots{}

Ich schwieg. Ich lächelte triumphierend (und wahrscheinlich
töricht) und blickte ihr in die Augen, in denen ich mein
Spiegelbild sah: winzig klein, kaum sichtbar, für immer in dieses
dunkle Gefängnis eingeschlossen. Dann spürte ich ihre weichen,
brennenden Lippen auf meinem Mund, fühlte den süßen Schmerz des
Blühens. In jeder Nummer ist ein unsichtbares, leise tickendes
Metronom eingebaut, so dass wir, ohne auf die Uhr zu sehen, die
Zeit auf fünf Minuten genau bestimmen können. Doch an diesem Abend
war das Metronom in mir stehen geblieben, ich wusste nicht, wie
viel Zeit vergangen war, und zog erschrocken mein Abzeichen mit der
Uhr unter dem Kopfkissen hervor\ldots{}

Dem Wohltäter sei Dank! Noch zwanzig Minuten. Die Minuten waren
lächerlich kurz, sie entflohen unaufhaltsam, und ich musste ihr
doch so viel erzählen, alles, alles erzählen: von O.s Brief, von
jenem schrecklichen Abend, als ich ein Kind mit ihr zeugte, von
meiner Kindheit (warum, weiß ich nicht), von unserem
Mathematiklehrer Plapa, von der Wurzel aus minus eins und wie ich
zum ersten Mal in meinem Leben am Tag der Einstimmigkeit teilnahm
und bitterlich weinte, weil ich einen Tintenklecks auf der Uniform
hatte — und das an diesem Feiertag! I richtete sich auf und stützte
den Kopf in die Hand. „Vielleicht werde ich an diesem Tag\ldots{} “ Sie
brach mitten im Satz ab und runzelte die dunklen Brauen. Dann nahm
sie meine Hand und drückte sie fest. „Sag, wirst du mich nie
vergessen, wirst du immer an mich denken?“ „Warum fragst du mich
das? Was meinst du damit? I, Liebste!“

Sie antwortete nicht, und ihre Augen blickten an mir vorbei in
weite Fernen. Da hörte ich plötzlich, wie der Wind mit
Riesenflügeln gegen das Haus schlug (er hatte es die ganze Zeit
getan, aber ich merkte es erst jetzt), und ich musste an das
durchdringende Geschrei der Vögel über der Grünen Mauer denken.

I machte eine Kopfbewegung, als wollte sie etwas von sich
abschütteln. Ein letztes Mal berührte sie mich eine Sekunde lang
mit dem ganzen Körper, wie ein Flugzeug federnd die Erde berührt,
bevor es landet. „Gib mir meine Strümpfe, schnell!“

Die Strümpfe lagen auf meinem Manuskript (Seite 193) auf dem
Schreibtisch. In der Eile stieß ich dagegen, die einzelnen Blätter
gerieten durcheinander, und ich konnte sie nicht ordnen, wie sehr
ich mich auch bemühte. Nun, was machte es schon! Eine richtige
Ordnung wäre ja doch nicht mehr zustande gekommen, denn es blieben
so viele Strudel, Abgründe und unbekannte Größen zurück. „Ich kann
das nicht ertragen“, sagte ich. „Du bist hier, neben mir, und
scheinst doch so fern, als wärst du durch eine undurchsichtige
Mauer von mir getrennt. Ich höre Geräusche und Stimmen hinter der
Mauer und kann die Worte nicht verstehen; ich weiß nicht, was dort
ist. Ich halte das nicht länger aus. Du verschweigst mir die ganze
Zeit etwas, du hast mir nie gesagt, wohin ich damals im Alten Haus
geraten war, was das für Korridore sind, und warum der kleine
Doktor\ldots{} Oder ist alles gar nicht geschehen?“

I legte mir die Hand auf die Schulter und blickte mir tief in die
Augen: „Möchtest du alles wissen?“ „Ja, ich will es wissen, ich
muss es wissen.“ „Und du hast keine Angst, mir überallhin zu
folgen, bis zum Ende, ganz gleich, wohin ich dich führen werde?“

„Nein, ich habe keine Angst. Ich folge dir, wohin du mich auch
führst.“

„Gut. Ich verspreche dir, wenn der Feiertag vorbei ist, wenn
erst\ldots{} Übrigens, da fällt mir ein — wie weit bist du eigentlich
mit dem Integral? Ist er bald fertig? Ich hätte fast vergessen,
dich danach zu fragen.“ „Was bedeutet dieses \glq{}Wenn
erst\ldots{}\grq{}?“ Sie stand schon an der Tür. „Du wirst es
sehen\ldots{} “ Ich war wieder allein. Alles, was von ihr blieb, war ein
feiner Duft, der mich an den trockenen Blütenstaub aus dem Land
jenseits der Mauer erinnerte. Und noch etwas blieb: quälende
Fragen, die sich wie spitze Haken in mich bohrten\ldots{} Warum hatte
sie auf einmal vom Integral gesprochen?

\section{EINTRAGUNG NR. 24}

\uebersicht{\emph{Übersicht:} Die Grenze der Funktion. Ostern.
Alles ausstreichen.}
Ich bin wie eine Maschine, die auf eine zu große Umdrehungszahl
eingestellt worden ist. Die Lager sind heißgelaufen, noch eine
Minute, und das geschmolzene Metall wird tröpfeln, alles wird sich
in Nichts auflösen. Schnell, kaltes Wasser her, Logik! Ich begieße
die Maschine mit ganzen Eimern kalten Wassers, aber die Logik
zischt auf den glühenden Lagern und entweicht als weißer Dampf, der
sich nicht greifen lässt.

Wenn man die wirkliche Bedeutung einer Funktion bestimmen will,
muss man ihren Grenzwert nehmen, das ist völlig klar. Also war
meine lächerliche „Auflösung im Weltall“, von der ich gestern
sprach, wenn man sie als

Limes auffasst, nichts anderes als der Tod. Denn der Tod ist die
totale Auflösung des Ich im Weltall. Daraus folgt: Wenn man die
Liebe mit L bezeichnet, den Tod mit T, dann ist L — f (T), das
bedeutet, dass Liebe eine Funktion des Todes\ldots{}

Ja, so ist es, so und nicht anders! Deswegen fürchte ich I,
deswegen ringe ich mit ihr und will mich ihr nicht unterwerfen.

Aber warum stehen „ich will nicht“ und „ich will“ in mir
nebeneinander? Das Entsetzliche ist, dass ich diesen seligen Tod
von gestern herbeisehne. Das Entsetzliche ist, dass mich sogar
jetzt, da ich die logische Funktion integriert habe, da ich weiß,
dass sie den Tod in sich trägt, mit Lippen, Händen, mit meiner
Brust, mit jedem Millimeter meines Körpers nach ihr verlangt\ldots{}
Morgen ist der Tag der Einstimmigkeit. Sie wird dort sein, ich
werde sie sehen, freilich nur von weitem — das wird mir sehr wehe
tun, denn ich muss neben ihr sein, ich fühle mich unwiderstehlich
zu ihr hingezogen, ich muss ihre Knie, ihre Schultern, ihr Haar
spüren\ldots{} Aber ich will ja diesen Schmerz, ich brauche ihn.

Großer Wohltäter! Welch absurder Gedanke — nach Schmerz verlangen!
Jeder weiß, dass Schmerzen negative Größen sind und die Summe
verringern, die wir Glück nennen. Daraus folgt\ldots{} Nichts, gar
nichts folgt daraus. Öde, Leere. Abends:

Durch die gläsernen Mauern des Hauses blicke ich in den fiebrig
glühenden Sonnenuntergang. Ich stelle meinen Sessel so, dass ich
diesen triumphierenden roten Schein nicht mehr sehe, und blättere
in meinem Manuskript. Dabei muss ich feststellen: Ich habe wieder
einmal vergessen, dass ich nicht für mich schreibe, sondern für
Sie, unbekannte Leser, die ich liebe und bedaure, weil Sie sich
noch irgendwo tief unten abmühen wie die Menschen vergangener
Jahrhunderte. Nun will ich von dem Tag der Einstimmigkeit
berichten, dem herrlichsten aller Tage. Ich habe ihn von frühester
Jugend an geliebt. Ich glaube, für Sie ist dieser Tag etwas
Ähnliches wie das Osterfest unserer Ahnen. Ich kann mich noch genau
erinnern, dass ich mir als Kind am Tag vor dem Fest einen kleinen
Kalender anlegte und feierlich eine Stunde nach der anderen
ausstrich: jede Stunde, die ich ausstrich, hieß eine Stunde weniger
warten. Wenn ich wüsste, dass niemand es sähe, würde ich auch heute
— mein Ehrenwort — einen solchen Kalender bei mir tragen und mich
jeden Augenblick vergewissern, wie viel Zeit noch bis morgen
bleibt, wann ich sie endlich sehen werde, wenn auch nur von
weitem\ldots{} Ich wurde unterbrochen, der Schneider brachte meine neue
Uniform. Nach altem Brauch erhalten sämtliche Nummern neue
Uniformen für das Fest. Morgen werde ich ein Schauspiel erleben,
das sich jahraus, jahrein wiederholt und uns jedes Mal von neuem
begeistert: die gewaltige Schale der Einstimmigkeit, andächtig
erhobene Hände. Morgen ist der Tag der alljährlich wiederkehrenden
Wahl des Wohltäters. Morgen werden wir IHM die Schlüssel zu der
unbezwinglichen Feste unseres Glückes für ein weiteres Jahr
übergeben. Der Tag der Einstimmigkeit hat natürlich nichts mit
jenen ungeordneten, unorganisierten Wahlen unserer Vorfahren zu
tun, deren Ergebnis nicht im voraus bekannt war. Es gibt nichts
Unsinnigeres, als einen Staat auf blinde Zufälligkeiten zu gründen.
Aber es mussten ganze Jahrhunderte vergehen, bevor die Menschen das
einsahen. Ich brauche Ihnen deshalb wohl nicht zu sagen, dass es
bei uns keinen Raum für irgendwelche Zufälligkeiten, für
unerwartete Ereignisse gibt. Unsere Wahlen haben eher eine
symbolische Bedeutung: sie erinnern uns daran, dass wir einen
einzigen, gewaltigen Organismus bilden, der aus Millionen Zellen
besteht, dass wir — in den Worten des Evangeliums gesagt — die
Einzige Kirche sind. In der Geschichte des Einzigen Staates ist es
noch niemals vorgekommen, dass auch nur eine Stimme sich erdreistet
hätte, das machtvolle Unisono dieses feierlichen Tages zu stören.

Es heißt, unsere Vorfahren hätten ihre „Wahlen“ „geheim“
durchgeführt, sich also wie Diebe versteckt. Einige Historiker
behaupten sogar, sie seien maskiert zur Wahl erschienen (ich stelle
mir dieses phantastisch-düstere Schauspiel ungefähr so vor: Nacht,
ein freier Platz, dunkel gekleidete Gestalten, die an den Mauern
entlangschleichen, im Wind flackern rote Fackeln\ldots{}). Den Grund für
diese Geheimnistuerei haben wir bis auf den heutigen Tag nicht
eindeutig feststellen können. Wahrscheinlich hingen diese Wahlen
mit irgendwelchen mystischen, abergläubischen oder sogar
verbrecherischen Vorgängen zusammen. Wir aber haben nichts zu
verbergen und brauchen uns für nichts zu schämen: wir halten unsere
Wahlen in aller Öffentlichkeit am hellen Tag ab. Ich sehe, wie alle
für den Wohltäter stimmen, alle anderen sehen, wie ich dem
Wohltäter meine Stimme gebe — und es kann auch gar nicht anders
sein, denn alle und ich — das ist das große Wir. Unsere
Wahlmethoden erziehen die Menschen zu einer edlen Gesinnung, sie
sind viel aufrichtiger und besser als die feige, verlogene
Geheimniskrämerei von einst. Außerdem sind sie weit zweckmäßiger.
Nehmen wir einmal an, das Unmögliche würde geschehen, und ein
falscher Ton schliche sich in die Monophonie ein. Die unsichtbaren
Beschützer, die mitten in unseren Reihen sitzen, würden
es sofort bemerken und die auf Abwege geratenen Nummern
zurückhalten und sie so vor weiteren Entgleisungen bewahren. Aber
noch etwas kommt hinzu\ldots{} Mein Blick fällt auf die gläserne Wand
des Nachbarzimmers: eine Frau steht vor dem Spiegel und knöpft
eilig ihre Uniform auf. Eine Sekunde lang sehe ich Augen, Lippen
und zwei spitze rosige Punkte. Dann schließen sich die Vorhänge.
Meine Erlebnisse von gestern fallen mir plötzlich wieder ein, und
ich weiß nicht mehr, was „noch hinzukommt“. Ich will es auch gar
nicht wissen. Ich will nur eins: I! Ich will, dass sie fortan immer
bei mir ist, bei mir allein. Was ich von dem „Fest“ geschrieben
habe, ist lauter Unsinn. Ich möchte alles durchstreichen,
zerreißen, wegwerfen. Denn ich weiß, dass es für mich nur dann ein
Feiertag ist, wenn ich sie bei mir habe, wenn ihre Schulter die
meine berührt. (Vielleicht ist dies ein frevelhafter Gedanke, aber
es ist die Wahrheit.) Ohne I wird die Sonne von morgen nur eine
Blechscheibe sein, der Himmel ein Stück blaubemaltes Blech, und ich
selber\ldots{}

Ich nahm den Telefonhörer ab: „I, sind Sie?s?“

„Ja. Warum rufen Sie so spät an?“

„Ich\ldots{} ich wollte Sie bitten\ldots{} ich möchte gern, dass Sie morgen
neben mir sitzen. Liebste\ldots{} “ Liebste — ich hauchte das nur. Sie
gab lange keine Antwort. Mir war, als hörte ich in I.s Zimmer
jemand flüstern. Endlich sagte sie: „Nein, ich kann nicht. Sie
wissen, wie gern ich es möchte, aber es geht wirklich nicht. Warum?
Morgen werden Sie es sehen.“ Nacht.

\section{EINTRAGUNG NR. 25}

\uebersicht{\emph{Übersicht:} Niederfahrt vom Himmel. Die größte
Katastrophe der Geschichte. Alles unbestimmt.}
Vor der Wahl, als sich alle erhoben und die feierlichen Klänge der
Hymne über unseren Köpfen brausten, vergaß ich für eine Sekunde,
was I von diesem Feiertag gesagt und was mich so sehr beunruhigt
hatte. Ja, ich glaube, ich vergaß sogar sie. Ich war wieder der
kleine Junge, der an diesem Tag einmal bitterlich geweint hatte,
weil er einen winzigen, ihm allein sichtbaren Fleck auf seiner
Uniform entdeckte. Wenn auch keiner der rings um mich Stehenden
sah, wie viele schwarze Flecken jetzt auf mir waren, so wusste ich
doch allzu gut, dass ein Verbrecher wie ich unter diesen Menschen
mit den offenen, ehrlichen Gesichtern nichts zu suchen hatte. Ach,
ich wäre am liebsten aufgesprungen, um mit tränenerstickter Stimme
die ganz Wahrheit über mich herauszuschreien. Mag es auch mein Ende
sein, dachte ich, was tut es? Wenn ich mich nur eine einzige
Sekunde lang so rein und gedankenlos fühlen könnte wie dieser
kindlich-blaue Himmel! Aller Augen blickten zum Himmel auf; in dem
morgendlich keuschen Blau zitterte ein kaum erkennbarer Punkt, bald
dunkel, bald im Licht blitzend. Das war Er, der von den Himmeln zu
uns herniederstieg, ein neuer Jehova im Flugzeug, weise, gütig und
streng wie der Gott der Alten. Mit jeder Minute kam Er näher und
näher, immer höher schlugen Ihm Millionen Herzen entgegen. Jetzt
musste Er uns sehen! Im Geist schaute ich mit Ihm auf die Menge
herab, auf die punktierten Linien der konzentrisch angeordneten
Tribünen, die wie Kreise eines Spinnennetzes waren. Im Zentrum
dieses Netzes würde sich gleich eine
weiße weise Spinne niederlassen, der Wohltäter in weißer Uniform,
der uns in seiner Weisheit unsere Hände und Füße mit den starken
Fäden des Glückes gebunden hat. Seine erhabene Niederfahrt war
beendet, die brausende Hymne verstummte, alle hatten sich wieder
hingesetzt. Da erkannte ich plötzlich, dies war wirklich ein
hauchdünnes Spinnennetz, zum Zerreißen gespannt, ja, im nächsten
Augenblick musste es reißen, und dann würde etwas Unglaubliches
geschehen.

Ich richtete mich ein wenig auf und schaute mich um. Mein Blick
begegnete einem Augenpaar, das ein Gesicht nach dem anderen liebend
und besorgt musterte. Da hob jemand die Hand und gab einem anderen
ein Zeichen. Und da — das Antwortsignal. Noch einmal\ldots{} Das waren
sie, die Beschützer. Irgend etwas beunruhigte sie, das Spinnennetz
war so straff gespannt, dass es bebte. Auf dem Podium verlas ein
Dichter die Wahleröffnungsrede, doch ich nahm die einzelnen Worte
nicht auf, ich hörte nur den gleichmäßigen Rhythmus der Hexameter,
der mich an das Ticken einer Uhr erinnerte. Mit jedem Schwung des
Pendels rückte eine bestimmte Sekunde näher, immer näher. Wie im
Fieber schweiften meine Augen über die Reihen, doch ich konnte das
eine Gesicht, das ich suchte, nirgends entdecken. Ich musste es so
schnell wie möglich finden, ich musste sie finden, denn gleich
würde die Uhr ticken, und dann\ldots{}

Er — ja, das war er! Rosa Ohren, eine S-förmige Schlinge flogen an
dem Podium vorüber. Er eilte durch die Gänge zwischen den
Tribünen.

S, I — zwischen den beiden bestand irgendeine Verbindung. (Ich
vermute das seit langer Zeit, aber ich weiß bis jetzt nicht,
welcher Art diese Verbindung ist. Irgendwann werde ich es
erfahren.) Ich blickte ihm nach. Plötzlich
blieb er stehen\ldots{} Es war wie ein elektrischer Schlag — etwas
durchzuckte mich, presste mich zu einem Bündel zusammen. S stand in
unserer Reihe, höchstens 40\textcelsius{} von mir entfernt, und beugte sich zu
jemand hinab. Ich sah I; neben ihr den grinsenden R-13 mit seinen
widerlichen Negerlippen.

Mein erster Gedanke war, zu ihr zu stürzen und laut zu schreien:
„Warum bist du heute mit ihm zusammen? Warum wolltest du mich nicht
bei dir haben?“ Aber eine unsichtbare Spinne fesselte mich an
Händen und Füßen. Die Zähne aufeinander beißend, blieb ich sitzen,
ohne sie aus den Augen zu lassen. In diesem Augenblick empfand ich
einen starken physischen Schmerz im Herzen, den ich jetzt noch
spüre.

Das war mein Ich. Ich weiß noch, dass ich dachte: Wenn
nichtphysische Ursachen einen physischen Schmerz hervorrufen
können, dann ist es klar, dass\ldots{} Leider dachte ich diesen Gedanken
nicht zu Ende, ich kann mich nur ganz dunkel entsinnen, dass mir
jene törichte Redensart unserer Vorfahren einfiel: Es zerreißt mir
die Seele. Ich erstarrte: die Hexameter waren verstummt. Jetzt
würde es geschehen\ldots{}

Eine Pause von fünf Minuten folgte. Doch dieses Schweigen war kein
andächtiges Gebet wie sonst, es hatte etwas Bedrückendes, wie in
alten Zeiten, als man noch keine Akkumulatorentürme kannte, als
bisweilen ein Gewitter über den ungezähmten Himmel jagte. Die Luft
— durchsichtiges Erz, man möchte mit weit aufgerissenem Mund
atmen.

Mein zum Zerreißen gespanntes Ohr registrierte irgendwo hinter mir
ein aufgeregtes Flüstern. Es klang wie das Nagen einer Maus. Unter
gesenkten Lidern spähte ich die ganze Zeit zu I und R hinüber, und
auf meinen Knien
zitterten meine behaarten Hände, die nicht mir gehörten, die ich
hasste\ldots{} Alle hielten die Abzeichen mit der Uhr in der Hand. Eine
Minute verstrich, zwei, drei, fünf Minuten\ldots{} Auf dem Podium sprach
eine eherne Stimme klar und langsam: „Wer dafür ist, hebe die
Hand.“

Wenn ich ihm nur in die Augen sehen könnte, wie früher, aufrichtig
und ergeben: „Da bin ich. Ganz! Nimm mich hin!“ Mit großer
Anstrengung, als wären meine Gelenke eingerostet, hob ich die
Hand.

Ein Rauschen von Millionen Händen, ein gedämpftes „Ach!“ Ich
fühlte, dass jetzt etwas begann, dass etwas jäh umstürzte, doch ich
wusste nicht, was, denn mir fehlte die Kraft, hinzublicken, ich
wagte es nicht\ldots{} „Wer ist dagegen?“

Das war stets der feierlichste Augenblick des ganzen Tages: Alle
blieben regungslos sitzen und beugten sich freudig dem wohltätigen
Joch der Nummer aller Nummern. Plötzlich hörte ich zu meinem
Entsetzen wieder ein Geräusch; es war leiser als ein Windhauch und
lauter als die brausenden Töne der Hymne. Es klang wie der letzte
schwache Seufzer eines Sterbenden — und alle Gesichter ringsum
erbleichten, allen trat kalter Schweiß auf die Stirn.

Ich blickte auf, nur eine Hundertstelsekunde\ldots{} Tausende von Händen
flogen empor — „dagegen“ — und fielen herab. Ich sah I.s blasses,
von einem Kreuz gezeichnetes Gesicht, sah ihre erhobene Hand. Mir
wurde schwarz vor den Augen.

Stille, mein Puls hämmerte wild. Dann, als hätte ein wahnsinniger
Dirigent das Zeichen zum Einsatz gegeben, auf allen Tribünen Lärm,
laute Rufe, wie ein Sturmwind vorüberwirbelnde Uniformträger,
hilflos hin und her
laufende Beschützer, Stiefelabsätze in der Luft, dicht vor meinem
Gesicht, neben den Absätzen ein aufgerissener, von einem unhörbaren
Schrei verzerrter Mund. Ringsum Tausende brüllender Münder — wie
auf einer imaginären, riesigen Leinwand.

Sekundenlang sah ich die weißen Lippen von O, sie duckte sich in
dem schmalen Durchgang dicht an die Mauer, den Leib mit beiden
Händen schützend. Im Nu war sie verschwunden, weggespült, oder ich
hatte sie vergessen, denn\ldots{}

Aber das spielte sich nicht mehr auf einer Leinwand ab, sondern in
mir selbst, in meinem zusammengepressten Herzen, in meinen
hämmernden Schläfen. Links über mir sprang plötzlich R-13 auf eine
Bank, rot, rasend. Auf seinen Armen I, totenbleich, die Uniform von
der Schulter bis zur Brust aufgerissen, auf der weißen Haut Blut.
Sie hielt sich an seinem Hals fest, und er trug sie, wie ein
widerlicher Gorilla, flink von Bank zu Bank setzend, aus dem
Gewühl.

Ich weiß nicht, wo ich die Kraft hernahm — ich drängte mich durch
die Menge, sprang über Schultern und Bänke, und schon hatte ich sie
eingeholt und packte R am Kragen. „Loslassen! Sofort loslassen!“
(Zum Glück konnte das keiner hören, denn alle schrieen, alle
rannten.) R wandte sich um, seine Lippen bebten, er dachte wohl,
die Beschützer seien ihm auf den Fersen. „Ich dulde das nicht, ich
will das nicht! Hände weg, sofort!“

Er schnalzte ärgerlich mit den dicken Lippen, schüttelte den Kopf
und eilte weiter. Und da — es ist mir sehr peinlich, dass ich dies
niederschreiben muss, aber ich glaube, ich darf es nicht
verschweigen, damit Sie, unbekannte Leser, die Geschichte meiner
Krankheit bis ins kleinste studieren können — da holte ich aus und
schlug ihn nieder.

Stellen Sie sich vor — ich schlug ihn nieder! Ich kann mich noch
genau daran erinnern. I glitt rasch aus seinen Armen.

„Gehen Sie!“ schrie sie R zu, „gehen Sie! Sie sehen doch, er\ldots{}
Gehen Sie, R! Schnell!“

R fletschte die weißen Zähne, spritzte mir irgendein Wort ins
Gesicht und tauchte in der Menge unter. Ich nahm I auf meine Arme,
drückte sie fest an meine Brust und trug sie fort.

Mein Herz schlug wie wild, und ich hörte in mir die jubelnde Stimme
der Freiheit. Mochte dort unten alles zu Scherben werden — es
kümmerte mich nicht! Ich wollte nichts weiter als sie tragen,
tragen, tragen\ldots{}

Abends, 22 Uhr.

Die verwirrenden Ereignisse von heute morgen haben mich so
erschöpft, dass ich die Feder kaum halten kann. Sind die
schützenden, ewigen Mauern des Einzigen Staates wirklich
zusammengestürzt? Sind wir wieder ohne Obdach, in wilder Freiheit,
wie unsere Urväter? Gibt es tatsächlich keinen Wohltäter mehr?
Dagegen\ldots{} am Tag der Einstimmigkeit — dagegen? Ich schäme mich für
diese Nummern, ich leide, ich fürchte mich für sie. Übrigens — wer
ist das eigentlich — sie? Und wer bin ich selber — sie oder wir?

Ich hatte I zu der obersten Reihe der Tribüne getragen. Sie saß auf
einer gläsernen Bank in der Sonne. Die rechte Schulter und der
Ansatz der herrlichen, unberechenbaren Kurve weiter unten waren
entblößt und eine dünne rote Schlange ringelte sich darauf — Blut.
Sie hatte offenbar nicht bemerkt, dass ihre Brust entblößt war,
dass sie blutete\ldots{} nein, sie sah es, sie wusste, dass es so sein
musste, und wäre ihre Uniform geschlossen gewesen, sie hätte sie
jetzt aufgerissen\ldots{}

„Morgen“, sie atmete gierig durch die zusammengepressten weißen
Zähne, „was morgen geschieht, weiß keiner. Hörst du — weder ich
weiß es noch sonst jemand — keiner! Alle Gewissheit ist zu Ende.
Jetzt kommt etwas Neues, Unglaubliches, Unerhörtes!“

Drunten herrschte noch immer ein wildes Gewühl. Doch das alles war
so fern, ich hörte es nicht mehr, denn sie schaute mich an und
saugte mich langsam durch die schmalen goldenen Fenster ihrer Augen
in sich hinein. Da fiel mir ein, dass ich einmal durch die Grüne
Mauer in die geheimnisvollen gelben Augen eines seltsamen Wesens
geblickt hatte.

„Höre, wenn morgen nichts Besonderes geschieht, werde ich dich
dorthin führen — du weißt doch, was ich meine.“ Nein, ich wusste es
nicht, doch ich nickte stumm. Ich hatte mich aufgelöst, ich war
unendlich klein, ein Punkt\ldots{} Aber in dieser Vorstellung lag ja
eine gewisse Logik des heutigen Tages, denn der Punkt ist eine
völlig unbekannte Größe — er braucht sich nur zu regen, sich
fortzubewegen — und er kann sich in tausend verschiedene Kurven, in
Hunderte von Körpern verwandeln. Ich habe Angst, mich zu bewegen —
in was werde ich mich verwandeln? Mir scheint, dass alle gleich mir
die leiseste Bewegung fürchten. Jetzt, da ich diese Zeilen
schreibe, sitzen sie in ihren gläsernen Käfigen und warten auf
irgend etwas. Kein summender Lift, kein Lachen, keine Schritte sind
im Korridor zu hören wie sonst um diese Stunde. Hin und wieder
kommen Nummern an meinem Zimmer vorbei; sie gehen auf Zehenspitzen,
blicken sich verstohlen um und flüstern miteinander: Was wird
morgen sein? In was werde ich mich morgen verwandeln?

\section{EINTRAGUNG NR. 26}

\uebersicht{\emph{Übersicht:} Die Welt existiert. Aussatz. 41\textcelsius{}.}
Morgen. Durch die Zimmerdecke blickt der Himmel zu mir herein, fest
wie immer, rund, rotwangig. Ich glaube, ich würde mich weniger
wundern, wenn ich statt dessen eine viereckige Sonne, Menschen in
bunten Kleidern aus Tierwolle und undurchsichtige Steinwände
erblicken würde. Sie existiert also noch, die Welt, unsere Welt.
Vielleicht ist es aber nur die Trägheit der Materie — der Generator
ist schon abgestellt, doch die Getriebe knirschen und drehen sich
noch — zwei Umdrehungen, drei, und nach der vierten stehen sie
still\ldots{}

Kennen Sie dieses seltsame Gefühl: Man wacht mitten in der Nacht
auf, öffnet die Augen, blickt in die Finsternis, glaubt, man habe
sich verirrt. Hastig tastet die Hand ins Dunkel, sucht die Wand,
die Lampe, den Stuhl. So suchte ich in der Staatszeitung nach etwas
— da, da war es: Gestern fand die von allen Nummern ungeduldig
erwartete Feier der Einstimmigkeit statt. Der Wohltäter, der schon
so oft seine unfehlbare Weisheit bewiesen hat, wurde zum
achtundvierzigsten Male einstimmig wieder gewählt. Einige Feinde
des Glückes versuchten die Feier zu stören. Durch ihr
staatsfeindliches Verhalten haben sie das Recht verwirkt, Bausteine
des gestern erneuerten Fundaments des Einzigen Staates zu sein.
Ihren Stimmen Bedeutung beizumessen, wäre genauso töricht, wie wenn
man glauben würde, ein Husten im Konzertsaal gehöre zu einer
heroischen Sinfonie. Jeder von uns weiß das. Oh, Allerweisester!
Sind wir wirklich gerettet, trotz allem, was geschehen? In der Tat,
was könnte man auf diesen kristallklaren Syllogismus erwidern?

Dann folgten noch drei Zeilen:

Heute um 12 Uhr wird die Staatsverwaltung gemeinsam mit dem
Gesundheitsamt und den Beschützern über einen wichtigen Staatsakt
beraten.

Nein, noch standen die Mauern, ich fühlte sie. Das seltsame
Empfinden, ich hätte mich verirrt, war mit einemmal verschwunden,
es kam mir nicht mehr sonderbar vor, dass ich einen blauen Himmel
und eine kreisrunde Sonne sah. Alle gingen wie sonst zur Arbeit.

Festen Schrittes eilte ich über den Prospekt. Ich kam zur Kreuzung
und bog in eine Seitenstraße ein. Merkwürdig, dachte ich, die Leute
machen einen Bogen um das Eckgebäude, als wäre dort ein Rohr
gebrochen, als schösse kaltes Wasser auf den Gehsteig.

Noch zehn, fünf Schritte — und ich war gleichfalls mit kaltem
Wasser übergossen, ich taumelte und sprang auf den Fahrdamm\ldots{} etwa
zwei Meter über mir klebte ein viereckiges Stück Papier an der
Mauer, und darauf las ich in giftgrünen Lettern das unverständliche
Wort MEPHI. Vor dem Plakat ein krummer Rücken, durchsichtige,
flügelähnliche Ohren, die vor Zorn oder vor Erregung bebten. Den
rechten Arm erhoben, den linken wie einen zerbrochenen Flügel
hilflos nach hinten streckend, sprang er in die Höhe, um das Plakat
abzureißen, doch er konnte es nicht erreichen.

Wahrscheinlich dachte jeder der Vorübergehenden: Wenn ich zu ihm
hingehe und ihm helfe, dann denkt er bestimmt, ich fühlte mich
irgendwie schuldig, und deswegen\ldots{} Ich gestehe, auch ich hatte
diesen Gedanken. Doch ich erinnerte mich daran, wie oft er mein
Schutzengel gewesen, wie oft er mich gerettet hatte, und so trat
ich entschlossen auf ihn zu, streckte die Hand aus und riss das
Plakat herunter.

S wandte sich um, seine Augen bohrten sich in mich hinein, bis auf
den tiefsten Grund, suchten und fanden dort etwas. Dann deutete er
mit einer leichten Bewegung der linken Augenbraue auf die Mauer, wo
das viereckige Papier mit dem Wort MEPHI gehangen hatte, und über
sein Gesicht huschte ein Lächeln, das zu meiner Verwunderung fast
heiter wirkte. Doch wieso wunderte ich mich eigentlich darüber? Die
beängstigend langsam ansteigende Temperatur der Inkubationszeit ist
dem Arzt stets unangenehmer als ein Ausschlag und 40\textcelsius{} Fieber — denn
in diesem Fall sieht man wenigstens sofort, um welche Krankheit es
sich handelt. Das Wort MEPHI, das heute auf allen Mauern stand, war
wie ein Ausschlag. Ich verstand das Lächeln von S\ldots{}

In der Untergrundbahn sah ich überall diesen weißen, entsetzlichen
Ausschlag — auf Wänden, Bänken und Spiegeln klebten Zettel mit der
Aufschrift MEPHI. Die Nummern saßen stumm auf ihren Plätzen. In der
Stille konnte ich das Geräusch der Räder deutlich hören; es klang
wie das Rauschen entzündeten Blutes. Einer stieß seinen Nachbarn
mit der Schulter an, der darauf zusammenfuhr und ein Paket mit
Papieren fallen ließ. Die Nummer links neben mir las in einer
Zeitung, immer die gleiche Zeile, und die Zeitung zitterte leise in
seinen Händen. Überall, in Rädern, Händen, Zeitungen, Augenwimpern
schlug ein Puls, schneller, immer schneller, und wenn ich heute mit
I dorthin ginge, würde das Thermometer vielleicht auf 39, 40, 41
Grad steigen.

Auf der Werft herrschte eine drückende, von dem Summen eines
fernen, unsichtbaren Propellers kaum unterbrochene Stille. Die
Maschinen standen schweigend und finster da. Nur die Kräne bewegten
sich langsam, lautlos, wie auf Zehenspitzen. Sie senkten sich,
packten mit ihren Greifern
blaue Klumpen gefrorener Luft und luden sie in die Zisternen des
Integral. Wir trafen Vorbereitungen zum Probeflug.

„Ob wir wohl noch diese Woche fertig werden?“ fragte ich den
zweiten Konstrukteur. Er hat ein Gesicht wie aus Porzellan, auf das
Augen und Lippen wie blaue und rosa Blümchen gemalt sind; aber
heute waren die Blumen verblichen, wie ausgewaschen.

Wir rechneten. Plötzlich verstummte ich und riss vor Entsetzen den
Mund auf. Ganz oben unter der Kuppel, auf dem blauen Klumpen Luft
in den Greifern des Krans, schimmerte ein winziges weißes Quadrat,
ein Plakat. Ich zitterte am ganzen Leibe — wahrscheinlich vor
Lachen. Ja, ich hörte, wie ich lachte. (Wissen Sie, wie das ist,
wenn man sein eigenes Lachen hört?)

„Stellen Sie sich vor“, sagte ich zu dem zweiten Konstrukteur, „Sie
säßen in einem altmodischen Flugzeug, der Höhenmesser zeigt auf
5000, der eine Flügel ist gebrochen, Sie stürzen wie ein Stein in
die Tiefe, und unterwegs rechnen Sie aus: morgen von 12 bis 2, von
2 bis 6, um 6 Abendessen\ldots{} Lächerlich, nicht wahr? Genauso
unsinnig wie unsere ganze Rechnerei.“ Die blauen Blümchen öffneten
sich weit. Wenn ich aus Glas wäre, könnte er sehen, dass ich in
drei, vier Stunden\ldots{} Was würde er wohl dazu sagen?

\section{EINTRAGUNG NR. 27}

\uebersicht{\emph{Übersicht:} Keine Übersicht — ich kann nicht.}
Ich bin allein in den endlosen unterirdischen Gängen. Über mir ein
stummer Betonhimmel. Irgendwo tröpfelt Wasser von den Steinen
herab. Da ist die schwere, undurchsichtige Tür, dahinter ein
dumpfes Geräusch. Sie hat gesagt, sie werde pünktlich um 16 Uhr zu
mir kommen. Doch es ist schon fünf, zehn, fünfzehn Minuten nach 16
Uhr, und niemand ist zu sehen. Ich will noch fünf Minuten warten.
Wenn sie dann nicht kommt\ldots{}

Irgendwo tröpfelt Wasser von den Steinen herab. Traurig, glücklich
denke ich: Gerettet! Ich kehre um und gehe langsam durch den
Korridor zurück. Das zitternde Licht der Deckenlampen verblasst\ldots{}

Plötzlich wird hinter mir eine Tür aufgerissen, schnelle Schritte,
die dumpf und weich von den Wänden widerhallen, nähern sich — und
sie steht vor mir, ein wenig außer Atem vom schnellen Laufen.

„Ich wusste, dass du kommen würdest, ich wusste es!“ Sie blickte
auf und\ldots{} Wie soll ich beschreiben, wie es ist, wenn ihre Lippen
die meinen berühren? Durch welche Formel soll ich den Sturm in
meiner Seele ausdrücken, der alles, alles außer ihr, hinwegfegt?
Ja, in meiner Seele — lacht darüber, wenn ihr wollt. Sie schlug
langsam die Augen auf und sagte leise: „Genug\ldots{} später. Wir müssen
gehen.“ Sie öffnete eine Tür. Alte, ausgetretene Stufen, ein
unerträglicher Lärm, Pfeifen, Licht\ldots{}

Inzwischen sind fast vierundzwanzig Stunden vergangen. Ich habe
mich zwar ein wenig beruhigt, aber immer noch
fällt es mir schwer, meine Erlebnisse einigermaßen genau zu
schildern. In meinem Kopf sieht es aus, als wäre eine Bombe
explodiert — geöffnete Münder, Flügel, Schreie, Blätter, Worte,
Steine, ein wüstes Durcheinander. Ich entsinne mich, dass mein
erster Gedanke war: Zurück! Schnell zurück! Denn ich wusste:
Während ich in den unterirdischen Gängen wartete, zerstörten die
Bewohner des unbekannten Landes die Grüne Mauer und brachen in
unsere Stadt ein, die wir mit großer Mühe von der niedrigen Welt
gesäubert hatten. Etwas Ähnliches hatte ich wohl zu I gesagt; sie
lächelte:

„Aber nein! Wir sind nur hinter der Grünen Mauer.“ Da öffnete ich
die Augen und erblickte ein Reich, das bisher jeder nur durch das
trübe Glas der Grünen Mauer gesehen hatte.

Die Sonne schien\ldots{} doch es war nicht unsere gleichmäßig über die
spiegelnde Fläche der Straße verteilte Sonne, es waren lebendige
Splitter, tanzende Flecke, die die Augen blendeten und mich
schwindlig machten. Kerzengerade Bäume, hoch wie der Himmel, stumme
grüne Fontänen, knorrige Äste\ldots{} und all das bewegte sich,
flüsterte und rauschte. Dicht vor mir sprang ein struppiges Knäuel
auf und stürzte in wilder Flucht davon. Ich blieb wie angewurzelt
stehen, ich konnte nicht mehr weiter, denn unter meinen Füßen war
keine glatte, ebene Fläche, sondern etwas widerlich Weiches,
Lebendiges, Grünes. Ich war wie betäubt, ich glaubte, ich müsse
ersticken — ja, ersticken, das ist das richtige Wort. Unbeweglich
stand ich da und hielt mich mit beiden Händen an einem schwankenden
Ast fest.

„Fürchte dich nicht! Das ist nur am Anfang so, das geht vorüber!
Nur Mut!“ sagte I. Auf dem wie toll hüpfenden grünen Netz entdeckte
ich
neben I ein scharfes, wie aus Papier geschnittenes Profil\ldots{} Der
Doktor! Ich erkannte ihn sofort. Die beiden nahmen mich an den
Händen und zogen mich lachend mit sich fort. Ich stolperte weiter,
glitt alle Augenblicke aus. Um mich Gekrächze, Moos, Erdhaufen,
Adlerschreie, Äste, Baumstämme, Flügel, Blätter, schrille Pfiffe\ldots{}
Der Wald lichtete sich, ich sah eine große Wiese und viele
Menschen, oder, besser gesagt, Lebewesen.

Jetzt komme ich zu dem, was am schwersten zu beschreiben ist. Denn
was ich auf dieser Lichtung erblickte, ging über alle Grenzen des
Wahrscheinlichen hinaus. Jäh wurde mir klar, warum I stets so
hartnäckig geschwiegen hatte, denn ich hätte ihr niemals geglaubt,
nein, selbst ihr nicht. Es ist durchaus möglich, dass ich morgen
auch mir selbst und meinen Aufzeichnungen nicht mehr glaube. Um
einen kahlen Felsblock inmitten der Wiese drängten sich drei- oder
viertausend \ldots{} Menschen. Nun, nennen wir sie Menschen, anders kann
man diese Geschöpfe wohl nicht bezeichnen. Anfangs erkannte ich nur
unsere blaugrauen Uniformen unter der Menge, und in der nächsten
Sekunde entdeckte ich neben ihnen schwarze, rote, gelbe, braune,
graue und weiße Menschen — ja, es konnten nur Menschen sein. Sie
waren alle ohne Kleider und hatten ein kurzes, glänzendes Fell wie
das ausgestopfte Pferd im Prähistorischen Museum. Aber die Weibchen
hatten die gleichen Gesichter wie unsere Frauen, zart und rosig,
und ihre Brüste, groß, fest, von schöner geometrischer Form, waren
nicht behaart. Bei den Männchen war nur ein Teil des Gesichts nicht
mit Haaren bedeckt, wie bei unseren Ahnen.

Dieser Anblick war so verblüffend, dass ich stehen blieb und sie
anstarrte. Plötzlich war ich allein, I stand nicht mehr neben mir;
ich
konnte mir nicht erklären, wie und wohin sie verschwunden war.
Rings um mich nur diese Geschöpfe, deren Fell in der Sonne wie
Atlas glänzte. Ich fasste einen von ihnen an der warmen,
dunkelbraunen Schulter: „Hören Sie, um des Wohltäters willen, haben
Sie gesehen, wohin sie gegangen ist? Sie war eben noch hier\ldots{} “ Er
warf mir einen finsteren Blick zu: „Pst! Leise!“ und deutete auf
den Felsblock in der Mitte der Lichtung. Dort oben stand sie, hoch
über den Köpfen der Menge. Die Sonne schien mir genau ins Gesicht,
so dass ich I als kohlschwarze, eckige Silhouette auf der blauen
Leinwand des Himmels sah. Niedrige Wolken flogen vorbei, und mir
war, als glitten nicht die Wolken über den Himmel, sondern der
Felsblock, und darauf sie selbst, und mit ihr die Menschenmenge und
die ganze Lichtung, als glitte alles lautlos dahin wie ein Schiff.
Die Erde wurde leicht, ganz leicht und schwamm mir unter den Füßen
weg \ldots{} „Brüder\ldots{}“, sprach I, „Brüder! Ihr alle wisst, dass sie in
der Stadt hinter der Mauer den Integral bauen. Ihr wisst, dass der
Tag naht, da wir diese Mauer — und alle Mauern — niederreißen
werden, auf dass der frische, grüne Wind die ganze Erde erfasse.
Aber der Integral wird diese Mauern dort hinauftragen, zu tausend
anderen Welten, die euch in der Nacht als helle Feuer durch die
schwarzen Blätter zublinken\ldots{} “

Um den Felsblock wogte, schäumte, heulte es: „Nieder mit dem
Integral! Nieder!“

„Nein, Brüder, nein! Der Integral muss unser werden. Wenn er zum
ersten Mal in den Himmelsraum emporjagt, werden wir an Bord sein.
Denn der Erbauer des Integral ist einer der Unseren. Er hat den
Mauern den Rücken gekehrt, er ist mit mir hierher gekommen, um bei
euch zu sein. Es lebe der Erbauer des Integral!“

Eine Sekunde — und ich stand irgendwo hoch oben, unter mir Köpfe,
schreiende Münder, erhobene Arme. Ein seltsames, berauschendes
Gefühl: Ich war über alle anderen erhoben, ich war ein Einzelwesen,
eine Welt, ich hatte aufgehört, eine Nummer zu sein\ldots{}

Müde und glücklich wie nach einer Umarmung sprang ich von dem
Felsen herunter. Sonne, Stimmen von oben und I.s Lächeln. Eine
goldhaarige, nach Kräutern duftende Frau mit atlasglänzendem Körper
trat auf mich zu. Sie hielt eine hölzerne Schale in den Händen,
setzte sie an die roten Lippen, und dann reichte sie sie mir. Ich
trank gierig, mit geschlossenen Augen, um das Feuer in mir
auszulöschen, ich trank süße, kalte Funken. Und das Blut in meinen
Adern, die ganze Welt, kreiste tausendmal schneller, die leichte
Erde flog wie eine Daunenfeder. Und alles wurde schwerelos, einfach
und klar.

Plötzlich erkannte ich auf dem Felsblock die riesigen Buchstaben
MEPHI. Ich wunderte mich nicht darüber, denn das war ja der starke
Faden, der alles verband. Dann sah ich eine Zeichnung, ich glaube,
ebenfalls auf diesem Stein: ein geflügelter Jüngling mit
durchsichtigem Körper, und an der Stelle des Herzens saß eine
rotglühende Kohle. Ich begriff, was diese Kohle bedeutete, nein,
ich fühlte es, so, wie ich jedes Wort von I (sie stand wieder auf
dem Felsblock und sprach) fühlte, ohne zuzuhören. Ich fühlte, dass
alle im gleichen Takt atmeten, dass alle irgendwohin flogen, wie
damals die Vögel über der Grünen Mauer\ldots{} In dem atmenden Dickicht
der Leiber rief eine tiefe Stimme:

„Aber das ist ja Wahnsinn!“

Ich glaube, ich — ja, ich bin ganz sicher, ich war es — sprang auf
den Felsblock und schrie: „Ja, es ist Wahnsinn! Alle müssen den
Verstand verlieren, alle, alle — so schnell wie möglich! Es muss
sein, ich weiß es!“

I stand lächelnd an meiner Seite. In mir glühte eine rote Kohle\ldots{}
Von dem, was dann geschah, blieben nur kleine Splitter in meinem
Gedächtnis haften. Langsam glitt ein Vogel vorbei. Ich sah, dass er
lebendig war wie ich: er wandte den Kopf nach rechts und nach links
und starrte mich mit seinen schwarzen, runden Augen an\ldots{}

Ein Rücken, mit glänzendem, elfenbeinfarbenem Fell bedeckt. Ein
dunkles Insekt mit winzigen durchsichtigen Flügeln kroch über
diesen Rücken. Der Rücken zuckte, um den Käfer abzuschütteln,
zuckte ein zweites Mal\ldots{} Verschlungenes grünes Gitterwerk —
Blätter. In ihrem Schatten lagen Menschen und kauten etwas, das
mich an die legendäre Nahrung unserer fernen Ahnen erinnerte — eine
lange, gelbe Frucht und etwas, das dunkel aussah. Eine Frau gab mir
davon, und das belustigte mich, denn ich wusste nicht, ob dieses
Zeug überhaupt genießbar war. Dann wieder eine Menschenmenge,
Köpfe, Beine, Arme, Münder. Eine Sekunde lang erkannte ich die
Gesichter ganz deutlich — aber im nächsten Augenblick waren sie
verschwunden, zerplatzt wie Seifenblasen. Durchsichtige,
flügelähnliche Ohren huschten vorüber — oder kam mir das nur so
vor? Ich zupfte I am Ärmel. Sie wandte sich um: „Was ist?“ „Er ist
hier\ldots{} “ „Wer?“

„S .. Er war eben hier .. in der Menge\ldots{} “ Die kohlschwarzen
feinen Brauen hoben sich, sie lächelte. Ich konnte nicht verstehen,
warum sie lachte. „Begreifst du denn nicht, was es bedeutet, wenn
er oder irgendeiner von ihnen hier ist?" flüsterte ich erregt.

„Was willst du? Keinem von ihnen würde es je in den Sinn kommen,
uns hier zu suchen. Denke doch einmal nach! Hättest du dir
vorstellen können, dass wir hier sind, dass so etwas überhaupt
möglich ist? In der Stadt können sie uns vielleicht festnehmen,
aber nicht hier. Du träumst.“

Sie lächelte übermütig, und ich lachte gleichfalls. Die Erde unter
meinen Füßen schwamm, trunken, heiter und leicht\ldots{}

\section{EINTRAGUNG NR. 28}

\uebersicht{\emph{Übersicht:} Beide. Entropie und Energie. Ein
undurchsichtiger Körperteil.}
Liebe Leser, wenn Ihre Welt der Welt unserer fernen Ahnen gleicht,
dann stellen Sie sich einmal vor, Sie hätten irgendwo im Ozean den
sechsten, siebten Weltteil, irgendein Atlantis, entdeckt und
erblickten dort seltsame Städtelabyrinthe, Menschen, die ohne
Flügel und ohne Flugzeug in der Luft schweben, Steine, die von der
bloßen Kraft des Blickes emporgehoben werden\ldots{} Nicht einmal mit
der glühendsten Phantasie hätten Sie sich dergleichen ausmalen
können. So etwa war mir gestern zumute, denn seit dem 200jährigen
Krieg war noch nie einer von uns hinter der Grünen Mauer, wie ich
Ihnen bereits sagte. Ich weiß, es ist meine Pflicht, Ihnen
ausführlich von dieser sonderbaren Welt zu berichten, die sich mir
gestern auftat. Aber ich bin heute einfach nicht fähig, darauf
zurückzukommen. Neues, immer wieder Neues stürmt auf mich ein, eine
wahre Flut von Ereignissen, und ich vermag nichts davon
festzuhalten\ldots{}

Als erstes hörte ich Stimmengewirr vor meiner Tür und erkannte die
Stimme von I, geschmeidig, und eine andere, monoton, starr wie ein
hölzernes Lineal — die von U. Dann sprang die Tür krachend auf, und
beide stürzten zugleich in mein Zimmer.

I stützte die Hand auf die Lehne meines Sessels und sah die andere
mit bösem Lächeln an:

„Hören Sie“, sagte sie zu mir, „diese Frau will Sie offenbar vor
mir beschützen. Sie tut, als wären Sie ein kleines Kind.“

Darauf entgegnete die andere mit zitternden Kiemen: „Er ist ja auch
ein Kind! Darum sieht er nicht, was Sie von ihm wollen und dass Sie
nur Komödie spielen. Jawohl! Und ich fühle mich verpflichtet\ldots{}“

Einen Augenblick lang sah ich im Spiegel die gekrümmte Linie meiner
Brauen. Ich sprang auf, schüttelte meine behaarten Fäuste und
schrie: „Hinaus! Sofort hinaus!“

Die Kiemen schwollen an, verfärbten sich ziegelrot, fielen sofort
wieder zusammen und wurden aschfahl. Sie öffnete den Mund, wollte
etwas sagen, brachte aber kein Wort hervor; sie schluckte nur und
ging hinaus. Ich stürzte auf I zu: „Das kann ich mir nie verzeihen,
nie! Sie hat es gewagt, dich zu kränken\ldots{} Aber konntest du dir
denn nicht denken, dass ich glaubte, sie wolle\ldots{} Sie hat das nur
getan, weil sie sich auf mich einschreiben möchte, aber ich\ldots{} “

„Sie wird keine Zeit mehr dazu haben. Und wenn es tausend solcher
Frauen wie sie gäbe. Ich weiß, dass du nicht diesen tausend
glaubst, sondern mir allein. Denn nach dem, was gestern geschehen
ist, bin ich ganz dein, wie du es gewollt hast. Ich habe mich in
deine Hände gegeben, du kannst in jedem beliebigen Augenblick\ldots{} “

„Was kann ich in jedem beliebigen Augenblick?“ plötzlich begriff
ich, was sie meinte. Das Blut schoss mir in Wangen und Ohren, und
ich schrie: „Hör auf! Nie mehr ein Wort davon! Du weißt doch, dass
mein früheres Ich nicht mehr existiert!“ „Ach, der Mensch ist wie
ein Roman; bevor man nicht die letzten Seiten gelesen hat, kennt
man das Ende nicht. Wäre es anders, dann lohnte es das Lesen gar
nicht\ldots{}“

I strich mir übers Haar. Ich konnte ihr Gesicht nicht sehen, aber
ich spürte an ihrer Stimme, dass sie in weite Fernen blickte, dass
sie mit den Augen einer Wolke folgte, die langsam und lautlos über
den Himmel zog, ohne dass man wusste, wohin.

Nach einer Weile schob sie mich mit einer zärtlichen Handbewegung
von sich fort:

„Höre, ich bin gekommen, um dir etwas zu sagen. Weißt du, dass seit
heute Abend in allen Auditorien Veränderungen vor sich gehen?“
„Veränderungen?“

„Ja. Ich kam vorhin vorbei und sah drinnen lange Tische stehen und
Ärzte in weißen Kitteln.“ „Was mag das bedeuten?“

„Ich weiß es nicht, keiner weiß es bis jetzt, und das ist das
Schlimmste dabei\ldots{} Aber sie kommen vielleicht zu spät\ldots{}“

Ich kann schon längst nicht mehr unterscheiden, wer sie und wer wir
sind, und so konnte ich nicht sagen, was mir lieber war: ob sie zu
spät oder nicht zu spät kommen sollten. Nur eines war mir klar: I
stand jetzt dicht am Rande des Abgrunds.

„Aber das ist ja Wahnsinn“, sagte ich. „Ihr gegen den Einzigen
Staat. Es ist genauso, als wollte man die Hand
gegen die Gewehrmündung pressen, um die Kugel zurückzuhalten. Das
ist heller Wahnsinn!“

Sie lächelte: „Alle müssen den Verstand verlieren, so schnell wie
möglich! Erst neulich hat das jemand gesagt. Erinnerst du dich
noch?“

Ja, so stand es in meinen Notizen. Also war es wirklich so gewesen.
Ich blickte sie schweigend an; das dunkle Kreuz auf ihrem Gesicht
trat heute besonders deutlich hervor.

„I, Liebste, lass uns handeln, ehe es zu spät ist\ldots{} Wenn du
willst, werde ich alles aufgeben, alles vergessen und mit dir in
das Land hinter der Grünen Mauer fliehen, zu diesen\ldots{} ich weiß
nicht, wer sie sind.“ Sie schüttelte den Kopf. Daran erkannte ich,
dass es schon zu spät war. Sie erhob sich und wollte gehen. Ich
nahm ihre Hand:

„Nein! Bleib noch ein wenig, nur noch ein wenig, um des Wohl\ldots{}“

Sie hob meine Hand langsam zum Licht, meine behaarte Hand, die ich
so sehr hasste. Ich wollte sie ihr entwinden, doch sie hielt sie
fest.

„Deine Hand\ldots{} Du weißt ja nicht — nur wenige wissen es —, dass
Frauen von hier, aus unserer Stadt, jene Menschen liebten. Auch in
dir ist wahrscheinlich ein Tropfen Sonnen- und Waldblut. Vielleicht
habe ich dich deswegen\ldots{} “

Pause. Seltsam; diese Pause, diese Leere, dieses Nichts ließ mein
Herz so wild pochen, dass es fast zersprang. Ich schrie: „Nein, ich
lasse dich nicht gehen, nicht bevor du mir von ihnen erzählt und
mir gesagt hast, warum du sie so liebst. Ich weiß nicht einmal, wer
sie sind und woher sie kommen. Wer sind sie? Die Hälfte, die wir
verloren haben\ldots{} H und O — aber wenn h3O entstehen soll, Bäche,
Meere, Wasserfälle, Wogen, Strudel, dann müssen sich die beiden
Hälften vereinigen\ldots{} “ Ich kann mich noch deutlich an jede ihrer
Bewegungen erinnern. Ich erinnere mich, dass sie mein gläsernes
Dreieck vom Tisch nahm und es gegen ihr Gesicht presste, während
sie sprach. Auf ihrer Wange blieb eine weiße Spur zurück, die sich
rötete und dann verschwand. Doch ich kann mich nicht ihrer Worte
entsinnen, nur einzelner Bilder und Farben. Ich glaube, sie
erzählte etwas vom 200jährigen Krieg. Zuerst sah ich rote Flecke
auf grünem Gras, auf der dunklen Erde, auf dem bläulich-weißen
Schnee, rote Lachen, die nicht trocknen wollten. Dann gelbes, von
der Sonne versengtes Gras, nackte, gelbe, zerzauste Menschen und
gelbe, struppige Hunde — und schließlich aufgedunsene Kadaver,
vielleicht waren es Hunde, vielleicht auch Menschen\ldots{} Das alles
spielte sich natürlich jenseits der Grünen Mauer ab, denn die Stadt
hatte bereits gesiegt, in der Stadt gab es damals schon unsere
künstliche Nahrung.

Ich hörte das Rauschen schwerer, schwarzer Falten, die von der Erde
bis zum Himmel reichten — Rauchsäulen über brennenden Wäldern und
Dörfern. Dumpfes Getöse: endlose Scharen von Menschen werden in die
Stadt getrieben, um mit Gewalt gerettet zu werden und das Glück zu
lernen.

„Hast du das alles gewusst?“ fragte sie mich. „Ja, fast alles.“

„Aber du wusstest nicht, dass eine kleine Gruppe von ihnen trotzdem
unversehrt geblieben ist und jenseits der Mauern weiterlebte. Nackt
flohen sie in die Wälder. Dort gingen sie bei Bäumen, Tieren,
Vögeln, Blumen und bei der Sonne in die Schule. Im Lauf der Zeit
bedeckte sich ihr Körper mit Haaren, aber unter diesem Fell
bewahrten
sie ihr heißes, rotes Blut. Ihr seid viel schlimmer als sie, ihr
seid mit Zahlen bewachsen, ihr wimmelt von Zahlen wie von Läusen.
Man muss alles von euch herunterreißen und euch nackt in die Wälder
jagen. Ihr müsst lernen, vor Angst, vor Freude, vor Hass und vor
Wut, vor Kälte zu zittern, ihr müsst das Feuer anbeten. Und wir,
die Mephi, wir wollen\ldots{} “ „Mephi, was ist das?“

„Mephi? Das ist ein uralter Name, das ist jener, der\ldots{} Auf dem
Felsblock dort ist ein Jüngling dargestellt, erinnerst du dich
noch? Nein, ich will es dir lieber in deiner Sprache erklären,
damit du es besser verstehst. Es gibt zwei Kräfte in der Welt,
Entropie und Energie. Die eine schafft selige Ruhe und glückliches
Gleichgewicht, die andere führt zur Zerstörung des Gleichgewichts,
zu qualvoll-unendlicher Bewegung. Unsere oder, richtig gesagt, eure
Vorfahren, die Christen, haben die Entropie als Gott verehrt. Wir
aber, die Antichristen, wir\ldots{} “ In diesem Augenblick klopfte
jemand leise an die Tür, und jener Mensch mit der gewölbten Stirn,
der mir einmal Nachricht von I gebracht hatte, trat ein. Er eilte
auf uns zu, blieb stehen, rang nach Luft und brachte lange kein
Wort heraus. Er musste aus Leibeskräften gerannt sein. „Nun, was
gibt?s? Was ist geschehen?“ fragte I und nahm ihn am Arm.

„Sie kommen\ldots{} hierher\ldots{} “, stieß er endlich hervor.
„Beschützer\ldots{} mit ihnen jener — jener Bucklige.“ „S?“

„Ja! Sie sind schon im Nachbarhaus. Gleich werden sie hier sein!
Schnell, schnell!“

„Unsinn! Wir haben noch Zeit\ldots{}“ Sie lachte, in ihren Augen tanzten
lustige kleine Flammen. Es war entweder tollkühner Mut oder etwas
anderes, das ich nicht begriff.

„I, ich bitte dich“, sagte ich flehend, „um des Wohltäters willen!
Verstehst du denn nicht\ldots{}“

„Um des Wohltäters willen?“ Ein spöttisches Lächeln. „Nun, dann um
meinetwillen \ldots{} ich bitte dich, geh.“ „Ich habe eigentlich noch
etwas mit dir zu besprechen \ldots{} aber gut, verschieben wir?s auf
morgen\ldots{} “ Sie nickte mir vergnügt (ja, vergnügt!) zu und ging mit
dem Mann hinaus. Ich war allein.

Schnell an den Schreibtisch. Ich schlug mein Manuskript auf und
nahm die Feder in die Hand, damit sie mich bei dieser Arbeit
fänden, die dem Nutzen des Einzigen Staates diente. Plötzlich
bewegte sich jedes einzelne Haar auf meinem Kopf, als wäre es
lebendig: „Wenn sie das Manuskript in die Hand nehmen und eine
Seite lesen, eine von den letzten? Was dann?“

Ohne mich zu rühren, saß ich am Schreibtisch. Die Wände bebten, die
Feder zitterte in meiner Hand, die Buchstaben verschwammen vor
meinen Augen\ldots{} Verstecken? Aber wo? — alles war ja aus Glas!
Verbrennen? Man hätte es vom Korridor und von den anderen Zimmern
aus gesehen. Außerdem hatte ich nicht die Kraft, dieses Stück
meiner selbst, das mir vielleicht teurer ist als alles übrige, zu
vernichten. Nein, ich konnte es nicht. Stimmen, Schritte im
Korridor. Da waren sie. Ich konnte gerade noch einen Stoß Blätter
ergreifen und mich draufsetzen. Jetzt war ich an den Sessel
geschmiedet, der mit jedem einzelnen Atom bebte. Der Boden meines
Zimmers war wie das Deck eines Schiffes, er hob und senkte sich\ldots{}
Zusammengekauert blickte ich verstohlen auf. Sie gingen von Zimmer
zu Zimmer, sie kamen näher, immer näher. Einige Nummern saßen wie
ich erstarrt da, andere eilten ihnen entgegen und rissen die Tür
weit auf. Die Glücklichen! Wenn ich doch auch\ldots{}

„Der Wohltäter ist eine für die ganze Menschheit unumgänglich
notwendige, totale Desinfektion, und infolgedessen gibt es in dem
Einzigen Staat keine Peristaltik\ldots{} “ Diesen völlig sinnlosen Satz
schrieb ich mit tanzender Feder nieder und beugte mich dabei noch
tiefer über den Tisch. In meinem Kopf hämmerte eine Grille wie
rasend, mit dem Rücken zur Tür lauschte ich angestrengt. Da klirrte
die Klinke, ich fühlte einen Luftzug, der Sessel begann sich zu
drehen\ldots{}

Mit unsäglicher Mühe riss ich mich von meinem Manuskript los und
wandte mich zu den Eintretenden (wie schwer ist es, Komödie zu
spielen\ldots{} ach, wer hatte mir denn heute etwas von einer Komödie
gesagt?). S kam als erster herein, schweigend, finster. Seine Augen
bohrten sich in mich, in meinen Sessel, in das Blatt Papier unter
meinen zitternden Händen. Dann sah ich bekannte Gesichter im
Türrahmen, eines von ihnen löste sich aus der Gruppe — aufgeblähte
dunkelrote Kiemen\ldots{} Mir fiel alles wieder ein, was sich vor einer
halben Stunde in meinem Zimmer abgespielt hatte, und mir war völlig
klar, dass sie sogleich\ldots{} Mein Herz, mein ganzes Wesen pochte und
hüpfte in jenem (zum Glück undurchsichtigen) Teil meines Körpers,
unter dem ich mein Manuskript verborgen hatte.

U trat von hinten an S heran und flüsterte: „Das ist D-503, der
Konstrukteur des Integral. Sie haben doch sicherlich schon von
seiner Erfindung gehört? Er sitzt immer am Schreibtisch\ldots{} Er gönnt
sich niemals Ruhe.“ Was sollte ich dazu sagen? Welch wunderbare
Frau. S kam leise auf mich zu, beugte sich über meine Schulter und
blickte auf den Tisch. Ich stemmte die Ellbogen auf mein
Manuskript, doch er sagte in strengem Ton: „Zeigen Sie, was Sie da
haben!“

Glühend rot vor Scham reichte ich ihm das Blatt. Er überflog es,
und ich sah, wie in seinen Augen ein Lächeln aufschimmerte, über
sein Gesicht huschte und im rechten Mundwinkel haften blieb.

„Etwas doppeldeutig, aber immerhin\ldots{} Nun, machen Sie weiter, wir
wollen Sie nicht länger stören.“ Er schlurfte zur Tür, und mit
jedem seiner Schritte kehrte allmählich Leben in meine Füße, Hände
und Finger zurück. Meine Seele verteilte sich gleichmäßig durch den
ganzen Körper, ich atmete befreit auf\ldots{} Aber U stand immer noch
neben mir. Sie beugte sich zu mir herab und flüsterte mir ins Ohr:
„Ihr Glück, dass ich\ldots{} “ Ich weiß nicht, was sie damit sagen
wollte. Am Abend erfuhr ich, dass sie drei verhaftet hatten.
Natürlich wagte keiner von uns, laut von diesem Vorfall zu sprechen
(der erzieherische Einfluss der unsichtbar in unserer Mitte
weilenden Beschützer). Unsere Gespräche drehten sich vor allem um
das rasche Fallen des Barometers und den bevorstehenden
Witterungsumschlag.

\section{EINTRAGUNG NR. 29}

\uebersicht{\emph{Übersicht:} Fäden im Gesicht. Keime. Eine
widernatürliche Kompression.}
Merkwürdig — das Barometer fällt, aber noch immer haben wir keinen
Wind, es herrscht tiefe Stille. Dort oben hat es schon begonnen,
das für uns unhörbare Gewitter. Die Wolken stürmen über den Himmel.
Bis jetzt sind es nur einzelne gezackte Fetzen. Es hat den
Anschein, als wäre man dort oben schon dabei, irgendeine Stadt zu
zerstören, als flögen dicke Brocken von Mauern und Türmen zu uns
herab. Diese Brocken wachsen mit erschreckender Schnelligkeit vor
meinen Augen, sie kommen immer näher, doch sie müssen noch tagelang
durch die blaue Unendlichkeit fliegen, bevor sie auf die Erde
aufschlagen.

Hier unten ist alles grabesstill. In der Luft schweben hauchfeine,
fast unsichtbare Fäden. Jedes Jahr trägt sie der Herbst aus dem
Land hinter der Mauer in die Stadt herüber, und auf einmal spürt
man etwas Fremdes, Unsichtbares im Gesicht, man möchte es
wegwischen, doch man kann sich nicht davon befreien\ldots{} An der
Grünen Mauer, wo ich heute morgen spazierenging, gibt es besonders
viele dieser Fäden. I hatte mir gesagt, ich solle sie in „unserer
Wohnung“ im Alten Haus treffen.

Unweit von dem dunklen Alten Haus hörte ich kurze, eilige Schritte
und einen keuchenden Atem hinter mir. Ich wandte mich um — es war
O!

Sie sah ganz anders aus als sonst, rund und straff. Die Uniform
spannte sich über ihrem Körper, der mir so vertraut war. Bald würde
dieser Körper den dünnen Stoff zerreißen und nach außen drängen, an
die Sonne, ans Licht. Ich musste an die grünen Schluchten jenseits
der Mauer denken, wo im Frühling die Keime die Erde durchstoßen, um
Zweige, Blätter und Blüten zu treiben. O schaute mich schweigend
an, ihre blauen Augen strahlten. Dann sagte sie:

„Ich habe Sie gesehen — damals, am Tag der Einstimmigkeit.“

„Ich habe Sie auch gesehen\ldots{} “ Ich erinnerte mich sogleich, dass
sie in dem engen Durchgang gestanden hatte, dicht an die Mauer
geduckt, den Leib mit den Händen
schützend. Unwillkürlich blickte ich auf diesen Leib, der sich
unter der Uniform wölbte. Sie schien es bemerkt zu haben, denn sie
wurde rot und lächelte verlegen: „Ich bin so glücklich, so
glücklich\ldots{} Ich sehe und höre nichts von dem, was um mich ist, ich
lausche nur in mich hinein\ldots{}“

Ich entgegnete kein Wort. Auf meinem Gesicht haftete etwas Fremdes,
Störendes, und ich konnte mich nicht davon befreien. Plötzlich nahm
sie meine Hand und führte sie an ihre Lippen\ldots{} Diese altmodische
Liebkosung, die ich zum ersten Mal in meinem Leben spürte,
beschämte und schmerzte mich so sehr, dass ich ihr meine Hand
heftig entriss.

„Sind Sie wahnsinnig?\ldots{} Worüber freuen Sie sich eigentlich? Haben
Sie denn vergessen, was Sie erwartet? Nicht sofort natürlich, aber
in ein, zwei Monaten \ldots{} “ Sie erbleichte, alle Rundungen ihres
Körpers fielen zusammen. Ich fühlte in meinem Herzen eine
unangenehme, krankhafte Kompression, die von dem so genannten
Mitleid herrührte. (Das Herz ist nichts anders als eine ideale
Pumpe. Eine Pumpe kann niemals eine Kompression bewirken, sie kann
niemals eine Flüssigkeit aufsaugen — das wäre technisch unmöglich,
geradezu absurd. Daraus folgt, wie unsinnig, widernatürlich und
krankhaft Liebe, Mitleid und Ähnliches ist, das eine solche
Kompression hervorruft.)

Stille. Neben mir das trübe grüne Glas der Mauer, vor mir ein
dunkelroter Steinhaufen\ldots{} Die Resultante dieser beiden Kräfte war
eine glänzende Idee: „Halt! Ich weiß, wie Sie gerettet werden
können. Ihr Kind gebären und dann sterben — vor diesem Schicksal
will ich Sie bewahren. Sie sollen Ihr Kind großziehen und sehen,
wie es auf Ihren Armen heranwächst.“

Am ganzen Leibe zitternd, klammerte sie sich an mich.

„Sie erinnern sich wohl noch an jene Frau“, sagte ich, „damals auf
dem Spaziergang. Sie ist jetzt im Alten

Haus. Wir gehen gleich zu ihr, und ich werde alles Nötige
veranlassen.“

Im Geist sah ich uns zu dritt durch die unterirdischen Gänge gehen,
und schon war O in jenem Land, inmitten von Blumen und Gräsern und
Blättern\ldots{} Aber sie wich betroffen zurück, ihre Mundwinkel zuckten
und neigten sich nach unten.

„Das ist doch dieselbe\ldots{} “, sagte sie.

„Hm, sie ist\ldots{}“, stotterte ich verlegen, „ja, sie ist dieselbe.“

„Und Sie verlangen von mir, dass ich zu ihr gehe, dass ich sie
bitte\ldots{} dass ich\ldots{}? Unterstehen Sie sich, noch ein Wort davon zu
sagen!“

Gesenkten Hauptes eilte sie davon. Plötzlich wandte sie sich noch
einmal um, als hätte sie etwas vergessen, und rief:

„Was ist schon dabei, dass ich sterben muss? Und Ihnen ist ja
sowieso alles gleich!“

Stille. Von oben fallen Brocken von zerstörten Mauern und Türmen
herab und wachsen mit erschreckender Schnelligkeit vor meinen
Augen; doch sie müssen noch Stunden oder gar Tage durch die
Unendlichkeit fliegen.

Langsam gleiten unsichtbare Fäden durch die Luft, sie haften an
meinem Gesicht, aber ich kann mich nicht von ihnen befreien.

Ich ging zum Alten Haus. In meinem Herzen eine absurde, quälende
Kompression \ldots{}

\section{EINTRAGUNG NR. 30}

\uebersicht{\emph{Übersicht:} Die letzte Zahl. Galileis Irrtum. Ist
es nicht besser\ldots{}}
Gestern traf ich mich mit I im Alten Haus. Unter dem verwirrenden
Lärm roter, grüner, bronzegelber und weißlicher Farbtöne, der mich
am logischen Denken hinderte, und unter dem marmornen Lächeln des
krummnasigen alten Dichters hatten wir ein langes Gespräch. Ich
will es hier Wort für Wort wiedergeben, da es, wie mir scheint, von
entscheidender Bedeutung nicht nur für den Einzigen Staat, sondern
für das Weltall ist. Unvermittelt sagte I:

„Ich weiß, dass der Integral übermorgen zum ersten Probeflug
startet. Wir werden ihn in die Hand bekommen.“ „Wie? Übermorgen?“

„Ja. Setz dich, reg dich nicht auf. Wir dürfen keine Minute
verlieren. Unter den hundert, die gestern auf bloßen Verdacht hin
verhaftet wurden, sind zwölf Mephi. Wenn wir noch zwei, drei Tage
warten, sind sie verloren.“ Ich schwieg.

„Ihr müsst Elektrotechniker, Mechaniker, Ärzte und Meteorologen als
Beobachter an Bord nehmen. Um zwölf, wenn es zur Mittagspause
läutet und alle zum Essen gehen, bleiben wir im Korridor und
schließen sie in der Kantine ein. Dann ist der Integral unser\ldots{} Es
muss sein, um jeden Preis, hörst du? Der Integral ist unsere Waffe,
mit deren Hilfe wir allem mit einem Schlag ein Ende machen können.
Ihre Flugzeuge\ldots{} ha! Ein elender Mückenschwarm gegen einen Geier!
Und sollte es wirklich nicht ohne Gewalt gehen, dann brauchen wir
nur den Auspuff des Integral nach unten zu richten. Das genügt.“

Ich sprang auf:

„Das ist ja Wahnsinn! Ist dir nicht klar, dass das, was du da
planst, eine Revolution ist?“

„Ja, es ist eine Revolution! Und warum soll es Wahnsinn sein?“

„Weil unsere Revolution die letzte war. Es kann keine neue
Revolution mehr geben. Das wissen alle.“ Sie zog spöttisch die
Augenbrauen hoch: „Mein Lieber, du bist doch Mathematiker, mehr
noch, du bist ein Philosoph. Bitte nenn mir die letzte Zahl.“ „Was
meinst du damit? Ich\ldots{} ich verstehe nicht, welche letzte Zahl?“

„Nun, die letzte, höchste, die allergrößte Zahl.“ „Aber I, das ist
ja alles dummes Zeug. Die Anzahl der Zahlen ist doch unendlich. Was
für eine letzte Zahl willst du also?“

„Und was für eine letzte Revolution willst du? Es gibt keine letzte
Revolution, die Anzahl der Revolutionen ist unendlich. Die letzte —
das ist etwas für kleine Kinder. Die Kinder fürchten sich vor der
Unendlichkeit, doch sie müssen unter allen Umständen ruhig
schlafen, und deshalb\ldots{} “

„Aber was für einen Sinn hat das alles, um des Wohltäters willen?
Was für einen Sinn hat das alles, da doch alle glücklich sind?“

„Schön, nehmen wir an, wir seien tatsächlich glücklich. Aber wie
geht es weiter?“

„Lächerlich! Eine ganz kindische Frage! Erzähle Kindern eine
Geschichte, erzähle sie zu Ende — sie werden trotzdem fragen: Und
wie geht es weiter?“ „Kinder sind eben die einzig kühnen
Philosophen. Und kühne Philosophen sind Kinder. Wie die Kinder muss
man immer fragen: Und wie geht es weiter?“

„Es gibt kein Weiter! Punkt, aus! Überall, im ganzen Weltall muss
Gleichheit und Gleichmaß herrschen\ldots{} “ „Aha! Gleichmäßigkeit,
überall! Da haben wir sie, die Entropie, die psychologische
Entropie. Ist dir als Naturwissenschaftler denn nicht klar, dass
nur in der Verschiedenartigkeit\ldots{} in Temperaturunterschieden, in
Wärmekontrasten — Leben ist? Wenn aber überall, im ganzen Weltall,
gleichartig-warme oder gleichartig-kalte Körper sind\ldots{} nun, dann
muss man sie zusammenstoßen, damit Feuer, eine Explosion, die Hölle
entsteht. Und wir werden sie zusammenstoßen.“

„Aber I, begreife doch — gerade das haben ja unsere Vorfahren im
200jährigen Krieg getan\ldots{} “ „Oh, sie hatten recht, tausendmal
recht. Sie haben nur einen Fehler gemacht: Sie glaubten später, sie
seien die letzte Zahl, etwas, das es in der Natur niemals gibt,
niemals. Ihr Irrtum war der Irrtum Galileis — er hatte recht mit
seiner Behauptung, dass sich die Erde um die Sonne dreht, doch er
wusste nicht, dass sich das ganze Sonnenzentrum um ein anderes
Zentrum bewegt, er wusste nicht, dass die wirkliche, nicht die
relative Bahn der Erde durchaus kein naiver Kreis ist\ldots{}“ „Und
ihr?“

„Wir? Bis jetzt wissen wir, dass es keine letzte Zahl gibt.
Vielleicht werden wir es einmal vergessen. Ja, wir werden es ganz
gewiss vergessen, wenn wir alt werden. Und dann fallen auch wir
unaufhaltsam hinab, wie im Herbst die Blätter von den Bäumen
fallen, wie ihr übermorgen fallen werdet\ldots{} Nein, nein, Liebster,
du nicht, du bist ja einer der Unseren!“

Stürmisch, mit funkelnden Augen, von Leidenschaft glühend — noch
nie hatte ich sie so gesehen — umarmte sie mich. Dann sah sie mir
fest in die Augen und sagte:

„Also denk daran — Punkt zwölf!“ „Ja, ich vergesse es nicht“,
antwortete ich. Sie ging hinaus, und ich war allein mit dem
vielstimmigen Lärm von Blau, Rot, Grün, Bronze und Orange. Ja, um
zwölf\ldots{} Plötzlich spürte ich etwas Fremdes in meinem Gesicht, von
dem ich mich nicht befreien konnte. Der gestrige Morgen fiel mir
ein. U, die Worte, die sie I zugeschrien hatte\ldots{} Wie kam sie dazu?
Verrückt! Ich machte mich eilig auf den Heimweg. Hinter mir ein
gellender Vogelschrei über der Grünen Mauer. Vor mir die Stadt im
Schein der untergehenden Sonne, ganz aus himbeerrotem,
kristallisiertem Feuer — die runden Kuppeln, die riesigen
Häuserwürfel, die Spitzen der Akkumulatorentürme, die am Himmel
erstarrten Blitzen glichen. All das, diese ganze makellose
geometrische Schönheit, sollte ich mit meinen eigenen Händen
vernichten\ldots{} Gab es wirklich keinen Ausweg?

Ich kam an einem Auditorium vorbei (die Nummer habe ich vergessen).
Drinnen ein Haufen aufgestapelter Bänke, in der Mitte Tische, mit
Tüchern aus schneeweißem Glas bedeckt, auf dem Weiß ein Fleck
rötlichen Sonnenblutes. In all dem war ein unbekanntes und deshalb
beängstigendes Morgen verborgen. Es ist doch höchst widernatürlich,
dachte ich, wenn ein denkendes, sehendes Wesen unter
Regellosigkeiten, Unbekannten und allerlei x leben muss. Das ist
genauso, als ob man einem Menschen die Augen verbinden und ihn
zwingen würde, zu gehen, zu tasten, zu stolpern. Er weiß, dass
irgendwo in der Nähe ein Abgrund ist; noch ein Schritt — und nur
ein plattgedrücktes, unförmiges Stück Fleisch bleibt von ihm übrig.
Was aber, wenn man einfach kopfüber hinunterspringt? Wäre das nicht
das einzig Richtige, würde es nicht alles mit einem Male lösen?

\section{EINTRAGUNG NR. 31}

\uebersicht{\emph{Übersicht:} Die große Operation. Ich habe alles
verziehen. Zusammenstoß.}
Gerettet! Im allerletzten Augenblick, da es schon so aussah, als
gäbe es nirgends einen Halt, als wäre alles aus\ldots{} So, als wäre man
schon die Stufen zur Maschine des Wohltäters hinaufgestiegen, als
hätte sich der gläserne Schirm mit dumpfem Klirren über einem
geschlossen, als sähe man sich zum letzten Mal mit weitgeöffneten
Augen an dem blauen Himmel satt\ldots{}

Und plötzlich entdeckt man, dass alles nur ein Traum war. Die Sonne
strahlt rosig und heiter, und die Mauern — wie wohl tut es, mit der
Hand die kalte Mauer zu streicheln — und das Kopfkissen — wie
berauscht man sich an der kleinen Vertiefung, die der Kopf in das
Kissen gedrückt hat\ldots{} Das alles empfand ich, als ich heute morgen
die Staatszeitung las. Ja, was ich gestern erlebt habe, war nur ein
entsetzlicher Traum, und nun ist er vorbei. Dabei hatte ich in
meiner Verzagtheit, in meinem kläglichen Kleinmut schon an
Selbstmord gedacht! Ich schäme mich, jetzt die letzten Seiten zu
lesen, die ich gestern geschrieben habe. Nun, mögen sie stehen
bleiben als Erinnerung an jenes Unglaubliche, das wohl hätte sein
können, das aber nie sein wird, nie!

Auf der ersten Seite der Staatszeitung stand in großen Lettern:
Freut euch, denn von nun an seid ihr vollkommen! Bis zum heutigen
Tage waren eure Kinder, die Mechanismen, vollkommener als ihr.
Wodurch?

Jeder Funke des Dynamos ist ein Funke der reinsten Vernunft, jeder
Stoß des Kolbens ist ein reiner Syllogismus. Ist diese unfehlbare
Vernunft nicht auch in euch? Die Philosophie von Kranen, Pressen
und Pumpen ist geschlossen und klar wie der Kreis eines Zirkels.
Bewegt sich eure Philosophie nicht auch in Kreisen? Die Schönheit
des Mechanismus liegt in seinem Rhythmus, der unveränderlich und
genau ist wie der eines Pendels. Seid ihr, die ihr von Kindesbeinen
an nach dem System Taylors erzogen wurdet, nicht auch so exakt wie
ein Pendel?

Aber entscheidender als dies alles ist: Mechanismen haben keine
Phantasie. Habt ihr je gesehen, dass bei der Arbeit auf dem Gesicht
eines Pumpenzylinders ein entrücktes, töricht-verträumtes Lächeln
spielt? Habt ihr je gehört, dass die Krane nachts, in Stunden, die
der Ruhe dienen sollen, sich unruhig hin-und herwerfen und seufzen?
Nein!

An euch aber — schämt euch! — entdecken die Beschützer dieses
Lächeln und Seufzen immer öfter. Und — schlagt schuldbewusst die
Augen nieder — die Historiker des Einzigen Staates bitten um
Entlassung, weil sie diese schmachvollen Ereignisse nicht
beschreiben wollen. Doch das alles ist nicht eure Schuld — ihr seid
krank. Eure Krankheit heißt Phantasie.

Die Phantasie ist ein Wurm, der schwarze Furchen in eure Stirnen
frisst, ein Fieber, das euch treibt, immer weiterzueilen — wenn
auch dieses „weiter“ dort beginnt, wo das Glück endet. Die
Phantasie ist das letzte Hindernis auf dem Weg zum Glück. Freut
euch, dieses Hindernis ist beseitigt. Der Weg ist frei.
nie staatliche Wissenschaft hat vor kurzem eine wichtige Entdeckung
gemacht: das Zentrum der Phantasie ist ein winziger Knoten an der
Gehirnbasis. Eine dreimalige Bestrahlung dieses Knotens — und ihr
seid von der Phantasie geheilt. Für immer.

Ihr seid vollkommen, ihr seid wie Maschinen, der Weg zum
vollkommenen Glück ist frei. Kommt in die Auditorien und lasst euch
operieren. Es lebe die Große Operation, es lebe der Einzige Staat!
Es lebe der Wohltäter! Liebe Leser, wenn Sie dies in meinen
Aufzeichnungen lesen würden, die einem sonderbaren, altmodischen
Roman gleichen, wenn Sie die nach frischer Druckerschwärze
riechende Staatszeitung in Ihren zitternden Händen hielten, wenn
Sie wüssten, dass es die wirklichste Wirklichkeit ist — wenn nicht
heute, so doch morgen —, dann würden Sie gewiss das gleiche
empfinden wie ich jetzt. Der Kopf würde Ihnen schwindeln, bange,
süße, eisige Schauer würden Ihnen über Arme und Rücken laufen. Sie
würden glauben, Sie seien ein Gigant, ein Atlas, der unfehlbar an
die gläserne Decke stößt, wenn er sich aufrichtet. Ich nahm den
Telefonhörer ab:

„I-330, ja, 330.“ Gleich darauf stammelte ich: „Sind Sie zu Hause?
Haben Sie?s schon gelesen? Ist das nicht wunderbar?"

„Ja\ldots{}“ Ein langes, dunkles Schweigen. „Ich muss Sie heute
unbedingt sehen. Kommen Sie nach 16 Uhr zu mir.“ Liebste! Liebste!
„Unbedingt!\ldots{}“ Ich lächelte — ich konnte nicht länger an mich
halten. Lächelnd ging ich durch die Straßen. Der Wind sprang mich
an. Er brauste, wirbelte, pfiff, peitschte mein Gesicht. Aber das
stimmte mich nur noch froher. Tobe nur, heule nur, jetzt wirst du
unsere Mauern nicht mehr umwerfen! Über meinem Kopf jagten
bleigraue Wolken dahin; nun — sie werden
die Sonne nicht verdunkeln — wir haben sie an den Zenit
geschmiedet, wir, die Nachfolger von Nazareth. An der Ecke drängten
sich die Nummern. Sie pressten die Stirnen an die gläsernen Mauern
des Auditoriums. Drinnen lag einer auf dem blendendweißen Tisch.
Unter einem weißen Tuch sahen die nackten gelblichen Fußsohlen
hervor, Ärzte in weißen Kitteln beugten sich über das Kopfende des
Tisches, eine weiße Hand hielt eine Spritze mit irgendeiner
Flüssigkeit.

„Warum gehen Sie nicht auch hinein?“ fragte ich einen oder,
richtiger, alle.

„Und Sie?“ Einer drehte sich zu mir um. „Ich gehe später. Ich muss
vorher noch \ldots{} “ Ein wenig verwirrt ging ich weiter. Ich musste
tatsächlich I noch vorher sehen. Aber warum denn „vorher“ — darauf
fand ich keine Antwort\ldots{}

Auf der Werft. Der Integral leuchtete wie bläuliches Eis. Im
Maschinenraum heulte der Dynamo, zärtlich ein und dasselbe Wort
wiederholend, mein Wort. Ich bückte mich und streichelte den
langen, kalten Auspuff des Motors. Morgen wirst du leben, morgen
wird ein Funkenregen in deinem Leibe sprühen und dich erheben
lassen\ldots{} Mit welchen Augen würde ich dieses gläserne Untier sehen,
wenn alles noch so geblieben wäre wie gestern? Wenn ich gewusst
hätte, dass ich es morgen um zwölf Uhr verraten würde? Ja,
verraten\ldots{}

Jemand zupfte mich am Ellbogen. Ich wandte mich um: das flache
Tellergesicht des zweiten Konstrukteurs. „Wissen Sie schon?“ fragte
er. „Was? Die Operation? Ja, eine großartige Sache.“ „Nein, das
meine ich nicht. Der Probeflug ist auf übermorgen verschoben
worden. Alles wegen dieser Operation\ldots{} Wir haben uns umsonst
abgehetzt.“

Alles wegen der Operation! Lächerlicher, beschränkter Mensch! Wenn
es keine Operation gäbe, säße er morgen im gläsernen Käfig, würde
wie wahnsinnig hin und her rennen\ldots{}

In meinem Zimmer. 12.30 Uhr. Als ich hereinkam, saß U an meinem
Schreibtisch, knochig, kerzengerade, die rechte Wange auf die Hand
gestützt. Sie musste lange gewartet haben, denn als sie aufsprang
und mir entgegeneilte, sah ich auf ihrer Wange fünf Vertiefungen,
die von den Fingern herrührten.

Eine Sekunde lang dachte ich an jenen unglücklichen Morgen — sie
hatte neben I am Schreibtisch gestanden, voller Wut\ldots{} Aber das
dauerte nur einen Augenblick, dann hatte die heutige Sonne schon
alles weggewischt. Es war, wie wenn man an einem klaren Tag ins
Zimmer kommt und zerstreut das Licht einschaltet — die Lampe
brennt, doch sie scheint nicht dazusein, so lächerlich, armselig
und überflüssig ist sie.

Ohne Zögern streckte ich ihr die Hand hin; ich verzieh ihr alles\ldots{}
Sie nahm meine Hände und drückte sie fest. Ihre Backen, die wie ein
altmodischer Zierat über die Kinnladen herabhingen, begannen zu
zittern. Sie sagte: „Ich habe auf Sie gewartet\ldots{} Ich wollte nur
eine Minute\ldots{} Ich wollte Ihnen nur sagen, wie glücklich ich bin,
wie sehr ich mich für Sie freue! Morgen oder übermorgen sind Sie
genesen, wie neugeboren\ldots{} “ Auf dem Schreibtisch sah ich die
beiden letzten Seiten meiner gestrigen Aufzeichnungen; sie lagen
noch genauso da, wie ich sie gestern Abend hatte liegenlassen. Wenn
sie gesehen hätte, was ich da geschrieben habe\ldots{} Nun, das wäre
jetzt auch gleich. Das alles gehört bereits der Geschichte an, es
ist schon so weit weg, als sähe ich es durch ein umgekehrtes
Fernglas\ldots{}

„Ja“, erwiderte ich. „Übrigens, auf dem Prospekt habe ich eben
etwas Seltsames beobachtet: ein paar Menschen gingen vor mir her,
und denken Sie, ihr Schatten leuchtete! Ich glaube ganz gewiss,
dass es morgen überhaupt keine Schatten mehr geben wird, weder von
einem Menschen noch von einem Gegenstand; die Sonne wird alles
durchdringen\ldots{} “

„Sie sind ein Träumer! Meinen Kindern würde ich nicht erlauben, so
zu reden \ldots{} “, sagte sie mit zärtlicher Strenge und erzählte mir,
dass sie ihre ganze Schulklasse zur Operation geführt habe und dass
man die Kinder an den Tischen hatte festbinden müssen. Doch man
müsse ja „erbarmungslos“ lieben, und sie habe sich nun endlich
entschlossen\ldots{} Sie lächelte mir ermunternd zu und ging. Zum Glück
war die Sonne heute noch nicht stehen geblieben; es war 16 Uhr. Mit
pochendem Herzen klopfte ich an die Tür von I.s Zimmer. „Herein!“

Ich kniete vor ihrem Sessel nieder, umschlang ihre Beine, warf den
Kopf zurück und blickte ihr in die Augen\ldots{} Jenseits der Mauer
stieg ein Gewitter auf, die Wolken wurden immer dunkler. Ich
stammelte wirres Zeug: Ich fliege mit der Sonne irgendwohin\ldots{}
nein, nicht irgendwohin, jetzt kennen wir unsere Flugrichtung.
Hinter mir flammensprühende Planeten, in denen feurige, laut
tönende Blumen wachsen, dann stumme blaue Planeten, wo vernünftige
Steine zu einer organisierten Gesellschaft vereinigt sind,
Planeten, die gleich unserer Erde den Gipfel des höchsten, des
vollkommenen Glücks erreicht haben\ldots{}

Plötzlich sprach eine Stimme von oben: „Du glaubst doch nicht, dass
diese vernünftigen Steine der Gipfel sind?“

Immer spitzer, immer dunkler wurde das Dreieck in ihrem Gesicht:

„Und das Glück? Wünsche sind etwas Qualvolles, findest du nicht
auch? Man ist nur dann glücklich, wenn man keinen Wunsch mehr
hat\ldots{} Welch ein Irrtum, welch ein lächerliches Vorurteil, dass wir
bis heute ein Pluszeichen vor das Glück gesetzt haben, vor das
absolute Glück hingegen ein Minuszeichen, das göttliche Minus.“ Ich
erinnere mich, dass ich zerstreut murmelte: „Das absolute Minus —
273 Grad\ldots{} “

„Stimmt, minus 273. Ein wenig kühl, aber beweist das nicht, dass
wir auf dem Gipfel stehen?“ Sie sprach gleichsam meine eigenen
Gedanken aus, wie sie es schon einmal getan hatte. Doch das hatte
etwas Beängstigendes für mich, ich konnte es nicht ertragen, und
mit großer Mühe zwang ich mir ein Nein ab. „Nein“, sagte ich,
„du\ldots{} du scherzest.“ Sie lachte laut, zu laut. Sie stand auf,
legte die Hände auf meine Schultern und blickte mich lange an. Dann
zog sie mich an sich, und ich vergaß alles, fühlte nur noch ihre
heißen Lippen. „Leb wohl!“

Das kam aus weiter Ferne, ganz von oben, und erreichte mich
vielleicht erst nach zwei, drei Minuten. „Wieso, leb wohl?“

„Du bist doch krank, du hast um meinetwillen ein Verbrechen
begangen. Hat dich das nie gequält? Jetzt gehst du zur Operation —
und dann bist du von mir geheilt. Das bedeutet: Leb wohl.“ „Nein!“
schrie ich. „Wie, du verschmähst das Glück?“

Mein Kopf zersprang in zwei Hälften, zwei logische Züge stießen
zusammen und entgleisten\ldots{}

„Wähle: die Operation und das vollkommene Glück oder\ldots{}“

„Ich kann ohne dich nicht leben“, murmelte ich, oder vielleicht
dachte ich es auch nur, aber sie hatte es gehört. „Ich weiß“,
antwortete sie, und während ihre Hände immer noch auf meinen
Schultern ruhten und sie mir tief in die Augen blickte, fuhr sie
fort: „Dann bis morgen. Morgen um zwölf.“ „Nein, es ist um einen
Tag aufgeschoben\ldots{} übermorgen.“

„Um so besser für uns. Also übermorgen\ldots{} “ Ich ging allein durch
die dämmrige Straße. Der Wind packte mich, trug mich fort wie einen
Papierfetzen, von dem gusseisernen Himmel stürzten dicke Brocken
herab — sie müssen noch ein, zwei Tage durch die Unendlichkeit
fliegen\ldots{} Die Uniformen, die mir begegneten, hielten mich an, aber
ich ging allein weiter. Mir war klar: Alle waren gerettet, doch für
mich gab es keine Rettung mehr; ich wollte nicht gerettet werden.

\section{EINTRAGUNG NR. 32}

\uebersicht{\emph{Übersicht:} Ich glaube es nicht. Traktoren. Ein
armseliges Bündel Mensch.}
Können Sie sich vorstellen, dass Sie sterben werden? Nun, der
Mensch ist sterblich, und da ich ein Mensch bin\ldots{} Aber das wissen
Sie ja. Dennoch: haben Sie es sich je vorstellen können, nicht nur
mit dem Verstand, sondern mit dem ganzen Körper und so deutlich,
dass Sie es geradezu fühlten: diese Finger, die jetzt dieses Blatt
Papier halten, werden dereinst gelb und eiskalt sein\ldots{}

Nein, Sie können es sich natürlich nicht vorstellen, und deshalb
sind Sie auch bis jetzt noch nicht vom zehnten Stockwerk auf die
Straße gesprungen, deshalb essen Sie noch, blättern eine Seite in
einem Buch um, rasieren sich, lächeln, schreiben.

So war mir heute zumute, genau so. Ich weiß, dass der kleine
schwarze Zeiger der Uhr nach Mitternacht hinabschleicht, wieder
langsam emporklettert, irgendeinen letzten Strich überschreitet —
und dann beginnt es, das unvorstellbare Morgen. Ja, ich weiß es,
und trotzdem glaube ich es nicht, oder vielleicht kommen mir
vierundzwanzig Stunden wie vierundzwanzig Jahre vor. Deshalb kann
ich noch etwas tun, Fragen beantworten, die Treppe zum Integral
hinaufsteigen. Er schaukelt auf dem Wasser, das fühle ich noch, und
ich weiß, dass man sich am Geländer festhalten muss. Ich spüre das
kalte Glas unter meiner Hand. Ich sehe, wie die durchsichtigen,
lebendigen Krane ihre Kranichhälse biegen, die Schnäbel vorstrecken
und die Motoren des Integral mit grauenhaft explosiver Nahrung
füttern. Und drunten auf dem Fluss erkenne ich deutlich blaue, im
Winde anschwellende Wasseradern und Wasserknoten. All das ist sehr
weit weg von mir, fremd, flach, wie eine Zeichnung auf einem Blatt
Papier. Mir kommt es ganz merkwürdig vor, dass das flache Gesicht
des zweiten Konstrukteurs plötzlich sagt: „Wie viel Treibstoff
nehmen wir mit? Rechnen wir drei oder dreieinhalb Stunden\ldots{} “

„Fünfzehn Tonnen genügen. Nein, nehmen Sie besser hundert mit\ldots{}“

Ich wusste ja, was morgen geschehen würde! „Hundert? Warum so viel?
Das reicht ja für eine ganze Woche! Für viel länger sogar!“
„Unterwegs kann allerlei passieren.“

„Ja, ich weiß.“

Der Wind pfiff, die Luft war von etwas Unsichtbarem erfüllt. Es
machte mir große Mühe, zu atmen, zu gehen. Schwerfällig, langsam,
ohne eine Sekunde stehenzubleiben, bewegte sich der Uhrzeiger am
Akkumulatorenturm vorwärts. Die in den Wolken verborgene Turmspitze
sog mit dumpfem Getöse Elektrizität ein. Die Schornsteine der
Musikfabrik heulten. Die Nummern marschierten wie immer zu vieren,
doch die Reihen waren nicht geschlossen, sie schienen im Wind zu
schwanken. Da, an der Ecke stießen sie mit etwas zusammen, wichen
zurück und wurden zu einem starren, atemlosen Haufen. Mit einemmal
hatten sie alle lange Gänsehälse.

„Sehen Sie! Sehen Sie nur! Dort, schnell!“ „Sie! Das sind sie!“

„Ich gehe um keinen Preis hin! Nein, lieber den Kopf auf die
Maschine des Wohltäters\ldots{}“ „Still! Wahnsinniger\ldots{} “

Die Tür des Auditoriums an der Ecke war weit geöffnet, langsam kam
eine Kolonne von etwa fünfzig Menschen herausgestampft. Menschen
ist nicht das richtige Wort — nein, es waren keine Füße, sondern
schwere, von einem unsichtbaren Triebwerk bewegte Räder, es waren
keine Menschen, sondern Traktoren in Menschengestalt. Über ihren
Köpfen knatterte eine weiße Fahne, auf die eine goldene Sonne
gestickt war, und in den Sonnenstrahlen las ich:

Wir sind die ersten! Wir sind operiert! Alle nach uns! Unaufhaltsam
pflügten sie durch die Menge, und wenn statt unserer blaugrauen
Reihen ihnen eine Mauer, ein Baum, ein Haus im Weg gestanden hätte,
hätten sie alles niedergewalzt. Inzwischen waren sie zur Mitte des
Prospekts gelangt und bildeten eine Kette, wobei sie die Gesichter
uns zuwandten. Wie Gänse die Hälse reckend, warteten wir beklommen,
dunkle Wolken glitten über den Himmel, der Wind pfiff.
plötzlich schwenkten beide Flügel der Kette ein, sie kamen
schneller, immer schneller auf uns zu, und schon hatten sie uns
eingekreist. Sie drängten uns zu der weit geöffneten Tür des
Auditoriums, hinein in den Saal\ldots{} Da, ein markerschütternder
Schrei: „Rettet euch! Lauft!“

Und alle stürmten davon. Dicht an der Mauer war noch ein schmaler
Durchgang frei; alle rannten zu dieser Stelle, mit den Köpfen
voraus — die Köpfe wurden im Nu zu spitzen Keilen. Stampfende Füße,
fuchtelnde Hände, Uniformen sprühten wie ein Wasserstrahl aus einem
Feuerlöschschlauch auseinander und zerstreuten sich. Ein S-förmiger
Körper tauchte vor mir auf und war im nächsten Augenblick
verschwunden, als hätte ihn der Erdboden verschluckt. Ich war
allein unter Armen und Beinen. Ich lief, so schnell ich konnte\ldots{}
Wenig später, als ich in einem Hauseingang stand, um Atem zu holen,
flog, vom Wind gleich einem Span hergeweht, eine Gestalt auf mich
zu.

„Ich bin die ganze Zeit\ldots{} hinter Ihnen hergelaufen\ldots{} Ich will
nicht\ldots{} Ich will nicht operiert werden\ldots{} Helfen Sie mir.“

Kleine, runde Hände legten sich auf meinen Arm, blaue, runde Augen
blickten mich flehend an. O! Sie setzte sich auf die kalten
Treppenstufen und kauerte sich zu einem Bündel zusammen. Ich beugte
mich über sie, streichelte ihr Haar, ihre Wangen — meine Hände
waren nass. Sie schlug die Hände vors Gesicht und sagte kaum
vernehmlich:

„Jede Nacht, wenn ich allein bin, denke ich an das Kind, wie es
aussehen wird, wie ich es\ldots{} Und wenn sie mich heilen, wozu soll
ich dann noch leben? Ich kann nicht mehr, Sie müssen, Sie müssen
mir helfen.“ Ein dummes Gefühl, aber ich war tatsächlich davon
überzeugt, dass ich es musste. Dumm, weil diese meine Pflicht ein
neues Verbrechen war, dumm, weil Weiß nicht zugleich Schwarz sein
kann, weil Pflicht und Verbrechen sich niemals decken können.
Vielleicht gibt es aber im Leben weder Schwarz noch Weiß vielleicht
hängt die Farbe nur von dem logischen Grundsatz ab, von dem man
ausgeht. Und wenn das der Grundsatz war, dass ich gegen das Gesetz
ein Kind mit ihr gezeugt hatte\ldots{} „Gut“, sagte ich. „Hören Sie auf
zu weinen! Ich werde Sie zu I bringen, wie ich Ihnen damals
vorschlug.“ „Ja“, antwortete sie leise, immer noch die Hände vorm
Gesicht.

Ich half ihr aufstehen. Schweigend, jeder mit seinen eigenen oder
beide vielleicht mit den gleichen Gedanken beschäftigt, gingen wir
durch die dunkle Straße, an stummen, bleifarbenen Häusern vorbei.
Plötzlich hörte ich schlurfende Schritte hinter mir. An einer
Straßenecke drehte ich mich um — und inmitten der jagenden Wolken,
die sich in dem trüben, gläsernen Pflaster spiegelten, erblickte
ich S. Meine Arme kamen sofort aus dem Takt und vollführten
unsichere Bewegungen, als gehörten sie nicht mir. Ich berichtete O
mit lauter Stimme, dass morgen der Integral zum ersten Mal starten
werde und dass das etwas Gewaltiges und Wunderbares sei. O schaute
mich verwundert mit ihren runden blauen Augen an und sah auf meine
sinnlos gestikulierenden Hände. Ich ließ sie nicht zu Wort kommen,
ich redete, redete. Aber in mir — das konnte nur ich hören —
summte und hämmerte der Gedanke: „Unmöglich\ldots{} ich kann sie
unmöglich jetzt zu I bringen.“ Statt links einzubiegen, ging ich
nach rechts. Die Brücke bot uns dreien, mir, O und S, der uns
folgte, ihren sklavisch gekrümmten Rücken dar. Aus den
hellerleuchteten Häusern am jenseitigen Ufer fiel Lichtschein auf
das Wasser und zersprühte in tausend tanzende Funken. Der Wind
dröhnte wie eine Basssaite, die aus einem dicken Tau besteht. Und
im Dröhnen des Basses vernahmen wir die ganze Zeit schlurfende
Schritte hinter uns. Wir kamen zu dem Haus, in dem ich wohne. O
sagte: „Sie haben doch versprochen\ldots{} “

Aber ich ließ sie nicht ausreden und schob sie hastig durch die
Tür. Über dem Tisch im Vestibül sah ich die zitternden Hängebacken
von U; ein Haufen Nummern drängte sich zeternd und streitend um
sie. Ich zog O in die entgegengesetzte Ecke, setzte sie mit dem
Rücken zur Wand auf einen Stuhl (ich hatte bemerkt, dass draußen
ein dunkler Schatten mit einem großen Kopf über das Pflaster
huschte) und nahm mein Notizbuch aus der Tasche. O saß da, als wäre
ihr Körper unter der Uniform verdampft, als wäre sie nur noch eine
leere Hülle mit Augen, aus denen einen eine blaue Leere anblickte.
Müde sagte sie: „Warum haben Sie mich hierher geführt? Wollen Sie
mich betrügen?“

„Nein! Still! Sehen Sie, draußen vor dem Haus\ldots{}“ „Ja, ein
Schatten.“

„Er ist schon die ganze Zeit hinter mir her. Ich kann Sie nicht zu
I bringen. Verstehen Sie es doch! Ich schreibe rasch ein paar
Zeilen, die nehmen Sie mit und gehen allein. Ich weiß, dass er hier
bleiben wird.“ Unter ihrer Uniform regte sich wieder ein Körper,
ihr Leib rundete sich allmählich, ihr Gesicht hellte sich auf.

Ich gab ihr den Zettel und drückte ihre kalte Hand. Meine Augen
ruhten zum letzten Mal in ihren blauen

Augen.

„Leben Sie wohl! Vielleicht werden wir uns wieder sehen\ldots{} “

Sie zog ihre Hand zurück. Mit hängendem Kopf machte sie zwei
Schritte, drehte sich rasch um und stand wieder neben mir. Ihre
Lippen zuckten, ihre Augen, ihr Mund, ihr ganzer Körper sagten mir
ein Wort, immer das gleiche Wort — welch unerträgliches Lächeln in
ihrem Gesicht, welcher Schmerz\ldots{}

Dann sah ich das zusammengeduckte Bündel Mensch in der Tür und
gleich darauf einen winzigen Schatten vor dem Haus. Sie entfernte
sich, ohne sich noch einmal umzusehen.

Ich trat zu U an den Tisch. Sie blies erregt die Kiemen auf und
sagte:

„Der da behauptet, er habe beim Alten Haus einen nackten Menschen
gesehen, der ganz mit Haaren bedeckt war. Verstehen Sie das? Die
sind ja alle verrückt geworden.“

Aus der Gruppe rief eine Stimme: „Jawohl! Ich habe ihn auch
gesehen!“

„Was sagen Sie dazu? Ist das nicht Irrsinn?“ sagte U zu mir. Sie
sprach das Wort „Irrsinn“ so überzeugt aus, dass ich mich fragte:

„Ist alles, was in letzter Zeit mit mir und um mich herum geschehen
ist, vielleicht nichts weiter als Irrsinn?“ Ich blickte auf meine
behaarten Hände, und da fiel mir ein: „In dir ist gewiss noch ein
Tropfen Waldblut\ldots{} Vielleicht habe ich dich deswegen\ldots{} “ Nein,
zum Glück ist es kein Irrsinn. Nein, zum Unglück ist es keiner.

\section{EINTRAGUNG NR. 33}

\uebersicht{\emph{Übersicht:} Ohne Übersicht. In aller Eile ein
letztes Wort.}
Der bewusste Tag war angebrochen. — Rasch ein Blick in die Zeitung,
vielleicht war dort\ldots{} Ich las die Zeitung nur mit den Augen (meine
Augen sind wie eine Feder, wie ein Rechenschieber, den man in den
Händen hält, etwas Fremdes, ein Instrument). Eine riesige
Schlagzeile zog sich über die erste Seite: Die Feinde des Glückes
schlafen nicht. Haltet das Glück mit beiden Händen fest! Morgen
ruht jegliche Arbeit — sämtliche Nummern werden zur Operation
antreten. Wer nicht erscheint, endet durch die Maschine des
Wohltäters. Morgen! Kann es überhaupt noch ein Morgen geben? Die
Macht der Gewohnheit ließ mich die Hand nach dem Bücherbrett
ausstrecken und die heutige Zeitung zu den übrigen in eine Mappe
mit Goldprägung legen. Dabei dachte ich: Wozu? Ich werde nie wieder
in dieses Zimmer zurückkehren!

Die Zeitung fiel auf den Boden. Ich stand da und blickte mich im
Zimmer um; hastig raffte ich alles zusammen, was ich mitnehmen
wollte, und warf es in einen imaginären Koffer. Den Tisch, die
Bücher, den Sessel. Auf diesem Sessel hatte I damals gesessen\ldots{}
und ich hatte vor ihr auf der Erde gekniet. Das Bett\ldots{} Dann
wartete ich eine, zwei Minuten lang töricht auf irgendein Wunder;
vielleicht würde das Telefon rasseln, vielleicht würde sie mir
sagen, dass\ldots{} Nein, es geschah kein Wunder\ldots{}

Ich gehe jetzt hinaus, ins Unbekannte. Dies sind meine letzten
Worte. Leben Sie wohl, liebe unbekannte Leser, mit denen ich so
vieles durchlebt, denen ich, der an dem unheilbaren Leiden Seele
Erkrankte, mich ganz enthüllt habe, bis zum letzten abgebrochenen
Schräubchen, bis zur letzten geplatzten Sprungfeder\ldots{} Ich gehe.

\section{EINTRAGUNG NR. 34}

\uebersicht{\emph{Übersicht:} Die Freigelassenen. Sonnennacht. Die
Radio-Walküre.}
Ach, hätte ich nur mich selbst und alle anderen in tausend Stücke
zerspringen lassen, wäre ich nur mit ihr irgendwo jenseits der
Grünen Mauer, unter Tieren, die ihre gelben Hauer fletschen, wäre
ich nur nie wieder hierher gekommen! Dann wäre mir tausendmal,
Millionen Mal leichter. Aber jetzt — was nun? Soll ich sie
erwürgen, diese — Doch würde das noch irgend jemandem helfen? Nein,
nein, nein! Nimm dich fest in die Hand, D-503. Ergreife den Hebel
der Logik mit aller Kraft und drehe gleich einem Sklaven der Alten
die Mühlsteine des Syllogismus, bis du alles niederschreibst, was
geschehen ist.

Als ich an Bord des Integral ging, waren schon alle auf ihren
Plätzen. Durch das gläserne Deck sah ich tief unten neben
Telegrafen, Dynamos, Transformatoren, Höhenmessern, Ventilen,
Zeigern, Motoren, Pumpen und Röhren winzige Menschen wimmeln. Im
Kommandoraum beugten sich die Leute vom Amt für Wissenschaft über
Tabellen und Instrumente, neben ihnen stand der zweite Konstrukteur
mit seinen beiden Assistenten.

Die drei hatten ihre Köpfe wie Schildkröten eingezogen, ihre
Gesichter waren herbstlich-fahl und stumpf.

„Was ist?“ fragte ich.

„Ach, nichts weiter\ldots{} “, antwortete einer von ihnen mit einem
grauen, matten Lächeln. „Vielleicht müssen wir irgendwo landen. Wer
weiß, was alles passiert. Überhaupt weiß niemand\ldots{} “

Ihr Anblick war mir unerträglich; ich konnte sie nicht länger
ansehen, sie, die ich in einer Stunde mit meinen Händen für immer
von der mütterlichen Brust des Einzigen Staates losreißen sollte.
Sie erinnerten mich an die tragischen Gestalten der Freigelassenen,
deren Geschichte jedes Schulkind kennt. Dieses Epos erzählt davon,
wie drei Nummern versuchsweise einen Monat lang von der Arbeit
befreit wurden: Tut, was ihr wollt, geht, wohin ihr wollt. (Das war
im dritten Jahrhundert nach der Schaffung der Gesetzestafel.)

Die Unglücklichen lungerten in der Nähe ihrer früheren
Arbeitsstätte herum und starrten mit gierigen Blicken auf die
anderen; sie blieben mitten auf der Straße stehen und vollführten
stundenlang jene Bewegungen, die zu bestimmten Tageszeiten bereits
ein Bedürfnis für ihren Organismus geworden waren: sie sägten und
hobelten Luft, sie bearbeiteten mit unsichtbaren Hämmern
unsichtbare Eisenstangen. Am zehnten Tag hielten sie es nicht mehr
aus: sie fassten einander bei den Händen und gingen ins Wasser.
Unter den Klängen unseres Marsches sanken sie immer tiefer, bis die
Wellen ihren Qualen ein Ende machten.

Ich wiederhole: Es war bedrückend für mich, sie anzusehen, und ich
ging eilig hinaus. „Ich will nur rasch zum Maschinenraum“, sagte
ich und machte mich auf den Weg.

Man fragte mich etwas, ich glaube, wie viel Volt man für die
Startexplosion nehmen solle, wie viel Wasserballast
wir für die Heckzisterne brauchten. In mir war ein Grammofon, das
schnell und genau alle Fragen beantwortete, während ich selber
meinen eigenen Gedanken nachhing. Aber plötzlich begegnete mir
etwas in dem engen Korridor, und damit fing alles erst richtig an.
Graue Uniformen, graue Gesichter huschten vorbei, und eine Sekunde
lang sah ich unter ihnen tief in die Stirn hängende Haare und
tiefliegende Augen — es war der Mann von damals. Ich wusste, dass
sie hier waren und dass es für mich kein Entrinnen gab, mir blieben
nur noch zehn Minuten\ldots{} Ich spürte ein leichtes Zittern im ganzen
Körper (es hörte bis zuletzt nicht mehr auf); es war, als hätte man
einen riesigen Motor in mich einmontiert, doch mein Körper war zu
leicht gebaut, so dass sämtliche Wände, Schotten, Kabel, Balken und
Lichter zitterten\ldots{} Ich wusste nicht, ob sie schon da war. Aber
ich hatte keine Zeit, mich zu vergewissern, ich musste zum
Kommandoraum zurück; Zeit zum Start — wohin? Graue, glanzlose
Gesichter. Drunten auf dem Wasser geschwollene blaue Adern. Ein
bleischwerer Himmel, so schwer wie meine Hand, die den Telefonhörer
abnahm: „Start — fünfundvierzig Grad!“

Eine dumpfe Explosion — ein schwacher Stoß — ein zischender
weißlich-grüner Wasserberg im Heck — das Deck schwankte unter
meinen Füßen, und alles stürzte in die Tiefe, das ganze Leben, für
immer\ldots{} Eine Sekunde lang sah ich die eisblauen Umrisse der Stadt,
die runden Blasen der Kuppeln, den einsamen Finger des
Akkumulatorenturms. Dann ein Wolkenvorhang aus grauer Watte — wir
stießen hindurch — und Sonne, blauer Himmel. Sekunden, Minuten,
Meilen — das Blau wurde härter und dunkler, die Sterne glichen
silbernen Schweißtropfen\ldots{}
plötzlich war es Nacht, eine unheimlich beklemmende, glühende
Sternennacht, Sonnennacht. Wir hatten die Erdatmosphäre verlassen.
Aber die Verwandlung war so jäh, dass alle betroffen schwiegen. Mir
allein wurde es leichter ums Herz unter dieser gespenstisch stummen
Sonne, als hätte ich die unvermeidliche Schwelle überschritten, als
wäre mein Körper irgendwo dort unten geblieben, während ich durch
eine neue Welt jagte, wo alles sich in sein Gegenteil verkehrte\ldots{}

„Gleichen Kurs halten!“ rief ich in den Maschinenraum; nein, nicht
ich rief es, sondern das Grammofon in mir, das mit seinem
Scharnierarm dem zweiten Konstrukteur den Hörer reichte. Ich selber
war ganz in ein leises Zittern gehüllt, das nur ich fühlen konnte:
ich lief nach unten, um sie zu suchen.

Die Tür zur Messe — in einer Stunde würde sie sich klirrend
schließen\ldots{} Neben der Tür stand ein Unbekannter mit einem
Dutzendgesicht, das in der Menge nicht auffällt; nur seine Arme
waren ungewöhnlich lang, sie reichten bis zu den Knien, als wären
sie aus Versehen von einer anderen Menschenart genommen. Er
streckte abwehrend die langen Arme aus: „Wohin?“ Offensichtlich
hatte er keine Ahnung, dass ich alles wusste\ldots{}

Ich sagte in ziemlich scharfem Ton:

„Ich bin der Konstrukteur des Integral und überwache alle Versuche.
Verstanden?“ Die Arme sanken herab.

In der Messe. Über Karten und Instrumenten gelbliche Glatzen. Ich
musterte sie mit einem kurzen Blick, machte kehrt und eilte in den
Maschinenraum. Glühende Röhren verbreiteten eine unerträgliche
Hitze, blitzende Hebel drehten sich in trunkenem Tanz, leise bebend
bewegten
sich die Zeiger auf den Uhren, ohne eine Sekunde stillzustehen.

Endlich sah ich jenen Mann mit den dichten, überhängenden Brauen;
er saß mit einem Notizbuch in der Hand am Tachometer.

„Hören Sie, ist sie hier? Wo ist sie?“ Er lachte: „Sie? In der
Funkkabine.“ Ich eilte dorthin.

In der Funkkabine saßen drei Menschen, alle mit Kopfhörern, die
mich an geflügelte Helme erinnerten. Sie schien um einen Kopf
größer als sonst, strahlend, wie eine Walküre aus alter Zeit.

„Irgend jemand\ldots{} nein, würden Sie vielleicht\ldots{}“, sagte ich zu
ihr, vom schnellen Laufen noch ganz außer Atem, „ich muss einen
Funkspruch durchgeben \ldots{} Kommen Sie, ich diktiere ihn.“

Neben dem Armaturenraum befand sich eine kleine Kajüte. Wir setzten
uns an den Tisch. Ich ergriff ihre Hand und drückte sie fest: „Was
wird nun geschehen?“

„Ich weiß es nicht. Herrlich, so zu fliegen, ohne zu wissen,
wohin\ldots{} Bald ist es zwölf Uhr\ldots{} Wo werden wir beide heute Nacht
sein? Vielleicht auf einer Wiese, auf welken Blättern\ldots{} “ Ich
zitterte immer stärker:

„Schreiben Sie“, sagte ich, immer noch ein wenig atemlos (natürlich
vom schnellen Laufen). „Zeit: 11.30. Geschwindigkeit: 6800\ldots{} “
Ohne aufzublicken, sagte sie leise:

„Gestern Abend kam sie mit deinem Zettel\ldots{} Ich weiß alles, du
brauchst mir nichts zu erklären. Es ist doch dein Kind, nicht wahr?
Ich habe sie fortgebracht, sie ist schon hinter der Mauer. Sie wird
leben\ldots{} “

Im Kommandoraum. Das dunkle Gewölbe der Nacht mit unzähligen,
funkelnden Sternen und einer die Augen blendenden Sonne; der Zeiger
der Wanduhr hinkte schwerfällig von einer Minute zur nächsten,
alles war in einen leichten Nebel getaucht, alles bebte. „Gut, dass
der Aufstand nicht hier, sondern irgendwo weiter unten, in der Nähe
der Erde, ausbrechen wird“, schoss es mir durch den Kopf. „Stopp!“
rief ich in den Maschinenraum. Unsere Geschwindigkeit nahm
allmählich ab. Plötzlich hing der Integral einen Augenblick
unbeweglich in der Luft, und dann stürzte er wie ein Stein
hinunter, schneller, immer schneller. Ohne ein Wort zu sagen,
flogen wir zehn Minuten lang — ich hörte meinen Pulsschlag — der
Zeiger der Uhr näherte sich der Zwölf. Ich war ein Stein, I die
Erde, und ich, der Stein, den jemand hochwarf, musste fallen und
die Erde erreichen\ldots{}

Unter mir sah ich schon den dichten blauen Rauch der Wolken\ldots{} Was
aber sollte werden, wenn unser Plan scheiterte?

Das Grammofon in mir nahm den Hörer und kommandierte: „Halbe Kraft
voraus!“ Der Stein stand still. Vier niedrige Klötze, zwei am Heck
und zwei am Bug, wurden hinausgelassen, um den Flug des Integral zu
stoppen, und wir schwebten etwa einen Kilometer über der Erde in
der Luft.

Alle kamen an Deck (es war gleich zwölf). Sie beugten sich über die
gläserne Reling und betrachteten die unbekannte Welt jenseits der
Mauer, die unter uns lag. Bernsteingelb, grün, blau — ein
Herbstwald, eine Wiese, ein See. Am Rande der kleinen blauen Schale
standen gelbe Ruinen, und daneben drohte ein grauer, verdorrter
Finger — das musste der Turm einer alten

Kirche sein, der wie durch ein Wunder unversehrt geblieben war.

„Sehen Sie! Schnell! Dort, rechts!“

Über die grüne Fläche unter uns flog ein brauner Schatten.
Mechanisch hob ich mein Fernglas: eine Herde brauner Pferde
galoppierte mit wehenden Schweifen über die Wiese, und auf ihrem
Rücken saßen braune, weiße und schwarze Menschen.

Hinter mir sagte eine Stimme: „Sie können es mir glauben, ich habe
ein Gesicht gesehen!“ „Erzählen Sie das einem anderen!“ „Dann sehen
Sie doch einmal durch das Glas\ldots{}“ Doch sie waren schon
verschwunden. Endlos dehnte sich die grüne Einöde\ldots{} Ein schrilles
Läuten: Mittagessen, eine Minute vor zwölf.

Ich ging zur Messe. Auf der Treppe lag ein goldenes Abzeichen; es
krachte unter meinem Fuß. Jemand sagte: „Und es war doch ein
Gesicht!“ Ein dunkles Quadrat — die offene Tür zur Messe. Neben mir
zusammengepresste weiße Zähne\ldots{} Unendlich langsam begann die Uhr
zu schlagen, die vordersten Reihen setzten sich in Bewegung.
Plötzlich versperrten zwei überlange Arme den Eingang: „Halt!“

Harte Finger gruben sich in meine Handfläche — es war 1 Sie
flüsterte: „Was soll das? Kennst du ihn?“ „Nein. Ist das\ldots{} ist das
denn nicht\ldots{}“ Der Mann mit dem Dutzendgesicht sagte: „Alles
herhören! Im Namen des Wohltäters! Wir wissen Bescheid. Wir kennen
zwar noch nicht eure Nummern, aber wir wissen alles. Ihr sollt den
Integral nicht haben! Untersteht euch, auch nur eine Bewegung zu
machen. Der Probeflug wird zu Ende geführt. Und dann\ldots{} Das ist
alles, was ich euch zu sagen habe.“

Schweigen. Die gläsernen Platten unter meinen Füßen waren weich wie
Watte, wie meine Beine. I sprühte wilde, blaue Funken. Sie zischte
mir ins Ohr: „Sie waren?s also! Sie haben Ihre
\glq{}Pflicht\grq{} erfüllt!“ Sie riss ihre Hand aus
der meinen und ließ mich stehen. Ich ging allein in die Messe,
schweigend wie die anderen\ldots{} „Aber ich habe es doch gar nicht
getan! Ich habe keinem ein Wort davon gesagt, außer diesen stummen
weißen Seiten\ldots{} “, schrie ich ihr in Gedanken verzweifelt zu. Sie
saß mir gegenüber am Tisch und würdigte mich keines Blickes. Ich
hörte, wie sie zu der gelblichen Glatze neben ihr sagte:

„Edelmut? Aber, lieber Professor, eine philologische Analyse dieses
Wortes zeigt ja schon, dass es sich hier nur um ein Vorurteil
handelt, um ein Überbleibsel aus feudalen Zeiten. Wir aber\ldots{} “

Ich fühlte, wie ich erbleichte; gleich mussten es alle merken\ldots{}
Doch das Grammofon in mir vollführte automatisch die für jeden
Bissen vorgeschriebenen fünfzig Kaubewegungen, und ich zog mich in
mich selbst zurück wie in ein undurchsichtiges Haus, ich wälzte
Steine vor die Tür und verhängte die Fenster \ldots{}

Dann hielt ich den Telefonhörer in der Hand, und der Flug durch die
Wolken in die eisige, sternhelle Sonnennacht begann. Wahrscheinlich
lief die ganze Zeit in meinem Inneren ein logischer Motor auf
vollen Touren, denn nirgendwo im blauen Raum sah ich mit einem Male
dieses Bild: mein Schreibtisch, darauf ein Blatt meiner
Aufzeichnungen und darüber die kiemenähnlichen Backen von U. Sie
allein konnte uns verraten haben. Schnell in die Funkkabine\ldots{} Ich
erinnere mich, dass ich irgend etwas zu ihr sagte und dass sie
durch mich hindurchsah, als wäre ich aus Glas:

„Ich bin beschäftigt. Ich nehme gerade einen Funkspruch von unten
auf. Diktieren Sie meiner Kollegin.“ Ich überlegte eine Minute und
sagte mit fester Stimme: „Zeit: 14.40. Landen! Motoren abstellen.
Alles zu Ende.“ Wieder im Kommandoraum. Das Maschinenherz des
Integral stand still, wir fielen, und mein Herz, das keine Zeit
hatte zu fallen, blieb stehen und stieg plötzlich immer höher, bis
zur Kehle. Wolken, in der Ferne ein grüner Fleck, er jagte wie ein
Sturmwind auf uns zu — gleich ist alles vorüber. —

Das porzellanweiße, verzerrte Gesicht des zweiten Konstrukteurs.
Ich glaube, er versetzte mir mit aller Kraft einen Stoß. Ich schlug
irgendwo mit dem Kopf auf und hörte gerade noch wie durch einen
dichten Nebel: „Heckmotoren — äußerste Kraft voraus!“ Ein jäher
Sprung nach oben — was dann geschah, weiß ich nicht.

\section{EINTRAGUNG NR. 35}

\uebersicht{\emph{Übersicht:} Ein Reif um meinen Kopf. Eine Möhre.
Ein Mord.}
Ich habe die ganze Nacht nicht geschlafen. Ich dachte unablässig an
ein und dasselbe.

Mein Kopf ist seit meinem gestrigen Unfall fest mit Binden
umwickelt. Aber mir ist, als wären es keine Binden, sondern ein
Reif, ein schmerzender Reif aus gläsernem Stahl, und ich bewegte
mich immerzu in dem gleichen magischen Kreis: Töten! Töten! Töten —
und dann zu ihr gehen und sagen: „Glaubst du mir jetzt?“
Widerwärtig, dass das Töten ein schmutziges Handwerk ist.

Dieser Gedanke erzeugt in meinem Mund einen abscheulich süßen
Geschmack, und ich kann den Speichel nicht herunterschlucken, ich
spucke ihn in ein Taschentuch. Mein Mund ist trocken.

In meinem Schrank lag eine schwere Kolbenstange, die beim Gießen
geplatzt war (ich wollte die Bruchstelle unter dem Mikroskop
betrachten). Ich steckte meine Aufzeichnungen in eine Rolle, schob
die Kolbenstange hinein und ging hinunter ins Vestibül. Die Treppe
nahm kein Ende, die Stufen waren schlüpfrig und weich, die ganze
Zeit musste ich mir mit dem Taschentuch den Mund abwischen.

U war nicht da, ihr Platz war leer. Da fiel mir ein, dass heute
jegliche Arbeit ruhte. Alle mussten ja zur Operation. Was sollte
sie also hier?

Ich verließ das Haus. Wind. Graue Eisenklumpen wirbelten über den
Himmel. Die ganze Welt war in spitze Späne gespalten; sie stürzten
herab, hingen eine Sekunde lang vor mir in der Luft und verdampften
spurlos. Auf der Straße ein wildes Gewühl. Die Menschen
marschierten nicht in Reih und Glied wie sonst, sondern rannten
kopflos hin und her. Ich lief, so schnell ich konnte. Plötzlich
blieb ich wie angewurzelt stehen: im zweiten Stock eines Hauses, in
einem Glaskäfig, der in der Luft zu schweben schien, sah ich einen
Mann und eine Frau in enger Umarmung. Ein letzter Kuss, ein
Abschied für immer\ldots{}

An irgendeiner Straßenecke ein schwankender, stachliger Strauch von
vielen Köpfen. Darüber knatterte eine Fahne: Nieder mit der
Maschine! Nieder mit der Operation! Kann denn auch sie ein Schmerz
quälen, von dem man sie nur zu befreien vermag, indem man ihnen das
Herz herausreißt? Und müssen sie alle noch etwas tun, bevor
man sie heilt? durchzuckte es mich. Einen Augenblick lang
existierte nichts mehr auf der ganzen Welt außer meiner behaarten
Hand mit der bleischweren Rolle. Da kam mir ein Schuljunge laut
heulend entgegengerannt. Ich hielt ihn an und fragte ihn nach U.
„Sie ist bestimmt noch in der Schule“, antwortete er, „aber beeilen
Sie sich.“

Schnell zur nächsten U-Bahn-Station. Am Eingang rief mir jemand zu:
„Heute fährt kein Zug! Dort unten\ldots{} “ Und schon hastete er
weiter.

Ich ging die Treppe hinunter. Ein leerer, kalter Zug. Auf dem
Bahnsteig eine dichte Menschenmenge. Schweigen. In der Stille eine
Stimme. Ich konnte sie nicht sehen, aber das war ihre Stimme, ich
kannte sie, ich kannte sie nur allzu gut! Ich schrie: „Lasst mich
durch! Platz! Ich muss\ldots{} “ Jemand packte mich an den Armen und
hielt mich fest. Die Stimme sagte in schneidendem Ton: „Nein, geht
nur hinauf! Dort werdet ihr geheilt, dort werdet ihr mit Glück
gefüttert. Ihr werdet satt und zufrieden sein, ihr werdet friedlich
schlafen und im Takt schnarchen — hört ihr sie, die große
Schnarchsymphonie? Narren, man will euch von allen Fragezeichen
befreien, die an euch nagen wie Würmer. Doch ihr steht hier und
hört mir zu. Schnell hinauf, zur großen Operation! Was kümmert es
euch, dass ich allein hier bleibe? Was geht es euch an, dass ich
das Unmögliche will\ldots{}“ Eine andere Stimme sagte:

„Das Unmögliche! Jag du nur deinen törichten Phantasien nach,
solange du magst, sie zeigen dir doch bloß den Schwanz. Nein, wir
werden ihn packen, diesen Schwanz, und dann\ldots{} “ „Und dann fresst
ihr euch voll und schnarcht und braucht
bald einen neuen Schwanz vor eurer Nase. In alten Zeiten gab es ein
Tier, das unsere Vorfahren Esel nannten. Um ihn zu zwingen, immer
vorwärts zu gehen, immer weiter, banden sie ihm dicht vor den
Nüstern ein Bündel Möhren an die Deichsel, und zwar so, dass er es
nicht erreichen konnte. Aber wenn er es erwischte, fraß er es mit
einemmal auf\ldots{} “

Plötzlich war ich frei; ich stürzte zur Mitte, wo sie sprach — und
im gleichen Augenblick stoben alle auseinander. Von oben ein
Schrei:

„Sie kommen, sie kommen!“ Das Licht erlosch — offenbar hatte jemand
die Leitung durchgeschnitten. Eine Lawine, Schreie, Röcheln, Köpfe,
Finger\ldots{} Ich weiß nicht, wie lange wir durch den stockfinsteren
Tunnel gingen. Endlich Stufen, Licht, Helle — wir standen auf der
Straße. Die Menge zerstreute sich, ich war allein. Wind, graue
Wolken, dicht über meinem Kopf, Dämmerung. Auf dem nassen
Straßenpflaster spiegelten sich Lichter, Mauern, Gestalten. Die
bleischwere Rolle in meiner Hand zog mich fast zu Boden. U saß
immer noch nicht an ihrem Tisch im Vestibül; ihr Zimmer war
dunkel.

Ich ging in mein Zimmer und machte Licht. Der Reif um meine
hämmernden Schläfen zog sich fester zusammen; ich ging vom Tisch,
auf den ich die schwere Rolle gelegt hatte, zum Bett, zur Tür,
zurück zum Tisch, als wäre ich in einen magischen Kreis gebannt. Im
Zimmer links waren die Vorhänge geschlossen. Rechts — eine höckrige
Glatze, die sich über ein Buch beugte: die Stirn — eine riesige
gelbe Parabel, die Runzeln darauf — undeutliche gelbe Zeilen. Jedes
Mal, wenn sich unsere Blicke begegneten, spürte ich, dass diese
Zeilen mir galten\ldots{} Punkt 21 Uhr. U kam zu mir. Ich atmete so
laut, dass ich meinen eigenen Atem hörte; ich versuchte, mich zu
beherrschen — es gelang mir nicht. U setzte sich und zog den Rock
über die Knie. Die rosa Kiemen zitterten:

„Ach, mein Lieber, Sie sind wirklich verletzt? Ich habe es eben
erfahren\ldots{}“

Die Rolle lag vor mir auf dem Tisch. Keuchend sprang ich auf. Sie
hielt mitten im Satz inne und erhob sich gleichfalls. Ich starrte
auf jene Stelle an ihrem Kopf und hatte einen widerlich süßen
Geschmack im Mund\ldots{} Das Tuch\ldots{} Ich fand es nicht und spuckte auf
den Boden. Mein Zimmernachbar durfte es nicht sehen. Das würde
alles noch verschlimmern\ldots{} Ich drückte auf den Knopf, obwohl ich
kein Recht dazu hatte, aber jetzt war alles gleich. Die Vorhänge
schlossen sich.

Sie begriff offenbar, was ich vorhatte, und eilte zur Tür. Ich kam
ihr zuvor; schwer atmend und die Stelle an ihrem Kopf keine Sekunde
aus den Augen lassend, stand ich vor ihr.

„Sie\ldots{} Sie sind wahnsinnig! Unterstehen Sie sich\ldots{}“ Sie wich
zurück und setzte sich, vielmehr, sie fiel auf das Bett. Zitternd,
die Hände im Schoß gefaltet, hockte sie dort. Sie immer noch fest
anblickend und alle Kraft zusammennehmend, streckte ich die Hand
nach dem Tisch aus und ergriff die Rolle.

„Ich bitte Sie! Ein Tag — nur noch ein Tag! Morgen will ich kommen
und alles tun, was Sie wollen\ldots{}“ Was meinte sie damit? Ich holte
aus\ldots{} Ja, ich habe sie getötet. Sie, unbekannte Leser, können mich
einen Mörder nennen. Ich weiß, ich hätte die Rolle auf ihren Kopf
niedersausen lassen, wenn sie nicht geschrieen hätte: „Um des
Wohltäters willen\ldots{} ich bin bereit\ldots{}“

Und mit zitternden Händen riss sie sich die Uniform vom Leib. Ihr
fetter, gelber schlaffer Körper lag auf dem Bett. Sie hatte
gedacht, darum hätte ich die Vorhänge geschlossen! Das war so
grotesk, dass ich in lautes Gelächter ausbrach. Im selben
Augenblick riss eine zu straff gespannte Feder in mir, meine Hand
sank kraftlos herab, und die Rolle glitt über den Boden. Lachen ist
die tödlichste Waffe. Mit Lachen kann man alles töten, sogar den
Mord selbst, erkannte ich plötzlich.

Ich saß am Tisch und schüttelte mich vor Lachen. Es war ein Lachen
der Verzweiflung. Ich weiß nicht, was geschehen wäre, wenn alles
seinen natürlichen Lauf genommen hätte. Doch da rasselte unvermutet
das Telefon. Ich nahm hastig den Hörer ab — vielleicht war es I.
Eine bekannte Stimme sagte: „Einen Augenblick!“ Endloses,
qualvolles Summen. Dann hörte ich schwere Schritte; sie kamen
näher, wurden lauter, dröhnten wie Erz:

„D-503? Hier spricht der Wohltäter. Kommen Sie sofort zu mir!“

U lag immer noch auf dem Bett, die Augen geschlossen, die Kiemen zu
einem breiten Grinsen verzogen. Ich raffte ihre Kleider vom
Fußboden auf, warf sie ihr zu und fuhr sie an:

„Machen Sie, dass Sie hier wegkommen!“ Sie richtete sich auf und
glotzte mich verdutzt an. „Wie?“

„Ziehen Sie sich an! Los, los!“

Sie nahm ihr Kleid und duckte sich. Mit gepresster Stimme sagte
sie:

„Drehen Sie sich um.“

Ich wandte mich ab und drückte die Stirn an die gläserne Mauer. Auf
dem schwarzen, nassen Spiegel zitterten

Lichter, Gestalten, Funken. Nein, das war ja ich, das war in mir\ldots{}
Warum hatte er mich gerufen? Wusste er bereits alles?

U ging fertig angekleidet zur Tür. Zwei Schritte, und ich presste
ihre Hände so fest zusammen, als wollte ich Tropfen um Tropfen
herauspressen, was ich erfahren musste:

„Hören Sie\ldots{} Haben Sie ihren Namen — Sie wissen, wen ich meine —
haben Sie ihren Namen angegeben? Sagen Sie mir die Wahrheit, ich
muss es unbedingt wissen. Alles andere spielt keine Rolle\ldots{} “
„Nein.“

„Nein? Warum denn nicht — da Sie doch Anzeige erstattet haben\ldots{}“

Sie schob die Unterlippe vor, und über ihre Wangen rannen dicke
Tränen. „Weil\ldots{} weil ich fürchtete\ldots{} wenn sie verhaftet würde\ldots{}
dann würden Sie mich vielleicht\ldots{} nicht mehr\ldots{} lieben\ldots{} Ach, ich
kann nicht mehr\ldots{}“

Ja, das war die Wahrheit, die dumme, lächerliche, menschliche
Wahrheit! Ich öffnete die Tür.

\section{EINTRAGUNG NR. 36}

\uebersicht{\emph{Übersicht:} Leere Seiten. Der Gott der Christen.
Meine Mutter.}
Sonderbar, mein Gedächtnis ist wie eine leere weiße Seite: ich weiß
nicht mehr, wie es war, als ich zu Ihm ging und auf Ihn wartete
(ich erinnere mich nur noch, dass ich warten musste). Auf keinen
Laut, auf kein Gesicht, auf keine einzige Geste kann ich mich
besinnen. Als wären
sämtliche Verbindungen zwischen mir und der Umwelt abgeschnitten.

Ich kam erst wieder zu mir, als ich vor Ihm stand; ich hatte
entsetzliche Angst, aufzublicken. Ich sah nur die riesigen Hände
auf seinen Knien. Diese Hände schienen Ihn zu erdrücken, die Knie
gaben nach unter ihrem Gewicht. Er bewegte langsam die Finger. Sein
Antlitz war irgendwo hoch oben in einem Nebel, und wohl nur
deshalb, weil seine Stimme aus einer solchen Höhe zu mir drang,
grollte sie nicht wie Donner und betäubte mich nicht, sondern klang
wie eine gewöhnliche, menschliche Stimme:

„Also auch Sie? Sie, der Konstrukteur des Integral? Sie, der
berufen war, ein großer Konquistador zu werden, Sie, dessen Name
einen neuen glanzvollen Abschnitt in der Geschichte des Einzigen
Staates einleiten sollte\ldots{} Sie?“ Mir schoss das Blut in die Wangen
— und wieder eine leere Seite. Ich spürte nur meinen Pulsschlag in
den Schläfen und hörte die donnernde Stimme aus der Höhe, nahm aber
keines der Worte auf. Als er verstummt war, kam ich wieder zum
Bewusstsein und sah, dass die Hand vor mir sich bewegte.
Zentnerschwer kroch sie langsam näher, und ein Finger deutete auf
mich: „Nun? Warum schweigen Sie? Bin ich ein Henker? Ja oder
nein?“

„Ja“, antwortete ich demütig. Ich verstand alles, was er sagte:

„Denken Sie etwa, ich fürchte dieses Wort? Haben Sie schon einmal
versucht, die Hülle zu zerreißen und nachzusehen, was sich dahinter
verbirgt? Ich werde es Ihnen zeigen. Ein blauer Hügel, ein Kreuz,
und davor eine Menschenmenge. Erinnern Sie sich? Die einen,
blutbespritzt, schlagen einen Körper ans Kreuz, die anderen,
tränenüberströmt, sehen zu. Glauben Sie nicht auch, dass die Rolle
der ersten oben die schwierigste und wichtigste ist? Denn wie hätte
sich ohne sie die erhabene Tragödie vollenden können? Sie wurden
von der dunklen Menge ausgepfiffen, und darum musste der Autor der
Tragödie, Gott, sie um so reichlicher belohnen. Und der barmherzige
Christengott, der alle Abtrünnigen im Höllenfeuer schmoren lässt,
ist er vielleicht kein Henker? Und ist die Zahl derer, die von den
Christen auf dem Scheiterhaufen verbrannt wurden, kleiner als die
Zahl der Christen, die in der Hölle schmoren? Dennoch — hören Sie —
dennoch hat man diesen Gott, die Liebe dieses Gottes,
jahrhundertelang gepriesen. Absurd? Nein, im Gegenteil — es ist das
mit Blut geschriebene Patent auf die unausrottbare Vernunft des
Menschen. Selbst damals, als er noch wild und ungesittet war,
begriff der Mensch: Die wahre Liebe zur Menschheit ist
unmenschlich, und das Kennzeichen der Wahrheit ist ihre
Grausamkeit! So, wie es das Kennzeichen des Feuers ist, dass es
brennt. Können Sie mir ein Feuer nennen, das nicht brennt? Nun,
nennen Sie es mir doch, widersprechen Sie mir doch!“ Wie sollte ich
Ihm widersprechen — das, was Er sagte, waren ja meine eigenen
Gedanken, nur hatte ich nie verstanden, sie in einen so festen,
glänzenden Panzer zu kleiden. Ich schwieg\ldots{}

„Wenn Ihr Schweigen bedeutet, dass Sie mir recht geben, dann lassen
Sie uns miteinander reden wie erwachsene Menschen, wenn die Kinder
im Bett sind. Ich frage Sie: Worum haben die Menschen von
Kindesbeinen an gebetet, wovon haben sie geträumt, womit haben sie
sich gequält? Dass irgendeiner ihnen ein für allemal sage, was das
Glück ist und sie mit einer Kette an dieses Glück schmiede. Und ist
dies nicht gerade das, was wir
tun? Der uralte Traum vom Paradies\ldots{} Kennen Sie ihn? Im Paradies
haben die Menschen keine Wünsche mehr, sie kennen kein Mitleid,
keine Liebe, dort gibt es nur Selige, denen man die Phantasie
herausoperiert hat (sonst wären sie nicht glücklich), Engel,
Knechte Gottes\ldots{} Und in dem Augenblick, da wir diesen Traum
verwirklichen konnten“ — er ballte die Faust, als wollte er Saft
aus einem Stein pressen —, „als wir die Beute nur noch auszuweiden
und zu verteilen brauchten, da kamen Sie\ldots{} “ Das eherne Dröhnen
verstummte plötzlich. Ich war wie eine glühende Eisenstange unter
den Schlägen des Hammers\ldots{} Plötzlich fragte Er: „Wie alt sind
Sie?“ „Zweiunddreißig.“

„Aha, zweiunddreißig, und doppelt so naiv wie ein Junge von
sechzehn! Ist es Ihnen denn noch nie in den Sinn gekommen, dass
diese Leute — ihre Namen kennen wir noch nicht, aber ich bin
sicher, dass wir sie von Ihnen erfahren —, dass diese Leute Sie nur
brauchten, weil sie der Konstrukteur des Integral sind, nur weil
man durch Sie\ldots{}“

„Nein, nein!“ schrie ich\ldots{}. Es war genauso, als wollte man sich
mit bloßen Händen gegen eine Kugel schützen und ihr diese Worte
zurufen; man hört noch sein lächerliches „Nein, nein!“, aber die
Kugel hat einen schon getroffen, und man wälzt sich auf der Erde.

Ja, den Konstrukteur des Integral, den brauchten sie\ldots{} Ich sah das
wütende Gesicht von U vor mir, als sie an jenem Morgen mit
ziegelroten, bebenden Kiemen in mein Zimmer zusammen mit I\ldots{} Ich
lachte schallend und blickte auf. Vor mir saß ein Mensch mit einer
Glatze wie Sokrates, und auf der Glatze standen kleine
Schweißtropfen.

Wie einfach alles war, wie banal und lächerlich einfach! Vor Lachen
fast berstend, hielt ich die Hand vor den Mund und rannte hinaus.

Stufen, Wind, feuchte, wirbelnde Splitter von Lichtern und
Gesichtern. „Ich muss sie sehen, noch ein einziges Mal!“ dachte
ich. Nun folgt wieder eine leere Seite. Ich kann mich nur noch an
eines entsinnen: Füße, keine Menschen, sondern Füße. Tausende von
Füßen auf dem Pflaster, ein schwerer Regen von Füßen. Ein keckes,
wildes Lied und ein Ruf, der wahrscheinlich mir galt: „He, he!
Hierher, zu uns!“

Ein öder Platz, in der Mitte eine dunkle, drohende Masse: die
Maschine des Wohltäters. Sie rief mir ein schreckliches Bild ins
Gedächtnis: ein blendendweißes Kissen, ein nach hinten geneigter
Kopf darauf mit geschlossenen Augen, ein Streifen weißer Zähne\ldots{}
All das war auf unheimliche Weise mit der Maschine verbunden — ich
wusste, wie, aber ich wollte es nicht sehen, ich wollte es nicht
laut aussprechen, ich konnte es nicht. Ich schloss die Augen und
setzte mich auf die Stufen, die zu der Maschine hinaufführten. Ich
glaube, es regnete, denn mein Gesicht war mit einemmal ganz nass.
In der Ferne dumpfe Schreie. Aber niemand, niemand hörte, wie ich
schrie: „Rettet mich, rettet mich!“ Wenn ich eine Mutter hätte, wie
unsere Vorfahren, eine Mutter\ldots{} Für sie wäre ich nicht der
Konstrukteur des Integral, nicht die Nummer D-503, nicht ein
Molekül des Einzigen Staates, sondern nur ein Mensch, ein Teil von
ihr selbst — zertreten, erdrückt, verstoßen\ldots{} Sie würde mich hören
und mich trösten\ldots{}

\section{EINTRAGUNG NR. 37}

\uebersicht{\emph{Übersicht:} Infusorien. Weltuntergang. Ihr
Zimmer.}
Morgens beim Frühstück. Mein linker Nachbar flüsterte mir
erschrocken zu: „Essen Sie doch, man beobachtet Sie!“

Ich lächelte mit großer Anstrengung, und dabei hatte ich ein
Gefühl, als wäre mein Gesicht in zwei Hälften gespalten. Die Spalte
öffnete sich immer weiter: es war ein unerträglicher Schmerz.

Ich versuchte zu essen. Doch kaum führte ich einen Bissen zum Mund,
da zitterte die Gabel in meiner Hand und fiel klirrend auf den
Teller. Eine ungeheure Detonation erschütterte das ganze Haus, die
Tische, die Wände, die Teller, die Luft bebten und klirrten.
Bleiche, verzerrte Gesichter, offene Münder, in der Luft erstarrte
Gabeln. Dann sprang alles aus den jahrhundertealten Geleisen; alle
fuhren von ihren Plätzen auf (ohne die Hymne zu Ende zu singen!),
kauend, einander drängend und stoßend: „Was war das? Was ist
geschehen?“ Und wie Splitter einer jäh zerstörten Maschine, die
eben noch vorzüglich funktioniert hatte, flogen alle in wildem
Durcheinander zum Lift und zu den Treppen. Auf den Stufen eilige
Schritte, Stampfen, abgerissene Worte. In allen Nachbarhäusern das
gleiche. Eine Minute später glich der Prospekt einem Wassertropfen
unter dem Mikroskop: unzählige Infusorien schossen bald hierhin,
bald

„Aha!“ rief eine triumphierende Stimme. Vor mir ein Nacken und ein
zum Himmel erhobener Finger — ich sehe den gelblich-rosa
Fingernagel mit dem weißen Halbmond noch deutlich vor mir. Dieser
Finger war wie ein

Kompass — aller Augen blickten zum Himmel. Dort oben jagten Wolken,
sprangen eine über die andere, daneben die Flugzeuge der Beschützer
mit ihren langen, nach unten gerichteten Fernrohren, und im Westen
so etwas wie\ldots{}

Zuerst begriff keiner, was es war, selbst ich nicht, der (zum
Unglück) mehr wusste als die anderen. Es sah aus wie ein ungeheurer
Schwarm schwarzer Flugzeuge. Sie kamen rasch näher; Vögel schwebten
mit heiserem Geschrei über unseren Köpfen. Der Sturm packte sie und
stieß sie hinab, und sie ließen sich auf Kuppeln, Dächern und
Baikonen nieder.

„Aha!“ Der Mann vor mir wandte sich um — und ich erkannte jenen
Menschen mit den buschigen Brauen. Doch er sah völlig verändert
aus, er war unter seiner gewölbten Stirn hervorgekrochen, und um
seine Augen und Lippen schimmerten helle Strahlen: er lächelte.
„Die Mauer ist niedergerissen! Die Mauer ist niedergerissen!“
schrie er mir durch das Pfeifen des Windes und der Vogelschwingen
zu.

Am Ende des Prospekts fliehende Gestalten, die mit vorgestreckten
Köpfen in die Häuser rannten. In der Mitte der Straße die schwere
Lawine der Operierten; sie wälzte sich nach Westen, zur Grünen
Mauer. Ich fasste den Mann am Arm:

„Sagen Sie, wo ist sie, wo ist I? Hinter der Mauer, oder hier? Ich
muss sie sehen! Verstehen Sie! Sofort!“ „Sie ist hier in der Stadt,
sie arbeitet!“ rief er mir strahlend zu. „Ja, wir arbeiten, und
wie!“ Um ihn scharrten sich etwa fünfzig solcher Leute wie er — sie
waren unter ihren finsteren Stirnen hervorgekrochen, ihre weißen
Zähne blitzten. Gierig den Wind einatmend und mit elektrisch
geladenen Knuten winkend

(wo hatten sie sie nur her?), marschierten sie hinter den
Operierten nach Westen, aber auf einem Umweg\ldots{} Ich eilte zu ihrem
Haus. Wozu? Ich wusste es nicht. Leere Straßen, eine fremde, wilde
Stadt, unaufhörliches, triumphierendes Vogelgekrächze,
Weltuntergang. In einigen Häusern umarmten sich männliche und
weibliche Nummern, ohne die Vorhänge zu schließen, ohne rosa
Billett, am helllichten Tage\ldots{}

Ein Haus — ihr Haus. Die Tür stand offen. Am Kontrolltisch im
Vestibül kein Mensch. Der Lift hing in der Mitte des Schachtes.
Keuchend lief ich die endlosen Stufen hinauf. Ein Korridor. Nummer
320, 326, 330\ldots{} I-330. In ihrem Zimmer herrschte eine wüste
Unordnung. Der Stuhl lag auf dem Boden und streckte seine Beine wie
ein verendetes Tier in die Luft. Das Bett war von der Wand
weggerückt und stand schief im Raum. Der Fußboden war mit
zerknitterten rosa Billetts übersät. Ich bückte mich und hob eine
Handvoll auf. Auf allen stand mein Name, D-503\ldots{} Nein, sie durften
nicht auf dem Boden liegen bleiben, niemand sollte darauf treten.
Ich raffte sie zusammen, legte sie auf den Tisch, glättete sie
sorgfältig, betrachtete sie und — lachte laut. Jetzt weiß ich
etwas, das ich früher nicht wusste: Das Lachen kann verschiedene
Gründe haben. Es ist nichts anderes als ein fernes Echo einer
inneren Explosion: vielleicht sind rote, blaue und goldene Raketen
mit lustigem Geknatter zerplatzt, vielleicht sind die Fetzen eines
menschlichen Körpers in die Luft geflogen\ldots{} Auf einem Billett las
ich einen mir gänzlich unbekannten Namen. An die Zahl erinnere ich
mich nicht mehr, nur noch an den Buchstaben: F. Ich fegte die rosa
Fetzen vom Tisch, trampelte wütend auf ihnen herum und ging
hinaus.

Im Korridor setzte ich mich auf die Fensterbank und wartete lange.
Links von mir näherten sich schlurfende Schritte. Ein alter Mann;
sein Gesicht war eine aufgestochene, leere, runzlige Blase, und aus
dem Einstichloch tropfte etwas über seine Wangen. Langsam und
dunkel begriff ich: Tränen. Erst als der Alte schon weit weg war,
kam ich wieder zu mir und rief ihm nach: „Hören Sie mal, kennen Sie
I-330?“

Er drehte sich um, winkte verzweifelt ab und hinkte weiter\ldots{} Gegen
Abend kehrte ich nach Hause zurück. Im Westen zuckte der Himmel in
blassblauem Krampf, und auf jedes Zucken folgte ein dumpfes
Grollen. Die Dächer waren von schwarzem, erloschenem Feuerbrand
bedeckt: Vögel.

Ich legte mich zu Bett — und sogleich fiel mich der Schlaf an wie
ein wildes Tier\ldots{}

\section{EINTRAGUNG NR. 38}

\uebersicht{\emph{Übersicht:} (Ich weiß nicht, welche. Vielleicht
ist das die ganze \emph{Übersicht:} Die weggeworfene Zigarette.)}
Ich wachte auf — grelles Licht im Zimmer. Ich kniff die Augen
zusammen; in meinem Kopf beißender blauer Rauch, alles wie Nebel.
Durch den Nebel drang der Gedanke: „Ich habe ja kein Licht gemacht,
wieso\ldots{} “ Ich fuhr auf. Am Tisch saß I, das Kinn in die Hand
gestützt, und sah mich spöttisch an\ldots{} An diesem Tisch sitze ich
jetzt und schreibe. Die zehn oder fünfzehn Minuten, die sie hier
war, sind längst vorbei, doch mir ist, als hätte sich erst eben die
Tür hinter ihr geschlossen, als könnte ich sie noch einholen, ihre

Hand nehmen und\ldots{} vielleicht würde sie lachen und sagen\ldots{}

I saß am Tisch. Ich sprang aus dem Bett: „Du, du! Ich war\ldots{} ich
habe dein Zimmer gesehen\ldots{} ich dachte, du seiest\ldots{}“

Auf halbem Wege stieß ich gegen ihre spitzen, unbeweglichen Wimpern
und blieb stehen. Mir fiel ein, dass sie mich im Integral genauso
angesehen hatte. Deshalb musste ich ihr sofort alles erzählen, so,
dass sie es glaubte. „Höre, I, ich will dir alles sagen. Ich will
nur vorher einen Schluck Wasser trinken.“

Mein Mund war so trocken, als wäre er mit Löschpapier ausgelegt.
Ich goss mir Wasser ein und konnte nicht trinken. Ich stellte das
Glas auf den Tisch und umklammerte die Karaffe mit beiden Händen.

Jetzt sah ich, dass der blaue Rauch von einer Zigarette kam. Sie
tat einen tiefen Zug und sagte: „Lass doch. Schweig. Es ist alles
gleich. Du siehst, ich bin trotzdem gekommen. Drunten warten sie
auf mich, wir haben nur zehn Minuten \ldots{} “

Sie warf die Zigarette auf den Boden und beugte sich über die
Sessellehne nach hinten (dort an der Wand ist der Knopf, man kann
ihn schwer erreichen), so dass der Sessel kippte und nur auf zwei
Beinen stand. Dann schlossen sich die Vorhänge.

Sie trat auf mich zu und presste mich fest an sich. Die Berührung
ihrer Knie war ein süßes Gift, über dem ich alles vergaß\ldots{} Und
plötzlich\ldots{} Sie haben das gewiss schon einmal erlebt: Man liegt in
tiefem Schlaf, und plötzlich zuckt man zusammen, richtet sich auf
und ist wieder hellwach. So war es jetzt mit mir; ich dachte an den
Buchstaben F und an irgendeine Zahl\ldots{} Das alles ballte sich zum
Klumpen in mir zusammen.

Nicht einmal jetzt kann ich sagen, was für ein Gefühl ich dabei
empfand, jedenfalls drückte ich sie so fest an mich, dass sie vor
Schmerz aufschrie.

Eine Minute später lag ihr Kopf mit geschlossenen Augen auf dem
blendend weißen Kissen. Dies erinnerte mich die ganze Zeit an
etwas, an das ich um keinen Preis denken durfte. Ich presste sie
immer zärtlicher, immer leidenschaftlicher an mich und immer
stärker zeichneten sich die blauen Flecken unter meinen Fingern
ab\ldots{} Ohne die Augen zu öffnen, sagte sie:

„Ich habe gehört, du seiest beim Wohltäter gewesen. Stimmt das?“
„Ja, es stimmt.“

Da schlug sie die Augen auf, und ich beobachtete entzückt, wie ihr
Gesicht erblasste, erlosch, verschwand; nur die Augen blieben. Ich
erzählte ihr alles. Nur eines verschwieg ich ihr — ich weiß nicht,
warum, nein, das ist nicht wahr, ich weiß es — ich verschwieg ihr,
was Er zuallerletzt gesagt hatte, dass sie mich nur brauchten, weil
ich der Konstrukteur des Integral bin. Langsam, wie eine
fotografische Platte im Entwickler, nahm ihr Gesicht wieder Gestalt
an — die Wangen, der weiße Streifen der Zähne, die Lippen. Sie
stand auf und ging zum Spiegelschrank. Mein Mund war trocken. Ich
goss mir Wasser ein, mochte aber nicht trinken. Ich stellte das
Glas auf den Tisch und fragte:

„Bist du nur gekommen, weil du alles erfahren wolltest?“ Im Spiegel
sah ich ihre spöttisch hochgezogenen Brauen. Sie wandte sich um,
als wollte sie etwas sagen, doch sie sagte kein Wort. Ich wusste
auch so alles. Sollte ich von ihr Abschied nehmen? Ich machte eine
Bewegung; meine Beine, die nicht mir gehörten, wankten. Ich stieß
gegen den Stuhl — er fiel um und blieb wie tot liegen. Ihre Lippen
waren kalt — so kalt war einmal der Boden meines Zimmers gewesen,
hier neben dem Bett.

Als sie gegangen war, hockte ich auf dem Boden und beugte mich über
die Zigarette, die sie weggeworfen
hatte.

Ich kann nicht mehr schreiben, ich will nicht mehr schreiben!

\section{EINTRAGUNG NR. 39}

\uebersicht{\emph{Übersicht:} Das Ende.}
Es war wie das letzte Salzkörnchen, das man in eine gesättigte
Lösung wirft: die Kristalle schließen sich zu Nadeln zusammen,
werden fest und erkalten. Ja, alles war entschieden, morgen früh
würde ich es tun. Es kam zwar einem Selbstmord gleich, aber
vielleicht würde ich danach auferstehen. Weil ja nur auferstehen
kann, wer tot ist.

Im Westen zuckte der Himmel unaufhörlich in blauem Krampf. Mein
Kopf glühte und pochte. So saß ich die ganze Nacht und schlief erst
gegen sieben Uhr ein, als das Dunkel sich grün verfärbte und man
die mit schwarzen Vögeln übersäten Dächer schon erkennen konnte.
Ich wachte um zehn Uhr auf (heute hatte es offenbar nicht
geläutet). Auf dem Tisch stand noch das Glas Wasser von gestern.
Ich leerte es in einem Zug und eilte fort: ich musste alles so
schnell wie möglich erledigen. Der Himmel war blau, leer, bis auf
den Grund vom Gewitter ausgelaugt. Man fürchtete sich, die scharfen
Kanten der Schatten anzufassen, die aus der blauen Herbstluft
herausgeschnitten schienen, denn sie mussten
bei der leisesten Berührung zerbrechen und in gläsernen Staub
zerfallen. In mir die gleichen zerbrechlichen Schatten. Nein, ich
durfte nicht denken, ich durfte nicht denken, sonst\ldots{}

Ich dachte nichts, vielleicht sah ich nicht einmal richtig, sondern
registrierte nur. Auf dem Pflaster lagen Zweige mit grünen, roten
und braunen Blättern. Am Himmel schossen Vögel und Flugzeuge hin
und her. Da — Köpfe, weit geöffnete Münder, Hände, die mit Zweigen
winkten. Ich glaube, das alles brüllte, krächzte und summte\ldots{} Dann
öde, wie von der Pest leergefegte Straßen. Ich erinnerte mich, dass
ich über etwas unangenehm Weiches und dennoch Starres, Festes
stolperte. Ich bückte mich: eine Leiche. Der Tote lag auf dem
Rücken, die Beine waren gespreizt. Sein Gesicht — ich erkannte ihn
an seinen dicken, wulstigen Lippen. Mit zwinkernden Augen lachte er
mich an. Ich schritt über ihn hinweg und hastete weiter, ich konnte
nicht mehr, ich musste alles so schnell wie möglich hinter mich
bringen, sonst würde ich zugrunde gehen. Zum Glück hatte ich nur
noch zwanzig Meter zu gehen — da leuchtete schon das Schild mit der
goldenen Aufschrift: Schutzamt. Bevor ich eintrat, blieb ich eine
Weile auf der Schwelle stehen und sog so viel Luft in mich ein, wie
ich konnte.

Im Korridor eine endlose Schlange von Nummern mit Blättern und
dicken Heften unter dem Arm. Sie bewegten sich einen Schritt
vorwärts und blieben dann wieder stehen.

Ich eilte an der Schlange vorüber. Köpfe wandten sich mir zornig
zu. Ich sank in die Knie und flehte sie wie ein Todkranker um ein
Mittel an, das allem mit einem sekundenlangen, furchtbaren Schmerz
ein Ende macht. Aus einer Tür kam eine Frau, die den Gürtel eng um
die

Uniform geschnallt hatte; die beiden Halbkugeln ihres Gesäßes
traten deutlich hervor, und sie wand sich hin und her, als ob dort
ihre Augen säßen. Als sie mich sah, rief sie: „Er hat Bauchweh!
Führt ihn zur Toilette, dort, zweite Tür rechts!“

Alle lachten. Dieses Gelächter würgte mich in der Kehle, ich musste
schreien oder\ldots{} Da fasste mich jemand von hinten am Ellbogen. Ich
drehte mich um — durchsichtige Flügelohren. Aber sie waren nicht
rosa, wie sonst, sondern dunkelrot; der Adamsapfel hüpfte, gleich
musste die dünne Haut zerreißen.

„Warum sind Sie hier?“ fragte er mich mit durchbohrendem Blick. Ich
klammerte mich verzweifelt an ihn. „Schnell in Ihr Zimmer! Ich muss
Ihnen sofort alles erzählen. Gut, dass ich Sie getroffen habe\ldots{}
Vielleicht ist es entsetzlich, dass gerade Sie es sind\ldots{} Nein, es
ist doch gut so.“

Er kannte sie ebenfalls, und das machte alles noch qualvoller für
mich, aber vielleicht würde es ihn schaudern, wenn er meinen
Bericht hörte, und dann würden wir zu zweien töten, dann wäre ich
in meiner letzten Sekunde nicht allein\ldots{}

Die Tür fiel zu; eine eigenartige, luftleere Stille trat ein, wie
unter einer Glasglocke. Hätte er nur ein Wort gesagt, und wäre es
das unsinnigste gewesen, dann hätte ich ihm sofort alles erzählt.
Doch er schwieg. Ohne aufzublicken, begann ich endlich: „Ich
glaube, ich habe sie stets gehasst, von Anfang an. Ich habe
erbittert mit mir selbst gerungen\ldots{} Nein, das ist nicht wahr, ich
konnte und wollte nicht gerettet werden, ich wollte zugrunde gehen,
das war mir mehr wert als alles andere\ldots{} das heißt, ich wollte nur
sie\ldots{} So empfinde ich auch jetzt noch, da ich alles weiß\ldots{} Haben
Sie
gehört, dass der Wohltäter mich zu sich gerufen hat?“ „Ja.“

„Aber, was der Wohltäter zu mir sagte, das wissen Sie wohl nicht\ldots{}
Es war, als würde einem der Boden unter den Füßen weggezogen\ldots{} So,
als ob Sie plötzlich mit diesem Schreibtisch, mit dem Papier und
der Tinte darauf versänken\ldots{} Die Tinte spritzt, und alles ist ein
einziger Klecks\ldots{}“

„Weiter, weiter! Beeilen Sie sich, draußen warten noch andere.“

Stockend und verwirrt schilderte ich ihm alles, was gewesen ist und
was ich auf diesen Seiten festgehalten habe. Von meinem wahren und
von dem anderen Ich, und das, was sie auf dem Spaziergang von
meinen Händen gesagt hatte — ja, damit hatte es angefangen —, und
wie ich meine Pflicht versäumte, wie ich mich selbst betrog, wie
sie mir Atteste besorgte, wie ich mich immer mehr verstrickte, wie
ich in die unterirdischen Gänge und in das Land hinter der Grünen
Mauer kam. Die spöttischen, S-förmig geschwungenen Lippen schoben
mir lächelnd die Stichworte zu — ich nickte dankbar. Aber — was war
das? Plötzlich sprach er statt meiner, und ich hörte zu! Es
überlief mich eiskalt. Ich fragte: „Wieso wissen Sie das? Sie
können es doch von niemandem erfahren haben\ldots{} “

Er lächelte noch spöttischer. Nach einer Weile sagte er: „Sie
wollten mir etwas verheimlichen. Sie haben mir alle genannt, die
Sie hinter der Mauer entdeckt haben, aber Sie haben einen
vergessen. Erinnern Sie sich nicht mehr, dass Sie mich dort gesehen
haben? Ja, mich!“ Pause.

Plötzlich durchfuhr mich ein schamloser Gedanke: Er gehört auch zu
ihnen! Alle Pein, die ich erlitten, alles, was
ich mit letzter Kraft tapfer hierher geschleppt hatte, war nur noch
lächerlich wie die alte Geschichte von Abraham und Isaak. Abraham,
in kalten Schweiß gebadet, hatte schon das Messer gegen seinen
Sohn, gegen sich selbst gezückt — da sprach eine Stimme in der
Höhe: „Lass! Ich habe nur gescherzt.“

Ohne den Blick von diesem spöttischen Lächeln zu wenden, stemmte
ich mich mit beiden Beinen gegen die Tischkante und schob mich
langsam mit meinem Sessel zurück. Dann sprang ich mit einem Satz
auf und stürzte an schreienden Menschen vorbei zum Ausgang. Ich
weiß nicht, wie ich in den Waschraum der U-Bahn-Station kam. Oben
war alles zerstört, die höchste und vernünftigste Zivilisation der
Geschichte vernichtet, doch hier unten war durch irgendeine Ironie
des Schicksals alles noch so schön wie früher. Aber auch dies würde
zerfallen, würden von dichtem Gras überwuchert werden, und über uns
würden die Mephi herrschen. Ein entsetzlicher Gedanke! Ich stöhnte
laut. Da streichelte mich jemand zärtlich am Arm. Es war mein
Zimmernachbar, der auf dem Sitz links neben mir saß. Seine Stirn —
eine riesige gelbe Parabel, mit wirren Zeilen darauf, die mir
galten. „Ich verstehe Sie, ich verstehe Sie vollkommen“, sagte er.
„Aber beruhigen Sie sich, jede Erregung ist überflüssig. Es wird
alles wiederkehren. Nur müssen zuvor alle von meiner Entdeckung
erfahren. Sie sind der erste, dem ich davon berichte. Ich habe
festgestellt, dass es keine Unendlichkeit gibt!“ Ich sah ihn
verstört an.

„Ja, es gibt keine Unendlichkeit! Wenn die Welt unendlich wäre,
dann müsste die mittlere Dichte ihrer Materie Null sein. Da sie
jedoch nicht Null ist — wie wir wissen —, muss das Weltall endlich
sein, es hat sphärische

Form und das Quadrat des Weltradius, y2 = mittlere Dichte,
multipliziert mit\ldots{} Jetzt muss ich nur noch den Koeffizienten
berechnen, und dann\ldots{} nun, dann ist alles ganz einfach. Dann
werden wir philosophisch siegen, verstehen Sie? Aber, Verehrtester,
Sie stören mich bei meinen Berechnungen, Sie schreien\ldots{} “

Ich weiß nicht, was mich mehr erschütterte, seine Entdeckung oder
seine Ruhe in dieser apokalyptischen Stunde. Er hielt sein
Notizbuch und eine Logarithmentafel in der Hand (das bemerkte ich
erst jetzt). Da dachte ich: Bevor alles vernichtet wird, muss ich
meine Aufzeichnungen abschließen, das bin ich meinen Lesern
schuldig. Ich bat meinen Nachbar um Papier und schrieb diese Zeilen
nieder. Ich wollte schon einen Punkt machen, so, wie unsere
Vorfahren über den Gruben, in die sie ihre Toten warfen, ein Kreuz
errichteten, aber plötzlich zitterte der Bleistift in meiner Hand
und rollte auf den Boden. „Hören Sie“, ich packte meinen Nachbarn
am Arm, „beantworten Sie mir diese Frage. Sie müssen sie mir
beantworten. Was ist dort, wo Ihr endliches Weltall aufhört, was
ist dort?“

Er hatte keine Zeit mehr, zu antworten; stampfende Schritte kamen
die Treppe herunter\ldots{}

\section{EINTRAGUNG NR. 40}

\uebersicht{\emph{Übersicht:} Fakten. Die Glocke. Ich bin
überzeugt.}
Tag. Hell. Barometerstand 760.

Habe ich, D-503, tatsächlich all diese Seiten geschrieben? Habe ich
das wirklich jemals empfunden, was ich hier aufgezeichnet habe,
habe ich es mir nur eingebildet?

Ja, es ist meine Handschrift. Auch auf dieser Seite ist es meine
Handschrift\ldots{} Aber hier ist nicht mehr von Phantasien und Gefühlen
die Rede, sondern nur noch von Fakten. Ich bin nämlich wieder
gesund, völlig gesund. Unwillkürlich muss ich lächeln, ich kann
nicht anders: man hat mir einen Splitter aus dem Kopf gezogen, und
ich spüre eine große Leere und Erleichterung. Nein, keine Leere, es
ist nur nichts mehr da, was mich am Lächeln hindert (das Lächeln
ist der Normalzustand eines normalen Menschen).

Nun zu den Fakten: Gestern Abend wurden mein Nachbar, der die
Endlichkeit des Weltalls entdeckt hatte, ich und sämtliche Nummern,
die keine Bescheinigung über die Operation besaßen, verhaftet und
ins nächste Auditorium geführt. Man band uns fest und operierte
uns. Heute morgen ging ich, D-503, zum Wohltäter und berichtete ihm
alles, was ich von den Feinden des Glückes wusste. Ich kann jetzt
einfach nicht begreifen, weshalb mir das früher so schwierig
vorkam. Es gibt nur eine Erklärung dafür: meine Krankheit, die
Seele. Am Abend saß ich an einem Tisch mit dem Wohltäter in der
Gaskammer. Die Nummer I-330 wurde hereingeführt. Sie sollte in
meiner Gegenwart ein umfassendes Geständnis ablegen. Sie schwieg
hartnäckig und lächelte nur. Ich bemerkte, dass sie scharfe, sehr
schöne weiße Zähne hatte. Man setzte sie unter die Gasglocke. Ihr
Gesicht wurde bleich, aber ihre großen dunklen Augen leuchteten
dadurch nur noch mehr. Als die Luft aus der Glocke gepumpt wurde,
warf sie den Kopf zurück, senkte die Lider und presste die Lippen
zusammen — das erinnerte mich an irgend etwas. Sie hielt sich
krampfhaft an den Sessellehnen fest und blickte mich an, bis ihr
die Augen zufielen. Dann wurde sie aus der Gasglocke herausgeholt,
durch einen Stromstoß wieder zu sich gebracht und von neuem unter
die Glocke gesetzt. Dreimal wiederholte sich dies, doch sie sagte
kein Wort. Die anderen, die zusammen mit ihr hereingeführt wurden,
benahmen sich weniger widerspenstig: die meisten sprachen schon
beim ersten Male. Morgen werden sie alle die Stufen zur Maschine
des Wohltäters hinaufsteigen.

Wir müssen handeln, die Sache duldet keinen Aufschub, denn in den
westlichen Vierteln gibt es immer noch Chaos, Gebrüll, Leichen,
Tiere und leider auch eine bedeutende Zahl von Nummern, die die
Vernunft verraten haben. Aber es ist uns gelungen, auf dem 40.
Prospekt eine provisorische Mauer aus Starkstrom zu errichten. Ich
hoffe, dass wir siegen. Ich bin sogar fest von unserem Sieg
überzeugt. Die Vernunft muss siegen!

\end{document}
