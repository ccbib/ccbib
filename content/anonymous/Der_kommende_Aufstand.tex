\usepackage[ngerman,french]{babel}
\usepackage[T1]{fontenc}

\newcommand\extrapar\medskip
\newcommand{\satz}[1]{#1\extrapar\noindent}

\setcounter{tocdepth}{1}

\begin{document}
\raggedbottom
\selectlanguage{ngerman}

\title{Der kommende Aufstand}
\author{Unsichtbares Komitee}
\date{}
\maketitle

\textit{Ein Aufstand, wir können uns nicht mal mehr vorstellen, wo er 
beginnt. Sechzig Jahre der Befriedung, ausgesetzter historischer 
Umwälzungen, sechzig Jahre demokratischer Anästhesie und Verwaltung 
der Ereignisse haben in uns eine gewisse abrupte Wahrnehmung des Realen 
geschwächt, den parteilichen Sinn für den laufenden Krieg. Es ist diese 
Wahrnehmung, die wir wiedererlangen müssen, um zu beginnen.}

~

Warum eine deutschsprachige Ausgabe?

Bevor wir uns daran machten „L‘insurrection qui vient“ zu
übersetzen, waren wir eigentlich der Meinung, es nicht zu tun. Im
Grunde dachten wir, dass dieses Buch zu speziell auf die
französische Situation zugeschnitten ist, in den Beispielen wie in
der Schwerpunktsetzung.

Warum haben wir es trotzdem getan?

\extrapar{}

Der wichtigste Grund ist wohl, dass wir es satt haben, politische
Pamphlete zu lesen, die sich mit der Darstellung der schlechten
Verhältnisse begnügen, ohne konkrete Schritte zu ihrer Aufhebung in
die Diskussion zu werfen. „Der kommende Aufstand“ beschreibt die
bröckeligen Fundamente der gegenwärtigen Ordnung nicht, um
aufzurütteln oder Therapien zu ihrer Rettung vorzuschlagen, im
Gegenteil. Die Zerbrechlichkeit der verschiedenen Aspekte dieser
Welt der Domestizierung und Vernutzung, ihre neusten
Transformationen werden nur durchgespielt, um endlich ihre
Zerstörung konkret ins Auge zu fassen. Die Selbst-Zurichtung der
Individuen, die sich mit Pillen im Rennen der Vermarktung halten,
die Gewöhnung schon der Kleinsten daran, dass ihr Leben in der
Selektion für eine Arbeitswelt bestehen wird, deren einziger Zweck
der Erhalt des Hamsterrades selbst ist; der Angriff auf unser Leben
wird nur geschildert, damit wir uns darin erkennen und dagegen in
Stellung bringen können. Die Rundreise durch das trostlose
Existieren der Metropole ist Aufklärung nicht im mythischen,
sondern im militärischen Sinne: die Klärung eines gemeinsamen
Ausgangspunktes, der operativen Bedingungen einer
Real-Exit-Strategie aus der globalen Misere, und nicht zuletzt der
praktischen Hebel, die uns in diesem Kampf zur Verfügung stehen.
\extrapar{}

Die Tatsache, dass der soziale Angriff der Eliten in den
verschiedenen Staaten der westlichen Welt zusehends ähnliche Formen
annimmt, kann eine derartige Positionsbestimmung über alte
kulturelle Grenzen hinweg tragen. Die Vermessung und Verwaltung des
Menschenmaterials kommt uns allen bekannt vor, auch wenn Dateien
und Polizeien andere Namen tragen. Der Klartext hinter der
Billig-Propaganda von Managern und Kriegsherren tritt international
so klar zutage wie die Erfahrung sich verschärfender Ohnmacht
angesichts der unbelehrbaren Arroganz der Macht. Die Kolonisierung
verarmter Viertel, der expandierende Assimilationsdruck nicht nur
in den Banlieues schärft das Verständnis eines Widerstands, der
sich nicht mehr mit Forderungen aufhält. Der sich in der Tat
organisiert.
\extrapar{}

Wir schätzen den Text der französischen GenossInnen als Beitrag zur
uralten, immer wieder neu aufflammenden Debatte darüber, wie wir
uns diesen Quatsch vom Hals schaffen können, diese ewig gleiche
Reduzierung der Welt auf die Verwertung der Welt. Eine
Gebrauchsanweisung für die Revolte ist das Buch nicht. Das wäre
vollkommen absurd. Jedes Aufbegehren ist so einzigartig wie die
Revoltierenden selbst; eine Vielfalt an Traditionen,
Kampferfahrungen und Träumen, erkennbar aber nicht vereinheitlicht
durch den Glutkern ihrer Sehnsucht nach Befreiung. Viele
Überlegungen über den kommenden Aufstand finden wir inspirierend,
manche Taktiken direkt übertragbar, und einiges führt uns zu
anderen Schlussfolgerungen, weil unsere Stärken andere, unsere
Kämpfe nicht identisch sind. Richtig und gut für die Überwindung
hiesiger Defizite finden wir, dass der in diesem Buch vertretene
Zugang helfen könnte, den Status Quo linker Teilbereichskämpfe
aufzubrechen, der im Horizont der Opposition oft an die
unverbundene Aufzählung unzähliger Antis gebunden bleibt und
dadurch nahe legt, sich in feindlichen Kategorien einzukästeln.
\extrapar{}

Die Bedeutung der Diskussionen und praktischen Versuche, die in den
letzten Jahren um die Idee des Aufstands kreisen, sehen wir in der
Erneuerung einer lebendigen revolutionären Perspektive, im
handgreiflichen Ringen um die Wiedervereinigung von Denken und
Handeln - nicht in 500 Jahren oder am anderen Ende der Welt,
sondern hier und heute. Das Experimentieren mit Alternativen und
der Kampf gegen das Establishment sind nicht nur nicht unvereinbar,
sie ergänzen sich und sind unmittelbar aufeinander angewiesen. Das
unsichtbare Komitee fordert uns auf, die Waffe der Kritik nicht in
den Dienst der eigenen Entwaffnung zu stellen und den Kampf auf der
Strasse nicht in den Abschied vom Nachdenken darüber münden zu
lassen, wie wir uns langfristig halten können, denn der Aufbau
autonomer Strukturen ist notwendige Basis für jeden ernst gemeinten
Angriff auf dieses System. Der allerdings hat viele Formen und
bringt unterschiedlichste Kommunen hervor.
\extrapar{}

„Der kommende Aufstand“ wird uns nicht ersparen, in unseren eigenen
Zusammenhängen klar zu kriegen, wer unsere Feinde sind und wo ihre
Schwachstellen liegen, wie sie angegriffen werden können und welche
Fallen es zu vermeiden gilt. Wir können den Text nutzen, um unsere
Ideen und unsere Praxis weiterzuentwickeln.Wir können kritisch
darauf antworten und die internationale Debatte um unser
Tortenstück Erfahrung bereichern, und so dazu beitragen, eine neue
Sprache der Rebellion zu entwickeln. Notwendig ist das Buch dafür
keinesfalls. Um unsere Situation zu erkennen und daraus
aufzubrechen braucht es keinen Masterplan, keine eine Wahrheit, die
uns offenbart werden muss. Wir können an jeder Ecke loslegen, unser
Leben in die eigenen Hände nehmen und uns gegen die herrschenden
Verhältnisse verbünden. Was die unsichtbaren GenossInnen allerdings
zu bedenken geben, ist, dass es zum Schmieden einer nachhaltigen,
den absehbaren Attacken der Gegenseite gewachsenen Strategie Sinn
macht, eine gewisse Koordination zwischen den einzelnen Gruppen zu
kultivieren, um eine gemeinsame Verteidigung zu ermöglichen.
\extrapar{}

Als Diskussionsvorschlag in Richtung einer derartigen strategischen
Kooperation finden wir das Buch der französischen GenossInnen gut,
nicht als neue Schule oder Kult. Diese undogmatische Weigerung
beinhaltet, dass es uns im Grunde gleichgültig ist, ob die
AutorInnen die Emanzipation der anderen Kollektive zentral am
Herzen liegt oder ob sie lieber alle nach ihrem Ebenbild schaffen
würden. Es geht nicht darum ja oder nein zu ihrem Vorschlag zu
sagen, sondern um die Eskalation einer Diskussion, zu der auch sie
nur beigetragen. Ein libertärer Aufstand wird sich nicht
ausbreiten, von der Revolution ganz zu schweigen, solange der Ruf
nach FührerInnen uns innerlich schwächt. Wie alle Bücher hat auch
„Der kommende Aufstand“ seine blinden Flecken, und die
Unterschätzung der Möglichkeit einer autoritären Wendung der ganzen
Angelegenheit zählt mit Sicherheit dazu. Was nichts anderes heisst,
als dass diese Auseinandersetzung noch darauf harrt, von uns
geführt und aufgeschrieben zu werden! Wenn es gelingt zu vermeiden,
die Diskussion über den kommenden Aufstand auf eine banale
Zugehörigkeitsmaschine einzudampfen, könnten folgende Analysen und
Vorschläge helfen, uns mit organisierteren Strukturen gegen die
fortschreitende Zerstörung unserer Lebensgrundlagen zu formieren
und eine lange nicht gesehene Schlagkraft zu entfalten.

\begin{center}\rule{3in}{0.4pt}\end{center}
\newpage
\tableofcontents

\section{Aus welcher Sicht ...}

Aus welcher Sicht man sie auch betrachtet, die Gegenwart ist ohne
Ausweg. Das ist nicht die geringste ihrer Tugenden. Denjenigen, die
unbedingt hoffen möchten, raubt sie jeden Halt. Diejenigen, die
vorgeben Lösungen zu haben, werden sofort entkräftet. Es ist
bekannt, dass alles nur noch schlimmer werden kann. »Die Zukunft
hat keine Zukunft mehr« ist die Weisheit jener Epoche, die unter
dem Anschein einer extremen Normalität auf der Bewusstseinsebene
der ersten Punks angelangt ist.
\extrapar{}

Die Sphäre der politischen Repräsentation schließt sich. Von Links
nach Rechts nimmt die gleiche Nichtigkeit mal die Pose von Mackern,
mal ein jungfräuliches Gehabe an, sind es die gleichen
Ausverkäufer, die ihre Rede gemäß den neuesten Erfindungen der
Abteilungen für Öffentlichkeitsarbeit austauschen. Diejenigen, die
noch wählen, scheinen dies nur noch mit der Absicht zu tun, die
Urnen durch pure Proteststimmen hochgehen zu lassen. Man fängt an
zu erraten, dass gegen die Wahlen selbst weiter gewählt wird.
Nichts von dem, was sich ergibt, ist auch nur im Entferntesten der
Situation angemessen. In ihrer Stille selbst scheint die
Bevölkerung unendlich viel erwachsener als all die Hampelmänner,
die sich darum zanken, sie zu regieren. Jeder Chibani\footnote{
Slang für alte Menschen
}
aus
Belleville ist weiser in seinen Worten als jeder einzelne unserer
sogenannten Führer in seinen Verlautbarungen. Der Deckel des
sozialen Kochtopfs verschließt sich dreifach, während der innere
Druck stetig steigt. Von Argentinien aus beginnt das Gespenst des
»Que se vayan todos!« ernsthaft in den führenden Köpfen umzugehen.
\extrapar{}

Die Feuer vom November 2005 werfen unaufhörlich ihre Schatten auf
jedes Bewusstsein. Diese ersten Freudenfeuer sind die Taufe eines
Jahrhunderts voller Versprechungen. Fehlt es dem medialen Märchen
der Banlieues-gegen-die-Republik nicht an Effizienz, fehlt es ihm
an Wahrheit. Bis in die Stadtzentren hinein haben sich die Feuer
gebrannt, die methodisch verschwiegen wurden. Ganze Straßen in
Barcelona haben in Solidarität gebrannt, ohne dass, außer ihren
Bewohnern, irgendwer etwas davon mitbekommen hat. Und es stimmt
nicht einmal, dass das Land seither aufgehört hat zu brennen. Unter
den Beschuldigten sind ganz unterschiedliche Profile, die nur noch
der Hass auf die bestehende Gesellschaft eint, nicht mal die
Zugehörigkeit zu einer Klasse, einer Rasse oder einem Stadtteil.
Das Einmalige besteht nicht in einer »Revolte der Banlieues«, die
bereits 1980 nichts Neues war, sondern in dem Bruch mit den
etablierten Formen. Die Angreifer hören auf niemanden mehr, weder
auf die großen Brüder, noch auf den lokalen Verein, welcher die
Rückkehr zum Normalen verwalten sollte. Kein SOS Racisme\footnote{
antirassistische Organisation
}
kann
seine krebserregenden Wurzeln in dieses Ereignis schlagen, dem nur
die Müdigkeit, die Verfälschung und die mediale Omertà\footnote{
Gesetz des Schweigens der Mafia} 
ein
vorgetäuschtes Ende bereiten konnten. Die ganze Serie nächtlicher
Anschläge, anonymer Angriffe und der wortlosen Zerstörung hat den
Verdienst, die größtmögliche Kluft zwischen die Politik und das
Politische zu reißen. Niemand kann ernsthaft die Offenkundigkeit
des Angriffes verneinen, der keine Forderung stellte, der keine
andere Botschaft hatte als die Bedrohung; der nichts mit der
Politik zu schaffen hatte. Man muss blind sein, um das rein
Politische nicht zu sehen, das in dieser entschlossenen Verneinung
der Politik steckt; oder keine Ahnung von den autonomen Bewegungen
der Jugend der letzten dreißig Jahre haben. Wie verlorene Kinder
haben wir den Nippes einer Gesellschaft verbrannt, die nicht mehr
Aufmerksamkeit verdient als die Monumente in Paris zum Ende der
Blutigen Woche\footnote{
Niederschlagung der Pariser Kommune}%
, und die sich dessen bewusst ist.
\extrapar{}

Für die gegenwärtige Situation wird es keine soziale Lösung geben.
Zuerst, weil die verschwommene Anhäufung von Milieus, Institutionen
und individuellen Blasen, die ironischerweise als »Gesellschaft«
bezeichnet wird, keine Konsistenz hat, des Weiteren, weil keine
Sprache mehr für die gemeinsame Erfahrung existiert. Und Reichtümer
können nicht geteilt werden, wenn man keine Sprache teilt. Es
bedurfte eines halben Jahrhundert des Kampfes um die Aufklärung, um
die Französische Revolution zu ermöglichen, und ein Jahrhundert von
Kämpfen um die Arbeit, um den furchterregenden Wohlfahrtsstaat
hervorzubringen. Die Kämpfe schaffen die Sprache, in der sich die
neue Ordnung ausdrückt. Anders heute. Europa ist ein Kontinent, der
pleitegegangen ist, der im Geheimen bei Lidl einkauft und der
Billigflüge nutzt, um überhaupt noch reisen zu können. Keines der
in der sozialen Sprache formulierten »Probleme« führt darin zu
einer Lösung. Die Fragen der »Renten«, jene der »Prekarität«, der
»Jugendlichen« und ihrer »Gewalt« können nur in der Schwebe
bleiben, während jene Gewalt, die immer verblüffender zur Tat
übergeht, mit polizeilichen Maßnahmen verwaltet wird. Es wird sich
nicht beschönigen lassen, den Hintern von alten Leuten zum
Spottpreis zu wischen, die von den Ihren verlassen wurden und
nichts zu sagen haben. Diejenigen, welche auf kriminellem Wege
weniger Erniedrigung und mehr Profit gefunden haben als in der
Arbeit als Reinigungskraft, werden ihre Waffen nicht strecken. Das
Gefängnis wird ihnen nicht die Liebe zur Gesellschaft eintrichtern.
Die auf Genuß abgehenden Horden von Rentnern werden die drastischen
Kürzungen ihres monatlichen Einkommens nicht auf den Knien
hinnehmen und sich nur noch mehr aufregen, dass eine breite
Fraktion der Jugend die Arbeit verweigert. Wieso schließlich sollte
ein Grundeinkommen, gewährt nach einem Beinahe-Aufruhr, den
Grundstein für einen neuen New Deal, einen neuen Pakt, einen neuen
Frieden legen. Dafür ist vom sozialen Empfinden zu viel verdampft.
\extrapar{}

Als Lösung wird sich der Druck, dass nichts passiert, und mit ihm
auch das polizeiliche Raster des Territoriums stetig verstärken.
Die Drone, die laut eigenem Eingeständnis der Polizei am letzten
14. Juli\footnote{
14. Juli 2007}
Seine-Saint-Denis überflog, zeichnet die Zukunft in viel
deutlicheren Farben als der humanistische Dunst. Dass man sich die
Mühe gemacht hat zu präzisieren, dass sie nicht bewaffnet war,
zeigt ziemlich deutlich, auf welchem Weg wir uns befinden. Das
Territorium wird in immer dichter abgeriegelte Zonen zerstückelt
werden. Autobahnen, die am Rand eines »Problemviertels« gebaut
werden, bilden eine unsichtbare Mauer, um es ganz und gar von den
Reihenhäusern abzutrennen. Was auch immer die guten
republikanischen Seelen darüber denken mögen, das Verwalten von
Stadtteilen »durch Communities« ist bekanntlich am effizientesten.
Die rein metropolitanen Stücke des Territoriums, die wichtigsten
Stadtzentren, werden in einer immer hinterhältigeren, immer
ausgefeilteren, immer eklatanteren Dekonstruktion ihr Luxusleben
führen. Mit ihrem Bordelllicht werden sie den ganzen Planeten
beleuchten, während die Patrouillen der BAC\footnote{
»Brigade Anti Criminalité«, städtische Einheit der Polizei}%
, begleitet von
privaten Sicherheitsdiensten, kurz: die Milizen, sich unendlich
vervielfachen und dabei eine immer unverschämtere Deckung durch
die Justiz genießen werden.
Die Sackgasse der Gegenwart, überall wahrnehmbar, wird überall
geleugnet. Noch nie haben so viele Psychologen, Soziologen und
Literaten sich damit beschäftigt, jeder in seinem speziellen
Jargon, in dem die Schlussfolgerung auf spezielle Art abwesend ist.
Es reicht aus, sich die Lieder der Epoche anzuhören, die Schnulzen
des »neuen französischen Chanson«, in welchen das Kleinbürgertum
seine seelischen Zustände seziert, und die Kriegserklärungen von
Mafia K‘1Fry\footnote{
französische Hip-Hop-Gruppe}%7
, um zu wissen, dass die Koexistenz bald aufhören
wird, dass eine Entscheidung naht.

\extrapar{}

Dieses Buch ist mit dem Namen eines imaginären Kollektivs
unterzeichnet. Seine Redakteure sind nicht seine Autoren. Sie haben
sich damit zufrieden gegeben, ein bisschen Ordnung in die
verschiedenen Allgemeinplätze dieser Epoche zu bringen, in das, was
an den Tischen der Bars, was hinter verschlossenen
Schlafzimmertüren gemurmelt wird. Sie haben nur die nötigen
Wahrheiten fixiert, deren universelle Verdrängung die
psychiatrischen Kliniken und die Blicke mit Schmerz füllt. Sie
haben sich zu den Schreibern der Situation gemacht. Es ist das
Privileg der radikalen Umstände, dass die Richtigkeit in logischer
Konsequenz zur Revolution führt. Es reicht aus, das zu benennen,
was einem unter die Augen kommt, und dabei nicht der
Schlussfolgerung auszuweichen.

\section{Erster Kreis}

\satz{»I AM WHAT I AM«}

»I AM WHAT I AM.« Das ist die letzte Opfergabe des Marketing an die
Welt, das letzte Entwicklungsstadium der Werbung, und vor, weit vor
all den Mahnungen, anders zu sein, man selbst zu sein, und Pepsi zu
trinken. Jahrzehnte von Konzepten, um dort anzukommen, bei der
reinen Tautologie. ICH = ICH. Er rennt auf einem Laufband vor dem
Spiegel in seinem Fitnesscenter. Sie fährt am Steuer ihres Smart
von der Arbeit nach Hause zurück. Werden sie sich treffen?
»ICH BIN WAS ICH BIN.« Mein Körper gehört mir. Ich bin Ich, Du bist
Du und es geht schlecht. Die Personalisierung der Masse. Die
Individualisierung aller Bedingungen – des Lebens, der Arbeit, des
Unglücks. Diffuse Schizophrenie. Schleichende Depression.
Atomisierung in feine paranoide Partikel. Hysterisierung des
Kontakts. Je mehr ich Ich sein will, desto mehr habe ich das Gefühl
einer Leere. Je mehr ich mich ausdrücke, desto mehr trockne ich
aus. Je mehr ich hinter mir herlaufe, desto erschöpfter bin ich.
Ich betreibe, du betreibst, wir betreiben unser Ich wie einen
geschäftigen Schalter. Wir sind die Vertreter unserer selbst
geworden – dieser seltsame Handel, die Garanten einer
Personalisierung, die letzten Endes einer Amputation ähnelt. In
mehr oder weniger versteckter Unbeholfenheit schaffen wir es zum
Bankrott.

Bis es soweit ist, verwalte ich, kriege ich es hin. Die Suche nach
sich selbst, mein Blog, meine Wohnung, der letzte angesagte Scheiß,
die Pärchengeschichten, die Affairen... was es an Prothesen
braucht, um ein Ich zusammenzuhalten! Wäre »die Gesellschaft« nicht
zu dieser definitiven Abstraktion geworden, würde sie die ganzen
existentiellen Krücken bezeichnen, die mir gereicht werden, damit
ich mich weiterschleppen kann, die ganzen Abhängigkeiten, die ich
um den Preis meiner Identität eingegangen bin. Der Behinderte ist
das Vorbild der kommenden Bürgerlichkeit. Es ist nicht ohne jede
Vorahnung, dass die Vereine, die ihn ausbeuten, ein
existenzsicherndes Grundeinkommen für ihn fordern.

\extrapar{}

Die allgegenwärtige Anordnung, »jemand zu sein«, erhält den
pathologischen Zustand, der diese Gesellschaft notwendig macht. Die
Anordnung, stark zu sein, erzeugt die Schwäche, durch die sie sich
erhält, so dass alles einen therapeutischen Aspekt zu bekommen
scheint, sogar arbeiten, sogar lieben. All die »Wie geht‘s?«, die
jeden Tag ausgetauscht werden, lassen ans Fiebermessen denken,
wodurch die einen den anderen die Patientengesellschaft aufzwingen.
Gesellschaftlichkeit besteht heute aus tausend kleinen Nischen, aus
tausend kleinen Unterschlüpfen, in denen man sich warm hält. Wo es
immer besser ist als draußen in der großen Kälte. Wo alles falsch
ist, weil alles nur ein Vorwand ist, um sich aufzuwärmen. Wo nichts
entstehen kann, weil man dort taub wird beim gemeinsamen
Schlottern. Diese Gesellschaft wird bald nur noch durch die
Spannung zwischen allen sozialen Atomen in Richtung einer
illusorischen Heilung zusammengehalten. Sie ist ein Werk, das seine
Kraft aus einem gigantischen Staudamm von Tränen zieht, der ständig
kurz vor dem Überlaufen ist.
\extrapar{}

»I AM WHAT I AM.« Niemals hatte eine Herrschaft ein
unverdächtigeres Motto gefunden. Das Erhalten des Ich in einem
Zustand permanenten Halb-Verfalls, chronischer Halb-Ohnmacht ist
das bestgehütete Geheimnis der aktuellen Ordnung der Dinge. Das
schwache, deprimierte, selbstkritische, virtuelle Ich ist im
Wesentlichen dieses unendlich anpassbare Subjekt, das von einer
Produktion gefordert wird, die auf Innovation beruht, auf dem
beschleunigten Veralten der Technologien, dem stetigen Umbruch der
sozialen Normen und der verallgemeinerten Flexibilität. Es ist
gleichzeitig der gefräßigste Konsument und, paradoxerweise, das
produktivste Ich, welches sich mit einem Maximum an Energie und
Gier auf das kleinste Projekt stürzt, um später wieder in seinen
larvenartigen Originalzustand zurückzukehren.
\extrapar{}

»DAS, WAS ICH BIN«, ja und? Von Kindheit an durchdrungen von
Flüssen von Milch, Gerüchen, Geschichten, Klängen, Affektionen,
Reimen, Substanzen, Gesten, Ideen, Eindrücken, Blicken, Gesängen
und Fressen. Was ich bin? Von allen Seiten gebunden an Orte,
Leiden, Ahnen, Freunde, Liebschaften, Ereignisse, Sprachen,
Erinnerungen, an Dinge aller Art, die mit aller Offenkundigkeit
nicht Ich sind. Alles, was mich an die Welt bindet, alle
Verbindungen, die mich ausmachen, alle Kräfte, die mir innewohnen,
verstricken sich nicht zu einer Identität, die zur Schau zu stellen
wir aufgefordert werden, sondern zu einer Existenz: einzigartig,
gemeinschaftlich, lebendig, aus der stellenweise, im Moment, dieses
Wesen aufsteigt, das »ich« sagt. Unser Gefühl der Inkonsistenz ist
nur eine Auswirkung dieses dummen Glaubens an die Permanenz des
Ich, und der wenigen Sorgfalt, die wir dem entgegenbringen, was uns
ausmacht.
Es macht schwindelig, das »I AM WHAT I AM« von Reebok an einem
Wolkenkratzer von Schanghai thronen zu sehen. Das Abendland rückt
überall vor wie sein bevorzugtes trojanisches Pferd, diese tödliche
Antinomie zwischen dem Ich und der Welt, dem Individuum und der
Gruppe, der Verbundenheit und der Freiheit. Die Freiheit ist nicht
die Geste, uns von unseren Verbundenheiten loszulösen, sondern die
praktische Fähigkeit, auf sie einzuwirken, sich in ihnen zu
bewegen, sie zu erschaffen oder zu durchtrennen. Die Familie
existiert nur für denjenigen als Familie, das heißt als Hölle, der
darauf verzichtet hat, ihre Mechanismen zu verderben, die uns
schwachsinnig machen, oder der nicht weiß wie. Die Freiheit sich
loszureißen war schon immer das Gespenst der Freiheit. Wir
entledigen uns nicht von dem, was uns fesselt, ohne gleichzeitig
das zu verlieren, worauf sich unsere Kräfte ausüben könnten.

»I AM WHAT I AM«, also, keine bloße Lüge, keine bloße
Werbekampagne, sondern ein Feldzug, ein Kriegsschrei, gerichtet
gegen alles, was es zwischen den Wesen gibt, gegen alles, was
ununterscheidbar zirkuliert, alles, was sie unsichtbar miteinander
verbindet, alles, was die perfekte Verwüstung hindert, gegen alles,
was bewirkt, dass wir existieren und dass die Welt nicht überall
wie eine Autobahn aussieht, wie ein Vergnügungspark oder eine
Trabantenstadt: pure Langeweile, ohne Leidenschaft und wohl
geordnet, leerer Raum, eiskalt, nur noch durchquert von
registrierten Körpern, automobilen Molekülen und idealen Waren.

Frankreich ist nicht das Vaterland der Anxiolytika, das Paradies
der Antidepressiva und das Mekka der Neurose, ohne gleichzeitig
Europameister der Stundenproduktivität zu sein. Die Krankheit, die
Müdigkeit, die Depression können als individuelle Symptome dessen
wahrgenommen werden, was geheilt werden muss. Sie arbeiten also am
Erhalt der existierenden Ordnung, an meiner folgsamen Anpassung an
dumme Normen, an der Modernisierung meiner Krücken. Sie umfassen
die Selektion der opportunen, konformen und produktiven Neigungen
in mir, von jenen, die brav zu betrauern sind. »Man muss sich
verändern können, weißt du.« Werden sie aber als Fakten angenommen,
können meine Störungen auch zur Zerschlagung der Hypothese des Ich
führen. Sie werden also zu Akten des Widerstandes im laufenden
Krieg. Sie werden zur Rebellion und zum Energiezentrum gegen alles,
was sich verschwört, uns zu normalisieren, uns zu amputieren. Es
ist nicht das Ich, was bei uns in der Krise ist, sondern die Form,
die man uns aufzuzwingen versucht. Es sollen wohl abgegrenzte, wohl
getrennte Ichs aus uns gemacht werden, zuordenbar und zählbar nach
Qualitäten, kurz: kontrollierbar; während wir Kreaturen unter
Kreaturen sind, Einzigartigkeiten unter unseresgleichen, lebendiges
Fleisch, welches das Gewebe der Welt bildet. Entgegen dem, was uns
seit der Kindheit immer wieder erzählt wird, besteht Intelligenz
nicht darin, sich anpassen zu können – oder wenn das eine
Intelligenz ist, dann die der Sklaven. Unsere Unfähigkeit zur
Anpassung und unsere Müdigkeit sind keine Probleme, außer aus der
Sicht dessen, was uns unterwerfen will. Vielmehr deuten sie auf
einen Ausgangspunkt, einen Verbindungspunkt für neue
Komplizenschaften. Sie eröffnen den Blick auf eine noch viel
verfallenere, aber unendlich viel teilbarere Landschaft als all die
Trugbilder, die diese Gesellschaft von sich selbst bereithält.

Wir sind nicht deprimiert, wir streiken. Wer sich weigert, sich zu
verwalten, für den ist die »Depression« kein Zustand, sondern ein
Übergang, ein Auf-Wiedersehen, ein Schritt zur Seite hin zur
Aufkündigung einer politischen Zugehörigkeit. Davon ausgehend gibt
es keine andere Schlichtung als die medikamentöse und die
polizeiliche. Genau deswegen scheut sich diese Gesellschaft nicht,
ihren zu lebhaften Kindern Ritalin aufzuzwingen, zu jeder
Gelegenheit Laufleinen pharmazeutischer Abhängigkeiten zu flechten,
und vorzugeben, schon bei Dreijährigen »Verhaltensstörungen«
festzustellen. Weil die Hypothese des Ich überall Risse bekommt.

\section{Zweiter Kreis}

\satz{»Unterhaltung ist ein Grundbedürfnis«}

Eine Regierung, die den Notstand über 15jährige verhängt. Ein Land,
das sein Heil in die Hände einer Fußballmannschaft legt. Ein Bulle
in einem Krankenhausbett, der klagt, zum Opfer von »Gewalt«
geworden zu sein. Ein Präfekt, der eine Verordnung erlässt gegen
jene, die sich Häuser in Bäumen bauen. Zwei Kinder im Alter von 10
Jahren aus Chelles, die der Brandstiftung an einer Ludothek\footnote{
Einrichtung, wo Spiele Kindern zur Verfügung gestellt werden
}
beschuldigt werden. Diese Epoche zeichnet sich durch eine gewisse
Groteske der Situation aus, der sie sich jedes Mal zu entziehen
scheint. Dazu muss man sagen, dass die Medialen keine Mühen
scheuen, mit ihrer Tonlage der Klage und Entrüstung das schallende
Gelächter zu ersticken, das solche Neuigkeiten eigentlich begleiten
sollte.
Eine Explosion schallenden Gelächters ist die angemessene Antwort
auf all die ernsten »Fragen«, die es der Tagesschau zu stellen
gefällt. Angefangen mit der abgedroschensten: Es gibt keine »Frage
der Immigration«. Wer wächst noch da auf, wo er geboren wurde? Wer
wohnt da, wo er aufgewachsen ist? Wer arbeitet da, wo er wohnt? Wer
wohnt dort, wo seine Vorfahren gelebt haben? Und von wem sind die
Kinder dieser Epoche, vom Fernsehen oder von ihren Eltern? Die
Wahrheit ist, dass wir massenhaft aus jeder Zugehörigkeit gerissen
wurden, dass wir von nirgendwo mehr herkommen, und dass sich
daraus, gleichzeitig mit einer ungewöhnlichen Neigung zum
Tourismus, ein nicht zu leugnendes Leiden ergibt. Unsere Geschichte
ist jene der Kolonisierungen, der Migrationen, Kriege, Exile, der
Zerstörung sämtlicher Verwurzelungen. Es ist die Geschichte all
dessen, was uns zu Fremden in dieser Welt gemacht hat, zu Gästen in
unserer eigenen Familie. Wir wurden unserer Sprache enteignet durch
die Schule, unserer Lieder durch die Hitparade, unseres Fleisches
durch die Massenpornographie, unserer Stadt durch die Polizei,
unserer Freunde durch die Lohnarbeit. Dazu kommt in Frankreich die
erbarmungslose, jahrhundertelange Arbeit der Staatsmacht an der
Individualisierung, die ihre Untertanen vom jüngsten Alter an
verzeichnet, vergleicht, diszipliniert und trennt, die aus Instinkt
jegliche Solidaritäten, die sich ihr entziehen, zermalmt, bis nur
noch die Staatsbürgerschaft bleibt, die reine, eingebildete
Zugehörigkeit zur Republik. Der Franzose ist mehr als alle andern
der Enteignete, der Elende. Sein Hass auf die Ausländer mischt sich
unter seinen Hass auf sich als Fremden. Seine Eifersucht, gemischt
mit dem Entsetzen über die »Banlieues«, drückt nur sein
Ressentiment aus, über all das, was er verloren hat. Er kann es
sich nicht verkneifen, diese sogenannten Stadtteile der
»Verbannung« zu beneiden. Stadtteile, in denen noch ein bißchen
gemeinschaftliches Leben besteht, etwas Verbundenheit zwischen den
Menschen, ein paar nichtstaatliche Solidaritäten, eine informelle
Ökonomie, eine Organisation, die noch nicht von denen getrennt ist,
die sie organisieren. Wir sind an einem Punkt des Verlusts
angelangt, an dem die einzige Art und Weise, sich als Franzose zu
fühlen, ist, Immigranten zu beschimpfen, diejenigen, die sichtbarer
Fremde sind als ich. Die Immigranten haben in diesem Land eine
seltsame Position der Souveränität: Wären sie nicht da, würden die
Franzosen vielleicht nicht mehr existieren.

\extrapar{}

Frankreich ist ein Produkt seiner Schule, und nicht umgekehrt. Wir
leben in einem übermäßig verschulten Land, wo man sich an das
Bestehen des Abiturs als einen prägenden Moment des Lebens
erinnert. Wo Rentner vierzig Jahre später noch von ihrem
Durchfallen in dieser oder jener Prüfung erzählen und davon, wie
dies ihre ganze Karriere, und ihr ganzes Leben belastet hat. Seit
eineinhalb Jahrhunderten hat die Schule der Republik eine Art
verstaatlichter Subjektivitäten gebildet, die unter allen anderen
erkennbar sind. Leute, welche die Selektion und den Wettbewerb
unter der Bedingung akzeptieren, dass die Chancen gleich verteilt
sind. Die vom Leben erwarten, dass jeder wie in einem Wettbewerb
belohnt wird, nach seinem Verdienst. Die immer um Erlaubnis fragen,
bevor sie etwas nehmen. Die ganz still die Kultur, die Regeln und
die Klassenbesten respektieren. Selbst ihre Verbundenheit mit ihren
großen kritischen Intellektuellen und ihre Ablehnung des
Kapitalismus sind von dieser Liebe zur Schule geprägt. Diese
staatliche Konstruktion der Subjektivitäten ist es, welche unter
der Last der Dekadenz der schulischen Institution Tag für Tag mehr
zusammenbricht. Das Wiederauftauchen der Schule in den letzten 20
Jahren sowie der Straßenkultur in Konkurrenz zur Schule der
Republik und ihrer Pappkultur ist das tiefste Trauma, das der
französische Universalismus aktuell erleidet. An diesem Punkt
versöhnt sich die extremste Rechte im Voraus mit der giftigsten
Linken. Allein schon der Name Jules Ferry, Minister von Thiers
während der Zerschlagung der Pariser Kommune und Theoretiker der
Kolonisierung\footnote{
Knapp zehn Jahre nach Zerschlagung der Pariser Kommune setzte
Ferry 1880 in abgeschwächter Form eine zentrale Forderung der
KommunardInnen um: den unentgeltlichen und verpflichtenden
Grundschulbesuch.
}%9
, sollte doch reichen, um uns diese Institution
suspekt zu machen.
Was uns angeht, wenn wir hören, wie Lehrer aus irgendeinem
„Bürgerkomitee für Sicherheit und Sauberkeit“ in den
Abendnachrichten heulen, dass ihnen ihre Schule abgefackelt wurde,
dann erinnern wir uns daran, wie oft wir als Kinder genau davon
geträumt haben. Wenn wir einen linken Intellektuellen hören, wie er
sich über die Barbarei der Banden von Jugendlichen auskotzt, welche
Passanten auf der Strasse anquatschen, im Supermarkt klauen, Autos
anzünden und mit der CRS\footnote{
»Compagnies républicaines de sécurité«, Einheit der Polizei
}
Katz und Maus spielen, dann erinnern wir
uns daran, was 1960 über die »Blouson Noirs«\footnote{
In Lederjacken gekleidete Halbstarke und Rocker}%11
, oder besser noch,
was über die Apachen während der »Belle Époque« erzählt wurde:
»Seit einigen Jahren ist es Mode geworden«, schrieb im Jahr 1907
ein Richter am Gericht »la Seine«, »unter dem Gattungsnamen Apachen
alle gefährliche Individuen zu bezeichnen; Sammelbecken der
Wiederholungstäter, Feinde der Gesellschaft, nicht Vaterland noch
Familie, Deserteure aller Pflichten, bereit zu wagemutigsten
Handgreiflichkeiten, zu jedem Attentat auf Leben oder Eigentum«.
Diese Banden, die der Arbeit entfliehen, sich nach ihrem Stadtteil
benennen und gegen die Polizei kämpfen, sind der Albtraum des guten
Bürgers, individualisiert à la française: Sie verkörpern all das,
worauf er verzichtet hat, all die mögliche Freude, die zu erreichen
ihm nie möglich sein wird. Es ist eine Frechheit, in einem Land zu
existieren, wo ein Kind, das singt, wie ihm der Sinn steht,
unvermeidlich beschimpft wird »Hör auf, es wird noch anfangen zu
regnen!«, wo die schulische Kastration am laufenden Band
Generationen von polizierten Angestellten ausstößt. Die
fortbestehende Aura von Mesrine\footnote{
Jaques Mesrine, französischer Bankräuber und bewusster
Verbrecher. Er kritisierte den Knast als System und wurde 1979 von
der Polizei ermordet.
}
hat weniger mit seiner
Geradlinigkeit und seiner Unverfrorenheit zu tun als mit seinem
Unterfangen, sich daran zu rächen, woran wir uns alle rächen
sollten. Oder vielmehr, woran wir uns direkt rächen sollten, statt
auszuweichen, statt es hinauszuschieben. Denn es gibt keinen
Zweifel, dass der Franzose sich rächt, ständig und an allem, mit
tausenden von unauffälligen Niederträchtigkeiten, mit allen Arten
von Verleumdungen, mit einer kleinen, eisigen Boshaftigkeit und
einer giftigen Freundlichkeit rächt er sich für das
Zertretenwerden, dem er sich ergeben hat. Es war an der Zeit, dass
das Fick die Polizei! das Jawohl Herr Polizist! ersetzte. In diesem
Sinne drückt die nuancenlose Feindseligkeit bestimmter Banden nur
ein bisschen weniger gedämpft die schlechte Stimmung aus, den
grundsätzlich schlechten Geist, die Lust nach erlösender
Zerstörung, in der sich dieses Land verzehrt.

\extrapar{}

Das Volk von Fremden, in deren Mitte wir leben, als »Gesellschaft«
zu bezeichnen, stellt einen derartigen Betrug dar, dass sich sogar
die Soziologen überlegen, sich von einem Konzept zu verabschieden,
das ein Jahrhundert lang ihr Broterwerb war. Sie bevorzugen heute
die Metapher des Netzwerks, um die Art und Weise zu beschreiben,
wie sich die kybernetischen Einsamkeiten verbinden, wie sich die
schwachen Interaktionen, bekannt unter den Namen »Kollege«,
»Kontakt«, »Kumpel«, »Beziehung« oder »Affaire« verknüpfen. Dennoch
passiert es, dass diese Netzwerke zu einem Milieu verdampfen, wo
man nichts teilt außer Codes und nichts auf dem Spiel steht außer
der unaufhörlichen Wiederherstellung einer Identität.
\extrapar{}

Es wäre Zeitverschwendung, alles genau aufzuführen, was in den
existierenden sozialen Verhältnissen im Sterben liegt. Die Familie
kehrt zurück, sagt man, das Paar kehrt zurück. Doch die Familie,
die zurückkehrt, ist nicht die, die gegangen ist. Ihre Rückkehr ist
nur eine Vertiefung der herrschenden Trennung, die sie vertuschen
soll, wobei sie selbst zur Täuschung wird. Jeder kann die Dosen von
Traurigkeit bezeugen, die sich Jahr für Jahr an Familienfesten
kristallisieren, dieses mühsame Lächeln, diese Verlegenheit, alle
vergeblich simulieren zu sehen, dieses Gefühl, dass da ein Kadaver
liegt, auf dem Tisch, und alle tun so, als ob nichts wäre. Vom
Flirt zur Scheidung, vom Konkubinat zum Patchwork empfindet jeder
die Sinnlosigkeit der traurigen Kernfamilie, doch die meisten
scheinen die Einschätzung zu haben, dass es noch trauriger wäre,
darauf zu verzichten. Die Familie, das ist nicht mehr so sehr das
Ersticken unter dem mütterlichen Einfluss oder das Patriarchat der
Ohrfeigen, sondern vielmehr dieses kindliche Sich-Gehenlassen in
einer flauschigen Abhängigkeit, in der alles bekannt ist, dieser
Moment von Sorglosigkeit gegenüber einer Welt, von der niemand mehr
leugnen kann, dass sie in sich zusammenbricht. Eine Welt, in der
»autonom werden« ein Euphemismus dafür ist, »einen Chef gefunden zu
haben«. Man möchte in der biologischen Vertrautheit die
Entschuldigung erkennen, jede auch nur ein bisschen angriffige
Entschlossenheit in uns zu zerfressen, uns zum Verzicht anzuregen;
zum Verzicht darauf, ganz erwachsen zu werden, sowie zum Verzicht
auf die Ernsthaftigkeit, die schon in der Kindheit steckt, unter
dem Vorwand, dass man uns hat aufwachsen sehen. Vor diesem
Zerfressen-werden müssen wir uns bewahren.
Das Paar ist wie die letzte Stufe des großen, gesellschaftlichen
Debakels. Es ist die Oase in der Mitte der menschlichen Wüste. In
ihm wird unter dem heiligen Schutz »des Intimen« all das gesucht,
was so offenkundig alle zwischenmenschlichen Beziehungen heutzutage
verlassen hat: die Wärme, die Einfachheit, die Wahrheit, ein Leben
ohne Theater und Zuschauer. Aber ist der Liebestaumel vorbei, dann
lässt die »Intimität« die Hosen runter: Sie ist selbst eine soziale
Erfindung, sie spricht die Sprache der Frauenzeitschriften und der
Psychologie, sie ist wie der Rest bis zum Erbrechen voll mit
Strategien. Es gibt darin nicht mehr Wahrheit als irgendwo sonst,
denn auch hier herrschen die Lüge und die Gesetze der
Fremdhaftigkeit. Und wird sie darin gefunden, glücklicherweise,
diese Wahrheit, dann ruft sie ein Teilen hervor, das der Form des
Paares selbst widerspricht. Denn das, wodurch die Menschen sich
lieben, kann auch das sein, was sie liebenswert macht, und was jede
Utopie des Autismus zu zweit zerstört.

In Wirklichkeit ist die Zersetzung aller gesellschaftlichen Formen
ein Glücksfall. Sie ist für uns die ideale Bedingung für ein wildes
Massenexperiment, in neuen Zusammensetzungen, mit neuen Treuen. Die
sogenannte »elterliche Vernachlässigung« hat uns eine Konfrontation
mit der Welt aufgenötigt, die in uns eine frühreife Hellsichtigkeit
erzwungen hat, und ein paar schöne Revolten erahnen lässt. Im Tod
des Paares sehen wir verwirrende Formen kollektiver Affektivität
aufsteigen, jetzt, wo der Sex bis zum Verschleiß abgerieben ist, wo
Männlichkeit und Weiblichkeit zu mottenzerfressenen Kostümen
verkommen sind, wo drei Jahrzehnte fortgesetzter pornographischer
Innovationen jeden Reiz an der Überschreitung und der Befreiung
genommen haben. Aus dem, was es an Unbedingtem in
verwandtschaftlichen Verbindungen gibt, beabsichtigen wir das
Gerüst für eine politische Solidarität zu errichten, die für den
staatlichen Zugriff so undurchdringbar ist wie ein Zigeunerlager.
Sogar in den endlosen Subventionen, die viele Eltern ihrem
proletarisierten Nachwuchs zu zahlen gezwungen sind, gibt es
nichts, was nicht zu einer Art Mäzenentum für die soziale
Subversion werden könnte. »Autonom werden« könnte auch gut heißen:
lernen, auf der Straße zu kämpfen, sich leere Häuser zu nehmen,
nicht zu arbeiten, sich wie verrückt zu lieben und in den
Supermärkten zu klauen.

\section{Dritter Kreis}

\satz{»Das Leben, die Gesundheit und die Liebe sind prekär, wieso sollte
sich die Arbeit diesem Gesetz entziehen?«}

Es gibt in Frankreich keine verworrenere Frage als die der Arbeit,
es gibt keine verdrehtere Beziehung als die der Franzosen zur
Arbeit. Geht nach Andalusien, Algerien oder Neapel. Dort verachtet
man die Arbeit im Grunde. Geht nach Deutschland, in die USA oder
nach Japan. Dort wird die Arbeit verehrt. Die Dinge ändern sich,
das ist wahr. Es gibt sehr wohl auch Otaku in Japan, Glückliche
Arbeitslose in Deutschland und Workaholics in Andalusien. Aber die
sind zur Zeit nur Kuriositäten. In Frankreich reißt man sich Arme
und Beine aus, um in der Hierarchie aufzusteigen, aber im Privaten
rühmt man sich, nichts zu tun. Man bleibt bis um zehn Uhr abends
auf der Arbeit, wenn man überfordert ist, aber man hat keine
Skrupel, dort Büromaterial zu klauen oder sich bei den Waren im
Lager zu bedienen und diese bei Gelegenheit zu verkaufen. Man hasst
die Chefs, will aber um jeden Preis Angestellter sein. Eine Arbeit
zu haben ist eine Ehre und arbeiten ein Zeichen der
Unterwürfigkeit. Kurz: das perfekte Krankheitsbild der Hysterie.
Man liebt, indem man hasst, und man hasst, indem man liebt. Und
jeder weiß, welche Verblüffung und Verwirrung den Hysterischen
schlägt, wenn er sein Opfer, seinen Herrn verliert. In den meisten
Fällen erholt er sich davon nie wieder.
In diesem grundlegend politischen Land, das Frankreich ist, war die
industrielle Macht stets der staatlichen unterworfen. Die
wirtschaftlichen Aktivitäten waren niemals ohne die argwöhnisch
rahmende Betreuung einer kleinlichen Verwaltung. Die großen Chefs,
die nicht vom staatlichen Adel à la École Polytechnique oder ENA\footnote{
Eliteschulen, die alle wichtigen Staatsfunktionäre,
PolitikerInnen und Wirtschaftseliten durchlaufen
}
stammen, sind die Paria der Geschäftswelt, wo zugegeben wird,
hinter der Kulisse, dass sie ein bisschen Mitleid erregen. Bernard
Tapie\footnote{
französischer Geschäftsmann und Schauspieler mit zweifelhaftem
Ruf
}
ist ihr tragischer Held: an einem Tag vergöttert, am
nächsten Tag im Knast, immer unberührbar. Dass er jetzt auf der
Bühne steht, hat nichts Erstaunliches. Indem das französische
Publikum ihn bewundert, wie man ein Monster bewundert, hält es
sichere Distanz, durch das Spektakel einer so faszinierenden
Ruchlosigkeit schützt es sich vor dem Kontakt. Trotz dem grossen
Bluff der 1980er Jahre hat in Frankreich der Kult des Unternehmens
nie Fuss gefasst. Wer immer ein Buch schreibt, um ihn zu
verunglimpfen, ist sich eines Bestsellers sicher. Die Manager mit
ihren Sitten und ihrer Literatur können sich zwar in der
Öffentlichkeit präsentieren, es umgibt sie aber ein Sperrgürtel aus
Gekicher, ein Ozean aus Verachtung, ein Meer aus Sarkasmus. Der
Unternehmer gehört nicht zur Familie. In der Hierarchie des
Abscheus wird ihm der Bulle vorgezogen. Beamter zu sein bleibt,
gegen Wind und Wetter, gegen Golden Boys und Privatisierung, nach
allgemeinem Verständnis die Definition guter Arbeit. Man kann den
Reichtum derjenigen beneiden, die keine sind, um ihre Stellen
beneidet man sie nicht.

Vor dem Hintergrund dieser Neurose können die aufeinander folgenden
Regierungen noch immer der Arbeitslosigkeit den Krieg erklären und
vorgeben, sich die »Schlacht um die Beschäftigung« zu liefern,
während Ex-Führungs\-kräfte mit ihren Handies in den Zelten der
Médecins du monde\footnote{
internationale humanitäre Hilfsorganisation
}
am Ufer der Seine campen. Während es den
massiven Streichungen der ANPE\footnote{
»Agence Nationale Pour l‘Emploi«, Agentur für Arbeit
}
und allen statistischen Tricks
nicht gelingt, die Zahl der Arbeitslosen unter zwei Millionen zu
senken. Während das RMI\footnote{
»Revenu Minimum d‘Insertion«, Fürsorgezahlungen
}
und die kleinen Gaunereien selbst nach
der Meinung der Nachrichtendienste die einzigen Faktoren sind, die
vor einer sozialen Explosion schützen, die zu jedem Moment möglich
ist. Die psychische Ökonomie der Franzosen genauso wie die
politische Stabilität des Landes ist es, welche bei der
Aufrechterhaltung dieser arbeiterischen Fiktion auf dem Spiel
stehen.

Mit Verlaub, das ist uns scheissegal.

Wir gehören einer Generation an, die sehr gut ohne diese Fiktion
lebt. Die noch nie auf die Rente oder das Arbeitsrecht und schon
gar nicht auf das Recht auf Arbeit gezählt hat. Die nicht einmal
prekär ist, wie es die fortschrittlichsten Fraktionen des linken
Aktivismus gerne theoretisieren. Weil prekär sein noch immer heißt,
sich in Bezug auf die Sphäre der Arbeit zu definieren, in diesem
Fall: in Bezug auf ihren Zerfall. Wir anerkennen die Notwendigkeit,
Geld zu finden, mit welchen Mitteln auch immer, denn es ist zur
Zeit unmöglich, darauf zu verzichten. Wir anerkennen nicht die
Notwendigkeit der Arbeit. Außerdem arbeiten wir nicht mehr: wir
jobben. Das Unternehmen ist kein Ort, an dem wir existieren, es ist
ein Ort, den wir durchqueren. Wir sind nicht zynisch, wir scheuen
nur, uns missbrauchen zu lassen. Der Diskurs der Motivation, der
Qualität und des persönlichen Einbringens perlt zur großen
Verzweiflung aller Verwalter des Humankapitals an uns ab. Man sagt,
wir seien von den Unternehmen enttäuscht, dass diese die Loyalität
unserer Eltern nicht honoriert hätten, sie hätten sie zu schnell
entlassen. Man lügt. Um enttäuscht zu werden, muss man einst
gehofft haben. Und wir haben uns von den Unternehmen nie etwas
erhofft: wir sehen sie als das, was sie sind und was zu sein sie
nie aufgehört haben, ein doppeltes Spiel mit wechselhaftem Komfort.
Wir bedauern vor allem unsere Eltern, dass sie in diese Falle
getappt sind, die zwei zumindest, die daran geglaubt haben.

Die sentimentale Verwirrung um die Frage der Arbeit lässt sich
folgendermaßen erklären: Der Begriff der Arbeit umfasste schon
immer zwei gegensätzliche Dimensionen: Eine Dimension der
Ausbeutung und eine Dimension der Teilnahme. Ausbeutung der
Arbeitskraft, individuell und kollektiv, durch die Aneignung des
Mehrwerts, privat oder sozial; Teilnahme an einem gemeinsamen Werk
durch die Verbindungen, die sich zwischen denen knüpfen, die im
Universum der Produktion kooperieren. Im Begriff der Arbeit sind
diese zwei Dimensionen pervers verworren, was im Grunde die
Gleichgültigkeit erklärt, mit der die Arbeiter der marxistischen
Rhetorik, welche die Dimension der Teilnahme leugnet, ebenso
begegnen wie der Rhetorik der Manager, welche die Dimension der
Ausbeutung leugnet. Daher auch die Ambivalenz des Verhältnisses zur
Arbeit, zugleich verabscheut, weil sie uns von dem entfremdet, was
wir tun, und verehrt, da sich ein Teil von uns selbst darin
abspielt. Das Desaster ist dabei vorbedingt: es besteht in allem,
das es zu zerstören galt, in all denjenigen, die es zu entwurzeln
galt, damit die Arbeit letzten Endes als einzige Art zu existieren
erschien. Der Horror der Arbeit liegt weniger in der Arbeit selbst
als in der methodischen Verwüstung, seit Jahrzehnten, all dessen,
was sie nicht ist: Vertrautheiten des Stadtteils, des Berufs, des
Dorfes, des Kampfes, der Verwandtschaft, die Verbundenheit mit
Orten, mit Lebewesen, mit Jahreszeiten, mit der Art und Weise zu
handeln und zu sprechen.

Hierin besteht das aktuelle Paradox: Die Arbeit hat restlos über
alle anderen Formen der Existenz triumphiert, genau zu der Zeit, in
der die Arbeiter überflüssig geworden sind. Die Steigerungen der
Produktivität, die Auslagerung, die Mechanisierung, die
Automatisierung und die Digitalisierung der Produktion sind derart
fortgeschritten, dass sie die Menge an lebendiger, zur Herstellung
der Ware benötigter Arbeit, auf beinahe nichts reduziert haben. Wir
erleben das Paradox einer Gesellschaft von Arbeitern ohne Arbeit,
in der die Ablenkung, der Konsum, das Vergnügen nur noch den Mangel
daran beklagen, wovon sie uns eigentlich ablenken sollten. Die Mine
von Carmaux, die ein Jahrhundert lang aufgrund ihrer gewalttätigen
Streiks Berühmtheit erlangte, wurde in den Cap Découverte
umgerüstet. Es handelt sich dabei um ein »multiples
Vergnügungszentrum«, wo Skateboard und Fahrrad gefahren wird, und
das sich durch ein »Minenmuseum« auszeichnet, wo für die Urlauber
Steinschläge simuliert werden.

In den Unternehmen teilt sich die Arbeit immer offensichtlicher in
hochqualifizierte Arbeitsplätze in Forschung, Entwicklung,
Kontrolle, Koordination und Kommunikation, im Zusammenhang mit dem
Einbringen des notwendigen Wissens in die neuen kybernetisierten
Produktionsprozesse, und in entqualifizierte Arbeitsplätze zur
Instandhaltung und Überwachung dieser Prozesse.

Erstere sind von geringer Zahl. Sehr gut bezahlt und dadurch
dermaßen begehrt, dass die Minderheit, die sie vereinnahmt, nicht
auf die Idee käme, sich auch nur ein Krümelchen davon entgehen zu
lassen. Ihre Arbeit und sie verschmelzen in einer angsterfüllten
Umklammerung effektiv zu Einem. Manager, Wissenschaftler,
Lobbyisten, Forscher, Programmierer, Entwickler, Berater und
Ingenieure hören im wahrsten Sinne nie auf zu arbeiten. Selbst ihre
Fickbeziehungen sollen ihre Produktivität steigern. »Die
kreativsten Unternehmen sind zugleich diejenigen, in denen die
Intimbeziehungen am zahlreichsten sind«, theoretisiert ein
Philosoph des Humankapitals\footnote{
»DRH (Direction des Ressources Humaines)«, die Abteilung
Humankapital.
}%18
. »Die Mitarbeiter der Firma«,
bestätigt der Abteilungsleiter Humankapital von Daimler Benz,
»gehören zum Kapital der Firma. […] Ihre Motivation, ihr Know-How,
ihr Innovationsvermögen und ihre Sorge darum, was die Kundschaft
verlangt, sind der Rohstoff für innovative Dienstleistungen. […]
Ihr Verhalten, ihre soziale und emotionale Kompetenz sind von
zunehmender Bedeutung für die Evaluation ihrer Arbeit. […] Diese
wird nicht mehr anhand der Zahl der Anwesenheitsstunden bewertet,
sondern auf Basis der erreichten Ziele und der Qualität der
Resultate. Sie sind Unternehmer.«

Alle Aufgaben, die nicht an die Automation delegiert werden
konnten, bilden ein Nebelfeld von Arbeitsplätzen, die, weil nicht
durch Maschinen besetzbar, mit irgendwelchen Menschen zu besetzen
sind – Handlanger, Lagerarbeiter, Fließbandarbeiter,
Saisonarbeiter, etc.. Diese flexible, undifferenzierte
Arbeitskraft, die von einer Aufgabe zur nächsten wechselt und nie
lange in einem Unternehmen bleibt, kann sich nicht mehr zu einer
Kraft verdichten. Dies, weil sie nie im Mittelpunkt des
Produktionsprozesses steht, sondern wie pulverisiert in einer
Vielzahl von Zwischenräumen damit beschäftigt ist, die Löcher
dessen zu stopfen, was nicht mechanisiert wurde. Der Leiharbeiter
ist die Figur dieses Arbeiters, der keiner mehr ist, der keinen
Beruf mehr hat, sondern Fähigkeiten, die er bei seinen Einsätzen
verkauft, und deren Bereitstellung eine weitere Arbeit ist.

Am Rande des Kerns dieser effektiven, für das reibungslose
Funktionieren der Maschine notwendigen Arbeiterschaft breitet sich
nunmehr eine überzählig gewordene Mehrheit aus, die zwar dem Absatz
der Produktion dient, doch kaum zu mehr, und die das Risiko birgt,
dass sie in ihrer Untätigkeit damit beginnt, die Maschine zu
sabotieren. Die Drohung einer generellen Demobilisierung ist das
Gespenst, welches im gegenwärtigen Produktionssystem umgeht. Auf
die Frage »Warum also arbeiten?« antworten nicht alle wie diese
ehemalige Sozialhilfeempfängerin gegenüber der Libération\footnote{
Tageszeitung
}%19
: »Für
mein Wohlbefinden. Ich musste mich beschäftigen«. Es besteht das
ernsthafte Risiko, dass wir letztendlich dazu kommen, in unserer
Untätigkeit eine Beschäftigung zu finden.

Diese treibende Bevölkerung muss beschäftigt oder im Zaum gehalten
werden. Denn bis heute wurde keine bessere Methode zur
Disziplinierung gefunden als die Lohnarbeit. Deshalb muss der Abbau
der »sozialen Errungenschaften« weitergeführt werden, um die
Widerspenstigsten in den Schoß der Lohnarbeit zu treiben,
diejenigen, die sich nur angesichts der Alternative zwischen
Sterben an Hunger und Verfaulen im Knast ergeben. Die Explosion des
Sklavenhändlersektors der »Personaldienstleistungen« muss
weitergehen: Frauen in Putzjobs, Gastronomie, Massage,
Haushaltshilfe, Prostitution, Pflege, Privatunterricht,
therapeutische Vergnügungen, psychologische Betreuung etc.. All
dies begleitet von sich stetig verschärfenden Standards der
Sicherheit, der Hygiene, des Benehmens und der Kultur, von einer
Beschleunigung der Flüchtigkeit der Moden, die allein die
Notwendigkeit solcher Dienstleistungen sichern. In Rouen haben die
Parkuhren ihren Platz an »menschliche Parkuhren« abgetreten:
Jemand, der sich auf der Straße langweilt, stellt Ihnen einen
Parkschein aus und vermietet Ihnen, gegebenenfalls, bei schlechtem
Wetter einen Regenschirm.

\extrapar{}

Die Ordnung der Arbeit war die Ordnung einer Welt. Die
Offenkundigkeit ihres Ruins schlägt bereits die Idee davon, was
alles daraus folgt, mit Lähmung. Arbeiten bezieht sich heutzutage
weniger auf die wirtschaftliche Notwendigkeit, Waren zu
produzieren, als auf die politische Notwendigkeit, Produzenten und
Konsumenten zu produzieren, um mit allen Mitteln die Ordnung der
Arbeit zu retten. Sich selbst zu produzieren ist dabei, zur
vorherrschenden Beschäftigung einer Gesellschaft zu werden, in der
die Produktion zwecklos geworden ist: wie ein Tischler, dem seine
Werkstatt enteignet wurde und der in seiner Verzweiflung beginnt,
sich selbst zu hobeln. Daher das Spektakel all dieser jungen
Menschen, die für ihr Vorstellungsgespräch lächeln üben, die sich
die Zähne für ihre Beförderung bleichen, die in Clubs gehen, um den
Teamgeist anzuregen, die Englisch lernen, um ihre Karriere zu
boosten, die sich scheiden lassen oder heiraten, um wieder in Gang
zu kommen, die Theater-Workshops besuchen, um Führungskräfte zu
werden, oder Schulungen in »Persönlichkeitsentwicklung«, um
»Konflikte besser zu managen« - »Die intimste
›Persönlichkeitsentwicklung‹«, behauptet irgendso ein Guru, »wird
zu einer besseren emotionalen Stabilität führen, einer größeren
Offenheit für Beziehungen, einer zielgerichteteren intellektuellen
Sinnesschärfe und damit zu einer verbesserten wirtschaftlichen
Leistung.« Das Gewimmel dieser kleinen Welt, die ungeduldig darauf
wartet, ausgewählt zu werden, während sie trainiert natürlich zu
sein, enthüllt den Versuch, die Ordnung der Arbeit durch eine Ethik
der Mobilisierung zu retten. Mobilisiert zu sein heißt sich auf die
Arbeit nicht als Aktivität zu beziehen, sondern als Möglichkeit.
Wenn der Arbeitslose sich seine Piercings entfernt, zum Friseur
geht, und seine Projekte entwickelt, ernsthaft »an seiner
Beschäftigungsfähigkeit« arbeitet, wie man sagt, dann zeugt er
dadurch von seiner Mobilisierung. Die Mobilisierung ist dieses
leichte Ablösen von sich selbst, dieses belanglose Ausreißen von
dem, was uns ausmacht, dieser Zustand der Fremdartigkeit, von dem
aus das Ich als Objekt der Arbeit gehalten werden kann, aus dem
heraus es möglich wird, sich selbst zu verkaufen und nicht seine
Arbeitskraft, sich entlohnen zu lassen, nicht für das, was man tut,
sondern für das, was man ist, für unsere exquisite Beherrschung
sozialer Codes, unsere zwischenmenschlichen Talente, unser Lächeln
oder unsere Art zu präsentieren. Dies ist die neue Norm der
Sozialisierung. Die Mobilisierung führt die Fusion der zwei
widersprüchlichen Pole der Arbeit herbei: Hier nimmt man an seiner
Ausbeutung teil und beutet jegliche Teilnahme aus. Man ist an sich
idealerweise sein kleines Unternehmen, sein eigener Chef und sein
eigenes Produkt. Ob man arbeitet oder nicht, es geht darum, die
Kontakte, Kompetenzen, das »Netzwerk« auszubauen, kurz: das
»Humankapital«. Die planetare Anweisung, sich beim geringsten
Vorwand zu mobilisieren – der Krebs, der »Terrorismus«, ein
Erdbeben, die Obdachlosen – fasst die Entschlossenheit der
herrschenden Mächte zusammen, die Herrschaft der Arbeit über ihr
physisches Verschwinden hinaus aufrecht zu erhalten.

Der gegenwärtige Produktionsapparat ist nun einerseits diese
gigantische Maschine zur physischen und psychischen Mobilisierung,
zum Abpumpen der Energie der überflüssig gewordenen Menschen, und
andererseits diese Sortiermaschine, die den konformen
Subjektivitäten das Überleben gewährt und all die
»Risiko-Individuen« fallen lässt, all jene, die einen anderen
Gebrauch des Lebens verkörpern und ihr somit Widerstand leisten. So
ruft man einerseits die Gespenster ins Leben, und lässt
andererseits die Lebendigen sterben. Dies ist die eigentliche
politische Funktion des gegenwärtigen Produktionsapparats.

\extrapar{}

Sich darüber hinaus und gegen die Arbeit zu organisieren, kollektiv
vom Regime der Mobilisierung zu desertieren, der Existenz einer
Vitalität und einer Disziplin in der Demobilisierung selbst
Ausdruck zu verleihen, ist ein Verbrechen, das diese Zivilisation
in äußerster Bedrängnis nicht bereit ist uns zu vergeben; denn dies
ist die einzige Art, sie zu überleben.

\section{Vierter Kreis}

\satz{»Einfacher, spaßiger, mobiler, sicherer!«}

Erzählt uns nichts mehr von »der Stadt« und »dem Land«, und noch
weniger von ihrer althergebrachten Opposition. Was sich um uns
herum ausbreitet, ähnelt dem weder von nah noch von fern: Eine
einzige urbane Schwade ist es, ohne Form und Ordnung, eine
trostlose Zone, unbestimmt und unbegrenzt, ein weltweites Kontinuum
von musealisierten Mega-Cities und Naturschutzgebieten, von
Hochhaussiedlungen und riesigen Agrarbetrieben, von industriellen
Zonen und Reihenhaussiedlungen, Landgasthöfen und Yuppie-Kneipen:
die Metropole. Da war die antike Stadt, die mittelalterliche Stadt
oder die moderne Stadt; die metropolitane Stadt gibt es nicht. Die
Metropole strebt nach der Synthese aller Territorien. In ihr lebt
alles zusammen, weniger im geographischen Sinn, als durch Verwebung
ihrer Netzwerke.
Gerade weil sie vollends am verschwinden ist, wird die Stadt jetzt
fetischisiert, als Geschichte. Die Manufakturen von Lille werden zu
Sälen für das Spektakel, das zubetonierte Stadtzentrum von Le Havre
ist Weltkulturerbe der Unesco. In Peking werden die Hutongs um die
Verbotene Stadt zerstört und als Imitationen wieder aufgebaut,
etwas weiter weg, für die Neugierigen. In Troyes klebt man
Fachwerk-Fassaden auf Betonbauten. Eine Kunstform der Pastiche, die
einen zwangsläufig an die Boutiquen viktorianischen Stils im
Disneyland Paris erinnert. Die historischen Stadtzentren, einst
Herd des Aufruhrs, finden brav ihren Platz im Organigramm der
Metropole. Sie sind dem Tourismus und dem zur Schau gestellten
Konsum preisgegeben. Sie sind die Inselchen der zauberhaften
Warenwelt, die durch Messen, Ästhetik und auch mit Gewalt aufrecht
erhalten werden. Die Abgeschmacktheiten der Weihnachtsmärkte lassen
sich mit immer mehr Wachleuten und Stadtpolizeistreifen bezahlen.
Die Kontrolle fügt sich wunderbar in die Welt der Waren ein und
zeigt jedem, der es sehen will, ihr autoritäres Gesicht. Trend der
Epoche ist die Durchmischung von fader Musik mit
Teleskopschlagstöcken und Zuckerwatte. Was sie an polizeilicher
Überwachung bedarf, diese Bezauberung!

Dieser Geschmack für das Authentische-in-Anführungszeichen und die
damit einhergehende Kontrolle begleitet das »Kleinbürgertum« bei
ihrer Kolonisierung der Arbeiterviertel. Vertrieben aus den Zentren
der Mega-Cities sucht es dort das »nachbarschaftliche Miteinander«,
das es inmitten von Einfamilienhäusern niemals finden wird. Und
indem es die Armen, die Autos und die Immigranten vertreibt, indem
es den Platz aufräumt, indem es alle Mikroben auszupft,
pulverisiert es genau das, was es eigentlich gesucht hat. Auf einem
Werbeplakat der Stadt gibt ein Strassenfeger einem Security die
Hand; ein Slogan: »Montauban, saubere Stadt«.

Der Anstand, der die Urbanisten zwingt, nicht mehr von »der Stadt«
zu reden, die sie zerstört haben, sondern »vom Urbanen«, sollte sie
auch anstacheln, nicht mehr von »dem Land« zu reden, das nicht mehr
existiert. Was es an seiner Stelle noch gibt, ist eine Landschaft,
die den gestressten und entwurzelten Massen zur Schau stellt wird,
eine Vergangenheit, die jetzt, da die Bauern auf so wenige
reduziert worden sind, inszeniert werden kann. Das über einem
»Territorium« ausgebreitete Marketing, wo alles in Wert gesetzt
werden muss und alles ein Erbe darstellen soll. Es ist immer
dieselbe eisige Leere, die sich bis zum hintersten und letzten
Kirchturm entfaltet.

Die Metropole ist der gleichzeitige Tod der Stadt und des Landes an
der Kreuzung, an der alle Mittelklassen zusammenlaufen, in diesem
Milieu der Klasse-der-Mitte, das sich in der Landflucht
um-die-Urbanisation-herum unendlich räkelt. Der Verglasung des
globalen Territoriums folgt passgenau der Zynismus der
zeitgenössischen Architektur. Ein Gymnasium, ein Krankenhaus und
eine Mediothek sind verschiedene Variationen ein und desselben
Themas: Transparenz, Neutralität, Uniformität. Massive und
fließende Gebäude, die geplant werden, ohne ihren genauen Zweck
kennen zu müssen, die hier sein könnten, genau so gut wie überall
sonst. Was sollen wir mit den Bürotürmen der Défense, der Part Dieu
oder der Euralille machen?\footnote{
Stadtteile in Paris, Lyon und Lille, in denen fast
ausschließlich Bürogebäude stehen.
}
Der Ausdruck »brandneu« beinhaltet ihr
ganzes Schicksal. Ein schottischer Reisender beschreibt, nachdem
die Aufständischen im Mai 1871 das Rathaus in Paris angezündet
hatten, die einzigartige Herrlichkeit der Macht in Flammen: »[…]
niemals hatte ich mir etwas Schöneres vorgestellt; es ist
wunderbar. Die Leute der Commune sind schreckliche Schurken, das
leugne ich nicht; aber welche Künstler! Und sie waren sich ihres
Werks noch nicht mal bewusst! […] Ich habe die Ruinen von Amalfi
gesehen, wie sie in den blauen Fluten des Mittelmeers baden, die
Ruinen der Tempel Tung-hoor im Punjab gesehen; ich habe Rom gesehen
und viel mehr: nichts davon kann mit dem verglichen werden, was ich
heute Abend vor Augen hatte.“

\extrapar{}

Es bleiben sehr wohl einige Fragmente der Stadt und einige
Abfallprodukte des Landes in der metropolitanen Verwebung bestehen.
Aber das Lebendige hat sein Quartier in den Orten des totalen
Ausschlusses aufgeschlagen. Das Paradox will, dass die
augenscheinlich unbewohnbarsten Orte die einzigen noch in
irgendeiner Art und Weise bewohnten sind. Eine alte besetzte
Baracke wird immer einen viel bewohnteren Eindruck machen als all
diese steifen Apartments, wo man seine Möbel hinstellen und die
Ausstattung perfektionieren kann, während man auf den nächsten
Umzug wartet. Die Slums sind in vielen Mega-Cities die letzten
lebendigen und lebenswerten Orte; aber auch, was keine Überraschung
ist, die tödlichsten. Sie sind die Rückseite des elektronischen
Bühnenbilds der Weltmetropole. Die Schlafstädte der Banlieue
nördlich von Paris, verlassen von einem Kleinbürgertum auf der
Jagd nach Traumhäusern, wieder zum Leben erwacht durch die
Massenarbeitslosigkeit, strahlen nun viel intensiver als das
„Quartier Latin“. Durch die Worte genauso wie durch das Feuer.
Die Brände vom November 2005 sind nicht aus der extremen Enteignung
geboren, wie es so oft dahergeredet wurde, sondern aus dem
vollständigen Besitz eines Territoriums. Man kann Autos aus
Langeweile anzünden, aber um einen Aufstand über einen Monat zu
verbreiten und die Polizei dauerhaft in Schach zu halten, muss man
sich zu organisieren wissen, man muss auf Komplizenschaften zählen,
das Territorium perfekt kennen, eine Sprache teilen und einen
gemeinsam Feind haben. Weder die Kilometer noch die Wochen konnten
die Ausbreitung des Feuers verhindern. Den ersten Feuergluten wurde
mit anderen geantwortet, da, wo sie am wenigsten erwartet wurden.
Das Geraune kann nicht abgehört werden.

\extrapar{}

Die Metropole ist das Terrain eines unaufhörlichen Konflikts
niedriger Intensität, dessen Höhepunkte die Einnahme von Basra,
Mogadischu oder Nablus sind. Lange war die Stadt für die Militärs
ein Ort, dem es auszuweichen oder den es zu belagern galt. Die
Metropole ist selbst mit dem Krieg völlig kompatibel. Der
bewaffnete Konflikt ist nur ein Moment ihrer permanenten
Umgestaltung. Die von den Großmächten geschlagenen Schlachten
ähneln der immer zu wiederholenden Polizeiarbeit in den schwarzen
Löchern der Metropole - »ob in Burkina Faso, der Süd-Bronx, in
Kamagasaki, in Chiapas oder in La Courneuve«. Die »Polizeieinsätze«
zielen weder auf den Sieg, noch darauf, Ordnung und Frieden wieder
herzustellen, sondern auf das Fortsetzen einer
Sicherheitsunternehmung, die immer schon am Werk ist. Der Krieg
kann nicht mehr isoliert werden in der Zeit, denn er teilt sich auf
in eine Reihe militärischer und polizeilicher Mikro-Operationen, um
die Sicherheit zu gewährleisten.
Die Polizei und das Militär passen sich parallel zueinander Schritt
für Schritt an. Ein Kriminologe verlangt von der CRS, sich in
kleinen mobilen und professionalisierten Einheiten zu organisieren.
Die militärische Institution, Wiege der disziplinarischen Methoden,
stellt ihre hierarchische Organisation in Frage. Ein Offizier der
NATO wendet auf sein Grenadier-Bataillon eine »partizipative
Methode« an, »die jeden einzelnen in Analyse, Vorbereitung,
Ausführung und Evaluation einer Aktion einbezieht. Tagelang wird
der Plan wieder und wieder diskutiert, im Verlauf der Übung und
nach Erhalt der letzten Nachrichten. […] Es gibt nichts Besseres
als einen gemeinsam erarbeiteten Plan, um Motivation und
Zielstrebigkeit zu fördern«.

Die bewaffneten Kräfte passen sich nicht nur an die Metropole an,
sie bilden sie auch. So die israelischen Soldaten, die sich seit
der Schlacht um Nablus zu Innenarchitekten machen. Von der
palästinensischen Guerilla gezwungen, die Straßen zu verlassen,
weil sie zu gefährlich sind, lernen sie, wie man horizontal und
vertikal innerhalb der urbanen Bauwerke vorankommt, indem sie
Mauern und Decken sprengen, um sich zu bewegen. Ein Offizier der
israelischen Verteidigungskräfte mit einem Diplom in Philosophie
erklärt: »Der Feind interpretiert den Raum in klassischer und
traditioneller Weise und ich weigere mich, seiner Interpretation zu
folgen und in seine Fallen zu tappen. […] Ich will ihn überraschen!
Hierin liegt die Essenz des Krieges. Ich muss gewinnen. […] Deshalb
habe ich mir die Methode ausgesucht, die mich die Mauern
durchqueren lässt... Wie ein Wurm, der sich fortbewegt, indem er
alles, was er auf dem Weg findet, frisst.« Das Urbane ist mehr als
die Bühne der Konfrontation, es ist ein Mittel dafür. All dies
nicht ohne an Blanquis Tips zu erinnern, diesmal an die Partei des
Aufstands, der den zukünftigen Aufständischen in Paris empfahl, die
Häuser in den verbarrikadierten Straßen zu besetzen, um ihre
Positionen zu schützen, dann die Mauern zu durchbrechen, um sie zu
verbinden, die Treppen im Erdgeschoss zu zerstören und die Decken
zu durchlöchern, um sich gegebenenfalls gegen Angreifer zu
verteidigen, die Türen herauszureißen, um damit die Fenster zu
verbarrikadieren und aus jedem Stockwerk einen Schützenposten zu
machen.

\extrapar{}

Die Metropole ist nicht nur diese urbanisierte Anhäufung, dieses
finale Aufeinanderprallen zwischen der Stadt und dem Lande, sie ist
gleichermaßen ein Fluss von Wesen und Dingen. Ein Strom, der durch
ein ganzes Netzwerk aus Fiberglasleitungen, TGV-Linien, Satelliten
und Videoüberwachungskameras fließt, damit diese Welt nie aufhört
ihrem Untergang hinterherzurennen. Ein Strom, der in seiner
hoffnungslosen Mobilität alles mitreißen will, der jeden
mobilisiert. Wo man von Informationen wie von genau so vielen
feindlichen Kräften angegriffen wird. Wo nichts anderes bleibt, als
zu rennen. Wo es schwierig wird zu warten, selbst auf die x-te
U-Bahn.
Die Vervielfältigung der Transport- und Kommunikationsmittel
entreißt uns unablässig dem Hier und Jetzt, durch die Verführung,
immer woanders zu sein. Einen TGV, eine RER oder ein Telefon
nehmen, um bereits dort zu sein. Diese Mobilität beinhaltet nur
Zerrissenheit, Isolation und Exil. Sie wäre für jedermann
unerträglich, wenn sie nicht auch gleichzeitig die Mobilität des
privaten Raumes wäre, des tragbaren Inneren. Die private Blase
platzt nicht, sie beginnt zu treiben. Dies ist nicht das Ende des
Cocooning, nur sein In-Bewegung-Setzen. Von einem Bahnhof, einem
Einkaufszentrum, einer Geschäftsbank oder einem Hotel zum anderen;
überall diese Befremdung, so banal, so bekannt, dass sie die letzte
Vertrautheit ersetzt.

\extrapar{}

Die Üppigkeit der Metropole ist die zufallsbedingte Mischung
definierter Stimmungen, in der Lage, sich unendlich wieder aufs
Neue zusammenzusetzen. Dabei bieten sich die Stadtzentren nicht als
identische Orte an, sondern als eigenartige Angebote von
Stimmungen, in denen wir uns bewegen, nach Lust und Laune die eine
auswählend, die andere verlassend, in einer Art existentiellem
Shopping des Stils der Bar, der Leute, des Designs oder der
playlist eines ipod. »Mit meinem MP3-Player bin ich der Herr meiner
Welt.« Die einzige Möglichkeit, die umgebende Uniformität zu
überleben, ist es, die eigene innere Welt unaufhörlich wieder
herzustellen, wie ein Kind, das überall wieder die gleiche Hütte
aufbaut. Wie Robinson, der sein Tante-Emma-Laden-Universum auf der
einsamen Insel reproduziert, mit dem Unterschied, dass unsere
einsame Insel die Zivilisation selbst ist, und dass Milliarden von
uns unaufhörlich dort landen.
Eben weil sie diese Architektur der Flüsse ist, ist die Metropole
eine der verwundbarsten menschlichen Formationen, die jemals
existiert hat. Biegsam, subtil, aber verwundbar. Eine brutale
Schließung der Grenzen aufgrund einer wütenden Epidemie, irgendein
Mangel in der lebenswichtigen Versorgung, eine organisierte
Blockade der Kommunikationswege, und das gesamte Bühnenbild bricht
zusammen, schafft es nicht mehr, die Szenen des Gemetzels zu
verdecken, die es zu jeder Zeit heimsuchen. Diese Welt wäre nicht
so schnell, wenn sie nicht stetig von der Nähe ihres eigenen
Zusammenbruchs verfolgt würde.

Ihre netzartige Struktur, all ihre technologische Infrastruktur der
Knoten und Verbindungen, ihre dezentralisierte Architektur möchte
die Metropole vor den unvermeidlichen Betriebsstörungen schützen.
Das Internet muss einem nuklearen Angriff standhalten. Die
permanente Kontrolle der Flüsse von Informationen, Menschen und
Waren muss die metropolitane Mobilität sichern, die
Rückverfolgbarkeit, sicherstellen, dass niemals eine Palette im
Warenbestand fehlt, dass niemals ein gestohlener Geldschein auf dem
Markt zu finden ist oder ein Terrorist im Flugzeug. Dank einem
RFID-Chip, einem biometrischen Pass, einer DNA-Datenbank.

Aber die Metropole produziert auch die Mittel ihrer eigenen
Zerstörung. Ein amerikanischer Sicherheitsexperte erklärt die
Niederlage im Irak mit der Fähigkeit der Guerilla, aus den neuen
Kommunikationsmethoden Profit zu schlagen. Mit ihrer Invasion haben
die USA weniger die Demokratie als die kybernetischen Netzwerke
eingeführt. Mit sich brachten sie eine der Waffen ihrer Niederlage.
Die Vervielfachung der Handies und Internetzugänge haben die
Guerilla mit neuartigen Mitteln versorgt, sich zu organisieren und
sich selbst so schwer angreifbar zu machen.

Jedes Netzwerk hat seine Schwachpunkte, Knoten, die aufgemacht
werden können, um die Zirkulation zu stoppen, um das Netz
implodieren zu lassen. Der letzte große europäische Stromausfall
hat es gezeigt: Ein Zwischenfall auf einer Hochspannungsleitung
reichte, um einen Großteil des Kontinents ins Dunkel zu stürzen.
Die erste Geste, damit etwas mitten in der Metropole hervorbrechen
kann, damit sich andere Möglichkeiten eröffnen, besteht darin ihr
Perpetuum Mobile zu stoppen. Das ist es, was die thailändischen
Rebellen verstanden haben, die Umspannwerke hochgehen lassen. Das
ist es, was die Anti-CPE\footnote{
Bewegung u.a. gegen die Einführung eines Arbeitsvertrages für
Menschen unter 25 Jahren, die fristlose Kündigungen ohne jede
Begründung innerhalb einer Probezeit von 2 Jahren vorsah. Das CPE
war Teil eines Maßnahmenpakets, das die Regierung in Reaktion auf
die Revolte von 2005 erlassen wollte, das auch eine Verschärfung
des Aufenthaltsrechts beinhaltete. Das CPE wurde zurückgeschlagen
und führte zur Politisierung einer ganzen Generation.
}
verstanden haben, die die Universitäten
blockierten, um dann zu versuchen, die Wirtschaft zu blockieren.
Das ist es auch, was die im Oktober 2002 streikenden amerikanischen
Hafenarbeiter verstanden haben, die für den Erhalt von 300
Arbeitsplätzen zehn Tage lang die wichtigsten Häfen der West-Küste
blockierten. Die amerikanische Wirtschaft ist von den
kontinuierlichen Flüssen aus Asien derart abhängig, dass sich die
Kosten der Blockade auf eine Milliarde Euro am Tag beliefen. Mit
zehntausend Leuten kann man die größte Wirtschaftsmacht ins Wanken
bringen. Hätte die Bewegung noch einen Monat länger gedauert, wäre
es laut mancher »Experten« zu »einer Rückkehr der USA in die
Rezession und einem wirtschaftlichen Albtraum für Süd-Ost-Asien«
gekommen.

\section{Fünfter Kreis}

\satz{»Weniger Güter, mehr Bindungen!«}

Dreißig Jahre Massenarbeitslosigkeit, »Krise« und Wachstum auf
Halbmast, und noch immer sollen wir an die Wirtschaft glauben.
Dreißig Jahre, die zugegebenermaßen durch ein paar illusorische
Pausen betont wurden: die Pause von 1981 – 83, die Illusion, dass
eine linke Regierung das Glück des Volkes herbeiführen könnte; die
Pause der »Jahre des Geldes« (1986 – 89), in denen es hieß, dass
wir alle reich, Geschäftsleute und Börsianer geworden waren, die
Pause des Internet (1998 – 2001), in denen es hieß, dass wir alle
eine virtuelle Stelle gefunden haben, weil wir alle total auf Draht
waren, in der das vielfarbige, aber geeinte Frankreich,
multikulturell und kultiviert, alle Weltmeisterschaften gewinnen
würde.\footnote{
1998 hatte Frankreich die Fußball-WM ausgerichtet und gewonnen.
Die Euphorie des Sieges wurde genutzt, um kurzzeitig die verlogene
Idee eines mulitkulturellen Frankreich, des »France black blanc
beur« zu propagieren.
}
Aber die Wahrheit ist, dass wir schon alle Ersparnisse an
Illusionen ausgegeben haben; wir sind am Boden, wir sind blank,
wenn nicht sogar im Minus.
Gezwungenermaßen haben wir folgendes verstanden: Nicht die
Wirtschaft ist in der Krise, die Wirtschaft ist die Krise; die
Arbeit fehlt nicht, die Arbeit ist zuviel; wohl überlegt deprimiert
uns nicht die Krise, sondern das Wachstum. Wir müssen zugeben: Die
Litanei der Börsenkurse berührt uns ungefähr so viel wie eine Messe
in Latein. Glücklicherweise sind wir eine gewisse Anzahl an Leuten,
die zu diesem Schluss gekommen sind. Dabei reden wir nicht von all
denen, die von diversen Betrügereien leben, von Machenschaften
aller Art, oder die seit zehn Jahren Sozialhilfe beziehen. Auch
nicht von all denen, die sich nicht mehr mit ihrem Job
identifizieren können und sich für die Freizeit aufsparen. Von all
denen, die aufs Abstellgleis geschoben wurden, all den
Drückebergern, all denjenigen, die ein Minimum machen und ein
Maximum sind. Von all denen, die mit dieser seltsamen
Gleichgültigkeit der Masse geschlagen sind, die durch das Beispiel
der Rentner oder die zynische Überausbeutung einer flexibilisierten
Arbeitskraft verstärkt wird. Wir reden nicht von ihnen, auch wenn
sie auf die ein oder andere Weise zu einem ähnlichen Schluss kommen
müssten.

Das, wovon wir reden, das sind all diese Länder, diese ganzen
Kontinente, die den wirtschaftlichen Glauben verloren haben, da sie
die Boeings des IWF mit Krach und Verderben haben vorbeidonnern
sehen, weil sie ein bisschen von der Weltbank probiert haben.
Nichts ist dort zu erkennen von der Krise der Berufung, welche die
Wirtschaft im Abendland erdulden muss. Das, worum es in Guinea, in
Russland, in Argentinien, in Bolivien geht, ist die gewaltsame und
dauerhafte Diskreditierung dieser Religion und ihres Klerus. »Was
sind tausend Ökonomen des IWF, die auf dem Meeresboden liegen? -
Ein guter Anfang«, wird bei der Weltbank gespottet. Ein russischer
Witz: Treffen sich zwei Ökonomen. Fragt einer den anderen:
»Verstehst du, was passiert?« Antwortet der andere: »Warte, ich
erklär es dir.« - »Nein nein« widerspricht der erste, »erklären ist
nicht schwer, ich bin auch Ökonom. Was ich dich frage, ist:
Verstehst du es?« Teile des Klerus selbst heucheln, dass sie
abtrünnig geworden wären und das Dogma kritisieren würden. Die
letzte noch etwas lebendige Strömung der so genannten
»Wirtschaftswissenschaft« - eine Strömung, die sich ohne Witz
»nicht autistische Wirtschaft« nennt – hat heute den Schwerpunkt,
die Anmaßungen, die Zaubertricks, die gefälschten Zahlen einer
Wissenschaft zu demontieren, deren einzig greifbare Rolle darin
besteht, die Monstranz vor den Hirngespinsten der Mächtigen
herzutragen, ihre Aufrufe zur Unterwerfung mit ein bisschen
Zeremonie zu umgeben und endlich, wie es die Religionen immer
gemacht haben, Erklärungen zu liefern. Denn das allgemeine Unglück
wird unerträglich, sobald es als das erscheint, was es wirklich
ist: ohne Grund und Ursache.

\extrapar{}

Nirgendwo wird das Geld mehr respektiert, weder von denen, die es
haben, noch von denen, denen es fehlt. Zwanzig Prozent der jungen
Deutschen antworten, wenn man sie fragt, was sie später machen
wollen: »Künstler«. Die Arbeit wird nicht mehr als Gegebenheit
menschlichen Daseins ausgehalten. Die Buchführung der Firmen gibt
zu, dass sie nicht mehr weiß, wo der Wert entsteht. Sein schlechter
Ruf hätte den Markt seit einem guten Jahrzehnt überwunden, wären da
nicht die Wut und die umfangreichen Mittel seiner Apologeten. Der
Fortschritt ist überall im allgemeinen Verständnis zum Synonym von
Desaster geworden. In der Welt der Wirtschaft flüchtet alles, wie
in der UdSSR in der Epoche von Andropov alles flüchtete. Wer sich
ein bisschen mit den letzten Jahren der UdSSR auseinandergesetzt
hat, wird in den ganzen Aufrufen unserer Regierenden zum
Voluntarismus, in all den lyrischen Ausschweifungen über eine
Zukunft, von der wir jede Spur verloren haben, in all diesen
Glaubensbekenntnissen zur »Reform« von allem und irgendwas ohne
Mühe das erste Knacken in der Struktur der Mauer hören. Der
Zusammenbruch des Sozialistischen Blocks hat nicht den Triumph des
Kapitalismus verankert, sondern nur das Scheitern einer seiner
Formen bewiesen. Übrigens war die Tötung der UdSSR nicht die Tat
eines Volkes im Aufstand, sondern einer Nomenklatura in
Umstrukturierung. Indem sie das Ende des Sozialismus proklamierte,
befreite sich eine Fraktion der herrschenden Klasse zunächst von
allen anachronistischen Aufgaben, die sie mit dem Volk verbanden.
Sie übernahm die private Kontrolle über das, was sie zuvor im Namen
aller kontrollierte. »Da sie so tun, als würden sie uns bezahlen,
tun wir so, als würden wir arbeiten«, wurde in den Fabriken gesagt.
»Wenn dem so ist, hören wir auf, so zu tun« antwortete die
Oligarchie. Für die einen die Rohstoffe, die industrielle
Infrastruktur, der militärisch-industrielle Komplex, die Banken und
die Nachtklubs und für die anderen das Elend oder die Emigration.
So wie man in der UdSSR unter Andropov nicht mehr geglaubt hat, so
glaubt man heute in Frankreich in den Sitzungssälen, in den
Ateliers und in den Büros nicht mehr. »Wie dem so ist!«, antworten
die Chefs und die Regierenden, die sich nicht mal mehr die Mühe
machen, die »harten Gesetze der Wirtschaft« zu mildern, die eine
Fabrik über Nacht räumen und der Belegschaft am frühen Morgen die
Schließung bekannt geben, die nicht mehr zögern die
Spezialeinsatzkommandos zu schicken, um einen Streik zu beenden –
wie beim Streik bei der korsischen Schifffahrtsgesellschaft SNCM\footnote{
Staatliche, d.h. französische Gesellschaft mit Monopol auf den
Passagierverkehr zwischen Frankreich und Korsika. Im Sommer 2005
wurde ein Schiff der SNCM von sieben korsischen Arbeitern nach
Korsika »zurückgebracht«, um die drohende Privatisierung nach
EU-Recht zu verhindern. Das Schiff wurde gestürmt.
}
oder bei der Besetzung des Postverteilzentrums in Rennes. Die ganze
mörderische Aktivität der gegenwärtigen Macht besteht einerseits
darin, diese Ruine zu verwalten, und andererseits die Basis für
eine »Neue Wirtschaft« zu legen.
\extrapar{}

Wir hatten uns doch ganz schön dran gewöhnt, an die Wirtschaft.
Seit Generationen werden wir diszipliniert, befriedet, wurden aus
uns Untertanen gemacht, auf natürliche Art produktiv, zufrieden mit
dem Konsum. Und dann enthüllt sich alles, das wir uns bemüht hatten
zu vergessen: dass die Wirtschaft eine Politik ist. Und dass diese
Politik heute eine Politik der Selektion der Menschheit ist, die in
ihrer Masse überflüssig geworden ist. Von Colbert zu De Gaulle und
vorbei bei Napoléon III hat der Staat die Wirtschaft immer als
Politik wahrgenommen, nicht weniger als die Bourgeoisie, die ihren
Profit daraus zieht, und die Proletarier, die sie bekämpfen. Bloß
diese seltsame Zwischenschicht der Bevölkerung, diese komische
kraftlose Anhäufung aus denen, die nicht Partei ergreifen, das
»Kleinbürgertum«, das immer so getan hat, als würde es an die
Wirtschaft wie an eine Realität glauben – weil es dadurch seine
Neutralität aufrecht erhalten konnte. Es sind die kleinen Händler,
kleinen Chefs, kleinen Funktionäre, Führungskräfte, Professoren,
Journalisten und Zwischengeschaltete
aller Art, die in Frankreich diese Nicht-Klasse bilden, diese
soziale Gallerte, die aus der Masse derer besteht, die einfach ihr
kleines Privatleben außerhalb der Geschichte und ihrer Tumulte
verbringen möchten. Dieser Sumpf ist per Veranlagung der
Weltmeister des falschen Gewissens, zu allem bereit, um die Augen
in seinem Halbschlaf geschlossen zu halten vor dem Krieg, der
rundherum tobt. Jede Erhellung der Front ist in Frankreich
gezeichnet von der Erfindung einer neuen Schrulle. In den letzten
zehn Jahren war dies ATTAC\footnote{
\foreignlanguage{french}{24 Association pour Taxation des Transaction 
financières et pour l´Action Citoyenne}
}
mit ihrer unglaublichen Tobin-Steuer –
deren Einführung nichts Geringeres bedarf als die Schaffung einer
Weltregierung - mit ihrer Apologie der »Realwirtschaft« im
Gegensatz zu den Finanzmärkten und ihrer berührenden Nostalgie für
den Staat. Die Komödie dauerte, solange sie dauerte, und endete in
platter Maskerade. Eine Schrulle ersetzt die andere, es folgt die
Wachstumsrücknahme\footnote{
»decroissance«
}%25
. Wenn ATTAC mit ihren Abendkursen versuchte,
die Wirtschaft als Wissenschaft zu retten, dann behauptet die
Wachstumsrücknahme, sie als Moral zu retten. Die einzige
Alternative zur vorrückenden Apokalypse: zurücknehmen. Konsumieren
und weniger produzieren. Mit Freuden genügsam werden. Bio essen,
mit dem Fahrrad fahren, aufhören zu rauchen und alle Produkte
streng kontrollieren, die gekauft werden. Sich mit dem absolut
Nötigen zufrieden geben. Freiwillige Anspruchslosigkeit. »Den
wahren Reichtum entdecken im Aufblühen von geselligen sozialen
Beziehungen in einer gesunden Welt.« »Aus unserem natürlichen
Kapital nichts abschöpfen.« Hin zu einer »gesunden Wirtschaft«.
»Der Regulierung durch das Chaos zuvorkommen.« »Keine soziale Krise
generieren, die Demokratie und Humanismus in Frage stellt.« Kurz:
Wirtschafter werden. Zurück zur Ökonomie von Papa, ins goldene
Zeitalter des Kleinbürgertums: die Fünfziger Jahre. »Wenn das
Individuum ein guter Wirtschafter wird, dann erfüllt dessen
Eigentum genau seinen Zweck, ihm zu ermöglichen, sein eigenes Leben
zu genießen, abseits der öffentlichen Existenz oder in der privaten
Einfriedung seines Lebens.«

\extrapar{}

Ein Grafik-Designer im handgemachten Pullover trinkt auf der
Terrasse eines Ethno-Cafés unter Freunden einen Frucht-Cocktail.
Man ist redegewandt, herzlich, man macht kleine Späße, man macht
nicht zuviel Lärm und ist auch nicht zu still, man schaut sich mit
einem Lächeln an, etwas selbstgefällig: man ist so zivilisiert.
\extrapar{}

Später werden die einen die Erde eines Stadtgartens etwas
auflockern, während die andern ein bisschen Töpfern, etwas Zen
machen oder einen Animationsfilm drehen. Man zelebriert die
Kommunion im berechtigten Gefühl, eine neue Menschheit zu bilden,
die weiseste, raffinierteste, die letzte. Und man hat recht. Apple
und die Wachstumsrücknahme sind sich erstaunlich einig über die
Zivilisation der Zukunft. Die Idee der einen, von der Rückkehr zur
Wirtschaft von einst, ist der günstige Nebel, in dem die Idee der
anderen vom großen technologischen Sprung voranschreitet. Denn in
der Geschichte gibt es keine Rückkehr. Die Mahnrufe, in die
Vergangenheit zurückzukehren, stellen niemals etwas anderes dar als
eine der Formen des Bewusstseins der Zeit, und selten des
modernsten. Die Wachstumsrücknahme ist nicht zufällig das Banner
der dissidenten Werbemanager der Zeitschrift Casseurs de pub.\footnote{
Gruppen, die sich auf unterschiedlichste Art und Weise gegen die
allgegenwärtige Werbung wehren.\\
\texttt{http://www.dailymotion.com/video/x6cgln\_casseurs-anti-pub\_news}
}
Die
Erfinder des Nullwachstums – Der Club of Rome 1972 – waren selbst
eine Gruppe von Industriellen und Beamten, die sich auf einen
Bericht von Kybernetikern des MIT\footnote{
Massachusetts Institute of Technology
}
stützten.
Dieses Zusammenkommen ist kein Zufall. Es reiht sich ein ins
erzwungene Hasten, eine Nachfolge für die Wirtschaft zu finden. Der
Kapitalismus hat zum eigenen Profit alles auseinandergebrochen, was
an sozialen Verbindungen noch übrig blieb, und macht sich jetzt an
den neuen Wiederaufbau auf seiner eigenen Grundlage. Die aktuelle
metropolitane Gesellschaftlichkeit ist die Brutstätte dafür. Auf
gleiche Art hat sie die natürlichen Welten verwüstet und macht sich
nun an die verrückte Idee, sie als kontrollierte Umgebungen
nachzubilden, ausgestattet mit geeigneten Sensoren. Dieser neuen
Menschheit entspricht eine neue Wirtschaft, die nicht mehr eine von
der Existenz getrennte Sphäre sein möchte, sondern ihr Gewebe, die
der Stoff der menschlichen Beziehungen sein möchte; eine neue
Definition der Arbeit als Arbeit an sich selbst, und des Kapitals
als Humankapital; eine neue Idee der Produktion als Produktion von
Beziehungsgütern und des Konsums als Konsum von Situationen; und
vor allem eine neue Idee des Werts, die alle Qualitäten der
Lebewesen umfasst. Diese »Bio-Ökonomie« in Vorbereitung begreift
den Planeten als zu verwaltendes geschlossenes System, und gibt
vor, die Basis für eine Wissenschaft zu legen, die alle Parameter
des Lebens integrieren will. Solche Wissenschaft könnte uns eines
Tages die schöne Zeit der trügerischen Statistiken vermissen
lassen, als man noch vorgab, das Glück des Volkes am Wachstum des
BIP messen zu können, aber als wenigstens niemand daran glaubte.

„Die nicht-wirtschaftlichen Aspekte des Lebens wieder aufwerten“
ist zugleich Motto der Wachstumsrücknahme und Reformprogramm des
Kapitals. Öko-Dörfer, Videoüberwachung, Spiritualität,
Biotechnologie und Geselligkeit sind Teil des selben, sich
formierenden »zivilisatorischen Paradigmas«, dem der totalen
Wirtschaft, generiert von Grund auf. Ihre intellektuelle Matrix ist
keine andere als die Kybernetik, die Wissenschaft der Systeme, das
heißt ihrer Kontrolle. Um Wirtschaft mit ihrer Ethik der Arbeit und
der Gier definitiv durchzusetzen, musste man im Laufe des 17.
Jahrhunderts die gesamte Fauna der Müßiggänger, der Bettler, der
Hexen, der Verrückten, der Genießer und weiterer Armer ohne
Schuldbekenntnis einsperren und eliminieren, eine ganze Menschheit,
die allein durch ihre Existenz der Ordnung der Interessen und der
Selbstbeschränkung widersprach. Ohne eine derartige Selektion der
mutationsfähigen Subjekte und Zonen wird sich die neue Wirtschaft
nicht durchsetzen. Das häufig angekündigte Chaos wird die
Gelegenheit für dieses Aussortieren sein, oder unser Sieg über
dieses hassenswerte Projekt.


\section{Sechster Kreis}

\satz{»Die Umwelt ist eine industrielle Herausforderung«}

Die Ökologie ist die Entdeckung des Jahres. Nach dreißig Jahren, in
denen man sie den Grünen überließ, am Sonntag genüsslich darüber
lachte, um am Montag wieder einen betroffenen Ausdruck anzunehmen.
Und jetzt holt sie uns ein. Wie ein Sommerhit erobert sie die
Frequenzen, weil es im Dezember 20 Grad warm ist.
\extrapar{}

Ein Viertel der Fischarten ist aus den Ozeanen verschwunden. Der
Rest hat nicht mehr lange.
Vogelgrippealarm: Es wird versprochen, Zugvögel zu Hunderttausenden
abzuknallen.

In der Muttermilch ist die Quecksilberquote zehnmal höher als die
in der Kuhmilch zugelassene. Und diese Lippen, die anschwellen,
wenn ich in den Apfel beiße – er kam doch vom Markt.

Die einfachsten Gesten sind giftig geworden. Man stirbt mit
fünfunddreißig Jahren »an einer langen Krankheit«, die man
verwalten wird, wie man den ganzen Rest verwaltet hat. Man hätte
die Schlussfolgerungen ziehen sollen, bevor sie uns hierher
bringen, zum Gebäude B der Notfallstation.

Geben wir es zu: diese ganze »Katastrophe«, mit der man uns so laut
unterhält, berührt uns nicht. Zumindest nicht, bevor sie uns mit
einer ihrer vorhersehbaren Konsequenzen schlägt. Sie betrifft uns
vielleicht, aber sie berührt uns nicht. Und das gerade ist die
Katastrophe.

Es gibt keine »Umweltkatastrophe«. Jene Katastrophe ist die Umwelt.
Die Umwelt ist das, was dem Menschen bleibt, wenn er alles verloren
hat. Jene, die einen Stadtteil, eine Straße, ein Tal, einen Krieg,
eine Werkstatt bewohnen, haben keine »Umwelt«, sie bewegen sich in
einer Welt bevölkert von Anwesenheiten, Gefahren, Freunden,
Feinden, Punkten des Lebens und Punkten des Todes, von allerlei
Wesen. Diese Welt hat ihre Konsistenz, die variiert mit der
Intensität und der Qualität der Verbindungen, die uns an all diese
Wesen bindet, an all diese Orte. Wir, Kinder der endgültigen
Enteignung, Verbannte der letzten Stunde – wir, die in Betonwürfeln
auf die Welt kommen, Obst in den Supermärkten pflücken und im
Fernsehen das Echo der Welt belauern – wir sind die einzigen, die
eine Umwelt haben. Wir sind die einzigen, die unserer eigenen
Vernichtung zusehen, als ginge es um einen simplen
Stimmungswechsel. Die sich über die letzten Fortschritte des
Desasters empören und mit Geduld seine Enzyklopädie
zusammenstellen.

\extrapar{}

Was sich als Umwelt herauskristallisiert, ist ein Verhältnis zur
Welt, das auf Verwaltung, also auf Fremdheit aufbaut. Ein
Verhältnis zur Welt, in dem wir nicht mehr ebenso gut aus dem
Rascheln der Bäume, dem Fritiergeruch der Gebäude, dem Rieseln des
Wassers, dem Getöse des Schulunterrichts oder der Schwüle der
Sommerabende bestehen, ein Verhältnis zur Welt, in dem es mich und
meine Umwelt gibt, die mich umgibt, ohne mich jemals auszumachen.
Wir sind zu Nachbarn in einer planetaren
Wohnungseigentümerversammlung geworden. Man kann sich kaum eine
wahrhaftigere Hölle vorstellen. Kein materielles Milieu hat jemals
den Namen »Umwelt« verdient, heute vielleicht mit Ausnahme der
Metropole. Die digitalisierte Stimme des Ansagers, das Kreischen
der Straßenbahn des 21. Jahrhunderts, das bläuliche Licht der
Straßenlaternen in der Form riesiger Streichhölzer, Fußgänger in
Verkleidung gescheiterter Models, stiller Schwenk einer
Videoüberwachungskamera, klares Klappern der Schranken in der
Metro-Station, Supermarktkassen, Stechuhren, elektronische
Internetcafé-Stimmung, der Überfluss an Plasmabildschirmen,
Schnellstraßen und Latex. Niemals war ein Bühnenbild ohne die es
durchquerenden Seelen ausgekommen. Kein Milieu war je
automatischer. Kein Kontext war je gleichgültiger und verlangte
dafür, um darin zu überleben, eine gleichmäßige Gleichgültigkeit
zurück. Die Umwelt ist schließlich nichts anderes als das: das
eigenartige Verhältnis zur Welt, das sich auf alles projeziert, was
ihr entgleitet.
\extrapar{}

Die Situation ist folgende: man hat sich unserer Eltern bedient, um
diese Welt zu zerstören, nun möchte man uns an ihrem Wiederaufbau
arbeiten lassen, und der soll noch dazu profitabel sein. Die
morbide Erregung, die Journalisten und Werbemanager bei jedem neuen
Beweis für die Klimaerwärmung erfasst, enthüllt das eiserne Lächeln
des neuen grünen Kapitalismus, jenes, der sich seit den 1970ern
ankündigte, auf den man wartete und der nicht kam. Et bien, le
voilá! Die Ökologie, das ist er! Die alternativen Lösungen, das ist
er! Das Heil des Planeten, das ist er immer noch! Kein Zweifel:
grün liegt in der Luft; die Umwelt wird das Drehmoment der
politischen Ökonomie des 21. Jahrhunderts sein. Auf jeden Schub
Katastrophismus folgt eine Salve »industrieller Lösungen«.
Der Erfinder der Wasserstoffbombe, Edward Teller, schlägt vor,
Millionen Tonnen Metallstaub in die Stratosphäre zu zerstreuen, um
die Klimaerwärmung zu stoppen. Die Nasa, frustriert, weil sie ihre
großsartige Idee eines Antiraketenabwehrschildes im Museum der
Fantasien des Kalten Krieges verstauen musste, verspricht die
Errichtung eines riesigen Spiegels jenseits der Mondumlaufbahn, der
uns vor den tödlichen Sonnenstrahlen schützen soll. Eine andere
Zukunftsvision: eine motorisierte Menschheit, die von Sao Paulo
nach Stockholm mit Bioethanol fährt; ein Traum der
Getreidehersteller aus der Beauce\footnote{
Die Beauce ist eine Region südlich von Paris, in der einige
Großbauern riesige landwirtschaftliche Produktionsflächen
bewirtschaften.
}%28
, welcher alles in allem nichts
anderes als die Umwandlung sämtlichen Ackerlandes des Planeten in
Soja- und Zuckerrübenfelder voraussetzt. Beim Durchblättern der
Seiten der Hochglanzmagazine koexistieren ökologische Autos,
saubere Energie, Environmental Counsulting ohne Schwierigkeit mit
der neusten Chanel-Werbung.

Es heißt, dass die Umwelt das unvergleichliche Verdienst hat, das
erste globale Problem zu sein, das sich der Menschheit stellt. Ein
globales Problem, also ein Problem, wofür nur diejenigen die Lösung
haben können, die global organisiert sind. Und diese, die kennen
wir. Es sind dies die Gruppen, die seit fast einem Jahrhundert die
Avantgarde des Desasters sind und fest entschlossen dies zu
bleiben, zum minimalen Preis eines Logo-Wechsels. Dass EDF\footnote{
Staatliche Energiewerke Frankreichs
}%29
die
Unverschämtheit hat, uns ihr Atomprogramm als neue Lösung für die
weltweite Energiekrise wieder aufzutischen, sagt genug darüber, wie
sehr die neuen Lösungen den alten Problemen ähneln.

Von den Büros der Staatssekretäre zu den Hinterzimmern der
alternativen Cafés wird die Besorgnis in den gleichen Worten
geäußert, die eigentlich die selben wie immer sind. Es geht darum
zu mobilisieren. Nicht für den Wiederaufbau wie nach dem Krieg,
nicht für die Äthiopier wie in den 1980er Jahren, nicht für den
Arbeitsplatz wie in den 1990ern. Nein, diesmal ist es für die
Umwelt. Sie dankt es euch so sehr. Al Gore, die Ökologie à la
Hulot\footnote{
Nicolas Hulot war lange Fernseh-Moderator für ein Umweltprogramm
namens »Ushuaia« - eine bedrohte Insel irgendwo – später trat er in
eine der grünen Partei Frankreichs ein. Ushuaia tauchte etwas
später als Name einer bekannten Körperpflegeserie wieder auf.
}%30
und die Wachstumsrücknahme stellen sich an die Seite der
ewigen großen Seelen der Republik, um ihre Rolle der Wiederbelebung
des Volkes der kleinen Leute und des wohlbekannten Idealismus der
Jugend zu spielen. Indem sie ihre Fahne der freiwilligen
Selbstbeschränkung schwenken, arbeiten sie freiwillig und konform
zum »kommenden ökologischen Ausnahmezustand«. Die runde und
klebrige Masse ihrer Schuld fällt auf unsere müden Schultern und
möchte uns dazu antreiben, unseren Garten zu bepflanzen, unseren
Abfall zu trennen, die Reste des makaberen Festmahls, in dem und
für das wir aufgepeppelt wurden, biologisch zu kompostieren.

Den Ausstieg aus der Atomenergie, den CO2-Überschuss in der
Atmosphäre, den Gletscherschwund, die Orkane, die Epidemien, die
weltweite Überbevölkerung, die Bodenerosion, das massive
Verschwinden der lebenden Spezies verwalten... dies wird unsere
Bürde sein. »Es liegt an jedem Einzelnen, sein Verhalten zu
ändern«, sagen sie, wenn wir unser schönes Zivilisationsmodell
retten wollen. Es muss wenig konsumiert werden, um noch konsumieren
zu können. Biologisch produzieren um noch produzieren zu können.
Sich selbst zwingen, um noch zwingen zu können. Und so mag die
Logik einer Welt überleben, indem sie sich den Anschein eines
historischen Bruchs gibt. So möchte man uns davon überzeugen, uns
an den vorrückenden industriellen Herausforderungen dieses
Jahrhunderts zu beteiligen. Bekloppt wie wir sind, wären wir
bereit, in die Arme derer zu springen, welche die Verwüstung
angeführt haben, damit sie uns da rausholen.

\extrapar{}

Die Ökologie ist nicht nur die Logik der totalen Ökonomie, sie ist
auch die neue Moral des Kapitals. Der interne Krisenzustand des
Systems und die Unerbittlichkeit der sich abspielenden Selektion
sind so hart, dass erneut ein Kriterium benötigt wird, um in
dessen Namen ein solches Aussortieren durchführen zu können. Die
Idee der Tugend war durch die Epochen nie etwas anderes als die
Erfindung der Liederlichkeit. Ohne die Ökologie könnte die
gegenwärtige Existenz zweier Ernährungsstränge nicht gerechtfertigt
werden, der eine »gesund und biologisch« für die Reichen und ihre
Kinder, der andere bekanntlich schädlich für den Pöbel und ihre zur
Fettleibigkeit verdammten Sprösslinge. Die planetare
Hyper-Bourgeoisie könnte ihren Lebenstil nicht als respektabel
gelten lassen, wären ihre letzten Launen nicht aufs Strengste
»umweltfreundlich». Nichts hätte ohne die Ökologie noch genug
Autorität, jeden Einspruch gegen den immensen Fortschritt der
Kontrolle zum Schweigen zu bringen.
Nachvollziehbarkeit, Transparenz, Zertifizierung, Öko-Steuern,
Umwelt-Exzellenzinitiativen der Regionen und Wasserqualitätspolizei
sind die Vorzeichen des sich ankündigenden ökologischen
Ausnahmezustands. Alles ist einer Macht erlaubt, die im Namen der
Natur, der Gesundheit und des Wohlbefindens agiert.

»Wenn einmal die neue Wirtschafts- und Verhaltenskultur zu den
Sitten gehören wird, werden die Zwangsmaßnahmen ohne Zweifel von
alleine greifen.« Es bedarf der ganzen lächerlichen Dreistigkeit
eines Abenteurers der Fernsehstudios\footnote{
Bezieht sich auf die Programme des Nicolas Hulot, der auf der
Spur der Umweltkatastrophe um die Welt reiste.
}%31
, um eine derart einfrierende
Perspektive zu verfechten, uns gleichzeitig dazu aufzufordern,
genügend »Planetenschmerz« aufzuweisen, um uns zu mobilisieren und
ausreichend betäubt zu bleiben, um alledem zurückhaltend und
zivilisiert zuzusehen. Der neue Bio-Asketismus ist die
Selbstkontrolle, die von allen verlangt wird, um über die
Rettungsoperation zu verhandeln, in die sich das System selbst
getrieben hat. Im Namen der Ökologie wird man nun den Gürtel enger
schnallen müssen, wie gestern im Namen der Wirtschaft. Natürlich
könnten sich die Straßen in Radwege verwandeln, wir könnten sogar
in unseren Breitengraden mit einem garantierten Grundeinkommen
beehrt werden, aber nur zum Preis einer vollkommen therapeutischen
Existenz. Wer behauptet, dass die verallgemeinerte Selbstkontrolle
uns das Erleiden einer Umweltdiktatur ersparen wird, lügt: das Eine
wird das Andere in die Wege leiten und wir werden beides kriegen.

Solange es den Menschen und die Umwelt geben wird, wird die Polizei
zwischen ihnen stehen.

\extrapar{}

In den Diskursen der Umweltschützer gilt es, alles umzustürzen. Da,
wo sie von »Katastrophen« reden, um die Entgleisungen des
gegenwärtigen Regimes der Verwaltung von Wesen und Dingen zu
beschreiben, sehen wir nichts als die Katastrophe seines perfekten
Funktionierens. Die größte bis heute erlebte Hungersnot in den
Tropen (1876-1879) fällt mit einer weltweiten Dürre, aber vor allem
mit dem Höhepunkt der Kolonisierung zusammen. Die Zerstörung der
bäuerlichen Welten und der Praktiken der Subsistenz hatte die
Mittel zur Bekämpfung der Not verschwinden lassen. Mehr als der
Wassermangel waren es die Auswirkungen der sich in vollem
Aufschwung befindlichen kolonialen Wirtschaft, welche die gesamte
tropische Zone mit ausgehungerten Leichen bedeckte. Was sich
allerorts als Umweltkatastrophe präsentiert, hat nie aufgehört, in
erster Linie Ausdruck eines verheerenden Verhältnisses zur Welt zu
sein. Nichts zu bewohnen macht uns verwundbar bei der geringsten
Erschütterung des Systems, bei der geringsten klimatischen
Zufälligkeit. Als der vergangene Tsunami nahte und die Touristen
weiter in den Wellen herumtollten, flüchteten die Jäger und Sammler
der Insel, den Vögeln folgend, eilig von den Küsten. Das
gegenwärtige Paradox der Ökologie ist es, dass sie unter dem
Vorwand, die Erde zu retten, lediglich das Fundament dessen rettet,
was aus ihr dieses verödete Gestirn gemacht hat.
\extrapar{}

Die Regelmäßigkeit des globalen Funktionierens verdeckt in normalen
Zeiten unseren wahrhaft katastrophalen Zustand der Enteignung. Was
Katastrophe genannt wird, ist nichts als die notgedrungene
Aufhebung dieses Zustands, einer der wenigen Momente, in dem wir
ein wenig Anwesenheit in dieser Welt zurückgewinnen. Auf dass wir
früher als erwartet an die Grenzen der Erdölreserven gelangen, auf
dass die internationalen Ströme, die das Tempo der Metropole
aufrechterhalten, unterbrochen werden, auf dass wir großer sozialer
Unordnung entgegengehen, dass die »Verwilderung der Bevölkerungen«,
die »planetare Bedrohung«, das »Ende der Zivilisation« geschehe!
Irgendein Kontrollverlust ist jedem Krisenverwaltungs-Szenario
vorzuziehen. Die besten Tips sind von nun an nicht bei den Experten
für nachhaltige Entwicklung zu suchen. In den Funktionsstörungen,
den Kurzschlüssen des Systems erscheinen die Elemente logischer
Antworten, auf das, was aufhören könnte ein Problem zu sein. Die
einzigen Länder unter den Unterzeichnern des Kyoto-Protokolls, die
ihren Verpflichtungen unfreiwillig gerecht werden, sind die Ukraine
und Rumänien. Erratet warum. Das im weltweiten Vergleich am
Weitesten fortgeschrittene Experimentieren in Sachen »biologische
Landwirtschaft« findet seit 1989 auf Cuba statt. Erratet warum.
Entlang der afrikanischen Pisten und nicht woanders, hat die
Automechanik den Rang einer Volkskunst erreicht. Erratet wie.
Was die Krise wünschenswert macht, ist, dass die Umwelt in ihr
aufhört Umwelt zu sein. Wir sind im Begriff, einen Kontakt
wiederzuknüpfen, auch wenn er fatal ist, mit dem, was da ist, die
Rhytmen der Realität wiederzufinden. Was uns umgibt, ist nicht mehr
Landschaft, Panorama, Schauplatz, sondern was es zu bewohnen gilt,
womit wir uns abfinden sollen und wovon wir lernen können. Wir
werden uns nicht von denen berauben lassen, die sie verursacht
haben, die möglichen Inhalte der »Katastrophe«. Während sich die
Verwalter platonisch fragen, wie das Tempo gedrosselt werden kann,
»ohne alles zu zertrümmern«, sehen wir keine realistischere Option
als so früh wie möglich «alles zu zertrümmern», und bis es so weit
ist, jeden Zusammenbruch des Systems auszunutzen, um an Stärke zu
gewinnen.

\extrapar{}

Einige Tage, nachdem New Orleans vom Hurrikan Cathrina heimgesucht
wurde. In dieser apokalyptischen Atmosphäre reorganisiert sich hier
und dort ein Leben. Vor der Untätigkeit der Behörden, die eher mit
der Reinigung des Touristenviertels »Carré français« und dem Schutz
der Geschäfte beschäftigt waren, als den armen Stadtbewohnern Hilfe
zu leisten, erwachten vergessene Formen zu neuem Leben. Trotz den
manchmal energischen Versuchen, die Zone zu evakuieren, trotz der
von White-Supremacist-Milizen\footnote{
In den USA weitverbreitete Form des Rassismus, die auf der Idee
allgemeiner Überlegenheit der Weißen beruht.
}%32
bei diesem Anlass eröffneten
»Negerjagd«, wollten viele das Gebiet nicht verlassen. Für
diejenigen, die sich weigerten, als »Umweltflüchtlinge« in alle
Ecken des Landes deportiert zu werden, und für diejenigen von
überall her, die sich nach dem Aufruf eines ehemaligen Black
Panther entschieden, sich ihnen solidarisch anzuschließen, tauchte
die Offenkundigkeit der Selbstorganisierung wieder auf. Innerhalb
weniger Wochen wurde die Common Ground Clinic auf die Beine
gestellt. Dieses waschechte Landkrankenhaus bietet vom ersten Tage
an, dank dem unaufhörlichen Strom von Freiwilligen, immer
effizientere, kostenlose Pflege an. Seit nun einem Jahr bildet die
Klinik die Basis eines tagtäglichen Widerstands gegen die
Tabula-Rasa-Aktion der Regierungsbulldozer, die darauf abzielt, den
ganzen Stadtteil dem Erdboden gleichgemacht an den Immobilienmakler
zu übergeben. Volksküchen, Versorgung, Straßenmedizin, wilde
Beschlagnahmungen, Bau von Notunterkünften: Das von den Einen und
Anderen im Laufe des Lebens angehäufte praktische Wissen hat seinen
Ort der Entfaltung gefunden. Weit weg von den Uniformen und
Sirenen.
Wer die mittellose Freude in diesen Vierteln von New Orleans vor
der Katastrophe gekannt hat, das Misstrauen gegenüber dem Staat,
das vorher schon dort herrschte, und die massiven Praktiken des
Zurechtkommens gekannt hat, wird sich nicht darüber wundern, dass
all dies hier möglich gewesen ist. Wer hingegen im blutarmen und
atomisierten Alltag unserer Wohnwüsten gefangen ist, wird an einer
derartigen Entschlossenheit zweifeln. Nach Jahren des
normalisierten Lebens an diese Gesten anzuknüpfen ist jedoch der
einzig gangbare Weg, um nicht mit dieser Welt unterzugehen. Auf
dass eine Zeit komme, in die wir uns verlieben.

\section{Siebter Kreis}

\satz{»Hier errichten wir einen zivilisierten Raum«}

Das erste globale Gemetzel, das von 1914 bis 1918 ermöglichte, sich
auf einen Schlag eines großen Teils des ländlichen und städtischen
Proletariats zu entledigen, wurde im Namen der Freiheit, der
Demokratie und der Zivilisation geführt. Nur dem Anschein nach im
Namen der gleichen Werte setzt sich seit fünf Jahren, von gezielten
Morden bis hin zu Spezialeinsätzen, der berühmte »Krieg gegen den
Terrorismus« fort. Die Parallele endet damit beim Anschein. Die
Zivilisation ist nicht mehr diese Offenkundigkeit, im
Schnellverfahren zu den Eingeborenen transportiert. Freiheit ist
nicht mehr der Name, der an die Wände geschrieben wird\footnote{
Paul Éluard, »Liberté«, 1942 »...et par le pouvoir des mots
\textbar{} je recommence ma vie \textbar{} je suis née pour te
connaître \textbar{} pour te nommer.« - »...und durch die Macht der
Worte \textbar{} fange ich mein Leben von Neuem an \textbar{} ich
bin geboren, um dich zu kennen \textbar{} um dich zu nennen«
}%33
, auf den
nun sein Schatten folgt, der Name »Sicherheit«. Und die Demokratie
ist hinreichend bekannt dafür, sich in den reinsten
Ausnahmegesetzen aufzulösen – so zum Beispiel in der
Wiedereinführung der Folter in den USA oder dem „Perben II“-Gesetz
in Frankreich\footnote{
2004 definierte der französische Staat »Organisierte
Kriminalität« neu, führte einen Straftatenkatalog und damit ein
komplett neues Verfahren ein, das gewöhnliche Straftaten in ein
Geflecht von Sondergesetzen integriert.
}%34
.
In einem Jahrhundert wurden die Freiheit, die Demokratie und die
Zivilisation in den Zustand von Hypothesen versetzt. Die ganze
Arbeit der Führenden besteht von nun an im Einrichten materieller
und moralischer, symbolischer und sozialer Bedingungen, unter denen
diese Hypothesen beinahe bestätigt werden, im Konfigurieren von
Räumen, wo sie scheinbar funktionieren. Alle Mittel sind zum
Erreichen dieses Ziels recht, auch die am wenigsten demokratischen,
die am wenigsten zivilisierten und die am meisten mit
Sicherheitsmaßnahmen verbundenen. Im vergangenen Jahrhundert hat
die Demokratie regelmäßig der Geburt des Faschismus vorgestanden,
hat die Zivilisation nicht aufgehört die Arien Wagners oder Iron
Maidens mit Vernichtung in Einklang zu bringen, hat die Freiheit
eines Tages 1929 die Doppelgesichtigkeit eines Bankiers angenommen,
der sich aus dem Fenster stürzt, und einer Arbeiterfamilie, die an
Hunger stirbt. Man hat sich seit – sagen wir 1945 – darauf
geeinigt, dass die Manipulation der Massen, die Aktivitäten der
Geheimdienste, die Einschränkung der öffentlichen Freiheiten und
die vollständige Souveränität der verschiedenen Polizeien
angemessene Mittel zur Sicherung von Demokratie, Freiheit und
Zivilisation sind. Im letzten Stadium dieser Evolution: der erste
sozialistische Bürgermeister von Paris, der letzte Hand anlegt an
die urbane Befriedung, an die polizeiliche Erneuerung eines
Arbeiterviertels, und sich mit sorgfältig abgewogenen Worten
rechtfertigt: »Hier wird ein zivilisierter Raum errichtet«. Nichts
ist dem hinzuzufügen, alles ist zu zerstören.

\extrapar{}

Hinter ihrem Anschein von Allgemeinheit hat jene Frage der
Zivilisation nichts von einer philosophischen Frage. Eine
Zivilisation ist keine Abstraktion, die das Leben überragt. Sie ist
vielmehr, was herrscht, belagert und kolonisiert, die
alltäglichste, die persönlichste Existenz. Sie ist, was die
intimste und die allgemeinste Dimension zusammenhält. In Frankreich
ist die Zivilisation nicht vom Staat zu trennen. Je stärker und
älter ein Staat, desto weniger ist er eine Suprastruktur, das
Exoskelett einer Gesellschaft, desto mehr ist er in der Tat die
Form der Subjektivitäten, die ihn bewohnen. Der französische Staat
ist das eigentliche Gewebe der französischen Subjektivitäten, der
Aspekt, der nach der jahrhundertelangen Kastration seiner
Untertanen bleibt. Es erstaunt daher nicht, dass man sich in der
Psychatrie die Welt anhand von politischen Figuren zusammenspinnt,
dass man sich darin einig ist, den Ursprung all unseren Übels in
unseren Führern zu sehen, dass es uns so gefällt, über sie zu
meckern, und dass diese Meckereien die Jubelrufe sind, mit denen
wir sie als unsere Herrscher inthronisieren. Denn hier sorgt man
sich nicht um die Politik als eine fremde Realität, sondern als
Teil seiner selbst. Das Leben, das wir in diese Figuren stecken,
ist dasjenige, das uns geraubt wurde.
\extrapar{}

Wenn es eine französische Ausnahme gibt, stammt sie von dort\footnote{
Die französische Ausnahme bezeichnet das Selbstbild einer
herausragenden französischen Kultur, die es z.B. gegen die
US-amerikanische Kulturindustrie zu behaupten gilt.
}%35
. Es
gibt nichts, bis hin zur weltweiten Ausstrahlung der französischen
Literatur, was nicht die Frucht dieser Amputation wäre. Die
Literatur ist in Frankreich der Raum, den man selbstherrlich zur
Unterhaltung der Kastrierten zugelassen hat. Sie ist die formelle
Freiheit, die denen gewährt wurde, die sich nicht an die
Nichtigkeit ihrer realen Freiheit gewöhnen. Wo seit Jahrhunderten
unaufhörlich obszönes Augenzwinkern ausgetauscht wird, in diesem
Land, zwischen Männern des Staates und Männern der Schrift, wo die
Einen sich gerne den Anzug der Anderen leihen und umgekehrt. Wo
auch die Intellektuellen es gewohnt sind, wichtig daherzureden,
obwohl sie ganz unbedeutend sind, um immer im entscheidenden Moment
zu scheitern, im einzigen, der ihrer Existenz einen Sinn gegeben,
sie aber auch aus ihrem Beruf verbannt hätte.
Es ist eine verteidigte und zu verteidigende These, dass die
moderne Literatur mit Baudelaire, Heine und Flaubert als
Nachwirkung des staatlichen Massakers vom Juni 1848 geboren wurde.
Im Blut der Pariser Aufständischen und gegen das Schweigen, welches
das Gemetzel umhüllt, werden die modernen literarischen Formen
geboren – Spleen, Ambivalenz, Fetischismus der Form und eine
morbide Distanziertheit. Die neurotische Zuneigung, welche die
Franzosen ihrer Republik geloben – in deren Namen jeder Übergriff
wieder zu seiner Würde, und jede Schurkerei wieder ihren
Ritterschlag findet – verlängert stetig das Verdrängen des
Gründungsopfers. Die Tage des Juni 1848 - tausendfünfhundert Tote
während der Kämpfe, und mehrere tausend mehr in den darauf
folgenden Hinrichtungen von Gefangenen, die Nationalversammlung,
welche die Kapitulation der letzten Barrikade mit dem Ruf »Es lebe
die Republik« begrüßt - und die Blutige Woche sind Geburtsmale, die
auszulöschen keine Chirurgie vermag.

\extrapar{}

Kojève\footnote{
Alexandre Kojève (1902-1968) russisch-französischer Philosoph,
der zur Wiederentdeckung Hegels in Frankreich beitrug.
}
schrieb 1945: »Das ›offizielle‹ politische Ideal
Frankreichs und der Franzosen ist noch heute das des
Nationalstaates, der ›geeinten und unteilbaren Republik‹.
Andererseits nimmt das Land in der Tiefe seiner Seele die
Unzulänglichkeit dieses Ideals wahr, den politischen Anachronismus
einer ausschließlich ›nationalen‹ Idee. Zwar hat dieses Gefühl noch
nicht die Ebene einer klaren und deutlichen Idee erreicht: Das Land
kann und will dies noch nicht offen formulieren. Gerade wegen des
unvergleichbaren Glanzes seiner nationalen Vergangenheit, fällt es
Frankreich besonders schwer, das Ende der ›nationalen‹ Periode der
Geschichte in aller Deutlichkeit anzunehmen und wirklich zu
akzeptieren, und alle Konsequenzen daraus zu ziehen. Es ist hart
für ein Land, das aus dem Nichts das ideologische Gerüst des
Nationalismus geschaffen und in die ganze Welt exportiert hat,
zuzugestehen, dass es sich dabei bloß um ein zu klassifizierendes
Archivgut handelt, das in die Geschichte gehört.«
Die Frage des Nationalstaates und dessen Trauer bilden seit nunmehr
einem halben Jahrhundert das Herz dessen, was man wohl als
französisches Unbehagen bezeichnen muss. Höflich wird dieser
gelähmte Aufschub Alternance\footnote{
politsches Wechselspiel, anerkennende Bezeichnung für die im
Rahmen der Nation alternierenden Regierungen
}
genannt, diese Art des Pendelns, von
Links nach Rechts, dann von Rechts nach Links, so wie die manische
Phase auf die depressive Phase folgt, die eine weitere vorbereitet,
so wie in Frankreich die wortgewandteste Kritik des Individualismus
und der heftigste Zynismus, die größte Großzügigkeit und die Angst
vor der Masse Hand in Hand gehen. Seit 1945 hat dieses Unbehagen,
das sich nur in der Gunst des Mai 68 und seiner aufständischen
Leidenschaftlichkeit zu lichten schien, nicht aufgehört sich zu
vertiefen. Die Ära der Staaten, Nationen und Republiken schließt
sich wieder; das Land, das ihnen alles geopfert hat, was es an
Lebenskraft enthielt, bleibt benommen zurück. Die Detonation,
welche der Satz Jospins »Der Staat kann nicht alles« auslöste,
lässt jene erahnen, die früher oder später verursacht wird durch
die Offenbarung, dass der Staat gar nichts mehr kann. Dieses
Gefühl, hereingelegt worden zu sein, hört nicht auf um sich zu
greifen, eitrig. Es begründet die latente Wut, die bei jeder
Gelegenheit hochkommt. Dass die Ära der Nationen nie betrauert
wurde, ist der Schlüssel des französischen Anachronismus und der
revolutionären Möglichkeiten, die er in Reserve hält.

\extrapar{}

Was auch immer das Resultat der nächsten Wahlen sein wird\footnote{
»L‘insurrection qui vient« erschien unmittelbar vor den
Präsidentschaftswahl 2007, die Sarkozy an die Macht spülte
}%38
, ihre
Rolle ist, das Signal für das Ende der französischen Illusion zu
geben, die historische Blase zum Platzen zu bringen, in der wir
leben, und Ereignisse wie die Bewegung gegen das CPE zu
ermöglichen, die im Ausland beobachtet wird wie ein schlechter
Traum, der den 1970ern entflohen ist. Deshalb wünscht sich im
Grunde genommen niemand diese Wahlen. Frankreich ist wahrlich die
rote Laterne der westlichen Zone.
\extrapar{}

Das Abendland, das ist heute ein GI, der in einem Abraham M1 Panzer
nach Falloudja rast und volle Pulle Hardrock hört. Es ist ein
Tourist, der verloren mitten in den Ebenen der Mongolei von allen
verlacht seine Kreditkarte umklammert wie den letzten Strohhalm. Es
ist ein Manager, der auf nichts schwört außer auf das Spiel Go. Es
ist ein junges Mädchen, das sein Glück bei Klamotten, Männern und
Feuchtigkeitscremes sucht. Es ist ein schweizer
Menschenrechtsaktivist, der um alle vier Ecken des Planeten reist,
solidarisch mit allen Revolten, sofern sie niedergeschlagen werden.
Es ist ein Spanier, der auf die politische Freiheit scheißt, seit
ihm die sexuelle Freiheit gewährt wurde. Es ist ein Kunstfreund,
der zur erstarrten Bewunderung und als letzten Ausdruck des Genies
der Moderne ein Jahrhundert an Künstlern darbietet, die, vom
Surrealismus bis zum Wiener Aktionismus, darum konkurrieren, wer am
zielgenausten auf das Gesicht der Zivilisation spuckt. Es ist
schließlich ein Kybernetiker, der im Buddhismus eine realistische
Theorie des Bewusstseins gefunden hat und ein Teilchenphysiker, der
in der Metaphysik des Hinduismus nach Inspiration für seine neusten
Entdeckungen sucht.
\extrapar{}

Das Abendland, das ist jene Zivilisation, die alle Prophezeiungen
über ihren Untergang durch eine eigenartige List überlebt hat. So
wie das Bürgertum sich als Klasse verneinen musste, um die
Verbürgerlichung der Gesellschaft vom Arbeiter bis zum Baron zu
ermöglichen. Wie sich das Kapital als Lohnverhältnis opfern musste,
um sich als soziales Verhältnis durchzusetzen, um dadurch zu
kulturellem Kapital und gesundheitlichem Kapital, wie auch zu
finanziellem Kapital zu werden. Wie das Christentum sich als
Religion opfern musste, um als affektive Struktur zu überleben, als
diffuse Mahnung zu Demut, Mitgefühl und Ohnmacht, das Abendland hat
sich als besondere Zivilisation geopfert, um sich als universelle
Kultur durchzusetzen. Das Vorgehen lässt sich wie folgt
zusammenfassen: Ein im Sterben liegendes Gebilde opfert sich als
Inhalt, um als Form zu überleben.
\extrapar{}

Das in tausend Teile zerbrochene Individuum rettet sich dank der
»spirituellen« Technik des Coaching als Form. Das Patriachat, indem
es den Frauen alle peinlichen Attribute des Männchens aufbürdet:
Willenskraft, Selbstkontrolle und Unempfindsamkeit. Die zerfallene
Gesellschaft, indem sie eine Epidemie der Geselligkeit und der
Zerstreuung propagiert. Folglich halten sich all die verfaulten
Fiktionen des Abendlandes durch Kunstgriffe, von denen sie Punkt
für Punkt widerlegt werden.
\extrapar{}

Es gibt keinen »Zivilisationsschock«. Was es gibt, ist eine
Zivilisation in klinisch totem Zustand, die an sämtliche
lebenserhaltenden Apparate angeschlossen wird und die in der
planetaren Atmosphäre einen charakteristischen Gestank verbreitet.
An diesem Punkt gibt es keinen einzigen ihrer »Werte«, an den sie
noch irgendwie glauben kann, und jede Behauptung wirkt auf sie wie
eine Unverschämtheit, eine Provokation, die es auszuwaiden, zu
dekonstruieren und in den Zustand des Zweifels zu versetzen gilt.
Der abendländische Imperialismus ist heute jener des Relativismus
der »Sichtweise«, der böse Blick aus dem Augenwinkel, oder das
verletzte Protestieren gegen alles, was dumm genug, primitiv genug
oder selbstgefällig genug ist, um noch an etwas zu glauben, für
irgendetwas einzustehen. Er ist jener Dogmatismus der
Fragestellung, des komplizenhaften Augenzwinkern der universitären
und literarischen Intelligentsia. Keine Kritik ist den postmodernen
Denkern zu radikal, solange sie ein Nichts an Gewissheit umhüllt.
Noch vor einem Jahrhundert lag der Skandal in jeder etwas
auffälligen Verneinung, heute liegt er in jeder unerschütterlichen
Behauptung.
\extrapar{}

Keine soziale Ordnung kann dauerhaft auf dem Prinzip aufbauen, dass
nichts wahr ist. Also muss sie zusammengehalten werden. Die
Anwendung des Konzepts der »Sicherheit« auf jede einzelne Sache ist
heutzutage Ausdruck des Projekts, die ideale Ordnung in die Wesen
selbst, in Verhalten und Orte zu integrieren, eine Ordnung, der
sich zu unterwerfen sie nicht mehr bereit sind. »Nichts ist wahr«
sagt nichts über die Welt, sondern alles über das abendländische
Konzept der Wahrheit. Die Wahrheit wird hier nicht als Attribut der
Wesen oder Dinge wahrgenommen, sondern als ihre Repräsentation.
Eine Repräsentation gilt als echt, wenn sie erfahrungskonform ist.
Die Wissenschaft ist in letzter Instanz dieses Imperium der
universellen Verifizierung. Aber alle Formen menschlichen
Verhaltens, von den einfachsten zu den gelehrtesten, ruhen auf
einem Sockel ungleich formulierter Offenkundigkeiten, alle
Praktiken gehen von einem Punkt aus, in dem Dinge und
Repräsentationen ununterscheidbar verbunden sind, jedes Leben
beinhaltet eine Dosis Wahrheit, die das abendländische Konzept
ignoriert. Wenn hier einmal von »echten Leuten« geredet wird, dann
unweigerlich, um sich über diese geistig Armen lustig zu machen.
Von daher werden die Abendländler von denen, die sie kolonisierten,
weltweit für Lügner und Heuchler gehalten. Von daher werden sie um
das beneidet, was sie haben, ihren technologischen Fortschritt, nie
um das, was sie sind, wofür sie zurecht verachtet werden. Sade,
Nietzsche und Arteaud könnten in den Gymnasien nicht unterrichtet
werden, wäre dieser Begriff der Wahrheit nicht zuvor
disqualifiziert worden. Alle Behauptungen ohne Ende zu
unterdrücken, Schritt für Schritt alle Gewissheiten zu
deaktivieren, die fatalerweise ans Licht kommen, darin besteht die
langwierige Arbeit der abendländischen Intelligenz. Die Polizei und
die Philosophie sind zwei in die gleiche Richtung weisende Mittel,
obgleich verschieden in der Form.
Selbstverständlich findet der Imperialismus des Relativen in
irgendeinem leeren Dogmatismus, in irgendeinem
Marxismus-Leninismus, irgendeiner Salafiyya, in irgendeinem
Neo-Nazismus einen angemessenen Gegner, jemand, der, wie die
Abendländler, Behauptung mit Provokation verwechselt.

\extrapar{}

In diesem Stadium macht sich jeder ausschließlich soziale Protest,
der sich weigert anzuerkennen, dass das, was uns gegenübersteht,
nicht die Krise einer Gesellschaft ist, sondern der Untergang einer
Zivilisation, zum Komplizen ihres Fortbestehens. Es ist nunmehr
sogar eine verbreitete Strategie, diese Gesellschaft zu kritisieren
in der vergeblichen Hoffnung, diese Zivilisation zu retten.
\extrapar{}

Genau. Wir haben einen Kadaver auf dem Rücken, aber den werden wir
nicht so einfach los. Vom Ende der Zivilisation, ihrem klinischen
Tod, haben wir nichts zu erwarten. So wie sie ist, kann sie nur
Historiker interessieren. Das ist eine Tatsache, aus der eine
Entscheidung werden muss. Die Tatsachen können vertuscht werden,
die Entscheidung bleibt politisch. Sich für den Tod der
Zivilisation zu entscheiden, in die Hand zu nehmen, wie dies
geschieht: Nur durch die Entscheidung werden wir uns des Kadavers
entledigen.

\section{AUF GEHT‘S!}

Ein Aufstand, wir können uns nicht mal mehr vorstellen, wo er
beginnt. Sechzig Jahre der Befriedung, ausgesetzter historischer
Umwälzungen, sechzig Jahre demokratischer Anästhesie und Verwaltung
der Ereignisse haben in uns eine gewisse abrupte Wahrnehmung des
Realen geschwächt, den parteilichen Sinn für den laufenden Krieg.
Es ist diese Wahrnehmung, die wir wiedererlangen müssen, um zu
beginnen.
\extrapar{}

Es gibt keinen Grund, sich darüber zu entrüsten, dass seit fünf
Jahren ein bekanntermaßen verfassungswidriges Gesetz angewandt
wird, das Gesetz über die Alltägliche Sicherheit. Es ist
vergeblich, auf legalem Wege gegen die vollendete Implosion des
legalen Rahmens zu protestieren. Entsprechend muss man sich
organisieren.
\extrapar{}

Es gibt keinen Grund, sich in diesem oder jenem Bürgerkollektiv zu
engagieren, in dieser oder jener Sackgasse der radikalen Linken, in
der letzten vereinten Hochstapelei. Alle Organisationen, die
vorgeben, die gegenwärtige Ordnung anzufechten, haben selbst wie
Marionetten die Form, die Sitten und die Sprache von
Miniaturstaaten. Alle Anwandlungen, »Politik anders zu machen«,
haben bis zum heutigen Tag nur zur unbestimmten Ausdehnung des
staatlichen biomechanischen Apparats beigetragen.
\extrapar{}

Es gibt keinen Grund mehr, auf die neusten Nachrichten zu
reagieren, vielmehr ist jede Information als Operation in einem
feindlichen Feld von Strategien zu verstehen, die zu durchschauen
ist, Operationen, die gerade zum Ziel haben, bei diesem oder jenem
diese oder jene Reaktion hervorzurufen; und diese Operation für die
wirkliche Information zu halten, welche in den sichtbaren
Nachrichten verborgen ist.
\extrapar{}

Es gibt keinen Grund mehr zu warten – auf eine Aufheiterung, die
Revolution, die atomare Apokalypse oder eine soziale Bewegung. Noch
zu warten ist Wahnsinn. Die Katastrophe ist nicht, was kommt,
sondern was da ist. Wir verorten uns bereits jetzt in der Bewegung
des Zusammenbruchs einer Zivilisation. Dort ist es, wo man Partei
ergreifen muss.
\extrapar{}

Nicht mehr zu warten heißt, auf die eine oder andere Weise in die
aufständische Logik einzutreten. Es bedeutet, aufs Neue das leicht
erschreckte Zittern in der Stimme unserer Regierenden zu hören, das
sie nie verlässt. Denn regieren war niemals etwas anderes als mit
tausend Listen den Moment, wo die Menge sie aufhängen wird, zu
verschieben, und jeder Akt des Regierens ist nichts als die Weise,
die Kontrolle über die Bevölkerung nicht zu verlieren.
\extrapar{}

Wir gehen aus von einem Punkt der extremen Isolation, der extremen
Ohnmacht. Alles ist aufzubauen im aufständischen Prozess. Nichts
scheint unwahrscheinlicher als ein Aufstand, aber nichts ist
notwendiger.

\section{SICH FINDEN}

\subsection{Sich binden an das, was man als wahr erkennt.\\
Davon ausgehen}

Eine Begegnung, eine Entdeckung, eine große Streikbewegung, ein
Erdbeben: jedes Ereignis erzeugt Wahrheit, indem es unsere Art
verändert, auf der Welt zu sein. Umgekehrt hat sich eine
Feststellung, die uns gleichgültig ist, die uns unverändert lässt,
die zu nichts verpflichtet, noch nicht den Namen Wahrheit verdient.
In jeder Geste gibt es eine unterschwellige Wahrheit, in jeder
Praxis, in jeder Beziehung und in jeder Situation. Die Gewohnheit
ist, dem auszuweichen, das zu verwalten, was die charakteristische
Verwirrung der Allermeisten in dieser Epoche produziert. In
Wirklichkeit verpflichtet alles zu allem. Noch das Gefühl, in der
Lüge zu leben, ist eine Wahrheit. Es geht darum, es nicht
loszulassen, davon sogar auszugehen. Eine Wahrheit ist nicht eine
Sicht auf die Welt, sondern das, was uns auf unreduzierbare Art mit
ihr verbunden hält. Eine Wahrheit ist nichts, was man besitzt,
sondern etwas, das einen trägt. Sie stellt mich her und sie löst
mich auf, sie macht mich als Individuum aus und sie zersetzt mich
als solches, sie entfernt mich von vielen und verbindet mich mit
jenen, die sie erkennen. Das vereinsamte Wesen, das sich daran
bindet, trifft unausweichlich auf seinesgleichen. Im Grunde
genommen geht jeder aufständische Prozess von einer Wahrheit aus,
von der wir nicht abrücken. Es war in Hamburg in den 1980 Jahren,
wo eine Handvoll Bewohner eines besetzten Hauses entschieden hatte,
dass man von nun an über ihre Leiche gehen muss, um sie zu räumen.
Es gab einen belagerten Stadtteil mit Panzern und Helikoptern,
tagelange Straßenschlachten, gewaltige Demonstrationen – und eine
Stadtregierung, die am Ende kapitulierte. Georges Guingouin, der
»erste Partisan in Frankreich«, hatte als Ausgangspunkt 1940 nur
die Sicherheit seiner Ablehnung der Besatzung. Damals war er für
die Kommunistische Partei nur so »ein Spinner, der im Wald lebt«;
bis es zwanzigtausend Spinner waren, die im Wald lebten, und sie
Limoges befreiten.

\subsection{Nicht davor zurückweichen, was jede Freundschaft an 
Politischem mit sich bringt}

Man hat uns an eine neutrale Idee der Freundschaft gewöhnt, wie
reine Zuneigung ohne Konsequenzen. Aber jegliche Affinität ist
Affinität in einer gemeinsamen Wahrheit. Jede Begegnung ist eine
Begegnung in einer gemeinsamen Behauptung, und sei es die der
Zerstörung. Man verbindet sich nicht unschuldig in einer Epoche, in
der an etwas festzuhalten und sich nicht von etwas abbringen zu
lassen regelmäßig in die Arbeitslosigkeit führt, in der man lügen
muss, um zu arbeiten, und dann arbeiten muss, um die Mittel der
Lüge zu behalten. Wenn sich Wesen ausgehend von der Quantenphysik
schwören würden, in allen Bereichen alle Konsequenzen zu ziehen,
würden sie sich nicht auf weniger politische Weise verbinden als
Genossen, die einen Kampf gegen einen Nahrungsmittel-Multi führen.
Sie würden früher oder später entweder nicht erscheinen oder beim
Kampf ankommen.
Die Triebkräfte der Arbeiterbewegung hatten einst die Werkstätten,
dann die Fabriken, um sich zu finden. Sie hatten den Streik, um
sich zu zählen und die Verräter zu demaskieren. Sie hatten das
Lohnverhältnis, das die Partei des Kapitals und die Partei der
Arbeit gegeneinander aufbringt, um weltweit Solidaritäten und
Fronten aufzuspüren. Wir haben die Totalität des sozialen Raumes,
um uns zu finden. Wir haben das alltägliche Verhalten der
Aufsässigkeit, um uns zu zählen und die Verräter zu demaskieren.
Wir haben die Feindschaft gegenüber der Zivilisation, um weltweit
Solidaritäten und Fronten aufzuspüren.

\subsection{Nichts von den Organisationen erwarten.\\
Allen bestehenden Milieus misstrauen,\\
und zuallererst verhindern, zu einem zu werden}

Es ist nicht selten, dass man im Verlauf eines konsequenten
Austritts den Organisationen begegnet – politischen,
gewerkschaftlichen, humanitären, vereinten, etc.. Es kann sogar
vorkommen, dass man einigen aufrichtigen, aber hoffnungslosen, oder
enthusiastischen, aber durchtriebenen Wesen begegnet. Der Reiz der
Organisationen besteht in ihrer augenscheinlichen Beschaffenheit –
sie haben eine Geschichte, einen Sitz, einen Namen, Mittel, einen
Chef, eine Strategie und einen Diskurs. Nichtsdestotrotz sind sie
leere Architekturen, die der Respekt vor ihren heroischen
Ursprüngen nur mühsam mit Leben zu füllen vermag. In allen Dingen
wie auf jeder internen Ebene kümmern sie sich zuerst um das
Überleben als Organisationen, und um nichts anderes. Ihr
wiederholter Verrat also hat sie am meisten von der Verbindung zu
ihrer Basis entfremdet. Darum trifft man dort manchmal
schätzenswerte Wesen. Aber das in der Begegnung enthaltene
Versprechen kann nur außerhalb der Organisation verwirklicht werden
und, notwendigerweise, gegen sie.
\extrapar{}

Viel fürchterlicher noch sind die Milieus mit ihrer weichen
Struktur, ihrem Getratsche und ihren informellen Hierarchien. Alle
Milieus sind zu fliehen. Jedes einzelne von ihnen ist beauftragt,
eine Wahrheit zu neutralisieren. Die literarischen Milieus sind da,
die Offenkundigkeiten der Schriften zu ersticken. Die libertären
Milieus, die der direkten Aktion. Die naturwissenschaftlichen
Milieus, um zurückzuhalten, was ihre Forschungen ab heute für die
allermeisten mit sich bringen. Die sportlichen Milieus, um die
verschiedenen Lebensformen in ihren Sporthallen zu halten, welche
die verschiedenen Sportarten hervorbringen könnten. Insbesondere zu
fliehen sind die kulturellen und politischen Milieus. Sie sind die
zwei Hospize, in denen traditionellerweise alles revolutionäre
Verlangen zerschellt. Die Aufgabe der kulturellen Milieus besteht
darin, alle aufkeimenden Intensitäten aufzuspüren und den Sinn
dessen, was Ihr tut, zu unterschlagen, durch das Ausstellen; die
Aufgabe der politischen Milieus, Euch die Energie wegzunehmen, es
zu tun. Die politischen Milieus erstrecken ihre diffusen Netzwerke
über das ganze französische Territorium und stehen jeglichem
revolutionären Werden im Weg. Sie sind nur Träger der Anzahl ihrer
Niederlagen und der daraus erwachsenden Bitterkeit. Ihr Verschleiß
genauso wie ihr Übermaß an Ohnmacht hat sie unfähig gemacht, die
Möglichkeiten der Gegenwart aufzugreifen. Außerdem wird dort viel
zu viel geredet, um eine unglückliche Passivität einzurichten; was
sie polizeilich unsicher macht. So wie es vergeblich ist, von ihnen
etwas zu erhoffen, ist es dumm, von ihrer Sklerose enttäuscht zu
sein. Es reicht, sie verrecken zu lassen.
Alle Milieus sind konterrevolutionär, da ihr einziges Anliegen der
Erhalt ihrer miesen Bequemlichkeit ist.

\subsection{Sich als Kommunen zusammentun}

Die Kommune ist, was passiert, wenn Wesen sich finden, sich
verstehen und entscheiden, gemeinsam voranzuschreiten. Die Kommune
entscheidet sich vielleicht in dem Moment, wo es Brauch ist, sich
zu trennen. Sie ist die Freude des Zusammentreffens, die ihre
unerläßliche Erstickung überlebt. Sie ist, was bewirkt, dass wir
»wir« sagen, und dass dies ein Ereignis ist. Das Seltsame ist
nicht, dass Wesen, die sich verstehen, eine Kommune bilden, sondern
dass sie getrennt bleiben. Warum können sich die Kommunen nicht ins
Unendliche vermehren? In jeder Fabrik, in jeder Straße, in jedem
Dorf, in jeder Schule. Endlich die Herrschaft der Basiskomitees!
Kommunen aber, die akzeptieren würden, zu sein, was sie sind, wo
sie sind. Und möglicherweise eine Vielfalt von Kommunen, welche die
Institutionen des Staates ersetzen würden: die Familie, die Schule,
die Gewerkschaft, den Sportverein, etc.. Kommunen, die sich nicht
fürchten würden, sich neben ihren rein politischen Aktivitäten für
das materielle und emotionale Überleben eines jeden ihrer
Mitglieder zu organisieren und für all die Verlorenen, die sie
umgeben. Kommunen, die sich – anders als es Kollektive im
Allgemeinen tun – nicht über ein Drinnen und ein Draußen
definieren, sondern über die Dichte der Beziehungen in ihrem
Inneren. Nicht über die Personen, die sie zusammensetzten, sondern
über den Geist, der sie treibt.
\extrapar{}

Eine Kommune bildet sich jedes Mal, wenn einige, befreit von der
individuellen Zwangsjacke, sich entscheiden nur auf sich selbst zu
zählen und ihre Kraft an der Realität zu messen. Jeder wilde Streik
ist eine Kommune, jedes kollektiv besetzte Haus, das auf einer
klaren Basis steht, ist eine Kommune, die Aktionskomitees von 68
waren Kommunen, so wie es die Cimarrones geflohener Sklaven in den
Vereinigten Staaten waren, oder Radio Alice in Bologna im Jahre
1977. Jede Kommune will sich selbst die Basis sein. Sie will die
Frage der Bedürfnisse auflösen. Sie will gleichzeitig mit jeglicher
wirtschaftlichen Abhängigkeit jede politische Unterwerfung
zerschlagen, und sie degeneriert zum Milieu, sobald sie den Kontakt
zu den Wahrheiten verliert, die sie begründen. Es gibt allerlei
Kommunen, die weder die Zahl, noch die Mittel, noch den »richtigen
Moment«, der nie kommen wird, abwarten, um sich zu organisieren.

\section{SICH ORGANISIEREN}

\subsection{Sich organisieren, um nicht mehr arbeiten zu müssen}

Low Intensity Arbeitsplätze sind selten geworden und, um die
Wahrheit zu sagen, bedeutet es oft, zu viel Zeit zu verlieren, sich
dort weiter zu langweilen. Sie zeichnen sich außerdem durch
schlechte Bedingungen für die Siesta oder die Lektüre aus.
Es ist wohlbekannt, dass das Individuum so wenig existiert, dass es
sich sein Leben verdienen muss, dass es seine Zeit gegen ein
bisschen soziale Existenz tauschen muss. Persönliche Zeit gegen
soziale Existenz: so ist die Arbeit, so ist der Markt. Die Zeit der
Kommune entzieht sich sofort der Arbeit, sie fällt nicht auf den
Trick herein, sie bevorzugt andere. Gruppen argentinischer
Piqueteros luchsen dem Staat kollektiv eine Art lokale Sozialhilfe
ab, die an ein paar Arbeitsstunden geknüpft ist; sie leisten die
Stunden nicht ab, schmeißen ihren Gewinn zusammen und statten sich
aus mit Schneidereien, einer Bäckerei und bauen die Gärten auf, die
sie benötigen.

Es gilt, Geld für die Kommune zu suchen, auf keinen Fall muss das
Leben verdient werden. Alle Kommunen haben ihre schwarzen Kassen.
Die Tricks sind vielfältig. Neben dem RMI gibt es das Kindergeld,
das Krankfeiern, mehrfache Stipendien, erschwindelte Prämien für
fiktive Geburten, alle Arten von Geschäften und so viele andere
Mittel, die bei jeder Mutation der Kontrolle entstehen. Es liegt
weder an uns, sie zu verteidigen, noch es uns in diesen schützenden
Verschlägen bequem zu machen oder sie wie ein Privileg für
Eingeweihte zu bewahren. Was wichtig ist zu kultivieren, zu
verbreiten, ist jene notwendige Bereitschaft zum Betrug und zum
Teilen seiner Innovationen. Für die Kommunen stellt sich die Frage
der Arbeit nur im Verhältnis zu den anderen vorhandenen Einkommen.
Dabei darf nicht vernachlässigt werden, was es in gewissen Berufen,
Ausbildungen oder gut platzierten Posten nebenbei alles an
nützlichen Erkenntnissen zu sammeln gibt.

\extrapar{}

Der Anspruch der Kommune ist es, für alle so viel Zeit wie möglich
freizumachen. Ein Anspruch, der sich nicht nur, nicht im
Wesentlichen, an der Zahl der Stunden misst, die frei von
lohnabhängiger Ausbeutung sind. Die befreite Zeit schickt uns nicht
in die Ferien. Die unbesetzte Zeit, die tote Zeit, die Zeit der
Leere und der Angst vor der Leere, das ist die Zeit der Arbeit. Von
nun an gibt es keine Zeit mehr zu füllen, aber eine Befreiung von
Energie, die keine »Zeit« beinhaltet; Linien, die sich abzeichnen,
die deutlicher werden, denen wir nach Belieben folgen können, bis
zum Ende, bis wir sehen, wie sie andere kreuzen.

\subsection{Plündern, anbauen, herstellen}

Ehemalige Arbeiter von Metaleurop werden eher Räuber als Schließer.
Die Angestellten von EDF lassen ihren Freundeskreis wissen, wie man
einen Stromzähler überbrückt. Die »vom Lastwagen gefallene« heiße
Ware wird schnell weiterverkauft. Eine Welt, die sich selbst derart
offen zynisch erklärt, konnte seitens der Proletarier kaum viel
Loyalität erwarten.
Einerseits kann eine Kommune nicht auf die Ewigkeit des
Wohlfahrtsstaates zählen, andererseits kann sie nicht damit
rechnen, auf lange Sicht von Ladendiebstahl, vom Containern des
Abfalls aus den Mülltonnen der Supermärkte oder des Nachts aus den
Warenlagern der Industriezonen, vom Abzweigen von Subventionen, vom
Versicherungs- und sonstigen Betrug, kurz: vom Plündern zu leben.
Sie muss sich also permanent damit beschäftigen, wie sie das Niveau
und die Ausbreitung ihrer Selbstorganisation steigert. Nichts wäre
logischer, als dass die Drehbänke, die Fräsen und Fotokopierer, die
bei Schließung einer Fabrik mit Rabatt verkauft werden, später zur
Bekräftigung irgendeiner Verschwörung gegen die Warengesellschaft
dienen.

Das Gefühl des bevorstehenden Zusammenbruchs ist heutzutage überall
so akut, dass es schwerfällt, alle laufenden Experimente in den
Bereichen Bau, Energie, Materialien, Illegalismus und
Landwirtschaft aufzuzählen. Ein ganzes Ensemble von Wissen und
Techniken wartet nur darauf, geplündert und seiner Verpackung
entrissen zu werden, sei diese moralistisch, kleinkriminell oder
ökologisch. Dieses Ensemble aber macht nur einen Teil aller
Intuition, allen Know-Hows und der den Slums eigenen Erfindungsgabe
aus, die wir an den Tag legen müssen, um die metropolitane Wüste
wieder zu bevölkern und mittelfristig die Lebensfähigkeit eines
Aufstandes zu sichern.

Wie kommunizieren und sich bewegen, wenn alle Flüsse unterbrochen
sind? Wie können wir die Subsistenz in den ländlichen Gebieten
wiederherstellen, bis diese in der Lage sind, die
Bevölkerungsdichte zu tragen, wie dies vor sechzig Jahren noch der
Fall war? Wie können wir die betonierten Räume in städtische
Gemüsegärten verwandeln, wie dies einst Cuba tat, um das
amerikanische Embargo und die Liquidierung der UdSSR zu
verkraften?

\subsection{Ausbilden und sich formieren}

Was bleibt uns, die wir soviel Gebrauch gemacht haben von den
autorisierten Vergnügungen, welche uns die marktwirtschaftliche
Demokratie zugesteht? Was hat uns einst dazu getrieben, am Sonntag
morgen joggen zu gehen? Was fesselt all die Karate-Fanatiker, die
Liebhaber der Bastelei, des Angelns oder der Pilzkunde? Was außer
der Notwendigkeit, die vollkommene Untätigkeit zu füllen, die
eigene Arbeitskraft oder das eigene »Ge\-sund\-heits-Ka\-pi\-tal«
wiederherzustellen. Die meisten Vergnügungen könnten mit
Leichtigkeit ihren absurden Charakter ablegen und zu mehr als nur
Vergnügungen werden. Das Boxen war nicht immer für die Vorführungen
auf Spenden-Galas und die Spektakel großer Wettkämpfe reserviert.
Zu Beginn des zwanzigsten Jahrhunderts, im von Horden von
Kolonisten ausgewaideten und aufgrund langer Trockenheit hungernden
China, organisierten sich hunderttausende armer Bauern rund um
unzählige Boxclubs unter freiem Himmel, um sich von Reichen und
Kolonisten zurückzuholen, was ihnen geraubt worden war. Dies war
die Revolte der Boxer. Wir können nicht früh genug damit beginnen,
zu lernen und anwenden, was weniger befriedete, weniger
vorhersehbare Zeiten von uns verlangen werden. Unsere Abhängigkeit
von der Metropole – von ihrer Medizin, ihrer Landwirtschaft, ihrer
Polizei – ist so groß, gegenwärtig, dass wir sie nicht angreifen
können, ohne uns selbst in Gefahr zu bringen. Es ist das
unausgesprochene Bewusstsein dieser Verletzbarkeit, das die
unaufgeforderte Selbstbeschränkung der aktuellen sozialen
Bewegungen ausmacht, das uns die Krisen fürchten und nach
»Sicherheit« streben lässt. Ihm ist es zu verdanken, dass die
Streiks den Horizont der Revolution gegen den der Rückkehr zur
Normalität eingetauscht haben. Aus diesem Schicksal auszubrechen
verlangt nach einem langen und stichhaltigen Lernprozess, nach
vielfältigen massiven Experimenten. Es geht darum, kämpfen zu
können, Schlösser zu knacken, Knochenbrüche ebenso zu heilen wie
eine Angina, einen Piratensender zu bauen, Volksküchen
einzurichten, genau zu zielen, aber auch darum, zerstreutes Wissen
zu sammeln und eine Landwirtschaft des Krieges zu schaffen, die
Biologie des Plankton und die Zusammensetzung des Bodens zu
verstehen, das Zusammenwirken der Pflanzen zu studieren und dadurch
die verlorene Intuition, alle Formen der Nutzung wiederzuentdecken,
alle möglichen Bindungen an unsere unmittelbare Umgebung, und die
Grenzen, über die hinaus wir sie aufbrauchen würden; und dies ab
heute, für die Zeit, in der wir mehr als einen symbolischen Teil
unserer Ernährung und Pflege aus ihr beschaffen müssen.

\subsection{Territorien schaffen. Die Zonen der Undurchdringlichkeit vermehren.}

Die Reformisten sind sich heute zunehmend einig, dass es mit »dem
näherrückenden Peak Oil« und, »um Treibhausgase zu reduzieren«,
einer »Re-Lokalisierung der Wirtschaft« bedarf, der Förderung
regionaler Versorgung, kurzer Vertriebswege, des Verzichts auf die
Bequemlichkeit von Importen aus der Ferne, etc.. Was sie dabei
vergessen, ist, dass es die Eigentümlichkeit dieser lokalen
wirtschaftlichen Tätigkeiten ist, das sie im Schatten stattfinden,
auf »informelle« Art; dass diese einfache ökologische Maßnahme der
Re-Lokalisierung der Wirtschaft nicht weniger impliziert, als sich
aus der staatlichen Kontrolle zu befreien, oder sich ihr
bedingungslos zu unterwerfen.
Das aktuelle Territorium ist das Produkt mehrerer Jahrhunderte
polizeilicher Operationen. Das Volk wurde von seinem Land, von
seinen Straßen, dann aus seinen Stadtteilen und schließlich aus
seinen Treppenhäusern gedrängt\footnote{
2001 verwandelt das »Gesetz über die alltägliche Sicherheit« die
›Besatzung‹ der Eingangsbereiche der Wohnhäuser in ein Delikt.
Seither kann die Polizei Jugendliche für den Aufenthalt vor ihrer
Haustür verhaften.
}%39
, in der verrückten Hoffnung, alles
Leben in den vier schwitzenden Wänden des Privaten in Schach zu
halten. Für uns stellt sich die Frage des Territoriums nicht in
gleicher Weise wie für den Staat. Es geht nicht darum, es zu
halten. Es geht darum, auf lokaler Ebene die Kommunen, die
Zirkulation und die Solidaritäten zu verdichten, bis zu dem Punkt,
an dem das Territorium unlesbar, undurchdringlich wird für jegliche
Autorität. Es geht nicht darum, ein Territorium zu besetzen,
sondern es zu sein.

Jede Praxis lässt ein Territorium existieren – ein Territorium für
den Drogenhandel oder die Jagd, ein Territorium der Spiele für
Kinder, der Verliebten oder der Unruhen, ein Territorium des
Bauern, des Ornithologen oder des Flaneurs. Die Regel ist simpel:
je mehr Territorien sich in einer bestimmten Zone überlagern, desto
mehr Zirkulation gibt es zwischen ihnen, und umso weniger
Angriffsfläche findet die Macht. Kneipen, Druckereien, Sporthallen,
Brachflächen, Antiquariate, Dächer von Wohnblocks, unangemeldete
Märkte, Dönerläden und Garagen können ihrer offiziellen Bestimmung
einfach entkommen, wenn sich dort ausreichend Komplizenschaften
finden. Indem sie der staatlichen Kartographie ihre eigene
Geographie aufzwingt, sie verschwimmen lässt, sie löscht,
produziert die lokale Selbstorganisierung ihre eigene Sezession.

\subsection{Reisen. Unsere eigenen Kommunikationswege anlegen.}

Das Prinzip der Kommunen ist nicht, der Metropole und ihrer
Mobilität die lokale Verwurzelung und die Langsamkeit
entgegenzusetzen. Die sich ausbreitende Bewegung der Bildung von
Kommunen muss diejenige der Metropole unterirdisch überholen. Es
gibt keinen Grund, die Möglichkeiten des Reisens und der
Kommunikation, die uns die Infrastrukturen des Marktes bieten,
abzulehnen, es genügt, ihre Grenzen zu kennen. Man muss nur
vorsichtig genug, unauffällig genug sein. Sich zu besuchen ist
allemal sicherer, hinterlässt keine Spur und schafft Verbindungen,
die gehaltvoller sind als alle Kontaktlisten im Internet. Das
Privileg, »frei zu reisen«, quer durch den Kontinent und ohne
größere Probleme in die ganze Welt, das vielen von uns zugestanden
wird, ist ein nicht zu vernachlässigender Trumpf für die
Kommunikation zwischen den Herden der Konspiration. Einer der Reize
der Metropole ist es, Amerikanern, Griechen, Mexikanern und
Deutschen zu erlauben, sich heimlich für die Zeit einer
Strategiediskussion in Paris wiederzutreffen.
Die permanente Bewegung zwischen den befreundeten Kommunen gehört
zu den Dingen, die sie vor dem Austrocknen und dem Verhängnis des
Verzichts bewahrt. Genossen zu empfangen, sich über ihre
Initiativen auf dem Laufenden zu halten, über ihre Erfahrungen
nachzusinnen, sich die Techniken, die sie beherrschen, anzueignen,
bringen einer Kommune mehr, als sterile Selbstprüfungen hinter
verschlossenen Türen. Es wäre falsch zu unterschätzen, was an
diesen Abenden an Entscheidendem erarbeitet werden kann, an denen
wir uns mit unseren Ansichten über den laufenden Krieg
auseinandersetzen.

\subsection{Alle Hindernisse umstürzen, eins nach dem anderen.}

Wie man weiß, laufen die Straßen über vor Unhöflichkeiten. Zwischen
dem, was sie wirklich sind, und dem, was sie sein sollten, steht
die Zentripetalkraft jeglicher Polizei, die sich abmüht die Ordnung
wiederherzustellen; und ihr gegenüber gibt es uns, das heißt die
gegenläufige Bewegung, die Zentrifugalkraft. Überall wo Erregung
und Unordnung auftauchen, können wir uns über sie nur freuen. Es
erstaunt nicht, dass diese Nationalfeiern, die nichts mehr feiern,
nun systematisch verderben. Funkelnagelneu oder klapprig, das
urbane Mobiliar – aber wo fängt es an? Wo hört es auf? -
materialisiert unsere gemeinsame Enteignung. Hartnäckig in seiner
Nichtigkeit, verlangt es nur danach auf ewig wiederzukehren.
Beobachten wir aufmerksam, was uns umgibt: all dies wartet, dass
seine Stunde schlägt, die Metropole nimmt nostalgische Züge an, wie
dies sonst nur Ruinen tun.
Auf dass die Unhöflichkeiten methodisch werden, dass sie
systematisch werden, sich zu einer diffusen, effizienten Guerilla
vereinen, die uns wieder zu unserer wesentlichen Unregierbarkeit
zurückführt, zu unserer Undiszipliniertheit. Es ist verwirrend,
dass gerade die Undiszipiniertheit zu den Tugenden des Partisanen
gezählt wird. Schlussendlich hätte man die Wut nie von der Politik
lösen sollen. Ohne erstere verliert sich letztere im Diskurs, und
ohne letztere erschöpft sich erstere im Gebrüll. Begriffe wie »die
Wütenden« und »die Fanatiker«\footnote{
Als Wütende, »enragés«, und Fanatiker, »exaltés«, wurden während
der Französischen Revolution jene radikale Gruppen bezeichnet,
deren Ideen einer direkt vom Volk ausgeübten Souveränität und
Kritik jeder Form der Repräsentation sie zu politischen Gegnern der
Jakobiner wie der Bergpartei werden ließ.
}%40
tauchen in der Politik nie ohne
Warnschüsse wieder auf.

\extrapar{}

Was die Methode angeht, behalten wir für die Sabotage folgendes
Prinzip: ein Minimum an Risiko bei der Aktion, ein Minimum an Zeit,
ein Maximum an Schaden. Für die Strategie, erinnern wir uns daran,
dass ein umgestürztes, aber nicht ausgeräumtes Hindernis – ein
befreiter, aber nicht bewohnter Raum – einfach durch ein weiteres
Hindernis zu ersetzen ist, das beständiger und schwerer anzugreifen
ist.
Es bringt nichts, sich mit den drei Typen der Arbeitersabotage
abzumühen: die Arbeit bremsen, vom »locker nehmen« zum
Bummelstreik; die Maschinen zerstören oder ihre Abläufe
beeinträchtigen; Firmengeheimnisse ausplaudern. Auf die Dimensionen
der gesellschaftlichen Fabrik ausgeweitet, verallgemeinern sich die
Prinzipien der Sabotage von der Produktion in die Zirkulation. Die
technische Infrastruktur der Metropole ist verletzbar: ihre Flüsse
bestehen nicht nur im Transport von Personen und Waren,
Informationen und Energie zirkulieren durch Netze aus Kabeln,
Glasfasern und Rohren, die angegriffen werden können. Die soziale
Maschine mit einiger Auswirkung zu sabotieren, bedeutet heutzutage
sich die Mittel zur Unterbrechung ihrer Netze wieder anzueignen und
neu zu erfinden. Wie können eine TGV-Linie oder ein Stromnetz
unbrauchbar gemacht werden? Wie können die Schwachstellen der
Computer-Netzwerke gefunden, wie die Radiofrequenzen gestört und
die Flimmerkiste wieder zum Rauschen gebracht werden?

Was die ernsten Hindernisse anbelangt, ist es falsch, ihre
Zerstörung für unmöglich zu halten. Das Promethische dabei besteht
und lässt sich zusammenfassen in einer gewissen Aneignung des
Feuers, jenseits jeglichen blinden Voluntarismus. 356 v. Chr.
brannte Herostratos den Tempel der Artemis nieder, eines der sieben
Weltwunder. In unseren Zeiten der vollendeten Dekadenz haben die
Tempel nichts Imposantes mehr, außer der finsteren Wahrheit, dass
sie bereits Ruinen sind.

Dieses Nichts zu vernichten hat nichts von einer traurigen Aufgabe.
Das Handeln findet darin zu neuer Jugend. Alles macht Sinn, alles
ordnet sich plötzlich, Raum, Zeit, Freundschaft. Aus allem Holz
wird ein Pfeil gemacht, man findet die Verwendung wieder – ist ganz
Pfeil. Im Elend der Zeit dient »scheiß auf alles« vielleicht –
nicht ohne Grund, wie man zugeben muss – als letzte kollektive
Verführung.

\subsection{Die Sichtbarkeit fliehen. Die Anonymität in eine 
offensive Position wenden.}

Während einer Demonstration reißt eine Gewerkschafterin die Maske
eines Anonymen runter, der gerade eine Scheibe eingeschlagen hat:
»Steh zu dem, was du tust, anstatt dich zu verstecken«. Sichtbar zu
sein bedeutet ohne Deckung, das heißt vor allem verletzbar zu sein.
Wenn die Linken aller Länder nicht aufhören ihre Sache »sichtbar«
zu machen - sei es die der Obdachlosen, der Frauen oder der
Sans-Papiers – in der Hoffnung, dass man sich darum kümmert, tun
sie genau das Gegenteil dessen, was getan werden müsste. Nicht sich
sichtbar zu machen, sondern die Anonymität, in die wir abgeschoben
wurden, zu unserem Vorteil zu wenden und daraus, mittels der
Verschwörung, der nächtlichen oder vermummten Aktion, eine
unangreifbare Position des Angriffs zu machen. Das Feuer von
November 2005 bietet dafür das Vorbild. Kein Führer, keine
Forderung, keine Organisation, sondern Worte, Gesten,
Komplizenschaften. Gesellschaftlich nichts zu sein ist kein
erniedrigender Stand, die Quelle eines tragischen Mangels an
Anerkennung – anerkannt: von wem? - vielmehr ist es die Bedingung
einer maximalen Aktionsfreiheit. Seine Untaten nicht zu
unterzeichnen, mit sinnlosen Kürzeln bekannt zu machen – man
erinnert sich noch der kurzlebigen BAFT (Anti-Bullen-Brigade
Tarterêts)\footnote{
»Brigade Anti-Flic de Tarterêts«, einer Hochhaussiedlung im
Département Saint-Denis.
}%41
 – ist eine Art, diese Freiheit zu bewahren.
Offenkundig ist das Konstruieren eines Subjekts »Banlieue« als
Akteur der »Unruhen von 2005« eines der ersten defensiven Manöver
des Regimes gewesen. Sich die Fressen derjenigen anzusehen, die in
dieser Gesellschaft jemand sind kann helfen die Freude zu
verstehen, dort niemand zu sein.
Die Sichtbarkeit ist zu fliehen. Aber eine Kraft, die sich im
Dunkeln sammelt, kann ihr nicht auf ewig ausweichen. Uns geht es
darum, unser Erscheinen als Kraft bis zum günstigen Zeitpunkt zu
verschieben. Denn je später uns die Sichtbarkeit findet, umso
stärker findet sie uns. Erst einmal in der Sichtbarkeit, sind
unsere Tage gezählt. Entweder sind wir in der Lage, ihre Herrschaft
kurzfristig zu pulverisieren, oder sie wird uns ohne Verzögerung
zerquetschen.

\subsection{Die Selbstverteidigung organisieren.}

Wir leben unter Besatzung, unter polizeilicher Besatzung. Die
Razzien gegen Sans-Papiers auf offener Straße, die Zivilstreifen,
welche den Boulevard hoch und runter fahren, die Befriedung von
Stadtteilen der Metropole mittels Techniken, die in den Kolonien
geschmiedet wurden, die Vorträge des Innenministers gegen die
»Banden«, die jenen aus dem Algerienkrieg ähneln, erinnern uns
täglich daran. Genügend Motive, sich nicht mehr zerquetschen zu
lassen, in die Selbstverteidigung zu gehen.
Im Zuge ihres Wachsens und Ausstrahlens wird eine Kommune nach und
nach mit Operationen der Macht konfrontiert, die aufs Korn nehmen,
was sie ausmacht. Diese Gegenangriffe nehmen die Form der
Verführung an, der Vereinahmung und in letzter Instanz jene der
rohen Gewalt. Die Selbstverteidigung muss eine kollektive
Offenkundigkeit für die Kommunen sein, sowohl praktisch als auch
theoretisch. Eine Festnahme abzuwehren, sich blitzschnell und
zahlreich gegen Abschiebungen oder Räumungen zu versammeln, einen
der unseren zu verstecken, werden in den kommenden Zeiten keine
überflüssigen Reflexe sein. Wir können nicht permanent unsere
Stützpunkte wieder aufbauen. Hören wir auf die Repression zu
beklagen, bereiten wir uns darauf vor.

Dies ist keine einfache Sache, denn indem von der Bevölkerung ein
Überschuss an polizeilicher Arbeit erwartet wird – von der
Denunziation über die gelegentliche Beteiligung an Bürgerwehren –
verschwinden die Polizeikräfte in der Masse. Das
Passepartout-Modell der polizeilichen Intervention, auch in
aufständischen Situationen, ist nun der Bulle in zivil. Die
Effizienz der Polizei während der letzten Demos gegen das CPE ist
diesen Zivilen zu verdanken, die sich unter die Menge mischten,
wartend auf den Moment, sich zu enttarnen: mit Gas, Schlagstock,
Gummigeschoss und Festnahme; das Ganze in Koordination mit den
Ordnungsdiensten der Gewerkschaften. Allein die Möglichkeit ihrer
Anwesenheit reicht, den Verdacht unter den Demonstranten zu wecken:
Wer ist wer?, und das Handeln zu lähmen. Ausgehend davon, dass eine
Demonstration nicht ein Mittel ist, sich zu zählen, sondern zu
handeln, müssen wir uns die Mittel aneignen, die Zivilen zu
demaskieren, sie zu verjagen und gegenbenfalls ihnen diejenigen zu
entreißen, die sie versuchen festzunehmen.

Die Polizei ist nicht unbesiegbar auf der Straße, sie hat nur die
Mittel sich zu organisieren, sich zu trainieren und permanent neue
Waffen zu testen. Im Vergleich werden unsere Waffen immer
rudimentär, gebastelt und oft vor Ort improvisiert sein. Diese
geben auf keinen Fall vor mit deren Feuerkraft zu konkurrieren,
sondern zielen darauf ab, sie auf Distanz zu halten, die
Aufmerksamkeit abzulenken, psychologischen Druck auszuüben oder
durch Überraschen eine Bresche zu schlagen und Terrain zu gewinnen.
Offensichtlich reicht die ganze Innovation, die sich in den
Ausbildungszentren zur Stadt-Guerilla der französischen Gendarmerie
entfaltet, nicht aus, und wird wahrscheinlich nie genügen, um rasch
genug auf eine sich bewegende Vielfalt zu reagieren, die vielerorts
gleichzeitig zuschlagen kann und vor allem immer versucht die
Initiative zu behalten.

Die Kommunen sind natürlich verletzbar durch Überwachung und
polizeiliche Ermittlungen, durch Kriminaltechnik und
Geheimdienste. Die Verhaftungenwellen von Anarchisten in Italien
und Ecowarriors in den USA wurden durch Abhören ermöglicht. Jede
zeitweilige Gewahrsamnahme\footnote{
»Garde à vue« bedeutet juristisch, eine Person bis zu drei Tage
›der Polizei zur Verfügung zu stellen‹. Sie kann auf verschiedenste
Gesetzesverstöße angewendet werden, auch auf Verdacht, und betrifft
jährlich 900.000 Menschen.
}
veranlasst mittlerweile eine
DNA-Entnahme und trägt bei zu einer immer vollständigeren
Datenbank. Ein Hausbesetzer aus Barcelona wurde gefasst, weil er
Fingerabdrücke auf Flugblättern hinterließ, die er verteilt hatte.
Die Methoden der Datenspeicherung verbessern sich stetig, vor allem
durch die Biometrie. Und sollte der elektronische Ausweis
eingeführt werden, würde dies unsere Aufgabe nur noch komplizieren.
Die Pariser Kommune hatte das Problem der Datenspeicherung
teilweise gelöst: Mit dem Niederbrennen des Ratshauses zerstörten
die Brandstifter die Archive der Zivilverwaltung. Eine Möglichkeit
elektronische Daten auf immer zu zerstören, muss erst noch gefunden
werden.

\section{AUFSTAND}

Die Kommune ist die elementare Einheit der Realität der Partisanen.
Eine aufständische Erhebung ist vielleicht nichts anderes als eine
Vervielfachung der Kommunen, ihrer Verbindungen und ihres
Zusammenspiels. Im Lauf der Ereignisse verschmelzen die Kommunen zu
größeren Einheiten oder splittern sich auf. Zwischen einer Bande
von Brüdern und Schwestern, verbunden »auf Leben und Tod«, und der
Zusammenkunft einer Vielzahl von Gruppen, Komitees und Banden um
die Versorgung und Selbstverteidigung eines Stadtteils, oder sogar
einer aufständischen Region, gibt es nur einen Unterschied im
Umfang, sie sind ununterscheidbar Kommunen.
\extrapar{}

Jegliche Kommune kann nur zwangsläufig nach Selbstversorgung
streben und in ihrem Innern Geld als etwas Lächerliches und genau
gesagt Deplaziertes empfinden. Die Macht des Geldes besteht darin,
eine Bindung zu schaffen zwischen denen, die ohne Bindung sind,
Fremde als Fremde zu verbinden und dadurch, dass jedes Einzelne als
äquivalent gesetzt wird, alles in Zirkulation zu versetzen. Die
Fähigkeit des Geldes, alles zu verbinden, wird mit der
Oberflächlichkeit dieser Verbindung bezahlt, in der die Lüge zur
Regel wird. Der Argwohn ist das Fundament des Kreditverhältnisses.
Die Herrschaft des Geldes muss deswegen immer die Herrschaft der
Kontrolle sein. Die praktische Abschaffung des Geldes kann nur
durch die Ausweitung der Kommunen geschehen. Bei der Ausweitung der
Kommunen muss jede einzelne der Sorge Rechnung tragen, eine gewisse
Größe nicht zu überschreiten, eine Größe, ab der sie den Kontakt zu
sich selbst verliert und damit unweigerlich eine dominante Kaste
ins Leben ruft. Die Kommune zieht es also vor, sich aufzuspalten
und sich auf diese Weise auszudehnen, wodurch sie gleichzeitig
einem unglücklichem Ende zuvorkommt.
Der Aufstand der algerischen Jugend, der im Frühling 2001 die ganze
Kabylei in Brand setzte, erreichte eine beinahe totale
Wiederaneignung des Terrritoriums, mit Attacken auf
Gendarmerieposten, Gerichte und staatliche Vertretungen, die
Unruhen verallgemeinernd, bis zum einseitigen Rückzug der
Ordnungskräfte, bis zum physischen Verhindern der Wahlen. Die
Stärke der Bewegung lag in der diffusen Komplementarität
unterschiedlicher Komponenten – die nur zu einem kleinen Teil in
den endlosen und hoffnungslos männlichen Versammlungen der
Dorfkomitees und anderer Volkskomitees vertreten waren. Mal trugen
die »Kommunen« des noch immer schwelenden algerischen Aufstands das
Gesicht dieser »gebrannten« Jugendlichen mit Basecap, die vom Dach
eines Gebäudes in Tizi Ouzou Gasflaschen auf die CNS (CRS) warfen,
mal das spöttische Lächeln eines alten, in seinen Burnous gehüllten
Maquisards, und mal die Energie der Frauen eines Bergdorfes, die
allen Umständen zum Trotz den traditionellen Anbau und das Vieh
unterhielten, ohne welche die regionale Wirtschaft nie dermaßen
wiederholt und systematisch hätte blockiert werden können.

\subsection{Aus jeder Krise ein Feuer machen}

»Man muss außerdem hinzufügen, dass nicht die gesamte französische
Bevölkerung behandelt werden könnte. Daher gilt es eine Auswahl zu
treffen.« So fasst ein Experte der Virologie am 7. September 2005
in »le Monde« zusammen, was sich im Falle einer Pandemie der
Vogelgrippe ereignen würde. »Terroristische Bedrohungen«,
»Naturkatastrophen«, »Virenalarm«, »Soziale Bewegungen« und
»Städtische Gewalt« sind für die Verwalter der Gesellschaft
gleichermaßen Momente der Instabilität, in denen sie ihre Macht
durch die Selektion dessen sichern, was ihnen gefällt, und durch
die Vernichtung dessen, was sie stört. Dies ist also,
logischerweise, die Gelegenheit für jegliche andere Kraft, sich zu
festigen oder zu verstärken, indem sie dagegen Partei ergreift. Die
Unterbrechung der Warenflüsse, das Aussetzen der Normalität – es
genügt sich anzusehen, was bei einem plötzlichen Stromausfall an
sozialem Leben in ein Gebäude zurückkehrt, um sich vorzustellen, zu
was das Leben werden könnte in einer Stadt, in der alles versagt –
und der polizeilichen Kontrolle, setzen an Möglichkeiten der
Selbstorganisierung frei, was unter anderen Umständen unvorstellbar
wäre. Das ist allen bewusst. Die revolutionäre Arbeiterbewegung
hatte dies wohl verstanden, die aus den Krisen der bürgerlichen
Wirtschaft Höhepunkte ihrer wachsenden Stärke machte. Heutzutage
sind die islamischen Parteien nirgends so stark wie dort, wo es
ihnen gelingt, für die Schwäche des Staates auf intelligente Art
Ersatz zu bieten, zum Beispiel: bei der Einrichtung der Nothilfe
nach dem Erdbeben von Boumerdès in Algerien, oder bei der
alltäglichen Unterstützung der Bevölkerung im Süd-Libanon, der von
der israelischen Armee zerstört wurde.
Wie wir bereits oben erwähnten, hat die Verwüstung von New Orleans
durch den Hurrikan Cathrina einem wesentlichen Teil der
nordamerikanischen anarchistischen Bewegung Gelegenheit gegeben, zu
einer bisher unbekannten Konsistenz zu gelangen, indem sie sich an
die Seite all jener stellte, die sich vor Ort der Zwangsumsiedlung
widersetzten. Die Straßenküchen setzen voraus, sich Gedanken über
die Versorgung gemacht zu haben, die Erste Hilfe verlangt das
Aneignen des nötigen Wissens und Materials, genau wie das
Einrichten freier Radios. Das, was solche Erfahrungen an Freude
enthalten, das Überwinden des individuellen Durchwurstelns, diese
greifbare, nicht dem Alltag der Ordnung und der Arbeit unterworfene
Realität, garantiert ihre politische Fruchtbarkeit.

In einem Land wie Frankreich, wo die radioaktiven Wolken an der
Grenze anhalten, und wo man sich nicht fürchtet ein Krebszentrum
auf dem als Seveso-Zone\footnote{
U-Richtlinie von 1996 »zur Beherrschung der Gefahren bei
schweren Betriebsunfällen mit gefährlichen Stoffen und zur
Begrenzung der Unfallfolgen«, benannt nach der Giftmüll-Katastrophe
von Seveso, klassifiziert Verseuchungsgrade.
}
eingestuften ehemaligen Standort der
Fabrik AZF\footnote{
Am 21. September 2001 kam es in der Total-Fina-Elf gehörenden
Düngemittelfabrik AZF in Toulouse zu einer Explosion mehrerer
hundert Tonnen Ammoniumnitrat in einer Deponie für chemische
Abfälle. Ein Teil der Stadt wurde beschädigt, 31 Menschen starben,
mehrere Tausend wurden verletzt.
}%44
zu errichten, gilt es weniger, auf die »natürlichen«
Krisen zu setzen als auf die sozialen. Es liegt meist an den
sozialen Bewegungen, den normalen Ablauf des Desasters zu
unterbrechen. Sicher boten die verschiedenen Streiks der letzten
Jahre vor allem der Macht und den Unternehmensleitungen die
Gelegenheit, ihre Fähigkeit zur Bereitstellung eines immer
umfangreicheren »Mindest-Services« zu testen, bis sie die
Arbeitsniederlegung auf die rein symbolischen Dimension reduziert
haben – kaum schädlicher als ein Schneesturm oder ein Selbstmord
auf den Bahngleisen. Aber indem die Kämpfe der Schüler von 2005 und
gegen das CPE die etablierten aktivistischen Praxen der
systematischen Besetzung von Schulen und der hartnäckigen Blockaden
auf den Kopf stellten, haben sie die Fähigkeit zur Störung und zur
diffusen Offensive großer Bewegungen in Erinnerung gerufen. Mit all
den Banden, die ihrem Kielwasser entsprangen, ließen diese Kämpfe
erahnen, unter welchen Bedingungen Bewegungen zu einem Ort werden,
an dem sich neue Kommunen bilden.

\subsection{Jegliche Instanz der Repräsentation sabotieren\\
Das Palaver verallgemeinern\\
Die Vollversammlungen abschaffen}

Als erstes Hindernis, lange noch vor der eigentlichen Polizei,
trifft die soziale Bewegung auf die gewerkschaftlichen Kräfte und
all die Mikrobürokratie, deren Berufung es ist, die Kämpfe
einzuhegen. Die Kommunen, die Basisgruppen, die Banden misstrauen
ihnen spontan. Deswegen haben die Parabürokraten vor zwanzig Jahren
die Bündnisse erfunden, die durch ihre Abwesenheit eines Etiketts
einen unschuldigeren Eindruck machen, aber dadurch nicht weniger
das ideale Terrain ihrer Manöver bleiben. Wenn ein
orientierungsloses Kollektiv sich in Autonomie übt, hören sie nicht
auf, es von jeglichem Inhalt zu leeren, indem sie gute Fragen
entschlossen wegwischen. Sie sind erbittert, sie erregen sich;
nicht aus Leidenschaft für die Debatte, sondern in ihrer Berufung,
sie zu bannen. Und sobald das Kollektiv von ihrer unnachgiebigen
Verteidigung der Apathie überwältigt wurde, erklären sie dessen
Scheitern durch den Mangel an politischem Bewusstsein. Man muss
sagen, dass es den jungen Aktivisten in Frankreich, dank der
wahnsinnigen Aktivitäten der verschiedenen trotzkistischen Sekten,
nicht an der Kunst der politischen Manipulation fehlt. Sie sind es
nicht, die aus den Flammen des November 2005 folgende Lehre zu
ziehen wussten: Alle Bündnisse sind da überflüssig, wo man sich
verbündet, die Organisationen sind immer da zuviel, wo man sich
organisiert.
\extrapar{}

Ein weiterer Reflex ist, bei der kleinsten Bewegung eine
Vollversammlung einzuberufen und abzustimmen. Das ist ein Fehler.
Allein, was mit der Wahl, der Entscheidung zu gewinnen, auf dem
Spiel steht, reicht, die Versammlung in einen Alptraum zu
verwandeln, in ein Theater, in dem sich alle Ansprüche auf die
Macht gegenüberstehen. Wir stehen hier unter dem Einfluss des
schlechten Vorbilds der bürgerlichen Parlamente. Die Versammlung
ist nicht für die Entscheidung gemacht, sondern für das Palaver,
für das freie, ziellos ausgeübte Wort.
Das Bedürfnis, sich zu versammeln, ist so konstant bei den
Menschen, wie die Notwendigkeit, Entscheidungen zu fällen selten
ist. Sich zu versammeln entspricht der Freude, eine gemeinsame
Stärke zu erleben. Entscheiden ist nur in Notsituationen
lebenswichtig, wo die Ausübung der Demokratie ohnehin fraglich ist.
In der restlichen Zeit besteht das Problem des »demokratischen
Charakters des Entscheidungsprozesses« nur für Fanatiker der
Prozedur. Es geht nicht darum, die Versammlungen zu kritisieren
oder sich von ihnen abzuwenden, sondern in ihnen die Worte, die
Gesten und Spiele zwischen den Wesen zu befreien. Es reicht zu
erkennen, dass jeder nicht nur mit einem Standpunkt, einem Antrag,
sondern mit Wünschen, Verbundenheit, Fähigkeiten, Stärken,
Traurigkeiten, und einer gewissen Verfügbarkeit dorthin kommt.
Sollte es nun gelingen, das Hirngespinst der Vollversammlung zu
zerreißen, zugunsten einer Versammlung der Anwesenheiten, wenn es
gelingt, der immer wieder auflebenden Versuchung der Hegemonie zu
widerstehen, wenn wir aufhören, uns auf die Entscheidung als Zweck
festzulegen, dann gibt es einige Chancen, dass sich eine dieser
kritischen Massen ergibt, eines jener Phänomene der kollektiven
Kristallisation, in der eine Entscheidung die Wesen ergreift, in
ihrer Gesamtheit oder nur zum Teil.

\extrapar{}

Das Gleiche gilt bei der Entscheidung über Aktionen. Vom Prinzip
auszugehen, dass »sich der Verlauf einer Versammlung an der Aktion
ausrichten soll«, verunmöglicht sowohl das Aufwallen der Debatte
wie die effiziente Aktion. Eine Versammlung von zahlreichen Fremden
verdammt sich dazu, Spezialisten der Aktion zu schaffen, was
bedeutet, die Aktion zugunsten ihrer Kontrolle zu vernachlässigen.
Auf der einen Seite sind die Beauftragten per Definition in ihrer
Aktion eingeschränkt, auf der anderen Seite hindert sie nichts
daran, alle zu bevormunden.
\extrapar{}

Es geht nicht darum, der Aktion eine ideale Form zuzuweisen.
Hauptsache, die Aktion gibt sich eine Form, bringt diese hervor und
ordnet sich ihr nicht unter. Dies setzt voraus, eine gleiche
politische, geographische Position zu teilen – wie einst die
Sektionen der Pariser Kommune während der Französischen Revolution
– sowie ein gleiches, zirkulierendes Wissen. Geht es darum, über
Aktionen zu entscheiden, so könnte das Prinzip lauten: Jeder holt
eigene Erkundungen ein, die Übereinstimmung der Nachrichten wird
geprüft, und die Entscheidung wird von alleine kommen, ergreift uns
mehr, als wir sie ergreifen. Die Zirkulation des Wissens hebt die
Hierarchie auf, sie gleicht alles von oben an. Horizontale, um sich
greifende Kommunikation, dies ist auch die beste Form, die
verschiedenen Kommunen zu koordinieren, um sich von der Hegemonie
zu verabschieden.

\subsection{Die Wirtschaft blockieren, aber unsere Stärke zu 
blockieren an unserem Niveau der Selbstorganisierung messen.}

Ende Juni 2006 mehren sich im ganzen Staat von Oaxaca die
Besetzungen der Rathäuser, die Aufständischen besetzen öffentliche
Gebäude. In einigen Gemeinden vertreiben sie die Bürgermeister und
beschlagnahmen die offiziellen Fahrzeuge. Einen Monat später sind
die Zugänge zu gewissen Hotels und Tourismuskomplexen blockiert.
Der Tourismusminister spricht von einer »mit dem Hurrikan Wilma
vergleichbaren« Katastrophe. Einige Jahre zuvor war die Blockade zu
einer der wichtigsten Aktionsformen der argentinischen Bewegung der
Revolte geworden, dabei unterstützten sich die verschiedenen
lokalen Gruppen gegenseitig, indem sie diese oder jene
Verkehrsachse blockierten, und durch ihre gemeinsame Aktion ständig
drohten, das gesamte Land zu lähmen, sollten ihre Forderungen nicht
erfüllt werden. Eine derartige Drohung war lange Zeit ein
kraftvoller Hebel in den Händen der Bahnarbeiter, Gasarbeiter,
Elektriker und Lastwagenfahrer. Die Bewegung gegen das CPE zögerte
nicht Bahnhöfe, Ringstraßen, Fabriken, Autobahnen, Supermärkte und
sogar Flughäfen zu blockieren. In Rennes bedurfte es nicht mehr als
dreihundert Personen, um die Umgehungsstrasse für Stunden
lahmzulegen und vierzig Kilometer Stau zu verursachen.
Alles blockieren ist deshalb der erste Reflex all dessen, was sich
gegen die gegenwärtige Ordnung richtet. In einer ausgelagerten
Wirtschaft, in der die Unternehmen »Just in time« funktionieren, wo
der Wert sich aus ihrer Verbindung zum Netzwerk herleitet, wo die
Autobahnen Glieder der entmaterialisierten Produktionskette sind,
die von Subunternehmer zu Subunternehmer und von da zur
Montagefabrik gehen, heißt die Produktion zu blockieren ebenso die
Zirkulation zu blockieren.

Aber es kann nicht darum gehen, mehr zu blockieren, als es die
Fähigkeit zur Versorgung und Kommunikation der Aufständischen, die
tatsächliche Selbstorganisierung der verschiedenen Kommunen
erlaubt. Wie können wir uns ernähren, wenn alles lahmgelegt ist?
Die Geschäfte zu plündern, wie dies in Argentinien gemacht wurde,
hat seine Grenzen; so gewaltig die Konsumtempel auch sind, sie sind
keine unerschöpfliche Vorratskammer. Auf Dauer die Fähigkeit zu
erlangen, sich seine grundlegende Versorgung selbst zu schaffen,
bedingt also, sich die Mittel ihrer Produktion anzueignen. Und an
diesem Punkt scheint es wohl unnötig, noch länger zu warten. Zwei
Prozent der Bevölkerung die Nahrungsmittelproduktion für alle
anderen zu überantworten, wie es heute geschieht, ist historischer
sowie strategischer Unsinn.

\subsection{Das Territorium von der polizeilichen Besatzung befreien.\\
Soweit möglich die direkte Konfrontation vermeiden.}

»Diese Geschichte macht klar, dass wir es hier nicht mit
Jugendlichen zu tun haben, die ein Mehr an Sozialem fordern,
sondern mit Individuen, die der Republik den Krieg erklären«,
bemerkte ein hellsichtiger Bulle angesichts der kürzlich gelegten
Hinterhalte\footnote{
Nach der Wahl von Sarkozy wurden in der Pariser Blanlieue immer
wieder Polizeifahrzeuge, aber auch viele Linienbusse in Hinterhalte
gelockt und angezündet.
}%45
. Die Offensive zur Befreiung des Territoriums von
seiner polizeilichen Besatzung hat bereits angefangen, und kann auf
die unerschöpflichen Reserven von Ressentiments zählen, welche
diese Kräfte gegen sich gesammelt haben. In den »sozialen
Bewegungen« selbst verbreitet sich nach und nach die Revolte,
ebenso wie unter den Feiernden in Rennes, die im Jahr 2005 jeden
Donnerstagabend der CRS gegenüber traten, oder jene in Barcelona,
welche jüngst anlässlich eines Botellons eine der kommerziellen
Arterien der Stadt verwüsteten. Die CPE-Bewegung hat die
regelmäßige Rückkehr des Molotov Cocktails erlebt. Aber in diesem
Punkt bleiben einige Banlieues unübertroffen. Vor allem in der
Technik, die sich schon seit langem hält: der Hinterhalt. Wie der
vom 13. Oktober 2006, in Epinay: Eine Einsatzgruppe der BAC fährt
um 23 Uhr nach einem Anruf wegen Diebstahls aus einem Auto umher;
bei ihrer Ankunft fand sich eine der Einsatzgruppen blockiert »von
zwei quer über die Straße stehenden Fahrzeugen [...] und von mehr
als dreißig Individuen, mit Eisenstangen und Handfeuerwaffen
ausgerüstet, die Steine auf das Fahrzeug warfen und Tränengas gegen
die Polizisten einsetzten«. In kleinerem Maßstab denkt man an die
Stadtteil-Polizeiwachen, die außerhalb der Öffnungszeiten
angegriffen werden: Eingeschlagene Scheiben, abgefackelte Autos.
Eine der Errungenschaften der letzten Bewegungen ist, dass eine
echte Demonstration eine »wilde« ist, ohne Anmeldung bei der
Präfektur. Was die Wahl des Terrains anbelangt, täten wir gut
daran, uns ein Beispiel am Schwarzen Block 2001 in Genua zu nehmen,
die roten Zonen zu umgehen, die direkte Konfrontation zu fliehen
und, in Entscheidung des Weges, die Bullen zum Laufen zu bringen,
statt von ihnen zum Laufen gebracht zu werden – vor allem die der
Gewerkschaften, vor allem die der Pazifisten. Es hat sich gezeigt,
dass es tausend Entschlossenen gelingen kann, ganze Wannen der
Carabinieri zurückzudrängen, um sie schließlich in Brand zu
stecken. Wichtig ist nicht, der besser Bewaffnete zu sein, sondern
die Initiative zu ergreifen. Der Mut ist nichts, das Vertrauen in
den eigenen Mut alles. Die Initiative ergreifen und beitragen.

Alles anstacheln, dennoch, die direkte Konfrontation als Punkte der
Fixierung der gegnerischen Kräfte erlaubt es, abzuwarten und
woanders anzugreifen – auch ganz nah. Dass man nicht verhindern
kann, dass eine Konfrontation stattfindet, verbietet nicht, sie als
einfache Ablenkung zu nutzen. Noch mehr als um die Aktionen muss
man sich um ihre Koordination kümmern. Die Polizei zu belästigen,
heißt so zu handeln, dass sie, indem sie überall präsent ist,
nirgendwo mehr effizient ist.

Jeder Akt der Belästigung belebt die 1842 ausgesprochene Wahrheit
wieder: »Das Leben des Polizeibeamten ist peinlich; seine Position
inmitten der Gesellschaft ist genauso erniedrigend und verachtet
wie das Verbrechen selbst. […] Die Scham und die Infamie
umschließen ihn von allen Seiten, die Gesellschaft verjagt ihn aus
ihrem Innern, isoliert ihn wie einen Paria, speit ihm ihre
Missachtung mit seinem Lohn entgegen, ohne Reue, ohne Bedauern,
ohne Mitleid, […] der Polizeiausweis in seiner Tasche ist ein
Zeugnis der Schande.« Am 21. November 2006 haben demonstrierende
Feurwehrmänner die CRS mit Hammerschlägen angegriffen und fünfzehn
von ihnen verletzt. Dies, um in Erinnerung zu rufen, dass die
»Berufung zum Helfen« nie eine gültige Entschuldigung dafür sein
wird, die Polizei mit einzubinden.

\subsection{Bewaffnet sein. Alles daran setzen, die Nutzung der Waffen 
überflüssig zu machen. Gegen die Armee ist der Sieg politisch.}

Es gibt keinen friedlichen Aufstand. Waffen sind notwendig: Es geht
darum, alles daran zu setzen, ihre Nutzung überflüssig zu machen.
Ein Aufstand ist vielmehr ein Griff zu den Waffen, eine »bewaffnete
Permanenz« als ein Schritt in den bewaffneten Kampf. Wir haben
alles Interesse daran, die Bewaffnung von der Nutzung der Waffen zu
unterscheiden. Die Waffen sind eine revolutionäre Konstante, auch
wenn ihre Nutzung selten war, oder selten entscheidend war, in den
Momenten großer Umwälzungen: Der 10. August 1792, der 18. März
1871, Oktober 1917. Wenn die Macht in der Gosse liegt, genügt es,
sie zu zertreten.
In der Distanz, die uns von ihnen trennt, haben Waffen jenen
doppelten Charakter von Faszination und Abscheu angenommen, der nur
durch ihre Handhabung überwunden werden kann. Ein authentischer
Pazifismus kann nicht in der Ablehnung der Waffen bestehen, sondern
in der ihrer Nutzung. Pazifist zu sein, ohne feuern zu können, ist
nur die Theoretisierung einer Ohnmacht. Dieser Pazifismus a priori
entspricht einer Art präventiver Entwaffnung, er ist eine rein
polizeiliche Operation. In Wahrheit stellt sich die Frage des
Pazifismus ernsthaft nur für den, der die Macht hat zu feuern. Und
in diesem Fall wäre der Pazifismus im Gegenteil ein Zeichen der
Stärke, denn nur aus einer Position extremer Stärke heraus ist man
von der Notwendigkeit zu feuern befreit.

Von einem strategischen Blickpunkt erscheint die indirekte,
asymmetrische Aktion die lohnendste, die der Epoche am besten
angepasste: Man greift eine Besatzungsarmee nicht frontal an. Auch
wenn die Perspektive einer Stadtguerilla wie im Irak, die sich ohne
Möglichkeit zur Offensive eingräbt, mehr zu fürchten als zu
wünschen ist. Die Militarisierung des Bürgerkriegs ist das
Scheitern des Aufstands. Die Roten mögen 1921 triumphieren, die
Russische Revolution ist bereits verloren.

Wir müssen zwei Formen staatlicher Reaktion in Betracht ziehen.
Eine der klaren Feindseligkeit, die andere heimtückischer,
demokratisch. Die erste, zur wortlosen Zerstörung aufrufend, die
zweite, von einer subtilen, aber erbarmungslosen Feindseligkeit:
Sie wartet nur darauf, uns einzuziehen. Man kann von der Diktatur
besiegt werden oder von der Tatsache, darauf reduziert zu sein,
sich gegen nichts anderes als die Diktatur zu wehren. Die
Niederlage besteht ebenso im Verlieren eines Kriegs wie im Verlust
der Wahl des zu führenden Krieges. Beides bleibt möglich, wie
Spanien 1936 beweist: Durch den Faschismus und durch die Republik
wurden die Revolutionäre doppelt besiegt.

Wenn die Sache ernst wird, besetzt die Armee das Territorium. Ihr
Einsatz scheint weniger offenkundig. Dafür bräuchte es einen zu
einem Blutbad entschlossenen Staat, was nur als Drohung aktuell
ist, in etwa wie der Einsatz von Atomwaffen seit einem halben
Jahrhundert. Aber es bleibt dabei, die seit langem verletzte
staatliche Bestie ist gefährlich. Es bleibt dabei, dass es
gegenüber einer Armee einer zahlreichen Menschenmenge bedarf,
welche die Ränge durchdringt, sich verbrüdert. Es braucht den 18.
März 1871. Die Armee in den Strassen ist eine aufständische
Situation. Die Armee im Einsatz ist das sich beschleunigende Ende.
Jeder ist aufgefordert Position zu beziehen, sich zu entscheiden
für die Anarchie oder die Angst vor der Anarchie. Ein Aufstand
triumphiert als politische Kraft. Politisch ist es nicht unmöglich,
eine Armee zu besiegen.

\subsection{Die Autoritäten lokal absetzen.}

Die Frage, die sich einem Aufstand stellt, ist es, sich unumkehrbar
zu machen. Die Unumkehrbarkeit ist erreicht, wenn gleichzeitig mit
den Autoritäten auch der Bedarf nach Autorität besiegt wird,
gleichzeitig mit dem Eigentum auch die Lust nach Aneignung,
gleichzeitig mit der Hegemonie auch das Streben nach Hegemonie.
Deswegen trägt der aufständische Prozess die Form seines Sieges
oder seines Scheiterns in sich selbst. Im Hinblick auf die
Unumkehrbarkeit reichte die Zerstörung nie aus. Alles liegt im Wie.
Es gibt Arten der Zerstörung, die unweigerlich die Rückkehr dessen
hervorrufen, was man vernichtet hat. Wer sich im Kadaver einer
Ordnung verbeißt, stellt sicher, dass die Berufung geweckt wird,
sie zu rächen. Deshalb gilt es überall, wo die Wirtschaft blockiert
und die Polizei neutralisiert ist, so wenig Pathos wie möglich in
den Umsturz der Autoritäten zu legen. Sie sind mit bedingungsloser
Frechheit und Spott abzusetzen.
\extrapar{}

Der Dezentralisierung der Macht entspricht in dieser Epoche das
Ende der revolutionären Zentralitäten. Winterpaläste gibt es wohl
immer noch, sie werden aber vielmehr Ziel des Sturms von Touristen
als von Aufständischen. Paris, Rom oder Buenos Aires können
heutzutage eingenommen werden, ohne damit eine Entscheidung zu
erringen. Die Einnahme Rungis\footnote{
Stadt südlich von Paris, in der alle Lebensmittel für die
Pariser Region angeliefert, gelagert und verteilt werden.
}
hätte sicher mehr Auswirkungen als
die des Elysée Palastes. Die Macht konzentriert sich nicht mehr an
einem Punkt der Welt, sie selbst ist diese Welt, ihre Flüsse und
Straßen, ihre Menschen und Normen, ihre Codes und Technologien. Die
Macht ist die Organisation der Metropole selbst. Sie ist die
makellose Totalität der Warenwelt in all ihren Punkten. Wer sie
lokal besiegt, produziert quer durch die Netzwerke eine planetare
Schockwelle. Die Angreifer von Clichy-sous-Bois haben in mehr als
einem amerikanischen Haushalt für Freude gesorgt, während die
Aufständischen von Oaxaca Komplizen im Herzen von Paris gefunden
haben. Für Frankreich bedeutet der Verlust der Zentralität der
Macht das Ende der revolutionären Zentralität von Paris. Dies wird
von jeder neuen Bewegung seit den Streiks von 1995 bestätigt. Die
gewagtesten, die konsistentesten Handlungen tauchen dort nicht mehr
auf. Schließlich hebt sich Paris nur noch als bloßes Ziel von
Razzien, als pures Terrain der Plünderung und der Verwüstung ab.
Kurze und brutale Einfälle von außerhalb, die sich daran machen,
die metropolitanen Flüsse am Punkt maximaler Dichte anzugreifen.
Spuren der Wut, welche die Wüste dieses künstlichen Überflusses
durchziehen und sich verlieren. Ein Tag wird kommen, an dem dieses
fürchterliche Konzentrat der Macht, die Hauptstadt, in großem Stil
zerfallen sein wird, dies aber im Abschluss eines Prozesses, der
überall sonst schon weiter fortgeschritten sein wird als dort.

\subsection{Alle Macht den Kommunen!}

\textit{In der Metro ist keine Spur mehr zu finden vom Schleier der 
Befangenheit, der üblichweise die Gesten der Passagiere hemmt. Die 
Fremden reden miteinander, sie sprechen sich nicht mehr an. Eine 
Bande im geheimer Absprache an einer Straßenecke. Größere 
Zusammenkünfte auf den Hauptstraßen, die ernstlich diskutieren. 
Die Angriffe antworten aufeinander von einer Stadt zur andern, von 
einem Tag zum andern. Eine weitere Kaserne wurde geplündert und 
niedergebrannt. Die Bewohner eines geräumten Wohnheims haben die 
Verhandlungen mit dem Rathaus abgebrochen: sie wohnen dort. In einer 
Anwandlung von Klarheit erstarrt ein Manager inmitten einer Sitzung 
einer Handvoll Kollegen. Akten mit den persönlichen Adressen aller 
Polizisten und Gendarmen sowie der Angestellten des Strafvollzugs 
sickern durch und haben zu einer bislang unbekannten Welle 
überstürzter Umzüge geführt. In die alte Épicerie-Bar des Dorfes wird 
der produzierte Überschuss gebracht und geholt, was uns fehlt. Dort 
trifft man sich auch, um zu diskutieren, über die allgemeine Situation 
und das notwendige Material für die Werkstatt. Das Radio informiert 
die Aufständischen über den Rückzug der Regierungstruppen. Eine 
Granate hat gerade eben die Mauer des Gefängnisses Clairveaux 
aufgerissen. Unmöglich zu sagen, ob ein Monat oder Jahre vergangen 
sind, seit die »Ereignisse« begonnen haben. Der Premierminister mit 
seinen Mahnungen zur Ruhe scheint ziemlich alleine dazustehen.}
\end{document}
