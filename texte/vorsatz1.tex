% This file was generated with something like 
% pandoc -o vorsatz1.tex was.mdwn wie.mdwn
% and hand-edited afterwards ...

\usepackage[T1]{fontenc}
\usepackage{ngerman}

\begin{document}
\mbox{}\newpage

\section{Was die Creative Commons Bibliothek ist}

Wir wollen Texte, die im Internet frei verfügbar sind, weil sie
entweder unter einer freien Lizenz stehen oder das Urheberrecht
bereits verjährt ist, auch in gedruckter Form zugänglich machen.

Als frei verfügbar gelten Texte für uns, wenn sie sowohl ohne
rechtliche EinschrAñnkung als auch ohne technische Hindernisse
kopiert, gedruckt und weitergegeben werden können. Das schließt
auch eine Veränderung von Layout und Satz des Textes vor dem
Drucken ein.

Für alle Texte, die bei uns in gedruckter Form im Regal stehen,
haben wir auch die nötigen digitalen Daten, um sie in
gleichwertiger Form neu auszudrucken. Daher ist es nicht weiter
schlimm, falls einmal ein Buch verschwindet und wir führen auch
keine Aufzeichnungen, wer wann welches Buch entlehnt hat. Dennoch
erwarten wir uns, dass unsere Bücher sorgsam behandelt und so bald
als möglich zurückgebracht werden. Für die Rückgabe eines Buchs ist
nicht allein die Person verantwortlich, die es ausgeborgt hat,
sondern wer auch immer das Buch gerade besitzt. Zum Beispiel ist es
in unserem Sinn, wenn ein entlehntes Buch im Freundeskreis
herumgereicht wird und der letzte Leser es dann zurück bringt.

\section{Wie funktioniert die Creative Commons Bibliothek?}

Im einfachsten Fall steht das Buch, das du lesen willst, bereits
bei uns im Regal. Du kannst es sofort ausborgen. Wir erwarten dafür
keine Spende und du musst auch keine Daten hergeben. Du bekommst
also einen großen Vertrauensvorschuss, bitte geh' damit
verantwortungsvoll um und bring' das Buch sobald wie möglich
zurück

\subsection{Wenn alle Exemplare entlehnt sind oder du das Buch selbst drucken willst}

Es kann passieren, dass alle Exemplare verborgt sind, und wir
wissen in diesem Fall auch nicht, wann wir das nächste Exemplar
zurückbekommen. Es ist deine Entscheidung, ob du warten willst oder
ob du ein neues Exemplar druckst.

Wenn du dich für das Drucken entscheidest, dann helfen wir dir
natürlich beim Drucken und Binden. Besonders beim ersten Mal
brauchst du also nichts außer das Interesse zu lernen, wie Bücher
hergestellt werden können. Außerdem kannst du die Gestaltung des
Buchs -- also Format, Schriftgröße, Layout, Cover, etc. selbst
bestimmen. Wir haben dafür auch schon einige Standardvorlagen
vorbereitet. Ob du das Buch, das du produzierst, der Creative
Commons Bibliothek gibst oder lieber ins eigene Regal stellst,
bleibt natürlich dir überlassen.

In allen Fällen -- also auch wenn du ein Buch für die Bibliothek
druckst -- erwarten wir, dass du die Materialkosten (für Papier,
\discretionary{Druk-}{ker}{Drucker}\-far\-be, etc.) ersetzt.

\subsection{Ein neues Buch in die CCBib aufnehmen}

Bevor ein Buch das erste Mal gedruckt werden kann sind einige
Vorbereitungsarbeiten nötig: Im Internet sind Texte typischerweise
als HTML-Datei vorhanden. Um daraus einen Druck von hoher Qualität
zu bekommen, muss der Text in ein Dateiformat, das sich für
Computersatz eignet, umgewandelt werden. Wir versuchen das so gut
es geht zu automatisieren, aber die Erfahrung zeigt, dass ein
gewisses Maß an manueller Arbeit notwendig ist. Außerdem muss das
Ergebnis dieses Umwandlungsprozess vor dem Drucken nocheinmal
gesichtet werden.

Grundsätzlich können alle legal (das bezieht sich primär auf unser
extrem restriktives Urheberrecht) druckbaren Texte in die
Bibliothek aufgenommen werden, sobald jemand die oben genannten
Vorbereitungsarbeiten erledigt hat. Leider ist dafür ein gewisses
Maß an Arbeit, Wissen und Erfahrung notwendig. Wir bemühen uns
dieses Wissen zu vermitteln, aber es ist klar, dass wir in den
meisten Fällen helfen müssen. Wir versuchen diesen Prozess künftig
einfacher und unkomplizierter zu machen. Es ist unser Anspruch,
dass nicht wir entscheiden welche Texte durch die Bibliothek
verfügbar sind, sondern jene, die diese Texte tatsächlich lesen
bzw. drucken wollen. -- Trotzdem entscheidet natürlich jeder selbst,
welche Texte er mit der eigenen Arbeit unterstützt.

\end{document}
