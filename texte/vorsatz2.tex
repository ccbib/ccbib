% This file was generated with something like
% pandoc -o vorsatz2.tex warum.mdwn
% and hand-edited afterwards ...

\usepackage[T1]{fontenc}
\usepackage{ngerman}

\begin{document}
\mbox{}\newpage

\section{Warum das Alles? Warum gerade in der Form?}

Das Konzept der Creative-Commons Bibliothek, wie wir es versuchen
umzusetzen, hat ein paar Eigenschaften, die uns mehr oder weniger
gut gefallen:

\begin{itemize}
\item 
  Es gibt keine (zentrale) Instanz, die entscheidet welche Texte in
  die Bibliothek aufgenommen werden.
\item 
  Niemand kann die Aufnahme eines freien Textes blockieren.
\item 
  Es gibt keinen Verwaltungsaufwand darüber wer wann welches Buch
  ausgeborgt hat und ob es schon zurück ist oder nicht.
\item 
  Jeder entscheidet selbst wie und wieviel er mitarbeitet. Jeder kann
  über das Produkt seiner Arbeit frei bestimmen -- sowohl was die Form
  betrifft wie auch die Verwendung.
\item 
  Die klassische Arbeitsteilung (Autor -- Verlegerin -- Typograph --
  Setzerin -- Drucker -- Binderin -- Händler -- Leser), wie sie für
  kapitalistische Produktionsverhältnisse typisch ist, wird
  durchbrochen. Stattdessen gibt es einen selbstregulierenden
  Prozess, der von den Interessen der Beteiligten ausgeht.
\end{itemize}
Daneben machen wir uns noch Hoffnungen, die die Entwicklung unseres
Konzepts beeinflusst haben:

Es wäre sehr schön, wenn das Vorhandensein einer Creative-Commons
Bibliothek mehr Autoren dazu anregt, ihre eigenen Texte unter einer
freien Lizenz zu veröffentlichen. Wir wollen zeigen, dass sie
nicht länger ihre Rechte uneingeschränkt an einen großen Verlagskonzern
abtreten müssen, um eine hohe Reichweite zu erzielen. Es ist ihre
Entscheidung, ob ihnen ungehinderter Zugang zu ihren Texten wichtig
ist oder maximaler Profit.

Leser finden vielleicht Projekte wie die
"`Distributed Proofreaders"' \texttt{http://www.pgdp.net/} interessant.

Das Konzept der Creative-Commons Bibliothek orientiert sich stark
daran wie die Entwicklung freier Software (z.\,B. Linux)
funktioniert: Freie Software ist ein beeindruckendes Beispiel
dafür, was mit Solidarökonomie alles möglich ist. Dort aktiv
mitzumachen ist eine große Erfahrung, die die Zuversicht erzeugt,
dass eine andere Welt tatsächlich möglich ist. Leider stellt das
dafür nötige Wissen doch eine schwere Anfangshürde dar -- wir
hoffen, dass die Bibliothek ein ähnliches Gefühl vermitteln kann
mit geringerem Lernaufwand.

Natürlich hoffen wir, dass die Gesellschaft beginnt, das
Urheberrecht in seiner aktuellen Form -- noch dazu seit einigen
Jahren zeitlich massiv verlängert -- in Frage zu stellen und über
Alternativen nachzudenken.

\end{document}
